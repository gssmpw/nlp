
\documentclass[10pt]{article} % For LaTeX2e
% \usepackage{tmlr}
% If accepted, instead use the following line for the camera-ready submission:
%\usepackage[accepted]{tmlr}
% To de-anonymize and remove mentions to TMLR (for example for posting to preprint servers), instead use the following:
\usepackage[preprint]{tmlr}

% Optional math commands from https://github.com/goodfeli/dlbook_notation.
%%%%% NEW MATH DEFINITIONS %%%%%

\usepackage{amsmath,amsfonts,bm}
\usepackage{derivative}
% Mark sections of captions for referring to divisions of figures
\newcommand{\figleft}{{\em (Left)}}
\newcommand{\figcenter}{{\em (Center)}}
\newcommand{\figright}{{\em (Right)}}
\newcommand{\figtop}{{\em (Top)}}
\newcommand{\figbottom}{{\em (Bottom)}}
\newcommand{\captiona}{{\em (a)}}
\newcommand{\captionb}{{\em (b)}}
\newcommand{\captionc}{{\em (c)}}
\newcommand{\captiond}{{\em (d)}}

% Highlight a newly defined term
\newcommand{\newterm}[1]{{\bf #1}}

% Derivative d 
\newcommand{\deriv}{{\mathrm{d}}}

% Figure reference, lower-case.
\def\figref#1{figure~\ref{#1}}
% Figure reference, capital. For start of sentence
\def\Figref#1{Figure~\ref{#1}}
\def\twofigref#1#2{figures \ref{#1} and \ref{#2}}
\def\quadfigref#1#2#3#4{figures \ref{#1}, \ref{#2}, \ref{#3} and \ref{#4}}
% Section reference, lower-case.
\def\secref#1{section~\ref{#1}}
% Section reference, capital.
\def\Secref#1{Section~\ref{#1}}
% Reference to two sections.
\def\twosecrefs#1#2{sections \ref{#1} and \ref{#2}}
% Reference to three sections.
\def\secrefs#1#2#3{sections \ref{#1}, \ref{#2} and \ref{#3}}
% Reference to an equation, lower-case.
\def\eqref#1{equation~\ref{#1}}
% Reference to an equation, upper case
\def\Eqref#1{Equation~\ref{#1}}
% A raw reference to an equation---avoid using if possible
\def\plaineqref#1{\ref{#1}}
% Reference to a chapter, lower-case.
\def\chapref#1{chapter~\ref{#1}}
% Reference to an equation, upper case.
\def\Chapref#1{Chapter~\ref{#1}}
% Reference to a range of chapters
\def\rangechapref#1#2{chapters\ref{#1}--\ref{#2}}
% Reference to an algorithm, lower-case.
\def\algref#1{algorithm~\ref{#1}}
% Reference to an algorithm, upper case.
\def\Algref#1{Algorithm~\ref{#1}}
\def\twoalgref#1#2{algorithms \ref{#1} and \ref{#2}}
\def\Twoalgref#1#2{Algorithms \ref{#1} and \ref{#2}}
% Reference to a part, lower case
\def\partref#1{part~\ref{#1}}
% Reference to a part, upper case
\def\Partref#1{Part~\ref{#1}}
\def\twopartref#1#2{parts \ref{#1} and \ref{#2}}

\def\ceil#1{\lceil #1 \rceil}
\def\floor#1{\lfloor #1 \rfloor}
\def\1{\bm{1}}
\newcommand{\train}{\mathcal{D}}
\newcommand{\valid}{\mathcal{D_{\mathrm{valid}}}}
\newcommand{\test}{\mathcal{D_{\mathrm{test}}}}

\def\eps{{\epsilon}}


% Random variables
\def\reta{{\textnormal{$\eta$}}}
\def\ra{{\textnormal{a}}}
\def\rb{{\textnormal{b}}}
\def\rc{{\textnormal{c}}}
\def\rd{{\textnormal{d}}}
\def\re{{\textnormal{e}}}
\def\rf{{\textnormal{f}}}
\def\rg{{\textnormal{g}}}
\def\rh{{\textnormal{h}}}
\def\ri{{\textnormal{i}}}
\def\rj{{\textnormal{j}}}
\def\rk{{\textnormal{k}}}
\def\rl{{\textnormal{l}}}
% rm is already a command, just don't name any random variables m
\def\rn{{\textnormal{n}}}
\def\ro{{\textnormal{o}}}
\def\rp{{\textnormal{p}}}
\def\rq{{\textnormal{q}}}
\def\rr{{\textnormal{r}}}
\def\rs{{\textnormal{s}}}
\def\rt{{\textnormal{t}}}
\def\ru{{\textnormal{u}}}
\def\rv{{\textnormal{v}}}
\def\rw{{\textnormal{w}}}
\def\rx{{\textnormal{x}}}
\def\ry{{\textnormal{y}}}
\def\rz{{\textnormal{z}}}

% Random vectors
\def\rvepsilon{{\mathbf{\epsilon}}}
\def\rvphi{{\mathbf{\phi}}}
\def\rvtheta{{\mathbf{\theta}}}
\def\rva{{\mathbf{a}}}
\def\rvb{{\mathbf{b}}}
\def\rvc{{\mathbf{c}}}
\def\rvd{{\mathbf{d}}}
\def\rve{{\mathbf{e}}}
\def\rvf{{\mathbf{f}}}
\def\rvg{{\mathbf{g}}}
\def\rvh{{\mathbf{h}}}
\def\rvu{{\mathbf{i}}}
\def\rvj{{\mathbf{j}}}
\def\rvk{{\mathbf{k}}}
\def\rvl{{\mathbf{l}}}
\def\rvm{{\mathbf{m}}}
\def\rvn{{\mathbf{n}}}
\def\rvo{{\mathbf{o}}}
\def\rvp{{\mathbf{p}}}
\def\rvq{{\mathbf{q}}}
\def\rvr{{\mathbf{r}}}
\def\rvs{{\mathbf{s}}}
\def\rvt{{\mathbf{t}}}
\def\rvu{{\mathbf{u}}}
\def\rvv{{\mathbf{v}}}
\def\rvw{{\mathbf{w}}}
\def\rvx{{\mathbf{x}}}
\def\rvy{{\mathbf{y}}}
\def\rvz{{\mathbf{z}}}

% Elements of random vectors
\def\erva{{\textnormal{a}}}
\def\ervb{{\textnormal{b}}}
\def\ervc{{\textnormal{c}}}
\def\ervd{{\textnormal{d}}}
\def\erve{{\textnormal{e}}}
\def\ervf{{\textnormal{f}}}
\def\ervg{{\textnormal{g}}}
\def\ervh{{\textnormal{h}}}
\def\ervi{{\textnormal{i}}}
\def\ervj{{\textnormal{j}}}
\def\ervk{{\textnormal{k}}}
\def\ervl{{\textnormal{l}}}
\def\ervm{{\textnormal{m}}}
\def\ervn{{\textnormal{n}}}
\def\ervo{{\textnormal{o}}}
\def\ervp{{\textnormal{p}}}
\def\ervq{{\textnormal{q}}}
\def\ervr{{\textnormal{r}}}
\def\ervs{{\textnormal{s}}}
\def\ervt{{\textnormal{t}}}
\def\ervu{{\textnormal{u}}}
\def\ervv{{\textnormal{v}}}
\def\ervw{{\textnormal{w}}}
\def\ervx{{\textnormal{x}}}
\def\ervy{{\textnormal{y}}}
\def\ervz{{\textnormal{z}}}

% Random matrices
\def\rmA{{\mathbf{A}}}
\def\rmB{{\mathbf{B}}}
\def\rmC{{\mathbf{C}}}
\def\rmD{{\mathbf{D}}}
\def\rmE{{\mathbf{E}}}
\def\rmF{{\mathbf{F}}}
\def\rmG{{\mathbf{G}}}
\def\rmH{{\mathbf{H}}}
\def\rmI{{\mathbf{I}}}
\def\rmJ{{\mathbf{J}}}
\def\rmK{{\mathbf{K}}}
\def\rmL{{\mathbf{L}}}
\def\rmM{{\mathbf{M}}}
\def\rmN{{\mathbf{N}}}
\def\rmO{{\mathbf{O}}}
\def\rmP{{\mathbf{P}}}
\def\rmQ{{\mathbf{Q}}}
\def\rmR{{\mathbf{R}}}
\def\rmS{{\mathbf{S}}}
\def\rmT{{\mathbf{T}}}
\def\rmU{{\mathbf{U}}}
\def\rmV{{\mathbf{V}}}
\def\rmW{{\mathbf{W}}}
\def\rmX{{\mathbf{X}}}
\def\rmY{{\mathbf{Y}}}
\def\rmZ{{\mathbf{Z}}}

% Elements of random matrices
\def\ermA{{\textnormal{A}}}
\def\ermB{{\textnormal{B}}}
\def\ermC{{\textnormal{C}}}
\def\ermD{{\textnormal{D}}}
\def\ermE{{\textnormal{E}}}
\def\ermF{{\textnormal{F}}}
\def\ermG{{\textnormal{G}}}
\def\ermH{{\textnormal{H}}}
\def\ermI{{\textnormal{I}}}
\def\ermJ{{\textnormal{J}}}
\def\ermK{{\textnormal{K}}}
\def\ermL{{\textnormal{L}}}
\def\ermM{{\textnormal{M}}}
\def\ermN{{\textnormal{N}}}
\def\ermO{{\textnormal{O}}}
\def\ermP{{\textnormal{P}}}
\def\ermQ{{\textnormal{Q}}}
\def\ermR{{\textnormal{R}}}
\def\ermS{{\textnormal{S}}}
\def\ermT{{\textnormal{T}}}
\def\ermU{{\textnormal{U}}}
\def\ermV{{\textnormal{V}}}
\def\ermW{{\textnormal{W}}}
\def\ermX{{\textnormal{X}}}
\def\ermY{{\textnormal{Y}}}
\def\ermZ{{\textnormal{Z}}}

% Vectors
\def\vzero{{\bm{0}}}
\def\vone{{\bm{1}}}
\def\vmu{{\bm{\mu}}}
\def\vtheta{{\bm{\theta}}}
\def\vphi{{\bm{\phi}}}
\def\va{{\bm{a}}}
\def\vb{{\bm{b}}}
\def\vc{{\bm{c}}}
\def\vd{{\bm{d}}}
\def\ve{{\bm{e}}}
\def\vf{{\bm{f}}}
\def\vg{{\bm{g}}}
\def\vh{{\bm{h}}}
\def\vi{{\bm{i}}}
\def\vj{{\bm{j}}}
\def\vk{{\bm{k}}}
\def\vl{{\bm{l}}}
\def\vm{{\bm{m}}}
\def\vn{{\bm{n}}}
\def\vo{{\bm{o}}}
\def\vp{{\bm{p}}}
\def\vq{{\bm{q}}}
\def\vr{{\bm{r}}}
\def\vs{{\bm{s}}}
\def\vt{{\bm{t}}}
\def\vu{{\bm{u}}}
\def\vv{{\bm{v}}}
\def\vw{{\bm{w}}}
\def\vx{{\bm{x}}}
\def\vy{{\bm{y}}}
\def\vz{{\bm{z}}}

% Elements of vectors
\def\evalpha{{\alpha}}
\def\evbeta{{\beta}}
\def\evepsilon{{\epsilon}}
\def\evlambda{{\lambda}}
\def\evomega{{\omega}}
\def\evmu{{\mu}}
\def\evpsi{{\psi}}
\def\evsigma{{\sigma}}
\def\evtheta{{\theta}}
\def\eva{{a}}
\def\evb{{b}}
\def\evc{{c}}
\def\evd{{d}}
\def\eve{{e}}
\def\evf{{f}}
\def\evg{{g}}
\def\evh{{h}}
\def\evi{{i}}
\def\evj{{j}}
\def\evk{{k}}
\def\evl{{l}}
\def\evm{{m}}
\def\evn{{n}}
\def\evo{{o}}
\def\evp{{p}}
\def\evq{{q}}
\def\evr{{r}}
\def\evs{{s}}
\def\evt{{t}}
\def\evu{{u}}
\def\evv{{v}}
\def\evw{{w}}
\def\evx{{x}}
\def\evy{{y}}
\def\evz{{z}}

% Matrix
\def\mA{{\bm{A}}}
\def\mB{{\bm{B}}}
\def\mC{{\bm{C}}}
\def\mD{{\bm{D}}}
\def\mE{{\bm{E}}}
\def\mF{{\bm{F}}}
\def\mG{{\bm{G}}}
\def\mH{{\bm{H}}}
\def\mI{{\bm{I}}}
\def\mJ{{\bm{J}}}
\def\mK{{\bm{K}}}
\def\mL{{\bm{L}}}
\def\mM{{\bm{M}}}
\def\mN{{\bm{N}}}
\def\mO{{\bm{O}}}
\def\mP{{\bm{P}}}
\def\mQ{{\bm{Q}}}
\def\mR{{\bm{R}}}
\def\mS{{\bm{S}}}
\def\mT{{\bm{T}}}
\def\mU{{\bm{U}}}
\def\mV{{\bm{V}}}
\def\mW{{\bm{W}}}
\def\mX{{\bm{X}}}
\def\mY{{\bm{Y}}}
\def\mZ{{\bm{Z}}}
\def\mBeta{{\bm{\beta}}}
\def\mPhi{{\bm{\Phi}}}
\def\mLambda{{\bm{\Lambda}}}
\def\mSigma{{\bm{\Sigma}}}

% Tensor
\DeclareMathAlphabet{\mathsfit}{\encodingdefault}{\sfdefault}{m}{sl}
\SetMathAlphabet{\mathsfit}{bold}{\encodingdefault}{\sfdefault}{bx}{n}
\newcommand{\tens}[1]{\bm{\mathsfit{#1}}}
\def\tA{{\tens{A}}}
\def\tB{{\tens{B}}}
\def\tC{{\tens{C}}}
\def\tD{{\tens{D}}}
\def\tE{{\tens{E}}}
\def\tF{{\tens{F}}}
\def\tG{{\tens{G}}}
\def\tH{{\tens{H}}}
\def\tI{{\tens{I}}}
\def\tJ{{\tens{J}}}
\def\tK{{\tens{K}}}
\def\tL{{\tens{L}}}
\def\tM{{\tens{M}}}
\def\tN{{\tens{N}}}
\def\tO{{\tens{O}}}
\def\tP{{\tens{P}}}
\def\tQ{{\tens{Q}}}
\def\tR{{\tens{R}}}
\def\tS{{\tens{S}}}
\def\tT{{\tens{T}}}
\def\tU{{\tens{U}}}
\def\tV{{\tens{V}}}
\def\tW{{\tens{W}}}
\def\tX{{\tens{X}}}
\def\tY{{\tens{Y}}}
\def\tZ{{\tens{Z}}}


% Graph
\def\gA{{\mathcal{A}}}
\def\gB{{\mathcal{B}}}
\def\gC{{\mathcal{C}}}
\def\gD{{\mathcal{D}}}
\def\gE{{\mathcal{E}}}
\def\gF{{\mathcal{F}}}
\def\gG{{\mathcal{G}}}
\def\gH{{\mathcal{H}}}
\def\gI{{\mathcal{I}}}
\def\gJ{{\mathcal{J}}}
\def\gK{{\mathcal{K}}}
\def\gL{{\mathcal{L}}}
\def\gM{{\mathcal{M}}}
\def\gN{{\mathcal{N}}}
\def\gO{{\mathcal{O}}}
\def\gP{{\mathcal{P}}}
\def\gQ{{\mathcal{Q}}}
\def\gR{{\mathcal{R}}}
\def\gS{{\mathcal{S}}}
\def\gT{{\mathcal{T}}}
\def\gU{{\mathcal{U}}}
\def\gV{{\mathcal{V}}}
\def\gW{{\mathcal{W}}}
\def\gX{{\mathcal{X}}}
\def\gY{{\mathcal{Y}}}
\def\gZ{{\mathcal{Z}}}

% Sets
\def\sA{{\mathbb{A}}}
\def\sB{{\mathbb{B}}}
\def\sC{{\mathbb{C}}}
\def\sD{{\mathbb{D}}}
% Don't use a set called E, because this would be the same as our symbol
% for expectation.
\def\sF{{\mathbb{F}}}
\def\sG{{\mathbb{G}}}
\def\sH{{\mathbb{H}}}
\def\sI{{\mathbb{I}}}
\def\sJ{{\mathbb{J}}}
\def\sK{{\mathbb{K}}}
\def\sL{{\mathbb{L}}}
\def\sM{{\mathbb{M}}}
\def\sN{{\mathbb{N}}}
\def\sO{{\mathbb{O}}}
\def\sP{{\mathbb{P}}}
\def\sQ{{\mathbb{Q}}}
\def\sR{{\mathbb{R}}}
\def\sS{{\mathbb{S}}}
\def\sT{{\mathbb{T}}}
\def\sU{{\mathbb{U}}}
\def\sV{{\mathbb{V}}}
\def\sW{{\mathbb{W}}}
\def\sX{{\mathbb{X}}}
\def\sY{{\mathbb{Y}}}
\def\sZ{{\mathbb{Z}}}

% Entries of a matrix
\def\emLambda{{\Lambda}}
\def\emA{{A}}
\def\emB{{B}}
\def\emC{{C}}
\def\emD{{D}}
\def\emE{{E}}
\def\emF{{F}}
\def\emG{{G}}
\def\emH{{H}}
\def\emI{{I}}
\def\emJ{{J}}
\def\emK{{K}}
\def\emL{{L}}
\def\emM{{M}}
\def\emN{{N}}
\def\emO{{O}}
\def\emP{{P}}
\def\emQ{{Q}}
\def\emR{{R}}
\def\emS{{S}}
\def\emT{{T}}
\def\emU{{U}}
\def\emV{{V}}
\def\emW{{W}}
\def\emX{{X}}
\def\emY{{Y}}
\def\emZ{{Z}}
\def\emSigma{{\Sigma}}

% entries of a tensor
% Same font as tensor, without \bm wrapper
\newcommand{\etens}[1]{\mathsfit{#1}}
\def\etLambda{{\etens{\Lambda}}}
\def\etA{{\etens{A}}}
\def\etB{{\etens{B}}}
\def\etC{{\etens{C}}}
\def\etD{{\etens{D}}}
\def\etE{{\etens{E}}}
\def\etF{{\etens{F}}}
\def\etG{{\etens{G}}}
\def\etH{{\etens{H}}}
\def\etI{{\etens{I}}}
\def\etJ{{\etens{J}}}
\def\etK{{\etens{K}}}
\def\etL{{\etens{L}}}
\def\etM{{\etens{M}}}
\def\etN{{\etens{N}}}
\def\etO{{\etens{O}}}
\def\etP{{\etens{P}}}
\def\etQ{{\etens{Q}}}
\def\etR{{\etens{R}}}
\def\etS{{\etens{S}}}
\def\etT{{\etens{T}}}
\def\etU{{\etens{U}}}
\def\etV{{\etens{V}}}
\def\etW{{\etens{W}}}
\def\etX{{\etens{X}}}
\def\etY{{\etens{Y}}}
\def\etZ{{\etens{Z}}}

% The true underlying data generating distribution
\newcommand{\pdata}{p_{\rm{data}}}
\newcommand{\ptarget}{p_{\rm{target}}}
\newcommand{\pprior}{p_{\rm{prior}}}
\newcommand{\pbase}{p_{\rm{base}}}
\newcommand{\pref}{p_{\rm{ref}}}

% The empirical distribution defined by the training set
\newcommand{\ptrain}{\hat{p}_{\rm{data}}}
\newcommand{\Ptrain}{\hat{P}_{\rm{data}}}
% The model distribution
\newcommand{\pmodel}{p_{\rm{model}}}
\newcommand{\Pmodel}{P_{\rm{model}}}
\newcommand{\ptildemodel}{\tilde{p}_{\rm{model}}}
% Stochastic autoencoder distributions
\newcommand{\pencode}{p_{\rm{encoder}}}
\newcommand{\pdecode}{p_{\rm{decoder}}}
\newcommand{\precons}{p_{\rm{reconstruct}}}

\newcommand{\laplace}{\mathrm{Laplace}} % Laplace distribution

\newcommand{\E}{\mathbb{E}}
\newcommand{\Ls}{\mathcal{L}}
\newcommand{\R}{\mathbb{R}}
\newcommand{\emp}{\tilde{p}}
\newcommand{\lr}{\alpha}
\newcommand{\reg}{\lambda}
\newcommand{\rect}{\mathrm{rectifier}}
\newcommand{\softmax}{\mathrm{softmax}}
\newcommand{\sigmoid}{\sigma}
\newcommand{\softplus}{\zeta}
\newcommand{\KL}{D_{\mathrm{KL}}}
\newcommand{\Var}{\mathrm{Var}}
\newcommand{\standarderror}{\mathrm{SE}}
\newcommand{\Cov}{\mathrm{Cov}}
% Wolfram Mathworld says $L^2$ is for function spaces and $\ell^2$ is for vectors
% But then they seem to use $L^2$ for vectors throughout the site, and so does
% wikipedia.
\newcommand{\normlzero}{L^0}
\newcommand{\normlone}{L^1}
\newcommand{\normltwo}{L^2}
\newcommand{\normlp}{L^p}
\newcommand{\normmax}{L^\infty}

\newcommand{\parents}{Pa} % See usage in notation.tex. Chosen to match Daphne's book.

\DeclareMathOperator*{\argmax}{arg\,max}
\DeclareMathOperator*{\argmin}{arg\,min}

\DeclareMathOperator{\sign}{sign}
\DeclareMathOperator{\Tr}{Tr}
\let\ab\allowbreak


\usepackage{url}
\usepackage{caption}
\usepackage{amssymb}
\usepackage{graphicx}
\usepackage{hyperref}
%\usepackage[dvipdfmx]{graphicx}
%\usepackage[dvipdfmx]{hyperref}
\hypersetup{hidelinks} % リンクの枠と色を完全に非表示
% \usepackage{pxjahyper}

\newtheorem{theorem}{Theorem}

% %%%%%%%擬似コード用%%%%%%%%%%%%%%%%%
\usepackage{algorithm}
\usepackage{algcompatible}
\renewcommand{\algorithmicrequire}{\textbf{Input:}}
\renewcommand{\algorithmicensure}{\textbf{Output:}}
\newcommand{\LLL}{\mbox{1}\hspace{-0.25em}\mbox{l}}


% 作成コマンド
%%%%%%%%%%%%%%%%%%%%%%%%%%%%%%%%%%%%%
\newcommand\halfopen[2]{\ensuremath{[#1,#2)}}
%%%%%%%%%%%%%%%%%%%%%%%%%%%%%%%%%%%%%


\title{Time Series Anomaly Detection in the Frequency Domain \\with Statistical Reliability}

% Authors must not appear in the submitted version. They should be hidden
% as long as the tmlr package is used without the [accepted] or [preprint] options.
% Non-anonymous submissions will be rejected without review.

% \author{\name Akifumi Yamada \\ %\email yamada.akifumi.m7@s.mail.nagoya-u.ac.jp
%       \addr Nagoya University
%       \AND
%       \name Tomohiro Shiraishi \\
%       \addr Nagoya University
%       \AND
%       \name Shuichi Nishino \\
%       \addr Nagoya University
%       \AND
%       \name Teruyuki Katsuoka \\
%       \addr Nagoya University
%       \AND
%       \name Kouichi Taji \\
%       \addr Nagoya University
%       \AND
%       \name Ichiro Takeuchi 
%       \email takeuchi.ichiro.n6@f.mail.nagoya-u.ac.jp \\
%       \addr Nagoya University \& RIKEN 
% }

\author{\name Akifumi Yamada, Tomohiro Shiraishi, Shuichi Nishino, Teruyuki Katsuoka, Kouichi Taji \\ %\email yamada.akifumi.m7@s.mail.nagoya-u.ac.jp
      \addr Nagoya University
      \AND
      \name Ichiro Takeuchi 
      \email takeuchi.ichiro.n6@f.mail.nagoya-u.ac.jp \\
      \addr Nagoya University \& RIKEN 
}

% The \author macro works with any number of authors. Use \AND 
% to separate the names and addresses of multiple authors.

\newcommand{\fix}{\marginpar{FIX}}
\newcommand{\new}{\marginpar{NEW}}

\def\month{11}  % Insert correct month for camera-ready version
\def\year{2024} % Insert correct year for camera-ready version
\def\openreview{\url{https://openreview.net/forum?id=XXXX}} % Insert correct link to OpenReview for camera-ready version


\begin{document}

\maketitle

\begin{abstract}

Effective anomaly detection in complex systems requires identifying change points (CPs) in the frequency domain, as abnormalities often arise across multiple frequencies.
%
This paper extends recent advancements in statistically significant CP detection, based on Selective Inference (SI), to the frequency domain.
%
The proposed SI method quantifies the statistical significance of detected CPs in the frequency domain using $p$-values, ensuring that the detected changes reflect genuine structural shifts in the target system.
%
We address two major technical challenges to achieve this.
%
First, we extend the existing SI framework to the frequency domain by appropriately utilizing the properties of discrete Fourier transform (DFT).
 %by appropriately incorporating frequency transformations into selection events.}
%
Second, we develop an SI method that provides valid $p$-values for CPs where changes occur across multiple frequencies.
%
Experimental results demonstrate that the proposed method reliably identifies genuine CPs with strong statistical guarantees, enabling more accurate root-cause analysis in the frequency domain of complex systems.
      
\end{abstract}

\section{Introduction}
\label{sec:intro}
%
In this study, we consider semi-supervised anomaly detection (AD) using the $k$-nearest neighbor (NN) approach~\citep{breunig2000lof,ramaswamy2000efficient,mehrotra2017anomaly}.
%
Semi-supervised AD detects anomalies using only normal training instances.
%
In many practical cases, such as industrial AD, anomalous instances are rare, making semi-supervised AD essential.
%
We focus on the $k$-nearest neighbor anomaly detection ($k$NNAD) among various semi-supervised AD methods.
%
The $k$NNAD approach is simple yet effective, offering flexibility, minimal data assumptions, and adaptability to different distance metrics.

An important challenge in semi-supervised AD is quantifying the reliability of detected anomalies~\citep{barnett1994outliers,chandola2009anomaly,montgomery2020introduction}.
%
Without anomalous training instances in training, estimating detection accuracy is challenging.
%
Furthermore, modeling anomaly distributions is difficult since similar anomalies may not occur repeatedly.
%
To address this issue, we formulate semi-supervised $k$NNAD as a statistical test to quantify false AD probability using $p$-values.
%
If the $p$-values are accurately calculated and, anomalies with $p$-values below a desired significance level (e.g., 5\%) can be detected, ensuring that the detected anomalies are statistically significantly different from normal instances in the specified significance level.

However, a critical challenge emerges when formulating semi-supervised AD as a statistical test.
%
The primary issue is conducting both detection and testing of anomalies on the same data, which makes accurate $p$-value calculation intractable. 
%
In traditional statistics, selecting and evaluating a hypothesis on the same data causes selection bias in $p$-values, leading to inaccuracies—a problem known as \emph{double dipping}~\citep{breiman1992little,kriegeskorte2009circular,benjamini2020selective}.
%
In semi-supervised AD, since only normal instances are available, both the detection and evaluation must rely on the same data.
%
Thus, a naive statistical test formulation cannot avoid the double dipping issue.

To address this issue, we employ \emph{Selective Inference (SI)}, a statistical framework gaining attention in the past decade~\citep{fithian2014optimal,taylor2015statistical,lee2016exact}.
%
SI ensures valid statistical inferences after data-driven selection of hypotheses by correcting selection biases, ensuring accurate \( p \)-values and confidence intervals.
%
SI was originally designed to assess feature selection reliability, enabling accurate significance evaluation even when selection and evaluation use the same dataset~\citep{lee2016exact,tibshirani2016exact,duy2022more}.
%
The key principle of SI is to perform statistical inference conditioned on the selected hypothesis.
%
By using conditional probability distributions, SI effectively mitigates selection bias from double dipping.
%
In this study, we propose \emph{Statistically Significant $k$NNAD (Stat-$k$NNAD)}, a method that performs statistical hypothesis test conditioned on anomalies detected by the $k$NNAD algorithm.
%
The Stat-$k$NNAD offers theoretical guarantees and precise quantification of false anomaly detection probability.

Our contributions are summarized as follows.
%
First, we formulate semi-supervised AD using $k$NNAD as a statistical test within the SI framework, enabling accurate reliability quantification of detected anomalies.
%
While $k$NNAD is widely used, no existing method theoretically and accurately quantifies the false identification probability of detected anomalies.
%
Second, to enable conditional inference for $k$NNAD, we decompose it into tractable selection events (linear or quadratic inequalities) within the SI framework, Notably, applying $k$NNAD in deep learning-based latent spaces requires representing complex deep learning operations in a tractable form, posing a significant technical challenge (details in \S~\ref{sec:SI}). 
%
Finally, through experiments on various datasets and industrial product AD, we validate the effectiveness of Stat-$k$NNAD. Specifically, for industrial product images, we show that applying $k$NNAD in the latent space of a pretrained CNN effectively addresses practical challenges.
%
A more comprehensive discussion on related work, as well as the scope and limitations of this work, is presented in \S\ref{sec:relatedWorks}.

\subsection*{Large-scale multi-center cervical data collection and annotation}\label{subsec2-1}

\begin{figure*}[htbp]
    \centering
    \includegraphics[width=\linewidth]{fig/F2_v8.pdf}%
    \caption{\textbf{Overview of CCS-127K data and annotations in this study.} \textbf{a}. Class-wise distribution of 127,471 WSIs collected from 48 medical centers, including 124118 samples from 45 retrospective centers and 3,353 samples from 3 prospective centers. Note: RCM represents the centers merged due to limited sample size for each center. \textbf{b}. Abnormal cell annotation statistics: 104,979 abnormal lesion cells were annotated into 6 categories, ASC-US, LSIL, ASC-H, HSIL, SCC, and AGC, termed as CCS-Cell dataset. \textbf{c}. Designed flowchart of the study for the development and validation of the proposed Smart-CCS system. The orange flow represents retrospective studies, while the blue flow represents prospective studies.}
    \label{F2_data}
\end{figure*}
% \clearpage
To develop and validate the Smart-CCS system for generalizable cancer screening, we curated the large-scale and multi-center dataset, named CCS-127K. As illustrated in Fig. \ref{F2_data}(a), we collected a total of 127,471 cytology WSIs from 48 centers. These WSIs were obtained from 128,423 specimens after slide preparation, scanning, and quality control. Details of these processes are elaborated in the Methods section. According to the TBS guidelines \cite{nayar2015bethesda}, cytologists provided slide-level and cell-level annotations. The slide-level labels contained seven major cytology grades including negative for intraepithelial lesion or malignancy (NILM), atypical squamous cells of undetermined significance (ASC-US), low–grade squamous intraepithelial lesion (LSIL), atypical squamous cells cannot exclude an HSIL (ASC-H), high–grade squamous intraepithelial lesions (HSIL), squamous cell carcinoma (SCC), and atypical glandular cells (AGC). In total, cytologists annotated 69,940 retrospective WSIs from 45 centers and 3,353 prospective WSIs from 3 centers.
For cell-level annotations, cytologists delineated regions of interest (RoIs) and annotated abnormal cells using bounding boxes. As shown in Fig. \ref{F2_data}(b), they identified a total of 104,979 abnormal cells from 13 retrospective centers, spanning six abnormal cell types: ASC-US, LSIL, ASC-H, HSIL, SCC, and AGC, named as CCS-Cell dataset. More abbreviations are detailed in Extended Data Table \ref{ST_abb}.

Building on this large and diverse dataset, we designed a structured workflow, illustrated in Fig. \ref{F2_data}(c), that incorporates both retrospective and prospective cohorts. The retrospective cohort was used for model development and evaluation, and it was further divided into three subsets: pretraining, finetuning, and testing. The pretraining cohort, consisting of 112,062 slides from 39 centers, was used for self-supervised learning to initialize the CCS model. This was followed by the finetuning cohort, which included 49,063 labeled slides from 27 centers, to train the model for WSI classification task. For testing, the model was evaluated on a testing cohort divided into two datasets: an internal test dataset with 11,722 slides from 11 centers and an external test dataset with 9,155 samples from 6 independent centers.  Additionally, a prospective cohort of 3,353 slides from three centers was used to evaluate the clinical effectiveness in real-world scenarios. Finally, we collected the 738 corresponding histology diagnoses, which serve as the ground truth to evaluate the cancer diagnosis capability of the Smart-CCS system.

\subsection*{Overview of proposed Smart-CCS system}\label{subsec2-2}
As shown in Extended Data Fig. \ref{SF_method}, we developed a thorough AI-assisted CCS paradigm called Smart-CCS, consisting of three sequential stages: 1) large-scale self-supervised pretraining, 2) CCS model finetuning, and 3) test-time adaptation.

The framework of Smart-CCS is illustrated in Fig. \ref{F3_method}. In the pretraining stage, the curated cytology dataset CCS-127K was fully leveraged and exploited in a self-supervised learning manner \cite{oquabdinov2}. Thus, our Smart-CCS could effectively capture and represent inherent and generalizable cytological knowledge such as cellular instance features (cytoplasm, nucleus), semantic features (morphology), and global information (distribution).
In the finetuning stage, the pretrained models were further specialized to CCS tasks under both slide-level and cell-level supervision. Specifically, we followed the two-step CCS scheme involving two models: an abnormal cell detector and a WSI classifier. The detector identified suspicious cells across the entire slide, while the classifier aggregated these cell candidates to generate the final slide classification results. During the adaptation stage, the WSI classification model was optimized to handle unseen samples before making predictions. This approach could enhance the adaptability and generalizability of the Smart-CCS system, enabling it to perform effectively under diverse and complex screening conditions.

%%%%%%%%%%%%%%%%%%%%%%%%%%%%%%%%%%%%%%%%%%%%%
\begin{figure*}[t]
    \centering
    \includegraphics[width=1.0\linewidth]{fig/F3_v8.pdf}
    \caption{\textbf{Overview of the Smart-CCS Paradigm.} The Smart-CCS paradigm consists of three sequential stages. \textbf{a.} the \textbf{pretraining stage}, which involves large-scale self-supervised pretraining on diverse cytology images from various centers to build a generalizable feature extraction model. \textbf{b.} the \textbf{finetuning stage}, which specializes the pretrained model for cancer screening tasks, including two components: an abnormal cell detector for identifying abnormal cells and a WSI classifier for slide-level  predictions. \textbf{c.} the \textbf{adaptation stage}, which further optimizes trained model for diverse clinical settings via adapting and refining predictions.
    }
    \label{F3_method}
\end{figure*}

\subsection*{Self-supervised pretraining for cervical cancer screening}\label{subsec2-3}
Large-scale self-supervised pretraining empowers the model with strong generalization capability by yielding robust and off-the-shelf representation in computational cytology. 
In this study, we first investigated the effectiveness of self-supervised pretraining for two crucial tasks in CCS, namely cell-level and WSI-level classification. 

In cell-level tasks, we utilized two public datasets, HErlev \cite{jantzen2005pap} and SIPaKMeD \cite{plissiti2018sipakmed}, along with our collected CCS-Cell dataset to evaluate cell classification performance. We first evaluated scaling law of self-supervised pretraining, the results are illustrated in Fig. \ref{FE1}(a). Overall, the reported cell classification results indicate that pretraining can benefit downstream tasks across all three datasets. Specifically, pretraining using 1 million (M) cytology images yielded top-1 accuracy gains of 4.5\% on HErlev, 1\% on SIPaKMeD, and 3.5\% on CCS-Cell. As scaling pretraining data, classification performance improved steadily and continuously, with increases of 7.2\%, 3.4\%, and 5.6\%, respectively. Ultimately, three datasets reached top-1 accuracies of 0.914 (95\% CI: 0.873–0.955), 0.960 (95\% CI: 0.946–0.974), and 0.883 (95\% CI: 0.868–0.898) when pretrained using 100M cytology patches. 
 To further illustrate the effectiveness, we used t-SNE to visualize cell features with and without pretraining in Extended Data Fig. \ref{SF_tsne}. These visualizations reveal the enhanced feature aggregation capability after pretraining, where cell features are tightly clustered within each category and well-separated from neighboring categories. This aggregation capability potentially addresses category ambiguity issues in classifying cytology grades \cite{jiang2024holistic}.

In WSI-level tasks, we included three retrospective centers consisting of 25,571 WSIs from CCS-127K to investigate the impact of pretraining and investigate the scaling law. We gradually scaled up the pretraining data from 0 to 100M. Shown in Fig. \ref{FE1}(b), the results demonstrated the significant efficacy of pretraining, evidenced by its application in cancer screening through the detection of epithelial cell abnormalities (denoted as ECA), as well as in fine-grained cytology WSI classification tasks (denoted as ALL). As the pretraining data increased, the experimental results steadily improved, showing an overall accuracy increase of 10.34\% in cancer screening and 8.61\% in fine-grained cytology classification, reaching up to 100M data. Ultimately, the three centers achieved 0.950 (95\%CI: 0.942 - 0.959), 0.915 (95\%CI: 0.902-0.929) and 0.958 (95\%CI: 0.946-0.970) accuracies for cancer screening. Details regarding cell and WSI classification are provided in Extended Data Tables. \ref{ST_pretrain_cell}-\ref{ST_pretrain_wsi}.

 Additionally, ablation experiments were conducted on pretraining backbones (ViT-Large, ViT-Gaint) and algorithms (DINOv2 \cite{oquabdinov2}, MoCov3 \cite{chen2021empirical}). Based on the results, we employed the ViT-Large architecture pretrained with DINOv2 using 100 million pretraining data in the following experiments. More experimental results are provided in Extended Data Table. \ref{ST_pretrain_ablation}. 

\subsection*{Retrospective evaluation of abnormal cell detection}\label{subsec2-4}
\begin{figure*}[htbp]
    \centering
    \includegraphics[width=\linewidth]{fig/FE1_v4.pdf}
    \caption{\textbf{Performance of Smart-CCS in retrospective study.} \textbf{a.} Evaluation of cell-level cytology task using cell classification datasets,  SIPaKMeD ($N$ = 4,049), HErlev ($N$ = 918), and CCS-Cell ($N$ = 9,008). \textbf{b.} Evaluation of the WSI-level cytology task using retrospective cervical cytology datasets ($N$ = 5,189, 11,986, 8,396) to assess cancer screening (ECA) and fine-grained classification (ALL) performances. \textbf{c.} Comparison of abnormal cell detection performance among DDETR, DETR, RetinaNet, Faster R-CNN and YOLOv3 on CCS-Cell dataset. \textbf{d.} The external testing performances are evaluated by metric AUC with different settings, Base denotes the typical two-step CCS model, w/ P is introducing pretraining, w/ P\&A refers to our proposed Smart-CCS with pretraining and adaptation. \textbf{e.} Internal and external data distribution, along with the results of cervical cancer screening evaluations.}
    \label{FE1}
\end{figure*}
Abnormal cell detection typically serves as the prerequisite for AI-assisted CCS, where abnormal cells with high confidence scores are identified from the whole slide as candidates and then aggregated for slide-level classification. Thus, we aim to build a strong cell detector to screen out all abnormal and suspicious cells within each WSI. 
As shown in Fig. \ref{FE1}(c), we compared several state-of-the-art (SOTA) detectors used in previous CCS studies, including Faster R-CNN \cite{ren2015faster}, YOLOv3 \cite{zhu2021hybrid,wei2021efficient}, RetinaNet \cite{wang2024artificial}, transformer-based detector DETR \cite{carion2020end}, and its deformable variant DDETR \cite{zhu2020deformable}. 
The experimental results showed that DDETR achieved the best performance and was selected as the detector in Smart-CCS. DDETR surpassed YOLOv3, Faster R-CNN, and RetinaNet by 14.3\%, 5.9\%, and 7.3\% in AP50 (Average Precision with an IoU threshold of 0.50), respectively, benefiting from the powerful visual representation capability of the transformer and the adaptability of deformable attention to multi-resolution \cite{zhu2020deformable}.
The per-class detection results showed that DDETR not only addressed the issue of missed detections for tiny objects, such as SCC (naked nuclei and small nucleoli), achieving a 10.3\% AP50 increase compared to YOLOv3, but also enhanced the detection of morphologically diverse atypical cells. Specifically, it achieved a +13.1\% AP50 improvement in ASC-US and a +8.5\% AP50 improvement in ASC-H (Extended Data Table \ref{ST_det}-\ref{ST_det_p2}). Subsequently, the cell classification performance was evaluated using a confusion matrix, shown in Extended Data Fig. \ref{SF_det}. From this matrix, we observed a clear distinction in the model's ability to identify glandular cells (class AGC) compared to other squamous cell abnormalities. This is attributed to the distinct morphological abnormalities and unique arrangement characteristics of glandular cells. We also found a strong correlation between ASC-US and LSIL, as well as between ASC-H and HSIL, which aligns with clinical practice \cite{nayar2015bethesda,vandenbussche2022cytologic}. Overall, our detector outperforms previous work in differentiating fine-grained cell categories, which serves as the first step in WSI classification model for cancer screening.

\subsection*{Internal testing for multi-center evaluation}\label{subsec2-5}
In the retrospective study, a total of 112,062 cervical cytology samples from pretraining and finetuning cohorts were included for the Smart-CCS system development. Then, internal testing included 11,722 samples from 11 centers for multi-center evaluation of the developed Smart-CCS (Fig. \ref{FE1}(e)). The positive rates among these centers range from 5.48\% 
 to 49.26\% with different grades of epithelial abnormalities.

The quantitative performances in testing centers are presented in Fig. \ref{FE1}(e). 
Overall, Smart-CCS achieved an overall of 0.965 (95\% CI: 0.961–0.969) AUC, 0.965 (95\% CI: 0.961–0.969) accuracy, 0.913 (95\% CI: 0.907–0.919) sensitivity, and  0.896 (95\% CI: 0.889–0.902) specificity on internal testing centers. Specifically, 9 centers achieved AUC metrics greater than  95\% among the 11 internal testing centers, such as 0.971 (95\% CI 0.965–0.978) in RC2, 0.990 (95\% CI: 0.985–0.995) in RC3, 0.985 (95\% CI: 0.977–0.992) in RC5, and 0.971 (95\% CI: 0.960–0.982) in RC7. We also found that centers with a low positive rate, such as RC1, often showed lower classification accuracies. Besides, the system screening capabilities are highly determined by accurately detecting early-stage intraepithelial abnormalities from a large number of samples, as indicated by the metric, sensitivity. Smart-CCS achieved a high overall sensitivity of 0.913 (95\% CI: 0.907–0.919), with 9 internal centers demonstrating sensitivities greater than 85\%, indicating strong screening capabilities.

In terms of different cytology grades, ASC-US+, considered as the squamous cell abnormality, Smart-CCS exhibited high classification capability with an overall AUC of 0.961 (95\% CI: 0.957–0.965) and a sensitivity of 0.910 (95\% CI: 0.904–0.916). Then, the critical cytology group LSIL+, to be considered for further colposcopy, maintained advanced screening performance of 0.958 (95\% CI: 0.953–0.962) AUC with 0.910 (95\% CI: 0.903–0.916) sensitivity. Especially in RC1, the challenging center with a quite low positive rate (5.48\%), Smart-CCS increased from 0.767 (95\% CI: 0.750–0.784) in ECA to 0.902 (95\% CI: 0.890–0.914) in LSIL+, demonstrating the effect of ASC-US, the ambiguous category \cite{jiang2024holistic, nayar2015bethesda}. The HSIL+ group typically refers to malignant neoplasia, with superior classification results ranging from 0.918 (95\% CI: 0.904–0.931) to 0.994 (95\% CI: 0.987–1.001) in AUC within internal testing centers. Furthermore, we also reported the F1 score metric to assess the model stability under class imbalance. Smart-CCS achieved favorable F1 scores of 0.903 (95\% CI: 0.896–0.909), 0.897 (95\% CI: 0.890–0.903), 0.891 (95\% CI: 0.884–0.897), and 0.950 (95\% CI: 0.945–0.955) in subgroups ECA, ASC-US+, LSIL+, and HSIL+. More result details are appended in Extended Data Tables. \ref{ST_wsi}-\ref{ST_wsi_3}.

\subsection*{External testing for generalizability evaluation}\label{subsec2-5}
The cytology samples collected from different centers present diverse and large variations, which brings challenges for system generalizability. Under the proposed Smart-CCS paradigm, the pretrained model could extract general and strong cytology features that are domain-invariant across diverse clinical centers.  To evaluate the effectiveness and generalizability of developed Smart-CCS, we retrospectively included 9,155 cytology WSIs from six independent centers (RC8-RC13).

In the external testing, the positive rates among six centers ranged from 8.81\% (RC8) to 52.65\% (RC12), resulting in an overall rate of 23.89\%. The results are shown in Fig. \ref{FE1}(e), where we obtained an overall screening performance of 0.912 (95\% CI: 0.906–0.918) accuracy, 0.950 (95\% CI: 0.945–0.954) AUC, and 0.854 (95\% CI: 0.847–0.861) sensitivity. Specifically, Smart-CCS achieved consistent and favorable classification results ranging from 0.880 (95\% CI: 0.859–0.901) to 0.963 (95\% CI: 0.954–0.972) AUC across six external centers.
The external testing performance reveals the strong generalization capabilities of Smart-CCS.

To further investigate the efficacy of the proposed pretraining and adaptation strategies in Smart-CCS, we conducted ablation studies comparing three experimental groups: a two-step CCS model (denoted as Base), CCS with pretraining (denoted as w/ P), and our Smart-CCS with both pretraining and adaptation (denoted as w/ P\&A). As illustrated in Fig. \ref{FE1}(d), we report the area under the receiver operating characteristic curve (AUC) for cancer screening across different centers, as well as the overall performance. Compared to the CCS baseline model, pretraining consistently increased the AUC, from 0.897 (95\% CI: 0.877–0.918) to 0.930 (95\% CI: 0.913–0.948) in RC9, and from 0.830 (95\% CI: 0.805–0.855) to 0.911 (95\% CI: 0.892–0.930) in RC10, among others. Furthermore, when both pretraining and adaptation were applied, Smart-CCS achieved a 6.3\% AUC improvement against the CCS baseline model. The demonstrated effectiveness of the Smart-CCS paradigm in external testing underscores its potential clinical applicability in complex scenarios. Additional details regarding the ablation studies can be found in Extended Data Table \ref{ST_ext}. Moreover, we compared classifiers, including MeanMIL, MaxMIL, ABMIL \cite{ilse2018attention}, DSMIL \cite{li2021dual}, CLAM \cite{shao2021transmil}, TransMIL \cite{lu2021data}, and S4MIL \cite{fillioux2023structured}, in terms of internal and external testing performance, as shown in Extended Data Table \ref{ST_clas}.

\subsection*{Prospective study for clinical validation}\label{subsec2-6}
Between January 1 2024 and July 31, 2024, we recruited a total of 3,353 participants from three prospective centers, PC1, PC2 and PC3 to form the prospective cohort. 
As shown in Fig. \ref{FE2}(a), three prospective centers individually contributed 998, 1,311, and 1,044 samples, with positive rates of 43.24\%, 26.32\%, and 28.35\%. Besides, we also obtained 185, 258, and 295 corresponding histological diagnosis results as the gold standard for further system evaluation.

%%%%%%%%%%%%%%%%%%%%%%%%%%%%%%%%%%%%%%%%%%%%%
\begin{figure*}[htbp]
    \centering
    \includegraphics[width=1.0\linewidth]{fig/FE2_v5.pdf}
    \caption{\textbf{Performance of Smart-CCS in the prospective study.} \textbf{a}. Grade distribution among three prospective centers (PC1, PC2 and PC3) and reported evaluation results from our Smart-CCS system. \textbf{b}. The diagnostic performance of Smart-CCS in cancer detection using histological results as the gold standard. \textbf{c}. Visualizations from cervical cancer screenings, which include interpretable results at both the cell-level and slide-level, derived from three sample tests.}
    \label{FE2}
\end{figure*}

Overall, the evaluation results of Smart-CCS for cancer screening demonstrated consistent generalization capabilities in prospective centers, as shown in Fig. \ref{FE2}(a) and Extended Data Table. \ref{ST_pros}. Specifically, Smart-CCS achieved accuracies of  0.877 (95\% CI: 0.856–0.897), 0.862 (95\% CI: 0.843–0.881), and 0.950 (95\% CI: 0.937–0.963) across three prospective centers, with high sensitivities of 0.893 (95\% CI: 0.874–0.912), 0.881 (95\% CI: 0.864–0.899), and 0.946 (95\% CI: 0.932–0.960). The high AUC scores ranging from 0.924 (95\% CI: 0.910–0.938) to 0.986 (95\% CI: 0.979–0.993) also reveal the robust and strong cancer screening capability of Smart-CCS system in clinical scenarios.

To further validate Smart-CCS for cervical cancer diagnosis, we utilized histological biopsy results as the gold standard \cite{ouh2021discrepancy}. Histology provides a more confirmative diagnosis through direct tissue examination compared to the individual cell analysis of cytology, making it a reliable benchmark for assessing the accuracy of the cytology-based Smart-CCS. The overall and different grades (i.e., CIN1, CIN2, CIN3) results are presented in Fig. \ref{FE2}(b). Overall, Smart-CCS obtained a sensitivity of 0.943 (95\% CI 0.927-0.960) and an AUC of 0.902 (95\% CI 0.880-0.923) in cervical cancer diagnosis. Specifically, the three prospective centers reported sensitivities of 0.967 (95\% CI: 0.942–0.993), 0.992 (95\% CI: 0.981–1.000), 0.949 (95\% CI: 0.924–0.974), with corresponding AUCs of 0.984 (95\% CI: 0.965–1.000), 0.921(95\% CI: 0.888–0.954), and N/A due to the lack of negative samples in PC3. For histology-positive samples, Smart-CCS achieved an overall of 0.957 (95\% CI: 0.942–0.973) sensitivity in three centers. The high sensitivities can be observed across all groups, ranging from 0.935 (95\% CI: 0.902–0.968) to 1.000 (95\% CI: 1.000–1.000). We also provided the results of different histology grades, CIN1, CIN2, and CIN3. We found there were no missed samples in CIN1 and CIN3 in two centers (PC1, PC3). The CIN2 group achieved 0.971 (95\% CI: 0.943–0.999), 0.937 (95\% CI: 0.894–0.979) and 0.935 (95\% CI: 0.902–0.968) sensitivities.
Notably, Smart-CCS yielded sensitivities close to 1.00 for HSIL+ across all three prospective centers, demonstrating a high consistency between cancer screening and diagnosis on higher cytology grades. 
Therefore, generalizable cancer screening could potentially avoid unnecessary biopsies and colposcopies for patients at risk of cervical cancer. 

\subsection*{Interpretability analysis}
To support clinical practice, we illustrate the decision-making process of Smart-CCS in Fig. \ref{FE2}(c), providing interpretable insights from diverse perspectives.
 At the slide-level, Smart-CCS predicts positive scores for precancerous abnormalities based on cervical cytology specimens. It provides detailed prediction categories and scores for malignancy grade evaluation, which inform subsequent colposcopy and biopsy procedures. For instance, in Fig. \ref{FE2}(c), the left WSI sample achieves a confidence score of 0.9999 for the NILM category, while the middle sample yields a positive score of 0.9999, with a high-grade HSIL confidence of 0.9915. This indicates a high probability of cervical neoplasia and malignancy.

In the context of cell-level screening, suspicious cells are highlighted using distinct markers in each WSI. The visualization illustrates that these suspicious cells are randomly distributed across the WSIs due to the uniform mixing and centrifugation during specimen preparation \cite{jiang2023deep}. Furthermore, detected suspicious cells, along with confidence score statistics and characteristic morphologies, provide reliable clues for the final diagnosis. In the middle sample illustrated in Fig. \ref{FE2}(c), Smart-CCS detected a few  LSIL cells (i.e., koilocytotic cells with typical perinuclear halos) alongside a significant number of HSIL cells, which demonstrated markedly enlarged nuclei, reduced cytoplasmic areas, and hyperchromatic clustering. This predicted information aligns well with the ground truth for this sample, which is HSIL. Similarly, the AGC sample reveals a substantial presence of AGC cells arranged in a fence-like pattern, alongside other atypical cells, as shown in Fig. \ref{FE2}(c)(right). In the case of NILM, despite a slide-level prediction score of nearly 1.00, some cells exhibit deepening and enlargement of nuclei, presenting as non-neoplastic cellular variations or hard mimics \cite{nayar2015bethesda}. By integrating these insights within an interactive interface, we have developed Smart-CCS into a comprehensive system, as depicted in Extended Data Fig. \ref{SF_integrate}. This system aims to provide cytologists with interpretable AI-assisted results, ensuring reliable screening outcomes and guiding subsequent final diagnoses, thereby enhancing the overall diagnostic workflow.

Cervical cancer has been targeted for global elimination under the WHO initiative \cite{world2020global}. Leveraging AI to assist cytologists can significantly accelerate cervical cancer screening, especially in large-scale precancerous screening conditions. In recent years, several AI-based computational cytology studies preliminarily explored its feasibility, particularly a recent study demonstrated the effectiveness of AI-assisted cytology in cancer identification and tumor origin prediction, highlighting the potential of AI in computational cytology \cite{tian2024prediction,rassy2024predicting, li2024new}. However, previous AI-assisted CCS systems often faced challenges in generalization and robustness across different clinical scenarios, particularly concerning variations in slide preparation and imaging protocols \cite{cheng2021robust,wang2024artificial,zhu2021hybrid,wu2024development}. 

In this study, we introduced Smart-CCS, a generalizable cervical cancer screening paradigm. This pioneering paradigm consists of three stages: self-supervised pretraining, screening model finetuning, and test-time adaptation. It leverages: 1) large-scale self-supervised pretraining for robust cytology representations, 2) effective utilization of both cell-level and slide-level supervision, and 3) model adaptation during clinical evaluation. To support this, we constructed a large, multi-center dataset, CCS-127K, which includes 127,471 cervical cytology WSIs from 48 centers. To the best of our knowledge, this is the first comprehensive paradigm for cervical cancer screening with multi-center and prospective validation.

The experimental results of internal testing, external testing, and prospective studies revealed the extensive effectiveness and potential clinical benefits of the proposed Smart-CCS for generalizing cancer screening. 
Notably, the significant improvements in cell-level and WSI-level downstream tasks demonstrated the efficacy of pretraining in capturing general cytology information. For retrospective evaluation of abnormal cell detection, we compared SOTA detection models and selected the best as the first step in WSI classification model. In terms of internal testing, Smart-CCS achieved an overall AUC of 0.965 (95\% CI: 0.961–0.969) across 11 centers for cancer screening. 
In external testing, Smart-CCS achieved an AUC from 0.880 (95\% CI: 0.859–0.901) to 0.941 (95\% CI: 0.927–0.955)  across 6 external centers with high sensitivities, validating its applicability in diverse clinical scenarios. Additionally, prospective studies in PC1-PC3 yielded high AUC scores of 0.947 (95\% CI: 0.933–0.961), 0.924 (95\% CI: 0.910–0.938), and 0.986 (95\% CI: 0.979–0.993), respectively. Further histological evaluation and interpretability analysis yielded consistent conclusions with clinical knowledge, which highlights the reliability and superiority of the proposed Smart-CCS paradigm.


This study has several limitations. 
First, the availability of large-scale data is critical for pretraining robust and generalizable models. Although we have constructed one of the largest cytology datasets to date, comprising 127,471 WSIs, it remains relatively small compared to datasets commonly used in the histology domain. For instance, Virchow2 was pretrained on 3.1M WSIs \cite{vorontsov2024foundation}, and CONCH utilized 1.17M image-caption pairs \cite{lu2024visual}, leveraging extensive public histology datasets such as TCGA \cite{weinstein2013cancer}. Similarly, recent advancements in cytology have also focused on scaling up data resources to enhance model development \cite{yu2023ai}. To address this limitation, we are actively working to incorporate larger datasets for both pretraining and validation, with the goal of advancing cancer screening and computational cytology.
Second, from a methodological perspective, the ability to distinguish individual cell instances is critical for guiding accurate screening decisions. While we have developed a detection-based WSI classification framework, integrating fine-grained cellular information could further improve the modeling of cell-to-WSI relationships. For example, transitioning from patch-level pretraining to cell-level pretraining or incorporating quantified morphological features, such as cytoplasmic segmentation masks and nucleus-to-cytoplasm ratios, may enhance the performance and interpretability of the Smart-CCS system \cite{zhu2021hybrid}.
Third, regarding system validation, although this study included independent testing across 11 external centers and prospective evaluation in 3 centers, broader deployment and validation in diverse clinical settings are essential to facilitate widespread adoption. Future evaluations should consider a wider range of demographic variables, including ethnicity, age, geographic regions, and variations in slide preparation and imaging protocols, to ensure the system's generalizability and robustness across heterogeneous clinical scenarios.

Ultimately, to establish a trustworthy cancer screening system for large-scale clinical applications, several in-depth exploration directions follow this study. First, establishing a unified cytology WSI benchmark is essential to tackle unique challenges such as cell instance distinguishability and ambiguous categorization of atypical findings (e.g., ASC-US, ASC-H) \cite{jiang2024holistic}.  
Secondly, while Smart-CCS demonstrated the effectiveness for cervical cancer screening, the paradigm has significant potential for broader applications. Generalizing Smart-CCS to other cancers, such as urine \cite{wu2024development}, thyroid \cite{wang2024deep}, and pleural effusion samples \cite{tian2024prediction}, could contribute to a cytology foundation model, enhancing its applicability in computational cytology. Finally, while cytology primarily focuses on cancer screening, large-scale data-driven approaches could expand to more clinical applications, including cancer diagnosis \cite{wu2024development}, survival prognosis \cite{sigel2017cytology,valletti2021gastric}, biomarker prediction \cite{tian2024prediction,rassy2024predicting,li2024new} and molecular-level discovery \cite{caputo2024current, ohori2024molecular}.


In summary, the Smart-CCS paradigm demonstrates strong potential for advancing cervical cancer screening while paving the way for broader applications in computational cytology.


\section{Selective Inference on CP candidate Locations}
\label{sec:SI}
%
In this section, using the SI framework, we quantify the statistical significance of all locations selected as CP candidates by the algorithm $\mathcal{A}$ in Section~\ref{sec:CpSelection} in the form of $p$-values.
%
By setting the significance level $\alpha$ (e.g., 0.05 or 0.01) and considering CP candidate locations with the $p$-value $< \alpha$ as the final CPs, 
it is theoretically guaranteed that the false positive detection probabilities (type I error rates) of these final CPs is controlled below the specified significance level $\alpha$.

To formalize the SI framework, let us write the algorithm in Section~\ref{sec:CpSelection} as  
\begin{equation}
\mathcal{A}: \bm{X} \mapsto \bm{\mathcal{T}}, \notag
\end{equation}
and the CP candidates obtained by applying the algorithm to the actual observed time series data \(\bm{x}\) are expressed as  
\begin{equation}
\bm{\mathcal{T}}^{\rm det} = \left({\bm{\tau}^{\rm det}}^{(0)}, \ldots, {\bm{\tau}^{\rm det}}^{(D-1)} \right) = \mathcal{A}(\bm{x}), \label{eq:detected_obs}
\end{equation}
where ${\bm{\tau}^{\rm det}}^{(d)} = \{{\tau_1^{\rm det}}^{(d)}, \ldots, {\tau_{K^{(d)}}^{\rm det}}^{\hspace{-1.6mm}(d)}\}$ 
is the set of CP candidates for frequency $d \in \{0, \ldots, D-1\}$~\footnote{
Note that, to ensure deterministic behavior of the algorithm $\mathcal{A}$, 
the random seed is fixed to a constant value at the beginning of the procedure.}.
%
Furthermore,
let
$\bm{\tau}^{\rm det} = \{\tau_1^{\rm det}, \ldots, \tau_K^{\rm det}\}$
be the ordered set of CP candidate locations, 
$\mathcal{D}_{\tau_k^{\rm det}} \subseteq \{0, \dots, D - 1\}$
be the set of frequencies that have CP candidates at a CP location $\tau_k^{\rm det}$,
and
$k^{(d)} \in [K^{(d)}]$
be the correponding index of the CP candidates for
$d$
and 
$k \in [K]$,
i.e., ${\tau_{k^{(d)}}^{\text{det}}}^{\hspace{-0.7mm}(d)} = \tau_k^{\text{det}}$ holds for all $d \in \mathcal{D}_{\tau_k^{\text{det}}}$~\footnote{
  These notations are somewhat intricate. For example, in the case of Figure~\ref{fig:fig2}, 
  \begin{itemize}  
   \item $D=5$  
   \item $\bm{\mathcal{T}}^{\rm det} = \left(\{97\}, \{28\}, \{28, 70\}, \{70\}, \{\}\right)$  
   \item $\bm{\tau}^{\rm det} = \{28, 70, 97\}$  
   \item $K=3$  
   \item $\mathcal{D}_{\tau_1^{\rm det}} = \{d_1, d_2\}, \mathcal{D}_{\tau_2^{\rm det}} = \{d_2, d_3\}, \mathcal{D}_{\tau_3^{\rm det}} = \{d_0\}$  
   \item $k = 3, 1, 1, 2, 2$ for $(d, k^{(d)}) = (d_0, 1), (d_1, 1), (d_2, 1), (d_2, 2), (d_3, 1)$, respectively.
  \end{itemize}  
  }.

\subsection{Statistical Test on CP Locations}
%
\textbf{Hypotheses.}
For testing the statistical significance of CP location $\tau^{\rm det}_k$ for $k \in [K]$, we consider the following null hypothesis ${\rm H}_{0, k}$ and alternative hypothesis ${\rm H}_{1, k}$:
% 
\begin{align}
 {\rm H}_{0, k}: 
 \frac{1}{{\tau_{k^{(d)}}^{\text{det}}}^{\hspace{-0.7mm}(d)} - {\tau_{k^{(d)}-1}^{\rm det}}^{\hspace{-4.3mm}(d)}} \,
 \sum_{t={\tau_{k^{(d)}-1}^{\rm det}}^{\hspace{-4.3mm}(d)} \, +1}^{{\tau_{k^{(d)}}^{\text{det}}}^{\hspace{-1.2mm}(d)}}
 \mu_t^{(d)}
 &= 
 \frac{1}{{\tau_{k^{(d)}+1}^{\rm det}}^{\hspace{-4.3mm}(d)} - {\tau_{k^{(d)}}^{\text{det}}}^{\hspace{-0.7mm}(d)}}
 \sum_{t={\tau_{k^{(d)}}^{\text{det}}}^{\hspace{-1.2mm}(d)} +1}^{{\tau_{k^{(d)}+1}^{\rm det}}^{\hspace{-4.3mm}(d)}}
 \mu_t^{(d)}, ~~~
 \forall d \in \mathcal{D}_{\tau_k^{\rm det}}, \label{eq:H0}
 \\
 &\text{~v.s.} \notag \\
 {\rm H}_{1, k}:
 \frac{1}{{\tau_{k^{(d)}}^{\text{det}}}^{\hspace{-0.7mm}(d)} - {\tau_{k^{(d)}-1}^{\rm det}}^{\hspace{-4.3mm}(d)}} \,
 \sum_{t={\tau_{k^{(d)}-1}^{\rm det}}^{\hspace{-4.3mm}(d)} \, +1}^{{\tau_{k^{(d)}}^{\text{det}}}^{\hspace{-1.2mm}(d)}}
 \mu_t^{(d)}
 &\neq
 \frac{1}{{\tau_{k^{(d)}+1}^{\rm det}}^{\hspace{-4.3mm}(d)} - {\tau_{k^{(d)}}^{\text{det}}}^{\hspace{-0.7mm}(d)}}
 \sum_{t={\tau_{k^{(d)}}^{\text{det}}}^{\hspace{-1.2mm}(d)} +1}^{{\tau_{k^{(d)}+1}^{\rm det}}^{\hspace{-4.3mm}(d)}}
 \mu_t^{(d)}, ~~~
 \exists d \in \mathcal{D}_{\tau_k^{\rm det}}. \label{eq:H1}
\end{align}
%
The above null hypothesis ${\rm H}_{0, k}$ indicates that, at the location $\tau_k^{\rm det}$, the means of $\mu_t^{(d)}$ in the left segment and the right segment are equal for all frequencies $d \in \mathcal{D}_{\tau_k^{\rm det}}$.
%
On the other hand, the alternative hypothesis ${\rm H}_{1, k}$ implies that the means of the left and right segments are not equal in at least one of the frequencies.

% \textbf{Formulation of hypotheses.}
% We quantify the statistical significance of the detected change time points $\bm{\tau}^{\text{det}}$.
% For the $k$-th detected change time point $\tau_k^{\text{det}} \in \bm{\tau}^{\text{det}}$, 
% let $\mathcal{D}_{\tau_k^{\text{det}}}$ be the set of frequencies % in which $\tau_k^{\text{det}}$ appears. 
% in which $\tau_k^{\text{det}}$ is an element of ${\bm{\tau}^\text{det}}^{(d)}$, 
% and 
% \begin{align}
%   \mu_{\text{seg}}^{(d)}\left(\tau_{k^{(d)}-1}^{\text{det}} : \tau_{k^{(d)}}^{\text{det}}\right) &= \mu_{{\tau_{k^{(d)}-1}^{\text{det}}}+1}^{(d)} = \cdots = \mu_{\tau_{k^{(d)}}^{\text{det}}}^{(d)}, \notag  \\
%   \mu_{\text{seg}}^{(d)}\left(\tau_{k^{(d)}}^{\text{det}} : \tau_{k^{(d)}+1}^{\text{det}}\right) &= \mu_{{\tau_{k^{(d)}}^{\text{det}}}+1}^{(d)} = \cdots = \mu_{\tau_{{k^{(d)}+1}}^{\text{det}}}^{(d)}, \notag
% \end{align}
% where $\tau_{k^{(d)}}^{\text{det}} = \tau_k^{\text{det}}$ for all $d \in \mathcal{D}_{\tau_k^{\text{det}}}$, 
% be the mean of the each segment before and after the detected change time point. 
% Then, we consider the statistical test for the mean-shift problem, 
% where the null and alternative hypotheses are formulated as:
% \begin{align}
%   \text{H}_{0,k}: \forall d \in \mathcal{D}_{\tau_k^{\text{det}}}, \hspace{1mm} \mu_{\text{seg}}^{(d)}\left(\tau_{k^{(d)}-1}^{\text{det}} : \tau_{k^{(d)}}^{\text{det}}\right) &= \mu_{\text{seg}}^{(d)}\left(\tau_{k^{(d)}}^{\text{det}} : \tau_{k^{(d)}+1}^{\text{det}}\right) \notag \\
%   &\text{\hspace{1mm}vs.} \label{hypotheses} \\
%   \text{H}_{1,k}: \exists d \in \mathcal{D}_{\tau_k^{\text{det}}}, \hspace{1mm} \mu_{\text{seg}}^{(d)}\left(\tau_{k^{(d)}-1}^{\text{det}} : \tau_{k^{(d)}}^{\text{det}}\right) &\neq \mu_{\text{seg}}^{(d)}\left(\tau_{k^{(d)}}^{\text{det}} : \tau_{k^{(d)}+1}^{\text{det}}\right). \notag
% \end{align}
% Under the null hypothesis, it is assumed that the mean within each segment remains constant before and after the detected change time point.

\textbf{Test statistic.}
For testing with the null hypothesis~(\ref{eq:H0}) and alternative hypothesis~(\ref{eq:H1}), 
we consider the test statistic constructed by 
computing the differences between averages of the complex spectra in the two segments before and after the CP location $\tau_k^{\text{det}}$ for each frequency $d \in \mathcal{D}_{\tau_k^{\text{det}}}$, 
and then aggregating their squared absolute values. 
Formally, the test statistic is defined as follows:
\begin{align}
 \label{eq:teststatistic}
 T_k(\bm{X}) 
  = \sigma^{-1} \sqrt{\sum_{d \in \mathcal{D}_{\tau_k^{\text{det}}}} \hspace{-1mm} a_{k^{(d)}} \bigg{|} \bar{F}_{{{\tau_{k^{(d)}-1}^{\rm det}}^{\hspace{-4.3mm}(d)}+1:{\tau_{k^{(d)}}^{\text{det}}}^{\hspace{-1.2mm}(d)}}}^{(d)}
  - \bar{F}_{{\tau_{k^{(d)}}^{\text{det}}}^{\hspace{-1.2mm}(d)}+1:{\tau_{k^{(d)}+1}^{\rm det}}^{\hspace{-4.3mm}(d)}}^{(d)} \, \bigg{|}^2}, 
\end{align}
% \begin{equation}
%   \sigma^{-1} ||P_k \bm{X}|| = \sigma^{-1} \sqrt{\sum_{d \in \mathcal{D}_{\tau_k^{\text{det}}}} a^{(d)} \Bigg{|} {\frac{1}{\tau_{k^{(d)}}^{\text{det}} - \tau_{k^{(d)}-1}^{\text{det}}}} \sum_{t = \tau_{k^{(d)}-1}^{\text{det}}+1}^{\tau_{k^{(d)}}^{\text{det}}} \hspace{-2mm} {{\bm{W}_{M}}^{(t, d)}}^T \bm{X} 
%   - \frac{1}{\tau_{k^{(d)}+1}^{\text{det}} - \tau_{k^{(d)}}^{\text{det}}}\sum_{t=\tau_{k^{(d)}}^{\text{det}}+1}^{\tau_{k^{(d)}+1}^{\text{det}}} \hspace{-2mm} {{\bm{W}_{M}}^{(t, d)}}^T \bm{X} \Bigg{|}^2}, \notag 
% \end{equation} 
where
\begin{equation}
  a_{k^{(d)}} = \frac{\left({{\tau_{k^{(d)}}^{\text{det}}}^{\hspace{-0.7mm}(d)} - {\tau_{k^{(d)}-1}^{\rm det}}^{\hspace{-4.3mm}(d)}}\right) \left({\tau_{k^{(d)}+1}^{\rm det}}^{\hspace{-4.3mm}(d)} - {\tau_{k^{(d)}}^{\text{det}}}^{\hspace{-0.7mm}(d)}\right) c_{\text{sym}}^{(d)}}{\left({{\tau_{k^{(d)}+1}^{\rm det}}^{\hspace{-4.3mm}(d)} - {\tau_{k^{(d)}-1}^{\rm det}}^{\hspace{-4.3mm}(d)}}\right)M} \in \mathbb{R} \notag
\end{equation}
is required for scaling the test statistic, 
and the spectral averages in the two segments before and after the CP location $\tau_k^{\text{det}}$ for frequency $d \in \mathcal{D}_{\tau_k^{\text{det}}}$ are respectively denoted as 
\begin{align}
  \bar{F}_{{{\tau_{k^{(d)}-1}^{\rm det}}^{\hspace{-4.3mm}(d)}+1:{\tau_{k^{(d)}}^{\text{det}}}^{\hspace{-1.2mm}(d)}}}^{(d)} 
  &= {\frac{1}{{\tau_{k^{(d)}}^{\text{det}}}^{\hspace{-0.7mm}(d)} - {\tau_{k^{(d)}-1}^{\rm det}}^{\hspace{-4.3mm}(d)}}} \, \sum_{t = {\tau_{k^{(d)}-1}^{\rm det}}^{\hspace{-4.3mm}(d)}+1}^{{\tau_{k^{(d)}}^{\text{det}}}^{\hspace{-1.2mm}(d)}} F_t^{(d)} \notag \\
  &= {\frac{1}{{\tau_{k^{(d)}}^{\text{det}}}^{\hspace{-0.7mm}(d)} - {\tau_{k^{(d)}-1}^{\rm det}}^{\hspace{-4.3mm}(d)}}} \left(\bm{1}_{{\tau_{k^{(d)}-1}^{\rm det}}^{\hspace{-4.3mm}(d)}+1:{\tau_{k^{(d)}}^{\text{det}}}^{\hspace{-1.2mm}(d)}} \otimes \bm{w}_M^{(d)}\right)^\top \bm{X} \in \mathbb{C}, \notag
\end{align}
\begin{align}
  \bar{F}_{{\tau_{k^{(d)}}^{\text{det}}}^{\hspace{-1.2mm}(d)}+1:{\tau_{k^{(d)}+1}^{\rm det}}^{\hspace{-4.3mm}(d)}}^{(d)} 
  &= \frac{1}{{\tau_{k^{(d)}+1}^{\rm det}}^{\hspace{-4.3mm}(d)} - {\tau_{k^{(d)}}^{\text{det}}}^{\hspace{-0.7mm}(d)}}\sum_{t = {\tau_{k^{(d)}}^{\text{det}}}^{\hspace{-1.2mm}(d)}+1}^{{\tau_{k^{(d)}+1}^{\rm det}}^{\hspace{-4.3mm}(d)}} F_t^{(d)} \notag \\
  &= \frac{1}{{\tau_{k^{(d)}+1}^{\rm det}}^{\hspace{-4.3mm}(d)} - {\tau_{k^{(d)}}^{\text{det}}}^{\hspace{-0.7mm}(d)}} \left(\bm{1}_{{\tau_{k^{(d)}}^{\text{det}}}^{\hspace{-1.2mm}(d)}+1:{\tau_{k^{(d)}+1}^{\rm det}}^{\hspace{-4.3mm}(d)}} \otimes \bm{w}_M^{(d)}\right)^\top \bm{X} \in \mathbb{C}. \notag
\end{align}
%
By introducing a projection matrix defined as %depends on the result of the algorithm A
\begin{equation}
  P_k = \sum_{d \in \mathcal{D}_{\tau_k^{\text{det}}}} a_{k^{(d)}} {\bm{v}}_{k^{(d)}} {{\bm{v}}^{\ast}}_{k^{(d)}}^\top \in \mathbb{R}^{N \times N}, \notag
\end{equation}
the test statistic in~(\ref{eq:teststatistic}) is simply written as
\begin{align}
 \label{eq:teststatistic2}
 T_k(\bm{X}) = \sigma^{-1} ||P_k \bm{X}||,
\end{align}
where
\begin{equation}
  {\bm{v}}_{k^{(d)}} = {\frac{1}{{\tau_{k^{(d)}}^{\text{det}}}^{\hspace{-0.7mm}(d)} - {\tau_{k^{(d)}-1}^{\rm det}}^{\hspace{-4.3mm}(d)}}} \left(\bm{1}_{{\tau_{k^{(d)}-1}^{\rm det}}^{\hspace{-4.3mm}(d)}+1:{\tau_{k^{(d)}}^{\text{det}}}^{\hspace{-1.2mm}(d)}} \otimes \bm{w}_M^{(d)}\right) 
  - \frac{1}{{\tau_{k^{(d)}+1}^{\rm det}}^{\hspace{-4.3mm}(d)} - {\tau_{k^{(d)}}^{\text{det}}}^{\hspace{-0.7mm}(d)}} \left(\bm{1}_{{\tau_{k^{(d)}}^{\text{det}}}^{\hspace{-1.2mm}(d)}+1:{\tau_{k^{(d)}+1}^{\rm det}}^{\hspace{-4.3mm}(d)}} \otimes \bm{w}_M^{(d)}\right) \in \mathbb{C}^N, \notag
\end{equation}
and ${{\bm{v}}^{\ast}}_{k^{(d)}}$ is the complex conjugate of ${\bm{v}}_{k^{(d)}}$. 

\textbf{Naive $p$-value.}
%
The sampling distribution of the test statistic in~(\ref{eq:teststatistic}) and (\ref{eq:teststatistic2}) is highly complicated.
%
However, if we ``forget'' the fact that the CP candidates are selected by looking at the sequence $\bm X$, the matrix $P_k$ in~(\ref{eq:teststatistic2}) does not depend on $\bm X$, meaning that the test statistic $T_k(\bm X)$ simply follows a $\chi$-distribution with ${\rm tr}(P_k)$ degrees of freedom under the null hypothesis. 
%
The $p$-values obtained by this sampling distribution are referred to as \emph{naive $p$-values} and computed as 
\begin{equation}
  p_k^{\text{naive}} = \mathbb{P}_{\text{H}_0,k} (T_k(\bm{X}) \geq T_k(\bm{x})). \notag %\label{p_naive}
\end{equation}
%
Of course, since the CP candidates are selected based on the sequence \(\bm{X}\), it is not appropriate to quantify the statistical significance using naive $p$-values.  
%
If naive $p$-values are mistakenly used, the type I error rate can become significantly larger than the significance level $\alpha$.

\subsection{Computing Selective $p$-values}

\textbf{Selective Inference (SI).}
%
To address the above issue, we consider the sampling distribution of the test statistic conditional on the event that the detected CP candidates $\bm{\mathcal{T}}$ for a random sequence $\bm{X}$ is the same as $\bm{\mathcal{T}}^{\text{det}}$ for the observed sequence $\bm{x}$, that is, 
\begin{equation}
  T_k(\bm{X}) \, | \, \{\mathcal{A}(\bm{X}) = \mathcal{A}(\bm{x})\}. \label{condition}
\end{equation}
%
To compute a selective $p$-value based on the conditional sampling distribution in~(\ref{condition}), we also introduce a condition on the sufficient statistic of the nuisance parameter $\mathcal{Q}(\bm{X})$, which is defined as % which is independent of the test statistic $T_k(\bm{X})$ and defined as
\begin{equation}
  \mathcal{Q}(\bm{X}) = \{\mathcal{V}(\bm{X}), \, \mathcal{U}(\bm{X})\} \label{nuisance}
\end{equation} 
with
\begin{equation}
  \mathcal{V}(\bm{X}) = \frac{\sigma P_k \bm{X}}{\| P_k \bm{X} \|} \in \mathbb{R}^N, ~~~ \mathcal{U}(\bm{X}) = (I_{N} - P_k) \bm{X} \in \mathbb{R}^N. \notag
\end{equation}
%
This additional conditioning on $\mathcal{Q}(\bm{X})$ is a standard approach for computational tractability in the SI literature.
% \footnote{The nuisance parameter $\mathcal{Q}(\bm{X})$ corresponds to the component $\{\bm{w}, \bm{z}\}$ in the seminal paper~\citep{chen2020valid} (see Section 3, Theorem 3.7) and this technique is used in almost all the SI-related works that we cited.} 
%
Taking into account~(\ref{condition}) and (\ref{nuisance}), the selective $p$-value is defined as
\begin{equation}
  p_k^{\text{selective}} = \mathbb{P}_{\text{H}_0,k} (T_k(\bm{X}) \geq T_k(\bm{x}) \, | \, \bm{X} \in \mathcal{X}), \label{p_selective}
\end{equation}
where the conditioning event $\mathcal{X}$ is defined as
\begin{equation}
  \mathcal{X} = \{\bm{X} \in \mathbb{R}^N \, | \, \mathcal{A}(\bm{X}) = \mathcal{A}(\bm{x}), \mathcal{Q}(\bm{X}) =\mathcal{Q}(\bm{x})\}. \label{data_space}
\end{equation}
%
The subspace $\mathcal{X}$ is restricted to one-dimensional data space in $\mathbb{R}^N$~\citep{lee2016exact, liu2018more}, as stated in the following theorem.
\begin{theorem}
  \label{thm:truncation}
  The set $\mathcal{X}$ in~(\ref{data_space}) can be rewritten using a scalar parameter $z \in \mathbb{R}$ as
  \begin{equation}
    \mathcal{X} = \{\bm{X} = \bm{a} + \bm{b} z \, | \, z \in \mathcal{Z}\}, \label{one-D}
  \end{equation}
  where $\bm{a}, \bm{b} \in \mathbb{R}^N$, and the truncation region $\mathcal{Z}$ are defined as
  \begin{equation}
    \bm{a} =\mathcal{U}(\bm{x}), ~~~ \bm{b} = \mathcal{V}(\bm{x}), \label{ab}
  \end{equation}
  \begin{equation}
    \mathcal{Z} = \{z \in \mathbb{R} \, | \, \mathcal{A}( \bm{a} + \bm{b} z) = \mathcal{A}(\bm{x})\}. \label{truncatedarea}
  \end{equation}
\end{theorem}

The proof of Theorem~\ref{thm:truncation} is deferred to Appendix~\ref{app:proofs:truncation}. 
%
Let us denote a random variable $Z = T_k(\bm{X}) \in \mathbb{R}$ and its observation $z^{\text{obs}} = T_k(\bm{x}) \in \mathbb{R}$ that satisfy $\bm{X} = \bm{a} + \bm{b} Z$ and $\bm{x} = \bm{a} + \bm{b} z^{\text{obs}}$, respectively.
%
The selective $p$-value in~(\ref{p_selective}) can be rewritten as
\begin{equation}
  p_k^{\text{selective}} = \mathbb{P}_{\text{H}_0,k} (Z \geq z^{\text{obs}} \, | \, Z \in \mathcal{Z}). \label{psel_z}
\end{equation} 
%
Since the uncnoditional variable $Z \sim \chi(\mathrm{tr}(P_k))$ under the null hypothesis, the conditional variable $Z \, | \, Z \in \mathcal{Z}$ follows a truncated $\chi$-distribution with $\mathrm{tr}(P_k)$ degrees of freedom and truncation region $\mathcal{Z}$.
%
Once the truncation region $\mathcal{Z}$ is identified, the selective $p$-value in~(\ref{psel_z}) can be computed as
\begin{equation}
  p_k^{\text{selective}} = 1 - {F_{\text{tr}\left(P_{k}\right)}^{\text{cdf}}}^{\hspace{-3mm} \mathcal{Z}} \, (z^{\text{obs}}), \notag % ただし$F_\nu^\mathcal{Z}$は, 自由度$\nu$, 切断区間$\mathcal{Z}$の切断カイ分布の累積分布関数
\end{equation}
where ${F_{\text{tr}(P_{k})}^{\text{cdf}}}^{\hspace{-3mm} \mathcal{Z}}$ is the cumulative distribution function of the truncated $\chi$-distribution with $\mathrm{tr}(P_k)$ degrees of freedom and the truncation region $\mathcal{Z}$.
%
A schematic illustration of the SI framework is shown in Figure~\ref{fig5}.

\begin{figure}[H]
  \centering
  \includegraphics[width=0.7\hsize]{figure/fig5.pdf}
  \caption{
  %
  Schematic illustration of the SI framework.
  %
  A point in the data space $\mathbb{R}^N$ corresponds to a sequence with length $N$.
  %
  %  The blue regions in the data space indicate that, if we input a point in these regions into the algorithm $\mathcal{A}$, the same CP candidates $\bm{\mathcal{T}}^{\rm det}$ which are obtained from the observed sequence $\bm x$. 
  The darkly shaded regions in the data space indicate that, if we input a point in these regions into the algorithm $\mathcal{A}$, 
  the CP candidates are the same as $\bm{\mathcal{T}}^{\rm det}$ obtained from the observed sequence $\bm x$.
  %
  By conditioning on these regions and $\mathcal{Q}(\bm{X})$, the conditional sampling distribution of the test statistic $T_k(\bm X)$ is represented as a truncated $\chi$-distribution. 
  %
  Selective $p$-values are defined based on the tail probability of such a truncated $\chi$-distribution.
  }
  \label{fig5}
\end{figure}

\begin{theorem}
  \label{thm:uniform}
  The selective $p$-value satisfies the property of a valid $p$-value: 
  \begin{equation}
   \mathbb{P}_{\rm{H}_0,k} (p_k^{\rm{selective}}\leq\alpha) = \alpha, \, \forall \alpha \in [0, 1]. \notag
  \end{equation}
\end{theorem}
%
The proof of Theorem~\ref{thm:uniform} is deferred to Appendix~\ref{app:proofs:uniform}.
%
This theorem guarantees that the type I error rate can be controlled at any significance level $\alpha$ by using the selective $p$-value. 
%
%In the next section, the identification of the truncation region $\mathcal{Z}$ which is required to compute the selective $p$-value will be discussed.



Several methods for computing selective $p$-values have been proposed in the SI community. 
In this study, we adopt the method based on parametric programming~\citep{le2022more} 
for the identification of the truncation region $\mathcal{Z}$.
%
For further details, refer to Appendix~\ref{app:truncation}.

% If we only consider the first condition $\mathcal{A}(\bm{X}) = \mathcal{A}(\bm{x})$, 
% the conditional data space is represented as a subset of $\mathbb{R}^N$, 
% thus the identification is computationally difficult. 
% However, thanks to the second condition $\mathcal{Q}(\bm{X}) =\mathcal{Q}(\bm{x})$, %we only need to consider (考えれば十分である など)
% the subspace $\mathcal{X}$ is restricted to one-dimensional data space in $\mathbb{R}^N$~\citep{lee2016exact, liu2018more}, % N次元ではなく, 
% as stated in the following theorem.

% Thanks to the condition on $\mathcal{Q}(\bm{X})$, 
% we do not need to consider the $N$-dimensional data space. 
% Instead, we only need to consider the one-dimensional projected data space $\mathcal{Z}$ in~(\ref{truncatedarea}). 
% In this case, the conditional sampling distribution of the test statistic follows a truncated $\chi$-distribution. 

\section*{Data availability}\label{sec5-1}
This study involved public and private cytology datasets for system development. The public datasets comprising HErlev (\url{https://mde-lab.aegean.gr/index.php/downloads/}) and SIPaKMeD (\url{https://www.cs.uoi.gr/~marina/sipakmed.html}) are utilized for downstream task evaluation.
The details of private CCS-127K dataset are shown in Fig. \ref{F2_data}(a) and Extended Data Table. \ref{ST_patches}. 

\section*{Code availability}\label{sec5-1}
Source code for model development and the implementation details
will be publicly available upon
paper acceptance. 

\section*{Acknowledgements}\label{sec5-1}
This work was supported by the National Natural Science Foundation of China (No. 62202403), Hong Kong Innovation and Technology Commission (Project No. PRP/034/22FX and ITCPD/17-9), Shenzhen Science and Technology Program (Project No. KCXFZ20230731094059008) and the Project of Hetao Shenzhen-Hong Kong Science and Technology Innovation Cooperation Zone (HZQB-KCZYB-2020083). 

\section*{Author contributions}\label{sec5-1}
H.C., H.J., C.J., and H.L. investigated and designed the study. H.L. and L.D. collected a large-scale clinical cytology dataset. H.L., J.H., and L.D. provided the cell and WSI annotations. Data quality control and preprocessing were completed by H.S., C.J., and Y.Q.. C.J. developed the library for cytology slide reading. Drafting of the original manuscript was contributed by H.J.. C.J. designed and implemented the SSL algorithm. H.J., C.J., and J.M. were responsible for coding and implementing the cell-level and WSI-level screening algorithms. H.J. and Y.Z. developed the adaptation algorithm. H.J. conducted retrospective validation experiments. Clinical guidance was provided by R.C.K.C., A.H., and C.L., who also screened and interpreted WSIs in prospective validation. L.L., X.W., Q.W. and H.C. reviewed and edited the manuscript. Visualization and analysis of experiments were provided by R.L. and Z.C.. H.J. and C.J. contributed to data statistics and analysis. H.C. supervised this research.

\section*{Competing interests}\label{sec5-1}
The authors have no conflicts of
interest to declare.

\section*{Ethics approval}\label{sec5-1}
This study has been reviewed and approved by the Human and Artefacts Research Ethics Committee (HAREC) at the Hong Kong University of Science and Technology, with the protocol number HREP-2024-0216.

\section*{Additional information}\label{sec5-1}

\textbf{Correspondence and requests for materials} should be addressed to Hao Chen.
%%%%%%%%%%%%%%%%%%%%%%%%%%%%%%%%%%%%%%%%%%%%%%%%%%%%%%%%%%%%%%%%%%%%%%%%%%%%%%%%%%%%%%%%
\newpage
\section*{Extended Data}\label{secA1}

\setcounter{figure}{0}
\begin{figure*}[h]
\renewcommand{\figurename}{Extended Data Fig.}
    \centering
    \includegraphics[width=1.0\linewidth]{fig/F1_v6.pdf}
    \caption{\textbf{Overview of cervical cancer screening and computational cytology.} \textbf{a}. Illustration of cervical cells infected with human papillomavirus (HPV), leading to cervical cancer. The cytology sample collection involves sampling, centrifugation, staining, imaging, for cytologist examinations with screening reports. \textbf{b}. Key challenges in CCS include cytomorphology similarity, sparse abnormal cell distribution, identifying abnormal cells in gigapixel-sized whole slide images (WSI), and data variability. \textbf{c}. A general AI-assisted cancer screening pipeline, comprising a cell detector and a slide classifier, provides quantitative and visualized predictions for both cell-level and slide-level screening.}
    \label{F1_intro}
\end{figure*}


\begin{figure*}[h]
\renewcommand{\figurename}{Extended Data Fig.}
    \centering
    \includegraphics[width=1.0\linewidth]{fig/SF_method.pdf}
    \caption{\textbf{The conceptual illustration of proposed Smart-CCS paradigm.} It consists of three sequential stages: 1) large-scale self-supervised pretraining, 2) CCS model finetuning, and 3) test-time adaptation. }
    \label{SF_method}
\end{figure*}

\clearpage
\begin{figure*}[h]
\renewcommand{\figurename}{Extended Data Fig.}
    \centering
    \includegraphics[width=1.0\linewidth]{fig/SF_tsne.pdf}
    \caption{\textbf{Visualization of the effectiveness of pretraining from SIPaKMeD \cite{plissiti2018sipakmed} data using t-SNE, showcasing cervical cell images and feature distributions both without and with pretraining.} Five colors (dark blue, light blue, light orange, deep orange, green) denote five cell categories: superficial-intermediate (normal), parabasal (normal), metaplastic (benign), koilocytotic (abnormal), dyskeratotic (abnormal). Cell images are extracted with LVD-142M pretrained ViT-Large and our CCS-127K pretrained ViT-Large encoder to obtain 1024-dimensional features, subsequently reduced by t-SNE and plotted as a scatter plot. }
    \label{SF_tsne}
\end{figure*}

\clearpage
\begin{figure*}[h]
\renewcommand{\figurename}{Extended Data Fig.}
    \centering
    \includegraphics[width=1.0\linewidth]{fig/SF_det_v3.pdf}
    \caption{\textbf{Cervical cell detection and classification performance.} a) Performance comparison of per-class AP50 of detectors, demonstrating consistent performance improvement across all categories. b) Confusion matrix illustrating the classification performance of detectors, showing distinguishability between cell types such as glandular cells and squamous cells, as well as category correlations between ASC-US and LSIL, ASC-H and HSIL.}
    \label{SF_det}
\end{figure*}

\clearpage
\begin{figure*}[h]
\renewcommand{\figurename}{Extended Data Fig.}
    \centering
    \includegraphics[width=1.0\linewidth]{fig/SF2_platform.pdf}
    \caption{\textbf{The integrated and interactive interface of the developed cervical cancer screening system.} This integrated CCS system provides both local and global views of cytology samples, highlighting screened suspicious cells (LSIL, HSIL, etc.) and ultimately delivering TBS diagnostic suggestions (ASC-H).
    }
    \label{SF_integrate}
\end{figure*}


\clearpage
\begin{figure*}[h]
\renewcommand{\figurename}{Extended Data Fig.}
    \centering
    \includegraphics[width=1.0\linewidth]{fig/SF8_patching.pdf}
    \caption{\textbf{Whole slide image preprocessing.} Central circular foreground is extracted from the whole-slide image (69,888 pixels in the given example) using CLAM toolkit \cite{lu2021data}, and then cut into 1,200$\times$1,200 patches for further processing. }
    \label{SF_preprocessing}
\end{figure*}


\clearpage
\begin{figure*}[h]
\renewcommand{\figurename}{Extended Data Fig.}
    \centering
    \includegraphics[width=1.0\linewidth]{fig/SF9_qualitycontrol_v4.pdf}
    \caption{\textbf{Workflow of quality control for WSI.} Quality control for WSI involves considerations of cellularity ($>$5,000 per slide \cite{nayar2015bethesda}), preparation, staining, and scanning. Typical samples that are excluded include poor staining quality, out-of-focus, artifacts, impurities, image corruption, dried specimens, and interfering labeling.}
    \label{SF_qualitycontrol}
\end{figure*}

\clearpage
\begin{table}[h] 
\renewcommand{\arraystretch}{1.5}
\renewcommand{\tablename}{Extended Data Table.}
\centering 
\caption{\textbf{Abbreviations and nomenclature in this study.}}
\begin{tabular}{c|c} 
\hline
\rowcolor{cusyellow} \textbf{Abbreviations} & \textbf{Nomenclature} \\ 
\hline
 CC & Cervical cancer \\ 
\rowcolor{cusyellowl} CCS & Cervical cancer screening \\ 
 NILM & Negative for intraepithelial lesion or malignancy \\ 
\rowcolor{cusyellowl} ASC-US & Atypical squamous cells of undetermined significance \\ 
 LSIL & Low–grade squamous intraepithelial lesion \\ 
\rowcolor{cusyellowl} ASC-H & Atypical squamous cells cannot exclude an HSIL \\ 
 HSIL & High–grade squamous intraepithelial lesions \\ 
\rowcolor{cusyellowl} SCC & Squamous cell carcinoma \\ 
 AGC & Atypical glandular cells \\ 
\rowcolor{cusyellowl} AGC-FN & Atypical glandular cells, favor neoplastic \\ 
 AGC-NOS & Atypical glandular cells, not otherwise specified \\ 
\rowcolor{cusyellowl} AIS & Adenocarcinoma in situ \\ 
 ADC & Adenocarcinoma \\ 
\rowcolor{cusyellowl} TBS & The Bethesda system \\ 
 SIL & Squamous intraepithelial lesion \\ 
\rowcolor{cusyellowl} HPV & Human papillomavirus \\ 
 ECA & Epithelial cell abnormalities \\ 
\rowcolor{cusyellowl} CIN & Cervical intraepithelial neoplasia \\ 
\hline 
\end{tabular} 
\label{ST_abb}
\end{table}


\clearpage
\begin{table}[h] 
\renewcommand{\arraystretch}{1.5}
\renewcommand{\tablename}{Extended Data Table.}
\centering 
\caption{\textbf{Investigation of the effectiveness and scaling laws of self-supervised pretraining for cell classification tasks (Top-1 accuracy)}. Three cell classification datasets are evaluated, SIPaKMeD\cite{plissiti2018sipakmed}, HErlev \cite{jantzen2005pap}, and CCS-Cell. The best results are highlighted in bold.}
\begin{tabular}{c|ccc} 
\hline
\rowcolor{cusyellow} \textbf{Pretrain settings }& \textbf{SIPaKMeD} &\textbf{HErlev} & \textbf{CCS-Cell} \\ 
\hline
0M (w/o Pretrain)&0.926 (0.908 - 0.944) & 0.842 (0.789 - 0.895)&0.827 (0.810 - 0.844)\\
\rowcolor{cusyellowl} 1M& 0.936 (0.919 - 0.953) & 0.887 (0.841 - 0.933) & 0.859 (0.843 - 0.875) \\
6M&0.937 (0.920 - 0.954) & 0.890 (0.845 - 0.935)&0.863 (0.847 - 0.879)\\
\rowcolor{cusyellowl} 30M&0.940 (0.924 - 0.956) & 0.895 (0.851 - 0.939)&0.872 (0.857 - 0.887)\\
60M&0.942 (0.926 - 0.958) & 0.905 (0.863 - 0.947)&\textbf{0.883 (0.868 - 0.898)}\\
\rowcolor{cusyellowl} 100M&\textbf{0.960 (0.946 - 0.974)} & \textbf{0.914 (0.873 - 0.955)}&\textbf{0.883 (0.868 - 0.898)}\\
\hline
\end{tabular} 
\label{ST_pretrain_cell}
\end{table}

\clearpage
\begin{table}[h] 
\renewcommand{\arraystretch}{1.5}
\renewcommand{\tablename}{Extended Data Table.}
\centering 
\caption{\textbf{Investigation of the effectiveness and scaling laws of self-supervised pretraining for WSI classification tasks (Top-1 accuracy)}. Three retrospective centers are included, and we report the overall fine-grained classification results (ALL) along with the performance of different groups (ECA, ASC-US+, LSIL+, and HSIL+).}
\begin{tabular}{cc|ccc} 
\hline
\rowcolor{cusyellow} \multicolumn{2}{c|}{\multirow{1}{*}{\textbf{Pretrain settings}}} & \textbf{CCS-127K RC2}& \textbf{CCS-127K RC4} & \textbf{CCS-127K RC5} \\ 
\hline
\multirow{1}{*}{0M }&ECA &0.945 (0.936 - 0.954) &0.857 (0.840 - 0.873)& 0.712 (0.684 - 0.740)\\
& ASC-US+  & 0.940 (0.930 - 0.949) & 0.841 (0.824 - 0.858) & 0.692 (0.664 - 0.721)\\
& LSIL+ & 0.949 (0.940 - 0.958) & 0.842 (0.824 - 0.859) & 0.788 (0.763 - 0.813) \\
& HSIL+ & 0.986 (0.981 - 0.991) & 0.932 (0.920 - 0.944) & 0.887 (0.868 - 0.906) \\
& ALL & 0.895 (0.883 - 0.908) & 0.687 (0.665 - 0.709) & 0.638 (0.608 - 0.667) \\

 \rowcolor{cusyellowl} \multirow{1}{*}{1M}&ECA & 0.943 (0.934 - 0.952) & 0.887 (0.872 - 0.899) & 0.932 (0.916 - 0.947) \\
\rowcolor{cusyellowl} & ASC-US+ & 0.935 (0.925 - 0.944) & 0.873 (0.857 - 0.886) & 0.917 (0.900 - 0.934) \\
\rowcolor{cusyellowl} & LSIL+  & 0.942 (0.932 - 0.951) & 0.837 (0.820 - 0.852)  & 0.876 (0.856 - 0.896) \\
\rowcolor{cusyellowl} & HSIL+   & 0.983 (0.978 - 0.988) & 0.916 (0.903 - 0.927)  & 0.946 (0.933 - 0.960)\\
\rowcolor{cusyellowl} & ALL     & 0.889 (0.876 - 0.902) & 0.711 (0.689 - 0.729)  & 0.792 (0.767 - 0.816)\\

\multirow{1}{*}{6M}&ECA  & 0.952 (0.943 - 0.961) & 0.875 (0.859 - 0.891)& 0.937 (0.922 - 0.952) \\
&ASC-US+  & 0.942 (0.933 - 0.951) & 0.858 (0.842 - 0.875) & 0.927 (0.911 - 0.943)\\
&LSIL+    & 0.945 (0.935 - 0.954) & 0.832 (0.814 - 0.850) & 0.895 (0.876 - 0.914)\\
&HSIL+   & 0.985 (0.980 - 0.990) & 0.917 (0.904 - 0.930) & 0.948 (0.935 - 0.962) \\
&ALL     & 0.897 (0.884 - 0.909) & 0.693 (0.671 - 0.715) & 0.808 (0.784 - 0.832) \\

  \rowcolor{cusyellowl} \multirow{1}{*}{30M}&ECA & 0.952 (0.944 - 0.961) & 0.882 (0.866 - 0.897) & 0.944 (0.929 - 0.958) \\
  \rowcolor{cusyellowl} &ASC-US+ & 0.945 (0.936 - 0.954) & 0.868 (0.852 - 0.884) & 0.936 (0.921 - 0.951) \\
  \rowcolor{cusyellowl} &LSIL+  & 0.945 (0.936 - 0.954) & 0.848 (0.830 - 0.865)  & 0.884 (0.865 - 0.904) \\
  \rowcolor{cusyellowl} &HSIL+    & 0.984 (0.979 - 0.989) & 0.927 (0.915 - 0.940) & 0.946 (0.933 - 0.960)\\
  \rowcolor{cusyellowl} &ALL    & 0.896 (0.884 - 0.908) & 0.717 (0.695 - 0.738) & 0.808 (0.784 - 0.832) \\ 

\multirow{1}{*}{60M}&ECA   & 0.954 (0.946 - 0.963) & 0.915 (0.902 - 0.929)& 0.949 (0.936 - 0.963)\\
&ASC-US+  & 0.947 (0.938 - 0.956) & 0.900 (0.886 - 0.914)& 0.941 (0.926 - 0.955)\\
&LSIL+   & 0.947 (0.938 - 0.956) & 0.857 (0.840 - 0.874)  &  0.905 (0.887 - 0.923)\\
&HSIL+  & 0.985 (0.981 - 0.990) & 0.930 (0.918 - 0.942)  & 0.946 (0.932 - 0.959) \\
&ALL & 0.907 (0.895 - 0.919) & 0.742 (0.721 - 0.763) & 0.816 (0.792 - 0.840) \\

  \rowcolor{cusyellowl} \multirow{1}{*}{100M}&ECA  & 0.950 (0.942 - 0.959) & 0.915 (0.902 - 0.929) & 0.958 (0.946 - 0.970) \\
  \rowcolor{cusyellowl} &ASC-US+  & 0.942 (0.933 - 0.951) & 0.898 (0.884 - 0.913) & 0.949 (0.936 - 0.963) \\
  \rowcolor{cusyellowl} &LSIL+ & 0.951 (0.943 - 0.960) & 0.855 (0.838 - 0.872) & 0.914 (0.897 - 0.931)\\
  \rowcolor{cusyellowl} &HSIL+   & 0.985 (0.981 - 0.990) & 0.927 (0.915 - 0.940) & 0.956 (0.944 - 0.969)\\
  \rowcolor{cusyellowl} &ALL & 0.902 (0.890 - 0.913) & 0.742 (0.721 - 0.763) & 0.835 (0.812 - 0.857) \\ 

\hline 
\end{tabular} 
\label{ST_pretrain_wsi}
\end{table}


\clearpage
\begin{table}[h] 
\renewcommand{\arraystretch}{1.5}
\renewcommand{\tablename}{Extended Data Table.}
\centering 
\caption{\textbf{Ablation experiments for self-supervised pretraining backbone (ViT-Large and ViT-Gaint) and algorithm (DINOv2 \cite{oquabdinov2} and MoCov3 \cite{chen2021empirical}) (Top-1 accuracy)}. The best results are in bold.}
\begin{tabular}{c|ccc} 
\hline
 \rowcolor{cusyellow} \textbf{Pretrain settings} & \textbf{SIPaKMeD }&\textbf{HErlev} & \textbf{CCS-Cell} \\ 
\hline
w/o Pretrain&0.926 (0.908 - 0.944) & 0.842 (0.789 - 0.895)&0.827 (0.810 - 0.844)\\
 \rowcolor{cusyellowl} ViT-Large+DINOv2 &\textbf{0.942 (0.926 - 0.958)} & 0.905 (0.863 - 0.947)&\textbf{0.883 (0.868 - 0.898)}\\
ViT-Gaint+DINOv2 &0.934 (0.917 - 0.951) & \textbf{0.906 (0.864 - 0.948)}&0.877 (0.862 - 0.892)\\
 \rowcolor{cusyellowl}ViT-Large+MoCov3 &0.927 (0.909 - 0.945) & 0.892 (0.847 - 0.937)&0.868 (0.852 - 0.884)\\
\hline 
\end{tabular} 
\label{ST_pretrain_ablation}
\end{table}




\clearpage
\begin{table}[h] 
\renewcommand{\arraystretch}{1.5}
\renewcommand{\tablename}{Extended Data Table.}
\centering 
\caption{\textbf{Overall performance and comparison of SOTA  methods for abnormal cell detection.} The best results are in bold.}
    \begin{tabular}{c|ccc}
    \hline
    \rowcolor{cusyellow} \textbf{Model} & \textbf{mAP} & \textbf{AP50} & \textbf{mAR} \\ 
    \hline
    \textbf{YOLOv3} & 0.134 (0.122–0.146) & 0.263 (0.248–0.278) & 0.349 (0.333–0.365) \\ 
     \rowcolor{cusyellowl} \textbf{Faster R-CNN} & 0.206 (0.192–0.220) & 0.347 (0.331–0.363) & 0.432 (0.415–0.449) \\ 
    \textbf{RetinaNet} & 0.201 (0.187–0.215) & 0.333 (0.317–0.349) & 0.463 (0.446–0.480) \\ 
     \rowcolor{cusyellowl} \textbf{DETR} & 0.209 (0.195–0.223) & 0.390 (0.374–0.406) & 0.470 (0.453–0.487) \\ 
    \textbf{DDETR} & \textbf{0.239 (0.225–0.253)} & \textbf{0.406 (0.389–0.423)} & \textbf{0.490 (0.473–0.507)} \\ 
    \hline
\end{tabular}
\label{ST_det}
\end{table}

\begin{table}[h] 
\renewcommand{\arraystretch}{1.5}
\renewcommand{\tablename}{Extended Data Table.}
\centering 
\caption{\textbf{Class-wise performance and comparison of SOTA methods for abnormal cell detection.} The best results are in bold.}
    \begin{tabular}{c|ccc}
    \hline
    \rowcolor{cusyellow} \textbf{Model} & \textbf{ASC-US} & \textbf{LSIL} & \textbf{ASC-H} \\ 
    \hline
    \textbf{YOLOv3} & 0.316 (0.300–0.332) & 0.375 (0.359–0.391) & 0.125 (0.114–0.136) \\ 
    \rowcolor{cusyellowl} \textbf{Faster R-CNN} & 0.350 (0.334–0.366) & 0.494 (0.477–0.511) & 0.138 (0.126–0.150) \\ 
    \textbf{RetinaNet} & 0.360 (0.344–0.376) & 0.491 (0.474–0.508) & 0.138 (0.126–0.150) \\ 
    \rowcolor{cusyellowl} \textbf{DETR} & 0.431 (0.414–0.448) & 0.513 (0.496–0.530) & \textbf{0.219 (0.205–0.233)} \\ 
    \textbf{DDETR} & \textbf{0.447 (0.430–0.464)} & \textbf{0.549 (0.532–0.566)} & 0.210 (0.196–0.224) \\ 
    \hline
\end{tabular}
\label{ST_det_p1}
\end{table}

\begin{table}[h] 
\renewcommand{\arraystretch}{1.5}
\renewcommand{\tablename}{Extended Data Table.}
\centering 
\caption{\textbf{Class-wise performance and comparison of SOTA methods for abnormal cell detection (continued).} The best results are in bold.}
    \begin{tabular}{c|ccc}
    \hline
    \rowcolor{cusyellow} \textbf{Model} & \textbf{HSIL} & \textbf{SCC} & \textbf{AGC} \\ 
    \hline
    \textbf{YOLOv3} & 0.253 (0.238–0.268) & 0.021 (0.016–0.026) & 0.488 (0.471–0.505) \\ 
    \rowcolor{cusyellowl} \textbf{Faster R-CNN} & 0.336 (0.320–0.352) & 0.136 (0.124–0.148) & 0.631 (0.615–0.647) \\ 
    \textbf{RetinaNet} & 0.314 (0.298–0.330) & 0.077 (0.068–0.086) & 0.618 (0.602–0.634) \\ 
    \rowcolor{cusyellowl} \textbf{DETR} & 0.397 (0.380–0.414) & 0.096 (0.086–0.106) & 0.671 (0.655–0.687) \\ 
    \textbf{DDETR} & \textbf{0.408 (0.391–0.425)} & \textbf{0.124 (0.113–0.135)} & \textbf{0.701 (0.686–0.716)} \\ 
    \hline
\end{tabular}
\label{ST_det_p2}
\end{table}

\clearpage
\begin{table}[h] 
\renewcommand{\arraystretch}{2}
\renewcommand{\tablename}{Extended Data Table.}
\centering 
\caption{\textbf{Internal testing results of Smart-CCS in retrospective study (RC1-RC3).}}
\begin{tabular}{cc|c|c|c} 
\hline
\rowcolor{cusyellow} \multicolumn{2}{c|}{\multirow{1}{*}{\textbf{
 Metrics}}} & \textbf{RC1} & \textbf{RC2} & \textbf{RC3} \\ 
\hline
\multirow{1}{*}{ECA}&Accuracy (95\% CI) & 0.663 (0.644–0.682) & 0.940 (0.931–0.949) & 0.968 (0.960–0.976) \\
& AUC (95\% CI) & 0.767 (0.750–0.784) & 0.971 (0.965–0.978) & 0.990 (0.985–0.995) \\
& F1 Score (95\% CI) & 0.710 (0.692–0.728) & 0.940 (0.931–0.950) & 0.968 (0.960–0.976) \\
& Sensitivity (95\% CI) & 0.713 (0.695–0.731) & 0.859 (0.845–0.873) & 0.938 (0.927–0.949) \\
& Specificity (95\% CI) & 0.654 (0.635–0.673) & 0.961 (0.954–0.969) & 0.978 (0.971–0.985) \\

 \rowcolor{cusyellowl}  \multirow{1}{*}{ASC-US+ }&Accuracy (95\% CI) & 0.726 (0.709–0.743) & 0.934 (0.924–0.944) & 0.967 (0.959–0.975) \\
 \rowcolor{cusyellowl}  & AUC (95\% CI) & 0.767 (0.750–0.784) & 0.967 (0.960–0.974) & 0.990 (0.985–0.995)\\
 \rowcolor{cusyellowl}  & F1 Score (95\% CI)  & 0.759 (0.742–0.776) & 0.934 (0.925–0.944) & 0.967 (0.959–0.975)  \\
 \rowcolor{cusyellowl}  & Sensitivity (95\% CI) & 0.626 (0.607–0.645) & 0.852 (0.838–0.866) & 0.932 (0.920–0.944) \\
 \rowcolor{cusyellowl}  & Specificity (95\% CI)      & 0.743 (0.726–0.760) & 0.954 (0.946–0.962) & 0.978 (0.971–0.985) \\

\multirow{1}{*}{LSIL+}&Accuracy (95\% CI) & 0.807 (0.792–0.822) & 0.911 (0.900–0.922) & 0.848 (0.831–0.865) \\
& AUC (95\% CI) & 0.902 (0.890–0.914) & 0.975 (0.969–0.982) & 0.926 (0.914–0.938) \\
& F1 Score (95\% CI) & 0.859 (0.845–0.873) & 0.923 (0.913–0.933) & 0.896 (0.882–0.910)  \\
& Sensitivity (95\% CI) & 0.816 (0.801–0.831) & 0.929 (0.919–0.939) & 0.957 (0.948–0.966)  \\
& Specificity (95\% CI) & 0.807 (0.792–0.822) & 0.909 (0.898–0.921) & 0.844 (0.827–0.861) \\

 \rowcolor{cusyellowl}   \multirow{1}{*}{HSIL+}&Accuracy (95\% CI) & 0.967 (0.960–0.974) & 0.970 (0.964–0.977) & 0.898 (0.884–0.912)  \\
 \rowcolor{cusyellowl}  & AUC (95\% CI) & 0.943 (0.934–0.952) & 0.982 (0.977–0.987) & 0.934 (0.923–0.945) \\
 \rowcolor{cusyellowl}  & F1 Score (95\% CI)  & 0.979 (0.973–0.985) & 0.976 (0.970–0.982) & 0.936 (0.925–0.947) \\
 \rowcolor{cusyellowl}  & Sensitivity (95\% CI) & 0.750 (0.733–0.767) & 0.826 (0.811–0.841) & 0.800 (0.782–0.818)\\
 \rowcolor{cusyellowl}  & Specificity (95\% CI) & 0.968 (0.961–0.975) & 0.973 (0.967–0.980) & 0.899 (0.885–0.913)\\
\hline
\end{tabular} 
\label{ST_wsi}
\end{table}



\clearpage
\begin{table}[h] 
\renewcommand{\arraystretch}{2}
\renewcommand{\tablename}{Extended Data Table.}
\centering 
\caption{\textbf{Internal testing results of Smart-CCS in retrospective study (RC4-RC6).}}
\begin{tabular}{cc|c|c|c} 
\hline
 \rowcolor{cusyellow} \multicolumn{2}{c|}{\multirow{1}{*}{\textbf{
 Metrics}}} & \textbf{RC4} & \textbf{RC5} & \textbf{RC6}\\ 
\hline
\multirow{1}{*}{ECA}&Accuracy (95\% CI) & 0.894 (0.879–0.908) & 0.943 (0.928–0.957) & 0.821 (0.796–0.846)  \\
& AUC (95\% CI) & 0.968 (0.959–0.976) & 0.985 (0.977–0.992) & 0.890 (0.870–0.910) \\
& F1 Score (95\% CI) & 0.894 (0.879–0.908) & 0.943 (0.928–0.957) & 0.839 (0.816–0.863)  \\
& Sensitivity (95\% CI) & 0.961 (0.951–0.970) & 0.948 (0.934–0.961) & 0.814 (0.789–0.839) \\
& Specificity (95\% CI) & 0.837 (0.819–0.855) & 0.938 (0.923–0.952) & 0.822 (0.798–0.847)  \\

 \rowcolor{cusyellowl}  \multirow{1}{*}{ASC-US+ }&Accuracy (95\% CI) & 0.877 (0.861–0.893) & 0.925 (0.908–0.941) & 0.823 (0.799–0.848) \\
 \rowcolor{cusyellowl}  & AUC (95\% CI)  & 0.957 (0.947–0.967) & 0.978 (0.968–0.987) & 0.888 (0.868–0.909)  \\
 \rowcolor{cusyellowl}  & F1 Score (95\% CI)  & 0.877 (0.862–0.893)& 0.925 (0.908–0.941) & 0.841 (0.817–0.864) \\
 \rowcolor{cusyellowl}  & Sensitivity (95\% CI) & 0.961 (0.952–0.971) & 0.947 (0.933–0.961) & 0.800 (0.774–0.826) \\
 \rowcolor{cusyellowl}  & Specificity (95\% CI)   & 0.812 (0.794–0.831)    & 0.905 (0.887–0.923) & 0.828 (0.803–0.852) ) \\

\multirow{1}{*}{LSIL+}&Accuracy (95\% CI) & 0.828 (0.810–0.846) & 0.870 (0.849–0.891) & 0.831 (0.807–0.855) \\
& AUC (95\% CI) & 0.916 (0.903–0.930) & 0.965 (0.953–0.976) & 0.946 (0.931–0.960) \\
& F1 Score (95\% CI) & 0.838 (0.820–0.855)& 0.875 (0.855–0.896) & 0.864 (0.842–0.886)  \\
& Sensitivity (95\% CI)& 0.842 (0.824–0.859) & 0.957 (0.945–0.970) & 0.935 (0.919–0.951) \\
& Specificity (95\% CI) & 0.824 (0.806–0.842) & 0.837 (0.814–0.860) & 0.821 (0.797–0.846) \\

 \rowcolor{cusyellowl}   \multirow{1}{*}{HSIL+}&Accuracy (95\% CI) & 0.878 (0.862–0.894)& 0.911 (0.893–0.928) & 0.909 (0.891–0.928) \\
 \rowcolor{cusyellowl}  & AUC (95\% CI) & 0.918 (0.904–0.931) & 0.972 (0.962–0.983) & 0.973 (0.963–0.984)  \\
 \rowcolor{cusyellowl}  & F1 Score (95\% CI)   & 0.895 (0.880–0.909)& 0.918 (0.901–0.935) & 0.930 (0.913–0.946) \\
 \rowcolor{cusyellowl}  & Sensitivity (95\% CI)  & 0.818 (0.800–0.837)& 0.895 (0.876–0.914) & 0.897 (0.878–0.917) \\
 \rowcolor{cusyellowl}  & Specificity (95\% CI)  &  0.884 (0.869–0.899) & 0.913 (0.896–0.930) & 0.910 (0.892–0.928) \\

\hline 
\end{tabular} 
\label{ST_wsi_2}
\end{table}


\clearpage
\begin{table}[h] 
\renewcommand{\arraystretch}{2}
\renewcommand{\tablename}{Extended Data Table.}
\centering 
\caption{\textbf{Internal testing results of Smart-CCS in retrospective study (RC7-RC17).} Note: R14-R17 were merged due to the limited sample size at these four centers.}
\begin{tabular}{cc|c|c} 
\hline
 \rowcolor{cusyellow} \multicolumn{2}{c|}{\multirow{1}{*}{\textbf{
 Metrics}}}  & \textbf{RC7 }& \textbf{RC14-17 }\\ 
\hline
\multirow{1}{*}{ECA}&Accuracy (95\% CI) & 0.942 (0.926–0.958) & 0.968 (0.952–0.984) \\
& AUC (95\% CI)  & 0.971 (0.960–0.982) & 0.971 (0.956–0.986) \\
& F1 Score (95\% CI) & 0.942 (0.926–0.958) & 0.968 (0.952–0.984) \\
& Sensitivity (95\% CI)  & 0.857 (0.833–0.881) & 0.940 (0.918–0.961) \\
& Specificity (95\% CI)  & 0.968 (0.956–0.980) & 0.977 (0.964–0.991) \\

 \rowcolor{cusyellowl}  \multirow{1}{*}{ASC-US+ }&Accuracy (95\% CI) & 0.948 (0.933–0.963) & 0.970 (0.955–0.986) \\
 \rowcolor{cusyellowl}  & AUC (95\% CI)  & 0.971 (0.960–0.982) & 0.980 (0.967–0.992) \\
 \rowcolor{cusyellowl}  & F1 Score (95\% CI)  & 0.948 (0.933–0.963) & 0.970 (0.955–0.986) \\
 \rowcolor{cusyellowl}  & Sensitivity (95\% CI)  & 0.887 (0.866–0.908) & 0.947 (0.927–0.968) \\
 \rowcolor{cusyellowl}  & Specificity (95\% CI)    & 0.965 (0.953–0.977) & 0.978 (0.964–0.991) \\

\multirow{1}{*}{LSIL+}&Accuracy (95\% CI)  & 0.818 (0.792–0.844) & 0.968 (0.952–0.984) \\
& AUC (95\% CI) & 0.888 (0.867–0.909) & 0.996 (0.989–1.000) \\
& F1 Score (95\% CI) & 0.846 (0.822–0.870) & 0.969 (0.953–0.985) \\
& Sensitivity (95\% CI) & 0.795 (0.768–0.822) & 1.000 (1.000–1.000) \\
& Specificity (95\% CI) & 0.820 (0.794–0.846) & 0.960 (0.942–0.977) \\

 \rowcolor{cusyellowl}   \multirow{1}{*}{HSIL+}&Accuracy (95\% CI) & 0.890 (0.869–0.911) & 0.921 (0.897–0.946) \\
 \rowcolor{cusyellowl}  & AUC (95\% CI)  & 0.932 (0.915–0.949) & 0.994 (0.987–1.001) \\
 \rowcolor{cusyellowl}  & F1 Score (95\% CI)  & 0.913 (0.894–0.932) & 0.939 (0.917–0.961) \\
 \rowcolor{cusyellowl}  & Sensitivity (95\% CI)  & 0.778 (0.750–0.806) & 1.000 (1.000–1.000) \\
 \rowcolor{cusyellowl}  & Specificity (95\% CI)  & 0.895 (0.874–0.916) & 0.918 (0.893–0.943) \\

\hline 
\end{tabular} 
\label{ST_wsi_3}
\end{table}


\clearpage
\begin{table}[h] 
\renewcommand{\arraystretch}{2}
\renewcommand{\tablename}{Extended Data Table.}
\centering 
\caption{\textbf{External testing results of Smart-CCS in the retrospective study.} Base refers to the typical two-step CCS model, w/ P is adding pretraining, w/ P\&A denotes our proposed Smart-CCS with pretraining and adaptation.}
\begin{tabular}{cc|ccc} 
\hline
 \rowcolor{cusyellow} \multicolumn{2}{c|}{\multirow{1}{*}{\textbf{Metrics}}} & \textbf{Base} &\textbf{w/ P }&\textbf{w/ P\&A (ours)}\\ 
\hline
\multirow{1}{*}{RC8}&Accuracy (95\% CI) & 0.680 (0.650–0.711)&0.802 (0.776–0.828)&0.837 (0.820–0.854)\\
& AUC (95\% CI) &0.812 (0.787–0.837) & 0.851 (0.828–0.874)& 0.923 (0.911–0.935)\\
& F1 (95\% CI) &0.750 (0.722–0.778)&0.840 (0.816–0.864) &0.867 (0.851–0.882)\\
& Sensitivity (95\% CI) &0.762 (0.735–0.790)& 0.775 (0.748–0.802) & 0.891 (0.877–0.905)\\
& Specificity (95\% CI) &0.673 (0.642–0.703)& 0.805 (0.779–0.830)&0.832 (0.815–0.849)\\




 \rowcolor{cusyellowl}  \multirow{1}{*}{RC9}&Accuracy (95\% CI) & 0.812 (0.785–0.838)&0.897 (0.876–0.918)&0.938 (0.927–0.949)\\
 \rowcolor{cusyellowl} & AUC (95\% CI) &0.897 (0.877–0.918) &0.930 (0.913–0.948)&0.963 (0.954–0.972)\\
 \rowcolor{cusyellowl} & F1 (95\% CI) & 0.823 (0.797–0.849) &0.898 (0.878–0.919)&0.937 (0.926–0.949) \\
 \rowcolor{cusyellowl} & Sensitivity (95\% CI) & 0.821 (0.795–0.847) &0.816 (0.789–0.842) &0.833 (0.816–0.851)\\
 \rowcolor{cusyellowl} & Specificity (95\% CI) & 0.809 (0.783–0.836)&0.921 (0.903–0.939) &0.967 (0.958–0.975)\\




\multirow{1}{*}{RC10}&Accuracy (95\% CI) &0.741 (0.712–0.770) &0.871 (0.849–0.893) &0.906 (0.892–0.919)\\
& AUC (95\% CI) &0.830 (0.805–0.855)&0.911 (0.892–0.930) &0.953 (0.943–0.963)\\
& F1 (95\% CI) &0.771 (0.743–0.799)& 0.879 (0.857–0.901)& 0.911 (0.898–0.924)\\
& Sensitivity (95\% CI) &0.781 (0.753–0.808)& 0.808 (0.782–0.834) &0.892 (0.878–0.907) \\
& Specificity (95\% CI) &0.733 (0.704–0.762)&0.884 (0.863–0.905)& 0.909 (0.895–0.922) \\



 \rowcolor{cusyellowl}  \multirow{1}{*}{RC11 }&Accuracy (95\% CI) & 0.815 (0.788–0.841) &0.886 (0.864–0.907)& 0.921 (0.908–0.934)\\
 \rowcolor{cusyellowl} & AUC (95\% CI) & 0.875 (0.853–0.898)&0.935 (0.919–0.952)&0.962 (0.953–0.971)\\
 \rowcolor{cusyellowl} & F1 (95\% CI) &0.815 (0.789–0.841) &0.887 (0.865–0.908) &0.921 (0.908–0.933)\\
 \rowcolor{cusyellowl} & Sensitivity (95\% CI) & 0.761 (0.732–0.790)   &0.895 (0.874–0.916) &0.871 (0.855–0.887)\\
 \rowcolor{cusyellowl} & Specificity (95\% CI) & 0.846 (0.822–0.871)  &0.880 (0.859–0.902)&0.951 (0.941–0.961)\\




\multirow{1}{*}{RC12}&Accuracy (95\% CI) & 0.804 (0.771–0.836)  &0.828 (0.797–0.859)&0.871 (0.851–0.891)\\
& AUC (95\% CI) &0.871 (0.843–0.898)& 0.916 (0.894–0.939) &0.941 (0.927–0.955) \\
& F1 (95\% CI) &0.804 (0.772–0.836)& 0.826 (0.796–0.857) & 0.871 (0.851–0.891)\\
& Sensitivity (95\% CI) &0.786 (0.752–0.819)&0.899 (0.875–0.924)& 0.884 (0.865–0.903)\\
& Specificity (95\% CI) &0.824 (0.793–0.855) & 0.747 (0.712–0.783)  &0.857 (0.836–0.878)\\



 \rowcolor{cusyellowl}  \multirow{1}{*}{RC13}&Accuracy (95\% CI) & 0.644 (0.522–0.766)&0.793 (0.761–0.825)& 0.828 (0.804–0.852)\\
 \rowcolor{cusyellowl} & AUC (95\% CI) & 0.635 (0.513–0.758)  &0.815 (0.784–0.846) &0.880 (0.859–0.901)\\
 \rowcolor{cusyellowl} & F1 (95\% CI) & 0.644 (0.522–0.766) &0.814 (0.783–0.845)&0.846 (0.823–0.869)\\
 \rowcolor{cusyellowl} & Sensitivity (95\% CI) & 0.659 (0.621–0.697)&0.950 (0.939–0.960)&0.771 (0.744–0.798)\\
 \rowcolor{cusyellowl} & Specificity (95\% CI) & 0.594 (0.468–0.719) &0.816 (0.785–0.847) &0.838 (0.814–0.861)\\

\hline 
\end{tabular} 
\label{ST_ext}
\end{table}



\clearpage
\begin{table}[h] 
\renewcommand{\arraystretch}{2}
\renewcommand{\tablename}{Extended Data Table.}
\centering 
\caption{\textbf{Performance comparisons of different classification methods in WSI classification task (AUC (95\% CI)).} The best results are in bold.}
\begin{tabular}{c|cc} 
\hline
 \rowcolor{cusyellow} \textbf{Methods} & \textbf{Internal Test}  & \textbf{External Test} \\ 
\hline
MeanMIL & 0.965 (0.961–0.969)&0.950 (0.945–0.954) \\
 \rowcolor{cusyellowl} MaxMIL & 0.957 (0.953–0.961) &0.933 (0.928–0.938)\\
ABMIL \cite{ilse2018attention} & 0.965 (0.961–0.969) &0.947 (0.942–0.952) \\
 \rowcolor{cusyellowl} DSMIL \cite{li2021dual} & 0.959 (0.955–0.964) & 0.938 (0.933–0.943)  \\
CLAM-SB \cite{shao2021transmil} & 0.964 (0.960–0.968) & 0.919 (0.914–0.925) \\
 \rowcolor{cusyellowl}  TransMIL \cite{lu2021data} & 0.966 (0.962–0.970)  & 0.951 (0.946–0.955)\\
S4MIL \cite{fillioux2023structured} &\textbf{0.969 (0.966–0.973)} & \textbf{0.953 (0.948–0.957)}\\
\hline 
\end{tabular} 
\label{ST_clas}
\end{table}

\clearpage
\begin{table}[h] 
\renewcommand{\arraystretch}{2}
\renewcommand{\tablename}{Extended Data Table.}
\centering 
\caption{\textbf{Performance of cervical cytology cancer screening in the prospective study across three centers (PC1-PC3).}}
\begin{tabular}{c|ccc} 
\hline
 \rowcolor{cusyellow} \textbf{Metrics} & \textbf{PC1 ($N$=998)} & \textbf{PC2 ($N$=1,311)} & \textbf{PC3 ($N$=1,044)}\\ 
\hline
Accuracy (95\% CI) & 0.877 (0.856–0.897)&0.862 (0.843–0.881) &0.950 (0.937–0.963)\\
 \rowcolor{cusyellowl} AUC (95\% CI) &0.947 (0.933–0.961)   & 0.924 (0.910–0.938)& 0.986 (0.979–0.993) \\
F1 Score (95\% CI) & 0.877 (0.857–0.898) & 0.867 (0.848–0.885)   &0.951 (0.938–0.964)\\
 \rowcolor{cusyellowl} Sensitivity (95\% CI) &0.893 (0.874–0.912) & 0.881 (0.864–0.899) & 0.946 (0.932–0.960)\\
Specificity (95\% CI) &0.864 (0.843–0.885)& 0.855 (0.836–0.874)  &0.952 (0.939–0.965)\\

\hline 
\end{tabular} 
\label{ST_pros}
\end{table}


\newpage
\begin{table}[h] 
\renewcommand{\arraystretch}{1.5}
\renewcommand{\tablename}{Extended Data Table.}
\centering 
\caption{\textbf{Statistics of the CCS-127K dataset with the number of WSIs and corresponding patches at each center.}}
\begin{tabular}{c|c|c|c|c|c} 
\hline
 \rowcolor{cusyellow} \textbf{Center} & \textbf{WSI} & \textbf{Patch} & \textbf{Center} & \textbf{WSI} & \textbf{Patch} \\
\hline
 RC1 & 12,598 & 26,980,754   & RC26 & 60 & 63,215\\
 \rowcolor{cusyellowl}  RC2 & 14,529 & 17,400,701&  RC27 & 239 & 102,381\\
 RC3 & 9,039  & 27,416,532 & RC28 & 33& 22,435\\
 \rowcolor{cusyellowl}  RC4 &8,432 & 14,811,441 &  RC29 & 20 & 23,921 \\
 RC5 & 17,623 & 33,845,451& RC30 & 13 & 63,745\\
 \rowcolor{cusyellowl}  RC6 & 5,773 & 15,692,295 &  RC31 & 10& 5,527\\
 RC7 & 8,139 & 31,573,365 & RC32 & 5,516 & 2,290,698\\
 \rowcolor{cusyellowl}  RC8 & 1,873 & 2,002,922&  RC33 & 160 &354,062\\
 RC9 & 2,013 & 6,085,661 & RC34 & 1,319 & 1,732,829\\
 \rowcolor{cusyellowl}  RC10 &  1,797 & 1,776,018 &  RC35 & 543 & 1,921,444 \\
 RC11 & 3,400 & 8,420,141 & RC36 & 491 & 2,970,771\\
 \rowcolor{cusyellowl}  RC12 & 1,097 & 1,867,367&  RC37 & 6& 10,337\\
 RC13 &  1,876 & 371,098 & RC38 & 39 & 42,763 \\
 \rowcolor{cusyellowl}  RC14 & 12,823 & 12,832,172 &  RC39 & 13& 14,433\\
 RC15 & 657 & 979,460 & RC40 & 127 & 181,802 \\
 \rowcolor{cusyellowl}  RC16 &588 & 68,299 &  RC41 & 190 & 63,359\\
 RC17 & 707 & 231,009 & RC42 & 7,002 & 9,228,021\\
 \rowcolor{cusyellowl}  RC18 & 491 & 451,371 &  RC43 & 330 & 54,372\\
 RC19 &494 & 300,671 &  RC44 & 451 & 428,447 \\
 \rowcolor{cusyellowl}  RC20 & 933 & 1,749,139 &  RC45 & 14 & 45,227 \\
 RC21 & 234 & 85,197 & PC1 & 998 & 421,901\\
 \rowcolor{cusyellowl}  RC22 & 2,045 & 372,867 &  PC2 & 1,311 & 788,410\\
 RC23 & 192 & 223,526 & PC3 &1,044 & 602,583\\
 \rowcolor{cusyellowl}  RC24 & 108 & 51,613&  Total & 127,471& 227,266,352\\
RC25 & 81 & 244,599& & & \\
\hline
\end{tabular} 
\label{ST_patches}
\end{table}

\newpage
\begin{table}[h] 
\renewcommand{\arraystretch}{1.5}
\renewcommand{\tablename}{Extended Data Table.}
\centering 
\caption{\textbf{Hyperparameters used for self-supervised pretraining.}}

\begin{tabular}{c|c|p{2cm}c} 
\hline 
 \rowcolor{cusyellow} & \textbf{Hyperparameters} & \textbf{Value} \\
\hline 
\multirow{7}{*}{Model} & Layer number  & 24 \\
& Feature dimension & 1,024 \\
& Patch size & 14 \\
& Heads number & 16 \\
& FFN layer & MLP \\
& Drop path ratio & 0.4 \\
& Layer scale  & 1.00e-05 \\
\hline 
 \rowcolor{cusyellowl} Loss weight & DINO & 1 \\
 \rowcolor{cusyellowl}& iBOT & 1 \\
\hline 
\multirow{12}{*}{Optimization} & Teacher momentum & 0.994\\
& Total batch size & 1,024 \\
& Base learning rate & 1.00e-04 \\
& Minimum learning rate & 1.00e-06 \\
& Global crops scale & 0.32, 1.0 \\
& Global crops size & 224 \\
& Local crops scale & 0.02, 0.32 \\
& Local crops size & 98 \\
& Local crops number & 8 \\
& Gradient clip & 3\\
& Warmup iterations & 50,000 \\
& Total iterations & 500,000\\ 
\hline 
\end{tabular} 
\label{ST_pretrain_para}
\end{table}

\clearpage
\begin{table}[h] 
\renewcommand{\arraystretch}{1.5}
\renewcommand{\tablename}{Extended Data Table.}
\centering 
\caption{\textbf{Hyperparameters used for cell detector and WSI classifier in CCS model.}}
\begin{tabular}{c|c} 
\hline
 \rowcolor{cusyellow} \multicolumn{2}{c}{\multirow{1}{*}{\textbf{Abnormal Cell Detector}}} \\ 
\hline
backbone & ResNet50\\ 
 \rowcolor{cusyellowl} num\_encoder\_layers & 6\\ 
num\_decoder\_layers & 6\\ 
 \rowcolor{cusyellowl} num\_queries & 300\\ 
num\_classes & 7\\ 
 \rowcolor{cusyellowl} dropout & 0.1\\ 
epoch & 100\\ 

\hline
 \rowcolor{cusyellow} \multicolumn{2}{c}{\multirow{1}{*}{\textbf{WSI Classifier}}} \\ 
\hline
classifier & ABMIL, MeanMIL, MaxMIL, CLAM, DSMIL, TransMIL, S4MIL\\ 
 \rowcolor{cusyellowl} top-k & 50\\ 
epoch & 50\\ 
 \rowcolor{cusyellowl} num\_classes & 7\\ 
in\_dim & 1,024\\ 
\hline 
\end{tabular} 
\label{ST_det_para}
\end{table}

% \subsubsection*{Broader Impact Statement}
% \subsubsection*{Author Contributions}
\subsubsection*{Acknowledgments}
This work was partially supported by MEXT KAKENHI (20H00601), JST CREST (JPMJCR21D3, JPMJCR22N2), JST Moonshot R\&D (JPMJMS2033-05), JST AIP Acceleration Research (JPMJCR21U2), NEDO (JPNP18002, JPNP20006) and RIKEN Center for Advanced Intelligence Project.
% \documentclass{article}
\usepackage{amsmath}
\usepackage{authblk}
\usepackage{cite}
\usepackage{graphicx}

\newtheorem{lemma}{Lemma}

\title{Tensor Evolution: A Framework for Fast Evaluation of Tensor Computations using Recurrences}

\author[1]{Javed Absar}
\author[2]{Samarth Narang }
\author[3]{Muthu Baskaran }
\affil[1]{Qualcomm Technologies International}
\affil[2]{Qualcomm Technologies, Inc.}
\affil[3]{Qualcomm Technologies, Inc.}

\begin{document}
\maketitle

\begin{abstract}
This paper introduces a new mathematical framework for analysis and optimization of tensor
expressions within an enclosing loop. Tensors are multi-dimensional arrays of values.
They are common in high performance computing (HPC)  and machine learning
\cite{TensorComprehension} domains. Our framework extends Scalar Evolution
\cite{scev_j} -- an important optimization pass implemented in both LLVM and GCC -- to tensors.
Scalar Evolution (SCEV) relies on the theory of `Chain of Recurrences' \cite{Zima} for its
mathematical underpinnings. We use the same theory for \textbf{Tensor Evolution} (TeV).
While some concepts from SCEV map easily to TeV -- e.g. element-wise operations;
tensors introduce new operations \cite{Attention, mlir,TVM,Glow} such as concatenation,
slicing, broadcast, reduction, and reshape which have no equivalent in scalars and SCEV.
Not all computations are amenable to TeV analysis but it can play a part in the optimization
and analysis parts of ML and HPC compilers. Also, for many mathematical/compiler ideas,
applications may go beyond what was initially envisioned, once others build on it and
take it further. We hope for a similar trajectory for the tensor-evolution concept.
\end{abstract}

\section{Introduction}
In this work, we introduce \textbf{Tensor Evolution} (TeV), a framework designed for analysis and optimization
 of tensor computations within loops. TeV provides a mathematical framework and mechanism to trace tensor
  values as they evolve over iterations of the enclosing loops. This approach enables optimizations that
   simplify loop-carried tensor computations to reduce computational overhead.

As machine learning (ML) models grow in complexity and scale, the demand for efficient computation has
spurred the development of specialized ML compilers, such as MLIR\cite{mlir}, Glow\cite{Glow}, and
TVM\cite{TVM}. These compilers convert computation in frameworks such as PyTorch and TensorFlow into
IR (intermediate representation) graphs where nodes represent operations
(e.g., matrix multiplications, additions, etc.), and edges represent tensors.
While traditional compiler optimizations, such as Constant Propagation, Loop Strength Reduction,
and Loop Invariant Code Motion have been somewhat adapted to these graph compilers,
more advanced transformations for tensor operations remain unexplored. Notably,
scalar evolution which is widely successful in traditional compilers such as LLVM for
analyzing scalar variables in loops lacks a counterpart in tensor operations within deep learning models or HPC. 

The paper is organized as follows: To explain TeV, we first introduce the basic concepts of recurrences
 and SCEV. Then, we present a formal definition of TeV. We outline a set of lemmas (re-write rules)
  that simplify TeV expressions. Next, we provide a worked-out example. Finally, we survey related
   work and conclude the paper.

Please note that part of this work was previously presented at the LLVM Developer
Summit \cite{tev_j} in the form of `Technical Talk`.

\section{Chain of Recurrences and Scalar Evolution}
Consider the function $S(n) = 1+2+3...+n$. We know that $S(n)=n(n+1)/2$, and this closed form can help
 calculate $S$ at any point, bypassing the need for a long chain of sequential additions.
  `Chain of Recurrences' (CR) is essentially a more complex algebraic generalization of this
concept \cite{ZimaRealPaper}.CRs are a compact representation of a series of computations. 
They provide a set of lemmas to simplify CR expressions and offer a closed-form function to evaluate
 these representations at any fixed point. Scalar Evolution (SCEV), implemented in LLVM and GCC,
  incorporates a simplified form of CR. To understand CR and SCEV, let us consider an example.


% ---    SCEV code examples  ---
\begin{verbatim}
    foo(...) { 
      int f = k0;
      for (int i = 0; i < n; ++i) {
         ... 
         f = f + k1;
      }
      ... = f // loop exit  value of `f`
\end{verbatim}

Here the value of $f$ can be expressed as a `basic recurrence'($BR$) $f = \{k_0, +, k_1\}_i$. 
In this $BR$ notation, $f$ has initial value $k_0$ and in each iteration of the loop with 
induction variable $i$, $f$ increases by $k_1$. In compiler terminology, $i$ is a basic 
induction variable, while $f$ is a derived induction variable. To help appreciate the algebraic
 power of recurrences, we show a few lemmas on basic recurrence.
    
% ---  Basic Recurrence Lemmas ---

\section{Basic Recurrence Lemmas}
Let $f = \{a, \odot, b\}$ represent a basic recurrence, where $\odot$ denotes an operator (e.g., addition or multiplication), and let $c$ be a constant. The following lemmas express  properties of basic recurrences:
    
    \begin{lemma}[Addition of a Constant]
    For a constant $c$ added to a basic recurrence:
    \[
    c + \{a, +, b\} \implies \{c + a, +, b\}.
    \]
    \end{lemma}
    
    \begin{lemma}[Multiplication by a Constant]
    For a constant $c$ multiplied with a basic recurrence:
    \[
    c \cdot \{a, +, b\} \implies \{c \cdot a, +, c \cdot b\}.
    \]
    \end{lemma}
    
    \begin{lemma}[Exponentiation by a Constant]
    For a constant $c$ raised to the power of a basic recurrence:
    \[
    c^{\{a, +, b\}} \implies \{c^a, \cdot, c^b\}.
    \]
    \end{lemma}
    
    \section*{Combining Basic Recurrences}
    
    Let $f = \{a, \odot, b\}$ and $g = \{c, \odot, d\}$ represent two basic recurrences. It can be proven that:
    
    \begin{lemma}[Addition of Two Recurrences]
    For the addition of two basic recurrences:
    \[
    \{a, +, b\} + \{c, +, d\} \implies \{a + c, +, b + d\}.
    \]
    \end{lemma}
    
    \begin{lemma}[Multiplication of Two Recurrences]
    For the multiplication of two basic recurrences:
    \[
    \{a, +, b\} \cdot \{c, +, d\} \implies \{ac, +, \{a, \odot, b\}d + \{c, \odot, d\}b + bd\}.
    \]
    \end{lemma}
    
    \section*{Chain of Recurrences}    
    A chain of recurrences (CR) is defined recursively as a function over $N$ with
     constants $\{c_0, c_1, \dots, c_{k-1}\}$, a function $f_k$, and an operator $\odot$ (either $+$ or $\cdot$). 
    
    \begin{lemma}[Chain of Recurrences]
    A chain of recurrences $\phi(i)$ is defined as:
    \[
    \phi(i) = \{c_0, \odot, \{c_1, \odot, c_2, \dots, \odot, f_k\}\}(i).
    \]
    \end{lemma}
    
    In practice, the nested braces are omitted, and the chain of recurrences
    simply written out as (LLVM SCEV prints out in same format):
    \[
    \phi = \{c_0, \odot, c_1, \odot, c_2, \dots, \odot, f_k\}.
    \]
    
    Consider the CR $\phi = \{7, +, 6, +, 10, +, 6\}$. This CR can be simplified to the closed expression:
    \[
    \phi(i) = i^3 + 2i^2 + 3i + 7.
    \]
    
    
This closed expression can be derived by applying the recurrence lemmas outlined
above and is described in detail in \cite{ZimaRealPaper, Zima, OlafRecurrence}.
The application of chains of recurrences in LLVM, including their usage for 
induction variables, is discussed in \cite{scev_j}.
    
    % ---  Tensor Evolution   ---
    \section{Tensor Evolution}
    Both Scalar Evolution and Tensor Evolution are founded on the principle of `compact representation',
     which enables computational reducibility \cite{Wolfram}—the ability to compute the value at a distant
      iteration point directly, without sequentially processing each iteration of the loop.
    
    \subsection{Code Example}
    Consider the following code -    
    \begin{verbatim}
        def forward(a: torch.Tensor, x: torch.Tensor) -> torch.Tensor:
            for _ in range(15):
                x = a + x 
            return x
    \end{verbatim}
    
    
    By analyzing the code and understanding the loop's behavior, we observe that the
     return (final) value is  $a+15x$, where $x$ is the initial input tensor value.
      A tensor can be viewed as a collection of scalar values. Thus, we can see the 
      connection between what we refer to as \textit{Tensor Evolution} and the concepts discussed
       in the previous section—\textit{Scalar Evolution} and \textit{Chain of Recurrences}.

       
    % -- Tensor Evolution - Basic Formulation ---
    \subsection{Basic Formulation}    
    \begin{figure}[h!]
    \centering
    \includegraphics[width=0.75\linewidth]{evol_basic.png}
    \caption{Basic Tensor Evolution}
    \label{fig:basic_TEV}
    \end{figure}
 

Given a constant or loop-invariant tensor $T_c$,  a function $\tau_1$ over the natural
numbers $N$ that produces a tensor of same shape as $T_c$, and an element-wise operator
$\odot$, then the basic evolution of the tensor value represented by tuple
$\tau=\{T_c, \odot, \tau_1\}$ is defined as a function $\tau(i)$ over $N$
(see Fig. \ref{fig:basic_TEV}):
    
\begin{equation}
\{T_c, \odot, \tau_1\}(i) = T_c \odot \tau_1(1) \odot \tau_1(2) ... \odot \tau_1(i)
\end{equation}
    
    For instance, given a zeros tensor $T_0$, a ones tensor $T_1$, and the operator `+'
     representing tensor addition, the tensor evolution expression $\tau=\{T_0, +, T_1\}$
      produces the sequence of tensors $T_0$, $T_1$, ..., $T_k$ for subsequent values
       of $\tau(i)$. Note in this case, the function $\tau_1(i) = T_i, \forall i \in N$.
    
    Next, given loop-invariant tensors $T_{c_0}, T_{c_1}, T_{c_2}, ...,T_{c_{i-1}}$,
     a function $\tau_k$ defined over $N$, and operators $\odot_{1}, \odot_{2}, ..., \odot_k$,
      a chain of evolution of the tensor value is represented by the tuple 
    
    \begin{equation}
    \tau = \{T_{c_0}, \odot_{1},T_{c_1}, \odot_{2}, ..., \odot_{k}, \tau_{k} \}    
    \end{equation}
    
    which is defined recursively as 
    \begin{equation}
    \tau(i) = \{T_{c_0}, \odot_{1}, \{T_{c_1}, \odot_{2}, ..., \odot_{k}, \tau_{k} \} \}(i)
    \end{equation}
    
    The operators $\odot_{i}$ could all be the same (e.g., $+$ or $*$) or different. Additionally,
     if we drop the braces and understand that the application of evolution proceeds 
     from right to left, the chain of evolution can be rewritten simply as: 
    
    \begin{equation}
    \tau(i) = \{T_{c_0}, \odot_{1}, T_{c_1}, \odot_{2}, ..., \odot_{k}, \tau_{k} \}(i)
    \end{equation}
    
    % -- Tensor Evolution Lemmas --
    \section{Tensor Evolution Lemmas}
    Next, we develop a set of lemmas for Tensor Evolution. These lemmas are algebraic 
    properties (re-write rules) designed to simplify computations involving tensor evolution expressions.
    
    \iffalse
    \begin{figure}[h!]
    \centering
      \includegraphics[scale=0.75]{images/evol_add_basic.png}
      \caption{Adding a loop invariant tensor `K' to a TeV expression}
    \end{figure}
    \fi

    % ADD CONSTANT TO TeV    
    % \setcounter{lemma}{0}
    \begin{lemma}{Add or multiply TeV by a loop invariant tensor}
    \label{lem::evol_add_basic}
    \newline Let $\{A,\odot,\tau\}$ be a tensor evolution expression where $\odot$ is either element wise add, subtract or multiply (see Fig. \ref{lem::evol_add_basic}).
    And, let $K$ be a loop invariant tensor. Then, 
    \[
    K+\{A, +, \tau \} \Rightarrow \{K+A, +, \tau \}
    \]
    \[
    K *\{A,+,\tau\} \Rightarrow \{K*A,+,K*\tau \}
    \]
    \end{lemma}
    
    
    % RESHAPE LEMMA
    \begin{lemma}{Reshape of tensor evolution expression}
    \label{lem::Reshape}
    \newline Let $\{A,\odot,\tau\}$ be a tensor evolution expression where $\odot$ is an element-wise operator.
    Let $\mathcal{R}(A)$ represent the reshaping of the tensor 
    A, which includes the permutation of dimensions and the ordinary transpose.
    Then,
    \[
     \mathcal{R}(\{A, \odot, \tau \}) 
     \Rightarrow
     \{\mathcal{R}(A), \odot, \mathcal{R}(\tau)\}
    \]
    \end{lemma}
    
    % SLICE LEMMA
    
    
    \begin{lemma}{Slice of tensor evolution expression}
    \label{lem::Slice}
    \newline Let $\{A,\odot,\tau\}$ be a tensor evolution expression where $\odot$ is an element-wise operator.
    Let $\mathcal{S}(A)$ represent slicing of the tensor, where slicing selects a subset of elements to form the resulting tensor.
    Then,
    \[
     \mathcal{S}(\{A, \odot, \tau \}) 
     \Rightarrow 
     \{\mathcal{S}(A), \odot, \mathcal{S}(\tau)\}
    \]
    \end{lemma}
    
    % CONCAT LEMMA
    
    \begin{lemma}{Concatenation of TeV expressions}
    \label{lem::Concat}
    \newline Let $\{A,\odot,\tau_1\}$ and $\{B,\odot,\tau_2\}$ be two tensor evolution expressions
    where $\odot$ is an element-wise operator.
    Let $\mathcal{C}(T_1,T_2)$ represent concatenation of two tensors.
    Then,
    \[
     \mathcal{C}(\{A,\odot,\tau_1\}, \{B,\odot,\tau_2\})
     \Rightarrow 
     \{\mathcal{C}(A,B),\odot,\mathcal{C}(\tau_1,\tau_2)\} 
    \]
    \end{lemma}
    
    % ADD TWO TeVS TO SSA VALUE    
    \begin{figure}[h!]
    \centering
      \includegraphics[width=0.75\linewidth]{evol_add_chain.png}
      \caption{Adding two TeV expressions}
      \label{fig::evol_add_chain}
    \end{figure}
    

    
    \begin{lemma}{Adding two TeV Expressions}
    \label{lem::evol_add_chain}
    \newline Let $\{A,\odot,\tau_1\}$ and $\{B,\odot,\tau_2\}$ be two tensor evolution expressions (see Fig. \ref{fig::evol_add_chain}). Then, 
    \[
    \{A, +, \tau_1 \} + \{B, +, \tau_2\} \Rightarrow \{A+B, +, \tau_1 + \tau_2\}
    \]
    \end{lemma}
    
    % Multiple TWO TeVS 
    
    \begin{lemma}{Multiply two TeV Expressions}
    \label{lem::evol_mul_chain}
    \newline Let $\{A,\odot,\tau_1\}$ and $\{B,\odot,\tau_2\}$ be two tensor evolution expressions. Then, 
    \[
    \{A, +, \tau_1 \} * \{B, +, \tau_2\} \Rightarrow \big\{AB, +, \tau_1\{B, +, \tau_2\} + \tau_2\{A, +, \tau_1 \} + \tau_1\tau_2\big\}
    \]
    \end{lemma}
    
    % ---    Evolution of chain  ---
   
    \begin{lemma}{TeV input to a loop variant tensor}
    \label{lem::evol_mem_chain}
    \newline Let $X$ be a loop variant tensor with an initial value of $A$. At each iteration,
     its value evolves as $A + \{B,\odot_2,\tau_2\}$, meaning the TeV $\{B,\odot_2,\tau_2\}$ is 
     added to the current value of $X$. This is represented as 
    $\{B, +, \tau_2\} \xrightarrow[+]{} A$. Then,
    \newline
    \[
    \{B, +, \tau_2\} \xrightarrow[+]{} A \Rightarrow \{A, +, B, +, \tau\}
    \]
    In other words, this results in a chain of evolution of depth two.
    \end{lemma}
    
    There are additional lemmas based on variations in composition and operators beyond those
     discussed above. However, the lemmas presented so far are already sufficient to address
      a range of interesting problems.
    % ---  Table of  TeV Lemmas  ---
    \begin{table}[h!]
    \centering
    \begin{tabular}{|c|c|c|}
    \hline
    \textbf{Operator} & \textbf{Tensor Expression} & \textbf{Rewrite-rule} \\ \hline
    reshape & $\mathcal{R}(\{A, +, \tau\})$ & $\{\mathcal{R}(A), +, \mathcal{R}(\tau)\}$ \\ 
            & $\mathcal{R}(\{A, \ast, \tau\})$ & $\{\mathcal{R}(A), \ast, \mathcal{R}(\tau)\}$ \\ \hline
    slice & $\mathcal{S}(\{A, +, \tau\})$ & $\{\mathcal{S}(A), +, \mathcal{S}(\tau)\}$ \\ 
          & $\mathcal{S}(\{A, \ast, \tau\})$ & $\{\mathcal{S}(A), \ast, \mathcal{S}(\tau)\}$ \\ \hline
    broadcast & $\mathcal{B}(\{A, +, \tau\})$ & $\{\mathcal{B}(A), +, \mathcal{B}(\tau)\}$ \\ 
              & $\mathcal{B}(\{A, \ast, \tau\})$ & $\{\mathcal{B}(A), \ast, \mathcal{B}(\tau)\}$ \\ \hline
    concat & $\mathcal{C}(\{\{A, +, \tau_1\}, \{B, \odot, \tau_2\}\})$ & $\{\mathcal{C}(A, B), +, \mathcal{C}(\tau_1, \tau_2)\}$ \\ 
           & $\mathcal{C}(\{\{A, \ast, \tau_1\}, \{B, \odot, \tau_2\}\})$ & $\{\mathcal{C}(A, B), \ast, \mathcal{C}(\tau_1, \tau_2)\}$ \\ \hline
    log & $\log(\{A, \ast, \tau\})$ & $\{\log(A), +, \log(\tau)\}$ \\ \hline
    exponent & $\exp(\{A, +, \tau\})$ & $\{\exp(A), \ast, \exp(\tau)\}$ \\ \hline
    add const & $K + \{A, +, \tau\}$ & $\{K + A, +, \tau\}$ \\ \hline
    multiply by const & $K \ast \{A, +, \tau\}$ & $\{K \ast A, +, K \ast \tau\}$ \\ \hline
    add two TeV & $\{A, +, \tau_1\} + \{B, +, \tau_2\}$ & $\{A + B, +, \tau_1 + \tau_2\}$ \\ \hline
    inject TeV into TeV & $\{B, +, \tau_2\} \to A$ & $\{A, +, B, +, \tau\}$ \\ \hline
    \end{tabular}
    \caption{Useful Lemmas (re-write Rules) for Tensor Evolution Evaluation.}
    \label{table:tev}
    \end{table}
    
    \section{Application of Tensor Evolution}
    
    Consider the following code which operates on tensors -
    \begin{verbatim}
        def forward(a: torch.Tensor, x: torch.Tensor) -> torch.Tensor:
          y = torch.zeros_like(x)
          for _ in range(15):
            x = x + a
            z = x[1,:]
            y = y + z
          return y
    \end{verbatim}
    The interpretation of how the tensor values are evolving across iterations of the loop is shown in \ref{fig:TEV_Code_Example}. 
    
    \begin{figure}[h!]
    \centering
      \includegraphics[width=\linewidth]{simplify_ex.png}
      \caption{Calculate tensor evolution expressions of given code.}
      \label{fig:TEV_Code_Example}
    \end{figure}
    

    Now, using the re-write rules of Tensor Evolution that were laid out before, we can construct and simplify the TeV expressions.
    
    \begin{equation}
    \begin{cases}
    Y_k = \left\{  Y_0, +, \mathcal{S}( \left\{ X_0, +, A\right\})    \right\}_k \\
    Y_k= \left\{  Y_0, +,  \left\{\mathcal{S}(X_0), +, \mathcal{S}(A)\right\}    \right\}_k \\
    Y_k= \left\{  Y_0, +,  \mathcal{S}(X_0), +, \mathcal{S}(A) \right\}_k \\
    Y_k= Y_0 + k * \mathcal{S}(X_0) + k*(k+1)/2 *  \mathcal{S}(A)    
    \end{cases}
    \end{equation}
    
    Based on the above simplifications (rewrites) the same code can be re-written now as 
    \begin{verbatim}
    def forward(self, a: torch.Tensor, x:torch.Tensor)->torch.Tensor: 
      y = torch.zeros_like(x)
      return y + 15*x[1,:] + 15*(15+1)/2*a[1,:]
    \end{verbatim}
    
    \section{Background Art and Related Work}
    Tensor Comprehensions (TC) \cite{TensorComprehension} is another representation for
     computations on tensors, based on generalized Einstein notation. For example, let $X$ and $Y$
      be two input tensors, and $Z$ the output tensor.  The TC statement $Z(r) += X(r,c) * Y(c)$ 
      introduces two index variables, $r$ and $c$. Their ranges are inferred from their indexing 
      of tensors $X$ and $Y$. Since $c$ appears only on the right-hand side, the computation reduces
       over $c$ when storing values into $Z$. TC provides a compact representation of computations. 
       However, it does not address the \textit{computational reducibility} aspect of the problem.
        In contrast, SCEV and TeV include algebraic rewrite rules that simplify computations. 
        While not all computations are reducible, areas of reducibility often exist in general code,
         and both TeV and SCEV are specifically designed to identify and simplify these reducible computations.
    
    As mentioned earlier, Tensor Evolution is based on the concept of Chain of Recurrences (CR).
     The paper by \cite{Zima} on CR develops rewrite rules for long chains and complex operations,
      including trigonometric functions and exponents of exponent operations. Compilers such as LLVM 
      and GCC implement CR as Scalar Evolution (SCEV), but only for simple operations like addition
       and subtraction, and limited to two levels of recurrence. While extending beyond this is 
       mathematically possible, it is not practical for general-purpose compilers and has limited
        usability in real-world applications.
     
    Many Domain-Specific Languages (DSLs) and compilers have been developed for machine 
    learning (ML) and high-performance computing (HPC). DSLs leverage domain-specific constructs
     to capture parallelism, locality, and other application-specific characteristics. 
     An example is Halide \cite{HALIDE}, a DSL for image processing. Halide defines its inputs
      as images over an infinite range, whereas Tensor Comprehensions (TC) sets fixed sizes for
       each dimension using range inference. In ML applications, most computations are performed 
       on fixed-size tensors with higher temporal locality than images. Additionally, TC is less
        verbose than Halide, as Halide carries the syntactic burden of pre-declaring stage names 
        and free variables.
    
    The polyhedral framework is another abstraction used for the analysis and transformation of loop nests.
     One of the first high-level language polyhedral compilers was the R-Stream 3.0 Compiler \cite{Lethin2008RStream3C}
      and several tools and libraries have been developed to harness its benefits, such as GCC (Graphite) and LLVM
       (Polly). Polyhedral techniques have also been adapted for domain-specific purposes. Examples include 
       PolyMage \cite{Polymage}, designed for image processing pipelines, and PENCIL \cite{PENCIL}, 
       an approach for constructing parallelizing compilers for DSLs.


    \section{Conclusion}       
    In this paper, we presented a new mathematical framework—Tensor Evolution (TeV).
     TeV extends the theory of Chain of Recurrences (CR) to the evolution of tensor values
      within loops found in machine learning (ML) and high-performance computing (HPC) graphs.
       While CR was originally developed for scalar values and successfully implemented in LLVM and GCC as Scalar Evolution (SCEV), TeV introduces a novel approach by generalizing these concepts to tensors.
    We demonstrated the usefulness of TeV in specific contexts through illustrative code examples.
     The mathematical framework we propose has potential applications in general ML and HPC
      scenarios when further extended and applied creatively. For example, SCEV is widely
       used in optimizations like loop strength reduction (LSR) and loop-exit value computation.
        Similarly, TeV could be leveraged in cases where parts of a tensor evolve predictably,
         enabling the effective application of recurrence lemmas to optimize computations.

\bibliography{reference}{}
\bibliographystyle{unrst}
    
\end{document}

\bibliography{main}
\bibliographystyle{tmlr}

\section{Proof of Theorem~\ref{theo:main}}
\label{app:proof_theo_main}
%
Firstly, we show that the conditional distribution
%
\begin{equation}
    T(\bm{Y}) \mid
    \left\{
    \mathcal{N}_{\bm{Y}} = \mathcal{N}_{\bm{y}},
    \mathcal{K}_{\bm{Y}} = \mathcal{K}_{\bm{y}},
    \mathcal{S}_{\bm{Y}} = \mathcal{S}_{\bm{y}},
    \mathcal{Q}_{\bm{Y}} = \mathcal{Q}_{\bm{y}}
    \right\}
\end{equation}
%
is a truncated normal distribution.
%
Let we define the two vectors $\bm{a},\bm{b}\in\mathbb{R}^{(n+1)d}$ as $\bm{a}=\mathcal{Q}_{\bm{y}}$ and $\bm{b}=\tilde{\Sigma}\bm{\eta}/\bm{\eta}^\top\tilde{\Sigma}\bm{\eta}$, respectively.
%
Then, from the condition on $\mathcal{Q}_{\bm{Y}}=\mathcal{Q}_{\bm{y}}$, we have
%
\begin{equation}
    \mathcal{Q}_{\bm{Y}} = \mathcal{Q}_{\bm{y}} \Leftrightarrow
    \left(
    I_{(n+1)d} -
    \frac{\tilde{\Sigma}\bm{\eta}\bm{\eta}^\top}{\bm{\eta}^\top\tilde{\Sigma}\bm{\eta}}
    \right)
    \bm{Y} =
    \mathcal{Q}_{\bm{y}}
    \Leftrightarrow
    \bm{Y} = \bm{a} + \bm{b}z,
\end{equation}
%
where $z=T(\bm{Y})=\bm{\eta}^\top\bm{Y}\in\mathbb{R}$. Thus, we have
%
\begin{align}
      &
    \{
    \bm{Y}\in\mathbb{R}^{(n+1)d}\mid
    \mathcal{N}_{\bm{Y}} = \mathcal{N}_{\bm{y}},
    \mathcal{K}_{\bm{Y}} = \mathcal{K}_{\bm{y}},
    \mathcal{S}_{\bm{Y}} = \mathcal{S}_{\bm{y}},
    \mathcal{Q}_{\bm{Y}} = \mathcal{Q}_{\bm{y}}
    \}  \\
    = &
    \{
    \bm{Y}\in\mathbb{R}^{(n+1)d}\mid
    \mathcal{N}_{\bm{Y}} = \mathcal{N}_{\bm{y}},
    \mathcal{K}_{\bm{Y}} = \mathcal{K}_{\bm{y}},
    \mathcal{S}_{\bm{Y}} = \mathcal{S}_{\bm{y}},
    \bm{Y} = \bm{a} + \bm{b}z, z\in\mathbb{R}
    \}  \\
    = &
    \{
    \bm{a} + \bm{b}z\in\mathbb{R}^{(n+1)d}\mid
    \mathcal{N}_{\bm{a}+\bm{b}z} = \mathcal{N}_{\bm{y}},
    \mathcal{K}_{\bm{a}+\bm{b}z} = \mathcal{K}_{\bm{y}},
    \mathcal{S}_{\bm{a}+\bm{b}z} = \mathcal{S}_{\bm{y}},
    z\in\mathbb{R}
    \}  \\
    = &
    \{
    \bm{a} + \bm{b}z\in\mathbb{R}^{(n+1)d}\mid
    z\in \mathcal{Z}
    \},
\end{align}
%
where truncated interval $\mathcal{Z}$ is defined as
%
\begin{equation}
    \mathcal{Z} =
    \{
    z\in\mathbb{R}\mid
    \mathcal{N}_{\bm{a}+\bm{b}z} = \mathcal{N}_{\bm{y}},
    \mathcal{K}_{\bm{a}+\bm{b}z} = \mathcal{K}_{\bm{y}},
    \mathcal{S}_{\bm{a}+\bm{b}z} = \mathcal{S}_{\bm{y}}
    \}.
\end{equation}
%
Therefore, we obtain
%
\begin{equation}
    T(\bm{Y}) \mid
    \{
    \mathcal{N}_{\bm{Y}} = \mathcal{N}_{\bm{y}},
    \mathcal{K}_{\bm{Y}} = \mathcal{K}_{\bm{y}},
    \mathcal{S}_{\bm{Y}} = \mathcal{S}_{\bm{y}},
    \mathcal{Q}_{\bm{Y}} = \mathcal{Q}_{\bm{y}}
    \}
    \sim
    \mathrm{TN}(\bm{\eta}^\top\bm{\mu}, \bm{\eta}^\top\tilde{\Sigma}\bm{\eta}, \mathcal{Z}),
\end{equation}
%
which is the truncated normal distribution with the mean $\bm{\eta}^\top\bm{\mu}$, the variance $\bm{\eta}^\top\tilde{\Sigma}\bm{\eta}$, and the truncation intervals $\mathcal{Z}$.
%

From the above result, we can compute the selective $p$-value as defined in Eq.~\eqref{eq:selective_pvalue} by using the truncated normal distribution.
%
Therefore, by probability integral transformation, under the null hypothesis, we have
%
\begin{equation}
    p_\mathrm{selective} \mid
    \{
    \mathcal{N}_{\bm{Y}} = \mathcal{N}_{\bm{y}},
    \mathcal{K}_{\bm{Y}} = \mathcal{K}_{\bm{y}},
    \mathcal{S}_{\bm{Y}} = \mathcal{S}_{\bm{y}},
    \mathcal{Q}_{\bm{Y}} = \mathcal{Q}_{\bm{y}}
    \}
    \sim
    \mathrm{Unif}(0, 1),
\end{equation}
%
which leads to
%
\begin{equation}
    \mathbb{P}_{\mathrm{H}_0}
    \left(
    p_\mathrm{selective} \leq \alpha \mid
    \mathcal{N}_{\bm{Y}} = \mathcal{N}_{\bm{y}},
    \mathcal{K}_{\bm{Y}} = \mathcal{K}_{\bm{y}},
    \mathcal{S}_{\bm{Y}} = \mathcal{S}_{\bm{y}},
    \mathcal{Q}_{\bm{Y}} = \mathcal{Q}_{\bm{y}}
    \right)
    =\alpha,\
    \forall\alpha\in(0,1).
\end{equation}
%
For any $\alpha\in(0,1)$, by marginalizing over all the values of the nuisance parameters, we obtain
%
\begin{align}
      &
    \mathbb{P}_{\mathrm{H}_0}
    \left(
    p_\mathrm{selective} \leq \alpha \mid
    \mathcal{N}_{\bm{Y}} = \mathcal{N}_{\bm{y}},
    \mathcal{K}_{\bm{Y}} = \mathcal{K}_{\bm{y}},
    \mathcal{S}_{\bm{Y}} = \mathcal{S}_{\bm{y}}
    \right)                                                                                                       \\
    = &
    \begin{multlined}
        \int_{\mathbb{R}^{n^\prime}}
        \mathbb{P}_{\mathrm{H}_0}
        \left(
        p_\mathrm{selective} \leq \alpha \mid
        \mathcal{N}_{\bm{Y}} = \mathcal{N}_{\bm{y}},
        \mathcal{K}_{\bm{Y}} = \mathcal{K}_{\bm{y}},
        \mathcal{S}_{\bm{Y}} = \mathcal{S}_{\bm{y}},
        \mathcal{Q}_{\bm{Y}} = \mathcal{Q}_{\bm{y}}
        \right) \\
        \mathbb{P}_{\mathrm{H}_0}
        \left(
        \mathcal{Q}_{\bm{Y}} = \mathcal{Q}_{\bm{y}} \mid
        \mathcal{N}_{\bm{Y}} = \mathcal{N}_{\bm{y}},
        \mathcal{K}_{\bm{Y}} = \mathcal{K}_{\bm{y}},
        \mathcal{S}_{\bm{Y}} = \mathcal{S}_{\bm{y}},
        \right)
        d\mathcal{Q}_{\bm{y}}
    \end{multlined} \\
    = & \alpha \int_{\mathbb{R}^{(n+1)d}}
    \mathbb{P}_{\mathrm{H}_0}
    \left(
    \mathcal{Q}_{\bm{Y}} = \mathcal{Q}_{\bm{y}} \mid
    \mathcal{N}_{\bm{Y}} = \mathcal{N}_{\bm{y}},
    \mathcal{K}_{\bm{Y}} = \mathcal{K}_{\bm{y}},
    \mathcal{S}_{\bm{Y}} = \mathcal{S}_{\bm{y}}
    \right)
    d\mathcal{Q}_{\bm{y}} = \alpha.
\end{align}
%
Therefore, we also obtain
%
\begin{align}
      &
    \mathbb{P}_{\mathrm{H}_0}(p_{\mathrm{selective}}\leq \alpha)                                              \\
    = &
    \sum_{\mathcal{N}_{\bm{y}}\in 2^{[n]}}
    \sum_{\mathcal{K}_{\bm{y}}\in \{0, 1\}}
    \sum_{\mathcal{S}_{\bm{y}}\in \{-1, 1\}^d}
    \begin{multlined}
        \mathbb{P}_{\mathrm{H}_0}
        (
        \mathcal{N}_{\bm{Y}} = \mathcal{N}_{\bm{y}},
        \mathcal{K}_{\bm{Y}} = \mathcal{K}_{\bm{y}},
        \mathcal{S}_{\bm{Y}} = \mathcal{S}_{\bm{y}}
        ) \\
        \mathbb{P}_{\mathrm{H}_0}
        \left(
        p_\mathrm{selective} \leq \alpha \mid
        \mathcal{N}_{\bm{Y}} = \mathcal{N}_{\bm{y}},
        \mathcal{K}_{\bm{Y}} = \mathcal{K}_{\bm{y}},
        \mathcal{S}_{\bm{Y}} = \mathcal{S}_{\bm{y}}
        \right)
    \end{multlined} \\
    = &
    \alpha
    \sum_{\mathcal{N}_{\bm{y}}\in 2^{[n]}}
    \sum_{\mathcal{K}_{\bm{y}}\in \{0, 1\}}
    \sum_{\mathcal{S}_{\bm{y}}\in \{-1, 1\}^d}
    \mathbb{P}_{\mathrm{H}_0}(\mathcal{N}_{\bm{y}}, \mathcal{K}_{\bm{y}}, \mathcal{S}_{\bm{y}})
    = \alpha.
\end{align}
%

\section{Selection Events of the Deep Learning Models}
\label{app:selection_events_of_dnn}
%
We explain the selection events regarding the deep learning model that transforms an image instance $\bm x_i \in \RR^d$ to a latent feature vector $\bm{z}_i \in \RR^{\tilde{d}}$.
%
We consider a deep learning model that consists of sequential piecewise-linear functions (e.g., convolution, ReLU activation, max pooling, and up-sampling).
% 
Obviously, the composite function of those piecewise-linear functions maintains its piecewise-linear nature.
%
Thus, within a specific real space in $\RR^d$, the deep learning model simplifies to a linear function, which can be expressed as:
%
\begin{equation}
 \cA_{\rm{DL}}(\bm{x}_i) 
 = \bm{B} + \bm{W}\bm{x}_i
 \;\;\; \text{if } \bm{x}_i \in \cP,
\end{equation}
%
where $\bm B \in \RR^{\tilde{d}}$ and $\bm W \in \RR^{\tilde{d}}$ represent the bias and weight matrices, and $\cP \subseteq \RR^d$ is a polytope where $\cA_{\rm{DL}}$ acts as a linear function.
%
The polytope can be characterized by a set of linear inequalities.
%
For details on computing these linear inequalities, see \citet{katsuoka2025si4onnx}.
% 
Let the selection event denote the set of polytopes for all instances in $\bm{Y}$:
%
\begin{equation}
 \cD_{\bm{Y}} := \{\cP \mid \bm{X}_i \in \bm{Y}, \bm{X}_i \in \cP\}.
\end{equation}
% 
For $k$NNAD using feature representations from the deep learning model, we can compute the selective $p$-value by adding the conditioning $\cD_{\bm{Y}} = \cD_{\bm{y}}$ into Eq.\eqref{eq:selective_pvalue}, as follows:
%
\begin{equation}
 p_{\rm selective}
 :=
 \PP_{\rm H_0}
 \left(
 T(\bm Y) \ge T(\bm y)
 \middle|
 \cD_{\bm{Y}} = \cD_{\bm{y}},
 \cN_{\bm{Y}} = \cN_{\bm{y}},
 \cK_{\bm{Y}} = \cK_{\bm{y}},
 \cS_{\bm{Y}} = \cS_{\bm{y}},
 \cQ_{\bm{Y}} = \cQ_{\bm{y}}
 \right).
\end{equation}

\section{Details of the Experiments}
\label{app:appC}

\subsection{Additional Type I Error Rate Results}
\label{app:appC1}
%
We also conducted experiments to investigate the type I error rate when the data dimension $d$ and the number of neighbors $k$ were changed.
%
Specifically, we changed $d \in \{1, 2, 5, 10\}$ and $k \in \{1, 2, 5, 10\}$, while setting the default parameters as $n=100$, $d=2$, and $k=1$.
% Furthermore, we conducted additional experiments where $d$ was changed, considering the case where $k \in \{1, 2, 5, 10\}$ was adaptively selected in a data-driven manner.
\red{
In addition, experiments with changing $d$ were also considered the case where $k$ was selected adaptively from $\{1, 2, 5, 10\}$ in a data-driven manner.
}
%
In all cases, we generated the datasets in the same way as \red{in the experiments on synthetic datasets} (\S{\ref{subsec:experiment_of_synthetic_data}}), and the results are shown in Figure \ref{fig:fpr_d}.
%
\begin{figure}[htbp]
  \begin{minipage}[b]{0.32\linewidth}
      \centering
      \includegraphics[width=0.98\linewidth]{Fig/fpr/zinkou_knn_d.pdf}
      \subcaption{Data Dimension}
  \end{minipage}
  \begin{minipage}[b]{0.32\linewidth}
      \centering
      \includegraphics[width=0.98\linewidth]{Fig/fpr/zinkou_knn_k.pdf}
      \subcaption{Number of Neighbors}
  \end{minipage}
  \begin{minipage}[b]{0.32\linewidth}
    \centering
    \includegraphics[width=0.98\linewidth]{Fig/fpr/zinkou_knn_kchoose_d.pdf}
    \subcaption{Adaptively Selected $k$}
\end{minipage}
  
  \caption{
      Type I error rate when changing the data dimension $d$ and the number of neighbors $k$.
      %
      Our proposed method (\texttt{Stat-kNNAD}), the ablation study (\texttt{w/o-pp}), and the Bonferroni method (\texttt{bonferroni}) successfully control the type I error rate across all settings.
      %
      \red{However, the naive method (\texttt{naive}) fails.
      %
      The results of the \texttt{bonferroni} are almost zero, because it is too conservative.
      }
  }
  \label{fig:fpr_d}
\end{figure}
%


\subsection{Additional Power Results}
\label{app:appC2}
%
We also conducted experiments to investigate the power when \red{the number of training data $n$}, the data dimension $d$ and the number of neighbors $k$ are changed.
%
\red{
We changed $n \in \{100, 200, 500, 1000\}$, $d \in \{1, 2, 5, 10\}$ and $k \in \{1, 2, 5, 10\}$  while setting the default parameters as $n=100$, $d=2$, $k=1$ and signal strength $\delta=5$.
}
%
Furthermore, we conducted additional experiments where $n$ and $d$ was changed, considering the case where $k$ was adaptively selected from $\in \{1, 2, 5, 10\}$ in a data-driven manner.
%
In all cases, we generated the datasets in the same way as \red{in the experiments on synthetic datasets} (\S{\ref{subsec:experiment_of_synthetic_data}}), and the results are shown in Figure~\ref{fig:power_fixed_k} and Figure~\ref{fig:power_adaptive_k}.


\begin{figure}[H]
  \centering
  \begin{minipage}[b]{0.32\linewidth}
      \centering
      \includegraphics[width=0.98\linewidth]{Fig/power/zinkou_knn_pwr_n.pdf}
      \subcaption{Datasize }
  \end{minipage}
  \begin{minipage}[b]{0.32\linewidth}
      \centering
      \includegraphics[width=0.98\linewidth]{Fig/power/zinkou_knn_pwr_d.pdf}
      \subcaption{Data Dimension }
  \end{minipage}
  \begin{minipage}[b]{0.32\linewidth}
      \centering
      \includegraphics[width=0.98\linewidth]{Fig/power/zinkou_knn_pwr_k.pdf}
      \subcaption{Number of Neighbors}
  \end{minipage}
  \caption{
      Power for a fixed number of neighbors $k$. 
      %
      The results show the effect of changing the training dataset size $n$, the data dimension $d$, and $k$. 
      %
      Our proposed method (\texttt{Stat-kNNAD}) outperformed other methods \red{across} all settings.
  }
  \label{fig:power_fixed_k}
\end{figure}

\begin{figure}[H]
  \centering
  \begin{minipage}[b]{0.32\linewidth}
      \centering
      \includegraphics[width=0.98\linewidth]{Fig/power/zinkou_knn_pwr_kchoose_n.pdf}
      \subcaption{Datasize}
  \end{minipage}
  \begin{minipage}[b]{0.32\linewidth}
      \centering
      \includegraphics[width=0.98\linewidth]{Fig/power/zinkou_knn_pwr_kchoose_d.pdf}
      \subcaption{Data Dimension}
  \end{minipage}
  \caption{
      Power for an adaptively selected number of neighbors $k$. 
      %
      The results show the effect of changing the training dataset size $n$ and the data dimension $d$.
      %
      Our proposed method (\texttt{Stat-kNNAD}) outperformed other methods \red{across} all settings.
  }
  \label{fig:power_adaptive_k}
\end{figure}

\subsection{Details of \red{Tabular} Datasets}
\label{app:appC3}
We used the following \red{10} real datasets from the Kaggle Repository. All datasets are licensed under the CC BY 4.0 license.

\begin{itemize}
  \item \textit{Heart}: Dataset for predicting heart attacks
  \item \textit{Money}: Dataset on financial transactions in a virtual environment
  \item \textit{Fire}: Dataset on fires in the MUGLA region in June
  \item \textit{Cancer}: Dataset related to breast cancer diagnosis
  \item \textit{Credit}: Dataset on credit card transactions
  \item \textit{Student}: Dataset related to student performance
  \item \textit{Bankruptcy}: Dataset on company bankruptcies
  \item \textit{Drink}: Dataset on the quality of drinking water
  \item \textit{Nuclear}: Dataset on pressurized nuclear reactors
  \item \textit{Network}: Dataset on anomaly detection in virtual network environments
\end{itemize}

%
\subsection{\red{Experimental Results on Image Data Examples}}
\label{app:appC4}
\red{
  We evaluated \texttt{Stat-kNNAD} and \texttt{naive} on the 10 datasets from MVTec AD dataset.  
  %
  The datasets used in this study are \textit{Carpet}, \textit{Grid}, \textit{Leather}, \textit{Tile}, \textit{Wood}, \textit{Bottle}, \textit{Capsule}, \textit{Metal Nut}, \textit{Transistor}, and \textit{Zipper}.
  %  
  Examples from each dataset are shown in Figure~\ref{fig:mvtec_examples}.
  %
  In each example, we present patches corresponding to true negative and true positive cases, along with both the naive $p$-value and the selective $p$-value.
}


%
% Let me show the case study in the latent feature space.
%
% 10 kinds of real data examples from MVtec dataset are shown, along with naive $p$ and selective $p$ values.
% %
% As well as the real data Experiments results in~\ref{subsec:experiment_of_image_data}, 'Carpet', 'Grid', 'Leather', 'Tile', 'Wood', 'Bottle', 'Capsule', 'Metal Nut', 'Transistor' and 'Zipper' data examples.
% %
% In each data example, we show true negative and true positive patches with both the naive $p$ value and the selective $p$ value.
%

\begin{figure}[H]
    \centering
    {
    \begin{minipage}[b]{0.45\linewidth}
        \centering
        \subcaption*{
        \centering
        \textit{Carpet}}
        \includegraphics[width=0.6\linewidth]{Fig/appC2/patch_example_carpet.pdf}
        % \subcaption*{$p_{\text{naive}}=0.011$, $p_{\text{selective}}=0.25$, $p_{\text{naive}}=0.001$, $p_{\text{selective}}=0.022$ }
        \subcaption*{
                    Normal Example (Left):  $p_{\text{naive}}=0.011$, $p_{\text{selective}}=0.250$ \\ 
                    Anomaly Example (Right): $p_{\text{naive}}=0.001$, $p_{\text{selective}}=0.022$ }

      \end{minipage}
      \begin{minipage}[b]{0.45\linewidth}
        \centering
        \subcaption*{
        \centering
        \textit{Grid}}
        \includegraphics[width=0.6\linewidth]{Fig/appC2/patch_example_grid.pdf}
        \subcaption*{
                Normal Example (Left): $p_{\text{naive}}=0.031$, $p_{\text{selective}}=0.491$ \\ 
                Anomaly Example (Right): $p_{\text{naive}}=0.001$, $p_{\text{selective}}=0.013$ }
    \end{minipage}
    }\\
    \vspace{0.5cm}
    {
    \begin{minipage}[b]{0.45\linewidth}
        \centering
        \subcaption*{
        \centering
        \textit{Leather}}
        \includegraphics[width=0.6\linewidth]{Fig/appC2/patch_example_leather.pdf}
        \subcaption*{
          Normal Example (Left): $p_{\text{naive}}=0.040$, $p_{\text{selective}}=0.640$\\
          Anomaly Example (Right): $p_{\text{naive}}=0.004$, $p_{\text{selective}}=0.021$ }
    \end{minipage}
    \begin{minipage}[b]{0.45\linewidth}
        \centering
        \subcaption*{
        \centering
        \textit{Tile}}
        \includegraphics[width=0.6\linewidth]{Fig/appC2/patch_example_tile.pdf}
        \subcaption*{
          Normal Example (Left): $p_{\text{naive}}=0.011$, $p_{\text{selective}}=0.309$\\
          Anomaly Example (Right): $p_{\text{naive}}=0.009$, $p_{\text{selective}}=0.046$
          }
    \end{minipage}
    }\\
    \vspace{0.5cm}
    \begin{minipage}[b]{0.45\linewidth}
        \centering
        \subcaption*{
        \centering
        \textit{Wood}}
        \includegraphics[width=0.6\linewidth]{Fig/appC2/patch_example_wood.pdf}
        \subcaption*{
          Normal Example (Left): $p_{\text{naive}}=0.028$, $p_{\text{selective}}=0.488$\\
          Anomaly Example (Right): $p_{\text{naive}}=0.001$, $p_{\text{selective}}=0.034$ }
    \end{minipage}
    \begin{minipage}[b]{0.45\linewidth}
        \centering
        \subcaption*{
        \centering
        \textit{Bottle}}
        \includegraphics[width=0.6\linewidth]{Fig/appC2/patch_example_bottle.pdf}
        \subcaption*{
          Normal Example (Left): $p_{\text{naive}}=0.027$, $p_{\text{selective}}=0.460$\\
          Anomaly Example (Right): $p_{\text{naive}}=0.003$, $p_{\text{selective}}=0.017$ }
    \end{minipage}
    % \caption{
    %   \red{
    %   Experimental results of 6 datasets from MVTec AD: \textit{Carpet}, \textit{Grid}, \textit{Leather}, \textit{Tile}, \textit{Wood}, and \textit{Bottle}.
    %   %
    %   For each dataset, one normal example (left) and one anomaly example (right) are showed.  
    %   %
    %   For each example, the top row displays the original image used for testing along with the patch location (marked in red), while the bottom row presents the extracted patch image.
    %   %
    %   For all normal examples, the naive $p$-value is below the significance level $\alpha = 0.05$ (false positive), whereas the proposed selective $p$-value correctly results in a true negative.  
    %   %
    %   For all anomaly examples, the selective $p$-value successfully detects anomalies.
    %   }
    % }
    % \label{fig:mvtec_examples_1}
\end{figure}


\addtocounter{figure}{-1}
\begin{figure}[H]
  \centering
  \begin{minipage}[b]{0.45\linewidth}
      \centering
      \subcaption*{
      \centering
      \textit{Capsule}}
      \includegraphics[width=0.6\linewidth]{Fig/appC2/patch_example_capsule.pdf}
      \subcaption*{
        Normal Example (Left): $p_{\text{naive}}=0.026$, $p_{\text{selective}}=0.505$\\
      Anomaly Example (Right): $p_{\text{naive}}=0.002$, $p_{\text{selective}}=0.047$ }
  \end{minipage}
  \begin{minipage}[b]{0.45\linewidth}
      \centering
      \subcaption*{
      \centering
      \textit{Metal Nut}}
      \includegraphics[width=0.6\linewidth]{Fig/appC2/patch_example_metalnut.pdf}
      \subcaption*{
        Normal Example (Left): $p_{\text{naive}}=0.010$, $p_{\text{selective}}=0.283$\\
      Anomaly Example (Right): $p_{\text{naive}}=0.009$, $p_{\text{selective}}=0.038$ }
  \end{minipage}
  \\
  \vspace{0.5cm}
  \begin{minipage}[b]{0.45\linewidth}
      \centering
      \subcaption*{
      \centering
      \textit{Transistor}}
      \includegraphics[width=0.6\linewidth]{Fig/appC2/patch_example_transistor.pdf}
      \subcaption*{
        Normal Example (Left): $p_{\text{naive}}=0.017$, $p_{\text{selective}}=0.585$\\
      Anomaly Example (Right): $p_{\text{naive}}=0.001$, $p_{\text{selective}}=0.015$ }
  \end{minipage}
  \begin{minipage}[b]{0.45\linewidth}
      \centering
      \subcaption*{
      \centering
      \textit{Zipper}}
      \includegraphics[width=0.6\linewidth]{Fig/appC2/patch_example_zipper.pdf}
      \subcaption*{
        Normal Example (Left): $p_{\text{naive}}=0.030$, $p_{\text{selective}}=0.471$\\
      Anomaly Example (Right): $p_{\text{naive}}=0.001$, $p_{\text{selective}}=0.048$ }
  \end{minipage}
  \caption{
    \red{
    Experimental results of 10 datasets from MVTec AD dataset.
    % \textit{Capsule}, \textit{Metal Nut}, \textit{Transistor}, and \textit{Zipper}.
    % %
    % The interpretation of this figure is the same as that of Figure~\ref{fig:mvtec_examples_1}.
    % %
    For each dataset, one normal example (left) and one anomaly example (right) are showed.  
    %
    For each example, the top row displays the original image used for testing along with the patch location (marked in red), while the bottom row presents the extracted patch image.
    %
    For all normal examples, the naive $p$-value is below the significance level $\alpha = 0.05$ (false positive), whereas the proposed selective $p$-value correctly results in a true negative.  
    %
    For all anomaly examples, the selective $p$-value successfully detects anomalies.
    }
  }
  \label{fig:mvtec_examples}
\end{figure}



\begin{figure}[H]
  \centering
  \begin{minipage}[b]{0.46\linewidth}
      \centering
      \includegraphics[width=0.98\linewidth]{Fig/normal_example.pdf}
      \subcaption{\\The naive $p$-value is 0.03\\
       (false positive)\\ 
      The selective $p$-value is 0.46\\
       (true negative)}
  \end{minipage}
  \begin{minipage}[b]{0.46\linewidth}
      \centering
      \includegraphics[width=0.98\linewidth]{Fig/anomaly_example.pdf}
      \subcaption{\\The naive $p$-value is 0.00\\
      (true positive)\\ 
      The selective $p$-value is 0.02\\
      (true positive)}
  \end{minipage}
  \caption{
    Experimental results on the \textit{Bottle} image from the MVTec AD dataset.
    %
    For the normal image (left), the conventional $p$-value falls below the significance level $\alpha = 0.05$, leading to a false positive, whereas the proposed selective $p$-value correctly indicates a true negative.
    %
    For the anomaly image (right), the proposed method accurately detects the anomaly.
    }
  \label{fig:mvtec_bottle_example}
\end{figure}



%10個の事例とそれぞれ2種類のp値を載せる。
%\caption{
%Across all 10 data examples of true negative patches, naive $p$ value are under significant level $\alpha$. On the other hand, selective $p$ value are over significant level $\alpha$.
%This shows that the proposed method correctly rejects the null hypothesis.
%}

\section{Details of the Numerical Experiments}
\label{Experimental_Details}

\subsection{Detailed descriptions of comparison methods}
\label{Detailed_descriptions_of_comparison_methods}
In our experiments, we compared the proposed method (\texttt{Proposed}) with the following methods.
\begin{itemize}
  \item \texttt{OC}: In this method, we consider $p$-values conditioned on the process of the algorithm $\mathcal{A}$. 
        The over-conditioned $p$-value is computed as
  \begin{equation}
    p_k^{\text{oc}} = \mathbb{P}_{\text{H}_0,k} \left(Z \geq z^{\text{obs}} \, | \, Z \in \mathcal{Z}^{\text{oc}}\left(\bm{a}+\bm{b}z^{\text{obs}}\right)\right). \notag
  \end{equation}
  This method is computationally efficient, however, its power is low due to over-conditioning.
  \item \texttt{OptSeg-SI-oc}~\citep{duy2020computing}: This method uses a $p$-value conditioned only on the process of dynamic programming algorithm. 
  The over-conditioned $p$-value is computed as
  \begin{equation}
    p_k^{\text{OptSeg-SI-oc}} = \mathbb{P}_{\text{H}_0,k} \left(Z \geq z^{\text{obs}} \, | \, Z \in  \left\{r \in \mathbb{R} \, | \, \mathcal{S}_{\text{DP}}(r) = \mathcal{S}_{\text{DP}}\left(z^{\text{obs}}\right)\right\}\right). \notag
  \end{equation}
  \item \texttt{OptSeg-SI}~\citep{duy2020computing}: This method removes over-conditionning from OptSeg-SI-oc, that is, 
        the $p$-value is conditioned only on the result of dynamic programming. 
  \item \texttt{Naive}: This method uses a conventional $p$-value without conditioning, which is computed as
  \begin{equation}
    p_k^{\text{naive}} = \mathbb{P}_{\text{H}_0,k} \left(Z \geq z^{\text{obs}}\right). \notag
  \end{equation}
  \item \texttt{Bonferroni}: This is a method to control the type I error rate by applying Bonferroni correction which is widely used as multiple testing correction.
        Since the number of all possible hypotheses is $m = (2^{D}-1)(T-1)$, the bonferroni $p$-value is computed by $p_k^{\text{bonferroni}} = \min(1, m \cdot p_k^{\text{naive}})$.
\end{itemize}

\subsection{Computational Time and Computer Resources}
\label{Computational_Time}
We measured the computational time of our proposed method for the synthetic data experiments presented in Section~\ref{subsec:Synthetic_Data_Experiments} 
and computed the medians for each settings. 
The results for the type I error rate and power experiments are shown in Figure~\ref{fig_time}. 
Panels (a) and (b) indicate that the computational time increases exponentially with the sequence length. 
In addition, the computational time becomes shorter as the signal intensity increases in panels (c) and (d). 
This may be because the results of hypothesis testing in the case of high intensity are more likely to be obvious, and
the inference process can be terminated early.
%the number of interval computations in the parametric programming decreases.
All numerical experiments were conducted on a computer with a 96-core 3.60GHz CPU and 512GB of memory.

\begin{figure}[H]
  \centering
  \begin{minipage}[t]{0.24\hsize}
      \centering
      \includegraphics[width=0.95\textwidth]{figure/exp/tmlr2025_fpr512_time_median.pdf}
      \captionsetup{justification=centering}
      \caption*{(a) Type I error rate \\ \text{\hspace{1em}} ($M=512$)}
  \end{minipage}
  \hfill
  \begin{minipage}[t]{0.24\hsize}
      \centering
      \includegraphics[width=0.95\textwidth]{figure/exp/tmlr2025_fpr1024_time_median.pdf}
      \captionsetup{justification=centering}
      \caption*{(b) Type I error rate \\ \text{\hspace{1em}} ($M=1024$)}
  \end{minipage}
  \begin{minipage}[t]{0.24\hsize}
    \centering
    \includegraphics[width=0.95\textwidth]{figure/exp/tmlr2025_tpr512_time_median.pdf}
    \captionsetup{justification=centering}
    \caption*{(c) Power ($M=512$)}
\end{minipage}
\hfill
\begin{minipage}[t]{0.24\hsize}
    \centering
    \includegraphics[width=0.95\textwidth]{figure/exp/tmlr2025_tpr1024_time_median.pdf}
    \captionsetup{justification=centering}
    \caption*{(d) Power ($M=1024$)}
\end{minipage}
  \caption{Computational time in the type I error rate and the power experiments.}
  \label{fig_time}
\end{figure}

\subsection{Robustness of Type I Error Rate Control}
\label{Robustness_of_Type_I_Error_Rate_Control}
We evaluated the robustness of the \texttt{Proposed} in terms of the type I error rate control. 
For this purpose, we conducted experiments under three distinct noise conditions: 
(i) unknown noise variance, 
(ii) non-Gaussian noise, and
(iii) correlated noise.
The details of each experiment are given below.

\textbf{Unknown noise variance.}
We generated 1000 null sequences in the same manner as the type I error rate experiment with known variance presented in Section~\ref{subsec:Synthetic_Data_Experiments}.
To estimate the variance $\sigma^2$ from the same data, we first applied CP candidate selection algorithm to identify the segments, and then computed the empirical variance of each segment for all frequencies. 
Since the estimated variance tended to be smaller than the true value, we adopted the maximum value for each frequency. 
Given $\hat{\sigma}_f$ as the average of the values, $\sigma$ could be estimated from the property of DFT as $\hat{\sigma} = \frac{\hat{\sigma}_f}{\sqrt{M}}$.
The results for the significance levels $\alpha = 0.01, 0.05, 0.1$ are shown in Figure~\ref{fig_fpr_est} and the \texttt{Proposed} still could properly control the type I error rate. 

\textbf{Non-Gaussian noise.}
We considered the case where the noise followed the five non-Gaussian distributions:
\begin{itemize}
  \item \texttt{skewnorm}: Skew normal distribution family.
  \item \texttt{exponnorm}: Exponentially modified normal distribution family.
  \item \texttt{gennormsteep}: Generalized normal distribution family whose shape parameter $\beta$ is limited to be steeper than the normal distribution, i.e., $\beta < 2$.
  \item \texttt{gennormflat}: Generalized normal distribution family whose shape parameter $\beta$ is limited to be flatter than the normal distribution, i.e., $\beta > 2$.
  \item \texttt{t}: Student's t distribution family.
\end{itemize}

To generate sequences used in the experiment, 
we first obtained a noise distribution such that the 1-Wasserstein distance from the standard normal distribution $\mathcal{N}(0,1)$ was $\{0.01,0.02,0.03,0.04\}$ in each aforementioned distribution family. 
Subsequently, we standardized the distribution to have a mean of 0 and a variance of 1.
Then, we generated 1000 null sequences $\bm{x}=(x_1, \dots, x_N)^\top$, 
where the mean vector was specified in the same manner as described in the type I error rate experiment with known variance, 
and the noise followed the obtained distribution, for $T=60$. 
We applyed hypothesis testing using the test statistic with $\sigma=1$ for the detected CP candidate locations.
The results for the significance levels $\alpha = 0.05$ are shown in Figure~\ref{fig_fpr_nongaussian} 
and the \texttt{Proposed} could properly control the type I error rate for all non-Gaussian distributions.

\textbf{Correlated noise.}
We generated 10000 null sequences 
$\bm{x}=(x_1, \dots, x_N)^\top \sim \mathcal{N}(\bm{s}, \Sigma)$, 
where mean vector $\bm{s}$ was specified in the same manner as described in the type I error rate experiment with known variance, 
and covariance matrix $\Sigma$ was defined as $\Sigma = \sigma^2\left(\rho^{|i-j|}\right)_{ij} \in \mathbb{R}^{N \times N}$ with $\sigma=1$ and $\rho \in \{0.025, 0.05, 0.075, 0.1\}$, for $T=60$.
After CP candidate selection, we conducted the hypothesis testing using the test statistic with $\sigma=1$. 
The results for the significance levels $\alpha = 0.01, 0.05, 0.1$ are shown in Figure~\ref{fig_fpr_cov}.
For weak covariance, the \texttt{Proposed} could properly control the type I error rate. 
However, the type I error rate could not be controlled with increasing noise correlation. %when the noise is highly correlated.
This remains a challenge for future work. 

\subsection{More Results on Real Data Experiments}
\label{More_Results_on_Real_Data_Experiment}
Additional results for the signals of bearing~2, 3, and 4 
before and after the anomaly of bearing~1 occurred
are shown in Figures~\ref{fig_bearing2}, \ref{fig_bearing3}, and \ref{fig_bearing4}. 
In panel~(a) of each figure, the time variation of frequency spectra where CP candidate locations were falsely detected are shown for the period of 0.25--2.25~days when the frequency anomaly in bearing~1 did not actually exist. 
The results indicate that the inferences using $p$-values of the \texttt{Proposed} and \texttt{OC} are valid.
In panel~(b), the time variations of the BPFO harmonics where CP candidate locations were correctly detected are presented for the period of 4--6~days (bearing~2), 4.75--6.75~days (bearing~3), and 4--6~days (bearing~4), respectively.
In these cases, although the detection was delayed by several days relative to the occurrence of the frequency anomaly in bearing~1, 
the outer race fault signatures were successfully identified in all bearings using the \texttt{Proposed}.

\begin{figure}[H]
  \centering
  \begin{minipage}[t]{0.45\hsize}
      \centering
      \includegraphics[width=0.95\textwidth]{figure/exp/tmlr2025_est_fpr512.pdf}
      \caption*{(a) $M=512$}
  \end{minipage}
  \hfill
  \begin{minipage}[t]{0.45\hsize}
      \centering
      \includegraphics[width=0.95\textwidth]{figure/exp/tmlr2025_est_fpr1024.pdf}
      \caption*{(b) $M=1024$}
  \end{minipage}
  \caption{Robustness of type I error control for estimated variance.}
  \label{fig_fpr_est}
\end{figure}

\begin{figure}[H]
  \centering
  \begin{minipage}[t]{0.45\hsize}
      \centering
      \includegraphics[width=0.95\textwidth]{figure/exp/tmlr2025_nongaussian_fpr512.pdf}
      \caption*{(a) $M=512$}
  \end{minipage}
  \hfill
  \begin{minipage}[t]{0.45\hsize}
      \centering
      \includegraphics[width=0.95\textwidth]{figure/exp/tmlr2025_nongaussian_fpr1024.pdf}
      \caption*{(b) $M=1024$}
  \end{minipage}
  \caption{Robustness of type I error control for non-Gaussian noise.}
  \label{fig_fpr_nongaussian}
\end{figure}

\begin{figure}[H]
  \centering
  \begin{minipage}[t]{0.45\hsize}
      \centering
      \includegraphics[width=0.95\textwidth]{figure/exp/tmlr2025_cov_fpr512.pdf}
      \caption*{(a) $M=512$}
  \end{minipage}
  \hfill
  \begin{minipage}[t]{0.45\hsize}
      \centering
      \includegraphics[width=0.95\textwidth]{figure/exp/tmlr2025_cov_fpr1024.pdf}
      \caption*{(b) $M=1024$}
  \end{minipage}
  \caption{Robustness of type I error control for correlation of noise.}
  \label{fig_fpr_cov}
\end{figure}

\begin{figure}[t]
  \centering
  \begin{minipage}[t]{0.4\hsize}
    \centering
    \includegraphics[width=0.95\textwidth]{figure/exp/tmlr2025_bearing2_fpr.pdf}
    \caption*{(a) Inference on a falsely detected CP candidate location for 3260 Hz (around the 14th harmonic) on 0.25--2.25~days}
  \end{minipage}
  \hfill
  \begin{minipage}[t]{0.4\hsize}
    \centering
    \includegraphics[width=0.95\textwidth]{figure/exp/tmlr2025_bearing2_tpr.pdf}
    \caption*{(b) Inference on a truely detected CP candidate location for 1420 Hz (the 6th harmonic) on 4--6~days}
  \end{minipage}
  \caption{Results of bearing~2. In panel (b), $p$-values were actually computed by considering CP candidates of the 6th harmonic, 3240 Hz, and 3460 Hz (around the 14th and 15th harmonics).}
  \label{fig_bearing2}
\end{figure}

\begin{figure}[t]
  \centering
  \vspace{1mm}
  \begin{minipage}[t]{0.4\hsize}
    \centering
    \includegraphics[width=0.95\textwidth]{figure/exp/tmlr2025_bearing3_fpr.pdf}
    \caption*{(a) Inference on a falsely detected CP candidate location for 3380 Hz (around the 14th harmonic) on 0.25--2.25~days}
  \end{minipage}
  \hfill
  \begin{minipage}[t]{0.4\hsize}
    \centering
    \includegraphics[width=0.95\textwidth]{figure/exp/tmlr2025_bearing3_tpr.pdf}
    \caption*{(b) Inferences on truely detected CP candidate locations for 1420 Hz (the 6th harmonic) on 4.75--6.75~days}
  \end{minipage}
  \caption{Results of bearing~3.}
  \label{fig_bearing3}
\end{figure}

\begin{figure}[H]
  \centering
  \vspace{-1mm}
  \begin{minipage}[t]{0.4\hsize}
    \centering
    \includegraphics[width=0.95\textwidth]{figure/exp/tmlr2025_bearing4_fpr.pdf}
    \caption*{(a) Inference on a falsely detected CP candidate location for 3540 Hz (the 15th harmonic) on 0.25--2.25~days}
  \end{minipage}  
  \hfill
  \begin{minipage}[t]{0.4\hsize}
    \centering
    \includegraphics[width=0.95\textwidth]{figure/exp/tmlr2025_bearing4_tpr.pdf}
    \caption*{(b) Inference on a truely detected CP candidate location for 1420 Hz (the 6th harmonic) on 4--6~days}
  \end{minipage}  
  \caption{Results of bearing~4. In panel (b), $p$-values were actually computed by considering not only a CP candidate of the 6th harmonic but also a CP candidate of 3440 Hz (around the 15th harmonic).}
  \label{fig_bearing4}
\end{figure}


\end{document}
