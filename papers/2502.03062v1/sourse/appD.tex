\section{Details of the Numerical Experiments}
\label{Experimental_Details}

\subsection{Detailed descriptions of comparison methods}
\label{Detailed_descriptions_of_comparison_methods}
In our experiments, we compared the proposed method (\texttt{Proposed}) with the following methods.
\begin{itemize}
  \item \texttt{OC}: In this method, we consider $p$-values conditioned on the process of the algorithm $\mathcal{A}$. 
        The over-conditioned $p$-value is computed as
  \begin{equation}
    p_k^{\text{oc}} = \mathbb{P}_{\text{H}_0,k} \left(Z \geq z^{\text{obs}} \, | \, Z \in \mathcal{Z}^{\text{oc}}\left(\bm{a}+\bm{b}z^{\text{obs}}\right)\right). \notag
  \end{equation}
  This method is computationally efficient, however, its power is low due to over-conditioning.
  \item \texttt{OptSeg-SI-oc}~\citep{duy2020computing}: This method uses a $p$-value conditioned only on the process of dynamic programming algorithm. 
  The over-conditioned $p$-value is computed as
  \begin{equation}
    p_k^{\text{OptSeg-SI-oc}} = \mathbb{P}_{\text{H}_0,k} \left(Z \geq z^{\text{obs}} \, | \, Z \in  \left\{r \in \mathbb{R} \, | \, \mathcal{S}_{\text{DP}}(r) = \mathcal{S}_{\text{DP}}\left(z^{\text{obs}}\right)\right\}\right). \notag
  \end{equation}
  \item \texttt{OptSeg-SI}~\citep{duy2020computing}: This method removes over-conditionning from OptSeg-SI-oc, that is, 
        the $p$-value is conditioned only on the result of dynamic programming. 
  \item \texttt{Naive}: This method uses a conventional $p$-value without conditioning, which is computed as
  \begin{equation}
    p_k^{\text{naive}} = \mathbb{P}_{\text{H}_0,k} \left(Z \geq z^{\text{obs}}\right). \notag
  \end{equation}
  \item \texttt{Bonferroni}: This is a method to control the type I error rate by applying Bonferroni correction which is widely used as multiple testing correction.
        Since the number of all possible hypotheses is $m = (2^{D}-1)(T-1)$, the bonferroni $p$-value is computed by $p_k^{\text{bonferroni}} = \min(1, m \cdot p_k^{\text{naive}})$.
\end{itemize}

\subsection{Computational Time and Computer Resources}
\label{Computational_Time}
We measured the computational time of our proposed method for the synthetic data experiments presented in Section~\ref{subsec:Synthetic_Data_Experiments} 
and computed the medians for each settings. 
The results for the type I error rate and power experiments are shown in Figure~\ref{fig_time}. 
Panels (a) and (b) indicate that the computational time increases exponentially with the sequence length. 
In addition, the computational time becomes shorter as the signal intensity increases in panels (c) and (d). 
This may be because the results of hypothesis testing in the case of high intensity are more likely to be obvious, and
the inference process can be terminated early.
%the number of interval computations in the parametric programming decreases.
All numerical experiments were conducted on a computer with a 96-core 3.60GHz CPU and 512GB of memory.

\begin{figure}[H]
  \centering
  \begin{minipage}[t]{0.24\hsize}
      \centering
      \includegraphics[width=0.95\textwidth]{figure/exp/tmlr2025_fpr512_time_median.pdf}
      \captionsetup{justification=centering}
      \caption*{(a) Type I error rate \\ \text{\hspace{1em}} ($M=512$)}
  \end{minipage}
  \hfill
  \begin{minipage}[t]{0.24\hsize}
      \centering
      \includegraphics[width=0.95\textwidth]{figure/exp/tmlr2025_fpr1024_time_median.pdf}
      \captionsetup{justification=centering}
      \caption*{(b) Type I error rate \\ \text{\hspace{1em}} ($M=1024$)}
  \end{minipage}
  \begin{minipage}[t]{0.24\hsize}
    \centering
    \includegraphics[width=0.95\textwidth]{figure/exp/tmlr2025_tpr512_time_median.pdf}
    \captionsetup{justification=centering}
    \caption*{(c) Power ($M=512$)}
\end{minipage}
\hfill
\begin{minipage}[t]{0.24\hsize}
    \centering
    \includegraphics[width=0.95\textwidth]{figure/exp/tmlr2025_tpr1024_time_median.pdf}
    \captionsetup{justification=centering}
    \caption*{(d) Power ($M=1024$)}
\end{minipage}
  \caption{Computational time in the type I error rate and the power experiments.}
  \label{fig_time}
\end{figure}

\subsection{Robustness of Type I Error Rate Control}
\label{Robustness_of_Type_I_Error_Rate_Control}
We evaluated the robustness of the \texttt{Proposed} in terms of the type I error rate control. 
For this purpose, we conducted experiments under three distinct noise conditions: 
(i) unknown noise variance, 
(ii) non-Gaussian noise, and
(iii) correlated noise.
The details of each experiment are given below.

\textbf{Unknown noise variance.}
We generated 1000 null sequences in the same manner as the type I error rate experiment with known variance presented in Section~\ref{subsec:Synthetic_Data_Experiments}.
To estimate the variance $\sigma^2$ from the same data, we first applied CP candidate selection algorithm to identify the segments, and then computed the empirical variance of each segment for all frequencies. 
Since the estimated variance tended to be smaller than the true value, we adopted the maximum value for each frequency. 
Given $\hat{\sigma}_f$ as the average of the values, $\sigma$ could be estimated from the property of DFT as $\hat{\sigma} = \frac{\hat{\sigma}_f}{\sqrt{M}}$.
The results for the significance levels $\alpha = 0.01, 0.05, 0.1$ are shown in Figure~\ref{fig_fpr_est} and the \texttt{Proposed} still could properly control the type I error rate. 

\textbf{Non-Gaussian noise.}
We considered the case where the noise followed the five non-Gaussian distributions:
\begin{itemize}
  \item \texttt{skewnorm}: Skew normal distribution family.
  \item \texttt{exponnorm}: Exponentially modified normal distribution family.
  \item \texttt{gennormsteep}: Generalized normal distribution family whose shape parameter $\beta$ is limited to be steeper than the normal distribution, i.e., $\beta < 2$.
  \item \texttt{gennormflat}: Generalized normal distribution family whose shape parameter $\beta$ is limited to be flatter than the normal distribution, i.e., $\beta > 2$.
  \item \texttt{t}: Student's t distribution family.
\end{itemize}

To generate sequences used in the experiment, 
we first obtained a noise distribution such that the 1-Wasserstein distance from the standard normal distribution $\mathcal{N}(0,1)$ was $\{0.01,0.02,0.03,0.04\}$ in each aforementioned distribution family. 
Subsequently, we standardized the distribution to have a mean of 0 and a variance of 1.
Then, we generated 1000 null sequences $\bm{x}=(x_1, \dots, x_N)^\top$, 
where the mean vector was specified in the same manner as described in the type I error rate experiment with known variance, 
and the noise followed the obtained distribution, for $T=60$. 
We applyed hypothesis testing using the test statistic with $\sigma=1$ for the detected CP candidate locations.
The results for the significance levels $\alpha = 0.05$ are shown in Figure~\ref{fig_fpr_nongaussian} 
and the \texttt{Proposed} could properly control the type I error rate for all non-Gaussian distributions.

\textbf{Correlated noise.}
We generated 10000 null sequences 
$\bm{x}=(x_1, \dots, x_N)^\top \sim \mathcal{N}(\bm{s}, \Sigma)$, 
where mean vector $\bm{s}$ was specified in the same manner as described in the type I error rate experiment with known variance, 
and covariance matrix $\Sigma$ was defined as $\Sigma = \sigma^2\left(\rho^{|i-j|}\right)_{ij} \in \mathbb{R}^{N \times N}$ with $\sigma=1$ and $\rho \in \{0.025, 0.05, 0.075, 0.1\}$, for $T=60$.
After CP candidate selection, we conducted the hypothesis testing using the test statistic with $\sigma=1$. 
The results for the significance levels $\alpha = 0.01, 0.05, 0.1$ are shown in Figure~\ref{fig_fpr_cov}.
For weak covariance, the \texttt{Proposed} could properly control the type I error rate. 
However, the type I error rate could not be controlled with increasing noise correlation. %when the noise is highly correlated.
This remains a challenge for future work. 

\subsection{More Results on Real Data Experiments}
\label{More_Results_on_Real_Data_Experiment}
Additional results for the signals of bearing~2, 3, and 4 
before and after the anomaly of bearing~1 occurred
are shown in Figures~\ref{fig_bearing2}, \ref{fig_bearing3}, and \ref{fig_bearing4}. 
In panel~(a) of each figure, the time variation of frequency spectra where CP candidate locations were falsely detected are shown for the period of 0.25--2.25~days when the frequency anomaly in bearing~1 did not actually exist. 
The results indicate that the inferences using $p$-values of the \texttt{Proposed} and \texttt{OC} are valid.
In panel~(b), the time variations of the BPFO harmonics where CP candidate locations were correctly detected are presented for the period of 4--6~days (bearing~2), 4.75--6.75~days (bearing~3), and 4--6~days (bearing~4), respectively.
In these cases, although the detection was delayed by several days relative to the occurrence of the frequency anomaly in bearing~1, 
the outer race fault signatures were successfully identified in all bearings using the \texttt{Proposed}.

\begin{figure}[H]
  \centering
  \begin{minipage}[t]{0.45\hsize}
      \centering
      \includegraphics[width=0.95\textwidth]{figure/exp/tmlr2025_est_fpr512.pdf}
      \caption*{(a) $M=512$}
  \end{minipage}
  \hfill
  \begin{minipage}[t]{0.45\hsize}
      \centering
      \includegraphics[width=0.95\textwidth]{figure/exp/tmlr2025_est_fpr1024.pdf}
      \caption*{(b) $M=1024$}
  \end{minipage}
  \caption{Robustness of type I error control for estimated variance.}
  \label{fig_fpr_est}
\end{figure}

\begin{figure}[H]
  \centering
  \begin{minipage}[t]{0.45\hsize}
      \centering
      \includegraphics[width=0.95\textwidth]{figure/exp/tmlr2025_nongaussian_fpr512.pdf}
      \caption*{(a) $M=512$}
  \end{minipage}
  \hfill
  \begin{minipage}[t]{0.45\hsize}
      \centering
      \includegraphics[width=0.95\textwidth]{figure/exp/tmlr2025_nongaussian_fpr1024.pdf}
      \caption*{(b) $M=1024$}
  \end{minipage}
  \caption{Robustness of type I error control for non-Gaussian noise.}
  \label{fig_fpr_nongaussian}
\end{figure}

\begin{figure}[H]
  \centering
  \begin{minipage}[t]{0.45\hsize}
      \centering
      \includegraphics[width=0.95\textwidth]{figure/exp/tmlr2025_cov_fpr512.pdf}
      \caption*{(a) $M=512$}
  \end{minipage}
  \hfill
  \begin{minipage}[t]{0.45\hsize}
      \centering
      \includegraphics[width=0.95\textwidth]{figure/exp/tmlr2025_cov_fpr1024.pdf}
      \caption*{(b) $M=1024$}
  \end{minipage}
  \caption{Robustness of type I error control for correlation of noise.}
  \label{fig_fpr_cov}
\end{figure}

\begin{figure}[t]
  \centering
  \begin{minipage}[t]{0.4\hsize}
    \centering
    \includegraphics[width=0.95\textwidth]{figure/exp/tmlr2025_bearing2_fpr.pdf}
    \caption*{(a) Inference on a falsely detected CP candidate location for 3260 Hz (around the 14th harmonic) on 0.25--2.25~days}
  \end{minipage}
  \hfill
  \begin{minipage}[t]{0.4\hsize}
    \centering
    \includegraphics[width=0.95\textwidth]{figure/exp/tmlr2025_bearing2_tpr.pdf}
    \caption*{(b) Inference on a truely detected CP candidate location for 1420 Hz (the 6th harmonic) on 4--6~days}
  \end{minipage}
  \caption{Results of bearing~2. In panel (b), $p$-values were actually computed by considering CP candidates of the 6th harmonic, 3240 Hz, and 3460 Hz (around the 14th and 15th harmonics).}
  \label{fig_bearing2}
\end{figure}

\begin{figure}[t]
  \centering
  \vspace{1mm}
  \begin{minipage}[t]{0.4\hsize}
    \centering
    \includegraphics[width=0.95\textwidth]{figure/exp/tmlr2025_bearing3_fpr.pdf}
    \caption*{(a) Inference on a falsely detected CP candidate location for 3380 Hz (around the 14th harmonic) on 0.25--2.25~days}
  \end{minipage}
  \hfill
  \begin{minipage}[t]{0.4\hsize}
    \centering
    \includegraphics[width=0.95\textwidth]{figure/exp/tmlr2025_bearing3_tpr.pdf}
    \caption*{(b) Inferences on truely detected CP candidate locations for 1420 Hz (the 6th harmonic) on 4.75--6.75~days}
  \end{minipage}
  \caption{Results of bearing~3.}
  \label{fig_bearing3}
\end{figure}

\begin{figure}[H]
  \centering
  \vspace{-1mm}
  \begin{minipage}[t]{0.4\hsize}
    \centering
    \includegraphics[width=0.95\textwidth]{figure/exp/tmlr2025_bearing4_fpr.pdf}
    \caption*{(a) Inference on a falsely detected CP candidate location for 3540 Hz (the 15th harmonic) on 0.25--2.25~days}
  \end{minipage}  
  \hfill
  \begin{minipage}[t]{0.4\hsize}
    \centering
    \includegraphics[width=0.95\textwidth]{figure/exp/tmlr2025_bearing4_tpr.pdf}
    \caption*{(b) Inference on a truely detected CP candidate location for 1420 Hz (the 6th harmonic) on 4--6~days}
  \end{minipage}  
  \caption{Results of bearing~4. In panel (b), $p$-values were actually computed by considering not only a CP candidate of the 6th harmonic but also a CP candidate of 3440 Hz (around the 15th harmonic).}
  \label{fig_bearing4}
\end{figure}
