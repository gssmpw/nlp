\appendix

\section{Determination of Penalty Parameters in the Optimization Problem}
\label{app:penalty}

In this section, we derive the penalty term in~(\ref{objective_func}) used for detecting CP candidates in the frequency domain.
We begin by assuming that the true mean vector of $\bm{f}^{(d)}$ for frequency $d \in \{0, \dots, D-1\}$ is piecewise constant as follows
\begin{equation}
  f_t^{(d)} \sim \mathcal{CN}\left(\mu_{\text{seg}}^{(d)}\left(\tau_{k-1}^{(d)} : \tau_{k}^{(d)}\right), \sigma_f^2\right), \, t \in \left\{\tau_{k-1}^{(d)}+1, \dots, \tau_k^{(d)}\right\}, \, k \in [K^{(d)}+1], \label{mean_structure}
\end{equation}
where $\mu_{\text{seg}}^{(d)}\left(\tau_{k-1}^{(d)} : \tau_{k}^{(d)}\right)$ represents the true mean of the $k$-th segment in frequency $d$, and $\sigma_f^2$ denotes the known variance such that $\sigma_f^2 = M \sigma^2$ by the property of DFT\footnote{
Although we impose the assumption of the piecewise constant mean structure for model selection based on BIC, 
the theoretical validity of $p$-values obtained by our proposed SI method is guaranteed even when this assumption does not hold.
}.
Then, we introduce BIC for deriving the values of the penalty parameters $\bm{\beta}$ in~(\ref{objective_func}). 
Given the assumptions of~(\ref{mean_structure}), 
the unknown parameters $\bm{\theta}$ of the sequence $\bm{f}^{(d)}$ are expressed as 
\begin{equation}
  \bm{\theta} = \left(\tau_1^{(d)}, \dots, \tau_{K^{(d)}}^{(d)}, \mu_{\text{seg}}^{(d)}\left(\tau_{0}^{(d)} : \tau_{1}^{(d)}\right), \dots, \mu_{\text{seg}}^{(d)}\left(\tau_{K^{(d)}}^{(d)} : \tau_{K^{(d)}+1}^{(d)}\right)\right). \notag 
\end{equation}
Therefore, the degrees of freedom can be computed as $K^{(d)}+c_{\text{sym}}^{(d)}\left(K^{(d)}+1\right)$. 
That is because the mean vector consists of real numbers for frequency $d = 0, \frac{M}{2}$, 
and complex numbers for the other frequencies. 
Consequently, the BIC of this model is given by 
\begin{equation}
  \text{BIC} = -2 \log L(\bm{\theta}) + \left(K^{(d)}+c_{\text{sym}}^{(d)}\left(K^{(d)}+1\right)\right) \log T, \notag
  % &= c^{(d)}(\frac{1}{\sigma_f^2} \sum |f - \bar{f}|^2 + T \log (2 \pi \sigma_f^2/c^{(d)})) + (c^{(d)}(K^{(d)}+1)+K^{(d)}) \log T, \notag 
  % \begin{cases}
  %   2\{\frac{1}{\sigma_f^2} \sum |f - \bar{f}|^2 + T \log (\pi \sigma_f^2)\} +  (DOF) \log T (d \neq 0,M/2) \notag
  %   \{\frac{1}{\sigma_f^2} \sum |f - \bar{f}|^2 + T \log (2 \pi \sigma_f^2)\} +  (DOF) \log T (d=0,M/2) \notag
  % \end{cases}
\end{equation}
where $L$ denotes the likelihood function for this model, % the model which minimizes the BIC is selected
and it can be rewritten by ignoring the terms that do not contribute to its minimization as 
\begin{equation}
  \text{BIC} = \sum_{k=1}^{{K}^{(d)}+1} \mathcal{C} \left(\bm{f}_{\tau_{k - 1}^{(d)}+1 : \tau_k^{(d)}}^{(d)}\right) +  \left(\left(c_{\text{sym}}^{(d)}+1\right) M \sigma^2 \log T\right) K^{(d)}. \notag
\end{equation}
Comparing the BIC with~(\ref{PO}), $\beta^{(d)}$ can be defined as
\begin{equation}
  \beta^{(d)} = \left(c_{\text{sym}}^{(d)}+1\right) M \sigma^2 \log T. \label{beta}
\end{equation}
While $\bm{\beta}$ can be determined based on the BIC, 
to the best of our knowledge, 
there is no theoretical method to define $\gamma$.  
Therefore, referring to~(\ref{beta}), 
we employ the value scaled by $M \sigma^2 \log T$ as $\gamma$, 
which is given by
\begin{equation}
  \gamma = \kappa M \sigma^2 \log T, \notag
\end{equation}
where $\kappa$ is a hyper-parameter determined by the user based on problem settings. 
We heuristically set $\kappa = 0.5$ in the synthetic data experiments, 
and $\kappa = 3$ in the real data experiment.