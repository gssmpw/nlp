\section{Numerical Experiments}
\label{sec:Experiment}

\subsection{Methods for Comparison}
In our experiments, we compared the proposed method (\texttt{Proposed}) using $p_k^{\text{selective}}$ in~(\ref{psel_z}) with the following methods 
% over-conditioning (\texttt{OC}) method, 
% SI for optimal partitioning with over-conditioning~\citep{duy2020computing} (\texttt{OptSeg-SI-oc}), %OC DP
% SI that removed over-conditionning from OptSeg-SI-oc~\citep{duy2020computing} (\texttt{OptSeg-SI}), %Parametric DP
% naive test (\texttt{Naive}) 
% and Bonferroni correction (\texttt{Bonferroni}), 
% The details of these methods for comparison are provided in the Appendix~\ref{app:methods}.
in terms of type I error rate control and power.
\begin{itemize}
  \item \texttt{OC}: In this method, that is, a simple extension of SI literature to our setting, 
  we consider $p$-values with additional conditioning (over-conditioning) described in Appendix~\ref{subsec:over-conditioning}.
  \item \texttt{OptSeg-SI-oc}, \texttt{OptSeg-SI}~\citep{duy2020computing}: 
  These methods use $p$-values conditioned only on the dynamic programming algorithm, disregarding the conditioning on simulated annealing.
  \item \texttt{Naive}: This method is a conventional statistical inference.
  \item \texttt{Bonferroni}: This method applies Bonferroni correction for multiple testing correction.
\end{itemize}
The details of these comparison methods are provided in Appendix~\ref{Detailed_descriptions_of_comparison_methods}.

\subsection{Synthetic Data Experiments}
\label{subsec:Synthetic_Data_Experiments}
\textbf{Experimental setup.}
In all synthetic experiments, we set window size $M \in \{512, 1024\}$, 
the number of frequencies $D = \left\lfloor\frac{M}{2} \right\rfloor + 1$, 
the length of sequence $N = M \cdot T$, where $T$ was specified for each experiment, 
the sampling rate $f_s = 20480$, 
and each element of mean vector $\bm{s}$ as 
\begin{equation}
  s_n = \sum_{d \in \{d_1, d_2, d_3\}} A_n^{(d)} \sin\left(\omega^{(d)}(n-1)\right) ~~~ (1 \leq n \leq N), \notag
\end{equation}
where frequencies $d_1, d_2, d_3 \in \{0, ..., D-1\}$ were randomly selected without replacement for each simulation, %such that $|\{d_1, d_2, d_3\}| = 3$
$A_n^{(d)}$ was defined for each experiment, 
and $\omega^{(d)} \in \left\{2 \pi (\frac{f_s}{M}) d \, \big{|} \, d = 0, \dots, \left\lfloor\frac{M}{2} \right\rfloor\right\}$.
We used BIC for the choice of penalty parameters $\bm{\beta}$ and $\gamma$ as indicated in Appendix~\ref{app:penalty}, 
and set the parameters of simulated annealing as $c_0^+=1000, \lambda^+=1.5, \chi(c_0)=0.5$ and $\lambda=0.8$ in Section~\ref{subsec:SA}. 
After detecting CP candidates, a CP location $\tau_k^{\text{det}}$ randomly selected from $\bm{\tau}^{\text{det}}$ was tested at the significance level $\alpha = 0.05$.

In the experiments conducted to evaluate the control of type I error rate, 
we generated $1000$ null sequences, which did not contain true CPs in the frequency domain, $\bm{x}=(x_1, \dots, x_N)^\top \sim \mathcal{N}\left(\bm{s}, \sigma^2 I_N\right)$, where $A_n^{(d)} = A^{(d)}$ 
was randomly sampled from $\halfopen{0}{1}$ for $d$ in each simulation, 
and $\sigma=1$, for each $T \in \{40, 60, 80, 100\}$.

Regarding the experiments to compare the power, 
we generated sequences $\bm{x}=(x_1, \dots, x_N)^\top \sim \mathcal{N}\left(\bm{s}, \sigma^2 I_N\right)$, 
where
\begin{equation}
  A_n^{(d)} = 
  \begin{cases}
    A^{(d)} & \left(1 \leq t \leq M \cdot t_1^{(d)}\right) \\
    \vspace{-3mm}\\
    A^{(d)} + \Delta & \left(M \cdot t_1^{(d)} + 1 \leq t \leq M \cdot t_2^{(d)}\right) \\
    \vspace{-3mm}\\
    A^{(d)} + 2\Delta & \left(M \cdot t_2^{(d)}+1 \leq t \leq T\right)
  \end{cases}, \notag
\end{equation}
with $A^{(d_1)}, A^{(d_2)}, A^{(d_3)} \in \halfopen{0}{1}$ which were randomly sampled in each simulation, $\left(t_1^{(d_1)}, t_1^{(d_2)}, t_1^{(d_3)}\right) = (18, 20, 22), \left(t_2^{(d_1)}, t_2^{(d_2)}, t_2^{(d_3)}\right) = (38, 40, 42)$, an intensity of the change $\Delta \in \{0.04, 0.08, 0.12, 0.16\}$ and $\sigma=1$, for $T=60$.
In each case, we ran $1000$ trials. 
Since we tested only when a CP candidate location was correctly detected, the power was defined as follows
\begin{equation}
  \text{Power (or Conditional Power)} = \frac{\# \text{ correctly detected \& rejected}}{\# \text{ correctly detected}}. \notag
\end{equation}
We considered the CP candidate location $\tau_k^{\text{det}}$ to be correctly detected if it satisfied the following two conditions:
\vspace{-3mm}
\begin{itemize}
\item The set $\mathcal{D}^{\text{det}}$ of frequencies containing at least one CP candidate was a subset of $\{d_1, d_2, d_3\}$.
\vspace{-2mm}
\item For $\mathcal{D}^{\text{det}}$ satisfying the above condition, either $\underset{d \in \mathcal{D}^{\text{det}}}{\min}{t_1^{(d)}} \leq \tau_k^{\text{det}} \leq \underset{d \in \mathcal{D}^{\text{det}}}{\max} {t_1^{(d)}}$ or $\underset{d \in \mathcal{D}^{\text{det}}}{\min}{t_2^{(d)}} \leq \tau_k^{\text{det}} \leq \underset{d \in \mathcal{D}^{\text{det}}}{\max} {t_2^{(d)}}$ held, 
      that is, $\tau_k^{\text{det}}$ was detected within the true CP locations for the frequencies in $\mathcal{D}^{\text{det}}$.
\end{itemize}
\vspace{-3mm}

\textbf{Experimental results.}
The results of experiments regarding the control of the type I error rate are shown in Figure~\ref{fig_fpr}.
The \texttt{Proposed}, \texttt{OC}, and \texttt{Bonferroni} successfully controlled the type I error rate below the significance level, 
whereas the \texttt{OptSeg-SI}, \texttt{OptSeg-SI-oc}, and \texttt{Naive} could not. 
That was because the \texttt{OptSeg-SI} and \texttt{OptSeg-SI-oc} 
used $p$-values conditioned only on the dynamic programming algorithm, 
excluding the conditioning on simulated annealing,
and the \texttt{Naive} employed conventional $p$-values without conditioning. 
Since the \texttt{OptSeg-SI}, \texttt{OptSeg-SI-oc}, and \texttt{Naive} failed to control the type I error rate, 
we omitted the analysis of their power.
The results of power experiments are shown in Figure~\ref{fig_tpr}. 
Based on these results, the \texttt{Proposed} was the most powerful of all methods that controlled the type I error rate.
The power of the \texttt{OC} was lower than that of the \texttt{Proposed} due to redundant conditions (see Appendix~\ref{app:truncation} for details). %, which caused the loss of power.
Furthermore, the \texttt{Bonferroni} method had the lowest power because it was a highly conservative approach that accounted for the huge number of all possible hypotheses.
Additionally, we provide the computational time of the \texttt{Proposed} in both experiments and the information on the computer resources in Appendix~\ref{Computational_Time}.
% used in the experiments

\begin{figure}[H]
  \centering
  \begin{minipage}[t]{0.45\hsize}
      \centering
      \includegraphics[width=0.95\textwidth]{figure/exp/tmlr2025_fpr512_skip.pdf}
      \caption*{(a) $M=512$}
  \end{minipage}
  \hfill
  \begin{minipage}[t]{0.45\hsize}
      \centering
      \includegraphics[width=0.95\textwidth]{figure/exp/tmlr2025_fpr1024_skip.pdf}
      \caption*{(b) $M=1024$}
  \end{minipage}
  \caption{Type I Error Rate}
  \label{fig_fpr}
\end{figure}
% \vspace{-5mm}
\begin{figure}[H]
  \centering
  \begin{minipage}[t]{0.45\hsize}
      \centering
      \includegraphics[width=0.95\textwidth]{figure/exp/tmlr2025_tpr512.pdf}
      \caption*{(a) $M=512$}
  \end{minipage}
  \hfill
  \begin{minipage}[t]{0.45\hsize}
      \centering
      \includegraphics[width=0.95\textwidth]{figure/exp/tmlr2025_tpr1024.pdf}
      \caption*{(b) $M=1024$}
  \end{minipage}
  \caption{Power}
  \label{fig_tpr}
\end{figure}

\textbf{Robustness of type I error rate control.}
We also conducted the following experiments to investigate the robustness of the \texttt{Proposed} in terms of type I error rate control.
\begin{itemize}
  \item Unknown noise variance: We considered the case where the variance $\sigma^2$ was estimated from the same data. 
  \item Non-Gaussian noise: We also considered the case where the noise followed the five types of standardized non-Gaussian distributions.
  \item Correlated noise: Furthermore, we considered the sequence whose noise was correlated, i.e., the covariance matrix $\Sigma \neq \sigma^2 I_N$. 
        In this case, although the test statistic did not theoretically follow a $\chi$-distribution, 
        we conducted the hypothesis testing using our proposed framework.
\end{itemize}
These details and results are shown in Appendix~\ref{Robustness_of_Type_I_Error_Rate_Control}.

\subsection{Real Data Experiments}
To demonstrate the practical applicability of the \texttt{Proposed}, we applied the \texttt{Proposed}, \texttt{OC}, and \texttt{Naive} to a real-world dataset.
We used the set No.2 of the IMS bearing dataset, 
which is provided by the Center for Intelligent Maintenance Systems (IMS), University of Cincinnati~\citep{qiu2006wavelet} 
and is available from the Prognostic data repository of NASA~\citep{Lee2007bearing}. 
The experimental apparatus consisted of four identical bearings installed on a common shaft, driven at a constant rotation speed by an AC motor under applied radial loading.
In this dataset, the vibration signals were measured using accelerometers until the outer race of bearing~1 failed at the end of the experiment, 
as shown in Figure~\ref{fig_bearing_signals}. 
The analysis for the time signal of bearing~1 in the frequency domain had revealed that anomalies were detected in the harmonics of the characteristic frequency ($236$ Hz) associated with the outer race failure (Ball Pass Frequency Outer race, BPFO) on 3--4~days~\citep{gousseau2016analysis}. 
Based on this previous study, we conducted CP candidate selection for the sensor data of bearing~1 in the frequency domain (1400--4000 Hz) for two periods: 
0.25--2.25~days when no anomalies existed 
and 2.25--4.25~days when the BPFO harmonics exhibited anomalous patterns. 
Subsequently, we tested the detected CP candidate locations to evaluate whether each of them was a genuine anomaly.
Since the signal with 20480 samples per second had been recorded every 10 minutes, 
we computed the DFT of $M = 1024$ consecutive points in the 20480 samples and repeated the procedure $T = 288$ times. %$N = MT$ 
Additionally, the variance $\sigma^2$ was estimated from the data on 0--0.25~days that was not used in all experiments.
The results for the signal of bearing~1 on 0.25--2.25~days and 2.25--4.25~days are shown in Figure~\ref{fig_bearing1}.
In panel~(a), the time variation of a frequency spectrum (1920 Hz) where a CP candidate location was falsely detected is shown for the period of 0.25--2.25~days when the frequency anomaly did not actually exist. 
It shows that $p$-values of the \texttt{Proposed} and \texttt{OC} are above the significance level $0.05$, 
and therefore the result provides the validity of the inference, 
while $p$-value of the \texttt{Naive} is too small. 
In panel~(b), the time variations of the 8th and 15th harmonics of the BPFO where CP candidate locations were correctly detected are presented for the period of 2.25--4.25~days when the frequency anomaly truly existed.
In this case, $p$-values of the \texttt{Proposed} are below the significance level $\frac{0.05}{2} = 0.025$ decided by Bonferroni correction,
thus it indicates that the inference is valid. 
In contrast, $p$-values of the \texttt{OC} are too large, 
due to the loss of power caused by the redundant conditions. 
In addition, since even the time sequences of the healthy bearings~2, 3, and 4 had been reported to indicate frequency characteristics associated with the outer race fault in bearing~1~\citep{gousseau2016analysis}, 
we performed the same analysis for the three signals.
The results are shown in Appendix~\ref{More_Results_on_Real_Data_Experiment}.
% 記載した 𝑝 値は, その周辺の周波数成分や, 15, 16 次以外の高調波の変化点も考慮した値となっていることに注意されたい.

\begin{figure}[t]
  \centering
  \begin{minipage}[t]{0.24\hsize}
      \centering
      \includegraphics[width=0.95\textwidth]{figure/exp/tmlr2025_bearing1.pdf}
      \captionsetup{justification=centering}
      \caption*{(a) Bearing 1}
  \end{minipage}
  \begin{minipage}[t]{0.24\hsize}
    \centering
    \includegraphics[width=0.95\textwidth]{figure/exp/tmlr2025_bearing2.pdf}
    \captionsetup{justification=centering}
    \caption*{(b) Bearing 2}
  \end{minipage}
  \begin{minipage}[t]{0.24\hsize}
    \centering
    \includegraphics[width=0.95\textwidth]{figure/exp/tmlr2025_bearing3.pdf}
    \captionsetup{justification=centering}
    \caption*{(c) Bearing 3}
  \end{minipage}
  \begin{minipage}[t]{0.24\hsize}
    \centering
    \includegraphics[width=0.95\textwidth]{figure/exp/tmlr2025_bearing4.pdf}
    \captionsetup{justification=centering}
    \caption*{(d) Bearing 4}
  \end{minipage}
  \caption{Vibration signals of the four bearings.}
  \label{fig_bearing_signals}
\end{figure}

\begin{figure}[t]
  \centering
  \begin{minipage}[t]{0.4\hsize}
    \centering
    \includegraphics[width=0.95\textwidth]{figure/exp/tmlr2025_bearing1_fpr.pdf}
    \caption*{(a) Inference on a falsely detected CP candidate location for 1920 Hz (around the 8th harmonic) on 0.25--2.25 days}
  \end{minipage}
  \hfill
  \begin{minipage}[t]{0.4\hsize}
    \centering
    \includegraphics[width=0.95\textwidth]{figure/exp/tmlr2025_bearing1_tpr.pdf}
    \caption*{(b) Inferences on truely detected CP candidate locations for 1880 Hz (the 8th harmonic, above) and 3540 Hz (the 15th harmonic, below) on 2.25--4.25~days}
  \end{minipage}
  \caption{Results of the CP candidate selection for the signal of bearing~1 in the frequency domain and the subsequent inference for the detected CP candidate locations.
  In panel (b), note that $p$-values for the first CP candidate location were actually computed by considering not only a CP candidate of the 15th harmonic but also a CP candidate of 1900 Hz (around the 8th harmonic).}
  \label{fig_bearing1}
\end{figure}
