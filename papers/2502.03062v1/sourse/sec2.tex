\section{Problem Setup}
\label{sec:Problem_Setup}

In this section, we first desrcribe the probabilistic model of time series on which the statistical inference is based, 
and then formulate the problem of CP detection in the frequency domain.

\subsection{Probabilistic Model for Time Series Data}
%
For statistical inference, we interpret that the observed time series data is a realization of a random sequence following a certain probabilistic model.
%
Let us denote the univariate random sequence with length $N$ by 
\begin{equation}
  \bm{X} = (X_1, \dots, X_N)^\top = \bm{s} + \bm{\epsilon}, \, \bm{\epsilon} \sim \mathcal{N}(\bm{0}, \sigma^2 I_{N}) \label{signal},  
\end{equation}
where 
$\bm{s} \in \mathbb{R}^N$ is the unknown true signal vector, % mean
and $\bm{\epsilon} \in \mathbb{R}^N$ is the normally distributed noise vector with the covariance matrix $\sigma^2 I_{N}$~\footnote{
%
In the following discussion, the noise $\bm{\epsilon}$ must be an independent and identically distributed Gaussian vector with predetermined covariance matrix.
%
The robustness of our proposed method for unknown noise variance, non-Gaussian noise, and correlated noise is discussed in Appendix~\ref{Robustness_of_Type_I_Error_Rate_Control}.
}.
%
In other words, we assume that the observed time series data is sampled from the probabilistic model $\bm X \sim \mathcal{N}\left(\bm s, \sigma^2 I_{N} \right)$.

Then, consider applying discrete short-time Fourier transform (STFT) to the random sequence $\bm{X}$ using rectangular window of width $M$ without overlapping.
%
In this case, the computations of DFT are required $T = \left\lfloor \frac{N}{M} \right\rfloor$ times, where we assume $T = \frac{N}{M}$, i.e., $N$ is a multiple of $M$ for simplicity.
%
We denote $M$-point DFT matrix as
\begin{equation}
  W_{M} = \left(\bm{w}_M^{(0)}, \dots, \bm{w}_M^{(M-1)}\right) \in \mathbb{C}^{M \times M}, \notag
\end{equation}
where $\bm{w}_M^{(d)} = \left(\omega_M^0, \omega_M^d, \dots, \omega_M^{(M-1)d}\right)^\top$, and $\omega_M = e^{-j \left(\frac{2\pi}{M}\right)}$ for frequency $d \in \{0, \dots, M-1\}$ in which $j$ is the imaginary unit.
%
Using $\bm{w}_M^{(d)}$, we consider multiple spectral sequences across $D = \left\lfloor\frac{M}{2} \right\rfloor + 1$ frequencies due to the symmetry property of the spectrum.
%
For frequency $d$,
the sequence of spectra is written as
\begin{equation}
  \bm{F}^{(d)} = \left({F}_1^{(d)}, \dots, {F}_T^{(d)}\right)^\top \in \mathbb{C}^T, \, d \in \{0, \ldots, D-1\}, \notag %\label{freq_representation}
\end{equation}
where ${F}_t^{(d)} = \left(\bm{1}_{t:t} \otimes \bm{w}_M^{(d)}\right)^\top \bm{X}$, and $\bm{1}_{s:e} \in \mathbb{R}^T$ is a vector whose elements from position $s$ to $e$ are set to $1$, and $0$ otherwise for $1 \leq s \leq e \leq T$.

\subsection{Statistically Significant CP Detection in the Frequency Domain}
%
As mentioned above, for statistical inference, we interpret that the observed time series is randomly sampled from the probabilistic model in~(\ref{signal}), which is denoted by
\begin{equation}
 \bm{x} = (x_1, \dots, x_N)^\top \in \mathbb{R}^N. \label{eq:observedX} % \sim \mathcal{N}(\bm s, \sigma^2 I_N)
\end{equation}
%
Similarly, the sequence of spectra for frequency $d$, obtained by applying the aforementioned STFT to the observed time series $\bm{x}$, is written as
\begin{equation}
\bm{f}^{(d)} = \left({f}_1^{(d)}, \dots, {f}_T^{(d)}\right)^\top \in \mathbb{C}^T, \, d \in \{0, \dots, D-1\}. \label{eq:observedF}
\end{equation}
%
The goal of this study is to detect changes in the true signals of frequency spectral sequences $\{\bm{F}^{(d)}\}_{d \in \{0, \ldots, D-1\}}$ based on the observed sequences $\{\bm{f}^{(d)}\}_{d \in \{0, \ldots, D-1\}}$.

Among variety of changes, we focus in this paper on \emph{mean-shift} of frequency spectral sequences.
%
Let us denote the mean spectrum of frequency $d$ at time point $t$ by 
\begin{equation}
 \mu_t^{(d)} = \mathbb{E}[F_t^{(d)}], \, (d, t) \in \{0, \ldots, D-1\} \times [T], \notag
\end{equation}
where the expectation operator $\mathbb{E}[\cdot]$ is taken with respect to the probabilistic model in~(\ref{signal})\footnote{
  Since the frequency spectral sequences are obtained through linear transformations of Gaussian random variables, 
  the existence of their expectations is guaranteed.  
}, and $[T] = {\{1, \dots, T \}}$ indicates the set of natural numbers up to $T$.

%
Considering a segment from time points $s$ to $e$ with $1 \le s \le e \le T$, we say that there is a \emph{mean-shift change} in the frequency $d$ at time point $t \in \{s, \ldots, e-1\}$ if and only if
\begin{equation}
 \frac{1}{t - s + 1} \sum_{t^\prime = s}^{t} \mu_{t^\prime}^{(d)}
 \neq
 \frac{1}{e - t} \sum_{t^\prime = t+1}^{e} \mu_{t^\prime}^{(d)}. \notag
\end{equation}


As discussed in Section~\ref{sec:Introduction}, we have prior knowledge that simultaneous changes occur across multiple distinct frequencies.
%
In Section~\ref{sec:CpSelection}, we introduce a heuristic algorithm that generates CP candidates by incorporating this prior knowledge.
%
Subsequently, in Section~\ref{sec:SI}, we present an SI framework to quantify the statistical significance of each CP candidate in the form of $p$-values.
%
Finally, we select the candidates with $p$-values smaller than the user-specified significance level (e.g., 0.05 or 0.01) as our final CPs.
%
The flow of CP detection in the frequency domain is illustrated in Figure~\ref{fig:fig2}.
%
\begin{figure}[H]
\centering
\includegraphics[width=0.7\hsize]{figure/fig2.pdf}
 \caption{
 Schematic illustration of CP detection process in the frequency domain.
 %
 In this example, at CP~28, changes occur in frequencies $d_1$ and $d_2$; at CP~70, in $d_2$ and $d_3$; and at CP~97, only in $d_0$.
 %
 Our proposed SI method in Section~\ref{sec:SI} can provide valid $p$-values for each CP.
 }
\label{fig:fig2}
\end{figure}
