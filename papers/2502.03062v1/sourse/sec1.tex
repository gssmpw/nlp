\section{Introduction}
\label{sec:Introduction}
%
To detect failures in complex systems, it is crucial to accurately identify the frequencies where the anomalies occur.
%
By pinpointing these specific frequencies, the root causes of faults in the systems can be identified.
%
In this paper, we address the change point (CP) detection problem in the frequency domain and propose a method that can identify \emph{statistically significant} changes in specific frequencies.
%
Quantifying statistical significance in CP detection ensures that the detected changes in the frequency domain reflect genuine structural alterations rather than random noise.
%
Measures such as $p$-values provide a quantitative framework to differentiate real changes from spurious ones, effectively mitigating the risk of incorrect decisions in data-driven systems.

This study was motivated by recent developments in statistically significant CP detection based on \emph{Selective Inference (SI)}~\citep{taylor2015statistical, fithian2015selective, lee2014exact}.
%
Prior to SI, quantifying the significance of detected CPs was challenging due to the issue of \emph{double dipping}, i.e., using the same data to both identify and test CPs inflates false positive findings.
%
Traditional statistical approaches prior to SI primarily focused on testing only whether a CP exists or not within a certain range using multiple testing framework based on asymptotic distribution~\citep{page1954continuous, mika1999fisher}.
%
Therefore, providing statistical significance measures, such as $p$-values, for specific points or frequencies was not feasible.
%
SI is a novel statistical inference framework designed for data-driven hypotheses.
%
In the context of CP detection, it enables the evaluation for the statistical significance of detected CPs conditional on a specific CP detection algorithm, 
thereby resolving the aforementioned double dipping issue.
%
The use of SI for statistically significant CP detection in the time domain has been actively studied recently~\citep{hyun2018exact, duy2020computing}.
%
Our contribution in this study is to extend these methods and realize statistically significant CP detection in the frequency domain.
% 

One of the difficulties in frequency-domain CP detection is that changes often appear across multiple frequencies due to shared underlying phenomena influencing broad signal characteristics.
%
Figure~\ref{fig_demo} shows an example of frequency-domain CP detection problems along with the statistically significant CPs identified using our proposed SI method. 
%
Panel (a) shows a time series transformed into signals across five frequencies.
% 
Panel (b) presents the CP detection results for each individual frequency $d \in \{d_0, d_1, d_2, d_3, d_4\}$~\footnote{
  Note that while the ``amplitude'' spectra are presented in the figures of this paper to visualize the temporal variations of spectral sequences and the means for each segment, 
  the ``complex'' spectra are utilized in the actual CP detection and the hypothesis testing
  (see Section~\ref{sec:Problem_Setup} and the subsequent sections for details).
}.
%
Panel (c) integrates CPs across multiple frequencies, detecting three changes A, B, and C (at CP A, changes occur in frequencies $d_1$ and $d_2$; at CP B, in $d_2$ and $d_3$; and at CP C, only in $d_0$).
%
In table (d), the obtained $p$-values for changes A, B, and C using the proposed method are 0.008, 0.006, and 0.752 (denoted as \emph{selective $p$-values}), respectively, indicating that at a significance level of $\frac{0.05}{3} \approx 0.0167$ decided by Bonferroni correction, changes A and B are statistically significant CPs\footnote{
%
In Figure~\ref{fig_demo}, the values labeled as naive $p$-values are inappropriate because they fail to account for double-dipping, leading to excessively small values and an inflated rate of false positive findings (see Section~\ref{sec:SI} for details). %and subsequent sections
%
}.

\begin{figure}[H]
  \centering
  \begin{minipage}[t]{0.3\hsize}
      \centering
      \includegraphics[width=0.66\textwidth]{figure/tmlr_demo_time.pdf}
      \caption*{(a) Time series signal.}
  \end{minipage}  
  \hspace{5mm} 
  \begin{minipage}[t]{0.3\hsize}
      \centering
      \includegraphics[width=0.66\textwidth]{figure/tmlr_demo_dp.pdf}
      \caption*{(b) CP detection result for each frequency component.}
  \end{minipage}
  \hspace{5mm} 
  \begin{minipage}[t]{0.3\hsize}
      \centering
      \includegraphics[width=0.66\textwidth]{figure/tmlr_demo_sa.pdf}
      \caption*{(c) CP detection result after integrating CPs across multiple frequencies.}
  \end{minipage}
  
  \vspace{3mm}

  \begin{minipage}[t]{0.6\hsize}
    \centering
    \caption*{(d) Statistical test for the CPs detected in (c).}
    \includegraphics[width=0.7\textwidth]{figure/fig_demo_p.pdf}
  \end{minipage}

  \caption{ Demonstration of the proposed method. 
            Panel~(a) shows the original time series signal. 
            Panel~(b) illustrates the result of CP detection for each time variation of the five frequency components, 
            and panel~(c) represents the CPs merged across multiple frequencies.
            In panel~(c), the CPs for frequency $d_1$ at time point~$28$ and $d_2$ at $30$, 
            and those for $d_2$ at $68$ and $d_3$ at $74$ merge into one CP at $28$~(A) and $70$~(B), respectively.
            Actually, the CPs~A and B are truely detected, while a CP C for $d_0$ at $97$ is wrong detection. 
            Table~(d) shows the results of statistical test for the CPs detected in~(c). 
            The proposed $p$-value is enough large for falsely detected CP~C, 
            while the naive metod causes a false positive because the $p$-value is too small. 
            Furthermore, the proposed method provides sufficiently small $p$-values for truely detected CPs~A~and~B, 
            which are statistically significant.
          }
  \label{fig_demo}
\end{figure}
%

To perform statistically significant CP detection, as illustrated in Figure~\ref{fig_demo}, two key technical challenges must be addressed.
%
The first challenge is to extend the SI framework to the frequency domain.
% 
We tackle this by formulating the test statistic and the conditioning on the hypothesis selection
that accurately account for the properties of discrete Fourier transform (DFT).
% We tackle this by introducing a projection matrix that accurately 
% accounts for the properties of discrete Fourier transform (DFT) 
% to define the test statistic and the hypothesis selection event.
% leveraging the linearity of frequency transformation and 
% {\color{red}carefully considering the probability law in the frequency domain.}
% {\color{red}representing each of the selection events in the frequency domain as a quadratic inequality with respect to the original data.}
%
The second challenge involves quantifying the statistical significance of CPs shared across multiple frequencies.
%
Addressing this requires solving a combinatorial optimization problem, which is computationally infeasible to solve globally optimally, necessitating a heuristic approach to obtain the approximate solution.
%
In this study, we employ a method based on simulated annealing~\citep{kirkpatrick1983optimization, cerny1985thermodynamical} and implement the SI framework to appropriately quantify the statistical significance of the approximate solution derived from the heuristic algorithm. 
% 

The proposed CP detection method consists of two stages.
%
In the first stage, CP candidates in the frequency domain are selected.
%
Since the selection of these CP candidates is formulated as a combinatorial optimization problem, a heuristic algorithm is used to derive an approximate solution.
%
In the second stage, the statistical significance of each CP candidate selected in the first stage is quantified in the form of $p$-values using the SI framework.
%
Among the CP candidates, only those with $p$-values below a significance level (e.g., 0.05 or 0.01) are eventually detected as final CPs.
%
The probability of the final detected CPs being false positives is theoretically guaranteed to be below the specified significance level.

The rest of the paper is organized as follows.
%
Section~\ref{sec:Problem_Setup} formulates the problem.
%
Section~\ref{sec:CpSelection} explains the heuristic algorithm used to select CP candidates in the first stage of the proposed method.
%
Section~\ref{sec:SI} describes the method for quantifying the statistical significance of CP candidates using the SI framework in the second stage.
%
In Section~\ref{sec:Experiment}, we demonstrate the effectiveness of our proposed method through comprehensive numerical experiments using both synthetic and real-world data.
For reproducibility, our implementation is available at \url{https://github.com/Takeuchi-Lab-SI-Group/si_for_frequency-domain_anomaly_detection}.
% vibration-based fault diagnosis of bearings.
% presents numerical experiments, 
%
Finally, Section~\ref{sec:conclusion} concludes the paper.
%
We note that the selection of CP candidates in stage~1 (Section~\ref{sec:CpSelection}) does not involve any particularly technical contributions.
%
Our primary contribution lies in stage~2 (Section~\ref{sec:SI}), where the statistical significance of the approximate solution is evaluated using the SI framework.

% 
% In summary, our contributions are as follows: 
% \begin{itemize}
%   \item[1.] We extend the SI framework to the frequency domain. 
%             To the best of our knowledge, this is the first study to consider selection events including frequency transformations.
% \end{itemize} 

\textbf{Related Work.}
%
The CP detection problem has long been studied with various applications in a variety of fields, such as finance~\citep{fryzlewicz2014multiple, pepelyshev2017real}, bioinformatics~\citep{chen2008statistical, muggeo2011efficient, pierre2015performance}, climatology~\citep{reeves2007review, beaulieu2012change}, and machine monitoring~\citep{lu2017novel, lu2018graph}.  
%
In the statistics and machine learning communities, various methods have been proposed for identifying multiple CPs from univariate sequences.
%
The most straightforward approach is repeatedly applying single CP detection algorithms, such as binary segmentation~\citep{scott1974cluster}.
%
Examples of such approaches include circular binary segmentation~\citep{olshen2004circular} and wild binary segmentation~\citep{fryzlewicz2014wild}.  
%
Another line of research has proposed numerous approaches based on penalized likelihood, such as Segment Neighbourhood~\citep{auger1989algorithms}, Optimal Partitioning~\citep{jackson2005algorithm}, PELT~\citep{killick2012optimal}, and FPOP~\citep{maidstone2017optimal}.
%
In these studies, dynamic programming is employed to solve the penalized likelihood minimization problem.  
%
While most CP detection studies focus on the time domain, a few studies have targeted the frequency domain, such as \citet{adak1998time}, \citet{last2008detecting}, and \citet{preuss2015detection}.
%
% In anomaly detection for complex systems, detecting CPs in the frequency domain {\color{red}has helped} identify the root causes of abnormalities in the systems. 

Most existing studies for statistical inference on detected CPs have relied on asymptotic theory under restrictive assumptions, such as weak dependency.
%
For single CP detection problems, approaches such as the CUSUM score~\citep{page1954continuous}, Fisher discriminant score~\citep{mika1999fisher, harchaoui2009kernel}, and MMD~\citep{li2015m} conduct statistical inference on the detected CPs based on asymptotic distribution of some discrepancy measures.
%
For multiple CP detection problems, methods such as SMUCE~\citep{frick2014multiscale} and the MOSUM procedure~\citep{eichinger2018mosum} also employ asymptotic inference.
%
However, these methods primarily focus on testing whether a CP exists within a certain range rather than directly quantifying the statistical significance of the location of the CP itself.
%
Furthermore, asymptotic approaches often fail to control the false positive (type I error) rate effectively, or resulting in conservative testing with low statistical power~\citep{hyun2018exact}.

SI was initially introduced as a method of statistical inference for feature selection in linear models~\citep{taylor2015statistical, fithian2015selective} and later extended to various feature selection algorithms, including marginal screening~\citep{lee2014exact}, stepwise feature selection~\citep{tibshirani2016exact}, and Lasso~\citep{lee2016exact}. 
% 
The core concept of SI is to derive the exact null distribution of the test statistic conditional on the hypothesis selection event, 
thereby enabling valid statistical inference with controlled type I error rate.
%
Recently, significant attention has been given to applying SI to more complex supervised learning algorithms, such as kernel models~\citep{yamada2018post}, boosting~\citep{rugamer2020inference}, tree-structured models~\citep{neufeld2022tree}, and neural networks~\citep{duy2022quantifying, miwa2023valid, shiraishi2024statistical}.
%
Furthermore, SI has proven valuable for unsupervised learning tasks, including clustering~\citep{lee2015evaluating, chen2023selective, gao2024selective}, outlier detection~\citep{chen2020valid, tsukurimichi2022conditional}, domain adaptation~\citep{le2024cad}, and segmentation~\citep{tanizaki2020computing, duy2022quantifying}.
%
SI was first used for CP detection problem in~\citep{hyun2018exact}, which focused on Fused Lasso algorithm.
%
Since then, the framework has been explored in various CP detection algorithms, including the CUSUM-based method~\citep{umezu2017selective}, binary segmentation and its variants~\citep{hyun2021post}, dynamic programming~\citep{duy2020computing}, and other related problems~\citep{sugiyama2021valid, jewell2022testing, carrington2024post, shiraishi2024selective}.
%
Existing studies on CP detection have focused on the time domain, making this the first to offer valid statistical inferences for frequency-domain anomalies using the SI framework.
