\section{Proofs}
\label{app:proofs}

\subsection{Proof of Theorem~\ref{thm:truncation}}
\label{app:proofs:truncation}
\textbf{Proof.}
According to the second condition in~(\ref{data_space}), we have 
\begin{align}
  \mathcal{U}(\bm{X}) &= \mathcal{U}(\bm{x}) \notag \\
  \Leftrightarrow (I_{N} - P_k) \bm{X} &= \mathcal{U}(\bm{x}) \notag \\
  \Leftrightarrow \bm{X} &= \mathcal{U}(\bm{x}) + \mathcal{V}(\bm{X}) z, \notag \\
  \Leftrightarrow \bm{X} &= \mathcal{U}(\bm{x}) + \mathcal{V}(\bm{x}) z, \, (\because \mathcal{V}(\bm{X}) = \mathcal{V}(\bm{x})) \notag \\
  \Leftrightarrow \bm{X} &= \bm{a} + \bm{b} z, \notag 
\end{align}
where $\bm{a} = \mathcal{U}(\bm{x}), \bm{b} = \mathcal{V}(\bm{x})$, and $z = T_k(\bm{X}) = \sigma^{-1} ||P_k \bm{X}||$. 

Then, we have
\begin{align}
  \mathcal{X} 
  &= \{\bm{X} \in \mathbb{R}^N \, | \, \mathcal{A}(\bm{X}) = \mathcal{A}(\bm{x}), \mathcal{Q}(\bm{X}) =\mathcal{Q}(\bm{x})\} \notag \\
  &= \{\bm{X} \in \mathbb{R}^N \, | \, \mathcal{A}(\bm{X}) = \mathcal{A}(\bm{x}), \bm{X} = \bm{a} + \bm{b} z, z \in \mathbb{R}\} \notag \\
  &= \{\bm{X} = \bm{a} + \bm{b} z \in \mathbb{R}^N \, | \, \mathcal{A}(\bm{a} + \bm{b} z) = \mathcal{A}(\bm{x}), z \in \mathbb{R}\} \notag \\
  &= \{\bm{X} = \bm{a} + \bm{b} z \in \mathbb{R}^N \, | \, z \in \mathcal{Z} \}. \notag 
\end{align}
Therefore, we obtain the result in Theorem~\ref{thm:truncation}. 

\subsection{Proof of Theorem~\ref{thm:uniform}}
\label{app:proofs:uniform}
\textbf{Proof.}
The sampling distribution of the test statistic conditional on $\mathcal{A}(\bm{X})$ and $\mathcal{Q}(\bm{X})$ denoted by
\begin{equation}
  T_k(\bm{X}) \, | \, \{\mathcal{A}(\bm{X}) = \mathcal{A}(\bm{x}), \mathcal{Q}(\bm{X}) =\mathcal{Q}(\bm{x})\} \notag
\end{equation}
follows a truncated $\chi$-distribution with $\mathrm{tr}(P_k)$ degrees of freedom and the truncation region $\mathcal{Z}$ defined in~(\ref{truncatedarea}).
Thus, by applying the probability integral transform, under the null hypothesis, 
\begin{equation}
  p_k^{\rm{selective}} \, | \, \{\mathcal{A}(\bm{X}) = \mathcal{A}(\bm{x}), \mathcal{Q}(\bm{X}) =\mathcal{Q}(\bm{x})\} \sim \rm{Unif}(0,1), \notag
\end{equation}
which leads to 
\begin{equation}
  \mathbb{P}_{\rm{H}_0,k} \left(p_k^{\rm{selective}}\leq\alpha \, | \, \mathcal{A}(\bm{X}) = \mathcal{A}(\bm{x}), \mathcal{Q}(\bm{X}) =\mathcal{Q}(\bm{x})\right) = \alpha, \, \forall \alpha \in [0, 1]. \notag
\end{equation}
Next, for any $\alpha \in [0, 1]$, we have
\begin{align}
  &\mathbb{P}_{\rm{H}_0,k} \left(p_k^{\rm{selective}}\leq\alpha \, | \, \mathcal{A}(\bm{X}) = \mathcal{A}(\bm{x})\right) \notag \\
  &= \int \mathbb{P}_{\rm{H}_0,k} \left(p_k^{\rm{selective}}\leq\alpha \, | \, \mathcal{A}(\bm{X}) = \mathcal{A}(\bm{x}), \mathcal{Q}(\bm{X}) =\mathcal{Q}(\bm{x})\right) \,
  \mathbb{P}_{\rm{H}_0,k} \left(\mathcal{Q}(\bm{X}) =\mathcal{Q}(\bm{x}) \, | \, \mathcal{A}(\bm{X}) = \mathcal{A}(\bm{x})\right) d\mathcal{Q}(\bm{x}) \notag \\
  &= \alpha \int \mathbb{P}_{\rm{H}_0,k} (\mathcal{Q}(\bm{X}) =\mathcal{Q}(\bm{x}) \, | \, \mathcal{A}(\bm{X}) = \mathcal{A}(\bm{x})) d\mathcal{Q}(\bm{x}) \notag \\
  &= \alpha. \notag
\end{align}
Therefore, we obtain the result in Theorem~\ref{thm:uniform} as follows:
\begin{align}
  \mathbb{P}_{\rm{H}_0,k} \left(p_k^{\rm{selective}}\leq\alpha\right) 
  &= \sum_{\mathcal{A}(\bm{x})} \mathbb{P}_{\rm{H}_0,k} \left(p_k^{\rm{selective}}\leq\alpha \, | \, \mathcal{A}(\bm{X}) = \mathcal{A}(\bm{x})\right) \, 
  \mathbb{P}_{\rm{H}_0,k} (\mathcal{A}(\bm{X}) = \mathcal{A}(\bm{x})) \notag \\
  &= \alpha \sum_{\mathcal{A}(\bm{x})} \mathbb{P}_{\rm{H}_0,k} (\mathcal{A}(\bm{X}) = \mathcal{A}(\bm{x})) \notag \\
  &= \alpha. \notag
\end{align}
