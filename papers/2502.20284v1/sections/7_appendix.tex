\newpage

\section{Gripper Planning Problem}
\label{app: grippers}
The types and predicates collaboratively define the states of a planning environment. Then, the actions define the transition of the environment. Each action consists of a set of input parameters, a precondition, and an effect. We consider the precondition and effect as the initial and final state of the action. 

The types of objects of the gripper problem are defined as
\vspace{4pt}
\begin{lstlisting}[language=completion]
(:types room ball robot gripper) 
<completion>; there are several balls distributed in several rooms and a robot with two grippers.</completion>
\end{lstlisting}
and the predicates are defined as
\vspace{4pt}
\begin{lstlisting}[language=completion]
(:predicates (at-robby ?r - robot ?x - room) <completion>; a predicate indicating the robot's location</completion>
    (at ?o - ball ?x - room) <completion>; a predicate indicating the ball's location</completion>
    (free ?r - robot ?g - gripper) <completion>; a predicate indicating whether the robot's gripper is free</completion>
    (carry ?r - robot ?o - ball ?g - gripper)) <completion>; indicating the ball carried by a gripper</completion>
\end{lstlisting}

An action moving the robot from a room to another room is defined as
\vspace{4pt}
\begin{lstlisting}[language=completion]
(:action move
    :parameters  (?r - robot ?from ?to - room) <completion>; we specify the initial and target rooms</completion>
    :precondition (and  (at-robby ?r ?from)) <completion>; the robot has to be in the initial room</completion>
    :effect (and  (at-robby ?r ?to) (not (at-robby ?r ?from))))
\end{lstlisting}
Furthermore, we have actions ``pick (a ball with a gripper)'' and ``drop (a ball).''
\vspace{4pt}
\begin{lstlisting}[language=completion]
(:action pick
       :parameters (?r - robot ?obj - object ?room - room ?g - gripper)
       :precondition  (and  (at ?obj ?room) (at-robby ?r ?room) (free ?r ?g))
       :effect (and (carry ?r ?obj ?g)
		    (not (at ?obj ?room)) 
		    (not (free ?r ?g))))

(:action drop
    :parameters (?r - robot ?obj - object ?room - room ?g - gripper)
    :precondition  (and  (carry ?r ?obj ?g) (at-robby ?r ?room))
    :effect (and (at ?obj ?room)
		(free ?r ?g)
		(not (carry ?r ?obj ?g)))))
\end{lstlisting}

An example of the initial and goal states are
\vspace{4pt}
\begin{lstlisting}[language=completion]
(:init (at-robby robot1 room1) (free robot1 rgripper1) (free robot1 lgripper1)
    (at ball1 room1) (at ball2 room3) (at ball3 room1) (at ball4 room2) )
(:goal (and (at ball1 room1) (at ball2 room3) (at ball3 room1) (at ball4 room2) ) )
\end{lstlisting}

\section{Additional Details on LLM Planner}
\label{app: llm-planner}

Figure \ref{fig: in-context}, \ref{fig: prompt}, and \ref{fig: refine-prompt} show the complete prompt for querying the language model to solve a planning problem. The blue text represents the prompts to the language model, while the red text corresponds to the responses generated by the language model.

\begin{figure}[H]
    \centering
    \begin{lstlisting}[language=completion]
    <prompt>User: Given the following planning domain:
    (define (domain gripper-strips)
       (:requirements :strips :typing) 
       (:types room object robot gripper)
       (:predicates (at-robby ?r - robot ?x - room)
                  (at ?o - object ?x - room)
                  (free ?r - robot ?g - gripper)
                  (carry ?r - robot ?o - object ?g - gripper))
       (:action move
           :parameters  (?r - robot ?from ?to - room)
           :precondition (and  (at-robby ?r ?from))
           :effect (and  (at-robby ?r ?to) (not (at-robby ?r ?from))))
       (:action pick
           :parameters (?r - robot ?obj - object ?room - room ?g - gripper)
           :precondition  (and  (at ?obj ?room) (at-robby ?r ?room) (free ?r ?g))
           :effect (and (carry ?r ?obj ?g) (not (at ?obj ?room)) (not (free ?r ?g))))
       (:action drop
           :parameters (?r - robot ?obj - object ?room - room ?g - gripper)
           :precondition  (and  (carry ?r ?obj ?g) (at-robby ?r ?room))
           :effect (and (at ?obj ?room) (free ?r ?g) (not (carry ?r ?obj ?g)))))
           
    Please solve the following PDDL problem and return the solution only.
    (define (problem gripper-1-2-1)
        (:domain gripper-strips)
        (:objects robot1 - robot
                  rgripper1 lgripper1 - gripper
                  room1 room2 - room
                  ball1 - object)
        (:init (at-robby robot2 room1)
               (free robot1 rgripper1)
               (free robot1 lgripper1)
               (at ball1 room1) )
        (:goal (at ball1 room2) ) )
    
    Assistant:
    1. Robot1 moves from room2 to room1. 
    2. Robot1 picks up ball1 with its left gripper in room1.
    3. Robot1 moves from room1 to room2. 
    4. Robot1 drops ball1 in room2 with its left gripper. </prompt>
    \end{lstlisting}
    \Description{In-context example.}
    \caption{An in-context example provided to the language model to constrain its output format.}
    \label{fig: in-context}
\end{figure}

\begin{figure}[H]
    \centering
    \begin{lstlisting}[language=completion]
    <prompt><In-context example from the previous textbox>
    
    User: Given the same planning domain:
    Please solve the following PDDL problem and return the solution only.
    (define (problem gripper-1-4-4)
        (:domain gripper-strips)
        (:objects robot1 - robot
                  rgripper1 lgripper1 - gripper
                  room1 room2 room3 room4 - room
                  ball1 ball2 ball3 ball4 - object)
        (:init (at-robby robot1 room1)
            (free robot1 rgripper1)
            (free robot1 lgripper1)
            (at ball1 room1)
            (at ball2 room3)
            (at ball3 room1)
            (at ball4 room2) )
        (:goal (and (at ball1 room1)
                (at ball2 room2)
                (at ball3 room2)
                (at ball4 room4) ) ) )
    Assistant: </prompt><completion>
    1. Robot1 picks up ball3 with its left gripper in room1.
    2. Robot1 moves from room1 to room2.
    3. Robot1 picks up ball4 with its right gripper in room2.
    4. Robot1 drops ball3 in room2 with its left gripper.
    5. Robot1 moves from room2 to room3.
    6. Robot1 picks up ball2 with its left gripper in room3.
    7. Robot1 moves from room3 to room2.
    8. Robot1 drops ball2 in room2 with its left gripper.
    9. Robot1 moves from room2 to room4.
    10. Robot1 drops ball4 in room4 with its right gripper.</completion>
    \end{lstlisting}
    \Description{Prompt}
    \caption{Complete prompt for querying a plan from the language model.}
    \label{fig: prompt}
\end{figure}

\begin{figure}[H]
% \begin{wrapfigure}{r}{0.4\textwidth}
    \centering
    \begin{lstlisting}[language=completion]
    <prompt>Given the planning domain and problem:
    (define (domain gripper-strips)
        ......)
    (define (problem gripper-1-2-1)
        (:domain gripper-strips)......)
    
    Please refine the plan from step 2:
    1. Robot1 moves from room2 to room1. 
    2. Robot1 picks up ball1 with its left gripper in room1.
    3. Robot1 picks up ball2 with its right gripper in room2.
    4. Robot1 drops ball3 in room2 with its left gripper. </prompt>
    <completion>1. Robot1 moves from room2 to room1. 
    2. Robot1 picks up ball1 with its left gripper in room1.
    3. Robot1 moves from room1 to room2. 
    4. Robot1 drops ball1 in room2 with its left gripper. </completion>
    \end{lstlisting}
    \Description{Refine prompt}
    \caption{Prompt for querying the LLM Planner to refine an existing plan.}
    \label{fig: refine-prompt}
% \end{wrapfigure}
\end{figure}

\begin{figure}[t]
    \centering
    \includegraphics[width=0.49\linewidth, trim=0 160 0 50, clip]{figures/correct-int.png}
    \includegraphics[width=0.49\linewidth, trim=0 170 0 50, clip]{figures/wrong.jpeg}
    \caption{Examples where LLM Planner correctly and incorrectly generates plans for the gripper planning problems.}
    \Description{Planning Problem Correctness}
    \label{fig: correctness}
\end{figure}

\begin{figure}[t]
    \centering
    \includegraphics[width=0.49\linewidth, trim=0 350 0 0, clip]{figures/explain-correct.jpeg}
    \includegraphics[width=0.49\linewidth, trim=0 350 0 0, clip]{figures/explain-wrong.jpeg}
    \caption{Explanations to the plans generated by the LLM Planner. The left example shows the explanation of a correctly generated plan while the right example shows the explanation of a incorrect plan.}
    \Description{Planning Problem Explanation}
    \label{fig: explain}
\end{figure}

\begin{figure}[t]
    \centering
    \includegraphics[height=0.24\linewidth]{figures/refine.jpeg}
    \includegraphics[height=0.24\linewidth, trim=500 405 10 0, clip]{figures/refine-correct.jpeg}
    \includegraphics[height=0.24\linewidth, trim=500 400 20 0, clip]{figures/refine-wrong.jpeg}
    \caption{Plan refinement by the LLM Planner. The left figure shows the planning problem (initial and goal state). The text boxes from left to right present two choices of plan refinement (where the refinement starts), a correctly refined plan refined from step 2, and an incorrect plan refined from step 5.}
    \Description{Refinement}
    \label{fig: refine}
\end{figure}

\clearpage
\section{User Study Details}\label{appendix:user_study_details}

\subsection{Main Session Questionnaires}
\paragraph{Before Intervention.}
Participants were presented with different planners and asked to rate their trust in the plans and decide whether to accept or reject them. The setup differed slightly between planner groups, as shown in \Cref{fig:before_intervention_ab,fig:before_intervention_cd}.

\begin{figure}[ht]
    \centering
    \begin{minipage}[t]{0.48\linewidth}
        \centering
        \includegraphics[width=\linewidth]{figures/planner_a2.pdf}
        \caption{Planners A and B before intervention: Plan + Trust Rating + Accept/Reject.}
        \label{fig:before_intervention_ab}
    \end{minipage}
    \hfill % Add spacing between the two minipages
    \begin{minipage}[t]{0.48\linewidth}
        \centering
        \includegraphics[width=\linewidth]{figures/planner_c2.pdf}
        \caption{Planners C and D before intervention: Plan + Trust Rating.}
        \label{fig:before_intervention_cd}
    \end{minipage}
    % \caption{Before Intervention}
\end{figure}


\paragraph{During Intervention.}
Each planner involved a specific type of intervention, as detailed in \Cref{fig:during_intervention_ab,fig:during_intervention_c,fig:during_intervention_d}.

\begin{figure}[ht]
    \centering
    \begin{minipage}[t]{0.48\linewidth}
        \centering
        \includegraphics[width=\linewidth]{figures/planner_a3.pdf}
    \label{fig:during_intervention_ab}
    \caption{Planners A and B: Participants observed the consequences of the proposed plan.}
    \end{minipage}
    \begin{minipage}[t]{0.48\linewidth}
        \centering
        \includegraphics[width=\linewidth]{figures/planner_c3.pdf}
        \caption{Planner C: Participants received an explanation of the plan.}
        \label{fig:during_intervention_c}
    \end{minipage}
\end{figure}


\begin{figure}[ht]
    \centering
    \begin{minipage}[t]{0.48\linewidth}
        \centering
        \includegraphics[width=\linewidth]{figures/planner_d3.pdf}
        \caption{Participants reviewed the problem (initial and goal state) and plan with two lines highlighted. The text boxes from left to right present two choices of plan refinement (where the refinement starts)}
    \end{minipage}
    \begin{minipage}[t]{0.48\linewidth}
    \centering
    \includegraphics[width=\linewidth]{figures/planner_d4.pdf}
    \caption{Participants received the refined plan.}
\end{minipage}
    \caption{Planner D.}
    \label{fig:during_intervention_d}
\end{figure}




\subsection{Propensity to Trust Scale}\label{appendix:propensity_to_trust_scale}

At the end of each session, participants rated their agreement with statements about the planner they just interacted with. For example, at the end of a session with Planner A, they were asked: ``\textit{Please indicate your level of agreement with the following statements about Planner A:}".

\begin{table}[ht]
\centering
\label{tab:trust_scale}
\resizebox{\textwidth}{!}{
\begin{tabular}{lccccc}
\toprule
 & \textbf{Strongly Disagree} & \textbf{Somewhat Disagree} & \textbf{Neither Agree Nor Disagree} & \textbf{Somewhat Agree} & \textbf{Strongly Agree} \\
 \textbf{Statement} & \textbf{(1)} & \textbf{(2)} & \textbf{(3)} & \textbf{(4)} & \textbf{(5)} \\
\midrule
I usually trust Planner A until there is a reason not to. & $\circ$ & $\circ$ & $\circ$ & $\circ$ & $\circ$ \\ 
For the most part, I distrust Planner A. & $\circ$ & $\circ$ & $\circ$ & $\circ$ & $\circ$ \\ 
In general, I would rely on Planner A to assist me. & $\circ$ & $\circ$ & $\circ$ & $\circ$ & $\circ$ \\ 
My tendency to trust Planner A is high. & $\circ$ & $\circ$ & $\circ$ & $\circ$ & $\circ$ \\ 
It is easy for me to trust Planner A to do their job. & $\circ$ & $\circ$ & $\circ$ & $\circ$ & $\circ$ \\ 
I am likely to trust Planner A even when I have little knowledge about it. & $\circ$ & $\circ$ & $\circ$ & $\circ$ & $\circ$\\
\bottomrule
\end{tabular}}
\caption{Propensity to trust scale for Planner A session. The planner label is replaced with B, C, or D in other sessions.}
\end{table} 


The results for the complete propensity to trust scale is presented in \Cref{fig:propensity_full} for reference.
\begin{figure}
    \centering
    \includegraphics[width=\linewidth]{figures/trust_propensity_full.pdf}
    \caption{Complete Propensity to trust scale result}
    \label{fig:propensity_full}
\end{figure}