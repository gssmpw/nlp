%%
%% This is file `sample-sigconf-authordraft.tex',
%% generated with the docstrip utility.
%%
%% The original source files were:
%%
%% samples.dtx  (with options: `all,proceedings,bibtex,authordraft')
%% 
%% IMPORTANT NOTICE:
%% 
%% For the copyright see the source file.
%% 
%% Any modified versions of this file must be renamed
%% with new filenames distinct from sample-sigconf-authordraft.tex.
%% 
%% For distribution of the original source see the terms
%% for copying and modification in the file samples.dtx.
%% 
%% This generated file may be distributed as long as the
%% original source files, as listed above, are part of the
%% same distribution. (The sources need not necessarily be
%% in the same archive or directory.)
%%
%%
%% Commands for TeXCount
%TC:macro \cite [option:text,text]
%TC:macro \citep [option:text,text]
%TC:macro \citet [option:text,text]
%TC:envir table 0 1
%TC:envir table* 0 1
%TC:envir tabular [ignore] word
%TC:envir displaymath 0 word
%TC:envir math 0 word
%TC:envir comment 0 0
%%
%%
%% The first command in your LaTeX source must be the \documentclass
%% command.
%%
%% For submission and review of your manuscript please change the
%% command to \documentclass[manuscript, screen, review]{acmart}.
%%
%% When submitting camera ready or to TAPS, please change the command
%% to \documentclass[sigconf]{acmart} or whichever template is required
%% for your publication.
%%
%%
% \documentclass[manuscript,review,anonymous]{acmart} % for anonymous review
\documentclass[manuscript,nonacm]{acmart} % for anonymous review
% \documentclass[sigconf]{acmart} % for camera-ready

% basic
%\usepackage{color,xcolor}
\usepackage{color}
\usepackage{epsfig}
\usepackage{graphicx}
\usepackage{algorithm,algorithmic}
% \usepackage{algpseudocode}
%\usepackage{ulem}

% figure and table
\usepackage{adjustbox}
\usepackage{array}
\usepackage{booktabs}
\usepackage{colortbl}
\usepackage{float,wrapfig}
\usepackage{framed}
\usepackage{hhline}
\usepackage{multirow}
% \usepackage{subcaption} % issues a warning with CVPR/ICCV format
% \usepackage[font=small]{caption}
\usepackage[percent]{overpic}
%\usepackage{tikz} % conflict with ECCV format

% font and character
\usepackage{amsmath,amsfonts,amssymb}
% \let\proof\relax      % for ECCV llncs class
% \let\endproof\relax   % for ECCV llncs class
\usepackage{amsthm} 
\usepackage{bm}
\usepackage{nicefrac}
\usepackage{microtype}
\usepackage{contour}
\usepackage{courier}
%\usepackage{palatino}
%\usepackage{times}

% layout
\usepackage{changepage}
\usepackage{extramarks}
\usepackage{fancyhdr}
\usepackage{lastpage}
\usepackage{setspace}
\usepackage{soul}
\usepackage{xspace}
\usepackage{cuted}
\usepackage{fancybox}
\usepackage{afterpage}
%\usepackage{enumitem} % conflict with IEEE format
%\usepackage{titlesec} % conflict with ECCV format

% ref
% commenting these two out for this submission so it looks the same as RSS example
% \usepackage[breaklinks=true,colorlinks,backref=True]{hyperref}
% \hypersetup{colorlinks,linkcolor={black},citecolor={MSBlue},urlcolor={magenta}}
\usepackage{url}
\usepackage{quoting}
\usepackage{epigraph}

% misc
\usepackage{enumerate}
\usepackage{paralist,tabularx}
\usepackage{comment}
\usepackage{pdfpages}
% \usepackage[draft]{todonotes} % conflict with CVPR/ICCV/ECCV format



% \usepackage{todonotes}
% \usepackage{caption}
% \usepackage{subcaption}

\usepackage{pifont}% http://ctan.org/pkg/pifont

% extra symbols
\usepackage{MnSymbol}


%%
%% \BibTeX command to typeset BibTeX logo in the docs
\AtBeginDocument{%
  \providecommand\BibTeX{{%
    Bib\TeX}}}

%% Rights management information.  This information is sent to you
%% when you complete the rights form.  These commands have SAMPLE
%% values in them; it is your responsibility as an author to replace
%% the commands and values with those provided to you when you
%% complete the rights form.
% \setcopyright{acmlicensed}
% \copyrightyear{2018}
% \acmYear{2018}
% \acmDOI{XXXXXXX.XXXXXXX}
\setcopyright{none}
\acmConference[Anonymous Submission]{}{Year}{Location}
\acmBooktitle{}
\acmPrice{}
\acmDOI{}
\acmISBN{}
\renewcommand\footnotetextcopyrightpermission[1]{} % Remove copyright footnote
\settopmatter{printacmref=false} % Remove ACM Reference Format

%% These commands are for a PROCEEDINGS abstract or paper.
% \acmConference[Conference acronym 'XX]{Make sure to enter the correct
%   conference title from your rights confirmation emai}{June 03--05,
%   2018}{Woodstock, NY}
%%
%%  Uncomment \acmBooktitle if the title of the proceedings is different
%%  from ``Proceedings of ...''!
%%
%%\acmBooktitle{Woodstock '18: ACM Symposium on Neural Gaze Detection,
%%  June 03--05, 2018, Woodstock, NY}
% \acmISBN{978-1-4503-XXXX-X/18/06}


%%
%% Submission ID.
%% Use this when submitting an article to a sponsored event. You'll
%% receive a unique submission ID from the organizers
%% of the event, and this ID should be used as the parameter to this command.
%%\acmSubmissionID{123-A56-BU3}

%%
%% For managing citations, it is recommended to use bibliography
%% files in BibTeX format.
%%
%% You can then either use BibTeX with the ACM-Reference-Format style,
%% or BibLaTeX with the acmnumeric or acmauthoryear sytles, that include
%% support for advanced citation of software artefact from the
%% biblatex-software package, also separately available on CTAN.
%%
%% Look at the sample-*-biblatex.tex files for templates showcasing
%% the biblatex styles.
%%

%%
%% The majority of ACM publications use numbered citations and
%% references.  The command \citestyle{authoryear} switches to the
%% "author year" style.
%%
%% If you are preparing content for an event
%% sponsored by ACM SIGGRAPH, you must use the "author year" style of
%% citations and references.
%% Uncommenting
%% the next command will enable that style.
%%\citestyle{acmauthoryear}


%%
%% end of the preamble, start of the body of the document source.
\begin{document}

%%
%% The "title" command has an optional parameter,
%% allowing the author to define a "short title" to be used in page headers.
% \title{Evaluating Human Trust in LLM-Based Planners: The Role of Correctness, Explanation, and Refinement in Structured Planning Tasks}
\title{Evaluating Human Trust in LLM-Based Planners: A Preliminary Study}
% Preliminary Evaluation of Human Trust in LLM Planners

%%
%% The "author" command and its associated commands are used to define
%% the authors and their affiliations.
%% Of note is the shared affiliation of the first two authors, and the
%% "authornote" and "authornotemark" commands
%% used to denote shared contribution to the research.
\author{Shenghui Chen}
\affiliation{%
  \institution{University of Texas at Austin}
  \city{Austin}
  \state{Texas}
  \country{USA}
}
\email{shenghui.chen@utexas.edu}

\author{Yunhao Yang}
\affiliation{%
  \institution{University of Texas at Austin}
  \city{Austin}
  \state{Texas}
  \country{USA}
}
\email{yunhaoyang234@utexas.edu}

\author{Kayla Boggess}
\affiliation{
  \institution{University of Virginia}
  \city{Charlottesville}
  \state{Virginia}
  \country{USA}
}
\email{kjb5we@virginia.edu}

\author{Seongkook Heo}
\affiliation{
  \institution{University of Virginia}
  \city{Charlottesville}
  \state{Virginia}
  \country{USA}
}
\email{seongkook@virginia.edu}

\author{Lu Feng}
\affiliation{
  \institution{University of Virginia}
  \city{Charlottesville}
  \state{Virginia}
  \country{USA}
}
\email{lu.feng@virginia.edu}

\author{Ufuk Topcu}
\affiliation{
  \institution{University of Texas at Austin}
  \city{Austin}
  \state{Texas}
  \country{USA}
}
\email{utopcu@utexas.edu}

%%
%% By default, the full list of authors will be used in the page
%% headers. Often, this list is too long, and will overlap
%% other information printed in the page headers. This command allows
%% the author to define a more concise list
%% of authors' names for this purpose.
% \renewcommand{\shortauthors}{Chen et al.}

%%
%% The abstract is a short summary of the work to be presented in the
%% article.
\begin{abstract}
    \begin{abstract}


The choice of representation for geographic location significantly impacts the accuracy of models for a broad range of geospatial tasks, including fine-grained species classification, population density estimation, and biome classification. Recent works like SatCLIP and GeoCLIP learn such representations by contrastively aligning geolocation with co-located images. While these methods work exceptionally well, in this paper, we posit that the current training strategies fail to fully capture the important visual features. We provide an information theoretic perspective on why the resulting embeddings from these methods discard crucial visual information that is important for many downstream tasks. To solve this problem, we propose a novel retrieval-augmented strategy called RANGE. We build our method on the intuition that the visual features of a location can be estimated by combining the visual features from multiple similar-looking locations. We evaluate our method across a wide variety of tasks. Our results show that RANGE outperforms the existing state-of-the-art models with significant margins in most tasks. We show gains of up to 13.1\% on classification tasks and 0.145 $R^2$ on regression tasks. All our code and models will be made available at: \href{https://github.com/mvrl/RANGE}{https://github.com/mvrl/RANGE}.

\end{abstract}


\end{abstract}

% \begin{CCSXML}
% <ccs2012>
%    <concept>
%        <concept_id>10003120.10003121.10011748</concept_id>
%        <concept_desc>Human-centered computing~Empirical studies in HCI</concept_desc>
%        <concept_significance>500</concept_significance>
%        </concept>
%  </ccs2012>
% \end{CCSXML}

% \ccsdesc[500]{Human-centered computing~Empirical studies in HCI}

%%
%% Keywords. The author(s) should pick words that accurately describe
%% the work being presented. Separate the keywords with commas.
\keywords{Trust, Large Language Models (LLMs), Explainable AI, Planning}


%% A "teaser" image appears between the author and affiliation
%% information and the body of the document, and typically spans the
%% page.

% \begin{teaserfigure}
%   \includegraphics[width=\textwidth]{sampleteaser.pdf}
%   \caption{Seattle Mariners at Spring Training, 2010.}
%   \Description{Enjoying the baseball game from the third-base
%   seats. Ichiro Suzuki preparing to bat.}
%   \label{fig:teaser}
% \end{teaserfigure}

% \received{20 February 2007}
% \received[revised]{12 March 2009}
% \received[accepted]{5 June 2009}

%%
%% This command processes the author and affiliation and title
%% information and builds the first part of the formatted document.
\maketitle

\section{Introduction} \label{sec:intro} %1.5p (including title and absract)
\section{Introduction}
Backdoor attacks pose a concealed yet profound security risk to machine learning (ML) models, for which the adversaries can inject a stealth backdoor into the model during training, enabling them to illicitly control the model's output upon encountering predefined inputs. These attacks can even occur without the knowledge of developers or end-users, thereby undermining the trust in ML systems. As ML becomes more deeply embedded in critical sectors like finance, healthcare, and autonomous driving \citep{he2016deep, liu2020computing, tournier2019mrtrix3, adjabi2020past}, the potential damage from backdoor attacks grows, underscoring the emergency for developing robust defense mechanisms against backdoor attacks.

To address the threat of backdoor attacks, researchers have developed a variety of strategies \cite{liu2018fine,wu2021adversarial,wang2019neural,zeng2022adversarial,zhu2023neural,Zhu_2023_ICCV, wei2024shared,wei2024d3}, aimed at purifying backdoors within victim models. These methods are designed to integrate with current deployment workflows seamlessly and have demonstrated significant success in mitigating the effects of backdoor triggers \cite{wubackdoorbench, wu2023defenses, wu2024backdoorbench,dunnett2024countering}.  However, most state-of-the-art (SOTA) backdoor purification methods operate under the assumption that a small clean dataset, often referred to as \textbf{auxiliary dataset}, is available for purification. Such an assumption poses practical challenges, especially in scenarios where data is scarce. To tackle this challenge, efforts have been made to reduce the size of the required auxiliary dataset~\cite{chai2022oneshot,li2023reconstructive, Zhu_2023_ICCV} and even explore dataset-free purification techniques~\cite{zheng2022data,hong2023revisiting,lin2024fusing}. Although these approaches offer some improvements, recent evaluations \cite{dunnett2024countering, wu2024backdoorbench} continue to highlight the importance of sufficient auxiliary data for achieving robust defenses against backdoor attacks.

While significant progress has been made in reducing the size of auxiliary datasets, an equally critical yet underexplored question remains: \emph{how does the nature of the auxiliary dataset affect purification effectiveness?} In  real-world  applications, auxiliary datasets can vary widely, encompassing in-distribution data, synthetic data, or external data from different sources. Understanding how each type of auxiliary dataset influences the purification effectiveness is vital for selecting or constructing the most suitable auxiliary dataset and the corresponding technique. For instance, when multiple datasets are available, understanding how different datasets contribute to purification can guide defenders in selecting or crafting the most appropriate dataset. Conversely, when only limited auxiliary data is accessible, knowing which purification technique works best under those constraints is critical. Therefore, there is an urgent need for a thorough investigation into the impact of auxiliary datasets on purification effectiveness to guide defenders in  enhancing the security of ML systems. 

In this paper, we systematically investigate the critical role of auxiliary datasets in backdoor purification, aiming to bridge the gap between idealized and practical purification scenarios.  Specifically, we first construct a diverse set of auxiliary datasets to emulate real-world conditions, as summarized in Table~\ref{overall}. These datasets include in-distribution data, synthetic data, and external data from other sources. Through an evaluation of SOTA backdoor purification methods across these datasets, we uncover several critical insights: \textbf{1)} In-distribution datasets, particularly those carefully filtered from the original training data of the victim model, effectively preserve the model’s utility for its intended tasks but may fall short in eliminating backdoors. \textbf{2)} Incorporating OOD datasets can help the model forget backdoors but also bring the risk of forgetting critical learned knowledge, significantly degrading its overall performance. Building on these findings, we propose Guided Input Calibration (GIC), a novel technique that enhances backdoor purification by adaptively transforming auxiliary data to better align with the victim model’s learned representations. By leveraging the victim model itself to guide this transformation, GIC optimizes the purification process, striking a balance between preserving model utility and mitigating backdoor threats. Extensive experiments demonstrate that GIC significantly improves the effectiveness of backdoor purification across diverse auxiliary datasets, providing a practical and robust defense solution.

Our main contributions are threefold:
\textbf{1) Impact analysis of auxiliary datasets:} We take the \textbf{first step}  in systematically investigating how different types of auxiliary datasets influence backdoor purification effectiveness. Our findings provide novel insights and serve as a foundation for future research on optimizing dataset selection and construction for enhanced backdoor defense.
%
\textbf{2) Compilation and evaluation of diverse auxiliary datasets:}  We have compiled and rigorously evaluated a diverse set of auxiliary datasets using SOTA purification methods, making our datasets and code publicly available to facilitate and support future research on practical backdoor defense strategies.
%
\textbf{3) Introduction of GIC:} We introduce GIC, the \textbf{first} dedicated solution designed to align auxiliary datasets with the model’s learned representations, significantly enhancing backdoor mitigation across various dataset types. Our approach sets a new benchmark for practical and effective backdoor defense.




% \section{Related Work} \label{sec:related} %1p
% \section{Related Work}
\label{sec:related-works}
\subsection{Novel View Synthesis}
Novel view synthesis is a foundational task in the computer vision and graphics, which aims to generate unseen views of a scene from a given set of images.
% Many methods have been designed to solve this problem by posing it as 3D geometry based rendering, where point clouds~\cite{point_differentiable,point_nfs}, mesh~\cite{worldsheet,FVS,SVS}, planes~\cite{automatci_photo_pop_up,tour_into_the_picture} and multi-plane images~\cite{MINE,single_view_mpi,stereo_magnification}, \etal
Numerous methods have been developed to address this problem by approaching it as 3D geometry-based rendering, such as using meshes~\cite{worldsheet,FVS,SVS}, MPI~\cite{MINE,single_view_mpi,stereo_magnification}, point clouds~\cite{point_differentiable,point_nfs}, etc.
% planes~\cite{automatci_photo_pop_up,tour_into_the_picture}, 


\begin{figure*}[!t]
    \centering
    \includegraphics[width=1.0\linewidth]{figures/overview-v7.png}
    %\caption{\textbf{Overview.} Given a set of images, our method obtains both camera intrinsics and extrinsics, as well as a 3DGS model. First, we obtain the initial camera parameters, global track points from image correspondences and monodepth with reprojection loss. Then we incorporate the global track information and select Gaussian kernels associated with track points. We jointly optimize the parameters $K$, $T_{cw}$, 3DGS through multi-view geometric consistency $L_{t2d}$, $L_{t3d}$, $L_{scale}$ and photometric consistency $L_1$, $L_{D-SSIM}$.}
    \caption{\textbf{Overview.} Given a set of images, our method obtains both camera intrinsics and extrinsics, as well as a 3DGS model. During the initialization, we extract the global tracks, and initialize camera parameters and Gaussians from image correspondences and monodepth with reprojection loss. We determine Gaussian kernels with recovered 3D track points, and then jointly optimize the parameters $K$, $T_{cw}$, 3DGS through the proposed global track constraints (i.e., $L_{t2d}$, $L_{t3d}$, and $L_{scale}$) and original photometric losses (i.e., $L_1$ and $L_{D-SSIM}$).}
    \label{fig:overview}
\end{figure*}

Recently, Neural Radiance Fields (NeRF)~\cite{2020NeRF} provide a novel solution to this problem by representing scenes as implicit radiance fields using neural networks, achieving photo-realistic rendering quality. Although having some works in improving efficiency~\cite{instant_nerf2022, lin2022enerf}, the time-consuming training and rendering still limit its practicality.
Alternatively, 3D Gaussian Splatting (3DGS)~\cite{3DGS2023} models the scene as explicit Gaussian kernels, with differentiable splatting for rendering. Its improved real-time rendering performance, lower storage and efficiency, quickly attract more attentions.
% Different from NeRF-based methods which need MLPs to model the scene and huge computational cost for rendering, 3DGS has stronger real-time performance, higher storage and computational efficiency, benefits from its explicit representation and gradient backpropagation.

\subsection{Optimizing Camera Poses in NeRFs and 3DGS}
Although NeRF and 3DGS can provide impressive scene representation, these methods all need accurate camera parameters (both intrinsic and extrinsic) as additional inputs, which are mostly obtained by COLMAP~\cite{colmap2016}.
% This strong reliance on COLMAP significantly limits their use in real-world applications, so optimizing the camera parameters during the scene training becomes crucial.
When the prior is inaccurate or unknown, accurately estimating camera parameters and scene representations becomes crucial.

% In early works, only photometric constraints are used for scene training and camera pose estimation. 
% iNeRF~\cite{iNerf2021} optimizes the camera poses based on a pre-trained NeRF model.
% NeRFmm~\cite{wang2021nerfmm} introduce a joint optimization process, which estimates the camera poses and trains NeRF model jointly.
% BARF~\cite{barf2021} and GARF~\cite{2022GARF} provide new positional encoding strategy to handle with the gradient inconsistency issue of positional embedding and yield promising results.
% However, they achieve satisfactory optimization results when only the pose initialization is quite closed to the ground-truth, as the photometric constrains can only improve the quality of camera estimation within a small range.
% Later, more prior information of geometry and correspondence, \ie monocular depth and feature matching, are introduced into joint optimisation to enhance the capability of camera poses estimation.
% SC-NeRF~\cite{SCNeRF2021} minimizes a projected ray distance loss based on correspondence of adjacent frames.
% NoPe-NeRF~\cite{bian2022nopenerf} chooses monocular depth maps as geometric priors, and defines undistorted depth loss and relative pose constraints for joint optimization.
In earlier studies, scene training and camera pose estimation relied solely on photometric constraints. iNeRF~\cite{iNerf2021} refines the camera poses using a pre-trained NeRF model. NeRFmm~\cite{wang2021nerfmm} introduces a joint optimization approach that simultaneously estimates camera poses and trains the NeRF model. BARF~\cite{barf2021} and GARF~\cite{2022GARF} propose a new positional encoding strategy to address the gradient inconsistency issues in positional embedding, achieving promising results. However, these methods only yield satisfactory optimization when the initial pose is very close to the ground truth, as photometric constraints alone can only enhance camera estimation quality within a limited range. Subsequently, 
% additional prior information on geometry and correspondence, such as monocular depth and feature matching, has been incorporated into joint optimization to improve the accuracy of camera pose estimation. 
SC-NeRF~\cite{SCNeRF2021} minimizes a projected ray distance loss based on correspondence between adjacent frames. NoPe-NeRF~\cite{bian2022nopenerf} utilizes monocular depth maps as geometric priors and defines undistorted depth loss and relative pose constraints.

% With regard to 3D Gaussian Splatting, CF-3DGS~\cite{CF-3DGS-2024} also leverages mono-depth information to constrain the optimization of local 3DGS for relative pose estimation and later learn a global 3DGS progressively in a sequential manner.
% InstantSplat~\cite{fan2024instantsplat} focus on sparse view scenes, first use DUSt3R~\cite{dust3r2024cvpr} to generate a set of densely covered and pixel-aligned points for 3D Gaussian initialization, then introduce a parallel grid partitioning strategy in joint optimization to speed up.
% % Jiang et al.~\cite{Jiang_2024sig} proposed to build the scene continuously and progressively, to next unregistered frame, they use registration and adjustment to adjust the previous registered camera poses and align unregistered monocular depths, later refine the joint model by matching detected correspondences in screen-space coordinates.
% \gjh{Jiang et al.~\cite{Jiang_2024sig} also implemented an incremental approach for reconstructing camera poses and scenes. Initially, they perform feature matching between the current image and the image rendered by a differentiable surface renderer. They then construct matching point errors, depth errors, and photometric errors to achieve the registration and adjustment of the current image. Finally, based on the depth map, the pixels of the current image are projected as new 3D Gaussians. However, this method still exhibits limitations when dealing with complex scenes and unordered images.}
% % CG-3DGS~\cite{sun2024correspondenceguidedsfmfree3dgaussian} follows CF-3DGS, first construct a coarse point cloud from mono-depth maps to train a 3DGS model, then progressively estimate camera poses based on this pre-trained model by constraining the correspondences between rendering view and ground-truth.
% \gjh{Similarly, CG-3DGS~\cite{sun2024correspondenceguidedsfmfree3dgaussian} first utilizes monocular depth estimation and the camera parameters from the first frame to initialize a set of 3D Gaussians. It then progressively estimates camera poses based on this pre-trained model by constraining the correspondences between the rendered views and the ground truth.}
% % Free-SurGS~\cite{freesurgs2024} matches the projection flow derived from 3D Gaussians with optical flow to estimate the poses, to compensate for the limitations of photometric loss.
% \gjh{Free-SurGS~\cite{freesurgs2024} introduces the first SfM-free 3DGS approach for surgical scene reconstruction. Due to the challenges posed by weak textures and photometric inconsistencies in surgical scenes, Free-SurGS achieves pose estimation by minimizing the flow loss between the projection flow and the optical flow. Subsequently, it keeps the camera pose fixed and optimizes the scene representation by minimizing the photometric loss, depth loss and flow loss.}
% \gjh{However, most current works assume camera intrinsics are known and primarily focus on optimizing camera poses. Additionally, these methods typically rely on sequentially ordered image inputs and incrementally optimize camera parameters and scene representation. This inevitably leads to drift errors, preventing the achievement of globally consistent results. Our work aims to address these issues.}

Regarding 3D Gaussian Splatting, CF-3DGS~\cite{CF-3DGS-2024} utilizes mono-depth information to refine the optimization of local 3DGS for relative pose estimation and subsequently learns a global 3DGS in a sequential manner. InstantSplat~\cite{fan2024instantsplat} targets sparse view scenes, initially employing DUSt3R~\cite{dust3r2024cvpr} to create a densely covered, pixel-aligned point set for initializing 3D Gaussian models, and then implements a parallel grid partitioning strategy to accelerate joint optimization. Jiang \etal~\cite{Jiang_2024sig} develops an incremental method for reconstructing camera poses and scenes, but it struggles with complex scenes and unordered images. 
% Similarly, CG-3DGS~\cite{sun2024correspondenceguidedsfmfree3dgaussian} progressively estimates camera poses using a pre-trained model by aligning the correspondences between rendered views and actual scenes. Free-SurGS~\cite{freesurgs2024} pioneers an SfM-free 3DGS method for reconstructing surgical scenes, overcoming challenges such as weak textures and photometric inconsistencies by minimizing the discrepancy between projection flow and optical flow.
%\pb{SF-3DGS-HT~\cite{ji2024sfmfree3dgaussiansplatting} introduced VFI into training as additional photometric constraints. They separated the whole scene into several local 3DGS models and then merged them hierarchically, which leads to a significant improvement on simple and dense view scenes.}
HT-3DGS~\cite{ji2024sfmfree3dgaussiansplatting} interpolates frames for training and splits the scene into local clips, using a hierarchical strategy to build 3DGS model. It works well for simple scenes, but fails with dramatic motions due to unstable interpolation and low efficiency.
% {While effective for simple scenes, it struggles with dramatic motion due to unstable view interpolation and suffers from low computational efficiency.}

However, most existing methods generally depend on sequentially ordered image inputs and incrementally optimize camera parameters and 3DGS, which often leads to drift errors and hinders achieving globally consistent results. Our work seeks to overcome these limitations.


\section{Methods} \label{sec:method} %1.5p
\section{Analysis}

\textbf{3D Structure.}
The vanilla RoPE defines a matrix $\bm{A}_{t_1,t_2}$ that represents the relative positional encoding between two positions $t_1$ and $t_2$ in a 1D sequence:
\begin{equation}\label{eq:vanilla_rope}
% \vspace{-6pt}
\begin{aligned}
% \small
\bm{A}_{t_1,t_2}&=\left(\bm{q}_{t_1}\bm{R}_{t_1}\right){\left(\bm{k}_{t_2}\bm{R}_{t_2}\right)}^\top
% = \bm{q}_{t_1}\bm{R}_{t_1}\bm{R}_{t_2}^\top\bm{k}_{t_2}^\top
= \bm{q}_{t_1}\bm{R}_{\Delta t}\bm{k}_{t_2}^\top,
\end{aligned}
% \vspace{-6pt}
\end{equation}
where $\Delta t=t_1-t_2$, the symbols $\bm{q}_{t_1}$ and $\bm{k}_{t_2}$ are the query and key vectors at positions $t_1$ and $t_2$.
The \textit{relative rotation matrix} $\bm{R}_{\Delta t}$ is defined as $\bm{R}_{\Delta t} = \exp(\Delta ti\theta_{n})$, while $i$ is the imaginary unit, $\theta_{n} = \beta^{-2n/d}$ is the frequency of rotation applied to a specific $n$-th pair of $d$ dimensions ($n=0,\ldots,d/2-1$), and $\beta$ is the frequency base parameter.
The vanilla RoPE uses $d=128$, thus $n=0,\ldots,63$.
Consequently, the $\bm{A}_{t_1,t_2}$ in Eq. (\ref{eq:vanilla_rope}) can be extended as:
% \begin{equation}\label{equ:rope}
% \vspace{-6pt}
% \resizebox{0.5\textwidth}{!}{
% \scriptsize
% \begin{gathered}
% \begin{pmatrix}
% q^{(0)}\\q^{(1)}\\\vdots\\q^{(126)}\\q^{(127)}
% \end{pmatrix}^\top
% \begin{pmatrix}\cos{\theta_0\Delta t}& -\sin{\theta_0\Delta t}&\cdots&0&0\\ \sin{\theta_0\Delta t}&\cos{\theta_0\Delta t}&\cdots&0&0 \\ \vdots&\vdots&\ddots&\vdots&\vdots\\ 0&0&\cdots&\cos{\theta_{63}\Delta t}&  \sin{\theta_{63}\Delta t}\\ 0&0&\cdots&\sin{\theta_{63}\Delta t}&\cos{\theta_{63}\Delta t}
% \end{pmatrix}
% \begin{pmatrix}
% k^{(0)}\\k^{(1)}\\\vdots\\k^{(126)}\\k^{(127)}
% \end{pmatrix}.
% \end{gathered}
% }
% \end{equation}
\begin{equation}\label{equ:rope}
\vspace{-6pt}
\resizebox{0.5\textwidth}{!}{$
\scriptsize
\left(
\begin{array}{c}
q^{(0)}\\q^{(1)}\\\vdots\\q^{(126)}\\q^{(127)}
\end{array}
\right)^{\top}
\left(
\begin{array}{ccccc}
\cos{\theta_0\Delta t} & -\sin{\theta_0\Delta t} & \cdots & 0 & 0 \\ 
\sin{\theta_0\Delta t} & \cos{\theta_0\Delta t} & \cdots & 0 & 0 \\ 
\vdots & \vdots & \ddots & \vdots & \vdots \\  
0 & 0 & \cdots & \cos{\theta_{63}\Delta t} &  \sin{\theta_{63}\Delta t} \\  
0 & 0 & \cdots & \sin{\theta_{63}\Delta t} & \cos{\theta_{63}\Delta t} 
\end{array}
\right)
\left(
\begin{array}{c}
k^{(0)}\\k^{(1)}\\\vdots\\k^{(126)}\\k^{(127)}
\end{array}
\right)
$}
\end{equation}



While the vanilla RoPE operates on 1D sequences, it can also be applied to higher-dimensional input by flattening the input into a 1-D sequence.
However, the flattening process discards crucial neighborhood information, increases the sequence length, and hinders the capture of long-range dependencies.
Therefore, preserving the inherent 3D structure is essential when adapting RoPE for video data.
Some recent RoPE-variants (e.g., M-RoPE in Qwen2-VL \cite{wang2024qwen2}) incorporate the $3$D structure.
The corresponding relative matrix $\bm{A}_{(t_1,x_1,y_1)}$ is computed as:
\begin{equation}
% \small
\bm{A}_{(t_1,x_1,y_1),(t_2,x_2,y_2)}=\bm{q}_{(t_1,x_1,y_1)}\bm{R}_{\Delta t,\Delta x,\Delta y}\bm{k}_{(t_2,x_2,y_2)}^\top,
\end{equation}
where $\Delta t=t_1-t_2$, $\Delta x=x_1-x_2$, and $\Delta y=y_1-y_2$.
M-RoPE divides the $d=128$ feature dimensions into 3 groups: the first 32 for temporal positions ($t$), the middle 48 for horizontal positions ($x$), and the last 48 for vertical positions ($y$). As shown in Eq~(\ref{equ:mrope}), $\bm{A}_{(t_1,x_1,y_1),(t_2,x_2,y_2)}$ in M-RoPE is extended as:
\begin{equation}
\vspace{-6pt}
\resizebox{0.5\textwidth}{!}{$
\scriptsize
\begin{gathered}
\underbrace{\begingroup
\setlength\arraycolsep{1pt}
\begin{pmatrix}q^{(0)}\\q^{(1)}\\q^{(2)}\\q^{(3)}\\\vdots\\q^{(30)}\\q^{(31)}\end{pmatrix}^\top
\begin{pmatrix}
% \setstacktabbedgap{2pt}
\cos{\theta_0\Delta t}& -\sin{\theta_0\Delta t}&0&0&\cdots&0&0\\
\sin{\theta_0\Delta t}&\cos{\theta_0\Delta t}&0&0&\cdots&0&0 \\
0&0&\cos{\theta_1\Delta t}& -\sin{\theta_1\Delta t}&\cdots&0&0\\
0&0&\sin{\theta_1\Delta t}&\cos{\theta_1\Delta t}&\cdots&0&0 \\ 
\vdots&\vdots&\vdots&\vdots&\ddots&\vdots&\vdots\\
0&0&0&0&\cdots&\cos{\theta_{15}\Delta t}& -\sin{\theta_{15}\Delta t}\\
0&0&0&0&\cdots&\sin{\theta_{15}\Delta t}&\cos{\theta_{15}\Delta t}
\end{pmatrix}
\begin{pmatrix}k^{(0)}\\k^{(1)}\\k^{(2)}\\k^{(3)}\\\vdots\\k^{(30)}\\k^{(31)}\end{pmatrix}
\endgroup}_\text{\normalsize modeling temporal dependency with higher frequency} \\
+ \underbrace{\begingroup
\setlength\arraycolsep{1pt}
\begin{pmatrix}q^{(32)}\\q^{(33)}\\q^{(34)}\\q^{(35)}\\\vdots\\q^{(78)}\\q^{(79)}\end{pmatrix}^\top
\begin{pmatrix}
% \setstacktabbedgap{2pt}
\cos{\theta_{16}\Delta x}& -\sin{\theta_{16}\Delta x}&0&0&\cdots&0&0\\
\sin{\theta_{16}\Delta x}&\cos{\theta_{16}\Delta x}&0&0&\cdots&0&0 \\
0&0&\cos{\theta_{17}\Delta x}& -\sin{\theta_{17}\Delta x}&\cdots&0&0\\
0&0&\sin{\theta_{17}\Delta x}&\cos{\theta_{17}\Delta x}&\cdots&0&0 \\ 
\vdots&\vdots&\vdots&\vdots&\ddots&\vdots&\vdots\\
0&0&0&0&\cdots&\cos{\theta_{39}\Delta x}& -\sin{\theta_{39}\Delta x}\\
0&0&0&0&\cdots&\sin{\theta_{39}\Delta x}&\cos{\theta_{39}\Delta x}
\end{pmatrix}
\begin{pmatrix}k^{(32)}\\k^{(33)}\\k^{(34)}\\k^{(35)}\\\vdots\\k^{(78)}\\k^{(79)}\end{pmatrix}
\endgroup}_\text{\normalsize modeling horizontal dependency with intermediate frequency} \\
+ \underbrace{\begingroup
\setlength\arraycolsep{1pt}
\begin{pmatrix}q^{(80)}\\q^{(81)}\\q^{(82)}\\q^{(83)}\\\vdots\\q^{(126)}\\q^{(127)}\end{pmatrix}^\top
\begin{pmatrix}
% \setstacktabbedgap{2pt}
\cos{\theta_{40}\Delta y}& -\sin{\theta_{40}\Delta y}&0&0&\cdots&0&0\\
\sin{\theta_{40}\Delta y}&\cos{\theta_{40}\Delta y}&0&0&\cdots&0&0 \\
0&0&\cos{\theta_{41}\Delta y}& -\sin{\theta_{41}\Delta y}&\cdots&0&0\\
0&0&\sin{\theta_{41}\Delta y}&\cos{\theta_{41}\Delta y}&\cdots&0&0 \\ 
\vdots&\vdots&\vdots&\vdots&\ddots&\vdots&\vdots\\
0&0&0&0&\cdots&\cos{\theta_{63}\Delta y}& -\sin{\theta_{63}\Delta y}\\
0&0&0&0&\cdots&\sin{\theta_{63}\Delta y}&\cos{\theta_{63}\Delta y}
\end{pmatrix}
\begin{pmatrix}k^{(80)}\\k^{(81)}\\k^{(82)}\\k^{(83)}\\\vdots\\k^{(126)}\\k^{(127)}\end{pmatrix}
\endgroup}_\text{\normalsize modeling vertical dependency with lower frequency}
\end{gathered}
$}
\label{equ:mrope}
\end{equation}
% \begin{equation}
% \vspace{-6pt}
% \resizebox{0.5\textwidth}{!}{$
% \scriptsize
% \underbrace{
% \left(
% \begin{array}{c}
% q^{(0)}\\q^{(1)}\\\vdots\\q^{(31)}
% \end{array}
% \right)^{\top}
% \left(
% \begin{array}{cccccc}
% \cos{\theta_0\Delta t} & -\sin{\theta_0\Delta t} & \cdots & 0 \\ 
% \sin{\theta_0\Delta t} & \cos{\theta_0\Delta t} & \cdots & 0 \\ 
% \vdots & \vdots & \ddots & \vdots \\ 
% 0 & 0 & \cdots & \cos{\theta_{15}\Delta t} 
% \end{array}
% \right)
% \left(
% \begin{array}{c}
% k^{(0)}\\k^{(1)}\\\vdots\\k^{(31)}
% \end{array}
% \right)
% }_{\text{\normalsize modeling temporal dependency with higher frequency}}
% +
% \underbrace{
% \left(
% \begin{array}{c}
% q^{(32)}\\q^{(33)}\\\vdots\\q^{(79)}
% \end{array}
% \right)^{\top}
% \left(
% \begin{array}{cccccc}
% \cos{\theta_{16}\Delta x} & -\sin{\theta_{16}\Delta x} & \cdots & 0 \\ 
% \sin{\theta_{16}\Delta x} & \cos{\theta_{16}\Delta x} & \cdots & 0 \\ 
% \vdots & \vdots & \ddots & \vdots \\ 
% 0 & 0 & \cdots & \cos{\theta_{39}\Delta x} 
% \end{array}
% \right)
% \left(
% \begin{array}{c}
% k^{(32)}\\k^{(33)}\\\vdots\\k^{(79)}
% \end{array}
% \right)
% }_{\text{\normalsize modeling horizontal dependency with intermediate frequency}}
% +
% \underbrace{
% \left(
% \begin{array}{c}
% q^{(80)}\\q^{(81)}\\\vdots\\q^{(127)}
% \end{array}
% \right)^{\top}
% \left(
% \begin{array}{cccccc}
% \cos{\theta_{40}\Delta y} & -\sin{\theta_{40}\Delta y} & \cdots & 0 \\ 
% \sin{\theta_{40}\Delta y} & \cos{\theta_{40}\Delta y} & \cdots & 0 \\ 
% \vdots & \vdots & \ddots & \vdots \\ 
% 0 & 0 & \cdots & \cos{\theta_{63}\Delta y} 
% \end{array}
% \right)
% \left(
% \begin{array}{c}
% k^{(80)}\\k^{(81)}\\\vdots\\k^{(127)}
% \end{array}
% \right)
% }_{\text{\normalsize modeling vertical dependency with lower frequency}}
% $}
% \end{equation}


\noindent \textbf{Frequency Allocation.}
% Note that the frequency encoding in vanilla RoPE (Eq. \ref{equ:rope}) assigns higher frequencies (via larger $\theta_{n}$ values) to lower dimensions.
Incorporating 3D structure raises the question of how to allocate the temporal ($t$), horizontal ($x$), and vertical ($y$) components within the $d$ dimensions.
Note that different allocation strategies are not equivalent in the rotation frequency $\theta_{n} = \beta^{-2n/d}$.
As shown in Eq. (\ref{equ:mrope}), M-RoPE assigns higher frequencies (corresponding to lower dimensions) to the temporal dimension ($t$).

To highlight the importance of frequency allocation, we introduce a challenging retrieval task \textbf{V}isual \textbf{N}eedle-\textbf{I}n-\textbf{A}-\textbf{H}astack-\textbf{D}istractor (\textbf{V-NIAH-D}).
V-NIAH-D builds upon V-NIAH \cite{zhang2024longva}, a benchmark designed to evaluate visual long-context understanding.
However, the straightforward retrieval-based task has been shown to provide only a superficial form of long-context understanding~\cite{hsieh2024ruler,yuan2024lv}.
Therefore, We enhance V-NIAH by incorporating semantically similar distractors, obtained using Google Image Search~\cite{googleimagesearch} or Flux ~\cite{flux2023}, to mitigate the possibility of correct answers through random chance.
These distractors are designed to be unambiguous to the question in Fig. \ref{fig:v-ruler}.

\begin{figure}[t]
\centering
\includegraphics[width=.94\linewidth]{figures/files/attention_analysis.pdf}
\vspace{-6pt}
\caption{\footnotesize Attention-based frequential allocation analysis.
\textbf{Middle}: M-RoPE's temporal dimension ($t$) is limited to local information, resulting in a diagonal layout.
\textbf{Bottom}: \methodname effectively retrieves the needle using the temporal dimension.
The x and y coordinates represent the video frame number, e.g., 50 for 50 frames.
For more details see Appendix \ref{app:attention_analysis}.
% We use 8k-context input, with video tokens from the same frame aggregated via average pooling.
}
\vspace{-12pt}
\label{fig:attention_analysis}
\end{figure}



As shown in Fig. \ref{fig:v-ruler}, M-RoPE exhibits a clear performance drop from V-NIAH to V-NIAH-D. To investigate this decline, we follow previous works \citep{xiao2023efficient,liu2023scaling,barbero2024round} to visualize the attention scores in Fig. \ref{fig:attention_analysis}. We decompose the attention scores into their corresponding temporal ($t$), horizontal ($x$), and vertical ($y$) components for visualization.

Fig.~ \ref{fig:attention_analysis} reveals unusual attention patterns in M-RoPE, despite its ability to locate the needle image but fails to answer the multi-choice question.
According to the attention of M-RoPE, the needle is located primarily through vertical positional information, rather than temporal features.
Thus, the temporal dimension fails to capture long-range semantic dependencies, focusing instead on local relationships.
Conversely, the spatial dimensions exhibit a tendency to capture long-range rather than local semantic information.
Lastly, the horizontal and vertical dimensions display distinct characteristics, with the vertical dimension exhibiting phenomena reminiscent of attention sinks \cite{xiao2023efficient}.
These observations suggest that the performance decline primarily results from the sub-optimal frequency allocation designs of M-RoPE.

\noindent \textbf{Spatial Symmetry.} Given the text tokens $T$ and the visual tokens $T_v$, spatial symmetry \cite{kexuefm10352} claims that the distance between the end of the preceding textual input ($T_{\text{pre}}$) and the beginning of the visual input ($T_v^{\text{start}}$) is equal to the distance between the end of the visual input ($T_v^{\text{end}}$) and the beginning of the subsequent textual input ($T_{\text{sub}}$):
\begin{equation}
    T_{v}^{\text{start}} - T_{\text{pre}} =
    T_{\text{sub}} - T_{v}^{\text{end}}.
\end{equation}
The spatial symmetrical structure can potentially simplify the learning process and reduce bias toward input order.
However, existing 3D RoPE variants such as M-RoPE do not meet the spatial symmetry, we will elaborate related discussion in Fig. \ref{fig:spatial}.

\begin{figure*}[t]
\begin{minipage}{0.98\textwidth}
    \begin{subfigure}[b]{0.49\linewidth}
        \centering
\includegraphics[width=0.95\linewidth]{figures/files/video_rope-period_low-MRoPE.pdf}
        \caption{Temporal Frequency Allocation in M-RoPE}
        \label{fig:temporal_mrope}
    \end{subfigure}
    \hfill
    \begin{subfigure}[b]{0.49\linewidth}
        \centering
        \includegraphics[width=0.95\linewidth]{figures/files/video_rope-period_low-VideoRoPE.pdf}
        \caption{Temporal Frequency Allocation in \methodname (ours)}
        \label{fig:temporal_videorope}
    \end{subfigure}
    \vspace{-6pt}
    \caption{\footnotesize \textbf{(a)} M-RoPE \cite{wang2024qwen2} models temporal dependencies using the \textit{first} 16 rotary angles, which exhibit higher frequencies and more pronounced oscillations. \textbf{(b)} In contrast, \methodname models temporal dependencies using the \textit{last} 16 rotary angles, characterized by significantly wider, monotonic intervals. Our frequency allocation effectively mitigates the misleading influence of distractors in V-NIAH-D. For a more detailed analysis, please refer to Appendix \ref{app:supp_explain_modules}.
    % Take the first 3 rotary angles as an example, the position embedding for temporal modeling is free from oscillation~\cite{men2024base}.
    }
    \label{fig:period_mono}
    \vspace{-12pt}
\end{minipage}
\end{figure*}

\noindent \textbf{Temporal Index Scaling.}
The frame index in video and the token index in text are inherently different \cite{kexuefm10352,li2024temporal}.
Recognizing this difference, methods like TAD-RoPE, a 1D RoPE adaptation for Video LLMs, introduce distinct step offsets for image and text token indices: $\gamma$ for image tokens and $\gamma+1$ for text tokens.
Consequently, an ideal RoPE design for video data should permit scaling of the temporal index to meet the inherent difference between the frame index and the text index.

\section{\methodname}\label{subsec:step_size}

Based on some previous research and the above analysis, we claim that a good RoPE design for Video LLMs, especially for long videos, should satisfy four requirements.
% : 3D structure, Appreciate Frequency Allocation, Spatial Symmetry, and Temporal Index Scaling.
The first requirement has been solved by RoPE-Tie~\cite{kexuefm10040} and the subsequent M-RoPE~\cite{wang2024qwen2}.
To solve the last three requirements and mitigate the performance decline observed in V-NIAH-D, we propose our \methodname, comprising the following three key components.
% (1) Low-frequency Temporal Allocation; (2) Diagonal Layout; and (3) Adjustable Temporal Spacing.
% \textbf{\textit{Multi-Modal Compatibility}}, whether RoPE can simultaneously describe the spatiotemporal position in multi-modals and sequential position in text-only inputs~\cite{wang2024qwen2,kexuefm10040,kexuefm10352}, \textbf{\textit{Appropriate Dimension Distribution}}, whether the feature dimension can process the semantic relationship where it is responsibility~\cite{peng2023yarn,barbero2024round,liu2024kangaroo}, \textbf{\textit{Spatial Symmetry}}, whether the distance between the end of precedent textual input and start of visual input equals the distance between the end of visual input and the start of subsequent textual input~\cite{kexuefm10352}, and \textbf{\textit{Temporal Alignment}}, whether the alignment of sequential feature in different modality is considered~\cite{gao2024tc}.

\noindent \textbf{Low-frequency Temporal Allocation (LTA).} 
As shown in Eq. (\ref{equ:rope}), the vanilla RoPE~\cite{su2024roformer} uses all dimensions to model the 1D position information. And as indicated in Eq. (\ref{equ:mrope}), M-RoPE~\cite{wang2024qwen2} uses dimensions to model temporal, horizontal, and vertical dimensions sequentially.
However, previous frequency allocation strategies are suboptimal because different RoPE dimensions capture dependencies at varying ranges.
As shown in Fig.  \ref{fig:attention_analysis}, an interesting observation is that the local attention branch (as reported in \cite{han2024lm}) corresponds to lower dimensions, while the global branch (or attention sink, as in \cite{xiao2023efficient}) corresponds to higher dimensions.
To sum up, lower dimensions (higher frequency, shorter monotonic intervals, larger $\theta_n$) tend to capture relative distances and local semantics \cite{men2024base,barbero2024round}, while higher dimensions (lower frequency, wider monotonic intervals, smaller $\theta_n$) capture longer-range dependencies \cite{barbero2024round}.

Based on our analysis, \methodname uses higher dimensions for temporal features in longer contexts and lower dimensions for spatial features, which are limited by resolution and have a fixed range.
To avoid the gap between horizontal and vertical positions, we interleave the dimensions responsible for these spatial features.
The dimension distribution for \methodname is shown in Eq. (\ref{equ:videorope}):

% $\bm{A}_{(t_1,x_1,y_1),(t_2,x_2,y_2)}=\bm{q}_{(t_1,x_1,y_1)}\bm{k}_{(t_2,x_2,y_2)}^\top$

% To make full use of these properties of RoPE, \methodname uses higher dimensions to model temporal features in longer contexts and lower dimensions to model spatial features since spatial features tend to be limited by resolution and have a relatively fixed range. To avoid the gap between horizontal and vertical positions, we interleave the dimensions responsible for those two spatial features. Therefore, the dimension distribution for \methodname is shown in Equation~\ref{equ:videorope}.
% 48, 48, 32; 0, 47, 48, 95; 96, 127
\begin{equation}
\resizebox{0.5\textwidth}{!}{$
\scriptsize
\begin{gathered}
\underbrace{\begingroup
\setlength\arraycolsep{1pt}
\begin{pmatrix}q^{(96)}\\q^{(97)}\\q^{(98)}\\q^{(99)}\\\vdots\\q^{(126)}\\q^{(127)}\end{pmatrix}^\top
\begin{pmatrix}
% \setstacktabbedgap{2pt}
\cos{\theta_{48}\Delta t}& -\sin{\theta_{48}\Delta t}&0&0&\cdots&0&0\\
\sin{\theta_{48}\Delta t}&\cos{\theta_{48}\Delta t}&0&0&\cdots&0&0 \\
0&0&\cos{\theta_{49}\Delta t}& -\sin{\theta_{49}\Delta t}&\cdots&0&0\\
0&0&\sin{\theta_{49}\Delta t}&\cos{\theta_{49}\Delta t}&\cdots&0&0 \\ 
\vdots&\vdots&\vdots&\vdots&\ddots&\vdots&\vdots\\
0&0&0&0&\cdots&\cos{\theta_{63}\Delta t}& -\sin{\theta_{63}\Delta t}\\
0&0&0&0&\cdots&\sin{\theta_{63}\Delta t}&\cos{\theta_{63}\Delta t}
\end{pmatrix}
\begin{pmatrix}k^{(96)}\\k^{(97)}\\k^{(98)}\\k^{(99)}\\\vdots\\k^{(126)}\\k^{(127)}\end{pmatrix}
\endgroup}_\text{\normalsize modeling temporal dependency with lower frequency} \\
+ \underbrace{\begingroup
\setlength\arraycolsep{1pt}
\begin{pmatrix}q^{(0)}\\q^{(1)}\\q^{(4)}\\q^{(5)}\\\vdots\\q^{(92)}\\q^{(93)}\end{pmatrix}^\top
\begin{pmatrix}
% \setstacktabbedgap{2pt}
\cos{\theta_{0}\Delta x}& -\sin{\theta_{0}\Delta x}&0&0&\cdots&0&0\\
\sin{\theta_{0}\Delta x}&\cos{\theta_{0}\Delta x}&0&0&\cdots&0&0 \\
0&0&\cos{\theta_{2}\Delta x}& -\sin{\theta_{2}\Delta x}&\cdots&0&0\\
0&0&\sin{\theta_{2}\Delta x}&\cos{\theta_{2}\Delta x}&\cdots&0&0 \\ 
\vdots&\vdots&\vdots&\vdots&\ddots&\vdots&\vdots\\
0&0&0&0&\cdots&\cos{\theta_{46}\Delta x}& -\sin{\theta_{46}\Delta x}\\
0&0&0&0&\cdots&\sin{\theta_{46}\Delta x}&\cos{\theta_{46}\Delta x}
\end{pmatrix}
\begin{pmatrix}k^{(0)}\\k^{(1)}\\k^{(4)}\\k^{(5)}\\\vdots\\k^{(92)}\\k^{(93)}\end{pmatrix}
\endgroup}_\text{\normalsize modeling horizontal dependency with interleaved high frequency} \\
+ \underbrace{\begingroup
\setlength\arraycolsep{1pt}
\begin{pmatrix}q^{(2)}\\q^{(3)}\\q^{(6)}\\q^{(7)}\\\vdots\\q^{(94)}\\q^{(95)}\end{pmatrix}^\top
\begin{pmatrix}
% \setstacktabbedgap{2pt}
\cos{\theta_{1}\Delta y}& -\sin{\theta_{1}\Delta y}&0&0&\cdots&0&0\\
\sin{\theta_{1}\Delta y}&\cos{\theta_{1}\Delta y}&0&0&\cdots&0&0 \\
0&0&\cos{\theta_{3}\Delta y}& -\sin{\theta_{3}\Delta y}&\cdots&0&0\\
0&0&\sin{\theta_{3}\Delta y}&\cos{\theta_{3}\Delta y}&\cdots&0&0 \\ 
\vdots&\vdots&\vdots&\vdots&\ddots&\vdots&\vdots\\
0&0&0&0&\cdots&\cos{\theta_{47}\Delta y}& -\sin{\theta_{47}\Delta y}\\
0&0&0&0&\cdots&\sin{\theta_{47}\Delta y}&\cos{\theta_{47}\Delta y}
\end{pmatrix}
\begin{pmatrix}k^{(2)}\\k^{(3)}\\k^{(6)}\\k^{(7)}\\\vdots\\k^{(94)}\\k^{(95)}\end{pmatrix}
\endgroup}_\text{\normalsize modeling vertical dependency with interleaved high frequency} \\
% \Delta t=t_1-t_2,\quad \Delta x=x_1-x_2,\quad \Delta y=y_1-y_2 \\
% \theta_n=\beta^{-\dfrac{2n}{d}},\quad n=0,\cdots,d/2-1
\end{gathered}
$
}
% \raisebox{-5.5ex}{.}
\label{equ:videorope}
\end{equation}
The horizontal position $x$ and vertical position $y$ are interleaved to occupy the lower dimensions, followed by temporal $t$, which occupies the higher dimensions. We keep the same allocation number for $x$, $y$, and $t$ as M-RoPE for a fair comparison, with values of 48, 48, and 32, respectively.
The advantages of this distribution are evident in Fig.  \ref{fig:period_mono}. 
For a RoPE-based LLM with a 128-dimensional head (64 rotary angles $\theta_n$), we visualize the function of $\cos{\theta_n t}$ for 3 dimensions using parallel blue planes.

As shown in Fig. \ref{fig:period_mono} (\textbf{a}), M-RoPE's temporal position embeddings are significantly distorted by periodic oscillations \cite{men2024base}, leading to identical embeddings for distant positions.
For instance, considering the last three rotary angles, the temporal embeddings are severely affected by these oscillations due to their short monotonic intervals (and even shorter intervals in lower dimensions).
This periodicity creates ``hash collisions'' (red planes), where distant positions share near-identical embeddings, making the model susceptible to distractor influence.
Fortunately, our \methodname (Fig. \ref{fig:period_mono} (\textbf{b})) is free from oscillation and Hash collision in temporal modeling.
The visualized relationship between the periodicity, monotonicity, and temporal modeling.

\begin{figure}[t]
\centering
\includegraphics[width=0.98\linewidth]{figures/files/video_rope_figure_spatial_v2.pdf}
\vspace{-6pt}
\caption{\footnotesize The position embeddings of adjacent text tokens for Vanilla RoPE (\textbf{top} row), the corresponding visual tokens in adjacent frames for M-RoPE (\textbf{middle} row) and our \methodname (\textbf{bottom} row) with interleaved spatial and temporal last design.}
\vspace{-12pt}
\label{fig:spatail_index}
\end{figure}

\begin{figure*}[t]
\begin{minipage}{0.98\textwidth}
    \begin{subfigure}[b]{0.3\linewidth}
        \centering
        \includegraphics[width=0.95\linewidth]{figures/files/vanilla_rope.pdf}
        \caption{3D visualization for Vanilla RoPE.}
        \label{fig:vanilla_rope}
    \end{subfigure}
    \hfill
    \begin{subfigure}[b]{0.3\linewidth}
        \centering
        \includegraphics[width=0.95\linewidth]{figures/files/m_rope.pdf}
        \caption{3D visualization for M-RoPE.}
        \label{fig:m_rope}
    \end{subfigure}
    \hfill
    \begin{subfigure}[b]{0.3\linewidth}
        \centering
        \includegraphics[width=0.95\linewidth]{figures/files/m_modify_rope.pdf}
        \caption{3D visualization for \methodname.}
        \label{fig:video_rope}
    \end{subfigure}
    \hfill
    \vspace{-6pt}
    \caption{\footnotesize The 3D visualization for different position embedding. \textbf{(a)} The vanilla 1D RoPE~\cite{su2024roformer} does not incorporate spatial modeling.
    \textbf{(b)} M-RoPE~\cite{wang2024qwen2}, while have the 3D structure, introduces a discrepancy in index growth for visual tokens across frames, with some indices remaining constant.
    \textbf{(c)} In contrast, our \methodname achieves the desired balance, maintaining the consistent index growth pattern of vanilla RoPE while simultaneously incorporating spatial modeling. 
    }
    %3D visualization of different position embeddings. \textbf{(a)} The vanilla 1D RoPE~\cite{su2024roformer} lacks spatial modeling. \textbf{(b)} M-RoPE~\cite{wang2024qwen2}, while have the 3D structure, introduces a discrepancy in index growth for visual tokens across frames, with some indices remaining constant. \textbf{(c)} Our \methodname balances the index growth of vanilla RoPE while incorporating spatial modeling. For more details on the index, see Appendix \ref{app:supp_explain_modules}.
    \vspace{-12pt}
    \label{fig:spatial}
\end{minipage}
\end{figure*}

\noindent \textbf{Diagonal Layout.}
Fig. \ref{fig:spatial} provides a visual comparison of spatial symmetry in positional encodings.
For vanilla RoPE (Fig. ~\ref{fig:vanilla_rope}), no spatial relation is considered and the index for every dimension increases directly.
While M-RoPE (Fig. \ref{fig:m_rope}), incorporates spatial information within each frame, it introduces two significant discontinuities between textual and visual tokens.
This arises from M-RoPE's placement strategy, if the first visual token is at $(0, 0)$, the last token in each frame will always be placed at $(W-1, H-1)$, creating a stack in the bottom-left corner.
Furthermore, like vanilla RoPE, M-RoPE's indices increase with input length across all dimensions.

To address these limitations, \methodname arranges the entire input along the diagonal, see Fig. \ref{fig:video_rope}.
The central patch's 3D position for each video frame is $(t,t,t)$, with other patches offset in all directions.
Our \textbf{Diagonal Layout} has two advantages: (1) our design preserves the relative positions of visual tokens and ensures approximate equidistance from the image corners to the center, preventing text tokens from being overly close to any corner. (2) It maintains the indexing pattern of vanilla RoPE (Fig.  \ref{fig:spatail_index}), as the position index increment between corresponding spatial locations in adjacent frames mirrors that of adjacent textual tokens.

\noindent \textbf{Adjustable Temporal Spacing.}
To scale the temporal index, we introduce a scaling factor $\delta$ to better align temporal information between visual and textual tokens.

Suppose the symbol $\tau$ denotes the token index, for the starting text ($0 \leq \tau < T_s$), the temporal, horizontal, and vertical indices are simply set to the raw token index $\tau$.
For the video input ($T_s \leq \tau < T_s + T_v$), The difference $\tau - T_s$ represents the index of the current frame relative to the start of the video, which is then scaled by $\delta$ to control the space in the temporal dimension.
For the ending text ($T_s + T_v \leq \tau < T_s + T_v + T_e$), the temporal, horizontal, and vertical index are the same, creating a linear progression.

According to our adjustable temporal spacing design, for a multi-modal input that consists of a text with $T_s$ tokens, a following video with $T_v$ frame with $W\times H$ patches in each frame, and an ending text with $T_e$ tokens, the position indices $(t, x, y)$ of \methodname for $\tau$-th textual token or $(\tau, w, h)$-th visual token are defined as Eq. (\ref{equ:index}):
\begin{equation}
\vspace{-3pt}
\resizebox{0.5\textwidth}{!}{$
    \footnotesize
    (t,x,y) =
    \begin{cases}
        (\tau, \tau, \tau) & \text{if } 0 \leq \tau < T_s \\[3ex]
        \left( 
        \begin{array}{l}
            T_s + \delta (\tau - T_s), \\
            T_s + \delta (\tau - T_s) + w - \frac{W}{2}, \\
            T_s + \delta (\tau - T_s) + h - \frac{H}{2}
        \end{array}
        \right) & \text{if } T_s \leq \tau < T_s + T_v \\[6ex]
        \left( 
        \begin{array}{l}
            T_s + \delta T_v + \tau, \\
            T_s + \delta T_v + \tau, \\
            T_s + \delta T_v + \tau
        \end{array}
        \right) & \text{if } T_s + T_v \leq \tau < T_s + T_v + T_e
    \end{cases}
$}
\raisebox{-9.5ex}{,}
\label{equ:index}
\end{equation}
where $w$ and $h$ represent the horizontal and vertical indices of the visual patch within the frame, respectively.

In summary, the parameter $\delta$ in our adjustable temporal spacing allows for a flexible and consistent way to encode the relative positions of text and video tokens.

\section{User Study Design} \label{sec:study} %1.5p
\section{User Study}
\label{section:user_study}

\subsection{Procedure}

The first author used FeedforwardChat for four weeks.

\subsection{Results and Reflections}





\section{Results \& Analysis} \label{sec:results} %1p
% \begin{table}[!t]
% \centering
% \scalebox{0.68}{
%     \begin{tabular}{ll cccc}
%       \toprule
%       & \multicolumn{4}{c}{\textbf{Intellipro Dataset}}\\
%       & \multicolumn{2}{c}{Rank Resume} & \multicolumn{2}{c}{Rank Job} \\
%       \cmidrule(lr){2-3} \cmidrule(lr){4-5} 
%       \textbf{Method}
%       &  Recall@100 & nDCG@100 & Recall@10 & nDCG@10 \\
%       \midrule
%       \confitold{}
%       & 71.28 &34.79 &76.50 &52.57 
%       \\
%       \cmidrule{2-5}
%       \confitsimple{}
%     & 82.53 &48.17
%        & 85.58 &64.91
     
%        \\
%        +\RunnerUpMiningShort{}
%     &85.43 &50.99 &91.38 &71.34 
%       \\
%       +\HyReShort
%         &- & -
%        &-&-\\
       
%       \bottomrule

%     \end{tabular}
%   }
% \caption{Ablation studies using Jina-v2-base as the encoder. ``\confitsimple{}'' refers using a simplified encoder architecture. \framework{} trains \confitsimple{} with \RunnerUpMiningShort{} and \HyReShort{}.}
% \label{tbl:ablation}
% \end{table}
\begin{table*}[!t]
\centering
\scalebox{0.75}{
    \begin{tabular}{l cccc cccc}
      \toprule
      & \multicolumn{4}{c}{\textbf{Recruiting Dataset}}
      & \multicolumn{4}{c}{\textbf{AliYun Dataset}}\\
      & \multicolumn{2}{c}{Rank Resume} & \multicolumn{2}{c}{Rank Job} 
      & \multicolumn{2}{c}{Rank Resume} & \multicolumn{2}{c}{Rank Job}\\
      \cmidrule(lr){2-3} \cmidrule(lr){4-5} 
      \cmidrule(lr){6-7} \cmidrule(lr){8-9} 
      \textbf{Method}
      & Recall@100 & nDCG@100 & Recall@10 & nDCG@10
      & Recall@100 & nDCG@100 & Recall@10 & nDCG@10\\
      \midrule
      \confitold{}
      & 71.28 & 34.79 & 76.50 & 52.57 
      & 87.81 & 65.06 & 72.39 & 56.12
      \\
      \cmidrule{2-9}
      \confitsimple{}
      & 82.53 & 48.17 & 85.58 & 64.91
      & 94.90&78.40 & 78.70& 65.45
       \\
      +\HyReShort{}
       &85.28 & 49.50
       &90.25 & 70.22
       & 96.62&81.99 & \textbf{81.16}& 67.63
       \\
      +\RunnerUpMiningShort{}
       % & 85.14& 49.82
       % &90.75&72.51
       & \textbf{86.13}&\textbf{51.90} & \textbf{94.25}&\textbf{73.32}
       & \textbf{97.07}&\textbf{83.11} & 80.49& \textbf{68.02}
       \\
   %     +\RunnerUpMiningShort{}
   %    & 85.43 & 50.99 & 91.38 & 71.34 
   %    & 96.24 & 82.95 & 80.12 & 66.96
   %    \\
   %    +\HyReShort{} old
   %     &85.28 & 49.50
   %     &90.25 & 70.22
   %     & 96.62&81.99 & 81.16& 67.63
   %     \\
   % +\HyReShort{} 
   %     % & 85.14& 49.82
   %     % &90.75&72.51
   %     & 86.83&51.77 &92.00 &72.04
   %     & 97.07&83.11 & 80.49& 68.02
   %     \\
      \bottomrule

    \end{tabular}
  }
\caption{\framework{} ablation studies. ``\confitsimple{}'' refers using a simplified encoder architecture. \framework{} trains \confitsimple{} with \RunnerUpMiningShort{} and \HyReShort{}. We use Jina-v2-base as the encoder due to its better performance.
}
\label{tbl:ablation}
\end{table*}

\section{Results}
\label{sec:results}

In this section, we present detailed results demonstrating \emph{CellFlow}'s state-of-the-art performance in cellular morphology prediction under perturbations, outperforming existing methods across multiple datasets and evaluation metrics.

\subsection{Datasets}

Our experiments were conducted using three cell imaging perturbation datasets: BBBC021 (chemical perturbation)~\cite{caie2010high}, RxRx1 (genetic perturbation)~\cite{sypetkowski2023rxrx1}, and the JUMP dataset (combined perturbation)~\cite{chandrasekaran2023jump}. We followed the preprocessing protocol from IMPA~\cite{palma2023predicting}, which involves correcting illumination, cropping images centered on nuclei to a resolution of 96×96, and filtering out low-quality images. The resulting datasets include 98K, 171K, and 424K images with 3, 5, and 6 channels, respectively, from 26, 1,042, and 747 perturbation types. Examples of these images are provided in Figure~\ref{fig:comparison}. Details of datasets are provided in \S\ref{sec:data}.

\subsection{Experimental Setup}

\textbf{Evaluation metrics.} We evaluate methods using two types of metrics: (1) FID and KID, which measure image distribution similarity via Fréchet and kernel-based distances, computed on 5K generated images for BBBC021 and 100 randomly selected perturbation classes for RxRx1 and JUMP; we report both overall scores across all samples and conditional scores per perturbation class. (2) Mode of Action (MoA) classification accuracy, which assesses biological fidelity by using a trained classifier to predict a drug’s effect from perturbed images and comparing it to its known MoA from the literature.

\textbf{Baselines.} We compare our approach against two baselines, PhenDiff~\cite{bourou2024phendiff} and IMPA~\cite{palma2023predicting}, the only two baselines that incorporate control images into their model design --- a crucial setup for distinguishing true perturbation effects from artifacts such as batch effects. PhenDiff uses diffusion models to first map control images to noise and then transform the noise into target images. In contrast, IMPA employs GANs with an AdaIN layer to transfer the style of control images to target images, specifically designed for paired image-to-image mappings. Our method uses flow matching, which is tailored for distribution-to-distribution mapping, providing a more suitable solution for our problem. We reproduce these baselines with official codes.

\textbf{Training details.} \emph{CellFlow} employs a UNet-based velocity field with a four-stage design. Perturbations are encoded following IMPA~\cite{palma2023predicting}. Training is conducted for 100 epochs on 4 A100 GPUs. Details are in \S\ref{sec:experimental}.


\begin{table*}[ht!]
\centering
\footnotesize
\renewcommand{\arraystretch}{1.2}
\begin{tabular}{p{2.15cm} p{14.05cm}}
\toprule
 & \includegraphics[height=2.6cm]{figure/qualitative_example.pdf}\\
\midrule
Original \phantom{00000} Description & Bromazepam binds to the GABA receptor GABA-A, causing a conformational change and \textbf{increasing inhibitory effects of GABA}. Other neurotransmitters are not influenced. One of the \textbf{benzodiazepines} that is \textbf{used in the treatment of anxiety disorders.} \\
\midrule
User & Explain the features of the given molecule that are most distinct from those of other molecules. \\
\midrule
Mol-LLaMA (Ours) & The given molecule is a member of the \textcolor{tabblue}{benzodiazepine class}, which is characterized by a fusion of a benzene ring and a seven-membered diazepine ring. This structure is crucial for its biological activity, particularly its role as \textcolor{tabblue}{a sedative and hypnotic agent}. ...$<$omitted$>$...\\

& The given molecule is also characterized by a ketone group within the diazepine ring, which is a common feature in benzodiazepines. \textcolor{tabblue}{This functional group is important for the interaction with the GABA-A receptor, enhancing the inhibitory effects of GABA}. ...$<$omitted$>$...\\

\midrule
GPT-4o & 1. **Quinazoline Core**: The molecule contains a \textcolor{tabred}{quinazoline core}, ...$<$omitted$>$... \\
 & Overall, the combination of a \textcolor{tabred}{quinazoline} structure with unique substitution patterns, including bromine halogenation and the presence of a pyridinyl group, ...$<$omitted$>$... . \\
\midrule
LLaMo & The molecule has a \textcolor{tabblue}{benzodiazepine} structure with a bromo substituent at the 7-position and \textcolor{tabred}{a methyl group at the 1-position}. It is distinct from other molecules in that it contains a bromine atom, a nitrogen atom, and a methyl group, which are not present in the other molecules. \\
\midrule
3D-MoLM & ...$<$omitted$>$... It belongs to the class of compounds called \textcolor{tabred}{quinazolines}, which are characterized by a bicyclic structure consisting of a benzene ring fused to a pyrimidine ring. ...$<$omitted$>$...\\
& The molecule's structure suggests potential applications in medicinal chemistry, as quinazolines have been found to possess various biological activities, including \textcolor{tabred}{antitumor, antimicrobial, and anti-inflammatory properties.} \\
\midrule
Mol-Instructions & The molecule is a \textcolor{tabred}{quinoxaline derivative}. \\
\bottomrule
\end{tabular}
\vspace{-0.1in}
\caption{\small Case study to compare molecular understanding and reasoning ability. Mol-LLaMA accurately understands the molecular features, answering a correct molecular taxonomy and providing its distinct properties that are relevant to the given molecule.}
\label{tab:qualitative}
\vspace{-0.1in}
\end{table*}

\subsection{Main Results}

\textbf{\emph{CellFlow} generates highly realistic cell images.}  
\emph{CellFlow} outperforms existing methods in capturing cellular morphology across all datasets (Table~\ref{tab:results}a), achieving overall FID scores of 18.7, 33.0, and 9.0 on BBBC021, RxRx1, and JUMP, respectively --- improving FID by 21\%–45\% compared to previous methods. These gains in both FID and KID metrics confirm that \emph{CellFlow} produces significantly more realistic cell images than prior approaches.

\textbf{\emph{CellFlow} accurately captures perturbation-specific morphological changes.}  
As shown in Table~\ref{tab:results}a, \emph{CellFlow} achieves conditional FID scores of 56.8 (a 26\% improvement), 163.5, and 84.4 (a 16\% improvement) on BBBC021, RxRx1, and JUMP, respectively. These scores are computed by measuring the distribution distance for each specific perturbation and averaging across all perturbations.   
Table~\ref{tab:results}b further highlights \emph{CellFlow}’s performance on six representative chemical and three genetic perturbations. For chemical perturbations, \emph{CellFlow} reduces FID scores by 14–55\% compared to prior methods.
The smaller improvement (5–12\% improvements) on RxRx1 is likely due to the limited number of images per perturbation type.

\textbf{\emph{CellFlow} preserves biological fidelity across perturbation conditions.} 
Table~\ref{tab:ablation}a presents mode of action (MoA) classification accuracy on the BBBC021 dataset using generated cell images. MoA describes how a drug affects cellular function and can be inferred from morphology. To assess this, we train an image classifier on real perturbed images and test it on generated ones. \emph{CellFlow} achieves 71.1\% MoA accuracy, closely matching real images (72.4\%) and significantly surpassing other methods (best: 63.7\%), demonstrating its ability to maintain biological fidelity across perturbations. Qualitative comparisons in Figure~\ref{fig:comparison} further highlight \emph{CellFlow}’s accuracy in capturing key biological effects. For example, demecolcine produces smaller, fragmented nuclei, which other methods fail to reproduce accurately.

\textbf{\emph{CellFlow} generalizes to out-of-distribution (OOD) perturbations.}  
On BBBC021, \emph{CellFlow} demonstrates strong generalization to novel chemical perturbations never seen during training (Table~\ref{tab:ablation}b). It achieves 6\% and 28\% improvements in overall and conditional FID over the best baseline. This OOD generalization is critical for biological research, enabling the exploration of previously untested interventions and the design of new drugs.

\textbf{Ablations highlight the importance of each component in \emph{CellFlow}.}  
Table~\ref{tab:ablation}c shows that removing conditional information, classifier-free guidance, or noise augmentation significantly degrades performance, leading to higher FID scores. These underscore the critical role of each component in enabling \emph{CellFlow}’s state-of-the-art performance.  

\begin{figure*}[!tb]
    \centering
     \includegraphics[width=\linewidth]{imgs/interpolation.pdf}
     \vspace{-2em}
    \caption{
    \textbf{\emph{CellFlow} enables new capabilities.} 
\textit{(a.1) Batch effect calibration.}  
\emph{CellFlow} initializes with control images, enabling batch-specific predictions. Comparing predictions from different batches highlights actual perturbation effects (smaller cell size) while filtering out spurious batch effects (cell density variations).  
\textit{(a.2) Interpolation trajectory.}  
\emph{CellFlow}'s learned velocity field supports interpolation between cell states, which might provide insights into the dynamic cell trajectory. 
\textit{(b) Diffusion model comparison.}  
Unlike flow matching, diffusion models that start from noise cannot calibrate batch effects or support interpolation.  
\textit{(c) Reverse trajectory.}  
\emph{CellFlow}'s reversible velocity field can predict prior cell states from perturbed images, offering potential applications such as restoring damaged cells.
    }
    \label{fig:interpolation}
    \vspace{-1em}
\end{figure*}

\subsection{New Capabilities}

\textbf{\emph{CellFlow} addresses batch effects and reveals true perturbation effects.}  
\emph{CellFlow}’s distribution-to-distribution approach effectively addresses batch effects, a significant challenge in biological experimental data collection. As shown in Figure~\ref{fig:interpolation}a, when conditioned on two distinct control images with varying cell densities from different batches, \emph{CellFlow} consistently generates the expected perturbation effect (cell shrinkage due to mevinolin) while recapitulating batch-specific artifacts, revealing the true perturbation effect. Table~\ref{tab:ablation}d further quantifies the importance of conditioning on the same batch. By comparing generated images conditioned on control images from the same or different batches against the target perturbation images, we find that same-batch conditioning reduces overall and conditional FID by 21\%. This highlights the importance of modeling control images to more accurately capture true perturbation effects—an aspect often overlooked by prior approaches, such as diffusion models that initialize from noise (Figure~\ref{fig:interpolation}b).

\textbf{\emph{CellFlow} has the potential to model cellular morphological change trajectories.}
Cell trajectories could offer valuable information about perturbation mechanisms, but capturing them with current imaging technologies remains challenging due to their destructive nature. Since \emph{CellFlow} continuously transforms the source distribution into the target distribution, it can generate smooth interpolation paths between initial and final predicted cell states, producing video-like sequences of cellular transformation based on given source images (Figure~\ref{fig:interpolation}a). This suggests a possible approach for simulating morphological trajectories during perturbation response, which diffusion methods cannot achieve (Figure~\ref{fig:interpolation}b). Additionally, the reversible distribution transformation learned through flow matching enables \emph{CellFlow} to model backward cell state reversion (Figure~\ref{fig:interpolation}c), which could be useful for studying recovery dynamics and predicting potential treatment outcomes.


\section{Discussion} \label{sec:conclu} %0.5p 
\paragraph{Summary}
Our findings provide significant insights into the influence of correctness, explanations, and refinement on evaluation accuracy and user trust in AI-based planners. 
In particular, the findings are three-fold: 
(1) The \textbf{correctness} of the generated plans is the most significant factor that impacts the evaluation accuracy and user trust in the planners. As the PDDL solver is more capable of generating correct plans, it achieves the highest evaluation accuracy and trust. 
(2) The \textbf{explanation} component of the LLM planner improves evaluation accuracy, as LLM+Expl achieves higher accuracy than LLM alone. Despite this improvement, LLM+Expl minimally impacts user trust. However, alternative explanation methods may influence user trust differently from the manually generated explanations used in our approach.
% On the other hand, explanations may help refine the trust of the planner to a more appropriate level by indicating planner shortcomings.
(3) The \textbf{refinement} procedure in the LLM planner does not lead to a significant improvement in evaluation accuracy; however, it exhibits a positive influence on user trust that may indicate an overtrust in some situations.
% This finding is aligned with prior works showing that iterative refinements based on user feedback would increase user trust~\cite{kunkel2019let, sebo2019don}.
Finally, the propensity-to-trust analysis identifies correctness as the primary determinant of user trust, whereas explanations provided limited improvement in scenarios where the planner's accuracy is diminished.

% In conclusion, our results indicate that the planner's correctness is the dominant factor for both evaluation accuracy and user trust. Therefore, selecting high-quality training data and optimizing the training procedure of AI-based planners to improve planning correctness is the top priority. Once the AI planner achieves a similar correctness level to traditional graph-search planners, strengthening its capability to explain and refine plans will further improve user trust compared to traditional planners.

\paragraph{Future Research} Future steps in this research include expanding user studies with larger sample sizes to improve generalizability and including additional planning problems per session for a more comprehensive evaluation. Next, we will explore alternative methods for generating plan explanations beyond manual creation to identify approaches that more effectively enhance user trust. 
Additionally, we will examine user trust by employing multiple LLM-based planners with varying levels of planning accuracy to better understand the interplay between planning correctness and user trust. 
Furthermore, we aim to enable real-time user-planner interaction, allowing users to provide feedback and refine plans collaboratively, thereby fostering a more dynamic and user-centric planning process.


%%
%% The acknowledgments section is defined using the "acks" environment
%% (and NOT an unnumbered section). This ensures the proper
%% identification of the section in the article metadata, and the
%% consistent spelling of the heading.
% \begin{acks}
% acknowledgement if any
% \end{acks}

%%
%% The next two lines define the bibliography style to be used, and
%% the bibliography file.
\bibliographystyle{ACM-Reference-Format}
\bibliography{references}


%%
%% If your work has an appendix, this is the place to put it.
\appendix
\section{Secure Token Pruning Protocols}
\label{app:a}
We detail the encrypted token pruning protocols $\Pi_{prune}$ in Figure \ref{fig:protocol-prune} and $\Pi_{mask}$ in Figure \ref{fig:protocol-mask} in this section.

%Optionally include supplemental material (complete proofs, additional experiments and plots) in appendix.
%All such materials \textbf{SHOULD be included in the main submission.}
\begin{figure}[h]
%vspace{-0.2in}
\begin{protocolbox}
\noindent
\textbf{Parties:} Server $P_0$, Client $P_1$.

\textbf{Input:} $P_0$ and $P_1$ holds $\{ \left \langle Att \right \rangle_{0}^{h}, \left \langle Att \right \rangle_{1}^{h}\}_{h=0}^{H-1} \in \mathbb{Z}_{2^{\ell}}^{n\times n}$ and $\left \langle x \right \rangle_{0}, \left \langle x \right \rangle_{1} \in \mathbb{Z}_{2^{\ell}}^{n\times D}$ respectively, where H is the number of heads, n is the number of input tokens and D is the embedding dimension of tokens. Additionally, $P_1$ holds a threshold $\theta \in \mathbb{Z}_{2^{\ell}}$.

\textbf{Output:} $P_0$ and $P_1$ get $\left \langle y \right \rangle_{0}, \left \langle y \right \rangle_{1} \in \mathbb{Z}_{2^{\ell}}^{n'\times D}$, respectively, where $y=\mathsf{Prune}(x)$ and $n'$ is the number of remaining tokens.

\noindent\rule{13.2cm}{0.1pt} % This creates the horizontal line
\textbf{Protocol:}
\begin{enumerate}[label=\arabic*:, leftmargin=*]
    \item For $h \in [H]$, $P_0$ and $P_1$ compute locally with input $\left \langle Att \right \rangle^{h}$, and learn the importance score in each head $\left \langle s \right \rangle^{h} \in \mathbb{Z}_{2^{\ell}}^{n} $, where $\left \langle s \right \rangle^{h}[j] = \frac{1}{n} \sum_{i=0}^{n-1} \left \langle Att \right \rangle^{h}[i,j]$.
    \item $P_0$ and $P_1$ compute locally with input $\{ \left \langle s \right \rangle^{i} \in \mathbb{Z}_{2^{\ell}}^{n}  \}_{i=0}^{H-1}$, and learn the final importance score $\left \langle S \right \rangle \in \mathbb{Z}_{2^{\ell}}^{n}$ for each token, where  $\left \langle S \right \rangle[i] = \frac{1}{H} \sum_{h=0}^{H-1} \left \langle s \right \rangle^{h}[i]$.
    \item  For $i \in [n]$, $P_0$ and $P_1$ invoke $\Pi_{CMP}$ with inputs  $\left \langle S \right \rangle$ and $ \theta $, and learn  $\left \langle M \right \rangle \in \mathbb{Z}_{2^{\ell}}^{n}$, such that$\left \langle M \right \rangle[i] = \Pi_{CMP}(\left \langle S \right \rangle[i] - \theta) $, where: \\
    $M[i] = \begin{cases}
        1  &\text{if}\ S[i] > \theta, \\
        0  &\text{otherwise}.
            \end{cases} $
    % \item If the pruning location is insensitive, $P_0$ and $P_1$ learn real mask $M$ instead of shares $\left \langle M \right \rangle$. $P_0$ and $P_1$ compute $\left \langle y \right \rangle$ with input $\left \langle x \right \rangle$ and $M$, where  $\left \langle x \right \rangle[i]$ is pruned if $M[i]$ is $0$.
    \item $P_0$ and $P_1$ invoke $\Pi_{mask}$ with inputs  $\left \langle x \right \rangle$ and pruning mask $\left \langle M \right \rangle$, and set outputs as $\left \langle y \right \rangle$.
\end{enumerate}
\end{protocolbox}
\setlength{\abovecaptionskip}{-1pt} % Reduces space above the caption
\caption{Secure Token Pruning Protocol $\Pi_{prune}$.}
\label{fig:protocol-prune}
\end{figure}




\begin{figure}[h]
\begin{protocolbox}
\noindent
\textbf{Parties:} Server $P_0$, Client $P_1$.

\textbf{Input:} $P_0$ and $P_1$ hold $\left \langle x \right \rangle_{0}, \left \langle x \right \rangle_{1} \in \mathbb{Z}_{2^{\ell}}^{n\times D}$ and  $\left \langle M \right \rangle_{0}, \left \langle M \right \rangle_{1} \in \mathbb{Z}_{2^{\ell}}^{n}$, respectively, where n is the number of input tokens and D is the embedding dimension of tokens.

\textbf{Output:} $P_0$ and $P_1$ get $\left \langle y \right \rangle_{0}, \left \langle y \right \rangle_{1} \in \mathbb{Z}_{2^{\ell}}^{n'\times D}$, respectively, where $y=\mathsf{Prune}(x)$ and $n'$ is the number of remaining tokens.

\noindent\rule{13.2cm}{0.1pt} % This creates the horizontal line
\textbf{Protocol:}
\begin{enumerate}[label=\arabic*:, leftmargin=*]
    \item For $i \in [n]$, $P_0$ and $P_1$ set $\left \langle M \right \rangle$ to the MSB of $\left \langle x \right \rangle$ and learn the masked tokens $\left \langle \Bar{x} \right \rangle \in Z_{2^{\ell}}^{n\times D}$, where
    $\left \langle \Bar{x}[i] \right \rangle = \left \langle x[i] \right \rangle + (\left \langle M[i] \right \rangle << f)$ and $f$ is the fixed-point precision.
    \item $P_0$ and $P_1$ compute the sum of $\{\Pi_{B2A}(\left \langle M \right \rangle[i]) \}_{i=0}^{n-1}$, and learn the number of remaining tokens $n'$ and the number of tokens to be pruned $m$, where $m = n-n'$.
    \item For $k\in[m]$, for $i\in[n-k-1]$, $P_0$ and $P_1$ invoke $\Pi_{msb}$ to learn the highest bit of $\left \langle \Bar{x}[i] \right \rangle$, where $b=\mathsf{MSB}(\Bar{x}[i])$. With the highest bit of $\Bar{x}[i]$, $P_0$ and $P_1$ perform a oblivious swap between $\Bar{x}[i]$ and $\Bar{x}[i+1]$:
    $\begin{cases}
        \Tilde{x}[i] = b\cdot \Bar{x}[i] + (1-b)\cdot \Bar{x}[i+1] \\
        \Tilde{x}[i+1] = b\cdot \Bar{x}[i+1] + (1-b)\cdot \Bar{x}[i]
    \end{cases} $ \\
    $P_0$ and $P_1$ learn the swapped token sequence $\left \langle \Tilde{x} \right \rangle$.
    \item $P_0$ and $P_1$ truncate $\left \langle \Tilde{x} \right \rangle$ locally by keeping the first $n'$ tokens, clear current MSB (all remaining token has $1$ on the MSB), and set outputs as $\left \langle y \right \rangle$.
\end{enumerate}
\end{protocolbox}
\setlength{\abovecaptionskip}{-1pt} % Reduces space above the caption
\caption{Secure Mask Protocol $\Pi_{mask}$.}
\label{fig:protocol-mask}
%\vspace{-0.2in}
\end{figure}

% \begin{wrapfigure}{r}{0.35\textwidth}  % 'r' for right, and the width of the figure area
%   \centering
%   \includegraphics[width=0.35\textwidth]{figures/msb.pdf}
%   \caption{Runtime of $\Pi_{prune}$ and $\Pi_{mask}$ in different layers. We compare different secure pruning strategies based on the BERT Base model.}
%   \label{fig:msb}
%   \vspace{-0.1in}
% \end{wrapfigure}

% \begin{figure}[h]  % 'r' for right, and the width of the figure area
%   \centering
%   \includegraphics[width=0.4\textwidth]{figures/msb.pdf}
%   \caption{Runtime of $\Pi_{prune}$ and $\Pi_{mask}$ in different layers. We compare different secure pruning strategies based on the BERT Base model.}
%   \label{fig:msb}
%   % \vspace{-0.1in}
% \end{figure}

\textbf{Complexity of $\Pi_{mask}$.} The complexity of the proposed $\Pi_{mask}$ mainly depends on the number of oblivious swaps. To prune $m$ tokens out of $n$ input tokens, $O(mn)$ swaps are needed. Since token pruning is performed progressively, only a small number of tokens are pruned at each layer, which makes $\Pi_{mask}$ efficient during runtime. Specifically, for a BERT base model with 128 input tokens, the pruning protocol only takes $\sim0.9$s on average in each layer. An alternative approach is to invoke an oblivious sort algorithm~\citep{bogdanov2014swap2,pang2023bolt} on $\left \langle \Bar{x} \right \rangle$. However, this approach is less efficient because it blindly sort the whole token sequence without considering $m$. That is, even if only $1$ token needs to be pruned, $O(nlog^{2}n)\sim O(n^2)$ oblivious swaps are needed, where as the proposed $\Pi_{mask}$ only need $O(n)$ swaps. More generally, for an $\ell$-layer Transformer with a total of $m$ tokens pruned, the overall time complexity using the sort strategy would be $O(\ell n^2)$ while using the swap strategy remains an overall complexity of $O(mn).$ Specifically, using the sort strategy to prune tokens in one BERT Base model layer can take up to $3.8\sim4.5$ s depending on the sorting algorithm used. In contrast, using the swap strategy only needs $0.5$ s. Moreover, alternative to our MSB strategy, one can also swap the encrypted mask along with the encrypted token sequence. However, we find that this doubles the number of swaps needed, and thus is less efficient the our MSB strategy, as is shown in Figure \ref{fig:msb}.

\section{Existing Protocols}
\label{app:protocol}
\noindent\textbf{Existing Protocols Used in Our Private Inference.}  In our private inference framework, we reuse several existing cryptographic protocols for basic computations. $\Pi_{MatMul}$ \citep{pang2023bolt} processes two ASS matrices and outputs their product in SS form. For non-linear computations, protocols $\Pi_{SoftMax}, \Pi_{GELU}$, and $\Pi_{LayerNorm}$\citep{lu2023bumblebee, pang2023bolt} take a secret shared tensor and return the result of non-linear functions in ASS. Basic protocols from~\citep{rathee2020cryptflow2, rathee2021sirnn} are also utilized. $\Pi_{CMP}$\citep{EzPC}, for example, inputs ASS values and outputs a secret shared comparison result, while $\Pi_{B2A}$\citep{EzPC} converts secret shared Boolean values into their corresponding arithmetic values.

\section{Polynomial Reduction for Non-linear Functions}
\label{app:b}
The $\mathsf{SoftMax}$ and $\mathsf{GELU}$ functions can be approximated with polynomials. High-degree polynomials~\citep{lu2023bumblebee, pang2023bolt} can achieve the same accuracy as the LUT-based methods~\cite{hao2022iron-iron}. While these polynomial approximations are more efficient than look-up tables, they can still incur considerable overheads. Reducing the high-degree polynomials to the low-degree ones for the less important tokens can imporve efficiency without compromising accuracy. The $\mathsf{SoftMax}$ function is applied to each row of an attention map. If a token is to be reduced, the corresponding row will be computed using the low-degree polynomial approximations. Otherwise, the corresponding row will be computed accurately via a high-degree one. That is if $M_{\beta}'[i] = 1$, $P_0$ and $P_1$ uses high-degree polynomials to compute the $\mathsf{SoftMax}$ function on token $x[i]$:
\begin{equation}
\mathsf{SoftMax}_{i}(x) = \frac{e^{x_i}}{\sum_{j\in [d]}e^{x_j}}
\end{equation}
where $x$ is a input vector of length $d$ and the exponential function is computed via a polynomial approximation. For the $\mathsf{SoftMax}$ protocol, we adopt a similar strategy as~\citep{kim2021ibert, hao2022iron-iron}, where we evaluate on the normalized inputs $\mathsf{SoftMax}(x-max_{i\in [d]}x_i)$. Different from~\citep{hao2022iron-iron}, we did not used the binary tree to find max value in the given vector. Instead, we traverse through the vector to find the max value. This is because each attention map is computed independently and the binary tree cannot be re-used. If $M_{\beta}[i] = 0$, $P_0$ and $P_1$ will approximate the $\mathsf{SoftMax}$ function with low-degree polynomial approximations. We detail how $\mathsf{SoftMax}$ can be approximated as follows:
\begin{equation}
\label{eq:app softmax}
\mathsf{ApproxSoftMax}_{i}(x) = \frac{\mathsf{ApproxExp}(x_i)}{\sum_{j\in [d]}\mathsf{ApproxExp}(x_j)}
\end{equation}
\begin{equation}
\mathsf{ApproxExp}(x)=\begin{cases}
    0  &\text{if}\ x \leq T \\
    (1+ \frac{x}{2^n})^{2^n} &\text{if}\ x \in [T,0]\\
\end{cases}
\end{equation}
where the $2^n$-degree Taylor series is used to approximate the exponential function and $T$ is the clipping boundary. The value $n$ and $T$ determines the accuracy of above approximation. With $n=6$ and $T=-13$, the approximation can achieve an average error within $2^{-10}$~\citep{lu2023bumblebee}. For low-degree polynomial approximation, $n=3$ is used in the Taylor series.

Similarly, $P_0$ or $P_1$ can decide whether or not to approximate the $\mathsf{GELU}$ function for each token. If $M_{\beta}[i] = 1$, $P_0$ and $P_1$ use high-degree polynomials~\citep{lu2023bumblebee} to compute the $\mathsf{GELU}$ function on token $x[i]$ with high-degree polynomial:
% \begin{equation}
% \mathsf{GELU}(x) = 0.5x(1+\mathsf{Tanh}(\sqrt{2/\pi}(x+0.044715x^3)))
% \end{equation}
% where the $\mathsf{Tanh}$ and square root function are computed via a OT-based lookup-table.

\begin{equation}
\label{eq:app gelu}
\mathsf{ApproxGELU}(x)=\begin{cases}
    0  &\text{if}\ x \leq -5 \\
    P^3(x), &\text{if}\ -5 < x \leq -1.97 \\
    P^6(x), &\text{if}\ -1.97 < x \leq 3  \\
    x, &\text{if}\ x >3 \\
\end{cases}
\end{equation}
where $P^3(x)$ and $P^6(x)$ are degree-3 and degree-6 polynomials respectively. The detailed coefficient for the polynomial is: 
\begin{equation*}
    P^3(x) = -0.50540312 -  0.42226581x - 0.11807613x^2 - 0.01103413x^3
\end{equation*}
, and
\begin{equation*}
    P^6(x) = 0.00852632 + 0.5x + 0.36032927x^2 - 0.03768820x^4 + 0.00180675x^6
\end{equation*}

For BOLT baseline, we use another high-degree polynomial to compute the $\mathsf{GELU}$ function.

\begin{equation}
\label{eq:app gelu}
\mathsf{ApproxGELU}(x)=\begin{cases}
    0  &\text{if}\ x < -2.7 \\
    P^4(x), &\text{if}\   |x| \leq 2.7 \\
    x, &\text{if}\ x >2.7 \\
\end{cases}
\end{equation}
We use the same coefficients for $P^4(x)$ as BOLT~\citep{pang2023bolt}.

\begin{figure}[h]
 % \vspace{-0.1in}
    \centering
    \includegraphics[width=1\linewidth]{figures/bumble.pdf}
    % \captionsetup{skip=2pt}
    % \vspace{-0.1in}
    \caption{Comparison with prior works on the BERT model. The input has 128 tokens.}
    \label{fig:bumble}
\end{figure}

If $M_{\beta}'[i] = 0$, $P_0$ and $P_1$ will use low-degree 
polynomial approximation to compute the $\mathsf{GELU}$ function instead. Encrypted polynomial reduction leverages low-degree polynomials to compute non-linear functions for less important tokens. For the $\mathsf{GELU}$ function, the following degree-$2$ polynomial~\cite{kim2021ibert} is used:
\begin{equation*}
\mathsf{ApproxGELU}(x)=\begin{cases}
    0  &\text{if}\ x <  -1.7626 \\
    0.5x+0.28367x^2, &\text{if}\ x \leq |1.7626| \\
    x, &\text{if}\ x > 1.7626\\
\end{cases}
\end{equation*}


\section{Comparison with More Related Works.}
\label{app:c}
\textbf{Other 2PC frameworks.} The primary focus of CipherPrune is to accelerate the private Transformer inference in the 2PC setting. As shown in Figure \ref{fig:bumble}, CipherPrune can be easily extended to other 2PC private inference frameworks like BumbleBee~\citep{lu2023bumblebee}. We compare CipherPrune with BumbleBee and IRON on BERT models. We test the performance in the same LAN setting as BumbleBee with 1 Gbps bandwidth and 0.5 ms of ping time. CipherPrune achieves more than $\sim 60 \times$ speed up over BOLT and $4.3\times$ speed up over BumbleBee.

\begin{figure}[t]
 % \vspace{-0.1in}
    \centering
    \includegraphics[width=1\linewidth]{figures/pumab.pdf}
    % \captionsetup{skip=2pt}
    % \vspace{-0.1in}
    \caption{Comparison with MPCFormer and PUMA on the BERT models. The input has 128 tokens.}
    \label{fig:pumab}
\end{figure}

\begin{figure}[h]
 % \vspace{-0.1in}
    \centering
    \includegraphics[width=1\linewidth]{figures/pumag.pdf}
    % \captionsetup{skip=2pt}
    % \vspace{-0.1in}
    \caption{Comparison with MPCFormer and PUMA on the GPT2 models. The input has 128 tokens. The polynomial reduction is not used.}
    \label{fig:pumag}
\end{figure}

\textbf{Extension to 3PC frameworks.} Additionally, we highlight that CipherPrune can be also extended to the 3PC frameworks like MPCFormer~\citep{li2022mpcformer} and PUMA~\citep{dong2023puma}. This is because CipherPrune is built upon basic primitives like comparison and Boolean-to-Arithmetic conversion. We compare CipherPrune with MPCFormer and PUMA on both the BERT and GPT2 models. CipherPrune has a $6.6\sim9.4\times$ speed up over MPCFormer and $2.8\sim4.6\times$ speed up over PUMA on the BERT-Large and GPT2-Large models.


\section{Communication Reduction in SoftMax and GELU.}
\label{app:e}

\begin{figure}[h]
    \centering
    \includegraphics[width=0.9\linewidth]{figures/layerwise.pdf}
    \caption{Toy example of two successive Transformer layers. In layer$_i$, the SoftMax and Prune protocol have $n$ input tokens. The number of input tokens is reduced to $n'$ for the Linear layers, LayerNorm and GELU in layer$_i$ and SoftMax in layer$_{i+1}$.}
    \label{fig:layer}
\end{figure}

\begin{table*}[h]
\captionsetup{skip=2pt}
\centering
\scriptsize
\caption{Communication cost (in MB) of the SoftMax and GELU protocol in each Transformer layer.}
\begin{tblr}{
    colspec = {c |c c c c c c c c c c c c},
    row{1} = {font=\bfseries},
    row{2-Z} = {rowsep=1pt},
    % row{4} = {bg=LightBlue},
    colsep = 2.5pt,
    }
\hline
\textbf{Layer Index} & \textbf{0}  & \textbf{1}  & \textbf{2} & \textbf{3} & \textbf{4} & \textbf{5} & \textbf{6} & \textbf{7} & \textbf{8} & \textbf{9} & \textbf{10} & \textbf{11} \\
\hline
Softmax & 642.19 & 642.19 & 642.19 & 642.19 & 642.19 & 642.19 & 642.19 & 642.19 & 642.19 & 642.19 & 642.19 & 642.19 \\
Pruned Softmax & 642.19 & 129.58 & 127.89 & 119.73 & 97.04 & 71.52 & 43.92 & 21.50 & 10.67 & 6.16 & 4.65 & 4.03 \\
\hline
GELU & 698.84 & 698.84 & 698.84 & 698.84 & 698.84 & 698.84 & 698.84 & 698.84 & 698.84 & 698.84 & 698.84 & 698.84\\
Pruned GELU  & 325.10 & 317.18 & 313.43 & 275.94 & 236.95 & 191.96 & 135.02 & 88.34 & 46.68 & 16.50 & 5.58 & 5.58\\
\hline
\end{tblr}
\label{tab:layer}
\end{table*}

{
In Figure \ref{fig:layer}, we illustrate why CipherPrune can reduce the communication overhead of both  SoftMax and GELU. Suppose there are $n$ tokens in $layer_i$. Then, the SoftMax protocol in the attention module has a complexity of $O(n^2)$. CipherPrune's token pruning protocol is invoked to select $n'$ tokens out of all $n$ tokens, where $m=n-n'$ is the number of tokens that are removed. The overhead of the GELU function in $layer_i$, i.e., the current layer, has only $O(n')$ complexity (which should be $O(n)$ without token pruning). The complexity of the SoftMax function in $layer_{i+1}$, i.e., the following layer, is reduced to $O(n'^2)$ (which should be $O(n^2)$ without token pruning). The SoftMax protocol has quadratic complexity with respect to the token number and the GELU protocol has linear complexity. Therefore, CipherPrune can reduce the overhead of both the GELU protocol and the SoftMax protocols by reducing the number of tokens. In Table \ref{tab:layer}, we provide detailed layer-wise communication cost of the GELU and the SoftMax protocol. Compared to the unpruned baseline, CipherPrune can effectively reduce the overhead of the GELU and the SoftMax protocols layer by layer.
}

\section{Analysis on Layer-wise redundancy.}
\label{app:f}

\begin{figure}[h]
    \centering
    \includegraphics[width=0.9\linewidth]{figures/layertime0.pdf}
    \caption{The number of pruned tokens and pruning protocol runtime in different layers in the BERT Base model. The results are averaged across 128 QNLI samples.}
    \label{fig:layertime}
\end{figure}

{
In Figure \ref{fig:layertime}, we present the number of pruned tokens and the runtime of the pruning protocol for each layer in the BERT Base model. The number of pruned tokens per layer was averaged across 128 QNLI samples, while the pruning protocol runtime was measured over 10 independent runs. The mean token count for the QNLI samples is 48.5. During inference with BERT Base, input sequences with fewer tokens are padded to 128 tokens using padding tokens. Consistent with prior token pruning methods in plaintext~\citep{goyal2020power}, a significant number of padding tokens are removed at layer 0.  At layer 0, the number of pruned tokens is primarily influenced by the number of padding tokens rather than token-level redundancy.
%In Figure \ref{fig:layertime}, we demonstrate the number of pruned tokens and the pruning protocol runtime in each layer in the BERT Base model. We averaged the number of pruned tokens in each layer across 128 QNLI samples and then tested the pruning protocol runtime in 10 independent runs. The mean token number of the QNLI samples is 48.5. During inference with BERT Base, input sequences with small token number are padded to 128 tokens with padding tokens. Similar to prior token pruning methods in the plaintext~\citep{goyal2020power}, a large number of padding tokens can be removed at layer 0. We remark that token-level redundancy builds progressively throughout inference~\citep{goyal2020power, kim2022LTP}. The number of pruned tokens in layer 0 mostly depends on the number of padding tokens instead of token-level redundancy.
}

{
%As shown in Figure \ref{fig:layertime}, more tokens are removed in the intermediate layers, e.g., layer $4$ to layer $7$. This suggests there is more redundant information in these intermediate layers. 
In CipherPrune, tokens are removed progressively, and once removed, they are excluded from computations in subsequent layers. Consequently, token pruning in earlier layers affects computations in later layers, whereas token pruning in later layers does not impact earlier layers. As a result, even if layers 4 and 7 remove the same number of tokens, layer 7 processes fewer tokens overall, as illustrated in Figure \ref{fig:layertime}. Specifically, 8 tokens are removed in both layer $4$ and layer $7$, but the runtime of the pruning protocol in layer $4$ is $\sim2.4\times$ longer than that in  layer $7$.
}

\section{Related Works}
\label{app:g}

{
In response to the success of Transformers and the need to safeguard data privacy, various private Transformer Inferences~\citep{chen2022thex,zheng2023primer,hao2022iron-iron,li2022mpcformer, lu2023bumblebee, luo2024secformer, pang2023bolt}  are proposed. To efficiently run private Transformer inferences, multiple cryptographic primitives are used in a popular hybrid HE/MPC method IRON~\citep{hao2022iron-iron}, i.e., in a Transformer, HE and SS are used for linear layers, and SS and OT are adopted for nonlinear layers. IRON and BumbleBee~\citep{lu2023bumblebee} focus on optimizing linear general matrix multiplications; SecFormer~\cite{luo2024secformer} improves the non-linear operations like the exponential function with polynomial approximation; BOLT~\citep{pang2023bolt} introduces the baby-step giant-step (BSGS) algorithm to reduce the number of HE rotations, proposes a word elimination (W.E.) technique, and uses polynomial approximation for non-linear operations, ultimately achieving state-of-the-art (SOTA) performance.
}

{Other than above hybrid HE/MPC methods, there are also works exploring privacy-preserving Transformer inference using only HE~\citep{zimerman2023converting, zhang2024nonin}. The first HE-based private Transformer inference work~\citep{zimerman2023converting} replaces \mysoftmax function with a scaled-ReLU function. Since the scaled-ReLU function can be approximated with low-degree polynomials more easily, it can be computed more efficiently using only HE operations. A range-loss term is needed during training to reduce the polynomial degree while maintaining high accuracy. A training-free HE-based private Transformer inference was proposed~\citep{zhang2024nonin}, where non-linear operations are approximated by high-degree polynomials. The HE-based methods need frequent bootstrapping, especially when using high-degree polynomials, thus often incurring higher overhead than the hybrid HE/MPC methods in practice.
}


\end{document}
\endinput
%%
%% End of file `sample-sigconf-authordraft.tex'.
