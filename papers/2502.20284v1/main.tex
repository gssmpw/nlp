%%
%% This is file `sample-sigconf-authordraft.tex',
%% generated with the docstrip utility.
%%
%% The original source files were:
%%
%% samples.dtx  (with options: `all,proceedings,bibtex,authordraft')
%% 
%% IMPORTANT NOTICE:
%% 
%% For the copyright see the source file.
%% 
%% Any modified versions of this file must be renamed
%% with new filenames distinct from sample-sigconf-authordraft.tex.
%% 
%% For distribution of the original source see the terms
%% for copying and modification in the file samples.dtx.
%% 
%% This generated file may be distributed as long as the
%% original source files, as listed above, are part of the
%% same distribution. (The sources need not necessarily be
%% in the same archive or directory.)
%%
%%
%% Commands for TeXCount
%TC:macro \cite [option:text,text]
%TC:macro \citep [option:text,text]
%TC:macro \citet [option:text,text]
%TC:envir table 0 1
%TC:envir table* 0 1
%TC:envir tabular [ignore] word
%TC:envir displaymath 0 word
%TC:envir math 0 word
%TC:envir comment 0 0
%%
%%
%% The first command in your LaTeX source must be the \documentclass
%% command.
%%
%% For submission and review of your manuscript please change the
%% command to \documentclass[manuscript, screen, review]{acmart}.
%%
%% When submitting camera ready or to TAPS, please change the command
%% to \documentclass[sigconf]{acmart} or whichever template is required
%% for your publication.
%%
%%
% \documentclass[manuscript,review,anonymous]{acmart} % for anonymous review
\documentclass[manuscript,nonacm]{acmart} % for anonymous review
% \documentclass[sigconf]{acmart} % for camera-ready

%%%%%%%%%%%%%%%%%%%%%%%%%%%%%%%%%%%%%%%%%%%%%%%%%%%%%%%%%%%%%%%%%%%%%%%%%%%%%%

%% Beautiful mathematics
\usepackage{amsmath, amssymb, amsfonts} 
\usepackage{nicefrac}
\usepackage{mathtools}
\usepackage{bm, bbm}
\usepackage[scr=boondoxo,scrscaled=1.05]{mathalfa}

%% References in the correct format 
%\usepackage[square,numbers]{natbib}
%\def\bibfont{\footnotesize} % fix to have the same font size as without natbib

\usepackage[sort, compress, space]{cite}            


%% Enumerate nicely 
\usepackage{enumitem}

%% Different color comments and commenting large parts of the text
\usepackage{xcolor}
\usepackage{comment}
\usepackage{soul}

%% Hyper references
\usepackage{hyperref}
\usepackage{cleveref}
%\usepackage[numbers]{natbib}

\usepackage{tikz}
%\usepackage{thm-restate}
%% Appendix package
%\usepackage{appendix}

%% Random text to test spacing 
\usepackage{blindtext}

\usepackage{afterpage}

\usepackage{algorithm, algorithmic}    



\usepackage{dsfont}

\usepackage{tikz}
\usepackage{graphicx}
\usepackage{tikzscale}
\usepackage{pgfplots}
\pgfplotsset{compat=newest}
\usepackage{xfrac}

\usepackage{thm-restate}

%\usepackage{subcaption}

\usepackage{balance}

\usepackage{cite}
\usepackage{amsmath,amssymb,amsfonts}
\usepackage{balance}
\usepackage{algorithmic}
\usepackage{graphicx}
\usepackage{textcomp}
\usepackage{xcolor}
\usepackage{amsmath}
\usepackage{amssymb}
\usepackage[mathscr]{euscript}
\usepackage{comment}
\usepackage{xcolor}
\usepackage{enumitem} 
\usepackage{amsthm}


%%
%% \BibTeX command to typeset BibTeX logo in the docs
\AtBeginDocument{%
  \providecommand\BibTeX{{%
    Bib\TeX}}}

%% Rights management information.  This information is sent to you
%% when you complete the rights form.  These commands have SAMPLE
%% values in them; it is your responsibility as an author to replace
%% the commands and values with those provided to you when you
%% complete the rights form.
% \setcopyright{acmlicensed}
% \copyrightyear{2018}
% \acmYear{2018}
% \acmDOI{XXXXXXX.XXXXXXX}
\setcopyright{none}
\acmConference[Anonymous Submission]{}{Year}{Location}
\acmBooktitle{}
\acmPrice{}
\acmDOI{}
\acmISBN{}
\renewcommand\footnotetextcopyrightpermission[1]{} % Remove copyright footnote
\settopmatter{printacmref=false} % Remove ACM Reference Format

%% These commands are for a PROCEEDINGS abstract or paper.
% \acmConference[Conference acronym 'XX]{Make sure to enter the correct
%   conference title from your rights confirmation emai}{June 03--05,
%   2018}{Woodstock, NY}
%%
%%  Uncomment \acmBooktitle if the title of the proceedings is different
%%  from ``Proceedings of ...''!
%%
%%\acmBooktitle{Woodstock '18: ACM Symposium on Neural Gaze Detection,
%%  June 03--05, 2018, Woodstock, NY}
% \acmISBN{978-1-4503-XXXX-X/18/06}


%%
%% Submission ID.
%% Use this when submitting an article to a sponsored event. You'll
%% receive a unique submission ID from the organizers
%% of the event, and this ID should be used as the parameter to this command.
%%\acmSubmissionID{123-A56-BU3}

%%
%% For managing citations, it is recommended to use bibliography
%% files in BibTeX format.
%%
%% You can then either use BibTeX with the ACM-Reference-Format style,
%% or BibLaTeX with the acmnumeric or acmauthoryear sytles, that include
%% support for advanced citation of software artefact from the
%% biblatex-software package, also separately available on CTAN.
%%
%% Look at the sample-*-biblatex.tex files for templates showcasing
%% the biblatex styles.
%%

%%
%% The majority of ACM publications use numbered citations and
%% references.  The command \citestyle{authoryear} switches to the
%% "author year" style.
%%
%% If you are preparing content for an event
%% sponsored by ACM SIGGRAPH, you must use the "author year" style of
%% citations and references.
%% Uncommenting
%% the next command will enable that style.
%%\citestyle{acmauthoryear}


%%
%% end of the preamble, start of the body of the document source.
\begin{document}

%%
%% The "title" command has an optional parameter,
%% allowing the author to define a "short title" to be used in page headers.
% \title{Evaluating Human Trust in LLM-Based Planners: The Role of Correctness, Explanation, and Refinement in Structured Planning Tasks}
\title{Evaluating Human Trust in LLM-Based Planners: A Preliminary Study}
% Preliminary Evaluation of Human Trust in LLM Planners

%%
%% The "author" command and its associated commands are used to define
%% the authors and their affiliations.
%% Of note is the shared affiliation of the first two authors, and the
%% "authornote" and "authornotemark" commands
%% used to denote shared contribution to the research.
\author{Shenghui Chen}
\affiliation{%
  \institution{University of Texas at Austin}
  \city{Austin}
  \state{Texas}
  \country{USA}
}
\email{shenghui.chen@utexas.edu}

\author{Yunhao Yang}
\affiliation{%
  \institution{University of Texas at Austin}
  \city{Austin}
  \state{Texas}
  \country{USA}
}
\email{yunhaoyang234@utexas.edu}

\author{Kayla Boggess}
\affiliation{
  \institution{University of Virginia}
  \city{Charlottesville}
  \state{Virginia}
  \country{USA}
}
\email{kjb5we@virginia.edu}

\author{Seongkook Heo}
\affiliation{
  \institution{University of Virginia}
  \city{Charlottesville}
  \state{Virginia}
  \country{USA}
}
\email{seongkook@virginia.edu}

\author{Lu Feng}
\affiliation{
  \institution{University of Virginia}
  \city{Charlottesville}
  \state{Virginia}
  \country{USA}
}
\email{lu.feng@virginia.edu}

\author{Ufuk Topcu}
\affiliation{
  \institution{University of Texas at Austin}
  \city{Austin}
  \state{Texas}
  \country{USA}
}
\email{utopcu@utexas.edu}

%%
%% By default, the full list of authors will be used in the page
%% headers. Often, this list is too long, and will overlap
%% other information printed in the page headers. This command allows
%% the author to define a more concise list
%% of authors' names for this purpose.
% \renewcommand{\shortauthors}{Chen et al.}

%%
%% The abstract is a short summary of the work to be presented in the
%% article.
\begin{abstract}
    \begin{abstract}

% Recent works to jointly reconstruct 3D human and object from a single RGB image, are mostly model-based, that fail to capture the fine details of the clothed human body and object surface. In this paper, we introduce ReCHOR, a novel, model-free, first-method to produce realistic clothed human-object reconstructions from a monocular view. This is extremely challenging due to human-object occlusions, diverse interactions and depth ambiguity, as it needs to infer both 3D spatial awareness and high resolution details. Our core idea is based on estimating neural implicit representations for human and object respectively by an attention-based neural implicit model that attends to pixel-aligned features from both the global human-object image for spatial awareness and  the local separate view of human and object images for high quality details. Additionally, the network is conditioned on semantic features from an initial estimated human-object pose prior and a generative diffusion model that inpaints occluded regions, thus enabling the retrieval of details from them.
% We also propose a synthetic dataset with rendered scenes of diverse, inter-occluded 3D human and object scans, to train our network. We evaluate our method on the synthetic and real world BEHAVE dataset. Our experiments show that our method outperforms the SOTA in achieving realistic clothed human-object reconstructions.
Recent approaches to jointly reconstruct 3D humans and objects from a single RGB image represent 3D shapes with template-based or coarse models, which fail to capture details of loose clothing on human bodies. In this paper, we introduce a novel implicit approach for jointly reconstructing realistic 3D clothed humans and objects from a monocular view. For the first time, we model both the human and the object with an implicit representation, allowing to capture more realistic details such as clothing. This task is extremely challenging due to human-object occlusions and the lack of 3D information in 2D images, often leading to poor detail reconstruction and depth ambiguity. To address these problems, we propose a novel attention-based neural implicit model that leverages image pixel alignment from both the input human-object image for a global understanding of the human-object scene and from local separate views of the human and object images to improve realism with, for example, clothing details. Additionally, the network is conditioned on semantic features derived from an estimated human-object pose prior, which provides 3D spatial information about the shared space of humans and objects. To handle human occlusion caused by objects, we use a generative diffusion model that inpaints the occluded regions, recovering otherwise lost details. For training and evaluation, we introduce a synthetic dataset featuring rendered scenes of inter-occluded 3D human scans and diverse objects. Extensive evaluation on both synthetic and real-world datasets demonstrates the superior quality of the proposed human-object reconstructions over competitive methods.
\end{abstract}
\end{abstract}

% \begin{CCSXML}
% <ccs2012>
%    <concept>
%        <concept_id>10003120.10003121.10011748</concept_id>
%        <concept_desc>Human-centered computing~Empirical studies in HCI</concept_desc>
%        <concept_significance>500</concept_significance>
%        </concept>
%  </ccs2012>
% \end{CCSXML}

% \ccsdesc[500]{Human-centered computing~Empirical studies in HCI}

%%
%% Keywords. The author(s) should pick words that accurately describe
%% the work being presented. Separate the keywords with commas.
\keywords{Trust, Large Language Models (LLMs), Explainable AI, Planning}


%% A "teaser" image appears between the author and affiliation
%% information and the body of the document, and typically spans the
%% page.

% \begin{teaserfigure}
%   \includegraphics[width=\textwidth]{sampleteaser.pdf}
%   \caption{Seattle Mariners at Spring Training, 2010.}
%   \Description{Enjoying the baseball game from the third-base
%   seats. Ichiro Suzuki preparing to bat.}
%   \label{fig:teaser}
% \end{teaserfigure}

% \received{20 February 2007}
% \received[revised]{12 March 2009}
% \received[accepted]{5 June 2009}

%%
%% This command processes the author and affiliation and title
%% information and builds the first part of the formatted document.
\maketitle

\section{Introduction} \label{sec:intro} %1.5p (including title and absract)
\section{Introduction}
\label{sec:intro}
% Image editing methods in diffusion models depend on user-defined control directions - users can unlock their creativity using these methods by specifying the desired manipulation through prompts~\cite{gandikota2023concept}, reference images~\cite{ruiz2022dreambooth, kumari2022customdiffusion, gal2022image, chen2024trainingfreeregionalpromptingdiffusion}, or attribute vectors~\cite{parmar2023zero,hertz2022prompt}. In this work, we ask a fundamentally different question: \emph{Can we automatically discover the underlying visual structure of a concept within diffusion model's knowledge?} %Rather than requiring user-specified controls, we aim to decompose the model's internal knowledge into meaningful directions.

% This question touches on a fundamental limitation in how we interact with diffusion models. Current control methods ~\cite{zhang2023addingconditionalcontroltexttoimage, gandikota2023concept, ye2023ipadaptertextcompatibleimage,ye2023ipadaptertextcompatibleimage, hertz2024stylealignedimagegeneration, li2023photomaker, shi2024instantbooth, chen2024trainingfreeregionalpromptingdiffusion} require users to specify their desired manipulations in advance, limiting interactive creativity. This contrasts with natural human artistic workflows, where creators dynamically explore creative ideas while jointly refining them toward meaningful artistic outcomes~\cite{hoffmann2016modeling}. This synergy between specification and exploration is not new to generative models. Early GAN architectures naturally developed disentangled latent spaces that enabled continuous\cite{harkonen2020ganspace,radford2015unsupervised, wu2021stylespace, shen2020interfacegan}, compositional control over generated images. Users could explore these spaces to discover interesting variations that would be difficult to describe in words~\cite{wu2021stylespace}, then combine them to achieve their creative goals~\cite{grabe2022towards}. 


% While diffusion models have largely superseded GANs in conditional image synthesis~\cite{dhariwal2021diffusion},  their underlying structure remains less understood. Diffusion models achieve remarkable diversity through high-dimensional latents, unlike GANs' compact latent spaces.  With a single prompt, diffusion models can generate radically different variations through different random initializations of input noise. We ask - Is it possible to discover interpretable structure within this vast space of variations?

Text-to-image diffusion models are capable of generating remarkable visual variations from a single prompt through different random initializations. However, this vast creative potential remains largely opaque to users---while we can generate diverse images, we lack understanding of the underlying structure of these variations. This presents a fundamental challenge: how can we discover and expose the latent visual capabilities encoded within these models?

\let\thefootnote\relax \footnote{$^{*}$Correspondence to \texttt{gandikota.ro@northeastern.edu}}

The challenge touches on a key limitation in how we interact with diffusion models today. Current control methods require users to explicitly specify their desired edits in advance through prompts~\cite{gandikota2023concept}, reference images~\cite{zhang2023addingconditionalcontroltexttoimage, chen2024trainingfreeregionalpromptingdiffusion, ruiz2022dreambooth,kumari2022customdiffusion, Ryu_lora, hu2021lora}, or attribute vectors~\cite{ye2023ipadaptertextcompatibleimage, hertz2024stylealignedimagegeneration, li2023photomaker, shi2024instantbooth,parmar2023zero,hertz2022prompt}. That contrasts sharply with natural human creative workflows, where artists dynamically explore creative ideas and jointly refine them toward meaningful artistic outcomes~\cite{hoffmann2016modeling}. The need for pre-specified controls creates a barrier between users and the full creative potential of these models.

Interestingly, earlier generative models like GANs~\cite{gans,karras2019style,brock2018large} naturally developed more interpretable internal structures. Their compact latent spaces often exhibited emergent disentanglement~\cite{harkonen2020ganspace,radford2015unsupervised, wu2021stylespace, shen2020interfacegan}, enabling continuous and compositional control over generated images. Users could explore these spaces to discover interesting variations that would be difficult to describe in words~\cite{wu2021stylespace}, then combine them to achieve their creative goals~\cite{grabe2022towards}.

Diffusion models have largely superseded GANs in conditional image synthesis~\cite{dhariwal2021diffusion}, achieving greater diversity through much higher-dimensional latents. And yet an understanding of the underlying structure of these larger latent spaces has remained elusive. In this work, we ask a fundamental question: \emph{Can we automatically discover the visual structure within a diffusion model's knowledge of a concept?} Rather than requiring user-specified controls, we aim to decompose the model's internal representations into expressive directions that users can explore and combine.

To address these needs, we present \textbf{SliderSpace}, a framework that brings systematic explorability to diffusion models. Given just a text prompt, SliderSpace discovers a canonical set of meaningful, diverse, and controllable directions within the model's knowledge of that concept. Each direction is implemented as a low-rank adapter~\cite{hu2021lora} that can be scaled and composed with others, allowing users to explore and smoothly combine different aspects of variation, as shown in Figure~\ref{fig:intro}.

We ground SliderSpace discovery in three key requirements for meaningful decomposition of a diffusion model's visual manifold: 
\begin{enumerate}
    \item \textbf{Unsupervised Discovery:} The decomposition process should emerge from the intrinsic structure of the model's learned representation, rather than being guided by predefined attributes. This ensures we capture the true topology of the model's knowledge space rather than projecting our assumptions onto it.
    
    \item \textbf{Semantic Orthogonality:} Each discovered control must represent a distinct semantic direction. This is enforced in a semantic feature space, like CLIP, where every slider has an orthogonal effect in embeddings. This prevents discovering multiple controls that create similar semantic effects, making the system more efficient and easier.
    
    \item \textbf{Distribution Consistency:} Directions must induce consistent transformations across both random seeds and prompt variations. 
\end{enumerate}

These requirements naturally lead to our proposed framework, which we formalize in Section~\ref{sec:method}. As we show in our experiments, SliderSpace is architecture-agnostic, working with both conventional U-Net based models like Stable Diffusion~\cite{rombach2022high, rombach2022sd20, podell2023sdxl, turbo, dmd} and recent transformer-based architectures like Flux~\cite{flux}.

We demonstrate the expressiveness of SliderSpace through three applications: First, we show how SliderSpace can decompose high-level concepts into diverse and expressive components, revealing the natural axes of variation in the model's understanding. Second, we explore artistic style variation, where SliderSpace discovers directions that match or exceed the diversity of manually curated artist lists while being judged more useful by human evaluators. Finally, we show how SliderSpace can help reverse the mode collapse commonly observed in distilled diffusion models, restoring diversity while maintaining generation speed.

Beyond providing practical creative control, SliderSpace opens new avenues for understanding and utilizing the latent capabilities of diffusion models. By mapping these models' visual potential into intuitive, composable directions, we take a step toward making their creative possibilities more accessible and interpretable to users.

% Image editing methods in diffusion models unlock the creativity of users. In this work we ask an alternate question: \emph{Can we organize and expose what of the diffusion model is already capable of?}.
% Existing methods for controlling image generation typically require users to manually specify edit directions for desired changes. This process is time-consuming, requires technical expertise, and limits the spontaneity of the creative process. For instance, if a user wants to adjust the smile of a generated person, they must explicitly request this edit, often through imprecise prompt engineering or model fine-tuning. This approach of predefined controls or manual specifications restricts users from fully exploring the latent capabilities of the model. There may be interesting stylistic variations or attributes that the model can generate, but users have no easy way to discover or utilize these.

% Natural visual disentanglement was an emergent property in the latent space of Generative Adversarial Models (GANs) \cite{harkonen2020ganspace,radford2015unsupervised, wu2021stylespace, shen2020interfacegan}. In particular, it has been observed that StyleGAN~\cite{karras2019style} stylespace neurons offer detailed control over many meaningful aspects of images that would be difficult to describe in words~\cite{wu2021stylespace}. However, diffusion models do not share such a compact latent space~\cite{park2023unsupervised}; and efforts to uncover such a space in the semantic embeddings of the text conditioning have met with limited success \nik{Nick - is there a specific citation you were thinking about?}.

% In this work we introduce \textbf{SliderSpace}, which takes a step towards uncovering an analogous low dimensional representation of diffusion models' visual breadth; in essence treating the diffusion model as many generators sharing parameters, where a particular generator is defined by a specific prompt. For a given prompt we sample many random seeds (and optionally prompt expansions using an LLM), generate the corresponding images, and apply an off the shelf feature extractor (in this work CLIP, but our method can be applied to any differentiable feature extractor). We use PCA to analyze these features, and for each of the leading $k$ principal components we train a LoRA \cite{} which causes the diffusion model to produces images which increase the feature magnitude along that component when passed back through the same feature extractor. This leads to a 'Slider' for each principal component, because each LoRA can be scaled and applied to the original diffusion model, continuously varying those visual features in the generated results (as measured, in our case, by CLIP).

% There are many other works that enhance the controllability of diffusion models. One common approach is enabling users to add spatial constraints to a generation either manually, or via a reference image \cite{zhang2023addingconditionalcontroltexttoimage, chen2024trainingfreeregionalpromptingdiffusion}, a second is leveraging more abstract embeddings (e.g. identity, style) extracted from a reference image \cite{ye2023ipadaptertextcompatibleimage, hertz2024stylealignedimagegeneration, li2023photomaker, shi2024instantbooth}, a third is finetuning a foundation model to better generate a concept important to the user \cite{ruiz2022dreambooth, kumari2022customdiffusion, Ryu_lora, hu2021lora}, and a fourth (most relevant to this work) is finding low-rank adaptors of the model based on a prompt or small training set which can be scaled to provide continous control over one aspect of generated image (e.g. night vs day, basic vs luxury, etc.) \cite{gandikota2023concept}. SliderSpace is complementary to all of these methods and offers something distinct. All of the other methods we are aware require the user (and / or model designer) to know in advance what type of control they want. In contrast SliderSpace assists users in discovering and controlling hidden capabilities present in the diffusion model's distribution of possible generations.

%We propose that truly intuitive creative control in a text-to-image model should meet three key criteria: \emph{discoverability}, \emph{intuitiveness}, and \emph{specificity}. The model should reveal controllable attributes that may not be immediately obvious, offer controls that are easy to understand and manipulate, and ensure each control affects a distinct attribute of the generated image.

% We demonstrate the utility and power of SliderSpace using three applications built on top of SDXL-DMD \cite{dmd}, because its fast generation speed lends itself well to the continuous control offered by SliderSpace.

% First, we study concept decomposition (Section \ref{sec:concept_exp}), where we learn sliders for a specific concept (e.g. 'monster', 'waterfall', 'car'). Through quantitative metrics of diversity and text alignment we demonstrate that the learned sliders dramatically boost the diversity of generations when randomly applied without harming text alignment; we also ask humans to qualitatively judge these results in a user study where they find the SliderSpace results to be more 'Diverse', 'Useful', and 'Creative' than our baselines.

% Second, we attempt to compare the automatic discoveries of SliderSpace to a large scale manual study of artistic styles (Section \ref{sec:art_exp}), open-sourced by ParrotZone \cite{parrotzone}. In this study SDXL was prompted with over 4300 artist names,  and based on visual inspection the cases of successful stylistic mimicry recorded. Quantitatively SliderSpace more closely matches the distribution of artistic variation discovered by ParrotZone than other baselines, and in our user studies was judged to be significantly more 'Diverse' and 'Useful' than the baselines. To our surprise humans even judged SliderSpace results to be slightly more 'Diverse' than the results generated by the manually discovered artist names of \cite{parrotzone}.

% Third, we attempt to use SliderSpace to reverse the mode collapse commonly observed in distilled few-step diffusion models relative to the original teacher model (Section \ref{sec:diverse_exp}). We quantitatively demonstrate that applying SliderSpace to SDXL-DMD leads to more closely matching the distribution of images by the original teacher, SDXL.

%Through extensive experiments on various state-of-the-art text-to-image models, we demonstrate that SliderSpace significantly enhances user control and creative expression in AI-assisted image generation tasks. Our method enables a range of applications, including concept decomposition and control, diversity improvement in generated images, customization dissection and edits, and the exploration of artistic styles inherent in the model.

% SliderSpace goes beyond providing a practical tool for enhanced creative control. By mapping the visual potential of diffusion models it can open new avenues for generative creativity and deepens our understanding of each model's hidden potential.

% \section{Related Work} \label{sec:related} %1p
% \section{Related Work}

\paragraph{LLMs for Agent tasks.}

Our research is related to deploying large language models (LLMs) as agents for decision-making tasks in interactive environments~\citep{liu2023agentbench,zhou2023webarena,shridhar2020alfred,toyama2021androidenv}. Earlier works, such as~\citep{yao2023webshopscalablerealworldweb}, fine-tuned models like BERT~\citep{devlin2019bertpretrainingdeepbidirectional} for decision-making in simplified environments, such as online shopping or mobile phone manipulation. With the advent of large language models~\citep{brown2020languagemodelsfewshotlearners,openai2024gpt4technicalreport}, it became feasible to perform decision-making tasks through zero-shot or few-shot in-context learning. To better assess the capabilities of LLMs as agents, several models have been developed~\citep{deng2024mind2web,xiong2024watch,hong2023cogagent,yan2023gpt}. Most approaches~\citep{zheng2024seeact,deng2024mind2web} provide the agent with observation and action history, and the language model predicts the next action via in-context learning. Additionally, some methods~\citep{zhang2023building,li2023camel,song2024trial} attempt to distill trajectories from state-of-the-art language models to train more effective policy models. In contrast, our paper introduces a novel framework that automatically learns a reward model from LLM agent navigation, using it to guide the agents in making more effective plans.

\textbf{LLM Planning.} Our paper is also related to planning with large language models. Early researchers~\citep{brown2020languagemodelsfewshotlearners} often prompted large language models to directly perform agent tasks. Later, \citet{yao2022react} proposed ReAct, which combined LLMs for action prediction with chain-of-thought prompting~\citep{wei2022chain}. Several other works~\citep{yao2023treethoughtsdeliberateproblem,hao2023reasoning,zhao2023large,qiao2024agentplanningworldknowledge} have focused on enhancing multi-step reasoning capabilities by integrating LLMs with tree search methods. Our model differs from these previous studies in several significant ways. First, rather than solely focusing on text generation tasks, our pipeline addresses multi-step action planning tasks in interactive environments, where we must consider not only historical input but also multimodal feedback from the environment. Additionally, our pipeline involves automatic learning of the reward model from the environment without relying on human-annotated data, whereas previous works rely on prompting-based frameworks that require large commercial LLMs like GPT-4~\citep{openai2024gpt4technicalreport} to learn action prediction. Furthermore, \Model supports a variety of planning algorithms beyond tree search.

\textbf{Learning from AI Feedback.} In contrast to prior work on LLM planning, our approach also draws on recent advances in learning from AI feedback~\citep{bai2022constitutional,lee2023rlaif,yuan2024self,sharma2024critical,pan2024autonomous,koh2024tree}. These studies initially prompt state-of-the-art large language models to generate text responses that adhere to predefined principles and then potentially fine-tune the LLMs with reinforcement learning. Like previous studies, we also prompt large language models to generate synthetic data. However, unlike them, we focus not on fine-tuning a better generative model but on developing a classification model that evaluates how well action trajectories fulfill the intended instructions. This approach is simpler, requires no reliance on state-of-the-art LLMs, and is more efficient. We also demonstrate that our learned reward model can integrate with various LLMs and planning algorithms, consistently improving their performance.

\textbf{Inference-Time Scaling.} ~\citet{snell2024scaling} validates the efficacy of inference-time scaling for language models. Based on inference-time scaling, various methods have been proposed, such as random sampling~\citep{wang2022self} and tree-search methods~\citep{hao2023reasoning, zhang2024accessing, guan2025rstar}. Concurrently, several works have also leveraged inference-time scaling to improve the performance of agentic tasks. ~\citet{koh2024tree} adopts a training-free approach, employing MCTS to enhance policy model performance during inference and prompting the LLM to return the reward. ~\citet{gu2024your} introduces a novel speculative reasoning approach to bypass irreversible actions by leveraging LLMs or VLMs. It also employs tree search to improve performance and prompts an LLM to output rewards. ~\citet{yu2024exact} proposes Reflective-MCTS to perform tree search and fine-tune the GPT model, leading to improvements in ~\citet{koh2024visualwebarena}. ~\citet{putta2024agent} also utilizes MCTS to enhance performance on web-based tasks such as ~\citet{yao2023webshopscalablerealworldweb} and real-world booking environments. ~\cite{lin2025qlass} utilizes the stepwise reward to give effective intermediate guidance across different agentic tasks. Our work differs from previous efforts in two key aspects: (1) Broader Application Domain. Unlike prior studies that primarily focus on tasks from a single domain, our method demonstrates strong generalizability across web agents, mathematical reasoning, and scientific discovery domains, further proving its effectiveness. (2) Flexible and Effective Reward Modeling. Instead of simply prompting an LLM as a reward model, we finetune a small scale VLM~\citep{lin2023vila} to evaluate input trajectories. %Our reward scores range continuously between 0 and 1, in contrast to existing methods that rely on discrete scoring (e.g., 0 and 1, or 0, 0.5, and 1) through direct LLM prompting.

% Concurrently, several works have also leveraged inference-time scaling to improve the performance of agentic tasks. ~\citet{pan2024autonomous} demonstrates that LLMs and VLMs, such as the GPT series, can function as evaluators or reward models to provide guidance for fine-tuning or reflection, thereby enhancing digital agents. This lays the groundwork for subsequent studies that directly prompt LLMs as reward models. ~\citet{koh2024tree} adopts a training-free approach, employing MCTS to enhance policy model performance during inference. However, it is limited to web environments~\citep{koh2024visualwebarena}. Moreover, its value function relies on prompting an LLM, which is less effective than our proposed method. We validate our approach through ablation studies, demonstrating that our fine-tuned reward model is more effective. ~\citet{gu2024your} introduces a novel speculative reasoning approach to bypass irreversible actions, such as purchasing a product, by leveraging LLMs or VLMs. It also employs tree search to improve performance, but it remains restricted to the web domain~\citep{koh2024visualwebarena, deng2024mind2web}. Additionally, it lacks reward modeling and instead prompts an LLM to output rewards. ~\citet{yu2024exact} proposes Reflective-MCTS to perform tree search and fine-tune the GPT model, leading to improvements in ~\citep{koh2024visualwebarena}. However, this work focuses solely on a single web agent task, and its reward modeling is derived from multi-agent debate, differing from our more effective and efficient reward modeling approach. ~\citet{putta2024agent} also utilizes MCTS to enhance performance, but it is limited to web-based tasks such as ~\citep{yao2023webshopscalablerealworldweb} and real-world booking environments.

\section{Methods} \label{sec:method} %1.5p
\section{Analysis}

\textbf{3D Structure.}
The vanilla RoPE defines a matrix $\bm{A}_{t_1,t_2}$ that represents the relative positional encoding between two positions $t_1$ and $t_2$ in a 1D sequence:
\begin{equation}\label{eq:vanilla_rope}
% \vspace{-6pt}
\begin{aligned}
% \small
\bm{A}_{t_1,t_2}&=\left(\bm{q}_{t_1}\bm{R}_{t_1}\right){\left(\bm{k}_{t_2}\bm{R}_{t_2}\right)}^\top
% = \bm{q}_{t_1}\bm{R}_{t_1}\bm{R}_{t_2}^\top\bm{k}_{t_2}^\top
= \bm{q}_{t_1}\bm{R}_{\Delta t}\bm{k}_{t_2}^\top,
\end{aligned}
% \vspace{-6pt}
\end{equation}
where $\Delta t=t_1-t_2$, the symbols $\bm{q}_{t_1}$ and $\bm{k}_{t_2}$ are the query and key vectors at positions $t_1$ and $t_2$.
The \textit{relative rotation matrix} $\bm{R}_{\Delta t}$ is defined as $\bm{R}_{\Delta t} = \exp(\Delta ti\theta_{n})$, while $i$ is the imaginary unit, $\theta_{n} = \beta^{-2n/d}$ is the frequency of rotation applied to a specific $n$-th pair of $d$ dimensions ($n=0,\ldots,d/2-1$), and $\beta$ is the frequency base parameter.
The vanilla RoPE uses $d=128$, thus $n=0,\ldots,63$.
Consequently, the $\bm{A}_{t_1,t_2}$ in Eq. (\ref{eq:vanilla_rope}) can be extended as:
% \begin{equation}\label{equ:rope}
% \vspace{-6pt}
% \resizebox{0.5\textwidth}{!}{
% \scriptsize
% \begin{gathered}
% \begin{pmatrix}
% q^{(0)}\\q^{(1)}\\\vdots\\q^{(126)}\\q^{(127)}
% \end{pmatrix}^\top
% \begin{pmatrix}\cos{\theta_0\Delta t}& -\sin{\theta_0\Delta t}&\cdots&0&0\\ \sin{\theta_0\Delta t}&\cos{\theta_0\Delta t}&\cdots&0&0 \\ \vdots&\vdots&\ddots&\vdots&\vdots\\ 0&0&\cdots&\cos{\theta_{63}\Delta t}&  \sin{\theta_{63}\Delta t}\\ 0&0&\cdots&\sin{\theta_{63}\Delta t}&\cos{\theta_{63}\Delta t}
% \end{pmatrix}
% \begin{pmatrix}
% k^{(0)}\\k^{(1)}\\\vdots\\k^{(126)}\\k^{(127)}
% \end{pmatrix}.
% \end{gathered}
% }
% \end{equation}
\begin{equation}\label{equ:rope}
\vspace{-6pt}
\resizebox{0.5\textwidth}{!}{$
\scriptsize
\left(
\begin{array}{c}
q^{(0)}\\q^{(1)}\\\vdots\\q^{(126)}\\q^{(127)}
\end{array}
\right)^{\top}
\left(
\begin{array}{ccccc}
\cos{\theta_0\Delta t} & -\sin{\theta_0\Delta t} & \cdots & 0 & 0 \\ 
\sin{\theta_0\Delta t} & \cos{\theta_0\Delta t} & \cdots & 0 & 0 \\ 
\vdots & \vdots & \ddots & \vdots & \vdots \\  
0 & 0 & \cdots & \cos{\theta_{63}\Delta t} &  \sin{\theta_{63}\Delta t} \\  
0 & 0 & \cdots & \sin{\theta_{63}\Delta t} & \cos{\theta_{63}\Delta t} 
\end{array}
\right)
\left(
\begin{array}{c}
k^{(0)}\\k^{(1)}\\\vdots\\k^{(126)}\\k^{(127)}
\end{array}
\right)
$}
\end{equation}



While the vanilla RoPE operates on 1D sequences, it can also be applied to higher-dimensional input by flattening the input into a 1-D sequence.
However, the flattening process discards crucial neighborhood information, increases the sequence length, and hinders the capture of long-range dependencies.
Therefore, preserving the inherent 3D structure is essential when adapting RoPE for video data.
Some recent RoPE-variants (e.g., M-RoPE in Qwen2-VL \cite{wang2024qwen2}) incorporate the $3$D structure.
The corresponding relative matrix $\bm{A}_{(t_1,x_1,y_1)}$ is computed as:
\begin{equation}
% \small
\bm{A}_{(t_1,x_1,y_1),(t_2,x_2,y_2)}=\bm{q}_{(t_1,x_1,y_1)}\bm{R}_{\Delta t,\Delta x,\Delta y}\bm{k}_{(t_2,x_2,y_2)}^\top,
\end{equation}
where $\Delta t=t_1-t_2$, $\Delta x=x_1-x_2$, and $\Delta y=y_1-y_2$.
M-RoPE divides the $d=128$ feature dimensions into 3 groups: the first 32 for temporal positions ($t$), the middle 48 for horizontal positions ($x$), and the last 48 for vertical positions ($y$). As shown in Eq~(\ref{equ:mrope}), $\bm{A}_{(t_1,x_1,y_1),(t_2,x_2,y_2)}$ in M-RoPE is extended as:
\begin{equation}
\vspace{-6pt}
\resizebox{0.5\textwidth}{!}{$
\scriptsize
\begin{gathered}
\underbrace{\begingroup
\setlength\arraycolsep{1pt}
\begin{pmatrix}q^{(0)}\\q^{(1)}\\q^{(2)}\\q^{(3)}\\\vdots\\q^{(30)}\\q^{(31)}\end{pmatrix}^\top
\begin{pmatrix}
% \setstacktabbedgap{2pt}
\cos{\theta_0\Delta t}& -\sin{\theta_0\Delta t}&0&0&\cdots&0&0\\
\sin{\theta_0\Delta t}&\cos{\theta_0\Delta t}&0&0&\cdots&0&0 \\
0&0&\cos{\theta_1\Delta t}& -\sin{\theta_1\Delta t}&\cdots&0&0\\
0&0&\sin{\theta_1\Delta t}&\cos{\theta_1\Delta t}&\cdots&0&0 \\ 
\vdots&\vdots&\vdots&\vdots&\ddots&\vdots&\vdots\\
0&0&0&0&\cdots&\cos{\theta_{15}\Delta t}& -\sin{\theta_{15}\Delta t}\\
0&0&0&0&\cdots&\sin{\theta_{15}\Delta t}&\cos{\theta_{15}\Delta t}
\end{pmatrix}
\begin{pmatrix}k^{(0)}\\k^{(1)}\\k^{(2)}\\k^{(3)}\\\vdots\\k^{(30)}\\k^{(31)}\end{pmatrix}
\endgroup}_\text{\normalsize modeling temporal dependency with higher frequency} \\
+ \underbrace{\begingroup
\setlength\arraycolsep{1pt}
\begin{pmatrix}q^{(32)}\\q^{(33)}\\q^{(34)}\\q^{(35)}\\\vdots\\q^{(78)}\\q^{(79)}\end{pmatrix}^\top
\begin{pmatrix}
% \setstacktabbedgap{2pt}
\cos{\theta_{16}\Delta x}& -\sin{\theta_{16}\Delta x}&0&0&\cdots&0&0\\
\sin{\theta_{16}\Delta x}&\cos{\theta_{16}\Delta x}&0&0&\cdots&0&0 \\
0&0&\cos{\theta_{17}\Delta x}& -\sin{\theta_{17}\Delta x}&\cdots&0&0\\
0&0&\sin{\theta_{17}\Delta x}&\cos{\theta_{17}\Delta x}&\cdots&0&0 \\ 
\vdots&\vdots&\vdots&\vdots&\ddots&\vdots&\vdots\\
0&0&0&0&\cdots&\cos{\theta_{39}\Delta x}& -\sin{\theta_{39}\Delta x}\\
0&0&0&0&\cdots&\sin{\theta_{39}\Delta x}&\cos{\theta_{39}\Delta x}
\end{pmatrix}
\begin{pmatrix}k^{(32)}\\k^{(33)}\\k^{(34)}\\k^{(35)}\\\vdots\\k^{(78)}\\k^{(79)}\end{pmatrix}
\endgroup}_\text{\normalsize modeling horizontal dependency with intermediate frequency} \\
+ \underbrace{\begingroup
\setlength\arraycolsep{1pt}
\begin{pmatrix}q^{(80)}\\q^{(81)}\\q^{(82)}\\q^{(83)}\\\vdots\\q^{(126)}\\q^{(127)}\end{pmatrix}^\top
\begin{pmatrix}
% \setstacktabbedgap{2pt}
\cos{\theta_{40}\Delta y}& -\sin{\theta_{40}\Delta y}&0&0&\cdots&0&0\\
\sin{\theta_{40}\Delta y}&\cos{\theta_{40}\Delta y}&0&0&\cdots&0&0 \\
0&0&\cos{\theta_{41}\Delta y}& -\sin{\theta_{41}\Delta y}&\cdots&0&0\\
0&0&\sin{\theta_{41}\Delta y}&\cos{\theta_{41}\Delta y}&\cdots&0&0 \\ 
\vdots&\vdots&\vdots&\vdots&\ddots&\vdots&\vdots\\
0&0&0&0&\cdots&\cos{\theta_{63}\Delta y}& -\sin{\theta_{63}\Delta y}\\
0&0&0&0&\cdots&\sin{\theta_{63}\Delta y}&\cos{\theta_{63}\Delta y}
\end{pmatrix}
\begin{pmatrix}k^{(80)}\\k^{(81)}\\k^{(82)}\\k^{(83)}\\\vdots\\k^{(126)}\\k^{(127)}\end{pmatrix}
\endgroup}_\text{\normalsize modeling vertical dependency with lower frequency}
\end{gathered}
$}
\label{equ:mrope}
\end{equation}
% \begin{equation}
% \vspace{-6pt}
% \resizebox{0.5\textwidth}{!}{$
% \scriptsize
% \underbrace{
% \left(
% \begin{array}{c}
% q^{(0)}\\q^{(1)}\\\vdots\\q^{(31)}
% \end{array}
% \right)^{\top}
% \left(
% \begin{array}{cccccc}
% \cos{\theta_0\Delta t} & -\sin{\theta_0\Delta t} & \cdots & 0 \\ 
% \sin{\theta_0\Delta t} & \cos{\theta_0\Delta t} & \cdots & 0 \\ 
% \vdots & \vdots & \ddots & \vdots \\ 
% 0 & 0 & \cdots & \cos{\theta_{15}\Delta t} 
% \end{array}
% \right)
% \left(
% \begin{array}{c}
% k^{(0)}\\k^{(1)}\\\vdots\\k^{(31)}
% \end{array}
% \right)
% }_{\text{\normalsize modeling temporal dependency with higher frequency}}
% +
% \underbrace{
% \left(
% \begin{array}{c}
% q^{(32)}\\q^{(33)}\\\vdots\\q^{(79)}
% \end{array}
% \right)^{\top}
% \left(
% \begin{array}{cccccc}
% \cos{\theta_{16}\Delta x} & -\sin{\theta_{16}\Delta x} & \cdots & 0 \\ 
% \sin{\theta_{16}\Delta x} & \cos{\theta_{16}\Delta x} & \cdots & 0 \\ 
% \vdots & \vdots & \ddots & \vdots \\ 
% 0 & 0 & \cdots & \cos{\theta_{39}\Delta x} 
% \end{array}
% \right)
% \left(
% \begin{array}{c}
% k^{(32)}\\k^{(33)}\\\vdots\\k^{(79)}
% \end{array}
% \right)
% }_{\text{\normalsize modeling horizontal dependency with intermediate frequency}}
% +
% \underbrace{
% \left(
% \begin{array}{c}
% q^{(80)}\\q^{(81)}\\\vdots\\q^{(127)}
% \end{array}
% \right)^{\top}
% \left(
% \begin{array}{cccccc}
% \cos{\theta_{40}\Delta y} & -\sin{\theta_{40}\Delta y} & \cdots & 0 \\ 
% \sin{\theta_{40}\Delta y} & \cos{\theta_{40}\Delta y} & \cdots & 0 \\ 
% \vdots & \vdots & \ddots & \vdots \\ 
% 0 & 0 & \cdots & \cos{\theta_{63}\Delta y} 
% \end{array}
% \right)
% \left(
% \begin{array}{c}
% k^{(80)}\\k^{(81)}\\\vdots\\k^{(127)}
% \end{array}
% \right)
% }_{\text{\normalsize modeling vertical dependency with lower frequency}}
% $}
% \end{equation}


\noindent \textbf{Frequency Allocation.}
% Note that the frequency encoding in vanilla RoPE (Eq. \ref{equ:rope}) assigns higher frequencies (via larger $\theta_{n}$ values) to lower dimensions.
Incorporating 3D structure raises the question of how to allocate the temporal ($t$), horizontal ($x$), and vertical ($y$) components within the $d$ dimensions.
Note that different allocation strategies are not equivalent in the rotation frequency $\theta_{n} = \beta^{-2n/d}$.
As shown in Eq. (\ref{equ:mrope}), M-RoPE assigns higher frequencies (corresponding to lower dimensions) to the temporal dimension ($t$).

To highlight the importance of frequency allocation, we introduce a challenging retrieval task \textbf{V}isual \textbf{N}eedle-\textbf{I}n-\textbf{A}-\textbf{H}astack-\textbf{D}istractor (\textbf{V-NIAH-D}).
V-NIAH-D builds upon V-NIAH \cite{zhang2024longva}, a benchmark designed to evaluate visual long-context understanding.
However, the straightforward retrieval-based task has been shown to provide only a superficial form of long-context understanding~\cite{hsieh2024ruler,yuan2024lv}.
Therefore, We enhance V-NIAH by incorporating semantically similar distractors, obtained using Google Image Search~\cite{googleimagesearch} or Flux ~\cite{flux2023}, to mitigate the possibility of correct answers through random chance.
These distractors are designed to be unambiguous to the question in Fig. \ref{fig:v-ruler}.

\begin{figure}[t]
\centering
\includegraphics[width=.94\linewidth]{figures/files/attention_analysis.pdf}
\vspace{-6pt}
\caption{\footnotesize Attention-based frequential allocation analysis.
\textbf{Middle}: M-RoPE's temporal dimension ($t$) is limited to local information, resulting in a diagonal layout.
\textbf{Bottom}: \methodname effectively retrieves the needle using the temporal dimension.
The x and y coordinates represent the video frame number, e.g., 50 for 50 frames.
For more details see Appendix \ref{app:attention_analysis}.
% We use 8k-context input, with video tokens from the same frame aggregated via average pooling.
}
\vspace{-12pt}
\label{fig:attention_analysis}
\end{figure}



As shown in Fig. \ref{fig:v-ruler}, M-RoPE exhibits a clear performance drop from V-NIAH to V-NIAH-D. To investigate this decline, we follow previous works \citep{xiao2023efficient,liu2023scaling,barbero2024round} to visualize the attention scores in Fig. \ref{fig:attention_analysis}. We decompose the attention scores into their corresponding temporal ($t$), horizontal ($x$), and vertical ($y$) components for visualization.

Fig.~ \ref{fig:attention_analysis} reveals unusual attention patterns in M-RoPE, despite its ability to locate the needle image but fails to answer the multi-choice question.
According to the attention of M-RoPE, the needle is located primarily through vertical positional information, rather than temporal features.
Thus, the temporal dimension fails to capture long-range semantic dependencies, focusing instead on local relationships.
Conversely, the spatial dimensions exhibit a tendency to capture long-range rather than local semantic information.
Lastly, the horizontal and vertical dimensions display distinct characteristics, with the vertical dimension exhibiting phenomena reminiscent of attention sinks \cite{xiao2023efficient}.
These observations suggest that the performance decline primarily results from the sub-optimal frequency allocation designs of M-RoPE.

\noindent \textbf{Spatial Symmetry.} Given the text tokens $T$ and the visual tokens $T_v$, spatial symmetry \cite{kexuefm10352} claims that the distance between the end of the preceding textual input ($T_{\text{pre}}$) and the beginning of the visual input ($T_v^{\text{start}}$) is equal to the distance between the end of the visual input ($T_v^{\text{end}}$) and the beginning of the subsequent textual input ($T_{\text{sub}}$):
\begin{equation}
    T_{v}^{\text{start}} - T_{\text{pre}} =
    T_{\text{sub}} - T_{v}^{\text{end}}.
\end{equation}
The spatial symmetrical structure can potentially simplify the learning process and reduce bias toward input order.
However, existing 3D RoPE variants such as M-RoPE do not meet the spatial symmetry, we will elaborate related discussion in Fig. \ref{fig:spatial}.

\begin{figure*}[t]
\begin{minipage}{0.98\textwidth}
    \begin{subfigure}[b]{0.49\linewidth}
        \centering
\includegraphics[width=0.95\linewidth]{figures/files/video_rope-period_low-MRoPE.pdf}
        \caption{Temporal Frequency Allocation in M-RoPE}
        \label{fig:temporal_mrope}
    \end{subfigure}
    \hfill
    \begin{subfigure}[b]{0.49\linewidth}
        \centering
        \includegraphics[width=0.95\linewidth]{figures/files/video_rope-period_low-VideoRoPE.pdf}
        \caption{Temporal Frequency Allocation in \methodname (ours)}
        \label{fig:temporal_videorope}
    \end{subfigure}
    \vspace{-6pt}
    \caption{\footnotesize \textbf{(a)} M-RoPE \cite{wang2024qwen2} models temporal dependencies using the \textit{first} 16 rotary angles, which exhibit higher frequencies and more pronounced oscillations. \textbf{(b)} In contrast, \methodname models temporal dependencies using the \textit{last} 16 rotary angles, characterized by significantly wider, monotonic intervals. Our frequency allocation effectively mitigates the misleading influence of distractors in V-NIAH-D. For a more detailed analysis, please refer to Appendix \ref{app:supp_explain_modules}.
    % Take the first 3 rotary angles as an example, the position embedding for temporal modeling is free from oscillation~\cite{men2024base}.
    }
    \label{fig:period_mono}
    \vspace{-12pt}
\end{minipage}
\end{figure*}

\noindent \textbf{Temporal Index Scaling.}
The frame index in video and the token index in text are inherently different \cite{kexuefm10352,li2024temporal}.
Recognizing this difference, methods like TAD-RoPE, a 1D RoPE adaptation for Video LLMs, introduce distinct step offsets for image and text token indices: $\gamma$ for image tokens and $\gamma+1$ for text tokens.
Consequently, an ideal RoPE design for video data should permit scaling of the temporal index to meet the inherent difference between the frame index and the text index.

\section{\methodname}\label{subsec:step_size}

Based on some previous research and the above analysis, we claim that a good RoPE design for Video LLMs, especially for long videos, should satisfy four requirements.
% : 3D structure, Appreciate Frequency Allocation, Spatial Symmetry, and Temporal Index Scaling.
The first requirement has been solved by RoPE-Tie~\cite{kexuefm10040} and the subsequent M-RoPE~\cite{wang2024qwen2}.
To solve the last three requirements and mitigate the performance decline observed in V-NIAH-D, we propose our \methodname, comprising the following three key components.
% (1) Low-frequency Temporal Allocation; (2) Diagonal Layout; and (3) Adjustable Temporal Spacing.
% \textbf{\textit{Multi-Modal Compatibility}}, whether RoPE can simultaneously describe the spatiotemporal position in multi-modals and sequential position in text-only inputs~\cite{wang2024qwen2,kexuefm10040,kexuefm10352}, \textbf{\textit{Appropriate Dimension Distribution}}, whether the feature dimension can process the semantic relationship where it is responsibility~\cite{peng2023yarn,barbero2024round,liu2024kangaroo}, \textbf{\textit{Spatial Symmetry}}, whether the distance between the end of precedent textual input and start of visual input equals the distance between the end of visual input and the start of subsequent textual input~\cite{kexuefm10352}, and \textbf{\textit{Temporal Alignment}}, whether the alignment of sequential feature in different modality is considered~\cite{gao2024tc}.

\noindent \textbf{Low-frequency Temporal Allocation (LTA).} 
As shown in Eq. (\ref{equ:rope}), the vanilla RoPE~\cite{su2024roformer} uses all dimensions to model the 1D position information. And as indicated in Eq. (\ref{equ:mrope}), M-RoPE~\cite{wang2024qwen2} uses dimensions to model temporal, horizontal, and vertical dimensions sequentially.
However, previous frequency allocation strategies are suboptimal because different RoPE dimensions capture dependencies at varying ranges.
As shown in Fig.  \ref{fig:attention_analysis}, an interesting observation is that the local attention branch (as reported in \cite{han2024lm}) corresponds to lower dimensions, while the global branch (or attention sink, as in \cite{xiao2023efficient}) corresponds to higher dimensions.
To sum up, lower dimensions (higher frequency, shorter monotonic intervals, larger $\theta_n$) tend to capture relative distances and local semantics \cite{men2024base,barbero2024round}, while higher dimensions (lower frequency, wider monotonic intervals, smaller $\theta_n$) capture longer-range dependencies \cite{barbero2024round}.

Based on our analysis, \methodname uses higher dimensions for temporal features in longer contexts and lower dimensions for spatial features, which are limited by resolution and have a fixed range.
To avoid the gap between horizontal and vertical positions, we interleave the dimensions responsible for these spatial features.
The dimension distribution for \methodname is shown in Eq. (\ref{equ:videorope}):

% $\bm{A}_{(t_1,x_1,y_1),(t_2,x_2,y_2)}=\bm{q}_{(t_1,x_1,y_1)}\bm{k}_{(t_2,x_2,y_2)}^\top$

% To make full use of these properties of RoPE, \methodname uses higher dimensions to model temporal features in longer contexts and lower dimensions to model spatial features since spatial features tend to be limited by resolution and have a relatively fixed range. To avoid the gap between horizontal and vertical positions, we interleave the dimensions responsible for those two spatial features. Therefore, the dimension distribution for \methodname is shown in Equation~\ref{equ:videorope}.
% 48, 48, 32; 0, 47, 48, 95; 96, 127
\begin{equation}
\resizebox{0.5\textwidth}{!}{$
\scriptsize
\begin{gathered}
\underbrace{\begingroup
\setlength\arraycolsep{1pt}
\begin{pmatrix}q^{(96)}\\q^{(97)}\\q^{(98)}\\q^{(99)}\\\vdots\\q^{(126)}\\q^{(127)}\end{pmatrix}^\top
\begin{pmatrix}
% \setstacktabbedgap{2pt}
\cos{\theta_{48}\Delta t}& -\sin{\theta_{48}\Delta t}&0&0&\cdots&0&0\\
\sin{\theta_{48}\Delta t}&\cos{\theta_{48}\Delta t}&0&0&\cdots&0&0 \\
0&0&\cos{\theta_{49}\Delta t}& -\sin{\theta_{49}\Delta t}&\cdots&0&0\\
0&0&\sin{\theta_{49}\Delta t}&\cos{\theta_{49}\Delta t}&\cdots&0&0 \\ 
\vdots&\vdots&\vdots&\vdots&\ddots&\vdots&\vdots\\
0&0&0&0&\cdots&\cos{\theta_{63}\Delta t}& -\sin{\theta_{63}\Delta t}\\
0&0&0&0&\cdots&\sin{\theta_{63}\Delta t}&\cos{\theta_{63}\Delta t}
\end{pmatrix}
\begin{pmatrix}k^{(96)}\\k^{(97)}\\k^{(98)}\\k^{(99)}\\\vdots\\k^{(126)}\\k^{(127)}\end{pmatrix}
\endgroup}_\text{\normalsize modeling temporal dependency with lower frequency} \\
+ \underbrace{\begingroup
\setlength\arraycolsep{1pt}
\begin{pmatrix}q^{(0)}\\q^{(1)}\\q^{(4)}\\q^{(5)}\\\vdots\\q^{(92)}\\q^{(93)}\end{pmatrix}^\top
\begin{pmatrix}
% \setstacktabbedgap{2pt}
\cos{\theta_{0}\Delta x}& -\sin{\theta_{0}\Delta x}&0&0&\cdots&0&0\\
\sin{\theta_{0}\Delta x}&\cos{\theta_{0}\Delta x}&0&0&\cdots&0&0 \\
0&0&\cos{\theta_{2}\Delta x}& -\sin{\theta_{2}\Delta x}&\cdots&0&0\\
0&0&\sin{\theta_{2}\Delta x}&\cos{\theta_{2}\Delta x}&\cdots&0&0 \\ 
\vdots&\vdots&\vdots&\vdots&\ddots&\vdots&\vdots\\
0&0&0&0&\cdots&\cos{\theta_{46}\Delta x}& -\sin{\theta_{46}\Delta x}\\
0&0&0&0&\cdots&\sin{\theta_{46}\Delta x}&\cos{\theta_{46}\Delta x}
\end{pmatrix}
\begin{pmatrix}k^{(0)}\\k^{(1)}\\k^{(4)}\\k^{(5)}\\\vdots\\k^{(92)}\\k^{(93)}\end{pmatrix}
\endgroup}_\text{\normalsize modeling horizontal dependency with interleaved high frequency} \\
+ \underbrace{\begingroup
\setlength\arraycolsep{1pt}
\begin{pmatrix}q^{(2)}\\q^{(3)}\\q^{(6)}\\q^{(7)}\\\vdots\\q^{(94)}\\q^{(95)}\end{pmatrix}^\top
\begin{pmatrix}
% \setstacktabbedgap{2pt}
\cos{\theta_{1}\Delta y}& -\sin{\theta_{1}\Delta y}&0&0&\cdots&0&0\\
\sin{\theta_{1}\Delta y}&\cos{\theta_{1}\Delta y}&0&0&\cdots&0&0 \\
0&0&\cos{\theta_{3}\Delta y}& -\sin{\theta_{3}\Delta y}&\cdots&0&0\\
0&0&\sin{\theta_{3}\Delta y}&\cos{\theta_{3}\Delta y}&\cdots&0&0 \\ 
\vdots&\vdots&\vdots&\vdots&\ddots&\vdots&\vdots\\
0&0&0&0&\cdots&\cos{\theta_{47}\Delta y}& -\sin{\theta_{47}\Delta y}\\
0&0&0&0&\cdots&\sin{\theta_{47}\Delta y}&\cos{\theta_{47}\Delta y}
\end{pmatrix}
\begin{pmatrix}k^{(2)}\\k^{(3)}\\k^{(6)}\\k^{(7)}\\\vdots\\k^{(94)}\\k^{(95)}\end{pmatrix}
\endgroup}_\text{\normalsize modeling vertical dependency with interleaved high frequency} \\
% \Delta t=t_1-t_2,\quad \Delta x=x_1-x_2,\quad \Delta y=y_1-y_2 \\
% \theta_n=\beta^{-\dfrac{2n}{d}},\quad n=0,\cdots,d/2-1
\end{gathered}
$
}
% \raisebox{-5.5ex}{.}
\label{equ:videorope}
\end{equation}
The horizontal position $x$ and vertical position $y$ are interleaved to occupy the lower dimensions, followed by temporal $t$, which occupies the higher dimensions. We keep the same allocation number for $x$, $y$, and $t$ as M-RoPE for a fair comparison, with values of 48, 48, and 32, respectively.
The advantages of this distribution are evident in Fig.  \ref{fig:period_mono}. 
For a RoPE-based LLM with a 128-dimensional head (64 rotary angles $\theta_n$), we visualize the function of $\cos{\theta_n t}$ for 3 dimensions using parallel blue planes.

As shown in Fig. \ref{fig:period_mono} (\textbf{a}), M-RoPE's temporal position embeddings are significantly distorted by periodic oscillations \cite{men2024base}, leading to identical embeddings for distant positions.
For instance, considering the last three rotary angles, the temporal embeddings are severely affected by these oscillations due to their short monotonic intervals (and even shorter intervals in lower dimensions).
This periodicity creates ``hash collisions'' (red planes), where distant positions share near-identical embeddings, making the model susceptible to distractor influence.
Fortunately, our \methodname (Fig. \ref{fig:period_mono} (\textbf{b})) is free from oscillation and Hash collision in temporal modeling.
The visualized relationship between the periodicity, monotonicity, and temporal modeling.

\begin{figure}[t]
\centering
\includegraphics[width=0.98\linewidth]{figures/files/video_rope_figure_spatial_v2.pdf}
\vspace{-6pt}
\caption{\footnotesize The position embeddings of adjacent text tokens for Vanilla RoPE (\textbf{top} row), the corresponding visual tokens in adjacent frames for M-RoPE (\textbf{middle} row) and our \methodname (\textbf{bottom} row) with interleaved spatial and temporal last design.}
\vspace{-12pt}
\label{fig:spatail_index}
\end{figure}

\subsection{Spatial Transcriptomics}\label{sec:td_spatial}
%\subsubsection{Applications}
% \AB{General applications of TD in spatial transcriptomics. Mostly focused on integrating and analyzing single-cell spatial transcriptomic data~\cite{broadbent2024deciphering, song2023gntd, liu2017characterizing, armingol2022context} by tensor decomposition or extracting latent embeddings. We need to motivate usage of tensor decomposition in integrating spatial transcriptomics by regions of interest. }

Spatial transcriptomics enables the mapping of gene expression patterns within tissue architecture, providing critical insights into cellular organization and function. However, analyzing such high-dimensional data remains a computational challenge, especially when integrating multiple regions of interest (ROIs) to uncover spatially organized molecular patterns. As discussed above, tensor decomposition is a powerful approach to extract meaningful latent structures from spatial transcriptomic datasets, facilitating deconvolution of cell–cell interactions, spatial clustering of gene expression, and integration of multimodal spatial omics data. Armingol et al.~\cite{armingol2022context} developed \texttt{Tensor-cell2cell}, a non-negative tensor component analysis method, which is an extension of NMF to higher-order tensors—to analyze cell–cell communication while incorporating spatial context. In their work, a 4D tensor representing cell–cell interactions was reconstructed, and a comprehensive error analysis was subsequently performed. Traditional approaches often rely on bulk transcriptomic deconvolution, which disregards spatial positioning, but \texttt{Tensor-cell2cell} leverages tensor decomposition to preserve high-order interactions between signaling pathways across different tissue regions. By integrating spatially resolved transcriptomic profiles, this method enhances the identification of key cell–cell communication networks that drive tissue function and disease progression. In another study~\cite{broadbent2024deciphering}, a graph-guided Tucker decomposition method was used to decipher high-order structures in spatial transcriptomic data. This approach combines TD with graph-based priors to account for spatial dependencies between neighboring regions of interest. Unlike conventional decomposition techniques, which treat gene expression as independent across space, this method embeds structural tissue information into the decomposition process, leading to improved reconstruction of spatial transcriptomes and more biologically meaningful clustering of tissue subregions. Low-rank TD methods have also been utilized in characterizing the spatiotemporal transcriptome of the human brain~\cite{liu2017characterizing}. By applying tensor factorization, the study extracted low-dimensional representations of gene expression across different brain regions and developmental time points, enabling the identification of major gene expression modules that define functional brain architecture. This method provides a scalable framework for integrating multi-region transcriptomic datasets, offering insights into both spatial organization and temporal dynamics of gene regulation in the brain. Tensor-based spatial transcriptomics analysis was further advanced by the introduction of Graph-Guided Neural Tensor Decomposition (\texttt{GNTD})~\cite{song2023gntd}.  This model integrates spatial and functional relationships by embedding gene co-expression networks into the TD framework. Unlike traditional tensor methods that primarily focus on spatial structure, \texttt{GNTD} combines neural network-based embeddings with graph-guided TD, allowing for the reconstruction of missing transcriptomic signals while preserving biologically relevant spatial interactions. This approach enhances the accuracy of transcriptomic reconstructions in sparsely sampled regions, improving the detection of functional tissue microdomains.

Collectively, these studies highlight the power of TD in integrating and analyzing spatial transcriptomic data across different regions of interest. Hence, tensor-based techniques can uncover latent gene expression patterns, reconstruct missing transcriptomic information, and improve spatially aware clustering of cells. As spatial transcriptomics technologies continue to evolve, tensor decomposition will play a crucial role in enabling more precise and scalable analyses of high-dimensional spatial omics datasets.

% \subsubsection{Challenges}
% %\AB{Phase transition/computational hardness discussions with plots following from the notebook shared with you}

% Despite the advantages of tensor decomposition in spatial transcriptomics, several challenges remain in its application to high-dimensional and spatially heterogeneous datasets. One key challenge is the complexity of spatial dependencies, as gene expression varies not only across different tissue regions but also within microenvironments, making it difficult to define an optimal tensor structure that captures both local and global expression patterns. Traditional tensor decomposition methods often assume a predefined rank or spatial resolution, which may not be suitable for highly dynamic and heterogeneous tissue architectures. Additionally, sparsity in spatial transcriptomic data—where many genes are not detected in all spatial locations—limits the accuracy of decomposition models and can lead to biased reconstructions, especially when missing data is not randomly distributed but influenced by biological or technical factors. Another challenge is computational scalability, as large-scale spatial transcriptomics datasets with thousands of genes and spatial positions require substantial memory and processing power, making traditional tensor-based approaches computationally expensive. Furthermore, biological interpretability of tensor decomposition outputs remains a challenge, as extracted latent components may not always correspond to clear biological pathways or cell-type-specific interactions, necessitating additional validation with external datasets or functional assays. Finally, integration of multimodal spatial omics data poses a challenge, as combining spatial transcriptomics with proteomics or epigenomic data using tensor methods requires designing decomposition strategies that can accommodate different data modalities, resolutions, and measurement biases. Addressing these challenges will be crucial for unlocking the full potential of tensor decomposition in spatial transcriptomics and improving its utility for biological discovery.

\noindent \textbf{Diagonal Layout.}
Fig. \ref{fig:spatial} provides a visual comparison of spatial symmetry in positional encodings.
For vanilla RoPE (Fig. ~\ref{fig:vanilla_rope}), no spatial relation is considered and the index for every dimension increases directly.
While M-RoPE (Fig. \ref{fig:m_rope}), incorporates spatial information within each frame, it introduces two significant discontinuities between textual and visual tokens.
This arises from M-RoPE's placement strategy, if the first visual token is at $(0, 0)$, the last token in each frame will always be placed at $(W-1, H-1)$, creating a stack in the bottom-left corner.
Furthermore, like vanilla RoPE, M-RoPE's indices increase with input length across all dimensions.

To address these limitations, \methodname arranges the entire input along the diagonal, see Fig. \ref{fig:video_rope}.
The central patch's 3D position for each video frame is $(t,t,t)$, with other patches offset in all directions.
Our \textbf{Diagonal Layout} has two advantages: (1) our design preserves the relative positions of visual tokens and ensures approximate equidistance from the image corners to the center, preventing text tokens from being overly close to any corner. (2) It maintains the indexing pattern of vanilla RoPE (Fig.  \ref{fig:spatail_index}), as the position index increment between corresponding spatial locations in adjacent frames mirrors that of adjacent textual tokens.

\noindent \textbf{Adjustable Temporal Spacing.}
To scale the temporal index, we introduce a scaling factor $\delta$ to better align temporal information between visual and textual tokens.

Suppose the symbol $\tau$ denotes the token index, for the starting text ($0 \leq \tau < T_s$), the temporal, horizontal, and vertical indices are simply set to the raw token index $\tau$.
For the video input ($T_s \leq \tau < T_s + T_v$), The difference $\tau - T_s$ represents the index of the current frame relative to the start of the video, which is then scaled by $\delta$ to control the space in the temporal dimension.
For the ending text ($T_s + T_v \leq \tau < T_s + T_v + T_e$), the temporal, horizontal, and vertical index are the same, creating a linear progression.

According to our adjustable temporal spacing design, for a multi-modal input that consists of a text with $T_s$ tokens, a following video with $T_v$ frame with $W\times H$ patches in each frame, and an ending text with $T_e$ tokens, the position indices $(t, x, y)$ of \methodname for $\tau$-th textual token or $(\tau, w, h)$-th visual token are defined as Eq. (\ref{equ:index}):
\begin{equation}
\vspace{-3pt}
\resizebox{0.5\textwidth}{!}{$
    \footnotesize
    (t,x,y) =
    \begin{cases}
        (\tau, \tau, \tau) & \text{if } 0 \leq \tau < T_s \\[3ex]
        \left( 
        \begin{array}{l}
            T_s + \delta (\tau - T_s), \\
            T_s + \delta (\tau - T_s) + w - \frac{W}{2}, \\
            T_s + \delta (\tau - T_s) + h - \frac{H}{2}
        \end{array}
        \right) & \text{if } T_s \leq \tau < T_s + T_v \\[6ex]
        \left( 
        \begin{array}{l}
            T_s + \delta T_v + \tau, \\
            T_s + \delta T_v + \tau, \\
            T_s + \delta T_v + \tau
        \end{array}
        \right) & \text{if } T_s + T_v \leq \tau < T_s + T_v + T_e
    \end{cases}
$}
\raisebox{-9.5ex}{,}
\label{equ:index}
\end{equation}
where $w$ and $h$ represent the horizontal and vertical indices of the visual patch within the frame, respectively.

In summary, the parameter $\delta$ in our adjustable temporal spacing allows for a flexible and consistent way to encode the relative positions of text and video tokens.

\section{User Study Design} \label{sec:study} %1.5p
\section{User Study}
\label{section:user_study}

\subsection{Procedure}

The first author used FeedforwardChat for four weeks.

\subsection{Results and Reflections}





\section{Results \& Analysis} \label{sec:results} %1p
% \section{Simulation Evaluation \& Results}\label{sec:results}

\subsection{Baseline Planners}

To evaluate the performance of \PlannerName, we compare it against several baseline methods. In the following section, we describe these baselines, their implementation details, and their respective advantages and limitations, particularly in the context of information gathering in large, high-dimensional search spaces. The simulation framework and vehicle parameters remain consistent across all planners, and each method is allowed to replan during testing.

\subsubsection{Monte-Carlo Tree Search}

Monte Carlo Tree Search (MCTS) can be a powerful technique for finding feasible and optimal paths in complex environments. It is a heuristic search algorithm that builds a search tree incrementally through repeated simulations. At each iteration, it selects a node to explore based on a selection policy (often the Upper Confidence Bound or UCB1 algorithm), expands the tree by adding possible actions from that node, runs a simulation from the newly added node, and updates the statistics of nodes along the path traversed during the simulation. 

The UCB1 (Upper Confidence Bound) algorithm is a technique commonly used in the context of multi-armed bandit problems and Monte Carlo Tree Search (MCTS) for balancing exploration and exploitation. It helps in selecting actions or nodes that are likely to yield high rewards while also exploring less-frequented options to gather more information about their potential rewards. 

We formulate our UCB score in the following manner, \\
\begin{equation*}
    UCB_\text{node} = \frac{I(X_{\text{node}})}{\alpha} + C \times \sqrt{\frac{\ln(N_\text{tree})}{N_\text{node}}}
\end{equation*}
%  $
% UCB_\text{node} = \frac{\overline{X_\text{node}}}{\alpha} + C \times \sqrt{\frac{\ln(N_\text{tree})}{N_\text{node}}}
% $ \\
Here $I(X_{\text{node}})$ denotes the estimated information gain from the node, $\alpha$ denotes the normalization factor which is given by $\frac{B}{v_\text{desired}}$, $B$ being the maximum planning budget and $v_\text{desired}$ being the desired speed of our UAV. $C$ denotes the exploration weight, and $N_\text{tree}$ denotes the number of visits to the tree root node while $N_\text{node}$ denotes the number of times the present node has been visited.

After selecting a candidate node, if it has been visited before, it is expanded by applying motion primitives to generate child nodes, growing the tree. Unvisited nodes skip this step. Following expansion, either the unvisited candidate node or one of its children is selected for the simulation phase, where the future values of nodes along the path are estimated to update the total potential information gain. This informs the selection policy in subsequent iterations. Once planning time is exhausted, the path with the highest information gain is returned.

% with authors goes here
\begin{figure}[t]
\centering
\includegraphics[trim={.7cm 0cm .5cm 1.4cm},clip,width=\columnwidth]{figs/5_/Results1v3.pdf}
\caption{The Monte Carlo simulation results for the planners. The plots show the average percent reduction in entropy over the course of the simulations, and the shading shows the 95\% confidence intervals. IA-TIGRIS outperforms all of the baselines.}
\label{fig:mc_results}
\end{figure}

While MCTS is probabilistically guaranteed to converge to the optimal path \cite{mcts_ref_1}, it is constrained to actions within a predefined set of motion primitives. Its reliance on random sampling to estimate the future value of nodes can result in poor approximations, particularly in environments with sparse, localized pockets of high information gain. This limitation is especially pronounced in large search areas or scenarios with large budgets constraints, where estimating future node values becomes increasingly expensive. As a result, in such scenarios, MCTS is often implemented with a finite planning horizon, which can restrict its ability to account for long-term consequences or dependencies in the environment.

% This property of MCTS, which causes unguided exploration of the environment, leads to increased convergence times on the optimal path, as a result of a lot of budget being spent in exploring information sparse areas of the map. 
% Also, the computation time of MCTS increases exponentially with the depth of the search tree. The time complexity of MCTS is given by $\mathcal{O}(\frac{T}{t_\text{iter}} \cdot |A|^d)$. Here, $T$ is the total planning time and $t_\text{iter}$ is the time taken per iteration of the planning loop. $|A|$ is the number of actions and $d$ represents the average depth of the search tree. 

% The above limitations are not inconsequential in the context of performing informative path planning in large high-dimensional search spaces. We compare MCTS with \PlannerName, in \ref{}, and empirically demonstrate its drawbacks and how \PlannerName, is able to outperform MCTS in the context of the mission parameters we examine in this work.  

\subsubsection{Greedy}

For the greedy planner, we iterated through each cell within the search bounds and calculated the reward for a given cell $i$ as $g_i = R(X_i) / d_i$ where $R(X_i)$ is given through \eqref{equ:reward} and $d_i$ represents the Euclidean distance between the current position the robot at the current time $t$ and the closest viewpoint to the cell. To compute this viewpoint, the yaw between the current pose of the robot and the intersected cell is first calculated. Using the robot's sensor configuration and this yaw, $x$ and $y$ coordinates are calculated that view the cell at the desired flight altitude. With this formulation, the planner prioritizes regions with a high ratio of entropy to distance. This can lead to locally optimal choices that contradict with paths that lead to higher information gain over the entire trajectory. 

% without authors goes here
% \begin{figure}[t]
% \centering
% \includegraphics[trim={.7cm 0cm .5cm 1.4cm},clip,width=\columnwidth]{figs/5_/Results1v3.pdf}
% \caption{The Monte Carlo simulation results for the planners. The plots show the average percent reduction in entropy over the course of the simulations, and the shading shows the 95\% confidence intervals. IA-TIGRIS outperforms all of the baselines.}
% \label{fig:mc_results}
% \end{figure}


\begin{figure*}[t]
    \centering
    \begin{subfigure}[b]{0.99\textwidth}
        \centering
        \includegraphics[trim={0cm 0.3cm 0cm 0cm},clip,width=\textwidth]{figs/5_/Fig2v1_target.png}
        % \caption{Slice by targets}
        % \vspace{.1cm}
    \end{subfigure}
    
    \begin{subfigure}[b]{0.99\textwidth}
        \centering
        \includegraphics[trim={0cm 0cm 0cm 0cm},clip,width=\textwidth]{figs/5_/Fig2v1_sigma.png}
        % \caption{Slice by sigma }
    \end{subfigure}
    \caption{A comparison of the methods based on the number of sampled prior clusters and the standard deviation of sampled prior clusters. IA-TIGRIS is most effective compared to the baselines when there is high variation in the search space. As the search space prior information becomes more evenly spread out, the performance gap between the methods tends to decrease.}
    \label{fig:targets_sigmas}
\end{figure*}

\subsubsection{Random}

The random planner operates by iteratively sampling points within the defined search bounds and calculating the minimum-cost path to observe each sampled point. This process is repeated until the available budget is fully expended. The random planner does not utilize any prior information about the environment or target distribution. Additionally, it does not optimize the sequence of actions, instead treating each sampled point independently without considering the global structure of the search problem. This simplicity allows the random planner to highlight the performance benefits of more sophisticated methods by providing a lower-bound comparison for evaluation.

\subsubsection{Coverage}

The coverage planner generates a plan that systematically covers the entire search space using a straightforward lawn-mower pattern. The spacing between each pass is set to match the width of the projected observation footprint at 20\% from the bottom, ensuring that no grid cells are missed. This spacing also maintains a distance that enables high-quality sensor measurements. However, due to the size of the search spaces considered, the coverage planner spends significant time surveying empty regions. This approach results in inefficient use of the budget, as it prioritizes full coverage with safe sensor overlap, even in areas with little or no valuable information. While simple and robust, this method highlights the tradeoff between exhaustive coverage and efficient, targeted exploration.

% \subsubsection{Branch and Bound}
% The branch and bound baseline is based on motion primitive planning. In each future step the drone has a set of motion primitives with future states and each of these future states also has a set of motion primitives. In this way, a tree can be built with multiple path candidates. The path candidate with the highest information gain will be selected and form the output. 

% By adding branch and bound, there will be an estimation of a node's upper bound information reward, using the node's current information reward, updated information map and the remaining budget. If this upper bound is already lower than the information reward of any other node in the tree, the corresponding node will be closed and not expanded in the future to accelerate the expansion of the tree. 



\subsection{Tests and Analysis}
% To evaluate the efficacy of IA-TIGRIS compared to the baseline methods, we conduct Monte Carlo testing as well as analyze how the prior and budget affect the performance of each method. In all of these test cases, there are no time-based or priority rewards and have horizon lengths set to the full budget. All tests were performed using an Intel Xeon CPU E5-2620 v4 @ 2.10GHz.
To evaluate the efficacy of IA-TIGRIS against baseline methods, we perform Monte Carlo testing and analyze the impact of the prior and budget on the performance of each method. In all test cases, rewards are calculated using \eqref{equ:reward}, and horizon lengths are set to match the full budget. The tests are conducted on an Intel Xeon CPU E5-2620 v4 @ 2.10GHz, ensuring consistent computational conditions across all evaluations.

% Random sample across which parameters.

% Quantitative ideas. Look into number and std of prior (metric for this? std of grid cell values, mediuan, mean,). 
% Uniform prior? 
% Split distinct regions, not smooth. 
% Compare to coverage and amount of time to reach specific amount. 
% Compare with different budgets. 
% Repeatability test. 
% Graph size vs time. 
% Look at coverage with different altitudes or widths. Something that shows long horizon vs not nature of things?
% Shape of search space?
% Time/budget to get x\% of all info gain. Have to do moving horizon. 
% Targets detected? 

% Key thought for results where I show time, our optimization does not optimize for time, only final value. Key thing to show across the different budgets. 

% \BM{Qualitative. Nayana idea of plot with example sampled case. Should add one here.} 



\subsubsection{Monte Carlo Testing}
Our simulated testing environment is a $5000\times5000$ m square with Gaussian-distributed prior information randomly placed throughout the search space. The number of prior clusters was sampled uniformly between $[4,20]$, with standard deviations between $[60,450]$, and maximum value between $[0.05,0.5]$. 

The results of $100$ Monte Carlo tests are shown in Fig.~\ref{fig:mc_results}. IA-TIGRIS clearly outperforms the other methods, achieving nearly a $40\%$ greater reduction in entropy than the next best method. Early in the simulation, the greedy method initially gains information more quickly, as expected, but this does not translate to better long-term performance. Since our method optimizes for total information gain, it generates paths that maximize information collection over the entire budget. MCTS performed slightly worse than the greedy approach.

The random paths slightly outperformed the coverage paths. This is likely because the lawnmower strategy requires sufficient overlap between passes to avoid missing areas, and its long straight paths often lead to redundant observations due to the UAV’s forward-facing camera. Changing the heading of the UAV is beneficial to viewing more of the search space, which may explain why random paths performed better.

We also conducted Monte Carlo tests where either the number of prior clusters or their standard deviation was held constant to analyze how variations in the information map affect planner performance. The results, shown in Fig.~\ref{fig:targets_sigmas}, include two cases: the upper figure fixes the number of priors, while the lower figure fixes their standard deviation. All other agent and simulation parameters remained unchanged.


% The first thing to note from these results is that for all tests the proportional performance gap between IA-TIGRIS and the baselines increases as the number and standard deviation of the Gaussian priors decreases. As the search space becomes more uniformly filled with entropy in the information map, the need for longer-horizon planning decreases and other simple or random approaches can perform satisfactorily given the testing budget. As the information becomes more sparsely distribution in the space, such as when the information is contained in separated pockets of areas, there is a greater need to plan longer-horizon paths that reason about the given budget.
% \BM{Could have figures here or refer to others}

Across these tests, the performance gap between IA-TIGRIS and the baselines widens as the number and standard deviation of the Gaussian priors decrease. When entropy is more uniformly distributed across the search space, simpler methods perform reasonably well within the given budget. However, when information is concentrated in sparse, distinct regions, longer-horizon planning becomes essential. In such cases, IA-TIGRIS demonstrates a significant advantage by effectively reasoning about the budget and prioritizing high-value regions.

% Show plot of first plans expected info gain versus planning time. (plans not executed)


\subsubsection{Budget Analysis}
To evaluate the impact of budget constraints on performance, we conducted additional tests beyond our initial Monte Carlo experiments, evaluating budgets of $5000$ m, $10000$ m, $30000$ m, and $60000$ m. Table~\ref{tab:budgets} summarizes the average entropy reduction across these budgets.

\definecolor{tabfirst}{rgb}{1, 0.7, 0.7} % red
\definecolor{tabsecond}{rgb}{1, 0.85, 0.7} % orange
\definecolor{tabthird}{rgb}{1, 1, 0.7} % yellow
\begin{table}[t]
    \centering
    \resizebox{\linewidth}{!}{
    \begin{tabular}{l|ccccc}
    & $5000$ m & 10000 m  & 15000 m& 30000 m& 60000 m\\ \hline

    % \hline
    IA-TIGRIS  &  \cellcolor{tabfirst}$9.41\pm1.0$ &  \cellcolor{tabfirst}$18.28\pm1.8$ & \cellcolor{tabfirst}$25.36\pm2.3$ & \cellcolor{tabfirst}$41.08\pm2.9$ & \cellcolor{tabfirst}$58.85\pm2.9$ \\
    Greedy  &  \cellcolor{tabsecond}$6.99\pm0.8$ &  \cellcolor{tabsecond}$13.10\pm1.5$ & \cellcolor{tabsecond}$17.97\pm2.0$ & \cellcolor{tabthird}$30.00\pm2.3$ & \cellcolor{tabsecond}$49.38\pm3.5$ \\
    MCTS  &  \cellcolor{tabthird}$6.06\pm0.7$ &  \cellcolor{tabthird}$11.80\pm1.1$ & \cellcolor{tabthird}$17.11\pm1.4$ & \cellcolor{tabsecond}$30.21\pm2.2$ & \cellcolor{tabthird}$48.68\pm2.7$ \\
    Random  &  $2.19\pm0.3$ & $4.29\pm0.7$ & $6.61\pm0.6$ & $17.50\pm1.2$ & $22.47\pm1.4$ \\
    Coverage  &  $1.58\pm0.3$ &  $2.82\pm0.4$ & $4.09\pm0.7$ & $12.04\pm1.9$ & $16.77\pm2.4$ \\

    \end{tabular}
    }
    \caption{Monte Carlo testing results given different budgets. The values are the average percent reduction in entropy and the 95\% confidence bounds. \mbox{IA-TIGRIS} had the best performance for all budgets.}
    \label{tab:budgets}
\end{table}
%$\uparrow$ 

IA-TIGRIS consistently achieved the highest entropy reduction across all budget constraints, with a statistically significant margin over alternative methods. Greedy generally ranked second but was slightly outperformed by MCTS at the $30000$ m budget level. Greedy and MCTS exhibited comparable performance throughout the tests, with their results closely tracking each other. Consistent with our previous findings, Random and Coverage methods yielded the lowest results.


Among the tested methods, only IA-TIGRIS and MCTS explicitly incorporate budget constraints into their planning algorithms. Notably, at lower budgets ($5000$ m and $10000$ m), these methods achieved higher entropy reduction compared to the equivalent time steps ($200$ s and $400$ s) in the $15000$ m budget scenario shown in Fig.~\ref{fig:mc_results}. This improved performance stems from IA-TIGRIS's optimization of total path reward under budget constraints, contrasting with the myopic next-best-action approach of the greedy method. The remaining methods---Greedy, Random, and Coverage---maintain consistent behavior regardless of budget constraints, as their planning strategies do not account for resource limitations.


The performance gap between IA-TIGRIS and the next-best method varied with budget size, showing margins of $34.6\%$, $39.5\%$, $41.1\%$, $36.0\%$, and $19.2\%$ in ascending budget order. This gap widened through the first three budget levels as problem complexity increased, before declining significantly at higher budgets. This performance pattern suggests that implementing a planning horizon could enhance efficiency by limiting tree search depth, enabling the planner to prioritize path quality optimization over exhaustive space exploration.


% percent improved from next best
% 34.6, 39.5, 41.1, 36.0, 19.2
% reasons, too long horizon is a larger search space, so less quality paths closer. Or larger horizon, more packing in


% with authors goes here
\begin{figure}[t] 
    \centering
    \renewcommand\arraystretch{0} % Adjust the height between rows here
    \setlength{\tabcolsep}{1pt} % Adjust the column separation here
    \begin{tabular}{c}
        \begin{tikzpicture}
            \node[anchor=south west, inner sep=0] (image) at (0,0) {
                \includegraphics[width=0.9\linewidth]{figs/5_/google_earth_prior.png}
            };
            \begin{scope}[x={(image.south east)},y={(image.north west)}]
                % \fill[OrangeRed] (0.02, 0.03) circle (2pt); 
                % \fill[OrangeRed] (0.51, 0.04) circle (2pt); 
                % \fill[OrangeRed] (0.61, 0.04) arc (0:90:2pt); 
                \fill[Orange, opacity=0.8] (0.74, 0.45) circle (3pt); % Adjust 
                \fill[Orange, opacity=0.8] (0.27, 0.42) circle (3pt); % Adjust 
                \fill[Orange, opacity=0.8] (0.39, 0.63) circle (3pt); % Adjust 
            \end{scope}
        \end{tikzpicture} \\
        % \includegraphics[width=0.9\linewidth]{figs/5_/google_earth_prior.png} \\
        \\
        \includegraphics[width=0.9\linewidth]{figs/5_/google_earth_path.png} 
    \end{tabular}
    \caption{Google Earth screenshots illustrating the mission planning process and execution. Top: Areas of high entropy targeted for search are highlighted in red, representing regions with a binary occupied/unoccupied probability of 0.2. Three points of particular interest, each assigned a 0.5 probability, are marked in orange. Bottom: The executed drone flight path (yellow) shows the optimized path for maximum information gain across the search space.} 
    \label{fig:google_earth}
\end{figure}
\begin{figure}[t]
\centering
% https://docs.google.com/presentation/d/1RjI-QqHpBRLHN60UAxzmQYs4EaWaVCOoSBkEkA39kk0/edit?usp=sharing
\includegraphics[width=\columnwidth]{figs/5_/m600_labeled.jpg}
\caption{Hexarotor system (DJI M600 Pro) with onboard compute and camera. Left image shows drone on the ground, right image shows drone in flight.}
\label{fig:m600}
\end{figure}


\section{Field Deployments}\label{sec:field}


\subsection{Hexarotor Deployment}
The first field experiment that we present uses a hexarotor drone to cover an urban area shown in Fig.~\ref{fig:fig1}.
We designed this field experiment to simulate classifying where cars are within a search area.  
Hence, we set the plan request to focus on parking lots at the field test site (Fig.~\ref{fig:google_earth}, top), with the addition of three chosen grid cells within the parking lots being marked as having a higher uncertainty. The plan request boundaries and priors were created with GPS coordinates in Google Earth, exported as kml files, and then converted into our plan request message format. 

The following sections details the hardware, autonomy, and experimental results for our hexarotor deployments.

% without the authors goes here
% \begin{figure}[t] 
%     \centering
%     \renewcommand\arraystretch{0} % Adjust the height between rows here
%     \setlength{\tabcolsep}{1pt} % Adjust the column separation here
%     \begin{tabular}{c}
%         \begin{tikzpicture}
%             \node[anchor=south west, inner sep=0] (image) at (0,0) {
%                 \includegraphics[width=0.9\linewidth]{figs/5_/google_earth_prior.png}
%             };
%             \begin{scope}[x={(image.south east)},y={(image.north west)}]
%                 % \fill[OrangeRed] (0.02, 0.03) circle (2pt); 
%                 % \fill[OrangeRed] (0.51, 0.04) circle (2pt); 
%                 % \fill[OrangeRed] (0.61, 0.04) arc (0:90:2pt); 
%                 \fill[Orange, opacity=0.8] (0.74, 0.45) circle (3pt); % Adjust 
%                 \fill[Orange, opacity=0.8] (0.27, 0.42) circle (3pt); % Adjust 
%                 \fill[Orange, opacity=0.8] (0.39, 0.63) circle (3pt); % Adjust 
%             \end{scope}
%         \end{tikzpicture} \\
%         % \includegraphics[width=0.9\linewidth]{figs/5_/google_earth_prior.png} \\
%         \\
%         \includegraphics[width=0.9\linewidth]{figs/5_/google_earth_path.png} 
%     \end{tabular}
%     \caption{Google Earth screenshots illustrating the mission planning process and execution. Top: Areas of high entropy targeted for search are highlighted in red, representing regions with a binary occupied/unoccupied probability of 0.2. Three points of particular interest, each assigned a 0.5 probability, are marked in orange. Bottom: The executed drone flight path (yellow) shows the optimized path for maximum information gain across the search space.} 
%     \label{fig:google_earth}
% \end{figure}
% \begin{figure}[t]
% \centering
% % https://docs.google.com/presentation/d/1RjI-QqHpBRLHN60UAxzmQYs4EaWaVCOoSBkEkA39kk0/edit?usp=sharing
% \includegraphics[width=\columnwidth]{figs/5_/m600_labeled.jpg}
% \caption{Hexarotor system (DJI M600 Pro) with onboard compute and camera. Left image shows drone on the ground, right image shows drone in flight.}
% \label{fig:m600}
% \end{figure}

\subsubsection{Hardware System}
The hardware consists of the DJI M600 Pro, shown in Fig.~\ref{fig:m600}, along with the physical sensing and onboard computer payload. The DJI M600 Pro contains a flight controller that handles pose estimation and position-based control. The DJI M600 Pro’s flight controller also handles teleloperation if human intervention is necessary. Beneath the drone's base, we mount a custom hardware payload.
That payload consists of an onboard computer, a Jetson Xavier, to run the autonomy software shown in Fig.~\ref{fig:functional_diagram}.
The payload also contains a downward-facing a camera for sensing the environment. The camera is a Seek S304SP thermal camera.
The camera intrinsics are used to calculate the frustum's intersection with the search map's cells in IA-TIGRIS.

% without authors goes here
\begin{figure}[t]
\centering
% https://lucid.app/lucidchart/f750ddb4-2809-4773-8361-d5fbb1ba49eb/edit?viewport_loc=-257%2C-116%2C2219%2C1140%2C0_0&invitationId=inv_56e8a3a9-e8cf-4cad-a280-48bd967ff651
\includegraphics[trim={0cm 0cm 0cm 0cm},clip,width=\columnwidth]{figs/5_/functional_diagram.jpeg}
\caption{Functional diagram of the DJI M600 Pro autonomy software.}
\label{fig:functional_diagram}
\end{figure}
\begin{figure}[b]
    \centering
    \begin{subfigure}[b]{0.48\columnwidth}
        \centering
        \includegraphics[width=1.0\linewidth]{figs/5_/field_test_altitude_over_time.png}
        \caption{}
        \label{fig:m600_altitude_over_time}
    \end{subfigure}
    \begin{subfigure}[b]{0.48\columnwidth}
        \centering
        \includegraphics[width=1.0\linewidth]{figs/5_/field_test_entropy_over_time.png}
        \caption{}
        \label{fig:m600_entropy_over_time}
    \end{subfigure}
    \caption{The results for our hexarotor field deployment. (a) Plot of flown altitude over time, showing large variation throughout the experiment. (b) Reduction in entropy percentage over time of field experiment.}
\end{figure}

\subsubsection{Autonomy System}
Fig.~\ref{fig:functional_diagram} illustrates the functional system diagram for the real world field test on the DJI M600. The user specifies the initial plan request prior to takeoff. The TIGRIS planner makes an initial plan on that plan request and sends a global path to the waypoint manager. The waypoint manager tracks the current waypoint within the plan and sends the next waypoint to the DJI software development kit, which then sends actuation commands to the motors. The position of the drone is used to calculate the distance from the drone to the ground and sends that distance parameter to the sensor model. The sensor model's true positive and false positive rate is used to calculate the per-cell entropy updates in the search map manager. The search map manager publishes the current information map, and the replanning node sends an updated plan request to the IA-TIGRIS planner every ten seconds.

The drone started at an altitude of $50$ m above the origin of the reference frame. The informed sampler in IA-TIGRIS was set to add states at altitudes of either $30$ m or $60$ m, creating a trade-off between observation area and detector accuracy. The budget was $2000$ m, the planning horizon was $600$ m, and the planning time was $10$ seconds. 

% % without authors goes here
% \begin{figure}[t]
% \centering
% % https://lucid.app/lucidchart/f750ddb4-2809-4773-8361-d5fbb1ba49eb/edit?viewport_loc=-257%2C-116%2C2219%2C1140%2C0_0&invitationId=inv_56e8a3a9-e8cf-4cad-a280-48bd967ff651
% \includegraphics[trim={0cm 0cm 0cm 0cm},clip,width=\columnwidth]{figs/5_/functional_diagram.jpeg}
% \caption{Functional diagram of the DJI M600 Pro autonomy software.}
% \label{fig:functional_diagram}
% \end{figure}
% \begin{figure}[b]
%     \centering
%     \begin{subfigure}[b]{0.48\columnwidth}
%         \centering
%         \includegraphics[width=1.0\linewidth]{figs/5_/field_test_altitude_over_time.png}
%         \caption{}
%         \label{fig:m600_altitude_over_time}
%     \end{subfigure}
%     \begin{subfigure}[b]{0.48\columnwidth}
%         \centering
%         \includegraphics[width=1.0\linewidth]{figs/5_/field_test_entropy_over_time.png}
%         \caption{}
%         \label{fig:m600_entropy_over_time}
%     \end{subfigure}
%     \caption{The results for our hexarotor field deployment. (a) Plot of flown altitude over time, showing large variation throughout the experiment. (b) Reduction in entropy percentage over time of field experiment.}
% \end{figure}

\subsubsection{Experimental Results}


The bottom image of Fig.~\ref{fig:google_earth} shows the path selected by IA-TIGRIS in the search area. The figure highlights how the planner dynamically adjusts altitudes over time to balance coverage and sensing resolution, maximizing information gain. Higher altitudes allow for broader area coverage, while lower altitudes provide more detailed observations where needed. Additionally, the planner prioritizes revisiting the three regions of higher uncertainty, recognizing the need for repeated observations reduce entropy. This adaptive strategy ensures that uncertain areas receive sufficient attention to improve the belief map. As a result, the entropy of the information map decreases to near zero by the end of the mission, as shown in Fig.~\ref{fig:m600_entropy_over_time}, indicating that the planner has effectively gathered the necessary information. This behavior demonstrates the planner’s ability to optimize sensing actions, balancing altitude selection, revisit frequency, and exploration to maximize mission success.

\begin{figure}[t]
\centering
% \includegraphics[width=2.5in]{fig1}
\includegraphics[trim={4cm 4cm 0cm 4cm},clip,width=\columnwidth]{figs/5_/TL1.jpg}
\caption{Fixed-wing platform used for autonomous flights with an onboard camera pitched at 10 degrees\cite{alarewebsite}}
\label{fig:tl1}
\end{figure}






\subsection{Fixed-wing Deployments}

Our proposed approach was extensively tested on the fixed-wing AlareTech TL-1 UAV, shown in Fig.~\ref{fig:tl1}. The UAV is equipped with an onboard camera pitched at 10 degrees, which introduces a more challenging planning problem due to the non-holonomic motion model and the camera's field of view. Over more than 20 flight hours and 100 flights running IA-TIGRIS, we validated our approach with the objective to search for objects of interest in a large search space across a variety of test scenarios, including different terrain types, varying environmental conditions, and diverse target distributions. An example mission from these tests is shown in Fig.~\ref{fig:fwd}. In this scenario, the planner was given the search bounds and a designated high-priority region. The resulting flight path prioritized revisiting the high-priority area twice, optimizing sensor use and ensuring maximum information gain. This strategy led to the successful detection of the object of interest, with its estimated position marked by the red dot in the figure. 

The map on the upper right in Fig.~\ref{fig:fwd} shows the information map after plan execution was complete. Due to the UAV's limited budget, the upper right and lower left corners of the map are not searched by the agent. The budget is instead utilized to search over the area of higher priority two times. Compared to the paths in Fig.~\ref{fig:google_earth}, we observe that the paths for the fixed wing are smoother and have a larger turning radius, demonstrating how IA-TIGRIS respects the motion constraints of the vehicle. We can also see the effect of wind on the path execution, where the flown path shown in green deviates from the planned path shown in yellow. This illustrates the importance of online planning in the cases where this deviation is large or would accumulate over the course of a longer mission and cause the expected observed area to be much different than actual observed area. 

\begin{figure}[t]
\centering
% \includegraphics[width=2.5in]{fig1}
% [trim={left bottom right top},clip]
\includegraphics[trim={3.0cm, 1.0cm, 3.0cm, 1.0cm},clip,width=\columnwidth]{figs/5_/ONRFig_v3.pdf}
\caption{An example path generated for the fixed-wing platform conducting a large-area search for an object of interest. The larger black rectangle denotes the search bounds, while the smaller black rectangle highlights a region of higher uncertainty. The red dot marks the estimated position of the detected object based on image detections. The upper-right map displays the information state after planning is complete, while the middle plot shows the percent change in entropy over mission time. The flown path illustrates a balance between allocating resources to the high-priority region and exploring other areas within the search space.}
\label{fig:fwd}
\end{figure}

% Also tested extensively on the AlareTech TL-1 (citation?) tube launched UAV seen in Fig.~\ref{fig:tl1}.

% Talk about amount of flights, hours. Platform. Compute. Show visualization fo example flight. Talk about objects of interest in a broad sense (no mention of water/ocean/land for targets). Follow similar figure format as previous section. Main thing we want to highlight is the differences introduced in plans by having a fixed-wing platform compared to a drone. Include image of Alare TL-1 somewhere.

% One big figure showing all the info we want to convey. 

% \BM{Pitch 10 degrees, onboard computer type, etc}


% \subsection{VTOL?}
% what would it bring?



\section{Discussion} \label{sec:conclu} %0.5p 
\section{Concluding Remarks}
In this paper, we proposed a novel approach utilizing multimodal LLMs to generate gesture-aware speech recognition transcripts for patients with language disorders. Our framework integrates verbal speech and iconic gestures, enabling the generation of enriched transcripts that capture the latent meaning conveyed through both modalities. Through extensive experimentation, we demonstrated that the proposed method effectively contextualizes incomplete or disfluent speech by incorporating gesture information, leading to more accurate and meaningful representations of the speaker's intent. These findings highlight the potential of our approach to significantly contribute to the field of speech and language therapy, offering innovative tools that can enhance the quality of life for individuals with language disorders by facilitating better communication and assessment methods.

\subsection{Ethical Statement} 
Our dataset was obtained from AphasiaBank with the approval of the Institutional Review Board (IRB) and adheres to the data sharing guidelines set by TalkBank\footnote{https://talkbank.org/share/ethics.html}. This includes complying with the Ground Rules for all TalkBank databases, which are based on the American Psychological Association Code of Ethics~\cite{american2002ethical}.

\subsection{Limitation \& Future Work} 
%This study represents a preliminary investigation into using multimodal LLMs to generate gesture-aware speech recognition transcripts. 
While the results are promising, we recognize several limitations and outline our plans to extend this work further.

One primary limitation is the absence of a definitive ground truth for quantitative evaluation. Since our model generates transcripts by synthesizing speech and gesture data from scratch, traditional benchmarks, such as comparisons with standard speech recognition outputs, are insufficient. Moreover, existing original transcripts lack gesture annotations, making direct comparisons challenging. In future work, we aim to address this gap by collaborating with certified pathologists to conduct qualitative assessments, such as A-B preference tests, to evaluate the effectiveness of gesture-enriched transcripts in accurately conveying the speaker's intentions.

To support quantitative evaluations, we plan to develop novel metrics that assess transcript quality, including grammar accuracy, semantic consistency, and the integration of multimodal information. Such metrics will provide a more objective basis for assessing our model's performance and facilitate comparisons with other multimodal and unimodal approaches.

Another limitation of this study is its focus on structured gestures from a specific task, the Peanut Butter Sandwich Task. While this task offers a controlled context for testing our approach, it does not encompass the diversity of gestures and communication patterns seen in everyday scenarios. As part of our future work, we plan to expand the scope of our model to include tasks such as the Cinderella Story Recall Task~\cite{bird1996cinderella}, which involves unstructured and complex narrative gestures. This expansion will allow us to evaluate the adaptability and robustness of our model in handling varied linguistic and gestural contexts.

In summary, while this study establishes a strong foundation for gesture-aware speech recognition, we aim to refine and extend our methods through collaborative qualitative evaluations, the development of robust quantitative metrics, and broader task applications. These efforts will ensure that our approach continues to evolve, ultimately contributing to more effective communication tools and interventions for individuals with language disorders.





%%
%% The acknowledgments section is defined using the "acks" environment
%% (and NOT an unnumbered section). This ensures the proper
%% identification of the section in the article metadata, and the
%% consistent spelling of the heading.
% \begin{acks}
% acknowledgement if any
% \end{acks}

%%
%% The next two lines define the bibliography style to be used, and
%% the bibliography file.
\bibliographystyle{ACM-Reference-Format}
\bibliography{references}


%%
%% If your work has an appendix, this is the place to put it.
\appendix

\section{Metric}
\label{sec:metric}

\textbf{Mean Squared Error (MSE)} Mean Squared Error (MSE) is a common statistical metric used to assess the difference between predicted and actual values. The formula is:
\begin{equation}
    MSE = \frac{1}{n} \sum_{i=1}^{n} (y_i - \hat{y}_i)^2
\end{equation}
where $ n $ is the number of samples, $ y_i $ is the actual value, and $ \hat{y}_i $ is the predicted value.

\textbf{Relative L2 Error} Relative L2 error measures the relative difference between predicted and actual values, commonly used in time series prediction. The formula is:
\begin{equation}
    \text{Relative L2 Error} = \frac{\| Y_{\text{pred}} - Y_{\text{true}} \|_2}{\| Y_{\text{true}} \|_2}
\end{equation}
where $ Y_{\text{pred}} $ is the predicted value and $ Y_{\text{true}} $ is the actual value.

\textbf{Structural Similarity Index Measure (SSIM)} The Structural Similarity Index (SSIM) measures the similarity between two images in terms of luminance, contrast, and structure. The formula is:
\begin{equation}
    SSIM(x, y) = \frac{(2\mu_x \mu_y + C_1)(2\sigma_{xy} + C_2)}{(\mu_x^2 + \mu_y^2 + C_1)(\sigma_x^2 + \sigma_y^2 + C_2)}
\end{equation}
where $ \mu_x $ and $ \mu_y $ are the mean values, $ \sigma_x $ and $ \sigma_y $ are the standard deviations, $ \sigma_{xy} $ is the covariance.

\section{Related Work}
\subsection{Deep Learning based Weather Forecasting}
\textbf{Global Weather Forecasting.} Global weather forecasting has seen significant progress with deep learning models. FourCastNet, based on Fourier neural operators, provides global forecasts comparable to traditional numerical methods like IFS, but at much higher speeds~\cite{pathak2022fourcastnet}. Pangu, utilizing the Swin Transformer, exceeds NWP methods, incorporating earth-specific location embeddings for better performance~\cite{bi2023accurate}. The Spherical Fourier Neural Operator (SFNO) extends Fourier methods using spherical harmonics, offering more stable long-term predictions~\cite{bonev2023spherical}. FuXi focuses on long-term forecasting, achieving a 15-day forecasts comparable to ECMWF~\cite{chen2023fuxi}. GraphCast leverages message-passing networks to improve efficiency and forecasting accuracy~\cite{lam2023learning}, and GenCast builds on this to enhance ensemble forecasting~\cite{price2023gencast}. Further, diffusion models like those in~\cite{li2024generative} generate probabilistic ensembles by sampling, while NeuralGCM~\cite{kochkov2024neural} focuses on atmospheric circulation with a dynamic core, offering climate simulation capabilities but at higher training and inference costs. 

\textbf{Regional Weather Forecasting.} The goal of regional weather forecasting is to enhance local prediction accuracy with high-resolution models. CorrDiff~\cite{mardani2023generative} combines U-Net and diffusion models to improve local forecasts. MetaWeather~\cite{kim2024metaweather} adapts global forecasts to regional contexts using meta-learning. GNNs are also widely applied in regional forecasting, with Graphcast~\cite{lam2023learning} enhancing accuracy by modeling complex spatial dependencies. MetNet-3~\cite{espeholt2022deep} offers high-accuracy forecasts for weather variables, such as precipitation, temperature, and wind speed, at 2-minute intervals and 1–4 km resolution, outperforming traditional models like HRRR. NowcastNet~\cite{zhang2023skilful} and DGMR~\cite{ravuri2021skilful} excel in short-term extreme precipitation forecasts using deep generative models and radar data. In spatiotemporal prediction, NMO~\cite{wu2024neural} models the evolution of physical dynamics, providing new insights for local weather forecasting. Similarly, SimVP~\cite{gao2022simvp} and PastNet~\cite{wu2024pastnet} achieve good results in forecasting local precipitation evolution using spatiotemporal convolution methods.
    
% Despite these advances, none of these methods effectively address the challenge of balancing global and regional high-resolution forecasts or capturing the fine-grained, dynamic interactions important for extreme event prediction.
    
\subsection{Numerical analysis methods}
Multigrid methods~\cite{mccormick1987multigrid,wesseling1995introduction,hackbusch2013multi,bramble2019multigrid,hiptmair1998multigrid,brandt1983multigrid,borzi2009multigrid} and nested grid strategies~\cite{miyakoda1977one,zhang2012nested,sullivan1996grid} are widely used to solve PDEs and handle multi-scale problems~\cite{debreu2008two,xue2000advanced}. Multigrid methods use grids of different resolutions to transfer information and accelerate iterations. They efficiently solve large-scale problems and improve computational accuracy. By eliminating low-frequency errors on coarse grids and high-frequency errors on fine grids, multigrid methods effectively handle error convergence at different scales~\cite{he2019mgnet,he2023mgno,shao2022fast}. Nested grid strategies embed higher-resolution fine grids into regions of interest based on a global coarse grid to capture local complex physical phenomena in detail. In weather forecasting, this method provides large-scale background fields on a global scale while refining the grid for target regions to accurately simulate the evolution of local weather systems and the occurrence of extreme events~\cite{bacon2000dynamically}. 

% Our proposed neural nested grid method helps address challenges like boundary information loss in regional forecasting and multi-scale feature capture.

\section{Additional Results}
%
We present more additional results in Figure \ref{fig_0.25-day}, \ref{fig_0.5-day}, \ref{fig_1.0-day} \ref{fig_1.5-day}, \ref{fig_2.0-day}, \ref{fig_2.5-day}, \ref{fig_3.0-day}, \ref{fig_3.5-day}, \ref{fig_4.0-day}, \ref{fig_4.5-day}, \ref{fig_5.0-day}, \ref{fig_5.5-day}, \ref{fig_6.0-day}, \ref{fig_6.5-day}, \ref{fig_7.0-day}, \ref{fig_7.5-day},
\ref{fig_8.0-day}, \ref{fig_8.5-day}, \ref{fig_9.0-day}, \ref{fig_9.5-day},
\ref{fig_10.0-day}, including 18 variables that are importmant to weather forecasting, each with results ranging from 6 hours to 10 days. These additional results further demonstrate the effectiveness of OneForecast. Same as the Figure \ref{fig:visual_results}
, the initial conditions is 00:00 UTC, 1 January 2020.


\begin{figure*}[h]
\centering
\includegraphics[width=1\linewidth]{figures/fig_0.25-day.jpg}
\vspace{-20pt}
\caption{6-hour forecast results of different models.}
\label{fig_0.25-day}
\end{figure*}

\begin{figure*}[h]
\centering
\includegraphics[width=1\linewidth]{figures/fig_0.5-day.jpg}
\vspace{-20pt}
\caption{0.5-day forecast results of different models.}
\label{fig_0.5-day}
\end{figure*}

\begin{figure*}[h]
\centering
\includegraphics[width=1\linewidth]{figures/fig_1.0-day.jpg}
\vspace{-20pt}
\caption{1-day forecast results of different models.}
\label{fig_1.0-day}
\end{figure*}

\begin{figure*}[h]
\centering
\includegraphics[width=1\linewidth]{figures/fig_1.5-day.jpg}
\vspace{-20pt}
\caption{1.5-day forecast results of different models.}
\label{fig_1.5-day}
\end{figure*}

\begin{figure*}[h]
\centering
\includegraphics[width=1\linewidth]{figures/fig_2.0-day.jpg}
\vspace{-20pt}
\caption{2-day forecast results of different models.}
\label{fig_2.0-day}
\end{figure*}


\begin{figure*}[h]
\centering
\includegraphics[width=1\linewidth]{figures/fig_2.5-day.jpg}
\vspace{-20pt}
\caption{2.5-day forecast results of different models.}
\label{fig_2.5-day}
\end{figure*}

\begin{figure*}[h]
\centering
\includegraphics[width=1\linewidth]{figures/fig_3.0-day.jpg}
\vspace{-20pt}
\caption{3-day forecast results of different models.}
\label{fig_3.0-day}
\end{figure*}

\begin{figure*}[h]
\centering
\includegraphics[width=1\linewidth]{figures/fig_3.5-day.jpg}
\vspace{-20pt}
\caption{3.5-day forecast results of different models.}
\label{fig_3.5-day}
\end{figure*}

\begin{figure*}[h]
\centering
\includegraphics[width=1\linewidth]{figures/fig_4.0-day.jpg}
\vspace{-20pt}
\caption{4-day forecast results of different models.}
\label{fig_4.0-day}
\end{figure*}

\begin{figure*}[h]
\centering
\includegraphics[width=1\linewidth]{figures/fig_4.5-day.jpg}
\vspace{-20pt}
\caption{4.5-day forecast results of different models.}
\label{fig_4.5-day}
\end{figure*}


\begin{figure*}[h]
\centering
\includegraphics[width=1\linewidth]{figures/fig_5.0-day.jpg}
\vspace{-20pt}
\caption{5.0-day forecast results of different models.}
\label{fig_5.0-day}
\end{figure*}

\begin{figure*}[h]
\centering
\includegraphics[width=1\linewidth]{figures/fig_5.5-day.jpg}
\vspace{-20pt}
\caption{5.5-day forecast results of different models.}
\label{fig_5.5-day}
\end{figure*}

\begin{figure*}[h]
\centering
\includegraphics[width=1\linewidth]{figures/fig_6.0-day.jpg}
\vspace{-20pt}
\caption{6.0-day forecast results of different models.}
\label{fig_6.0-day}
\end{figure*}

\begin{figure*}[h]
\centering
\includegraphics[width=1\linewidth]{figures/fig_6.5-day.jpg}
\vspace{-20pt}
\caption{6.5-day forecast results of different models.}
\label{fig_6.5-day}
\end{figure*}

\begin{figure*}[h]
\centering
\includegraphics[width=1\linewidth]{figures/fig_7.0-day.jpg}
\vspace{-20pt}
\caption{7.0-day forecast results of different models.}
\label{fig_7.0-day}
\end{figure*}

\begin{figure*}[h]
\centering
\includegraphics[width=1\linewidth]{figures/fig_7.5-day.jpg}
\vspace{-20pt}
\caption{7.5-day forecast results of different models.}
\label{fig_7.5-day}
\end{figure*}

\begin{figure*}[h]
\centering
\includegraphics[width=1\linewidth]{figures/fig_8.0-day.jpg}
\vspace{-20pt}
\caption{8.0-day forecast results of different models.}
\label{fig_8.0-day}
\end{figure*}

\begin{figure*}[h]
\centering
\includegraphics[width=1\linewidth]{figures/fig_8.5-day.jpg}
\vspace{-20pt}
\caption{8.5-day forecast results of different models.}
\label{fig_8.5-day}
\end{figure*}

\begin{figure*}[h]
\centering
\includegraphics[width=1\linewidth]{figures/fig_9.0-day.jpg}
\vspace{-20pt}
\caption{9.0-day forecast results of different models.}
\label{fig_9.0-day}
\end{figure*}

\begin{figure*}[h]
\centering
\includegraphics[width=1\linewidth]{figures/fig_9.5-day.jpg}
\vspace{-20pt}
\caption{9.5-day forecast results of different models.}
\label{fig_9.5-day}
\end{figure*}

\begin{figure*}[h]
\centering
\includegraphics[width=1\linewidth]{figures/fig_10.0-day.jpg}
\vspace{-20pt}
\caption{10.0-day forecast results of different models.}
\label{fig_10.0-day}
\end{figure*}


\section{Detailed Mathematical Proof}
\label{sec:proof}
\textbf{Proof of Theorem 1}

Now we have N augmented data and we need to select the best from them. We consider both the quality and the diversity of these data and get the sampling strategy from an optimization problem.

We model the sampling strategy as a multinomial distribution supported on all the augmented data $S = \{\mathbf{X}_j\}_{j=1}^N$, which means that the sampling strategy $\pi=(\pi_1,...,\pi_N)^\top$ is the corresponding probabilities of selecting $\mathbf{X}_1,...,\mathbf{X}_N$, then we can model the expectation of the similarity as:
$$\begin{aligned}
 & \mathbb{E}_{Y_x,Y_{x^{\prime}}\in\mathcal{C}}\{g(x,x^{\prime})\mid S\} \\
 & =\quad\int g(\mathbf{x},\mathbf{x}^{\prime})\boldsymbol{\pi}(\mathbf{x})\mathrm{Pr}_{S}(Y_{x}\in\mathcal{C}\mid\boldsymbol{x}=\mathbf{x})\boldsymbol{\pi}(\mathbf{x}^{\prime})\mathrm{Pr}_{S}(Y_{x}\in\mathcal{C}\mid\boldsymbol{x}=\mathbf{x}^{\prime})d\mathbf{x}d\mathbf{x}^{\prime} \\
 & =\quad\sum_{i,j=1}^Ng(\mathbf{X}_i,\mathbf{X}_j)\pi_i\pi_j\mathrm{Pr}_{S}(Y_x\in\mathcal{C}\mid\boldsymbol{x}=\mathbf{X}_i)\mathrm{Pr}_{S}(Y_x\in\mathcal{C}\mid\boldsymbol{x}=\mathbf{X}_j),
\end{aligned}$$
where the set $\mathcal{C}$ denotes the criterion of selection we are using, the function $g$ can be chosen as any similarity metric function and $x$ means a random variable.

The core to solving the above optimization problem is to use predictive inference to approximate the conditional probability of $\{Y_x\in\mathcal{C}\}$ given $x = \mathbf{X}$
Let $\mu ( \mathbf{x} ) : = \mathbb{E} ( Y\mid \mathbf{X} = \mathbf{x} )$ be the oracle associated with $( \mathbf{X} , Y) .$ Denote $\theta_j=\mathbb{I}\{Y_j\in\mathcal{C}\}$. As the augmented data
$\mathbf{X}_1,...,\mathbf{X}_N$ are independently identically distributed, $\theta_1,...,\theta_N$ can be regarded as independent Bernoulli($q)$ variables with $q=\Pr(Y_j\in\mathcal{C}).$ The probability distribution of the predicted result $W_j$ for $j=1,...,N$ is
$$\Pr(W_j\mid\theta_j)=(1-\theta_j)f_0+\theta_jf_1,\quad$$
where $f_0$ and $f_1$ are the conditional distributions of $W_j$ on $Y_j \in \mathcal{C}$ or not.

Denote $T(w) = \frac{(1-q)f_0(W_j)}{f(W_j)}$, we can rewrite the expectation of the similarity as
$$\mathbb{E}_{Y_x,Y_{x^{\prime}}\in\mathcal{C}}\{g(x,x^{\prime})|S\}=\sum_{i,j=1}^Ng(\mathbf{X}_i,\mathbf{X}_j)\pi_i\pi_j(1-T_i)(1-T_j)=\boldsymbol{\pi}^\top A_\mathbb{T}\boldsymbol{\pi},$$

Next, we use the expectation to control the quality of the data.
$$\mathbb{E}\{\mathbb{I}(Y_x\not\in\mathcal{C})\mid S\}=\sum_{i=1}^N\Pr(Y_i\not\in\mathcal{C}\mid\mathbf{X}_i)\pi_i=\sum_{i=1}^N\pi_iT_i\leq\alpha,$$

In all, the optimization problem can be modeled as 
\begin{align}
    & \arg\min_{\boldsymbol{\pi}}\quad h(\boldsymbol{\pi},\mathbb{T}):=\boldsymbol{\pi}^\top A_\mathbb{T}\boldsymbol{\pi}, \\
    & \text{subject to} \quad
        \begin{cases}
            \sum_{i = 1}^N\pi_iT_i\leq\alpha, \\
            \sum_{i = 1}^N\pi_i = 1, \\
            0\leq\pi_i\leq m^{-1}, \quad 1\leq i\leq N.
        \end{cases}
\end{align}

where $m$ is used to control the maximum selection.

The best selection of K is determined by the strategy $\pi$ which serves as the solution to the above optimization problem.

\section{Additional Experiments}
\label{sec:more_experiments}
\subsection{Long-term forecasting experiment expansion}

In the long-term forecasting experiments, we compare the performance of different backbone models on the SWE benchmark, evaluating the relative L2 error for three variables (U, V, and H). Our setup inputs 5 frames and predicts 50 frames. For the SimVP-v2 model, using \method{} reduces the relative L2 error for SWE (u) from 0.0187 to 0.0154, SWE (v) from 0.0387 to 0.0342, and SWE (h) from 0.0443 to 0.0397. We visualize SWE (h) in 3D as shown in Figure~\ref{fig:case} [\textcolor{red}{I}]. For the ConvLSTM model, applying \method{} reduces the relative L2 error for SWE (u) from 0.0487 to 0.0321, SWE (v) from 0.0673 to 0.0351, and SWE (h) from 0.0762 to 0.0432. For the FNO model, using \method{} reduces the relative L2 error for SWE (u) from 0.0571 to 0.0502, SWE (v) from 0.0832 to 0.0653, and SWE (h) from 0.0981 to 0.0911. Overall, \method{} significantly improves the long-term forecasting accuracy of different backbone models.

\begin{figure*}[h]
    \centering
    \includegraphics[width=\textwidth]{image/casestudy.pdf}
    \caption{
    \textcolor{red}{I.} 3D visualization of the SWE(h), showing Ground-truth, SimVP-V2+BeamVQ predictions, and Error at T=1, 10, 20, 30, 40, 50. The first row shows Ground-truth, the second SimVP-V2+BeamVQ predictions, and the third Error. \textcolor{red}{II.} A case study. Building fire simulation with ventilation settings added to Wu's Prometheus~\cite{wu2024prometheus}. (a) Layout and HRR growth. (b) Comparison of physical metrics for different methods. (c) Ground-truth, ResNet+BeamVQ, and ResNet predictions.
    }
    \label{fig:case} 
\end{figure*}


\subsection{Experiment Statistical Significance}
\label{sec:significance}
To measure the statistical significance of our main experiment results, we choose three backbones to train on two datasets to run 5 times. 
Table~\ref{tab:significance} records the average and standard deviation of the test MSE loss.
The results prove that our method is statistically significant to outperform the baselines
because our confidence interval is always upper than the confidence interval of the baselines. 
Due to limited computation resources, we do not cover all ten backbones and five datasets, 
but we believe these results have shown that our method has consistent advantages.


\begin{table}[h]
\label{tab:significance}
\centering
\begin{scriptsize}
    \begin{sc}
    \caption{ The average and standard deviation of MSE in 5 runs}
    \label{tab:significance}
    \centering
        \renewcommand{\multirowsetup}{\centering}
        \setlength{\tabcolsep}{10pt}
        \begin{tabular}{l|cc|cc}
            \toprule
            
            \multirow{4}{*}{Model} & \multicolumn{4}{c}{Benchmarks}  \\
            \cmidrule(lr){2-5}
            & \multicolumn{2}{c}{NSE} &   \multicolumn{2}{c}{SEVIR}   \\
            \cmidrule(lr){2-5}
           & Ori & + BeamVQ & Ori & + BeamVQ  \\
            \midrule
            ConvLSTM &0.4092$\pm$0.0002 &\textbf{0.1277$\pm$0.0001}  & 0.1762 0.0007  & \textbf{0.1279$\pm$0.0009}  \\
            FNO &  0.2227$\pm$0.0003 &\textbf{0.1007 $\pm$0.0002}& 0.0787$\pm$0.0012 & \textbf{ 0.0437$\pm$0.0013} \\
            CNO & 0.2192 $\pm$0.0008 &\textbf{ 0.1492$\pm$0.0011}& 0.0057$\pm$0.0005 & \textbf{ 0.0053$\pm$0.0006} \\
            \bottomrule
        \end{tabular}
    \end{sc}

\end{scriptsize}
\end{table}


\end{document}
\endinput
%%
%% End of file `sample-sigconf-authordraft.tex'.
