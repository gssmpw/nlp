\section{Related work}
\paragraph{Swap regret} There is a large line of recent work on designing low-swap regret algorithms and understanding their game-theoretic guarantees \citep{braverman2018selling, deng2019prior, deng2019strategizing, camara2020mechanisms, mansour2022strategizing, cai2023selling, brown2024learning, haghtalab2024calibrated}. Almost all of these results are constrained to the discrete setting for low dimension. Very recently  \cite{dagan2023external} and \cite{peng2023fast} presented algorithms that achieve full swap regret of $O(T/\log T)$, independent of $d$. This work improves on these results in the regime $d = o(\log(T)/\log\log(T))$. \cite{roth2024forecasting} and \cite{hu2024calibrationerrordecisionmaking} study the problem of designing forecasts such that any downstream agent incurs low swap regret; the problem faced by a single downstream agent (with potentially many actions but where the payoff only depends on a low-dimensional outcome) can be interpreted as a swap-regret minimization problem in a structured game. 

\paragraph{$\phi$-regret} Gordon et al. \citep{GordonGreenwaldMarks2008} introduced a generalization of swap regret called $\phi$-regret, where instead of competing only with standard swaps you compete with all transformation functions $\phi:\cK\rightarrow \cK$ belonging to a set $\Phi$. Our full swap regret corresponds to the extreme case of this where $\Phi$ contains every such function. \cite{mansour2022strategizing}, \cite{farina2024polynomial}, and \cite{daskalakis2024efficient} study a variant of $\phi$-regret called $\LSR$ that competes only against linear maps $\phi$. This is a much weaker notion than $\FSR$ and does not suffice for obtaining bounds on calibration error.  Other variants of $\phi$-regret minimization have found applications in designing learning algorithms for extensive-form games \citep{celli2021decentralized, bai2022efficient, zhang2024efficient}. 

\paragraph{Calibration} There is a well-established connection in the literature between calibration error and swap-regret \citep{cesa2006prediction} (in fact the very first low swap-regret algorithms worked via responding to calibrated loss estimates, \cite{foster1997calibrated}). Designing online forecasters with good calibration guarantees is a problem of major interest \citep{brier1950verification,murphy1972scalar,murphy1973new,foster1998asymptotic, kleinberg2023u, gopalan2022omnipredictors}, with the optimal online bounds for $\ell_1$ calibration a major open problem \citep{qiao2021stronger}. Recent work of \cite{dagan2024breakingt23barriersequential} improve upon the bounds for $\ell_1$ calibration from \cite{qiao2021stronger}. However, our work focuses on $\ell_2$ calibration, as the $\ell_1$ calibration loss cannot be written as the swap regret in some game. 

\paragraph{Structured games} Our definition of $d$-dimensional structured games is similar in many ways to the definition of convex games (also appearing under the names ``polyhedral games'' or ``polytope games'' in the literature), where both players pick actions in a convex set and obtain utility given by a bilinear function of both players' actions \citep{GordonGreenwaldMarks2008, chakrabarti2024efficient, mansour2022strategizing}. We use the term ``structured game'' throughout this paper to emphasize the fact that when computing the swap regret, the swap functions act on the pure strategies in the associated normal-form game, not on the embeddings (i.e., we think of such a game as a large normal-form game with additional structure).