\section{Conclusion}
\label{sec:conclusion}
In this paper, we point out several limitations in the existing evaluation metrics of MI attacks and propose a novel metric, Diversity and Distance Composite Score (DDCS), to alleviate those limitations and encourage a more comprehensive evaluation of MI attacks.
%Focusing on sample-level privacy instead of label level privacy, DDCS accounts for the reconstruction distance of each target sample so as to consider multiple important characteristics of MI attacks and be robust on redundant samples from the reconstructed dataset.
%It is worth noting that DDCS assists a better analysis of sample vulnerability against MI attacks and indicates the low coverage in recent MI attacks.
Furthermore, to enhance existing MI attacks, we further propose a GAN augmentation framework with transfer learning for state-of-the-art MI attacks.
%In GAN augmentation, a pre-trained GAN with natural gradient on entropy loss is utilized to improve the generative ability of the MI attack.
%Extensive experiments on face identification and dog breed classification confirm the comprehensiveness and robustness of DDCS, as well as the effectiveness of our approach.
Overall, we emphasize the importance of informing the academic community about potential threat models of MI attacks and introducing new perspectives on privacy measurement, to foster the development of more robust privacy-preserving ML algorithms.

%Applying other techniques for improving GAN transfer learning~\cite{freezed20,lecamdiv21cvpr} to MI attacks are interesting for further discussion. 