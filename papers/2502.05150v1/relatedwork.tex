\section{Related Work}
% {\bf Automatic Code Generation.} 
\paragraph{Automatic Code Generation.}
Code generation with multi-modal prompts has been explored by some earlier works such as \cite{desai2015program, gulwani2017program}. Recent works have either adopted the transformer architecture \cite{feng2020codebert, wang2021codet5} or leveraged the GPT \cite{brown2020language} skeleton with massive pretraining for code pretraining \cite{rozière2023code, li2023starcoder, luo2023wizardcoder, nijkamp2023codegen, zhu2024deepseek}.

\paragraph{Prompt-Tuning.} 
Various approaches to prompt-tuning \cite{white2023prompt} have been explored for various domains and modalities \cite{mullick2024intent, wu2023self}, such as Chain-of-Thought reasoning \cite{wei2023chainofthought}, Tree of Thoughts \cite{yao2023treethought}, discrete prompt optimization \cite{wen2023hard, shin2020autoprompt}, and few-shot learning \cite{brown2020language}. In the context of code generation, prompt engineering has been leveraged for human-in-loop debugging \cite{denny2022conversing}, correctness evaluation of generated code \cite{liu2023code}, multistep planning, and generation \cite{zheng2023outline}. Our work explores the effects of modalities in prompts on code generation, which can be further used for targeted prompt-tuning processes for better performance.

\paragraph{Causal Inference in Code/NLP.} 
Recent research has applied causal inference to the NLP domain \cite{vig2020investigating, finlayson2021causal, stolfo2022causal} to better understand model behavior, which is now formalized as causal NLP \cite{jin-etal-2022-causalnlp, feder2022causal}. 
In the context of code, prior approaches have applied causal framework for various classification tasks such as vulnerability detection \cite{rahman2024towards, 9793858} and code performance prediction \cite{cito2021counterfactualexplanationsmodelscode}.
%\citet{rahman2024towards} apply a causal framework for a vulnerability detection task, a classification problem. 
To the best of our knowledge, we are the first to apply causal inference to study modal effects on code generation task.