% MT for personal use emerged in the late 1990s with free online services,
% %like Babel Fish, 
% following the rise of personal computers and the Web \citep{yang1998systran, gaspari-hutchins-2007-online}. 
% Despite early reports of MT consumed by the public \citep{mccarthy2004does, somers2005round}, they represented secondary concerns. 
Traditionally, real-world applications of MT have regarded so-called ``mixed MT'' workflows \citep{wagner1983rapid}, where human intervention was responsible to revise---i.e. post-edit \citep{li2023post}---MT and ensure a reliable final translation.\bes{if space allows add that this use case was envisioned by Bar and clarify that this is still a realistic application (@bea)} 
With MT primarily consumed and used by \textit{professionals}, this context shaped its evolution \citep{church1993good}, influencing system development \citep{green2014human, bentivogli2015evaluation, daems2019interactive}, application interfaces ~\citep{vieira2011review, vela2019improving}, and evaluation criteria  \citep{popovic2011towards, bentivogli2016neural}, with empirical experiments on when MT could support \citep{koponen2016machine, moorkens2017assessing}, or interfere \citep{federico2014assessing, daems2017identifying} with translators' activity.\bes{If space allows, insert again=  Studies have examined how trust in MT affects adoption \citep{scansani2019translator}, which errors are most disruptive to translators (e.g., coherence issues being more problematic than local meaning shifts) \citep{federico2014assessing, daems2017identifying}, and the minimum quality threshold required for MT to actually support the translation workflow (). Similarly, \citet{turchi2015mt} explored whether quality estimation (QE) predictions of MT output quality could aid post-editors, as well as its potential and limitations in flagging word-level errors \citep{shenoy-etal-2021-investigating}. (@bea)}
This literature, though, while meaningful, does not directly inform how to support lay users of MT, which face different needs, limits, and potentially other desiderata in heterogeneous situations \citep{szymanski2024comparingcriteriadevelopmentdomain}.\bs{[migliorare raccordo tra paragrafi]\mnc{Direi che, se questo gap e' stato accettabile per lungo tempo, non lo e' piu' oggi con l'avvento di Transformer-based models che hanno made strides...}} 
% While meaningful, the literature on professional users does not directly transfer and reflect the desiderata or limits of MT lay users \citep{szymanski2024comparingcriteriadevelopmentdomain}. With professional users in mind, MT potential for social impact is also limited, as they have the expertise to oversee and grant control over automatic translations.

Powerful 
%transformed-based 
Transformer-based
models have made strides in capabilities, language coverage, and online presence.\footnote{e.g. \href{https://blog.google/products/translate/google-translate-new-languages-2024/}{Google Translate} expanding language coverage; MT is embedded into social media platforms like \href{https://slator.com/reddit-ceo-calls-machine-translation-big-unlock-for-growth/}{Reddit}.
% e.g. Google Translate adding 110 new languages in 2024 (\url{https://blog.google/products/translate/google-translate-new-languages-2024/}); MT is embedded into social media platforms like Reddit (see \url{https://slator.com/reddit-ceo-calls-machine-translation-big-unlock-for-growth/})
\citet{thompson-etal-2024-shocking} estimated that a huge portion of the web is automatically translated.} 
Such advancements have allowed the general public to consume \textit{raw} MT output in an expanded range of scenarios, e.g.
to gist content and for multilingual conversation \citep{conversation, pombal2024context}, in education (Yang et al., 2022), but also in high-stakes domains such as 
 healthcare \citep{khoong2019assessing}, migration support \citep{liebling-etal-2022-opportunities}, and emergency services \citep{TURNER2015136}.\footnote{Though critical, MT comes in where human translators are unavailable---such as in the COVID-19 pandemic, when interpreter shortages were widespread \citep{khoong2022research, anastasopoulos2020tico}.}
 However, MT \mnc{Calcherei parlando non di MT ma di ``unmediated MT'' (dall'intervento dei professionals).} is not infallible: mistranslations can cause discomfort, misunderstandings, or even lead to life-threatening errors \citep{taira2021pragmatic} or wrongful arrests.\footnote{Infamously, a construction worker was arrested due to a mistranslation of ``good morning'' as ``attack them'', as reported by \citet{guardian2017facebook}.}
 % Unlike professionals, non-experts struggle to critically engage with MT,  as they might lack linguistic expertise or knowledge of technology limitations, thus increasing the risk of taking translation at face value.  
 
 \bs{[following passage to better integrate}
 Unlike professionals, non-experts struggle to critically engage with MT, often lacking both linguistic expertise and an understanding of its limitations, which increases the risk of uncritically accepting translations at face value. Current MT tools are designed for passive consumption, offering little guidance on when or how users should verify or refine outputs. There is a significant gap in research on user behaviors, effective support methods, and actionable measures that could help users determine when MT interactions genuinely benefit them.\bs{]} 
 
 % Despite this transformation, we are overlooking how to best support non-experts, how to improve their interactions with MT systems and align it   
 % in making informed decisions on  research on lay MT users remains sparse. Studies have largely overlooked how non-experts evaluate MT quality, their level of skepticism, and their approach to cross-lingual communication.

% Such concerns have been brought to the fore with multilingual and LLMS, like ChatGPT offer multilingual and multitask capabilities, attracting millions of users with their versatile, interactive interfaces.\footnote{According to OpenAI, in the summer of 2024, the service reached 200 million weekly active users (\url{https://www.reuters.com/technology/artificial-intelligence/openai-says-chatgpts/weekly-users-have-grown-200-million-2024-08-29/}).}


\textbf{A shift to LLMs?}\bes{to revisit and integrate}
LLMs have revolutionized language technologies, and we are attesting to a potential shift from dedicated applications to multilingual and multitask solutions that can introduce new opportunities \citep{haque2022think}. Chat-based LLMs like ChatGPT have drawn in millions of users, also thanks to their impressive versatility and engaging interfaces that allow to verbalize lay user requests.\footnote{According to \href{https://www.reuters.com/technology/artificial-intelligence/openai-says-chatgpts/weekly-users-have-grown-200-million-2024-08-29/}{OpenAI}, in the summer of 2024  the service reached 200 millions weekly active users.} 
%
While automatic translation is not among the main tasks for which users interrogate LLMs \citep{ouyang-etal-2023-shifted}, it remains a significant area of exploration \citep{zhu2023multilingual, lyu2024paradigm}. The field of MT is progressively adopting and exploring LLM-based methods (see also Figure~\ref{fig:trend-overview}, \textsc{mt+lm} \emph{vs} \textsc{mt} on the \emph{left}), reflecting a shift in how we engage with NLP technologies. This change has sparked interest in studies that focus on real-world user contexts, helping us better understand user needs (e.g. \citet{tao-etal-2024-chatgpt, Skjuve_Brandtzaeg_Følstad_2024, Kim_2024, STOJANOV2024100243, bodonhelyi2024userintentrecognitionsatisfaction, wang2024user, HYUNBAEK2023102030}).
%
Furthermore, it is essential to rethink how MT integrates with LLMs, enabling more complex tasks like translating and summarizing lengthy texts or adjusting translation styles. These advancements are reshaping our approach to model design and user interaction.
%
% As we navigate this dynamic field, we must prioritize understanding end users and human factors in their interactions with these models. By centering our attention on user needs, we can ensure that these powerful technologies effectively facilitate meaningful communication across languages.
\bs{The transition from passive consumption to interactive participation raises fundamental questions about usability, trust, and digital literacy.} 

