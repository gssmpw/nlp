% This must be in the first 5 lines to tell arXiv to use pdfLaTeX, which is strongly recommended.
\pdfoutput=1
% In particular, the hyperref package requires pdfLaTeX in order to break URLs across lines.

\documentclass[11pt]{article}

% Remove the "review" option to generate the final version.
\usepackage[review]{ACL2023}

% Standard package includes
\usepackage{times}
\usepackage{latexsym}
\usepackage{graphicx} 

\usepackage{soul}
% For proper rendering and hyphenation of words containing Latin characters (including in bib files)
\usepackage[T1]{fontenc}
% For Vietnamese characters
% \usepackage[T5]{fontenc}
% See https://www.latex-project.org/help/documentation/encguide.pdf for other character sets

% This assumes your files are encoded as UTF8
\usepackage[utf8]{inputenc}

% This is not strictly necessary, and may be commented out.
% However, it will improve the layout of the manuscript,
% and will typically save some space.
\usepackage{microtype}

% This is also not strictly necessary, and may be commented out.
% However, it will improve the aesthetics of text in
% the typewriter font.
\usepackage{inconsolata}
\usepackage{xcolor}
\newcommand{\bs}[1]{\textcolor{teal}{#1}}
\newcommand{\mn}[1]{\textcolor{blue}{#1}}
\newcommand{\ar}[1]{\textcolor{cyan}{#1}}

% If the title and author information does not fit in the area allocated, uncomment the following
%
%\setlength\titlebox{<dim>}
%
% and set <dim> to something 5cm or larger.

\title{Human-centered MT: A Research Agenda}

% Author information can be set in various styles:
% For several authors from the same institution:
% \author{Author 1 \and ... \and Author n \\
%         Address line \\ ... \\ Address line}
% if the names do not fit well on one line use
%         Author 1 \\ {\bf Author 2} \\ ... \\ {\bf Author n} \\
% For authors from different institutions:
% \author{Author 1 \\ Address line \\  ... \\ Address line
%         \And  ... \And
%         Author n \\ Address line \\ ... \\ Address line}
% To start a seperate ``row'' of authors use \AND, as in
% \author{Author 1 \\ Address line \\  ... \\ Address line
%         \AND
%         Author 2 \\ Address line \\ ... \\ Address line \And
%         Author 3 \\ Address line \\ ... \\ Address line}

\author{First Author \\
  Affiliation / Address line 1 \\
  Affiliation / Address line 2 \\
  Affiliation / Address line 3 \\
  \texttt{email@domain} \\\And
  Second Author \\
  Affiliation / Address line 1 \\
  Affiliation / Address line 2 \\
  Affiliation / Address line 3 \\
  \texttt{email@domain} \\}

\begin{document}
\maketitle
\begin{abstract}
\bs{In this paper, we take stock of where we stand wrt to human-centered MT. Ideally, we highlight some challenges and opportunities and a path. \hl{They might be framed as such, or as a set of desiderata or research agenda.}} 
% The main challenge I see with this work--- besides bringing relevant arguments---is finding a way to make it specifically applicable to the MT application and not AI/NLP more generally. 
% \hl{What is specific to MT that cannot be more broadly said of NLP?}
%

Machine Translation (MT) has long been a cornerstone of NLP, with its significance continuing to grow as our interconnected and multilingual world generates ever-increasing volumes of content. Initially designed to address real-world needs for accessing foreign-language materials, MT has evolved into one of the most widely used, user-facing technologies. Today, it is deeply embedded in real-world scenarios and integrated into diverse platforms. This widespread adoption is especially impactful as most users accessing MT lack high proficiency in at least one of the target languages. Given its critical role, we argue for a human-centered approach to MT \mn{[We're not the first in arguing this: need to shape what we're pushing for, different, e.g., from Han EMNLP`23 on explicitation, or with respect to, e.g., the ``three directions'' outlined in Robertson et al 2021...Unless you aim at a survey, it is necessary to find a ``personal'' (new?) perspective on HCMT. To this aim, I think you need to define what you mean by ``Human-centered MT''; maybe, the differences wrt previous work and novelty will stem from this definition?]}. Such an approach broadens the scope of MT systems by enabling users to assess their benefits against potential risks, situating MT research within specific use cases to better characterize and mitigate the harms caused by mistranslations. The increasing diversity in MT users, contexts, and the integration of MT with chat-based LMs further underscores the need for this paradigm shift. This paper reviews these evolving dynamics \mn{[Ok, it looks like a survey. Is it the first one? If not: are previous works outdated or lacking in some aspects? Which ones?]}, highlights urgent challenges in the field, and proposes a forward-looking research agenda for human-centered MT. Our discussion spans the evolution of MT, its current landscape, and the steps needed to align MT systems more closely with the needs and experiences of their users.


\end{abstract}

\section{Introduction}

% \textbf{--}\textit{Technology is made by people for people. }
% \bigskip

% \noindent\textbf{--}\textit{The L in NLP means language, and language means people.} \bs{add cit.}

\bs{\underline{Motivation}:we are no longer within an abstract academic discipline but rather contributing to the shaping of deployed technologies that need to serve real users in real-world scenarios.  And yet, we still do not really know that much about how users interact with the technology, for what needs, in which contexts, and how we can support them properly. The development chain of models is often self-reflecting: the relevance of the proposed technologies is usually assessed wrt to values that are internal to the field}, e.g. overall performance, scalability, new domain, efficiency \citep{birhane-2022}.\footnote{Or this is at least what \citet{birhane-2022} finds in the broader Machine Learning (ML) field.} \textit{But are we empowering users and their needs? Are we putting them in the conditions to make sensible choices about how and when to use the technologies?} \mn{[Up to this point, I still stand with my first two comments and encourage more focus + stress on novelty aspects/gaps to be filled.]}

\bigskip

With the large-scale deployment of NLP models, ensuring that they genuinely serve users and can be used safely has become a critical concern. Dialogue-based interfaces have further revolutionized how users interact with these technologies. However, even before the advent of LLMs, certain applications, like Machine Translation (MT), have always been at the forefront of user-facing technologies. MT is used by billions of people globally, accessible to anyone with an internet connection, and has reached a level of quality that encourages widespread adoption.
Despite its ubiquity, our understanding of MT users—their needs, experiences, and decision-making processes—remains limited. \mn{[What are, in your opinion, the main unanswered questions? Can we start from a list to increase focus?]} 

Historically, the primary MT users were professional translators
%research and design efforts focused primarily on professional translators, 
but today’s users are far more diverse \mn{[+CIT.]}. Most are not proficient in the target language \mn{[+CIT.]}, making it difficult for them to judge the quality or reliability of translations. This creates a paradox: MT is increasingly trusted, yet many users lack the means to assess when this trust is misplaced \mn{[+CIT.]}.
The expansion of MT use cases, from low-stakes personal communication to high-stakes scenarios like employment or healthcare, further highlights the need to address this issue. For example, MT is not sufficiently reliable for migrant workers seeking jobs or healthcare information \mn{[+CIT.]}, where accuracy is critical. High user expectations often lead to over-reliance on MT in settings where errors can have serious consequences \mn{[+CIT.]} \ar{or where translations misrepresent languages and cultures~\citep{pawar2024surveyculturalawarenesslanguage}, with the effect of reinforcing biases and dominant values}.

The emergence of LLMs presents both challenges and opportunities for MT. These models enable new applications like gisting and context-sensitive translations \mn{[+CIT.]} but also raise questions about how to effectively address multilingualism and cross-linguality \mn{[+CIT.]}. To meet contemporary needs \mn{[Like?]}, we must rethink traditional paradigms and ground our research and design efforts in the realities of today’s diverse users and use cases. \mn{[Very strong (``we must rethink'') but it's not clear why. And, whatever this rethink will result in, will it apply to \textit{both} for MT and LLMs? Are you treating them as separate realms/technologies?]}


\textbf{MAIN POINT
}We argue that current research in MT, however, still overlooks human/user-centered approaches \mn{[What's the diff between human and user-centered? Are the two terms interchangeable? In the next sentence you use only ``human'', indeed.]} that span from the design of the technology, its assessment, the factors it considers, and the verification of user needs. Despite an increasing interest in human-centered directions more broadly in NLP \mn{[+CIT.]}, efforts remain scarce and/or scattered across subdisciplines. 
The time is ripe to make advances and create a set of common goals for human-centered MT \mn{[Again: how do you go beyond Robertson et al. 2021?]}, given i) its quality, ii) its widespread adoption by different users, for different usage contexts, iii) its high potential for human/social impact.

\textbf{RQ}:
What are current sore points and how do we move beyond traditional paradigms to address contemporary needs and ground them in uses and users?

\section{Preliminaries}
\bs{We introduce some preliminaries on human-centered (MT/research) and the MT trajectory. \hl{I am unsure the former is needed. Also, I am unsure about our focus: is what I am writing looking into user-centred or human-centred research?}}

\subsection{Human-centered research/approaches}
\begin{itemize}
    \item 
There is increasing interest in \textit{human-centered AI}, as highlighted in works like \cite{liao2023ai, shneiderman2022human}. This trend is also evident in the field of language technologies, which shows a growing focus on better accounting for users and adopting human-centered approaches. Examples include: \textit{i)} The \textit{Human-Centered NLP special theme} at NAACL 2022\footnote{\url{https://2022.naacl.org/blog/special-theme/}}; \textit{ii)} the introduction of a new \textit{Human-Centered track} at *CL conferences, starting with EMNLP 2023\footnote{\url{https://2023.emnlp.org/calls/main_conference_papers/}}; \textit{iii)} The \textit{HCI+NLP workshop series}, reaching its third edition in 2024~\citep{hcinlp-ws-2024-bridging}; 
%\footnote{\url{https://sites.google.com/view/hciandnlp/home}}; 
\textit{iv)} the \textit{Tutorial on Human-Centered Evaluation of Language Technologies} at EMNLP 2024~\citep{blodgett-etal-2024-human}.
%\footnote{\url{https://2024.emnlp.org/program/tutorials/}}.

\bs{\texttt{\textbf{Visualization-idea}}: query of papers matching given keywords (e.g. human-centred/centric etc) on semantic scholar of ACL anthology published in the last few years to attest trend.} \ar{If we will keep it, it would be nice to compare the trend between HC papers vs HCandMT papers to gauge if MT is actually going towards human-centeredness}


    
\item \textbf{What do we mean by human-centered?} 
As a working definition, consider \citep{kotnis2022human}:  ``Human-centric implies that human stakeholders, \underline{in addition to the researcher}, actively participate in the research project.''
Ideally, user, people benefit from the outcome of human-centered technology, and are explicitly accounted for the in the design, creation ad deployment process. \ar{This overlaps a lot with participatory design approaches which are well-studied in the HCI community but a bit neglected in NLP - we can mention it and stress that with HC we don't mean just better tech performance :)}
%\textit{Human-Centric Research (HCR)} ensures that human stakeholders, besides researcher,  benefit from research outcomes and actively participate in the research process.

This approach differs from other uses of this term, or similar, that I attested in the field:
\begin{itemize}
    \item Using \textit{human-centric} to describe development of models with human-like capabilities \citep{wang2024survey} (e.g., passing the Turing Test),\footnote{Rather, MT has been mostly developed to carry out the activity that a translator would. Now, we might even revisit if this is actually what we need to automatize if we focus on MT not for professionals, but for multilingual communication across end-users.}  or using, e.g. LLMs to perform as (human) test subjects instead of actual participants \citep{artificial}.
    \item The concept of \textit{human-in-the-loop} \citep{wang-etal-2021-putting}, which often places humans in a model-centric pipeline. For example, \citet{bird-yibarbuk-2024-centering} contrasts:
    \begin{itemize}
        \item Building machine capacity with a human in the loop.
        \item Building human capacity with a machine in the loop.
    \end{itemize}
\end{itemize}


\bs{\texttt{\textbf{Visualization-idea}}: breakdown of steps in technology design/creation/deployment where human-centered RQs and concern apply.} \ar{Tab 1 in~\citet{caselli-etal-2021-guiding} would be super useful for this}

    \end{itemize}

\subsection{The evolving MT role/paradigm}
Let us look at the "past", "present", and "future" of MT to understand what is the same, what has changed or might change, and why the discussion on human-centered MT is needed now. 
Core points regard: 

\begin{itemize}

\item \textbf{Past}

\begin{itemize}

\item MT as a task arose in the 50s from very concrete human needs\footnote{Though for espionage...}. 
\item More clear-cut distinction of MT uses: for \textit{i)} dissemination and \textit{ii)} assimilation, also called gisting. The former required professional revision, the latter was just raw MT output mostly for perishable content, not highly relevant. Now, I would say that this distinction is no longer that clear-cut. 
\item Primary focus on professionals \mn{[Do you have a CIT for this?]}: MT did not have the quality to be often used as is, and rather the focus was on integrating it into the flow of translators. At this time, CAT tool are born, from which stems \textit{interactive} MT. There are human-centered designs and studies---many from the translation studies field but not only--- though often aimed at studying productivity in the language industry. OR, studies on the trust of the translator toward MT. 

\end{itemize}

\item \textbf{Present}

\begin{itemize}

\item Democratization of translation has facilitated unprecedented levels of global communication and reach. Its influence extends into most domains of human activity. The implications are vast and with social relevance. 

\item New, high-level capabilities of MT (for many, not all languages) have expanded the range of users and scenarios of use (potentially also with misunderstanding on which langs are well-supported). Non-professional MT users is an extremely heterogeneous group. 

\item In highly interconnected multilingual society, uses of MT have expanded and are not always known.\footnote{Cool MT stories project seek to collect stories of people using MT: \url{https://mt-stories.com/}}

\item Several cases of known dangerous uses\footnote{Or ``misuses'', e.g. to vet refugees status \url{https://www.propublica.org/article/google-says-google-translate-cant-replace-human-translators-immigration-officials-have-used-it-to-vet-refugees}} of MT (some are funny, some are not). Increase number of risky situations.\footnote{See for instance ``Death by MT'' (\url{https://slate.com/technology/2022/09/machine-translation-accuracy-government-danger.html}).}

\item Online, chat communication (where a lot of MT happens), are not longer "irrelevant" situations. Sometimes, users do not known of being exposed to MT output online. 

\item Different type of stakeholders: the one \textit{using} MT vs the one \textit{receiving} an MT output. e.g. one party with authority (e.g. police officer, doctor) using MT on the other party


\end{itemize}

\bs{\texttt{\textbf{Visualization-idea}}: some (real) images of various MT errors, in different settings, with different levels of risks to guide discussion. e.g. see image \ref{fig:error} }

%%%%%%%%%%%%%%%%%%%%%%%%%%%%%%%%%
\begin{figure}
    \centering
    \includegraphics[width=1\linewidth]{img/mt.jpg}
    \caption{MT error}
      \label{fig:error}
\end{figure}
%%%%%%%%%%%%%%%%%%%%%%%%%%%%%%%%%%%%%%%%%%%%
      

\item \textbf{Future?}

- The literature \citep{tao-etal-2024-chatgpt, Skjuve_Brandtzaeg_Følstad_2024} indicates that MT is currently not a core task for users of large language models (LLMs). However, this could change as LLM capabilities evolve. Alternatively, the way we understand "translation" may shift entirely, involving more complex tasks. For instance, MT might include translating and summarizing long texts across languages\footnote{A case in point is multilingual speech summarization; see \url{https://iwslt.org/2025/instruction-following}} or translating while adjusting the style of the output. \ar{Moreover, it could provide explanations for alternative translations to inform users about the underlying language/culture assumptions, and therefore mitigate the perpetuation of (mostly) Western-centric biases.} These developments affect how we conceptualize models, design user interactions, and evaluate system performance. Chat-based LMs, for example, have already altered traditional notions of NLP tasks \citep{tao-etal-2024-chatgpt}. At the same time, they introduce new risks, such as hallucinations—errors where generated content deviates entirely from the source text—underscoring the need for careful, and novel scrutiny in MT.
\end{itemize}

\mn{\texttt{\textbf{Visualization-idea:}} Can we put these considerations on past present and future in a table. Would a structured presentation enhance clarity/effectiveness?}


\textit{Takeaways}: MT has evolved from a tool for specialists to a democratized technology, enabling global communication and cross-cultural understanding at an unprecedented scale. As MT integrates further into more complex systems and LMs, its societal impacts and potential risks demand attention. Also, the future of MT is not just about advancing technology but about recentering the focus on the people and their \hl{myriad uses of this technology,} \mn{[Love it:) Can this (i.e. the ``myriad uses'') be an aspect characterizing this paper wrt to others? Are we in the condition of contributing in this regard?]} tailoring systems to diverse needs and fostering meaningful interactions, and meaningfully involving and considering them in development.





\section{Standing issues}

\bs{Here I list some points that are relevant and might be interesting to discuss. I am currently dividing them into standing issues vs the section about future possibilities just for the sake of simplicity and to help us figure out what maps to what, not sure if it's the best way to organize them. }

\mn{[Connected to my previous comment: do you see any possibility to link the standing issues to the table I was proposing?]}

\paragraph{Trust}

- Distinguish between trustworthy systems (intrinsic reliability) and user trust (subjective perception).


- learn to find an \textit{appropriate} level of trust: achieving an
appropriate level of trust is especially critical for end-users to harness the benefits of AI systems without over-relying on flawed AI outputs. e.g. trust might be displaced when an output is \textit{believable}--though inaccurate---leading to potentially serious risks. 

- link between trust and transparency \citep{liao2023aitransparencyagellms}:  Transparency is fundamentally about supporting appropriate human understanding. Understanding also is related to trust.\footnote{\citep{liao2023aitransparencyagellms} set out common approaches that the community has taken to achieve transparency: model reporting,  publishing evaluation results, providing explanations, and communicating uncertainty.} 

- supporting a functional understanding of what the model (or system) can do, often by exposing the goals, functions, overall capabilities, and limitations vs "mechanistic" understanding with XAI. 

- I would argue it's not very clear how XAI can support the MT task right now.\footnote{Explanations should be actionable and enhance trust without overwhelming users with complexity.} 

 

\paragraph{Human Factors}

% The community did quite a pretty job in now realising sociodemographic variation of different users (e.g. age, dialect, gender etc) for developing models---though admittedly it is still not a priority. We have always had ``domains'' too as a notion in MT---for whatever that means, but it is about modeling certain linguistic patterns and specificities. We are not really accounting for e.g. stress, distraction, negative perception of technology etc and other very human conditions that can change depending on the setting and moment to reason about what could be needed. High vs low stake, and now also online communication is getting huge---when do people resolve to MT? Is there a negative perception?

The community has made progress in recognizing sociodemographic variations among users (e.g., age, dialect, gender) for model development, though it remains a secondary priority. While MT traditionally considers ``domains'' to model linguistic patterns, it often overlooks human conditions like stress, distraction, or negative perceptions of technology that vary by context. Factors such as high vs. low stakes and the growing prevalence of online communication also influence when and why people rely on MT, raising questions about user needs and perceptions.

-  Users don’t rely purely on rational criteria like system performance to determine when and how to use a system, but are influenced by emotional, social, and experiential factors. \ar{Negative perceptions/trust also include the misrepresentation of their specific language variant in MT, i.e., when (primarily-spoken) unstandardized language varieties that exhibit great inner variation are treated as monolithic entities with an artificially-imposed written standard (e.g., see the recent ``scaling'' efforts by Google Translate)}



\paragraph{Role of users and agency}

- ``people" mostly involved in MT I) to collect data, ii) as models's evaluators, iii) to correlate metrics. 

- their agency and experiences not really accounted for

- users often find their own ways to circle around limited understanding of MT, e.g. backtranslation was found as a super common way to judge the quality of MT


\paragraph{Evaluation}

- Current benchmarks are often abstract and detached from user-specific needs.

- Are metrics actionable—i.e., can users use them to meaningfully compare and select systems?


- Quality vs. usefulness/usability. e.g. for translating a letter, or literary text, or a movie, amusement, \textit{enjoyability} might have a high impact. We need to disentagle notion of quality

\ar{Representativeness. To meet user-specific needs, the system should provide translations that preserve the cultural, geographical, and linguistic aspects of the target language (in addition to the overall meaning). This calls for new participatory evaluation methods that handle continuous speech communities’ feedback~\citep{ramponi-2024-language}}

- Leaderboards have a competitive nature, and a comparative value that makes sense for MT developers. But what about users? And also, 
Is it possible to keep the benefits of (large) standardized benchmark evaluations while involving humans/users?

- Chatbot arena by HF is a positive example to a certain extent of more realistic evaluations \footnote{\url{https://huggingface.co/spaces/lmarena-ai/chatbot-arena-leaderboard}}

\paragraph{What do we maximize in development?}

- fluency? MT with translators-like capabilities? Or efficient human-computer interaction?

- Paradigm shifts: Moving away from perfect imitation of human translation to designing systems for diverse, practical user goals.


- Also, are we accounting for more complex uses of translations within multiple tasks? Is it MT or multilingual transfer?

\section{Interesting possibilities}

\paragraph{MT literacy}
to empower users: understand limits and \hl{possibilities} of MT. This might also increase healthy trust and distrust when necessary.


\paragraph{Interfaces:} to \hl{integrate functionalities} that allow for additional sources, and double checks, and showing model uncertainty to put users in the condition to better verify the MT output, and/or specify what they want. \ar{Having free text explanations for culturally-sensitive translations could also help in this regard.}

\paragraph{Beyond human-like capabilities}

We are very focused on having MT that does what a translator does, and maximixe human-like translation. But depending on the \hl{context of usage}, this might not be the most desirable quality to pursue. e.g. in educational contexts, I do not want MT to be robust to typos, I need to recognize them and see them to improve my language skills. Sometimes it's better to reduce fluency to ensure adequacy. 


\paragraph{Informative estimates}
study measurements that are informative for users, also depending on the \hl{context}. When releasing new models and measuring advancements, report on measures that are meaningful to non-experts--e.g. ``threshold of usability''.\footnote{e.g. when I buy a car, I have info that makes sense to me as a customer: levels of safety, gasoline consumption, years of duration etc etc---indeed this are not the metrics used by engineers to build a card, they are converted into something that makes sense for someone buying a car.} 

\paragraph{Interactivity and Adaptability}
Instruction-following models can support this. 
What roles or \hl{personas} would users want a translation agent to play, e.g. interpreter, educator, confidence checker?

\paragraph{\hl{Methodologies}}

To engage users, there is much we can learn from other disciplines. In general, translation studies literature has engaged more with reception studies and users. But there are also other disciplines: 

\begin{itemize}
    
\item HCI: to set up since the beginning user considerations to develop technology, and for interfaces. Notion of ``co-creation'' \citep{heuer-buschek-2021-methods}.

\item Participatory methods: but we need to be careful not to have an exploitative approach when e.g. we bring marginalized groups to the table. Citizen science also is an interesting approach, where we engage with direct and indirect stakeholders of MT\ar{, however, users should be involved also besides data collection/evaluation to make citized science fully participatory}.

\item - From the social sciences: learn how to reach out and engage with users, e.g. by identifying trustable intermediate figure, and understand limits of people involvement (e.g. one paper was trying to understand uses of migrant communities of MT, but women could not engage \cite{liebling}). Similar, platforms like \textit{prolific} are cool to compile survey, but they are of course skewed and do not have super representative pool of people there \citep{orolific}.  We can also learn how to 
%
%
%
\mn{[...shape this paper in a way that offers a ``personal'' perspective, thus 1) clearly going beyond previous works on HCMT, 2) offering an up-to-date discussion in light of what MT is \textbf{today} (I refer to the possible idea of linking the discussion to the ``myriad uses'', which might be promising). Side note: the role/impact of LLMs is somewhat marginal from Sec.3 on; could this instead be another possible line of reflection?]}

% \item \hl{the data sampline collection}: 
\end{itemize}

\section{Conclusion}

\section*{Limitations}
We are only discussing text-to-text MT. 

Some of the approaches we are discussing require slow-science.

\section*{Ethics Statement}



\bibliography{anthology,custom}
\bibliographystyle{acl_natbib}

\appendix

\section{Example Appendix}
\label{sec:appendix}

This is a section in the appendix.

\end{document}
