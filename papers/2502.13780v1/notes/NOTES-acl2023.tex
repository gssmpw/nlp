% This must be in the first 5 lines to tell arXiv to use pdfLaTeX, which is strongly recommended.
\pdfoutput=1
% In particular, the hyperref package requires pdfLaTeX in order to break URLs across lines.

\documentclass[11pt]{article}

% Remove the "review" option to generate the final version.
\usepackage[review]{ACL2023}

% Standard package includes
\usepackage{times}
\usepackage{latexsym}

\usepackage{soul}
% For proper rendering and hyphenation of words containing Latin characters (including in bib files)
\usepackage[T1]{fontenc}
% For Vietnamese characters
% \usepackage[T5]{fontenc}
% See https://www.latex-project.org/help/documentation/encguide.pdf for other character sets

% This assumes your files are encoded as UTF8
\usepackage[utf8]{inputenc}

% This is not strictly necessary, and may be commented out.
% However, it will improve the layout of the manuscript,
% and will typically save some space.
\usepackage{microtype}

% This is also not strictly necessary, and may be commented out.
% However, it will improve the aesthetics of text in
% the typewriter font.
\usepackage{inconsolata}
\usepackage{xcolor}
\newcommand{\bs}[1]{\textcolor{teal}{#1}}

% If the title and author information does not fit in the area allocated, uncomment the following
%
%\setlength\titlebox{<dim>}
%
% and set <dim> to something 5cm or larger.

\title{A draft Position paper on Human-centered MT or cross-lingual communication}

% Author information can be set in various styles:
% For several authors from the same institution:
% \author{Author 1 \and ... \and Author n \\
%         Address line \\ ... \\ Address line}
% if the names do not fit well on one line use
%         Author 1 \\ {\bf Author 2} \\ ... \\ {\bf Author n} \\
% For authors from different institutions:
% \author{Author 1 \\ Address line \\  ... \\ Address line
%         \And  ... \And
%         Author n \\ Address line \\ ... \\ Address line}
% To start a seperate ``row'' of authors use \AND, as in
% \author{Author 1 \\ Address line \\  ... \\ Address line
%         \AND
%         Author 2 \\ Address line \\ ... \\ Address line \And
%         Author 3 \\ Address line \\ ... \\ Address line}

\author{First Author \\
  Affiliation / Address line 1 \\
  Affiliation / Address line 2 \\
  Affiliation / Address line 3 \\
  \texttt{email@domain} \\\And
  Second Author \\
  Affiliation / Address line 1 \\
  Affiliation / Address line 2 \\
  Affiliation / Address line 3 \\
  \texttt{email@domain} \\}

\begin{document}
\maketitle
\begin{abstract}
In this paper, we take stock of where we stand wrt to human-centered MT. Ideally, we highlight some challenges and opportunities and a path. The main challenge I see with this work--- besides bringing relevant arguments---is finding a way to make it specifically applicable to the MT application and not AI/NLP more generally. 
\hl{What is specific to MT that cannot be more broadly said of NLP?}
\end{abstract}

\section{Introduction}

\textbf{--}\textit{Technology is made by people for people. }
\bigskip


\noindent\textbf{--}\textit{The L in NLP means language, and language means people.} \bs{add cit.}

\paragraph{Motivation} 
NLP technologies have evolved into vastly user-facing applications that are used by millions of people in so many contexts---some of which are also inherently risky, e.g. the medical domain. \bs{[others?]}
Of course, this requires a shift: we are no longer within an abstract academic discipline but rather contributing to the shaping of deployed technologies that need to serve real users in real-world scenarios.  And yet, we still do not really know that much about how users interact with the technology, for what needs, in which contexts, and how we can support them properly. \
The development chain of models is often self-reflecting: \hl{the relevance of the proposed technologies is usually assessed wrt to values that are internal to the field}, e.g. overall performance, scalability, new domain, efficiency \citep{birhane-2022}.\footnote{Or this is at least what \citet{birhane-2022} finds in the broader Machine Learning (ML) field.} \textit{But are we empowering users and their needs? Are we putting them in the conditions to make sensible choices about how and when to use the technologies?}

\paragraph{Meta-reflection on the NLP field:}
Overall, in the field, we are attesting a growing interest towards directions that better account for users, and human-centered approaches. A case in point regards: 

\noindent--- the \textbf{Human-centered NLP special theme}  at NAACL 2022\footnote{\url{https://2022.naacl.org/blog/special-theme/}}
    
     \noindent---the new \textbf{human-centered track} at *CL conferences, introduced with EMNLP 2023\footnote{\url{https://2023.emnlp.org/calls/main_conference_papers/}}
    
     \noindent---\textbf{HCI+NLP workshops}, in 2024 reaching its third edition \footnote{\url{https://sites.google.com/view/hciandnlp/home}}
   
    \noindent---\textbf{Tutorial on Human-centered evaluation of Language Technologies} at EMNLP 2024:\footnote{\url{https://2024.emnlp.org/program/tutorials/}} 
  
    \noindent--- ongoing workshop proposal on \textbf{Participatory approaches to NLP}---this still signals desire to include users/people in technology design
    
    Note that there also was at ACL 2024 the \textit{Human-centered LLM workshop}\footnote{https://aclanthology.org/2024.hucllm-1.0.pdf}, but here honestly human-centered is sort of used vaguely...
    Also, it is good to keep in mind that we had to underscore in this paper \citet{artificial} the need for works to state that using LLms as subjects, it not the same as actually including subjects. 




\bs{TODO: maybe prepare a query on the ACL anthology with just human-centered to see mentions}. Use as keywords human-centric, centred, centered.  

\paragraph{Why Automatic Translation?}


\section{Vague background}

There is a lack of work that foregrounds human engagement 
\citep{cercas-curry-etal-2020-conversational, mengesha2021don, wang2024measuring}. 


\section{What are human-centered approaches?}
According to the NAACL special theme: 


\begin{itemize}
    \item is human-centered an \hl{empty signifier}?
    \item what does the legislation says about human-centered? e.g. \hl{AI act}
\end{itemize}

\section{Examples of risky situations where people do not know how to assess MT quality}

\section{List of issues/desiderata for human-centered MT}

According to NAACL special theme

\textit{As NLP applications increasingly mediate people’s lives, it is crucial to understand how the design decisions made throughout the NLP research and development lifecycle impact people, whether they are users, developers, data providers or other stakeholders. For NAACL 2022, we invite submissions that address research questions that meaningfully incorporate stakeholders in the design, development, and evaluation of NLP resources, models and systems. We particularly encourage submissions that bring together perspectives and methods from NLP and Human-Computer Interaction. In addition to papers presenting research studies, we invite survey and position papers that take stock of past work in human-centered NLP and propose directions for framing future research.}

\textit{Topics of interest include (but are not limited to): usability studies of language technologies; needs-findings studies; studies of human factors in the NLP R\&D lifecycle, including interactive systems; human-centered fairness, accountability, explainability, transparency, and ethics in NLP systems; or human-centered evaluations of NLP technologies.}

\textit{Relevant methods include (but are not limited to) user-centered design, \hl{value-sensitive design},\footnote{\bs{check some FBK work on this}.} participatory design, assets-based design, and qualitative methods, such as grounded theory. We welcome contributions that use such methods to study NLP problems, as well as methodological innovations and tools that tailor these methods to NLP.}


\section{}

\subsection{Footnotes}

Footnotes are inserted with the \verb|\footnote| command.\footnote{This is a footnote.}

\subsection{Tables and figures}


\subsection{Hyperlinks}

Users of older versions of \LaTeX{} may encounter the following error during compilation: 
\begin{quote}
\tt\verb|\pdfendlink| ended up in different nesting level than \verb|\pdfstartlink|.
\end{quote}
This happens when pdf\LaTeX{} is used and a citation splits across a page boundary. The best way to fix this is to upgrade \LaTeX{} to 2018-12-01 or later.

\subsection{Citations}






The \LaTeX{} and Bib\TeX{} style files provided roughly follow the American Psychological Association format.
If your own bib file is named \texttt{custom.bib}, then placing the following before any appendices in your \LaTeX{} file will generate the references section for you:
\begin{quote}
\begin{verbatim}
\bibliographystyle{acl_natbib}
\bibliography{custom}
\end{verbatim}
\end{quote}
You can obtain the complete ACL Anthology as a Bib\TeX{} file from \url{https://aclweb.org/anthology/anthology.bib.gz}.
To include both the Anthology and your own .bib file, use the following instead of the above.
\begin{quote}
\begin{verbatim}
\bibliographystyle{acl_natbib}
\bibliography{anthology,custom}
\end{verbatim}
\end{quote}
Please see Section~\ref{sec:bibtex} for information on preparing Bib\TeX{} files.

\subsection{Appendices}

Use \verb|\appendix| before any appendix section to switch the section numbering over to letters. See Appendix~\ref{sec:appendix} for an example.

\section{Bib\TeX{} Files}
\label{sec:bibtex}



Please ensure that Bib\TeX{} records contain DOIs or URLs when possible, and for all the ACL materials that you reference.
Use the \verb|doi| field for DOIs and the \verb|url| field for URLs.
If a Bib\TeX{} entry has a URL or DOI field, the paper title in the references section will appear as a hyperlink to the paper, using the hyperref \LaTeX{} package.

\section*{Limitations}
ACL 2023 requires all submissions to have a section titled ``Limitations'', for discussing the limitations of the paper as a complement to the discussion of strengths in the main text. This section should occur after the conclusion, but before the references. It will not count towards the page limit.
The discussion of limitations is mandatory. Papers without a limitation section will be desk-rejected without review.

While we are open to different types of limitations, just mentioning that a set of results have been shown for English only probably does not reflect what we expect. 
Mentioning that the method works mostly for languages with limited morphology, like English, is a much better alternative.
In addition, limitations such as low scalability to long text, the requirement of large GPU resources, or other things that inspire crucial further investigation are welcome.

\section*{Ethics Statement}
Scientific work published at ACL 2023 must comply with the ACL Ethics Policy.\footnote{\url{https://www.aclweb.org/portal/content/acl-code-ethics}} We encourage all authors to include an explicit ethics statement on the broader impact of the work, or other ethical considerations after the conclusion but before the references. The ethics statement will not count toward the page limit (8 pages for long, 4 pages for short papers).

\section*{Acknowledgements}
This document has been adapted by Jordan Boyd-Graber, Naoaki Okazaki, Anna Rogers from the style files used for earlier ACL, EMNLP and NAACL proceedings, including those for
EACL 2023 by Isabelle Augenstein and Andreas Vlachos,
EMNLP 2022 by Yue Zhang, Ryan Cotterell and Lea Frermann,
ACL 2020 by Steven Bethard, Ryan Cotterell and Rui Yan,
ACL 2019 by Douwe Kiela and Ivan Vuli\'{c},
NAACL 2019 by Stephanie Lukin and Alla Roskovskaya, 
ACL 2018 by Shay Cohen, Kevin Gimpel, and Wei Lu, 
NAACL 2018 by Margaret Mitchell and Stephanie Lukin,
Bib\TeX{} suggestions for (NA)ACL 2017/2018 from Jason Eisner,
ACL 2017 by Dan Gildea and Min-Yen Kan, NAACL 2017 by Margaret Mitchell, 
ACL 2012 by Maggie Li and Michael White, 
ACL 2010 by Jing-Shin Chang and Philipp Koehn, 
ACL 2008 by Johanna D. Moore, Simone Teufel, James Allan, and Sadaoki Furui, 
ACL 2005 by Hwee Tou Ng and Kemal Oflazer, 
ACL 2002 by Eugene Charniak and Dekang Lin, 
and earlier ACL and EACL formats written by several people, including
John Chen, Henry S. Thompson and Donald Walker.
Additional elements were taken from the formatting instructions of the \emph{International Joint Conference on Artificial Intelligence} and the \emph{Conference on Computer Vision and Pattern Recognition}.

% Entries for the entire Anthology, followed by custom entries
\bibliography{anthology,custom}
\bibliographystyle{acl_natbib}

\appendix

\section{Example Appendix}
\label{sec:appendix}

This is a section in the appendix.

\end{document}
