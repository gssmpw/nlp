\paragraph{\faCheck \space Trust} %\bes{to shorten}
For trust to be established, lay users 
%need to acquire the knowledge of the functional capacity and origin of a system~\citep{davis-1979-applications} (i.e. intrinsic factors of \emph{system trustworthiness} such as data proveneance and model behavior)
should be provided with 
%sufficient 
information and tools to understand how a system impacts them when interacting with it 
%in a collaborative environment (i.e. \emph{user trust})~
\citep{litschko-etal-2023-establishing}. 
%Moreover, providing d
Details on the functional capacity \mnc{``Details'' sembra intrecciato a literacy. Sono functionalities che aumenterebbero la trust o dettagli su tali functionalities?} and origin of a system~\citep{davis-1979-applications} (i.e. intrinsic factors of \emph{system trustworthiness} such as 
%data provenance and 
model behavior) also help to strengthen user trust.\mnc{La parte iniziale del paragrafo si puo' accorciare.}
In user-facing technologies such as MT, it is paramount to understand how lay users %react and 
modulate their trust with respect to flawed translations.
%, and how to adjust translation outputs accordingly. 
In this regard, 
%the preliminary study by 
\citet{martindale-carpuat-2018-fluency} 
show that users give more weight to \emph{fluency} rather than \emph{adequacy} errors. This has serious consequences, since trust might be displaced when an output is believable but inaccurate, especially in high-stakes contexts 
%such as medical settings~
\citep{mehandru-etal-2023-physician}. Although chat-based LLMs open opportunities for adjusting outputs based on user requirements 
%(e.g. instructions for prioritizing \emph{adequacy} over \emph{fluency}), 
(i.e. via instructions), 
LLMs' efficacy, reliability, and coverage of users' feedback is currently unexplored.\mnc{Fin qui si e' discusso il primo aspetto indicato all'inizio del paragrafo (l'interazione). Il secondo  (functional capacity) sembra oggetto del prossimo passaggio si backtranslation ma non sono sicuro 1) che sia cosi', 2) che si colga.}

\lb{To harness the benefits of MT systems while avoiding over-reliance on flawed translations}, lay users often adopt %backward translation
back-translation\footnote{i.e. automatically translating a text to a \emph{target} language and then back to the \emph{source} language.} as a strategy to improve 
%their 
confidence~\citep{zouhar-etal-2021-backtranslation,mehandru-etal-2023-physician}. However, 
%back-translation has shown to decrease objective translation quality~\citep{zouhar-etal-2021-backtranslation},\bes{1) eviterei il termine QUALITY generica (vedi paragrafo precedente), 2) mi peace molto il focus che differenzia "correttezza linguistics" da "trust/confidence" come due elemente in relazione ma diverse. Mi sto chiendedo come mettere piu' a fuoco questo punto} therefore presenting a trade-off between trust and accuracy. Moreover, 
back-translation is performed manually 
%by lay users 
due to the lack of dedicated functionalities. 
%in current MT interfaces. 
The need for an \emph{appropriate level} of trust in MT outputs is becoming central~\citep{martindale-carpuat-2018-fluency,deng2022generalpurposemachinetranslation} \mnc{Questa need arriva quando uno dovrebbe gia' aver capito che trust e' importante. Quindi il mio dubbio e' che voglia alludere a qualcosa che non colgo, probabilmente legato ad ``appropriate'' in italic...Cosa si intende?} and \emph{transparency} is a fundamental concept to help lay users gain it, e.g. \lb{by communicating} uncertainty and explanations~\citep{liao2023aitransparencyagellms}, making sure that the latter are %actually 
digestible for lay users and not just for developers. What is currently missing is how to define and quantify an appropriate level of trust and how to handle it concretely in the development of MT systems and the user interfaces in which they are embedded.