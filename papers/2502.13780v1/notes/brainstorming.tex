
\textit{Section statement of intent}: Brief overview of converging technical and societal converging factors that have brought the MT task to the general public, and has made lay users---as opposed to professional users---a big (if not the biggest) base of people that consume MT. This also implies the consumption of raw MT output by lay users. The section concludes with reflections on the potential shift of the MT task to LLMs and speculations on how that might affect the MT task and lay user interactions. Overall, the section should also serve to let the aspects of literacy, trust and usability emerge as an anticipation for the next section.   

\begin{itemize}

\item \textbf{Early days and MT for assimilation}

\begin{itemize}

\item MT as a task arose in the 50s from very concrete human needs\footnote{Though for espionage...}. 
\item Historically, there was a more clear-cut distinction of MT uses: for \textit{i)} dissemination and \textit{ii)} assimilation, also called gisting. The former required professional revision, the latter was just raw MT output mostly for perishable content, not highly relevant.
\end{itemize}
\begin{itemize}
    \begin{itemize}
        \item Implicitely, this distinction also corresponds to experts interacting with an MT output to post-edit, vs non-experts receiving a raw output
        \item the distinction was more clear cut and needed also because the quality of MT was lower. METEO---and MT for weather forecasts in Canada---was one of the few succesfull applications of raw MT because bulletin had a super simple lexicon and structure, with low risks for errors
    \end{itemize}
\end{itemize}
\item For assimilation, MT first mostly deployed in the language industry.  
\item MT did not have the quality to be often used as is, and rather the focus was on integrating it into the flow of translators. At this time, CAT tool are born, from which stems \textit{interactive} MT. There are human-centered designs and studies---many from the translation studies field but not only--- though often aimed at studying productivity in the language industry. OR, studies on the trust of the translator toward MT.  Creation of tools for ``assiting'' translation

\end{itemize}

    \item \textbf{MT for the general public and gisting}

\begin{itemize}

\item Democratization of translation has facilitated unprecedented levels of global communication and reach. Its influence extends into most domains of human activity. The implications are vast and with social relevance. 
\begin{itemize}
    \item The web has arrived
    \item Digital content has arrived
    \item Stronger systems with  higher quality encourage wider adoption
    \item Emergence of online MT user-facing applications 
    \item MT also embebbed into online platforms
    \item MT "lay" stakeholders become a big user base---n.b. following VSD disctiontion between direct and indirect stakeholders, we focus on direct ones
    \begin{itemize}
        \item e.g. Different type of stakeholders: the one \textit{using} MT vs the one \textit{receiving} an MT output. e.g. one party with authority (e.g. police officer, doctor) using MT on the other party.
    \end{itemize}
\end{itemize}

\item In highly interconnected multilingual society, uses of MT have expanded and are not always known.\footnote{Cool MT stories project seek to collect stories of people using MT: \url{https://mt-stories.com/}}
\item User base is also diverse---depending on background, language knowledge, and life situation

\item Time for studies---mostly for translation studies---on everyday uses and users of MT, for learning, gisting content online, chats, medicine, migration, emergency situation etc. 
\item Issues and considerations:
\begin{itemize}
    \item lay users have different needs, comptences and can be vulnerable
    \item do not necessary recognize errors, how to resume from them, do not known when MT can be trusted --> now with LMs and most recent studies lay users---considered as crodworkers---cannot distinguish MT vs human output
    \item maybe do not have the knowledge to know which langs are best supported
    \item interfaces online are simple, and yet not very informative for users
    \item mobile apps
    \item this can bring \end{itemize}

\end{itemize}


\item Language Models

\bs{\texttt{\textbf{Visualization-idea}}: some (real) images of various MT errors, in different settings, with different levels of risks to guide discussion. e.g. see image \ref{fig:error} }

