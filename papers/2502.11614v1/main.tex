% This must be in the first 5 lines to tell arXiv to use pdfLaTeX, which is strongly recommended.
\pdfoutput=1
% In particular, the hyperref package requires pdfLaTeX in order to break URLs across lines.

\documentclass[11pt]{article}

% Remove the "review" option to generate the final version.
% \usepackage{EMNLP2023}
% \usepackage[review]{acl}
\usepackage{acl}

% Standard package includes
\usepackage{times}
\usepackage{latexsym}
\usepackage{adjustbox}
\usepackage[T1]{fontenc}
% For proper rendering and hyphenation of words containing Latin characters (including in bib files)
\usepackage[T1,T2A]{fontenc}
% For Vietnamese characters
% \usepackage[T5]{fontenc}
% See https://www.latex-project.org/help/documentation/encguide.pdf for other character sets

% This assumes your files are encoded as UTF8
\usepackage[utf8]{inputenc}

% This is not strictly necessary and may be commented out.
% However, it will improve the layout of the manuscript,
% and will typically save some space.
\usepackage{microtype}

% This is also not strictly necessary and may be commented out.
% However, it will improve the aesthetics of text in
% the typewriter font.

\usepackage{lipsum} % Optional, for filler text
\usepackage{arabtex}
%\usepackage[T2A,T1]{fontenc}


\usepackage[english, vietnamese, russian]{babel}
\usepackage{utf8}
\setcode{utf8}
\usepackage{inconsolata}
\usepackage{microtype}
\usepackage{graphicx}
\usepackage{import}
\usepackage{layout}
\usepackage{tabularx, makecell}
\usepackage{booktabs}
\usepackage{mathrsfs}
\usepackage{amssymb} 
\usepackage{url}
\usepackage{hyperref}
\usepackage{graphicx}
\usepackage{xspace,paralist}
\usepackage{times,latexsym}
\usepackage{amsmath}
\usepackage{appendix}
\usepackage{comment} 
\usepackage{enumitem}
\usepackage{makecell}
\usepackage{multirow}
\usepackage{xcolor}
\usepackage{arydshln}
\usepackage{cleveref}
\usepackage{todonotes}
\usepackage{longtable,supertabular}
\usepackage[russian,english]{babel}
\usepackage[T2A,T1]{fontenc}

\usepackage{amssymb}% http://ctan.org/pkg/amssymb
\usepackage{pifont}% http://ctan.org/pkg/pifont
\usepackage{CJKutf8}
\newcommand{\cn}[1]{\begin{CJK*}{UTF8}{gbsn}#1\end{CJK*}}
\newcommand{\cmark}{\ding{51}}%
\newcommand{\xmark}{\ding{55}}%

\newcommand{\eqnref}[1]{Eq~\eqref{#1}\xspace}
\newcommand{\tabref}[2][]{Table#1~\ref{#2}\xspace}
\newcommand{\figref}[1]{Figure~\ref{#1}\xspace}
\newcommand{\secref}[1]{Section~\ref{#1}\xspace}
\newcommand{\appref}[1]{Appendix~\ref{#1}\xspace}

\newcommand{\model}[1]{\text{#1}\xspace}
\newcommand{\factscore}{\model{FactScore}}
\newcommand{\factool}{\model{FacTool}}
\newcommand{\factor}{\model{FACTOR}}
\newcommand{\rarr}{\model{RARR}}
\newcommand{\cove}{\model{CoVe}}
\newcommand{\perplexityai}{\model{Perplexity.ai}}

\newcommand{\claude}{\model{Claude}}
\newcommand{\chatglmfour}{\model{ChatGLM-4-9B-chat}}
\newcommand{\chatgpt}{\model{ChatGPT}}
\newcommand{\gptfour}{\model{GPT-4}}
\newcommand{\gptfouro}{\model{GPT-4o}}
\newcommand{\qwenturbo}{\model{Qwen-turbo}}
\newcommand{\qwentwo}{\model{Qwen2}}
\newcommand{\bard}{\model{Bard}}
\newcommand{\llama}{\model{LLaMA}}
\newcommand{\llamatwo}{\model{LLaMA2}}
\newcommand{\llamathree}{\model{LLaMA3}}
\newcommand{\alpaca}{\model{Alpaca}}
\newcommand{\vicuna}{\model{Vicuna}}
\newcommand{\ecot}{\model{ECoT}}

\usepackage{CJKutf8}
\usepackage[utf8]{inputenc}
\newcommand{\zx}[1]{{\color{CadmiumOrange}{\bf{[ZX:]}} #1}}


\NewDocumentCommand{\revanth}
{ mO{} }{\textcolor{blue}{\textsuperscript{\textit{Revanth}}\textsf{\textbf{\small[#1]}}}}


% If the title and author information does not fit in the area allocated, uncomment the following
%
\setlength\titlebox{3in}
%
% and set <dim> to something 5cm or larger.

% \title{Towards human-like vs. liked-by-human Multilingual LLM: Comprehensive Human Detection and Preference over Multidomain and Multilingual MGT}
% \title{Is Human-Written Text Liked by Humans? \\Multilingual Human Detection and Preference Against AI}
\title{Is Human-Like Text Liked by Humans? \\Multilingual Human Detection and Preference Against AI}
% \title{Human Discernment of Human from AI Texts: Multilingual Case Study}
% \title{Human can Safeguard Against AI: Multilingual Case Study of Human Discernment of Human from AI texts}

% Author information can be set in various styles:
% For several authors from the same institution:
% \author{Author 1 \and ... \and Author n \\
%         Address line \\ ... \\ Address line}
% if the names do not fit well on one line use
%         Author 1 \\ {\bf Author 2} \\ ... \\ {\bf Author n} \\
% For authors from different institutions:
% \author{Author 1 \\ Address line \\  ... \\ Address line
%         \And  ... \And
%         Author n \\ Address line \\ ... \\ Address line}
% To start a separate ``row'' of authors use \AND, as in
% \author{Author 1 \\ Address line \\  ... \\ Address line
%         \AND
%         Author 2 \\ Address line \\ ... \\ Address line \And
%         Author 3 \\ Address line \\ ... \\ Address line}

% \author{First Author \\
%   Affiliation / Address line 1 \\
%   Affiliation / Address line 2 \\
%   Affiliation / Address line 3 \\
%   \texttt{email@domain} \\\And
%   Second Author \\
%   Affiliation / Address line 1 \\
%   Affiliation / Address line 2 \\
%   Affiliation / Address line 3 \\
%   \texttt{email@domain} \\}

\author{Yuxia Wang\textsuperscript{1},
Rui Xing\textsuperscript{1}, 
Jonibek Mansurov\textsuperscript{1},
Giovanni Puccetti\textsuperscript{5}, 
\\\bf
Zhuohan Xie\textsuperscript{1}, 
Minh Ngoc Ta\textsuperscript{9}, 
Jiahui Geng\textsuperscript{1}, 
Jinyan Su\textsuperscript{1, 10}, 
Mervat Abassy\textsuperscript{13},
\\\bf
Saad El Dine Ahmed\textsuperscript{13}, 
Kareem Elozeiri\textsuperscript{11}, 
Nurkhan Laiyk\textsuperscript{1}, 
Maiya Goloburda\textsuperscript{1}, 
\\\bf
Tarek Mahmoud\textsuperscript{1},
Raj Vardhan Tomar\textsuperscript{14},
Alexander Aziz\textsuperscript{15}, 
Ryuto Koike\textsuperscript{7},
\\\bf
Masahiro Kaneko\textsuperscript{1,7}, 
Artem Shelmanov\textsuperscript{1},
Ekaterina Artemova\textsuperscript{6}, 
Vladislav Mikhailov\textsuperscript{4}, 
\\\bf
Akim Tsvigun\textsuperscript{2,3}, 
Alham Fikri Aji\textsuperscript{1}, 
Nizar Habash\textsuperscript{1,8}, 
Iryna Gurevych\textsuperscript{1,12}, 
Preslav Nakov\textsuperscript{1}
\\
\textsuperscript{1}MBZUAI \quad
\textsuperscript{2}Nebius AI \quad 
\textsuperscript{3}KU Leuven \quad
\textsuperscript{4}University of Oslo 
% \textsuperscript{5}Institute of Information Science and Technologies \quad
\textsuperscript{5}ISTI-CNR\quad
\textsuperscript{6}Toloka AI
\\
\textsuperscript{7}Institute of Science Tokyo \quad
\textsuperscript{8}New York University Abu Dhabi
\\
% \textsuperscript{9}BKAI Research Center, Hanoi University of Science and Technology \quad
\textsuperscript{9}BKAI Research Center, Hanoi University of Science and Technology \quad
\textsuperscript{10}Cornell University
\\
\textsuperscript{11}Zewail City of Science and Technology \quad
\textsuperscript{12}TU Darmstadt \quad
\textsuperscript{13}Alexandria University
\\
\textsuperscript{14}Cluster Innovation Center, University of Delhi \quad
\textsuperscript{15}University of Florida
\\
\href{yuxia.wang@mbzuai.ac.ae}{\{yuxia.wang, preslav.nakov\}@mbzuai.ac.ae}
}


\begin{document}
\selectlanguage{english}
\maketitle

\begin{abstract}
Prior studies have shown that distinguishing text generated by large language models (LLMs) from human-written one is highly challenging, and often no better than random guessing. To verify the generalizability of this finding across languages and domains, we perform an extensive case study to identify the upper bound of human detection accuracy. Across 16 datasets covering 9 languages and 9 domains, 19 annotators achieved an average detection accuracy of 87.6\%, thus challenging previous conclusions. We find that major gaps between human and machine text lie in concreteness, cultural nuances, and diversity. Prompting by explicitly explaining the distinctions in the prompts can partially bridge the gaps in over 50\% of the cases. However, we also find that humans do not always prefer human-written text, particularly when they cannot clearly identify its source. 
% We release our dataset, the human labels, and the annotator metadata at \texttt{http://URL.withheld.for.review}.

\end{list}
\end{abstract}


\section{Introduction}
\label{sec:intro}
% Image editing methods in diffusion models depend on user-defined control directions - users can unlock their creativity using these methods by specifying the desired manipulation through prompts~\cite{gandikota2023concept}, reference images~\cite{ruiz2022dreambooth, kumari2022customdiffusion, gal2022image, chen2024trainingfreeregionalpromptingdiffusion}, or attribute vectors~\cite{parmar2023zero,hertz2022prompt}. In this work, we ask a fundamentally different question: \emph{Can we automatically discover the underlying visual structure of a concept within diffusion model's knowledge?} %Rather than requiring user-specified controls, we aim to decompose the model's internal knowledge into meaningful directions.

% This question touches on a fundamental limitation in how we interact with diffusion models. Current control methods ~\cite{zhang2023addingconditionalcontroltexttoimage, gandikota2023concept, ye2023ipadaptertextcompatibleimage,ye2023ipadaptertextcompatibleimage, hertz2024stylealignedimagegeneration, li2023photomaker, shi2024instantbooth, chen2024trainingfreeregionalpromptingdiffusion} require users to specify their desired manipulations in advance, limiting interactive creativity. This contrasts with natural human artistic workflows, where creators dynamically explore creative ideas while jointly refining them toward meaningful artistic outcomes~\cite{hoffmann2016modeling}. This synergy between specification and exploration is not new to generative models. Early GAN architectures naturally developed disentangled latent spaces that enabled continuous\cite{harkonen2020ganspace,radford2015unsupervised, wu2021stylespace, shen2020interfacegan}, compositional control over generated images. Users could explore these spaces to discover interesting variations that would be difficult to describe in words~\cite{wu2021stylespace}, then combine them to achieve their creative goals~\cite{grabe2022towards}. 


% While diffusion models have largely superseded GANs in conditional image synthesis~\cite{dhariwal2021diffusion},  their underlying structure remains less understood. Diffusion models achieve remarkable diversity through high-dimensional latents, unlike GANs' compact latent spaces.  With a single prompt, diffusion models can generate radically different variations through different random initializations of input noise. We ask - Is it possible to discover interpretable structure within this vast space of variations?

Text-to-image diffusion models are capable of generating remarkable visual variations from a single prompt through different random initializations. However, this vast creative potential remains largely opaque to users---while we can generate diverse images, we lack understanding of the underlying structure of these variations. This presents a fundamental challenge: how can we discover and expose the latent visual capabilities encoded within these models?

\let\thefootnote\relax \footnote{$^{*}$Correspondence to \texttt{gandikota.ro@northeastern.edu}}

The challenge touches on a key limitation in how we interact with diffusion models today. Current control methods require users to explicitly specify their desired edits in advance through prompts~\cite{gandikota2023concept}, reference images~\cite{zhang2023addingconditionalcontroltexttoimage, chen2024trainingfreeregionalpromptingdiffusion, ruiz2022dreambooth,kumari2022customdiffusion, Ryu_lora, hu2021lora}, or attribute vectors~\cite{ye2023ipadaptertextcompatibleimage, hertz2024stylealignedimagegeneration, li2023photomaker, shi2024instantbooth,parmar2023zero,hertz2022prompt}. That contrasts sharply with natural human creative workflows, where artists dynamically explore creative ideas and jointly refine them toward meaningful artistic outcomes~\cite{hoffmann2016modeling}. The need for pre-specified controls creates a barrier between users and the full creative potential of these models.

Interestingly, earlier generative models like GANs~\cite{gans,karras2019style,brock2018large} naturally developed more interpretable internal structures. Their compact latent spaces often exhibited emergent disentanglement~\cite{harkonen2020ganspace,radford2015unsupervised, wu2021stylespace, shen2020interfacegan}, enabling continuous and compositional control over generated images. Users could explore these spaces to discover interesting variations that would be difficult to describe in words~\cite{wu2021stylespace}, then combine them to achieve their creative goals~\cite{grabe2022towards}.

Diffusion models have largely superseded GANs in conditional image synthesis~\cite{dhariwal2021diffusion}, achieving greater diversity through much higher-dimensional latents. And yet an understanding of the underlying structure of these larger latent spaces has remained elusive. In this work, we ask a fundamental question: \emph{Can we automatically discover the visual structure within a diffusion model's knowledge of a concept?} Rather than requiring user-specified controls, we aim to decompose the model's internal representations into expressive directions that users can explore and combine.

To address these needs, we present \textbf{SliderSpace}, a framework that brings systematic explorability to diffusion models. Given just a text prompt, SliderSpace discovers a canonical set of meaningful, diverse, and controllable directions within the model's knowledge of that concept. Each direction is implemented as a low-rank adapter~\cite{hu2021lora} that can be scaled and composed with others, allowing users to explore and smoothly combine different aspects of variation, as shown in Figure~\ref{fig:intro}.

We ground SliderSpace discovery in three key requirements for meaningful decomposition of a diffusion model's visual manifold: 
\begin{enumerate}
    \item \textbf{Unsupervised Discovery:} The decomposition process should emerge from the intrinsic structure of the model's learned representation, rather than being guided by predefined attributes. This ensures we capture the true topology of the model's knowledge space rather than projecting our assumptions onto it.
    
    \item \textbf{Semantic Orthogonality:} Each discovered control must represent a distinct semantic direction. This is enforced in a semantic feature space, like CLIP, where every slider has an orthogonal effect in embeddings. This prevents discovering multiple controls that create similar semantic effects, making the system more efficient and easier.
    
    \item \textbf{Distribution Consistency:} Directions must induce consistent transformations across both random seeds and prompt variations. 
\end{enumerate}

These requirements naturally lead to our proposed framework, which we formalize in Section~\ref{sec:method}. As we show in our experiments, SliderSpace is architecture-agnostic, working with both conventional U-Net based models like Stable Diffusion~\cite{rombach2022high, rombach2022sd20, podell2023sdxl, turbo, dmd} and recent transformer-based architectures like Flux~\cite{flux}.

We demonstrate the expressiveness of SliderSpace through three applications: First, we show how SliderSpace can decompose high-level concepts into diverse and expressive components, revealing the natural axes of variation in the model's understanding. Second, we explore artistic style variation, where SliderSpace discovers directions that match or exceed the diversity of manually curated artist lists while being judged more useful by human evaluators. Finally, we show how SliderSpace can help reverse the mode collapse commonly observed in distilled diffusion models, restoring diversity while maintaining generation speed.

Beyond providing practical creative control, SliderSpace opens new avenues for understanding and utilizing the latent capabilities of diffusion models. By mapping these models' visual potential into intuitive, composable directions, we take a step toward making their creative possibilities more accessible and interpretable to users.

% Image editing methods in diffusion models unlock the creativity of users. In this work we ask an alternate question: \emph{Can we organize and expose what of the diffusion model is already capable of?}.
% Existing methods for controlling image generation typically require users to manually specify edit directions for desired changes. This process is time-consuming, requires technical expertise, and limits the spontaneity of the creative process. For instance, if a user wants to adjust the smile of a generated person, they must explicitly request this edit, often through imprecise prompt engineering or model fine-tuning. This approach of predefined controls or manual specifications restricts users from fully exploring the latent capabilities of the model. There may be interesting stylistic variations or attributes that the model can generate, but users have no easy way to discover or utilize these.

% Natural visual disentanglement was an emergent property in the latent space of Generative Adversarial Models (GANs) \cite{harkonen2020ganspace,radford2015unsupervised, wu2021stylespace, shen2020interfacegan}. In particular, it has been observed that StyleGAN~\cite{karras2019style} stylespace neurons offer detailed control over many meaningful aspects of images that would be difficult to describe in words~\cite{wu2021stylespace}. However, diffusion models do not share such a compact latent space~\cite{park2023unsupervised}; and efforts to uncover such a space in the semantic embeddings of the text conditioning have met with limited success \nik{Nick - is there a specific citation you were thinking about?}.

% In this work we introduce \textbf{SliderSpace}, which takes a step towards uncovering an analogous low dimensional representation of diffusion models' visual breadth; in essence treating the diffusion model as many generators sharing parameters, where a particular generator is defined by a specific prompt. For a given prompt we sample many random seeds (and optionally prompt expansions using an LLM), generate the corresponding images, and apply an off the shelf feature extractor (in this work CLIP, but our method can be applied to any differentiable feature extractor). We use PCA to analyze these features, and for each of the leading $k$ principal components we train a LoRA \cite{} which causes the diffusion model to produces images which increase the feature magnitude along that component when passed back through the same feature extractor. This leads to a 'Slider' for each principal component, because each LoRA can be scaled and applied to the original diffusion model, continuously varying those visual features in the generated results (as measured, in our case, by CLIP).

% There are many other works that enhance the controllability of diffusion models. One common approach is enabling users to add spatial constraints to a generation either manually, or via a reference image \cite{zhang2023addingconditionalcontroltexttoimage, chen2024trainingfreeregionalpromptingdiffusion}, a second is leveraging more abstract embeddings (e.g. identity, style) extracted from a reference image \cite{ye2023ipadaptertextcompatibleimage, hertz2024stylealignedimagegeneration, li2023photomaker, shi2024instantbooth}, a third is finetuning a foundation model to better generate a concept important to the user \cite{ruiz2022dreambooth, kumari2022customdiffusion, Ryu_lora, hu2021lora}, and a fourth (most relevant to this work) is finding low-rank adaptors of the model based on a prompt or small training set which can be scaled to provide continous control over one aspect of generated image (e.g. night vs day, basic vs luxury, etc.) \cite{gandikota2023concept}. SliderSpace is complementary to all of these methods and offers something distinct. All of the other methods we are aware require the user (and / or model designer) to know in advance what type of control they want. In contrast SliderSpace assists users in discovering and controlling hidden capabilities present in the diffusion model's distribution of possible generations.

%We propose that truly intuitive creative control in a text-to-image model should meet three key criteria: \emph{discoverability}, \emph{intuitiveness}, and \emph{specificity}. The model should reveal controllable attributes that may not be immediately obvious, offer controls that are easy to understand and manipulate, and ensure each control affects a distinct attribute of the generated image.

% We demonstrate the utility and power of SliderSpace using three applications built on top of SDXL-DMD \cite{dmd}, because its fast generation speed lends itself well to the continuous control offered by SliderSpace.

% First, we study concept decomposition (Section \ref{sec:concept_exp}), where we learn sliders for a specific concept (e.g. 'monster', 'waterfall', 'car'). Through quantitative metrics of diversity and text alignment we demonstrate that the learned sliders dramatically boost the diversity of generations when randomly applied without harming text alignment; we also ask humans to qualitatively judge these results in a user study where they find the SliderSpace results to be more 'Diverse', 'Useful', and 'Creative' than our baselines.

% Second, we attempt to compare the automatic discoveries of SliderSpace to a large scale manual study of artistic styles (Section \ref{sec:art_exp}), open-sourced by ParrotZone \cite{parrotzone}. In this study SDXL was prompted with over 4300 artist names,  and based on visual inspection the cases of successful stylistic mimicry recorded. Quantitatively SliderSpace more closely matches the distribution of artistic variation discovered by ParrotZone than other baselines, and in our user studies was judged to be significantly more 'Diverse' and 'Useful' than the baselines. To our surprise humans even judged SliderSpace results to be slightly more 'Diverse' than the results generated by the manually discovered artist names of \cite{parrotzone}.

% Third, we attempt to use SliderSpace to reverse the mode collapse commonly observed in distilled few-step diffusion models relative to the original teacher model (Section \ref{sec:diverse_exp}). We quantitatively demonstrate that applying SliderSpace to SDXL-DMD leads to more closely matching the distribution of images by the original teacher, SDXL.

%Through extensive experiments on various state-of-the-art text-to-image models, we demonstrate that SliderSpace significantly enhances user control and creative expression in AI-assisted image generation tasks. Our method enables a range of applications, including concept decomposition and control, diversity improvement in generated images, customization dissection and edits, and the exploration of artistic styles inherent in the model.

% SliderSpace goes beyond providing a practical tool for enhanced creative control. By mapping the visual potential of diffusion models it can open new avenues for generative creativity and deepens our understanding of each model's hidden potential.
% \section{Related Work}
\label{sec:related work}
% In this section, we review the existing literature on point cloud denoising and unsupervised image denoising.
%-------------------------------------------------------------------------
\subsection{Point cloud denoising}

    \subsubsection{Traditional methods}
Traditional approaches to point cloud denoising include statistical methods \cite{computingpointset2003,definingpointset2004,wlop2009HH}, filtering techniques\cite{pointsetsurfaces2001,Robustmoving2005, zaman2017density}, and optimization-based methods \cite{l1sparse2010,clop2014PR,digne2017bilateral,multi-projection2018duan,hu2020featuregraph} . These techniques often rely on handcrafted features and heuristics to distinguish signal from noise. For example, statistical methods may use distribution models to identify and remove outliers. Filtering methods, such as mean or median filters, operate under the assumption that noise is statistically different from the signal. Optimization-based methods formulate denoising as an energy minimization problem, where regularization terms constrain the solution to ensure certain smoothness cirterion or adherence to prior knowledge.

%-------------------------------------------------------------------------
    \subsubsection{Supervised learning based methods}
In recent years, several deep learning-based methods \cite{rakotosaona2020PCN,hermosilla2019TotalDenoising,luo2020DMR,luo_score-based_2021} have been proposed for point cloud denoising. NPD \cite{NPD2019} is the first learning-based point cloud denoising method that directly processes noisy data without requiring noise characteristics or neighboring point definitions. This approach combines local and global information by projecting noisy points onto estimated reference planes, effectively removing noise and enhancing robustness against variations in noise intensity and curvature. PointCleanNet\cite{rakotosaona2020PCN} first removes outlier points and then combines them with residual connectivity to predict the inverse displacement \cite{Guerrero2017PCPNetLL}, and iteratively shifts noisy points to remove noise. Pistilli \etal proposed GPDNet \cite{gpdnet2020}, which is a graph convolutional network to improve denoising robustness at high noise levels. Luo \etal also proposed  DMRDenoise \cite{luo2020DMR}, which filter
points by first downsampling the noisy inputs and reconstructing the local subsurface to perform point upsampling. However, this resampling method is difficult to maintain a good local shape. ScoreDenoise \cite{luo_score-based_2021} is proposed to tackle the aforementioned issues by iteratively updating the point position in implicit gradient fields learned by neural networks. For inference, they follows an iterative procedure with a decaying step size, which stabilizes point movement and prevents over-correction, allowing points to converge gradually toward the underlying geometry. The authors of \cite{de_Silva_Edirimuni_2023_CVPR} proposed IterativePFN - an iterative method that use a novel loss function that utilizes an adaptive ground truth target at each iteration to capture the relationship between intermediate filtering results during training. Zheng \etal proposed a end-to-end network for joint normal filtering-based point cloud denoising \cite{10173632}. They introduce an auxiliary normal filtering task to enhance the network's ability to remove noise while preserving geometric features more accurately.

Supervised methods can achieve impressive results, but are limited by the availability and quality of the training data, as they typically require paired noisy and clean point clouds to train the neural network. The absence of clean data in real-world scenario pose a significant challenge on applicability of these algorithms.

%-------------------------------------------------------------------------
    \subsubsection{Unsupervised learning methods}
Unsupervised learning-based methods for point cloud denoising do not require ground-truth clean data. Instead, these methods leverage the inherent structure or distribution of the point cloud to guide the denoising process. Unsupervised methods show promise in scenarios where clean data is absent or hard to obtain.

Hermosilla \etal first introduced Total Denoising (TotalDn) \cite{hermosilla2019TotalDenoising} as an unsupervised learning approach for point cloud denoising, relying solely on noisy data without requiring clean ground truth. TotalDn approximates the underlying surfaces by regressing points from the distribution of unstructured total noise, utilizing a spatial prior term to refine the estimation of geometry. 

In DMRDenoise \cite{luo2020DMR}, an unsupervised version is proposed which utilizes a loss function that identify local neighborhoods using a probabilistic Gaussian mask on the k-nearest neighbors, which selectively retains points likely to represent the underlying surface. By leveraging an Earth Mover's Distance (EMD) assignment, it achieves a one-to-one correspondence between input and predicted points, aligning them to reduce noise within local neighborhoods.
This approach enhances robustness to noise and adapts well to varied surface geometries. However, the probabilistic masking and EMD calculation add computational complexity, which can slow down inference in dense or noisy point clouds. 

ScoreDenoise \cite{luo_score-based_2021} also introduced an unsupervised version that employs ensemble score function and an adaptive neighborhood-covering loss for model training.  
Score-u is probably the most relevant work to our method. However, the defined score in \cite{luo_score-based_2021} is only an displacement-alike version of the score function and there is no explicit formula relating the estimated score to the final denoising result. Also, the iterative process is computationally expensive, and can suffer from fluctuation, leading to perturbed and unstable solution.

Most recently, Noise4Denoise \cite{noise4Wang2024} method is proposed that use an additional doubly-noisy point cloud to learn noise displacement by comparing the two noise levels. This approach effectively leverages synthetic noise for training, allowing the model to estimate residuals without relying on clean data.
However, in practical applications, noise parameters are often unknown and variable, making it challenging to replicate the exact conditions assumed during training. This reliance on predefined noise characteristics can limit the model's applicability to real-world scenarios where noise distributions may differ significantly from synthetic settings. 
%-------------------------------------------------------------------------
\subsection{Unsupervised image denoising}
Recently unsupervised image denoising has made significant progress. Non-Bayesian methods include PURE \cite{luisier2010image}, SURE \cite{SURE2018} \textit{etc.}, which are based on various unbiased risk estimator under certain noise distribution. Other methods explore self-similarity in natural images \cite{xu2015patch, doi:10.1137/23M1614456} or exploits the statistical properties of noise to achieve denoising effect \cite{gravel2004method}.  

Noise2Noise \cite{2018Noise2NoiseLI} is a pioneering method that does not require clean data, but it requires multiple noisy versions of the same image for training. To address this limitation, methods such as Noise2Void \cite{2018Noise2VoidL}, Noise2Self \cite{2019Noise2SelfBD}, \textit{etc.}, have been developed, which use only a single noisy image. These methods are particularly important for practical applications where obtaining clean images or multiple noisy realizations of the same image is difficult or impossible. Neighbor2Neighbor \cite{huang2021neighbor2neighbor} proposed a two-step method with a a random neighbor sub-sampler that generates training  pairs and a denosing network. Kim \etal proposed Noise2Score\cite{kim_noise2score_2021}, a novel Bayesian framework for self-supervised image denoising without clean data. The core of Noise2Score is the usage of Tweedie's formula, which provides an explicit representation of the denoised image through a score function. Combined with score function estimation, Noise2Score can be applied to image denoising with any exponential family noise. Kim \etal also proposed the Noise Distribution Adaptive Self-Supervised Image Denoising method \cite{kim_noise_2022}, which further extends the application of Noise2Score by combining the Tweedie distribution with score matching. This enables adaptive handling of various noise distributions and dynamically adjusts the denoising process by estimating noise parameters. On the other hand, Xie \etal \cite{scoreXie2024} broadened the denoising scope of Noise2Score by allowing it to handle complex noise models, such as multiplicative and structurally correlated noise, demonstrating broad applicability to diverse noise models.

These development of unsupervised image denoising method motivate us to explore similar ideas in 3D point cloud denoising.




\section{Case Study}
\label{sec:data-annotation}

Previous studies presented the difficulty humans face in distinguishing SOTA LLM-generated content from human-written text, often resulting in a random guess (see more in \appref{sec:relatedwork}). However, most evaluations focused on English and generations by GPT-3.5-turbo, leaving the detectability of MGT in other languages and LLMs uncertain.

To verify whether this observation can be generalized to other languages and more advanced LLMs, we collected LLM generations based on 16 datasets across nine domains and nine languages. 19 native speakers who are either LLM researchers or practitioners performed human evaluations, investigating (\emph{i})~whether humans can correctly discern human vs. AI outputs, and (\emph{ii})~whether humans prefer fellow human answers or LLM responses.


% \begin{table*}[t!]
%     \centering
%     \small
%     \resizebox{\textwidth}{!}{
%     \begin{tabular}{lllr|ccccccccr}
%     \toprule
%     \textbf{Language} & \textbf{Source/} & \textbf{Data} &  \textbf{Total} & \multicolumn{9}{c}{\textbf{Sampled Parallel Data}}  \\
%                       & \textbf{Domain} & \textbf{License} &\textbf{Human} & \textbf{Human} & \textbf{GPT-4o} & \textbf{Claude} & \textbf{Vikhr-Nemo-12B} & \textbf{Llama3-405B} & \textbf{ChatGLM4} & \textbf{Qwen2} & \textbf{Qwen-turbo} & \textbf{Total}  \\
%     \midrule
%     \multirow{4}{*}{Arabic} & Dialect Tweet &  Apache 2.0 &  1400 & 300 & 300* & & & & &300* & &900 \\
%     & ESAC & cc-by-sa-3.0 & 765 & 153 & 153 & & & & & & & 306 \\
%     & Youm7 News & --- & 21,000 & 1,000 & 1,000 & & & & & & & 3,000 \\
%     & SANAD & cc-by-4.0 & 194,797 & 100 & 100 & -- & -- & -- & -- & -- & -- & 200 \\
%     \midrule
%     \multirow{4}{*}{Chinese} & Zhihu-QA & cc-by-4.0 & 224761 & 588 & 588 & &&&&& 588 & 1,764 \\
%                              & Student essay & cc-by-4.0 & 93,002 & 600 &  & 300* & && 300* & & & 1,200 \\
%                              & Student essay & cc-by-4.0 & 51 & 51 & & 51 & && 51 & & & 153 \\
%                              & Government Report & \\
%     \midrule
%     English & Peersum~\citep{peersum_2023} & cc-by-sa-4.0 &  5158 & 400 & 200 & 200  & &&&& & 800\\
%     \midrule
%     Hindi & News & cc-by-4.0 & 3,995 & 600 & 600 & &&&&&& 1200 \\
%     \midrule
%     Italian & DICE & cc-by-sa &  10,518 & 300 & 300 & & & 300 & & & & 900\\
%     % & CItA & & 1352 & 300 & & & & 300 & & & \\
%     \midrule
%     Japanese & News & cc-by-nc-sa-4.0 & 7,110 & 300 & 300 & &&&&&& 600 \\
%      \midrule
%     Kazakh & Wikipedia & cc-by-sa-4.0 &  4,827 & 300 & 300  & &&&&&& 600\\
%     \midrule
%     \multirow{2}{*}{Russian} 
%     & News & MIT & 800,000 & 300 & 300 & & 300  &  & & & & 900 \\
%     & Academic summary & MIT & 31,000 & 300 & 300 & & 300 &  & & & & 900 \\
%     \midrule
%     \multirow{2}{*}{Vietnamese} 
%     & Wikipedia & --- & 600 & 600 & 600 & & &  & & & & 1,200 \\
%     & News & --- & 290,282 & 600 & 600 & & &  & & & & 1,200 \\
%     \midrule
%     \bf Total & -- & -- &  \\
%     \bottomrule
%     \end{tabular}
%    }
%     \caption{Statistics of multilingual data for human annotation. Machine data with * means non-parallel data. }
%     \label{tab:multilingual-data}
% \end{table*}



\subsection{Task and Dataset}
% 1. original dataset (license, size, domain, topic); 
% 2. how we did sampling and explain why (size and topic); 
% 3. how we did machine-generation (model, prompt, model generation configuration)

\paragraph{MGT detection} The goal is to identify whether the text was written by a human or generated by models given a single text, or to recognize which text is written by a human given a pair of texts: one human-written and one machine-generated.

In data collection, we focused on datasets from common domains such as community QA, news, tweets, and government reports, alongside domains requiring high-integrity LLM applications, including educational and academic contexts, such as accurate knowledge verification in Wikipedia-style texts, and identifying the authorship of student essays and peer reviews. 
For a given language and given a dataset, we sampled 300-600 human texts and then generated corresponding machine text using two SOTA LLMs: a multilingual model (e.g., from the GPT or the Claude series) and a language-specific model (e.g., ChatGLM or Qwen for Chinese and AceGPT for Arabic), to analyze the impact of different LLMs on detection performance, particularly for non-English languages.
% 
As shown in \tabref{tab:multilingual-data}, we collected data based on 16 datasets across nine languages. The generation prompts and collection details are shown in \appref{sec:datasets}. % Note that we generated more data and used the subset for human annotation.

% Yuxia Steps:
% 1. check original dataset section, organize into a appendix section
% 2. list what are empty and ask them to fill
% 3. check where is the collected data, and where is the evaluation results


\subsection{Human Detection Setups}
\label{sec:anno-setup}

\paragraph{Annotation Settings}
To mimic real-world machine-generated text detection scenarios, we set up four human evaluation settings.
Given human-written text and machine-generated text, representing by $hwt$ and $mgt$ respectively, human annotators are asked to identify which text was written by a human. Note that $mgt$ can be generated by multiple different LLMs, referring to as $mgt_i$, where $i \in [1,2, \cdots, n]$.

As shown in \tabref{tab:detection-settings}, according to the input and the output options, we categorize detection settings as I. \emph{pair-binary}, II. \emph{pair-four-class}, III. \emph{single-binary}, and IV. \emph{triplet-three-class}. 
For single text input, either $hwt$ or $mgt$, the goal is to recognize whether the text was written by a human, by answering just Yes or No. This is suitable for the scenario where for the human text there is no necessarily a corresponding machine-generated text, and thus they can be collected from different sources and for different topics, such as Arabic tweets.
Given a pair of texts (text1, text2), a binary output is easier than the four-class detection. The \emph{pair-binary} setting asks that either text1 or text2 is $hwt$, and the other one is $mgt$, while the \emph{pair-four-class} setting has no restrictions: each of text1 and text2 can be $hwt$ or $mgt$, regardless of the label of the other text. 
Sometimes, we want to compare human text to generations from different LLMs, in which case, we apply IV, which we limit to a three-class detection: human vs. LLM$_1$ vs. LLM$_2$.

Overall, I and IV are suitable for scenarios where there is a human text and its corresponding machine-generated text. II and III can be used if there are non-corresponding human-written and machine-generated texts.
% Settings I and III are commonly used in other studies, as well as this work.
If the annotators have seen some human-written and some machine-generated text before detection, we refer to this as a few-shot setting; otherwise, we have zero-shot.


\begin{table*}[t!]
\centering
%\tabcolsep3pt
\resizebox{\textwidth}{!}{\small
\begin{tabular}{l p{4cm} p{8cm} p{4cm} p{4cm}}
\toprule
\textbf{Setting ID} & \textbf{Input} & \textbf{Task Description} & \textbf{Output Options} & \textbf{Applicable Scenarios} \\
    \midrule 
    \multirow{3}{3cm}{\textbf{I. Pair-Binary}} & ($hwt$, $mgt$) \textbf{or} ($mgt$, $hwt$) & Given a pair of (text1, text2), identify which one is human-written? Either text1 or text2 must be $hwt$, and another is $mgt$ randomly sampled from $mgt_i$. & \textbf{A.} text1; \textbf{B.} text2 & parallel data is available. \\
    \midrule
    \multirow{2}{3cm}{\textbf{II. Pair-Four-Class}} & ($hwt$, $mgt$) \textbf{or} ($mgt$, $hwt$) \textbf{or} ($hwt$, $hwt$) \textbf{or} ($mgt$, $mgt$) & Given a pair of (text1, text2), identify which one is human-written? Both text1 and text2 can be $hwt$, and can be $mgt$. & \textbf{A.} text1; \textbf{B.} text2; \textbf{C.} none of them; \textbf{D.} both & parallel data is not necessary. \\
    \midrule
    \multirow{1}{3cm}{\textbf{III. Single-Binary}} & $hwt$ or $mgt$ & Given a piece of text, identify whether it is written by human? & \textbf{A.} Yes, human; \textbf{B.} No, machine & parallel data is not necessary. \\
    \midrule
    \multirow{3}{3.5cm}{\textbf{IV. Triplet-Three-Class}} & ($hwt$, $mgt_1$, $mgt_2$)  & Given a set of texts (text1, text2, text3), identify which one is human-written? One of the text1, text2 and text3 must be $hwt$, and others are $mgt$ randomly sampled from $mgt_i$. & \textbf{A.} text1; \textbf{B.} text2; \textbf{C.} text3 & parallel human and multiple LLM generations are collected to make comparisons. \\
    \bottomrule
\end{tabular}
}
\caption{The four human detection settings: the setting name refers to input-output options, pair/binary means the input is a pair of texts and the goal is to predict binary labels, and whether the text is human-written (Yes or No).}
% I and IV are suitable for the scenarios where there is a human text and its corresponding machine text (parallel data is available). II and III can be used if parallel data is not available, there are only separate human text and machine text.
\label{tab:detection-settings}
\end{table*}



\paragraph{Annotation Tool}
% Rui: annotation pipeline we used and why we used these two: quality and efficiency
To mitigate potential labeling biases arising from raw spreadsheet annotation and to enhance efficiency, we implemented two methods with optimized interfaces and workflows for our annotation: (1)~a custom pipeline using the Google Workspace suite, including Apps Script, Google Sheets, and Google Forms. The core idea was to store all data in Google Sheets, use Apps Script to extract data and generate a survey in Google Forms; and (2)~Label Studio, an open-source multi-type data labeling and annotation tool with a standardized output format. We designed a custom template for our annotation task and collected results using this platform. The annotators were given the choice to use their preferred tool. 

% \footnote{https://developers.google.com/apps-script}, 
% \footnote{https://github.com/HumanSignal/label-studio/}  


\paragraph{Annotator Background}
In order to explore the upper bound of human detection capability, instead of using crowd-sourcing annotators, we conducted in-house labeling. 
The annotators were BSc, MSc, and PhD students, as well as postdocs, who were familiar with NLP tasks and LLM generations. All annotators independently completed their individual annotations. For each language, the annotators were all native speakers of that language. See more detail in \appref{sec:whyexperts}.

% Name, Age, Gender, degree, mother language, personality test type, 
\section{Human Detection}
\label{sec:human-detection-acc}
% Previous human detection on English machine-generated text show that it is challenging for humans to discern MGT from human text.
We performed an extensive case study on 9K instances across nine languages to verify how difficult it is for native speakers to detect AI outputs in everyday domains.
\tabref{tab:detection-acc} demonstrates that the average human detection accuracy is 87.6\%.
This reveals that this is not particularly difficult for native human experts, contrary to what previous studies have reported. Below, we zoom into the impact of various factors.

\begin{table*}[t!]
    \centering
    \small
    % \resizebox{\textwidth}{!}{
    \begin{tabular}{llr cll c}
    \toprule
    \textbf{Language} & \textbf{Source/Model} & \textbf{\#Example} & \textbf{\#Annotator} & \textbf{Anno Setup} & \textbf{Shot} & \textbf{Avg. Acc} \\
    \midrule
    \multirow{4}{*}{Arabic} 
    & Dialect Tweet & 900 & 1 & III. Single-binary & Zero & 50.1  \\
    & EASC Summary & 100 & 1 & I. Pair-binary & Zero & 82.0  \\
    & Youm7 News & 1,000 & 1 & I. Pair-binary & Zero & 92.7  \\
    & SANAD News & 100 & 1 & I. Pair-binary & Zero & 100.0  \\
    \midrule
    \multirow{6}{*}{Chinese} 
    % & Zhihu-QA (\gptfouro) & 428 & 5 & I. Pair-binary & Zero & 0.99, 0.99, 1.0, 1.0, 1.0 & 0.996 \\
    & Zhihu-QA (\gptfouro) & 428 & 5 & I. Pair-binary & Zero & 99.6 \\
    & Zhihu-QA (\gptfouro) & 160 & 1 & I. Pair-binary & Few  & 100.0  \\
    % & Zhihu-QA (\qwenturbo) & 588 & 2 & I. Pair-binary & Zero & 0.99, 0.97 & 0.98 \\
    & Zhihu-QA (\qwenturbo) & 588 & 2 & I. Pair-binary & Zero & 98.0 \\
    & Student essay & 102 & 1 & I. Pair-binary & Few  & 98.0  \\
    % & Student essay & 600 & 3 & II. Pair-four-class & Zero & 0.96, 0.96, 0.99 & 0.97 \\
    & Student essay & 600 & 3 & II. Pair-four-class & Zero & 97.0 \\
    & Government Report & 500 & 1 & IV. Triplet-three-class & Few & 97.2  \\
    \midrule
    English & Peersum & 400 & 1 & I. Pair-binary & Few & 99.8   \\
    \midrule
    Hindi & News & 600 & 1 & I. Pair-binary & Few & 85.2   \\
    \midrule
    \multirow{3}{*}{Italian} 
    & DICE News (Anita) & 300 & 1 & I. Pair-binary & Few & 88.0  \\
    & DICE News (Llama3-405B) & 300 & 1 & I. Pair-binary & Few & 99.7  \\
    & DICE News (GPT-4o) & 300 & 1 & I. Pair-binary & Few & 100.0  \\
    % & CItA & \\
    \midrule
    Japanese & News & 300 & 2 & I. Pair-binary & Zero & 86.4  \\
    \midrule
    Kazakh & Wikipedia & 300 & 2 & I. Pair-binary & Zero & 79.7   \\
    \midrule
    \multirow{2}{*}{Russian} 
    & News & 300 & 1 & I. Pair-binary & Few & 100.0   \\
    & Academic summary & 300 & 1 & III. Single-binary & Few & 80.0   \\
    \midrule
    \multirow{2}{*}{Vietnamese} 
    & Wikipedia & 600 & 1 & I. Pair-binary & Zero & 50.7  \\
    & News & 600 & 1 & I. Pair-binary & Zero & 80.3  \\
    \midrule
    \bf Total & -- & 8,778 & 30 & & & 87.6\\
    \bottomrule
    \end{tabular}
   % }
    \caption{Human annotator detection accuracy over 16 datasets and 9 languages: we have a total of 30 annotation settings and 19 unique human annotators. The average accuracy of the human expert guesses is 87.6\%.}
    \label{tab:detection-acc}
\end{table*}


\subsection{Language}
% \paragraph{Language, Domain, Generator}
Human detection accuracy exceeds 87.6\% for Chinese, English, Arabic, Italian and Russian, while it falls below this level for Vietnamese, Kazakh, Hindi, and Japanese. This discrepancy is largely due to the challenge of Wikipedia text.  

\subsection{Domain}
Wikipedia is widely used as training data for LLMs, particularly for low-resource languages, due to the scarcity of alternative datasets. Consequently, models often memorize Wikipedia content, leading to generated text that closely resembles human-written Wikipedia. 
Arabic tweets also present challenges for detection due to their short length and limited context, along with summaries, e.g., for Arabic and Russian summaries, the expert detection accuracy is about 80\%. 
This conversely highlights the ability of language models to generate high-quality human-like text in the domains of Wikipedia, tweets, and summaries. In contrast, substantial differences between machine-generated and human-written text persist in news articles, QA, student essays, and peer reviews, making them much easier to recognize for human experts. 

% \begin{table}[t!]
%     \centering
%     \resizebox{\columnwidth}{!}{
%         \begin{tabular}{lcccc}
%             \toprule
%             \textbf{Dialect} & \textbf{Human} & \textbf{\gptfouro} & \textbf{\qwentwo-7.5B} & \textbf{Overall MGT} \\
%             \midrule
%             EGY & 52.00 & 53.33 & 58.67 & 56.00 \\
%             MOR & 54.00 & 53.33 & 48.00 & 50.67 \\
%             LEV & 69.33 & 14.67 & 58.67 & 36.00 \\
%             GULF & 81.33 & 26.67 & 30.67 & 28.67 \\
%             \bottomrule
%         \end{tabular}
%     }
%     \caption{Arabic dialect tweet human detection accuracy over human vs. \gptfouro vs. \qwentwo-7.5B. Machine-generated text is harder than human text to discern. \gptfouro is harder than \qwentwo.}
%     \label{tab:arabic-dialect_tweet-accuracy}
% \end{table}


\subsection{Generator} 

It is hard to detect MGT across generators and languages.
While there are minimal differences for Chinese (accuracy is 100\% vs. 98\% for \gptfouro vs. \qwenturbo), there are sizable differences for Italian and Arabic.
Based on Italian DICE News, the same annotator detected generations by Anita (an Italian fine-tuned Llama3-8B), Llama3-405B, and \gptfouro, achieving accuracy of 88\%, 99\%, and 100\%, respectively.
Similarly, for Arabic tweets, \gptfouro's outputs are more similar to human text and thus more difficult to detect compared to those by \qwentwo, as shown in \tabref{tab:arabic-dialect_tweet-accuracy}.

\subsection{Annotation Setting}
We conducted the majority of our annotations under setting I. \emph{pair-binary}: given a pair ($hwt$, $mgt$), it asks to identify which of the two texts is human-written. 
We assumed that more complex settings would be more challenging. For instance, II. \emph{pair-four-class} should be harder, as each of the texts could be human- or machine-generated, independently of the other. %, introduces additional options.
Yet, for Chinese student essays, the performance for II does not degrade compared to I.
Similarly, for government reports in IV. \emph{triplet-three-class}, where the annotators have to select the human-written text among three candidates, there was no degradation compared to I.  

However, III. \emph{single-binary} proves to be more difficult than I. \emph{pair-binary} for both Arabic and Russian. While domain differences do exist, e.g.,~tweets vs. summary vs. news in Arabic and news vs. summary for Russian, the substantial performance gap (>20\%) can still be partially attributed to the annotation settings.
Overall, comparing the results for I. vs. III., it is easier to distinguish machine-generated content when given a comparison pair, rather than for single answer. Yet, introducing text triplets or increasing the number of machine-generated or human-written texts had minimal impact on detection performance. 
Moreover, using few shots before detection boosted the confidence of the annotators, resulting in higher accuracy compared to zero-shot.
% \footnote{Slight improvement in our results can be largely owing to expert-level annotators, given that most of them are LLM researchers.}
For datasets with high accuracy, before seeing labeled samples, the annotators found the distinction to be obvious. After seeing a few examples, the annotator was extremely confident in distinguishing human vs. machine text based on indicative features of MGT. 




\subsection{Expert Annotators}
% \paragraph{What factors may influence individuals' discerning accuracy?}
% 1. Individual personal ability: given the same language and a same text, what matters?
For the same language and dataset, individual annotator ability influences detection accuracy but not significantly. 
For instance, in Chinese Zhihu-QA (GPT-4o vs. human), five annotators achieved accuracies of 99\%, 99\%, 100\%, 100\%, and 100\%. Similarly, for Zhihu-QA (Qwen-turbo vs. human), two annotators obtained 99\% and 97\%. In student essays (II. pair-four-class), three annotators recorded accuracies of 96\%, 96\%, and 99\%.
This may also result from the bias that all annotators are native speakers and expert-level LLM practitioners or researchers. Differences between individuals are minor in their cognitive abilities, language proficiency and domain knowledge.

% a. cognitive abilities: strong analytical skills may be better at detecting inconsistencies, logical flaws, or patterns that indicate AI-generated text; and attention to details, where individuals who pay close attention to grammar, syntax, and style nuances might notice subtle signs of AI output.
% b. language proficiency: A high level of language proficiency may make it easier to spot overly formal, repetitive, or mechanically structured phrasing typical of AI outputs; and cultural nuance awareness, understanding cultural or idiomatic expressions can help identify content that lacks human-like subtlety or creativity.
% c. domain knowledge (subject matter experts are more likely to spot factual errors or lack of depth in AI-generated text) and familiarity with AI output. Individuals with experience interacting with AI models may recognize their writing patterns, such as positive tone bias.
% d. personal biases: A person’s predisposition to trust or distrust technology may influence their discernment; Beliefs about what constitutes human creativity or uniqueness may shape how someone distinguishes human from AI text.
% e. other individual traits: gender, age, ...


\subsection{Distinguishable Factors}
\label{sec:dis-factor}
We summarize five remarkable distinguishable signals between machine-generated vs. human-written text across the 16 datasets and the 9 languages; see more details in \appref{sec:distinctionfactor}. % regarding detection setting, accuracy and distinction factors 

\begin{compactitem}
% \begin{itemize}
    \item \textbf{Human text is more informative and concrete.}
    Human-written text contains concrete numbers, specific names of people or institutions, exact places or dates, URLs, and other references, while machine-generated text tends to provide generic information, with little detail to support its statements. 
    
    \item \textbf{Machine-generated text lacks regional, cultural, and religious nuances.}
    For languages such as Arabic, Japanese, Hindi, Kazakh, and Chinese, human texts reflect the cultural and the religious nuances of the language, which is not true for machine-generated text.

    \item \textbf{Human-written text varies substantially in terms of length, structure, style, and sentiment.}
    Human texts show diversity and inconsistency with large deviations in length, structure, style and emotions, while machine-generated texts tend to use a formulaic structure and neutral/positive emotion. This can be partially attributed to LLMs rigorously following instructions, and thus losing on flexibility. 
    

    \item \textbf{Machine-generated text has formatting.}
    MGTs are generally well-segmented with bullet points for better readability, while human-written texts are typically just large block of plain text, which may be due to human text collection and conversion. Moreover, machine-generated texts often use Markdown style, e.g., \texttt{**} and \texttt{\#\#\#}, while human-written texts have typos, grammatical errors, hashtags, and other social media elements.

    \item \textbf{Machine-generated text shows a mixture of other languages.} Non-English language responses often contain some English parts, which is very rare in human text.
% \end{itemize}
\end{compactitem}

\section{Can Prompting Fill in the Gap?}

% As we saw above, there are distinctions between human-written and machine-generated text. % in \secref{sec:human-detection-acc}. 
Given that LLMs can strictly follow instructions and their outputs are heavily influenced by the system and the user prompts, we investigated whether explicitly instructing LLMs to mimic human style can help narrow the gap. 
Responding to the distinguishable factors summarized in \appref{sec:distinctionfactor} for each dataset, we asked the human annotators to craft new prompts, aiming to improve the generations and to reduce the gap between human-written and LLM-generated texts. This involved trying instructions that (1) incorporate specific details and references, (2) avoid formulaic structures and formats, e.g., bullet points and Markdown, and (3) generate texts of varying length, structure, and sentiment.
\tabref{tab:ori-improved-prompts} presents the results for both the original and the improved prompts for all datasets.  


\textbf{Measurements:}
We re-generated the machine-generated parts of the text pairs, using the same models with improved prompts, and then sampled 200–600 examples from each dataset to assess whether and to what extent, the prompting strategy narrowed the gap between human-written and machine-generated texts. 
We used two approaches: (1) \textit{fill-the-gap survey}, where the original annotators evaluated whether the newly-generated text bridged the gap (Yes, No, or Partially), and (2) \textit{downstream detection}, where we compared the detection accuracy before and after applying the improved prompts. A decline in detection accuracy indicated a reduced distinction between human and machine text, making the differentiation more difficult, and further revealing that prompting was effective. Our experiments involved both human annotator evaluation and automated detection. 

\begin{figure}[t!]
    \centering
    \includegraphics[scale=0.38]{section/images/dist-stackedbar.pdf} 
    \caption{Evaluating whether the new generations fill in the gap: Yes, Partially, or No.}
    \label{fig:dist-survey}
\end{figure}

% \begin{figure*}[t!]
%     \centering
%     \includegraphics[scale=0.6]{section/images/dist-stackedbar.pdf} 
%     \caption{Distribution of survey evaluating whether the new generations fill the gap? Yes, Partially or No.}
%     \label{fig:dist-survey}
% \end{figure*}

% \begin{figure*}[t!]
%     \centering
%     \includegraphics[scale=0.6]{section/images/IAA-heatmap.pdf}
%     \caption{Three annotator agreement on Chinese essays regarding whether the improved prompts mitigate the gap between human text and machine-generated text.}
%     \label{fig:iaa-heatmap}
% \end{figure*}

% \begin{figure*}[t!]
%     \centering
%     \includegraphics[scale=0.55]{section/images/acc-bars.pdf}
%     \caption{Human detection accuracy differences on original vs. improved generations.}
%     \label{fig:acc-diff}
% \end{figure*}

% \begin{figure*}[t!]
%     \centering
%     \includegraphics[scale=0.55]{section/images/auto-acc-bars.pdf}
%     \caption{\textbf{Detection accuracy differences} of 26 automatic machine-generated text detection approaches on original vs. improved generations.}
%     \label{fig:auto-acc-diff}
% \end{figure*}

\paragraph{Fill-the-gap Survey}
The original annotators who conducted the detection on earlier generations were asked to evaluate whether the new prompts addressed the gaps for each example: \emph{Yes}, \emph{No}, and \emph{Partially}.
The distributions across the six datasets in \figref{fig:dist-survey} shows that in about 50\% of the cases, the prompt adjustments were effective to  either fully or partially mitigate the gaps. Large improvements were observed for Kazakh Wikipedia and Arabic tweets.
For the former, the revised prompt reduced repetitive sentence patterns (more diverse), but the formulaic expressions were not entirely eliminated. New outputs also included more concrete information, such as dates and names, while the inclusion of culturally-nuanced details remained challenging. The newly-generated Arabic tweets could touch on relatable human topics and express genuine emotions tied to daily experiences; however, the frequently added irrelevant hashtags at the end of the tweets made them easily identifiable as machine-generated. Moreover, the tweets often leaned on an overly optimistic tone even when negative experiences were mentioned. 

The annotators for English peer reviews noted that, despite the prompt adjustments, the models remained highly formulaic in their outputs, the length of the reviews remained relatively uniform, and the overall structure lacked variance. This may be due to the inherent nature of the peer review domain, while human reviews exhibited more variability in both length and structure, feeling more organic.
Similar issues remained for Chinese student essays (formulaic structure by using ``firstly, then, moreover, finally'' persisted) and government reports (certain repetitive phrases). 

% \textit{What remains challenging to address?}
Overall, adjusting the prompts did fill some gaps, but cultural nuances, diversity of length, structure and phrases, and sentiment adaption to scenarios remained challenging. See more in \appref{sec:fill-gap-prompts}.
Since annotators' background may influence the survey results (see \ref{app:subjectivity-test}), we conducted a second round of MGT detection on the new generations with the hop   e for more objective results.


\begin{figure}[t!]
    \centering
    \includegraphics[scale=0.4]{section/images/acc-bars.pdf}
    \caption{Human detection accuracy for the original vs. the improved generations.}
    \label{fig:acc-diff}
\end{figure}

\paragraph{MGT Detection on Improved Text}
We performed human detection on 13 datasets under the same annotation setting as described in \tabref{tab:detection-acc}, with the exact same annotators.
We observed sizable accuracy declines across all datasets in \figref{fig:acc-diff}, with average accuracy dropping to 72.5\%.
This implies that the improved machine-generated text became more similar to the human-written text, making it harder to discern and thus resulting in lower detection accuracy.


% \paragraph{Automatic Detection}
We further analyzed the impact of the improved generations on automatic detection accuracies.
% provide a quantitative analysis,
We collected a total of 17,017 texts using the original prompts and 32,487 texts using the improved prompts (detailed statistical distribution in \tabref{tab:multilingual_prompt_dist}).
We reproduced 26 MGT detection approaches presented in the COLING 2025 GenAI shared task~\cite{wang-etal-2025-genai} and evaluated them on the collected MGTs. As shown in \figref{fig:auto-acc-diff}, 19 methods exhibited lower accuracy on the newly generated texts, indicating that the texts produced using the improved prompts are more challenging to distinguish compared to those generated with the original prompts.
This suggests that prompting strategies can help bridge some gaps between machine-generated and human-written text when explicitly designed to mimic human writing style.



\begin{comment}
Interestingly, when we explicitly tell annotators that the given text is newly-generated and ask them to analyze whether the gaps have been resolved, they can clearly identified what issues are remained. When they are asked to distinguish between the improved machine-generated text and human-written text, some cases are harder for them to differentiate, leading to lower accuracy.
% We believe further research could analyze the impact of human attention in such tasks.
\end{comment}
\section{Human-Like or Liked-by-Human?}
We used the prompting strategy to bridge the gap between human-written and machine-generated text, aiming to make machine outputs more human-like. However, do humans favor human-like text? 
% The common assumption is that humans expect machines to behave more like humans, but is this hypothesis true? 
Below, we examine human preferences among four options: (1) human-written text, (2) machine-generated text using the original prompt, (3) machine-generated text using the improved prompt, or (4) none of the above. 
%, to examine which type of text is preferred? 


\begin{figure}[t!]
    \centering
    \includegraphics[scale=0.37]{section/images/zh-preference.pdf}
    \caption{Human preferences for three Chinese datasets (five annotators): QA-emo is an emotion-rich question subset of Zhihu-QA with 100 examples.}
    \label{fig:pre-zh}
\end{figure}

% \begin{figure}[t!]
%     \centering
%     \includegraphics[scale=0.50]{section/images/ru-preference.pdf}
%     \includegraphics[scale=0.45]{section/images/ar-preference.pdf}
%     \caption{Human preferences for two Russian (three annotators) and two Arabic datasets (two annotators).}
%     \label{fig:pre-ru-ar}
% \end{figure}

\begin{figure}[t!]
    \centering
    \includegraphics[scale=0.45]{section/images/ru-preference.pdf}
    \includegraphics[scale=0.4]{section/images/ar-preference.pdf}
    \caption{Human preferences for two Russian (three annotators) and two Arabic datasets (two annotators).}
    \label{fig:pre-ru-ar}
\end{figure}


\paragraph{Preference Labeling Setup:}
We labeled the preferences for three languages: Chinese, Russian, and Arabic.
For Chinese, we annotated Zhihu-QA and student essays (300 examples for each), along with 100 responses particularly for Zhihu questions, where emotional and empathetic comforts are highlighted. Five unique annotators participated, identified by \textit{nationality-gender-degree}. For example, \textit{Zh-Male-PhD} refers to a Chinese male, who is a PhD student.  
Similarly, we labeled two datasets for Russian (three annotators) and two datasets for Arabic (two annotators). 


\paragraph{Do People Always Prefer Human Text?}
The answer is \emph{No}.
Analyzing the preferences of ten annotators across six datasets in \figref{fig:pre-zh} and \ref{fig:pre-ru-ar}, human text was preferred in about half of the cases. Notably, for Russian and Arabic, annotators tended to favor machine text. 
This is particularly evident for Russian summaries using the improved prompts (green bars) and Arabic summaries using the original prompts (orange bars).
% This echos the finding in detection that machine-produced summary reaches human-expert quality, closely resembling human-written ones and making differentiation challenging.


For Chinese datasets, including Zhihu QA and student essays, human-written text is generally preferred though there are exceptions.
% For instance, 
\textit{Zh-Male-postdoc} exhibits a unique preference distribution that deviates from the rest.
% \footnote{This difference is not due to inattentive annotation but rather reflects a unique philosophy of the annotator.} %  across various judgments
Interestingly, for emotion-rich questions (QA-emo), where human responses are expected to be more empathetic and be preferred, three out of four annotators actually prefered the machine-generated responses. The remaining annotator disliked both the human and the machine text in 22\% of the cases. The annotator feedback suggested that this was influenced by the presence of mean-spirited responses from Zhihu users, where some human answers expressed personal biases and lacked empathy (see \ref{app:humanpreference}).
% \ref{sec:chinese-distinguishable-signal}).


\paragraph{Why are Human-Written Essays Favored?}
% \textbf{Student Essay:}
% Essays written by humans are more coherent and sincere.
% For model essays, the coherence between sections is poor, presenting hard boundary between independent sections, sometimes repeating the title to make it superficially coherent. 
% Model tends to tell concepts, related concepts. It is hard for models to tell a good story, or connect several stories naturally with the same theme, and then return back to the main topic.
% Therefore, articles by models always lack substances, full of empty concepts and rhetoric, coming across as superficial and ungrounded, overly wordy but fails to deliver any real insights. They feel grandiose but devoid of meaningful content. Model tends to play a role of teacher, rather than a peer. Some article are written like answering questions. % (id 26)

Human-written essays exhibit greater coherence and sincerity. In contrast, machine-generated essays often lack cohesion, displaying abrupt transitions and sometimes repeating the title to create superficial continuity. While LLMs can generate related concepts, they struggle to construct a compelling narrative or to seamlessly connect multiple stories under a unified theme. As a result, their outputs often lack depth, relying on abstract concepts and rhetorical flourishes without delivering substantive insights. The text tends to be verbose yet superficial, giving an impression of grandiosity without meaning. Moreover, LLM-generated essays often adopt an instructive tone, resembling answers to questions rather than peer-level discourse.


% Human: 更具有文学气息,委婉的优美感,氛围感
% 好的文章,或生动可爱俏皮之感,或立意新奇,启发人深思,或首尾巧妙呼应,或真诚质朴,真情实感的动情,或优美凄婉,用词清新准确,情趣意境佳
From a literary perspective, a well-written human article may be lively, charming, and playful, or it may present a novel perspective that inspires deep reflection. It might feature a cleverly structured beginning and ending, convey sincerity and heartfelt emotions, or captivate with its elegance and melancholy. With fresh and precise wording, it creates a rich atmosphere, evoking a refined aesthetic and a sense of poetic depth.  
% Model: 没有值得回味的感觉,一眼就可以看清楚要说什么,不需要太多思考和回味,说教式议论文
In contrast, a piece generated by an LLM lacks this literary nuance, leaving little room for contemplation or lingering thoughts. The expression is immediately clear, requiring no further reflection, resembling more of a lecture than an engaging discourse.


\paragraph{Preference Distributions Vary Across Individuals.}
The annotators for Russian and Arabic exhibited similar preference distributions, whereas large differences occured for Chinese QA.
For example, in the Chinese Zhihu QA, the second annotator selected only seven human-written texts, while the fourth one chose 284 (95\%). In the QA-emo, the first and the second annotators prefered human-written responses, whereas the third and the fourth favored machine-generated text.
This variance shows the charm of collecting personal preferences and then optimizing models to align with individual philosophies. 
Our preference annotations can serve as a valuable resource for investigating the relationship between annotator characteristics (e.g., MBTI personality, gender, and age) and preferences. Also the data can guide models to match individual preferences in multilingual contexts.
% \footnote{We will release all annotator metadata to facilitate future research of the community.}



\paragraph{Human-Like or Liked-by-Human?}
Human texts are not always preferred. This inspires us to reflect the ultimate goal of building LLMs that are human-like vs. liked-by-human.
The goal of being human-like has a single target, i.e.,~mimicking human behavior, while to be liked-by-humans involves optimization towards billions of local optima, each shaped by individual preferences.  
Current language models establish a foundation by learning from human data towards being human-like. 
As they get more advanced, they can further adapt by incorporating personal data, thus transitioning from merely imitating human behavior to aligning with individual user preferences, and moving from human-like to liked-by-human.
\section{Conclusion and future directions} \label{sec:conclusion}

In this paper we proposed a nested MLMC framework that offers important computational savings by performing most calculations in low precision and exploiting approximate random normal variables for the low precision path calculations. The low precision calculations could be performed in fixed precision on an FPGA for greater efficiency, and we suggested a procedure to optimise the bit-widths of every variable at each Monte Carlo level. This is an important improvement over previous mixed precision MLMC frameworks which held the lower precision fixed \cite{Rounding_error_oliver} or defined uniform bit-width at every level heuristically \cite{brugger2014mixed}. Our numerical results suggest that for the first levels our procedure reduces the cost at these levels by a factor 5 or 7. Hence the overall savings are significant since most paths are calculated on the first levels. Our approach would be even more efficient for the Milstein scheme because its higher order strong convergence leads to a greater proportion of the computational costs being on the coarsest levels.

The next stage of the research project will be to implement the RNG methods and the nested framework on FPGAs to determine the hardware requirements and confirm the extent of the computational savings. It would also be good to compare the performance benefits to using half-precision floating point arithmetic on GPUs or CPUs for the low-accuracy computations.





% Entries for the entire Anthology, followed by custom entries
\bibliography{ref}
\bibliographystyle{acl_natbib}

\appendix

\section{Appendix: Prompt}
\label{sec:appendix}
``Here is a sketch of an image. 
$\{input\_color\_mask\}$, while the rest of the white space is the background. 
I need you to infer details of the image based on the given sketch.
The details should include the possible background likely to be present with the $\{input\_color\_mask\}$, the attribute of each object (like wearing, texture, color etc.), the state (including action, posture, etc.) of each object, the direction of each object and the relationships between objects.

You should first analyze the mask carefully, considering the size, location, and relative position of each object mask. Ensure that specific actions are analyzed based on the mask, and infer each aspect with a reasoning process before providing the final output.
The final output format should be: $\{format\_example\}$, and you should refer to the example: $\{few\_shot\}$. You are going to complete the "" in each item, you need to complete them in multiple short phrases based on your above reasoning.

The state and relationship should be as detailed as possible while ensuring they align with the mask, formatted as: objectA action/spatial relation objectB, with both objectA and objectB included.
You should properly refer to some examples of attributes of object $\{attributes\}$ and relationships $\{relationships\}$.
Do not include words like `or', `possibly' in your final output, there should no ambiguity in your output.
Make sure all aspects of given mask is filled.''



\end{document}
