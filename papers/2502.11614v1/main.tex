% This must be in the first 5 lines to tell arXiv to use pdfLaTeX, which is strongly recommended.
\pdfoutput=1
% In particular, the hyperref package requires pdfLaTeX in order to break URLs across lines.

\documentclass[11pt]{article}

% Remove the "review" option to generate the final version.
% \usepackage{EMNLP2023}
% \usepackage[review]{acl}
\usepackage{acl}

% Standard package includes
\usepackage{times}
\usepackage{latexsym}
\usepackage{adjustbox}
\usepackage[T1]{fontenc}
% For proper rendering and hyphenation of words containing Latin characters (including in bib files)
\usepackage[T1,T2A]{fontenc}
% For Vietnamese characters
% \usepackage[T5]{fontenc}
% See https://www.latex-project.org/help/documentation/encguide.pdf for other character sets

% This assumes your files are encoded as UTF8
\usepackage[utf8]{inputenc}

% This is not strictly necessary and may be commented out.
% However, it will improve the layout of the manuscript,
% and will typically save some space.
\usepackage{microtype}

% This is also not strictly necessary and may be commented out.
% However, it will improve the aesthetics of text in
% the typewriter font.

\usepackage{lipsum} % Optional, for filler text
\usepackage{arabtex}
%\usepackage[T2A,T1]{fontenc}


\usepackage[english, vietnamese, russian]{babel}
\usepackage{utf8}
\setcode{utf8}
\usepackage{inconsolata}
\usepackage{microtype}
\usepackage{graphicx}
\usepackage{import}
\usepackage{layout}
\usepackage{tabularx, makecell}
\usepackage{booktabs}
\usepackage{mathrsfs}
\usepackage{amssymb} 
\usepackage{url}
\usepackage{hyperref}
\usepackage{graphicx}
\usepackage{xspace,paralist}
\usepackage{times,latexsym}
\usepackage{amsmath}
\usepackage{appendix}
\usepackage{comment} 
\usepackage{enumitem}
\usepackage{makecell}
\usepackage{multirow}
\usepackage{xcolor}
\usepackage{arydshln}
\usepackage{cleveref}
\usepackage{todonotes}
\usepackage{longtable,supertabular}
\usepackage[russian,english]{babel}
\usepackage[T2A,T1]{fontenc}

\usepackage{amssymb}% http://ctan.org/pkg/amssymb
\usepackage{pifont}% http://ctan.org/pkg/pifont
\usepackage{CJKutf8}
\newcommand{\cn}[1]{\begin{CJK*}{UTF8}{gbsn}#1\end{CJK*}}
\newcommand{\cmark}{\ding{51}}%
\newcommand{\xmark}{\ding{55}}%

\newcommand{\eqnref}[1]{Eq~\eqref{#1}\xspace}
\newcommand{\tabref}[2][]{Table#1~\ref{#2}\xspace}
\newcommand{\figref}[1]{Figure~\ref{#1}\xspace}
\newcommand{\secref}[1]{Section~\ref{#1}\xspace}
\newcommand{\appref}[1]{Appendix~\ref{#1}\xspace}

\newcommand{\model}[1]{\text{#1}\xspace}
\newcommand{\factscore}{\model{FactScore}}
\newcommand{\factool}{\model{FacTool}}
\newcommand{\factor}{\model{FACTOR}}
\newcommand{\rarr}{\model{RARR}}
\newcommand{\cove}{\model{CoVe}}
\newcommand{\perplexityai}{\model{Perplexity.ai}}

\newcommand{\claude}{\model{Claude}}
\newcommand{\chatglmfour}{\model{ChatGLM-4-9B-chat}}
\newcommand{\chatgpt}{\model{ChatGPT}}
\newcommand{\gptfour}{\model{GPT-4}}
\newcommand{\gptfouro}{\model{GPT-4o}}
\newcommand{\qwenturbo}{\model{Qwen-turbo}}
\newcommand{\qwentwo}{\model{Qwen2}}
\newcommand{\bard}{\model{Bard}}
\newcommand{\llama}{\model{LLaMA}}
\newcommand{\llamatwo}{\model{LLaMA2}}
\newcommand{\llamathree}{\model{LLaMA3}}
\newcommand{\alpaca}{\model{Alpaca}}
\newcommand{\vicuna}{\model{Vicuna}}
\newcommand{\ecot}{\model{ECoT}}

\usepackage{CJKutf8}
\usepackage[utf8]{inputenc}
\newcommand{\zx}[1]{{\color{CadmiumOrange}{\bf{[ZX:]}} #1}}


\NewDocumentCommand{\revanth}
{ mO{} }{\textcolor{blue}{\textsuperscript{\textit{Revanth}}\textsf{\textbf{\small[#1]}}}}


% If the title and author information does not fit in the area allocated, uncomment the following
%
\setlength\titlebox{3in}
%
% and set <dim> to something 5cm or larger.

% \title{Towards human-like vs. liked-by-human Multilingual LLM: Comprehensive Human Detection and Preference over Multidomain and Multilingual MGT}
% \title{Is Human-Written Text Liked by Humans? \\Multilingual Human Detection and Preference Against AI}
\title{Is Human-Like Text Liked by Humans? \\Multilingual Human Detection and Preference Against AI}
% \title{Human Discernment of Human from AI Texts: Multilingual Case Study}
% \title{Human can Safeguard Against AI: Multilingual Case Study of Human Discernment of Human from AI texts}

% Author information can be set in various styles:
% For several authors from the same institution:
% \author{Author 1 \and ... \and Author n \\
%         Address line \\ ... \\ Address line}
% if the names do not fit well on one line use
%         Author 1 \\ {\bf Author 2} \\ ... \\ {\bf Author n} \\
% For authors from different institutions:
% \author{Author 1 \\ Address line \\  ... \\ Address line
%         \And  ... \And
%         Author n \\ Address line \\ ... \\ Address line}
% To start a separate ``row'' of authors use \AND, as in
% \author{Author 1 \\ Address line \\  ... \\ Address line
%         \AND
%         Author 2 \\ Address line \\ ... \\ Address line \And
%         Author 3 \\ Address line \\ ... \\ Address line}

% \author{First Author \\
%   Affiliation / Address line 1 \\
%   Affiliation / Address line 2 \\
%   Affiliation / Address line 3 \\
%   \texttt{email@domain} \\\And
%   Second Author \\
%   Affiliation / Address line 1 \\
%   Affiliation / Address line 2 \\
%   Affiliation / Address line 3 \\
%   \texttt{email@domain} \\}

\author{Yuxia Wang\textsuperscript{1},
Rui Xing\textsuperscript{1}, 
Jonibek Mansurov\textsuperscript{1},
Giovanni Puccetti\textsuperscript{5}, 
\\\bf
Zhuohan Xie\textsuperscript{1}, 
Minh Ngoc Ta\textsuperscript{9}, 
Jiahui Geng\textsuperscript{1}, 
Jinyan Su\textsuperscript{1, 10}, 
Mervat Abassy\textsuperscript{13},
\\\bf
Saad El Dine Ahmed\textsuperscript{13}, 
Kareem Elozeiri\textsuperscript{11}, 
Nurkhan Laiyk\textsuperscript{1}, 
Maiya Goloburda\textsuperscript{1}, 
\\\bf
Tarek Mahmoud\textsuperscript{1},
Raj Vardhan Tomar\textsuperscript{14},
Alexander Aziz\textsuperscript{15}, 
Ryuto Koike\textsuperscript{7},
\\\bf
Masahiro Kaneko\textsuperscript{1,7}, 
Artem Shelmanov\textsuperscript{1},
Ekaterina Artemova\textsuperscript{6}, 
Vladislav Mikhailov\textsuperscript{4}, 
\\\bf
Akim Tsvigun\textsuperscript{2,3}, 
Alham Fikri Aji\textsuperscript{1}, 
Nizar Habash\textsuperscript{1,8}, 
Iryna Gurevych\textsuperscript{1,12}, 
Preslav Nakov\textsuperscript{1}
\\
\textsuperscript{1}MBZUAI \quad
\textsuperscript{2}Nebius AI \quad 
\textsuperscript{3}KU Leuven \quad
\textsuperscript{4}University of Oslo 
% \textsuperscript{5}Institute of Information Science and Technologies \quad
\textsuperscript{5}ISTI-CNR\quad
\textsuperscript{6}Toloka AI
\\
\textsuperscript{7}Institute of Science Tokyo \quad
\textsuperscript{8}New York University Abu Dhabi
\\
% \textsuperscript{9}BKAI Research Center, Hanoi University of Science and Technology \quad
\textsuperscript{9}BKAI Research Center, Hanoi University of Science and Technology \quad
\textsuperscript{10}Cornell University
\\
\textsuperscript{11}Zewail City of Science and Technology \quad
\textsuperscript{12}TU Darmstadt \quad
\textsuperscript{13}Alexandria University
\\
\textsuperscript{14}Cluster Innovation Center, University of Delhi \quad
\textsuperscript{15}University of Florida
\\
\href{yuxia.wang@mbzuai.ac.ae}{\{yuxia.wang, preslav.nakov\}@mbzuai.ac.ae}
}


\begin{document}
\selectlanguage{english}
\maketitle

\begin{abstract}
Prior studies have shown that distinguishing text generated by large language models (LLMs) from human-written one is highly challenging, and often no better than random guessing. To verify the generalizability of this finding across languages and domains, we perform an extensive case study to identify the upper bound of human detection accuracy. Across 16 datasets covering 9 languages and 9 domains, 19 annotators achieved an average detection accuracy of 87.6\%, thus challenging previous conclusions. We find that major gaps between human and machine text lie in concreteness, cultural nuances, and diversity. Prompting by explicitly explaining the distinctions in the prompts can partially bridge the gaps in over 50\% of the cases. However, we also find that humans do not always prefer human-written text, particularly when they cannot clearly identify its source. 
% We release our dataset, the human labels, and the annotator metadata at \texttt{http://URL.withheld.for.review}.

\end{list}
\end{abstract}


\section{Introduction}
Backdoor attacks pose a concealed yet profound security risk to machine learning (ML) models, for which the adversaries can inject a stealth backdoor into the model during training, enabling them to illicitly control the model's output upon encountering predefined inputs. These attacks can even occur without the knowledge of developers or end-users, thereby undermining the trust in ML systems. As ML becomes more deeply embedded in critical sectors like finance, healthcare, and autonomous driving \citep{he2016deep, liu2020computing, tournier2019mrtrix3, adjabi2020past}, the potential damage from backdoor attacks grows, underscoring the emergency for developing robust defense mechanisms against backdoor attacks.

To address the threat of backdoor attacks, researchers have developed a variety of strategies \cite{liu2018fine,wu2021adversarial,wang2019neural,zeng2022adversarial,zhu2023neural,Zhu_2023_ICCV, wei2024shared,wei2024d3}, aimed at purifying backdoors within victim models. These methods are designed to integrate with current deployment workflows seamlessly and have demonstrated significant success in mitigating the effects of backdoor triggers \cite{wubackdoorbench, wu2023defenses, wu2024backdoorbench,dunnett2024countering}.  However, most state-of-the-art (SOTA) backdoor purification methods operate under the assumption that a small clean dataset, often referred to as \textbf{auxiliary dataset}, is available for purification. Such an assumption poses practical challenges, especially in scenarios where data is scarce. To tackle this challenge, efforts have been made to reduce the size of the required auxiliary dataset~\cite{chai2022oneshot,li2023reconstructive, Zhu_2023_ICCV} and even explore dataset-free purification techniques~\cite{zheng2022data,hong2023revisiting,lin2024fusing}. Although these approaches offer some improvements, recent evaluations \cite{dunnett2024countering, wu2024backdoorbench} continue to highlight the importance of sufficient auxiliary data for achieving robust defenses against backdoor attacks.

While significant progress has been made in reducing the size of auxiliary datasets, an equally critical yet underexplored question remains: \emph{how does the nature of the auxiliary dataset affect purification effectiveness?} In  real-world  applications, auxiliary datasets can vary widely, encompassing in-distribution data, synthetic data, or external data from different sources. Understanding how each type of auxiliary dataset influences the purification effectiveness is vital for selecting or constructing the most suitable auxiliary dataset and the corresponding technique. For instance, when multiple datasets are available, understanding how different datasets contribute to purification can guide defenders in selecting or crafting the most appropriate dataset. Conversely, when only limited auxiliary data is accessible, knowing which purification technique works best under those constraints is critical. Therefore, there is an urgent need for a thorough investigation into the impact of auxiliary datasets on purification effectiveness to guide defenders in  enhancing the security of ML systems. 

In this paper, we systematically investigate the critical role of auxiliary datasets in backdoor purification, aiming to bridge the gap between idealized and practical purification scenarios.  Specifically, we first construct a diverse set of auxiliary datasets to emulate real-world conditions, as summarized in Table~\ref{overall}. These datasets include in-distribution data, synthetic data, and external data from other sources. Through an evaluation of SOTA backdoor purification methods across these datasets, we uncover several critical insights: \textbf{1)} In-distribution datasets, particularly those carefully filtered from the original training data of the victim model, effectively preserve the model’s utility for its intended tasks but may fall short in eliminating backdoors. \textbf{2)} Incorporating OOD datasets can help the model forget backdoors but also bring the risk of forgetting critical learned knowledge, significantly degrading its overall performance. Building on these findings, we propose Guided Input Calibration (GIC), a novel technique that enhances backdoor purification by adaptively transforming auxiliary data to better align with the victim model’s learned representations. By leveraging the victim model itself to guide this transformation, GIC optimizes the purification process, striking a balance between preserving model utility and mitigating backdoor threats. Extensive experiments demonstrate that GIC significantly improves the effectiveness of backdoor purification across diverse auxiliary datasets, providing a practical and robust defense solution.

Our main contributions are threefold:
\textbf{1) Impact analysis of auxiliary datasets:} We take the \textbf{first step}  in systematically investigating how different types of auxiliary datasets influence backdoor purification effectiveness. Our findings provide novel insights and serve as a foundation for future research on optimizing dataset selection and construction for enhanced backdoor defense.
%
\textbf{2) Compilation and evaluation of diverse auxiliary datasets:}  We have compiled and rigorously evaluated a diverse set of auxiliary datasets using SOTA purification methods, making our datasets and code publicly available to facilitate and support future research on practical backdoor defense strategies.
%
\textbf{3) Introduction of GIC:} We introduce GIC, the \textbf{first} dedicated solution designed to align auxiliary datasets with the model’s learned representations, significantly enhancing backdoor mitigation across various dataset types. Our approach sets a new benchmark for practical and effective backdoor defense.



% \section{Related Work}
\label{sec:relatedwork}

\subsection{Current AI Tools for Social Service}
\label{subsec:relatedtools}
% the title I feel is quite broad

Harnessing technology for social good has always been a grand challenge in social service \cite{berzin_practice_2015}. As early as the 90s, artificial neural networks and predictive models have been employed as tools for risk assessments, decision-making, and workload management in sectors like child protective services and mental health treatment \cite{fluke_artificial_1989, patterson_application_1999}. The recent rise of generative AI is poised to further advance social service practice, facilitating the automation of administrative tasks, streamlining of paperwork and documentation, optimisation of resource allocation, data analysis, and enhancing client support and interventions \cite{fernando_integration_2023, perron_generative_2023}.

Today, AI solutions are increasingly being deployed in both policy and practice \cite{goldkind_social_2021, hodgson_problematising_2022}. In clinical social work, AI has been used for risk assessments, crisis management, public health initiatives, and education and training for practitioners \cite{asakura_call_2020, gillingham2019can, jacobi_functions_2023, liedgren_use_2016, molala_social_2023, rice_piloting_2018, tambe_artificial_2018}. AI has also been employed for mental health support and therapeutic interventions, with conversational agents serving as on-demand virtual counsellors to provide clinical care and support \cite{lisetti_i_2013, reamer_artificial_2023}.
% commercial solutions include Woebot, which simulates therapeutic conversation, and Wysa, an “emotionally intelligent” AI coach, powered by evidenced-based clinical techniques \cite{reamer_artificial_2023}. 
% Non-clinical AI agents like Replika and companion robots can also provide social support and reduce loneliness amongst individuals \cite{ahmed_humanrobot_2024, chaturvedi_social_2023, pani_can_2024, ta_user_2020}.

Present research largely focuses on \textit{\textbf{AI-based decision support tools}} in social service \cite{james_algorithmic_2023, kawakami2022improving}, especially predictive risk models (PRMs) used to predict social service risks and outcomes \cite{gillingham2019can, van2017predicting}, like the Allegheny Family Screening Tool (AFST), which assesses child abuse risk using data from US public systems \cite{chouldechova_case_2018, vaithianathan2017developing}. Elsewhere, researchers have also piloted PRMs to predict social service needs for the homeless using Medicaid data\cite{erickson_automatic_2018, pourat_easy_2023}, and AI-powered algorithms to promote health interventions for at-risk populations, such as HIV testing among Californian homeless \cite{rice_piloting_2018, yadav_maximizing_2017}.

\subsection{Generative AI and Human-AI Collaboration}
\label{subsec:relatedworkhaicollaboration}
Beyond decision-making algorithms and PRMs, advancements in generative AI, such as large language models (LLMs), open new possibilities for human-AI (HAI) collaboration in social services. 
LLMs have been called "revolutionary" \cite{fui2023generative} and a "seismic shift" \cite{cooper2023examining}, offering "content support" \cite{memmert2023towards} by generating realistic and coherent responses to user inputs \cite{cascella2023evaluating}. Their vastly improved capabilities and ubiquity \cite{cooper2023examining} makes them poised to revolutionise work patterns \cite{fui2023generative}. Generative AI is already used in fields like design, writing, music, \cite{han2024teams, suh2021ai, verheijden2023collaborative, dhillon2024shaping, gero2023social} healthcare, and clinical settings \cite{zhang2023generative, yu2023leveraging, biswas2024intelligent}, with promising results. However, the social service sector has been slower in adopting AI \cite{diez2023artificial, kawakami2023training}.

% Yet, the social service sector is one that could perhaps stand to gain the most from AI technologies. As Goldkind \cite{goldkind_social_2021} writes, social service, as a "values-centred profession with a robust code of ethics" (p. 372), is uniquely placed to inform the development of thoughtful algorithmic policy and practice. 
Social service, however, stands to benefit immensely from generative AI. SSPs work in time-poor environments \cite{tiah_can_2024}, often overwhelmed with tedious administrative work \cite{meilvang_working_2023} and large amounts of paperwork and data processing \cite{singer_ai_2023, tiah_can_2024}. 
% As such, workers often work in time-poor environments and are burdened with information overload and administrative tasks \cite{tiah_can_2024, meilvang_working_2023}. 
Generative AI is well-placed to streamline and automate tasks like formatting case notes, formulating treatment plans and writing progress reports, which can free up valuable time for more meaningful work like client engagement and enhance service quality \cite{fernando_integration_2023, perron_generative_2023, tiah_can_2024, thesocialworkaimentor_ai_nodate}. 

Given the immense potential, there has been emerging research interest in HAI collaboration and teamwork in the Human-Computer Interaction and Computer Supported Cooperative Work space \cite{wang_human-human_2020}. HAI collaboration and interaction has been postulated by researchers to contribute to new forms of HAI symbiosis and augmented intelligence, where algorithmic and human agents work in tandem with one another to perform tasks better than they could accomplish alone by augmenting each other's strengths and capabilities  \cite{dave_augmented_2023, jarrahi_artificial_2018}.

However, compared to the focus on AI decision-making and PRM tools, there is scant research on generative AI and HAI collaboration in the social service sector \cite{wykman_artificial_2023}. This study therefore seeks to fill this critical gap by exploring how SSPs use and interact with a novel generative AI tool, helping to expand our understanding of the new opportunities that HAI collaboration can bring to the social service sector.

\subsection{Challenges in AI Use in Social Service}
\label{subsec:relatedworkaiuse}

% Despite the immense potential of AI systems to augment social work practice, there are multiple challenges with integrating such systems into real-life practice. 
Despite its evident benefits, multiple challenges plague the integration of AI and its vast potential into real-life social service practice.
% Numerous studies have investigated the use of PRMs to help practitioners decide on a course of action for their clients. 
When employing algorithmic decision-making systems, practitioners often experience tension in weighing AI suggestions against their own judgement \cite{kawakami2022improving, saxena2021framework}, being uncertain of how far they should rely on the machine. 
% Despite often being instructed to use the tool as part of evaluating a client, 
Workers are often reluctant to fully embrace AI assessments due to its inability to adequately account for the full context of a case \cite{kawakami2022improving, gambrill2001need}, and lack of clarity and transparency on AI systems and limitations \cite{kawakami2022improving}. Brown et al. \cite{brown2019toward} conducted workshops using hypothetical algorithmic tools 
% to understand service providers' comfort levels with using such tools in their work,
and found similar issues with mistrust and perceived unreliability. Furthermore, introducing AI tools can  create new problems of its own, causing confusion and distrust amongst workers \cite{kawakami2022improving}. Such factors are critical barriers to the acceptance and effective use of AI in the sector.

\citeauthor{meilvang_working_2023} (2023) cites the concept of \textit{boundary work}, which explores the delineation between "monotonous" administrative labour and "professional", "knowledge based" work drawing on core competencies of SSPs. While computers have long been used for bureaucratic tasks like client registration, the introduction of decision support systems like PRMs stirred debate over AI "threatening professional discretion and, as such, the profession itself" \cite{meilvang_working_2023}. Such latent concerns arguably drive the resistance to technology adoption described above. Generative AI is only set to further push this boundary, 
% these concerns are only set to grow in tandem with the vast capabilities of generative and other modern AI systems. Compared to the relatively primitive AI systems in past years, perceived as statistical algorithms \cite{brown2019toward} turning preset inputs like client age and behavioural symptoms \cite{vaithianathan2017developing} into simple numerical outputs indicating various risk scores, modern AI systems are vastly more capable: LLMs 
with its ability to formulate detailed reports and assessments that encroach upon the "core" work of SSPs.
% accept unrestricted and unstructured inputs and return a range of verbose and detailed evaluations according to the user's instructions. 
Introducing these systems exacerbate previously-raised issues such as understanding the limitations and possibilities of AI systems \cite{kawakami2022improving} and risk of overreliance on AI \cite{van2023chatgpt}, and requires a re-examination of where users fall on the algorithmic aversion-bias scale \cite{brown2019toward} and how they detect and react to algorithmic failings \cite{de2020case}. We address these critical issues through an empirical, on-the-ground study that to our knowledge is the first of its kind since the new wave of generative AI.

% W 

% Yet, to date, we have limited knowledge on the real-world impacts and implications of human-AI collaboration, and few studies have investigated practitioners’ experiences working with and using such AI systems in practice, especially within the social work context \cite{kawakami2022improving}. A small number of studies have explored practitioner perspectives on the use of AI in social work, including Kawakami et al. \cite{kawakami2022improving}, who interviewed social workers on their experiences using the AFST; Stapleton et al. \cite{stapleton_imagining_2022}, who conducted design workshops with caseworkers on the use of PRMs in child welfare; and Wassal et al. \cite{wassal_reimagining_2024}, who interviewed UK social work professionals on the use of AI. A common thread from all these studies was a general disregard for the context and users, with many practitioners criticising the failure of past AI tools arising from the lack of participation and involvement of social workers and actual users of such systems in the design and development of algorithmic systems \cite{wassal_reimagining_2024}. Similarly, in a scoping review done on decision-support algorithms in social work, Jacobi \& Christensen \cite{jacobi_functions_2023} reported that the majority of studies reveal limited bottom-up involvement and interaction between social workers, researchers and developers, and that algorithms were rarely developed with consideration of the perspective of social workers.
% so the \cite{yang_unremarkable_2019} and \cite{holten_moller_shifting_2020} are not real-world impacts? real-world means to hear practitioner's voice? I feel this is quite important but i didnt get this point in intro!

% why mentioning 'which have largely focused on existing ADS tools (e.g., AFST)'? i can see our strength is more localized, but without basic knowledge of social work i didnt get what's the 'departure' here orz
% the paragraph is great! do we need to also add one in line 20 21?

\subsection{Designing AI for Social Service through Participatory Design}
\label{subsec:relatedworkpd}
% i think it's important! but maybe not a whole subsection? but i feel the strong connection with practitioners is indeed one of our novelties and need to highlight it, also in intro maybe
% Participatory design (PD) has long been used extensively in HCI \cite{muller1993participatory}, to both design effective solutions for a specific community and gain a deep understanding of that community. Of particular interest here is the rich body of literature on PD in the field of healthcare \cite{donetto2015experience}, which in this regard shares many similarities and concerns with social work. PD has created effective health improvement apps \cite{ryu2017impact}, 

% PD offers researchers the chance to gather detailed user requirements \cite{ryu2017impact}...

Participatory design (PD) is a staple of HCI research \cite{muller1993participatory}, facilitating the design of effective solutions for a specific community while gaining a deep understanding of its stakeholders. The focus in PD of valuing the opinions and perspectives of users as experts \cite{schuler_participatory_1993} 
% In recent years, the tech and social work sectors have awakened to the importance of involving real users in designing and implementing digital technologies, developing human-centred design processes to iteratively design products or technologies through user feedback 
has gained importance in recent years \cite{storer2023reimagining}. Responding to criticisms and failures of past AI tools that have been implemented without adequate involvement and input from actual users, HCI scholars have adopted PD approaches to design predictive tools to better support human decision-making \cite{lehtiniemi_contextual_2023}.
% ; accordingly, in social service, a line of research has begun studying and designing for human-AI collaboration with real-world users (e.g. \cite{holten_moller_shifting_2020, kawakami2022improving, yang_unremarkable_2019}).
Section \ref{subsec:relatedworkaiuse} shows a clear need to better understand SSP perspectives when designing and implementing AI tools in the social sector. 
Yet, PD research in this area has been limited. \citeauthor{yang2019unremarkable} (2019), through field evaluation with clinicians, investigated reasons behind the failure of previous AI-powered decision support tools, allowing them to design a new-and-improved AI decision-support tool that was better aligned with healthcare workers’ workflows. Similarly, \citeauthor{holten_moller_shifting_2020} (2020) ran PD workshops with caseworkers, data scientists and developers in public service systems to identify the expectations and needs that different stakeholders had in using ADS tools.

% Indeed, it is as Wise \cite{wise_intelligent_1998} noted so many years ago on the rise of intelligent agents: “it is perhaps when technologies are new, when their (and our) movements, habits and attitudes seem most awkward and therefore still at the forefront of our thoughts that they are easiest to analyse” (p. 411). 
Building upon this existing body of work, we thus conduct a study to co-design an AI tool \textit{for} and \textit{with} SSPs through participatory workshops and focus group discussions. In the process, we revisit many of the issues mentioned in Section \ref{subsec:relatedworkaiuse}, but in the context of novel generative AI systems, which are fundamentally different from most historical examples of automation technologies \cite{noy2023experimental}. This valuable empirical inquiry occurs at an opportune time when varied expectations about this nascent technology abound \cite{lehtiniemi_contextual_2023}, allowing us to understand how SSPs incorporate AI into their practice, and what AI can (or cannot) do for them. In doing so, we aim to uncover new theoretical and practical insights on what AI can bring to the social service sector, and formulate design implications for developing AI technologies that SSPs find truly meaningful and useful.
% , and drive future technological innovations to transform the social service sector not just within [our country], but also on a global scale.

 % with an on-the-ground study using a real prototype system that reflects the state of AI in current society. With the presumption that AI will continue to be used in social work given the great benefits it brings, we address the pressing need to investigate these issues to ensure that any potential AI systems are designed and implemented in a responsible and effective manner.

% Building upon these works, this study therefore seeks to adopt a participatory design methodology to investigate social workers’ perspectives and attitudes on AI and human-AI collaboration in their social work practice, thus contributing to the nascent body of practitioner-centred HCI research on the use of AI in social work. Yet, in a departure from prior work, which have largely focused on existing ADS tools (e.g., AFST) and were situated in a Western context, our paper also aims to expand the scope by piloting a novel generative AI tool that was designed and developed by the researchers in partnership with a social service agency based in Singapore, with aims of generating more insights on wider use cases of AI beyond what has been previously studied.

% i may think 'While the current lacunae of research on applications of AI in social work may appear to be a limitation, it simultaneously presents an exciting opportunity for further research and exploration \cite{dey_unleashing_2023},' this point is already convincing enough, not sure if we need to quote here
% I like this end! it's a good transition to our study design, do we need to mention the localization in intro as well? like we target at singapore

% Given the increasing prominence and acceptance of AI in modern society, 

% These increased capabilities vastly exacerbate the issues already present with a simpler tool like the AFST: the boundaries and limitations of an LLM system are significantly more difficult to understand and its possible use cases are exponentially greater in scope. 

% Put this in discussion section instead?
% Kawakami et al's work "highlights the importance of studying how collaborative decision-making... impacts how people rely upon and make sense of AI models," They conclude by recommending designing tools that "support workers in understanding the boundaries of [an AI system's] capabilities", and implementing design procedures that "support open cultures for critical discussion around AI decision making". The authors outline critical challenges of implementing AI systems, elucidating factors that may hinder their effectiveness and even negatively affect operations within the organisation.


% Is this needed?:
% talk about the strengths of PD in eliciting user viewpoints and knowledge, in particular when it is a field that is novel or where a certain system has not been used or developed or tested before
\section{Case Study}
\label{sec:data-annotation}

Previous studies presented the difficulty humans face in distinguishing SOTA LLM-generated content from human-written text, often resulting in a random guess (see more in \appref{sec:relatedwork}). However, most evaluations focused on English and generations by GPT-3.5-turbo, leaving the detectability of MGT in other languages and LLMs uncertain.

To verify whether this observation can be generalized to other languages and more advanced LLMs, we collected LLM generations based on 16 datasets across nine domains and nine languages. 19 native speakers who are either LLM researchers or practitioners performed human evaluations, investigating (\emph{i})~whether humans can correctly discern human vs. AI outputs, and (\emph{ii})~whether humans prefer fellow human answers or LLM responses.


% \begin{table*}[t!]
%     \centering
%     \small
%     \resizebox{\textwidth}{!}{
%     \begin{tabular}{lllr|ccccccccr}
%     \toprule
%     \textbf{Language} & \textbf{Source/} & \textbf{Data} &  \textbf{Total} & \multicolumn{9}{c}{\textbf{Sampled Parallel Data}}  \\
%                       & \textbf{Domain} & \textbf{License} &\textbf{Human} & \textbf{Human} & \textbf{GPT-4o} & \textbf{Claude} & \textbf{Vikhr-Nemo-12B} & \textbf{Llama3-405B} & \textbf{ChatGLM4} & \textbf{Qwen2} & \textbf{Qwen-turbo} & \textbf{Total}  \\
%     \midrule
%     \multirow{4}{*}{Arabic} & Dialect Tweet &  Apache 2.0 &  1400 & 300 & 300* & & & & &300* & &900 \\
%     & ESAC & cc-by-sa-3.0 & 765 & 153 & 153 & & & & & & & 306 \\
%     & Youm7 News & --- & 21,000 & 1,000 & 1,000 & & & & & & & 3,000 \\
%     & SANAD & cc-by-4.0 & 194,797 & 100 & 100 & -- & -- & -- & -- & -- & -- & 200 \\
%     \midrule
%     \multirow{4}{*}{Chinese} & Zhihu-QA & cc-by-4.0 & 224761 & 588 & 588 & &&&&& 588 & 1,764 \\
%                              & Student essay & cc-by-4.0 & 93,002 & 600 &  & 300* & && 300* & & & 1,200 \\
%                              & Student essay & cc-by-4.0 & 51 & 51 & & 51 & && 51 & & & 153 \\
%                              & Government Report & \\
%     \midrule
%     English & Peersum~\citep{peersum_2023} & cc-by-sa-4.0 &  5158 & 400 & 200 & 200  & &&&& & 800\\
%     \midrule
%     Hindi & News & cc-by-4.0 & 3,995 & 600 & 600 & &&&&&& 1200 \\
%     \midrule
%     Italian & DICE & cc-by-sa &  10,518 & 300 & 300 & & & 300 & & & & 900\\
%     % & CItA & & 1352 & 300 & & & & 300 & & & \\
%     \midrule
%     Japanese & News & cc-by-nc-sa-4.0 & 7,110 & 300 & 300 & &&&&&& 600 \\
%      \midrule
%     Kazakh & Wikipedia & cc-by-sa-4.0 &  4,827 & 300 & 300  & &&&&&& 600\\
%     \midrule
%     \multirow{2}{*}{Russian} 
%     & News & MIT & 800,000 & 300 & 300 & & 300  &  & & & & 900 \\
%     & Academic summary & MIT & 31,000 & 300 & 300 & & 300 &  & & & & 900 \\
%     \midrule
%     \multirow{2}{*}{Vietnamese} 
%     & Wikipedia & --- & 600 & 600 & 600 & & &  & & & & 1,200 \\
%     & News & --- & 290,282 & 600 & 600 & & &  & & & & 1,200 \\
%     \midrule
%     \bf Total & -- & -- &  \\
%     \bottomrule
%     \end{tabular}
%    }
%     \caption{Statistics of multilingual data for human annotation. Machine data with * means non-parallel data. }
%     \label{tab:multilingual-data}
% \end{table*}



\subsection{Task and Dataset}
% 1. original dataset (license, size, domain, topic); 
% 2. how we did sampling and explain why (size and topic); 
% 3. how we did machine-generation (model, prompt, model generation configuration)

\paragraph{MGT detection} The goal is to identify whether the text was written by a human or generated by models given a single text, or to recognize which text is written by a human given a pair of texts: one human-written and one machine-generated.

In data collection, we focused on datasets from common domains such as community QA, news, tweets, and government reports, alongside domains requiring high-integrity LLM applications, including educational and academic contexts, such as accurate knowledge verification in Wikipedia-style texts, and identifying the authorship of student essays and peer reviews. 
For a given language and given a dataset, we sampled 300-600 human texts and then generated corresponding machine text using two SOTA LLMs: a multilingual model (e.g., from the GPT or the Claude series) and a language-specific model (e.g., ChatGLM or Qwen for Chinese and AceGPT for Arabic), to analyze the impact of different LLMs on detection performance, particularly for non-English languages.
% 
As shown in \tabref{tab:multilingual-data}, we collected data based on 16 datasets across nine languages. The generation prompts and collection details are shown in \appref{sec:datasets}. % Note that we generated more data and used the subset for human annotation.

% Yuxia Steps:
% 1. check original dataset section, organize into a appendix section
% 2. list what are empty and ask them to fill
% 3. check where is the collected data, and where is the evaluation results


\subsection{Human Detection Setups}
\label{sec:anno-setup}

\paragraph{Annotation Settings}
To mimic real-world machine-generated text detection scenarios, we set up four human evaluation settings.
Given human-written text and machine-generated text, representing by $hwt$ and $mgt$ respectively, human annotators are asked to identify which text was written by a human. Note that $mgt$ can be generated by multiple different LLMs, referring to as $mgt_i$, where $i \in [1,2, \cdots, n]$.

As shown in \tabref{tab:detection-settings}, according to the input and the output options, we categorize detection settings as I. \emph{pair-binary}, II. \emph{pair-four-class}, III. \emph{single-binary}, and IV. \emph{triplet-three-class}. 
For single text input, either $hwt$ or $mgt$, the goal is to recognize whether the text was written by a human, by answering just Yes or No. This is suitable for the scenario where for the human text there is no necessarily a corresponding machine-generated text, and thus they can be collected from different sources and for different topics, such as Arabic tweets.
Given a pair of texts (text1, text2), a binary output is easier than the four-class detection. The \emph{pair-binary} setting asks that either text1 or text2 is $hwt$, and the other one is $mgt$, while the \emph{pair-four-class} setting has no restrictions: each of text1 and text2 can be $hwt$ or $mgt$, regardless of the label of the other text. 
Sometimes, we want to compare human text to generations from different LLMs, in which case, we apply IV, which we limit to a three-class detection: human vs. LLM$_1$ vs. LLM$_2$.

Overall, I and IV are suitable for scenarios where there is a human text and its corresponding machine-generated text. II and III can be used if there are non-corresponding human-written and machine-generated texts.
% Settings I and III are commonly used in other studies, as well as this work.
If the annotators have seen some human-written and some machine-generated text before detection, we refer to this as a few-shot setting; otherwise, we have zero-shot.


\begin{table*}[t!]
\centering
%\tabcolsep3pt
\resizebox{\textwidth}{!}{\small
\begin{tabular}{l p{4cm} p{8cm} p{4cm} p{4cm}}
\toprule
\textbf{Setting ID} & \textbf{Input} & \textbf{Task Description} & \textbf{Output Options} & \textbf{Applicable Scenarios} \\
    \midrule 
    \multirow{3}{3cm}{\textbf{I. Pair-Binary}} & ($hwt$, $mgt$) \textbf{or} ($mgt$, $hwt$) & Given a pair of (text1, text2), identify which one is human-written? Either text1 or text2 must be $hwt$, and another is $mgt$ randomly sampled from $mgt_i$. & \textbf{A.} text1; \textbf{B.} text2 & parallel data is available. \\
    \midrule
    \multirow{2}{3cm}{\textbf{II. Pair-Four-Class}} & ($hwt$, $mgt$) \textbf{or} ($mgt$, $hwt$) \textbf{or} ($hwt$, $hwt$) \textbf{or} ($mgt$, $mgt$) & Given a pair of (text1, text2), identify which one is human-written? Both text1 and text2 can be $hwt$, and can be $mgt$. & \textbf{A.} text1; \textbf{B.} text2; \textbf{C.} none of them; \textbf{D.} both & parallel data is not necessary. \\
    \midrule
    \multirow{1}{3cm}{\textbf{III. Single-Binary}} & $hwt$ or $mgt$ & Given a piece of text, identify whether it is written by human? & \textbf{A.} Yes, human; \textbf{B.} No, machine & parallel data is not necessary. \\
    \midrule
    \multirow{3}{3.5cm}{\textbf{IV. Triplet-Three-Class}} & ($hwt$, $mgt_1$, $mgt_2$)  & Given a set of texts (text1, text2, text3), identify which one is human-written? One of the text1, text2 and text3 must be $hwt$, and others are $mgt$ randomly sampled from $mgt_i$. & \textbf{A.} text1; \textbf{B.} text2; \textbf{C.} text3 & parallel human and multiple LLM generations are collected to make comparisons. \\
    \bottomrule
\end{tabular}
}
\caption{The four human detection settings: the setting name refers to input-output options, pair/binary means the input is a pair of texts and the goal is to predict binary labels, and whether the text is human-written (Yes or No).}
% I and IV are suitable for the scenarios where there is a human text and its corresponding machine text (parallel data is available). II and III can be used if parallel data is not available, there are only separate human text and machine text.
\label{tab:detection-settings}
\end{table*}



\paragraph{Annotation Tool}
% Rui: annotation pipeline we used and why we used these two: quality and efficiency
To mitigate potential labeling biases arising from raw spreadsheet annotation and to enhance efficiency, we implemented two methods with optimized interfaces and workflows for our annotation: (1)~a custom pipeline using the Google Workspace suite, including Apps Script, Google Sheets, and Google Forms. The core idea was to store all data in Google Sheets, use Apps Script to extract data and generate a survey in Google Forms; and (2)~Label Studio, an open-source multi-type data labeling and annotation tool with a standardized output format. We designed a custom template for our annotation task and collected results using this platform. The annotators were given the choice to use their preferred tool. 

% \footnote{https://developers.google.com/apps-script}, 
% \footnote{https://github.com/HumanSignal/label-studio/}  


\paragraph{Annotator Background}
In order to explore the upper bound of human detection capability, instead of using crowd-sourcing annotators, we conducted in-house labeling. 
The annotators were BSc, MSc, and PhD students, as well as postdocs, who were familiar with NLP tasks and LLM generations. All annotators independently completed their individual annotations. For each language, the annotators were all native speakers of that language. See more detail in \appref{sec:whyexperts}.

% Name, Age, Gender, degree, mother language, personality test type, 
\section{Human Detection}
\label{sec:human-detection-acc}
% Previous human detection on English machine-generated text show that it is challenging for humans to discern MGT from human text.
We performed an extensive case study on 9K instances across nine languages to verify how difficult it is for native speakers to detect AI outputs in everyday domains.
\tabref{tab:detection-acc} demonstrates that the average human detection accuracy is 87.6\%.
This reveals that this is not particularly difficult for native human experts, contrary to what previous studies have reported. Below, we zoom into the impact of various factors.

\begin{table*}[t!]
    \centering
    \small
    % \resizebox{\textwidth}{!}{
    \begin{tabular}{llr cll c}
    \toprule
    \textbf{Language} & \textbf{Source/Model} & \textbf{\#Example} & \textbf{\#Annotator} & \textbf{Anno Setup} & \textbf{Shot} & \textbf{Avg. Acc} \\
    \midrule
    \multirow{4}{*}{Arabic} 
    & Dialect Tweet & 900 & 1 & III. Single-binary & Zero & 50.1  \\
    & EASC Summary & 100 & 1 & I. Pair-binary & Zero & 82.0  \\
    & Youm7 News & 1,000 & 1 & I. Pair-binary & Zero & 92.7  \\
    & SANAD News & 100 & 1 & I. Pair-binary & Zero & 100.0  \\
    \midrule
    \multirow{6}{*}{Chinese} 
    % & Zhihu-QA (\gptfouro) & 428 & 5 & I. Pair-binary & Zero & 0.99, 0.99, 1.0, 1.0, 1.0 & 0.996 \\
    & Zhihu-QA (\gptfouro) & 428 & 5 & I. Pair-binary & Zero & 99.6 \\
    & Zhihu-QA (\gptfouro) & 160 & 1 & I. Pair-binary & Few  & 100.0  \\
    % & Zhihu-QA (\qwenturbo) & 588 & 2 & I. Pair-binary & Zero & 0.99, 0.97 & 0.98 \\
    & Zhihu-QA (\qwenturbo) & 588 & 2 & I. Pair-binary & Zero & 98.0 \\
    & Student essay & 102 & 1 & I. Pair-binary & Few  & 98.0  \\
    % & Student essay & 600 & 3 & II. Pair-four-class & Zero & 0.96, 0.96, 0.99 & 0.97 \\
    & Student essay & 600 & 3 & II. Pair-four-class & Zero & 97.0 \\
    & Government Report & 500 & 1 & IV. Triplet-three-class & Few & 97.2  \\
    \midrule
    English & Peersum & 400 & 1 & I. Pair-binary & Few & 99.8   \\
    \midrule
    Hindi & News & 600 & 1 & I. Pair-binary & Few & 85.2   \\
    \midrule
    \multirow{3}{*}{Italian} 
    & DICE News (Anita) & 300 & 1 & I. Pair-binary & Few & 88.0  \\
    & DICE News (Llama3-405B) & 300 & 1 & I. Pair-binary & Few & 99.7  \\
    & DICE News (GPT-4o) & 300 & 1 & I. Pair-binary & Few & 100.0  \\
    % & CItA & \\
    \midrule
    Japanese & News & 300 & 2 & I. Pair-binary & Zero & 86.4  \\
    \midrule
    Kazakh & Wikipedia & 300 & 2 & I. Pair-binary & Zero & 79.7   \\
    \midrule
    \multirow{2}{*}{Russian} 
    & News & 300 & 1 & I. Pair-binary & Few & 100.0   \\
    & Academic summary & 300 & 1 & III. Single-binary & Few & 80.0   \\
    \midrule
    \multirow{2}{*}{Vietnamese} 
    & Wikipedia & 600 & 1 & I. Pair-binary & Zero & 50.7  \\
    & News & 600 & 1 & I. Pair-binary & Zero & 80.3  \\
    \midrule
    \bf Total & -- & 8,778 & 30 & & & 87.6\\
    \bottomrule
    \end{tabular}
   % }
    \caption{Human annotator detection accuracy over 16 datasets and 9 languages: we have a total of 30 annotation settings and 19 unique human annotators. The average accuracy of the human expert guesses is 87.6\%.}
    \label{tab:detection-acc}
\end{table*}


\subsection{Language}
% \paragraph{Language, Domain, Generator}
Human detection accuracy exceeds 87.6\% for Chinese, English, Arabic, Italian and Russian, while it falls below this level for Vietnamese, Kazakh, Hindi, and Japanese. This discrepancy is largely due to the challenge of Wikipedia text.  

\subsection{Domain}
Wikipedia is widely used as training data for LLMs, particularly for low-resource languages, due to the scarcity of alternative datasets. Consequently, models often memorize Wikipedia content, leading to generated text that closely resembles human-written Wikipedia. 
Arabic tweets also present challenges for detection due to their short length and limited context, along with summaries, e.g., for Arabic and Russian summaries, the expert detection accuracy is about 80\%. 
This conversely highlights the ability of language models to generate high-quality human-like text in the domains of Wikipedia, tweets, and summaries. In contrast, substantial differences between machine-generated and human-written text persist in news articles, QA, student essays, and peer reviews, making them much easier to recognize for human experts. 

% \begin{table}[t!]
%     \centering
%     \resizebox{\columnwidth}{!}{
%         \begin{tabular}{lcccc}
%             \toprule
%             \textbf{Dialect} & \textbf{Human} & \textbf{\gptfouro} & \textbf{\qwentwo-7.5B} & \textbf{Overall MGT} \\
%             \midrule
%             EGY & 52.00 & 53.33 & 58.67 & 56.00 \\
%             MOR & 54.00 & 53.33 & 48.00 & 50.67 \\
%             LEV & 69.33 & 14.67 & 58.67 & 36.00 \\
%             GULF & 81.33 & 26.67 & 30.67 & 28.67 \\
%             \bottomrule
%         \end{tabular}
%     }
%     \caption{Arabic dialect tweet human detection accuracy over human vs. \gptfouro vs. \qwentwo-7.5B. Machine-generated text is harder than human text to discern. \gptfouro is harder than \qwentwo.}
%     \label{tab:arabic-dialect_tweet-accuracy}
% \end{table}


\subsection{Generator} 

It is hard to detect MGT across generators and languages.
While there are minimal differences for Chinese (accuracy is 100\% vs. 98\% for \gptfouro vs. \qwenturbo), there are sizable differences for Italian and Arabic.
Based on Italian DICE News, the same annotator detected generations by Anita (an Italian fine-tuned Llama3-8B), Llama3-405B, and \gptfouro, achieving accuracy of 88\%, 99\%, and 100\%, respectively.
Similarly, for Arabic tweets, \gptfouro's outputs are more similar to human text and thus more difficult to detect compared to those by \qwentwo, as shown in \tabref{tab:arabic-dialect_tweet-accuracy}.

\subsection{Annotation Setting}
We conducted the majority of our annotations under setting I. \emph{pair-binary}: given a pair ($hwt$, $mgt$), it asks to identify which of the two texts is human-written. 
We assumed that more complex settings would be more challenging. For instance, II. \emph{pair-four-class} should be harder, as each of the texts could be human- or machine-generated, independently of the other. %, introduces additional options.
Yet, for Chinese student essays, the performance for II does not degrade compared to I.
Similarly, for government reports in IV. \emph{triplet-three-class}, where the annotators have to select the human-written text among three candidates, there was no degradation compared to I.  

However, III. \emph{single-binary} proves to be more difficult than I. \emph{pair-binary} for both Arabic and Russian. While domain differences do exist, e.g.,~tweets vs. summary vs. news in Arabic and news vs. summary for Russian, the substantial performance gap (>20\%) can still be partially attributed to the annotation settings.
Overall, comparing the results for I. vs. III., it is easier to distinguish machine-generated content when given a comparison pair, rather than for single answer. Yet, introducing text triplets or increasing the number of machine-generated or human-written texts had minimal impact on detection performance. 
Moreover, using few shots before detection boosted the confidence of the annotators, resulting in higher accuracy compared to zero-shot.
% \footnote{Slight improvement in our results can be largely owing to expert-level annotators, given that most of them are LLM researchers.}
For datasets with high accuracy, before seeing labeled samples, the annotators found the distinction to be obvious. After seeing a few examples, the annotator was extremely confident in distinguishing human vs. machine text based on indicative features of MGT. 




\subsection{Expert Annotators}
% \paragraph{What factors may influence individuals' discerning accuracy?}
% 1. Individual personal ability: given the same language and a same text, what matters?
For the same language and dataset, individual annotator ability influences detection accuracy but not significantly. 
For instance, in Chinese Zhihu-QA (GPT-4o vs. human), five annotators achieved accuracies of 99\%, 99\%, 100\%, 100\%, and 100\%. Similarly, for Zhihu-QA (Qwen-turbo vs. human), two annotators obtained 99\% and 97\%. In student essays (II. pair-four-class), three annotators recorded accuracies of 96\%, 96\%, and 99\%.
This may also result from the bias that all annotators are native speakers and expert-level LLM practitioners or researchers. Differences between individuals are minor in their cognitive abilities, language proficiency and domain knowledge.

% a. cognitive abilities: strong analytical skills may be better at detecting inconsistencies, logical flaws, or patterns that indicate AI-generated text; and attention to details, where individuals who pay close attention to grammar, syntax, and style nuances might notice subtle signs of AI output.
% b. language proficiency: A high level of language proficiency may make it easier to spot overly formal, repetitive, or mechanically structured phrasing typical of AI outputs; and cultural nuance awareness, understanding cultural or idiomatic expressions can help identify content that lacks human-like subtlety or creativity.
% c. domain knowledge (subject matter experts are more likely to spot factual errors or lack of depth in AI-generated text) and familiarity with AI output. Individuals with experience interacting with AI models may recognize their writing patterns, such as positive tone bias.
% d. personal biases: A person’s predisposition to trust or distrust technology may influence their discernment; Beliefs about what constitutes human creativity or uniqueness may shape how someone distinguishes human from AI text.
% e. other individual traits: gender, age, ...


\subsection{Distinguishable Factors}
\label{sec:dis-factor}
We summarize five remarkable distinguishable signals between machine-generated vs. human-written text across the 16 datasets and the 9 languages; see more details in \appref{sec:distinctionfactor}. % regarding detection setting, accuracy and distinction factors 

\begin{compactitem}
% \begin{itemize}
    \item \textbf{Human text is more informative and concrete.}
    Human-written text contains concrete numbers, specific names of people or institutions, exact places or dates, URLs, and other references, while machine-generated text tends to provide generic information, with little detail to support its statements. 
    
    \item \textbf{Machine-generated text lacks regional, cultural, and religious nuances.}
    For languages such as Arabic, Japanese, Hindi, Kazakh, and Chinese, human texts reflect the cultural and the religious nuances of the language, which is not true for machine-generated text.

    \item \textbf{Human-written text varies substantially in terms of length, structure, style, and sentiment.}
    Human texts show diversity and inconsistency with large deviations in length, structure, style and emotions, while machine-generated texts tend to use a formulaic structure and neutral/positive emotion. This can be partially attributed to LLMs rigorously following instructions, and thus losing on flexibility. 
    

    \item \textbf{Machine-generated text has formatting.}
    MGTs are generally well-segmented with bullet points for better readability, while human-written texts are typically just large block of plain text, which may be due to human text collection and conversion. Moreover, machine-generated texts often use Markdown style, e.g., \texttt{**} and \texttt{\#\#\#}, while human-written texts have typos, grammatical errors, hashtags, and other social media elements.

    \item \textbf{Machine-generated text shows a mixture of other languages.} Non-English language responses often contain some English parts, which is very rare in human text.
% \end{itemize}
\end{compactitem}

\section{Can Prompting Fill in the Gap?}

% As we saw above, there are distinctions between human-written and machine-generated text. % in \secref{sec:human-detection-acc}. 
Given that LLMs can strictly follow instructions and their outputs are heavily influenced by the system and the user prompts, we investigated whether explicitly instructing LLMs to mimic human style can help narrow the gap. 
Responding to the distinguishable factors summarized in \appref{sec:distinctionfactor} for each dataset, we asked the human annotators to craft new prompts, aiming to improve the generations and to reduce the gap between human-written and LLM-generated texts. This involved trying instructions that (1) incorporate specific details and references, (2) avoid formulaic structures and formats, e.g., bullet points and Markdown, and (3) generate texts of varying length, structure, and sentiment.
\tabref{tab:ori-improved-prompts} presents the results for both the original and the improved prompts for all datasets.  


\textbf{Measurements:}
We re-generated the machine-generated parts of the text pairs, using the same models with improved prompts, and then sampled 200–600 examples from each dataset to assess whether and to what extent, the prompting strategy narrowed the gap between human-written and machine-generated texts. 
We used two approaches: (1) \textit{fill-the-gap survey}, where the original annotators evaluated whether the newly-generated text bridged the gap (Yes, No, or Partially), and (2) \textit{downstream detection}, where we compared the detection accuracy before and after applying the improved prompts. A decline in detection accuracy indicated a reduced distinction between human and machine text, making the differentiation more difficult, and further revealing that prompting was effective. Our experiments involved both human annotator evaluation and automated detection. 

\begin{figure}[t!]
    \centering
    \includegraphics[scale=0.38]{section/images/dist-stackedbar.pdf} 
    \caption{Evaluating whether the new generations fill in the gap: Yes, Partially, or No.}
    \label{fig:dist-survey}
\end{figure}

% \begin{figure*}[t!]
%     \centering
%     \includegraphics[scale=0.6]{section/images/dist-stackedbar.pdf} 
%     \caption{Distribution of survey evaluating whether the new generations fill the gap? Yes, Partially or No.}
%     \label{fig:dist-survey}
% \end{figure*}

% \begin{figure*}[t!]
%     \centering
%     \includegraphics[scale=0.6]{section/images/IAA-heatmap.pdf}
%     \caption{Three annotator agreement on Chinese essays regarding whether the improved prompts mitigate the gap between human text and machine-generated text.}
%     \label{fig:iaa-heatmap}
% \end{figure*}

% \begin{figure*}[t!]
%     \centering
%     \includegraphics[scale=0.55]{section/images/acc-bars.pdf}
%     \caption{Human detection accuracy differences on original vs. improved generations.}
%     \label{fig:acc-diff}
% \end{figure*}

% \begin{figure*}[t!]
%     \centering
%     \includegraphics[scale=0.55]{section/images/auto-acc-bars.pdf}
%     \caption{\textbf{Detection accuracy differences} of 26 automatic machine-generated text detection approaches on original vs. improved generations.}
%     \label{fig:auto-acc-diff}
% \end{figure*}

\paragraph{Fill-the-gap Survey}
The original annotators who conducted the detection on earlier generations were asked to evaluate whether the new prompts addressed the gaps for each example: \emph{Yes}, \emph{No}, and \emph{Partially}.
The distributions across the six datasets in \figref{fig:dist-survey} shows that in about 50\% of the cases, the prompt adjustments were effective to  either fully or partially mitigate the gaps. Large improvements were observed for Kazakh Wikipedia and Arabic tweets.
For the former, the revised prompt reduced repetitive sentence patterns (more diverse), but the formulaic expressions were not entirely eliminated. New outputs also included more concrete information, such as dates and names, while the inclusion of culturally-nuanced details remained challenging. The newly-generated Arabic tweets could touch on relatable human topics and express genuine emotions tied to daily experiences; however, the frequently added irrelevant hashtags at the end of the tweets made them easily identifiable as machine-generated. Moreover, the tweets often leaned on an overly optimistic tone even when negative experiences were mentioned. 

The annotators for English peer reviews noted that, despite the prompt adjustments, the models remained highly formulaic in their outputs, the length of the reviews remained relatively uniform, and the overall structure lacked variance. This may be due to the inherent nature of the peer review domain, while human reviews exhibited more variability in both length and structure, feeling more organic.
Similar issues remained for Chinese student essays (formulaic structure by using ``firstly, then, moreover, finally'' persisted) and government reports (certain repetitive phrases). 

% \textit{What remains challenging to address?}
Overall, adjusting the prompts did fill some gaps, but cultural nuances, diversity of length, structure and phrases, and sentiment adaption to scenarios remained challenging. See more in \appref{sec:fill-gap-prompts}.
Since annotators' background may influence the survey results (see \ref{app:subjectivity-test}), we conducted a second round of MGT detection on the new generations with the hop   e for more objective results.


\begin{figure}[t!]
    \centering
    \includegraphics[scale=0.4]{section/images/acc-bars.pdf}
    \caption{Human detection accuracy for the original vs. the improved generations.}
    \label{fig:acc-diff}
\end{figure}

\paragraph{MGT Detection on Improved Text}
We performed human detection on 13 datasets under the same annotation setting as described in \tabref{tab:detection-acc}, with the exact same annotators.
We observed sizable accuracy declines across all datasets in \figref{fig:acc-diff}, with average accuracy dropping to 72.5\%.
This implies that the improved machine-generated text became more similar to the human-written text, making it harder to discern and thus resulting in lower detection accuracy.


% \paragraph{Automatic Detection}
We further analyzed the impact of the improved generations on automatic detection accuracies.
% provide a quantitative analysis,
We collected a total of 17,017 texts using the original prompts and 32,487 texts using the improved prompts (detailed statistical distribution in \tabref{tab:multilingual_prompt_dist}).
We reproduced 26 MGT detection approaches presented in the COLING 2025 GenAI shared task~\cite{wang-etal-2025-genai} and evaluated them on the collected MGTs. As shown in \figref{fig:auto-acc-diff}, 19 methods exhibited lower accuracy on the newly generated texts, indicating that the texts produced using the improved prompts are more challenging to distinguish compared to those generated with the original prompts.
This suggests that prompting strategies can help bridge some gaps between machine-generated and human-written text when explicitly designed to mimic human writing style.



\begin{comment}
Interestingly, when we explicitly tell annotators that the given text is newly-generated and ask them to analyze whether the gaps have been resolved, they can clearly identified what issues are remained. When they are asked to distinguish between the improved machine-generated text and human-written text, some cases are harder for them to differentiate, leading to lower accuracy.
% We believe further research could analyze the impact of human attention in such tasks.
\end{comment}
\section{Human-Like or Liked-by-Human?}
We used the prompting strategy to bridge the gap between human-written and machine-generated text, aiming to make machine outputs more human-like. However, do humans favor human-like text? 
% The common assumption is that humans expect machines to behave more like humans, but is this hypothesis true? 
Below, we examine human preferences among four options: (1) human-written text, (2) machine-generated text using the original prompt, (3) machine-generated text using the improved prompt, or (4) none of the above. 
%, to examine which type of text is preferred? 


\begin{figure}[t!]
    \centering
    \includegraphics[scale=0.37]{section/images/zh-preference.pdf}
    \caption{Human preferences for three Chinese datasets (five annotators): QA-emo is an emotion-rich question subset of Zhihu-QA with 100 examples.}
    \label{fig:pre-zh}
\end{figure}

% \begin{figure}[t!]
%     \centering
%     \includegraphics[scale=0.50]{section/images/ru-preference.pdf}
%     \includegraphics[scale=0.45]{section/images/ar-preference.pdf}
%     \caption{Human preferences for two Russian (three annotators) and two Arabic datasets (two annotators).}
%     \label{fig:pre-ru-ar}
% \end{figure}

\begin{figure}[t!]
    \centering
    \includegraphics[scale=0.45]{section/images/ru-preference.pdf}
    \includegraphics[scale=0.4]{section/images/ar-preference.pdf}
    \caption{Human preferences for two Russian (three annotators) and two Arabic datasets (two annotators).}
    \label{fig:pre-ru-ar}
\end{figure}


\paragraph{Preference Labeling Setup:}
We labeled the preferences for three languages: Chinese, Russian, and Arabic.
For Chinese, we annotated Zhihu-QA and student essays (300 examples for each), along with 100 responses particularly for Zhihu questions, where emotional and empathetic comforts are highlighted. Five unique annotators participated, identified by \textit{nationality-gender-degree}. For example, \textit{Zh-Male-PhD} refers to a Chinese male, who is a PhD student.  
Similarly, we labeled two datasets for Russian (three annotators) and two datasets for Arabic (two annotators). 


\paragraph{Do People Always Prefer Human Text?}
The answer is \emph{No}.
Analyzing the preferences of ten annotators across six datasets in \figref{fig:pre-zh} and \ref{fig:pre-ru-ar}, human text was preferred in about half of the cases. Notably, for Russian and Arabic, annotators tended to favor machine text. 
This is particularly evident for Russian summaries using the improved prompts (green bars) and Arabic summaries using the original prompts (orange bars).
% This echos the finding in detection that machine-produced summary reaches human-expert quality, closely resembling human-written ones and making differentiation challenging.


For Chinese datasets, including Zhihu QA and student essays, human-written text is generally preferred though there are exceptions.
% For instance, 
\textit{Zh-Male-postdoc} exhibits a unique preference distribution that deviates from the rest.
% \footnote{This difference is not due to inattentive annotation but rather reflects a unique philosophy of the annotator.} %  across various judgments
Interestingly, for emotion-rich questions (QA-emo), where human responses are expected to be more empathetic and be preferred, three out of four annotators actually prefered the machine-generated responses. The remaining annotator disliked both the human and the machine text in 22\% of the cases. The annotator feedback suggested that this was influenced by the presence of mean-spirited responses from Zhihu users, where some human answers expressed personal biases and lacked empathy (see \ref{app:humanpreference}).
% \ref{sec:chinese-distinguishable-signal}).


\paragraph{Why are Human-Written Essays Favored?}
% \textbf{Student Essay:}
% Essays written by humans are more coherent and sincere.
% For model essays, the coherence between sections is poor, presenting hard boundary between independent sections, sometimes repeating the title to make it superficially coherent. 
% Model tends to tell concepts, related concepts. It is hard for models to tell a good story, or connect several stories naturally with the same theme, and then return back to the main topic.
% Therefore, articles by models always lack substances, full of empty concepts and rhetoric, coming across as superficial and ungrounded, overly wordy but fails to deliver any real insights. They feel grandiose but devoid of meaningful content. Model tends to play a role of teacher, rather than a peer. Some article are written like answering questions. % (id 26)

Human-written essays exhibit greater coherence and sincerity. In contrast, machine-generated essays often lack cohesion, displaying abrupt transitions and sometimes repeating the title to create superficial continuity. While LLMs can generate related concepts, they struggle to construct a compelling narrative or to seamlessly connect multiple stories under a unified theme. As a result, their outputs often lack depth, relying on abstract concepts and rhetorical flourishes without delivering substantive insights. The text tends to be verbose yet superficial, giving an impression of grandiosity without meaning. Moreover, LLM-generated essays often adopt an instructive tone, resembling answers to questions rather than peer-level discourse.


% Human: 更具有文学气息,委婉的优美感,氛围感
% 好的文章,或生动可爱俏皮之感,或立意新奇,启发人深思,或首尾巧妙呼应,或真诚质朴,真情实感的动情,或优美凄婉,用词清新准确,情趣意境佳
From a literary perspective, a well-written human article may be lively, charming, and playful, or it may present a novel perspective that inspires deep reflection. It might feature a cleverly structured beginning and ending, convey sincerity and heartfelt emotions, or captivate with its elegance and melancholy. With fresh and precise wording, it creates a rich atmosphere, evoking a refined aesthetic and a sense of poetic depth.  
% Model: 没有值得回味的感觉,一眼就可以看清楚要说什么,不需要太多思考和回味,说教式议论文
In contrast, a piece generated by an LLM lacks this literary nuance, leaving little room for contemplation or lingering thoughts. The expression is immediately clear, requiring no further reflection, resembling more of a lecture than an engaging discourse.


\paragraph{Preference Distributions Vary Across Individuals.}
The annotators for Russian and Arabic exhibited similar preference distributions, whereas large differences occured for Chinese QA.
For example, in the Chinese Zhihu QA, the second annotator selected only seven human-written texts, while the fourth one chose 284 (95\%). In the QA-emo, the first and the second annotators prefered human-written responses, whereas the third and the fourth favored machine-generated text.
This variance shows the charm of collecting personal preferences and then optimizing models to align with individual philosophies. 
Our preference annotations can serve as a valuable resource for investigating the relationship between annotator characteristics (e.g., MBTI personality, gender, and age) and preferences. Also the data can guide models to match individual preferences in multilingual contexts.
% \footnote{We will release all annotator metadata to facilitate future research of the community.}



\paragraph{Human-Like or Liked-by-Human?}
Human texts are not always preferred. This inspires us to reflect the ultimate goal of building LLMs that are human-like vs. liked-by-human.
The goal of being human-like has a single target, i.e.,~mimicking human behavior, while to be liked-by-humans involves optimization towards billions of local optima, each shaped by individual preferences.  
Current language models establish a foundation by learning from human data towards being human-like. 
As they get more advanced, they can further adapt by incorporating personal data, thus transitioning from merely imitating human behavior to aligning with individual user preferences, and moving from human-like to liked-by-human.
\section{Discussion}\label{sec:discussion}



\subsection{From Interactive Prompting to Interactive Multi-modal Prompting}
The rapid advancements of large pre-trained generative models including large language models and text-to-image generation models, have inspired many HCI researchers to develop interactive tools to support users in crafting appropriate prompts.
% Studies on this topic in last two years' HCI conferences are predominantly focused on helping users refine single-modality textual prompts.
Many previous studies are focused on helping users refine single-modality textual prompts.
However, for many real-world applications concerning data beyond text modality, such as multi-modal AI and embodied intelligence, information from other modalities is essential in constructing sophisticated multi-modal prompts that fully convey users' instruction.
This demand inspires some researchers to develop multimodal prompting interactions to facilitate generation tasks ranging from visual modality image generation~\cite{wang2024promptcharm, promptpaint} to textual modality story generation~\cite{chung2022tale}.
% Some previous studies contributed relevant findings on this topic. 
Specifically, for the image generation task, recent studies have contributed some relevant findings on multi-modal prompting.
For example, PromptCharm~\cite{wang2024promptcharm} discovers the importance of multimodal feedback in refining initial text-based prompting in diffusion models.
However, the multi-modal interactions in PromptCharm are mainly focused on the feedback empowered the inpainting function, instead of supporting initial multimodal sketch-prompt control. 

\begin{figure*}[t]
    \centering
    \includegraphics[width=0.9\textwidth]{src/img/novice_expert.pdf}
    \vspace{-2mm}
    \caption{The comparison between novice and expert participants in painting reveals that experts produce more accurate and fine-grained sketches, resulting in closer alignment with reference images in close-ended tasks. Conversely, in open-ended tasks, expert fine-grained strokes fail to generate precise results due to \tool's lack of control at the thin stroke level.}
    \Description{The comparison between novice and expert participants in painting reveals that experts produce more accurate and fine-grained sketches, resulting in closer alignment with reference images in close-ended tasks. Novice users create rougher sketches with less accuracy in shape. Conversely, in open-ended tasks, expert fine-grained strokes fail to generate precise results due to \tool's lack of control at the thin stroke level, while novice users' broader strokes yield results more aligned with their sketches.}
    \label{fig:novice_expert}
    % \vspace{-3mm}
\end{figure*}


% In particular, in the initial control input, users are unable to explicitly specify multi-modal generation intents.
In another example, PromptPaint~\cite{promptpaint} stresses the importance of paint-medium-like interactions and introduces Prompt stencil functions that allow users to perform fine-grained controls with localized image generation. 
However, insufficient spatial control (\eg, PromptPaint only allows for single-object prompt stencil at a time) and unstable models can still leave some users feeling the uncertainty of AI and a varying degree of ownership of the generated artwork~\cite{promptpaint}.
% As a result, the gap between intuitive multi-modal or paint-medium-like control and the current prompting interface still exists, which requires further research on multi-modal prompting interactions.
From this perspective, our work seeks to further enhance multi-object spatial-semantic prompting control by users' natural sketching.
However, there are still some challenges to be resolved, such as consistent multi-object generation in multiple rounds to increase stability and improved understanding of user sketches.   


% \new{
% From this perspective, our work is a step forward in this direction by allowing multi-object spatial-semantic prompting control by users' natural sketching, which considers the interplay between multiple sketch regions.
% % To further advance the multi-modal prompting experience, there are some aspects we identify to be important.
% % One of the important aspects is enhancing the consistency and stability of multiple rounds of generation to reduce the uncertainty and loss of control on users' part.
% % For this purpose, we need to develop techniques to incorporate consistent generation~\cite{tewel2024training} into multi-modal prompting framework.}
% % Another important aspect is improving generative models' understanding of the implicit user intents \new{implied by the paint-medium-like or sketch-based input (\eg, sketch of two people with their hands slightly overlapping indicates holding hand without needing explicit prompt).
% % This can facilitate more natural control and alleviate users' effort in tuning the textual prompt.
% % In addition, it can increase users' sense of ownership as the generated results can be more aligned with their sketching intents.
% }
% For example, when users draw sketches of two people with their hands slightly overlapping, current region-based models cannot automatically infer users' implicit intention that the two people are holding hands.
% Instead, they still require users to explicitly specify in the prompt such relationship.
% \tool addresses this through sketch-aware prompt recommendation to fill in the necessary semantic information, alleviating users' workload.
% However, some users want the generative AI in the future to be able to directly infer this natural implicit intentions from the sketches without additional prompting since prompt recommendation can still be unstable sometimes.


% \new{
% Besides visual generation, 
% }
% For example, one of the important aspect is referring~\cite{he2024multi}, linking specific text semantics with specific spatial object, which is partly what we do in our sketch-aware prompt recommendation.
% Analogously, in natural communication between humans, text or audio alone often cannot suffice in expressing the speakers' intentions, and speakers often need to refer to an existing spatial object or draw out an illustration of her ideas for better explanation.
% Philosophically, we HCI researchers are mostly concerned about the human-end experience in human-AI communications.
% However, studies on prompting is unique in that we should not just care about the human-end interaction, but also make sure that AI can really get what the human means and produce intention-aligned output.
% Such consideration can drastically impact the design of prompting interactions in human-AI collaboration applications.
% On this note, although studies on multi-modal interactions is a well-established topic in HCI community, it remains a challenging problem what kind of multi-modal information is really effective in helping humans convey their ideas to current and next generation large AI models.




\subsection{Novice Performance vs. Expert Performance}\label{sec:nVe}
In this section we discuss the performance difference between novice and expert regarding experience in painting and prompting.
First, regarding painting skills, some participants with experience (4/12) preferred to draw accurate and fine-grained shapes at the beginning. 
All novice users (5/12) draw rough and less accurate shapes, while some participants with basic painting skills (3/12) also favored sketching rough areas of objects, as exemplified in Figure~\ref{fig:novice_expert}.
The experienced participants using fine-grained strokes (4/12, none of whom were experienced in prompting) achieved higher IoU scores (0.557) in the close-ended task (0.535) when using \tool. 
This is because their sketches were closer in shape and location to the reference, making the single object decomposition result more accurate.
Also, experienced participants are better at arranging spatial location and size of objects than novice participants.
However, some experienced participants (3/12) have mentioned that the fine-grained stroke sometimes makes them frustrated.
As P1's comment for his result in open-ended task: "\emph{It seems it cannot understand thin strokes; even if the shape is accurate, it can only generate content roughly around the area, especially when there is overlapping.}" 
This suggests that while \tool\ provides rough control to produce reasonably fine results from less accurate sketches for novice users, it may disappoint experienced users seeking more precise control through finer strokes. 
As shown in the last column in Figure~\ref{fig:novice_expert}, the dragon hovering in the sky was wrongly turned into a standing large dragon by \tool.

Second, regarding prompting skills, 3 out of 12 participants had one or more years of experience in T2I prompting. These participants used more modifiers than others during both T2I and R2I tasks.
Their performance in the T2I (0.335) and R2I (0.469) tasks showed higher scores than the average T2I (0.314) and R2I (0.418), but there was no performance improvement with \tool\ between their results (0.508) and the overall average score (0.528). 
This indicates that \tool\ can assist novice users in prompting, enabling them to produce satisfactory images similar to those created by users with prompting expertise.



\subsection{Applicability of \tool}
The feedback from user study highlighted several potential applications for our system. 
Three participants (P2, P6, P8) mentioned its possible use in commercial advertising design, emphasizing the importance of controllability for such work. 
They noted that the system's flexibility allows designers to quickly experiment with different settings.
Some participants (N = 3) also mentioned its potential for digital asset creation, particularly for game asset design. 
P7, a game mod developer, found the system highly useful for mod development. 
He explained: "\emph{Mods often require a series of images with a consistent theme and specific spatial requirements. 
For example, in a sacrifice scene, how the objects are arranged is closely tied to the mod's background. It would be difficult for a developer without professional skills, but with this system, it is possible to quickly construct such images}."
A few participants expressed similar thoughts regarding its use in scene construction, such as in film production. 
An interesting suggestion came from participant P4, who proposed its application in crime scene description. 
She pointed out that witnesses are often not skilled artists, and typically describe crime scenes verbally while someone else illustrates their account. 
With this system, witnesses could more easily express what they saw themselves, potentially producing depictions closer to the real events. "\emph{Details like object locations and distances from buildings can be easily conveyed using the system}," she added.

% \subsection{Model Understanding of Users' Implicit Intents}
% In region-sketch-based control of generative models, a significant gap between interaction design and actual implementation is the model's failure in understanding users' naturally expressed intentions.
% For example, when users draw sketches of two people with their hands slightly overlapping, current region-based models cannot automatically infer users' implicit intention that the two people are holding hands.
% Instead, they still require users to explicitly specify in the prompt such relationship.
% \tool addresses this through sketch-aware prompt recommendation to fill in the necessary semantic information, alleviating users' workload.
% However, some users want the generative AI in the future to be able to directly infer this natural implicit intentions from the sketches without additional prompting since prompt recommendation can still be unstable sometimes.
% This problem reflects a more general dilemma, which ubiquitously exists in all forms of conditioned control for generative models such as canny or scribble control.
% This is because all the control models are trained on pairs of explicit control signal and target image, which is lacking further interpretation or customization of the user intentions behind the seemingly straightforward input.
% For another example, the generative models cannot understand what abstraction level the user has in mind for her personal scribbles.
% Such problems leave more challenges to be addressed by future human-AI co-creation research.
% One possible direction is fine-tuning the conditioned models on individual user's conditioned control data to provide more customized interpretation. 

% \subsection{Balance between recommendation and autonomy}
% AIGC tools are a typical example of 
\subsection{Progressive Sketching}
Currently \tool is mainly aimed at novice users who are only capable of creating very rough sketches by themselves.
However, more accomplished painters or even professional artists typically have a coarse-to-fine creative process. 
Such a process is most evident in painting styles like traditional oil painting or digital impasto painting, where artists first quickly lay down large color patches to outline the most primitive proportion and structure of visual elements.
After that, the artists will progressively add layers of finer color strokes to the canvas to gradually refine the painting to an exquisite piece of artwork.
One participant in our user study (P1) , as a professional painter, has mentioned a similar point "\emph{
I think it is useful for laying out the big picture, give some inspirations for the initial drawing stage}."
Therefore, rough sketch also plays a part in the professional artists' creation process, yet it is more challenging to integrate AI into this more complex coarse-to-fine procedure.
Particularly, artists would like to preserve some of their finer strokes in later progression, not just the shape of the initial sketch.
In addition, instead of requiring the tool to generate a finished piece of artwork, some artists may prefer a model that can generate another more accurate sketch based on the initial one, and leave the final coloring and refining to the artists themselves.
To accommodate these diverse progressive sketching requirements, a more advanced sketch-based AI-assisted creation tool should be developed that can seamlessly enable artist intervention at any stage of the sketch and maximally preserve their creative intents to the finest level. 

\subsection{Ethical Issues}
Intellectual property and unethical misuse are two potential ethical concerns of AI-assisted creative tools, particularly those targeting novice users.
In terms of intellectual property, \tool hands over to novice users more control, giving them a higher sense of ownership of the creation.
However, the question still remains: how much contribution from the user's part constitutes full authorship of the artwork?
As \tool still relies on backbone generative models which may be trained on uncopyrighted data largely responsible for turning the sketch into finished artwork, we should design some mechanisms to circumvent this risk.
For example, we can allow artists to upload backbone models trained on their own artworks to integrate with our sketch control.
Regarding unethical misuse, \tool makes fine-grained spatial control more accessible to novice users, who may maliciously generate inappropriate content such as more realistic deepfake with specific postures they want or other explicit content.
To address this issue, we plan to incorporate a more sophisticated filtering mechanism that can detect and screen unethical content with more complex spatial-semantic conditions. 
% In the future, we plan to enable artists to upload their own style model

% \subsection{From interactive prompting to interactive spatial prompting}


\subsection{Limitations and Future work}

    \textbf{User Study Design}. Our open-ended task assesses the usability of \tool's system features in general use cases. To further examine aspects such as creativity and controllability across different methods, the open-ended task could be improved by incorporating baselines to provide more insightful comparative analysis. 
    Besides, in close-ended tasks, while the fixing order of tool usage prevents prior knowledge leakage, it might introduce learning effects. In our study, we include practice sessions for the three systems before the formal task to mitigate these effects. In the future, utilizing parallel tests (\textit{e.g.} different content with the same difficulty) or adding a control group could further reduce the learning effects.

    \textbf{Failure Cases}. There are certain failure cases with \tool that can limit its usability. 
    Firstly, when there are three or more objects with similar semantics, objects may still be missing despite prompt recommendations. 
    Secondly, if an object's stroke is thin, \tool may incorrectly interpret it as a full area, as demonstrated in the expert results of the open-ended task in Figure~\ref{fig:novice_expert}. 
    Finally, sometimes inclusion relationships (\textit{e.g.} inside) between objects cannot be generated correctly, partially due to biases in the base model that lack training samples with such relationship. 

    \textbf{More support for single object adjustment}.
    Participants (N=4) suggested that additional control features should be introduced, beyond just adjusting size and location. They noted that when objects overlap, they cannot freely control which object appears on top or which should be covered, and overlapping areas are currently not allowed.
    They proposed adding features such as layer control and depth control within the single-object mask manipulation. Currently, the system assigns layers based on color order, but future versions should allow users to adjust the layer of each object freely, while considering weighted prompts for overlapping areas.

    \textbf{More customized generation ability}.
    Our current system is built around a single model $ColorfulXL-Lightning$, which limits its ability to fully support the diverse creative needs of users. Feedback from participants has indicated a strong desire for more flexibility in style and personalization, such as integrating fine-tuned models that cater to specific artistic styles or individual preferences. 
    This limitation restricts the ability to adapt to varied creative intents across different users and contexts.
    In future iterations, we plan to address this by embedding a model selection feature, allowing users to choose from a variety of pre-trained or custom fine-tuned models that better align with their stylistic preferences. 
    
    \textbf{Integrate other model functions}.
    Our current system is compatible with many existing tools, such as Promptist~\cite{hao2024optimizing} and Magic Prompt, allowing users to iteratively generate prompts for single objects. However, the integration of these functions is somewhat limited in scope, and users may benefit from a broader range of interactive options, especially for more complex generation tasks. Additionally, for multimodal large models, users can currently explore using affordable or open-source models like Qwen2-VL~\cite{qwen} and InternVL2-Llama3~\cite{llama}, which have demonstrated solid inference performance in our tests. While GPT-4o remains a leading choice, alternative models also offer competitive results.
    Moving forward, we aim to integrate more multimodal large models into the system, giving users the flexibility to choose the models that best fit their needs. 
    


\section{Conclusion}\label{sec:conclusion}
In this paper, we present \tool, an interactive system designed to help novice users create high-quality, fine-grained images that align with their intentions based on rough sketches. 
The system first refines the user's initial prompt into a complete and coherent one that matches the rough sketch, ensuring the generated results are both stable, coherent and high quality.
To further support users in achieving fine-grained alignment between the generated image and their creative intent without requiring professional skills, we introduce a decompose-and-recompose strategy. 
This allows users to select desired, refined object shapes for individual decomposed objects and then recombine them, providing flexible mask manipulation for precise spatial control.
The framework operates through a coarse-to-fine process, enabling iterative and fine-grained control that is not possible with traditional end-to-end generation methods. 
Our user study demonstrates that \tool offers novice users enhanced flexibility in control and fine-grained alignment between their intentions and the generated images.



% Entries for the entire Anthology, followed by custom entries
\bibliography{ref}
\bibliographystyle{acl_natbib}

% E5数据集介绍,数据集处理过程
% 基线模型介绍

\definecolor{titlecolor}{rgb}{0.9, 0.5, 0.1}
\definecolor{anscolor}{rgb}{0.2, 0.5, 0.8}
\definecolor{labelcolor}{HTML}{48a07e}
\begin{table*}[h]
	\centering
	
 % \vspace{-0.2cm}
	
	\begin{center}
		\begin{tikzpicture}[
				chatbox_inner/.style={rectangle, rounded corners, opacity=0, text opacity=1, font=\sffamily\scriptsize, text width=5in, text height=9pt, inner xsep=6pt, inner ysep=6pt},
				chatbox_prompt_inner/.style={chatbox_inner, align=flush left, xshift=0pt, text height=11pt},
				chatbox_user_inner/.style={chatbox_inner, align=flush left, xshift=0pt},
				chatbox_gpt_inner/.style={chatbox_inner, align=flush left, xshift=0pt},
				chatbox/.style={chatbox_inner, draw=black!25, fill=gray!7, opacity=1, text opacity=0},
				chatbox_prompt/.style={chatbox, align=flush left, fill=gray!1.5, draw=black!30, text height=10pt},
				chatbox_user/.style={chatbox, align=flush left},
				chatbox_gpt/.style={chatbox, align=flush left},
				chatbox2/.style={chatbox_gpt, fill=green!25},
				chatbox3/.style={chatbox_gpt, fill=red!20, draw=black!20},
				chatbox4/.style={chatbox_gpt, fill=yellow!30},
				labelbox/.style={rectangle, rounded corners, draw=black!50, font=\sffamily\scriptsize\bfseries, fill=gray!5, inner sep=3pt},
			]
											
			\node[chatbox_user] (q1) {
				\textbf{System prompt}
				\newline
				\newline
				You are a helpful and precise assistant for segmenting and labeling sentences. We would like to request your help on curating a dataset for entity-level hallucination detection.
				\newline \newline
                We will give you a machine generated biography and a list of checked facts about the biography. Each fact consists of a sentence and a label (True/False). Please do the following process. First, breaking down the biography into words. Second, by referring to the provided list of facts, merging some broken down words in the previous step to form meaningful entities. For example, ``strategic thinking'' should be one entity instead of two. Third, according to the labels in the list of facts, labeling each entity as True or False. Specifically, for facts that share a similar sentence structure (\eg, \textit{``He was born on Mach 9, 1941.''} (\texttt{True}) and \textit{``He was born in Ramos Mejia.''} (\texttt{False})), please first assign labels to entities that differ across atomic facts. For example, first labeling ``Mach 9, 1941'' (\texttt{True}) and ``Ramos Mejia'' (\texttt{False}) in the above case. For those entities that are the same across atomic facts (\eg, ``was born'') or are neutral (\eg, ``he,'' ``in,'' and ``on''), please label them as \texttt{True}. For the cases that there is no atomic fact that shares the same sentence structure, please identify the most informative entities in the sentence and label them with the same label as the atomic fact while treating the rest of the entities as \texttt{True}. In the end, output the entities and labels in the following format:
                \begin{itemize}[nosep]
                    \item Entity 1 (Label 1)
                    \item Entity 2 (Label 2)
                    \item ...
                    \item Entity N (Label N)
                \end{itemize}
                % \newline \newline
                Here are two examples:
                \newline\newline
                \textbf{[Example 1]}
                \newline
                [The start of the biography]
                \newline
                \textcolor{titlecolor}{Marianne McAndrew is an American actress and singer, born on November 21, 1942, in Cleveland, Ohio. She began her acting career in the late 1960s, appearing in various television shows and films.}
                \newline
                [The end of the biography]
                \newline \newline
                [The start of the list of checked facts]
                \newline
                \textcolor{anscolor}{[Marianne McAndrew is an American. (False); Marianne McAndrew is an actress. (True); Marianne McAndrew is a singer. (False); Marianne McAndrew was born on November 21, 1942. (False); Marianne McAndrew was born in Cleveland, Ohio. (False); She began her acting career in the late 1960s. (True); She has appeared in various television shows. (True); She has appeared in various films. (True)]}
                \newline
                [The end of the list of checked facts]
                \newline \newline
                [The start of the ideal output]
                \newline
                \textcolor{labelcolor}{[Marianne McAndrew (True); is (True); an (True); American (False); actress (True); and (True); singer (False); , (True); born (True); on (True); November 21, 1942 (False); , (True); in (True); Cleveland, Ohio (False); . (True); She (True); began (True); her (True); acting career (True); in (True); the late 1960s (True); , (True); appearing (True); in (True); various (True); television shows (True); and (True); films (True); . (True)]}
                \newline
                [The end of the ideal output]
				\newline \newline
                \textbf{[Example 2]}
                \newline
                [The start of the biography]
                \newline
                \textcolor{titlecolor}{Doug Sheehan is an American actor who was born on April 27, 1949, in Santa Monica, California. He is best known for his roles in soap operas, including his portrayal of Joe Kelly on ``General Hospital'' and Ben Gibson on ``Knots Landing.''}
                \newline
                [The end of the biography]
                \newline \newline
                [The start of the list of checked facts]
                \newline
                \textcolor{anscolor}{[Doug Sheehan is an American. (True); Doug Sheehan is an actor. (True); Doug Sheehan was born on April 27, 1949. (True); Doug Sheehan was born in Santa Monica, California. (False); He is best known for his roles in soap operas. (True); He portrayed Joe Kelly. (True); Joe Kelly was in General Hospital. (True); General Hospital is a soap opera. (True); He portrayed Ben Gibson. (True); Ben Gibson was in Knots Landing. (True); Knots Landing is a soap opera. (True)]}
                \newline
                [The end of the list of checked facts]
                \newline \newline
                [The start of the ideal output]
                \newline
                \textcolor{labelcolor}{[Doug Sheehan (True); is (True); an (True); American (True); actor (True); who (True); was born (True); on (True); April 27, 1949 (True); in (True); Santa Monica, California (False); . (True); He (True); is (True); best known (True); for (True); his roles in soap operas (True); , (True); including (True); in (True); his portrayal (True); of (True); Joe Kelly (True); on (True); ``General Hospital'' (True); and (True); Ben Gibson (True); on (True); ``Knots Landing.'' (True)]}
                \newline
                [The end of the ideal output]
				\newline \newline
				\textbf{User prompt}
				\newline
				\newline
				[The start of the biography]
				\newline
				\textcolor{magenta}{\texttt{\{BIOGRAPHY\}}}
				\newline
				[The ebd of the biography]
				\newline \newline
				[The start of the list of checked facts]
				\newline
				\textcolor{magenta}{\texttt{\{LIST OF CHECKED FACTS\}}}
				\newline
				[The end of the list of checked facts]
			};
			\node[chatbox_user_inner] (q1_text) at (q1) {
				\textbf{System prompt}
				\newline
				\newline
				You are a helpful and precise assistant for segmenting and labeling sentences. We would like to request your help on curating a dataset for entity-level hallucination detection.
				\newline \newline
                We will give you a machine generated biography and a list of checked facts about the biography. Each fact consists of a sentence and a label (True/False). Please do the following process. First, breaking down the biography into words. Second, by referring to the provided list of facts, merging some broken down words in the previous step to form meaningful entities. For example, ``strategic thinking'' should be one entity instead of two. Third, according to the labels in the list of facts, labeling each entity as True or False. Specifically, for facts that share a similar sentence structure (\eg, \textit{``He was born on Mach 9, 1941.''} (\texttt{True}) and \textit{``He was born in Ramos Mejia.''} (\texttt{False})), please first assign labels to entities that differ across atomic facts. For example, first labeling ``Mach 9, 1941'' (\texttt{True}) and ``Ramos Mejia'' (\texttt{False}) in the above case. For those entities that are the same across atomic facts (\eg, ``was born'') or are neutral (\eg, ``he,'' ``in,'' and ``on''), please label them as \texttt{True}. For the cases that there is no atomic fact that shares the same sentence structure, please identify the most informative entities in the sentence and label them with the same label as the atomic fact while treating the rest of the entities as \texttt{True}. In the end, output the entities and labels in the following format:
                \begin{itemize}[nosep]
                    \item Entity 1 (Label 1)
                    \item Entity 2 (Label 2)
                    \item ...
                    \item Entity N (Label N)
                \end{itemize}
                % \newline \newline
                Here are two examples:
                \newline\newline
                \textbf{[Example 1]}
                \newline
                [The start of the biography]
                \newline
                \textcolor{titlecolor}{Marianne McAndrew is an American actress and singer, born on November 21, 1942, in Cleveland, Ohio. She began her acting career in the late 1960s, appearing in various television shows and films.}
                \newline
                [The end of the biography]
                \newline \newline
                [The start of the list of checked facts]
                \newline
                \textcolor{anscolor}{[Marianne McAndrew is an American. (False); Marianne McAndrew is an actress. (True); Marianne McAndrew is a singer. (False); Marianne McAndrew was born on November 21, 1942. (False); Marianne McAndrew was born in Cleveland, Ohio. (False); She began her acting career in the late 1960s. (True); She has appeared in various television shows. (True); She has appeared in various films. (True)]}
                \newline
                [The end of the list of checked facts]
                \newline \newline
                [The start of the ideal output]
                \newline
                \textcolor{labelcolor}{[Marianne McAndrew (True); is (True); an (True); American (False); actress (True); and (True); singer (False); , (True); born (True); on (True); November 21, 1942 (False); , (True); in (True); Cleveland, Ohio (False); . (True); She (True); began (True); her (True); acting career (True); in (True); the late 1960s (True); , (True); appearing (True); in (True); various (True); television shows (True); and (True); films (True); . (True)]}
                \newline
                [The end of the ideal output]
				\newline \newline
                \textbf{[Example 2]}
                \newline
                [The start of the biography]
                \newline
                \textcolor{titlecolor}{Doug Sheehan is an American actor who was born on April 27, 1949, in Santa Monica, California. He is best known for his roles in soap operas, including his portrayal of Joe Kelly on ``General Hospital'' and Ben Gibson on ``Knots Landing.''}
                \newline
                [The end of the biography]
                \newline \newline
                [The start of the list of checked facts]
                \newline
                \textcolor{anscolor}{[Doug Sheehan is an American. (True); Doug Sheehan is an actor. (True); Doug Sheehan was born on April 27, 1949. (True); Doug Sheehan was born in Santa Monica, California. (False); He is best known for his roles in soap operas. (True); He portrayed Joe Kelly. (True); Joe Kelly was in General Hospital. (True); General Hospital is a soap opera. (True); He portrayed Ben Gibson. (True); Ben Gibson was in Knots Landing. (True); Knots Landing is a soap opera. (True)]}
                \newline
                [The end of the list of checked facts]
                \newline \newline
                [The start of the ideal output]
                \newline
                \textcolor{labelcolor}{[Doug Sheehan (True); is (True); an (True); American (True); actor (True); who (True); was born (True); on (True); April 27, 1949 (True); in (True); Santa Monica, California (False); . (True); He (True); is (True); best known (True); for (True); his roles in soap operas (True); , (True); including (True); in (True); his portrayal (True); of (True); Joe Kelly (True); on (True); ``General Hospital'' (True); and (True); Ben Gibson (True); on (True); ``Knots Landing.'' (True)]}
                \newline
                [The end of the ideal output]
				\newline \newline
				\textbf{User prompt}
				\newline
				\newline
				[The start of the biography]
				\newline
				\textcolor{magenta}{\texttt{\{BIOGRAPHY\}}}
				\newline
				[The ebd of the biography]
				\newline \newline
				[The start of the list of checked facts]
				\newline
				\textcolor{magenta}{\texttt{\{LIST OF CHECKED FACTS\}}}
				\newline
				[The end of the list of checked facts]
			};
		\end{tikzpicture}
        \caption{GPT-4o prompt for labeling hallucinated entities.}\label{tb:gpt-4-prompt}
	\end{center}
\vspace{-0cm}
\end{table*}

% \begin{figure}[t]
%     \centering
%     \includegraphics[width=0.9\linewidth]{Image/abla2/doc7.png}
%     \caption{Improvement of generated documents over direct retrieval on different models.}
%     \label{fig:comparison}
% \end{figure}

\begin{figure}[t]
    \centering
    \subfigure[Unsupervised Dense Retriever.]{
        \label{fig:imp:unsupervised}
        \includegraphics[width=0.8\linewidth]{Image/A.3_fig/improvement_unsupervised.pdf}
    }
    \subfigure[Supervised Dense Retriever.]{
        \label{fig:imp:supervised}
        \includegraphics[width=0.8\linewidth]{Image/A.3_fig/improvement_supervised.pdf}
    }
    
    % \\
    % \subfigure[Comparison of Reasoning Quality With Different Method.]{
    %     \label{fig:reasoning} 
    %     \includegraphics[width=0.98\linewidth]{images/reasoning1.pdf}
    % }
    \caption{Improvements of LLM-QE in Both Unsupervised and Supervised Dense Retrievers. We plot the change of nDCG@10 scores before and after the query expansion using our LLM-QE model.}
    \label{fig:imp}
\end{figure}
\section{Appendix}
\subsection{License}
The authors of 4 out of the 15 datasets in the BEIR benchmark (NFCorpus, FiQA-2018, Quora, Climate-Fever) and the authors of ELI5 in the E5 dataset do not report the dataset license in the paper or a repository. We summarize the licenses of the remaining datasets as follows.

MS MARCO (MIT License); FEVER, NQ, and DBPedia (CC BY-SA 3.0 license); ArguAna and Touché-2020 (CC BY 4.0 license); CQADupStack and TriviaQA (Apache License 2.0); SciFact (CC BY-NC 2.0 license); SCIDOCS (GNU General Public License v3.0); HotpotQA and SQuAD (CC BY-SA 4.0 license); TREC-COVID (Dataset License Agreement).

All these licenses and agreements permit the use of their data for academic purposes.

\subsection{Additional Experimental Details}\label{app:experiment_detail}
This subsection outlines the components of the training data and presents the prompt templates used in the experiments.


\textbf{Training Datasets.} Following the setup of \citet{wang2024improving}, we use the following datasets: ELI5 (sample ratio 0.1)~\cite{fan2019eli5}, HotpotQA~\cite{yang2018hotpotqa}, FEVER~\cite{thorne2018fever}, MS MARCO passage ranking (sample ratio 0.5) and document ranking (sample ratio 0.2)~\cite{bajaj2016ms}, NQ~\cite{karpukhin2020dense}, SQuAD~\cite{karpukhin2020dense}, and TriviaQA~\cite{karpukhin2020dense}. In total, we use 808,740 training examples.

\textbf{Prompt Templates.} Table~\ref{tab:prompt_template} lists all the prompts used in this paper. In each prompt, ``query'' refers to the input query for which query expansions are generated, while ``Related Document'' denotes the ground truth document relevant to the original query. We observe that, in general, the model tends to generate introductory phrases such as ``Here is a passage to answer the question:'' or ``Here is a list of keywords related to the query:''. Before using the model outputs as query expansions, we first filter out these introductory phrases to ensure cleaner and more precise expansion results.



\subsection{Query Expansion Quality of LLM-QE}\label{app:analysis}
This section evaluates the quality of query expansion of LLM-QE. As shown in Figure~\ref{fig:imp}, we randomly select 100 samples from each dataset to assess the improvement in retrieval performance before and after applying LLM-QE.

Overall, the evaluation results demonstrate that LLM-QE consistently improves retrieval performance in both unsupervised (Figure~\ref{fig:imp:unsupervised}) and supervised (Figure~\ref{fig:imp:supervised}) settings. However, for the MS MARCO dataset, LLM-QE demonstrates limited effectiveness in the supervised setting. This can be attributed to the fact that MS MARCO provides higher-quality training signals, allowing the dense retriever to learn sufficient matching signals from relevance labels. In contrast, LLM-QE leads to more substantial performance improvements on the NQ and HotpotQA datasets. This indicates that LLM-QE provides essential matching signals for dense retrievers, particularly in retrieval scenarios where high-quality training signals are scarce.


\subsection{Case Study}\label{app:case_study}
\begin{figure}[htb]
\small
\begin{tcolorbox}[left=3pt,right=3pt,top=3pt,bottom=3pt,title=\textbf{Conversation History:}]
[human]: Craft an intriguing opening paragraph for a fictional short story. The story should involve a character who wakes up one morning to find that they can time travel.

...(Human-Bot Dialogue Turns)... \textcolor{blue}{(Topic: Time-Travel Fiction)}

[human]: Please describe the concept of machine learning. Could you elaborate on the differences between supervised, unsupervised, and reinforcement learning? Provide real-world examples of each.

...(Human-Bot Dialogue Turns)... \textcolor{blue}{(Topic: Machine learning Concepts and Types)}


[human]: Discuss antitrust laws and their impact on market competition. Compare the antitrust laws in US and China along with some case studies

...(Human-Bot Dialogue Turns)... \textcolor{blue}{(Topic: Antitrust Laws and Market Competition)}

[human]: The vertices of a triangle are at points (0, 0), (-1, 1), and (3, 3). What is the area of the triangle?

...(Human-Bot Dialogue Turns)... \textcolor{blue}{(Topic: Area and Circumcircle of Triangle)}

...(Human-Bot Dialogue Interaction on Other Topics)... 

\end{tcolorbox}
\begin{tcolorbox}[left=3pt,right=3pt,top=3pt,bottom=3pt,title=\textbf{Question:}]
What title reflects the thematic essence of the Doctor's experience with time travel as described in the story?
\end{tcolorbox}
\begin{tcolorbox}[left=3pt,right=3pt,top=3pt,bottom=3pt,title=\textbf{Ground-Truth Answer:}]
Awakening to Eternity: The Doctor's New Dawn
\end{tcolorbox}
\begin{tcolorbox}[left=3pt,right=3pt,top=3pt,bottom=3pt,title=\textbf{Retrieved Memory (Turn-Level):}]
[human]: Describe the process and write out the balanced chemical equation for the reaction that occurs when solid calcium carbonate reacts with hydrochloric acid to form aqueous calcium chloride, carbon dioxide, and water. What type of reaction is this, and what observations might indicate that the reaction is taking place?

[bot]: When solid calcium carbonate ($\text{CaCO}_{3}$) reacts with hydrochloric acid (HCl), the products formed are aqueous calcium chloride ($\text{CaCl}_{2}$), carbon dioxide ($\text{CO}_{2}$), and water ($\text{H}_{2}\text{O}$). The balanced chemical equation for this reaction is: ... \textcolor{blue}{(Topic: Calcium Carbonate and Hydrochloric Acid Reaction)}

\textcolor{brown}{
[human]: Write a title for your story.}

\textcolor{brown}{
[bot]: "Awakening to Eternity: The Doctor's New Dawn".
} \textcolor{blue}{(Topic: Time-Travel Fiction)}

[human]: Analyze the following customer reviews from different sources for three different smartphones - the latest iPhone, Samsung Galaxy, and Google Pixel - and provide an overall rating for each phone on a scale of 1 to 10. Consider the following complex and contradictory reviews: - TechRadar's review of the latest iPhone: The new iPhone is a stunning triumph of engineering that sets a new bar for smartphone performance and camera quality. However, the incremental design and high price mean it lacks the 'wow' factor of previous iPhones. Still, its power and intelligence are unrivaled. - CNET's review of the latest Samsung Galaxy: ... Return the answer as a JSON object with the overall ratings for each phone out of 10, to one decimal place.

[bot]: ```json { "iPhone": 8.5, "Samsung Galaxy": 8.0, "Google Pixel": 7.5 } ```

\textcolor{blue}{(Topic: Smartphone Review Ratings Analysis)}
\end{tcolorbox}
\begin{tcolorbox}[left=3pt,right=3pt,top=3pt,bottom=3pt,title=\textbf{Answer (Turn-Level Retrieval):}]
As an AI language model, I do not have access to the story you mentioned earlier. Please provide me with the story or the necessary details so that I can assist you better.

\textbf{\textcolor{red}{(The next part of this figure~\ref{fig: case_study_segment_vs_turn} will be continued on the next page.)}}
\end{tcolorbox}
\end{figure}

\begin{figure}[htb]
% \ContinuedFloat
\small
\begin{tcolorbox}[left=3pt,right=3pt,top=3pt,bottom=3pt,title=\textbf{Retrieved Memory (Ours):}]
[human]: Craft an intriguing opening paragraph for a fictional short story. The story should involve a character who wakes up one morning to find that they can time travel.

[bot]: In the hushed pre-dawn hours, when the world still slumbers under the cloak of night, Alex found themselves caught in the fragile boundary between sleep and wakefulness. ...

[human]: Summarize the story with three bullet points using only nouns and adjectives, without verbs.

[bot]: - Alex, pre-dawn awakening, time-travel discovery - Disorientation, vertigo, reality's malleability - Sunlight, new day, astonishing truth

[human]: Is there any famous fictional character that can time travel? If so, can you rewrite your story with him or her?

[bot]: - \textcolor{brown}{Doctor Who, a renowned time-traveling character from British television, is a perfect fit for this scenario.} In the silent embrace of early morning, where shadows merge with the fading night, the Doctor found themselves in the thin veil separating dreams from consciousness. It was here, amidst the soft whispers of time, that an unusual sensation enveloped them—a sense of time's fabric slipping through their fingers like grains of sand. As dawn's light broke through the darkness, painting the room in hues of gold and amber, the Doctor's eyes opened to what appeared to be an ordinary day. \textcolor{brown}{However, upon standing, a wave of dizziness overwhelmed them, the surroundings blurred, and in that moment of confusion, the Doctor realized a profound truth: they had not only awakened to a new dawn but had also stumbled upon the incredible ability to navigate the currents of time at will.}

\textcolor{brown}{[human]: Write a title for your story.}

\textcolor{brown}{[bot]: "Awakening to Eternity: The Doctor's New Dawn"} \textcolor{blue}{(Topic: Time-Travel Fiction)}
\end{tcolorbox}
\begin{tcolorbox}[left=3pt,right=3pt,top=3pt,bottom=3pt,title=\textbf{Answer (Ours):}]
The title "Awakening to Eternity: The Doctor's New Dawn" reflects the thematic essence of the Doctor's experience with time travel as described in the story. It captures the idea of a new beginning, a sense of eternity, and the awakening of a new power that will change the course of the Doctor's life forever.
\end{tcolorbox}
\caption{Case study comparing our \sysname\ with the \textit{turn-level} memory baseline. When constructing the memory bank at the \textit{turn-level}, the agent fails to provide a satisfactory response due to fragmentation: the retrieved dialogue turns can not capture a complete human-agent interaction related to the query, missing critical information highlighted in \textcolor{brown}{brown}. The dialogue topic is emphasized in \textcolor{blue}{blue}. Some context is omitted (replaced by ...) due to length constraints.}
\label{fig: case_study_segment_vs_turn}
\end{figure}

\newpage

\begin{figure}[htbp]
\small
\begin{tcolorbox}[left=3pt,right=3pt,top=3pt,bottom=3pt,title=\textbf{Conversation History:}]
[human]: Explain what's base rate fallacy and list five specific examples of ...

... (Human-Bot Dialogue Turns) ... \textcolor{blue}{(Topic: Base Rate Fallacy In Politics)}


[human]: How many integers are in the solution of the inequality $|x + 5| < 10$?

... (Human-Bot Dialogue Turns) ... \textcolor{blue}{(Topic: Number of Integers in Inequalities)}

[human]: How have the Alps and Rhine River influenced settlement and agriculture in Western Europe?

... (Human-Bot Dialogue Turns) ... \textcolor{blue}{(Topic: Alps and Rhine River Impact)}

[human]: In a survey conducted at a local high school, preferences for a new school color were measured: 58\% of students liked the color blue, 45\% preferred green, and 22\% liked both colors. If we randomly pick a student from the school, what's the probability that they would like neither blue nor green?

... (Human-Bot Dialogue Turns) ... \textcolor{blue}{(Topic: Probability of Liking Colors in Survey)}

[human]: Imagine you are participating in a race with a group of people. If you have just overtaken the second person, what's your current position? Where is the person you just overtook?

... (Human-Bot Dialogue Turns) ... \textcolor{blue}{(Topic: Overtaking Positions in Race)}
\end{tcolorbox}
\begin{tcolorbox}[left=3pt,right=3pt,top=3pt,bottom=3pt,title=\textbf{Question:}]
[human]: In the inequality question, how many positive integers are in the solution of the second inequality?
\end{tcolorbox}
\begin{tcolorbox}[left=3pt,right=3pt,top=3pt,bottom=3pt,title=\textbf{Ground-Truth Answer:}]
In the second inequality, $(|x + 10| < 5)$, the solution range is $(-15 < x < -5)$. Within this range, there are no positive integers, as all values of $(x)$ that satisfy the inequality are negative.
\end{tcolorbox}
\begin{tcolorbox}[left=3pt,right=3pt,top=3pt,bottom=3pt,title=\textbf{Retrieved Memory (Session-Level):}]
[human]: Explain what's base rate fallacy and list five specific examples of ...

... (Human-Bot Dialogue Turns) ... \textcolor{blue}{(Topic: Base Rate Fallacy In Politics)}

[human]: How many integers are in the solution of the inequality $|x + 5| < 10$?

[bot]: To solve the inequality $|x + 5| < 10$, ... 

\textcolor{brown}{
[human]: What about $|x + 10| < 5$?
}

\textcolor{brown}{
[bot]: To solve the inequality $|x + 10| < 5$, we ...} \textcolor{gray}{(``overwhelmed'' by other irrelevant, redundant content)}

... (Human-Bot Dialogue Turns) ...  \textcolor{blue}{(Topic: Number of Integers in Inequalities)}

% \textcolor{brown}{
% [human]: What about $|x + 10| < 5$
% }

% \textcolor{brown}{
% [bot]: To solve the inequality $|x + 10| < 5$, ...
% } \textcolor{blue}{(Topic: Number of Integers in Inequalities)}

[human]: How have the Alps and Rhine River influenced settlement and agriculture in Western Europe?

... (Human-Bot Dialogue Turns) ... \textcolor{blue}{(Topic: Alps and Rhine River Impact)}

[human]: In a survey conducted at a local high school, ...

... (Human-Bot Dialogue Turns) ... \textcolor{blue}{(Topic: Probability of Liking Colors in Survey)}
\end{tcolorbox}
\begin{tcolorbox}[left=3pt,right=3pt,top=3pt,bottom=3pt,title=\textbf{Answer (Session-Level Retrieval):}]
I apologize for the confusion earlier. To answer your question, the second inequality was not provided in our previous conversation. If you could provide the inequality, I would be happy to help you find the solution and the number of positive integers in the solution set.

\textbf{\textcolor{red}{(The next part of this figure~\ref{fig: case_study_segment_vs_session} will be continued on the next page.)}}
\end{tcolorbox}
\end{figure}
\begin{figure}[htbp]
% \ContinuedFloat
\small
\begin{tcolorbox}[left=3pt,right=3pt,top=3pt,bottom=3pt,title=\textbf{Retrieved Memory (Ours):}]
[human]: How many integers are in the solution of the inequality $|x + 5| < 10$?

[bot]: To solve the inequality $|x + 5| < 10$, we break it into two cases based on the definition of absolute value: 
1. $x + 5 < 10$ 
2. $-(x + 5) < 10$, which simplifies to $x + 5 > -10$ 
Solving each case: 
1. $x < 5$ 
2. $x > -15$ 
Combining these, we get $-15 < x < 5$. This inequality means that $x$ can be any number greater than $-15$ and less than $5$. To find how many integers satisfy this condition, we list them: $-14, -13, -12, -11, -10, -9, -8, -7, -6, -5, -4, -3, -2, -1, 0, 1, 2, 3, 4$. There are 19 integers that satisfy the inequality $|x + 5| < 10$.

\textcolor{brown}{
[human]: What about $|x + 10| < 5$?
}

\textcolor{brown}{
[bot]: To solve the inequality $|x + 10| < 5$, we again consider two cases: 
1. $x + 10 < 5$ 
2. $-(x + 10) < 5$, which simplifies to $x + 10 > -5$ 
Solving each case: 
1. $x < -5$ 
2. $x > -15$ 
Combining these, we get $-15 < x < -5$. This inequality means that $x$ can be any number greater than $-15$ and less than $-5$. To find how many integers satisfy this condition, we list them: $-14, -13, -12, -11, -10, -9, -8, -7, -6$. There are 9 integers that satisfy the inequality $|x + 10| < 5$.
}

[human]: What about $|x + 10| < 10$?

[bot]: To solve the inequality $|x + 10| < 10$, we consider two cases: 
1. $x + 10 < 10$ 
2. $-(x + 10) < 10$, which simplifies to $x + 10 > -10$ 
Solving each case: 
1. $x < 0$ 
2. $x > -20$ 
Combining these, we get $-20 < x < 0$. This inequality means that $x$ can be any number greater than $-20$ and less than $0$. To find how many integers satisfy this condition, we list them: $-19, -18, -17, -16, -15, -14, -13, -12, -11, -10, -9, -8, -7, -6, -5, -4, -3, -2, -1$. There are 19 integers that satisfy the inequality $|x + 10| < 10$. \textcolor{blue}{(Topic: Number of Integers in Inequalities)}
\end{tcolorbox}
\begin{tcolorbox}[left=3pt,right=3pt,top=3pt,bottom=3pt,title=\textbf{Answer (Ours):}]
To find the number of positive integers that satisfy the inequality $(|x + 10| < 5)$, we need to look for positive integers in the solution set $(-15 < x < -5)$. Since there are no positive integers in this interval, the answer is 0.

\end{tcolorbox}
\caption{Case study comparing our \sysname\ with the \textit{session-level} memory baseline. When constructing the memory bank at the \textit{session-level}, the agent fails to provide a satisfactory response due to the inclusion of excessive irrelevant information within a session. This distracts the generator from focusing on the truly useful information highlighted in \textcolor{brown}{brown}. The dialogue topic is emphasized in \textcolor{blue}{blue}. Some context is omitted (replaced by ...) due to length constraints.}
\label{fig: case_study_segment_vs_session}
\end{figure}



\begin{figure}[htb]
\small
\begin{tcolorbox}[left=3pt,right=3pt,top=3pt,bottom=3pt,title=\textbf{Conversation History:}]
[human]: Photosynthesis is a vital process for life on Earth. Could you outline the two main stages of photosynthesis, including where they take place within the chloroplast, and the primary inputs and outputs for each stage? ... (Human-Bot Dialogue Turns)... \textcolor{blue}{(Topic: Photosynthetic Energy Production)}

[human]: Please assume the role of an English translator, tasked with correcting and enhancing spelling and language. Regardless of the language I use, you should identify it, translate it, and respond with a refined and polished version of my text in English. 

... (Human-Bot Dialogue Turns)...  \textcolor{blue}{(Topic: Language Translation and Enhancement)}

[human]: Suggest five award-winning documentary films with brief background descriptions for aspiring filmmakers to study.

\textcolor{brown}{[bot]: ...
5. \"An Inconvenient Truth\" (2006) - Directed by Davis Guggenheim and featuring former United States Vice President Al Gore, this documentary aims to educate the public about global warming. It won two Academy Awards, including Best Documentary Feature. The film is notable for its straightforward yet impactful presentation of scientific data, making complex information accessible and engaging, a valuable lesson for filmmakers looking to tackle environmental or scientific subjects.}

... (Human-Bot Dialogue Turns)... 
\textcolor{blue}{(Topic: Documentary Films Recommendation)}

[human]: Given the following records of stock prices, extract the highest and lowest closing prices for each month in the year 2022. Return the results as a CSV string, with one line allocated for each month. Date,Open,High,Low,Close,Volume ... ... (Human-Bot Dialogue Turns)...  \textcolor{blue}{(Topic: Stock Prices Analysis)}

[human]: The city of Vega intends to build a bridge that will span the Vegona River, covering a distance of 1.8 kilometers. The proposed location falls within a seismically active area that has experienced several high-magnitude earthquakes. Given these circumstances, what would be the best approach to constructing the bridge? ... (Human-Bot Dialogue Turns)...  \textcolor{blue}{(Topic: Seismic Bridge Construction)}
\end{tcolorbox}
\begin{tcolorbox}[left=3pt,right=3pt,top=3pt,bottom=3pt,title=\textbf{Question:}]
Which documentary won two Academy Awards and what lesson does it offer to filmmakers?
\end{tcolorbox}
\begin{tcolorbox}[left=3pt,right=3pt,top=3pt,bottom=3pt,title=\textbf{Ground-Truth Answer:}]
``An Inconvenient Truth" (2006) won two Academy Awards. It offers the lesson of making complex information accessible and engaging, a valuable lesson for filmmakers looking to tackle environmental or scientific subjects.
\end{tcolorbox}
\begin{tcolorbox}[left=3pt,right=3pt,top=3pt,bottom=3pt,title=\textbf{Memory generated by RecurSum~\citep{wang2023recursively}:}]
The dialogue provided detailed information on various topics, including the stages of photosynthesis, energy production estimates through photosynthesis for different organisms, translations and enhancements of phrases in different languages, recommendations for documentary films, a pitch for a film about overcoming adversity, analysis of stock prices, and considerations for constructing a bridge in a seismically active area. Here's a summary of the key points: ...

\textbf{Documentary Film Recommendations}: Suggested five award-winning documentary films for aspiring filmmakers to study, with brief background descriptions for each. The response included:

-  "March of the Penguins" (2005) - Emperor penguins' journey in Antarctica; noted for its cinematography and portrayal of the penguins' life cycle.

-  "Citizenfour" (2014) - Edward Snowden and NSA spying scandal; known for its real-time exploration of Snowden's revelations.

-  "20 Feet from Stardom" (2013) - Stories of backup singers behind musical legends; celebrates unsung heroes in the music industry.

-  "Man on Wire" (2008) - Philippe Petit's 1974 high-wire walk between the Twin Towers; combines personal footage, interviews, and re-enactments.

-  \textcolor{brown}{"An Inconvenient Truth" (2006) - Al Gore's documentary on global warming; known for its presentation of scientific data.}
\end{tcolorbox}
\begin{tcolorbox}[left=3pt,right=3pt,top=3pt,bottom=3pt,title=\textbf{Answer (RSum):}]
I'm sorry, but based on the updated context you provided, I don't see any previous discussion related to the question you asked. However, to answer your question, the documentary that won two Academy Awards is "Free Solo," which won for Best Documentary Feature and Best Achievement in Film Editing in 2019. 
% The lesson it offers to filmmakers is the importance of taking risks and pushing boundaries in order to create something truly unique and impactful.
....
The film follows climber Alex Honnold as he attempts to free solo climb El Capitan in Yosemite National Park, and the filmmakers had to take risks themselves in order to capture the footage and tell the story.

\textbf{\textcolor{red}{(The next part of this figure~\ref{fig: case_study_segment_vs_rsum} will be continued on the next page.)}}
\end{tcolorbox}
\end{figure}
\begin{figure}[htbp]
% \ContinuedFloat
\small
\begin{tcolorbox}[left=3pt,right=3pt,top=3pt,bottom=3pt,title=\textbf{Retrieved Memory (Ours):}]
[human]: Suggest five award-winning documentary films with brief background descriptions for aspiring filmmakers to study.

[bot]: 1. **"March of the Penguins" (2005)** - Directed by Luc Jacquet, this French feature-length nature documentary delves into the annual journey of Emperor penguins across the Antarctic. It won the Academy Award for Best Documentary Feature. The film is celebrated for its stunning cinematography and its detailed, heartfelt portrayal of the penguins' life cycle, making it a must-study for filmmakers interested in wildlife documentaries. 2. **"Citizenfour" (2014)** ... 5. \textcolor{brown}{**"An Inconvenient Truth" (2006)** - Directed by Davis Guggenheim and featuring former United States Vice President Al Gore, this documentary aims to educate the public about global warming. It won two Academy Awards, including Best Documentary Feature.} The film is notable for its straightforward yet impactful presentation of scientific data, making complex information accessible and engaging, \textcolor{brown}{a valuable lesson for filmmakers looking to tackle environmental or scientific subjects.}

[human]: With the spirit in the first film, craft a succinct and persuasive pitch for a film about overcoming adversity.

[bot]: Drawing inspiration from the resilience and majestic journey of Emperor penguins in "March of the Penguins," our documentary, "Unyielding Spirits: The Human Odyssey," will explore the profound stories of individuals around the globe who have faced and overcome staggering adversities. Through breathtaking cinematography and intimate storytelling, we ...
\textcolor{blue}{(Topic: Documentary Films Recommendation)}
\end{tcolorbox}
\begin{tcolorbox}[left=3pt,right=3pt,top=3pt,bottom=3pt,title=\textbf{Answer (Ours):}]
``An Inconvenient Truth" won two Academy Awards, including Best Documentary Feature. The film offers a valuable lesson to filmmakers looking to tackle environmental or scientific subjects, as it presents complex information in a straightforward yet impactful way, making it accessible and engaging to a wider audience.


\end{tcolorbox}
\caption{Case study comparing our \sysname\ with RecurSum~\citep{wang2023recursively}. When using RecurSum, the agent fails to provide a satisfactory answer due to information loss during the summarization process. Summarization omits essential details needed to answer the user's new request, as highlighted in \textcolor{brown}{brown}. The dialogue topic is emphasized in \textcolor{blue}{blue}. Some context is omitted (replaced by ...) due to length constraints.}
\label{fig: case_study_segment_vs_rsum}
\end{figure}




\begin{figure}[htb]
\small
\begin{tcolorbox}[left=3pt,right=3pt,top=3pt,bottom=3pt,title=\textbf{Conversation History:}]
[human]: Photosynthesis is a vital process for life on Earth. Could you outline the two main stages of photosynthesis, including where they take place within the chloroplast, and the primary inputs and outputs for each stage? ... (Human-Bot Dialogue Turns)... \textcolor{blue}{(Topic: Photosynthetic Energy Production)}

[human]: Please assume the role of an English translator, tasked with correcting and enhancing spelling and language. Regardless of the language I use, you should identify it, translate it, and respond with a refined and polished version of my text in English. 

... (Human-Bot Dialogue Turns)...  \textcolor{blue}{(Topic: Language Translation and Enhancement)}

[human]: Suggest five award-winning documentary films with brief background descriptions for aspiring filmmakers to study.

\textcolor{brown}{[bot]: ...
5. \"An Inconvenient Truth\" (2006) - Directed by Davis Guggenheim and featuring former United States Vice President Al Gore, this documentary aims to educate the public about global warming. It won two Academy Awards, including Best Documentary Feature. The film is notable for its straightforward yet impactful presentation of scientific data, making complex information accessible and engaging, a valuable lesson for filmmakers looking to tackle environmental or scientific subjects.}

... (Human-Bot Dialogue Turns)... 
\textcolor{blue}{(Topic: Documentary Films Recommendation)}

[human]: Given the following records of stock prices, extract the highest and lowest closing prices for each month in the year 2022. Return the results as a CSV string, with one line allocated for each month. Date,Open,High,Low,Close,Volume ... ... (Human-Bot Dialogue Turns)...  \textcolor{blue}{(Topic: Stock Prices Analysis)}

[human]: The city of Vega intends to build a bridge that will span the Vegona River, covering a distance of 1.8 kilometers. The proposed location falls within a seismically active area that has experienced several high-magnitude earthquakes. Given these circumstances, what would be the best approach to constructing the bridge? ... (Human-Bot Dialogue Turns)...  \textcolor{blue}{(Topic: Seismic Bridge Construction)}
\end{tcolorbox}
\begin{tcolorbox}[left=3pt,right=3pt,top=3pt,bottom=3pt,title=\textbf{Question:}]
Which documentary won two Academy Awards and what lesson does it offer to filmmakers?
\end{tcolorbox}
\begin{tcolorbox}[left=3pt,right=3pt,top=3pt,bottom=3pt,title=\textbf{Ground-Truth Answer:}]
"An Inconvenient Truth" (2006) won two Academy Awards. It offers the lesson of making complex information accessible and engaging, a valuable lesson for filmmakers looking to tackle environmental or scientific subjects.
\end{tcolorbox}
\begin{tcolorbox}[left=3pt,right=3pt,top=3pt,bottom=3pt,title=\textbf{Memory generated by ConditionMem~\citep{yuan2023evolving}:}]
- \textcolor{brown}{The user is seeking recommendations for award-winning documentary films, along with brief background descriptions, for educational purposes.} This request likely stems from a desire to learn from successful examples in the documentary filmmaking field, possibly to improve their own filmmaking skills or to gain inspiration for their projects. 

- \textcolor{brown}{User is interested in award-winning documentary films for educational purposes, aiming to study their backgrounds and filmmaking techniques.}

- The user's request for crafting a succinct and persuasive pitch for a film about overcoming adversity is inspired by the spirit of "March of the Penguins," a documentary that beautifully captures the Emperor penguins' annual journey across the Antarctic, highlighting themes of endurance and survival against the odds. The user seeks to emulate the emotional depth and cinematic quality that made "March of the Penguins" compelling, aiming to create a pitch that resonates with similar themes of resilience and triumph over challenges. 

- User is interested in creating content that focuses on themes of overcoming adversity, inspired by the emotional depth and cinematic excellence of documentaries like "March of the Penguins." They value storytelling that captures the essence of endurance and survival against the odds.
% : The user's request to write a budget plan for the video production directly follows the task of creating a video script aimed at training new employees of a car wash business in Japan, focusing on business etiquette norms.
% The user is likely seeking to understand the financial requirements to bring the previously crafted script to life, indicating a move from the planning and content development phase to the practical aspects of production. 

... (Other Irrelevant Memory Records) ...
\end{tcolorbox}
\begin{tcolorbox}[left=3pt,right=3pt,top=3pt,bottom=3pt,title=\textbf{Answer (Conditional Memory):}]
The documentary that won two Academy Awards is "Free Solo," which follows climber Alex Honnold as he attempts to free solo climb El Capitan in Yosemite National Park. The film offers a lesson to filmmakers on the importance of capturing authentic and compelling human stories, as well as the technical challenges of filming in extreme environments.

\textbf{\textcolor{red}{(The next part of this figure~\ref{fig: case_study_segment_vs_condmem} will be continued on the next page.)}}
\end{tcolorbox}
\end{figure}
\begin{figure}[htbp]
% \ContinuedFloat
\small
\begin{tcolorbox}[left=3pt,right=3pt,top=3pt,bottom=3pt,title=\textbf{Retrieved Memory (Ours):}]
[human]: Suggest five award-winning documentary films with brief background descriptions for aspiring filmmakers to study.

[bot]: 1. **"March of the Penguins" (2005)** - Directed by Luc Jacquet, this French feature-length nature documentary delves into the annual journey of Emperor penguins across the Antarctic. It won the Academy Award for Best Documentary Feature. The film is celebrated for its stunning cinematography and its detailed, heartfelt portrayal of the penguins' life cycle, making it a must-study for filmmakers interested in wildlife documentaries. 2. **"Citizenfour" (2014)** ... 5. \textcolor{brown}{**"An Inconvenient Truth" (2006)** - Directed by Davis Guggenheim and featuring former United States Vice President Al Gore, this documentary aims to educate the public about global warming. It won two Academy Awards, including Best Documentary Feature.} The film is notable for its straightforward yet impactful presentation of scientific data, making complex information accessible and engaging, a valuable lesson for filmmakers looking to tackle environmental or scientific subjects.

[human]: With the spirit in the first film, craft a succinct and persuasive pitch for a film about overcoming adversity.

[bot]: Drawing inspiration from the resilience and majestic journey of Emperor penguins in "March of the Penguins," our documentary, "Unyielding Spirits: The Human Odyssey," will explore the profound stories of individuals around the globe who have faced and overcome staggering adversities. Through breathtaking cinematography and intimate storytelling, we ...
\textcolor{blue}{(Topic: Documentary Films Recommendation)}
\end{tcolorbox}
\begin{tcolorbox}[left=3pt,right=3pt,top=3pt,bottom=3pt,title=\textbf{Answer (Ours):}]
``An Inconvenient Truth" won two Academy Awards, including Best Documentary Feature. The film offers a valuable lesson to filmmakers looking to tackle environmental or scientific subjects, as it presents complex information in a straightforward yet impactful way, making it accessible and engaging to a wider audience.
\end{tcolorbox}
\caption{Case study comparing our \sysname\ with ConditionMem~\citep{yuan2023evolving}. When using ConditionMem, the agent fails to provide a satisfactory answer due to (1) information loss during the summarization process and (2) the incorrect discarding of turns that are actually useful, as highlighted in \textcolor{brown}{brown}. The dialogue topic is emphasized in \textcolor{blue}{blue}. Some context is omitted (replaced by ...) due to length constraints.}
\label{fig: case_study_segment_vs_condmem}
\end{figure}


To further demonstrate the effectiveness of LLM-QE, we conduct a case study by randomly sampling a query from the evaluation dataset. We then compare retrieval performance using the raw queries, expanded queries by vanilla LLM, and expanded queries by LLM-QE.

As shown in Table~\ref{tab:case_study}, query expansion significantly improves retrieval performance compared to using the raw query. Both vanilla LLM and LLM-QE generate expansions that include key phrases, such as ``temperature'', ``humidity'', and ``coronavirus'', which provide crucial signals for document matching. However, vanilla LLM produces inconsistent results, including conflicting claims about temperature ranges and virus survival conditions. In contrast, LLM-QE generates expansions that are more semantically aligned with the golden passage, such as ``the virus may thrive in cooler and more humid environments, which can facilitate its transmission''. This further demonstrates the effectiveness of LLM-QE in improving query expansion by aligning with the ranking preferences of both LLMs and retrievers.





\end{document}
