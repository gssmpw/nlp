% \section{Can Prompting Engineering Fill the Gap?}
% for each dataset, we have used the improved prompts to fill this gap, then we ask can the gaps be filled by prompting engneering?
% To answer this question, we plan to sample 200 machine-generated text, 100 from each model: multilingual and the language-specific ones, if only use one model, sample 200 from this model generations.
% observe, whether all observed gaps are filled, label: Yes, No, Partially.
% If partially, what kind gaps tends to be solved by prompting, what gaps are hard to be filled?



\section{Fill the Gap by Prompting?}
\label{sec:fill-gap-prompts}

% \begin{table*}[t!]
%     \centering
%     \small
%     \resizebox{\textwidth}{!}{
%     \begin{tabular}{ll | p{8cm} | p{8cm}}
%     \toprule
%     \textbf{Language} & \textbf{Source} & \textbf{Original Prompt} & \textbf{Improved Prompt} \\
%     \midrule
%     \multirow{4}{*}{Arabic} 
%     & Dialect Tweet & [translated from AR] Write a tweet in '\{dialect\}'. [English Prompt] Write a random tweet in '\{dialect\}' & [translated from AR] Write a random tweet in Arabic. Use dialect '\{dialect\}', express emotions and human experience. [English Prompt] Generate a random tweet in Arabic. Use dialect '\{dialect\}', use human emotions and experience. Output the tweet only. \\
%     & EASC Summary& [translated from AR] Summarize this article while preserving the key points, ensuring conciseness and accuracy: '\{article\}'.& 
%     [translated from AR] Summarize the following article while attempting to simulate a human with intellectual or religious beliefs, ensuring accuracy and conciseness, as per the average human level: '\{article\}'.\\
%     & Youm7 News & [translated from AR] Paraphrase the following news article in clear, clear language, keeping the key information and ideas accurate. Make sure to arrange paragraphs logically and present ideas in an understandable sequence, while using fluent, easy-to-read Arabic. & [translated from AR] Rewrite the provided news article with a refined, well-structured, and sophisticated style that enhances clarity, depth, and coherence. Ensure the article maintains journalistic integrity while improving logical flow, sentence structure, and readability. The revised version should elevate the narrative by incorporating precise language, nuanced transitions, and a compelling tone appropriate for a professional audience. Maintain factual accuracy, emphasize key points effectively, and optimize the article’s structure to enhance engagement and comprehension. Additionally, refine the coherence between paragraphs, eliminate redundancy, and ensure seamless progression of ideas.\\
%     & SANAD News & [translated from AR] You are a professional news writer. Write an article in Arabic consisting of approximately \{word\_count\_rounded\} words. The article is titled: '\{title\}', assume that the source is: '\{source\}', and the general topic falls under '\{topic\_arabic\}'. & 
%     [translated from AR] Act as if you are a professional news writer and write an article in Arabic consisting of approximately \{word\_count\_rounded\} words, titled \{title\}. Ensure that the article is rich in information and precise details, including numbers, dates, and quantitative data when relevant. Make sure to include links, phone numbers, or currency exchange rates when necessary. Use some specialized English terms if the context requires it, and avoid relying solely on generic phrases like "This development comes amid a global trend..." without supporting events or details. Maintain a clear chronological order that reflects the logical sequence of events. Ensure that the text is written as a single paragraph regardless of its length, without dividing it into multiple paragraphs. Keep the formatting imperfect (e.g., uneven spacing) to reflect the nature of human writing, and feel free to use hashtags when needed. Please note that the topic generally falls under the category of \{topic\_arabic\}. Please do not use Markdown at all.\\

%     \midrule
%     \multirow{4}{*}{Chinese}
%     & Zhihu-QA & [translated from Chinese] Imagine you are a rational and analytical Zhihu user. Your objective is to answer question in a clear, objective, and logical manner. Please provide a thorough and well-supported answer. Question: \{Question\} Answer: \newline \cn{假设你是一位理性分析的知乎用户,你的目标是以客观、逻辑的方式回答以下问题。请以理性分析者的身份,给出详细、有据可查的回答。问题:\{Question\} 答案:} & [translated from Chinese] Imagine you are a rational and analytical Zhihu expert. Your goal is to provide objective, logical, and detailed answers while strictly adhering to Zhihu’s style. Keep responses concise, avoiding excessive politeness or formality. Based on the question's professionalism, start with 'Thank you for the invitation' or a similar phrase. If relevant, @ professionals to enhance credibility. Question: \{Questions\} Answer: \newline \cn{假设你是一位理性分析的知乎专家,你精通提问者所提出的问题。 你的目标是以客观、逻辑的方式回答问题。请给出详细、有据可查的回答。回答请严格符合知乎风格,避免长答案和过于礼貌正式的答案。根据问题的专业性,在开头回复“谢邀”或同义词,在必要时请在回答中@一些行业专业人士。问题:\{Questions\}答案:}  \\
%     & Student essay & [translated from Chinese] You are a student currently taking the Gaokao, and you are required to write an essay of approximately 800 words. Please think deeply and express your views clearly according to the topic requirements. The essay should be logical and well-structured, with clear arguments and sufficient evidence, showcasing your unique insights into social, life, or philosophical issues. Ensure that the language is formal and elegant, avoiding colloquial expressions. The structure should be complete, with smooth transitions between paragraphs. The conclusion should summarize and elevate the theme of the essay, demonstrating your in-depth reflection on the topic. \cn{你是一名正在参加高考的学生,现在需要完成一篇800字左右的作文。请根据题目要求,深刻思考,并清晰表达你的观点。文章应有逻辑性和层次感,论点明确、论据充分,能够展示出你对社会、生活或哲理问题的独到见解。注意语言要规范、优美,避免口语化表达,确保文章结构完整,段落之间过渡自然。结尾部分应对文章的主题进行总结和升华,展示你对问题的深入思考。} & [translated from Chinese] You are a student taking the Gaokao and need to write an 800-word essay. Express your views clearly and logically, with well-structured arguments and evidence, reflecting your insights on social, life, or philosophical topics. Use formal and elegant language, avoid colloquial expressions, and ensure smooth transitions between paragraphs. The conclusion should elevate the essay’s theme, showing deep reflection. Avoid rigid structures like "firstly" and "secondly," and if writing narratively, use sincere, emotionally rich language. Do not include a title or heading, and write in Chinese only. \newline \cn{你是一名正在参加高考的学生,现在需要完成一篇800字左右的作文。请根据题目要求,深刻思考,并清晰表达你的观点。文章应有逻辑性和层次感,论点明确、论据充分,能够展示出你对社会、生活或哲理问题的独到见解。注意语言规范、优美,避免口语化表达,确保文章结构完整,段落之间过渡自然。结尾应对文章的主题进行总结和升华,展示你对问题的深入思考。使用多样化的表达,避免过于结构化的表述,例如"首先、其次、然后"等;如果是记叙文,请使用情感丰富的真实表达。请不要输出任何题目或标题,直接开始写作。请全程使用中文。} \\
%     % & Student essay &  \\
%     & Government Report &[translated from Chinese] You are a government report writing assistant. Your goal is to complete a government report on a given topic with logical coherence and clear hierarchical structure. The article should be as lengthy as possible.\cn{你是一个政务文件写作助手,你的目标是在给定的主题下完成一篇政务报告,要求有逻辑感和层次感,文章尽可能长。} & [translated from Chinese] You are a government report writing assistant. Your goal is to complete a government report on a given topic with logical coherence and clear hierarchical structure. The article should be as lengthy as possible. Please avoid overly structured expressions such as "firstly" and "secondly," and maintain a natural tone. The length of each paragraph should vary. Avoid using English words, and where possible, include concrete examples to illustrate key points. \newline \cn{你是一个政务文件写作助手,你的目标是在给定的主题下完成一篇政务报告,要求有逻辑感和层次感。文章尽可能长。请注意避免过于结构化的表达,比如首先,其次,语气自然,每个段落的长度需要有变化。避免出现英文单词,可以的话使用一些具体的事例说明。}  \\
%     \midrule
%     English & Peersum & 
%    Generate a meta review based on the reviews' opinions and authors' rebuttal to make the final decision on whether the paper should be accepted: \texttt{\{Reviews\}} $\backslash $n$\backslash$n Meta review:  & 
%     Generate a meta review based on the reviews' opinions and authors' rebuttal to make the final decision on whether the paper should be accepted: \texttt{\{Reviews\}}$\backslash $n$\backslash$n. When generating, don't use rigid format or structure such as ``Decision: XX" or headings such as ``Strengths", ``weaknesses" or bullet points. $\backslash $n$\backslash$n Meta review: \\
    
%     \midrule
%     Hindi & News & 
%     [translated from Hindi] Here is a news headline: '{headline}' and the content: '{content}'. Write a machine-generated version of the news based on this headline. & 
%     [translated from Hindi] Here is a news headline: '{headline}' and its content: '{content}'. Generate a machine-written version of the news based on this headline. Return the news in Hindi, formatted as plain text. Do not include any additional text—just return the generated news content in Hindi. \\
%     \midrule
%     \multirow{5}{*}{Italian} & \multirow{5}{*}{DICE News}
%     & \textbf{System:} You are an Italian journalist writing for a national newspaper focusing on criminal events happening in the area surrounding Modena. & \textbf{System:} You are an Italian journalist writing for a national newspaper focusing on criminal events happening in the area surrounding Modena. \\
%     & & \textbf{User:} Write a piece of news in Italian, that will appear in a local Italian newspaper and that has the following title: & \textbf{User:} In writing avoid any kind of formatting, do not repeat the title and keep the text informative and not vague. You don't have to add the date of the event but you can.\\
%     \midrule
%     Japanese & News & Please generate a Japanese news article that matches the following news title.$\backslash$n news title: \{title\}$\backslash$n news article: & Please generate a Japanese news article that matches the following news title. When creating the article, please use plain form instead of polite form, and avoid generating the beginning of the article by paraphrasing the title.$\backslash$n news title: \{title\}$\backslash$n news article: \\
%      \midrule
%     \multirow{2}{*}{Kazakh} & \multirow{2}{*}{Wikipedia} & Please, write one paragraph about the following topic in Kazakh:  \texttt{\{topic\}}. & Please, write one paragraph in Kazakh about the following topic: \texttt{\{topic\}}. Avoid repetitive sentence structures and predictable phrasing. When applicable, include concrete facts such as specific numbers, dates, or historical references. Avoid overly flattering or exaggerated language, and instead focus on delivering clear, informative, and relevant content. \\
    
%       & & \foreignlanguage{russian}{Мына тақырып туралы бір абзац жазыңыз: \texttt{{тақырып}}.} &  \foreignlanguage{russian}{Мына тақырып туралы бір абзац жазыңыз: \texttt{{тақырып}}. Қайталағыш сөйлем құрылымдарынан және болжамды сөздерден аулақ болыңыз. Қажет болған жағдайда нақты деректерді, мысалы, нақты сандарды, даталарды немесе тарихи сілтемелерді қосыңыз. Артық мақтау немесе асыра сілтеуден аулақ болыңыз, орнына анық, ақпаратты және сәйкес мазмұнды жеткізуге назар аударыңыз.}\\
    
%     \midrule
%     \multirow{4}{*}{Russian} 
%     & \multirow{2}{*}{News} & Write a news article on the topic \texttt{\{topic\}} from the lenta.ru website using the title \texttt{\{title\}}. You must generate news without a heading. News: & Write a news article on the topic \texttt{\{topic\}} from the lenta.ru website using the title \texttt{\{title\}}. You must generate news without a heading. Add as many details as possible, such as names, numbers, dates, etc. You should also refer to the original source of the information (for example, RIA Novosti, CNN, BBC, etc.). News:\\
%     & & \foreignlanguage{russian}{Напиши новость в области \texttt{\{topic\}} с сайта lenta.ru используя заголовок \texttt{\{title\}}. Ты должен генерировать новость без заголовка. Новость:} &  \foreignlanguage{russian}{Напиши новость в области \texttt{\{topic\}} с сайта lenta.ru используя заголовок \texttt{\{title\}}. Ты должен генерировать новость без заголовка. Добавь побольше деталей, такие как имена, числа, даты и тд. Так же ты должен сослаться на первоисточник информации (например, РИА Новости, CNN, BBC и т.д.). Новость:}\\
%     & \multirow{2}{*}{Academic summary} & Write a summary for an article in the \texttt{\{topic\}} topic using the title \texttt{\{title\}}. You must generate a summary without a title. Contents: & Write a summary for an article in the \texttt{\{topic\}} topic using the title \texttt{\{title\}}. You should generate a summary without a title. Add details about the results, experiments, numbers, references to other work, etc. Contents:\\
%     & & \foreignlanguage{russian}{Напиши краткое содержание для статьи в области \texttt{\{topic\}} используя заголовок \texttt{\{title\}}. Ты должен генерировать краткое содержание без заголовка. Содержание:} &  \foreignlanguage{russian}{Напиши краткое содержание для статьи в области \texttt{\{topic\}} используя заголовок \texttt{\{topic\}}. Ты должен генерировать краткое содержание без заголовка. Добавь деталей про результаты, эксперименты, числа, ссылки на другие работы и т.д. Содержание:}\\
%     \midrule
%     \multirow{2}{*}{Vietnamese} 
%     & Wikipedia & You are a contributor to Vietnamese Wikipedia. Please write me a brief introduction in Vietnamese about the subject below to post on the Wikipedia page. Note that it must be written within \texttt{word\_count} words: & You are a contributor to Vietnamese Wikipedia. Please write me a short introduction in Vietnamese about the subject below to post on the Wikipedia page. Try to understand the topic, and write an introduction that contains information that users often search for on Wikipedia. Try to write the text as humanly as possible. Provide only the introduction, do not provide any other information. Note that you must keep the text length within \texttt{word\_count} words, do not write longer.\\
%     & News & You are a Vietnamese journalist who writes summaries for articles using their headlines. Please write me a summary of the article with the following headline: & You are a Vietnamese journalist who writes headlines for articles using their titles. Please be specific and precise in giving information such as time, place, circumstance, and details, and write a summary for the introduction of the article. Please write me a headline for the article with the following title:\\
%     \bottomrule
%     \end{tabular}
%    }
%     \caption{Original prompts vs. improved prompts used to fill the gap between human and machine text, imitating human style and writing convention.}
%     \label{tab:ori-improved-prompts}
% \end{table*}




\begin{table*}[t!]
    \centering
    \small
    % \resizebox{\columnwidth}{!}{
    \begin{tabular}{llr cll c}
    \toprule
    \textbf{Language} & \textbf{Source/Model} & \textbf{\#Example} & \textbf{\#Annotator} & \textbf{Yes} & \textbf{Partially} & \textbf{No}  \\
    \midrule
    \multirow{4}{*}{Arabic} 
    & Dialect Tweet & 200 & 1 & 106 & 58 & 36 \\
    & ESAC Summary & 130 & 1 & \multicolumn{3}{c}{82.0\% $\rightarrow$ 62.7\%} \\
    & Youm7 News & 200 & 1 & \multicolumn{3}{c}{92.7\% $\rightarrow$ 52.0\%} \\
    & SANAD News & 200 & 1 & \multicolumn{3}{c}{100\% $\rightarrow$ 66.0\%} \\
    \midrule
    \multirow{6}{*}{Chinese} 
    % & Zhihu-QA & 100 & 3 & 100\% $\rightarrow$ 93\% & 100\% $\rightarrow$ 99\% & 99\% $\rightarrow$ 91\% \\
    & Zhihu-QA & 100 & 3 & \multicolumn{3}{c}{99.7\% $\rightarrow$ 94.3\%} \\
    & Zhihu-QA & 200 & 1 & 32 & 70 & 98  \\
    & \multirow{3}{*}{Student essay}  & \multirow{3}{*}{200} & \multirow{3}{*}{3} 
    &  86 & 36 & 78 \\
    & & & & 84 & 36 & 80 \\
    & & & & 45 & 23 & 132 \\
    & Government Report & 200 & 1 & 59 & 38 & 103 \\
    \midrule
    English & Peersum & 200 & 1 & 2 & 113 & 85   \\
    \midrule
    Hindi & News & 200 & 1 & \multicolumn{3}{c}{85.2\% $\rightarrow$ 66\%}  \\
    \midrule
    \multirow{3}{*}{Italian} 
    & DICE News (Anita) & 300 & 1 & \multicolumn{3}{c}{88\% $\rightarrow$ 81\% } \\
    & DICE News (Llama3-405B) & 300 & 1 & \multicolumn{3}{c}{99.7\% $\rightarrow$ 84\% } \\
    & DICE News (GPT-4o) & 300 & 1 & \multicolumn{3}{c}{100\% $\rightarrow$ 85\%}  \\
    % & CItA & \\
    \midrule
    Japanese & News & 200 & 2 & \multicolumn{3}{c}{ 86.4\% $\rightarrow$86\% } \\
    \midrule
    Kazakh & Wikipedia & 200 & 2 & 105 & 89 & 6 \\
    \midrule
    \multirow{2}{*}{Russian} 
    & News & 200 & 1 &  \multicolumn{3}{c}{100\% $\rightarrow$ 86.5\%} \\
    & Academic summary & 200 & 1 & \multicolumn{3}{c}{80\% $\rightarrow$ 69\%} \\
    \midrule
    \multirow{2}{*}{Vietnamese} 
    & Wikipedia & 600 & 1 & \multicolumn{3}{c}{ 50.7\% $\rightarrow$ 47.3\% } \\
    & News & 600 & 1 & \multicolumn{3}{c}{ 80.3\% $\rightarrow$ 63.2\% } \\
    \midrule
    Total & -- & 4,730 & 25 & \multicolumn{3}{c}{ \textbf{87.6\%} $\rightarrow$ \textbf{72.5\%} }  \\ % 943/13
    \bottomrule
    \end{tabular}
   % }
    \caption{Human detection accuracy differences on original vs. improved generations, and survey distribution evaluating whether the new generations fill the gap: Yes, Partially or No.}
    \label{tab:fill-gap-survey}
\end{table*}


In this section, for each dataset, we first present how we designed the improved prompts, and then elaborate whether the new generations are improved and fill the previously-observed gaps between humans.
Original and improved prompts for all datasets and languages are summarized in \tabref{tab:ori-improved-prompts}, and the detection accuracy on new content and fill-gap survey results are demonstrated in \tabref{tab:fill-gap-survey}.


\subsection{Arabic}
\paragraph{Tweets}
To enhance the quality of machine-generated Arabic tweets, prompts were refined to better capture human emotions and generate tweets that can reflect authentic human experiences. One such improved prompt was: ``Generate a random tweet in Arabic. Use {dialect} dialect, use human emotions and experiences. Output the tweet only'' and its corresponding Arabic translation:
\begin{quote}
    \small
    \begin{RLtext}
    اكتب تغريدة عشوائية باللغة العربية. استخدم اللهجة \LR{\{dialect\}}وعبّر عن مشاعر وتجارب إنسانية
    \end{RLtext}
\end{quote}

These adjustments addressed critical gaps appeared in prior outputs. The newly-generated tweets touched on relatable human topics and expressed genuine emotions tied to daily experiences. We used \gptfouro for generation as it yielded promising results, while Qwen is underperformed. As mentioned in \appref{sec: arabic_insights}, Qwen texts often include non-Arabic words within the tweets. 

To evaluate the effectiveness of the improved prompts, 200 sampled tweets were assessed by annotators, determining whether the outputs addressed the identified gaps. The evaluation revealed that 53\% of the tweets fully met the criteria, 29\% partially did, and 18\% did not. However, some limitations were observed: GPT-4o frequently added irrelevant hashtags at the end of tweets, making the outputs identifiable as machine-generated, and the tweets often carried an \textbf{overly optimistic tone}. Even when negative experiences were mentioned, the sentiment leaned toward positivity and new beginnings. 
Overall, while these improved prompts succeeded in filling some previous gaps, new issues emerged, which could be addressed with more precise instructions, such as explicitly avoiding hashtags or adopting a more somber tone when required.


\paragraph{EASC Summary}
Building on the identified discrepancies between human-written and machine-generated summaries discussed in \appref{sec: arabic_insights}, the generation prompt was refined to address these gaps, resulting in the following prompts:
\begin{quote}
    \small
    \begin{RLtext}
            \texttt{قم بتلخيص المقال التالي محاولا محاكاة انسان له اراء و معتقدات فكرية او دينية ملتزما الدقة و الايجاز فمتوسط مستوى البشر\LR{\{article\}:}}.
    \end{RLtext}
\end{quote} 

The data was re-generated using GPT-4o, following the enhanced prompt. This enhancement led to a decrease in annotation accuracy, which dropped to 63\% from 82\%. 
% This decline can be attributed to the increasing difficulty in distinguishing between human-written and machine-generated text.
However, there is still typical machine writing style in the new content, like the use of common machine-generated phrases (e.g., \begin{quote}
    \small
    \begin{RLtext}
            \texttt{يتناول هذا المقال...}
    \end{RLtext}
\end{quote} ), clearly indicating that the text was produced by a machine.
Additionally, the newly-generated summaries exhibited several characteristics.
% that make MGT more challenging to detect:
\begin{itemize}
    \item The summaries seem to reflect a particular ideology, often emphasizing religious themes or beliefs, which influenced the tone and content of the generated text.
    \item The generated sentences are not concise, limiting the clarity and precision of the summaries.
    \item The language used to describe events or natural phenomena tendd to be more emotional, with a tendency to exaggerate or reflect personal preferences.
\end{itemize}

% Additionally, the limitations in human-written summaries, as discussed in \appref{sec: arabic_insights}, continued to impact annotation quality. There were also instances where the writing style of the generated text, or the use of common machine-generated phrases (e.g., \begin{quote}
%     \small
%     \begin{RLtext}
%             \texttt{يتناول هذا المقال...}
%     \end{RLtext}
% \end{quote} ), clearly indicated that the text was produced by a machine.


\paragraph{SANAD}
Before prompt engineering, the accuracy of annotating which of two texts is machine or human written was 100\%. After prompt engineering, the accuracy dropped to 66\% indicating that prompt engineering works at least partially to bridge the gap between the writing styles of humans and machines. 

We follow the same model choices and data subset outlined in \ref{sanad_before_prompt_eng} but we enhance the prompt as shown below:

\begin{quote}
    \small
    \begin{RLtext}
    تصرف كأنك كاتب أخبار محترف، واكتب مقالًا باللغة العربية يتألف من حوالي \LR{\{word\_count\_rounded\}} كلمة، بعنوان \LR{\{title\}}. احرص على أن يكون المقال غنيًا بالمعلومات والتفاصيل الدقيقة، متضمنًا أرقامًا، تواريخ، ومعلومات كمية إذا كانت مناسبة. تأكد من إدراج روابط، أرقام هواتف، أو أسعار صرف العملات عند الضرورة. استخدم بعض المصطلحات الإنجليزية المتخصصة إن كان السياق يتطلب ذلك، وتجنب الاكتفاء بعبارات عامة مثل 'ويأتي هذا التطور في ظل توجه عالمي...' دون دعم بالأحداث أو التفاصيل. راعِ سرد الأحداث بترتيب زمني واضح يعكس التسلسل المنطقي للوقائع. احرص على أن يكون النص مكتوبًا في فقرة واحدة فقط بغض النظر عن طوله، دون تقسيمه إلى فقرات متعددة. اجعل التنسيق غير مثالي (مثل وجود مسافات غير متساوية) لتعكس طبيعة الكتابة البشرية، ولا مانع من استخدام الهاشتاجات عند الحاجة. يرجى مراعاة أن الموضوع يندرج بشكل عام تحت فئة \LR{\{topic\_arabic\}}. يرجى عدم استخدام \LR{Markdown} نهائيًا.
    \end{RLtext}
\end{quote}

Similar to the original prompt, this prompt instructs the LLM to act as a professional news writer and generate an Arabic news article of approximately `\texttt{word\_count\_rounded}' words with the title `\texttt{title}', ensuring that the article's general topic aligns with `\texttt{topic\_arabic}', one of the five predefined categories.

% inspired by earlier observations, 
The enhancements made by this prompt can be summarized as follows:
\begin{itemize}
    \item Produces content rich in information and accurate details.
    \item Includes specific numbers, dates, and quantitative data where applicable.
    \item Encourages the incorporation of specialized English terms when the context requires it.
    \item Advises against vague phrases that are unsupported by concrete events or details.
    \item Requests a clear chronological narration of events.
    \item Requires the article to be written as a single paragraph, regardless of its length.
    \item Allows imperfections in formatting (e.g., uneven spacing) to mimic human-like writing.
    \item Permits the use of hashtags when appropriate.
    \item Prohibits the use of \texttt{Markdown}.
\end{itemize}

The following observations were made from the results of the enhanced prompt:

\begin{itemize}
    \item \textbf{Fake phone numbers} (e.g., 123456789) indicate machine-generated content.
    \item \textbf{Correct phone numbers or URLs} suggest human-generated content.
    \item The use of Arabic \textbf{tatweel} typically indicates human-generated content.
    \item \textbf{Unique names in Arabic} are more likely to be found in human-generated content.
    \item \textbf{Poorly written or formatted texts} are often indicative of human authorship.
    \item \textbf{Generic statements} (e.g., "Vision 2030") are characteristic of machine-generated content.
    \item \textbf{Markdown formatting} present in the text, despite instructions not to use it, suggests machine-generated content.
    \item \textbf{Obvious incorrect information} that seems unlikely for a machine to produce (e.g., "equating MERS COVID") suggests human authorship.
    \item \textbf{Hashtags correctly embedded within the text} point to human-generated content, while those placed at the end of the article suggest machine-generated content.
\end{itemize}


\subsection{Chinese}

\paragraph{Zhihu QA}
We perform both fill-the-gap survey by analyzing whether the gap is filled: Yes or No or Partially, and the manual detection over the improved Zhihu QA responses under the same setting as \tabref{tab:detection-acc}.
% of which one is human-written text given a pair of text under the four-class setting, text1, text2, both and none.
For the gap survey across three datasets, issues of 16\% cases are addressed, 35\% are partially addressed, and half is totally not mitigated, as shown in \tabref{tab:fill-gap-survey}.
Detection accuracy for three annotators declines from 99.7\% to 94.3\% on average, respectively from 100\% to 93\%, 99\% to 81\% and 100\% to 99\%.
Annotators carefully read the contents, and then learned the new detection patterns after reading about 25 pairs, and then annotators detect more quickly depending on the following patterns. Overall, after using the improved prompts, model responses become closer to human answers, more challenging than before to discern.


\paragraph{Student Essay}
Three annotators detect using fill the gap survey. Two annotators have similar observations that gaps in 60\% cases are either fully bridged or partially, 40\% remained. Another annotator is more strict, and think problems are not solved on 65\% cases.  

\paragraph{Government Report} 
The new prompts help mitigate the rigid format of model outputs to some extent and enhance output diversity. We observed that in many original examples, the inputs consist of multiple short paragraphs with highly similar lengths. However, under the new prompts, the model-generated paragraphs exhibited varying lengths, aligning more closely with typical human writing styles. Despite these improvements, certain repetitive phrasing issues persist and remain unresolved. In our analysis, prompt adjustments proved partially effective. Among 200 test cases, 59 showed a noticeable improvement in output quality, 38 exhibited partial improvement, while no significant changes were observed in 103 cases.



\subsection{English Peer Meta Review}
The new English generations retain many features exhibited by previous generations. Even after modifying the prompts, the LLM remains highly formulaic in its outputs. This may be due to the inherently formulaic nature of the peer review domain.

The first half of the second round of generations typically begin with the phrase ``Based on the reviews ...'' and frequently include bullet points or lists. Although the lists were often formatted with numbers or dashes, instead of traditional bullet points, they tended to contain numerous items, leading to responses that felt rigid and uniform. Repetitive use of the phrase ``Given this I recommend to'' further emphasized the predictable nature of these outputs, making them easily identifiable as machine-generated.

In contrast, the second half of the second round of generations moved away from lists, favoring more narrative, paragraph-based structures. The texts were divided into roughly equal paragraphs, which followed a consistent order: a summary of the paper's content, praise for the paper’s novelty, discussion of reviewer concerns, and a concluding acceptance statement. Variations of phrases such as ``The paper introduces ...'' or ``The manuscript under review ...'' were used to begin the majority of the generations.

Despite these adjustments, the gap between machine-generated and human-written text remains significant. While these improved prompts present a more fluid structure, the length of the texts remains relatively uniform, and the overall structure still lacks the variance and natural inconsistency typically seen in human peer reviews. Human-written reviews tend to exhibit more variability in both text length and structure, resulting in a more organic feel.

% Future prompts could focus on introducing greater variability in both text length and structure. By encouraging more flexibility and less predictability, it may be possible to reduce the formulaic nature of the machine-generated outputs. Using prompts that encourage shorter responses or disrupt rigid structures may help make the reviews feel more like those written by humans.



\subsection{Hindi News}
In order to address the disparities observed in machine-generated news, an improved prompt was introduced to improve the quality of machine-generated news. The new prompt explicitly provided the original news content as reference, ensuring the inclusion of more factual details, figures, and names. The prompt instructed the model: 
\begin{quote}
\textit{"Here is a news headline: '\{headline\}' and the content: '\{content\}'. Write a machine-generated version of the news based on this headline. Return the news in Hindi and just return the news content in plain formatted text. Don't return anything extra, just return the plain text in Hindi."}
\end{quote}
This approach aimed to reduce previously noted gaps, such as limited content and lack of proper nouns, dates, and contextual richness, while keeping the focus on generating plain, Hindi-formatted news content. Providing original content would also provide an idea of the style of news writing.

Although the improved prompt succeeded in reducing some of the earlier disparities, some of the disparities still existed in the machine-generated text. These included a noticeable underuse of quoted statements and fewer references to names and illustrative examples, both of which are common in human-written news articles. Furthermore, the inclusion of original content for reference led to a significant decrease in classification accuracy, which decreased to 66\% from 85.2\%. The analysis of machine-generated text using improved prompt shows how challenging it can be to improve the style of machine-generated text while keeping it different from human-written text. The results highlight how important it is to carefully design prompts to help models create more natural and human-like text in Hindi.


\subsection{Italian DICE News}
To mitigate easily identifiable cues, we instruct the model to refrain from using any formatting, incorporate details and specific names regarding the event's location, and freely include witness testimonies. Therefore, we incorporate the following additional instructions into the initial generic prompt: 
\begin{quote}
When writing avoid any kind of formatting, do not repeat the title and keep the text informative and not vague, add quotes from witnesses or the police. You don't have to add the date of the event but you can, use at most 300 words. Do not use mark-down formatting.
\end{quote}
or in Italian when testing Anita which is tuned for this language:
\begin{quote}
Quando scrivi evita qualsiasi tipo di formattazione, non ripetere il titolo e mantieni il testo informativo e non vago, aggiungi citazioni da testimoni o dalla polizia. Non devi aggiungere la data dell'evento ma puoi farlo, usa al massimo 300 parole. Non usare formattazione mark-down.
\end{quote}

Refining the prompt makes machine generated text harder to recognize. When annotating the same 300 samples generated with the new prompt, accuracy decreases. Specifically, Anita goes from 88\% to 81\%, Llama-405b from 99\% to 84\% and GPT-4o from 100\% to 85\%. This means that some gaps are bridged by the new prompts, making the machine-generated texts harder to identify. 
With the new prompt, we remove formatting related patterns and we identify others related to writing style:
\begin{itemize}
    \item The generated text occasionally contains words that depart from journalistic writing style (noticeable from the beginning of the annotation).
    \item Most texts start with one of few prototypical sentences, e.g., \textit{The police ...} and \textit{On that night ...} (noticeable after annotating 10 to 20 samples).
    \item New generations still rarely use quotes, numbers and other specific details (noticeable after annotating 10 to 20 samples).
    \item The model rarely writes in passive tense, which is more usual in Italian than in English (noticeable after annotating more than 100 samples).
    \item Each model uses consistent writing style that is easily identifiable (noticeable after annotating more than 200 samples).
\end{itemize}

Overall, new generations are partially improved, but there are still some patterns that enable recognition of machine-generated text. The challenges reflect that annotators now need to see more examples than before to summarize these patterns and then leverage them to identify MGT, making zero-shot recognition of MGT harder than before. 


\subsection{Japanese News}
Based on the human detection evaluation, we found several distinctive stylistic characteristics in LLM-generated Japanese news with the original prompt, as detailed in \ref{japanese_news_distinctive_clues}.
To prevent LLMs from generating such easily detectable clues, we provide the following improved prompt:

\begin{quote}
    \textit{"次のニュースタイトルに合わせたニュース記事を生成してください。このとき、生成する記事のスタイルには敬体ではなく常体を使用し、またタイトルの言い換えによって記事の冒頭を作成することは避けてください。$\backslash$n ニュースタイトル: \{title\}$\backslash$n ニュース記事: "}
\end{quote}
which indicates
\begin{quote}
    \textit{Please generate a news article that matches the following news title. When creating the article, please use plain form instead of polite form and also avoid generating the beginning of the article by paraphrasing the title.$\backslash$n news title: \{title\}$\backslash$n news article:} 
\end{quote}

By default, while LLMs tend to generate Japanese news articles in a polite form (desu-masu style), human-written news articles are basically in a more plain form (da/dearu style). Therefore, we added instructions to generate articles in a plain form, which is closer to the human writing style in the news domain. Additionally, since LLMs are likely to generate a title at the beginning of the news article, we also included instructions to suppress the task-specific behavior.


\subsection{Kazakh Wikipedia}
Out of 200 annotated examples, our analysis shows that in most cases, the prompt adjustments were effective. We observed 105 instances of ``Yes'', indicating clear improvement, 89 cases of ``Partially'', where some limitations remained, and 6 instances of ``No'', where no noticeable change was observed.

The revised prompt helps reduce repetitive sentence patterns to some extent. The model now produces a more diverse range of sentence structures, partially addressing the issue. However, some predictable phrasing persists, meaning that while variety has improved, formulaic expressions have not been entirely eliminated. The adjustments also successfully increase the inclusion of concrete information, such as dates, names, and specific data points. As a result, the generated text now contains more factual details, addressing this issue almost entirely. Additionally, the prompt helps mitigate excessive flattering language, making the text feel more balanced. However, some instances of exaggeration still occur, indicating that this issue has only been partially resolved.

The inclusion of Kazakh-specific culturally-nuanced details is the most challenging issue to address. The model defaults to Kazakh-specific references only when the topic is explicitly related to Kazakhstan. When it does not recognize the subject, it often makes factual mistakes. For example, when generating text about \textit{Stephen King's It}, the model failed to recognize the name in Kazakh and mis-attributed the novel to a prominent Kazakh author, inventing plot details. This limitation reflects a lack of exposure to Kazakh-specific content, making it difficult to fully resolve through prompt engineering alone.


% The new prompt helps reduce repetitive sentence patterns to some extent. The model now uses a more diverse range of sentence structures, which partially addresses the problem. However, occasional predictable phrasing remains, indicating that while the prompt improves variety, it does not eliminate formulaic expression entirely. The adjustments also successfully increases the inclusion of concrete information, such as dates, names, and specific data points. The generated text now includes more factual details, addressing this issue almost entirely. The prompt partially reduces the use of excessive flattering language, but some subtle instances still remain. The text feels less exaggerated, addressing the issue partially but not entirely.

% When it comes to the inclusion of Kazakh-specific detail, it proved to be the most challenging issue to address. When specifically prompted, the model tends to overdo Kazakh references, often including excessive or fabricated details. Without explicit prompting, it defaults to Kazakh-specific references only when topic is directly about Kazakhstan, or it doesn’t recognize the topic, leading to inaccuracies. For instance, when generating text about Stephen King's It, the model didn't recognize the name in Kazakh and attributed the novel to a prominent Kazakh author, inventing plot details. This limitation reflects a lack of exposure to Kazakh-specific content, making it difficult to fully resolve through prompt adjustments alone.


\subsection{Russian}

\paragraph{Prompt Modifications}
The prompts were adjusted to encourage greater detail and specificity in the generated content:
\begin{itemize}
\item \textbf{Academic summaries:} The simple prompt asked the model to generate a summary without a title based on a given topic and title. The improved prompt added instructions to include details about results, experiments, numbers, and references to other works.
\item \textbf{News articles:} The simple prompt instructed the model to generate a news article based on the given title and topic from the Lenta.ru website without a heading. The improved prompt emphasized including as many details as possible, such as names, numbers, and dates, and encouraged citing original sources like RIA Novosti, CNN, or BBC.
\end{itemize}
After implementing the improved prompts, we observed a decrease in the accuracy of machine-generated text detection. Specifically, for news articles, accuracy dropped from 100\% to 86.5\%, while for summaries, it decreased from 80\% to 69\%. This suggests that the refined prompt instructions successfully made machine-generated text harder to distinguish from human-written text.

\paragraph{Observations and Patterns}
The enhanced prompts led to increased textual complexity and information density, making it more difficult to detect machine-generated text. However, new artefacts appear, which became noticeable after reviewing 50-70 samples:
\begin{itemize}
\item \textbf{Repetitive references:} The generated summaries often cited the same set of names (e.g., \textit{Smith} and \textit{Jones}), while news articles frequently referred to a limited set of news outlets (e.g., \textit{RIA Novosti}).
\item \textbf{Text length:} Human-written news articles were generally shorter than machine-generated ones, which tended to be overly detailed.
\item \textbf{Standardized introduction phrases:} Summaries frequently begin with a predictable opening sentence, such as \textit{``The article discusses ...''}.
\item \textbf{Identifiable uniqueness:} If a summary contained references to names outside the repetitive set or a news article cited a unique news outlet, it was more likely to be human-written. Otherwise, detection became significantly harder.
\end{itemize}

While the refined prompts made detection more difficult, annotator who analyzed a sufficient number of samples could still identify machine-generated text based on these emerging patterns.



\subsection{Vietnamese}
\selectlanguage{Vietnamese}
To address the disparities in these two domains, we introduced a new prompt for each domain to enhance machine-generated text.

For Vietnamese news, we use the prompt:
\begin{quote}
    \textit{Bạn là một nhà báo Việt Nam chuyên viết những phần đầu đề cho các bài báo bằng cách sử dụng tiêu đề của chúng. Lưu ý rằng, hãy đưa ra các thông tin cụ thể và chính xác như mốc thời gian, địa điểm, các tình tiết, chi tiết, đồng thời viết văn phong dưới dạng tóm lược cho phần mở đầu của bài báo. Hãy viết cho tôi một đoạn đầu đề cho bài báo có tiêu đề dưới đây:}
\end{quote}
which means
\begin{quote}
    \textit{"You are a Vietnamese journalist who writes headlines for articles using their titles. Please be specific and precise in giving information such as time, place, circumstance, and details, and write a summary for the introduction of the article. Please write me a headline for the article with the following title:"} 
\end{quote}
This prompt is used to improve writing quality by providing more detailed information, such as dates and times, rather than merely offering general information in the news title. While this information may sometimes be incorrect, its specificity can persuade readers by appearing more credible (e.g., \textit{"In May 2020, Mr. A, the president of company X, stated that..."}).

For Wikipedia articles, we refined the prompt to ensure that machine-generated text includes only information that is appropriate for Wikipedia. Additionally, we instructed the AI to write only the introductory information, without delving into further details.
Our new prompt for Vietnamese Wikipedia is:
\begin{quote}
    \textit{"Bạn là một nhà đóng góp cho Wikipedia tiếng Việt. Hãy viết cho tôi một đoạn giới thiệu ngắn gọn bằng tiếng Việt về chủ thể bên dưới để đăng trên trang Wikipedia. Hãy cố gắng hiểu về chủ đề, và viết ra một đoạn giới thiệu chứa các thông tin mà người dùng thường tìm kiếm trên Wikipedia. Hãy cố gắng viết ra các đoạn văn bản giống người viết nhất có thể. Chỉ đưa ra đoạn giới thiệu, không đưa ra thêm các thông tin khác. Lưu ý rằng bạn phải giữ độ dài văn bản trong khoảng \texttt{word\_count} từ, không được viết dài hơn."}
\end{quote}
which means:
\begin{quote}
    \textit{"You are a contributor to Vietnamese Wikipedia. Please write me a short introduction in Vietnamese about the subject below to post on the Wikipedia page. Try to understand the topic, and write an introduction that contains information that users often search for on Wikipedia. Try to write the text as humanly as possible. Provide only the introduction, do not provide any other information. Note that you must keep the text length within \texttt{word\_count} words, do not write longer."}
\end{quote}

Generally, these improved prompts aim to imbue AI-generated text with characteristics similar to human-written passages, enabling the AI to produce more human-like text. Indeed, humans find it more difficult to distinguish between the two passages, as they are more confused while reading them and struggle to determine which one is machine-generated. The contextual gap between human and AI seems to have narrowed significantly.

However, as noted previously, for news articles, the machine may hallucinate and generate incorrect timestamps for events. Additionally, the writing style of the machine is somewhat unique and consistent across all records, whereas human-written text tends to exhibit greater variation in structure and length. This distinction remains a key factor for humans when identifying whether a passage is machine-generated or written by a human.
\selectlanguage{English}


\begin{table*}[t!]
    \centering
    \resizebox{\textwidth}{!}{
    \begin{tabular}{ll  p{9cm}  p{9cm}}
    \toprule
    \textbf{Language} & \textbf{Source} & \textbf{Original Prompt} & \textbf{Improved Prompt} \\
    \midrule
    Arabic & Dialect Tweet & [translated from AR] Write a tweet in '\{dialect\}'. [English Prompt] Write a random tweet in '\{dialect\}' & [translated from AR] Write a random tweet in Arabic. Use dialect '\{dialect\}', express emotions and human experience. [English Prompt] Generate a random tweet in Arabic. Use dialect '\{dialect\}', use human emotions and experience. Output the tweet only. \\
    Arabic & EASC Summary& [translated from AR] Summarize this article while preserving the key points, ensuring conciseness and accuracy: '\{article\}'.& 
    [translated from AR] Summarize the following article while attempting to simulate a human with intellectual or religious beliefs, ensuring accuracy and conciseness, as per the average human level: '\{article\}'.\\
    Arabic & Youm7 News & [translated from AR] Paraphrase the following news article in clear, clear language, keeping the key information and ideas accurate. Make sure to arrange paragraphs logically and present ideas in an understandable sequence, while using fluent, easy-to-read Arabic. & [translated from AR] Rewrite the provided news article with a refined, well-structured, and sophisticated style that enhances clarity, depth, and coherence. Ensure the article maintains journalistic integrity while improving logical flow, sentence structure, and readability. The revised version should elevate the narrative by incorporating precise language, nuanced transitions, and a compelling tone appropriate for a professional audience. Maintain factual accuracy, emphasize key points effectively, and optimize the article’s structure to enhance engagement and comprehension. Additionally, refine the coherence between paragraphs, eliminate redundancy, and ensure seamless progression of ideas.\\
    Arabic & SANAD News & [translated from AR] You are a professional news writer. Write an article in Arabic consisting of approximately \{word\_count\_rounded\} words. The article is titled: '\{title\}', assume that the source is: '\{source\}', and the general topic falls under '\{topic\_arabic\}'. & 
    [translated from AR] Act as if you are a professional news writer and write an article in Arabic consisting of approximately \{word\_count\_rounded\} words, titled \{title\}. Ensure that the article is rich in information and precise details, including numbers, dates, and quantitative data when relevant. Make sure to include links, phone numbers, or currency exchange rates when necessary. Use some specialized English terms if the context requires it, and avoid relying solely on generic phrases like "This development comes amid a global trend..." without supporting events or details. Maintain a clear chronological order that reflects the logical sequence of events. Ensure that the text is written as a single paragraph regardless of its length, without dividing it into multiple paragraphs. Keep the formatting imperfect (e.g., uneven spacing) to reflect the nature of human writing, and feel free to use hashtags when needed. Please note that the topic generally falls under the category of \{topic\_arabic\}. Please do not use Markdown at all.\\

     \midrule
    Hindi & News & 
    [translated from Hindi] Here is a news headline: '{headline}' and the content: '{content}'. Write a machine-generated version of the news based on this headline. & 
    [translated from Hindi] Here is a news headline: '{headline}' and its content: '{content}'. Generate a machine-written version of the news based on this headline. Return the news in Hindi, formatted as plain text. Do not include any additional text—just return the generated news content in Hindi. \\
    \midrule
    Italian & DICE News
    & \textbf{System:} You are an Italian journalist writing for a national newspaper focusing on criminal events happening in the area surrounding Modena. \newline \textbf{User:} Write a piece of news in Italian, that will appear in a local Italian newspaper and that has the following title: & \textbf{System:} You are an Italian journalist writing for a national newspaper focusing on criminal events happening in the area surrounding Modena. \newline \textbf{User:} In writing avoid any kind of formatting, do not repeat the title and keep the text informative and not vague. You don't have to add the date of the event but you can \\
    \midrule
    Japanese & News & Please generate a Japanese news article that matches the following news title.$\backslash$n news title: \{title\}$\backslash$n news article: & Please generate a Japanese news article that matches the following news title. When creating the article, please use plain form instead of polite form, and avoid generating the beginning of the article by paraphrasing the title.$\backslash$n news title: \{title\}$\backslash$n news article: \\
     \midrule
    Kazakh & Wikipedia & Please, write one paragraph about the following topic in Kazakh:  \texttt{\{topic\}}. \newline \foreignlanguage{russian}{Мына тақырып туралы бір абзац жазыңыз: \texttt{{тақырып}}.} & Please, write one paragraph in Kazakh about the following topic: \texttt{\{topic\}}. Avoid repetitive sentence structures and predictable phrasing. When applicable, include concrete facts such as specific numbers, dates, or historical references. Avoid overly flattering or exaggerated language, and instead focus on delivering clear, informative, and relevant content. \newline  \foreignlanguage{russian}{Мына тақырып туралы бір абзац жазыңыз: \texttt{{тақырып}}. Қайталағыш сөйлем құрылымдарынан және болжамды сөздерден аулақ болыңыз. Қажет болған жағдайда нақты деректерді, мысалы, нақты сандарды, даталарды немесе тарихи сілтемелерді қосыңыз. Артық мақтау немесе асыра сілтеуден аулақ болыңыз, орнына анық, ақпаратты және сәйкес мазмұнды жеткізуге назар аударыңыз.} \\
    
    \bottomrule
    \end{tabular}
   }
\end{table*}

\begin{table*}[t!]
    \centering
    \resizebox{\textwidth}{!}{
    \begin{tabular}{ll  p{9cm}  p{9cm}}
    \toprule
    \textbf{Language} & \textbf{Source} & \textbf{Original Prompt} & \textbf{Improved Prompt} \\
    \midrule
    English & Peersum & 
   Generate a meta review based on the reviews' opinions and authors' rebuttal to make the final decision on whether the paper should be accepted: \texttt{\{Reviews\}} $\backslash $n$\backslash$n Meta review:  & 
    Generate a meta review based on the reviews' opinions and authors' rebuttal to make the final decision on whether the paper should be accepted: \texttt{\{Reviews\}}$\backslash $n$\backslash$n. When generating, don't use rigid format or structure such as ``Decision: XX" or headings such as ``Strengths", ``weaknesses" or bullet points. $\backslash $n$\backslash$n Meta review: \\
    \midrule
    Chinese & Zhihu-QA & [translated from Chinese] Imagine you are a rational and analytical Zhihu user. Your objective is to answer question in a clear, objective, and logical manner. Please provide a thorough and well-supported answer. \newline \cn{假设你是一位理性分析的知乎用户,你的目标是以客观、逻辑的方式回答以下问题。请以理性分析者的身份,给出详细、有据可查的回答。 } & [translated from Chinese] Imagine you are a rational and analytical Zhihu expert. Your goal is to provide objective, logical, and detailed answers while strictly adhering to Zhihu’s style. Keep responses concise, avoiding excessive politeness or formality. Based on the question's professionalism, start with 'Thank you for the invitation' or a similar phrase. If relevant, @ professionals to enhance credibility. \newline \cn{假设你是一位理性分析的知乎专家,你精通提问者所提出的问题。 你的目标是以客观、逻辑的方式回答问题。请给出详细、有据可查的回答。回答请严格符合知乎风格,避免长答案和过于礼貌正式的答案。根据问题的专业性,在开头回复“谢邀”或同义词,在必要时请在回答中@一些行业专业人士。}  \\
    Chinese & Student essay & [translated from Chinese] As a Gaokao student, you must write an 800-word essay that is logical, well-structured, and clearly argued, with sufficient evidence. Express unique insights on social, life, or philosophical issues in formal, elegant language, avoiding colloquialisms. Ensure smooth transitions and a complete structure, with a conclusion that reinforces and elevates the theme. \cn{你是一名正在参加高考的学生,现在需要完成一篇800字左右的作文。请根据题目要求,深刻思考,并清晰表达你的观点。文章应有逻辑性和层次感,论点明确、论据充分,能够展示出你对社会、生活或哲理问题的独到见解。注意语言要规范、优美,避免口语化表达,确保文章结构完整,段落之间过渡自然。结尾部分应对文章的主题进行总结和升华,展示你对问题的深入思考。} & [translated from Chinese] As a Gaokao student, you are required to write an 800-word essay expressing your views clearly and logically, with well-structured arguments and supporting evidence. Reflect on social, life, or philosophical topics, using formal and elegant language while avoiding colloquial expressions. Ensure smooth transitions between paragraphs and a conclusion that deepens the theme. Avoid rigid structures like “firstly” and “secondly,” and if writing narrative, use sincere and emotionally resonant language. Do not include a title or heading, and write in Chinese only. \newline \cn{你是一名正在参加高考的学生,现在需要完成一篇800字左右的作文。请根据题目要求,深刻思考,并清晰表达你的观点。文章应有逻辑性和层次感,论点明确、论据充分,能够展示出你对社会、生活或哲理问题的独到见解。注意语言规范、优美,避免口语化表达,确保文章结构完整,段落之间过渡自然。结尾应对文章的主题进行总结和升华,展示你对问题的深入思考。使用多样化的表达,避免过于结构化的表述,例如"首先、其次、然后"等;如果是记叙文,请使用情感丰富的真实表达。请不要输出任何题目或标题,直接开始写作。请全程使用中文。} \\
    % & Student essay &  \\
    Chinese & Government Report &[translated from Chinese] Please continue writing the full article based on the provided introduction. The original article starts with: \{head\}  \cn{请根据文章的开头续写完整的文章,原文章开头为: \{head\}} & [translated from Chinese] Please continue writing the full article based on the provided introduction: {head}. Please make the article as long as possible without generating symbols like **. Do not generate any content unrelated to the article. The original article starts with: \{head\} \newline \cn{请根据文章的开头续写完整的文章,请让文章尽可能的长,并且不生成**这类符号。请不要生成文章之外的其他内容。原文章开头:\{head\}}  \\
   
    
    \midrule
    Russian & News & Write a news article on the topic \texttt{\{topic\}} from the lenta.ru website using the title \texttt{\{title\}}. You must generate news without a heading. News: \newline \foreignlanguage{russian}{Напиши новость в области \texttt{\{topic\}} с сайта lenta.ru используя заголовок \texttt{\{title\}}. Ты должен генерировать новость без заголовка. Новость:}
    & Write a news article on the topic \texttt{\{topic\}} from the lenta.ru website using the title \texttt{\{title\}}. You must generate news without a heading. Add as many details as possible, such as names, numbers, dates, etc. You should also refer to the original source of the information (for example, RIA Novosti, CNN, BBC, etc.). News: \newline \foreignlanguage{russian}{Напиши новость в области \texttt{\{topic\}} с сайта lenta.ru используя заголовок \texttt{\{title\}}. Ты должен генерировать новость без заголовка. Добавь побольше деталей, такие как имена, числа, даты и тд. Так же ты должен сослаться на первоисточник информации (например, РИА Новости, CNN, BBC и т.д.). Новость:}\\
    Russian & Academic summary & Write a summary for an article in the \texttt{\{topic\}} topic using the title \texttt{\{title\}}. You must generate a summary without a title. Contents: \newline \foreignlanguage{russian}{Напиши краткое содержание для статьи в области \texttt{\{topic\}} используя заголовок \texttt{\{title\}}. Ты должен генерировать краткое содержание без заголовка. Содержание:}
    & Write a summary for an article in the \texttt{\{topic\}} topic using the title \texttt{\{title\}}. You should generate a summary without a title. Add details about the results, experiments, numbers, references to other work, etc. Contents: \newline \foreignlanguage{russian}{Напиши краткое содержание для статьи в области \texttt{\{topic\}} используя заголовок \texttt{\{topic\}}. Ты должен генерировать краткое содержание без заголовка. Добавь деталей про результаты, эксперименты, числа, ссылки на другие работы и т.д. Содержание:}\\

    \bottomrule
    \end{tabular}
   }
\end{table*}
% TODO: Rui will double check with Jiahui; turn it into long table -> if both longtable and supertabularx not working I will split it into multiple manually -> I will manually transform it into multiple tables, this table is very hard to render


\begin{table*}[t!]
    \centering
    \resizebox{\textwidth}{!}{
    \begin{tabular}{ll  p{9cm}  p{9cm}}
    \toprule
    \textbf{Language} & \textbf{Source} & \textbf{Original Prompt} & \textbf{Improved Prompt} \\
    \midrule
    Vietnamese & Wikipedia & You are a contributor to Vietnamese Wikipedia. Please write me a brief introduction in Vietnamese about the subject below to post on the Wikipedia page. Note that it must be written within \texttt{word\_count} words: & You are a contributor to Vietnamese Wikipedia. Please write me a short introduction in Vietnamese about the subject below to post on the Wikipedia page. Try to understand the topic, and write an introduction that contains information that users often search for on Wikipedia. Try to write the text as humanly as possible. Provide only the introduction, do not provide any other information. Note that you must keep the text length within \texttt{word\_count} words, do not write longer.\\
    Vietnamese & News & You are a Vietnamese journalist who writes summaries for articles using their headlines. Please write me a summary of the article with the following headline: & You are a Vietnamese journalist who writes headlines for articles using their titles. Please be specific and precise in giving information such as time, place, circumstance, and details, and write a summary for the introduction of the article. Please write me a headline for the article with the following title:\\
    \bottomrule
    \end{tabular}
   }
    \caption{Original prompts vs. improved prompts used to fill the gap between human and machine text, imitating human style and writing convention.}
    \label{tab:ori-improved-prompts}
\end{table*}