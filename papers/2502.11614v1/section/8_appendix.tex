\clearpage
% \onecolumn
\section*{Appendix}
\appendix

\section{Related Work}
\label{sec:relatedwork}

\subsection{Current AI Tools for Social Service}
\label{subsec:relatedtools}
% the title I feel is quite broad

Harnessing technology for social good has always been a grand challenge in social service \cite{berzin_practice_2015}. As early as the 90s, artificial neural networks and predictive models have been employed as tools for risk assessments, decision-making, and workload management in sectors like child protective services and mental health treatment \cite{fluke_artificial_1989, patterson_application_1999}. The recent rise of generative AI is poised to further advance social service practice, facilitating the automation of administrative tasks, streamlining of paperwork and documentation, optimisation of resource allocation, data analysis, and enhancing client support and interventions \cite{fernando_integration_2023, perron_generative_2023}.

Today, AI solutions are increasingly being deployed in both policy and practice \cite{goldkind_social_2021, hodgson_problematising_2022}. In clinical social work, AI has been used for risk assessments, crisis management, public health initiatives, and education and training for practitioners \cite{asakura_call_2020, gillingham2019can, jacobi_functions_2023, liedgren_use_2016, molala_social_2023, rice_piloting_2018, tambe_artificial_2018}. AI has also been employed for mental health support and therapeutic interventions, with conversational agents serving as on-demand virtual counsellors to provide clinical care and support \cite{lisetti_i_2013, reamer_artificial_2023}.
% commercial solutions include Woebot, which simulates therapeutic conversation, and Wysa, an “emotionally intelligent” AI coach, powered by evidenced-based clinical techniques \cite{reamer_artificial_2023}. 
% Non-clinical AI agents like Replika and companion robots can also provide social support and reduce loneliness amongst individuals \cite{ahmed_humanrobot_2024, chaturvedi_social_2023, pani_can_2024, ta_user_2020}.

Present research largely focuses on \textit{\textbf{AI-based decision support tools}} in social service \cite{james_algorithmic_2023, kawakami2022improving}, especially predictive risk models (PRMs) used to predict social service risks and outcomes \cite{gillingham2019can, van2017predicting}, like the Allegheny Family Screening Tool (AFST), which assesses child abuse risk using data from US public systems \cite{chouldechova_case_2018, vaithianathan2017developing}. Elsewhere, researchers have also piloted PRMs to predict social service needs for the homeless using Medicaid data\cite{erickson_automatic_2018, pourat_easy_2023}, and AI-powered algorithms to promote health interventions for at-risk populations, such as HIV testing among Californian homeless \cite{rice_piloting_2018, yadav_maximizing_2017}.

\subsection{Generative AI and Human-AI Collaboration}
\label{subsec:relatedworkhaicollaboration}
Beyond decision-making algorithms and PRMs, advancements in generative AI, such as large language models (LLMs), open new possibilities for human-AI (HAI) collaboration in social services. 
LLMs have been called "revolutionary" \cite{fui2023generative} and a "seismic shift" \cite{cooper2023examining}, offering "content support" \cite{memmert2023towards} by generating realistic and coherent responses to user inputs \cite{cascella2023evaluating}. Their vastly improved capabilities and ubiquity \cite{cooper2023examining} makes them poised to revolutionise work patterns \cite{fui2023generative}. Generative AI is already used in fields like design, writing, music, \cite{han2024teams, suh2021ai, verheijden2023collaborative, dhillon2024shaping, gero2023social} healthcare, and clinical settings \cite{zhang2023generative, yu2023leveraging, biswas2024intelligent}, with promising results. However, the social service sector has been slower in adopting AI \cite{diez2023artificial, kawakami2023training}.

% Yet, the social service sector is one that could perhaps stand to gain the most from AI technologies. As Goldkind \cite{goldkind_social_2021} writes, social service, as a "values-centred profession with a robust code of ethics" (p. 372), is uniquely placed to inform the development of thoughtful algorithmic policy and practice. 
Social service, however, stands to benefit immensely from generative AI. SSPs work in time-poor environments \cite{tiah_can_2024}, often overwhelmed with tedious administrative work \cite{meilvang_working_2023} and large amounts of paperwork and data processing \cite{singer_ai_2023, tiah_can_2024}. 
% As such, workers often work in time-poor environments and are burdened with information overload and administrative tasks \cite{tiah_can_2024, meilvang_working_2023}. 
Generative AI is well-placed to streamline and automate tasks like formatting case notes, formulating treatment plans and writing progress reports, which can free up valuable time for more meaningful work like client engagement and enhance service quality \cite{fernando_integration_2023, perron_generative_2023, tiah_can_2024, thesocialworkaimentor_ai_nodate}. 

Given the immense potential, there has been emerging research interest in HAI collaboration and teamwork in the Human-Computer Interaction and Computer Supported Cooperative Work space \cite{wang_human-human_2020}. HAI collaboration and interaction has been postulated by researchers to contribute to new forms of HAI symbiosis and augmented intelligence, where algorithmic and human agents work in tandem with one another to perform tasks better than they could accomplish alone by augmenting each other's strengths and capabilities  \cite{dave_augmented_2023, jarrahi_artificial_2018}.

However, compared to the focus on AI decision-making and PRM tools, there is scant research on generative AI and HAI collaboration in the social service sector \cite{wykman_artificial_2023}. This study therefore seeks to fill this critical gap by exploring how SSPs use and interact with a novel generative AI tool, helping to expand our understanding of the new opportunities that HAI collaboration can bring to the social service sector.

\subsection{Challenges in AI Use in Social Service}
\label{subsec:relatedworkaiuse}

% Despite the immense potential of AI systems to augment social work practice, there are multiple challenges with integrating such systems into real-life practice. 
Despite its evident benefits, multiple challenges plague the integration of AI and its vast potential into real-life social service practice.
% Numerous studies have investigated the use of PRMs to help practitioners decide on a course of action for their clients. 
When employing algorithmic decision-making systems, practitioners often experience tension in weighing AI suggestions against their own judgement \cite{kawakami2022improving, saxena2021framework}, being uncertain of how far they should rely on the machine. 
% Despite often being instructed to use the tool as part of evaluating a client, 
Workers are often reluctant to fully embrace AI assessments due to its inability to adequately account for the full context of a case \cite{kawakami2022improving, gambrill2001need}, and lack of clarity and transparency on AI systems and limitations \cite{kawakami2022improving}. Brown et al. \cite{brown2019toward} conducted workshops using hypothetical algorithmic tools 
% to understand service providers' comfort levels with using such tools in their work,
and found similar issues with mistrust and perceived unreliability. Furthermore, introducing AI tools can  create new problems of its own, causing confusion and distrust amongst workers \cite{kawakami2022improving}. Such factors are critical barriers to the acceptance and effective use of AI in the sector.

\citeauthor{meilvang_working_2023} (2023) cites the concept of \textit{boundary work}, which explores the delineation between "monotonous" administrative labour and "professional", "knowledge based" work drawing on core competencies of SSPs. While computers have long been used for bureaucratic tasks like client registration, the introduction of decision support systems like PRMs stirred debate over AI "threatening professional discretion and, as such, the profession itself" \cite{meilvang_working_2023}. Such latent concerns arguably drive the resistance to technology adoption described above. Generative AI is only set to further push this boundary, 
% these concerns are only set to grow in tandem with the vast capabilities of generative and other modern AI systems. Compared to the relatively primitive AI systems in past years, perceived as statistical algorithms \cite{brown2019toward} turning preset inputs like client age and behavioural symptoms \cite{vaithianathan2017developing} into simple numerical outputs indicating various risk scores, modern AI systems are vastly more capable: LLMs 
with its ability to formulate detailed reports and assessments that encroach upon the "core" work of SSPs.
% accept unrestricted and unstructured inputs and return a range of verbose and detailed evaluations according to the user's instructions. 
Introducing these systems exacerbate previously-raised issues such as understanding the limitations and possibilities of AI systems \cite{kawakami2022improving} and risk of overreliance on AI \cite{van2023chatgpt}, and requires a re-examination of where users fall on the algorithmic aversion-bias scale \cite{brown2019toward} and how they detect and react to algorithmic failings \cite{de2020case}. We address these critical issues through an empirical, on-the-ground study that to our knowledge is the first of its kind since the new wave of generative AI.

% W 

% Yet, to date, we have limited knowledge on the real-world impacts and implications of human-AI collaboration, and few studies have investigated practitioners’ experiences working with and using such AI systems in practice, especially within the social work context \cite{kawakami2022improving}. A small number of studies have explored practitioner perspectives on the use of AI in social work, including Kawakami et al. \cite{kawakami2022improving}, who interviewed social workers on their experiences using the AFST; Stapleton et al. \cite{stapleton_imagining_2022}, who conducted design workshops with caseworkers on the use of PRMs in child welfare; and Wassal et al. \cite{wassal_reimagining_2024}, who interviewed UK social work professionals on the use of AI. A common thread from all these studies was a general disregard for the context and users, with many practitioners criticising the failure of past AI tools arising from the lack of participation and involvement of social workers and actual users of such systems in the design and development of algorithmic systems \cite{wassal_reimagining_2024}. Similarly, in a scoping review done on decision-support algorithms in social work, Jacobi \& Christensen \cite{jacobi_functions_2023} reported that the majority of studies reveal limited bottom-up involvement and interaction between social workers, researchers and developers, and that algorithms were rarely developed with consideration of the perspective of social workers.
% so the \cite{yang_unremarkable_2019} and \cite{holten_moller_shifting_2020} are not real-world impacts? real-world means to hear practitioner's voice? I feel this is quite important but i didnt get this point in intro!

% why mentioning 'which have largely focused on existing ADS tools (e.g., AFST)'? i can see our strength is more localized, but without basic knowledge of social work i didnt get what's the 'departure' here orz
% the paragraph is great! do we need to also add one in line 20 21?

\subsection{Designing AI for Social Service through Participatory Design}
\label{subsec:relatedworkpd}
% i think it's important! but maybe not a whole subsection? but i feel the strong connection with practitioners is indeed one of our novelties and need to highlight it, also in intro maybe
% Participatory design (PD) has long been used extensively in HCI \cite{muller1993participatory}, to both design effective solutions for a specific community and gain a deep understanding of that community. Of particular interest here is the rich body of literature on PD in the field of healthcare \cite{donetto2015experience}, which in this regard shares many similarities and concerns with social work. PD has created effective health improvement apps \cite{ryu2017impact}, 

% PD offers researchers the chance to gather detailed user requirements \cite{ryu2017impact}...

Participatory design (PD) is a staple of HCI research \cite{muller1993participatory}, facilitating the design of effective solutions for a specific community while gaining a deep understanding of its stakeholders. The focus in PD of valuing the opinions and perspectives of users as experts \cite{schuler_participatory_1993} 
% In recent years, the tech and social work sectors have awakened to the importance of involving real users in designing and implementing digital technologies, developing human-centred design processes to iteratively design products or technologies through user feedback 
has gained importance in recent years \cite{storer2023reimagining}. Responding to criticisms and failures of past AI tools that have been implemented without adequate involvement and input from actual users, HCI scholars have adopted PD approaches to design predictive tools to better support human decision-making \cite{lehtiniemi_contextual_2023}.
% ; accordingly, in social service, a line of research has begun studying and designing for human-AI collaboration with real-world users (e.g. \cite{holten_moller_shifting_2020, kawakami2022improving, yang_unremarkable_2019}).
Section \ref{subsec:relatedworkaiuse} shows a clear need to better understand SSP perspectives when designing and implementing AI tools in the social sector. 
Yet, PD research in this area has been limited. \citeauthor{yang2019unremarkable} (2019), through field evaluation with clinicians, investigated reasons behind the failure of previous AI-powered decision support tools, allowing them to design a new-and-improved AI decision-support tool that was better aligned with healthcare workers’ workflows. Similarly, \citeauthor{holten_moller_shifting_2020} (2020) ran PD workshops with caseworkers, data scientists and developers in public service systems to identify the expectations and needs that different stakeholders had in using ADS tools.

% Indeed, it is as Wise \cite{wise_intelligent_1998} noted so many years ago on the rise of intelligent agents: “it is perhaps when technologies are new, when their (and our) movements, habits and attitudes seem most awkward and therefore still at the forefront of our thoughts that they are easiest to analyse” (p. 411). 
Building upon this existing body of work, we thus conduct a study to co-design an AI tool \textit{for} and \textit{with} SSPs through participatory workshops and focus group discussions. In the process, we revisit many of the issues mentioned in Section \ref{subsec:relatedworkaiuse}, but in the context of novel generative AI systems, which are fundamentally different from most historical examples of automation technologies \cite{noy2023experimental}. This valuable empirical inquiry occurs at an opportune time when varied expectations about this nascent technology abound \cite{lehtiniemi_contextual_2023}, allowing us to understand how SSPs incorporate AI into their practice, and what AI can (or cannot) do for them. In doing so, we aim to uncover new theoretical and practical insights on what AI can bring to the social service sector, and formulate design implications for developing AI technologies that SSPs find truly meaningful and useful.
% , and drive future technological innovations to transform the social service sector not just within [our country], but also on a global scale.

 % with an on-the-ground study using a real prototype system that reflects the state of AI in current society. With the presumption that AI will continue to be used in social work given the great benefits it brings, we address the pressing need to investigate these issues to ensure that any potential AI systems are designed and implemented in a responsible and effective manner.

% Building upon these works, this study therefore seeks to adopt a participatory design methodology to investigate social workers’ perspectives and attitudes on AI and human-AI collaboration in their social work practice, thus contributing to the nascent body of practitioner-centred HCI research on the use of AI in social work. Yet, in a departure from prior work, which have largely focused on existing ADS tools (e.g., AFST) and were situated in a Western context, our paper also aims to expand the scope by piloting a novel generative AI tool that was designed and developed by the researchers in partnership with a social service agency based in Singapore, with aims of generating more insights on wider use cases of AI beyond what has been previously studied.

% i may think 'While the current lacunae of research on applications of AI in social work may appear to be a limitation, it simultaneously presents an exciting opportunity for further research and exploration \cite{dey_unleashing_2023},' this point is already convincing enough, not sure if we need to quote here
% I like this end! it's a good transition to our study design, do we need to mention the localization in intro as well? like we target at singapore

% Given the increasing prominence and acceptance of AI in modern society, 

% These increased capabilities vastly exacerbate the issues already present with a simpler tool like the AFST: the boundaries and limitations of an LLM system are significantly more difficult to understand and its possible use cases are exponentially greater in scope. 

% Put this in discussion section instead?
% Kawakami et al's work "highlights the importance of studying how collaborative decision-making... impacts how people rely upon and make sense of AI models," They conclude by recommending designing tools that "support workers in understanding the boundaries of [an AI system's] capabilities", and implementing design procedures that "support open cultures for critical discussion around AI decision making". The authors outline critical challenges of implementing AI systems, elucidating factors that may hinder their effectiveness and even negatively affect operations within the organisation.


% Is this needed?:
% talk about the strengths of PD in eliciting user viewpoints and knowledge, in particular when it is a field that is novel or where a certain system has not been used or developed or tested before
\section{Datasets}
\label{sec:datasets}
% describe 
% 1. original dataset (license, size, domain, topic); 
% 2. how we did sampling and explain why (size and topic); 
% 3. how we did machine-generation (model, prompt, model generation configuration)

\begin{table*}[ht!]
    \centering
    \small
    \resizebox{\textwidth}{!}{
    \begin{tabular}{lllr|ccclr}
    \toprule
    \textbf{Language} & \textbf{Source/} & \textbf{Data} &  \textbf{Total} & \multicolumn{5}{c}{\textbf{Sampled Parallel Data}}  \\
                      & \textbf{Domain} & \textbf{License} &\textbf{Human} & \textbf{Human} & \textbf{GPT-4o} & \textbf{Claude} &  \textbf{LLM2 Name (\#)} & \textbf{Total}  \\
                      % \textbf{Vikhr-Nemo-12B} & \textbf{Llama3-405B} & \textbf{ChatGLM4} & \textbf{Qwen2} & \textbf{Qwen-turbo} & 
    \midrule
    \multirow{4}{*}{Arabic} & Dialect Tweet &  Apache 2.0 &  1400 & 300 & 300* & -- & Qwen2 (300*) & 900 \\
    & EASC & cc-by-sa-3.0 & 765 & 153 & 153 & -- & -- & 306 \\
    & Youm7 News & --- & 21,000 & 1,000 & 1,000 & -- & AceGPT (1,000) & 3,000 \\
    & SANAD & cc-by-4.0 & 194,797 & 100 & 100 & -- & -- & 200 \\
    \midrule
    \multirow{4}{*}{Chinese} & Zhihu-QA & cc-by-4.0 & 224,761 & 588 & 588 & -- & Qwen-turbo (588) & 1,764 \\
                             & Student essay & cc-by-4.0 & 93,002 & 600 & -- & 300* & ChatGLM4 (300*) & 1,200 \\
                             & Student essay & cc-by-4.0 & 51 & 51 & -- & 51 &  ChatGLM4 (51) & 153 \\
                             & Government Report & MIT & 200,409& 500 & 500 & -- &  Baichuan2-13B (500) & 1,500 \\
    \midrule
    English & Peersum~\citep{peersum_2023} & cc-by-sa-4.0 &  5,158 & 400 & 200 & 200  & -- &  800\\
    \midrule
    Hindi & News & cc-by-4.0 & 3,995 & 600 & 600 & -- & -- & 1,200 \\
    \midrule
    Italian & DICE & cc-by-sa &  10,518 & 300 & 300 & Anita (300) &  Llama3-405B (300) & 1,200\\
    % & CItA & & 1352 & 300 & & & & 300 & & & \\
    \midrule
    Japanese & News & cc-by-nc-sa-4.0 & 7,110 & 300 & 300 & -- & -- & 600 \\
     \midrule
    Kazakh & Wikipedia & cc-by-sa-4.0 &  4,827 & 300 & 300  & -- & -- & 600\\
    \midrule
    \multirow{2}{*}{Russian} 
    & News & MIT & 800,000 & 300 & 300 & -- & Vikhr-Nemo-12B (300) & 900 \\
    & Academic summary & MIT & 31,000 & 300 & 300 & -- & Vikhr-Nemo-12B (300) & 900 \\
    \midrule
    \multirow{2}{*}{Vietnamese} 
    & Wikipedia & cc-by-sa-3.0 & 600 & 600 & 600 & -- & -- & 1,200 \\
    & News & Kaggle data & 290,282 & 600 & 600 & -- & -- & 1,200 \\
    \midrule
    \bf Total & -- & -- & 1,690,803	& 6,992 & 6,141 & 851 & 3,639 & \textbf{17,623} \\
    \bottomrule
    \end{tabular}
   }
    \caption{Statistics of multilingual data for human annotation. Machine data with * means non-parallel data. }
    \label{tab:multilingual-data}
\end{table*}



% \begin{table*}[t!]
%     \centering
%     \small
%     \tabcolsep2pt
% %    \resizebox{\textwidth}{!}{
%             % \setlength{\tabcolsep}{3pt}
%     \adjustbox{max width=\linewidth}{
%     \begin{tabular}{lllr|ccccccccr}
%     \toprule
%     \textbf{Language} & \textbf{Source/} & \textbf{Data} &  \textbf{Total} & \multicolumn{9}{c}{\textbf{Sampled Parallel Data}}  \\
%                       & \textbf{Domain} & \textbf{License} &\textbf{Human} & \textbf{Human} & \textbf{GPT-4o} & \textbf{Claude} & \textbf{Llama3-8B} & \textbf{Llama3-405B} & \textbf{ChatGLM4} & \textbf{Qwen2} & \textbf{Qwen-turbo} & \textbf{Total}  \\
%     \midrule
%      English & Peersum & cc-by-sa-4.0 &   & 400 & 200 & 200 \\
%      \midrule
%     \multirow{4}{*}{Chinese} & Zhihu-QA & cc-by-4.0 & 224761 & 588 & 588 & &&&&& 588 & 1,764 \\
%                              & Student essay & cc-by-4.0 & 93,002 & 600 &  & 600* & && 600* & & & 1,800 \\
%                              & Student essay & cc-by-4.0 & 51 & 51 & & 51 & && 51 & & & 153 \\
%                          & Government Report & \\
%     % \midrule
%     \midrule
%     \multirow{2}{*}{Russian} & News &  &   &  &  & & &  & & & \\
%     & Academic summaries & &  &  & & & &  & & & \\
%     \midrule
%     Italian & DICE & CC-by-sa &  10,518 & 300 & 300 & & & 300 & & & \\
%     % & CItA & & 1352 & 300 & & & & 300 & & & \\
%     \midrule
%     Kazakh & Wikipedia & cc-by-sa-4.0 &  4827 & 300 & 300 & -- \\
%     \midrule
%     Japanese & News & cc-by-nc-sa-4.0 & 7,110 & 300 & 300 & &&&&&& 600 \\
%     \midrule
%     \multirow{2}{*}{Vietnamese} & Wikipedia &  &   &  &  & & &  & & & \\
%     & News & &  &  & & & &  & & & \\
%     \midrule
%     \multirow{3}{*}{Arabic} & Arabic POS Dialect &  Apache 2.0 &  1400 & 300 & 300 & & & & &300 & &900 \\
%     & ESAC & CC BY-SA 3.0 & 765 & 153 & 153 & & & & & & & 306 \\
%     & KALIMAT& NA & & & & & & & & & & \\
%     & SANAD & CC BY 4.0 & 194,797 & 100 & 100 & -- & -- & -- & -- & -- & -- & 200 \\
%     \midrule
%     Hindi & News & CC BY 4.0 & 3,995 & 600 & 600 & &&&&&& 1200 \\
%     \midrule
%         \bf Total & -- & -- &  \\
%     \bottomrule
%     \end{tabular}
%    }
%     \caption{Statistical information of multilingual data for human evaluation. Machine data with * means non-parallel data. }
%     \label{tab:multilingual-data}
% \end{table*}

\subsection{Arabic}
We collected three categories of text for Arabic including tweets, summaries, and news, involving four datasets.

\paragraph{Arabic Dialect Tweets}
% Mervat
We randomly selected 300 tweets from the QCRI Arabic POS Dialect dataset\footnote{\url{https://huggingface.co/datasets/QCRI/arabic_pos_dialect}}, which includes tweets in four dialects: Egyptian, Moroccan, Gulf, and Levantine. The dataset was originally curated for part of speech segmentation and covers a broad range of topics. The selected tweets were cleaned by removing the usernames of the original tweets' authors and parsing the words from each tweet to organize into sentences.

We generated 600 machine-generated tweets: 150 per dialect using \gptfouro and Qwen-2 (7.5B). Instead of calling APIs, we collected generations by the chat interface in order to esaily observe the patterns of the machine-produced content. The prompt employed for tweet generation was: \textit{write a random tweet with \texttt{}{dialect}.} In this structure, \texttt{dialect} denotes one of the four targeted dialects. Additionally, the corresponding Arabic prompt was also used in generation: 
\begin{RLtext} \texttt{ اكتب تغريدة بال\LR{\{dialect\}}}\end{RLtext}
% \begin{quote}
%     \small
%     \begin{RLtext}
%             \texttt{ اكتب تغريدة بال\LR{\{dialect\}}}.
%     \end{RLtext}
% \end{quote}


% The data was then cleaned, with each sentence written on a new line, with the prompt used and the label in a JSON file.
\paragraph{EASC Articles Summaries}
% Kareem
We used the Essex Arabic Summaries Corpus, a dataset featuring 153 Arabic articles sourced from Wikipedia, Alrai Newspaper, and Alwatan Newspaper. Each article is accompanied by 5 human-written summaries from which we sampled one. The corpus has a diverse range of topics, including art, education, politics, health, science, and finance. To generate machine-produced counterparts, we utilized \gptfouro using the prompt below. The prompt emphasizes that the generated summary should be highly informative, reserving all key ideas while remaining concise and to the point.
\begin{quote}
    \small
    \begin{RLtext}
            \texttt{قم بتلخيص هذا المقال محافظا على اهم النقاط ملتزما الايجاز و الدقة\LR{\{article\}:}}.
    \end{RLtext}

\end{quote}


\paragraph{Youm7 News Articles}
% Saad
The Kalimat dataset consists of articles sourced from the Egyptian news journal \textit{Youm7}. This dataset comprises a total of 21,000 articles across various topics, providing a diverse base for text generation and analysis. For our study, we sampled 1,000 news and generated using GPT-4o (a multilingual LLM) and AceGPT (a Arabic-centric LLM). 

The text generation process was guided by a carefully crafted prompt to simulate the typical writing style of \textit{Youm7} articles. This prompt instructed the models to act as a professional journalist, ensuring coherence and alignment with the original content’s structure and tone. Below is the prompt used for this task:
\begin{quote}
    \small
    \begin{RLtext}
        \texttt{تصرف كأنك كاتب أخبار محترف، واكتب مقالًا باللغة العربية يتألف من حوالي \LR{\{word\_count\_rounded\}} كلمة، بعنوان \LR{\{title\}}. اعتمد في أسلوبك على طريقة كتابة مؤلفي \LR{\{source\}}، مع الأخذ في الاعتبار أن الموضوع يندرج تحت فئة \LR{\{topic\_arabic\}}.}
    \end{RLtext}
\end{quote}

The goal of the generated samples was to maintain narrative consistency and topical relevance while adhering to the typical article patterns found in \textit{Youm7}. 
% This dataset provided a valuable foundation for comparing human- and machine-generated text and evaluating model performance in generating coherent Arabic content.


\paragraph{SANAD News}
% Tarek
\label{sanad_before_prompt_eng}
The SANAD news dataset \cite{EINEA2019104076} is an extensive collection of Arabic news articles in Modern Standard Arabic (MSA), designed to support various Arabic NLP tasks. SANAD dataset comprises over $190,000$ articles from three prominent news websites: AlKhaleej, AlArabiya, and Akhbarona, and it is released under the license of CC-BY-4.0.

For this study, we generated a total of $500$ news articles using titles from the SANAD news dataset, with $100$ articles for each of five selected news category topics present in SANAD spanning finance, politics, sports, medicine, and technology. The articles were generated using the OpenAI's GPT-4o-2024-05-13 and the prompt used to generate these articles is shown below:

\begin{quote}
    \small
    \begin{RLtext}
    \texttt{تصرف كأنك كاتب أخبار محترف ، و اكتب مقالًا باللغة العربية يتألف من حوالي \LR{\{word\_count\_rounded\}} كلمة، بعنوان \LR{\{title\}}. اعتمد في أسلوبك على طريقة كتابة مؤلفي \LR{\{source\}}، مع الأخذ في الاعتبار أن الموضوع يندرج بشكل عام تحت فئة \LR{\{topic\_arabic\}}.}
\end{RLtext}
\end{quote}


In this prompt, we ask the LLM to act as a professional news writer, and write an Arabic news article consisting of around  `\texttt{word\_count\_rounded}' words carrying the title `\texttt{title}', using the writing style of authors of `\texttt{source}' while taking into account that the topic falls under the general umbrella of `\texttt{topic\_arabic}'. 

`\texttt{word\_count\_rounded}' represents the number of words rounded to the nearest hundred of the corresponding human-written article for the same title. `\texttt{source}' is the publisher name of the given title obtained from SANAD. `\texttt{topic\_arabic}' is one of the five topics mentioned earlier selected to match the topic of the given title.


\subsection{Chinese}
We collected Zhihu question answering, high school student essays and government reports.

\paragraph{Zhihu QA}
Based on a Chinese question answering dataset collected from Zhihu, where the users can post and answer questions as well as liking or sharing them, we randomly sampled 588 examples.\footnote{\url{https://huggingface.co/datasets/zirui3/zhihu_qa}} 
They span life, technology, art, science, emotion, law, fashion, humor, and China-specific cultures. 
For emotion-rich questions, we specially collected across 10 topics including breaking-up, quarrel, happiness, marriage issue, divorce, depression, inferiority, parents, forgiveness, future concern (10 questions for each), emphasizing emotional and humorous responses.
Then we collect corresponding responses from a multilingual LLM \gptfouro and a Chinese SOTA model \qwenturbo.
% topic distribution of 488 examples + 200 emotion-rich questions
% life               100
% technology         100
% art                 69
% science             59
% emo                 50
% Chinese-culture     45
% law                 34
% fashion             26
% humor                5

\paragraph{Junior and Senior High School Student Essays}
We sampled 600 Chinese student essays from \citet{song-etal-2020-multi}, attempting to ensure the uniform distribution over the genre of an essay and the grade of the student.\footnote{Each instance includes the essay, and the genre (character, narratives, scenery, objects, argumentative and prose) and the rating (bad, moderate, good, excellent) of the essay, and the grade of the student (7-9 is junior and 10-12 is senior high school).}
However, the dataset does not provide the corresponding problem statements or titles of essays, making it challenging to generate the counterpart by LLMs. To produce the high school student essays by LLMs, we approached by collecting:
(i) essay problem statements of Gaokao from 1977 to 2024, along with some junior high school problems, in total of 376 problems, and (ii) 51 (problem, essay) pairs, with the excellent essays published in the website as the human text.\footnote{\url{https://docs.google.com/spreadsheets/d/1iwtnQkamxoqnThWUT9F007WUHm1XmNESAwSVFM8xNdU/edit?usp=sharing}} 
% 
We sampled 300 problem statements from (i), prioritizing the latest years, and the whole set of (2), and then generate machine counterparts by \claude-3.5-Sonnet and \chatglmfour.



\paragraph{Government Report}
The GovReport dataset, sourced from the MNBVC GovReport subset~\cite{mnbvc}, includes titles and main bodies of reports from various entities, such as schools, corporations, and local government units, without additional labels or classifications. We randomly sampled 500 entries to generate a diverse, high-quality LLM dataset. To achieve varied outputs, we used a set of eight distinct prompts, encompassing continuation tasks based on titles or opening sentences and rephrasing tasks based on the full report content. These prompts, presented in Chinese, were alternated to guide LLMs in generating work reports, with responses collected from GPT-4o, Baichuan2-13B-Chat, and ChatGLM3-6B.



\subsection{English Peer Meta Review}
% Jinyan
We randomly sample 600 reviews (NeurIPS 2021-2022) from PeerSum dataset~\citep{peersum_2023}. 
Peersum crawled both meta reviews and reviews from each reviewer and rebuttal from authors from openreview\footnote{\url{https://openreview.net/}}. 
We leveraged this dataset to generate meta reviews based on comments from other reviewers and compared with human-written meta reviews.
We generate meta reviews using \texttt{\gptfouro-2024-08-06} and \texttt{claude-3.5-sonnet-20240620} respectively, with prompt ``Generate a meta review based on the reviews' opinions and authors' rebuttal to make the final decision on whether the paper should be accepted: \texttt{\{Reviews\}}$\backslash $n$\backslash$n Meta review:'', where ``\texttt{Reviews}'' contains both the reviewers' review and authors rebuttals. 
We sampled 400 machine-generated meta reviews, paired with humans, and asked annotators to discern. 






\subsection{Hindi News}
% : Raj
The BBC Hindi news article dataset comprises a diverse selection of around 4,000 Hindi news articles sourced from the BBC Hindi website, covering a wide range of topics and categories. Each article includes three primary components: headline, main content and the thematic category of the article.
% \begin{itemize}
%     \item \textbf{Headline}: The title of the article.
%     \item \textbf{Content}: The main text of the news article.
%     \item \textbf{Category}: The thematic category of the article.
% \end{itemize}

We sampled 600 articles from the original dataset, focusing on articles that span multiple prominent topics in Indian news to ensure a balanced representation. These articles were selected randomly to avoid any thematic bias and provide a broad scope for comparison.
We generated 600 machine-written samples using GPT-4o model, employing a prompt that guided the model to write concise and formal articles: ``Here is a news headline: '{headline}' and the content: '{content}'. Write a machine-generated version of the news based on this headline''.
% Improved Prompt - Here is a news headline: '{headline}' and its content: '{content}'. Generate a machine-written version of the news based on this headline. Return the news in Hindi, formatted as plain text. Do not include any additional text—just return the generated news content in Hindi.


% This prompt encouraged the model to generate content that aligned with the factual and narrative style typical of professional news articles in Hindi. We configured the model generation settings to ensure clarity and coherence, aiming to produce content that mirrors the structure and tone of the human-written articles in the dataset. % This setup provided a solid foundation for analyzing machine-generated Hindi text in terms of authenticity, language quality, and narrative consistency.


\subsection{Italian DICE News}
% Giovanni
For Italian, we used DICE~\citep{bonisoli2023} --- local news from La Gazzetta di Modena ($\approx 10,000$ samples) licensed as CC-by-nc-sa.
We sampled 300 news, where the original text has at least 1000 characters. 
We applied three large language models: Llama-3.1-405b-instruct, \gptfouro, and Anita (an Italian fine-tuned Llama-3.1-8B: \citet{polignano2024advanced}). 
% \begin{itemize}
% \item Local News: (1) DICE \citep{bonisoli2023}, news from La Gazzetta di Modena ($\approx 10,000$ samples) licensed as CC-by-nc-sa; (2) 300 samples where the original text is at least 1000 characters; (3) GPT-4o, Anita, Llama 3.1 405B Instruct \citep{llama3}.

% \item Student Essays: (1) CiTA \citep{barbagli-etal-2016-cita}, essays from lower secondary school students ($1,352$ samples); (2) 300 random samples; (3) Anita and Llama 3.1 405B Instruct.
% \end{itemize}

% \section{Generation Details}
% \subsection{Italian}
When generating in Italian, we use one of 4 possible system prompts as \tabref{tab:Italian-prompt}.
The model is always prompted as a journalist but the mother tongue and the newspaper scope may vary lightly.
% The prompt used for generation is composed of a system prompt and a user prompt, we use a few variant, but 
The general format is the following:
\begin{quote}
\textbf{System:} You are an Italian journalist writing for a national newspaper focusing on criminal events
happening in the area surrounding Modena. \\
\textbf{User:} Write a piece of news in Italian, that will appear in a local Italian newspaper and that has the following title: ...
\end{quote}


\begin{table*}[ht]
\centering
\begin{tabular}{|p{7cm}|p{7cm}|}
% \toprule
\hline
\multicolumn{2}{|c|}{English-first LLMs} \\
\hline
\multicolumn{2}{|c|}{System Prompt} \\
% \midrule
\hline
    You are an Italian journalist writing for a national newspaper focusing on criminal events happening in the area surrounding Modena & You are an Italian journalist writing for a local newspaper focusing on criminal events happening in the area surrounding Modena \\
    \cline{1-2}
    You are an Italian-French journalist writing in Italian about criminal events happening in the area surrounding Modena & You are an Italian-American journalist writing for a local newspaper focusing on criminal events happening in the area surrounding Modena \\
% \midrule
\hline
\multicolumn{2}{|c|}{User Prompt} \\
\hline
% \midrule
    Write a piece of news in Italian, that will appear in a local Italian newspaper and that has the following title: & Write a piece of news in Italian, that will appear in a local Italian newspaper and that has the following title: \\
    \cline{1-2}
    Write a piece of news in Italian, that will appear in a national Italian newspaper and that has the following title: &     Write a piece of news in Italian, that will appear in a local Italian newspaper and that has the following title: \\
    % \midrule
    \hline
\multicolumn{2}{|c|}{Italian-first LLMs} \\
\hline
\multicolumn{2}{|c|}{System Prompts} \\
    \cline{1-2}
    Sei un giornalista italiano che che scrive per un giornale nazionale focalizzandosi su eventi criminali che accadono a Modena & Sei un giornalista italo-francese che scrive in italiano su eventi criminali che accadono a Modena \\
    \cline{1-2}
    Sei un giornalista italiano che scrive per un giornale locale focalizzandosi su eventi criminali che accadono a Modena & Sei un giornalista italo-americano che scrive per un giornale locale focalizzandosi su eventi criminali che accadono a Modena \\
    \cline{1-2}
\multicolumn{2}{|c|}{User Prompts} \\
    \cline{1-2}
    Scrivi un articolo di giornale in italiano. L'articolo sarà pubblicato su un giornale locale e avrà il seguente titolo: & Scrivi un articolo di giornale in italiano. L'articolo sarà pubblicato su un giornale nazionale e avrà il seguente titolo: \\
    \cline{1-2}
    Scrivi un articolo di giornale in italiano. L'articolo sarà pubblicato su un giornale locale e avrà il seguente titolo: & Scrivi un articolo di giornale in italiano. L'articolo sarà pubblicato su un giornale locale e avrà il seguente titolo:\\
% \bottomrule
\hline
\end{tabular}
\caption{Italian machine generation prompts.}
\label{tab:Italian-prompt}
\end{table*}


\subsection{Japanese News}
% Masahiro and Ryuto
We randomly sample 300 news articles of the BBC news from the XLSUM dataset \citep{hasan-etal-2021-xl}.
XLSUM has title, content, and summary of a news article.
We generate the corresponding news content using \texttt{\gptfouro-2024-08-06} based on the article titles.
% The content of the news articles in XLSUM is human-written text, 
For both text generated with \gptfouro and the human-written text, we remove obvious formatting indicators (e.g., line breaks and template messages at the beginning and end of the texts).
% For output text length, it sets the max tokens parameter of LLMs to the number of tokens in each human-written text.
% We calculated the number of tokens using the \texttt{tiktoken} library\footnote{https://pypi.org/project/tiktoken/0.1.1/}.

We generate 75 articles for each of the four settings: simple prompt zero-shot, diverse expression zero-shot, content-rich zero-shot, and few-shot.
Simple prompt zero-shot, diverse expression zero-shot, and content-rich zero-shot use the following instructions, respectively: ``\CJK{UTF8}{min}{次のニュースタイトルに合わせたニュース記事を生成してください。}'' (Please generate a news article that matches the following news title.), ``\CJK{UTF8}{min}{次のニュースタイトルに合わせたニュース記事を多様な表現を使用して生成してください。}'' (Please generate a news article that matches the following news title, using diverse expressions.), and ``\CJK{UTF8}{min}{次のニュースタイトルに合わせたニュース記事を生成してください。このとき生成するニュース記事には、誰かに対するインタビューや実際の出来事を組み込んでください。}'' (Please generate a news article that matches the following news title. When creating the article, include interviews with individuals or actual events.).
The few-shot uses the same instruction as the simple prompt zero-shot.
Add ``\CJK{UTF8}{min}{ニュースタイトル:}'' (news title:) and ``\CJK{UTF8}{min}{ニュース記事:}'' (news article:) at the beginning of the news title and the content, respectively.
We randomly sample three contents as examples for few-shot, and use the same examples for all generations.


\subsection{Kazakh Wikipedia}
KazQad is a closed question-answering dataset focused on the Kazakh language~\citep{kazqad}. It contains 5,000 distinct passages covering 1,700 topics derived from Kazakh Wikipedia. These passages span a variety of domains, including art, science, history, sports, and other general topics.
 
We randomly selected 300 titles along with their corresponding paragraphs, ensuring that each paragraph contained at least 3-5 sentences. Additionally, the sampled texts were cleaned and merged by title to increase the length of the texts. This step was necessary because some samples in the original dataset contained extraneous elements such as references, markdown formatting symbols, and other unnecessary characters.
For the machine-generated data, we used the GPT-4o model to generate passages based on the sampled titles. The generation process was initiated with the following prompt: ``Please, write one paragraph about the following topic in Kazakh: [title].''





\subsection{Russian}
We generated machine text for both news and summaries for Russian.

\paragraph{News}
Based on Lenta.ru news from Corus with around 800,000 news samples, we sampled 50 cases for each topic among the six topics: Russia, World, Economy, Sport, Culture, and Science \& Technology. We prompted GPT-4o and Vikhr-Nemo-12B-Instruct-R-21-09-24 to generate to corresponding machine text using \foreignlanguage{russian}{''Напиши новость в области "{topic}" с сайта lenta.ru используя заголовок {title}. Ты должен генерировать новость без заголовка. Новость: ''}

\paragraph{Academic Article Summaries} 
Based on ai-forever/ru-scibench-grnti-clustering-p2p with 31 000 samples, we sampled 30 summaries for each topic among 'Psychology', 'Mechanical Engineering', 'Agriculture and Forestry', 'Geology', 'Biology', and 'Energy', and generated machine counterparts using GPT-4o and Vikhr-Nemo-12B-Instruct-R-21-09-24.  
We used the following prompt: \foreignlanguage{russian}{''Напиши краткое содержание для статьи в области "{topic}" используя заголовок {title}. Ты должен генерировать краткое содержание без заголовка. Содержание:''}



\subsection{Vietnamese}
\selectlanguage{Vietnamese}
\paragraph{Vietnamese News} 
The dataset is crawled from Lao Động newspaper\footnote{\url{https://www.kaggle.com/datasets/phamtheds/news-dataset-vietnameses}} before May 2022 (before the releases of all LLM that support Vietnamese). The dataset contains 290,282 articles with a headlines and a summary of an article in various topics, such as politics, lifestyles, legal, etc. We cleaned the data by removing some missing values and too short summaries (less than 20 words), then randomly selected 600 examples to generate the corresponding 600 machine-generated summaries with GPT-4o using title of articles. The prompt for generation process was
\begin{quote}
    \textit{Bạn là một nhà báo Việt Nam chuyên viết những mẩu tin tóm gọn cho các bài báo bằng cách sử dụng tiêu đề của chúng. Hãy viết cho tôi một đoạn tóm tắt bài báo tiêu đề dưới đây:}
\end{quote}
which means \textit{You are a Vietnamese journalist who writes summaries for articles using their headlines. Please write me a summary of the article with the following headline:} 


\paragraph{Vietnamese Wikipedia} 
We randomly crawled 600 sites from Vietnamese Wikipedia\footnote{\url{https://vi.wikipedia.org/}} with its ID, title, and the introduction part of each topic. For our studies, we generated 600 introduction given a subject using GPT-4o with the prompt:
\begin{quote}
    \textit{Bạn là một nhà đóng góp cho Wikipedia tiếng Việt. Hãy viết cho tôi một đoạn giới thiệu ngắn gọn bằng tiếng Việt về chủ thể bên dưới để đăng trên trang Wikipedia. Lưu ý bắt buộc viết với khoảng \texttt{word\_count} từ.}
\end{quote}
which means \textit{You are a contributor to Vietnamese Wikipedia. Please write me a brief introduction in Vietnamese about the subject below to post on the Wikipedia page. Note that it must be written within \texttt{word\_count} words:}

In this prompt, to avoid the newly generated text has too long passage, since GPT-4o tends to write much longer than human does for a Wikipedia topic, we set the word limits \texttt{word\_count} as word count of the original one.
\selectlanguage{English}


% \section{Distinguishable Signals}

% \subsection{Linguistic Features}
% We first analyze the representative linguistic features between human and machine~\cite{guo-etal-2023-hc3}, and then measure the correlation between linguistic features and detection accuracy~\cite{chein2024human}.

\section{Distinguishable Signals}
\label{sec:distinctionfactor}
This section elaborates detection setting, accuracy and distinction signals summarized by 18 annotators for 16 datasets one by one.

\begin{table}[t!]
    \centering
    \resizebox{\columnwidth}{!}{
        \begin{tabular}{lcccc}
            \toprule
            \textbf{Dialect} & \textbf{Human} & \textbf{\gptfouro} & \textbf{\qwentwo-7.5B} & \textbf{Overall MGT} \\
            \midrule
            EGY & 52.00 & 53.33 & 58.67 & 56.00 \\
            MOR & 54.00 & 53.33 & 48.00 & 50.67 \\
            LEV & 69.33 & 14.67 & 58.67 & 36.00 \\
            GULF & 81.33 & 26.67 & 30.67 & 28.67 \\
            \bottomrule
        \end{tabular}
    }
    \caption{Arabic dialect tweet human detection accuracy over human vs. \gptfouro vs. \qwentwo-7.5B. Machine-generated text is harder than human text to discern. \gptfouro is harder than \qwentwo.}
    \label{tab:arabic-dialect_tweet-accuracy}
\end{table}

\subsection{Arabic}
\label{sec: arabic_insights}
\paragraph{Arabic Dialect Tweets}
% Mervat
The annotation was conducted under the setting III. Single-binary, where a total of 900 tweets across four dialects were analyzed, with ratio of machine-generated vs. human-written text as 2:1 (GPT-4o and Qwen2-7.5B).
% demonstrated strong performance in generating casual Arabic across all four dialects, with GPT-4.0 outperforming Qwen. 

The annotators were not very confident about their choices when deciding whether the data is human-written or not. They achieved an overall accuracy of 50.06\%, highlighting the difficulty of distinguishing machine-generated content in short texts. 
Overall, 64.17\% of human-written tweets were correctly annotated, 36\% human-written tweets were mis-identified as machine-generated.
\tabref{tab:arabic-dialect_tweet-accuracy} presents the human detection accuracy on human text and machine-generated tweets across four dialects, machine-generated text is harder than human text to discern. \gptfouro outputs are more similar to human text, thus more difficult to distinguish compared to \qwentwo.

Since LLMs effectively replicate native speaker vocabulary, human detection cannot rely on distinguishing lexical distinctions. A key indicator of machine-generated text was the use of emojis in an overly formulaic manner. Additionally, some tweets addressed topics that would typically be considered trivial or unlikely for humans to post, further suggesting machine generation.  
Machine-generated tweets also exhibited unnatural tones, misused native expressions, or contained incomplete content. A notable issue in \qwentwo's output was the inclusion of words from other languages within sentences and the mixing of dialects, which is uncommon in natural usage.

% \begin{table}[t!]
%     \centering
%     \resizebox{\columnwidth}{!}{
%         \begin{tabular}{lcccc}
%             \toprule
%             \textbf{Dialect} & \textbf{Human} & \textbf{\gptfouro} & \textbf{\qwentwo-7.5B} & \textbf{Overall MGT} \\
%             \midrule
%             EGY & 52.00 & 53.33 & 58.67 & 56.00 \\
%             MOR & 54.00 & 53.33 & 48.00 & 50.67 \\
%             LEV & 69.33 & 14.67 & 58.67 & 36.00 \\
%             GULF & 81.33 & 26.67 & 30.67 & 28.67 \\
%             \bottomrule
%         \end{tabular}
%     }
%     \caption{Arabic dialect tweet human detection accuracy over human vs. \gptfouro vs. \qwentwo-7.5B.}
%     \label{tab:arabic-dialect_tweet-accuracy}
% \end{table}

% Since the LLMs were able to replicate words commonly used by native speakers, the annotation process primarily focused on detecting generation patterns rather than assessing the accuracy of the terminology. One frequent indicator of machine-generated text was the use of emojis that were considered overly formulaic for natural usage. Additionally, some tweets addressed topics that would generally be seen as trivial or unnecessary for humans to tweet about, further hinting at machine generation.
% Some machine-generated tweets also exhibited tones that were not typically used by native speakers, used native words and expressions in the wrong context, or had incomplete tweets. A notable issue in the data generated by \qwentwo was the inclusion of words from other languages within sentences, and some tweets contained a mixture of dialects, which is uncommon in natural usage.


\paragraph{EASC Articles Summaries}
% Kareem
% The annotation adhered to the first setting where a total of 306 articles summaries were analyzed. We achieved 82\% annotation accuracy. The differentiation between human-written and machine-generated text was based on a feature set empirically validated for its effectiveness in distinguishing the two. Those features:

We sampled 100 (human, \gptfouro) pairs to identify which text is written by human, achieving 82\% accuracy.
The annotator differentiated human-written text from the machine-generated text based on five empirical distinguishable signals.
\begin{itemize}
    \item \textbf{Informative:} LLM summaries are more informative, while humans may overlook key points due to emotional biases, personal perspectives, or incomplete understanding of articles.
    \item \textbf{Abstract and concise:} LLMs present ideas in a more abstract and concise way, without emphasis on specific points. Human summaries often inject personal opinions and beliefs.
     % \item \textbf{Conciseness} LLMs outperform humans in terms of summary conciseness.
    \item \textbf{Religious language:} Humans can accurately use religious language, whereas LLMs generate text in standard language.
    \item \textbf{Prompt reflection:} LLMs often start with words from the prompt, such as
    % For example machine-generated summaries mostly start with: 
    \begin{quote}
    \small
    \begin{RLtext}
            \texttt{هذا المقال}
    \end{RLtext}
    \end{quote}
    \item \textbf{Formatting hints} Human-written summaries contain typos or grammar errors. Machine-generated text includes markdown elements, making it more easily detectable as machine-generated.
\end{itemize}

% It is important to note that some bias exists in the annotation process. Aside from the characteristic features discussed, human-written summaries contained occasional spelling or linguistic mistakes that gave unintended hints to the annotators. Additionally, the machine-generated text included markdown elements, making it more easily detectable as machine-generated.


\paragraph{Youm7 News Articles}
% Saad
% We conducted a human detection task to evaluate the quality of the generated text. Human annotators were tasked with distinguishing between machine-generated and human-written paragraphs. The annotation process achieved an accuracy of \textbf{92.7\%,} highlighting the subtle yet distinguishable differences between the two text types.

% A key finding was that human-generated paragraphs were typically presented as a single, continuous block of text. In contrast, GPT's responses were paraphrased and divided into smaller sections, which provided a structural cue for annotators to identify machine-generated content. Aside from the structural differences, no significant variations in the sequence or thematic patterns of the articles were detected between human- and machine-generated samples.

% These results underscore the importance of structural features, such as paragraph segmentation, in identifying machine-generated text. Although the models preserved the overall narrative and sequence, the segmentation of the text into multiple paragraphs became a key indicator that helped annotators in their task.

We paired 1,000 (human, \gptfouro) examples to identify which text is written by human, achieving accuracy of 92.7\%.
A key finding was that human-written paragraphs were typically presented as a single continuous block of text, whereas GPT-generated article was segmented into smaller sections. This structural difference served as a crucial cue for identifying machine-generated content. No significant variations were observed in thematic patterns or narrative sequences, paragraph segmentation emerged as a key indicator, highlighting the role of structural features in detection.



\paragraph{SANAD News}
% : Tarek
Using the SANAD Arabic news dataset in Modern Standard Arabic (MSA), we evaluated 100 samples to identify human vs. GPT-4o text, achieving $100\%$ annotation accuracy under the setting I. Pair-binary.
We identified several key features that differentiate machine-generated text from human-written text.
\begin{itemize}
    \item \textbf{Markdown presence:} 
    Machine-generated text often contains markdown, which is absent in human-written text.

    \item \textbf{Formatting style:} 
    Machine-generated text is consistently formatted into structured paragraphs, whereas human text tends to appear as large, unstructured blocks of text.

    \item \textbf{Content density:} 
    Human text is richer in factual information, while machine-generated text includes generic statements. For instance, phrases such as 
    % \begin{quote}
    %     \textarabic{ويأتي هذا التطور في ظل توجه عالمي نحو تنويع مصادر الطاقة وتقليل الاعتماد على الوقود الأحفوري}
    % \end{quote}
    mentions of ``Saudi Vision 2030'' are more frequent in machine-generated text. 

    \item \textbf{Source attribution issues:} 
    Machine-generated text often contains source attributions, but these are sometimes incorrect or against the prompt. 
    % For example:
    % \begin{quote}
    %     \textarabic{الرياض (رويترز)}
    % \end{quote}
    % is used instead of the expected ``Khobar Reuters.''

    \item \textbf{Presence of numbers and specific meta data:} 
    Human-written text contains more supporting numerical details, including URLs, phone numbers, currency exchange rates, dates, and amounts, which are not as frequent in machine-generated text.

    \item \textbf{Use of English terms:} 
    Human text includes sporadic English terms, especially in specialized contexts, while machine text remains fully in Arabic.

    \item \textbf{Hashtags and social media elements:} 
    Hashtags are commonly found in human-written text, whereas machine-generated text lacks such social media elements.

    \item \textbf{Narrative structure:} 
    Human-written text follows a narrative structure with a sequence of events, dates, and timelines. On the other hand, machine-generated text tends to resemble an essay on a given topic rather than a chronological news report.

    \item \textbf{Formatting consistency:} Human text formatting varies significantly, often appearing as inconsistent or messy blocks of text with irregular spacing or newlines. We attribute this apparent inconsistency to the fact that the human texts are composed by a large number of authors, while machine text is written by a single model which provides text that is more polished and consistent.

    \item \textbf{Readability and grammar:} 
    Machine-generated text is generally more readable and grammatically correct, while human text may contain stylistic inconsistencies.
\end{itemize}


\subsection{Chinese}
\label{sec:chinese-distinguishable-signal}
\paragraph{Zhihu QA}
The detection involves six unique individuals, with three female and three male annotators. 
We pair (human, \gptfouro) and (human, \qwenturbo), and annotate both under setting I, achieving the average accuracy of 1.0 and 0.98 respectively. The distinguishable factors are summarized below.
\begin{itemize}
    \item \textbf{Humans share personal experience and feelings.} For emotion-rich questions, humans provide empathic comforts by sharing their real personal experiences. However, narratives or stories generated by models tend to appear contrived and artificial, making it easy for humans to detect that they are not grounded by facts, similar to stories created by kids. Model responses thus typically offer solutions, but they are theoretically sound, while often lack practical applicability.    
    
    \item \textbf{Human answers can be informal, mean and sharp.} 
    Human responses sometimes are mean, strongly opinionated, and influenced by personal biases, reflecting a more self-centered perspective. LLMs provide general and less engaging information, but often attempting to help users and offering problem-solving assistance. Some individuals appreciate human answers for their authenticity and directness, others may find them offensive.
    
    \item \textbf{Different Intent.} Zhihu users prioritize expressing their feelings and opinions, rather than assisting and directly addressing the seeker's needs. So they often do not respond to the question directly. Though LLMs aim to assist but typically offer general information related to the entities or concepts in the question, rarely responding with direct and sharp answers.
   
    \item \textbf{Human answers can be extremely short or long.} Some human answers are excessively long and lack proper paragraph segmentation, making them harder to read. LLM-generated responses are generally well-segmented and structured with bullet points or listed points with bold subtitles.

    \item \textbf{LLM responses lack deviation.} LLMs adhere rigidly to instructions, presenting limited flexibility in their responses. When prompted to answer emotion-rich questions with greater empathy, their generative patterns remain predictable, often relying on superficial expressions such as more emoji inserted instead of deeply integrating empathy into the content. For instance, responses like \textit{I'm sorry this happened to you; you should probably consult a professional}, offer minimal support and feel unhelpful.

    \item \textbf{Other indicators} Human answers sometimes contain timeline of answer update and references. 
\end{itemize}

\textit{Improved Prompts}
We carefully designed prompts using a few different user personas: one with a positive attitude, another with a rational and realistic outlook, and a humorous one. For emotion-rich questions, we also applied \gptfouro utilizing \ecot, a framework designed to enhance LLMs’ emotional and empathetic responses~\cite{li2024ecot}. However, the responses are often brief, with emojis but minimal information, which limits their usefulness.


% \textit{Human vs. GPT-4o under I:} 
% [Yuxia], [Bai]
% Machine-generated answers often provide general information related to the entities or concepts in the question, but rarely address it directly and sharply. 

% For emotion-rich questions, model responses typically offer solutions, while humans provide empathic comforts by sharing their personal experiences. Most importantly, these model solutions are theoretically sound, while often lack practical applicability. 
% 3. Some answers I believe only humans can provide, detailed life experience sharing, I believe it is important to read the people sharing the similar experience, instead of general answers.

% Additionally, narratives or stories generated by models tend to appear contrived and artificial, making it easy for humans to detect that they are not grounded by facts, similar to stories created by kids.



% \textbf{[Zhuohan, Rui]}
% Zhuohan Thoughts while annotating the data:
% Intents are different:
% Zhihuers are there to express their feelings, most of them do not care about the questioners while LLMs at least try to help.

% As a third person, I probably like answers from zhihuers, since they are mean, interesting, honest, authentic.
% But As a person who really ask the questions, they probably like the answers that can take their stands (Assume there are really people who asking these questions, some are probably posted by the platform themselves there just to attract attentions)
% 2. Long answers without formatting is a disaster!!! It is not related to nlp, but it is hard to ignore mentally during annotation.
% 3. Some answers I believe only humans can provide, detailed life experience sharing, I believe it is important to read the people sharing the similar experience, instead of general answers.
% 4. Also, consider how much similar responses they have seen.
% 5. We generally do not think what non-ecot generates is good, because we saw them too much, so they become predictable.
%       a. At first, I think ECOT is good, but after I learnt the pattern, it becomes as boring as non-ecot, even worse. It basically gives nothing, at least Chatgpt is trying to do something.
% 6. Maybe we need a classifier, I think some questions still are solution seeking, even though it is related to relationships. (Like, this problem happened to me, what should I do? Ecot always say: i am sorry it happened to you, you should probably consult a professional, which is, I think, useless.

% \textit{Human vs. Qwen-turbo under I:} 
% \textbf{[Jiahui, Jinyan]}
% (1) Human is more informal, and sometimes doesn't response directly to the question. 
% (2)  Human answer can be harsh and more selfish, while LLM answer are more in a moral high ground. 
% (3) LLM like to answer with structure such as listed points with bolded summarization.
% (4) human answer use personal example.
% (5) sometimes, human answer contains time line of answer update
% (6) Human answer can be extremely short or long



% Junior and Senior 
\paragraph{High School Student Essays}
We perform human detection for student essays in two settings.
One NLP postdoc who is a Chinese native speaker did the detection under the setting I for 102 parallel pairs ($hwt$, $mgt_1$) and ($hwt$, $mgt_2$), obtaining the accuracy of 98\%.
For 600 non-parallel cases under the setting II. Pair-four-class, with 150 pairs for each class, another NLP postdoc and two NLP PhD students participated in annotations. They achieved an accuracy of 0.96, 0.96 and 0.99 respectively. 

During the annotation of high school student essays, we identified four remarkable distinction signals that make machine-generated texts easily recognizable:
\begin{itemize}
    \item \textbf{Title:} From the perspective of formatting, machine text tends to begin by ``title: \cn{《xxx》} or title: xxx'', and have newlines between paragraphs while humans does not have.
    % \item For student essay, the most distinctive trait is that, the MGT almost always has the title of the essay, while the human written student essay doesn't contain the title. 
    
    \item \textbf{Formulaic structure:} From the structure and the content of essays, the structure of machine-generated essays is generally formulaic. They often adopt an argumentative style that begins paragraphs with phrases such as ''first, then, moreover, additionally, finally, and overall'' (i.e., \cn{首先,其次,然而,总之,最后}), while HWT is more flexible and the styles are more diverse. Some MGT even use bullet points in an essay, which is rare in human text.
    % \item Machine-generated essays tend to follow a predictable pattern, often adopting an argumentative style that begins with phrases such as \cn{``首先 (first of all), 其次 (secondly), 再次 (then), 最后 (finally).''}

    \item \textbf{Sentiment:} Machine-generated essays tend to adopt a more neutral or positive tone, generally expressing less emotion compared to human-written essays. While real students often express a range of emotions, including sadness, anger, confusion, and a sense of being lost. These emotions reflect the authentic feelings of young people at that age.
    
    \item \textbf{Style:} Machine-generated content may incorporate elements from other genres, such as official document styles. Additionally, machine-generated content may unexpectedly switch to multilingual content, for example, outputting an English paragraph during Chinese content generation.
    % \item Machine-generated text occasionally inserts English words amid Chinese characters, likely due to the mixture of English and Chinese in the pre-training dataset.
\end{itemize}

\textit{Original vs. Improved Prompt}
% Rui
Based on the findings above, we further refined the prompts in the following ways: (1) instructing the model to avoid outputting titles, as these often serve as clear detection cues; (2) discouraging excessive use of connecting words like 'first of all,' 'secondly,' 'then,' and 'finally'; and (3) preventing the mixing of languages other than Chinese.


% distinguishable factors
% There are five remarkable distinction signals between human text and MGT. 
% [Yuxia] distinguishable signal:
% \textit{Parallel Data under I:} 
% One NLP postdoc who is a Chinese native speaker did the detection, obtaining the accuracy of 98\%.



% From the perspective of formatting, machine text tends to begin by ``title: \cn{《xxx》} or title: xxx'', and have newlines between paragraphs while humans does not have.
% From the structure and the content of essays, the structure of machine-generated essays is generally fixed, organizing by words ''first, then, moreover, additionally, finally, and overall'' (i.e., \cn{首先,其次,然而,总之,最后}), while HWT is more flexible and the styles are more diverse. Some MGT even use bullet points in an essay, which is rare in human text.



% \textit{Separate Sources for Human Essays and MGT under II:} 
% [Zhuohan] 
% One NLP postdoc and two NLP PhD students participate in annotations under the setting II. Pair-four-class. They achieved an accuracy of 0.96, 0.96 and 0.99 respectively. During the annotation of high school student essays, we identified several distinguishable factors that make machine-generated texts easily recognizable:


% \textbf{[Rui]}: 
% add accuracy and annotator info above
% Few extra findings:
% Machine-generated essays tend to adopt a more journalistic and neutral tone, generally expressing less emotion compared to human-written essays. Machine-generated content may incorporate elements from other genres, such as official document styles. Additionally, machine-generated content may unexpectedly switch to multilingual content, for example, outputting an English paragraph during Chinese content generation.


% \textbf{[Jiahui]}
% \textbf{[Jinyan]}
% For student essay, the most distinctive trait is that, the MGT almost always has the title of the essay, while the human written student essay doesn't contain the title. 



\paragraph{Government Report}
% : Jiahui
Under setting IV. Triplet-three-class, we presented human-written texts vs. two model outputs across 500 samples to a native Chinese-speaker (postdoctoral researcher specializing in NLG), and asked the annotator to identify which text is human-written. 
% selected the texts they believed were human-written rather than machine-generated. 
% Results showed that the 
The annotator achieved an accuracy of 97.2\%. Human-authored texts were typically longer and contained richer details, often covering multiple topics, while machine-generated texts were generally shorter, lacking noticeable rhythm variations. Additionally, machine-generated texts occasionally included distinct symbols, such as bold formatting and English words.
Errors in distinguishing humans from machine-generated texts primarily occurred when the machine-generated outputs were of similar length to human-written ones. 


\subsection{English Peer Meta Review}
% Alex

% Distinguishable Signals
% After being given a couple of labeled sample data points I was extremely confident of which texts were human-written and which texts were machine-generated as I knew what features were indicative of MGT. However before I was given examples of labeled sample data I still think it is very obvious as to which texts were machine-generated.

% The English machine-generated peer reviews exhibited easily distinguishable features, tending to follow predictable patterns. It commonly gave decisions in a specific format such as ``Decision: Accepted'', which was rarely seen in human texts. The MGT also commonly used labeled headings like ``Strengths'' and Weaknesses to structure peer reviews. Additionally, the MGT sometimes explicitly summarizes multiple reviews while the human texts were never as explicit or rarely included a summarization of multiple reviews. 

% the text could be expressed as a paragraph. The choice of bullet symbol that was used showed little to no variation, and the frequent use of bullet points was extremely common among MGT. 
% Additionally, many MGT responses were of similar length and were generally longer than human texts.

% The Human texts do not seem to follow a pattern, uncommonly contained bulleted lists of any kind, and varied large amounts in length while still being shorter than the MGT majority of the time.

% \textit{Improved prompt:}
% ``Generate a meta review based on the reviews' opinions and authors' rebuttal to make the final decision on whether the paper should be accepted: \texttt{\{Reviews\}}$\backslash $n$\backslash$n. When generating, don't use rigid format or structure such as ``Decision: XX" or headings such as ``Strengths", ``weaknesses" or bullet points. $\backslash $n$\backslash$n
% Meta review:"

Human detection accuracy is 99.75\%.
Before seeing labeled samples, the annotator found the distinction was obvious. After reviewing a few examples, the annotator was extremely confident in distinguishing human-written from machine-generated text based on indicative features of MGT. 

Machine-generated peer reviews exhibited distinct, predictable patterns. They frequently followed a structured format, often providing explicit decisions as ``Decision: Accepted'', which was rare in human-written reviews. MGT also commonly used headings such as ``Strengths'' and ``Weaknesses''. MGT tended to explicitly summarize multiple reviews while the human reviewers rarely did. An example of MGT is like ``Based on the reviews, this paper presents methods for embedding numerical features to improve deep learning models for tabular data. The key points are:''.

MGT additionally presents a notable preference for bullet points, even when a paragraph format was feasible, with minimal variation in bullet style. MGT responses tend to be more uniform in length and generally longer than human reviews. In contrast, human-written meta reviews lacked a fixed structure, rarely used bullet points, and exhibited greater variation in length, though they were typically shorter than MGT.




\subsection{Hindi News}
% : Raj
An overall accuracy of 85.17\% was achieved in distinguishing between the machine-generated and human-written Hindi BBC news articles.
The analysis reveals key stylistic and content-based distinctions.
One significant observation was that machine-generated news content tended to be more concise, presenting less overall information compared to its human-written counterpart. Additionally, machine-generated articles included fewer mentions of names of persons and specific dates, elements that are typically embedded within human-authored news to enhance credibility and specificity.

Furthermore, stylistic disparities were evident, particularly in the absence of colloquial language elements commonly used by human authors. Human-written articles often incorporate idioms, culturally significant phrases, and even a blend of Urdu vocabulary, reflecting the linguistic diversity and nuance of the Hindi language. These elements, however, were noticeably missing in machine-generated content, which instead adhered strictly to the main topic, with minimal linguistic or thematic deviations. This adherence to topic, while adding to clarity, lacked the depth and regional authenticity often present in human-created Hindi news content. Also there was use of English text in the machine generated version.


\subsection{Italian News}
Based on human text from DICE dataset~\citep{bonisoli2023}, we randomly selected machine-generated text from GPT-4o, Anita \citep{polignano2024advanced} and Llama3-405B \citep{llama3} outputs, resulting in total of 300 (human, MGT) pairs.
A native Italian speaker was asked to choose which text is human-written. Detection accuracy is 88.0\% for Anita, 99\% on Llama3-405B and 100\% for GPT-4o.
% Annotation Setting 1: Two options (A and B) are presented, and the task is to choose which text is human-written. For machine-generated text, either GPT-4o, Anita \citep{polignano2024advanced} or Llama-405b \citep{llama3} outputs are randomly selected (300 samples).

% Annotator Background: The annotator is a native Italian speaker.
% Accuracy: (DIcE: anita: 88.0\%, Llama-405b: 99\%, GPT-4o: 100\%).

% \paragraph{DICE Dataset} 
We identified the following distinguishable signals. 
When generating news, all models especially the larger ones, Llama-405b and GPT-4o have consistent formatting, e.g., article title is always enclosed in markdown-like double asterisks ``**'', the city where the issue happens is always mentioned first. These cues are immediately recognizable. After seeing only 2-3 examples, human annotator was confident to make decisions and obtain close to 100\% detection accuracy.


% \paragraph{CiTA Dataset} The main identification style is that models tend to generate more imaginative texts when compared to kids. The latter tend to tell real stories and not go into too much detail about their stories, while the former tend to add irrealistic events seemingly overestimating the imaginative component of a text written by a 11 or 12 years old child.


\subsection{Japanese News}
\label{japanese_news_distinctive_clues}
% : Masahiro and Ryuto
We divided 300 pairs into two groups and two annotators independently annotated them, to identify which text was a human-written BBC news article.
% The dataset consisted of BBC news headlines with articles - one written by a human and the other generated by an LLM for the headline.
% Our experimental results showed that 
Human annotators achieved an average accuracy of 62\%. % in distinguishing between human and LLM-generated texts. 
We identified several distinctive characteristics of LLM-generated texts:
\begin{itemize}
    \item \textbf{Style:} Some texts failed to match news writing styles.
    For instance, in Japanese, writing typically either uses the \textit{desu/masu} or \textit{dearu} style.
    While news articles conventionally use the \textit{dearu} style, LLM-generated texts often use the \textit{desu/masu} style.
    \item \textbf{High reliance on headlines:} The opening sentences frequently relied on the title (headline), either through direct repetition or close paraphrasing.
    \item \textbf{Formulaic phrase:} LLM outputs sometimes contained formulaic phrases like \textit{This article is provided as fiction} or \textit{Here are five reasons: 1. ...}.
    \item \textbf{Typos and grammar issues:} We observed various typographical errors and fluency issues, such as \textit{This four has shocked the entire UK} or \textit{An event that shook the air in Myanmar}.
\end{itemize}
These findings suggest that LLMs have not yet fully mastered the stylistic conventions of news writing.
% It is known that explicit instructions in LLM detection significantly affect the results~\cite{koike-etal-2024-prompt}, indicating that it is necessary to provide 
Explicit instructions that indicate the domain-specific format and style may help the LLM output.




\subsection{Kazakh Wikipedia}
Two native Kazakh speakers annotated 300 (human, \gptfouro) pairs under the setting I, with detection accuracy of 79.67\%.
% Annotation Setting 1: Two options (A and B) are presented, and the task is to choose which text is human-written. For machine-generated text GPT-4o model output was presented.
% Annotator Background: The annotators are native Kazakh speakers.
% Accuracy: 79.67%.
We summarized the following features of MGT.
\begin{itemize}
    \item The generated text lacks diversity in expression, often using repetitive sentence patterns and predictable phrasing, such as frequently ending with ``bolyp tabylady'', which contributes to a mechanical and formulaic feel.
    \item The generated text rarely includes concrete facts, such as numbers or years, and when they do appear, their occurrence is minimal.
    \item The text also frequently includes flattering language, which can be another signal of its artificial nature.
    \item Human-written texts include more Kazakh-specific cultural references where applicable, making them more relatable and authentic, which serves as a clear signal of human-generated content over LLM-generated text.
\end{itemize}



\subsection{Russian}
% Joni
\paragraph{News Articles}
A native Russian speaker (NLP PhD student) annotated 300 examples under the setting I, where machine-generated text was randomly selected from either GPT-4 or Vikhr outputs. Accuracy is 100\%. 
We found that human texts often include specific details like exact dates, numbers, names of people, places, or things (especially those not mentioned in the title or from ordinary backgrounds), as well as ages and other specifics. Additionally, human-written texts tend to reference sources, which is a strong indicator.

% 1. Annotation Setting 1: Two options (A and B) are presented, and the task is to choose which text is human-written. For machine-generated text, either GPT-4 or Vikhr outputs are randomly selected (300 samples).
% 2. Annotator Background: The annotator is a native Russian speaker (myself).
% 3. Accuracy: 100\% (‘Russia,’ ‘World,’ ‘Economy,’ ‘Sport,’ ‘Culture,’ and ‘Science and Technology.’)
% 4. Human-Written Indicators: Human texts often include specific details like exact dates, numbers, names of people, places, or things (especially those not mentioned in the title or from ordinary backgrounds), as well as ages and other specifics. Additionally, human-written texts tend to reference sources, which is a strong indicator.


\paragraph{Academic Article Summaries}
A native Russian speaker (NLP PhD student) was asked to identify whether the given text is human-written or machine-generated under the setting III. Single-binary. A text is randomly chosen from human-written, GPT-4, or Vikhr outputs (300 samples).
The overall accuracy is 80\% (Psychology: 80.0\%, Mechanical Engineering: 74.0\%, Agriculture and Forestry: 74.0\%, Geology: 86.0\%, Biology: 86.0\%, and Energy: 80.0\%).
There are several indicators. Machine-generated texts, especially from Vikhr, often begin with a paraphrased version of the title, which is a strong indicator. Human-written texts typically contain details such as numbers, references, and names. In some Vikhr outputs, sentences may lack coherence, and even after preprocessing, certain artifacts remain visible in the text.


% 1. Annotation Setting 3: The task is to identify whether the text is human-written or machine-generated. The dataset consists of 300 samples, and for each title, a text is randomly chosen from human-written, GPT-4, or Vikhr outputs (300 samples).
% 2. Annotator Background: The annotator is a native Russian speaker (myself).
% 3. Accuracy: 80\% (Psychology (80.0\%), Mechanical Engineering (74.0\%), Agriculture and Forestry (74.0\%), Geology (86.0\%), Biology (86.0\%), and Energy (80.0\%)).
% 4. Machine-Generated Indicators: Machine-generated texts, especially from Vikhr, often begin with a paraphrased version of the title, which is a strong indicator. Human-written texts typically contain details such as numbers, references, and names. In some Vikhr outputs, sentences may lack coherence, and even after preprocessing, certain artefacts remain visible in the text.






\subsection{Vietnamese}
% : Minh
We annotated 600 news summaries and 600 Wikipedia introduction passages in the setting of I. Pair-binary. 
We obtained accuracy of 80.33\% on news summaries, but random guess 50.67\% on Wikipedia text, showing minimal differences between human-written and machine-generated Wikipedia passages.

A key finding was that, for news articles, human tends to provide more details to support the statement (such as date, location, and other specific information), while LLMs tend to offer a general summary. Besides, GPT-4o usually uses short sentences in summaries, but Vietnamese journalists usually write long sentences.

For Wikipedia passages, it is much harder for humans to distinguish since GPT-4o was well-trained on Wikipedia data.
% also, it has a good knowledge about some common topics on the Internet. 
% Since the passages are informative, 
The differences between human-written and machine-generated Wikipedia text is minor, but there are still some distinctions.
For example, when GPT-4o was asked to write about Hai Phong city, city demographics and geography are expected, while it generated a review-like sentence:
% GPT-4o tends to write a passage including some information that a normal human doesn't look for (e.g. When it writes about Hai Phong city, normally, we usually looking for the city demographics, geography, etc., but GPT-4o has a sentence 
\textit{Modern infrastructure, along with open investment policies, have helped Hai Phong become an attractive destination for domestic and foreign investors, contributing to the city's sustainable development.}
%. This is similar to a review article rather than a normal Wikipedia introduction.

% \begin{table}[t]
%     \centering
%     \resizebox{\columnwidth}{!}{
%         \begin{tabular}{lcccc}
%             \textbf{Dataset} & \textbf{Accuracy} \\
%             \midrule
%             Vietnamese news & 80.33 \\
%             Vietnamese Wikipedia introduction passage & 50.67 \\
%             \bottomrule
%             \label{tab:tweets aaccuracy}
%         \end{tabular}
%     }
%     \caption{Accuracy rates for Human and Machine-generated passages across different domains.}
%     \label{tab:vietnamese_acc}
% \end{table}

% \section{Can Prompting Engineering Fill the Gap?}
% for each dataset, we have used the improved prompts to fill this gap, then we ask can the gaps be filled by prompting engneering?
% To answer this question, we plan to sample 200 machine-generated text, 100 from each model: multilingual and the language-specific ones, if only use one model, sample 200 from this model generations.
% observe, whether all observed gaps are filled, label: Yes, No, Partially.
% If partially, what kind gaps tends to be solved by prompting, what gaps are hard to be filled?



\section{Fill the Gap by Prompting?}
\label{sec:fill-gap-prompts}

% \begin{table*}[t!]
%     \centering
%     \small
%     \resizebox{\textwidth}{!}{
%     \begin{tabular}{ll | p{8cm} | p{8cm}}
%     \toprule
%     \textbf{Language} & \textbf{Source} & \textbf{Original Prompt} & \textbf{Improved Prompt} \\
%     \midrule
%     \multirow{4}{*}{Arabic} 
%     & Dialect Tweet & [translated from AR] Write a tweet in '\{dialect\}'. [English Prompt] Write a random tweet in '\{dialect\}' & [translated from AR] Write a random tweet in Arabic. Use dialect '\{dialect\}', express emotions and human experience. [English Prompt] Generate a random tweet in Arabic. Use dialect '\{dialect\}', use human emotions and experience. Output the tweet only. \\
%     & EASC Summary& [translated from AR] Summarize this article while preserving the key points, ensuring conciseness and accuracy: '\{article\}'.& 
%     [translated from AR] Summarize the following article while attempting to simulate a human with intellectual or religious beliefs, ensuring accuracy and conciseness, as per the average human level: '\{article\}'.\\
%     & Youm7 News & [translated from AR] Paraphrase the following news article in clear, clear language, keeping the key information and ideas accurate. Make sure to arrange paragraphs logically and present ideas in an understandable sequence, while using fluent, easy-to-read Arabic. & [translated from AR] Rewrite the provided news article with a refined, well-structured, and sophisticated style that enhances clarity, depth, and coherence. Ensure the article maintains journalistic integrity while improving logical flow, sentence structure, and readability. The revised version should elevate the narrative by incorporating precise language, nuanced transitions, and a compelling tone appropriate for a professional audience. Maintain factual accuracy, emphasize key points effectively, and optimize the article’s structure to enhance engagement and comprehension. Additionally, refine the coherence between paragraphs, eliminate redundancy, and ensure seamless progression of ideas.\\
%     & SANAD News & [translated from AR] You are a professional news writer. Write an article in Arabic consisting of approximately \{word\_count\_rounded\} words. The article is titled: '\{title\}', assume that the source is: '\{source\}', and the general topic falls under '\{topic\_arabic\}'. & 
%     [translated from AR] Act as if you are a professional news writer and write an article in Arabic consisting of approximately \{word\_count\_rounded\} words, titled \{title\}. Ensure that the article is rich in information and precise details, including numbers, dates, and quantitative data when relevant. Make sure to include links, phone numbers, or currency exchange rates when necessary. Use some specialized English terms if the context requires it, and avoid relying solely on generic phrases like "This development comes amid a global trend..." without supporting events or details. Maintain a clear chronological order that reflects the logical sequence of events. Ensure that the text is written as a single paragraph regardless of its length, without dividing it into multiple paragraphs. Keep the formatting imperfect (e.g., uneven spacing) to reflect the nature of human writing, and feel free to use hashtags when needed. Please note that the topic generally falls under the category of \{topic\_arabic\}. Please do not use Markdown at all.\\

%     \midrule
%     \multirow{4}{*}{Chinese}
%     & Zhihu-QA & [translated from Chinese] Imagine you are a rational and analytical Zhihu user. Your objective is to answer question in a clear, objective, and logical manner. Please provide a thorough and well-supported answer. Question: \{Question\} Answer: \newline \cn{假设你是一位理性分析的知乎用户,你的目标是以客观、逻辑的方式回答以下问题。请以理性分析者的身份,给出详细、有据可查的回答。问题:\{Question\} 答案:} & [translated from Chinese] Imagine you are a rational and analytical Zhihu expert. Your goal is to provide objective, logical, and detailed answers while strictly adhering to Zhihu’s style. Keep responses concise, avoiding excessive politeness or formality. Based on the question's professionalism, start with 'Thank you for the invitation' or a similar phrase. If relevant, @ professionals to enhance credibility. Question: \{Questions\} Answer: \newline \cn{假设你是一位理性分析的知乎专家,你精通提问者所提出的问题。 你的目标是以客观、逻辑的方式回答问题。请给出详细、有据可查的回答。回答请严格符合知乎风格,避免长答案和过于礼貌正式的答案。根据问题的专业性,在开头回复“谢邀”或同义词,在必要时请在回答中@一些行业专业人士。问题:\{Questions\}答案:}  \\
%     & Student essay & [translated from Chinese] You are a student currently taking the Gaokao, and you are required to write an essay of approximately 800 words. Please think deeply and express your views clearly according to the topic requirements. The essay should be logical and well-structured, with clear arguments and sufficient evidence, showcasing your unique insights into social, life, or philosophical issues. Ensure that the language is formal and elegant, avoiding colloquial expressions. The structure should be complete, with smooth transitions between paragraphs. The conclusion should summarize and elevate the theme of the essay, demonstrating your in-depth reflection on the topic. \cn{你是一名正在参加高考的学生,现在需要完成一篇800字左右的作文。请根据题目要求,深刻思考,并清晰表达你的观点。文章应有逻辑性和层次感,论点明确、论据充分,能够展示出你对社会、生活或哲理问题的独到见解。注意语言要规范、优美,避免口语化表达,确保文章结构完整,段落之间过渡自然。结尾部分应对文章的主题进行总结和升华,展示你对问题的深入思考。} & [translated from Chinese] You are a student taking the Gaokao and need to write an 800-word essay. Express your views clearly and logically, with well-structured arguments and evidence, reflecting your insights on social, life, or philosophical topics. Use formal and elegant language, avoid colloquial expressions, and ensure smooth transitions between paragraphs. The conclusion should elevate the essay’s theme, showing deep reflection. Avoid rigid structures like "firstly" and "secondly," and if writing narratively, use sincere, emotionally rich language. Do not include a title or heading, and write in Chinese only. \newline \cn{你是一名正在参加高考的学生,现在需要完成一篇800字左右的作文。请根据题目要求,深刻思考,并清晰表达你的观点。文章应有逻辑性和层次感,论点明确、论据充分,能够展示出你对社会、生活或哲理问题的独到见解。注意语言规范、优美,避免口语化表达,确保文章结构完整,段落之间过渡自然。结尾应对文章的主题进行总结和升华,展示你对问题的深入思考。使用多样化的表达,避免过于结构化的表述,例如"首先、其次、然后"等;如果是记叙文,请使用情感丰富的真实表达。请不要输出任何题目或标题,直接开始写作。请全程使用中文。} \\
%     % & Student essay &  \\
%     & Government Report &[translated from Chinese] You are a government report writing assistant. Your goal is to complete a government report on a given topic with logical coherence and clear hierarchical structure. The article should be as lengthy as possible.\cn{你是一个政务文件写作助手,你的目标是在给定的主题下完成一篇政务报告,要求有逻辑感和层次感,文章尽可能长。} & [translated from Chinese] You are a government report writing assistant. Your goal is to complete a government report on a given topic with logical coherence and clear hierarchical structure. The article should be as lengthy as possible. Please avoid overly structured expressions such as "firstly" and "secondly," and maintain a natural tone. The length of each paragraph should vary. Avoid using English words, and where possible, include concrete examples to illustrate key points. \newline \cn{你是一个政务文件写作助手,你的目标是在给定的主题下完成一篇政务报告,要求有逻辑感和层次感。文章尽可能长。请注意避免过于结构化的表达,比如首先,其次,语气自然,每个段落的长度需要有变化。避免出现英文单词,可以的话使用一些具体的事例说明。}  \\
%     \midrule
%     English & Peersum & 
%    Generate a meta review based on the reviews' opinions and authors' rebuttal to make the final decision on whether the paper should be accepted: \texttt{\{Reviews\}} $\backslash $n$\backslash$n Meta review:  & 
%     Generate a meta review based on the reviews' opinions and authors' rebuttal to make the final decision on whether the paper should be accepted: \texttt{\{Reviews\}}$\backslash $n$\backslash$n. When generating, don't use rigid format or structure such as ``Decision: XX" or headings such as ``Strengths", ``weaknesses" or bullet points. $\backslash $n$\backslash$n Meta review: \\
    
%     \midrule
%     Hindi & News & 
%     [translated from Hindi] Here is a news headline: '{headline}' and the content: '{content}'. Write a machine-generated version of the news based on this headline. & 
%     [translated from Hindi] Here is a news headline: '{headline}' and its content: '{content}'. Generate a machine-written version of the news based on this headline. Return the news in Hindi, formatted as plain text. Do not include any additional text—just return the generated news content in Hindi. \\
%     \midrule
%     \multirow{5}{*}{Italian} & \multirow{5}{*}{DICE News}
%     & \textbf{System:} You are an Italian journalist writing for a national newspaper focusing on criminal events happening in the area surrounding Modena. & \textbf{System:} You are an Italian journalist writing for a national newspaper focusing on criminal events happening in the area surrounding Modena. \\
%     & & \textbf{User:} Write a piece of news in Italian, that will appear in a local Italian newspaper and that has the following title: & \textbf{User:} In writing avoid any kind of formatting, do not repeat the title and keep the text informative and not vague. You don't have to add the date of the event but you can.\\
%     \midrule
%     Japanese & News & Please generate a Japanese news article that matches the following news title.$\backslash$n news title: \{title\}$\backslash$n news article: & Please generate a Japanese news article that matches the following news title. When creating the article, please use plain form instead of polite form, and avoid generating the beginning of the article by paraphrasing the title.$\backslash$n news title: \{title\}$\backslash$n news article: \\
%      \midrule
%     \multirow{2}{*}{Kazakh} & \multirow{2}{*}{Wikipedia} & Please, write one paragraph about the following topic in Kazakh:  \texttt{\{topic\}}. & Please, write one paragraph in Kazakh about the following topic: \texttt{\{topic\}}. Avoid repetitive sentence structures and predictable phrasing. When applicable, include concrete facts such as specific numbers, dates, or historical references. Avoid overly flattering or exaggerated language, and instead focus on delivering clear, informative, and relevant content. \\
    
%       & & \foreignlanguage{russian}{Мына тақырып туралы бір абзац жазыңыз: \texttt{{тақырып}}.} &  \foreignlanguage{russian}{Мына тақырып туралы бір абзац жазыңыз: \texttt{{тақырып}}. Қайталағыш сөйлем құрылымдарынан және болжамды сөздерден аулақ болыңыз. Қажет болған жағдайда нақты деректерді, мысалы, нақты сандарды, даталарды немесе тарихи сілтемелерді қосыңыз. Артық мақтау немесе асыра сілтеуден аулақ болыңыз, орнына анық, ақпаратты және сәйкес мазмұнды жеткізуге назар аударыңыз.}\\
    
%     \midrule
%     \multirow{4}{*}{Russian} 
%     & \multirow{2}{*}{News} & Write a news article on the topic \texttt{\{topic\}} from the lenta.ru website using the title \texttt{\{title\}}. You must generate news without a heading. News: & Write a news article on the topic \texttt{\{topic\}} from the lenta.ru website using the title \texttt{\{title\}}. You must generate news without a heading. Add as many details as possible, such as names, numbers, dates, etc. You should also refer to the original source of the information (for example, RIA Novosti, CNN, BBC, etc.). News:\\
%     & & \foreignlanguage{russian}{Напиши новость в области \texttt{\{topic\}} с сайта lenta.ru используя заголовок \texttt{\{title\}}. Ты должен генерировать новость без заголовка. Новость:} &  \foreignlanguage{russian}{Напиши новость в области \texttt{\{topic\}} с сайта lenta.ru используя заголовок \texttt{\{title\}}. Ты должен генерировать новость без заголовка. Добавь побольше деталей, такие как имена, числа, даты и тд. Так же ты должен сослаться на первоисточник информации (например, РИА Новости, CNN, BBC и т.д.). Новость:}\\
%     & \multirow{2}{*}{Academic summary} & Write a summary for an article in the \texttt{\{topic\}} topic using the title \texttt{\{title\}}. You must generate a summary without a title. Contents: & Write a summary for an article in the \texttt{\{topic\}} topic using the title \texttt{\{title\}}. You should generate a summary without a title. Add details about the results, experiments, numbers, references to other work, etc. Contents:\\
%     & & \foreignlanguage{russian}{Напиши краткое содержание для статьи в области \texttt{\{topic\}} используя заголовок \texttt{\{title\}}. Ты должен генерировать краткое содержание без заголовка. Содержание:} &  \foreignlanguage{russian}{Напиши краткое содержание для статьи в области \texttt{\{topic\}} используя заголовок \texttt{\{topic\}}. Ты должен генерировать краткое содержание без заголовка. Добавь деталей про результаты, эксперименты, числа, ссылки на другие работы и т.д. Содержание:}\\
%     \midrule
%     \multirow{2}{*}{Vietnamese} 
%     & Wikipedia & You are a contributor to Vietnamese Wikipedia. Please write me a brief introduction in Vietnamese about the subject below to post on the Wikipedia page. Note that it must be written within \texttt{word\_count} words: & You are a contributor to Vietnamese Wikipedia. Please write me a short introduction in Vietnamese about the subject below to post on the Wikipedia page. Try to understand the topic, and write an introduction that contains information that users often search for on Wikipedia. Try to write the text as humanly as possible. Provide only the introduction, do not provide any other information. Note that you must keep the text length within \texttt{word\_count} words, do not write longer.\\
%     & News & You are a Vietnamese journalist who writes summaries for articles using their headlines. Please write me a summary of the article with the following headline: & You are a Vietnamese journalist who writes headlines for articles using their titles. Please be specific and precise in giving information such as time, place, circumstance, and details, and write a summary for the introduction of the article. Please write me a headline for the article with the following title:\\
%     \bottomrule
%     \end{tabular}
%    }
%     \caption{Original prompts vs. improved prompts used to fill the gap between human and machine text, imitating human style and writing convention.}
%     \label{tab:ori-improved-prompts}
% \end{table*}




\begin{table*}[t!]
    \centering
    \small
    % \resizebox{\columnwidth}{!}{
    \begin{tabular}{llr cll c}
    \toprule
    \textbf{Language} & \textbf{Source/Model} & \textbf{\#Example} & \textbf{\#Annotator} & \textbf{Yes} & \textbf{Partially} & \textbf{No}  \\
    \midrule
    \multirow{4}{*}{Arabic} 
    & Dialect Tweet & 200 & 1 & 106 & 58 & 36 \\
    & ESAC Summary & 130 & 1 & \multicolumn{3}{c}{82.0\% $\rightarrow$ 62.7\%} \\
    & Youm7 News & 200 & 1 & \multicolumn{3}{c}{92.7\% $\rightarrow$ 52.0\%} \\
    & SANAD News & 200 & 1 & \multicolumn{3}{c}{100\% $\rightarrow$ 66.0\%} \\
    \midrule
    \multirow{6}{*}{Chinese} 
    % & Zhihu-QA & 100 & 3 & 100\% $\rightarrow$ 93\% & 100\% $\rightarrow$ 99\% & 99\% $\rightarrow$ 91\% \\
    & Zhihu-QA & 100 & 3 & \multicolumn{3}{c}{99.7\% $\rightarrow$ 94.3\%} \\
    & Zhihu-QA & 200 & 1 & 32 & 70 & 98  \\
    & \multirow{3}{*}{Student essay}  & \multirow{3}{*}{200} & \multirow{3}{*}{3} 
    &  86 & 36 & 78 \\
    & & & & 84 & 36 & 80 \\
    & & & & 45 & 23 & 132 \\
    & Government Report & 200 & 1 & 59 & 38 & 103 \\
    \midrule
    English & Peersum & 200 & 1 & 2 & 113 & 85   \\
    \midrule
    Hindi & News & 200 & 1 & \multicolumn{3}{c}{85.2\% $\rightarrow$ 66\%}  \\
    \midrule
    \multirow{3}{*}{Italian} 
    & DICE News (Anita) & 300 & 1 & \multicolumn{3}{c}{88\% $\rightarrow$ 81\% } \\
    & DICE News (Llama3-405B) & 300 & 1 & \multicolumn{3}{c}{99.7\% $\rightarrow$ 84\% } \\
    & DICE News (GPT-4o) & 300 & 1 & \multicolumn{3}{c}{100\% $\rightarrow$ 85\%}  \\
    % & CItA & \\
    \midrule
    Japanese & News & 200 & 2 & \multicolumn{3}{c}{ 86.4\% $\rightarrow$86\% } \\
    \midrule
    Kazakh & Wikipedia & 200 & 2 & 105 & 89 & 6 \\
    \midrule
    \multirow{2}{*}{Russian} 
    & News & 200 & 1 &  \multicolumn{3}{c}{100\% $\rightarrow$ 86.5\%} \\
    & Academic summary & 200 & 1 & \multicolumn{3}{c}{80\% $\rightarrow$ 69\%} \\
    \midrule
    \multirow{2}{*}{Vietnamese} 
    & Wikipedia & 600 & 1 & \multicolumn{3}{c}{ 50.7\% $\rightarrow$ 47.3\% } \\
    & News & 600 & 1 & \multicolumn{3}{c}{ 80.3\% $\rightarrow$ 63.2\% } \\
    \midrule
    Total & -- & 4,730 & 25 & \multicolumn{3}{c}{ \textbf{87.6\%} $\rightarrow$ \textbf{72.5\%} }  \\ % 943/13
    \bottomrule
    \end{tabular}
   % }
    \caption{Human detection accuracy differences on original vs. improved generations, and survey distribution evaluating whether the new generations fill the gap: Yes, Partially or No.}
    \label{tab:fill-gap-survey}
\end{table*}


In this section, for each dataset, we first present how we designed the improved prompts, and then elaborate whether the new generations are improved and fill the previously-observed gaps between humans.
Original and improved prompts for all datasets and languages are summarized in \tabref{tab:ori-improved-prompts}, and the detection accuracy on new content and fill-gap survey results are demonstrated in \tabref{tab:fill-gap-survey}.


\subsection{Arabic}
\paragraph{Tweets}
To enhance the quality of machine-generated Arabic tweets, prompts were refined to better capture human emotions and generate tweets that can reflect authentic human experiences. One such improved prompt was: ``Generate a random tweet in Arabic. Use {dialect} dialect, use human emotions and experiences. Output the tweet only'' and its corresponding Arabic translation:
\begin{quote}
    \small
    \begin{RLtext}
    اكتب تغريدة عشوائية باللغة العربية. استخدم اللهجة \LR{\{dialect\}}وعبّر عن مشاعر وتجارب إنسانية
    \end{RLtext}
\end{quote}

These adjustments addressed critical gaps appeared in prior outputs. The newly-generated tweets touched on relatable human topics and expressed genuine emotions tied to daily experiences. We used \gptfouro for generation as it yielded promising results, while Qwen is underperformed. As mentioned in \appref{sec: arabic_insights}, Qwen texts often include non-Arabic words within the tweets. 

To evaluate the effectiveness of the improved prompts, 200 sampled tweets were assessed by annotators, determining whether the outputs addressed the identified gaps. The evaluation revealed that 53\% of the tweets fully met the criteria, 29\% partially did, and 18\% did not. However, some limitations were observed: GPT-4o frequently added irrelevant hashtags at the end of tweets, making the outputs identifiable as machine-generated, and the tweets often carried an \textbf{overly optimistic tone}. Even when negative experiences were mentioned, the sentiment leaned toward positivity and new beginnings. 
Overall, while these improved prompts succeeded in filling some previous gaps, new issues emerged, which could be addressed with more precise instructions, such as explicitly avoiding hashtags or adopting a more somber tone when required.


\paragraph{EASC Summary}
Building on the identified discrepancies between human-written and machine-generated summaries discussed in \appref{sec: arabic_insights}, the generation prompt was refined to address these gaps, resulting in the following prompts:
\begin{quote}
    \small
    \begin{RLtext}
            \texttt{قم بتلخيص المقال التالي محاولا محاكاة انسان له اراء و معتقدات فكرية او دينية ملتزما الدقة و الايجاز فمتوسط مستوى البشر\LR{\{article\}:}}.
    \end{RLtext}
\end{quote} 

The data was re-generated using GPT-4o, following the enhanced prompt. This enhancement led to a decrease in annotation accuracy, which dropped to 63\% from 82\%. 
% This decline can be attributed to the increasing difficulty in distinguishing between human-written and machine-generated text.
However, there is still typical machine writing style in the new content, like the use of common machine-generated phrases (e.g., \begin{quote}
    \small
    \begin{RLtext}
            \texttt{يتناول هذا المقال...}
    \end{RLtext}
\end{quote} ), clearly indicating that the text was produced by a machine.
Additionally, the newly-generated summaries exhibited several characteristics.
% that make MGT more challenging to detect:
\begin{itemize}
    \item The summaries seem to reflect a particular ideology, often emphasizing religious themes or beliefs, which influenced the tone and content of the generated text.
    \item The generated sentences are not concise, limiting the clarity and precision of the summaries.
    \item The language used to describe events or natural phenomena tendd to be more emotional, with a tendency to exaggerate or reflect personal preferences.
\end{itemize}

% Additionally, the limitations in human-written summaries, as discussed in \appref{sec: arabic_insights}, continued to impact annotation quality. There were also instances where the writing style of the generated text, or the use of common machine-generated phrases (e.g., \begin{quote}
%     \small
%     \begin{RLtext}
%             \texttt{يتناول هذا المقال...}
%     \end{RLtext}
% \end{quote} ), clearly indicated that the text was produced by a machine.


\paragraph{SANAD}
Before prompt engineering, the accuracy of annotating which of two texts is machine or human written was 100\%. After prompt engineering, the accuracy dropped to 66\% indicating that prompt engineering works at least partially to bridge the gap between the writing styles of humans and machines. 

We follow the same model choices and data subset outlined in \ref{sanad_before_prompt_eng} but we enhance the prompt as shown below:

\begin{quote}
    \small
    \begin{RLtext}
    تصرف كأنك كاتب أخبار محترف، واكتب مقالًا باللغة العربية يتألف من حوالي \LR{\{word\_count\_rounded\}} كلمة، بعنوان \LR{\{title\}}. احرص على أن يكون المقال غنيًا بالمعلومات والتفاصيل الدقيقة، متضمنًا أرقامًا، تواريخ، ومعلومات كمية إذا كانت مناسبة. تأكد من إدراج روابط، أرقام هواتف، أو أسعار صرف العملات عند الضرورة. استخدم بعض المصطلحات الإنجليزية المتخصصة إن كان السياق يتطلب ذلك، وتجنب الاكتفاء بعبارات عامة مثل 'ويأتي هذا التطور في ظل توجه عالمي...' دون دعم بالأحداث أو التفاصيل. راعِ سرد الأحداث بترتيب زمني واضح يعكس التسلسل المنطقي للوقائع. احرص على أن يكون النص مكتوبًا في فقرة واحدة فقط بغض النظر عن طوله، دون تقسيمه إلى فقرات متعددة. اجعل التنسيق غير مثالي (مثل وجود مسافات غير متساوية) لتعكس طبيعة الكتابة البشرية، ولا مانع من استخدام الهاشتاجات عند الحاجة. يرجى مراعاة أن الموضوع يندرج بشكل عام تحت فئة \LR{\{topic\_arabic\}}. يرجى عدم استخدام \LR{Markdown} نهائيًا.
    \end{RLtext}
\end{quote}

Similar to the original prompt, this prompt instructs the LLM to act as a professional news writer and generate an Arabic news article of approximately `\texttt{word\_count\_rounded}' words with the title `\texttt{title}', ensuring that the article's general topic aligns with `\texttt{topic\_arabic}', one of the five predefined categories.

% inspired by earlier observations, 
The enhancements made by this prompt can be summarized as follows:
\begin{itemize}
    \item Produces content rich in information and accurate details.
    \item Includes specific numbers, dates, and quantitative data where applicable.
    \item Encourages the incorporation of specialized English terms when the context requires it.
    \item Advises against vague phrases that are unsupported by concrete events or details.
    \item Requests a clear chronological narration of events.
    \item Requires the article to be written as a single paragraph, regardless of its length.
    \item Allows imperfections in formatting (e.g., uneven spacing) to mimic human-like writing.
    \item Permits the use of hashtags when appropriate.
    \item Prohibits the use of \texttt{Markdown}.
\end{itemize}

The following observations were made from the results of the enhanced prompt:

\begin{itemize}
    \item \textbf{Fake phone numbers} (e.g., 123456789) indicate machine-generated content.
    \item \textbf{Correct phone numbers or URLs} suggest human-generated content.
    \item The use of Arabic \textbf{tatweel} typically indicates human-generated content.
    \item \textbf{Unique names in Arabic} are more likely to be found in human-generated content.
    \item \textbf{Poorly written or formatted texts} are often indicative of human authorship.
    \item \textbf{Generic statements} (e.g., "Vision 2030") are characteristic of machine-generated content.
    \item \textbf{Markdown formatting} present in the text, despite instructions not to use it, suggests machine-generated content.
    \item \textbf{Obvious incorrect information} that seems unlikely for a machine to produce (e.g., "equating MERS COVID") suggests human authorship.
    \item \textbf{Hashtags correctly embedded within the text} point to human-generated content, while those placed at the end of the article suggest machine-generated content.
\end{itemize}


\subsection{Chinese}

\paragraph{Zhihu QA}
We perform both fill-the-gap survey by analyzing whether the gap is filled: Yes or No or Partially, and the manual detection over the improved Zhihu QA responses under the same setting as \tabref{tab:detection-acc}.
% of which one is human-written text given a pair of text under the four-class setting, text1, text2, both and none.
For the gap survey across three datasets, issues of 16\% cases are addressed, 35\% are partially addressed, and half is totally not mitigated, as shown in \tabref{tab:fill-gap-survey}.
Detection accuracy for three annotators declines from 99.7\% to 94.3\% on average, respectively from 100\% to 93\%, 99\% to 81\% and 100\% to 99\%.
Annotators carefully read the contents, and then learned the new detection patterns after reading about 25 pairs, and then annotators detect more quickly depending on the following patterns. Overall, after using the improved prompts, model responses become closer to human answers, more challenging than before to discern.


\paragraph{Student Essay}
Three annotators detect using fill the gap survey. Two annotators have similar observations that gaps in 60\% cases are either fully bridged or partially, 40\% remained. Another annotator is more strict, and think problems are not solved on 65\% cases.  

\paragraph{Government Report} 
The new prompts help mitigate the rigid format of model outputs to some extent and enhance output diversity. We observed that in many original examples, the inputs consist of multiple short paragraphs with highly similar lengths. However, under the new prompts, the model-generated paragraphs exhibited varying lengths, aligning more closely with typical human writing styles. Despite these improvements, certain repetitive phrasing issues persist and remain unresolved. In our analysis, prompt adjustments proved partially effective. Among 200 test cases, 59 showed a noticeable improvement in output quality, 38 exhibited partial improvement, while no significant changes were observed in 103 cases.



\subsection{English Peer Meta Review}
The new English generations retain many features exhibited by previous generations. Even after modifying the prompts, the LLM remains highly formulaic in its outputs. This may be due to the inherently formulaic nature of the peer review domain.

The first half of the second round of generations typically begin with the phrase ``Based on the reviews ...'' and frequently include bullet points or lists. Although the lists were often formatted with numbers or dashes, instead of traditional bullet points, they tended to contain numerous items, leading to responses that felt rigid and uniform. Repetitive use of the phrase ``Given this I recommend to'' further emphasized the predictable nature of these outputs, making them easily identifiable as machine-generated.

In contrast, the second half of the second round of generations moved away from lists, favoring more narrative, paragraph-based structures. The texts were divided into roughly equal paragraphs, which followed a consistent order: a summary of the paper's content, praise for the paper’s novelty, discussion of reviewer concerns, and a concluding acceptance statement. Variations of phrases such as ``The paper introduces ...'' or ``The manuscript under review ...'' were used to begin the majority of the generations.

Despite these adjustments, the gap between machine-generated and human-written text remains significant. While these improved prompts present a more fluid structure, the length of the texts remains relatively uniform, and the overall structure still lacks the variance and natural inconsistency typically seen in human peer reviews. Human-written reviews tend to exhibit more variability in both text length and structure, resulting in a more organic feel.

% Future prompts could focus on introducing greater variability in both text length and structure. By encouraging more flexibility and less predictability, it may be possible to reduce the formulaic nature of the machine-generated outputs. Using prompts that encourage shorter responses or disrupt rigid structures may help make the reviews feel more like those written by humans.



\subsection{Hindi News}
In order to address the disparities observed in machine-generated news, an improved prompt was introduced to improve the quality of machine-generated news. The new prompt explicitly provided the original news content as reference, ensuring the inclusion of more factual details, figures, and names. The prompt instructed the model: 
\begin{quote}
\textit{"Here is a news headline: '\{headline\}' and the content: '\{content\}'. Write a machine-generated version of the news based on this headline. Return the news in Hindi and just return the news content in plain formatted text. Don't return anything extra, just return the plain text in Hindi."}
\end{quote}
This approach aimed to reduce previously noted gaps, such as limited content and lack of proper nouns, dates, and contextual richness, while keeping the focus on generating plain, Hindi-formatted news content. Providing original content would also provide an idea of the style of news writing.

Although the improved prompt succeeded in reducing some of the earlier disparities, some of the disparities still existed in the machine-generated text. These included a noticeable underuse of quoted statements and fewer references to names and illustrative examples, both of which are common in human-written news articles. Furthermore, the inclusion of original content for reference led to a significant decrease in classification accuracy, which decreased to 66\% from 85.2\%. The analysis of machine-generated text using improved prompt shows how challenging it can be to improve the style of machine-generated text while keeping it different from human-written text. The results highlight how important it is to carefully design prompts to help models create more natural and human-like text in Hindi.


\subsection{Italian DICE News}
To mitigate easily identifiable cues, we instruct the model to refrain from using any formatting, incorporate details and specific names regarding the event's location, and freely include witness testimonies. Therefore, we incorporate the following additional instructions into the initial generic prompt: 
\begin{quote}
When writing avoid any kind of formatting, do not repeat the title and keep the text informative and not vague, add quotes from witnesses or the police. You don't have to add the date of the event but you can, use at most 300 words. Do not use mark-down formatting.
\end{quote}
or in Italian when testing Anita which is tuned for this language:
\begin{quote}
Quando scrivi evita qualsiasi tipo di formattazione, non ripetere il titolo e mantieni il testo informativo e non vago, aggiungi citazioni da testimoni o dalla polizia. Non devi aggiungere la data dell'evento ma puoi farlo, usa al massimo 300 parole. Non usare formattazione mark-down.
\end{quote}

Refining the prompt makes machine generated text harder to recognize. When annotating the same 300 samples generated with the new prompt, accuracy decreases. Specifically, Anita goes from 88\% to 81\%, Llama-405b from 99\% to 84\% and GPT-4o from 100\% to 85\%. This means that some gaps are bridged by the new prompts, making the machine-generated texts harder to identify. 
With the new prompt, we remove formatting related patterns and we identify others related to writing style:
\begin{itemize}
    \item The generated text occasionally contains words that depart from journalistic writing style (noticeable from the beginning of the annotation).
    \item Most texts start with one of few prototypical sentences, e.g., \textit{The police ...} and \textit{On that night ...} (noticeable after annotating 10 to 20 samples).
    \item New generations still rarely use quotes, numbers and other specific details (noticeable after annotating 10 to 20 samples).
    \item The model rarely writes in passive tense, which is more usual in Italian than in English (noticeable after annotating more than 100 samples).
    \item Each model uses consistent writing style that is easily identifiable (noticeable after annotating more than 200 samples).
\end{itemize}

Overall, new generations are partially improved, but there are still some patterns that enable recognition of machine-generated text. The challenges reflect that annotators now need to see more examples than before to summarize these patterns and then leverage them to identify MGT, making zero-shot recognition of MGT harder than before. 


\subsection{Japanese News}
Based on the human detection evaluation, we found several distinctive stylistic characteristics in LLM-generated Japanese news with the original prompt, as detailed in \ref{japanese_news_distinctive_clues}.
To prevent LLMs from generating such easily detectable clues, we provide the following improved prompt:

\begin{quote}
    \textit{"次のニュースタイトルに合わせたニュース記事を生成してください。このとき、生成する記事のスタイルには敬体ではなく常体を使用し、またタイトルの言い換えによって記事の冒頭を作成することは避けてください。$\backslash$n ニュースタイトル: \{title\}$\backslash$n ニュース記事: "}
\end{quote}
which indicates
\begin{quote}
    \textit{Please generate a news article that matches the following news title. When creating the article, please use plain form instead of polite form and also avoid generating the beginning of the article by paraphrasing the title.$\backslash$n news title: \{title\}$\backslash$n news article:} 
\end{quote}

By default, while LLMs tend to generate Japanese news articles in a polite form (desu-masu style), human-written news articles are basically in a more plain form (da/dearu style). Therefore, we added instructions to generate articles in a plain form, which is closer to the human writing style in the news domain. Additionally, since LLMs are likely to generate a title at the beginning of the news article, we also included instructions to suppress the task-specific behavior.


\subsection{Kazakh Wikipedia}
Out of 200 annotated examples, our analysis shows that in most cases, the prompt adjustments were effective. We observed 105 instances of ``Yes'', indicating clear improvement, 89 cases of ``Partially'', where some limitations remained, and 6 instances of ``No'', where no noticeable change was observed.

The revised prompt helps reduce repetitive sentence patterns to some extent. The model now produces a more diverse range of sentence structures, partially addressing the issue. However, some predictable phrasing persists, meaning that while variety has improved, formulaic expressions have not been entirely eliminated. The adjustments also successfully increase the inclusion of concrete information, such as dates, names, and specific data points. As a result, the generated text now contains more factual details, addressing this issue almost entirely. Additionally, the prompt helps mitigate excessive flattering language, making the text feel more balanced. However, some instances of exaggeration still occur, indicating that this issue has only been partially resolved.

The inclusion of Kazakh-specific culturally-nuanced details is the most challenging issue to address. The model defaults to Kazakh-specific references only when the topic is explicitly related to Kazakhstan. When it does not recognize the subject, it often makes factual mistakes. For example, when generating text about \textit{Stephen King's It}, the model failed to recognize the name in Kazakh and mis-attributed the novel to a prominent Kazakh author, inventing plot details. This limitation reflects a lack of exposure to Kazakh-specific content, making it difficult to fully resolve through prompt engineering alone.


% The new prompt helps reduce repetitive sentence patterns to some extent. The model now uses a more diverse range of sentence structures, which partially addresses the problem. However, occasional predictable phrasing remains, indicating that while the prompt improves variety, it does not eliminate formulaic expression entirely. The adjustments also successfully increases the inclusion of concrete information, such as dates, names, and specific data points. The generated text now includes more factual details, addressing this issue almost entirely. The prompt partially reduces the use of excessive flattering language, but some subtle instances still remain. The text feels less exaggerated, addressing the issue partially but not entirely.

% When it comes to the inclusion of Kazakh-specific detail, it proved to be the most challenging issue to address. When specifically prompted, the model tends to overdo Kazakh references, often including excessive or fabricated details. Without explicit prompting, it defaults to Kazakh-specific references only when topic is directly about Kazakhstan, or it doesn’t recognize the topic, leading to inaccuracies. For instance, when generating text about Stephen King's It, the model didn't recognize the name in Kazakh and attributed the novel to a prominent Kazakh author, inventing plot details. This limitation reflects a lack of exposure to Kazakh-specific content, making it difficult to fully resolve through prompt adjustments alone.


\subsection{Russian}

\paragraph{Prompt Modifications}
The prompts were adjusted to encourage greater detail and specificity in the generated content:
\begin{itemize}
\item \textbf{Academic summaries:} The simple prompt asked the model to generate a summary without a title based on a given topic and title. The improved prompt added instructions to include details about results, experiments, numbers, and references to other works.
\item \textbf{News articles:} The simple prompt instructed the model to generate a news article based on the given title and topic from the Lenta.ru website without a heading. The improved prompt emphasized including as many details as possible, such as names, numbers, and dates, and encouraged citing original sources like RIA Novosti, CNN, or BBC.
\end{itemize}
After implementing the improved prompts, we observed a decrease in the accuracy of machine-generated text detection. Specifically, for news articles, accuracy dropped from 100\% to 86.5\%, while for summaries, it decreased from 80\% to 69\%. This suggests that the refined prompt instructions successfully made machine-generated text harder to distinguish from human-written text.

\paragraph{Observations and Patterns}
The enhanced prompts led to increased textual complexity and information density, making it more difficult to detect machine-generated text. However, new artefacts appear, which became noticeable after reviewing 50-70 samples:
\begin{itemize}
\item \textbf{Repetitive references:} The generated summaries often cited the same set of names (e.g., \textit{Smith} and \textit{Jones}), while news articles frequently referred to a limited set of news outlets (e.g., \textit{RIA Novosti}).
\item \textbf{Text length:} Human-written news articles were generally shorter than machine-generated ones, which tended to be overly detailed.
\item \textbf{Standardized introduction phrases:} Summaries frequently begin with a predictable opening sentence, such as \textit{``The article discusses ...''}.
\item \textbf{Identifiable uniqueness:} If a summary contained references to names outside the repetitive set or a news article cited a unique news outlet, it was more likely to be human-written. Otherwise, detection became significantly harder.
\end{itemize}

While the refined prompts made detection more difficult, annotator who analyzed a sufficient number of samples could still identify machine-generated text based on these emerging patterns.



\subsection{Vietnamese}
\selectlanguage{Vietnamese}
To address the disparities in these two domains, we introduced a new prompt for each domain to enhance machine-generated text.

For Vietnamese news, we use the prompt:
\begin{quote}
    \textit{Bạn là một nhà báo Việt Nam chuyên viết những phần đầu đề cho các bài báo bằng cách sử dụng tiêu đề của chúng. Lưu ý rằng, hãy đưa ra các thông tin cụ thể và chính xác như mốc thời gian, địa điểm, các tình tiết, chi tiết, đồng thời viết văn phong dưới dạng tóm lược cho phần mở đầu của bài báo. Hãy viết cho tôi một đoạn đầu đề cho bài báo có tiêu đề dưới đây:}
\end{quote}
which means
\begin{quote}
    \textit{"You are a Vietnamese journalist who writes headlines for articles using their titles. Please be specific and precise in giving information such as time, place, circumstance, and details, and write a summary for the introduction of the article. Please write me a headline for the article with the following title:"} 
\end{quote}
This prompt is used to improve writing quality by providing more detailed information, such as dates and times, rather than merely offering general information in the news title. While this information may sometimes be incorrect, its specificity can persuade readers by appearing more credible (e.g., \textit{"In May 2020, Mr. A, the president of company X, stated that..."}).

For Wikipedia articles, we refined the prompt to ensure that machine-generated text includes only information that is appropriate for Wikipedia. Additionally, we instructed the AI to write only the introductory information, without delving into further details.
Our new prompt for Vietnamese Wikipedia is:
\begin{quote}
    \textit{"Bạn là một nhà đóng góp cho Wikipedia tiếng Việt. Hãy viết cho tôi một đoạn giới thiệu ngắn gọn bằng tiếng Việt về chủ thể bên dưới để đăng trên trang Wikipedia. Hãy cố gắng hiểu về chủ đề, và viết ra một đoạn giới thiệu chứa các thông tin mà người dùng thường tìm kiếm trên Wikipedia. Hãy cố gắng viết ra các đoạn văn bản giống người viết nhất có thể. Chỉ đưa ra đoạn giới thiệu, không đưa ra thêm các thông tin khác. Lưu ý rằng bạn phải giữ độ dài văn bản trong khoảng \texttt{word\_count} từ, không được viết dài hơn."}
\end{quote}
which means:
\begin{quote}
    \textit{"You are a contributor to Vietnamese Wikipedia. Please write me a short introduction in Vietnamese about the subject below to post on the Wikipedia page. Try to understand the topic, and write an introduction that contains information that users often search for on Wikipedia. Try to write the text as humanly as possible. Provide only the introduction, do not provide any other information. Note that you must keep the text length within \texttt{word\_count} words, do not write longer."}
\end{quote}

Generally, these improved prompts aim to imbue AI-generated text with characteristics similar to human-written passages, enabling the AI to produce more human-like text. Indeed, humans find it more difficult to distinguish between the two passages, as they are more confused while reading them and struggle to determine which one is machine-generated. The contextual gap between human and AI seems to have narrowed significantly.

However, as noted previously, for news articles, the machine may hallucinate and generate incorrect timestamps for events. Additionally, the writing style of the machine is somewhat unique and consistent across all records, whereas human-written text tends to exhibit greater variation in structure and length. This distinction remains a key factor for humans when identifying whether a passage is machine-generated or written by a human.
\selectlanguage{English}


\begin{table*}[t!]
    \centering
    \resizebox{\textwidth}{!}{
    \begin{tabular}{ll  p{9cm}  p{9cm}}
    \toprule
    \textbf{Language} & \textbf{Source} & \textbf{Original Prompt} & \textbf{Improved Prompt} \\
    \midrule
    Arabic & Dialect Tweet & [translated from AR] Write a tweet in '\{dialect\}'. [English Prompt] Write a random tweet in '\{dialect\}' & [translated from AR] Write a random tweet in Arabic. Use dialect '\{dialect\}', express emotions and human experience. [English Prompt] Generate a random tweet in Arabic. Use dialect '\{dialect\}', use human emotions and experience. Output the tweet only. \\
    Arabic & EASC Summary& [translated from AR] Summarize this article while preserving the key points, ensuring conciseness and accuracy: '\{article\}'.& 
    [translated from AR] Summarize the following article while attempting to simulate a human with intellectual or religious beliefs, ensuring accuracy and conciseness, as per the average human level: '\{article\}'.\\
    Arabic & Youm7 News & [translated from AR] Paraphrase the following news article in clear, clear language, keeping the key information and ideas accurate. Make sure to arrange paragraphs logically and present ideas in an understandable sequence, while using fluent, easy-to-read Arabic. & [translated from AR] Rewrite the provided news article with a refined, well-structured, and sophisticated style that enhances clarity, depth, and coherence. Ensure the article maintains journalistic integrity while improving logical flow, sentence structure, and readability. The revised version should elevate the narrative by incorporating precise language, nuanced transitions, and a compelling tone appropriate for a professional audience. Maintain factual accuracy, emphasize key points effectively, and optimize the article’s structure to enhance engagement and comprehension. Additionally, refine the coherence between paragraphs, eliminate redundancy, and ensure seamless progression of ideas.\\
    Arabic & SANAD News & [translated from AR] You are a professional news writer. Write an article in Arabic consisting of approximately \{word\_count\_rounded\} words. The article is titled: '\{title\}', assume that the source is: '\{source\}', and the general topic falls under '\{topic\_arabic\}'. & 
    [translated from AR] Act as if you are a professional news writer and write an article in Arabic consisting of approximately \{word\_count\_rounded\} words, titled \{title\}. Ensure that the article is rich in information and precise details, including numbers, dates, and quantitative data when relevant. Make sure to include links, phone numbers, or currency exchange rates when necessary. Use some specialized English terms if the context requires it, and avoid relying solely on generic phrases like "This development comes amid a global trend..." without supporting events or details. Maintain a clear chronological order that reflects the logical sequence of events. Ensure that the text is written as a single paragraph regardless of its length, without dividing it into multiple paragraphs. Keep the formatting imperfect (e.g., uneven spacing) to reflect the nature of human writing, and feel free to use hashtags when needed. Please note that the topic generally falls under the category of \{topic\_arabic\}. Please do not use Markdown at all.\\

     \midrule
    Hindi & News & 
    [translated from Hindi] Here is a news headline: '{headline}' and the content: '{content}'. Write a machine-generated version of the news based on this headline. & 
    [translated from Hindi] Here is a news headline: '{headline}' and its content: '{content}'. Generate a machine-written version of the news based on this headline. Return the news in Hindi, formatted as plain text. Do not include any additional text—just return the generated news content in Hindi. \\
    \midrule
    Italian & DICE News
    & \textbf{System:} You are an Italian journalist writing for a national newspaper focusing on criminal events happening in the area surrounding Modena. \newline \textbf{User:} Write a piece of news in Italian, that will appear in a local Italian newspaper and that has the following title: & \textbf{System:} You are an Italian journalist writing for a national newspaper focusing on criminal events happening in the area surrounding Modena. \newline \textbf{User:} In writing avoid any kind of formatting, do not repeat the title and keep the text informative and not vague. You don't have to add the date of the event but you can \\
    \midrule
    Japanese & News & Please generate a Japanese news article that matches the following news title.$\backslash$n news title: \{title\}$\backslash$n news article: & Please generate a Japanese news article that matches the following news title. When creating the article, please use plain form instead of polite form, and avoid generating the beginning of the article by paraphrasing the title.$\backslash$n news title: \{title\}$\backslash$n news article: \\
     \midrule
    Kazakh & Wikipedia & Please, write one paragraph about the following topic in Kazakh:  \texttt{\{topic\}}. \newline \foreignlanguage{russian}{Мына тақырып туралы бір абзац жазыңыз: \texttt{{тақырып}}.} & Please, write one paragraph in Kazakh about the following topic: \texttt{\{topic\}}. Avoid repetitive sentence structures and predictable phrasing. When applicable, include concrete facts such as specific numbers, dates, or historical references. Avoid overly flattering or exaggerated language, and instead focus on delivering clear, informative, and relevant content. \newline  \foreignlanguage{russian}{Мына тақырып туралы бір абзац жазыңыз: \texttt{{тақырып}}. Қайталағыш сөйлем құрылымдарынан және болжамды сөздерден аулақ болыңыз. Қажет болған жағдайда нақты деректерді, мысалы, нақты сандарды, даталарды немесе тарихи сілтемелерді қосыңыз. Артық мақтау немесе асыра сілтеуден аулақ болыңыз, орнына анық, ақпаратты және сәйкес мазмұнды жеткізуге назар аударыңыз.} \\
    
    \bottomrule
    \end{tabular}
   }
\end{table*}

\begin{table*}[t!]
    \centering
    \resizebox{\textwidth}{!}{
    \begin{tabular}{ll  p{9cm}  p{9cm}}
    \toprule
    \textbf{Language} & \textbf{Source} & \textbf{Original Prompt} & \textbf{Improved Prompt} \\
    \midrule
    English & Peersum & 
   Generate a meta review based on the reviews' opinions and authors' rebuttal to make the final decision on whether the paper should be accepted: \texttt{\{Reviews\}} $\backslash $n$\backslash$n Meta review:  & 
    Generate a meta review based on the reviews' opinions and authors' rebuttal to make the final decision on whether the paper should be accepted: \texttt{\{Reviews\}}$\backslash $n$\backslash$n. When generating, don't use rigid format or structure such as ``Decision: XX" or headings such as ``Strengths", ``weaknesses" or bullet points. $\backslash $n$\backslash$n Meta review: \\
    \midrule
    Chinese & Zhihu-QA & [translated from Chinese] Imagine you are a rational and analytical Zhihu user. Your objective is to answer question in a clear, objective, and logical manner. Please provide a thorough and well-supported answer. \newline \cn{假设你是一位理性分析的知乎用户,你的目标是以客观、逻辑的方式回答以下问题。请以理性分析者的身份,给出详细、有据可查的回答。 } & [translated from Chinese] Imagine you are a rational and analytical Zhihu expert. Your goal is to provide objective, logical, and detailed answers while strictly adhering to Zhihu’s style. Keep responses concise, avoiding excessive politeness or formality. Based on the question's professionalism, start with 'Thank you for the invitation' or a similar phrase. If relevant, @ professionals to enhance credibility. \newline \cn{假设你是一位理性分析的知乎专家,你精通提问者所提出的问题。 你的目标是以客观、逻辑的方式回答问题。请给出详细、有据可查的回答。回答请严格符合知乎风格,避免长答案和过于礼貌正式的答案。根据问题的专业性,在开头回复“谢邀”或同义词,在必要时请在回答中@一些行业专业人士。}  \\
    Chinese & Student essay & [translated from Chinese] As a Gaokao student, you must write an 800-word essay that is logical, well-structured, and clearly argued, with sufficient evidence. Express unique insights on social, life, or philosophical issues in formal, elegant language, avoiding colloquialisms. Ensure smooth transitions and a complete structure, with a conclusion that reinforces and elevates the theme. \cn{你是一名正在参加高考的学生,现在需要完成一篇800字左右的作文。请根据题目要求,深刻思考,并清晰表达你的观点。文章应有逻辑性和层次感,论点明确、论据充分,能够展示出你对社会、生活或哲理问题的独到见解。注意语言要规范、优美,避免口语化表达,确保文章结构完整,段落之间过渡自然。结尾部分应对文章的主题进行总结和升华,展示你对问题的深入思考。} & [translated from Chinese] As a Gaokao student, you are required to write an 800-word essay expressing your views clearly and logically, with well-structured arguments and supporting evidence. Reflect on social, life, or philosophical topics, using formal and elegant language while avoiding colloquial expressions. Ensure smooth transitions between paragraphs and a conclusion that deepens the theme. Avoid rigid structures like “firstly” and “secondly,” and if writing narrative, use sincere and emotionally resonant language. Do not include a title or heading, and write in Chinese only. \newline \cn{你是一名正在参加高考的学生,现在需要完成一篇800字左右的作文。请根据题目要求,深刻思考,并清晰表达你的观点。文章应有逻辑性和层次感,论点明确、论据充分,能够展示出你对社会、生活或哲理问题的独到见解。注意语言规范、优美,避免口语化表达,确保文章结构完整,段落之间过渡自然。结尾应对文章的主题进行总结和升华,展示你对问题的深入思考。使用多样化的表达,避免过于结构化的表述,例如"首先、其次、然后"等;如果是记叙文,请使用情感丰富的真实表达。请不要输出任何题目或标题,直接开始写作。请全程使用中文。} \\
    % & Student essay &  \\
    Chinese & Government Report &[translated from Chinese] Please continue writing the full article based on the provided introduction. The original article starts with: \{head\}  \cn{请根据文章的开头续写完整的文章,原文章开头为: \{head\}} & [translated from Chinese] Please continue writing the full article based on the provided introduction: {head}. Please make the article as long as possible without generating symbols like **. Do not generate any content unrelated to the article. The original article starts with: \{head\} \newline \cn{请根据文章的开头续写完整的文章,请让文章尽可能的长,并且不生成**这类符号。请不要生成文章之外的其他内容。原文章开头:\{head\}}  \\
   
    
    \midrule
    Russian & News & Write a news article on the topic \texttt{\{topic\}} from the lenta.ru website using the title \texttt{\{title\}}. You must generate news without a heading. News: \newline \foreignlanguage{russian}{Напиши новость в области \texttt{\{topic\}} с сайта lenta.ru используя заголовок \texttt{\{title\}}. Ты должен генерировать новость без заголовка. Новость:}
    & Write a news article on the topic \texttt{\{topic\}} from the lenta.ru website using the title \texttt{\{title\}}. You must generate news without a heading. Add as many details as possible, such as names, numbers, dates, etc. You should also refer to the original source of the information (for example, RIA Novosti, CNN, BBC, etc.). News: \newline \foreignlanguage{russian}{Напиши новость в области \texttt{\{topic\}} с сайта lenta.ru используя заголовок \texttt{\{title\}}. Ты должен генерировать новость без заголовка. Добавь побольше деталей, такие как имена, числа, даты и тд. Так же ты должен сослаться на первоисточник информации (например, РИА Новости, CNN, BBC и т.д.). Новость:}\\
    Russian & Academic summary & Write a summary for an article in the \texttt{\{topic\}} topic using the title \texttt{\{title\}}. You must generate a summary without a title. Contents: \newline \foreignlanguage{russian}{Напиши краткое содержание для статьи в области \texttt{\{topic\}} используя заголовок \texttt{\{title\}}. Ты должен генерировать краткое содержание без заголовка. Содержание:}
    & Write a summary for an article in the \texttt{\{topic\}} topic using the title \texttt{\{title\}}. You should generate a summary without a title. Add details about the results, experiments, numbers, references to other work, etc. Contents: \newline \foreignlanguage{russian}{Напиши краткое содержание для статьи в области \texttt{\{topic\}} используя заголовок \texttt{\{topic\}}. Ты должен генерировать краткое содержание без заголовка. Добавь деталей про результаты, эксперименты, числа, ссылки на другие работы и т.д. Содержание:}\\

    \bottomrule
    \end{tabular}
   }
\end{table*}
% TODO: Rui will double check with Jiahui; turn it into long table -> if both longtable and supertabularx not working I will split it into multiple manually -> I will manually transform it into multiple tables, this table is very hard to render


\begin{table*}[t!]
    \centering
    \resizebox{\textwidth}{!}{
    \begin{tabular}{ll  p{9cm}  p{9cm}}
    \toprule
    \textbf{Language} & \textbf{Source} & \textbf{Original Prompt} & \textbf{Improved Prompt} \\
    \midrule
    Vietnamese & Wikipedia & You are a contributor to Vietnamese Wikipedia. Please write me a brief introduction in Vietnamese about the subject below to post on the Wikipedia page. Note that it must be written within \texttt{word\_count} words: & You are a contributor to Vietnamese Wikipedia. Please write me a short introduction in Vietnamese about the subject below to post on the Wikipedia page. Try to understand the topic, and write an introduction that contains information that users often search for on Wikipedia. Try to write the text as humanly as possible. Provide only the introduction, do not provide any other information. Note that you must keep the text length within \texttt{word\_count} words, do not write longer.\\
    Vietnamese & News & You are a Vietnamese journalist who writes summaries for articles using their headlines. Please write me a summary of the article with the following headline: & You are a Vietnamese journalist who writes headlines for articles using their titles. Please be specific and precise in giving information such as time, place, circumstance, and details, and write a summary for the introduction of the article. Please write me a headline for the article with the following title:\\
    \bottomrule
    \end{tabular}
   }
    \caption{Original prompts vs. improved prompts used to fill the gap between human and machine text, imitating human style and writing convention.}
    \label{tab:ori-improved-prompts}
\end{table*}



\subsection{Subjectivity in Fill-the-gap Survey}
\label{app:subjectivity-test}
\begin{figure*}[t!]
    \centering
    \includegraphics[scale=0.6]{section/images/IAA-heatmap.pdf}
    \caption{Three annotator agreement on Chinese essays regarding whether the improved prompts mitigate the gap between human text and machine-generated text.}
    \label{fig:iaa-heatmap}
\end{figure*}

To examine the subjectivity of this survey, i.e., the variation between annotators when evaluating the same set of examples, we asked three annotators who previously detected Chinese student essays to assess the same set of newly generated essays and measure their inter-annotator agreement over 200 cases.
\figref{fig:iaa-heatmap} shows that the two postdoc annotators exhibit a higher correlation, whereas the female PhD annotator demonstrates significant disagreement with them.
This suggests that annotators' backgrounds influence the survey results, and individual biases are not negligible factors.
To obtain more objective results, we conducted a second round of detection on the new generations, as we believe detection accuracy provides a more objective measure.

\subsection{Automatic Detection Data and Results}
\tabref{tab:multilingual_prompt_dist} presents the distribution of data used in automatic MGT detection, and the results are shown in \figref{fig:auto-acc-diff}. 19 among 26 automatic detection approaches demonstrate lower detection accuracy on the improved generations produced by the adjusted prompts.

\begin{figure*}[ht!]
    \centering
    \includegraphics[scale=0.55]{section/images/auto-acc-bars.pdf}
    \caption{\textbf{Detection accuracy differences} of 26 automatic machine-generated text detection approaches on original vs. improved generations.}
    \label{fig:auto-acc-diff}
\end{figure*}

\begin{table*}[ht]
\centering
\scalebox{0.7}{
\begin{tabular}{>{\raggedright\arraybackslash}p{3cm}|>{\centering\arraybackslash}p{2cm}>{\raggedleft\arraybackslash}p{1.9cm}>{\raggedleft\arraybackslash}p{1.6cm}>{\raggedright\arraybackslash}p{12cm}}
\hline
\rule{0pt}{2.5ex}Source / Domain & Language & \# Improved & \# Original & LLM Generator List \rule[-1.2ex]{0pt}{0pt}\\
\hline
\rule{0pt}{2.5ex}QA & Chinese & 3422 & 9842 & GPT-4o (3421), GPT-4o-mini (6845), GPT-4o-2024-05-13 (2998)\rule[-1.2ex]{0pt}{0pt}\\
\hline
\rule{0pt}{2.5ex}Essay & Chinese & 849 & 702 & Claude-3-5-Sonnet (773), GLM-4-9B-Chat (778)\rule[-1.2ex]{0pt}{0pt}\\
\hline
\rule{0pt}{2.5ex}GovReport & Chinese & 16776 & 0 & Baichuan2-13B-Chat (5521), ChatGLM3-6B (5359), GPT-4o (5896) \rule[-1.2ex]{0pt}{0pt}\\
\hline
\rule{0pt}{2.5ex}News & Hindi & 0 & 1199 & GPT-4 (600), Human (599)\rule[-1.2ex]{0pt}{0pt}\\
\hline
\rule{0pt}{2.5ex}News & Japanese & 0 & 300 & GPT-4o (300) \rule[-1.2ex]{0pt}{0pt}\\
\hline
\rule{0pt}{2.5ex}News & Russian & 5915 & 600 & GPT-4o (3921), Vikhrmodels/Vikhr-Nemo-12B-Instruct-R-21-09-24 (3224)\rule[-1.2ex]{0pt}{0pt}\\
\hline
\rule{0pt}{2.5ex}Sunmary & Arabic & 153 & 153 & GPT-4o (306)\rule[-1.2ex]{0pt}{0pt}\\
\hline
\rule{0pt}{2.5ex}Sunmary & Russian & 5944 & 585 & Vikhrmodels/Vikhr-Nemo-12B-Instruct-R-21-09-24 (3279), GPT-4o (3300)\rule[-1.2ex]{0pt}{0pt}\\
\hline
\rule{0pt}{2.5ex}Tweets & Arabic & 2800 & 214 & GPT-4-Turbo (1400), GPT-4 (1545), Qwen2.5 72B (69)\rule[-1.2ex]{0pt}{0pt}\\
\hline
\rule{0pt}{2.5ex}Total & -- & 32,487 & 17,017 & -- \rule[-1.2ex]{0pt}{0pt}\\
\hline
\end{tabular}
}
\caption{Statistics of data used in automatic detection: generations using original and the improved prompts.}
\label{tab:multilingual_prompt_dist}
\end{table*}

\subsection{Why do Human Preference strongly differ in Zhihu QA?}
\label{app:humanpreference}
% \paragraph{Why do we strongly differ in Zhihu QA?}
We summarized three major reasons.

\textbf{Mean-spirited human Zhihu answers.}
One of the female annotators prefers responses that are sincere, authentic, and uplifting. However, many human-written responses tend to show the dark aspects of society by often adopting a strongly negative tone through harsh, sarcastic, or offensive remarks targeting specific groups. While these responses may contain some facts, they present a narrow, subjective perspective rather than a comprehensive view. In such cases, a more positive or neutral response is generally perceived as more mature and reasonable, aligning with Chinese cultural norms of expressing individual opinions in a mild and modest manner.  

This does not imply that machine-generated responses are ideal. When asked for personal opinions, LLMs typically provide generic, average viewpoints that lack depth, wisdom, and inspiration. Human annotators tend to prefer personalized, insightful reflections and suggestions informed by real experiences and critical thinking, rather than generic statements that merely reiterate common knowledge. This limitation may stem from their lack of personal experiences and the inability to internalize knowledge into a coherent philosophical perspective or behavioral framework.
Also, human answers would extend the answer from the current topic to other relevant topics, rather than only responding to this question.
They can tell from the history and naturally incorporate personal thoughts into it, while models do not have this ability or nature. 


\textbf{Humans more trust human suggestions.}
For questions asking for clinical suggestions, graduate program application experience, recommendation of restaurants, educational institution, tutoring courses, and teacher in a city, town or even a district, humans tend to trust more on human answers than model generations.
These questions require either expert knowledge or real personal experience.
Compared with model generic opinion and suggestions, humans offer advice in a way of real professional expertise or share firsthand experience with more details, feeling and suggestions, preferred by individuals.
 % qid = 264: 同大插班生和科兴教育辅导机构今年各学校都考走了多少学生?@科兴教育@同大插班生
For example, \textit{How many students did Tongda Students and Kexing Education Tutoring Institution get admitted to top universities this year?} Kexing Education and Tongda Students are two external tutoring schools to help improve scores of high school students in admission exams. This internal information is only known by a small group of people in these tutoring schools. LLMs generation must be less trustworthy and are even factually incorrect. 
% Question 250 :
% Quesion: 如何评价《濑户内海》这篇小说? 我的资讯--网易云阅读如题
% Inconsistency in model generations and factual errors


\textbf{Emotional and complex relationship issues are difficult for models.}
Sometimes it is difficult for models to understand complex relationship and emotional issues and provide practical suggestions.
For example, \textit{I have 87 days left until the college entrance exam as a student who attended this exam the second time, and my girlfriend, who is in university, and we have been together for over 450 days. When I went to see her, she told me she was exhausted and that I couldn't provide the support she needed. She said that I was not someone who could stand by her through tough times. Since this was my first relationship, I don't know how to do. She broke up with me decisively.}
In such a scenario, human responses are more realistic and practical, comforting from their own relationship experience and offering suggestions though some words can be more polite and kind.


