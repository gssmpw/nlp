\section{Related work}
\label{sec:related-work}
%% =============================================================================

\subsection{Art therapy and its role in mental health}
\label{subsec:ATMH}


\rev{
Art therapy has been recognized as an effective tool for promoting mental health in the clinical context____. 
Common forms of art therapy include creative activities such as drawing, painting, and sculpting. In these activities, artwork functions as a non-verbal medium for individuals to express emotions, process trauma, and engage in self-reflection, helping them to manage a range of psychological issues, including anxiety, depression, and Post-traumatic stress disorder (PTSD)____. Such an approach, however, requires the engagement of expert \rev{therapists} throughout the session and preparation, as well as specific conditions for patients to participate in creative activities. This restricts access for individuals with physical or contextual limitations.
}

\rev{
Visual engagement with artwork is another widely used form of art therapy in clinical settings, particularly for its effectiveness as a positive distraction____. It offers a broad range of applications for patients due to its less complex and demanding nature. For instance, visual exposure to art has been used for critically ill patients to address stress, anxiety, and perceived pain____, or to alleviate agitation and anxiety in patients with severe mental health conditions____. It has also been applied in operating rooms and recovery areas, where it not only reduced anxiety but also contributed to faster recovery times____.
}
\rev{
The effects of visual engagement with artwork are further exemplified in recent studies____, which employed a technique from narrative therapy____ to encourage prolonged and active engagement with the artwork. These studies demonstrated significant temporary mood regulation effects for former patients, including those who had been in the ICU. This shows promising potential for visual art engagement as an additional intervention for preventing PICS, alongside currently practiced methods such as ICU diary____ and post-ICU clinic____.
}

\rev{
In art therapy through visual engagement, it is crucial to use appropriate content in the artwork, much of which is dominated by nature-based themes, such as landscapes featuring trees and water____. The therapeutic effects of nature-based visual art are rooted in theories such as evolutionary theory____ and the Biophilia hypothesis____. These theories suggest that natural environments, which provided strategic survival advantages for early humans, elicit positive affective responses (for an overview of nature-based visual stimuli eliciting positive affective responses, see Kim et al.____). Following the positive outcomes of art therapy through visual exposure there is growing interest in exploring how art experiences might aid in the emotional and psychological recovery of PICS patients____. However, the content of the artwork can also have negative effects if not carefully chosen. A study by Ulrich____ demonstrated that abstract art with straight-edged forms can provoke strong negative reactions explaining that in highly stressful situations, patients may project their negative emotions onto such artwork, leading to adverse visual experiences. These examples underscore that while art therapy holds promises, its application requires careful customization to ensure both its effectiveness and safety. It highlights the importance of thoughtful selection and integration of artwork in a therapeutic practice, which can make its preparation and implementation time- and labour-intensive for the therapist.
}

%\rev{Technology-enabled Art therapy}
\rev{\subsection{State-of-the-Art Digital Art Therapy}}
\rev{
Recent advancements in digital technology can offer significant benefits in addressing the limitations of art therapy. 
On the one hand, for example, digital technology supports therapists with efficient scheduling and management of sessions 
while supporting patients to overcome their physical and contextual barriers,
by enabling patients to receive therapy in the comfort of their own homes.
On the other hand, digital technology also allows therapists to address anxieties or compulsions as they arise, 
thereby providing immediate support and guidance____. 
Digital technology also expands the tools available for art therapy: 
the use of digital tools for creating and appreciating art, 
as well as the introduction of innovative methods such as interventions based on videos or digital games____. 
By adopting and integrating these technologies, art therapy can empower patients in both physical and mental manners, 
enabling new avenues for self-expression and therapeutic engagement____.
}

\rev{
A recent study by Cooney and Menezes____ developed a robotic therapist 
that assists individuals in expressing emotions through stylized and symbolic means facilitating self-exploration, 
which might otherwise be limited by personal skill,
while allowing for a highly personalized therapeutic experience.
}\rev{Another study by Liu, Zhou, and An ____ integrated the use of image-based generative AI into expressive art therapy to help children and families express emotions and thoughts. While the study demonstrated the potential of AI as a tool to enhance participants' creative potential, it also highlighted the risks, particularly regarding safety due to the uncertainties in generative AI that could lead to inappropriate or unsuitable content. To reduce the risk, the study recommended the development of more age-appropriate generative AI interfaces through collaboration among parents, children, and therapists.}\rev{ 
 A study by Yilma et al.____ introduced AI-driven VA RecSys engines 
to recommend artworks tailored to the emotional and psychological needs of PICS patients. 
These systems leverage text-based, image-based, and multimodal methods to analyze and recommend art. 
Text-based VA RecSys employ Natural Language Processing (NLP) techniques, 
such as LDA____ and BERT____, 
to interpret textual descriptions of art. 
Image-based VA RecSys, which analyze visual features like color and composition, 
proved more effective in aligning recommendations with therapeutic goals. 
These systems prioritized calming and uplifting artworks, 
which are essential for emotional well-being during recovery____. 
Multimodal VA RecSys, such as BLIP____, integrating textual and visual inputs, 
offered a balanced approach that improved recommendation accuracy. 
}

\rev{
These findings highlight the importance of designing AI systems that incorporate healing elements, 
such as sensory pleasure and engagement, to maximize their therapeutic impact. 
However, those results also underscore a critical risk; 
some of the AI-based approaches generated recommendations of paintings 
that included imagery of ruins, destruction, and darker aesthetics—elements, 
which are particularly dangerous and inappropriate for therapeutic settings. 
These outcomes highlight that while AI shows considerable promise in supporting art therapy, it cannot be relied upon in isolation. 
Human intervention is pivotal to ensure that the therapeutic content is both safe and effective. 
To address this challenge, the present study explores human-AI collaboration 
by proposing a collaborative approach that integrates both therapist expertise and AI capabilities for art therapy in PICS interventions.
}





%% =============================================================================