%\documentclass[5p,times]{elsarticle}
%\documentclass[review]{elsarticle}
\documentclass[12p]{elsarticle}

%\AtBeginDocument{\renewcommand*{\thesubfigure}{\alphalph{\value{subfigure}}}}
\renewcommand{\footnotesize}{\scriptsize}
\usepackage{textcomp}
%\smartqed  % flush right qed marks, e.g. at end of proof
%\usepackage[breaklines]{listings} % Enable wrapping for inline code
\usepackage{listings} % For lstinline command
%\usepackage{graphicx} % Optional for text wrapping with parbox
\usepackage{graphicx}
%\usepackage{subfig}
\usepackage{caption}
\usepackage{subcaption}
%\usepackage{natbib}
%\usepackage{subfigure}
% \usepackage{etex} % Extend TeX capacity % to add more fig
% \reserveinserts{28} % Increase inserts to add more fig
%\usepackage[table]{xcolor}

\newtheorem{hypothesis}{Hypothesis}
\newtheorem{nullhypothesis}{Null Hypothesis}

\usepackage{csquotes}

\usepackage{tikz}
\usetikzlibrary{tikzmark}
 %% footnote fonts
\usetikzlibrary{fit} %% Used for putting dotted box in image
%\usepackage[margin=2cm]{geometry}
%\usepackage{beamerposter}
\usetikzlibrary{positioning}
\usetikzlibrary{arrows}
\usetikzlibrary{shapes.multipart}

\usepackage[hidelinks,bookmarks=false]{hyperref}
\usepackage[numbered]{bookmark}
\usepackage{color,soul}
\usepackage{booktabs}
\usepackage{multirow}
\usepackage{url}
%\usepackage[small,it]{caption}
\usepackage{tcolorbox}
%\usepackage{cite}
\usepackage{amsmath,amssymb,amsfonts}
\usepackage{algorithmic}
\usepackage{xcolor}
\usepackage{url}
\usepackage{tabularx} 
\usepackage{array}
\usepackage{lineno,hyperref}
\usepackage{longtable}
\renewcommand{\arraystretch}{1.2} % Adjust row height for readability
\modulolinenumbers[5]
\journal{Journal of \LaTeX\ Templates}
 
%\usepackage{tabularx}
\usepackage{pgfplots}
%\pgfplotsset{width=7cm,compat=1.8,tick label style={font=\small}}
\pgfplotsset{compat=1.14}



%\usepackage{graphicx}
\usepackage{adjustbox}
\usepackage{float}
\usepackage{rotating}
\usepackage{tablefootnote}
\usepackage{nth}
\DeclareCaptionType{TextBox}
\newcolumntype{L}{>{\centering\arraybackslash}m{3cm}}

\tcbset{tab1/.style={fonttitle=\bfseries\large,fontupper=\normalsize\sffamily,
colback=yellow!6!white,colframe=red!75!black,colbacktitle=black!75!black,
coltitle=white,center title,freelance,frame code={
\foreach \n in {north east,north west,south east,south west}
{\path [fill=red!75!black] (interior.\n) circle (3mm); };},}}

\tcbset{tab2/.style={enhanced,fonttitle=\bfseries,fontupper=\tiny\sffamily,
colback=yellow!6!white,colframe=red!50!black,colbacktitle=black!75!black,
coltitle=white,center title}}
%
\usepackage[T1]{fontenc}
\usepackage{array, booktabs, makecell}
\usepackage[utf8]{inputenc}
\usepackage{dirtytalk}
\usepackage{tcolorbox}

\usepackage[british]{babel}
\usepackage{enumitem}

\newlist{SubItemList}{itemize}{1}
\setlist[SubItemList]{label={$-$}}

\let\OldItem\item
\newcommand{\SubItemStart}[1]{%
    \let\item\SubItemEnd
    \begin{SubItemList}[resume]%
        \OldItem #1%
}
\newcommand{\SubItemMiddle}[1]{%
    \OldItem #1%
}
\newcommand{\SubItemEnd}[1]{%
    \end{SubItemList}%
    \let\item\OldItem
    \item #1%
}
\newcommand*{\SubItem}[1]{%
    \let\SubItem\SubItemMiddle%
    \SubItemStart{#1}%
} 


%%%%%%%% start: comments %%%%%%%%   
\newcommand{\todo}[1]{\textcolor{purple}{{[TODO: #1]}}}
\newcommand{\Eman}[1]{\textcolor{blue}{{\it [Eman: #1]}}}
\newcommand{\ali}[1]{\textcolor{red}{{\it [Ali: #1]}}}
\newcommand{\anthony}[1]{\textcolor{orange}{{\it [Anthony: #1]}}}
\newcommand{\christian}[1]{\textcolor{red}{{\it [Christian: #1]}}}
\newcommand{\mohamed}[1]{\textcolor{green}{{\it [mohamed: #1]}}}
%%%%%%%% end: comments comments %%%%%%%% 

%%%%%%%% start: dataset information %%%%%%%%  
\newcommand{\projectsNumber}{800\xspace}
\newcommand{\commitsNumber}{111,884\xspace}
\newcommand{\refactoringsNumber}{711,495\xspace}
\newcommand{\sarInitialNumber}{513\xspace}
\newcommand{\labeledCommitsNumber}{1,702\xspace}
\newcommand{\numberSAR}{230\xspace}

%%%%%%%% end: dataset information %%%%%%%% 
%%%%%%%%%%%%%%
\usepackage{framed}
\usepackage{mdframed}
\usepackage{pstricks}
\usepackage{xspace}
%%%%%%%%%%%%%%
\usepackage{pgf-pie}
\usepackage{comment}
\usepackage{colortbl}
\tcbuselibrary{skins}
\tcbuselibrary{listings}
\usepackage[normalem]{ulem}
\usetikzlibrary{patterns,}
\usepgfplotslibrary{colorbrewer}
\usepackage{verbatim}
%\usepackage{fullpage}
% Adding packages to fix RQ2 missing figure
\usepackage{tikz}
\usetikzlibrary{tikzmark}
\usetikzlibrary{fit} %% Used for putting dotted box in image
%\usepackage[margin=2cm]{geometry}
%\usepackage{beamerposter}
\usetikzlibrary{positioning}
\usetikzlibrary{arrows}
\usetikzlibrary{shapes.multipart}
% For rotating figures, tables, etc.
%  including their captions
\usepackage{lscape}
\usepackage{placeins}
\usepackage{afterpage}
\newlength\mylength
\usepackage{fontawesome}
\usepackage{xspace}
\newcommand{\ie}{\textit{i.e., \xspace}}
\newcommand{\eg}{\textit{e.g., \xspace}}
\newcommand{\etal}{\textit{et al. \xspace}}
\newcolumntype{L}{>{\arraybackslash}m{16cm}}
\usepackage[flushleft]{threeparttable}
\usepackage{mdframed}
%\usepackage{subfigure}
%\usepackage{tablenotes}
%%%%

\newtcolorbox{boxK}{
    sharpish corners, % better drop shadow
    boxrule = 0pt,
    toprule = 4.5pt, % top rule weight
    enhanced,
    fuzzy shadow = {0pt}{-2pt}{-0.5pt}{0.5pt}{black!35} % {xshift}{yshift}{offset}{step}{options} 
}
%\usepackage{apacite}
%\bibliographystyle{model5-names}\biboptions{authoryear} eswa
\bibliographystyle{elsarticle-num}
\begin{document}

\begin{frontmatter}

\title{An Empirical Study on the Impact of Code Duplication-aware Refactoring Practices on Quality Metrics}
%An Empirical Study on Refactoring Documentation in Open Source Software}
%\subtitle{Do you have a subtitle?\\ If so, write it here}


%% Group authors per affiliation:
\author[STEVENS]{Eman Abdullah AlOmar\corref{mycorrespondingauthor}}
\cortext[mycorrespondingauthor]{Corresponding author}
\ead{ealomar@stevens.edu}


%\author[RIT]{Nasir Safdari}
%\ead{nsafdari@rit.edu}
%\author[ETS]{Ali Ouni}
%\ead{ali.ouni@etsmtl.ca}

\address[STEVENS]{Stevens Institute of Technology, Hoboken, NJ, USA}



\begin{abstract}
%%% reduced version
\textbf{Context:} Code refactoring is widely recognized as an essential software engineering practice that improves the
understandability and maintainability of source code. Several studies attempted to detect refactoring activities through mining software repositories, allowing one to collect, analyze, and get actionable data-driven insights about refactoring practices within software projects. 

\noindent\textbf{Objective:} Our goal is to identify, among the various quality models presented in the literature, the ones that align with the developer's vision of eliminating duplicates of code, when they explicitly mention that they refactor the code to improve them. 

\noindent\textbf{Method:} We extract a corpus of 332 refactoring commits applied and documented by developers during their daily changes from 128 open-source Java projects. In particular, we extract 32 structural metrics 
from which we identify code duplicate removal commits with their corresponding refactoring operations, as perceived by software engineers. Thereafter, we empirically analyze the impact of these refactoring operations on a set of common state-of-the-art design quality metrics. 

\noindent\textbf{Results:} The statistical analysis of the results obtained shows that (i) some state-of-the-art metrics are capable of capturing the developer's intention of removing code duplication; and
(ii) some metrics are being more emphasized than others. We confirm that various structural metrics can effectively represent code duplication, leading to different impacts on software quality. Some metrics contribute to improvements, while others may lead to degradation. 

\noindent\textbf{Conclusion:} Most of the mapped metrics associated with the main quality attributes successfully capture developers' intentions for removing code duplicates, as is evident from the commit messages. However, certain metrics do not fully capture these intentions.

\end{abstract}
\begin{keyword}
Refactoring, Quality, Code Duplicates, Metrics
\end{keyword}

\end{frontmatter}

%\linenumbers



% humans are sensitive to the way information is presented.

% introduce framing as the way we address framing. say something about political views and how information is represented.

% in this paper we explore if models show similar sensitivity.

% why is it important/interesting.



% thought - it would be interesting to test it on real world data, but it would be hard to test humans because they come already biased about real world stuff, so we tested artificial.


% LLMs have recently been shown to mimic cognitive biases, typically associated with human behavior~\citep{ malberg2024comprehensive, itzhak-etal-2024-instructed}. This resemblance has significant implications for how we perceive these models and what we can expect from them in real-world interactions and decisionmaking~\citep{eigner2024determinants, echterhoff-etal-2024-cognitive}.

The \textit{framing effect} is a well-known cognitive phenomenon, where different presentations of the same underlying facts affect human perception towards them~\citep{tversky1981framing}.
For example, presenting an economic policy as only creating 50,000 new jobs, versus also reporting that it would cost 2B USD, can dramatically shift public opinion~\cite{sniderman2004structure}. 
%%%%%%%% 图1:  %%%%%%%%%%%%%%%%
\begin{figure}[t]
    \centering
    \includegraphics[width=\columnwidth]{Figs/01.pdf}
    \caption{Performance comparison (Top-1 Acc (\%)) under various open-vocabulary evaluation settings where the video learners except for CLIP are tuned on Kinetics-400~\cite{k400} with frozen text encoders. The satisfying in-context generalizability on UCF101~\cite{UCF101} (a) can be severely affected by static bias when evaluating on out-of-context SCUBA-UCF101~\cite{li2023mitigating} (b) by replacing the video background with other images.}
    \label{fig:teaser}
\end{figure}


Previous research has shown that LLMs exhibit various cognitive biases, including the framing effect~\cite{lore2024strategic,shaikh2024cbeval,malberg2024comprehensive,echterhoff-etal-2024-cognitive}. However, these either rely on synthetic datasets or evaluate LLMs on different data from what humans were tested on. In addition, comparisons between models and humans typically treat human performance as a baseline rather than comparing patterns in human behavior. 
% \gabis{looks good! what do we mean by ``most studies'' or ``rarely'' can we remove those? or we want to say that we don't know of previous work doing both at the same time?}\gili{yeah the main point is that some work has done each separated, but not all of it together. how about now?}

In this work, we evaluate LLMs on real-world data. Rather than measuring model performance in terms of accuracy, we analyze how closely their responses align with human annotations. Furthermore, while previous studies have examined the effect of framing on decision making, we extend this analysis to sentiment analysis, as sentiment perception plays a key explanatory role in decision-making \cite{lerner2015emotion}. 
%Based on this, we argue that examining sentiment shifts in response to reframing can provide deeper insights into the framing effect. \gabis{I don't understand this last claim. Maybe remove and just say we extend to sentiment analysis?}

% Understanding how LLMs respond to framing is crucial, as they are increasingly integrated into real-world applications~\citep{gan2024application, hurlin2024fairness}.
% In some applications, e.g., in virtual companions, framing can be harnessed to produce human-like behavior leading to better engagement.
% In contrast, in other applications, such as financial or legal advice, mitigating the effect of framing can lead to less biased decisions.
% In both cases, a better understanding of the framing effect on LLMs can help develop strategies to mitigate its negative impacts,
% while utilizing its positive aspects. \gabis{$\leftarrow$ reading this again, maybe this isn't the right place for this paragraph. Consider putting in the conclusion? I think that after we said that people have worked on it, we don't necessarily need this here and will shorten the long intro}


% If framing can influence their outputs, this could have significant societal effects,
% from spreading biases in automated decision-making~\citep{ghasemaghaei2024understanding} to reducing public trust in AI-generated content~\citep{afroogh2024trust}. 
% However, framing is not inherently negative -- understanding how it affects LLM outputs can offer valuable insights into both human and machine cognition.
% By systematically investigating the framing effect,


%It is therefore crucial to systematically investigate the framing effect, to better understand and mitigate its impact. \gabis{This paragraph is important - I think that right now it's saying that we don't want models to be influenced by framing (since we want to mitigate its impact, right?) When we talked I think we had a more nuanced position?}




To better understand the framing effect in LLMs in comparison to human behavior,
we introduce the \name{} dataset (Section~\ref{sec:data}), comprising 1,000 statements, constructed through a three-step process, as shown in Figure~\ref{fig:fig1}.
First, we collect a set of real-world statements that express a clear negative or positive sentiment (e.g., ``I won the highest prize'').
%as exemplified in Figure~\ref{fig:fig1} -- ``I won the highest prize'' positive base statement. (2) next,
Second, we \emph{reframe} the text by adding a prefix or suffix with an opposite sentiment (e.g., ``I won the highest prize, \emph{although I lost all my friends on the way}'').
Finally, we collect human annotations by asking different participants
if they consider the reframed statement to be overall positive or negative.
% \gabist{This allows us to quantify the extent of \textit{sentiment shifts}, which is defined as labeling the sentiment aligning with the opposite framing, rather then the base sentiment -- e.g., voting ``negative'' for the statement ``I won the highest prize, although I lost all my friends on the way'', as it aligns with the opposite framing sentiment.}
We choose to annotate Amazon reviews, where sentiment is more robust, compared to e.g., the news domain which introduces confounding variables such as prior political leaning~\cite{druckman2004political}.


%While the implications of framing on sensitive and controversial topics like politics or economics are highly relevant to real-world applications, testing these subjects in a controlled setting is challenging. Such topics can introduce confounding variables, as annotators might rely on their personal beliefs or emotions rather than focusing solely on the framing, particularly when the content is emotionally charged~\cite{druckman2004political}. To balance real-world relevance with experimental reliability, we chose to focus on statements derived from Amazon reviews. These are naturally occurring, sentiment-rich texts that are less likely to trigger strong preexisting biases or emotional reactions. For instance, a review like ``The book was engaging'' can be framed negatively without invoking specific cultural or political associations. 

 In Section~\ref{sec:results}, we evaluate eight state-of-the-art LLMs
 % including \gpt{}~\cite{openai2024gpt4osystemcard}, \llama{}~\cite{dubey2024llama}, \mistral{}~\cite{jiang2023mistral}, \mixtral{}~\cite{mistral2023mixtral}, and \gemma{}~\cite{team2024gemma}, 
on the \name{} dataset and compare them against human annotations. We find  that LLMs are influenced by framing, somewhat similar to human behavior. All models show a \emph{strong} correlation ($r>0.57$) with human behavior.
%All models show a correlation with human responses of more than $0.55$ in Pearson's $r$ \gabis{@Gili check how people report this?}.
Moreover, we find that both humans and LLMs are more influenced by positive reframing rather than negative reframing. We also find that larger models tend to be more correlated with human behavior. Interestingly, \gpt{} shows the lowest correlation with human behavior. This raises questions about how architectural or training differences might influence susceptibility to framing. 
%\gabis{this last finding about \gpt{} stands in opposition to the start of the statement, right? Even though it's probably one of the largest models, it doesn't correlate with humans? If so, better to state this explicitly}

This work contributes to understanding the parallels between LLM and human cognition, offering insights into how cognitive mechanisms such as the framing effect emerge in LLMs.\footnote{\name{} data available at \url{https://huggingface.co/datasets/gililior/WildFrame}\\Code: ~\url{https://github.com/SLAB-NLP/WildFrame-Eval}}

%\gabist{It also raises fundamental philosophical and practical questions -- should LLMs aim to emulate human-like behavior, even when such behavior is susceptible to harmful cognitive biases? or should they strive to deviate from human tendencies to avoid reproducing these pitfalls?}\gabis{$\leftarrow$ also following Itay's comment, maybe this is better in the dicsussion, since we don't address these questions in the paper.} %\gabis{This last statement brings the nuance back, so I think it contradicts the previous parapgraph where we talked about ``mitigating'' the effect of framing. Also, I think it would be nice to discuss this a bit more in depth, maybe in the discussion section.}






\section{Related Work}
%%%%%%%%%%%%%%
% Know-Item Retrieval and Query Simulation
%%%%%%%%%%%%%%
\subsection{Query Simulation and Know-Item Retrieval}

Query simulation methods have been used for various purposes, including document expansion \cite{nogueira2019docT5query} and synthetic test collection generation \cite{Rahmani24synthetic}. In the context of known-item retrieval, these methods have been explored to improve retrieval strategies \cite{OgilvieCallan03combining} and evaluation frameworks \cite{Azzopardi06testbeds, hagen2015corpus}.



%% Query Simulation
\textit{Simulating} the known-item queries has long been an active research area \cite{balog2006overviewWebclef, Azzopardi07SimulatedQueries, Kim09desktop, Elsweiler2011Seeding}.
Early work \cite{Azzopardi07SimulatedQueries} generated synthetic queries using term-based likelihood models, selecting query terms based on their likelihood within a randomly chosen document. Later studies adapted this approach for desktop search \cite{Kim09desktop} and email re-finding \cite{Elsweiler2011Seeding}, demonstrating its effectiveness for simulated evaluations of know-item retrieval models.
%
The \textit{validation} of these query simulators has also been a key focus.
System ranking correlation \cite{balog2006overviewWebclef}, retrieval score distribution comparisons \cite{Azzopardi07SimulatedQueries}, and synthetic versus human query resemblance \cite{Kim09desktop} have been used to assess their reliability.


While valuable, known-item search queries differ significantly from TOT queries, which are longer and more complex. Despite progress in simulating known-item queries, TOT retrieval remains unexplored. This paper bridges that gap by introducing novel TOT query elicitation methods and adapting established validation techniques \cite{zeigler2000theory} to ensure alignment with real-world queries, enabling scalable and accurate simulated evaluations.






%%%%%%%%%%%%%%
% TOT Datasets
%%%%%%%%%%%%%%
\subsection{TOT Datasets}
Several datasets have been developed to support research on TOT retrieval, primarily collected from online CQA platforms and focused on specific domains. MS-TOT \cite{arguello-movie-identification} was constructed from the \textit{IRememberThisMovie} website and human-annotated with tags in the Movie domain. It also includes qualitative coding of TOT queries and demonstrates significant room for improvement in current retrieval technologies for such information needs. Similarly, \citet{gameTOT} collected TOT queries from Reddit's \textit{/r/tipofmyjoystick} subreddit in the Game domain, providing coded tag information. Other datasets include Reddit-TOMT \cite{Bhargav-2022-wsdm}, focused on movies and books from Reddit's \textit{/r/tipofmytongue} subreddit; TOT-Music \cite{Bhargav23MusicTOT}, targeting the Music domain from the same subreddit; and Whatsthatbook \cite{lin-etal-2023-whatsthatbook}, sourced from \textit{GoodReads}, focused on the Book domain.



In response to the domain specificity of these datasets, recent efforts have aimed to expand TOT datasets across multiple areas. \citet{Meier21-complex-reddit} expanded to general casual leisure domains using data from six Reddit subreddits, including games, books, and music, although other identified domains, such as videos and people, remain underrepresented. Similarly, TOMT-KIS \cite{frobe2023-performance-pred} extended the collection from \textit{/r/tipofmytongue} by adapting \citet{Bhargav-2022-wsdm}'s approach with fewer filtering restrictions, resulting in 1.28 million TOT queries. However, only 47\% of these queries have identified answers, and the dataset continues to exhibit severe domain skewness toward a few topics. 


In this work, we develop and validate TOT query elicitation methods using the Movie domain for robust evaluation, then expand to Landmark and Person to assess applicability across underrepresented domains.



\section{Study Design}
\label{Section:methodology}


\begin{figure*}[t]
\centering 
\includegraphics[width=1.0\textwidth]{Images/Approach-v2.pdf}
\caption{\textcolor{black}{Overview of the empirical study design, highlighting the 3 main phases: Dataset Extraction, Selection of Quality Attributes and Software Metrics, and Data Analysis.}}
\label{fig:approach_overview}
\end{figure*}

Our primary objective is to explore the alignment between developers' perceptions of code duplicate removal (as anticipated by developers) and the actual improvement in software quality (as evaluated by quality metrics). Specifically, our aim is to address the following research questions.
\begin{boxK}
\textbf{RQ$_1$}: \textcolor{black}{What is the quantitative code quality assessment of code duplications that have been intentionally removed by developers?}

\textbf{RQ$_2$}: \textcolor{black}{What are the refactoring operations associated with code duplicate removal?}
\end{boxK}

To address our research questions, we conducted a three-phase empirical study. \textcolor{black}{An overview of the experiment methodology is depicted in Figure \ref{fig:approach_overview}. The initial phase involves extracting a substantial number of open-source Java projects along with their instances of refactoring throughout their development history, specifically focusing on commit-level code changes for each project under consideration. In the second phase, we select software quality metrics to compare their values before and after the identified refactoring commits. Subsequently, the third phase involves analyzing commit messages to identify refactoring commits where developers document their perception of code duplicate removal. In the next subsection, we discuss each phase in detail.}

\subsection{Extracted Dataset}

Our study uses the SmartSHARK MongoDB Release 2.2 dataset  \citep{trautsch2021msr}. This dataset contains a wide range of information for 128 open-source Java projects, such as commit history, issues, refactorings, code metrics, mailing lists, and continuous integration data. All Java projects are part of the Apache ecosystem and utilize GitHub as their version control repository and JIRA for issue tracking. SmartSHARK utilizes \texttt{RefDiff} \citep{silva2017refdiff} and \texttt{RefactoringMiner} \citep{tsantalis2018accurate} to mine refactoring operations. \textcolor{black}{This study is motivated to investigate code duplication-aware refactoring practices in Apache projects. A recent study \citep{xiao2024empirical} highlights the Apache Software Foundation as a prominent example of successful open-source software communities \citep{mockus2002two,mockus2000case,crowston2006assessing}. Both practitioners and researchers have been extracting valuable insights and gaining experience from Apache's effective practices to drive the open-source movement forward \citep{rigby2008open,duenas2007apache,weiss2006evolution}. Furthermore, Apache is a collaborative environment where engineers from major corporations such as IBM, Google, Yahoo, Sun, and Oracle volunteer to develop open-source software infrastructure \citep{severance2012apache}.} 

%  this study is motivated to investigate the usage of mocking frameworks in Apache projects since Apache Software Foundation has been widely recognized
% and researched as a distinguished example of successful open-source software communities (Mockus et al. 2002, 2000; Crowston and Howison 2006). Practitioners and researchers
% have been gaining experience and insights from the successful practices of Apache projects
% to lead the open-source movement (Rigby et al. 2008; Duenas et al. 2007; Weiss et al. 2006).
% In addition, Apache provides a meeting point where engineers from large companies like
% IBM, Google, Yahoo, Sun, and Oracle work as volunteers to build open-source software
% infrastructure (Severance 2012).

To extract the relevant information, we built custom scripts to extract data pertinent to our study (\ie commits, metrics, refactorings) from
the source dataset into an SQLite database for analysis. First, we extract all commits with the keyword `duplicat*' and `code clone', discussed later in Section \ref{dataanslysis}. Next,
we extract all refactoring operations. However, due to the use of two refactoring mining tools, there are duplicate operations in the source data. \textcolor{black}{Hence, our next step is to remove all duplicates by comparing the refactoring descriptions. After that, we select all
commits associated with a refactoring operation. Using both refactoring mining tools allowed us to mitigate the limitations of relying on a single tool and ensured a more diverse and thorough dataset.} Table \ref{Table:DATA_Overview} summarizes the extracted data.




\subsection{Quality Attributes \& Quality Metrics Selection}
To setup a comprehensive set of quality attributes for evaluation in our study, we initially analyze existing studies to identify commonly recognized software quality attributes \citep{chidamber1994metrics,lorenz1994object,mccabe1976complexity, henry1981software, nejmeh1988npath, Destefanis:2014:SMA:2813544.2813555}. Next, we assess whether the metrics evaluate various object-oriented design aspects, mapping each internal quality attribute to the corresponding structural metric(s). %For example, the Response For Class (RFC) metric is typically used to measure Coupling and Complexity quality attributes. 
  Additionally, we extract associations between metrics (such as the CK suite \citep{chidamber1994metrics}, McCabe \citep{mccabe1976complexity}, and Lorenz and Kidd's book \citep{lorenz1994object}) and internal quality attributes from the literature review. \textcolor{black}{Tables \ref{Table:Quality Metrics in Related Work} and \ref{Table:Quality Metrics in Related Work-v2} summarize the extracted metrics.}

Subsequently, we examined the extracted metrics to determine whether these metrics exist in the SmartSHARK dataset, calculated using OpenStaticAnalyzer\footnote{https://github.com/sed-inf-u-szeged/OpenStaticAnalyzer}. The extraction process results in 32 distinct structural metrics as shown in Table \ref{Table:Quality Metrics Used in This Study.}. The list of metrics is (1) well-known and defined in the literature, and (2) can assess different code-level elements, \ie method, class, package. % and (3) can be calculated by existing static analysis tools. 

%We also adopted NOA andNOO since they measure quality aspects of a class that are not takeninto account by the CK metrics

\subsection{Data Analysis}
\label{dataanslysis}

\begin{table}[h!]
\begin{center}
\caption{\textcolor{black}{Summary of the extracted data.}}
\label{Table:DATA_Overview}
\begin{adjustbox}{width=1.0\textwidth,center}
%\begin{adjustbox}{width=\textheight,totalheight=\textwidth,keepaspectratio}
\begin{tabular}{lllll}\hline
\toprule
\bfseries Item & \bfseries Count \\
\midrule
Total projects & 128 \\
%Total commits & \\
%Total projects with commits containing keyword `\textit{duplicat*}' & 73  \\
%Refactoring commits & 2169916  \\
\cellcolor{gray!30}Refactoring commits with keyword `\textit{duplicat*}' & \cellcolor{gray!30}2,169,916  \\
False positive commits & 22 \\
\cellcolor{gray!30}Refactoring commits after removing false positives & \cellcolor{gray!30}2,164,797 \\
(Distinct) Refactoring commits with keyword `\textit{duplicat*}' & 332  \\
%Refactoring commits w/ class code entity & 88,642 \\
\bottomrule
\end{tabular}
\end{adjustbox}
\end{center}
\end{table}

After extracting all refactoring commits, we want to keep only commits where refactoring is documented. We continue to filter them, using the content of their messages at this stage. We use a keyword-based search to find commits whose messages contain the keywords (\ie `duplicat*' or `code clone*'). We selected these keywords because these keywords are naturally used by developers to articulate their intent regarding code duplication \citep{alomar2019can,alomar2021we}. However, it is worth mentioning that we did not find any commits with the keyword `code clone'. Therefore, all the commits in our dataset solely include the keyword `duplicat'.

This keyword-based filtering selected 2,169,916 commit messages. %We notice that the ratio of these commits is very small compared to the total number of refactoring commits, \ie \hl{\#}. However, these observations are consistent with previous studies \citep{murphy2012we,szoke2014bulk} as developers typically do not provide details when documenting their refactorings. 
To ensure that these commits reported developers' intention to remove code duplication, we manually inspected and read through 322 distinct refactoring commits to remove false positives. An example of a discarded commit is: \say{\textit{DeferredDuplicates.java}}. We discarded this commit because the keyword `duplicat' is actually part of the identifier name of the class. In the case of doubts about including a certain commit, it was excluded. This step resulted in considering 322 commits. Our goal is to have a \textit{gold set} of commits in which the developers explicitly reported the removal of duplicate code. This \textit{gold set} will serve to check later if there is an alignment between the real quality metrics affected in the source code, and the code duplicate removal as documented by developers. 
 An example of commit messages belonging to the \textit{gold set}, is showcased in the following commit message  \say{\textit{Refactored JavaClass and FieldOrMethod to avoid a code duplication}}. %This commit message shown in Figure \ref{fig:commit message} suggests that the developers performed a code refactoring to eliminate code duplication. Figures \ref{fig:duplicate 1} and \ref{fig:duplicate 2} depict the two instances of code duplication present in the codebase. The developer employed the `Extract Method' refactoring technique (see Figure \ref{fig:method extraction}), which involves isolating a code fragment and relocating it to form a new method, subsequently replacing all occurrences of that fragment with a call to the newly created method.

\begin{table}
%\begin{minipage}{\columnwidth} 
  \centering
	 \caption{\textcolor{black}{Structural code quality metrics used in this study.}}
	 \label{Table:Quality Metrics Used in This Study.}
  \begin{threeparttable}
%\begin{sideways}
\begin{adjustbox}{width=1.0\textwidth,center}
%\begin{adjustbox}{width=\textheight,totalheight=\textwidth,keepaspectratio}
\begin{tabular}{llll}\hline
\toprule
\bfseries Quality Attribute & \bfseries Study &   \bfseries Metric & \bfseries Description \\
\midrule
%\multicolumn{2}{l}{\textbf{\textit{Internal Quality Attribute }}}\\
%\midrule
Cohesion & \cite{pantiuchina2018improving,chavez2017does} &↓ LCOM& Lack of Cohesion of Methods   \\ 
Coupling &  \cellcolor{gray!30}\cite{chavez2017does,pantiuchina2018improving} & \cellcolor{gray!30}↓ CBO&\cellcolor{gray!30}Coupling Between Objects    \\
         & \cite{pantiuchina2018improving} & ↓ RFC & Response For Class   \\
         & \cellcolor{gray!30}\cite{islam2018characteristics} & \cellcolor{gray!30}↓ NII &\cellcolor{gray!30}Number of Incoming Invocations  \\
         & \cite{islam2018characteristics} & ↓ NOI &Number of Outgoing Invocations \\
Complexity & \cellcolor{gray!30}\cite{chavez2017does} & \cellcolor{gray!30}↓ CC & \cellcolor{gray!30}Cyclomatic Complexity 
           \\
           & \cite{chavez2017does,pantiuchina2018improving,singh2012evaluation} & ↓ WMC& Weighted Method Count  \\
           & \cellcolor{gray!30}\cite{islam2018characteristics} & \cellcolor{gray!30}↓ NL & \cellcolor{gray!30}Nesting Level  \\
           & \cite{islam2018characteristics} & ↓ NLE &Nesting Level Else-if  \\
           & \cellcolor{gray!30}\cite{islam2018characteristics} & \cellcolor{gray!30}↓ HCPL & \cellcolor{gray!30}Hal. Calculated Program Length \\
            & \cite{islam2018characteristics} & ↓ HDIF & Hal. Difficulty  \\
             & \cellcolor{gray!30}\cite{islam2018characteristics} & \cellcolor{gray!30}↓ HEFF & \cellcolor{gray!30}Hal. Effort  \\
              & \cite{islam2018characteristics} & ↓ HNDB & Hal. Number of Delivered Bugs   \\
               & \cellcolor{gray!30}\cite{islam2018characteristics} &\cellcolor{gray!30}↓ HPL & \cellcolor{gray!30}Hal. Program Length   \\
                & \cite{islam2018characteristics} & ↓ HPV & Hal. Program Vocabulary  \\
                 & \cellcolor{gray!30}\cite{islam2018characteristics} & \cellcolor{gray!30}↓ HTRP &\cellcolor{gray!30}Hal. Time Required to Program  \\
                  & \cite{islam2018characteristics} & ↓ HVOL &Hal. Volume \\
                   & \cellcolor{gray!30}\cite{islam2018characteristics} &\cellcolor{gray!30}↑ MIMS & \cellcolor{gray!30}Maintainability Index (MS) \\
                    & \cite{islam2018characteristics} &↑ MI& Maintainability Index (OV) \\
                     & \cellcolor{gray!30}\cite{islam2018characteristics} &\cellcolor{gray!30}↑ MISEI& \cellcolor{gray!30}Maintainability Index (SEIV) \\
                      & \cite{islam2018characteristics} &↑ MISM&  Maintainability Index (SV)\\
Inheritance & \cellcolor{gray!30}\cite{chavez2017does,singh2012evaluation} & \cellcolor{gray!30}↓ DIT &\cellcolor{gray!30}Depth of Inheritance Tree 
  \\
   & \cite{chavez2017does,singh2012evaluation} & ↓ NOC &Number of Children   \\
 & \cellcolor{gray!30}\cite{bavota2015experimental} & \cellcolor{gray!30}↓ NOA & \cellcolor{gray!30}Number of Operations Added by Subclass   \\
        
Design Size & \cite{chavez2017does} & ↓ LOC & Lines of Code \\
& \cellcolor{gray!30}\cite{islam2018characteristics} &\cellcolor{gray!30}↓ TLOC &\cellcolor{gray!30}Total Lines of Code   \\
& \cite{chavez2017does} & ↓ LLOC &Logical Lines of Code   \\
& \cellcolor{gray!30}\cite{islam2018characteristics} & \cellcolor{gray!30}↓ TLLOC&\cellcolor{gray!30}Total Logical Lines of Code  \\
            & \cite{chavez2017does} & ↑  CLOC&Lines with Comments  \\
            & \cellcolor{gray!30}\cite{stroggylos2007refactoring} & \cellcolor{gray!30}↓ NPM &\cellcolor{gray!30}Number of Public Methods  \\
           % & & Total Number of Methods (TNM) \\
            &\cite{islam2018characteristics} &↓ NOS& Number of Statements  \\
            &\cellcolor{gray!30}\cite{islam2018characteristics} & \cellcolor{gray!30}↓ TNOS&\cellcolor{gray!30}Total Number of Statements  \\
           % & & Total Number of Accessor Methods (TNG) \\
           % & & Total Number of Attributes (TNA) \\
\bottomrule
\multicolumn{4}{l}{\tiny 
↑ by a metric indicates the higher the better for that metric; 
↓ by a metric indicates the lower the better for that metric.}
%Hal.=Halstead; MS= Microsoft version; OV=Original version; SEIV=SEI version; SV=SourceMeter version.}
% \begin{tablenotes}
 %    The first note
 %   \end{tablenotes}
\end{tabular}
\end{adjustbox}
%\end{sideways}
%\footnote{xxx}
%\end{minipage}
\end{threeparttable}
\end{table}
%\footnotesize{$^a$ The smallest spatial unit is county, $^b$ more details in appendix A}

We perform a qualitative analysis of intriguing instances of alignment or disparity between the removal of code duplication as perceived by developers and its evaluation through quality metrics. To do this, the author manually inspects the commits, which involves analyzing the diff code alongside the metrics profile of the affected code elements before and after the commit.

\begin{comment}

\begin{itemize}
    \item commit id for those commits that contain either “\%duplicat\%” or “\%code clone\%” in the commit message
    \item Refactoring type is `extract method'
    \item code entity is method
\end{itemize}



\begin{figure*}[htbp]
	\centering
    \includegraphics[scale = 0.35]{Images/CommitMessage.png}   
    \caption{Commit message indicating the removal of code duplication \citep{commons-bcel}.}
    \label{fig:commit message}

\vspace{0.70cm}

	\centering
    \includegraphics[width=0.5\textwidth]{Images/DuplicateMethod1.png}   
    \caption{Code snippet depicting the first instance of code duplication before refactoring \citep{commons-bcel}.}
    \label{fig:duplicate 1}

\vspace{0.70cm}

\centering 
\includegraphics[scale = 0.2]{Images/DuplicateMethod2.png}
\caption{Code snippet depicting the second instance of code duplication before refactoring \citep{commons-bcel}.}
\label{fig:duplicate 2}

\vspace{0.70cm}

\centering 
%\scalebox{0.8}{\includegraphics[width=\columnwidth]{Images/DiffPreconditionChecking.PNG}}
\includegraphics[scale = 0.2]{Images/ExtractedMethod.png}
\caption{Code snippet depicting the removal of the duplicated code through the `Extract Method' refactoring \citep{commons-bcel}.}
\label{fig:}
\label{fig:method extraction}
\vspace{0.70cm}


\end{figure*}
\end{comment}

The resulting commits correspond to our data points, each data point is represented by a set of \textit{pre-refactoring} and \textit{post-refactoring} Java files. These data points will be used in the experiments, to measure the effect of changes in terms of structural metrics, with respect to the quality attribute, announced in the commit message.


\section{Results \& Discussion}
\label{Section:Result}
\subsection{\textcolor{black}{What is the quantitative code quality assessment of code duplications that have been intentionally removed by developers?}}
For each refactoring commit in which developers document the removal of duplicate code, we extract its associated metric values  (see Table~\ref{Table:Quality Metrics Used in This Study.}) before and after the commit. 
 In other words, for commit messages related to the removal of code duplicates, we examine 32 corresponding metric values before and after the selected refactoring commit. As we evaluate metric values both pre- and post-refactoring, we want to distinguish, for each metric, whether there is a variation between its pair of values, whether this variation signifies an improvement, and whether the variation is statistically significant. Therefore, we use the Wilcoxon test \citep{wilcoxon1945individual}, a non-parametric test, to compare the group of metric values before and after the commit since these groups depend on each other. The null hypothesis is defined by no variation in the metric values of pre- and post-refactored code elements. Thus, the alternative hypothesis indicates a variation in the metric values. In each case, a decrease in the metric value is considered desirable (\ie an improvement), except for complexity metrics related to the maintainability index and CLOC (see Table~\ref{Table:Quality Metrics Used in This Study.}), where higher values are desirable. Furthermore, the variation between the values of both sets is considered significant if its associated \textit{p}-value is less than 0.05. Furthermore, we used the Cliff's delta ($\delta$) effect size to estimate the magnitude of the differences. Regarding its interpretation, we follow the guidelines reported by Grissom \etal \citep{trove.nla.gov.au/work/16432558}:

 \begin{itemize}
\item Negligible for $\mid \delta \mid< 0.147$
\item Small for $0.147 \leq \mid \delta \mid < 0.33$
\item Medium for $0.33 \leq \mid \delta \mid < 0.474$
\item Large for $\mid \delta \mid \geq 0.474$
\end{itemize}

%In the following, we report the results of our research questions. 


To answer our main research question, we provide a detailed analysis of each of the five quality attributes reported in Table \ref{Table:Quality Metrics Used in This Study.} and qualitatively analyze the cases with positive and negative impacts. Table~\ref{Table:Metrics Suites and Metrics Tools Summary} shows the overall impact of refactorings on quality. The boxplots in \textcolor{black}{Figures \ref{Chart:Boxplots_cohesion}, \ref{Chart:Boxplots_coupling}, \ref{Chart:Boxplots_complexity}, \ref{Chart:Boxplots_inheritance}, and \ref{Chart:Boxplots_design size}} show the distribution of each metric before and after each of the examined commits.


\begin{table}
  \centering
\caption{\textcolor{black}{Effect of duplicate code removal on structural metrics. (+ve) indicates positive impact; (-ve) indicates negative impact; (-) indicates metric remains unaffected, \textbf{bold} indicates statistical significance; \textit{italic} indicates improvement.}}
\label{Table:Metrics Suites and Metrics Tools Summary}
%\begin{sideways}
\begin{adjustbox}{width=1.0\textwidth,center}
%\begin{adjustbox}{width=\textheight,totalheight=\textwidth,keepaspectratio}
\begin{tabular}{lllll}\hline
\toprule
\bfseries Quality Attribute & \bfseries Metric & \bfseries Impact & \bfseries \textit{p}-value & \bfseries Cliff's delta ($\delta$) \\
\midrule
%\multicolumn{2}{l}{\textbf{\textit{Internal Quality Attribute }}}\\
%\midrule
Cohesion &  LCOM5  & +ve & \textit{\textbf{7.72e-41}} & 0.54 (Large)
%(Small)   
\\ 
Coupling &  \cellcolor{gray!30}CBO  & \cellcolor{gray!30}+ve & \cellcolor{gray!30}\textit{\textbf{9.49e-76}} & \cellcolor{gray!30}0.6 (Large)
%(Small) 
\\
         &  RFC & +ve & \textit{\textbf{1.25e-68}}  & 0.55 (Large)

\\
         &  \cellcolor{gray!30}NII & \cellcolor{gray!30}-ve &  \cellcolor{gray!30}\textbf{0} & \cellcolor{gray!30}0.47 (Large)

\\
         &  NOI & +ve & \textit{\textbf{0}}  & 0.26 (Small)

\\
Complexity &  \cellcolor{gray!30}CC & \cellcolor{gray!30}- & \cellcolor{gray!30}\textbf{0} & \cellcolor{gray!30}0.14 (Small)

\\
           &  WMC & +ve & \textit{\textbf{6.51e-70}} & 0.5 (Large)

\\
         &  \cellcolor{gray!30}NL &  \cellcolor{gray!30}- &  \cellcolor{gray!30}\textbf{3.92e-05}&  \cellcolor{gray!30}0.03 (Negligible)

\\
         &  NLE &  - & \textbf{0.004}  & 0.02 (Negligible)

\\
         &  \cellcolor{gray!30}HCPL & \cellcolor{gray!30}+ve &  \cellcolor{gray!30}\textit{\textbf{0}} &  \cellcolor{gray!30}0.14 (Negligible)

\\
         &  HDIF & +ve & \textit{\textbf{0}}  &  0.08 (Negligible)

\\
         &  \cellcolor{gray!30}HEFF & \cellcolor{gray!30}+ve &  \cellcolor{gray!30}\textit{\textbf{2.45e-271}} &  \cellcolor{gray!30}0.13 (Negligible)

\\
         &  HNDB & +ve &  \textit{\textbf{1.07e-266}} & 0.13  (Negligible)

\\
         &  \cellcolor{gray!30}HPL & \cellcolor{gray!30}+ve & \cellcolor{gray!30}\textit{\textbf{0}} & \cellcolor{gray!30}0.13  (Negligible)

\\
         &  HPV & +ve & \textit{\textbf{0}}  & 0.14  (Negligible)

\\
         &  \cellcolor{gray!30}HTRP & \cellcolor{gray!30}+ve &  \cellcolor{gray!30}\textit{\textbf{2.48e-271}} & \cellcolor{gray!30}0.13  (Negligible)

\\
         &  HVOL & +ve & \textit{\textbf{0}}  &  0.13  (Negligible)

\\
         &  \cellcolor{gray!30}MIMS & \cellcolor{gray!30}+ve & \cellcolor{gray!30}\textit{\textbf{7.23e-227}}  &  \cellcolor{gray!30}0.13  (Negligible)

\\
         &   MI & +ve &  \textit{\textbf{7.22e-227}} &  0.13  (Negligible)

\\
         &   \cellcolor{gray!30}MISEI & \cellcolor{gray!30}+ve & \cellcolor{gray!30}\textit{\textbf{0}} & \cellcolor{gray!30}0.16  (Small)

\\
         &   MISM &  +ve&  \textit{\textbf{0}}  & 0.16  (Small)

\\
Inheritance &   \cellcolor{gray!30}DIT & \cellcolor{gray!30}-ve & \cellcolor{gray!30}\textbf{3.81e-199} & \cellcolor{gray!30}0.6 (Large) 
 
\\
            &  NOC & +ve & \textbf{\textit{3.61e-130}} & 0.83 (Large)  

\\
            &  \cellcolor{gray!30}NOA & \cellcolor{gray!30}-ve & \cellcolor{gray!30}\textbf{2.37e-196}  & \cellcolor{gray!30}0.63 (Large)
 
\\ 
Design Size &  LOC & +ve & \textbf{\textit{0}} &   0.14 (Small)

\\
         &  \cellcolor{gray!30}TLOC & \cellcolor{gray!30}+ve &   \cellcolor{gray!30}\textit{\textbf{0}} &  \cellcolor{gray!30}0.16 (Small)

\\
            &  LLOC &  +ve &  \textbf{\textit{0}} &   0.13 (Negligible)

\\
         &  \cellcolor{gray!30}TLLOC & \cellcolor{gray!30}+ve & \cellcolor{gray!30}\textit{\textbf{0}}  &   \cellcolor{gray!30}0.15 (Small)

\\
            &   CLOC & - &  \textbf{1.43e-05} &   0.02 (Negligible)

\\
            &  \cellcolor{gray!30}NPM & \cellcolor{gray!30}- & \cellcolor{gray!30}\textbf{4.42e-193} & \cellcolor{gray!30}0.5 (Large)

\\
         &  NOS &  +ve&  \textit{\textbf{0}} &  0.07 (Negligible)

\\
         &  \cellcolor{gray!30}TNOS & \cellcolor{gray!30}+ve & \cellcolor{gray!30}\textit{\textbf{0}}  &  \cellcolor{gray!30}0.08 (Negligible)

\\
\bottomrule
\end{tabular}
\end{adjustbox}
%\end{sideways}
\end{table}
\begin{comment}
    


%\begin{sidewaystable}
\begin{table*}

\caption{The impact of duplicate code removal on quality metrics.}
\label{composites}


\fontsize{10}{14}\selectfont
	\tabcolsep=0.1cm
\resizebox{\textwidth}{!}{
\begin{tabular}{llcccccccccccccccccccccc} \toprule
\multicolumn{1}{c}{}                             & \multicolumn{1}{c}{}                          & \multicolumn{3}{c}{Cohesion}                                                                                                            & \multicolumn{5}{c}{Coupling}                                                                                                          & \multicolumn{4}{c}{Complexity}                                                                                                                                                         & \multicolumn{3}{c}{Inheritance}                                                         & \multicolumn{7}{c}{Design Size}                                                                                                                                                                                                                                                                         \\
\multicolumn{1}{c}{\multirow{-2}{*}{}} & \multicolumn{1}{c}{\multirow{-2}{*}{Measure}} & LCOM5                                        &                                          &                                          & CBO                                         & RFC                                         &     &    &                 & WMC                                         & CC                                         &                                  &                                          & DIT                                         & NOC                 &   NOA            & SLOC                                         & LLOC                                        & CLOC                &      NPM                                  &                                        &                                          &                                         \\ \hline
                                                 & Refactoring Impact                            & 0                                           & \cellcolor[HTML]{0350F8}1                   & \cellcolor[HTML]{0350F8}1                   & 0                                           & \cellcolor[HTML]{0350F8}-12                 & 0        & 0        & 0                   & \cellcolor[HTML]{0C56F7}-5                  & \cellcolor[HTML]{1B61F8}-3                  & \cellcolor[HTML]{1B61F8}-3                  & \cellcolor[HTML]{A3BBED}-1                  & 0                                           & 0                   & 0                   & \cellcolor[HTML]{7DA2F1}-2                  & 0                                           & 0                   & 7                                           & \cellcolor[HTML]{F8ADAD}2                   & \cellcolor[HTML]{0350F8}-20                 & 0                                           \\
                                                 & Behavior                                      & -                                           & \cellcolor[HTML]{0350F8}haut                & \cellcolor[HTML]{0350F8}haut                & -                                           & \cellcolor[HTML]{0350F8}bas                 & -        & -        & -                   & \cellcolor[HTML]{0C56F7}bas                 & \cellcolor[HTML]{1B61F8}bas                 & \cellcolor[HTML]{1B61F8}bas                 & \cellcolor[HTML]{A3BBED}bas                 & -                                           & -                   & -                   & \cellcolor[HTML]{7DA2F1}bas                 & -                                           & -                   & haut                                        & \cellcolor[HTML]{F8ADAD}haut                & \cellcolor[HTML]{0350F8}bas                 & -                                           \\
\multirow{-3}{*}{}                         & P-value ($\delta$)                            & 1 (N)                                       & \cellcolor[HTML]{0350F8}\textless{}0.05 (S) & \cellcolor[HTML]{0350F8}\textless{}0.05 (S) & 1 (N)                                       & \cellcolor[HTML]{0350F8}\textless{}0.05 (M) & 0.07 (N) & 0.07 (N) & \textless{}0.05 (N) & \cellcolor[HTML]{0C56F7}\textless{}0.05(S)  & \cellcolor[HTML]{1B61F8}\textless{}0.05 (S) & \cellcolor[HTML]{1B61F8}\textless{}0.05(S)  & \cellcolor[HTML]{A3BBED}\textless{}0.05(N)  & 0.08 (N)                                    & 0.23 (S)            & 0.19 (N)            & \cellcolor[HTML]{7DA2F1}0.06 (N)            & 0.16 (N)                                    & \textless{}0.05 (N) & 0.09 (S)                                    & \cellcolor[HTML]{F8ADAD}\textless{}0.05 (N) & \cellcolor[HTML]{0350F8}\textless{}0.05 (M) & 0.12 (N)                                    \\
 \bottomrule
\end{tabular}
}
\end{table*}
%\end{sidewaystable}

\end{comment}
%\subsection{Is the developer's perception of code duplicate removal aligned with the quantitative assessment of code quality?}
%%%%%%%%%%%%%%%%%%%%%%%%%%%%%%%%%%%%%%%%%%%%%%%%%%%%%%%%%%%%%%%%%%%%%%%%%%%%%%%%%%%%%%%%%
%\subsubsection{Cohesion}
\noindent\textbf{\textcolor{black}{Cohesion.}}  For commits wherein the messages indicate the removal of code duplication, the boxplot depicted in Figure \ref{BP:chesion-lcom} illustrates the pre- and post-refactoring results of the normalized LCOM. This metric, commonly used in the literature to assess cohesion, is crucial in estimating the strength of cohesion within classes. A lower LCOM metric value generally suggests that classes should be split into one or more classes with better cohesion. Therefore, a low value for this metric signifies strong class cohesiveness. We specifically chose the normalized LCOM metric as it has been widely recognized in the literature  \citep{pantiuchina2018improving,chavez2017does,henderson1995object} as being the alternative to the original LCOM, by addressing its main limitations (artificial outliers, misperception of getters and setters, etc.). As can be seen from the boxplot in Figure \ref{BP:chesion-lcom}, the median drops from 1 to 0. This result indicates that LCOM is improved after code duplicate removal. Furthermore, as shown in Table~\ref{Table:Metrics Suites and Metrics Tools Summary}, LCOM has a positive impact on cohesion quality, as it decreases in the refactored code. This implies that developers did improve the cohesion of their classes. Table~\ref{Table:Metrics Suites and Metrics Tools Summary} shows that the differences in LCOM are statistically significant and the magnitude of the differences is large.

\noindent\textbf{Example (Positive Impact):} To illustrate an improvement in cohesion when the removal of duplicate code was found in the Maven project\footnote{\textcolor{black}{\url{https://github.com/apache/maven-surefire/commit/d5de47a4f790ea2d18edb5e05c1ef2adcd2db8a2}}}, developers applied `Move Class' refactoring to move \texttt{JUnit4TestCheckerTest.MySuite2} to \texttt{JUnit3TestCheckerTest.MySuite2}. This results in its LCOM5 dropping from 2 to 1. The improvement in the LCOM5 metric after the removal of code duplication could be attributed to the simplification of method interactions, the modularization of logic, the enhancement of code clarity, and the abstraction of common functionality.

%https://github.com/apache/maven-surefire/commit/d5de47a4f790ea2d18edb5e05c1ef2adcd2db8a2
\noindent\textbf{Example (Negative Impact):} To illustrate a decrease in cohesion when the removal of duplicated code was found in the Maven project\footnote{\textcolor{black}{\url{https://github.com/apache/maven-surefire/commit/d5de47a4f790ea2d18edb5e05c1ef2adcd2db8a2}}}, developers applied `Move Class' refactoring to move \texttt{JUnit4TestCheckerTest.NestedTC} to \texttt{JUnit3TestChecker\break Test.NestedTC}. This results in its LCOM5 increasing from 0 to 2. The LCOM5 metric might not have improved after removing duplicated code, as removal of duplicated code might not have substantially altered the underlying design and interactions of methods.  


\begin{figure*}
% row-1
\centering
%\begin{subfigure}{3.5cm}
\centering\includegraphics[width=3.5cm]{Images/LCOM5.png}
\caption{Cohesion - LCOM5}
\label{BP:chesion-lcom}
%\end{subfigure}%
\caption{\textcolor{black}{Boxplots of cohesion metric values of pre- and post-refactored files.}}
\label{Chart:Boxplots_cohesion}
\end{figure*}
%https://github.com/apache/maven-surefire/commit/d5de47a4f790ea2d18edb5e05c1ef2adcd2db8a2
% \begin{boxK}
% \textit{\textbf{Summary.} The normalized LCOM metric not only serves as a suitable substitute for the original LCOM but also serves as a representation of the cohesion quality attribute. A positive variation in this metric aligns with the developer's intention to eliminate code duplication.}
% \end{boxK}

%%%%%%%%%%%%%%%%%%%%%%%%%%%%%%%%%%%%%%%%%%%%%%%%%%%%%%%%%%%%%%%%%%%%%%%%%%%%%%%%%%%%%%%%%
%\subsubsection{Coupling}
\noindent\textbf{\textcolor{black}{Coupling.}} For commits with messages indicating the removal of code duplication, the boxplots presented in Figures \ref{BP:coupling-cbo}, \ref{BP:coupling-rfc}, \ref{BP:coupling-nii}, and \ref{BP:coupling-noi} show the pre- and post-refactoring results of four structural metrics, \ie CBO, RFC, NII, and NOI, used in the literature to estimate the coupling. The figures reveal that three of the coupling metrics exhibited an improvement in median values. For instance, CBO, RFC, and NOI medians decreased, respectively, from 6 to 3, from 5 to 1, and from 6 to 3, respectively. CBO counts the number of classes coupled to a particular class through method or attribute calls. Calls are counted in both directions. CBO values have significantly decreased, making it a good coupling representative.  The RFC, which measures the visibility of a class to outsider classes, has been reduced as developers intend to optimize coupling. According to our results, the variations are statistically significant and the magnitude of the differences is large for both metrics.  NOI, which represents the number of outgoing invocations, has also decreased, and Cliff’s delta value
indicates a small effect size. However, NII exhibits the opposite variation, and the effect size is large.

The manual inspection of the refactored code indicates that developers typically decrease coupling by reducing (1) the strength of dependencies that exist between classes, (2) the message flow of the classes, and (3) the number of inputs a method uses plus the number of subprograms that call this method. The code was improved as expected from the developer's intentions in their commit message.

\noindent\textbf{Example (Positive Impact):} One of the examples showing an improvement in coupling was found in the Maven project\footnote{\textcolor{black}{\url{https://github.com/apache/maven-surefire/commit/d5de47a4f790ea2d18edb5e05c1ef2adcd2db8a2}}}. Developers applied `Move Class' refactoring to move \texttt{JUnit4TestCheckerTest.MySuite2} to \texttt{JUnit3Test\break CheckerTest.MySuite2}. This results in its CBO dropping from 1 to 0, and its RFC from 2 to 1.  The improvement in the CBO and RFC metrics after the removal of code duplication can be related to the elimination of external dependencies and the simplification of method interactions. However, its NII and NOI remain unchanged.

%https://github.com/apache/maven-surefire/commit/d5de47a4f790ea2d18edb5e05c1ef2adcd2db8a2
\noindent\textbf{Example (Negative Impact):} One of the examples showing an increase in coupling was found in the Archiva project\footnote{\textcolor{black}{\url{https://github.com/apache/archiva/commit/26e9c3b257bed850d0e2f0bc9dc2d7f11381b789}}}. The developer applied `Extract Superclass' refactoring to extract \texttt{AbstractDiscoverer} from \texttt{AbstractArtifact\break Discoverer}. This results in its CBO increasing from 0 to 1, its RFC from 4 to 6, and its NOI from 0 to 2. However, its NII improves from 3 to 0. The lack of improvement in CBO, RFC, and NOI metrics after the removal of code duplication could be attributed to the specific nature of the duplication and its limited impact on class interactions, method hierarchies, and message handling. 
%which indicates that the number of dependencies does not decrease after performing the refactoring to remove code duplication. 
%https://github.com/apache/archiva/commit/26e9c3b257bed850d0e2f0bc9dc2d7f11381b789
% \begin{boxK}
% \textit{\textbf{Summary.} CBO, RFC, and NOI generally improve as the developer intends to eliminate code duplication, and their variation is significant. NII exhibits opposite variations in
% coupling.}
% \end{boxK}
\begin{figure*}
% row-1
\centering
\begin{subfigure}{3.5cm}
\centering\includegraphics[width=3.5cm]{Images/CBO.png}
\caption{Coupling - CBO}
\label{BP:coupling-cbo}
\end{subfigure}%
\begin{subfigure}{3.5cm}
\centering\includegraphics[width=3.5cm]{Images/RFC.png}
\caption{Coupling - RFC}
\label{BP:coupling-rfc}
\end{subfigure}%\vspace{9pt}
\begin{subfigure}{3.5cm}
\centering\includegraphics[width=3.5cm]{Images/NII.png}
\caption{Coupling - NII}
\label{BP:coupling-nii}
\end{subfigure}%\vspace{9pt}
\begin{subfigure}{3.5cm}
\centering\includegraphics[width=3.5cm]{Images/NOI.png}
\caption{Coupling - NOI}
\label{BP:coupling-noi}
\end{subfigure}%
\caption{\textcolor{black}{Boxplots of coupling metric values of pre- and post-refactored files.}}
\label{Chart:Boxplots_coupling}
\end{figure*}
%%%%%%%%%%%%%%%%%%%%%%%%%%%%%%%%%%%%%%%%%%%%%%%%%%%%%%%%%%%%%%%%%%%%%%%%%%%%%%%%%%%%%%%%%
%RFC exhibits an opposite variation to coupling, but it is not statistically significant.
%\subsubsection{Complexity}
\noindent\textbf{\textcolor{black}{Complexity.}} Regarding complexity metrics, we consider 16 literature metrics, shown in Table \ref{Table:Quality Metrics Used in This Study.}, to investigate the removal of duplicate code as perceived by developers. As seen in the boxplots in Figures \ref{BP:Complexity-cc}, \ref{BP:Complexity-wmc}, \ref{BP:Complexity-nl}, \ref{BP:coupling-nle}, \ref{BP:Complexity-hcpl}, \ref{BP:Complexity-hdif}, \ref{BP:Complexityheff}, \ref{BP:Complexity-hndb}, \ref{BP:Complexity-hpl}, \ref{BP:Complexity-hpv}, \ref{BP:Complexity-htrp}, \ref{BP:Complexity-hvol}, \ref{BP:Complexity-mims}, \ref{BP:Complexity-mi}, \ref{BP:Complexity-misei}, and \ref{BP:Complexity-mism},  we observe that CC, NL, and NLE remain unchanged, whereas the other 13 metrics experienced an improvement in the median values. The refactored duplicate code exhibits higher values for the four maintenance index-related complexity (\ie MIMS, MI, MISEI, and MISM). The higher values are desirable for these metrics, as shown in Table \ref{Table:Quality Metrics Used in This Study.}. Additionally, the duplicate code refactored shows lower values for the other metrics (\ie WMC, HCPL, HDIF, HEEF, HNDB, HPL, HPV, HTRP, and HVOL), where lower values are desirable after the application of refactoring.

%Furthermore, all the variations are statistically significant and the magnitude of the differences is large for both metrics. 


%Our results indicate that the two metrics represent the quality attribute of complexity that improved. 

In particular, through a manual inspection of the collected dataset, we observe that developers tend to reduce the number of local methods, simplify the structure statements, reduce the number of paths in the body of the code, and lower the nesting level of the control statements (\eg selection and loop statements) in the method body. %On the other hand, when we observe a significant increase in RFC, we notice that developers lower the complexity of methods by pulling them up in the hierarchy, and so they increase the number of inherited methods. 

As seen in Table \ref{Table:Metrics Suites and Metrics Tools Summary}, the \textit{p}-values obtained from all complexity metrics are statistically significant. The effect sizes
calculated in Cliff 's delta ($\delta$) are found to be large only for WMC, small for CC, MISEI, and MISM, and negligible for the remaining 12 metrics.

\noindent\textbf{Example (Positive Impact):} As an illustrative example, we refer to commit\footnote{\textcolor{black}{\url{https://github.com/apache/maven-surefire/commit/d5de47a4f790ea2d18edb5e05c1ef2adcd2db8a2}}} which implements `Move Class' refactoring to move \texttt{JUnit4TestChecker\break Test.MySuite2} to \texttt{JUnit3TestCheckerTest.MySuite2}. Its CC, NL, and NLE remain unaffected, and its WMC improves from 2 to 1. The unchanged CC could be due to the specific nature of the duplicated code, which might not have affected the control flow patterns significantly. However, the improved WMC could be due to consolidation, optimization, or simplification of methods due to the removal of duplicates. In another example\footnote{\textcolor{black}{\url{https://github.com/apache/kafka/commit/f7b7b4745541a576eb0219468263487b07bac959}}}, `Extract Method' refactoring has been applied by developers to extract \texttt{resume} from \texttt{addStreamTasks} to eliminate duplication. Its four maintainability index metrics, \ie MIMS, MI, MISEI, and MISM improved (44.59 to 75.78, 76.25 to 129.6, 56.64 to 153.82, and 33.12 to 89.95), respectively. The remaining complexity metrics, \ie HCPL, HDIF, HNDB, HPL, HPV, HTRP, HVOL, have also improved (343.36 to 23.50, 52.85 to 3, 1515.44 to 44, 184 to 11, 67 to 10, 3277.43 to 6.09, and 1116.16 to 36.54, respectively).
%https://github.com/apache/maven-surefire/commit/d5de47a4f790ea2d18edb5e05c1ef2adcd2db8a2

\noindent\textbf{Example (Negative Impact):} As an illustrative example, we refer to commit\footnote{\textcolor{black}{\url{https://github.com/apache/sis/commit/c8ffc0116b86f39caa3d2f45dca5dec68049c93e}}} which implements `Extract Superclass' refactoring to extract \texttt{Element} from \texttt{Copyright} and \texttt{Person}. Its CC increases from 0 to 0.11, its WMC increases from 3 to 45, its NL and NLE increase from 0 to 2. When referring to commit\footnote{\textcolor{black}{\url{https://github.com/apache/ant-ivy/commit/b74264847ef8e9ffeaf06d5fa1fdead4a065b480}}}, the `Extract Method' refactoring to extract \texttt{getReportFile} from \texttt{getRealDependencyRevisionIds} to remove duplication. Its four maintainability index metrics have not improved (75.71 to 64.27, 129.47 to 109.90, 155.64 to 83.06, 91,02, for MIMS, MI, MISEI, and MISM, respectively).   The remaining complexity metrics, \ie HCPL, HDIF, HNDB, HPL, HPV, HTRP, HVOL, have also not improved (61.30 to 120.76, 12.25 to 21, 108.09 to 317.87, 22 to 55, 18 to 30, 62.43 to 314.85, and 91.73 to 269.87, respectively). The absence of improvement in complexity metrics could be due to factors such as the nature of duplicated code, the distribution of complexity across the codebase, and the potential compensatory complexity introduced during the code duplicate removal process. 


%https://github.com/apache/sis/commit/c8ffc0116b86f39caa3d2f45dca5dec68049c93e
% \begin{boxK}
% \textit{\textbf{Summary.} CC, NL, and NLE remain unchanged, and the remaining 13 complexity-related metrics generally improve as the developer intends to improve code duplicate, and all their variation is significant.}
% \end{boxK}




\begin{figure*}
% row-1
\centering
\begin{subfigure}{3.5cm}
\centering\includegraphics[width=3.5cm]{Images/CC.png}
\caption{Complexity - CC}
\label{BP:Complexity-cc}
\end{subfigure}%
\begin{subfigure}{3.5cm}
\centering\includegraphics[width=3.5cm]{Images/WMC.png}
\caption{Complexity - WMC}
\label{BP:Complexity-wmc}
\end{subfigure}%
\begin{subfigure}{3.5cm}
\centering\includegraphics[width=3.5cm]{Images/NL.png}
\caption{Complexity - NL}
\label{BP:Complexity-nl}
\end{subfigure}%
\begin{subfigure}{3.5cm}
\centering\includegraphics[width=3.5cm]{Images/NLE.png}
\caption{Coupling - NLE}
\label{BP:coupling-nle}
\end{subfigure}%
\vspace{9pt}
\begin{subfigure}{3.5cm}
\centering\includegraphics[width=3.5cm]{Images/HCPL.png}
\caption{Complexity - HCPL}
\label{BP:Complexity-hcpl}
\end{subfigure}%
\begin{subfigure}{3.5cm}
\centering\includegraphics[width=3.5cm]{Images/HDIF.png}
\caption{Complexity - HDIF}
\label{BP:Complexity-hdif}
\end{subfigure}%
\begin{subfigure}{3.5cm}
\centering\includegraphics[width=3.5cm]{Images/HEFF.png}
\caption{Complexity - HEFF}
\label{BP:Complexityheff}
\end{subfigure}%
\begin{subfigure}{3.5cm}
\centering\includegraphics[width=3.5cm]{Images/HNDB.png}
\caption{Complexity - HNDB}
\label{BP:Complexity-hndb}
\end{subfigure}%
\vspace{9pt}
\begin{subfigure}{3.5cm}
\centering\includegraphics[width=3.5cm]{Images/HPL.png}
\caption{Complexity - HPL}
\label{BP:Complexity-hpl}
\end{subfigure}%
\begin{subfigure}{3.5cm}
\centering\includegraphics[width=3.5cm]{Images/HPV.png}
\caption{Complexity - HPV}
\label{BP:Complexity-hpv}
\end{subfigure}%
\begin{subfigure}{3.5cm}
\centering\includegraphics[width=3.5cm]{Images/HTRP.png}
\caption{Complexity - HTRP}
\label{BP:Complexity-htrp}
\end{subfigure}%
\begin{subfigure}{3.5cm}
\centering\includegraphics[width=3.5cm]{Images/HVOL.png}
\caption{Complexity - HVOL}
\label{BP:Complexity-hvol}
\end{subfigure}%
\vspace{9pt}
\begin{subfigure}{3.5cm}
\centering\includegraphics[width=3.5cm]{Images/MIMS.png}
\caption{Complexity - MIMS}
\label{BP:Complexity-mims}
\end{subfigure}%
\begin{subfigure}{3.5cm}
\centering\includegraphics[width=3.5cm]{Images/MI.png}
\caption{Complexity - MI}
\label{BP:Complexity-mi}
\end{subfigure}%
\begin{subfigure}{3.5cm}
\centering\includegraphics[width=3.5cm]{Images/MISEI.png}
\caption{Complexity - MISEI}
\label{BP:Complexity-misei}
\end{subfigure}%
\begin{subfigure}{3.5cm}
\centering\includegraphics[width=3.5cm]{Images/MISM.png}
\caption{Complexity - MISM}
\label{BP:Complexity-mism}
\end{subfigure}%
\caption{\textcolor{black}{Boxplots of complexity metric values of pre- and post-refactored files.}}
\label{Chart:Boxplots_complexity}
\end{figure*}



% \begin{figure*}
% % row-1
% \centering
% \begin{subfigure}{3.5cm}
% \centering\includegraphics[width=3.5cm]{Images/LCOM5.png}
% \caption{Cohesion - LCOM5}
% \label{BP:chesion-lcom}
% \end{subfigure}%
% \begin{subfigure}{3.5cm}
% \centering\includegraphics[width=3.5cm]{Images/CBO.png}
% \caption{Coupling - CBO}
% \label{BP:coupling-cbo}
% \end{subfigure}%
% \begin{subfigure}{3.5cm}
% \centering\includegraphics[width=3.5cm]{Images/RFC.png}
% \caption{Coupling - RFC}
% \label{BP:coupling-rfc}
% \end{subfigure}%\vspace{9pt}
% \begin{subfigure}{3.5cm}
% \centering\includegraphics[width=3.5cm]{Images/NII.png}
% \caption{Coupling - NII}
% \label{BP:coupling-nii}
% \end{subfigure}%\vspace{9pt}





% % row-2 
% \begin{subfigure}{3.5cm}
% \centering\includegraphics[width=3.5cm]{Images/NOI.png}
% \caption{Coupling - NOI}
% \label{BP:coupling-noi}
% \end{subfigure}%
% \begin{subfigure}{3.5cm}
% \centering\includegraphics[width=3.5cm]{Images/CC.png}
% \caption{Complexity - CC}
% \label{BP:Complexity-cc}
% \end{subfigure}%
% \begin{subfigure}{3.5cm}
% \centering\includegraphics[width=3.5cm]{Images/WMC.png}
% \caption{Complexity - WMC}
% \label{BP:Complexity-wmc}
% \end{subfigure}%
% \begin{subfigure}{3.5cm}
% \centering\includegraphics[width=3.5cm]{Images/NL.png}
% \caption{Complexity - NL}
% \label{BP:Complexity-nl}
% \end{subfigure}%


% % row-2 
% \begin{subfigure}{3.5cm}
% \centering\includegraphics[width=3.5cm]{Images/NLE.png}
% \caption{Coupling - NLE}
% \label{BP:coupling-nle}
% \end{subfigure}%
% \begin{subfigure}{3.5cm}
% \centering\includegraphics[width=3.5cm]{Images/HCPL.png}
% \caption{Complexity - HCPL}
% \label{BP:Complexity-hcpl}
% \end{subfigure}%
% \begin{subfigure}{3.5cm}
% \centering\includegraphics[width=3.5cm]{Images/HDIF.png}
% \caption{Complexity - HDIF}
% \label{BP:Complexity-hdif}
% \end{subfigure}%
% \begin{subfigure}{3.5cm}
% \centering\includegraphics[width=3.5cm]{Images/HEFF.png}
% \caption{Complexity - HEFF}
% \label{BP:Complexityheff}
% \end{subfigure}%




% \begin{subfigure}{3.5cm}
% \centering\includegraphics[width=3.5cm]{Images/HNDB.png}
% \caption{Complexity - HNDB}
% \label{BP:Complexity-hndb}
% \end{subfigure}%
% \begin{subfigure}{3.5cm}
% \centering\includegraphics[width=3.5cm]{Images/HPL.png}
% \caption{Complexity - HPL}
% \label{BP:Complexity-hpl}
% \end{subfigure}%
% \begin{subfigure}{3.5cm}
% \centering\includegraphics[width=3.5cm]{Images/HPV.png}
% \caption{Complexity - HPV}
% \label{BP:Complexity-hpv}
% \end{subfigure}%
% \begin{subfigure}{3.5cm}
% \centering\includegraphics[width=3.5cm]{Images/HTRP.png}
% \caption{Complexity - HTRP}
% \label{BP:Complexity-htrp}
% \end{subfigure}%

% \begin{subfigure}{3.5cm}
% \centering\includegraphics[width=3.5cm]{Images/HVOL.png}
% \caption{Complexity - HVOL}
% \label{BP:Complexity-hvol}
% \end{subfigure}%
% \begin{subfigure}{3.5cm}
% \centering\includegraphics[width=3.5cm]{Images/MIMS.png}
% \caption{Complexity - MIMS}
% \label{BP:Complexity-mims}
% \end{subfigure}%
% \begin{subfigure}{3.5cm}
% \centering\includegraphics[width=3.5cm]{Images/MI.png}
% \caption{Complexity - MI}
% \label{BP:Complexity-mi}
% \end{subfigure}%
% \begin{subfigure}{3.5cm}
% \centering\includegraphics[width=3.5cm]{Images/MISEI.png}
% \caption{Complexity - MISEI}
% \label{BP:Complexity-misei}
% \end{subfigure}%

% \begin{subfigure}{3.5cm}
% \centering\includegraphics[width=3.5cm]{Images/MISM.png}
% \caption{Complexity - MISM}
% \label{BP:Complexity-mism}
% \end{subfigure}%
% \begin{subfigure}{3.5cm}
% \centering\includegraphics[width=3.5cm]{Images/DIT.png}
% \caption{Inheritance - DIT}
% \label{BP:Inheritance-dit}
% \end{subfigure}%\vspace{9pt}
% \begin{subfigure}{3.5cm}
% \centering\includegraphics[width=3.5cm]{Images/NOC.png}
% \caption{Inheritance - NOC}
% \label{BP:Inheritance-noc}
% \end{subfigure}%
% \begin{subfigure}{3.5cm}
% \centering\includegraphics[width=3.5cm]{Images/NOA.png}
% \caption{Inheritance - NOA}
% \label{BP:Inheritance-noa}
% \end{subfigure}%



% % row 3
% \begin{subfigure}{3.5cm}
% \centering\includegraphics[width=3.5cm]{Images/LOC.png}
% \caption{Design Size - LOC}
% \label{BP:Design Size-loc}
% \end{subfigure}%
% \begin{subfigure}{3.5cm}
% \centering\includegraphics[width=3.5cm]{Images/TLOC.png}
% \caption{Design Size - TLOC}
% \label{BP:Design Size-tloc}
% \end{subfigure}%
% \begin{subfigure}{3.5cm}
% \centering\includegraphics[width=3.5cm]{Images/LLOC.png}
% \caption{Design Size - LLOC}
% \label{BP:Design Size-lloc}
% \end{subfigure}%
% \begin{subfigure}{3.5cm}
% \centering\includegraphics[width=3.5cm]{Images/TLLOC.png}
% \caption{Design Size - TLLOC}
% \label{BP:Design Size-tlloc}
% \end{subfigure}%


% \begin{subfigure}{3.5cm}
% \centering\includegraphics[width=3.5cm]{Images/CLOC.png}
% \caption{Design Size - CLOC}
% \label{BP:Design Size-cloc}
% \end{subfigure}%
% \begin{subfigure}{3.5cm}
% \centering\includegraphics[width=3.5cm]{Images/NPM.png}
% \caption{Design Size - NPM}
% \label{BP:Design Size-npm}
% \end{subfigure}%
% \begin{subfigure}{3.5cm}
% \centering\includegraphics[width=3.5cm]{Images/NOS.png}
% \caption{Design Size - NOS}
% \label{BP:Design Size-nos}
% \end{subfigure}%
% \begin{subfigure}{3.5cm}
% \centering\includegraphics[width=3.5cm]{Images/TNOS.png}
% \caption{Design Size - TNOS}
% \label{BP:Design Size-tnos}
% \end{subfigure}

% \caption{Boxplots of metrics values of pre- and post-refactored files.} 
% \label{Chart:Boxplots_Al_V1}
% \end{figure*}
%%%%%%%%%%%%%%%%%%%%%%%%%%%%%%%%%%%%%%%%%%%%%%%%%%%%%%%%%%%%%%%%%%%%%%%%%%%%%%%%%%%%%%%%%
%\subsubsection{Inheritance}
\noindent\textbf{\textcolor{black}{Inheritance.}} For commits that involve the removal of code duplication, the boxplots depicted in Figures  \ref{BP:Inheritance-dit}, \ref{BP:Inheritance-noc} and \ref{BP:Inheritance-noa} showcase the pre- and post-refactoring results of three structural metrics:  \ie DIT, NOC, and NOA, used in the literature to estimate the inheritance. We observe that only one metric among the three experienced a degradation in median values. Specifically, the median for NOC decreased from 3 to 0, while the median for DIT and NOA increased from 2 to 3 and from 3 to 4, respectively. This suggests that developers may be increasing the depth of the hierarchy by adding more methods for a class to inherit, reducing the number of immediate subclasses, and increasing the number of methods added by a subclass. While some instances show improvement in inheritance, the overall depth of the inheritance tree and the number of methods added by a subclass did not decrease.  The interpretation of the metric improvement depends highly on the quality of the code and the developer's design decisions. The statistical test shows that the differences are statistically significant for DIT, NOC, and NOA. The magnitude of the difference between the three metrics is large.

\noindent\textbf{Example (Positive Impact):} One of the examples that demonstrated improvement in inheritance was found
in a particular commit in the Maven project\footnote{\textcolor{black}{\url{https://github.com/apache/maven-surefire/commit/d5de47a4f790ea2d18edb5e05c1ef2adcd2db8a2}}}.  The developer applied `Move Class' refactoring to move \texttt{JUnit4TestChecker\break Test.CustomSuiteOnlyTest} to \texttt{JUnit3TestCheckerTest.CustomSuiteOnlyTest}. Its DIT drops from 1 to 0, its NOC remains unaffected, and its NOA improves from 1 to 0. This increases the reuse of common code logic and leads to more effective inheritance relationships and a better-defined hierarchy.
%https://github.com/apache/maven-surefire/commit/d5de47a4f790ea2d18edb5e05c1ef2adcd2db8a2

\noindent\textbf{Example (Negative Impact):} One of the examples that showed improvement in inheritance was found
in a particular commit in the Archiva project\footnote{\textcolor{black}{\url{https://github.com/apache/archiva/commit/26e9c3b257bed850d0e2f0bc9dc2d7f11381b789}}}.  The developer applied `Extract Superclass' refactoring to extract \texttt{AbstractDiscoverer} from class \texttt{AbstractArtifactDiscoverer}. Its DIT increases from 0 to 1, its NOC remains unaffected, and its NOA increases from 0 to 1. This indicates that the refactoring applied to remove duplication does not always improve inheritance metrics due to either pre-existing inheritance challenges, or the focused nature of the duplication removal. 
%https://github.com/apache/archiva/commit/26e9c3b257bed850d0e2f0bc9dc2d7f11381b789
% \begin{boxK}
% \textit{\textbf{Summary.} NOC generally decreases as the developer intends to remove code duplication, and its variation is significant. DIT and NOA exhibit opposite variations in inheritance. }
% \end{boxK}


\begin{figure*}
% row-1
\centering
\begin{subfigure}{3.5cm}
\centering\includegraphics[width=3.5cm]{Images/DIT.png}
\caption{Inheritance - DIT}
\label{BP:Inheritance-dit}
\end{subfigure}%\vspace{9pt}
\begin{subfigure}{3.5cm}
\centering\includegraphics[width=3.5cm]{Images/NOC.png}
\caption{Inheritance - NOC}
\label{BP:Inheritance-noc}
\end{subfigure}%
\begin{subfigure}{3.5cm}
\centering\includegraphics[width=3.5cm]{Images/NOA.png}
\caption{Inheritance - NOA}
\label{BP:Inheritance-noa}
\end{subfigure}%
\caption{\textcolor{black}{Boxplots of inheritance metric values of pre- and post-refactored files.}}
\label{Chart:Boxplots_inheritance}
\end{figure*}
%%%%%%%%%%%%%%%%%%%%%%%%%%%%%%%%%%%%%%%%%%%%%%%%%%%%%%%%%%%%%%%%%%%%%%%%%%%%%%%%%%%%%%%%%%
%\subsubsection{Design Size}
\noindent\textbf{\textcolor{black}{Design Size.}} For commits whose messages report the removal of code duplicate, the boxplots sketched in Figures \ref{BP:Design Size-loc}, \ref{BP:Design Size-tloc}, \ref{BP:Design Size-lloc}, \ref{BP:Design Size-tlloc}, \ref{BP:Design Size-cloc},  \ref{BP:Design Size-npm}, \ref{BP:Design Size-nos}, and \ref{BP:Design Size-tnos}  show the pre- and post-refactoring results of four structural metrics, \ie LOC, TLOC, LLOC, TLLOC, CLOC,  NPM, NOS, and TNOS, used in the literature to estimate the design size. We notice the improvement of six metrics, namely LOC, TLOC, LLOC, TLLOC, NOS, and TNOS after the commits in which developers explicitly target the improvement of code duplication, their variations are statistically significant.  
 The magnitude of LOC, TLOC, and TLLOC is small, whereas the magnitude for LLOC, NOS, and TNOS is negligible. As seen in the box plots, the medians generally decreased. However, we note that the medians for CLOC and NPM remain unchanged. The differences in CLOC and NPM are statistically significant, and the magnitude of the difference is negligible and large, respectively. This indicates that developers generally retain the lines containing comments and maintain the same number of methods after applying refactoring. %This shows us that developers added more lines of code plus more declarative and executable statements after the application of refactoring that might be because developer intentions is to improve the readability and the clarity of the code. 

\noindent\textbf{Example (Positive Impact):} As an illustrative example, we refer to the commit\footnote{\textcolor{black}{\url{https://github.com/apache/maven-surefire/commit/d5de47a4f790ea2d18edb5e05c1ef2adcd2db8a2}}} which implements `Extract Method' refactoring to extract \texttt{accept\break(testClass)} from \texttt{invalidTest}. Its LOC, TLOC, LLOC, TLLOC drop from 6 to 4, and its CLOC, NOS, and TNOS remain unaffected.  Furthermore, when moving the class \texttt{JUnit4TestCheckerTest.AlsoValid}  to \texttt{JUnit3TestChecker\break Test.AlsoValid}, its NPM improves from 1 to 0.  In qualitative terms, the removal of code duplication and the introduction of a dedicated method have led to more modular, focused, and readable code. This shows that size metrics capture the removal of code duplication as perceived by the developer.
%https://github.com/apache/maven-surefire/commit/d5de47a4f790ea2d18edb5e05c1ef2adcd2db8a2

\noindent\textbf{Example (Negative Impact):} As an illustrative example, we refer to the commit\footnote{\textcolor{black}{\url{https://github.com/apache/commons-bcel/commit/67dfdf60f5f8ccb8ed910bfe9d1cdc6e84f0db36}}} which implements `Extract Method' refactoring to extract \texttt{accept\break(createAnnotationEntries)} from \texttt{getAnnotationEntries}. Its LOC and TLOC increased from 7 to 11, its LLOC, TLLOC increased from 6 to 10, and its NOS and TNOS increased from 3 to 6. Its CLOC decreases from 3 to 1. The observed lack of improvement, in this case, can be attributed to a couple of factors, including the nature of the changes made, the extent of duplication and additional compensatory changes. This results in an overall increase in the class size as assessed by these employed design size metrics.
%https://github.com/apache/commons-bcel/commit/67dfdf60f5f8ccb8ed910bfe9d1cdc6e84f0db36
\begin{figure*}
% row-1
\centering
% row 3
\begin{subfigure}{3.5cm}
\centering\includegraphics[width=3.5cm]{Images/LOC.png}
\caption{Design Size - LOC}
\label{BP:Design Size-loc}
\end{subfigure}%
\begin{subfigure}{3.5cm}
\centering\includegraphics[width=3.5cm]{Images/TLOC.png}
\caption{Design Size - TLOC}
\label{BP:Design Size-tloc}
\end{subfigure}%
\begin{subfigure}{3.5cm}
\centering\includegraphics[width=3.5cm]{Images/LLOC.png}
\caption{Design Size - LLOC}
\label{BP:Design Size-lloc}
\end{subfigure}%
\begin{subfigure}{3.5cm}
\centering\includegraphics[width=3.5cm]{Images/TLLOC.png}
\caption{Design Size - TLLOC}
\label{BP:Design Size-tlloc}
\end{subfigure}%
\vspace{9pt}
\begin{subfigure}{3.5cm}
\centering\includegraphics[width=3.5cm]{Images/CLOC.png}
\caption{Design Size - CLOC}
\label{BP:Design Size-cloc}
\end{subfigure}%
\begin{subfigure}{3.5cm}
\centering\includegraphics[width=3.5cm]{Images/NPM.png}
\caption{Design Size - NPM}
\label{BP:Design Size-npm}
\end{subfigure}%
\begin{subfigure}{3.5cm}
\centering\includegraphics[width=3.5cm]{Images/NOS.png}
\caption{Design Size - NOS}
\label{BP:Design Size-nos}
\end{subfigure}%
\begin{subfigure}{3.5cm}
\centering\includegraphics[width=3.5cm]{Images/TNOS.png}
\caption{Design Size - TNOS}
\label{BP:Design Size-tnos}
\end{subfigure}

\caption{\textcolor{black}{Boxplots of design size metric values of pre- and post-refactored files.}}
\label{Chart:Boxplots_design size}
\end{figure*}
% \begin{boxK}
% \textit{\textbf{Summary.} LOC, TLOC, LLOC, TLLOC, NOS, TNOS generally improve as developers intend to remove code duplication, and their variations are significant. These metrics have a significant positive variation which matches the developer's perception of removing code duplicates.}
% \end{boxK}


\noindent\textbf{\textcolor{black}{Summary.}} \textcolor{black}{This \textcolor{black}{section} summarizes our findings and their implications.}

\begin{itemize}
    \item \textcolor{black}{\textbf{Cohesion.} The normalized LCOM metric not only serves as a suitable substitute for the original LCOM but also serves as a representation of the cohesion quality attribute. A positive variation in this metric aligns with the developer's intention to eliminate code duplication.}
    \item \textcolor{black}{\textbf{Coupling.} CBO, RFC, and NOI generally improve as the developer intends to eliminate code duplication, and their variation is significant. NII exhibits opposite variations in
coupling.}
    \item \textcolor{black}{\textbf{Complexity.} CC, NL, and NLE remain unchanged, and the remaining 13 complexity-related metrics generally improve as the developer intends to improve code duplicate, and all their variation is significant.}
    \item \textcolor{black}{\textbf{Inheritance.} NOC generally decreases as the developer intends to remove code duplication, and its variation is significant. DIT and NOA exhibit opposite variations in inheritance.}
    \item \textcolor{black}{\textbf{Design Size.} LOC, TLOC, LLOC, TLLOC, NOS, TNOS generally improve as developers intend to remove code duplication, and their variations are significant. These metrics have a significant positive variation which matches the developer's perception of removing code duplicates.}
\end{itemize}

%\begin{comment}

\subsection{What are the refactoring operations that are associated with code duplicate removal?}
\begin{figure}[t]
\centering 
\begin{tikzpicture}
\begin{scope}[scale=0.8]
\pie[rotate = 180,pos ={0,0},text=inside,outside under=20,no number]{55.7/Extract Method\and55.7\%, 37.5/Move Method\and37.5\%, 3.8/Extract Superclass\and3.8\%,2.7/Move Attribute\and2.7\%,0.3/Move Class\and0.3\%}
\end{scope}
\end{tikzpicture}
\caption{Distribution of refactoring operations for code duplicate removal.}
\label{fig:refactoringtypes}
%\vspace{-.6cm}
\end{figure}
Looking at the refactoring operations that could play a role in code duplicate removal, Figure \ref{fig:refactoringtypes} depicts the percentages of refactoring operations. \textcolor{black}{As can be seen, the most common category concerns `Extract Method', representing 55.7\% of the commits. This observation is in line with the findings of previous studies describing that `Extract Method' refactoring is considered \say{Swiss army knife} of
refactorings as developers often apply it to eliminate duplicated code \citep{higo2004aries,higo2005aries,higo2008metric,tairas2012increasing,bian2013spape,yue2018automatic,yoshida2019proactive,arcelli2015duplicated,alomar2022anticopypaster,alomar2023just,alomar2024behind}. In fact, a recent study on extract method refactoring highlights that method extraction is one of the main refactorings that were defined when the area was established \citep{alomar2024behind,griswold1993automated}, as it is a common response to the need to keep methods concise and modular, and reduced the spread of shared responsibilities.} The next most common categories are `Move Method', representing 37.5\% of the commits. This indicates that developers might improve the quality of the code by moving the method containing duplication to a different class, effectively eliminating duplicated code. The category `Extract Superclass', `Move Attribute', and `Move Class' had the least number of commits, which had a ratio of 3.8\%, 2.7\%, and 0.3\%, respectively. 

When performing manual inspection of source code, we notice that these five refactoring operations contribute to the elimination of code duplication in several ways. By performing the `Extract Method' refactoring, redundant code segments can be consolidated into a single method that can be reused across different parts of the codebase. Additionally, when moving methods from one class to another using `Move Method' refactoring, it helps centralize logic and eliminate duplicate code that might have been present in multiple classes. Moreover, by extracting a superclass using `Extract Superclass' refactoring, it encapsulates common attributes and behaviors of related classes, allowing duplicated code to be consolidated. This can be followed by moving shared attributes to a common superclass using `Move Attribute' refactoring to reduce redundancy and ensures that changes to these attributes are reflected across all subclasses. Finally, moving the entire class to a common location using `Move Class' refactoring can help in reducing duplicated code, and it is useful when classes share similar functionality but exist in different parts of the codebase.

%\end{comment}

\section{Lessons Learned}
\label{Section:lesson}


\begin{figure*}[htbp]
\centering 
%\includegraphics[width=8cm,height=13cm,keepaspectratio]{Images/MSR-Example-v3.png}
%\includegraphics[width=16cm,height=8cm,keepaspectratio]{Images/MSR-Example-v3.png}
\includegraphics[width=1.2\textwidth]{Images/Example-type1.pdf}
%\vspace{-.5cm}
\caption{\textcolor{black}{Example of selected Type-1 code clone from kafka project.}} %\cite{kafka-type1}
\label{fig:example-casestudies-type1}
%\vspace{-.4cm}
\end{figure*}

 \begin{figure*}[htbp]
\centering 
% %\includegraphics[width=8cm,height=13cm,keepaspectratio]{Images/MSR-Example-v3.png}
% %\includegraphics[width=16cm,height=8cm,keepaspectratio]{Images/MSR-Example-v3.png}
 \includegraphics[width=1.2\textwidth]{Images/Example-type2.pdf}
% %\vspace{-.5cm}
 \caption{\textcolor{black}{Example of selected Type-2 code clone from cayenne project.}} %\cite{cayenne}
 \label{fig:example-casestudies-type2}
%\vspace{-.4cm}
\end{figure*}

\begin{figure*}[htbp]
\centering 
%\includegraphics[width=8cm,height=13cm,keepaspectratio]{Images/MSR-Example-v3.png}
%\includegraphics[width=16cm,height=8cm,keepaspectratio]{Images/MSR-Example-v3.png}
\includegraphics[width=1.3\textwidth]{Images/Example-type3.pdf}
%\vspace{-.5cm}
\caption{\textcolor{black}{Example of selected Type-3 code clone from pig project.}} %\cite{pig}
\label{fig:example-casestudies-type3}
%\vspace{-.4cm}
\end{figure*}

\noindent{\textbf{ \textcolor{black}{Lesson 1: Code clones associated with commits about duplicate removal are from different clone types.}}}  \textcolor{black}{There are various types of code clone exist in the literature (\ie Type-1, Type-2, Type-3, and Type-4) \citep{mondal2020survey}. When performing manual examination of commits associated with code clones, we realized that some commits with the explicit intention of removing duplication are associated with different clone types. Furthermore, in some commits associated with duplicate removal, developers can combine clone refactoring with other unrelated changes, such as feature updates, bug fixes, or general code cleanup. This observation is consistent with existing studies that show that developers interleave refactoring with other changes, and 11– 39\% of bug fixing commits include other changes \cite{silva2016we,alomar2021we,murphy2012we,nguyen2013filtering}.}



%As shown in previous studies (\eg \citep{pantiuchina2018improving,alomar2019impact}), this tangling can make it difficult to attribute the impact to a specific code change alone precisely. }

\noindent{\textbf{ \textcolor{black}{Lesson 2: Refactoring different types of clones can have different variations on metric values.}}}  \textcolor{black}{As illustrated in \textcolor{black}{Tables \ref{Table:Quality Metrics in Related Work} and \ref{Table:Quality Metrics in Related Work-v2}}, there have been two decades' worth of work on the relationship between refactoring and code quality. We can see that there is room for empirical investigation of the impact of clone removal refactorings on internal quality metrics. In this study, we observe that the impact of refactoring clones on software quality metrics can vary based on the type of clone being refactored. Moreover, developers may have various mechanisms that contribute to removing duplicates, and these strategies may dictate different variations on the metrics. However, locating refactored clone types for each instance presents multiple challenges: (1) a single commit can address multiple clone types simultaneously, making it difficult to attribute metric variations to a specific clone type;  (2) some clone types may occur less frequently in the dataset, further complicating efforts to draw conclusions regarding the influence of clone types on metric variations; and (3) manually determining clone types for each instance is time-consuming and prone to error, particularly when dealing with a large dataset. Although existing clone detection tools can detect the clone, they require additional configuration and setup by the users. In the following, we show an example of each type of clone and its refactoring:}
%However, it is challenging and time-consuming to determine the types of clones and manually refactor them. Although existing clone detection tools can detect the clone, they require additional configuration and setup by the users, which might make them reluctant to perform refactoring suggestions afterwords.}
\begin{itemize}
    \item \textcolor{black}{\textit{Type-1 code clone.} Figure \ref{fig:example-casestudies-type1} illustrates a Type-1 clone that has been refactored. The example demonstrates two duplicate instances, which represent a Type-1 clone (\ie identical code fragments). An `Extract Method' refactoring was applied, resulting in the extraction of the method \texttt{putNodeGroup\break Name(nodeName String, nodeGroupId int, nodeGroups Map, rootTo\break NodeGroup Map)} from \texttt{makeNodeGroups()} in the \texttt{InternalTopologyBuilder} class. For the complexity metrics, we observed varied behavior: CC remained unchanged, some metrics showed improvement (NL, NLE, HEFF, HPL, and HTRP), while others did not improve (HCPL, HDIF, HNDB, HPV, HVOL, MIMS, MI, MISEI, and MISM). Regarding the size metrics, none showed improvement. For coupling metrics, NII improved, whereas NOI did not.} %Extract Method private putNodeGroupName(nodeName String, nodeGroupId int, nodeGroups Map, rootToNodeGroup Map) : int extracted from private makeNodeGroups() : Map in class org.apache.kafka.streams.processor.internals.InternalTopologyBuilder
    \item \textcolor{black}{\textit{Type-2 code clone.} Figure \ref{fig:example-casestudies-type2} depicts a Type-2 clone that has been refactored. This example highlights two duplicate instances, categorized as a Type-2 clone (\ie syntactically identical fragments). The method \texttt{entitiesForCurrentMode()} was extracted from \texttt{generateClassPairs\_1\_1\break (classTemplate String, superTemplate String, superPrefix String)} in the \texttt{MapClassGenerator} class using the `Extract Method' refactoring operation. The complexity metrics have shown improvement, while the design size metrics have also improved, with the exception of CLOC. For coupling metrics, NOI has improved, whereas NII has not.}
    %Extract Method private entitiesForCurrentMode() : Collection extracted from private generateClassPairs_1_1(classTemplate String, superTemplate String, superPrefix String) : void in class org.apache.cayenne.gen.MapClassGenerator
    \item \textcolor{black}{\textit{Type-3 code clone.} Figure \ref{fig:example-casestudies-type3} shows a Type-3 clone that has been refactored. The example illustrates two duplicate instances, identified as a Type-3 clone (\ie copied fragments with further modifications such as changed, added, or removed statements). Through the `Extract Method' refactoring, the method \texttt{runSimpleScript(String name, String[] script)}  was extracted in the \texttt{TestScriptLanguage} class. The complexity metrics have improved overall, with the exception of HDIF, which has decreased, while NL and NLE remain unchanged. Size metrics have also improved, except for CLOC. For coupling metrics, NOI has improved, but NII has decreased. }
    %Extract Method private runPigRunner(name String, script String[]) : PigStats extracted from public pigRunnerTest() : void in class org.apache.pig.test.TestScriptLanguage
\end{itemize}








\noindent{\textbf{ \textcolor{black}{Lesson 3: Some state-of-the-art metrics can capture the developer’s intention of removing code duplication with different degrees of improvement and degradation of software quality.}}} \textcolor{black}{When removing code duplication, developers often perform `Extract Method' refactoring with the expectation of improving code quality. Yet, the state-of-the-art metrics may reflect varying levels of improvement or even degradation following these refactoring events.  For example, in Figure \ref{fig:example-casestudies-2}, we demonstrate the code snippet depicting the instances of code duplication before and after refactoring. We can see that refactoring mining tools detect `Extract Method' refactoring from project commoms-bcel\footnote{\textcolor{black}{\url{https://github.com/apache/commons-bcel/commit/67dfdf60f5f8ccb8ed910bfe9d1cdc6e84f0db36}}}  to extract \texttt{createAnnotationEntries} from \texttt{getAnnotationEntries}. This example emphasizes how refactoring can have mixed effects, positively influencing some metrics while negatively impacting others. As can be seen, its coupling metrics (NII and NOI) have been improved. However, its complexity metrics (CC, NL, NLE, HCPL, HDIF, HEFF, HNDB, HPL, HPV, HTRP, HVOL, MIMS, MI, MISEI, and MISM) and size metrics (LOC, TLOC, LLOC, TLLOC, CLOC, NOS, and TNOS) have not been improved. For metrics where the metrics do not capture the developer's intention, several possible explanations can be consideblack: }
\begin{itemize}
    \item \textcolor{black}{\textit{Inadequacy of the metrics for certain scenarios.} The metrics used to assess software quality, may not always be the most suitable for reflecting the specific intention behind a refactoring. For instance, a developer may intend to improve readability or maintainability, but standard structural metrics may not effectively quantify these aspects. This misalignment between developer goals and the measublack outcomes can lead to discrepancies in how the impact of refactoring is perceived.}
    \item \textcolor{black}{\textit{Limitations of the metrics.} The state-of-the-art metrics have inherent limitations and may not comprehensively capture the effects of refactoring. For example, metrics such as CC focus on the control flow but may overlook improvements in code modularity. This indicates a need to either refine existing metrics or introduce new ones that better align with developer goals, particularly in cases of complex refactoring.}
    \item \textcolor{black}{\textit{Deviation from developer intentions.} In some cases, developers' intentions, as stated in commit messages, may not align with the actual changes performed in the codebase. This could happen for various reasons. For example, a commit message may report the removal of duplicate code, but the implementation might only partially address the duplication or introduce new dependencies, resulting in no measurable improvement or even metric degradation.}
\end{itemize}

\begin{figure*}[htbp]
\centering 
%\includegraphics[width=8cm,height=13cm,keepaspectratio]{Images/MSR-Example-v3.png}
%\includegraphics[width=16cm,height=8cm,keepaspectratio]{Images/MSR-Example-v3.png}
\includegraphics[width=1.2\textwidth]{Images/Example2.pdf}
%\vspace{-.5cm}
%\captionsetup{justification=raggedright,singlelinecheck=false}  % Left-aligns caption

% \caption{\textcolor{black}{Example of selected commit message from commons-bcel project\textsuperscript{\ref{fig:example-casestudies-2}}.}}
 \caption{\textcolor{black}{Example of selected commit message from commons-bcel project.}}
 \label{fig:example-casestudies-2}
\end{figure*}

%https://github.com/apache/commons-bcel/commit/67dfdf60f5f8ccb8ed910bfe9d1cdc6e84f0db36
% \label{fig:example-casestudies-2}
% %\end{figure*}
% \footnotetext[\value{footnote}]{\label{fig:example-casestudies-2}\url{https://github.com/apache/commons-bcel/commit/67dfdf60f5f8ccb8ed910bfe9d1cdc6e84f0db36}.}



%\vspace{-.4cm}
%\end{figure*}
%\footnotetext{\url{https://github.com/apache/commons-bcel/commit/67dfdf60f5f8ccb8ed910bfe9d1cdc6e84f0db36}.} 
% \caption{\textcolor{black}{Example of selected commit message from commons-bcel project\footnotetext{\url{https://github.com/apache/commons-bcel/commit/67dfdf60f5f8ccb8ed910bfe9d1cdc6e84f0db36}.}}}
%%%% just an example %%%%
% \begin{figure}[h]
%     \centering
%     \includegraphics[width=0.7\textwidth]{example-image}  % Replace with actual image
%     \caption{\textcolor{black}{Example of selected commit message from commons-bcel project\textsuperscript{*}.}}
%     \label{fig:commit-example}
% \end{figure}

% \begin{center}
%     \footnotesize\textsuperscript{*}\url{https://github.com/apache/commons-bcel/commit/67dfdf60f5f8ccb8ed910bfe9d1cdc6e84f0db36}
% \end{center}


%For the cases that metrics  do not capture the intention of developer, it might be because these metrics are not the best metrics to consider on these cases, limitation of the metrics itself, or developers did not follow their intention reported in the commit messages.

%This is a thorough empirical investigation of a long standing question within the refactoring community: does refactoring improve code quality?

%There has been two decades worth of work on the relationship between refactoring and code quality but it has been nicely summarized herein. The authors indeed illustrate that there is room for an empirical investigation of the impact of clone removal refactorings on internal product metrics.

%If LCOM increases it may be correlated with removal of duplicated code but it will also be correlated with lots of other manipulations of code. And if NOC decreases it may be a symptom of clone removal but can just as well be a symptom of other code manipulations.

%One of the strengths is that the author shows anecdotal evidence (both positive and negative) of impact clone removing refactorings have on quality metrics. However, almost all positive anecdotal evidence stems from refactoring JUnit4TestCheckerTest. It has been shown that test code is sufficiently different from normal code with respect to clones. (cfr. Brent van Bladel and Serge Demeyer. A comparative study of test code clones and production code clones. Journal of Systems and Software, 176:110940
\section{Implications}
\label{Section:Implication}

%The main implications of this study are as follows.

 \noindent{\textbf{ Further advancing quality metrics and duplicate code removal.}} The existing literature discusses various automatic refactoring approaches aimed at assisting practitioners in detecting antipatterns or code smells. Baqais and Alshayeb \citep{baqais2020automatic} have highlighted the growing interest in automatic refactoring studies. The researchers explored the potential of machine learning to identify refactoring opportunities. Since features play a vital role in the quality of machine learning models obtained, this study can contribute to determining which metrics can serve as effective features in machine learning algorithms, facilitating the accurate recommendation of refactoring opportunities at different levels of granularity (\ie class, method, field), which can assist developers in automatically making their decisions. For example, incorporating the most impactful metrics as features in predicting whether a given piece of code should undergo a specific refactoring operation can enhance developers' confidence in accepting recommended refactorings or selecting the most suitable refactoring candidate. This knowledge is needed because, in practice, the built model should require as little data as possible. Furthermore, since we observe that some of the quality metrics did not capture any improvement, we plan to conduct more experiments to validate the effectiveness of these metrics to explore whether the observations are due to the appropriateness of the quality metrics or to the needed validation and clarity of the perception of the developers.

 \noindent{\textbf{ Putting developer in the loop when designing refactoring recommendation systems.}} Based on the findings, it becomes evident that different structural metrics have the capacity to depict code duplication, thereby influencing software quality in diverse ways. Certain metrics improve software quality, whereas others might result in its decline. %This variation highlights that not all metrics align with developers' intentions, as documented in their commit messages. 
 This underscores the importance of involving developers in the design of refactoring recommendation systems, effectively engaging them in the process. This approach emerges as effective in discerning meaningful refactorings that align with the perspectives of developers \citep{hall2012supervised,bavota2012putting,pantiuchina2018improving}.


 \noindent{\textbf{ Examining the code duplicate removal potentials with refactoring.} Our study reveals the context in which developers refactor the code to eliminate code duplicates. Our future research direction can focus on providing a comprehensive taxonomy for code duplication-aware refactoring practices. This taxonomy can show various contexts of code duplicates and refactoring and can demonstrate different forms of code reuse. Thereafter, researchers can build on top of our findings to better understand developer practices and investigate to what extent this taxonomy for refactoring with awareness of duplicate code improves the system's quality.


 \noindent\textbf{ Understanding the completeness of the quality metric capturing duplicate code removal as documented by developers.} We observe that not all quality metrics can capture the improvement in duplicate code removal perceived by developers in their commit messages. Although quality metrics can help pinpoint design flaws for refactoring recommendation systems, such a recommendation would be meaningful if qualitative insights from developers complemented it. Furthermore, the alignment or disparity between the enhancement of software quality as perceived by developers and its evaluation through quality metrics can be attributed to factors such as the focused nature of the duplication removal, the extent of duplication, and the potential compensatory changes. Future research is encouraged to consider the direct effect of duplicate removal and the broader context of code changes and their implications for quality metrics. 

%\faThumbTack \noindent{\textbf{ Lack of clarity of how the code duplicate removal tools leverage metrics and decide the associated threshold to make the decision.}} Existing `Extract Method' refactoring tools such as \texttt{AntiCopyPaster} \citep{alomar2022anticopypaster,alomar2023just} used 78 metrics related to size, complexity, coupling, and keywords to extract duplicate code. On the contrary, \texttt{Aries} \citep{higo2005aries} used six other metrics to identify the removal of the code clone. However, the implementation of these metrics may vary between these tools based on the context. In addition, there may be cases where different metric names are used to improve some quality attributes. This phenomenon might affect the interpretation of the accuracy of the recommended code duplicate removal tools. 

 \noindent{\textbf{ Investigating the characteristics and effects of eliminating code duplication on software quality.} The results advance our understanding of the effects of eliminating code duplication on software quality. It is evident that certain software quality metrics can be used as indicators for code fragments that are more likely to be extracted and identified as problematic and should be removed by refactoring. Consequently, a threshold can be established to show when quality metrics reach a level where duplicate code will have a negative effect on maintenance and need to be refactored.

%he outcome underscores that while code duplication removal can impact various aspects of a codebase, its effects on inheritance metrics might be context-dependent and influenced by the specific structure of the codebase.
\section{Threats to Validity}
\label{Section:Threats}

In this section, we describe potential threats to the validity of our research method and the actions we took to mitigate them.

\textbf{Internal Validity.} The accuracy of our analysis is primarily dependent on the precision of the refactoring mining tools, as these tools may miss the detection of some refactorings. However, previous studies \citep{silva2016we,tsantalis2018accurate,silva2017refdiff} report that \texttt{RefactoringMiner} and \texttt{RefDiff} have high precision and recall scores compared to other state-of-the-art refactoring detection tools, giving us confidence in using the tools. Another potential threat to validity is related to commit messages. \textcolor{black}{This study does not exclude commits containing tangle code changes \citep{herzig2016impact,kirinuki2014hey}, where developers made changes related to different tasks and one of these tasks could be related to quality improvement. If these changes were committed at once, there is a possibility that the individual changes merge and that the original task cannot be traced back. Similarly to the previous study \cite{pantiuchina2018improving}, we did not consider filtering out such changes in this study}. Moreover, our manual analysis is time-consuming and error-prone, which we tried to mitigate by focusing mainly on commits known to contain refactorings. 

Another potential threat to validity is sample bias, where the choice of the data can directly impact the results. Therefore, we explored a large sample of projects from the SmartSHARK dataset \citep{trautsch2021msr}, to ensure the quality of the findings and diversify the sources to reduce the bias of the data belonging to the same entity. The qualitative analysis was conducted by a single author, which could introduce bias into the process. However, commits that were debatable were discarded. We also provide our dataset online for further refinement and analysis. %During our qualitative analysis, we consideblack only commits where a consensus between authors was reached on whether a message clearly states the removal of duplicate code. Commits that were debatable were discarded. We also provide our dataset online for further refinement and analysis.

\textbf{Construct Validity.} A potential threat to construct validity relates to the set of metrics, as it may miss some properties of the selected internal quality attributes. To address this potential threat, we mitigate it by choosing well-known metrics that encompass various properties of each attribute, as reported in the literature \citep{chidamber1994metrics}.

\textbf{External Validity.} Our analysis was limited to only open-source Java projects. However, we were able to examine 128 projects, which were well-commented and exhibited diversity in terms of size, contributors, number of commits, and refactorings. \textcolor{black}{Still, we believe that the results found in this study are largely language-agnostic. However, certain language-specific characteristics, such as syntax complexity and tooling support, can influence duplication patterns. Although we expect similar trends across languages with similar paradigms, a comprehensive analysis encompassing various languages is recommended to confirm this generalization.}

%Still, we believe that the removal of duplicates is largely language-agnostic. However, certain language-specific characteristics, such as syntax complexity and tooling support, can influence duplication patterns. Although we expect similar trends across languages with similar paradigms, a comprehensive analysis encompassing various languages is suggested to is recommended to confirm this generalization.
\section{Conclusion}\label{sec:conclusion}
This work introduces a novel approach to TOT query elicitation, leveraging LLMs and human participants to move beyond the limitations of CQA-based datasets. Through system rank correlation and linguistic similarity validation, we demonstrate that LLM- and human-elicited queries can effectively support the simulated evaluation of TOT retrieval systems. Our findings highlight the potential for expanding TOT retrieval research into underrepresented domains while ensuring scalability and reproducibility. The released datasets and source code provide a foundation for future research, enabling further advancements in TOT retrieval evaluation and system development.
% \section{Introduction}

Large language models (LLMs) have achieved remarkable success in automated math problem solving, particularly through code-generation capabilities integrated with proof assistants~\citep{lean,isabelle,POT,autoformalization,MATH}. Although LLMs excel at generating solution steps and correct answers in algebra and calculus~\citep{math_solving}, their unimodal nature limits performance in plane geometry, where solution depends on both diagram and text~\citep{math_solving}. 

Specialized vision-language models (VLMs) have accordingly been developed for plane geometry problem solving (PGPS)~\citep{geoqa,unigeo,intergps,pgps,GOLD,LANS,geox}. Yet, it remains unclear whether these models genuinely leverage diagrams or rely almost exclusively on textual features. This ambiguity arises because existing PGPS datasets typically embed sufficient geometric details within problem statements, potentially making the vision encoder unnecessary~\citep{GOLD}. \cref{fig:pgps_examples} illustrates example questions from GeoQA and PGPS9K, where solutions can be derived without referencing the diagrams.

\begin{figure}
    \centering
    \begin{subfigure}[t]{.49\linewidth}
        \centering
        \includegraphics[width=\linewidth]{latex/figures/images/geoqa_example.pdf}
        \caption{GeoQA}
        \label{fig:geoqa_example}
    \end{subfigure}
    \begin{subfigure}[t]{.48\linewidth}
        \centering
        \includegraphics[width=\linewidth]{latex/figures/images/pgps_example.pdf}
        \caption{PGPS9K}
        \label{fig:pgps9k_example}
    \end{subfigure}
    \caption{
    Examples of diagram-caption pairs and their solution steps written in formal languages from GeoQA and PGPS9k datasets. In the problem description, the visual geometric premises and numerical variables are highlighted in green and red, respectively. A significant difference in the style of the diagram and formal language can be observable. %, along with the differences in formal languages supported by the corresponding datasets.
    \label{fig:pgps_examples}
    }
\end{figure}



We propose a new benchmark created via a synthetic data engine, which systematically evaluates the ability of VLM vision encoders to recognize geometric premises. Our empirical findings reveal that previously suggested self-supervised learning (SSL) approaches, e.g., vector quantized variataional auto-encoder (VQ-VAE)~\citep{unimath} and masked auto-encoder (MAE)~\citep{scagps,geox}, and widely adopted encoders, e.g., OpenCLIP~\citep{clip} and DinoV2~\citep{dinov2}, struggle to detect geometric features such as perpendicularity and degrees. 

To this end, we propose \geoclip{}, a model pre-trained on a large corpus of synthetic diagram–caption pairs. By varying diagram styles (e.g., color, font size, resolution, line width), \geoclip{} learns robust geometric representations and outperforms prior SSL-based methods on our benchmark. Building on \geoclip{}, we introduce a few-shot domain adaptation technique that efficiently transfers the recognition ability to real-world diagrams. We further combine this domain-adapted GeoCLIP with an LLM, forming a domain-agnostic VLM for solving PGPS tasks in MathVerse~\citep{mathverse}. 
%To accommodate diverse diagram styles and solution formats, we unify the solution program languages across multiple PGPS datasets, ensuring comprehensive evaluation. 

In our experiments on MathVerse~\citep{mathverse}, which encompasses diverse plane geometry tasks and diagram styles, our VLM with a domain-adapted \geoclip{} consistently outperforms both task-specific PGPS models and generalist VLMs. 
% In particular, it achieves higher accuracy on tasks requiring geometric-feature recognition, even when critical numerical measurements are moved from text to diagrams. 
Ablation studies confirm the effectiveness of our domain adaptation strategy, showing improvements in optical character recognition (OCR)-based tasks and robust diagram embeddings across different styles. 
% By unifying the solution program languages of existing datasets and incorporating OCR capability, we enable a single VLM, named \geovlm{}, to handle a broad class of plane geometry problems.

% Contributions
We summarize the contributions as follows:
We propose a novel benchmark for systematically assessing how well vision encoders recognize geometric premises in plane geometry diagrams~(\cref{sec:visual_feature}); We introduce \geoclip{}, a vision encoder capable of accurately detecting visual geometric premises~(\cref{sec:geoclip}), and a few-shot domain adaptation technique that efficiently transfers this capability across different diagram styles (\cref{sec:domain_adaptation});
We show that our VLM, incorporating domain-adapted GeoCLIP, surpasses existing specialized PGPS VLMs and generalist VLMs on the MathVerse benchmark~(\cref{sec:experiments}) and effectively interprets diverse diagram styles~(\cref{sec:abl}).

\iffalse
\begin{itemize}
    \item We propose a novel benchmark for systematically assessing how well vision encoders recognize geometric premises, e.g., perpendicularity and angle measures, in plane geometry diagrams.
	\item We introduce \geoclip{}, a vision encoder capable of accurately detecting visual geometric premises, and a few-shot domain adaptation technique that efficiently transfers this capability across different diagram styles.
	\item We show that our final VLM, incorporating GeoCLIP-DA, effectively interprets diverse diagram styles and achieves state-of-the-art performance on the MathVerse benchmark, surpassing existing specialized PGPS models and generalist VLM models.
\end{itemize}
\fi

\iffalse

Large language models (LLMs) have made significant strides in automated math word problem solving. In particular, their code-generation capabilities combined with proof assistants~\citep{lean,isabelle} help minimize computational errors~\citep{POT}, improve solution precision~\citep{autoformalization}, and offer rigorous feedback and evaluation~\citep{MATH}. Although LLMs excel in generating solution steps and correct answers for algebra and calculus~\citep{math_solving}, their uni-modal nature limits performance in domains like plane geometry, where both diagrams and text are vital.

Plane geometry problem solving (PGPS) tasks typically include diagrams and textual descriptions, requiring solvers to interpret premises from both sources. To facilitate automated solutions for these problems, several studies have introduced formal languages tailored for plane geometry to represent solution steps as a program with training datasets composed of diagrams, textual descriptions, and solution programs~\citep{geoqa,unigeo,intergps,pgps}. Building on these datasets, a number of PGPS specialized vision-language models (VLMs) have been developed so far~\citep{GOLD, LANS, geox}.

Most existing VLMs, however, fail to use diagrams when solving geometry problems. Well-known PGPS datasets such as GeoQA~\citep{geoqa}, UniGeo~\citep{unigeo}, and PGPS9K~\citep{pgps}, can be solved without accessing diagrams, as their problem descriptions often contain all geometric information. \cref{fig:pgps_examples} shows an example from GeoQA and PGPS9K datasets, where one can deduce the solution steps without knowing the diagrams. 
As a result, models trained on these datasets rely almost exclusively on textual information, leaving the vision encoder under-utilized~\citep{GOLD}. 
Consequently, the VLMs trained on these datasets cannot solve the plane geometry problem when necessary geometric properties or relations are excluded from the problem statement.

Some studies seek to enhance the recognition of geometric premises from a diagram by directly predicting the premises from the diagram~\citep{GOLD, intergps} or as an auxiliary task for vision encoders~\citep{geoqa,geoqa-plus}. However, these approaches remain highly domain-specific because the labels for training are difficult to obtain, thus limiting generalization across different domains. While self-supervised learning (SSL) methods that depend exclusively on geometric diagrams, e.g., vector quantized variational auto-encoder (VQ-VAE)~\citep{unimath} and masked auto-encoder (MAE)~\citep{scagps,geox}, have also been explored, the effectiveness of the SSL approaches on recognizing geometric features has not been thoroughly investigated.

We introduce a benchmark constructed with a synthetic data engine to evaluate the effectiveness of SSL approaches in recognizing geometric premises from diagrams. Our empirical results with the proposed benchmark show that the vision encoders trained with SSL methods fail to capture visual \geofeat{}s such as perpendicularity between two lines and angle measure.
Furthermore, we find that the pre-trained vision encoders often used in general-purpose VLMs, e.g., OpenCLIP~\citep{clip} and DinoV2~\citep{dinov2}, fail to recognize geometric premises from diagrams.

To improve the vision encoder for PGPS, we propose \geoclip{}, a model trained with a massive amount of diagram-caption pairs.
Since the amount of diagram-caption pairs in existing benchmarks is often limited, we develop a plane diagram generator that can randomly sample plane geometry problems with the help of existing proof assistant~\citep{alphageometry}.
To make \geoclip{} robust against different styles, we vary the visual properties of diagrams, such as color, font size, resolution, and line width.
We show that \geoclip{} performs better than the other SSL approaches and commonly used vision encoders on the newly proposed benchmark.

Another major challenge in PGPS is developing a domain-agnostic VLM capable of handling multiple PGPS benchmarks. As shown in \cref{fig:pgps_examples}, the main difficulties arise from variations in diagram styles. 
To address the issue, we propose a few-shot domain adaptation technique for \geoclip{} which transfers its visual \geofeat{} perception from the synthetic diagrams to the real-world diagrams efficiently. 

We study the efficacy of the domain adapted \geoclip{} on PGPS when equipped with the language model. To be specific, we compare the VLM with the previous PGPS models on MathVerse~\citep{mathverse}, which is designed to evaluate both the PGPS and visual \geofeat{} perception performance on various domains.
While previous PGPS models are inapplicable to certain types of MathVerse problems, we modify the prediction target and unify the solution program languages of the existing PGPS training data to make our VLM applicable to all types of MathVerse problems.
Results on MathVerse demonstrate that our VLM more effectively integrates diagrammatic information and remains robust under conditions of various diagram styles.

\begin{itemize}
    \item We propose a benchmark to measure the visual \geofeat{} recognition performance of different vision encoders.
    % \item \sh{We introduce geometric CLIP (\geoclip{} and train the VLM equipped with \geoclip{} to predict both solution steps and the numerical measurements of the problem.}
    \item We introduce \geoclip{}, a vision encoder which can accurately recognize visual \geofeat{}s and a few-shot domain adaptation technique which can transfer such ability to different domains efficiently. 
    % \item \sh{We develop our final PGPS model, \geovlm{}, by adapting \geoclip{} to different domains and training with unified languages of solution program data.}
    % We develop a domain-agnostic VLM, namely \geovlm{}, by applying a simple yet effective domain adaptation method to \geoclip{} and training on the refined training data.
    \item We demonstrate our VLM equipped with GeoCLIP-DA effectively interprets diverse diagram styles, achieving superior performance on MathVerse compared to the existing PGPS models.
\end{itemize}

\fi 

% \input{Sections/Self-Affirmed}
% \section{Related Work}
\label{sec:RelatedWork}

Within the realm of geophysical sciences, super-resolution/downscaling is a challenge that scientists continue to tackle. There have been several works involved in downscaling applications such as river mapping \cite{Yin2022}, coastal risk assessment \cite{Rucker2021}, estimating soil moisture from remotely sensed images \cite{Peng2017SoilMoisture} and downscaling of satellite based precipitation estimates \cite{Medrano2023PrecipitationDownscaling} to name a few. We direct the reader to \cite{Karwowska2022SuperResolutionSurvey} for a comprehensive review of satellite based downscaling applications and methods. Pertaining to our objective of downscaling \acp{WFM}, we can draw comparisons with several existing works. 
In what follows, we provide a brief review of functionally adjacent works to contrast the novelty of our proposed model and its role in addressing gaps in literature. 

When it comes to downscaling \ac{WFM}, several works use statistical downscaling techniques. These works downscale images by using statistical techniques that utilize relationships between neighboring water fraction pixels. For instance, \cite{Li2015SRFIM} treat the super-resolution task as a sub-pixel mapping problem, wherein the input fraction of inundated pixels must be exactly mapped to the output patch of inundated pixels. 
% In doing so, they are able to apply a discrete particle swarm optimization method to maximize the Flood Inundation Spatial Dependence Index (FISDI). 
\cite{Wang2019} improved upon these approaches by including a spectral term to fully utilize spectral information from multi spectral remote sensing image band. \cite{Wang2021} on the other hand also include a spectral correlation term to reduce the influence of linear and non-linear imaging conditions. All of these approaches are applied to water fraction obtained via spectral unmixing \cite{wang2013SpectralUnmixing} and are designed to work with multi spectral information from MODIS. However, we develop our model with the intention to be used with water fractions directly derived from the output of satellites. One such example is NOAA/VIIRS whose water fraction extraction method is described in \cite{Li2013VIIRSWFM}. \cite{Li2022VIIRSDownscaling} presented a work wherein \ac{WFM} at 375-m flood products from VIIRS were downscaled 30-m flood event and depth products by expressing the inundation mechanism as a function of the \ac{DEM}-based water area and the VIIRS water area.

On the other hand, the non-linear nature of the mapping task lends itself to the use of neural networks. Several models have been adapted from traditional single image digital super-resolution in computer vision literature \cite{sdraka2022DL4downscalingRemoteSensing}. Existing deep learning models in single image super-resolution are primarily dominated by \ac{CNN} based models. Specifically, there has been an upward trend in residual learning models. \acp{RDN} \cite{Zhang2018ResidualDenseSuperResolution} introduced residual dense blocks that employed a contiguous memory mechanism that aimed to overcome the inability of very deep \acp{CNN} to make full use of hierarchical features. 
\acp{RCAN} \cite{Zhang2018RCANSuperResolution} introduced an attention mechanism to exploit the inter-channel dependencies in the intermediate feature transformations. There have also been some works that aim to produce more lightweight \ac{CNN}-based architectures \cite{Zheng2019IMDN,Xiaotong2020LatticeNET}. Since the introduction of the vision transformer \cite{Vaswani2017Attention} that utilized the self-attention mechanism -- originally used for modeling text sequences -- by feeding a sequence 2D sub-image extracted from the original image. Using this approach \cite{LuESRT2022} developed a light-weight and efficient transformer based approach for single image super-resolution. 


For the task of super-resolution of \acp{WFM}, we discuss some works whose methodology is similar to ours even though they differ in their problem setting. \cite{Yin2022} presented a cascaded spectral spatial model for super-resolution of MODIS imagery with a scaling factor 10. Their architecture consists of two stages; first multi-spectral MODIS imagery is converted into a low-resolution \ac{WFM} via spectral unmixing by passing it through a deep stacked residual \ac{CNN}. The second stage involved the super-resolution mapping of these \acp{WFM} using a nested multi-level \ac{CNN} model. Similar to our work, the input fraction images are obtained with zero errors which may not be reflective of reality since there tends to be sensor noise, the spatial distribution of whom cannot be easily estimated. We also note that none of these works directly tackle flood inundation since they've been trained with river map data during non-flood circumstance and \textit{ergo} do not face a data scarcity problem as we do. 
% In this work, apart from the final product of \acp{WFM}, we are not presented with any additional spectral information about the low resolution image. This was intended to work directly with products that can generate \ac{WFM} either directly (VIIRS) or indirectly (Landsat).
\cite{Jia2019} used a deep \ac{CNN} for land mapping that consists of several classes such as building, low vegetation, background and trees. 
\cite{Kumar2021} similarly employ a \ac{CNN} based model for downscaling of summer monsoon rainfall data over the Indian subcontinent. Their proposed Super-Resolution Convolutional Neural Network (SRCNN) has a downscaling factor of 4. 
\cite{Shang2022} on the other hand, proposed a super-resolution mapping technique using Generative Adversarial Networks (GANs). They first generate high resolution fractional images, somewhat analogous to our \ac{WFM}, and are then mapped to categorical land cover maps involving forest, urban, agriculture and water classes. 
\cite{Qin2020} interestingly approach lake area super-resolution for Landsat and MODIS data as an unsupervised problem using a \ac{CNN} and are able to extend to other scaling factors. \cite{AristizabalInundationMapping2020} performed flood inundation mapping using \ac{SAR} data obtained from Sentinel-1. They showed that \ac{DEM}-based features helped to improve \ac{SAR}-based predictions for quadratic discriminant analysis, support vector machines and k-nearest neighbor classifiers. While almost all of the aforementioned works can be adapted to our task. We stand out in the following ways (i) We focus on downscaling of \acp{WFM} directly, \textit{i.e.,} we do not focus on the algorithm to compute the \ac{WFM} from multi-channel satellite data and (ii) We focus on producing high resolution maps only for instances of flood inundation. The latter point produces a data scarcity issue which we seek to remedy with synthetic data. 


%%%%%%%%%%%%%%%%% Additional unused information %%%%%%%%%%%%%%%%


%     \item[\cite{Wang2021}] Super-Resolution Mapping Based on Spatial–Spectral Correlation for Spectral Imagery
%     \begin{itemize}
%         \item Not a deep neural network approach. SRM based on spatial–spectral correlation (SSC) is proposed in order to overcome the influence of linear and nonlinear imaging conditions and utilize more accurate spectral properties.
%         \item (fig 1) there are two main SRM types: (1) the initialization-then-optimization SRM, where the class labels are allocated randomly to subpixels, and the location of each subpixel is optimized to obtain the final SRM result. and (2)soft-then-hard SRM, which involves two steps: the subpixel sharpening and the class allocation.  
%         \item SSC procedures: (1) spatial correlation is performed by the MSAM to reduce the influences of linear imaging conditions on image quality. (2) A spectral correlation that utilizes spectral properties based on the nonlinear KLD is proposed to reduce the influences of nonlinear imaging conditions. (3) spatial and spectral correlations are then combined to obtain an optimization function with improved linear and nonlinear performances. And finally (4) by maximizing the optimization function, a class allocation method based on the SA is used to assign LC labels to each subpixel, obtaining the final SRM result.
%         \item (Comparable) 
%     \end{itemize}
%     %--------------------------------------------------------------------
% \cite{Wang2021} account for the influence of linear and non-linear imaging conditions by involving more accurate spectral properties. 
%     %--------------------------------------------------------------------
%     \item[\cite{Yin2022}] A Cascaded Spectral–Spatial CNN Model for Super-Resolution River Mapping With MODIS Imagery
%     \begin{itemize}
%         \item produce  Landsat-like  fine-resolution (scale of 10)  river  maps  from  MODIS images. Notice the original coarse-resolution remotely sensed images, not the river fraction images.
%         \item combined  CNN  model that  contains  a spectral  unmixing  module  and  an  SRM  module, and the SRM module is made up of an encoder and a decoder that are connected through a series of convolutional blocks. 
%         \item With an adaptive cross-entropy loss function to address class imbalance.	
%         \item The overall accuracy, the omission error, the  commission  error,  and  the  mean  intersection  over  union (MIOU)  calculated  to  assess  the results.
%         \item partially comparable with ours, only the SRM module part
%     %--------------------------------------------------------------------

% To decouple the description of the objective and the \ac{ML} model architecture, the motivation for the model architecture is described in \secref{sec:Methodology}. 


%     \item[\cite{Wang2019}] Improving Super-Resolution Flood Inundation Mapping for Multi spectral Remote Sensing Image by Supplying More Spectral 
%     \begin{itemize}
%         \item proposed the SRFIM-MSI,where a new spectral term is added to the traditional SRFIM to fully utilize the spectral information from multi spectral remote sensing image band. 
%         \item The original SRFIM \cite{Huang2014, Li2015} obtains the sub pixel spatial distribution of flood inundation within mixed pixels by maximizing their spatial correlation while maintaining the original proportions of flood inundation within the mixed pixels. The SRFIM is formulated as a maximum combined optimization issue according to the principle of spatial correlation.
%         \item follow the terminology in \cite{Wang2021}, this is an initialization-then-optimization SRM. 
%         \item (Comparable) 
%     \end{itemize}
%     %--------------------------------------------------------------------


%--------------------------------------------------------------------
%     \item[\cite{Jia2019}] Super-Resolution Land Cover Mapping Based on the Convolutional Neural Network
%     \begin{itemize}
%         \item SRMCNN (Super-resolution mapping CNN) is proposed to obtain fine-scale land cover maps from coarse remote sensing images. Specifically, an encoder-decoder CNN is used to determined the labels (i.e., land cover classes) of the subpixels within mixed pixels.
%         \item There were three main parts in SRMCNN. The first part was a three-sequential convolutional layer with ReLU and pooling. The second part is up-sampling, for which a multi transposed-convolutional layer was adopted. To keep the feature learned in the previous layer, a skip connection was used to concatenate the output of the corresponding convolution layer. The last part was the softmax classifier, in which the feature in the antepenultimate layer was classified and class probabilities are obtained.
%         \item The loss: the optimal allocation of classes to the subpixels of mixed pixel is achieved by maximizing the spatial dependence between neighbor pixels under constraint that the class proportions within the mixed pixels are preserved.
%         \item (Preferred), this paper is designed to classify background, Building, Low Vegetation, or Tree in the land. But we can easily adapt to our problem and should compare with this paper.
%     \end{itemize}
%     %--------------------------------------------------------------------

%     \item[\cite{Kumar2021}] Deep learning–based downscaling of summer monsoon rainfall data over Indian region
%     \begin{itemize}
%         \item down-scaling (scale of 4) rainfall data. The output image is not binary image.
%         \item three algorithms: SRCNN, stacked SRCNN, and DeepSD are employed, based on \cite{Vandal2019}
%         \item mean square error and pattern correlation coefficient are used as evaluation metrics.
%         \item SRCNN: super-resolution-based convolutional neural networks (SRCNN) first upgrades the low-resolution image to the higher resolution size by using bicubic interpolation. Suppose the interpolated image is referred to as Y; SRCNNs’task is to retrieve from Y an image F(Y) which is close to the high-resolution ground truth image X.
%         \item stacked SRCNN: stack 2 or more SRCNN blocks to increasing the scaling factor.
%         \item DeepSD: uses topographies as an additional input to stacked SRCNN.
%         \item These algorithms are not designed for binary output images, but if prefer, the ``modified'' stacked SRCNN or DeepSD can be used as baseline algorithms.
%     
%     \item[\cite{Shang2022}] Super resolution Land Cover Mapping Using a Generative Adversarial Network
%     \begin{itemize}
%         \item propose an end-to-end SRM model based on a generative adversarial network (GAN), that is, GAN-SRM, to improve the two-step learning-based SRM methods. 
%         \item Two-step SRM method: The first step is fraction-image super-resolution (SR), which reconstructs a high-spatial-resolution fraction image from the low input, methods like SVR, or CNN has been widely adopted. The second step is converting the high-resolution fraction images to a categorical land cover map, such as with a soft-max function to assign each high-resolution pixel to a unique category value.
%         \item The proposed GAN-SRM model includes a generative network and a discriminative network, so that both the fraction-image SR and the conversion of the fraction images to categorical map steps are fully integrated to reduce the resultant uncertainty. 
%         \item applied to the National Land Cover Database (NLCD), which categorized land into four typical classes:forest, urban, agriculture,and water. scale factor of 8. 
%         \item (Preferred), we should compare with this work.
%     \end{itemize}
%     %--------------------------------------------------------------------

%   \item[\cite{Qin2020}] Achieving Higher Resolution Lake Area from Remote Sensing Images Through an Unsupervised Deep Learning Super-Resolution Method
%   \begin{itemize}
%       \item propose an unsupervised deep gradient network (UDGN) to generate a higher resolution lake area from remote sensing images.
%       \item UDGN models the internal recurrence of information inside the single image and its corresponding gradient map to generate images with higher spatial resolution. 
%       \item A single image super-resolution approach, not comparable
%   \end{itemize}
%     %--------------------------------------------------------------------




%     \item[\cite{Demiray2021}] D-SRGAN: DEM Super-Resolution with Generative Adversarial Networks
%     \begin{itemize}
%         \item A GAN based model is proposed to increase the spatial resolution of a given DEM dataset up to 4 times without additional information related to data.
%         \item Rather than processing each image in a sequence independently, our generator architecture uses a recurrent layer to update the state of the high-resolution reconstruction in a manner that is consistent with both the previous state and the newly received data. The recurrent layer can thus be understood as performing a Bayesian update on the ensemble member, resembling an ensemble Kalman filter. 
%         \item A single image super-resolution approach, not comparable
%     \end{itemize}
%     %--------------------------------------------------------------------
%     \item[\cite{Leinonen2021}] Stochastic Super-Resolution for Downscaling Time-Evolving Atmospheric Fields With a Generative Adversarial Network
%     \begin{itemize}
%         \item propose a super-resolution GAN that operates on sequences of two-dimensional images and creates an ensemble of predictions for each input. The spread between the ensemble members represents the uncertainty of the super-resolution reconstruction.
%         \item for sequence of input images, not comparable with ours.
%     \end{itemize} 
%     %--------------------------------------------------------------------

% \end{itemize}




% \input{Sections/EmpiricalStudySetup}
% \input{Sections/ExperimentalResults}
% \section{ Task Generalization Beyond i.i.d. Sampling and Parity Functions
}\label{sec:Discussion}
% Discussion: From Theory to Beyond
% \misha{what is beyond?}
% \amir{we mean two things: in the first subsection beyond i.i.d subsampling of parity tasks and in the second subsection beyond parity task}
% \misha{it has to be beyond something, otherwise it is not clear what it is about} \hz{this is suggested by GPT..., maybe can be interpreted as from theory to beyond theory. We can do explicit like Discussion: Beyond i.i.d. task sampling and the Parity Task}
% \misha{ why is "discussion" in the title?}\amir{Because it is a discussion, it is not like separate concrete explnation about why these thing happens or when they happen, they just discuss some interesting scenraios how it relates to our theory.   } \misha{it is not really a discussion -- there is a bunch of experiments}

In this section, we extend our experiments beyond i.i.d. task sampling and parity functions. We show an adversarial example where biased task selection substantially hinders task generalization for sparse parity problem. In addition, we demonstrate that exponential task scaling extends to a non-parity tasks including arithmetic and multi-step language translation.

% In this section, we extend our experiments beyond i.i.d. task sampling and parity functions. On the one hand, we find that biased task selection can significantly degrade task generalization; on the other hand, we show that exponential task scaling generalizes to broader scenarios.
% \misha{we should add a sentence or two giving more detail}


% 1. beyond i.i.d tasks sampling
% 2. beyond parity -> language, arithmetic -> task dependency + implicit bias of transformer (cannot implement this algorithm for arithmatic)



% In this section, we emphasize the challenge of quantifying the level of out-of-distribution (OOD) differences between training tasks and testing tasks, even for a simple parity task. To illustrate this, we present two scenarios where tasks differ between training and testing. For each scenario, we invite the reader to assess, before examining the experimental results, which cases might appear “more” OOD. All scenarios consider \( d = 10 \). \kaiyue{this sentence should be put into 5.1}






% for parity problem




% \begin{table*}[th!]
%     \centering
%     \caption{Generalization Results for Scenarios 1 and 2 for $d=10$.}
%     \begin{tabular}{|c|c|c|c|}
%         \hline
%         \textbf{Scenario} & \textbf{Type/Variation} & \textbf{Coordinates} & \textbf{Generalization accuracy} \\
%         \hline
%         \multirow{3}{*}{Generalization with Missing Pair} & Type 1 & \( c_1 = 4, c_2 = 6 \) & 47.8\%\\ 
%         & Type 2 & \( c_1 = 4, c_2 = 6 \) & 96.1\%\\ 
%         & Type 3 & \( c_1 = 4, c_2 = 6 \) & 99.5\%\\ 
%         \hline
%         \multirow{3}{*}{Generalization with Missing Pair} & Type 1 &  \( c_1 = 8, c_2 = 9 \) & 40.4\%\\ 
%         & Type 2 & \( c_1 = 8, c_2 = 9 \) & 84.6\% \\ 
%         & Type 3 & \( c_1 = 8, c_2 = 9 \) & 99.1\%\\ 
%         \hline
%         \multirow{1}{*}{Generalization with Missing Coordinate} & --- & \( c_1 = 5 \) & 45.6\% \\ 
%         \hline
%     \end{tabular}
%     \label{tab:generalization_results}
% \end{table*}

\subsection{Task Generalization Beyond i.i.d. Task Sampling }\label{sec: Experiment beyond iid sampling}

% \begin{table*}[ht!]
%     \centering
%     \caption{Generalization Results for Scenarios 1 and 2 for $d=10, k=3$.}
%     \begin{tabular}{|c|c|c|}
%         \hline
%         \textbf{Scenario}  & \textbf{Tasks excluded from training} & \textbf{Generalization accuracy} \\
%         \hline
%         \multirow{1}{*}{Generalization with Missing Pair}
%         & $\{4,6\} \subseteq \{s_1, s_2, s_3\}$ & 96.2\%\\ 
%         \hline
%         \multirow{1}{*}{Generalization with Missing Coordinate}
%         & \( s_2 = 5 \) & 45.6\% \\ 
%         \hline
%     \end{tabular}
%     \label{tab:generalization_results}
% \end{table*}




In previous sections, we focused on \textit{i.i.d. settings}, where the set of training tasks $\mathcal{F}_{train}$ were sampled uniformly at random from the entire class $\mathcal{F}$. Here, we explore scenarios that deliberately break this uniformity to examine the effect of task selection on out-of-distribution (OOD) generalization.\\

\textit{How does the selection of training tasks influence a model’s ability to generalize to unseen tasks? Can we predict which setups are more prone to failure?}\\

\noindent To investigate this, we consider two cases parity problems with \( d = 10 \) and \( k = 3 \), where each task is represented by its tuple of secret indices \( (s_1, s_2, s_3) \):

\begin{enumerate}[leftmargin=0.4 cm]
    \item \textbf{Generalization with a Missing Coordinate.} In this setup, we exclude all training tasks where the second coordinate takes the value \( s_2 = 5 \), such as \( (1,5,7) \). At test time, we evaluate whether the model can generalize to unseen tasks where \( s_2 = 5 \) appears.
    \item \textbf{Generalization with Missing Pair.} Here, we remove all training tasks that contain both \( 4 \) \textit{and} \( 6 \) in the tuple \( (s_1, s_2, s_3) \), such as \( (2,4,6) \) and \( (4,5,6) \). At test time, we assess whether the model can generalize to tasks where both \( 4 \) and \( 6 \) appear together.
\end{enumerate}

% \textbf{Before proceeding, consider the following question:} 
\noindent \textbf{If you had to guess.} Which scenario is more challenging for generalization to unseen tasks? We provide the experimental result in Table~\ref{tab:generalization_results}.

 % while the model struggles for one of them while as it generalizes almost perfectly in the other one. 

% in the first scenario, it generalizes almost perfectly in the second. This highlights how exposure to partial task structures can enhance generalization, even when certain combinations are entirely absent from the training set. 

In the first scenario, despite being trained on all tasks except those where \( s_2 = 5 \), which is of size $O(\d^T)$, the model struggles to generalize to these excluded cases, with prediction at chance level. This is intriguing as one may expect model to generalize across position. The failure  suggests that positional diversity plays a crucial role in the task generalization of Transformers. 

In contrast, in the second scenario, though the model has never seen tasks with both \( 4 \) \textit{and} \( 6 \) together, it has encountered individual instances where \( 4 \) appears in the second position (e.g., \( (1,4,5) \)) or where \( 6 \) appears in the third position (e.g., \( (2,3,6) \)). This exposure appears to facilitate generalization to test cases where both \( 4 \) \textit{and} \( 6 \) are present. 



\begin{table*}[t!]
    \centering
    \caption{Generalization Results for Scenarios 1 and 2 for $d=10, k=3$.}
    \resizebox{\textwidth}{!}{  % Scale to full width
        \begin{tabular}{|c|c|c|}
            \hline
            \textbf{Scenario}  & \textbf{Tasks excluded from training} & \textbf{Generalization accuracy} \\
            \hline
            Generalization with Missing Pair & $\{4,6\} \subseteq \{s_1, s_2, s_3\}$ & 96.2\%\\ 
            \hline
            Generalization with Missing Coordinate & \( s_2 = 5 \) & 45.6\% \\ 
            \hline
        \end{tabular}
    }
    \label{tab:generalization_results}
\end{table*}

As a result, when the training tasks are not i.i.d, an adversarial selection such as exclusion of specific positional configurations may lead to failure to unseen task generalization even though the size of $\mathcal{F}_{train}$ is exponentially large. 


% \paragraph{\textbf{Key Takeaways}}
% \begin{itemize}
%     \item Out-of-distribution generalization in the parity problem is highly sensitive to the diversity and positional coverage of training tasks.
%     \item Adversarial exclusion of specific pairs or positional configurations can lead to systematic failures, even when most tasks are observed during training.
% \end{itemize}




%################ previous veriosn down
% \textit{How does the choice of training tasks affect the ability of a model to generalize to unseen tasks? Can we predict which setups are likely to lead to failure?}

% To explore these questions, we crafted specific training and test task splits to investigate what makes one setup appear “more” OOD than another.

% \paragraph{Generalization with Missing Pair.}

% Imagine we have tasks constructed from subsets of \(k=3\) elements out of a larger set of \(d\) coordinates. What happens if certain pairs of coordinates are adversarially excluded during training? For example, suppose \(d=5\) and two specific coordinates, \(c_1 = 1\) and \(c_2 = 2\), are excluded. The remaining tasks are formed from subsets of the other coordinates. How would a model perform when tested on tasks involving the excluded pair \( (c_1, c_2) \)? 

% To probe this, we devised three variations in how training tasks are constructed:
%     \begin{enumerate}
%         \item \textbf{Type 1:} The training set includes all tasks except those containing both \( c_1 = 1 \) and \( c_2 = 2 \). 
%         For this example, the training set includes only $\{(3,4,5)\}$. The test set consists of all tasks containing the rest of tuples.

%         \item \textbf{Type 2:} Similar to Type 1, but the training set additionally includes half of the tasks containing either \( c_1 = 1 \) \textit{or} \( c_2 = 2 \) (but not both). 
%         For the example, the training set includes all tasks from Type 1 and adds tasks like \(\{(1, 3, 4), (2, 3, 5)\}\) (half of those containing \( c_1 = 1 \) or \( c_2 = 2 \)).

%         \item \textbf{Type 3:} Similar to Type 2, but the training set also includes half of the tasks containing both \( c_1 = 1 \) \textit{and} \( c_2 = 2 \). 
%         For the example, the training set includes all tasks from Type 2 and adds, for instance, \(\{(1, 2, 5)\}\) (half of the tasks containing both \( c_1 \) and \( c_2 \)).
%     \end{enumerate}

% By systematically increasing the diversity of training tasks in a controlled way, while ensuring no overlap between training and test configurations, we observe an improvement in OOD generalization. 

% % \textit{However, the question is this improvement similar across all coordinate pairs, or does it depend on the specific choices of \(c_1\) and \(c_2\) in the tasks?} 

% \textbf{Before proceeding, consider the following question:} Is the observed improvement consistent across all coordinate pairs, or does it depend on the specific choices of \(c_1\) and \(c_2\) in the tasks? 

% For instance, consider two cases for \(d = 10, k = 3\): (i) \(c_1 = 4, c_2 = 6\) and (ii) \(c_1 = 8, c_2 = 9\). Would you expect similar OOD generalization behavior for these two cases across the three training setups we discussed?



% \paragraph{Answer to the Question.} for both cases of \( c_1, c_2 \), we observe that generalization fails in Type 1, suggesting that the position of the tasks the model has been trained on significantly impacts its generalization capability. For Type 2, we find that \( c_1 = 4, c_2 = 6 \) performs significantly better than \( c_1 = 8, c_2 = 9 \). 

% Upon examining the tasks where the transformer fails for \( c_1 = 8, c_2 = 9 \), we see that the model only fails at tasks of the form \((*, 8, 9)\) while perfectly generalizing to the rest. This indicates that the model has never encountered the value \( 8 \) in the second position during training, which likely explains its failure to generalize. In contrast, for \( c_1 = 4, c_2 = 6 \), while the model has not seen tasks of the form \((*, 4, 6)\), it has encountered tasks where \( 4 \) appears in the second position, such as \((1, 4, 5)\), and tasks where \( 6 \) appears in the third position, such as \((2, 3, 6)\). This difference may explain why the model generalizes almost perfectly in Type 2 for \( c_1 = 4, c_2 = 6 \), but not for \( c_1 = 8, c_2 = 9 \).



% \paragraph{Generalization with Missing Coordinates.}
% Next, we investigate whether a model can generalize to tasks where a specific coordinate appears in an unseen position during training. For instance, consider \( c_1 = 5 \), and exclude all tasks where \( c_1 \) appears in the second position. Despite being trained on all other tasks, the model fails to generalize to these excluded cases, highlighting the importance of positional diversity in training tasks.



% \paragraph{Key Takeaways.}
% \begin{itemize}
%     \item OOD generalization depends heavily on the diversity and positional coverage of training tasks for the parity problem.
%     \item adversarial exclusion of specific pairs or positional configurations in the parity problem can lead to failure, even when the majority of tasks are observed during training.
% \end{itemize}


%################ previous veriosn up

% \paragraph{Key Takeaways} These findings highlight the complexity of OOD generalization, even in seemingly simple tasks like parity. They also underscore the importance of task design: the diversity of training tasks can significantly influence a model’s ability to generalize to unseen tasks. By better understanding these dynamics, we can design more robust training regimes that foster generalization across a wider range of scenarios.


% #############


% Upon examining the tasks where the transformer fails for \( c_1 = 8, c_2 = 9 \), we see that the model only fails at tasks of the form \((*, 8, 9)\) while perfectly generalizing to the rest. This indicates that the model has never encountered the value \( 8 \) in the second position during training, which likely explains its failure to generalize. In contrast, for \( c_1 = 4, c_2 = 6 \), while the model has not seen tasks of the form \((*, 4, 6)\), it has encountered tasks where \( 4 \) appears in the second position, such as \((1, 4, 5)\), and tasks where \( 6 \) appears in the third position, such as \((2, 3, 6)\). This difference may explain why the model generalizes almost perfectly in Type 2 for \( c_1 = 4, c_2 = 6 \), but not for \( c_1 = 8, c_2 = 9 \).

% we observe a striking pattern: generalization fails entirely in Type 1, regardless of the coordinate pair (\(c_1, c_2\)). However, in Type 2, generalization varies: \(c_1 = 4, c_2 = 6\) achieves 96\% accuracy, while \(c_1 = 8, c_2 = 9\) lags behind at 70\%. Why? Upon closer inspection, the model struggles specifically with tasks like \((*, 8, 9)\), where the combination \(c_1 = 8\) and \(c_2 = 9\) is entirely novel. In contrast, for \(c_1 = 4, c_2 = 6\), the model benefits from having seen tasks where \(4\) appears in the second position or \(6\) in the third. This suggests that positional exposure during training plays a key role in generalization.

% To test whether task structure influences generalization, we consider two variations:
% \begin{enumerate}
%     \item \textbf{Sorted Tuples:} Tasks are always sorted in ascending order.
%     \item \textbf{Unsorted Tuples:} Tasks can appear in any order.
% \end{enumerate}

% If the model struggles with generalizing to the excluded position, does introducing variability through unsorted tuples help mitigate this limitation?

% \paragraph{Discussion of Results}

% In \textbf{Generalization with Missing Pairs}, we observe a striking pattern: generalization fails entirely in Type 1, regardless of the coordinate pair (\(c_1, c_2\)). However, in Type 2, generalization varies: \(c_1 = 4, c_2 = 6\) achieves 96\% accuracy, while \(c_1 = 8, c_2 = 9\) lags behind at 70\%. Why? Upon closer inspection, the model struggles specifically with tasks like \((*, 8, 9)\), where the combination \(c_1 = 8\) and \(c_2 = 9\) is entirely novel. In contrast, for \(c_1 = 4, c_2 = 6\), the model benefits from having seen tasks where \(4\) appears in the second position or \(6\) in the third. This suggests that positional exposure during training plays a key role in generalization.

% In \textbf{Generalization with Missing Coordinates}, the results confirm this hypothesis. When \(c_1 = 5\) is excluded from the second position during training, the model fails to generalize to such tasks in the sorted case. However, allowing unsorted tuples introduces positional diversity, leading to near-perfect generalization. This raises an intriguing question: does the model inherently overfit to positional patterns, and can task variability help break this tendency?




% In this subsection, we show that the selection of training tasks can affect the quality of the unseen task generalization significantly in practice. To illustrate this, we present two scenarios where tasks differ between training and testing. For each scenario, we invite the reader to assess, before examining the experimental results, which cases might appear “more” OOD. 

% % \amir{add examples, }

% \kaiyue{I think the name of scenarios here are not very clear}
% \begin{itemize}
%     \item \textbf{Scenario 1:  Generalization Across Excluded Coordinate Pairs (\( k = 3 \))} \\
%     In this scenario, we select two coordinates \( c_1 \) and \( c_2 \) out of \( d \) and construct three types of training sets. 

%     Suppose \( d = 5 \), \( c_1 = 1 \), and \( c_2 = 2 \). The tuples are all possible subsets of \( \{1, 2, 3, 4, 5\} \) with \( k = 3 \):
%     \[
%     \begin{aligned}
%     \big\{ & (1, 2, 3), (1, 2, 4), (1, 2, 5), (1, 3, 4), (1, 3, 5), \\
%            & (1, 4, 5), (2, 3, 4), (2, 3, 5), (2, 4, 5), (3, 4, 5) \big\}.
%     \end{aligned}
%     \]

%     \begin{enumerate}
%         \item \textbf{Type 1:} The training set includes all tuples except those containing both \( c_1 = 1 \) and \( c_2 = 2 \). 
%         For this example, the training set includes only $\{(3,4,5)\}$ tuple. The test set consists of tuples containing the rest of tuples.

%         \item \textbf{Type 2:} Similar to Type 1, but the training set additionally includes half of the tuples containing either \( c_1 = 1 \) \textit{or} \( c_2 = 2 \) (but not both). 
%         For the example, the training set includes all tuples from Type 1 and adds tuples like \(\{(1, 3, 4), (2, 3, 5)\}\) (half of those containing \( c_1 = 1 \) or \( c_2 = 2 \)).

%         \item \textbf{Type 3:} Similar to Type 2, but the training set also includes half of the tuples containing both \( c_1 = 1 \) \textit{and} \( c_2 = 2 \). 
%         For the example, the training set includes all tuples from Type 2 and adds, for instance, \(\{(1, 2, 5)\}\) (half of the tuples containing both \( c_1 \) and \( c_2 \)).
%     \end{enumerate}

% % \begin{itemize}
% %     \item \textbf{Type 1:} The training set includes tuples \(\{1, 3, 4\}, \{2, 3, 4\}\) (excluding tuples with both \( c_1 \) and \( c_2 \): \(\{1, 2, 3\}, \{1, 2, 4\}\)). The test set contains the excluded tuples.
% %     \item \textbf{Type 2:} The training set includes all tuples in Type 1 plus half of the tuples containing either \( c_1 = 1 \) or \( c_2 = 2 \) (e.g., \(\{1, 2, 3\}\)).
% %     \item \textbf{Type 3:} The training set includes all tuples in Type 2 plus half of the tuples containing both \( c_1 = 1 \) and \( c_2 = 2 \) (e.g., \(\{1, 2, 4\}\)).
% % \end{itemize}
    
%     \item \textbf{Scenario 2: Scenario 2: Generalization Across a Fixed Coordinate (\( k = 3 \))} \\
%     In this scenario, we select one coordinate \( c_1 \) out of \( d \) (\( c_1 = 5 \)). The training set includes all task tuples except those where \( c_1 \) is the second coordinate of the tuple. For this scenario, we examine two variations:
%     \begin{enumerate}
%         \item \textbf{Sorted Tuples:} Task tuples are always sorted (e.g., \( (x_1, x_2, x_3) \) with \( x_1 \leq x_2 \leq x_3 \)).
%         \item \textbf{Unsorted Tuples:} Task tuples can appear in any order.
%     \end{enumerate}
% \end{itemize}




% \paragraph{Discussion of Results.} In the first scenario, for both cases of \( c_1, c_2 \), we observe that generalization fails in Type 1, suggesting that the position of the tasks the model has been trained on significantly impacts its generalization capability. For Type 2, we find that \( c_1 = 4, c_2 = 6 \) performs significantly better than \( c_1 = 8, c_2 = 9 \). 

% Upon examining the tasks where the transformer fails for \( c_1 = 8, c_2 = 9 \), we see that the model only fails at tasks of the form \((*, 8, 9)\) while perfectly generalizing to the rest. This indicates that the model has never encountered the value \( 8 \) in the second position during training, which likely explains its failure to generalize. In contrast, for \( c_1 = 4, c_2 = 6 \), while the model has not seen tasks of the form \((*, 4, 6)\), it has encountered tasks where \( 4 \) appears in the second position, such as \((1, 4, 5)\), and tasks where \( 6 \) appears in the third position, such as \((2, 3, 6)\). This difference may explain why the model generalizes almost perfectly in Type 2 for \( c_1 = 4, c_2 = 6 \), but not for \( c_1 = 8, c_2 = 9 \).

% This position-based explanation appears compelling, so in the second scenario, we focus on a single position to investigate further. Here, we find that the transformer fails to generalize to tasks where \( 5 \) appears in the second position, provided it has never seen any such tasks during training. However, when we allow for more task diversity in the unsorted case, the model achieves near-perfect generalization. 

% This raises an important question: does the transformer have a tendency to overfit to positional patterns, and does introducing more task variability, as in the unsorted case, prevent this overfitting and enable generalization to unseen positional configurations?

% These findings highlight that even in a simple task like parity, it is remarkably challenging to understand and quantify the sources and levels of OOD behavior. This motivates further investigation into the nuances of task design and its impact on model generalization.


\subsection{Task Generalization Beyond Parity Problems}

% \begin{figure}[t!]
%     \centering
%     \includegraphics[width=0.45\textwidth]{Figures/arithmetic_v1.png}
%     \vspace{-0.3cm}
%     \caption{Task generalization for arithmetic task with CoT, it has $\d =2$ and $T = d-1$ as the ambient dimension, hence $D\ln(DT) = 2\ln(2T)$. We show that the empirical scaling closely follows the theoretical scaling.}
%     \label{fig:arithmetic}
% \end{figure}



% \begin{wrapfigure}{r}{0.4\textwidth}  % 'r' for right, 'l' for left
%     \centering
%     \includegraphics[width=0.4\textwidth]{Figures/arithmetic_v1.png}
%     \vspace{-0.3cm}
%     \caption{Task generalization for the arithmetic task with CoT. It has $d =2$ and $T = d-1$ as the ambient dimension, hence $D\ln(DT) = 2\ln(2T)$. We show that the empirical scaling closely follows the theoretical scaling.}
%     \label{fig:arithmetic}
% \end{wrapfigure}

\subsubsection{Arithmetic Task}\label{subsec:arithmetic}











We introduce the family of \textit{Arithmetic} task that, like the sparse parity problem, operates on 
\( d \) binary inputs \( b_1, b_2, \dots, b_d \). The task involves computing a structured arithmetic expression over these inputs using a sequence of addition and multiplication operations.
\newcommand{\op}{\textrm{op}}

Formally, we define the function:
\[
\text{Arithmetic}_{S} \colon \{0,1\}^d \to \{0,1,\dots,d\},
\]
where \( S = (\op_1, \op_2, \dots, \op_{d-1}) \) is a sequence of \( d-1 \) operations, each \( \op_k \) chosen from \( \{+, \times\} \). The function evaluates the expression by applying the operations sequentially from left-to-right order: for example, if \( S = (+, \times, +) \), then the arithmetic function would compute:
\[
\text{Arithmetic}_{S}(b_1, b_2, b_3, b_4) = ((b_1 + b_2) \times b_3) + b_4.
\]
% Thus, the sequence of operations \( S \) defines how the binary inputs are combined to produce an integer output between \( 0 \) and \( d \).
% \[
% \text{Arithmetic}_{S} 
% (b_1,\,b_2,\,\dots,b_d)
% =
% \Bigl(\dots\bigl(\,(b_1 \;\op_1\; b_2)\;\op_2\; b_3\bigr)\,\dots\Bigr) 
% \;\op_{d-1}\; b_d.
% \]
% We now introduce an \emph{Arithmetic} task that, like the sparse parity problem, operates on $d$ binary inputs $b_1, b_2, \dots, b_d$. Specifically, we define an arithmetic function
% \[
% \text{Arithmetic}_{S}\colon \{0,1\}^d \;\to\; \{0,1,\dots,d\},
% \]
% where $S = (i_1, i_2, \dots, i_{d-1})$ is a sequence of $d-1$ operations, each $i_k \in \{+,\,\times\}$. The value of $\text{Arithmetic}_{S}$ is obtained by applying the prescribed addition and multiplication operations in order, namely:
% \[
% \text{Arithmetic}_{S}(b_1,\,b_2,\,\dots,b_d)
% \;=\;
% \Bigl(\dots\bigl(\,(b_1 \;i_1\; b_2)\;i_2\; b_3\bigr)\,\dots\Bigr) 
% \;i_{d-1}\; b_d.
% \]

% This is an example of our framework where $T = d-1$ and $|\Theta_t| = 2$ with total $2^d$ possible tasks. 




By introducing a step-by-step CoT, arithmetic class belongs to $ARC(2, d-1)$: this is because at every step, there is only $\d = |\Theta_t| = 2$ choices (either $+$ or $\times$) while the length is  $T = d-1$, resulting a total number of $2^{d-1}$ tasks. 


\begin{minipage}{0.5\textwidth}  % Left: Text
    Task generalization for the arithmetic task with CoT. It has $d =2$ and $T = d-1$ as the ambient dimension, hence $D\ln(DT) = 2\ln(2T)$. We show that the empirical scaling closely follows the theoretical scaling.
\end{minipage}
\hfill
\begin{minipage}{0.4\textwidth}  % Right: Image
    \centering
    \includegraphics[width=\textwidth]{Figures/arithmetic_v1.png}
    \refstepcounter{figure}  % Manually advances the figure counter
    \label{fig:arithmetic}  % Now this label correctly refers to the figure
\end{minipage}

Notably, when scaling with \( T \), we observe in the figure above that the task scaling closely follow the theoretical $O(D\log(DT))$ dependency. Given that the function class grows exponentially as \( 2^T \), it is truly remarkable that training on only a few hundred tasks enables generalization to an exponentially larger space—on the order of \( 2^{25} > 33 \) Million tasks. This exponential scaling highlights the efficiency of structured learning, where a modest number of training examples can yield vast generalization capability.





% Our theory suggests that only $\Tilde{O}(\ln(T))$ i.i.d training tasks is enough to generalize to the rest of unseen tasks. However, we show in Figure \ref{fig:arithmetic} that transformer is not able to match  that. The transformer out-of distribution generalization behavior is not consistent across different dimensions when we scale the number of training tasks with $\ln(T)$. \hongzhou{implicit bias, optimization, etc}
 






% \subsection{Task generalization Beyond parity problem}

% \subsection{Arithmetic} In this setting, we still use the set-up we introduced in \ref{subsec:parity_exmaple}, the input is still a set of $d$ binary variable, $b_1, b_2,\dots,b_d$ and ${Arithmatic_{S}}:\{0,1\}\rightarrow \{0, 1, \dots, d\}$, where $S = (i_1,i_2,\dots,i_{d-1})$ is a tuple of size $d-1$ where each coordinate is either add($+
% $) or multiplication ($\times$). The function is as following,

% \begin{align*}
%     Arithmatic_{S}(b_1, b_2,\dots,b_d) = (\dots(b1(i1)b2)(i3)b3\dots)(i{d-1})
% \end{align*}
    


\subsubsection{Multi-Step Language Translation Task}

 \begin{figure*}[h!]
    \centering
    \includegraphics[width=0.9\textwidth]{Figures/combined_plot_horiz.png}
    \vspace{-0.2cm}
    \caption{Task generalization for language translation task: $\d$ is the number of languages and $T$ is the length of steps.}
    \vspace{-2mm}
    \label{fig:language}
\end{figure*}
% \vspace{-2mm}

In this task, we study a sequential translation process across multiple languages~\cite{garg2022can}. Given a set of \( D \) languages, we construct a translation chain by randomly sampling a sequence of \( T \) languages \textbf{with replacement}:  \(L_1, L_2, \dots, L_T,\)
where each \( L_t \) is a sampled language. Starting with a word, we iteratively translate it through the sequence:
\vspace{-2mm}
\[
L_1 \to L_2 \to L_3 \to \dots \to L_T.
\]
For example, if the sampled sequence is EN → FR → DE → FR, translating the word "butterfly" follows:
\vspace{-1mm}
\[
\text{butterfly} \to \text{papillon} \to \text{schmetterling} \to \text{papillon}.
\]
This task follows an \textit{AutoRegressive Compositional} structure by itself, specifically \( ARC(D, T-1) \), where at each step, the conditional generation only depends on the target language, making \( D \) as the number of languages and the total number of possible tasks is \( D^{T-1} \). This example illustrates that autoregressive compositional structures naturally arise in real-world languages, even without explicit CoT. 

We examine task scaling along \( D \) (number of languages) and \( T \) (sequence length). As shown in Figure~\ref{fig:language}, empirical  \( D \)-scaling closely follows the theoretical \( O(D \ln D T) \). However, in the \( T \)-scaling case, we observe a linear dependency on \( T \) rather than the logarithmic dependency \(O(\ln T) \). A possible explanation is error accumulation across sequential steps—longer sequences require higher precision in intermediate steps to maintain accuracy. This contrasts with our theoretical analysis, which focuses on asymptotic scaling and does not explicitly account for compounding errors in finite-sample settings.

% We examine task scaling along \( D \) (number of languages) and \( T \) (sequence length). As shown in Figure~\ref{fig:language}, empirical scaling closely follows the theoretical \( O(D \ln D T) \) trend, with slight exceptions at $ T=10 \text{ and } 3$ in Panel B. One possible explanation for this deviation could be error accumulation across sequential steps—longer sequences require each intermediate translation to be approximated with higher precision to maintain test accuracy. This contrasts with our theoretical analysis, which primarily focuses on asymptotic scaling and does not explicitly account for compounding errors in finite-sample settings.

Despite this, the task scaling is still remarkable — training on a few hundred tasks enables generalization to   $4^{10} \approx 10^6$ tasks!






% , this case, we are in a regime where \( D \ll T \), we observe  that the task complexity empirically scales as \( T \log T \) rather than \( D \log T \). 


% the model generalizes to an exponentially larger space of \( 2^T \) unseen tasks. In case $T=25$, this is $2^{25} > 33$ Million tasks. This remarkable exponential generalization demonstrates the power of structured task composition in enabling efficient generalization.


% In the case of parity tasks, introducing CoT effectively decomposes the problem from \( ARC(D^T, 1) \) to \( ARC(D, T) \), significantly improving task generalization.

% Again, in the regime scaling $T$, we again observe a $T\log T$ dependency. Knowing that the function class is scaling as $D^T$, it is remarkable that training on a few hundreds tasks can generalize to $4^{10} \approx 1M$ tasks. 





% We further performed a preliminary investigation on a semi-synthetic word-level translation task to show that (1) task generalization via composition structure is feasible beyond parity and (2) understanding the fine-grained mechanism leading to this generalization is still challenging. 
% \noindent
% \noindent
% \begin{minipage}[t]{\columnwidth}
%     \centering
%     \textbf{\scriptsize In-context examples:}
%     \[
%     \begin{array}{rl}
%         \textbf{Input} & \hspace{1.5em} \textbf{Output} \\
%         \hline
%         \textcolor{blue}{car}   & \hspace{1.5em} \textcolor{red}{voiture \;,\; coche} \\
%         \textcolor{blue}{house} & \hspace{1.5em} \textcolor{red}{maison \;,\; casa} \\
%         \textcolor{blue}{dog}   & \hspace{1.5em} \textcolor{red}{chien \;,\; perro} 
%     \end{array}
%     \]
%     \textbf{\scriptsize Query:}
%     \[
%     \begin{array}{rl}
%         \textbf{Input} & \textbf{Output} \\
%         \hline
%         \textcolor{blue}{cat} & \hspace{1.5em} \textcolor{red}{?} \\
%     \end{array}
%     \]
% \end{minipage}



% \begin{figure}[h!]
%     \centering
%     \includegraphics[width=0.45\textwidth]{Figures/translation_scale_d.png}
%     \vspace{-0.2cm}
%     \caption{Task generalization behavior for word translation task.}
%     \label{fig:arithmetic}
% \end{figure}


\vspace{-1mm}
\section{Conclusions}
% \misha{is it conclusion of the section or of the whole paper?}    
% \amir{The whole paper. It is very short, do we need a separate section?}
% \misha{it should not be a subsection if it is the conclusion the whole thing. We can just remove it , it does not look informative} \hz{let's do it in a whole section, just to conclude and end the paper, even though it is not informative}
%     \kaiyue{Proposal: Talk about the implication of this result on theory development. For example, it calls for more fine-grained theoretical study in this space.  }

% \huaqing{Please feel free to edit it if you have better wording or suggestions.}

% In this work, we propose a theoretical framework to quantitatively investigate task generalization with compositional autoregressive tasks. We show that task to $D^T$ task is theoretically achievable by training on only $O (D\log DT)$ tasks, and empirically observe that transformers trained on parity problem indeed achieves such task generalization. However, for other tasks beyond parity, transformers seem to fail to achieve this bond. This calls for more fine-grained theoretical study the phenomenon of task generalization specific to transformer model. It may also be interesting to study task generalization beyond the setting of in-context learning. 
% \misha{what does this add?} \amir{It does not, i dont have any particular opinion to keep it. @Hongzhou if you want to add here?}\hz{While it may not introduce anything new, we are following a good practice to have a short conclusion. It provides a clear closing statement, reinforces key takeaways, and helps the reader leave with a well-framed understanding of our contributions. }
% In this work, we quantitatively investigate task generalization under autoregressive compositional structure. We demonstrate that task generalization to $D^T$ tasks is theoretically achievable by training on only $\tilde O(D)$ tasks. Empirically, we observe that transformers trained indeed achieve such exponential task generalization on problems such as parity, arithmetic and multi-step language translation. We believe our analysis opens up a new angle to understand the remarkable generalization ability of Transformer in practice. 

% However, for tasks beyond the parity problem, transformers appear to fail to reach this bound. This highlights the need for a more fine-grained theoretical exploration of task generalization, especially for transformer models. Additionally, it may be valuable to investigate task generalization beyond the scope of in-context learning.



In this work, we quantitatively investigated task generalization under the autoregressive compositional structure, demonstrating both theoretically and empirically that exponential task generalization to $D^T$ tasks can be achieved with training on only $\tilde{O}(D)$ tasks. %Our theoretical results establish a fundamental scaling law for task generalization, while our experiments validate these insights across problems such as parity, arithmetic, and multi-step language translation. The remarkable ability of transformers to generalize exponentially highlights the power of structured learning and provides a new perspective on how large language models extend their capabilities beyond seen tasks. 
We recap our key contributions  as follows:
\begin{itemize}
    \item \textbf{Theoretical Framework for Task Generalization.} We introduced the \emph{AutoRegressive Compositional} (ARC) framework to model structured task learning, demonstrating that a model trained on only $\tilde{O}(D)$ tasks can generalize to an exponentially large space of $D^T$ tasks.
    
    \item \textbf{Formal Sample Complexity Bound.} We established a fundamental scaling law that quantifies the number of tasks required for generalization, proving that exponential generalization is theoretically achievable with only a logarithmic increase in training samples.
    
    \item \textbf{Empirical Validation on Parity Functions.} We showed that Transformers struggle with standard in-context learning (ICL) on parity tasks but achieve exponential generalization when Chain-of-Thought (CoT) reasoning is introduced. Our results provide the first empirical demonstration of structured learning enabling efficient generalization in this setting.
    
    \item \textbf{Scaling Laws in Arithmetic and Language Translation.} Extending beyond parity functions, we demonstrated that the same compositional principles hold for arithmetic operations and multi-step language translation, confirming that structured learning significantly reduces the task complexity required for generalization.
    
    \item \textbf{Impact of Training Task Selection.} We analyzed how different task selection strategies affect generalization, showing that adversarially chosen training tasks can hinder generalization, while diverse training distributions promote robust learning across unseen tasks.
\end{itemize}



We introduce a framework for studying the role of compositionality in learning tasks and how this structure can significantly enhance generalization to unseen tasks. Additionally, we provide empirical evidence on learning tasks, such as the parity problem, demonstrating that transformers follow the scaling behavior predicted by our compositionality-based theory. Future research will  explore how these principles extend to real-world applications such as program synthesis, mathematical reasoning, and decision-making tasks. 


By establishing a principled framework for task generalization, our work advances the understanding of how models can learn efficiently beyond supervised training and adapt to new task distributions. We hope these insights will inspire further research into the mechanisms underlying task generalization and compositional generalization.

\section*{Acknowledgements}
We acknowledge support from the National Science Foundation (NSF) and the Simons Foundation for the Collaboration on the Theoretical Foundations of Deep Learning through awards DMS-2031883 and \#814639 as well as the  TILOS institute (NSF CCF-2112665) and the Office of Naval Research (ONR N000142412631). 
This work used the programs (1) XSEDE (Extreme science and engineering discovery environment)  which is supported by NSF grant numbers ACI-1548562, and (2) ACCESS (Advanced cyberinfrastructure coordination ecosystem: services \& support) which is supported by NSF grants numbers \#2138259, \#2138286, \#2138307, \#2137603, and \#2138296. Specifically, we used the resources from SDSC Expanse GPU compute nodes, and NCSA Delta system, via allocations TG-CIS220009. 

% \input{Sections/Threats}
% We present RiskHarvester, a risk-based tool to compute a security risk score based on the value of the asset and ease of attack on a database. We calculated the value of asset by identifying the sensitive data categories present in a database from the database keywords. We utilized data flow analysis, SQL, and Object Relational Mapper (ORM) parsing to identify the database keywords. To calculate the ease of attack, we utilized passive network analysis to retrieve the database host information. To evaluate RiskHarvester, we curated RiskBench, a benchmark of 1,791 database secret-asset pairs with sensitive data categories and host information manually retrieved from 188 GitHub repositories. RiskHarvester demonstrates precision of (95\%) and recall (90\%) in detecting database keywords for the value of asset and precision of (96\%) and recall (94\%) in detecting valid hosts for ease of attack. Finally, we conducted an online survey to understand whether developers prioritize secret removal based on security risk score. We found that 86\% of the developers prioritized the secrets for removal with descending security risk scores.




{\footnotesize\bibliography{references.bib}}
%{\footnotesize\bibliography{references.bib}}



\end{document}
