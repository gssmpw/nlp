%\documentclass[5p,times]{elsarticle}
%\documentclass[review]{elsarticle}
\documentclass[12p]{elsarticle}

%\AtBeginDocument{\renewcommand*{\thesubfigure}{\alphalph{\value{subfigure}}}}
\renewcommand{\footnotesize}{\scriptsize}
\usepackage{textcomp}
%\smartqed  % flush right qed marks, e.g. at end of proof
%\usepackage[breaklines]{listings} % Enable wrapping for inline code
\usepackage{listings} % For lstinline command
%\usepackage{graphicx} % Optional for text wrapping with parbox
\usepackage{graphicx}
%\usepackage{subfig}
\usepackage{caption}
\usepackage{subcaption}
%\usepackage{natbib}
%\usepackage{subfigure}
% \usepackage{etex} % Extend TeX capacity % to add more fig
% \reserveinserts{28} % Increase inserts to add more fig
%\usepackage[table]{xcolor}

\newtheorem{hypothesis}{Hypothesis}
\newtheorem{nullhypothesis}{Null Hypothesis}

\usepackage{csquotes}

\usepackage{tikz}
\usetikzlibrary{tikzmark}
 %% footnote fonts
\usetikzlibrary{fit} %% Used for putting dotted box in image
%\usepackage[margin=2cm]{geometry}
%\usepackage{beamerposter}
\usetikzlibrary{positioning}
\usetikzlibrary{arrows}
\usetikzlibrary{shapes.multipart}

\usepackage[hidelinks,bookmarks=false]{hyperref}
\usepackage[numbered]{bookmark}
\usepackage{color,soul}
\usepackage{booktabs}
\usepackage{multirow}
\usepackage{url}
%\usepackage[small,it]{caption}
\usepackage{tcolorbox}
%\usepackage{cite}
\usepackage{amsmath,amssymb,amsfonts}
\usepackage{algorithmic}
\usepackage{xcolor}
\usepackage{url}
\usepackage{tabularx} 
\usepackage{array}
\usepackage{lineno,hyperref}
\usepackage{longtable}
\renewcommand{\arraystretch}{1.2} % Adjust row height for readability
\modulolinenumbers[5]
\journal{Journal of \LaTeX\ Templates}
 
%\usepackage{tabularx}
\usepackage{pgfplots}
%\pgfplotsset{width=7cm,compat=1.8,tick label style={font=\small}}
\pgfplotsset{compat=1.14}



%\usepackage{graphicx}
\usepackage{adjustbox}
\usepackage{float}
\usepackage{rotating}
\usepackage{tablefootnote}
\usepackage{nth}
\DeclareCaptionType{TextBox}
\newcolumntype{L}{>{\centering\arraybackslash}m{3cm}}

\tcbset{tab1/.style={fonttitle=\bfseries\large,fontupper=\normalsize\sffamily,
colback=yellow!6!white,colframe=red!75!black,colbacktitle=black!75!black,
coltitle=white,center title,freelance,frame code={
\foreach \n in {north east,north west,south east,south west}
{\path [fill=red!75!black] (interior.\n) circle (3mm); };},}}

\tcbset{tab2/.style={enhanced,fonttitle=\bfseries,fontupper=\tiny\sffamily,
colback=yellow!6!white,colframe=red!50!black,colbacktitle=black!75!black,
coltitle=white,center title}}
%
\usepackage[T1]{fontenc}
\usepackage{array, booktabs, makecell}
\usepackage[utf8]{inputenc}
\usepackage{dirtytalk}
\usepackage{tcolorbox}

\usepackage[british]{babel}
\usepackage{enumitem}

\newlist{SubItemList}{itemize}{1}
\setlist[SubItemList]{label={$-$}}

\let\OldItem\item
\newcommand{\SubItemStart}[1]{%
    \let\item\SubItemEnd
    \begin{SubItemList}[resume]%
        \OldItem #1%
}
\newcommand{\SubItemMiddle}[1]{%
    \OldItem #1%
}
\newcommand{\SubItemEnd}[1]{%
    \end{SubItemList}%
    \let\item\OldItem
    \item #1%
}
\newcommand*{\SubItem}[1]{%
    \let\SubItem\SubItemMiddle%
    \SubItemStart{#1}%
} 


%%%%%%%% start: comments %%%%%%%%   
\newcommand{\todo}[1]{\textcolor{purple}{{[TODO: #1]}}}
\newcommand{\Eman}[1]{\textcolor{blue}{{\it [Eman: #1]}}}
\newcommand{\ali}[1]{\textcolor{red}{{\it [Ali: #1]}}}
\newcommand{\anthony}[1]{\textcolor{orange}{{\it [Anthony: #1]}}}
\newcommand{\christian}[1]{\textcolor{red}{{\it [Christian: #1]}}}
\newcommand{\mohamed}[1]{\textcolor{green}{{\it [mohamed: #1]}}}
%%%%%%%% end: comments comments %%%%%%%% 

%%%%%%%% start: dataset information %%%%%%%%  
\newcommand{\projectsNumber}{800\xspace}
\newcommand{\commitsNumber}{111,884\xspace}
\newcommand{\refactoringsNumber}{711,495\xspace}
\newcommand{\sarInitialNumber}{513\xspace}
\newcommand{\labeledCommitsNumber}{1,702\xspace}
\newcommand{\numberSAR}{230\xspace}

%%%%%%%% end: dataset information %%%%%%%% 
%%%%%%%%%%%%%%
\usepackage{framed}
\usepackage{mdframed}
\usepackage{pstricks}
\usepackage{xspace}
%%%%%%%%%%%%%%
\usepackage{pgf-pie}
\usepackage{comment}
\usepackage{colortbl}
\tcbuselibrary{skins}
\tcbuselibrary{listings}
\usepackage[normalem]{ulem}
\usetikzlibrary{patterns,}
\usepgfplotslibrary{colorbrewer}
\usepackage{verbatim}
%\usepackage{fullpage}
% Adding packages to fix RQ2 missing figure
\usepackage{tikz}
\usetikzlibrary{tikzmark}
\usetikzlibrary{fit} %% Used for putting dotted box in image
%\usepackage[margin=2cm]{geometry}
%\usepackage{beamerposter}
\usetikzlibrary{positioning}
\usetikzlibrary{arrows}
\usetikzlibrary{shapes.multipart}
% For rotating figures, tables, etc.
%  including their captions
\usepackage{lscape}
\usepackage{placeins}
\usepackage{afterpage}
\newlength\mylength
\usepackage{fontawesome}
\usepackage{xspace}
\newcommand{\ie}{\textit{i.e., \xspace}}
\newcommand{\eg}{\textit{e.g., \xspace}}
\newcommand{\etal}{\textit{et al. \xspace}}
\newcolumntype{L}{>{\arraybackslash}m{16cm}}
\usepackage[flushleft]{threeparttable}
\usepackage{mdframed}
%\usepackage{subfigure}
%\usepackage{tablenotes}
%%%%

\newtcolorbox{boxK}{
    sharpish corners, % better drop shadow
    boxrule = 0pt,
    toprule = 4.5pt, % top rule weight
    enhanced,
    fuzzy shadow = {0pt}{-2pt}{-0.5pt}{0.5pt}{black!35} % {xshift}{yshift}{offset}{step}{options} 
}
%\usepackage{apacite}
%\bibliographystyle{model5-names}\biboptions{authoryear} eswa
\bibliographystyle{elsarticle-num}
\begin{document}

\begin{frontmatter}

\title{An Empirical Study on the Impact of Code Duplication-aware Refactoring Practices on Quality Metrics}
%An Empirical Study on Refactoring Documentation in Open Source Software}
%\subtitle{Do you have a subtitle?\\ If so, write it here}


%% Group authors per affiliation:
\author[STEVENS]{Eman Abdullah AlOmar\corref{mycorrespondingauthor}}
\cortext[mycorrespondingauthor]{Corresponding author}
\ead{ealomar@stevens.edu}


%\author[RIT]{Nasir Safdari}
%\ead{nsafdari@rit.edu}
%\author[ETS]{Ali Ouni}
%\ead{ali.ouni@etsmtl.ca}

\address[STEVENS]{Stevens Institute of Technology, Hoboken, NJ, USA}



\begin{abstract}
%%% reduced version
\textbf{Context:} Code refactoring is widely recognized as an essential software engineering practice that improves the
understandability and maintainability of source code. Several studies attempted to detect refactoring activities through mining software repositories, allowing one to collect, analyze, and get actionable data-driven insights about refactoring practices within software projects. 

\noindent\textbf{Objective:} Our goal is to identify, among the various quality models presented in the literature, the ones that align with the developer's vision of eliminating duplicates of code, when they explicitly mention that they refactor the code to improve them. 

\noindent\textbf{Method:} We extract a corpus of 332 refactoring commits applied and documented by developers during their daily changes from 128 open-source Java projects. In particular, we extract 32 structural metrics 
from which we identify code duplicate removal commits with their corresponding refactoring operations, as perceived by software engineers. Thereafter, we empirically analyze the impact of these refactoring operations on a set of common state-of-the-art design quality metrics. 

\noindent\textbf{Results:} The statistical analysis of the results obtained shows that (i) some state-of-the-art metrics are capable of capturing the developer's intention of removing code duplication; and
(ii) some metrics are being more emphasized than others. We confirm that various structural metrics can effectively represent code duplication, leading to different impacts on software quality. Some metrics contribute to improvements, while others may lead to degradation. 

\noindent\textbf{Conclusion:} Most of the mapped metrics associated with the main quality attributes successfully capture developers' intentions for removing code duplicates, as is evident from the commit messages. However, certain metrics do not fully capture these intentions.

\end{abstract}
\begin{keyword}
Refactoring, Quality, Code Duplicates, Metrics
\end{keyword}

\end{frontmatter}

%\linenumbers



The increasing reliance on LLMs for multimodal tasks across far-reaching sectors such as healthcare, finance, and manufacturing underscores the need to assess the accuracy and reliability of the information they generate. Vision-Language Models (VLM) have achieved state-of-the-art (SoTA) performance on Visual Question-Answering (VQA) benchmarks, and these models often utilize Retrieval-Augmented Generation (RAG) to maintain factual accuracy and relevance in a dynamic information environment. However, this has led to uncertainty in the information the LLM bases its answer on, as it may choose between parametric memory and retrieved sources. When models rely on memorized information instead of dynamically retrieving information, they may inadvertently propagate outdated or incorrect information, causing serious legal and ethical risks and undermining trust and reliability in AI systems \citep{huang2023survey}.
% The ability to strike a balance between generalization and specialization in AI systems is therefore crucial for ensuring the safe, reliable use of these technologies in real-world applications.

Despite these concerns, the way that Vision-Language models (VLMs) memorize and retrieve information, particularly in complex multimodal tasks, remains under-explored. Current research often focuses on either the general capabilities of large language models (LLMs) or the specialized retrieval mechanisms in retrieval augmented generation systems (RAG) \citep{incontext_rag,chen_murag_2022,liu_universal_2023}. Particularly in the context of multimodal retrieval and multihop reasoning, few studies analyze the tradeoff between finetuning for specialized tasks and zero-shot prompting for general-purpose vision-language capabilities. A lack of consensus on how to approach this tradeoff motivates the development of measures to quantify reliance on parametric memory, as well as metrics for quantifying the potential performance impact of extending LLMs with RAG systems.

To address this gap, we investigate how multimodal QA models balance accuracy with memorization on the WebQA benchmark. We compare finetuned multimodal systems against zero-shot VLMs, analyzing how retrieval performance influences QA accuracy. In particular, we focus on cases where retrieval fails, allowing us to measure reliance on parametric memory through two proposed metrics---the \ppr (\PPR) which quantifies how much model accuracy is influenced by retrieval quality, contrasting performance in best-case versus worst-case retrieval scenarios, and the \ucr (\UCR) which measures how often correct QA responses are generated when the retriever fails, providing a proxy for memorization.

To enable this analysis, we make several methodological contributions. For the finetuned QA models, we investigate Vision-Transformer (ViT) architectures, which allow for multihop reasoning over multiple sources. To investigate the impact of retrieval performance on trained LMs, we propose a variable-input Fusion-in-Decoder (FiD) model \cite{tanaka_slidevqa_2023, nlvr2}, building upon the VoLTA architecture \citep{pramanick_volta_2023}. For the zero-shot case, we build upon previous research on In-Context Retrieval \citep{incontext_rag} by demonstrating that LLMs such as GPT-4o are capable of performing the final ranking step of the retrieval process. In doing so, we find that GPT-4o, a general-purpose LLM, achieves SoTA performance on the WebQA task, outperforming existing finetuned RAG models by a significant margin (7\% higher accuracy). 

Crucially, our results reveal that while retrieval-augmented models reduce memorization, the training paradigm plays an important role. Finetuned models exhibit higher reliance on parametric memory, whereas zero-shot RAG approaches have lower memorization scores at the cost of accuracy. This suggests that while retrieval modules may mitigate the risks associated with outdated or incorrect information, SoTA performance requires that they be coupled with specialized QA models. Our memorization measures contribute to the development of transparent and reliable AI systems, particularly in applications where the sourcing of up-to-date, factual information is critical.



% We investigate the impact of question complexity on the ability of these models to integrate multiple data sources—such as images, text, and external retrievers—and produce coherent and accurate answers. We also explore whether in-context retrieval can be a viable alternative to traditional retrieval-augmented systems, offering a more streamlined approach to multimodal QA.

% To achieve this, we first compare zero-shot prompting multimodal LLMs with finetuned multimodal systems. We evaluate both types of models on the WebQA benchmark, a dataset designed for complex question answering that requires reasoning across both image and text sources. For the finetuned models, we use a Fusion-in-Decoder (FiD) architecture, which allows for multihop reasoning over multiple sources. Additionally, we introduce the concept of In-Context Retrieval Language Modeling (RLM), where the LLM itself performs retrieval tasks without the need for external retrievers. This method builds upon existing research in in-context learning  and aims to explore the viability of LLMs retrieving relevant sources and generating accurate answers directly from their context window.

% In order to investigate source utilization in finetuned multimodal models and LLMs, three lines of inquiry are established; 
% \begin{itemize}
%     \item Study 1: retrieval vs QA performance on webQA (motivating example, does QA answer correctly even with incorrect sources?)
%     \item Study 2: performance on adversarial examples where parametric knowledge would be incorrect by design
%     \item Study 3: improving performance on adversarial examples by fine-tuning (i.e model robustness)
% \end{itemize}

% Note, there is one weakness in this plan which is tying in the work we've already done. 
% If we added something from adversarial generation to the retrieval experiment (like a combination of study 1 + 3) it would be complete. So for instance we could try fine-tuning the retriever with adversarial examples (and not just the QA model)

% \begin{figure}
%     \centering
%     \includegraphics[width=0.95\linewidth]{figures/segmentation/webqa_segment_infill.png}
%     \caption{Example of the segmentation substitution pipeline from the WebQA task.}
%     % d5c76d760dba11ecb1e81171463288e9
%     \label{fig:seg_sub_pipeline}
% \end{figure}



% Retrieval augmented generation (RAG) with zero-shot prompting and fine-tuning Large Language Models (LLMs) have become the go-to methods for tasks relying on information retrieval and text generation. In many cases the LLMs parametric memory can sufficiently generalize to answer questions without being provided with retrieval mechanisms for out-of-domain knowledge. However, LLMs often hallucinate and provide wrong information in certain scenarios. This problem is amplified even further on open-domain Question Answering (QA) tasks involving multiple modalities. Grounded text generation using retrieved sources \citep{lewis2021retrievalaugmented} has been extensively studied for text-to-text QA tasks, but its application in multimodal settings has not been studied as much.


% Multimodal reasoning and question answering have gained prominence in recent research endeavors, with an increasing emphasis on handling various forms of data, particularly text and images. In this study, we address a specific gap in the existing literature by focusing on the development of a versatile multihop model capable of accommodating varying numbers of input images.

% Our motivation for this research lies in the growing complexity of answering questions using information on the web, where the challenge of navigating the open-domain setting is further complicated by the presence of multiple modalities and sometimes requires reasoning over multiple sources. WebQA is an ideal dataset on which to compare performance of finetuned RAG systems against general purpose LLMs; it is multimodal, with correct answers requiring reasoning over image and text sources. It is multihop, requiring a complex reasoning process over multiple sources. Finally, WebQA questions from different categories can be broken down into subdomains to analyze performance over domains of varying cardinality.

% Motivated by the real-world challenges of building retrieval and question answering (QA) systems, we design and finetune a closed domain, multimodal, multihop QA model, that is capable of reasoning over a varying number of sources taken as input from an external retriever module. This research contributes to the relatively underexplored domain of multihop reasoning across various input sources and modalities. Our goal is to explore the challenges posed by these scenarios and develop strategies that enable QA models to retrieve relevant information, conduct logical or numerical reasoning across diverse modalities, and generate coherent responses in natural language. To our knowledge, this is the first application of the Fusion-in-Decoder (FiD) architecture \cite{tanaka_slidevqa_2023, nlvr2} that is shown to work with a variable number of inputs, enabling multi-hop reasoning over sources.

% In-Context Learning refers to the ability of LLMs to perform any task by simply providing examples in the input prompt \citep{dong2022survey,min2022rethinking}. Inspired by this research, we propose a method to use the LLM itself as a multimodal retriever, potentially eschewing the requirement of a distinct retrieval module, thereby allowing the design of simpler retrieval-augmented QA systems. We dub this method In-Context Retrieval Language Modeling (RLM). To the best of the authors knowledge, In-Content RLM is disparate from other retrieval augmented approaches which utilize external retrieval modules \citep{incontext_rag,chen_murag_2022,liu_universal_2023}. Despite being a natural extension of In-Context learning, In-Context RLM has not yet been studied empirically.

% To expand on our contribution of In-Context Retrieval, this stems from the well-researched in-context learning of LLMs. In-context learning is the ability of a model to perform any task given a sufficient context window \citep{dong2022survey,min2022rethinking}. Such tasks could include retrieval and ranking, but typically, the go-to solution for tasks requiring retrieval has been RAG. To the best of the authors knowledge, In-Context Retrieval is distinct from In-Context Retrieval Augmented Language Modelling (RALM), and despite being a natural extension of In-Context learning, In-Context Retrieval has not yet been shown empirically.

% Finally, we explore the tradeoff between using zero-shot prompting LLMs and the fine-tuning approach. While we find that, overall, GPT-4o obtains SoTA performance on the WebQA task, outperforming the accuracy of existing finetuned RAG approaches by 7\%, finetuned approaches still perform better on more restricted subdomains\footnote{``In-Context RLM" @ \url{https://eval.ai/web/challenges/challenge-page/1255/leaderboard/3168}}. Finally, we validate that GPT-4o is relying on retrieval abilities to solve the task; we find that GPT-4o is capable of retrieving relevant sources in the presence of distractors and furthermore, when GPT-4o fails to retrieve correct sources, it answers incorrectly 75\% of the time, meaning that it is not relying on parametric memory for this task.

% \paragraph{Contributions}
% Based on our experimentation and analysis on the WebQA benchmark, we make the following contributions:
% \begin{itemize}
%     \item Propose a new architecture for multimodal multihop QA that takes variable number of input sources inspired by the Fusion-in-Decoder method.
%     \item Comparison of general purpose LLMs vs specialized models on the WebQA benchmark.
%     \item Observation of In-Context Multimodal Retrieval abilities of GPT-4o and that it does not rely on parametric memory for multimodal QA.
%     \item Analysis of relationship between retrieval and QA task performance.
%     \item Analysis of task and query complexity on the performance of retrieval and QA tasks.
% \end{itemize}
















% Throughout this paper, we will present our methodology, experiments, and findings, emphasizing our approach to multihop reasoning over varying numbers of input images. We believe that our work contributes to a deeper understanding of multimodal reasoning and has the potential to enhance the capabilities of question-answering systems in the intricate, multimodal landscape of web-based information.
\section{Related Work}
\label{sec:relatedwork}
Traditionally, an experimental \ac{IR} collection includes three elements, a corpus, a set of topics, and the relevance judgments, defining which documents are relevant in response to the topics.
Over the last 30 years, since the first TREC campaign~\cite{DBLP:conf/trec/1992}, the most common strategy to obtain such relevance judgments has involved expert annotators, capable of providing the most accurate labels. 
The cost of this process can be partially reduced with pooling~\cite{croft2009search}, but the monetary and temporal costs of building an \ac{IR} experimental collection following this paradigm remain extremely high.

Automatic relevance judgment has recently received significant attention in the IR community. In earlier studies, ~\citet{faggioli2023perspectives} studied different levels of human and LLMs collaboration for automatic relevance judgment. They suggested the need for humans to support and collaborate with LLMs for a human-machine collaboration judgment. ~\citet{thomas2023large} leverage LLMs capabilities in judgment at scale, in Microsoft Bing. They used real searcher feedback to build an LLM and prompt in a way that matches the small sample of searcher preferences. Their experiments show that LLMs can be as good as human annotators in indicating the best systems. They also comprehensively investigated various prompts and prompt features for the task and revealed that LLM performance on judgments can vary with simple paraphrases of prompts. Recently, \citet{rahmani2024synthetic} have studied fully synthetic test collection using LLMs. In their study, they generated synthetic queries and synthetic judgment to build a full synthetic test collation for retrieval evaluation. They have shown that LLMs can generate a synthetic test collection that results in system ordering performance similar to evaluation results obtained using the real test collection.

On a different line, \citet{DBLP:conf/sigir/Dietz24} defines a LLM-based ``autograding'' approach. This evaluation strategy targets generated content that cannot be evaluated in a purely offline scenario and it consists of using a question bank as the evaluation test-bed. An \ac{LLM} measures the effectiveness of the generative model in answering the questions, possibly with the supervision of a human. The autograding approach proposed by \citet{DBLP:conf/sigir/Dietz24} includes an automatic passage evaluation whose task aligns with the one evaluated in \texttt{LLMJudge}.

\subsection{Criticisms and Open Challenges}
The use of \acp{LLM} as assessors comes with major bias risks and challenges that should not be neglected, especially considering the impact they might have in the development of \ac{IR} evaluation.

\partitle{Bias}
First and most importantly, \acp{LLM} are affected by bias~\cite{DBLP:conf/fat/BenderGMS21}. Their internal representation of the concepts is, by construction, conditioned on the context such concepts appear in~\cite{DBLP:conf/nips/VaswaniSPUJGKP17}. Thus, depending on the underlying data, the \ac{LLM} might form a biased notion of relevance that might reflect upon the relevance judgments generated by it. Quantifying the bias, identifying its source, and mitigating its consequences are still open issues that need to be addressed. We hope that the release of this collection will help the research community with the needed data to study how to deal with the bias in \ac{LLM}-generated relevance judgments.

\partitle{Circularity}
A second source of concern when it comes to using \acp{LLM} as assessors relates to the risk of \textit{circular evaluation}~\cite{faggioli2023perspectives,DBLP:journals/corr/abs-2409-15133}. For example, the same \ac{LLM} might be used to generate relevance judgments and as a document ranker. This would induce a strong bias on the validity and generalizability of the relevance judgments.

\partitle{Environmental Impact}
An often hidden cost of the \acp{LLM} concerns their environmental impact in terms of energy utilization, carbon emissions~\cite{DBLP:journals/corr/abs-2408-09713,DBLP:conf/sigir/ScellsZZ22}, and water consumption~\cite{DBLP:conf/ictir/ZucconSZ23}.
While \acp{LLM} might allow building collections at a fraction of the monetary and temporal cost, we should account for the environmental impact of such a process, limiting our reliance on ``disposable'' relevance judgments.

\partitle{Vulnerability to Attacks and Adversarial Misuse}
\citet{DBLP:conf/ecir/ParryFMPH24} and \citet{DBLP:conf/sigir-ap/Alaofi0SS24} illustrate the vulnerability of the \acp{LLM} to mischievous manipulations of the corpus. For example,~\citet{DBLP:conf/ecir/ParryFMPH24} show that, by introducing keywords such as the term ``relevant'' in a document, it will more likely considered relevant by an \ac{LLM}. Similar behavior is observed also by \citet{DBLP:conf/sigir-ap/Alaofi0SS24}, who notice that by introducing the query on the document, more probably an \ac{LLM} will consider the document relevant to such a query --- even if the rest of the document is composed by random terms.
More recently, \citet{DBLP:journals/corr/abs-2412-17156} show how, by properly crafting an adversarial run, it is possible to cheat an \ac{LLM} used as an assessor. \citet{DBLP:journals/corr/abs-2412-17156} crafted a run following the same approach used by~\citet{upadhyay2024umbrela} to pool the documents and build the \ac{LLM}-generated relevance judgments used for TREC 2024 RAG. Such a run achieved consistently higher effectiveness under the fully automatic evaluation paradigm compared to its performance based on manual relevance judgments. 

By releasing this collection of \ac{LLM}-generated relevance judgments we want to foster the analysis and study of possible sources of biases and systematic errors, to mitigate them and allow for the development of more effective and robust future solutions that involve \acp{LLM} as tools to support the annotation process.
\section{Study Design}
\label{Section:methodology}


\begin{figure*}[t]
\centering 
\includegraphics[width=1.0\textwidth]{Images/Approach-v2.pdf}
\caption{\textcolor{black}{Overview of the empirical study design, highlighting the 3 main phases: Dataset Extraction, Selection of Quality Attributes and Software Metrics, and Data Analysis.}}
\label{fig:approach_overview}
\end{figure*}

Our primary objective is to explore the alignment between developers' perceptions of code duplicate removal (as anticipated by developers) and the actual improvement in software quality (as evaluated by quality metrics). Specifically, our aim is to address the following research questions.
\begin{boxK}
\textbf{RQ$_1$}: \textcolor{black}{What is the quantitative code quality assessment of code duplications that have been intentionally removed by developers?}

\textbf{RQ$_2$}: \textcolor{black}{What are the refactoring operations associated with code duplicate removal?}
\end{boxK}

To address our research questions, we conducted a three-phase empirical study. \textcolor{black}{An overview of the experiment methodology is depicted in Figure \ref{fig:approach_overview}. The initial phase involves extracting a substantial number of open-source Java projects along with their instances of refactoring throughout their development history, specifically focusing on commit-level code changes for each project under consideration. In the second phase, we select software quality metrics to compare their values before and after the identified refactoring commits. Subsequently, the third phase involves analyzing commit messages to identify refactoring commits where developers document their perception of code duplicate removal. In the next subsection, we discuss each phase in detail.}

\subsection{Extracted Dataset}

Our study uses the SmartSHARK MongoDB Release 2.2 dataset  \citep{trautsch2021msr}. This dataset contains a wide range of information for 128 open-source Java projects, such as commit history, issues, refactorings, code metrics, mailing lists, and continuous integration data. All Java projects are part of the Apache ecosystem and utilize GitHub as their version control repository and JIRA for issue tracking. SmartSHARK utilizes \texttt{RefDiff} \citep{silva2017refdiff} and \texttt{RefactoringMiner} \citep{tsantalis2018accurate} to mine refactoring operations. \textcolor{black}{This study is motivated to investigate code duplication-aware refactoring practices in Apache projects. A recent study \citep{xiao2024empirical} highlights the Apache Software Foundation as a prominent example of successful open-source software communities \citep{mockus2002two,mockus2000case,crowston2006assessing}. Both practitioners and researchers have been extracting valuable insights and gaining experience from Apache's effective practices to drive the open-source movement forward \citep{rigby2008open,duenas2007apache,weiss2006evolution}. Furthermore, Apache is a collaborative environment where engineers from major corporations such as IBM, Google, Yahoo, Sun, and Oracle volunteer to develop open-source software infrastructure \citep{severance2012apache}.} 

%  this study is motivated to investigate the usage of mocking frameworks in Apache projects since Apache Software Foundation has been widely recognized
% and researched as a distinguished example of successful open-source software communities (Mockus et al. 2002, 2000; Crowston and Howison 2006). Practitioners and researchers
% have been gaining experience and insights from the successful practices of Apache projects
% to lead the open-source movement (Rigby et al. 2008; Duenas et al. 2007; Weiss et al. 2006).
% In addition, Apache provides a meeting point where engineers from large companies like
% IBM, Google, Yahoo, Sun, and Oracle work as volunteers to build open-source software
% infrastructure (Severance 2012).

To extract the relevant information, we built custom scripts to extract data pertinent to our study (\ie commits, metrics, refactorings) from
the source dataset into an SQLite database for analysis. First, we extract all commits with the keyword `duplicat*' and `code clone', discussed later in Section \ref{dataanslysis}. Next,
we extract all refactoring operations. However, due to the use of two refactoring mining tools, there are duplicate operations in the source data. \textcolor{black}{Hence, our next step is to remove all duplicates by comparing the refactoring descriptions. After that, we select all
commits associated with a refactoring operation. Using both refactoring mining tools allowed us to mitigate the limitations of relying on a single tool and ensured a more diverse and thorough dataset.} Table \ref{Table:DATA_Overview} summarizes the extracted data.




\subsection{Quality Attributes \& Quality Metrics Selection}
To setup a comprehensive set of quality attributes for evaluation in our study, we initially analyze existing studies to identify commonly recognized software quality attributes \citep{chidamber1994metrics,lorenz1994object,mccabe1976complexity, henry1981software, nejmeh1988npath, Destefanis:2014:SMA:2813544.2813555}. Next, we assess whether the metrics evaluate various object-oriented design aspects, mapping each internal quality attribute to the corresponding structural metric(s). %For example, the Response For Class (RFC) metric is typically used to measure Coupling and Complexity quality attributes. 
  Additionally, we extract associations between metrics (such as the CK suite \citep{chidamber1994metrics}, McCabe \citep{mccabe1976complexity}, and Lorenz and Kidd's book \citep{lorenz1994object}) and internal quality attributes from the literature review. \textcolor{black}{Tables \ref{Table:Quality Metrics in Related Work} and \ref{Table:Quality Metrics in Related Work-v2} summarize the extracted metrics.}

Subsequently, we examined the extracted metrics to determine whether these metrics exist in the SmartSHARK dataset, calculated using OpenStaticAnalyzer\footnote{https://github.com/sed-inf-u-szeged/OpenStaticAnalyzer}. The extraction process results in 32 distinct structural metrics as shown in Table \ref{Table:Quality Metrics Used in This Study.}. The list of metrics is (1) well-known and defined in the literature, and (2) can assess different code-level elements, \ie method, class, package. % and (3) can be calculated by existing static analysis tools. 

%We also adopted NOA andNOO since they measure quality aspects of a class that are not takeninto account by the CK metrics

\subsection{Data Analysis}
\label{dataanslysis}

\begin{table}[h!]
\begin{center}
\caption{\textcolor{black}{Summary of the extracted data.}}
\label{Table:DATA_Overview}
\begin{adjustbox}{width=1.0\textwidth,center}
%\begin{adjustbox}{width=\textheight,totalheight=\textwidth,keepaspectratio}
\begin{tabular}{lllll}\hline
\toprule
\bfseries Item & \bfseries Count \\
\midrule
Total projects & 128 \\
%Total commits & \\
%Total projects with commits containing keyword `\textit{duplicat*}' & 73  \\
%Refactoring commits & 2169916  \\
\cellcolor{gray!30}Refactoring commits with keyword `\textit{duplicat*}' & \cellcolor{gray!30}2,169,916  \\
False positive commits & 22 \\
\cellcolor{gray!30}Refactoring commits after removing false positives & \cellcolor{gray!30}2,164,797 \\
(Distinct) Refactoring commits with keyword `\textit{duplicat*}' & 332  \\
%Refactoring commits w/ class code entity & 88,642 \\
\bottomrule
\end{tabular}
\end{adjustbox}
\end{center}
\end{table}

After extracting all refactoring commits, we want to keep only commits where refactoring is documented. We continue to filter them, using the content of their messages at this stage. We use a keyword-based search to find commits whose messages contain the keywords (\ie `duplicat*' or `code clone*'). We selected these keywords because these keywords are naturally used by developers to articulate their intent regarding code duplication \citep{alomar2019can,alomar2021we}. However, it is worth mentioning that we did not find any commits with the keyword `code clone'. Therefore, all the commits in our dataset solely include the keyword `duplicat'.

This keyword-based filtering selected 2,169,916 commit messages. %We notice that the ratio of these commits is very small compared to the total number of refactoring commits, \ie \hl{\#}. However, these observations are consistent with previous studies \citep{murphy2012we,szoke2014bulk} as developers typically do not provide details when documenting their refactorings. 
To ensure that these commits reported developers' intention to remove code duplication, we manually inspected and read through 322 distinct refactoring commits to remove false positives. An example of a discarded commit is: \say{\textit{DeferredDuplicates.java}}. We discarded this commit because the keyword `duplicat' is actually part of the identifier name of the class. In the case of doubts about including a certain commit, it was excluded. This step resulted in considering 322 commits. Our goal is to have a \textit{gold set} of commits in which the developers explicitly reported the removal of duplicate code. This \textit{gold set} will serve to check later if there is an alignment between the real quality metrics affected in the source code, and the code duplicate removal as documented by developers. 
 An example of commit messages belonging to the \textit{gold set}, is showcased in the following commit message  \say{\textit{Refactored JavaClass and FieldOrMethod to avoid a code duplication}}. %This commit message shown in Figure \ref{fig:commit message} suggests that the developers performed a code refactoring to eliminate code duplication. Figures \ref{fig:duplicate 1} and \ref{fig:duplicate 2} depict the two instances of code duplication present in the codebase. The developer employed the `Extract Method' refactoring technique (see Figure \ref{fig:method extraction}), which involves isolating a code fragment and relocating it to form a new method, subsequently replacing all occurrences of that fragment with a call to the newly created method.

\begin{table}
%\begin{minipage}{\columnwidth} 
  \centering
	 \caption{\textcolor{black}{Structural code quality metrics used in this study.}}
	 \label{Table:Quality Metrics Used in This Study.}
  \begin{threeparttable}
%\begin{sideways}
\begin{adjustbox}{width=1.0\textwidth,center}
%\begin{adjustbox}{width=\textheight,totalheight=\textwidth,keepaspectratio}
\begin{tabular}{llll}\hline
\toprule
\bfseries Quality Attribute & \bfseries Study &   \bfseries Metric & \bfseries Description \\
\midrule
%\multicolumn{2}{l}{\textbf{\textit{Internal Quality Attribute }}}\\
%\midrule
Cohesion & \cite{pantiuchina2018improving,chavez2017does} &↓ LCOM& Lack of Cohesion of Methods   \\ 
Coupling &  \cellcolor{gray!30}\cite{chavez2017does,pantiuchina2018improving} & \cellcolor{gray!30}↓ CBO&\cellcolor{gray!30}Coupling Between Objects    \\
         & \cite{pantiuchina2018improving} & ↓ RFC & Response For Class   \\
         & \cellcolor{gray!30}\cite{islam2018characteristics} & \cellcolor{gray!30}↓ NII &\cellcolor{gray!30}Number of Incoming Invocations  \\
         & \cite{islam2018characteristics} & ↓ NOI &Number of Outgoing Invocations \\
Complexity & \cellcolor{gray!30}\cite{chavez2017does} & \cellcolor{gray!30}↓ CC & \cellcolor{gray!30}Cyclomatic Complexity 
           \\
           & \cite{chavez2017does,pantiuchina2018improving,singh2012evaluation} & ↓ WMC& Weighted Method Count  \\
           & \cellcolor{gray!30}\cite{islam2018characteristics} & \cellcolor{gray!30}↓ NL & \cellcolor{gray!30}Nesting Level  \\
           & \cite{islam2018characteristics} & ↓ NLE &Nesting Level Else-if  \\
           & \cellcolor{gray!30}\cite{islam2018characteristics} & \cellcolor{gray!30}↓ HCPL & \cellcolor{gray!30}Hal. Calculated Program Length \\
            & \cite{islam2018characteristics} & ↓ HDIF & Hal. Difficulty  \\
             & \cellcolor{gray!30}\cite{islam2018characteristics} & \cellcolor{gray!30}↓ HEFF & \cellcolor{gray!30}Hal. Effort  \\
              & \cite{islam2018characteristics} & ↓ HNDB & Hal. Number of Delivered Bugs   \\
               & \cellcolor{gray!30}\cite{islam2018characteristics} &\cellcolor{gray!30}↓ HPL & \cellcolor{gray!30}Hal. Program Length   \\
                & \cite{islam2018characteristics} & ↓ HPV & Hal. Program Vocabulary  \\
                 & \cellcolor{gray!30}\cite{islam2018characteristics} & \cellcolor{gray!30}↓ HTRP &\cellcolor{gray!30}Hal. Time Required to Program  \\
                  & \cite{islam2018characteristics} & ↓ HVOL &Hal. Volume \\
                   & \cellcolor{gray!30}\cite{islam2018characteristics} &\cellcolor{gray!30}↑ MIMS & \cellcolor{gray!30}Maintainability Index (MS) \\
                    & \cite{islam2018characteristics} &↑ MI& Maintainability Index (OV) \\
                     & \cellcolor{gray!30}\cite{islam2018characteristics} &\cellcolor{gray!30}↑ MISEI& \cellcolor{gray!30}Maintainability Index (SEIV) \\
                      & \cite{islam2018characteristics} &↑ MISM&  Maintainability Index (SV)\\
Inheritance & \cellcolor{gray!30}\cite{chavez2017does,singh2012evaluation} & \cellcolor{gray!30}↓ DIT &\cellcolor{gray!30}Depth of Inheritance Tree 
  \\
   & \cite{chavez2017does,singh2012evaluation} & ↓ NOC &Number of Children   \\
 & \cellcolor{gray!30}\cite{bavota2015experimental} & \cellcolor{gray!30}↓ NOA & \cellcolor{gray!30}Number of Operations Added by Subclass   \\
        
Design Size & \cite{chavez2017does} & ↓ LOC & Lines of Code \\
& \cellcolor{gray!30}\cite{islam2018characteristics} &\cellcolor{gray!30}↓ TLOC &\cellcolor{gray!30}Total Lines of Code   \\
& \cite{chavez2017does} & ↓ LLOC &Logical Lines of Code   \\
& \cellcolor{gray!30}\cite{islam2018characteristics} & \cellcolor{gray!30}↓ TLLOC&\cellcolor{gray!30}Total Logical Lines of Code  \\
            & \cite{chavez2017does} & ↑  CLOC&Lines with Comments  \\
            & \cellcolor{gray!30}\cite{stroggylos2007refactoring} & \cellcolor{gray!30}↓ NPM &\cellcolor{gray!30}Number of Public Methods  \\
           % & & Total Number of Methods (TNM) \\
            &\cite{islam2018characteristics} &↓ NOS& Number of Statements  \\
            &\cellcolor{gray!30}\cite{islam2018characteristics} & \cellcolor{gray!30}↓ TNOS&\cellcolor{gray!30}Total Number of Statements  \\
           % & & Total Number of Accessor Methods (TNG) \\
           % & & Total Number of Attributes (TNA) \\
\bottomrule
\multicolumn{4}{l}{\tiny 
↑ by a metric indicates the higher the better for that metric; 
↓ by a metric indicates the lower the better for that metric.}
%Hal.=Halstead; MS= Microsoft version; OV=Original version; SEIV=SEI version; SV=SourceMeter version.}
% \begin{tablenotes}
 %    The first note
 %   \end{tablenotes}
\end{tabular}
\end{adjustbox}
%\end{sideways}
%\footnote{xxx}
%\end{minipage}
\end{threeparttable}
\end{table}
%\footnotesize{$^a$ The smallest spatial unit is county, $^b$ more details in appendix A}

We perform a qualitative analysis of intriguing instances of alignment or disparity between the removal of code duplication as perceived by developers and its evaluation through quality metrics. To do this, the author manually inspects the commits, which involves analyzing the diff code alongside the metrics profile of the affected code elements before and after the commit.

\begin{comment}

\begin{itemize}
    \item commit id for those commits that contain either “\%duplicat\%” or “\%code clone\%” in the commit message
    \item Refactoring type is `extract method'
    \item code entity is method
\end{itemize}



\begin{figure*}[htbp]
	\centering
    \includegraphics[scale = 0.35]{Images/CommitMessage.png}   
    \caption{Commit message indicating the removal of code duplication \citep{commons-bcel}.}
    \label{fig:commit message}

\vspace{0.70cm}

	\centering
    \includegraphics[width=0.5\textwidth]{Images/DuplicateMethod1.png}   
    \caption{Code snippet depicting the first instance of code duplication before refactoring \citep{commons-bcel}.}
    \label{fig:duplicate 1}

\vspace{0.70cm}

\centering 
\includegraphics[scale = 0.2]{Images/DuplicateMethod2.png}
\caption{Code snippet depicting the second instance of code duplication before refactoring \citep{commons-bcel}.}
\label{fig:duplicate 2}

\vspace{0.70cm}

\centering 
%\scalebox{0.8}{\includegraphics[width=\columnwidth]{Images/DiffPreconditionChecking.PNG}}
\includegraphics[scale = 0.2]{Images/ExtractedMethod.png}
\caption{Code snippet depicting the removal of the duplicated code through the `Extract Method' refactoring \citep{commons-bcel}.}
\label{fig:}
\label{fig:method extraction}
\vspace{0.70cm}


\end{figure*}
\end{comment}

The resulting commits correspond to our data points, each data point is represented by a set of \textit{pre-refactoring} and \textit{post-refactoring} Java files. These data points will be used in the experiments, to measure the effect of changes in terms of structural metrics, with respect to the quality attribute, announced in the commit message.


\section{Results \& Discussion}
\label{Section:Result}
\subsection{\textcolor{black}{What is the quantitative code quality assessment of code duplications that have been intentionally removed by developers?}}
For each refactoring commit in which developers document the removal of duplicate code, we extract its associated metric values  (see Table~\ref{Table:Quality Metrics Used in This Study.}) before and after the commit. 
 In other words, for commit messages related to the removal of code duplicates, we examine 32 corresponding metric values before and after the selected refactoring commit. As we evaluate metric values both pre- and post-refactoring, we want to distinguish, for each metric, whether there is a variation between its pair of values, whether this variation signifies an improvement, and whether the variation is statistically significant. Therefore, we use the Wilcoxon test \citep{wilcoxon1945individual}, a non-parametric test, to compare the group of metric values before and after the commit since these groups depend on each other. The null hypothesis is defined by no variation in the metric values of pre- and post-refactored code elements. Thus, the alternative hypothesis indicates a variation in the metric values. In each case, a decrease in the metric value is considered desirable (\ie an improvement), except for complexity metrics related to the maintainability index and CLOC (see Table~\ref{Table:Quality Metrics Used in This Study.}), where higher values are desirable. Furthermore, the variation between the values of both sets is considered significant if its associated \textit{p}-value is less than 0.05. Furthermore, we used the Cliff's delta ($\delta$) effect size to estimate the magnitude of the differences. Regarding its interpretation, we follow the guidelines reported by Grissom \etal \citep{trove.nla.gov.au/work/16432558}:

 \begin{itemize}
\item Negligible for $\mid \delta \mid< 0.147$
\item Small for $0.147 \leq \mid \delta \mid < 0.33$
\item Medium for $0.33 \leq \mid \delta \mid < 0.474$
\item Large for $\mid \delta \mid \geq 0.474$
\end{itemize}

%In the following, we report the results of our research questions. 


To answer our main research question, we provide a detailed analysis of each of the five quality attributes reported in Table \ref{Table:Quality Metrics Used in This Study.} and qualitatively analyze the cases with positive and negative impacts. Table~\ref{Table:Metrics Suites and Metrics Tools Summary} shows the overall impact of refactorings on quality. The boxplots in \textcolor{black}{Figures \ref{Chart:Boxplots_cohesion}, \ref{Chart:Boxplots_coupling}, \ref{Chart:Boxplots_complexity}, \ref{Chart:Boxplots_inheritance}, and \ref{Chart:Boxplots_design size}} show the distribution of each metric before and after each of the examined commits.

\subsection{Pre-trained Model Selection (RQ1)} \label{subsec:rq1}

\sectopic{Methodology. }  We shortlist the ST models for investigation in our work based on the NLP  leaderboard, which reports the 38 most accurate pre-trained models\footnote{\url{https://www.sbert.net/docs/pretrained_models.html}}. These models have been extensively evaluated for their ability to generate sentence embeddings (i.e., capturing the semantics of the whole text) and their performance in semantic search (i.e., finding relevant answers to a given query). Both tasks closely align with our objectives. 
To identify trace links, we apply the pre-trained models in a zero-shot setting as follows. 
We let each model compute the similarity matrix equivalent to the output of step~5 in our approach (see Fig.~\ref{fig:approach}). 
We then predict a trace link if the similarity value exceeds 
a predefined threshold. Since zero-shot does not require training, we run EXPI on the entire \texttt{HIPAA} dataset. 


\sectopic{Evaluation Metrics. } To better assess the performance irrespective of the selected threshold, we compute the \textit{Area Under the Curve (AUC)} for the receiver operating characteristic (ROC) across different threshold values,  ranging from $0.1$ to $0.9$. 
The ROC curve captures the trade-off between the true positive rate (TPR) and the false positive rate (FPR). TPR is the proportion of positives correctly identified as such (i.e., the percentage of trace links correctly identified for a given threshold). FPR is the proportion of negatives incorrectly identified as positives (i.e., the percentage of trace links wrongly identified as not trace links). The AUC of the ROC curve (computed as micro-average over all the provisions to avoid the dominance of some provisions)  provides a single aggregate performance measure across all possible thresholds and, hence, is a suitable evaluation metric to compare the ST models.  We posit that the model with the highest AUC value demonstrates the best overall performance in identifying trace links in a zero-shot setting, as a higher AUC value indicates a better balance between correctly identifying true trace links (high TPR) and minimizing the identification of false links (low FPR). 
%
%
\sectopic{Results. }
Table~\ref{tab:rq1} presents the \texttt{AUC} values of the ST pre-trained models on the \texttt{HIPAA} dataset and also  reports $K$, indicating the ranking of the models in the NLP community based on their accuracy~\cite{Reimers:19}, as well as $K^\dag$, indicating the ranking based on \texttt{AUC} achieved on \texttt{HIPAA}. 



\begin{table}
%\footnotesize
\centering
\caption{AUC of ST models for LRT on \texttt{HIPAA} (\textbf{RQ1}). 
% \TBD{@Romina: come on! you don't leave footnotes on TRACES in the table. Please revise your changes. Also, you don't need "HIPAA" in the header if the results are only for HIPAA, @Romina: please remove and adjust the header accordingly}
}
\label{tab:rq1}
% \begin{threeparttable}[t]
\begin{tabularx}{\textwidth}{@{} p{0.05\textwidth} @{\hskip 0.5em} p{0.05\textwidth} @{\hskip 3em} p{0.05\textwidth} @{\hskip 20em} *{5}{>{\centering\arraybackslash}X}@{}}
    \toprule
    \multirow{1}{*}{$K$\tnote{1}} & \multirow{1}{*}{Model\tnote{2}} & \multirow{1}{*}{Name\tnote{1}} & \multirow{1}{*}{\texttt{AUC}\tnote{1}} & \multirow{1}{*}{$K^\dag$\tnote{1}} \\%\multicolumn{2}{c}{\texttt{HIPAA}} \\ %& \multicolumn{2}{c}{\texttt{TRACES}} & \multicolumn{2}{c}{Average}\\
    % \cmidrule(lr){4-5}
    % &&& \texttt{AUC} & $K^\dag$ \\ %&\texttt{AUC} & $K^\ddag$ &\texttt{AUC} & $K^*$  \\
    \midrule
1 &   \texttt{ST1}  & \texttt{all-mpnet-base-v2} & 0.744 & 16 \\ % & 0.331 & 29 & 0.538 & 27\\
2 &   \texttt{ST2}  & \texttt{gtr-t5-xxl} & 0.725 & 21 \\ % & \textbf{0.685} & 1 & 0.705 & 7\\
3 &   \texttt{ST3}  &\texttt{gtr-t5-xl} & 0.789 & 6 \\ % & 0.678 & 2 & 0.733 & 2\\
4 &   \texttt{ST4}  &\texttt{sentence-t5-xxl} & 0.720 & 22 \\ % & 0.666 & 3 & 0.693 & 8\\
5 &   \texttt{ST5}  &\texttt{gtr-t5-large} & 0.743 & 17 \\ % & 0.640 & 7 & 0.692 & 9\\
6 &   \texttt{ST6}  &\texttt{all-mpnet-base-v1} & 0.712 & 25 \\ % & 0.338 & 27 & 0.525 & 29\\
7 &   \texttt{ST7}  &\texttt{multi-qa-mpnet-base-dot-v1} & 0.688 & 27 \\ % & 0.631 & 8 & 0.659 & 12\\
8 &   \texttt{ST8}  &\texttt{multi-qa-mpnet-base-cos-v1} & 0.603 & 34 \\ % & 0.222 & 36 & 0.413 & 36\\
9 &   \texttt{ST9}  &\texttt{all-roberta-large-v1} & 0.601 & 35 \\ % & 0.333 & 28 & 0.467 & 34\\
10 &   \texttt{ST10}  &\texttt{sentence-t5-xl} & 0.769 & 10 \\ % & 0.644 & 6 & 0.706 & 5\\
11 &   \texttt{ST11}  &\texttt{all-distilroberta-v1} & 0.719 & 23 \\ % & 0.284 & 34 & 0.501 & 32\\
12 &   \texttt{ST12}  &\texttt{all-MiniLM-L12-v1} & 0.729 & 19 \\ % & 0.318 & 30 & 0.523 & 30\\
13 &   \texttt{ST13}  &\texttt{all-MiniLM-L12-v2} & 0.747 & 15 \\ % & 0.339 & 26 & 0.543 & 26\\
14 &   \texttt{ST14}  &\texttt{multi-qa-distilbert-dot-v1} & 0.563 & 36 \\ % & 0.546 & 17 & 0.555 & 25\\
15 &   \texttt{ST15}  &\texttt{multi-qa-distilbert-cos-v1} & 0.640 & 33 \\ % & 0.228 & 35 & 0.434 & 35\\
16 &   \texttt{ST16}  &\texttt{gtr-t5-base} & 0.770 & 9 \\ % & 0.655 & 5 & 0.712 & 4\\
17 &   \texttt{ST17}  &\texttt{sentence-t5-large} & 0.748 & 14 \\ % & 0.663 & 4 & 0.706 & 6\\
18 &   \texttt{ST18}  &\texttt{all-MiniLM-L6-v2} & 0.761 & 11 \\ % & 0.285 & 33 & 0.523 & 31\\
19 &   \texttt{ST19}  &\texttt{multi-qa-MiniLM-L6-cos-v1} & 0.670 & 29 \\ % & 0.313 & 31 & 0.492 & 33\\
20 &   \texttt{ST20}  &\texttt{all-MiniLM-L6-v1} & 0.749 & 13 \\ % & 0.307 & 32 & 0.528 & 28\\
21 &   \texttt{ST21}  &\texttt{paraphrase-mpnet-base-v2} & 0.850 & 2 \\ % & 0.587 & 14 & 0.719 & 3\\
22 &   \texttt{ST22}  &\texttt{msmarco-bert-base-dot-v5} & 0.644 & 32 \\ % & 0.503 & 20 & 0.574 & 24\\
23 &   \texttt{ST23}  & \texttt{multi-qa-MiniLM-L6-dot-v1} & 0.715 & 24 \\ % & 0.605 & 12 & 0.660 & 11\\
24 &   \texttt{ST24}  & \texttt{sentence-t5-base} & 0.726 & 20 \\ % & 0.584 & 15 & 0.655 & 13\\
25 &   \texttt{ST25}  & \texttt{msmarco-distilbert-base-tas-b} & 0.701 & 26 \\ % & 0.557 & 16 & 0.629 & 18\\
26 &   \texttt{ST26}  & \texttt{msmarco-distilbert-dot-v5} & 0.685 & 28 \\ % & 0.600 & 13 & 0.643 & 15\\
27 &   \texttt{ST27}  & \texttt{paraphrase-distilroberta-base-v2} & 0.801 & 4 \\ % & 0.455 & 24 & 0.628 & 19\\
28 &   \texttt{ST28}  & \texttt{paraphrase-MiniLM-L12-v2} & 0.794 & 5 \\ % & 0.496 & 22 & 0.645 & 14\\
29 &   \texttt{ST29}  & \texttt{paraphrase-multilingual-mpnet-base-v2} & \textbf{0.859} & 1 \\ % & 0.614 & 10 & \textbf{0.736} & 1\\
30 &   \texttt{ST30}  & \texttt{paraphrase-TinyBERT-L6-v2} & 0.787 & 7 \\ % & 0.464 & 23 & 0.625 & 21\\
31 &   \texttt{ST31}  & \texttt{paraphrase-MiniLM-L6-v2} & 0.770 & 8 \\ % & 0.511 & 18 & 0.641 & 16\\
32 &   \texttt{ST32}  & \texttt{paraphrase-albert-small-v2} & 0.737 & 18 \\ % & 0.499 & 21 & 0.618 & 22\\
33 &   \texttt{ST33}  & \texttt{paraphrase-multilingual-MiniLM-L12-v2} & 0.811 & 3 \\ % & 0.511 & 19 & 0.661 & 10\\
34 &   \texttt{ST34}  & \texttt{paraphrase-MiniLM-L3-v2} & 0.757 & 12 \\ % & 0.441 & 25 & 0.599 & 23\\
35 &   \texttt{ST35}  & \texttt{distiluse-base-multilingual-cased-v1} & 0.349 & 37 \\ % & 0.092 & 37 & 0.220 & 37\\
36 &   \texttt{ST36}  & \texttt{distiluse-base-multilingual-cased-v2} & 0.341 & 38 \\ % & 0.090 & 38 & 0.216 & 38\\
37 &   \texttt{ST37}  & \texttt{average\_word\_embeddings\_komninos} & 0.647 & 31 \\ % & 0.606 & 11 & 0.627 & 20\\
38 &   \texttt{ST38}  & \texttt{average\_word\_embeddings\_glove.6B.300d} & 0.636 & 30 \\ % & 0.625 & 9 & 0.630 & 17\\ 
\bottomrule
\end{tabularx}
\begin{tablenotes}
     \item[1] $K$: The average performance ranking of the models, as reported in the NLP community. $K^\dag$: The ranking of the models based on AUC values computed on \texttt{HIPAA} ($K=1$ indicates the highest AUC). 
      \item [2] \texttt{ST1}--\texttt{ST38} correspond to the models reported at this link (sorted by average accuracy in descending order):     \url{https://www.sbert.net/docs/pretrained_models.html}. %, where \texttt{ST29} is \texttt{paraphrase-multilingual-mpnet-base-v2}.
     \end{tablenotes}
 % \end{threeparttable}
 %\vspace*{-2em}
 \end{table}

 

The best-performing model on \texttt{HIPAA} is \texttt{ST29} ($K^\dag=1$), with an AUC value of 0.859. The next best performing model is \texttt{ST21} with an AUC value of 0.850. The difference between these two AUC values is only marginal. A possible explanation is that  \texttt{ST29} uses  \texttt{ST21} as its base model.  \texttt{ST29}  has been, however, trained on more (multi-lingual) data.   

Additionally, we observe a discrepancy in the performance of the models on the \texttt{HIPAA} dataset compared to that reported by the NLP community.  
The best NLP model, \texttt{ST1}, does not perform well  on \texttt{HIPAA}, ranked 16. 
This observation indicates that well-performing models in NLP are not necessarily as effective for RE-specific problems. 
%The datasets in RE are typically domain-specific increasing the level of complexity to deal with.    

\begin{tcolorbox}[arc=1mm,width=\columnwidth,
                  top=0mm,left=0mm,  right=0mm, bottom=0mm,
                  boxrule=1pt, colback=violet!15!white,colframe=white]
\textbf{The answer RQ1} is that \texttt{ST29} is the best-performing pre-trained model for LRT (corresponding to \texttt{paraphrase-multilingual-mpnet-base-v2}). 
\end{tcolorbox}%The goal of RQ1 is to select a robust ST model that performs consistently well across datasets. 
% Table~\ref{tab:rq1} shows the \texttt{AUC} values of the ST pre-trained models on the \texttt{HIPAA} and \texttt{TRACES} datasets. The table also reports $K$ indicating the ranking of the models in the NLP community based on their accuracy~\cite{Reimers:19}, as well as $K^\dag$,  $K^\ddag$ and $K^*$,  indicating the rankings based on \texttt{AUC} achieved on \texttt{HIPAA},  \texttt{TRACES} and on average across the two datasets, respectively. The AUC for the ROC curve metric enables fair comparison, irrespective of the selected threshold values. 

%\input{Files/tab1-RQ1}

% The table shows that the models perform considerably poorly on the \texttt{TRACES} dataset. A plausible reason is that \texttt{TRACES} has a total of 26 regulatory codes, some of which are seemingly closely related (e.g., the regulatory code \textit{TIM}---the period for which personal data is stored is semantically close to \textit{DUR}---the duration of data processing). 
% To reduce the degree of confusion that ST models exhibit, we compute the AUC values for \texttt{TRACES} at the category level \TBD{is this what we report in the table? (yes)}. Recall from Section~\ref{tab:datasets} that the 26 regulatory codes in  \texttt{TRACES} are grouped into 10 different categories (listed in Table~\ref{tab:datasets}). Once the ST model computes the similarity values of single regulatory codes, we then assign to each category the maximum similarity values among the single regulatory codes in that category. For example, \textit{TIM} and \textit{DUR} belong to the category \textit{data retention} (\textit{RTN} in Table~\ref{tab:datasets}). If the similarity value between a given requirement $r_i$ and \textit{TIM} and \textit{DUR} is 0.3 and 0.47, respectively, then we assign the similarity value 0.47 between $r_i$ and the category \textit{data retention}.  


% The table further shows a discrepancy in the performance of the models across our datasets compared to that reported by the NLP community.  
% The best NLP model, \texttt{ST1}, does not perform well on our datasets as it is ranked 16 on \texttt{HIPAA} and 29 on \texttt{TRACES}. This indicates that well-performing models in NLP are not necessarily robust for RE-specific problems where the models are confronted with datasets spanning specific-domains and potentially different requirement types.  

% The best-performing model on \texttt{HIPAA} is \texttt{ST29} ($K^\dag=1$), with an AUC value of 0.859. The same model, \texttt{ST29}, is however ranked 10 on \texttt{TRACES} with an AUC of 0.614, 0.07 lower than the best model \texttt{ST2} ($K^\ddag=1$). However, \texttt{ST2} yields  0.13 lower AUC value on \texttt{HIPAA} when compared with \texttt{ST29}. 
% Overall, \texttt{ST29} achieves the best average AUC value of 0.736 on both datasets \texttt{HIPAA} and \texttt{TRACES} ($K^*=1$), leaving \texttt{ST2} six ranks behind. 
% Additionally, we observe that, on average, \texttt{ST3} fares fairly close to \texttt{ST29}. Still, according to the NLP leaderboard, \texttt{ST29} has the advantage of being much faster and smaller in size than \texttt{ST3}: \texttt{ST29}'s size is 970 MB, whereas \texttt{ST3}'s size is 2370 MB. 

% \begin{tcolorbox}[arc=1mm,width=\columnwidth,
%                   top=0mm,left=0mm,  right=0mm, bottom=0mm,
%                   boxrule=1pt, colback=violet!15!white,colframe=white]
% In view of the above analysis, \textbf{the answer RQ1}, we select \texttt{ST29} (corresponding to \texttt{paraphrase-multilingual-mpnet-base-v2}) as the best-performing ST model in identifying trace links using a zero-shot setting. 
% \end{tcolorbox}


%\subsection{Is the developer's perception of code duplicate removal aligned with the quantitative assessment of code quality?}
%%%%%%%%%%%%%%%%%%%%%%%%%%%%%%%%%%%%%%%%%%%%%%%%%%%%%%%%%%%%%%%%%%%%%%%%%%%%%%%%%%%%%%%%%
%\subsubsection{Cohesion}
\noindent\textbf{\textcolor{black}{Cohesion.}}  For commits wherein the messages indicate the removal of code duplication, the boxplot depicted in Figure \ref{BP:chesion-lcom} illustrates the pre- and post-refactoring results of the normalized LCOM. This metric, commonly used in the literature to assess cohesion, is crucial in estimating the strength of cohesion within classes. A lower LCOM metric value generally suggests that classes should be split into one or more classes with better cohesion. Therefore, a low value for this metric signifies strong class cohesiveness. We specifically chose the normalized LCOM metric as it has been widely recognized in the literature  \citep{pantiuchina2018improving,chavez2017does,henderson1995object} as being the alternative to the original LCOM, by addressing its main limitations (artificial outliers, misperception of getters and setters, etc.). As can be seen from the boxplot in Figure \ref{BP:chesion-lcom}, the median drops from 1 to 0. This result indicates that LCOM is improved after code duplicate removal. Furthermore, as shown in Table~\ref{Table:Metrics Suites and Metrics Tools Summary}, LCOM has a positive impact on cohesion quality, as it decreases in the refactored code. This implies that developers did improve the cohesion of their classes. Table~\ref{Table:Metrics Suites and Metrics Tools Summary} shows that the differences in LCOM are statistically significant and the magnitude of the differences is large.

\noindent\textbf{Example (Positive Impact):} To illustrate an improvement in cohesion when the removal of duplicate code was found in the Maven project\footnote{\textcolor{black}{\url{https://github.com/apache/maven-surefire/commit/d5de47a4f790ea2d18edb5e05c1ef2adcd2db8a2}}}, developers applied `Move Class' refactoring to move \texttt{JUnit4TestCheckerTest.MySuite2} to \texttt{JUnit3TestCheckerTest.MySuite2}. This results in its LCOM5 dropping from 2 to 1. The improvement in the LCOM5 metric after the removal of code duplication could be attributed to the simplification of method interactions, the modularization of logic, the enhancement of code clarity, and the abstraction of common functionality.

%https://github.com/apache/maven-surefire/commit/d5de47a4f790ea2d18edb5e05c1ef2adcd2db8a2
\noindent\textbf{Example (Negative Impact):} To illustrate a decrease in cohesion when the removal of duplicated code was found in the Maven project\footnote{\textcolor{black}{\url{https://github.com/apache/maven-surefire/commit/d5de47a4f790ea2d18edb5e05c1ef2adcd2db8a2}}}, developers applied `Move Class' refactoring to move \texttt{JUnit4TestCheckerTest.NestedTC} to \texttt{JUnit3TestChecker\break Test.NestedTC}. This results in its LCOM5 increasing from 0 to 2. The LCOM5 metric might not have improved after removing duplicated code, as removal of duplicated code might not have substantially altered the underlying design and interactions of methods.  


\begin{figure*}
% row-1
\centering
%\begin{subfigure}{3.5cm}
\centering\includegraphics[width=3.5cm]{Images/LCOM5.png}
\caption{Cohesion - LCOM5}
\label{BP:chesion-lcom}
%\end{subfigure}%
\caption{\textcolor{black}{Boxplots of cohesion metric values of pre- and post-refactored files.}}
\label{Chart:Boxplots_cohesion}
\end{figure*}
%https://github.com/apache/maven-surefire/commit/d5de47a4f790ea2d18edb5e05c1ef2adcd2db8a2
% \begin{boxK}
% \textit{\textbf{Summary.} The normalized LCOM metric not only serves as a suitable substitute for the original LCOM but also serves as a representation of the cohesion quality attribute. A positive variation in this metric aligns with the developer's intention to eliminate code duplication.}
% \end{boxK}

%%%%%%%%%%%%%%%%%%%%%%%%%%%%%%%%%%%%%%%%%%%%%%%%%%%%%%%%%%%%%%%%%%%%%%%%%%%%%%%%%%%%%%%%%
%\subsubsection{Coupling}
\noindent\textbf{\textcolor{black}{Coupling.}} For commits with messages indicating the removal of code duplication, the boxplots presented in Figures \ref{BP:coupling-cbo}, \ref{BP:coupling-rfc}, \ref{BP:coupling-nii}, and \ref{BP:coupling-noi} show the pre- and post-refactoring results of four structural metrics, \ie CBO, RFC, NII, and NOI, used in the literature to estimate the coupling. The figures reveal that three of the coupling metrics exhibited an improvement in median values. For instance, CBO, RFC, and NOI medians decreased, respectively, from 6 to 3, from 5 to 1, and from 6 to 3, respectively. CBO counts the number of classes coupled to a particular class through method or attribute calls. Calls are counted in both directions. CBO values have significantly decreased, making it a good coupling representative.  The RFC, which measures the visibility of a class to outsider classes, has been reduced as developers intend to optimize coupling. According to our results, the variations are statistically significant and the magnitude of the differences is large for both metrics.  NOI, which represents the number of outgoing invocations, has also decreased, and Cliff’s delta value
indicates a small effect size. However, NII exhibits the opposite variation, and the effect size is large.

The manual inspection of the refactored code indicates that developers typically decrease coupling by reducing (1) the strength of dependencies that exist between classes, (2) the message flow of the classes, and (3) the number of inputs a method uses plus the number of subprograms that call this method. The code was improved as expected from the developer's intentions in their commit message.

\noindent\textbf{Example (Positive Impact):} One of the examples showing an improvement in coupling was found in the Maven project\footnote{\textcolor{black}{\url{https://github.com/apache/maven-surefire/commit/d5de47a4f790ea2d18edb5e05c1ef2adcd2db8a2}}}. Developers applied `Move Class' refactoring to move \texttt{JUnit4TestCheckerTest.MySuite2} to \texttt{JUnit3Test\break CheckerTest.MySuite2}. This results in its CBO dropping from 1 to 0, and its RFC from 2 to 1.  The improvement in the CBO and RFC metrics after the removal of code duplication can be related to the elimination of external dependencies and the simplification of method interactions. However, its NII and NOI remain unchanged.

%https://github.com/apache/maven-surefire/commit/d5de47a4f790ea2d18edb5e05c1ef2adcd2db8a2
\noindent\textbf{Example (Negative Impact):} One of the examples showing an increase in coupling was found in the Archiva project\footnote{\textcolor{black}{\url{https://github.com/apache/archiva/commit/26e9c3b257bed850d0e2f0bc9dc2d7f11381b789}}}. The developer applied `Extract Superclass' refactoring to extract \texttt{AbstractDiscoverer} from \texttt{AbstractArtifact\break Discoverer}. This results in its CBO increasing from 0 to 1, its RFC from 4 to 6, and its NOI from 0 to 2. However, its NII improves from 3 to 0. The lack of improvement in CBO, RFC, and NOI metrics after the removal of code duplication could be attributed to the specific nature of the duplication and its limited impact on class interactions, method hierarchies, and message handling. 
%which indicates that the number of dependencies does not decrease after performing the refactoring to remove code duplication. 
%https://github.com/apache/archiva/commit/26e9c3b257bed850d0e2f0bc9dc2d7f11381b789
% \begin{boxK}
% \textit{\textbf{Summary.} CBO, RFC, and NOI generally improve as the developer intends to eliminate code duplication, and their variation is significant. NII exhibits opposite variations in
% coupling.}
% \end{boxK}
\begin{figure*}
% row-1
\centering
\begin{subfigure}{3.5cm}
\centering\includegraphics[width=3.5cm]{Images/CBO.png}
\caption{Coupling - CBO}
\label{BP:coupling-cbo}
\end{subfigure}%
\begin{subfigure}{3.5cm}
\centering\includegraphics[width=3.5cm]{Images/RFC.png}
\caption{Coupling - RFC}
\label{BP:coupling-rfc}
\end{subfigure}%\vspace{9pt}
\begin{subfigure}{3.5cm}
\centering\includegraphics[width=3.5cm]{Images/NII.png}
\caption{Coupling - NII}
\label{BP:coupling-nii}
\end{subfigure}%\vspace{9pt}
\begin{subfigure}{3.5cm}
\centering\includegraphics[width=3.5cm]{Images/NOI.png}
\caption{Coupling - NOI}
\label{BP:coupling-noi}
\end{subfigure}%
\caption{\textcolor{black}{Boxplots of coupling metric values of pre- and post-refactored files.}}
\label{Chart:Boxplots_coupling}
\end{figure*}
%%%%%%%%%%%%%%%%%%%%%%%%%%%%%%%%%%%%%%%%%%%%%%%%%%%%%%%%%%%%%%%%%%%%%%%%%%%%%%%%%%%%%%%%%
%RFC exhibits an opposite variation to coupling, but it is not statistically significant.
%\subsubsection{Complexity}
\noindent\textbf{\textcolor{black}{Complexity.}} Regarding complexity metrics, we consider 16 literature metrics, shown in Table \ref{Table:Quality Metrics Used in This Study.}, to investigate the removal of duplicate code as perceived by developers. As seen in the boxplots in Figures \ref{BP:Complexity-cc}, \ref{BP:Complexity-wmc}, \ref{BP:Complexity-nl}, \ref{BP:coupling-nle}, \ref{BP:Complexity-hcpl}, \ref{BP:Complexity-hdif}, \ref{BP:Complexityheff}, \ref{BP:Complexity-hndb}, \ref{BP:Complexity-hpl}, \ref{BP:Complexity-hpv}, \ref{BP:Complexity-htrp}, \ref{BP:Complexity-hvol}, \ref{BP:Complexity-mims}, \ref{BP:Complexity-mi}, \ref{BP:Complexity-misei}, and \ref{BP:Complexity-mism},  we observe that CC, NL, and NLE remain unchanged, whereas the other 13 metrics experienced an improvement in the median values. The refactored duplicate code exhibits higher values for the four maintenance index-related complexity (\ie MIMS, MI, MISEI, and MISM). The higher values are desirable for these metrics, as shown in Table \ref{Table:Quality Metrics Used in This Study.}. Additionally, the duplicate code refactored shows lower values for the other metrics (\ie WMC, HCPL, HDIF, HEEF, HNDB, HPL, HPV, HTRP, and HVOL), where lower values are desirable after the application of refactoring.

%Furthermore, all the variations are statistically significant and the magnitude of the differences is large for both metrics. 


%Our results indicate that the two metrics represent the quality attribute of complexity that improved. 

In particular, through a manual inspection of the collected dataset, we observe that developers tend to reduce the number of local methods, simplify the structure statements, reduce the number of paths in the body of the code, and lower the nesting level of the control statements (\eg selection and loop statements) in the method body. %On the other hand, when we observe a significant increase in RFC, we notice that developers lower the complexity of methods by pulling them up in the hierarchy, and so they increase the number of inherited methods. 

As seen in Table \ref{Table:Metrics Suites and Metrics Tools Summary}, the \textit{p}-values obtained from all complexity metrics are statistically significant. The effect sizes
calculated in Cliff 's delta ($\delta$) are found to be large only for WMC, small for CC, MISEI, and MISM, and negligible for the remaining 12 metrics.

\noindent\textbf{Example (Positive Impact):} As an illustrative example, we refer to commit\footnote{\textcolor{black}{\url{https://github.com/apache/maven-surefire/commit/d5de47a4f790ea2d18edb5e05c1ef2adcd2db8a2}}} which implements `Move Class' refactoring to move \texttt{JUnit4TestChecker\break Test.MySuite2} to \texttt{JUnit3TestCheckerTest.MySuite2}. Its CC, NL, and NLE remain unaffected, and its WMC improves from 2 to 1. The unchanged CC could be due to the specific nature of the duplicated code, which might not have affected the control flow patterns significantly. However, the improved WMC could be due to consolidation, optimization, or simplification of methods due to the removal of duplicates. In another example\footnote{\textcolor{black}{\url{https://github.com/apache/kafka/commit/f7b7b4745541a576eb0219468263487b07bac959}}}, `Extract Method' refactoring has been applied by developers to extract \texttt{resume} from \texttt{addStreamTasks} to eliminate duplication. Its four maintainability index metrics, \ie MIMS, MI, MISEI, and MISM improved (44.59 to 75.78, 76.25 to 129.6, 56.64 to 153.82, and 33.12 to 89.95), respectively. The remaining complexity metrics, \ie HCPL, HDIF, HNDB, HPL, HPV, HTRP, HVOL, have also improved (343.36 to 23.50, 52.85 to 3, 1515.44 to 44, 184 to 11, 67 to 10, 3277.43 to 6.09, and 1116.16 to 36.54, respectively).
%https://github.com/apache/maven-surefire/commit/d5de47a4f790ea2d18edb5e05c1ef2adcd2db8a2

\noindent\textbf{Example (Negative Impact):} As an illustrative example, we refer to commit\footnote{\textcolor{black}{\url{https://github.com/apache/sis/commit/c8ffc0116b86f39caa3d2f45dca5dec68049c93e}}} which implements `Extract Superclass' refactoring to extract \texttt{Element} from \texttt{Copyright} and \texttt{Person}. Its CC increases from 0 to 0.11, its WMC increases from 3 to 45, its NL and NLE increase from 0 to 2. When referring to commit\footnote{\textcolor{black}{\url{https://github.com/apache/ant-ivy/commit/b74264847ef8e9ffeaf06d5fa1fdead4a065b480}}}, the `Extract Method' refactoring to extract \texttt{getReportFile} from \texttt{getRealDependencyRevisionIds} to remove duplication. Its four maintainability index metrics have not improved (75.71 to 64.27, 129.47 to 109.90, 155.64 to 83.06, 91,02, for MIMS, MI, MISEI, and MISM, respectively).   The remaining complexity metrics, \ie HCPL, HDIF, HNDB, HPL, HPV, HTRP, HVOL, have also not improved (61.30 to 120.76, 12.25 to 21, 108.09 to 317.87, 22 to 55, 18 to 30, 62.43 to 314.85, and 91.73 to 269.87, respectively). The absence of improvement in complexity metrics could be due to factors such as the nature of duplicated code, the distribution of complexity across the codebase, and the potential compensatory complexity introduced during the code duplicate removal process. 


%https://github.com/apache/sis/commit/c8ffc0116b86f39caa3d2f45dca5dec68049c93e
% \begin{boxK}
% \textit{\textbf{Summary.} CC, NL, and NLE remain unchanged, and the remaining 13 complexity-related metrics generally improve as the developer intends to improve code duplicate, and all their variation is significant.}
% \end{boxK}




\begin{figure*}
% row-1
\centering
\begin{subfigure}{3.5cm}
\centering\includegraphics[width=3.5cm]{Images/CC.png}
\caption{Complexity - CC}
\label{BP:Complexity-cc}
\end{subfigure}%
\begin{subfigure}{3.5cm}
\centering\includegraphics[width=3.5cm]{Images/WMC.png}
\caption{Complexity - WMC}
\label{BP:Complexity-wmc}
\end{subfigure}%
\begin{subfigure}{3.5cm}
\centering\includegraphics[width=3.5cm]{Images/NL.png}
\caption{Complexity - NL}
\label{BP:Complexity-nl}
\end{subfigure}%
\begin{subfigure}{3.5cm}
\centering\includegraphics[width=3.5cm]{Images/NLE.png}
\caption{Coupling - NLE}
\label{BP:coupling-nle}
\end{subfigure}%
\vspace{9pt}
\begin{subfigure}{3.5cm}
\centering\includegraphics[width=3.5cm]{Images/HCPL.png}
\caption{Complexity - HCPL}
\label{BP:Complexity-hcpl}
\end{subfigure}%
\begin{subfigure}{3.5cm}
\centering\includegraphics[width=3.5cm]{Images/HDIF.png}
\caption{Complexity - HDIF}
\label{BP:Complexity-hdif}
\end{subfigure}%
\begin{subfigure}{3.5cm}
\centering\includegraphics[width=3.5cm]{Images/HEFF.png}
\caption{Complexity - HEFF}
\label{BP:Complexityheff}
\end{subfigure}%
\begin{subfigure}{3.5cm}
\centering\includegraphics[width=3.5cm]{Images/HNDB.png}
\caption{Complexity - HNDB}
\label{BP:Complexity-hndb}
\end{subfigure}%
\vspace{9pt}
\begin{subfigure}{3.5cm}
\centering\includegraphics[width=3.5cm]{Images/HPL.png}
\caption{Complexity - HPL}
\label{BP:Complexity-hpl}
\end{subfigure}%
\begin{subfigure}{3.5cm}
\centering\includegraphics[width=3.5cm]{Images/HPV.png}
\caption{Complexity - HPV}
\label{BP:Complexity-hpv}
\end{subfigure}%
\begin{subfigure}{3.5cm}
\centering\includegraphics[width=3.5cm]{Images/HTRP.png}
\caption{Complexity - HTRP}
\label{BP:Complexity-htrp}
\end{subfigure}%
\begin{subfigure}{3.5cm}
\centering\includegraphics[width=3.5cm]{Images/HVOL.png}
\caption{Complexity - HVOL}
\label{BP:Complexity-hvol}
\end{subfigure}%
\vspace{9pt}
\begin{subfigure}{3.5cm}
\centering\includegraphics[width=3.5cm]{Images/MIMS.png}
\caption{Complexity - MIMS}
\label{BP:Complexity-mims}
\end{subfigure}%
\begin{subfigure}{3.5cm}
\centering\includegraphics[width=3.5cm]{Images/MI.png}
\caption{Complexity - MI}
\label{BP:Complexity-mi}
\end{subfigure}%
\begin{subfigure}{3.5cm}
\centering\includegraphics[width=3.5cm]{Images/MISEI.png}
\caption{Complexity - MISEI}
\label{BP:Complexity-misei}
\end{subfigure}%
\begin{subfigure}{3.5cm}
\centering\includegraphics[width=3.5cm]{Images/MISM.png}
\caption{Complexity - MISM}
\label{BP:Complexity-mism}
\end{subfigure}%
\caption{\textcolor{black}{Boxplots of complexity metric values of pre- and post-refactored files.}}
\label{Chart:Boxplots_complexity}
\end{figure*}



% \begin{figure*}
% % row-1
% \centering
% \begin{subfigure}{3.5cm}
% \centering\includegraphics[width=3.5cm]{Images/LCOM5.png}
% \caption{Cohesion - LCOM5}
% \label{BP:chesion-lcom}
% \end{subfigure}%
% \begin{subfigure}{3.5cm}
% \centering\includegraphics[width=3.5cm]{Images/CBO.png}
% \caption{Coupling - CBO}
% \label{BP:coupling-cbo}
% \end{subfigure}%
% \begin{subfigure}{3.5cm}
% \centering\includegraphics[width=3.5cm]{Images/RFC.png}
% \caption{Coupling - RFC}
% \label{BP:coupling-rfc}
% \end{subfigure}%\vspace{9pt}
% \begin{subfigure}{3.5cm}
% \centering\includegraphics[width=3.5cm]{Images/NII.png}
% \caption{Coupling - NII}
% \label{BP:coupling-nii}
% \end{subfigure}%\vspace{9pt}





% % row-2 
% \begin{subfigure}{3.5cm}
% \centering\includegraphics[width=3.5cm]{Images/NOI.png}
% \caption{Coupling - NOI}
% \label{BP:coupling-noi}
% \end{subfigure}%
% \begin{subfigure}{3.5cm}
% \centering\includegraphics[width=3.5cm]{Images/CC.png}
% \caption{Complexity - CC}
% \label{BP:Complexity-cc}
% \end{subfigure}%
% \begin{subfigure}{3.5cm}
% \centering\includegraphics[width=3.5cm]{Images/WMC.png}
% \caption{Complexity - WMC}
% \label{BP:Complexity-wmc}
% \end{subfigure}%
% \begin{subfigure}{3.5cm}
% \centering\includegraphics[width=3.5cm]{Images/NL.png}
% \caption{Complexity - NL}
% \label{BP:Complexity-nl}
% \end{subfigure}%


% % row-2 
% \begin{subfigure}{3.5cm}
% \centering\includegraphics[width=3.5cm]{Images/NLE.png}
% \caption{Coupling - NLE}
% \label{BP:coupling-nle}
% \end{subfigure}%
% \begin{subfigure}{3.5cm}
% \centering\includegraphics[width=3.5cm]{Images/HCPL.png}
% \caption{Complexity - HCPL}
% \label{BP:Complexity-hcpl}
% \end{subfigure}%
% \begin{subfigure}{3.5cm}
% \centering\includegraphics[width=3.5cm]{Images/HDIF.png}
% \caption{Complexity - HDIF}
% \label{BP:Complexity-hdif}
% \end{subfigure}%
% \begin{subfigure}{3.5cm}
% \centering\includegraphics[width=3.5cm]{Images/HEFF.png}
% \caption{Complexity - HEFF}
% \label{BP:Complexityheff}
% \end{subfigure}%




% \begin{subfigure}{3.5cm}
% \centering\includegraphics[width=3.5cm]{Images/HNDB.png}
% \caption{Complexity - HNDB}
% \label{BP:Complexity-hndb}
% \end{subfigure}%
% \begin{subfigure}{3.5cm}
% \centering\includegraphics[width=3.5cm]{Images/HPL.png}
% \caption{Complexity - HPL}
% \label{BP:Complexity-hpl}
% \end{subfigure}%
% \begin{subfigure}{3.5cm}
% \centering\includegraphics[width=3.5cm]{Images/HPV.png}
% \caption{Complexity - HPV}
% \label{BP:Complexity-hpv}
% \end{subfigure}%
% \begin{subfigure}{3.5cm}
% \centering\includegraphics[width=3.5cm]{Images/HTRP.png}
% \caption{Complexity - HTRP}
% \label{BP:Complexity-htrp}
% \end{subfigure}%

% \begin{subfigure}{3.5cm}
% \centering\includegraphics[width=3.5cm]{Images/HVOL.png}
% \caption{Complexity - HVOL}
% \label{BP:Complexity-hvol}
% \end{subfigure}%
% \begin{subfigure}{3.5cm}
% \centering\includegraphics[width=3.5cm]{Images/MIMS.png}
% \caption{Complexity - MIMS}
% \label{BP:Complexity-mims}
% \end{subfigure}%
% \begin{subfigure}{3.5cm}
% \centering\includegraphics[width=3.5cm]{Images/MI.png}
% \caption{Complexity - MI}
% \label{BP:Complexity-mi}
% \end{subfigure}%
% \begin{subfigure}{3.5cm}
% \centering\includegraphics[width=3.5cm]{Images/MISEI.png}
% \caption{Complexity - MISEI}
% \label{BP:Complexity-misei}
% \end{subfigure}%

% \begin{subfigure}{3.5cm}
% \centering\includegraphics[width=3.5cm]{Images/MISM.png}
% \caption{Complexity - MISM}
% \label{BP:Complexity-mism}
% \end{subfigure}%
% \begin{subfigure}{3.5cm}
% \centering\includegraphics[width=3.5cm]{Images/DIT.png}
% \caption{Inheritance - DIT}
% \label{BP:Inheritance-dit}
% \end{subfigure}%\vspace{9pt}
% \begin{subfigure}{3.5cm}
% \centering\includegraphics[width=3.5cm]{Images/NOC.png}
% \caption{Inheritance - NOC}
% \label{BP:Inheritance-noc}
% \end{subfigure}%
% \begin{subfigure}{3.5cm}
% \centering\includegraphics[width=3.5cm]{Images/NOA.png}
% \caption{Inheritance - NOA}
% \label{BP:Inheritance-noa}
% \end{subfigure}%



% % row 3
% \begin{subfigure}{3.5cm}
% \centering\includegraphics[width=3.5cm]{Images/LOC.png}
% \caption{Design Size - LOC}
% \label{BP:Design Size-loc}
% \end{subfigure}%
% \begin{subfigure}{3.5cm}
% \centering\includegraphics[width=3.5cm]{Images/TLOC.png}
% \caption{Design Size - TLOC}
% \label{BP:Design Size-tloc}
% \end{subfigure}%
% \begin{subfigure}{3.5cm}
% \centering\includegraphics[width=3.5cm]{Images/LLOC.png}
% \caption{Design Size - LLOC}
% \label{BP:Design Size-lloc}
% \end{subfigure}%
% \begin{subfigure}{3.5cm}
% \centering\includegraphics[width=3.5cm]{Images/TLLOC.png}
% \caption{Design Size - TLLOC}
% \label{BP:Design Size-tlloc}
% \end{subfigure}%


% \begin{subfigure}{3.5cm}
% \centering\includegraphics[width=3.5cm]{Images/CLOC.png}
% \caption{Design Size - CLOC}
% \label{BP:Design Size-cloc}
% \end{subfigure}%
% \begin{subfigure}{3.5cm}
% \centering\includegraphics[width=3.5cm]{Images/NPM.png}
% \caption{Design Size - NPM}
% \label{BP:Design Size-npm}
% \end{subfigure}%
% \begin{subfigure}{3.5cm}
% \centering\includegraphics[width=3.5cm]{Images/NOS.png}
% \caption{Design Size - NOS}
% \label{BP:Design Size-nos}
% \end{subfigure}%
% \begin{subfigure}{3.5cm}
% \centering\includegraphics[width=3.5cm]{Images/TNOS.png}
% \caption{Design Size - TNOS}
% \label{BP:Design Size-tnos}
% \end{subfigure}

% \caption{Boxplots of metrics values of pre- and post-refactored files.} 
% \label{Chart:Boxplots_Al_V1}
% \end{figure*}
%%%%%%%%%%%%%%%%%%%%%%%%%%%%%%%%%%%%%%%%%%%%%%%%%%%%%%%%%%%%%%%%%%%%%%%%%%%%%%%%%%%%%%%%%
%\subsubsection{Inheritance}
\noindent\textbf{\textcolor{black}{Inheritance.}} For commits that involve the removal of code duplication, the boxplots depicted in Figures  \ref{BP:Inheritance-dit}, \ref{BP:Inheritance-noc} and \ref{BP:Inheritance-noa} showcase the pre- and post-refactoring results of three structural metrics:  \ie DIT, NOC, and NOA, used in the literature to estimate the inheritance. We observe that only one metric among the three experienced a degradation in median values. Specifically, the median for NOC decreased from 3 to 0, while the median for DIT and NOA increased from 2 to 3 and from 3 to 4, respectively. This suggests that developers may be increasing the depth of the hierarchy by adding more methods for a class to inherit, reducing the number of immediate subclasses, and increasing the number of methods added by a subclass. While some instances show improvement in inheritance, the overall depth of the inheritance tree and the number of methods added by a subclass did not decrease.  The interpretation of the metric improvement depends highly on the quality of the code and the developer's design decisions. The statistical test shows that the differences are statistically significant for DIT, NOC, and NOA. The magnitude of the difference between the three metrics is large.

\noindent\textbf{Example (Positive Impact):} One of the examples that demonstrated improvement in inheritance was found
in a particular commit in the Maven project\footnote{\textcolor{black}{\url{https://github.com/apache/maven-surefire/commit/d5de47a4f790ea2d18edb5e05c1ef2adcd2db8a2}}}.  The developer applied `Move Class' refactoring to move \texttt{JUnit4TestChecker\break Test.CustomSuiteOnlyTest} to \texttt{JUnit3TestCheckerTest.CustomSuiteOnlyTest}. Its DIT drops from 1 to 0, its NOC remains unaffected, and its NOA improves from 1 to 0. This increases the reuse of common code logic and leads to more effective inheritance relationships and a better-defined hierarchy.
%https://github.com/apache/maven-surefire/commit/d5de47a4f790ea2d18edb5e05c1ef2adcd2db8a2

\noindent\textbf{Example (Negative Impact):} One of the examples that showed improvement in inheritance was found
in a particular commit in the Archiva project\footnote{\textcolor{black}{\url{https://github.com/apache/archiva/commit/26e9c3b257bed850d0e2f0bc9dc2d7f11381b789}}}.  The developer applied `Extract Superclass' refactoring to extract \texttt{AbstractDiscoverer} from class \texttt{AbstractArtifactDiscoverer}. Its DIT increases from 0 to 1, its NOC remains unaffected, and its NOA increases from 0 to 1. This indicates that the refactoring applied to remove duplication does not always improve inheritance metrics due to either pre-existing inheritance challenges, or the focused nature of the duplication removal. 
%https://github.com/apache/archiva/commit/26e9c3b257bed850d0e2f0bc9dc2d7f11381b789
% \begin{boxK}
% \textit{\textbf{Summary.} NOC generally decreases as the developer intends to remove code duplication, and its variation is significant. DIT and NOA exhibit opposite variations in inheritance. }
% \end{boxK}


\begin{figure*}
% row-1
\centering
\begin{subfigure}{3.5cm}
\centering\includegraphics[width=3.5cm]{Images/DIT.png}
\caption{Inheritance - DIT}
\label{BP:Inheritance-dit}
\end{subfigure}%\vspace{9pt}
\begin{subfigure}{3.5cm}
\centering\includegraphics[width=3.5cm]{Images/NOC.png}
\caption{Inheritance - NOC}
\label{BP:Inheritance-noc}
\end{subfigure}%
\begin{subfigure}{3.5cm}
\centering\includegraphics[width=3.5cm]{Images/NOA.png}
\caption{Inheritance - NOA}
\label{BP:Inheritance-noa}
\end{subfigure}%
\caption{\textcolor{black}{Boxplots of inheritance metric values of pre- and post-refactored files.}}
\label{Chart:Boxplots_inheritance}
\end{figure*}
%%%%%%%%%%%%%%%%%%%%%%%%%%%%%%%%%%%%%%%%%%%%%%%%%%%%%%%%%%%%%%%%%%%%%%%%%%%%%%%%%%%%%%%%%%
%\subsubsection{Design Size}
\noindent\textbf{\textcolor{black}{Design Size.}} For commits whose messages report the removal of code duplicate, the boxplots sketched in Figures \ref{BP:Design Size-loc}, \ref{BP:Design Size-tloc}, \ref{BP:Design Size-lloc}, \ref{BP:Design Size-tlloc}, \ref{BP:Design Size-cloc},  \ref{BP:Design Size-npm}, \ref{BP:Design Size-nos}, and \ref{BP:Design Size-tnos}  show the pre- and post-refactoring results of four structural metrics, \ie LOC, TLOC, LLOC, TLLOC, CLOC,  NPM, NOS, and TNOS, used in the literature to estimate the design size. We notice the improvement of six metrics, namely LOC, TLOC, LLOC, TLLOC, NOS, and TNOS after the commits in which developers explicitly target the improvement of code duplication, their variations are statistically significant.  
 The magnitude of LOC, TLOC, and TLLOC is small, whereas the magnitude for LLOC, NOS, and TNOS is negligible. As seen in the box plots, the medians generally decreased. However, we note that the medians for CLOC and NPM remain unchanged. The differences in CLOC and NPM are statistically significant, and the magnitude of the difference is negligible and large, respectively. This indicates that developers generally retain the lines containing comments and maintain the same number of methods after applying refactoring. %This shows us that developers added more lines of code plus more declarative and executable statements after the application of refactoring that might be because developer intentions is to improve the readability and the clarity of the code. 

\noindent\textbf{Example (Positive Impact):} As an illustrative example, we refer to the commit\footnote{\textcolor{black}{\url{https://github.com/apache/maven-surefire/commit/d5de47a4f790ea2d18edb5e05c1ef2adcd2db8a2}}} which implements `Extract Method' refactoring to extract \texttt{accept\break(testClass)} from \texttt{invalidTest}. Its LOC, TLOC, LLOC, TLLOC drop from 6 to 4, and its CLOC, NOS, and TNOS remain unaffected.  Furthermore, when moving the class \texttt{JUnit4TestCheckerTest.AlsoValid}  to \texttt{JUnit3TestChecker\break Test.AlsoValid}, its NPM improves from 1 to 0.  In qualitative terms, the removal of code duplication and the introduction of a dedicated method have led to more modular, focused, and readable code. This shows that size metrics capture the removal of code duplication as perceived by the developer.
%https://github.com/apache/maven-surefire/commit/d5de47a4f790ea2d18edb5e05c1ef2adcd2db8a2

\noindent\textbf{Example (Negative Impact):} As an illustrative example, we refer to the commit\footnote{\textcolor{black}{\url{https://github.com/apache/commons-bcel/commit/67dfdf60f5f8ccb8ed910bfe9d1cdc6e84f0db36}}} which implements `Extract Method' refactoring to extract \texttt{accept\break(createAnnotationEntries)} from \texttt{getAnnotationEntries}. Its LOC and TLOC increased from 7 to 11, its LLOC, TLLOC increased from 6 to 10, and its NOS and TNOS increased from 3 to 6. Its CLOC decreases from 3 to 1. The observed lack of improvement, in this case, can be attributed to a couple of factors, including the nature of the changes made, the extent of duplication and additional compensatory changes. This results in an overall increase in the class size as assessed by these employed design size metrics.
%https://github.com/apache/commons-bcel/commit/67dfdf60f5f8ccb8ed910bfe9d1cdc6e84f0db36
\begin{figure*}
% row-1
\centering
% row 3
\begin{subfigure}{3.5cm}
\centering\includegraphics[width=3.5cm]{Images/LOC.png}
\caption{Design Size - LOC}
\label{BP:Design Size-loc}
\end{subfigure}%
\begin{subfigure}{3.5cm}
\centering\includegraphics[width=3.5cm]{Images/TLOC.png}
\caption{Design Size - TLOC}
\label{BP:Design Size-tloc}
\end{subfigure}%
\begin{subfigure}{3.5cm}
\centering\includegraphics[width=3.5cm]{Images/LLOC.png}
\caption{Design Size - LLOC}
\label{BP:Design Size-lloc}
\end{subfigure}%
\begin{subfigure}{3.5cm}
\centering\includegraphics[width=3.5cm]{Images/TLLOC.png}
\caption{Design Size - TLLOC}
\label{BP:Design Size-tlloc}
\end{subfigure}%
\vspace{9pt}
\begin{subfigure}{3.5cm}
\centering\includegraphics[width=3.5cm]{Images/CLOC.png}
\caption{Design Size - CLOC}
\label{BP:Design Size-cloc}
\end{subfigure}%
\begin{subfigure}{3.5cm}
\centering\includegraphics[width=3.5cm]{Images/NPM.png}
\caption{Design Size - NPM}
\label{BP:Design Size-npm}
\end{subfigure}%
\begin{subfigure}{3.5cm}
\centering\includegraphics[width=3.5cm]{Images/NOS.png}
\caption{Design Size - NOS}
\label{BP:Design Size-nos}
\end{subfigure}%
\begin{subfigure}{3.5cm}
\centering\includegraphics[width=3.5cm]{Images/TNOS.png}
\caption{Design Size - TNOS}
\label{BP:Design Size-tnos}
\end{subfigure}

\caption{\textcolor{black}{Boxplots of design size metric values of pre- and post-refactored files.}}
\label{Chart:Boxplots_design size}
\end{figure*}
% \begin{boxK}
% \textit{\textbf{Summary.} LOC, TLOC, LLOC, TLLOC, NOS, TNOS generally improve as developers intend to remove code duplication, and their variations are significant. These metrics have a significant positive variation which matches the developer's perception of removing code duplicates.}
% \end{boxK}


\noindent\textbf{\textcolor{black}{Summary.}} \textcolor{black}{This \textcolor{black}{section} summarizes our findings and their implications.}

\begin{itemize}
    \item \textcolor{black}{\textbf{Cohesion.} The normalized LCOM metric not only serves as a suitable substitute for the original LCOM but also serves as a representation of the cohesion quality attribute. A positive variation in this metric aligns with the developer's intention to eliminate code duplication.}
    \item \textcolor{black}{\textbf{Coupling.} CBO, RFC, and NOI generally improve as the developer intends to eliminate code duplication, and their variation is significant. NII exhibits opposite variations in
coupling.}
    \item \textcolor{black}{\textbf{Complexity.} CC, NL, and NLE remain unchanged, and the remaining 13 complexity-related metrics generally improve as the developer intends to improve code duplicate, and all their variation is significant.}
    \item \textcolor{black}{\textbf{Inheritance.} NOC generally decreases as the developer intends to remove code duplication, and its variation is significant. DIT and NOA exhibit opposite variations in inheritance.}
    \item \textcolor{black}{\textbf{Design Size.} LOC, TLOC, LLOC, TLLOC, NOS, TNOS generally improve as developers intend to remove code duplication, and their variations are significant. These metrics have a significant positive variation which matches the developer's perception of removing code duplicates.}
\end{itemize}

%\begin{comment}

\subsection{What are the refactoring operations that are associated with code duplicate removal?}
\begin{figure}[t]
\centering 
\begin{tikzpicture}
\begin{scope}[scale=0.8]
\pie[rotate = 180,pos ={0,0},text=inside,outside under=20,no number]{55.7/Extract Method\and55.7\%, 37.5/Move Method\and37.5\%, 3.8/Extract Superclass\and3.8\%,2.7/Move Attribute\and2.7\%,0.3/Move Class\and0.3\%}
\end{scope}
\end{tikzpicture}
\caption{Distribution of refactoring operations for code duplicate removal.}
\label{fig:refactoringtypes}
%\vspace{-.6cm}
\end{figure}
Looking at the refactoring operations that could play a role in code duplicate removal, Figure \ref{fig:refactoringtypes} depicts the percentages of refactoring operations. \textcolor{black}{As can be seen, the most common category concerns `Extract Method', representing 55.7\% of the commits. This observation is in line with the findings of previous studies describing that `Extract Method' refactoring is considered \say{Swiss army knife} of
refactorings as developers often apply it to eliminate duplicated code \citep{higo2004aries,higo2005aries,higo2008metric,tairas2012increasing,bian2013spape,yue2018automatic,yoshida2019proactive,arcelli2015duplicated,alomar2022anticopypaster,alomar2023just,alomar2024behind}. In fact, a recent study on extract method refactoring highlights that method extraction is one of the main refactorings that were defined when the area was established \citep{alomar2024behind,griswold1993automated}, as it is a common response to the need to keep methods concise and modular, and reduced the spread of shared responsibilities.} The next most common categories are `Move Method', representing 37.5\% of the commits. This indicates that developers might improve the quality of the code by moving the method containing duplication to a different class, effectively eliminating duplicated code. The category `Extract Superclass', `Move Attribute', and `Move Class' had the least number of commits, which had a ratio of 3.8\%, 2.7\%, and 0.3\%, respectively. 

When performing manual inspection of source code, we notice that these five refactoring operations contribute to the elimination of code duplication in several ways. By performing the `Extract Method' refactoring, redundant code segments can be consolidated into a single method that can be reused across different parts of the codebase. Additionally, when moving methods from one class to another using `Move Method' refactoring, it helps centralize logic and eliminate duplicate code that might have been present in multiple classes. Moreover, by extracting a superclass using `Extract Superclass' refactoring, it encapsulates common attributes and behaviors of related classes, allowing duplicated code to be consolidated. This can be followed by moving shared attributes to a common superclass using `Move Attribute' refactoring to reduce redundancy and ensures that changes to these attributes are reflected across all subclasses. Finally, moving the entire class to a common location using `Move Class' refactoring can help in reducing duplicated code, and it is useful when classes share similar functionality but exist in different parts of the codebase.

%\end{comment}

\section{Lessons Learned}
\label{Section:lesson}


\begin{figure*}[htbp]
\centering 
%\includegraphics[width=8cm,height=13cm,keepaspectratio]{Images/MSR-Example-v3.png}
%\includegraphics[width=16cm,height=8cm,keepaspectratio]{Images/MSR-Example-v3.png}
\includegraphics[width=1.2\textwidth]{Images/Example-type1.pdf}
%\vspace{-.5cm}
\caption{\textcolor{black}{Example of selected Type-1 code clone from kafka project.}} %\cite{kafka-type1}
\label{fig:example-casestudies-type1}
%\vspace{-.4cm}
\end{figure*}

 \begin{figure*}[htbp]
\centering 
% %\includegraphics[width=8cm,height=13cm,keepaspectratio]{Images/MSR-Example-v3.png}
% %\includegraphics[width=16cm,height=8cm,keepaspectratio]{Images/MSR-Example-v3.png}
 \includegraphics[width=1.2\textwidth]{Images/Example-type2.pdf}
% %\vspace{-.5cm}
 \caption{\textcolor{black}{Example of selected Type-2 code clone from cayenne project.}} %\cite{cayenne}
 \label{fig:example-casestudies-type2}
%\vspace{-.4cm}
\end{figure*}

\begin{figure*}[htbp]
\centering 
%\includegraphics[width=8cm,height=13cm,keepaspectratio]{Images/MSR-Example-v3.png}
%\includegraphics[width=16cm,height=8cm,keepaspectratio]{Images/MSR-Example-v3.png}
\includegraphics[width=1.3\textwidth]{Images/Example-type3.pdf}
%\vspace{-.5cm}
\caption{\textcolor{black}{Example of selected Type-3 code clone from pig project.}} %\cite{pig}
\label{fig:example-casestudies-type3}
%\vspace{-.4cm}
\end{figure*}

\noindent{\textbf{ \textcolor{black}{Lesson 1: Code clones associated with commits about duplicate removal are from different clone types.}}}  \textcolor{black}{There are various types of code clone exist in the literature (\ie Type-1, Type-2, Type-3, and Type-4) \citep{mondal2020survey}. When performing manual examination of commits associated with code clones, we realized that some commits with the explicit intention of removing duplication are associated with different clone types. Furthermore, in some commits associated with duplicate removal, developers can combine clone refactoring with other unrelated changes, such as feature updates, bug fixes, or general code cleanup. This observation is consistent with existing studies that show that developers interleave refactoring with other changes, and 11– 39\% of bug fixing commits include other changes \cite{silva2016we,alomar2021we,murphy2012we,nguyen2013filtering}.}



%As shown in previous studies (\eg \citep{pantiuchina2018improving,alomar2019impact}), this tangling can make it difficult to attribute the impact to a specific code change alone precisely. }

\noindent{\textbf{ \textcolor{black}{Lesson 2: Refactoring different types of clones can have different variations on metric values.}}}  \textcolor{black}{As illustrated in \textcolor{black}{Tables \ref{Table:Quality Metrics in Related Work} and \ref{Table:Quality Metrics in Related Work-v2}}, there have been two decades' worth of work on the relationship between refactoring and code quality. We can see that there is room for empirical investigation of the impact of clone removal refactorings on internal quality metrics. In this study, we observe that the impact of refactoring clones on software quality metrics can vary based on the type of clone being refactored. Moreover, developers may have various mechanisms that contribute to removing duplicates, and these strategies may dictate different variations on the metrics. However, locating refactored clone types for each instance presents multiple challenges: (1) a single commit can address multiple clone types simultaneously, making it difficult to attribute metric variations to a specific clone type;  (2) some clone types may occur less frequently in the dataset, further complicating efforts to draw conclusions regarding the influence of clone types on metric variations; and (3) manually determining clone types for each instance is time-consuming and prone to error, particularly when dealing with a large dataset. Although existing clone detection tools can detect the clone, they require additional configuration and setup by the users. In the following, we show an example of each type of clone and its refactoring:}
%However, it is challenging and time-consuming to determine the types of clones and manually refactor them. Although existing clone detection tools can detect the clone, they require additional configuration and setup by the users, which might make them reluctant to perform refactoring suggestions afterwords.}
\begin{itemize}
    \item \textcolor{black}{\textit{Type-1 code clone.} Figure \ref{fig:example-casestudies-type1} illustrates a Type-1 clone that has been refactored. The example demonstrates two duplicate instances, which represent a Type-1 clone (\ie identical code fragments). An `Extract Method' refactoring was applied, resulting in the extraction of the method \texttt{putNodeGroup\break Name(nodeName String, nodeGroupId int, nodeGroups Map, rootTo\break NodeGroup Map)} from \texttt{makeNodeGroups()} in the \texttt{InternalTopologyBuilder} class. For the complexity metrics, we observed varied behavior: CC remained unchanged, some metrics showed improvement (NL, NLE, HEFF, HPL, and HTRP), while others did not improve (HCPL, HDIF, HNDB, HPV, HVOL, MIMS, MI, MISEI, and MISM). Regarding the size metrics, none showed improvement. For coupling metrics, NII improved, whereas NOI did not.} %Extract Method private putNodeGroupName(nodeName String, nodeGroupId int, nodeGroups Map, rootToNodeGroup Map) : int extracted from private makeNodeGroups() : Map in class org.apache.kafka.streams.processor.internals.InternalTopologyBuilder
    \item \textcolor{black}{\textit{Type-2 code clone.} Figure \ref{fig:example-casestudies-type2} depicts a Type-2 clone that has been refactored. This example highlights two duplicate instances, categorized as a Type-2 clone (\ie syntactically identical fragments). The method \texttt{entitiesForCurrentMode()} was extracted from \texttt{generateClassPairs\_1\_1\break (classTemplate String, superTemplate String, superPrefix String)} in the \texttt{MapClassGenerator} class using the `Extract Method' refactoring operation. The complexity metrics have shown improvement, while the design size metrics have also improved, with the exception of CLOC. For coupling metrics, NOI has improved, whereas NII has not.}
    %Extract Method private entitiesForCurrentMode() : Collection extracted from private generateClassPairs_1_1(classTemplate String, superTemplate String, superPrefix String) : void in class org.apache.cayenne.gen.MapClassGenerator
    \item \textcolor{black}{\textit{Type-3 code clone.} Figure \ref{fig:example-casestudies-type3} shows a Type-3 clone that has been refactored. The example illustrates two duplicate instances, identified as a Type-3 clone (\ie copied fragments with further modifications such as changed, added, or removed statements). Through the `Extract Method' refactoring, the method \texttt{runSimpleScript(String name, String[] script)}  was extracted in the \texttt{TestScriptLanguage} class. The complexity metrics have improved overall, with the exception of HDIF, which has decreased, while NL and NLE remain unchanged. Size metrics have also improved, except for CLOC. For coupling metrics, NOI has improved, but NII has decreased. }
    %Extract Method private runPigRunner(name String, script String[]) : PigStats extracted from public pigRunnerTest() : void in class org.apache.pig.test.TestScriptLanguage
\end{itemize}








\noindent{\textbf{ \textcolor{black}{Lesson 3: Some state-of-the-art metrics can capture the developer’s intention of removing code duplication with different degrees of improvement and degradation of software quality.}}} \textcolor{black}{When removing code duplication, developers often perform `Extract Method' refactoring with the expectation of improving code quality. Yet, the state-of-the-art metrics may reflect varying levels of improvement or even degradation following these refactoring events.  For example, in Figure \ref{fig:example-casestudies-2}, we demonstrate the code snippet depicting the instances of code duplication before and after refactoring. We can see that refactoring mining tools detect `Extract Method' refactoring from project commoms-bcel\footnote{\textcolor{black}{\url{https://github.com/apache/commons-bcel/commit/67dfdf60f5f8ccb8ed910bfe9d1cdc6e84f0db36}}}  to extract \texttt{createAnnotationEntries} from \texttt{getAnnotationEntries}. This example emphasizes how refactoring can have mixed effects, positively influencing some metrics while negatively impacting others. As can be seen, its coupling metrics (NII and NOI) have been improved. However, its complexity metrics (CC, NL, NLE, HCPL, HDIF, HEFF, HNDB, HPL, HPV, HTRP, HVOL, MIMS, MI, MISEI, and MISM) and size metrics (LOC, TLOC, LLOC, TLLOC, CLOC, NOS, and TNOS) have not been improved. For metrics where the metrics do not capture the developer's intention, several possible explanations can be consideblack: }
\begin{itemize}
    \item \textcolor{black}{\textit{Inadequacy of the metrics for certain scenarios.} The metrics used to assess software quality, may not always be the most suitable for reflecting the specific intention behind a refactoring. For instance, a developer may intend to improve readability or maintainability, but standard structural metrics may not effectively quantify these aspects. This misalignment between developer goals and the measublack outcomes can lead to discrepancies in how the impact of refactoring is perceived.}
    \item \textcolor{black}{\textit{Limitations of the metrics.} The state-of-the-art metrics have inherent limitations and may not comprehensively capture the effects of refactoring. For example, metrics such as CC focus on the control flow but may overlook improvements in code modularity. This indicates a need to either refine existing metrics or introduce new ones that better align with developer goals, particularly in cases of complex refactoring.}
    \item \textcolor{black}{\textit{Deviation from developer intentions.} In some cases, developers' intentions, as stated in commit messages, may not align with the actual changes performed in the codebase. This could happen for various reasons. For example, a commit message may report the removal of duplicate code, but the implementation might only partially address the duplication or introduce new dependencies, resulting in no measurable improvement or even metric degradation.}
\end{itemize}

\begin{figure*}[htbp]
\centering 
%\includegraphics[width=8cm,height=13cm,keepaspectratio]{Images/MSR-Example-v3.png}
%\includegraphics[width=16cm,height=8cm,keepaspectratio]{Images/MSR-Example-v3.png}
\includegraphics[width=1.2\textwidth]{Images/Example2.pdf}
%\vspace{-.5cm}
%\captionsetup{justification=raggedright,singlelinecheck=false}  % Left-aligns caption

% \caption{\textcolor{black}{Example of selected commit message from commons-bcel project\textsuperscript{\ref{fig:example-casestudies-2}}.}}
 \caption{\textcolor{black}{Example of selected commit message from commons-bcel project.}}
 \label{fig:example-casestudies-2}
\end{figure*}

%https://github.com/apache/commons-bcel/commit/67dfdf60f5f8ccb8ed910bfe9d1cdc6e84f0db36
% \label{fig:example-casestudies-2}
% %\end{figure*}
% \footnotetext[\value{footnote}]{\label{fig:example-casestudies-2}\url{https://github.com/apache/commons-bcel/commit/67dfdf60f5f8ccb8ed910bfe9d1cdc6e84f0db36}.}



%\vspace{-.4cm}
%\end{figure*}
%\footnotetext{\url{https://github.com/apache/commons-bcel/commit/67dfdf60f5f8ccb8ed910bfe9d1cdc6e84f0db36}.} 
% \caption{\textcolor{black}{Example of selected commit message from commons-bcel project\footnotetext{\url{https://github.com/apache/commons-bcel/commit/67dfdf60f5f8ccb8ed910bfe9d1cdc6e84f0db36}.}}}
%%%% just an example %%%%
% \begin{figure}[h]
%     \centering
%     \includegraphics[width=0.7\textwidth]{example-image}  % Replace with actual image
%     \caption{\textcolor{black}{Example of selected commit message from commons-bcel project\textsuperscript{*}.}}
%     \label{fig:commit-example}
% \end{figure}

% \begin{center}
%     \footnotesize\textsuperscript{*}\url{https://github.com/apache/commons-bcel/commit/67dfdf60f5f8ccb8ed910bfe9d1cdc6e84f0db36}
% \end{center}


%For the cases that metrics  do not capture the intention of developer, it might be because these metrics are not the best metrics to consider on these cases, limitation of the metrics itself, or developers did not follow their intention reported in the commit messages.

%This is a thorough empirical investigation of a long standing question within the refactoring community: does refactoring improve code quality?

%There has been two decades worth of work on the relationship between refactoring and code quality but it has been nicely summarized herein. The authors indeed illustrate that there is room for an empirical investigation of the impact of clone removal refactorings on internal product metrics.

%If LCOM increases it may be correlated with removal of duplicated code but it will also be correlated with lots of other manipulations of code. And if NOC decreases it may be a symptom of clone removal but can just as well be a symptom of other code manipulations.

%One of the strengths is that the author shows anecdotal evidence (both positive and negative) of impact clone removing refactorings have on quality metrics. However, almost all positive anecdotal evidence stems from refactoring JUnit4TestCheckerTest. It has been shown that test code is sufficiently different from normal code with respect to clones. (cfr. Brent van Bladel and Serge Demeyer. A comparative study of test code clones and production code clones. Journal of Systems and Software, 176:110940
\section{Implications}
\label{Section:Implication}

%The main implications of this study are as follows.

 \noindent{\textbf{ Further advancing quality metrics and duplicate code removal.}} The existing literature discusses various automatic refactoring approaches aimed at assisting practitioners in detecting antipatterns or code smells. Baqais and Alshayeb \citep{baqais2020automatic} have highlighted the growing interest in automatic refactoring studies. The researchers explored the potential of machine learning to identify refactoring opportunities. Since features play a vital role in the quality of machine learning models obtained, this study can contribute to determining which metrics can serve as effective features in machine learning algorithms, facilitating the accurate recommendation of refactoring opportunities at different levels of granularity (\ie class, method, field), which can assist developers in automatically making their decisions. For example, incorporating the most impactful metrics as features in predicting whether a given piece of code should undergo a specific refactoring operation can enhance developers' confidence in accepting recommended refactorings or selecting the most suitable refactoring candidate. This knowledge is needed because, in practice, the built model should require as little data as possible. Furthermore, since we observe that some of the quality metrics did not capture any improvement, we plan to conduct more experiments to validate the effectiveness of these metrics to explore whether the observations are due to the appropriateness of the quality metrics or to the needed validation and clarity of the perception of the developers.

 \noindent{\textbf{ Putting developer in the loop when designing refactoring recommendation systems.}} Based on the findings, it becomes evident that different structural metrics have the capacity to depict code duplication, thereby influencing software quality in diverse ways. Certain metrics improve software quality, whereas others might result in its decline. %This variation highlights that not all metrics align with developers' intentions, as documented in their commit messages. 
 This underscores the importance of involving developers in the design of refactoring recommendation systems, effectively engaging them in the process. This approach emerges as effective in discerning meaningful refactorings that align with the perspectives of developers \citep{hall2012supervised,bavota2012putting,pantiuchina2018improving}.


 \noindent{\textbf{ Examining the code duplicate removal potentials with refactoring.} Our study reveals the context in which developers refactor the code to eliminate code duplicates. Our future research direction can focus on providing a comprehensive taxonomy for code duplication-aware refactoring practices. This taxonomy can show various contexts of code duplicates and refactoring and can demonstrate different forms of code reuse. Thereafter, researchers can build on top of our findings to better understand developer practices and investigate to what extent this taxonomy for refactoring with awareness of duplicate code improves the system's quality.


 \noindent\textbf{ Understanding the completeness of the quality metric capturing duplicate code removal as documented by developers.} We observe that not all quality metrics can capture the improvement in duplicate code removal perceived by developers in their commit messages. Although quality metrics can help pinpoint design flaws for refactoring recommendation systems, such a recommendation would be meaningful if qualitative insights from developers complemented it. Furthermore, the alignment or disparity between the enhancement of software quality as perceived by developers and its evaluation through quality metrics can be attributed to factors such as the focused nature of the duplication removal, the extent of duplication, and the potential compensatory changes. Future research is encouraged to consider the direct effect of duplicate removal and the broader context of code changes and their implications for quality metrics. 

%\faThumbTack \noindent{\textbf{ Lack of clarity of how the code duplicate removal tools leverage metrics and decide the associated threshold to make the decision.}} Existing `Extract Method' refactoring tools such as \texttt{AntiCopyPaster} \citep{alomar2022anticopypaster,alomar2023just} used 78 metrics related to size, complexity, coupling, and keywords to extract duplicate code. On the contrary, \texttt{Aries} \citep{higo2005aries} used six other metrics to identify the removal of the code clone. However, the implementation of these metrics may vary between these tools based on the context. In addition, there may be cases where different metric names are used to improve some quality attributes. This phenomenon might affect the interpretation of the accuracy of the recommended code duplicate removal tools. 

 \noindent{\textbf{ Investigating the characteristics and effects of eliminating code duplication on software quality.} The results advance our understanding of the effects of eliminating code duplication on software quality. It is evident that certain software quality metrics can be used as indicators for code fragments that are more likely to be extracted and identified as problematic and should be removed by refactoring. Consequently, a threshold can be established to show when quality metrics reach a level where duplicate code will have a negative effect on maintenance and need to be refactored.

%he outcome underscores that while code duplication removal can impact various aspects of a codebase, its effects on inheritance metrics might be context-dependent and influenced by the specific structure of the codebase.
\section{Threats to Validity}
\label{Section:Threats}

In this section, we describe potential threats to the validity of our research method and the actions we took to mitigate them.

\textbf{Internal Validity.} The accuracy of our analysis is primarily dependent on the precision of the refactoring mining tools, as these tools may miss the detection of some refactorings. However, previous studies \citep{silva2016we,tsantalis2018accurate,silva2017refdiff} report that \texttt{RefactoringMiner} and \texttt{RefDiff} have high precision and recall scores compared to other state-of-the-art refactoring detection tools, giving us confidence in using the tools. Another potential threat to validity is related to commit messages. \textcolor{black}{This study does not exclude commits containing tangle code changes \citep{herzig2016impact,kirinuki2014hey}, where developers made changes related to different tasks and one of these tasks could be related to quality improvement. If these changes were committed at once, there is a possibility that the individual changes merge and that the original task cannot be traced back. Similarly to the previous study \cite{pantiuchina2018improving}, we did not consider filtering out such changes in this study}. Moreover, our manual analysis is time-consuming and error-prone, which we tried to mitigate by focusing mainly on commits known to contain refactorings. 

Another potential threat to validity is sample bias, where the choice of the data can directly impact the results. Therefore, we explored a large sample of projects from the SmartSHARK dataset \citep{trautsch2021msr}, to ensure the quality of the findings and diversify the sources to reduce the bias of the data belonging to the same entity. The qualitative analysis was conducted by a single author, which could introduce bias into the process. However, commits that were debatable were discarded. We also provide our dataset online for further refinement and analysis. %During our qualitative analysis, we consideblack only commits where a consensus between authors was reached on whether a message clearly states the removal of duplicate code. Commits that were debatable were discarded. We also provide our dataset online for further refinement and analysis.

\textbf{Construct Validity.} A potential threat to construct validity relates to the set of metrics, as it may miss some properties of the selected internal quality attributes. To address this potential threat, we mitigate it by choosing well-known metrics that encompass various properties of each attribute, as reported in the literature \citep{chidamber1994metrics}.

\textbf{External Validity.} Our analysis was limited to only open-source Java projects. However, we were able to examine 128 projects, which were well-commented and exhibited diversity in terms of size, contributors, number of commits, and refactorings. \textcolor{black}{Still, we believe that the results found in this study are largely language-agnostic. However, certain language-specific characteristics, such as syntax complexity and tooling support, can influence duplication patterns. Although we expect similar trends across languages with similar paradigms, a comprehensive analysis encompassing various languages is recommended to confirm this generalization.}

%Still, we believe that the removal of duplicates is largely language-agnostic. However, certain language-specific characteristics, such as syntax complexity and tooling support, can influence duplication patterns. Although we expect similar trends across languages with similar paradigms, a comprehensive analysis encompassing various languages is suggested to is recommended to confirm this generalization.
\section{Conclusion}\label{sec:conclusion}
This work introduces a novel approach to TOT query elicitation, leveraging LLMs and human participants to move beyond the limitations of CQA-based datasets. Through system rank correlation and linguistic similarity validation, we demonstrate that LLM- and human-elicited queries can effectively support the simulated evaluation of TOT retrieval systems. Our findings highlight the potential for expanding TOT retrieval research into underrepresented domains while ensuring scalability and reproducibility. The released datasets and source code provide a foundation for future research, enabling further advancements in TOT retrieval evaluation and system development.
% % 
% 
The widespread integration of communication networks and smart devices in modern control systems has increased the vulnerability of industrial systems to online cyber-attacks, e.g., Industroyer, Blackenergy, etc \citep{osti_1505628}.
% Modern control systems have seen a large push to include communication networks and smart devices to increase performance, made possible by improvements in communication device cost and energy consumption. This trend has been coupled with the usage of open-standard communication protocols among industrial control systems, making them vulnerable to online cyber-attacks such as Industroyer, Blackenergy, etc \citep{osti_1505628}. 
To counter this, methods have been developed to improve security by achieving attack detection, mitigation, and monitoring, among others \citep{sandberg2022secure}. This paper focuses on active attack diagnosis to mitigate stealthy attacks. 
%
%\subsection{Literature review}

Active diagnosis techniques rely on the inclusion of additional moduli to control systems
% inclusion within the control system of additional moduli 
to alter the behavior of the system compared to information known by the attacker. 
For instance, the concept of additive watermarking was introduced in \cite{mo2015physical}, where noise signals of known mean and variance are added at the plant and compensated for it at the controller. 
This compensation, however, is not exact, causing some performance degradation. Thus, trade-offs between performance and detectability  are necessary \citep{zhu2023detection}.
% A later work \citep{zhu2023detection} designs the watermark signal by trading performance for detection. Thus, although additive watermarking serves as a good detection scheme, they endure performance losses even in the nominal case. 

In encrypted control \citep{darup2021encrypted}, the sensor data is encrypted, sent to the controller, and then operated on directly. Encrypted input signals are sent back to the plant for decryption. Although encryption is widespread in IT security, in control systems it presents some concerns, such as the introduction of time delays \citep{stabile2024verifiable}, while it may present inherent weaknesses \citep{alisic2023model}.
% they are not preferred as they introduce time delays \citep{stabile2024verifiable} which can cause instability, and some encryption schemes can be very weak  \citep{alisic2023model}. 

In moving target defense \citep{griffioen2020moving}, the plant is augmented with fictitious dynamics, known to the controller. The plant output is transmitted to the controller along with the fictitious states over a network under attack. 
The additional measurements then aide in the detection of attacks. 
This comes at the cost of higher communication bandwidth needs, which increases rapidly with the dimension of the augmented systems.
% Since the dynamics of the fictitious dynamics are exactly known to the controller, the attack is detected easily. However, when the scale of the system increases, the communication bandwidth used by moving the target defense approach increases rapidly. 

Other recently proposed works include two-way coding \citep{fang2019two}, a weak encryuption technique, and dynamic masking \citep{abdalmoaty2023privacy}, which enhances privacy as well as security, have been shown to be effective against zero-dynamics attacks.
% Two-way coding \citep{fang2019two} and dynamic masking \citep{abdalmoaty2023privacy} are other recently proposed approaches. Two-way coding is another form of weak encryption technique whilst dynamic masking proposes an architecture that enhances both privacy and security. These schemes are shown to be effective against zero dynamics attacks but remain to be studied for other classes of attacks. 
% Recent extensions include \citep{mukherjee2021secure,ramos2024privacy}.
% Some other works which are related are \citep{mukherjee2021secure}, an extension of \cite{fang2019two}. The work \citep{ramos2024privacy} is an extension of moving target defense for multi-agent systems. 
Furthermore, filtering techniques for attack detection are proposed by \cite{murguia2020security,hashemi2022codesign,escudero2023safety}, while not focusing on stealthy attacks.
% The works \citep{murguia2020security,hashemi2022codesign,escudero2023safety} develop filtering techniques to guarantee safety, without being focused on stealthy covert attacks.

Multiplicative watermarking (mWM) has been proposed by the authors as a diagnosis technique \citep{ferrari2020switching}. mWM consists of a pair of filters on each communication channel between the plant and its controller; the scheme is affine to weak encryption, whereby ``encoding'' and ``decoding'' are done by changing signals' dynamic characteristics through inverse pairs of filters. This enables original signals to be recovered exactly, and thus does not lead to performance degradation.
% A multiplicative watermark is an affine to a weak encryption technique, through which the signal is ``encoded'' by a filter, changing its dynamic behavior. The use of inverse pairs means that the original signal can be recovered, through ``decoding'' via an inverse filter. As such, differently to techniques based on additive watermarking, no performance is lost due to the injection of noise, and there are no bandwidth limitations.

%\subsection{Contributions}
One of the critical features of multiplicative watermarking is that to detect stealthy attacks, the mWM filter parameters must be switched over time. In this paper, an algorithm to optimally design the mWM parameters after a switching event is presented, enhancing detection performance, without changing the switching time.
% This is done without changing the switching time, which is taken as given.

\textcolor{black}{
To formalize the filter design problem, we suppose the defender is interested in optimal performance against adversaries injecting covert attacks with matched system parameters \citep{smith2015covert}, including the mWM parameters prior to the switch. This scenario represents a worst case where malicious agents can take full control of the system while remaining undetected.
Thus, the attack strategy is explicitly included within the formulation of the closed-loop system, and the mWM filters are chosen by solving an optimization problem minimizing the attack-energy-constrained output-to-output gain (AEC-OOG) \citep{anand2023risk}, a variation of the output-to-output gain proposed in  \cite{teixeira2015strategic}.
}
The main contributions of this paper are:
% We consider an adversary injecting a covert attack with matched system parameters \citep{smith2015covert}, i.e., an attacker with full knowledge of the control system parameters, including those of the mWM filters before the switch. This scenario is taken as a worst case, as it has been shown that this class of attacks can be made stealthy. To quantitatively define a cost, the output-to-output gain (OOG) \citep{teixeira2015strategic} is leveraged,
% a metric introduced to evaluate the impact of an additive attack in a control system. %Specifically, OOG evaluates the worst-case performance loss that an attacker injecting an undetectable attack can obtain. 
% Here, the maximum performance loss caused by a stealthy adversary with limited energy is taken, the attack-energy-constrained OOG (AEC-OOG) \citep{anand2023risk}. The main contributions of this paper are:
\begin{enumerate}
%[label=\alph*.]
\item The problem of optimally designing the switching mWM filters is formulated as an optimization problem, with the AEC-OOG is taken as the objective;%where the AEC-OOG is taken as the impact metric; 
\item The worst-case scenario of a covert attack with exact knowledge of plant and mWM filter parameters is embedded within the design problem;
% The optimization problem is defined to incorporate the worst-case scenario of a covert attack with exact knowledge of plant and mWM filter parameters;
\item The feasibility of the optimization problem is shown to be dependent only on stability conditions; 
\item A solution scheme is proposed to promote randomization of the mWM filter parameters such that an eavesdropping adversary cannot remain stealthy.
\end{enumerate} 

This builds on the results of \cite{ferrari2020switching}, where the focus was on the design of the switching protocols, rather than the parameters themselves.
Compared to previous work \citep{gallo2021design}, this paper introduces an optimization problem which is always feasible (thanks to the use of AEC-OOG in the objective), while also considering a more sophisticated class of covert attacks, where the presence of watermark is known to the adversary. 
Moreover, this paper poses a different objective than \citep{zhang2023hybrid}; indeed, while \citep{zhang2023hybrid} provided a design strategy to ensure certain privacy properties, in this paper we address the problem of optimal parameter design following a switching event.


%\subsection{Organization}
The rest of the paper is organized as follows. 
After formulating the problem in Section~\ref{sec:PF}, we propose our design algorithm in Section~\ref{sec:main}, and analyze its properties. It is then evaluated through a numerical example in Section~\ref{sec:NE}, and concluding remarks are given Section~\ref{sec:Con}.
% We provide the problem background in Section~\ref{sec:PF}. We formulate the design problem in Section~\ref{sec:main}, together with an analysis of its properties. The proposed algorithm is evaluated through a numerical example in Section \ref{sec:NE}. Concluding remarks are offered in Section \ref{sec:Con}.
% \input{Sections/Self-Affirmed}
% \section{Related Work}

\subsection{View-Dependent Control}
View-dependent representations have a long history in computer graphics.
In his pioneering work, Rademacher proposed interpolating between \textit{key viewpoints} and associated \textit{key deformations} to manipulate 3D models~\cite{rademacher1999view}.
Other researchers have extended the idea to create 3D animation systems~\cite{10.1111:j.1467-8659.2004.00772.x}, streamline the modeling process~\cite{DBLP:journals/corr/abs-2103-15472}, and integrate physical simulation\cite{koyama2013view}.
Of particular note, Rivers et al.~\cite{rivers25Dcartoonmodels} introduced \textit{2.5D Cartoon Models}, a combination of planar meshes transformed, based upon view angle, so as to appears three dimensional.
Our work draws upon these works but is, to our knowledge, the first work to attempt to use view-dependent techniques to retarget 3D motion onto 2D characters.   

\subsection{Animation from 2D Images}

% output is still 2D
Many researchers have proposed different methods for creating animations from 2D images. Hornung et al.~\cite{Hornung2007anim2Dpicmotion} presented a method to deform a character from a photograph given user-provided joint annotations.
\textit{Toonsynth}~\cite{Dvoroznak18-SIG} and \textit{Neural Puppet}~\cite{poursaeed2020neural} both present methods to create new images of hand-drawn characters from examples.
% output is 3D model
Other researchers have proposed methods of obtaining 3D geometry from 2D sketches~\cite{igarashi2006teddy, Dvoroznak20-SA} and images~\cite{ArtiSketch,weng2019photo}.
% focus on sketches specifically
A number of works have specifically focused on childlike drawings.
Lingens et al.~\cite{lingens2020towards} proposed an evolutionary algorithm for animating children's drawings. 
\textit{MagicToon}~\cite{feng2017magictoon} creates a 3D model from childlike drawings for AR applications.
Similar to our work, Smith et al.~\cite{SmithHodgins} proposed a method for animating childlike drawings using 3D skeletal motion. 
However, the resulting animations are only suitable for use in 2D applications and the type of motions it supports are limited.

Unlike these previous works, our solution can be used in 3D contexts but does not create a 3D model. We instead relying upon a view-dependent formulation of the animated character.
% \input{Sections/EmpiricalStudySetup}
% \input{Sections/ExperimentalResults}
% \section{Discussion and Future Work}\label{sec:discussion}
This paper pioneers the novel approach of selective response, showing that withholding responses can be a powerful tool for GenAI systems. By opting not to answer every query as accurately as it can---particularly when new or complex topics emerge---GenAI can encourage user participation on community-driven platforms and thereby generate more high-quality data for future training. This mechanism ultimately enhances GenAI's long-term performance and revenue. From a welfare perspective, our results indicate that such selective engagement can also benefit users, leading to better solutions and increased overall satisfaction. Since this work is the first to address selective response strategies for GenAI, numerous promising directions remain for future research; we highlight some of them below. 

First, from a technical standpoint, all of the results in this paper rely on Assumption~\ref{assumption: data lip}, involving the lipshitz condition of the accuracy function and the sensitivity parameter $\beta$. Future work could seek to relax this assumption. Furthermore, our constrained optimization approach in Subsection~\ref{sec: welfare constrained revenue maximization} could be extended to approximate the optimal (continuous) strategy instead of the optimal discrete strategy.

Second, our stylized model adopts the simplifying---though unrealistic---assumption that only a single GenAI platform exists. Admittedly, this makes it easier to focus on the idea of selective responses, and indeed, this assumption is pivotal in keeping our analysis tractable. Future research could explore scenarios with multiple GenAI platforms and human-centered forums. In such settings, one platform's selective response might redirect users not only to forums but also to competing GenAI platforms, leading to the tragedy of the commons \cite{hardin1968tragedy}: Although all GenAI platforms benefit from fresh data generation, none may choose to respond selectively if it means losing users to competitors. 

Third, we assumed Forum behaves non-strategically. In reality, human-centered platforms often monetize their data by selling it to GenAI platforms, adding a further layer of strategic interaction for GenAI. Moreover, data transfer between the platforms can form the basis for collaboration: GenAI could employ selective response to bolster Forum content creation, and Forum could, in turn, attribute that content to GenAI for subsequent use in retraining.


%Third, we make the (again) simplifying assumption that Forum is non-strategic. However, in practice, human-centered platforms can sell their data to GenAI platforms. This adds additional considerations for GenAI. Furthermore, data transmission between the platforms can also become the basis for collaboration: GenAI can use selective response to ensure enough content is generated in Forum, and Forum could provide the data attributed to this mechanism back to GenAI. 


%Second, this paper makes the simplifying yet unrealistic assumption of the existence of one GenAI platform. Indeed, this simplifies many aspects and allows us to analyze selective responses. Future work could address the data generation process with more than one GenAI platform and possibly several human-centered forums. In such a case, selective response of one GenAI platform can either drive users to forums or to other GenAI platforms; thus, we might face a tragedy of the commons situation~\ref{hardin1968tragedy}, where all GenAI platforms are interested in fresh data generation but none volunteer to selectively respond and lose users. 

%This paper examines the competition between a generative AI platform and human-based platforms, challenging the assumption that always providing answers is optimal. We analyzed the impact of withholding answers on GenAI's revenue and developed an efficient approximately optimal algorithm for this purpose. We further explored how withholding affects users, showing that it can lead to better outcomes compared to always answering. Specifically, we demonstrated that withholding can Pareto-dominate this strategy and derived the necessary and sufficient conditions for that. Finally, we proposed a second approximately optimal algorithm that maximizes GenAI's revenue while ensuring users are better off than when GenAI answers all queries.

%On a more conceptual level, our model assumes that GenAI’s data comes solely from the competing platform (Forum). Future research could explore a scenario where GenAI can purchase additional data from a third party. This extension could provide valuable insights into the interplay between withholding answers and data purchasing, and whether these two strategies can complement each other or must be traded off.
% \input{Sections/Threats}
% Software development is increasingly conceived as a collaboration activity between developers and AIs. Indeed, IDEs already implement features to enable interactive development, with AI suggesting implementations that are reused by developers.

Although multiple studies show this interaction can be successful, there is still limited understanding of how the models must be configured and used in the context of code generation tasks. This study addresses this gap, systematically investigating the impact of several key parameters, including the repeated submission of a prompt to accommodate for the non-deterministic nature of the models.

Our study reveals several key findings about the usage of ChatGPT. In particular, we discovered how creativity, although up to a limited extent, is useful to increase the range of methods whose code can be generated correctly. A major role is played by parameter top-p, which is commonly underrated, and instead has a major impact on the correctness of the results, with lower values producing better results. Finally, prompts should be submitted multiple times, with $5$ repetitions combined with a temperature of $1.2$ resulting in an effective configuration in our experiments.  

Future work concerns two main research directions. One is about replicating this experiment with other AI assistants, to validate our findings in multiple contexts. The second research direction concerns finding strategies to deal with the need to submit the same prompt multiple times to obtain a useful result, and thus developing approaches able to select or merge multiple responses automatically. 




{\footnotesize\bibliography{references.bib}}
%{\footnotesize\bibliography{references.bib}}



\end{document}
