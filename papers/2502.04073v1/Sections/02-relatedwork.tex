\section{Related Work}
\label{Section:Background}

%\section{Related Work}
\label{sec:RelatedWork}

Within the realm of geophysical sciences, super-resolution/downscaling is a challenge that scientists continue to tackle. There have been several works involved in downscaling applications such as river mapping \cite{Yin2022}, coastal risk assessment \cite{Rucker2021}, estimating soil moisture from remotely sensed images \cite{Peng2017SoilMoisture} and downscaling of satellite based precipitation estimates \cite{Medrano2023PrecipitationDownscaling} to name a few. We direct the reader to \cite{Karwowska2022SuperResolutionSurvey} for a comprehensive review of satellite based downscaling applications and methods. Pertaining to our objective of downscaling \acp{WFM}, we can draw comparisons with several existing works. 
In what follows, we provide a brief review of functionally adjacent works to contrast the novelty of our proposed model and its role in addressing gaps in literature. 

When it comes to downscaling \ac{WFM}, several works use statistical downscaling techniques. These works downscale images by using statistical techniques that utilize relationships between neighboring water fraction pixels. For instance, \cite{Li2015SRFIM} treat the super-resolution task as a sub-pixel mapping problem, wherein the input fraction of inundated pixels must be exactly mapped to the output patch of inundated pixels. 
% In doing so, they are able to apply a discrete particle swarm optimization method to maximize the Flood Inundation Spatial Dependence Index (FISDI). 
\cite{Wang2019} improved upon these approaches by including a spectral term to fully utilize spectral information from multi spectral remote sensing image band. \cite{Wang2021} on the other hand also include a spectral correlation term to reduce the influence of linear and non-linear imaging conditions. All of these approaches are applied to water fraction obtained via spectral unmixing \cite{wang2013SpectralUnmixing} and are designed to work with multi spectral information from MODIS. However, we develop our model with the intention to be used with water fractions directly derived from the output of satellites. One such example is NOAA/VIIRS whose water fraction extraction method is described in \cite{Li2013VIIRSWFM}. \cite{Li2022VIIRSDownscaling} presented a work wherein \ac{WFM} at 375-m flood products from VIIRS were downscaled 30-m flood event and depth products by expressing the inundation mechanism as a function of the \ac{DEM}-based water area and the VIIRS water area.

On the other hand, the non-linear nature of the mapping task lends itself to the use of neural networks. Several models have been adapted from traditional single image digital super-resolution in computer vision literature \cite{sdraka2022DL4downscalingRemoteSensing}. Existing deep learning models in single image super-resolution are primarily dominated by \ac{CNN} based models. Specifically, there has been an upward trend in residual learning models. \acp{RDN} \cite{Zhang2018ResidualDenseSuperResolution} introduced residual dense blocks that employed a contiguous memory mechanism that aimed to overcome the inability of very deep \acp{CNN} to make full use of hierarchical features. 
\acp{RCAN} \cite{Zhang2018RCANSuperResolution} introduced an attention mechanism to exploit the inter-channel dependencies in the intermediate feature transformations. There have also been some works that aim to produce more lightweight \ac{CNN}-based architectures \cite{Zheng2019IMDN,Xiaotong2020LatticeNET}. Since the introduction of the vision transformer \cite{Vaswani2017Attention} that utilized the self-attention mechanism -- originally used for modeling text sequences -- by feeding a sequence 2D sub-image extracted from the original image. Using this approach \cite{LuESRT2022} developed a light-weight and efficient transformer based approach for single image super-resolution. 


For the task of super-resolution of \acp{WFM}, we discuss some works whose methodology is similar to ours even though they differ in their problem setting. \cite{Yin2022} presented a cascaded spectral spatial model for super-resolution of MODIS imagery with a scaling factor 10. Their architecture consists of two stages; first multi-spectral MODIS imagery is converted into a low-resolution \ac{WFM} via spectral unmixing by passing it through a deep stacked residual \ac{CNN}. The second stage involved the super-resolution mapping of these \acp{WFM} using a nested multi-level \ac{CNN} model. Similar to our work, the input fraction images are obtained with zero errors which may not be reflective of reality since there tends to be sensor noise, the spatial distribution of whom cannot be easily estimated. We also note that none of these works directly tackle flood inundation since they've been trained with river map data during non-flood circumstance and \textit{ergo} do not face a data scarcity problem as we do. 
% In this work, apart from the final product of \acp{WFM}, we are not presented with any additional spectral information about the low resolution image. This was intended to work directly with products that can generate \ac{WFM} either directly (VIIRS) or indirectly (Landsat).
\cite{Jia2019} used a deep \ac{CNN} for land mapping that consists of several classes such as building, low vegetation, background and trees. 
\cite{Kumar2021} similarly employ a \ac{CNN} based model for downscaling of summer monsoon rainfall data over the Indian subcontinent. Their proposed Super-Resolution Convolutional Neural Network (SRCNN) has a downscaling factor of 4. 
\cite{Shang2022} on the other hand, proposed a super-resolution mapping technique using Generative Adversarial Networks (GANs). They first generate high resolution fractional images, somewhat analogous to our \ac{WFM}, and are then mapped to categorical land cover maps involving forest, urban, agriculture and water classes. 
\cite{Qin2020} interestingly approach lake area super-resolution for Landsat and MODIS data as an unsupervised problem using a \ac{CNN} and are able to extend to other scaling factors. \cite{AristizabalInundationMapping2020} performed flood inundation mapping using \ac{SAR} data obtained from Sentinel-1. They showed that \ac{DEM}-based features helped to improve \ac{SAR}-based predictions for quadratic discriminant analysis, support vector machines and k-nearest neighbor classifiers. While almost all of the aforementioned works can be adapted to our task. We stand out in the following ways (i) We focus on downscaling of \acp{WFM} directly, \textit{i.e.,} we do not focus on the algorithm to compute the \ac{WFM} from multi-channel satellite data and (ii) We focus on producing high resolution maps only for instances of flood inundation. The latter point produces a data scarcity issue which we seek to remedy with synthetic data. 


%%%%%%%%%%%%%%%%% Additional unused information %%%%%%%%%%%%%%%%


%     \item[\cite{Wang2021}] Super-Resolution Mapping Based on Spatial–Spectral Correlation for Spectral Imagery
%     \begin{itemize}
%         \item Not a deep neural network approach. SRM based on spatial–spectral correlation (SSC) is proposed in order to overcome the influence of linear and nonlinear imaging conditions and utilize more accurate spectral properties.
%         \item (fig 1) there are two main SRM types: (1) the initialization-then-optimization SRM, where the class labels are allocated randomly to subpixels, and the location of each subpixel is optimized to obtain the final SRM result. and (2)soft-then-hard SRM, which involves two steps: the subpixel sharpening and the class allocation.  
%         \item SSC procedures: (1) spatial correlation is performed by the MSAM to reduce the influences of linear imaging conditions on image quality. (2) A spectral correlation that utilizes spectral properties based on the nonlinear KLD is proposed to reduce the influences of nonlinear imaging conditions. (3) spatial and spectral correlations are then combined to obtain an optimization function with improved linear and nonlinear performances. And finally (4) by maximizing the optimization function, a class allocation method based on the SA is used to assign LC labels to each subpixel, obtaining the final SRM result.
%         \item (Comparable) 
%     \end{itemize}
%     %--------------------------------------------------------------------
% \cite{Wang2021} account for the influence of linear and non-linear imaging conditions by involving more accurate spectral properties. 
%     %--------------------------------------------------------------------
%     \item[\cite{Yin2022}] A Cascaded Spectral–Spatial CNN Model for Super-Resolution River Mapping With MODIS Imagery
%     \begin{itemize}
%         \item produce  Landsat-like  fine-resolution (scale of 10)  river  maps  from  MODIS images. Notice the original coarse-resolution remotely sensed images, not the river fraction images.
%         \item combined  CNN  model that  contains  a spectral  unmixing  module  and  an  SRM  module, and the SRM module is made up of an encoder and a decoder that are connected through a series of convolutional blocks. 
%         \item With an adaptive cross-entropy loss function to address class imbalance.	
%         \item The overall accuracy, the omission error, the  commission  error,  and  the  mean  intersection  over  union (MIOU)  calculated  to  assess  the results.
%         \item partially comparable with ours, only the SRM module part
%     %--------------------------------------------------------------------

% To decouple the description of the objective and the \ac{ML} model architecture, the motivation for the model architecture is described in \secref{sec:Methodology}. 


%     \item[\cite{Wang2019}] Improving Super-Resolution Flood Inundation Mapping for Multi spectral Remote Sensing Image by Supplying More Spectral 
%     \begin{itemize}
%         \item proposed the SRFIM-MSI,where a new spectral term is added to the traditional SRFIM to fully utilize the spectral information from multi spectral remote sensing image band. 
%         \item The original SRFIM \cite{Huang2014, Li2015} obtains the sub pixel spatial distribution of flood inundation within mixed pixels by maximizing their spatial correlation while maintaining the original proportions of flood inundation within the mixed pixels. The SRFIM is formulated as a maximum combined optimization issue according to the principle of spatial correlation.
%         \item follow the terminology in \cite{Wang2021}, this is an initialization-then-optimization SRM. 
%         \item (Comparable) 
%     \end{itemize}
%     %--------------------------------------------------------------------


%--------------------------------------------------------------------
%     \item[\cite{Jia2019}] Super-Resolution Land Cover Mapping Based on the Convolutional Neural Network
%     \begin{itemize}
%         \item SRMCNN (Super-resolution mapping CNN) is proposed to obtain fine-scale land cover maps from coarse remote sensing images. Specifically, an encoder-decoder CNN is used to determined the labels (i.e., land cover classes) of the subpixels within mixed pixels.
%         \item There were three main parts in SRMCNN. The first part was a three-sequential convolutional layer with ReLU and pooling. The second part is up-sampling, for which a multi transposed-convolutional layer was adopted. To keep the feature learned in the previous layer, a skip connection was used to concatenate the output of the corresponding convolution layer. The last part was the softmax classifier, in which the feature in the antepenultimate layer was classified and class probabilities are obtained.
%         \item The loss: the optimal allocation of classes to the subpixels of mixed pixel is achieved by maximizing the spatial dependence between neighbor pixels under constraint that the class proportions within the mixed pixels are preserved.
%         \item (Preferred), this paper is designed to classify background, Building, Low Vegetation, or Tree in the land. But we can easily adapt to our problem and should compare with this paper.
%     \end{itemize}
%     %--------------------------------------------------------------------

%     \item[\cite{Kumar2021}] Deep learning–based downscaling of summer monsoon rainfall data over Indian region
%     \begin{itemize}
%         \item down-scaling (scale of 4) rainfall data. The output image is not binary image.
%         \item three algorithms: SRCNN, stacked SRCNN, and DeepSD are employed, based on \cite{Vandal2019}
%         \item mean square error and pattern correlation coefficient are used as evaluation metrics.
%         \item SRCNN: super-resolution-based convolutional neural networks (SRCNN) first upgrades the low-resolution image to the higher resolution size by using bicubic interpolation. Suppose the interpolated image is referred to as Y; SRCNNs’task is to retrieve from Y an image F(Y) which is close to the high-resolution ground truth image X.
%         \item stacked SRCNN: stack 2 or more SRCNN blocks to increasing the scaling factor.
%         \item DeepSD: uses topographies as an additional input to stacked SRCNN.
%         \item These algorithms are not designed for binary output images, but if prefer, the ``modified'' stacked SRCNN or DeepSD can be used as baseline algorithms.
%     
%     \item[\cite{Shang2022}] Super resolution Land Cover Mapping Using a Generative Adversarial Network
%     \begin{itemize}
%         \item propose an end-to-end SRM model based on a generative adversarial network (GAN), that is, GAN-SRM, to improve the two-step learning-based SRM methods. 
%         \item Two-step SRM method: The first step is fraction-image super-resolution (SR), which reconstructs a high-spatial-resolution fraction image from the low input, methods like SVR, or CNN has been widely adopted. The second step is converting the high-resolution fraction images to a categorical land cover map, such as with a soft-max function to assign each high-resolution pixel to a unique category value.
%         \item The proposed GAN-SRM model includes a generative network and a discriminative network, so that both the fraction-image SR and the conversion of the fraction images to categorical map steps are fully integrated to reduce the resultant uncertainty. 
%         \item applied to the National Land Cover Database (NLCD), which categorized land into four typical classes:forest, urban, agriculture,and water. scale factor of 8. 
%         \item (Preferred), we should compare with this work.
%     \end{itemize}
%     %--------------------------------------------------------------------

%   \item[\cite{Qin2020}] Achieving Higher Resolution Lake Area from Remote Sensing Images Through an Unsupervised Deep Learning Super-Resolution Method
%   \begin{itemize}
%       \item propose an unsupervised deep gradient network (UDGN) to generate a higher resolution lake area from remote sensing images.
%       \item UDGN models the internal recurrence of information inside the single image and its corresponding gradient map to generate images with higher spatial resolution. 
%       \item A single image super-resolution approach, not comparable
%   \end{itemize}
%     %--------------------------------------------------------------------




%     \item[\cite{Demiray2021}] D-SRGAN: DEM Super-Resolution with Generative Adversarial Networks
%     \begin{itemize}
%         \item A GAN based model is proposed to increase the spatial resolution of a given DEM dataset up to 4 times without additional information related to data.
%         \item Rather than processing each image in a sequence independently, our generator architecture uses a recurrent layer to update the state of the high-resolution reconstruction in a manner that is consistent with both the previous state and the newly received data. The recurrent layer can thus be understood as performing a Bayesian update on the ensemble member, resembling an ensemble Kalman filter. 
%         \item A single image super-resolution approach, not comparable
%     \end{itemize}
%     %--------------------------------------------------------------------
%     \item[\cite{Leinonen2021}] Stochastic Super-Resolution for Downscaling Time-Evolving Atmospheric Fields With a Generative Adversarial Network
%     \begin{itemize}
%         \item propose a super-resolution GAN that operates on sequences of two-dimensional images and creates an ensemble of predictions for each input. The spread between the ensemble members represents the uncertainty of the super-resolution reconstruction.
%         \item for sequence of input images, not comparable with ours.
%     \end{itemize} 
%     %--------------------------------------------------------------------

% \end{itemize}



 
% \begin{table*}
%   \centering
% 	 \caption{\textcolor{blue}{A summary of the literature on the impact of refactoring activities on software quality attributes.}}
% 	 \label{Table:Quality Metrics in Related Work}
% %\begin{sideways}
% \begin{adjustbox}{width=1.2\textwidth,center}
% \rowcolors{2}{gray!25}{white}
% %\begin{adjustbox}{width=\textheight,totalheight=\textwidth,keepaspectratio}
% \begin{tabular}{llllll}\hline
% \toprule
%  \bfseries No. & \bfseries Study & \bfseries Year & \bfseries Quality Metric & \bfseries Internal QA & \bfseries External QA  \\
% \midrule
%\centering % Centers the table on the page
%\begin{adjustbox}{width=1.2\textwidth,center} % Adjusts width to fit page, centers content
% \begin{center}
% \begin{longtable}{p{0.3cm} p{3.5cm} p{0.5cm} p{3.3cm} p{3.3cm} p{3.3cm}}
% %\centering

% \caption{\textcolor{black}{A summary of the literature on the impact of refactoring activities on software quality attributes.}} \label{Table:Quality Metrics in Related Work} \\
% %\begin{adjustbox}{width=1.2\textwidth,center}
% %\rowcolors{2}{gray!25}{white}
% %\centering
% \begin{adjustbox}{width=\textwidth,center}
% \toprule
% \bfseries No. & \bfseries Study & \bfseries Year & \bfseries Quality Metric & \bfseries Internal QA & \bfseries External QA \\
% \midrule
% \endhead


% \bottomrule
% \endlastfoot
% %\midrule

% 1 & Sahraoui \etal \cite{sahraoui2000can} & 2000  & CLD / NOC / NMO / NMI    & Inheritance / Coupling & Fault-proneness / Maintainability   \\ 
% & & &  NMA / SIX / CBO / DAC & & \\
% & & &  IH-ICP / OCAIC / DMMEC / OMMEC &  \\ \hline
% 2 & Stroulia \& Kapoor \cite{stroulia2001metrics} & 2001 & LOC / LCOM / CC & Size / Coupling & Design extensibility \\ \hline
% 3 & Kataoka \etal \cite{kataoka2002quantitative} & 2002 & Coupling measures &  Coupling & Maintainability   \\ \hline
% 4 & Demeyer \cite{demeyer2002maintainability} & 2002& N/A & Polymorphism  & Performance  \\ \hline
% 5 & Tahvildari \etal \cite{tahvildari2003quality} & 2003 & LOC / CC / CMT / Halstead's efforts & Complexity  & Performance / Maintainability   \\ \hline
% 6 & Leitch \& Stroulia \cite{leitch2003assessing}& 2003 & SLOC / No. of Procedure & Size & Maintainability  \\ \hline
% 7 & Bois \& Mens \cite{du2003describing} & 2003  & NOM / CC / NOC / CBO & Inheritance / Cohesion / Coupling / Size / Complexity & N/A  \\ 
% & & &  RFC / LCOM & & \\ \hline
% 8 & Tahvildari \& Kontogiannis \cite{tahvildari2003metric} & 2004  & LCOM / WMC / RFC / NOM   & Inheritance / Cohesion / Coupling / Complexity & Maintainability  \\ 
% & & &  CDE / DAC / TCC & & \\ \hline
% 9 & Bois \etal \cite{du2004refactoring} & 2004  & N/A & Cohesion / Coupling & Maintainability   \\ \hline
% 10 & Bois \etal \cite{du2005does} & 2005  & N/A & N/A &   Understandability   \\  \hline
% 11 & Geppert \etal \cite{geppert2005refactoring} & 2005  &  N/A & N/A & Changeability  \\ \hline
% 12 & Ratzinger \etal \cite{ratzinger2005improving} & 2005  & N/A &  Coupling & Evolvability \\ 
% & & & Analyzing code histories & \\ \hline
% 13 & Moser \etal \cite{moser2006does} & 2006  & CK / MCC / LOC   & Inheritance / Cohesion / Coupling / Complexity & Reusability \\ \hline
% 14 & Wilking \etal \cite{wilking2007empirical} & 2007  & CC / LOC  & Complexity & Maintainability / Modifiability   \\ \hline
% 15 & Stroggylos \& Spinells \cite{stroggylos2007refactoring} & 2007  & CK / Ca / NPM & Inheritance / Cohesion / Coupling / Complexity & N/A  \\ \hline
% 16 & Moser \etal \cite{moser2007case} & 2008 & CK / LOC / Effort (hour) & Cohesion / Coupling / Complexity & Productivity  \\ \hline
% 17 & Shrivastava \& Shrivastava \cite{shrivastava2008impact} & 2008 & NOA / NOC / NOM / CC & Inheritance / Complexity / Size & N/A\\ 
% & & & TLOC / DIT & \\ \hline 
% 18 & Higo \etal \cite{higo2008refactoring} & 2008 & CK & Inheritance / Cohesion / Coupling / Complexity & N/A\\ \hline
% 19 & Reddy \& Rao  \cite{reddy2009quantitative} & 2009 & DOCMA (CR) \ DOCMA (AR) & Complexity & N/A \\ \hline
% 20 & Alshayeb \cite{alshayeb2009empirical} & 2009 &   CK / LOC / FANOUT  & Inheritance / Cohesion / Coupling / Size & Adaptability / Maintainability / Testability / Reusability  \\ 
% & & & & &  Understandability  \\ \hline
% 21 & Alshayeb \cite{alshayeb2009refactoring} & 2009 & LCOM1 / LCOM2 / LCOM3 / LCOM4 / LCOM5 & Cohesion & N/A \\ \hline
% 22 & Usha \etal \cite{usha2009quantitative} & 2009 & LCOM / CBO / WMC / RFC / CC & Cohesion / Coupling / Complexity & Modifiability / Modularity \\ 
% & & &  CF / TCC / MHF / AHF & & \\ \hline 
% 23 & Hegedus \etal \cite{hegedHus2010effect} & 2010  & CK  & Coupling / Complexity / Size & Maintainability / Testability / Error Proneness / Changeability  \\
% & & & & &  Stability / Analizability \\ \hline
% 24 & Shatnawi \& Li \cite{shatnawi2011empirical} & 2011 & CK / QMOOD &  Inheritance / Cohesion / Coupling / Polymorphism / Size & Reusability / Flexibility / Extendibility / Effectiveness     \\ 
% & & & & Encapsulation / Composition / Abstraction / Messaging &    \\ \hline
% 25 & Fontana \& Spinelli \cite{fontana2011impact} & 2011   & DAC / LCOM / NOM / RFC & Cohesion / Coupling / Complexity & N/A\\ 
% & & &  TCC / WMC & &  \\ \hline 
% 26 & Alshayeb \cite{alshayeb2011impact} & 2011  &  CK / LOC / FANOUT  & Inheritance / Cohesion / Coupling / Size & Adaptability / Maintainability / Testability / Reusability  \\ 
% & & & & &  Understandability \\ \hline
% % check again paper \cite{lerthathairat2011approach}
% 27 & Lerthathairat \& Prompoon \cite{lerthathairat2011approach} & 2011 & NLOC / NILI / CC / ILCC / NOP & Cohesion / Encapsulation   & N/A \\
% & & & NOM / NFD / LCOM / LCOM-HS &  & \\ \hline
% 28 & {\'O} Cinn{\'e}ide \etal \cite{o2012experimental} & 2012 & LSCC / TCC / CC / SCOM / LCOM5 & Cohesion & N/A \\ \hline 
% 29 & Ibrahim \etal \cite{ibrahim2012identification} & 2012 & LCCI / LCCD / LCC / TCC  & Cohesion & N/A \\
% & & & CC / Coh / LCOM3 &  \\ \hline 
% 30 & Singh \& Kahlon \cite{singh2011effectiveness} & 2011 &  CK / LCOM4 / PuF / EncF / NOD & Inheritance / Cohesion / Coupling / Information hiding &  \\
% & & & & Polymorphism / Encapsulation / Abstraction & \\\hline
% 31 & Singh \& Kahlon \cite{singh2012effectiveness} & 2012 & CK / LCOM4 / PuF / EncF / NOD & Inheritance / Cohesion / Coupling / Information hiding &  \\
% & & & & Polymorphism / Encapsulation / Abstraction & \\\hline
% 32 & Murgia \etal \cite{murgia2012refactoring} & 2012 & FANIN / FANOUT & Coupling & N/A \\ \hline 
% 33 & Kannangara \& Wijayanake \cite{kannangara2013impact} & 2013 &  N/A  & N/A & Analysability / Changeability / Time Behaviour / Resource Utilization  \\ \hline 
% 34 & Veerappa \& Rachel \cite{veerappa2013empirical} & 2013 & RFC / DCC / CBO / MPC & Coupling & N/A \\ \hline
% 35 & Napoli \etal \cite{napoli2013using} & 2013 & LCOM / CBO & Cohesion / Coupling  & Modularity \\ \hline 
% 36 & Bavota \etal \cite{bavota2013empirical} & 2013  & ICP / IC-CD / CCBC & Coupling & N/A \\
% & &  & & & \\ \hline
% 37 & Kumari \& Saha \cite{kumari2014effect} & 2014 & DIT / CBO / RFC / WMC  & Inheritance / Cohesion / Coupling / Complexity & Maintainability / Reusability / Testability / Understandability \\ 
% & & & LCOM / NOM / LOC &  &  Fault proneness / Completeness / Stability / Adaptability \\ \hline 
% 38 & Szoke \etal \cite{szoke2014bulk} & 2014 & CC / U / NOA / NII / NAni & Size / Complexity & N/A \\
% & &  & LOC / NUMPAR / NMni / NA & &  \\ \hline
% 39 & Chaparro \etal \cite{chaparro2014impact} & 2014 &  RFC / CBO / DAC / MPC & Inheritance / Cohesion / Coupling / Size / Complexity & N/A\\ 
% & &  & LOC / NOM / CC / LCOM2 & &  \\ 
% & & & LCOM5 / NOC / DIT & & \\ \hline
% 40 & Bavota \etal \cite{bavota2015experimental} & 2015  &  CK / LOC / NOA / NOO  &  Inheritance / Cohesion / Coupling / Size / Complexity & N/A \\
% & &  & C3 / CCBC & & \\ \hline
% 41 & Kannangara \& Wijayanake \cite{kannangara2015empirical} & 2015 &  CC / DIT / CBO / LOC & Maintainability index / Complexity / Coupling / Inheritance & Analysability / Changeability / Time Behaviour / Resource Utilization  \\ \hline 
% 42 & Gatrell \&  Counsel \cite{gatrell2015effect} & 2015 & N/A & N/A & change \& fault-proneness \\ \hline 
% 43 & Cedrim at al. \cite{cedrim2016does} & 2016 & LOC / CBO / NOM / CC & Cohesion / Coupling / Complexity & N/A  \\
% & &  & FANOUT / FANIN & &  \\ \hline
% 44 & Malhotra \& Chug \cite{malhotra2016empirical} & 2016 & CK & & Understandability / Modifiability / Extensibility / Reusability   \\ 
% & & & & & Level of Abstraction \\ \hline 
% 45 & Mkaouer \etal \cite{mkaouer2016use} & 2016 & QMOOD & N/A & Reusability / Flexibility / Understandability / Functionality\\ 
% & & & && Extendibility / Effectiveness \\ \hline 
% 46 & Kaur \& Singh \cite{kaur2017improving} & 2017 & WMC / NOI / RFC / TCLOC  & Coupling / Complexity / Size  & Maintainability \\ 
% & & & TLLOC / TNOS /  CI  & & \\ \hline 
% 47 & Chavez \etal \cite{chavez2017does} & 2017  & CBO / WMC / DIT / NOC & Inheritance / Cohesion / Coupling / Size / Complexity & N/A  \\ 
% && &   LOC / LCOM2 / LCOM3 / WOC & &  \\
% && &  TCC / FANIN / FANOUT / CINT & &  \\
% && &  CDISP / CC / Evg / NPATH   & & \\
% && &  MaxNest / IFANIN / OR / CLOC & &  \\
% && & STMTC / CDL / NIV / NIM / NOPA & &  \\ \hline 
% 48 & Szoke \etal \cite{szHoke2017empirical} & 2017 & CK & N/A & Maintainability \\ \hline
% 49 & Bashir \etal \cite{bashir2017methodology} & 2017 & QMOOD & N/A & Modifiability / Analyzability / Understandability / Maintainability \\ \hline 
% 50 & Mumtaz \etal \cite{mumtaz2018empirical} & 2018 &  CK / CCC / CDP / CCDA / COA & N/A & Security \\ 
% & & &  CMW / CMAI / CAAI / CAIW & &  \\ \hline 
% 51 & Pantiuchina \etal \cite{pantiuchina2018improving} & 2018 &  LCOM / CBO / WMC / RFC  & Cohesion / Coupling / Complexity & Readability  \\
% & &   &  C3 / B\&W / SRead & & \\ \hline 
% 52 & Alizadeh \& Kessentini \cite{alizadeh2018reducing} & 2018 & QMOOD & N/A & Reusability / Flexibility / Understandability / Functionality \\ 
% & & & && Extendibility / Effectiveness   \\ \hline 
% 53 & Alizadeh \etal \cite{alizadeh2019refbot} & 2019 & QMOOD & N/A & Reusability / Flexibility / Understandability / Functionality \\ 
% & & & && Extendibility / Effectiveness   \\ \hline 
% 54 & Techapalokul \& Tilevich \cite{techapalokul2019code} & 2019 & LOC / Complex Script Dens / No. Literals & N/A & N/A  \\ 
% & & & Long Script Dens. / Procedure Dens. / No. Global Var & & \\
% & & & No. Create Clone Of. & & \\ \hline 
% 55 & Counsell \etal \cite{counsell2019relationship} & 2019 & CBO & Coupling & N/A  \\ \hline 
% 56 & Fakhoury \etal \cite{fakhoury2019improving} & 2019 & Buse \& Weimer / Dorn / Scalabrino / Posnett & Cohesion / Coupling / Size / Complexity  & Readability \\ 
% & & & LCOM 5 / WMC / RFC / MLOC / FLOC  & &  \\ 
% & & & Halstead Difficulty / Halstead Effort / Maintainability index / MCC & &  \\ 
% & & & Nesting level / Doc LOC / Comment Density / API Doc & & \\ 
% & & & Public Undoc API / Public Doc API / \# Parantheses / & &  \\ 
% & & & Number of Incoming Invocations & &  \\ \hline 
% 57 & AlOmar \etal \cite{alomar2019impact} & 2019 &  CK / FANIN / FANOUT / CC / NIV / NIM & Inheritance / Cohesion / Coupling / Complexity  & N/A \\ 
% & &   & Evg / NPath / MaxNest / IFANIN & Size / Polymorphism / Encapsulation / Abstraction &   \\ 
% & & &  LOC / CLOC / CDL / STMTC & &  \\ \hline 
% 58 & Rebai \etal \cite{rebai2019interactive} & 2019 & QMOOD & N/A & Reusability / Flexibility / Understandability / Functionality \\ 
% & & & && Extendibility / Effectiveness   \\ \hline 
% 59 & Alizadeh \etal \cite{alizadeh2019less} & 2019 & QMOOD & N/A & Reusability / Flexibility / Understandability / Functionality \\ 
% & & & && Extendibility / Effectiveness   \\ \hline 
% 60 & Alizadeh \etal \cite{alizadeh2018interactive} & 2020 & QMOOD & N/A & Reusability / Flexibility / Understandability / Functionality \\ 
% & & & && Extendibility / Effectiveness   \\ \hline 
% 61 & Rebai \etal \cite{rebai2020enabling} & 2020 & QMOOD & N/A & Reusability / Flexibility / Understandability / Functionality \\ 
% & & & && Extendibility / Effectiveness   \\ \hline 
% 62 & Fernandes \etal \cite{fernandes2020refactoring} & 2020    & CBO / WMC / DIT / NOC & Inheritance / Cohesion / Coupling / Size / Complexity & N/A  \\
% &&  & LOC / LCOM2 / LCOM3 / WOC & &  \\
% && &  TCC / FANIN / FANOUT / CINT & &  \\
% && &  CDISP / CC / Evg / NPATH   & & \\
% && &  MaxNest / IFANIN / OR / CLOC & &  \\
% && &  STMTC / CDL / NIV / NIM / NOPA & &  \\ \hline
% 63 & AlOmar \etal \cite{alomar2020developers} & 2020 & CK /  CC / LOC & Inheritance / Cohesion / Coupling / Complexity / Size  & Reusability  \\ \hline
% 64 & Bibiano \etal \cite{bibiano2020does} & 2020 & LCOM2 / CBO / MAXNest / CC &  Cohesion / Coupling / Complexity / Size & N/A \\ 
% & & & LOC / CLOC  / STMTC / NIV & &  \\ 
% && &  NIM / WMC & &  \\ \hline 
% 65 & Abid \etal \cite{abid2020does} & 2020 & QMOOD & N/A & Reusability / Flexibility / Understandability / Functionality \\ & & & && Extendibility / Effectiveness / Security  \\ \hline 
% 66 & Abid \etal \cite{abid2021prioritizing} & 2021 & QMOOD & N/A & Reusability / Flexibility / Understandability / Functionality  \\ 
% & & & & & Extendibility / Effectiveness / Security   \\ \hline  
% 67 & Riansyah \& Mursanto \cite{riansyah2020empirical} & 2020 & CINT / CDISP & Coupling & N/A \\ \hline
% 68 & Iyad \etal \cite{alazzam2020impact} & 2020 & CK / CC / TLOC / MFA / NBD & Inheritance / Cohesion / Coupling / Complexity / Size & N/A \\
% & & & NSC / CE & &  \\ \hline
% 69 & Hamdi \etal \cite{hamdi2021empirical} & 2021  & CBO / WMC / DIT / RFC & Inheritance / Cohesion / Coupling / Complexity / Size  & N/A\\ 
% & &    & LCOM / TCC / LOC / LCC & &  \\ 
% & & &  NOSI / VQTY & &  \\ \hline
% 70 & AlOmar \etal \cite{alomar2022refactoring} & 2021  & CK /  CC / LOC / NPATH & Inheritance / Cohesion / Coupling / Complexity / Size  & Reusability  \\ 
% & & &  MaxNest / IFANIN / CDL / CLOC & &  \\
% & & &  FANIN / FANOUT / STMTC / NIV & &  \\ \hline 
% 71 & Sellitto \etal \cite{sellittotoward} & 2021 & CIC / CIC\_syc / ITID / NMI / CR & N/A & Readability \\
% & & & NM / TC / NOC / NOC\_norm & & \\ \hline
% 72 & Ouni \etal \cite{ouni2023impact}    & 2023  & LCOM / CBO / NOSI / TCC / NIV / IFANIN& Coupling / Cohesion / Complexity / Inheritance / Size & N/A\\ 
% &  & & RFC / FANIN / WMC / VQYT / NIM & & \\
% &  &  & FANOUT / CC / Evg / MaxNest / DIT & & \\
% & & &  LOC / BLOC / CLOC / STMTC / NOC & & \\
% %\hline



% %\end{tabular}
% \end{adjustbox}
% %\end{sideways}
% \end{longtable}

% \end{center}

%%%%%%%%%%%%%%%%%%%%% table 1
\begin{table*}
  \centering
	 \caption{\textcolor{black}{A summary of the literature on the impact of refactoring activities on software quality attributes.}}
	 \label{Table:Quality Metrics in Related Work}
%\begin{sideways}
\begin{adjustbox}{width=1.6\textwidth,center}
\rowcolors{2}{gray!25}{white}
%\begin{adjustbox}{width=\textheight,totalheight=\textwidth,keepaspectratio}
\begin{tabular}{llllll}\hline
\toprule
 \bfseries No. & \bfseries Study & \bfseries Year & \bfseries Quality Metric & \bfseries Internal QA & \bfseries External QA  \\
\midrule

1 & Sahraoui \etal \cite{sahraoui2000can} & 2000  & CLD / NOC / NMO / NMI    & Inheritance / Coupling & Fault-proneness / Maintainability   \\ 
& & &  NMA / SIX / CBO / DAC & & \\
& & &  IH-ICP / OCAIC / DMMEC / OMMEC &  \\ \hline
2 & Stroulia \& Kapoor \cite{stroulia2001metrics} & 2001 & LOC / LCOM / CC & Size / Coupling & Design extensibility \\ \hline
3 & Kataoka \etal \cite{kataoka2002quantitative} & 2002 & Coupling measures &  Coupling & Maintainability   \\ \hline
4 & Demeyer \cite{demeyer2002maintainability} & 2002& N/A & Polymorphism  & Performance  \\ \hline
5 & Tahvildari \etal \cite{tahvildari2003quality} & 2003 & LOC / CC / CMT / Halstead's efforts & Complexity  & Performance / Maintainability   \\ \hline
6 & Leitch \& Stroulia \cite{leitch2003assessing}& 2003 & SLOC / No. of Procedure & Size & Maintainability  \\ \hline
7 & Bois \& Mens \cite{du2003describing} & 2003  & NOM / CC / NOC / CBO & Inheritance / Cohesion / Coupling / Size / Complexity & N/A  \\ 
& & &  RFC / LCOM & & \\ \hline
8 & Tahvildari \& Kontogiannis \cite{tahvildari2003metric} & 2004  & LCOM / WMC / RFC / NOM   & Inheritance / Cohesion / Coupling / Complexity & Maintainability  \\ 
& & &  CDE / DAC / TCC & & \\ \hline
9 & Bois \etal \cite{du2004refactoring} & 2004  & N/A & Cohesion / Coupling & Maintainability   \\ \hline
10 & Bois \etal \cite{du2005does} & 2005  & N/A & N/A &   Understandability   \\  \hline
11 & Geppert \etal \cite{geppert2005refactoring} & 2005  &  N/A & N/A & Changeability  \\ \hline
12 & Ratzinger \etal \cite{ratzinger2005improving} & 2005  & N/A &  Coupling & Evolvability \\ 
& & & Analyzing code histories & \\ \hline
13 & Moser \etal \cite{moser2006does} & 2006  & CK / MCC / LOC   & Inheritance / Cohesion / Coupling / Complexity & Reusability \\ \hline
14 & Wilking \etal \cite{wilking2007empirical} & 2007  & CC / LOC  & Complexity & Maintainability / Modifiability   \\ \hline
15 & Stroggylos \& Spinells \cite{stroggylos2007refactoring} & 2007  & CK / Ca / NPM & Inheritance / Cohesion / Coupling / Complexity & N/A  \\ \hline
16 & Moser \etal \cite{moser2007case} & 2008 & CK / LOC / Effort (hour) & Cohesion / Coupling / Complexity & Productivity  \\ \hline
17 & Shrivastava \& Shrivastava \cite{shrivastava2008impact} & 2008 & NOA / NOC / NOM / CC & Inheritance / Complexity / Size & N/A\\ 
& & & TLOC / DIT & \\ \hline 
18 & Higo \etal \cite{higo2008refactoring} & 2008 & CK & Inheritance / Cohesion / Coupling / Complexity & N/A\\ \hline
19 & Reddy \& Rao  \cite{Reddy2009quantitative} & 2009 & DOCMA (CR) \ DOCMA (AR) & Complexity & N/A \\ \hline
20 & Alshayeb \cite{alshayeb2009empirical} & 2009 &   CK / LOC / FANOUT  & Inheritance / Cohesion / Coupling / Size & Adaptability / Maintainability / Testability / Reusability  \\ 
& & & & &  Understandability  \\ \hline
21 & Alshayeb \cite{alshayeb2009refactoring} & 2009 & LCOM1 / LCOM2 / LCOM3 / LCOM4 / LCOM5 & Cohesion & N/A \\ \hline
22 & Usha \etal \cite{usha2009quantitative} & 2009 & LCOM / CBO / WMC / RFC / CC & Cohesion / Coupling / Complexity & Modifiability / Modularity \\ 
& & &  CF / TCC / MHF / AHF & & \\ \hline 
23 & Hegedus \etal \cite{hegedHus2010effect} & 2010  & CK  & Coupling / Complexity / Size & Maintainability / Testability / Error Proneness / Changeability  \\
& & & & &  Stability / Analizability \\ \hline
24 & Shatnawi \& Li \cite{shatnawi2011empirical} & 2011 & CK / QMOOD &  Inheritance / Cohesion / Coupling / Polymorphism / Size & Reusability / Flexibility / Extendibility / Effectiveness     \\ 
& & & & Encapsulation / Composition / Abstraction / Messaging &    \\ \hline
25 & Fontana \& Spinelli \cite{fontana2011impact} & 2011   & DAC / LCOM / NOM / RFC & Cohesion / Coupling / Complexity & N/A\\ 
& & &  TCC / WMC & &  \\ \hline 
26 & Alshayeb \cite{alshayeb2011impact} & 2011  &  CK / LOC / FANOUT  & Inheritance / Cohesion / Coupling / Size & Adaptability / Maintainability / Testability / Reusability  \\ 
& & & & &  Understandability \\ \hline
% check again paper \cite{lerthathairat2011approach}
27 & Lerthathairat \& Prompoon \cite{lerthathairat2011approach} & 2011 & NLOC / NILI / CC / ILCC / NOP & Cohesion / Encapsulation   & N/A \\
& & & NOM / NFD / LCOM / LCOM-HS &  & \\ \hline
28 & {\'O} Cinn{\'e}ide \etal \cite{o2012experimental} & 2012 & LSCC / TCC / CC / SCOM / LCOM5 & Cohesion & N/A \\ \hline 
29 & Ibrahim \etal \cite{ibrahim2012identification} & 2012 & LCCI / LCCD / LCC / TCC  & Cohesion & N/A \\
& & & CC / Coh / LCOM3 &  \\ \hline 
30 & Singh \& Kahlon \cite{singh2011effectiveness} & 2011 &  CK / LCOM4 / PuF / EncF / NOD & Inheritance / Cohesion / Coupling / Information hiding &  \\
& & & & Polymorphism / Encapsulation / Abstraction & \\\hline
31 & Singh \& Kahlon \cite{singh2012effectiveness} & 2012 & CK / LCOM4 / PuF / EncF / NOD & Inheritance / Cohesion / Coupling / Information hiding &  \\
& & & & Polymorphism / Encapsulation / Abstraction & \\\hline
32 & Murgia \etal \cite{murgia2012refactoring} & 2012 & FANIN / FANOUT & Coupling & N/A \\ \hline 
33 & Kannangara \& Wijayanake \cite{kannangara2013impact} & 2013 &  N/A  & N/A & Analysability / Changeability / Time Behaviour / Resource Utilization  \\ \hline 
34 & Veerappa \& Rachel \cite{veerappa2013empirical} & 2013 & RFC / DCC / CBO / MPC & Coupling & N/A \\ \hline
35 & Napoli \etal \cite{napoli2013using} & 2013 & LCOM / CBO & Cohesion / Coupling  & Modularity \\ \hline 
36 & Bavota \etal \cite{bavota2013empirical} & 2013  & ICP / IC-CD / CCBC & Coupling & N/A \\
& &  & & & \\ \hline
37 & Kumari \& Saha \cite{kumari2014effect} & 2014 & DIT / CBO / RFC / WMC  & Inheritance / Cohesion / Coupling / Complexity & Maintainability / Reusability / Testability / Understandability \\ 
& & & LCOM / NOM / LOC &  &  Fault proneness / Completeness / Stability / Adaptability \\ \hline 
38 & Szoke \etal \cite{szoke2014bulk} & 2014 & CC / U / NOA / NII / NAni & Size / Complexity & N/A \\
& &  & LOC / NUMPAR / NMni / NA & &  \\ \hline
39 & Chaparro \etal \cite{chaparro2014impact} & 2014 &  RFC / CBO / DAC / MPC & Inheritance / Cohesion / Coupling / Size / Complexity & N/A\\ 
& &  & LOC / NOM / CC / LCOM2 & &  \\ 
& & & LCOM5 / NOC / DIT & & \\ \hline
40 & Bavota \etal \cite{bavota2015experimental} & 2015  &  CK / LOC / NOA / NOO  &  Inheritance / Cohesion / Coupling / Size / Complexity & N/A \\
& &  & C3 / CCBC & & \\ \hline
41 & Kannangara \& Wijayanake \cite{kannangara2015empirical} & 2015 &  CC / DIT / CBO / LOC & Maintainability index / Complexity / Coupling / Inheritance & Analysability / Changeability / Time Behaviour / Resource Utilization  \\ \hline 
42 & Gatrell \&  Counsel \cite{gatrell2015effect} & 2015 & N/A & N/A & change \& fault-proneness \\ \hline 
43 & Cedrim at al. \cite{cedrim2016does} & 2016 & LOC / CBO / NOM / CC & Cohesion / Coupling / Complexity & N/A  \\
& &  & FANOUT / FANIN & &  \\ %\hline

\bottomrule
\end{tabular}
\end{adjustbox}
%\end{sideways}
\end{table*}
%%% table 2
\begin{table*}
  \centering
	 \caption{\textcolor{black}{Continued from previous page.}}
	 \label{Table:Quality Metrics in Related Work-v2}
%\begin{sideways}
\begin{adjustbox}{width=1.6\textwidth,center}
\rowcolors{2}{gray!25}{white}
%\begin{adjustbox}{width=\textheight,totalheight=\textwidth,keepaspectratio}
\begin{tabular}{llllll}\hline
\toprule
 \bfseries No. & \bfseries Study & \bfseries Year & \bfseries Quality Metric & \bfseries Internal QA & \bfseries External QA  \\
\midrule



44 & Malhotra \& Chug \cite{malhotra2016empirical} & 2016 & CK & Cohesion / Coupling / Complexity / Inheritance & Understandability / Modifiability / Extensibility / Reusability   \\ 
& & & & & Level of Abstraction \\ \hline 
45 & Mkaouer \etal \cite{mkaouer2016use} & 2016 & QMOOD & N/A & Reusability / Flexibility / Understandability / Functionality\\ 
& & & && Extendibility / Effectiveness \\ \hline 
46 & Kaur \& Singh \cite{kaur2017improving} & 2017 & WMC / NOI / RFC / TCLOC  & Coupling / Complexity / Size  & Maintainability \\ 
& & & TLLOC / TNOS /  CI  & & \\ \hline 
47 & Chavez \etal \cite{chavez2017does} & 2017  & CBO / WMC / DIT / NOC & Inheritance / Cohesion / Coupling / Size / Complexity & N/A  \\ 
&& &   LOC / LCOM2 / LCOM3 / WOC & &  \\
&& &  TCC / FANIN / FANOUT / CINT & &  \\
&& &  CDISP / CC / Evg / NPATH   & & \\
&& &  MaxNest / IFANIN / OR / CLOC & &  \\
&& & STMTC / CDL / NIV / NIM / NOPA & &  \\ \hline 
48 & Szoke \etal \cite{szHoke2017empirical} & 2017 & CK & N/A & Maintainability \\ \hline
49 & Bashir \etal \cite{bashir2017methodology} & 2017 & QMOOD & N/A & Modifiability / Analyzability / Understandability / Maintainability \\ \hline 
50 & Mumtaz \etal \cite{mumtaz2018empirical} & 2018 &  CK / CCC / CDP / CCDA / COA & N/A & Security \\ 
& & &  CMW / CMAI / CAAI / CAIW & &  \\ \hline 
51 & Pantiuchina \etal \cite{pantiuchina2018improving} & 2018 &  LCOM / CBO / WMC / RFC  & Cohesion / Coupling / Complexity & Readability  \\
& &   &  C3 / B\&W / SRead & & \\ \hline 
52 & Alizadeh \& Kessentini \cite{alizadeh2018reducing} & 2018 & QMOOD & N/A & Reusability / Flexibility / Understandability / Functionality \\ 
& & & && Extendibility / Effectiveness   \\ \hline 
53 & Alizadeh \etal \cite{alizadeh2019refbot} & 2019 & QMOOD & N/A & Reusability / Flexibility / Understandability / Functionality \\ 
& & & && Extendibility / Effectiveness   \\ \hline 
54 & Techapalokul \& Tilevich \cite{techapalokul2019code} & 2019 & LOC / Complex Script Dens / No. Literals & N/A & N/A  \\ 
& & & Long Script Dens. / Procedure Dens. / No. Global Var & & \\
& & & No. Create Clone Of. & & \\ \hline 
55 & Counsell \etal \cite{counsell2019relationship} & 2019 & CBO & Coupling & N/A  \\ \hline 
56 & Fakhoury \etal \cite{fakhoury2019improving} & 2019 & Buse \& Weimer / Dorn / Scalabrino / Posnett & Cohesion / Coupling / Size / Complexity  & Readability \\ 
& & & LCOM 5 / WMC / RFC / MLOC / FLOC  & &  \\ 
& & & Halstead Difficulty / Halstead Effort / Maintainability index / MCC & &  \\ 
& & & Nesting level / Doc LOC / Comment Density / API Doc & & \\ 
& & & Public Undoc API / Public Doc API / \# Parantheses / & &  \\ 
& & & Number of Incoming Invocations & &  \\ \hline 
57 & AlOmar \etal \cite{alomar2019impact} & 2019 &  CK / FANIN / FANOUT / CC / NIV / NIM & Inheritance / Cohesion / Coupling / Complexity  & N/A \\ 
& &   & Evg / NPath / MaxNest / IFANIN & Size / Polymorphism / Encapsulation / Abstraction &   \\ 
& & &  LOC / CLOC / CDL / STMTC & &  \\ \hline 
58 & Rebai \etal \cite{rebai2019interactive} & 2019 & QMOOD & N/A & Reusability / Flexibility / Understandability / Functionality \\ 
& & & && Extendibility / Effectiveness   \\ \hline 
59 & Alizadeh \etal \cite{alizadeh2019less} & 2019 & QMOOD & N/A & Reusability / Flexibility / Understandability / Functionality \\ 
& & & && Extendibility / Effectiveness   \\ \hline 
60 & Alizadeh \etal \cite{alizadeh2018interactive} & 2020 & QMOOD & N/A & Reusability / Flexibility / Understandability / Functionality \\ 
& & & && Extendibility / Effectiveness   \\ \hline 
61 & Rebai \etal \cite{rebai2020enabling} & 2020 & QMOOD & N/A & Reusability / Flexibility / Understandability / Functionality \\ 
& & & && Extendibility / Effectiveness   \\ \hline 
62 & Fernandes \etal \cite{fernandes2020refactoring} & 2020    & CBO / WMC / DIT / NOC & Inheritance / Cohesion / Coupling / Size / Complexity & N/A  \\
&&  & LOC / LCOM2 / LCOM3 / WOC & &  \\
&& &  TCC / FANIN / FANOUT / CINT & &  \\
&& &  CDISP / CC / Evg / NPATH   & & \\
&& &  MaxNest / IFANIN / OR / CLOC & &  \\
&& &  STMTC / CDL / NIV / NIM / NOPA & &  \\ \hline
63 & AlOmar \etal \cite{alomar2020developers} & 2020 & CK /  CC / LOC & Inheritance / Cohesion / Coupling / Complexity / Size  & Reusability  \\ \hline
64 & Bibiano \etal \cite{bibiano2020does} & 2020 & LCOM2 / CBO / MAXNest / CC &  Cohesion / Coupling / Complexity / Size & N/A \\ 
& & & LOC / CLOC  / STMTC / NIV & &  \\ 
&& &  NIM / WMC & &  \\ \hline 
65 & Abid \etal \cite{abid2020does} & 2020 & QMOOD & N/A & Reusability / Flexibility / Understandability / Functionality \\ & & & && Extendibility / Effectiveness / Security  \\ \hline 
66 & Abid \etal \cite{abid2021prioritizing} & 2021 & QMOOD & N/A & Reusability / Flexibility / Understandability / Functionality  \\ 
& & & & & Extendibility / Effectiveness / Security   \\ \hline  
67 & Riansyah \& Mursanto \cite{riansyah2020empirical} & 2020 & CINT / CDISP & Coupling & N/A \\ \hline
68 & Iyad \etal \cite{alazzam2020impact} & 2020 & CK / CC / TLOC / MFA / NBD & Inheritance / Cohesion / Coupling / Complexity / Size & N/A \\
& & & NSC / CE & &  \\ \hline
69 & Hamdi \etal \cite{hamdi2021empirical} & 2021  & CBO / WMC / DIT / RFC & Inheritance / Cohesion / Coupling / Complexity / Size  & N/A\\ 
& &    & LCOM / TCC / LOC / LCC & &  \\ 
& & &  NOSI / VQTY & &  \\ \hline
70 & AlOmar \etal \cite{alomar2022refactoring} & 2021  & CK /  CC / LOC / NPATH & Inheritance / Cohesion / Coupling / Complexity / Size  & Reusability  \\ 
& & &  MaxNest / IFANIN / CDL / CLOC & &  \\
& & &  FANIN / FANOUT / STMTC / NIV & &  \\ \hline 
71 & Sellitto \etal \cite{sellittotoward} & 2021 & CIC / CIC\_syc / ITID / NMI / CR & N/A & Readability \\
& & & NM / TC / NOC / NOC\_norm & & \\ \hline
72 & Ouni \etal \cite{ouni2023impact}    & 2023  & LCOM / CBO / NOSI / TCC / NIV / IFANIN& Coupling / Cohesion / Complexity / Inheritance / Size & N/A\\ 
&  & & RFC / FANIN / WMC / VQYT / NIM & & \\
&  &  & FANOUT / CC / Evg / MaxNest / DIT & & \\
& & &  LOC / BLOC / CLOC / STMTC / NOC & & \\
%\hline


\bottomrule
\end{tabular}
\end{adjustbox}
%\end{sideways}
\end{table*}


% %%%%%%% backup %%%%%%
% \begin{table*}
%   \centering
% 	 \caption{\textcolor{blue}{A summary of the literature on the impact of refactoring activities on software quality attributes.}}
% 	 \label{Table:Quality Metrics in Related Work}
% %\begin{sideways}
% \begin{adjustbox}{width=1.2\textwidth,center}
% \rowcolors{2}{gray!25}{white}
% %\begin{adjustbox}{width=\textheight,totalheight=\textwidth,keepaspectratio}
% \begin{tabular}{llllll}\hline
% \toprule
%  \bfseries No. & \bfseries Study & \bfseries Year & \bfseries Quality Metric & \bfseries Internal QA & \bfseries External QA  \\
% \midrule

% 1 & Sahraoui \etal \cite{sahraoui2000can} & 2000  & CLD / NOC / NMO / NMI    & Inheritance / Coupling & Fault-proneness / Maintainability   \\ 
% & & &  NMA / SIX / CBO / DAC & & \\
% & & &  IH-ICP / OCAIC / DMMEC / OMMEC &  \\ \hline
% 2 & Stroulia \& Kapoor \cite{stroulia2001metrics} & 2001 & LOC / LCOM / CC & Size / Coupling & Design extensibility \\ \hline
% 3 & Kataoka \etal \cite{kataoka2002quantitative} & 2002 & Coupling measures &  Coupling & Maintainability   \\ \hline
% 4 & Demeyer \cite{demeyer2002maintainability} & 2002& N/A & Polymorphism  & Performance  \\ \hline
% 5 & Tahvildari \etal \cite{tahvildari2003quality} & 2003 & LOC / CC / CMT / Halstead's efforts & Complexity  & Performance / Maintainability   \\ \hline
% 6 & Leitch \& Stroulia \cite{leitch2003assessing}& 2003 & SLOC / No. of Procedure & Size & Maintainability  \\ \hline
% 7 & Bois \& Mens \cite{du2003describing} & 2003  & NOM / CC / NOC / CBO & Inheritance / Cohesion / Coupling / Size / Complexity & N/A  \\ 
% & & &  RFC / LCOM & & \\ \hline
% 8 & Tahvildari \& Kontogiannis \cite{tahvildari2003metric} & 2004  & LCOM / WMC / RFC / NOM   & Inheritance / Cohesion / Coupling / Complexity & Maintainability  \\ 
% & & &  CDE / DAC / TCC & & \\ \hline
% 9 & Bois \etal \cite{du2004refactoring} & 2004  & N/A & Cohesion / Coupling & Maintainability   \\ \hline
% 10 & Bois \etal \cite{du2005does} & 2005  & N/A & N/A &   Understandability   \\  \hline
% 11 & Geppert \etal \cite{geppert2005refactoring} & 2005  &  N/A & N/A & Changeability  \\ \hline
% 12 & Ratzinger \etal \cite{ratzinger2005improving} & 2005  & N/A &  Coupling & Evolvability \\ 
% & & & Analyzing code histories & \\ \hline
% 13 & Moser \etal \cite{moser2006does} & 2006  & CK / MCC / LOC   & Inheritance / Cohesion / Coupling / Complexity & Reusability \\ \hline
% 14 & Wilking \etal \cite{wilking2007empirical} & 2007  & CC / LOC  & Complexity & Maintainability / Modifiability   \\ \hline
% 15 & Stroggylos \& Spinells \cite{stroggylos2007refactoring} & 2007  & CK / Ca / NPM & Inheritance / Cohesion / Coupling / Complexity & N/A  \\ \hline
% 16 & Moser \etal \cite{moser2007case} & 2008 & CK / LOC / Effort (hour) & Cohesion / Coupling / Complexity & Productivity  \\ \hline
% 17 & Shrivastava \& Shrivastava \cite{shrivastava2008impact} & 2008 & NOA / NOC / NOM / CC & Inheritance / Complexity / Size & N/A\\ 
% & & & TLOC / DIT & \\ \hline 
% 18 & Higo \etal \cite{higo2008refactoring} & 2008 & CK & Inheritance / Cohesion / Coupling / Complexity & N/A\\ \hline
% 19 & Reddy \& Rao  \cite{Reddy2009quantitative} & 2009 & DOCMA (CR) \ DOCMA (AR) & Complexity & N/A \\ \hline
% 20 & Alshayeb \cite{alshayeb2009empirical} & 2009 &   CK / LOC / FANOUT  & Inheritance / Cohesion / Coupling / Size & Adaptability / Maintainability / Testability / Reusability  \\ 
% & & & & &  Understandability  \\ \hline
% 21 & Alshayeb \cite{alshayeb2009refactoring} & 2009 & LCOM1 / LCOM2 / LCOM3 / LCOM4 / LCOM5 & Cohesion & N/A \\ \hline
% 22 & Usha \etal \cite{usha2009quantitative} & 2009 & LCOM / CBO / WMC / RFC / CC & Cohesion / Coupling / Complexity & Modifiability / Modularity \\ 
% & & &  CF / TCC / MHF / AHF & & \\ \hline 
% 23 & Hegedus \etal \cite{hegedHus2010effect} & 2010  & CK  & Coupling / Complexity / Size & Maintainability / Testability / Error Proneness / Changeability  \\
% & & & & &  Stability / Analizability \\ \hline
% 24 & Shatnawi \& Li \cite{shatnawi2011empirical} & 2011 & CK / QMOOD &  Inheritance / Cohesion / Coupling / Polymorphism / Size & Reusability / Flexibility / Extendibility / Effectiveness     \\ 
% & & & & Encapsulation / Composition / Abstraction / Messaging &    \\ \hline
% 25 & Fontana \& Spinelli \cite{fontana2011impact} & 2011   & DAC / LCOM / NOM / RFC & Cohesion / Coupling / Complexity & N/A\\ 
% & & &  TCC / WMC & &  \\ \hline 
% 26 & Alshayeb \cite{alshayeb2011impact} & 2011  &  CK / LOC / FANOUT  & Inheritance / Cohesion / Coupling / Size & Adaptability / Maintainability / Testability / Reusability  \\ 
% & & & & &  Understandability \\ \hline
% % check again paper \cite{lerthathairat2011approach}
% 27 & Lerthathairat \& Prompoon \cite{lerthathairat2011approach} & 2011 & NLOC / NILI / CC / ILCC / NOP & Cohesion / Encapsulation   & N/A \\
% & & & NOM / NFD / LCOM / LCOM-HS &  & \\ \hline
% 28 & {\'O} Cinn{\'e}ide \etal \cite{o2012experimental} & 2012 & LSCC / TCC / CC / SCOM / LCOM5 & Cohesion & N/A \\ \hline 
% 29 & Ibrahim \etal \cite{ibrahim2012identification} & 2012 & LCCI / LCCD / LCC / TCC  & Cohesion & N/A \\
% & & & CC / Coh / LCOM3 &  \\ \hline 
% 30 & Singh \& Kahlon \cite{singh2011effectiveness} & 2011 &  CK / LCOM4 / PuF / EncF / NOD & Inheritance / Cohesion / Coupling / Information hiding &  \\
% & & & & Polymorphism / Encapsulation / Abstraction & \\\hline
% 31 & Singh \& Kahlon \cite{singh2012effectiveness} & 2012 & CK / LCOM4 / PuF / EncF / NOD & Inheritance / Cohesion / Coupling / Information hiding &  \\
% & & & & Polymorphism / Encapsulation / Abstraction & \\\hline
% 32 & Murgia \etal \cite{murgia2012refactoring} & 2012 & FANIN / FANOUT & Coupling & N/A \\ \hline 
% 33 & Kannangara \& Wijayanake \cite{kannangara2013impact} & 2013 &  N/A  & N/A & Analysability / Changeability / Time Behaviour / Resource Utilization  \\ \hline 
% 34 & Veerappa \& Rachel \cite{veerappa2013empirical} & 2013 & RFC / DCC / CBO / MPC & Coupling & N/A \\ \hline
% 35 & Napoli \etal \cite{napoli2013using} & 2013 & LCOM / CBO & Cohesion / Coupling  & Modularity \\ \hline 
% 36 & Bavota \etal \cite{bavota2013empirical} & 2013  & ICP / IC-CD / CCBC & Coupling & N/A \\
% & &  & & & \\ \hline
% 37 & Kumari \& Saha \cite{kumari2014effect} & 2014 & DIT / CBO / RFC / WMC  & Inheritance / Cohesion / Coupling / Complexity & Maintainability / Reusability / Testability / Understandability \\ 
% & & & LCOM / NOM / LOC &  &  Fault proneness / Completeness / Stability / Adaptability \\ \hline 
% 38 & Szoke \etal \cite{szoke2014bulk} & 2014 & CC / U / NOA / NII / NAni & Size / Complexity & N/A \\
% & &  & LOC / NUMPAR / NMni / NA & &  \\ \hline
% 39 & Chaparro \etal \cite{chaparro2014impact} & 2014 &  RFC / CBO / DAC / MPC & Inheritance / Cohesion / Coupling / Size / Complexity & N/A\\ 
% & &  & LOC / NOM / CC / LCOM2 & &  \\ 
% & & & LCOM5 / NOC / DIT & & \\ \hline
% 40 & Bavota \etal \cite{bavota2015experimental} & 2015  &  CK / LOC / NOA / NOO  &  Inheritance / Cohesion / Coupling / Size / Complexity & N/A \\
% & &  & C3 / CCBC & & \\ \hline
% 41 & Kannangara \& Wijayanake \cite{kannangara2015empirical} & 2015 &  CC / DIT / CBO / LOC & Maintainability index / Complexity / Coupling / Inheritance & Analysability / Changeability / Time Behaviour / Resource Utilization  \\ \hline 
% 42 & Gatrell \&  Counsel \cite{gatrell2015effect} & 2015 & N/A & N/A & change \& fault-proneness \\ \hline 
% 43 & Cedrim at al. \cite{cedrim2016does} & 2016 & LOC / CBO / NOM / CC & Cohesion / Coupling / Complexity & N/A  \\
% & &  & FANOUT / FANIN & &  \\ \hline
% 44 & Malhotra \& Chug \cite{malhotra2016empirical} & 2016 & CK & Cohesion / Coupling / Complexity / Inheritance & Understandability / Modifiability / Extensibility / Reusability   \\ 
% & & & & & Level of Abstraction \\ \hline 
% 45 & Mkaouer \etal \cite{mkaouer2016use} & 2016 & QMOOD & N/A & Reusability / Flexibility / Understandability / Functionality\\ 
% & & & && Extendibility / Effectiveness \\ \hline 
% 46 & Kaur \& Singh \cite{kaur2017improving} & 2017 & WMC / NOI / RFC / TCLOC  & Coupling / Complexity / Size  & Maintainability \\ 
% & & & TLLOC / TNOS /  CI  & & \\ \hline 
% 47 & Chavez \etal \cite{chavez2017does} & 2017  & CBO / WMC / DIT / NOC & Inheritance / Cohesion / Coupling / Size / Complexity & N/A  \\ 
% && &   LOC / LCOM2 / LCOM3 / WOC & &  \\
% && &  TCC / FANIN / FANOUT / CINT & &  \\
% && &  CDISP / CC / Evg / NPATH   & & \\
% && &  MaxNest / IFANIN / OR / CLOC & &  \\
% && & STMTC / CDL / NIV / NIM / NOPA & &  \\ \hline 
% 48 & Szoke \etal \cite{szHoke2017empirical} & 2017 & CK & N/A & Maintainability \\ \hline
% 49 & Bashir \etal \cite{bashir2017methodology} & 2017 & QMOOD & N/A & Modifiability / Analyzability / Understandability / Maintainability \\ \hline 
% 50 & Mumtaz \etal \cite{mumtaz2018empirical} & 2018 &  CK / CCC / CDP / CCDA / COA & N/A & Security \\ 
% & & &  CMW / CMAI / CAAI / CAIW & &  \\ \hline 
% 51 & Pantiuchina \etal \cite{pantiuchina2018improving} & 2018 &  LCOM / CBO / WMC / RFC  & Cohesion / Coupling / Complexity & Readability  \\
% & &   &  C3 / B\&W / SRead & & \\ \hline 
% 52 & Alizadeh \& Kessentini \cite{alizadeh2018reducing} & 2018 & QMOOD & N/A & Reusability / Flexibility / Understandability / Functionality \\ 
% & & & && Extendibility / Effectiveness   \\ \hline 
% 53 & Alizadeh \etal \cite{alizadeh2019refbot} & 2019 & QMOOD & N/A & Reusability / Flexibility / Understandability / Functionality \\ 
% & & & && Extendibility / Effectiveness   \\ \hline 
% 54 & Techapalokul \& Tilevich \cite{techapalokul2019code} & 2019 & LOC / Complex Script Dens / No. Literals & N/A & N/A  \\ 
% & & & Long Script Dens. / Procedure Dens. / No. Global Var & & \\
% & & & No. Create Clone Of. & & \\ \hline 
% 55 & Counsell \etal \cite{counsell2019relationship} & 2019 & CBO & Coupling & N/A  \\ \hline 
% 56 & Fakhoury \etal \cite{fakhoury2019improving} & 2019 & Buse \& Weimer / Dorn / Scalabrino / Posnett & Cohesion / Coupling / Size / Complexity  & Readability \\ 
% & & & LCOM 5 / WMC / RFC / MLOC / FLOC  & &  \\ 
% & & & Halstead Difficulty / Halstead Effort / Maintainability index / MCC & &  \\ 
% & & & Nesting level / Doc LOC / Comment Density / API Doc & & \\ 
% & & & Public Undoc API / Public Doc API / \# Parantheses / & &  \\ 
% & & & Number of Incoming Invocations & &  \\ \hline 
% 57 & AlOmar \etal \cite{alomar2019impact} & 2019 &  CK / FANIN / FANOUT / CC / NIV / NIM & Inheritance / Cohesion / Coupling / Complexity  & N/A \\ 
% & &   & Evg / NPath / MaxNest / IFANIN & Size / Polymorphism / Encapsulation / Abstraction &   \\ 
% & & &  LOC / CLOC / CDL / STMTC & &  \\ \hline 
% 58 & Rebai \etal \cite{rebai2019interactive} & 2019 & QMOOD & N/A & Reusability / Flexibility / Understandability / Functionality \\ 
% & & & && Extendibility / Effectiveness   \\ \hline 
% 59 & Alizadeh \etal \cite{alizadeh2019less} & 2019 & QMOOD & N/A & Reusability / Flexibility / Understandability / Functionality \\ 
% & & & && Extendibility / Effectiveness   \\ \hline 
% 60 & Alizadeh \etal \cite{alizadeh2018interactive} & 2020 & QMOOD & N/A & Reusability / Flexibility / Understandability / Functionality \\ 
% & & & && Extendibility / Effectiveness   \\ \hline 
% 61 & Rebai \etal \cite{rebai2020enabling} & 2020 & QMOOD & N/A & Reusability / Flexibility / Understandability / Functionality \\ 
% & & & && Extendibility / Effectiveness   \\ \hline 
% 62 & Fernandes \etal \cite{fernandes2020refactoring} & 2020    & CBO / WMC / DIT / NOC & Inheritance / Cohesion / Coupling / Size / Complexity & N/A  \\
% &&  & LOC / LCOM2 / LCOM3 / WOC & &  \\
% && &  TCC / FANIN / FANOUT / CINT & &  \\
% && &  CDISP / CC / Evg / NPATH   & & \\
% && &  MaxNest / IFANIN / OR / CLOC & &  \\
% && &  STMTC / CDL / NIV / NIM / NOPA & &  \\ \hline
% 63 & AlOmar \etal \cite{alomar2020developers} & 2020 & CK /  CC / LOC & Inheritance / Cohesion / Coupling / Complexity / Size  & Reusability  \\ \hline
% 64 & Bibiano \etal \cite{bibiano2020does} & 2020 & LCOM2 / CBO / MAXNest / CC &  Cohesion / Coupling / Complexity / Size & N/A \\ 
% & & & LOC / CLOC  / STMTC / NIV & &  \\ 
% && &  NIM / WMC & &  \\ \hline 
% 65 & Abid \etal \cite{abid2020does} & 2020 & QMOOD & N/A & Reusability / Flexibility / Understandability / Functionality \\ & & & && Extendibility / Effectiveness / Security  \\ \hline 
% 66 & Abid \etal \cite{abid2021prioritizing} & 2021 & QMOOD & N/A & Reusability / Flexibility / Understandability / Functionality  \\ 
% & & & & & Extendibility / Effectiveness / Security   \\ \hline  
% 67 & Riansyah \& Mursanto \cite{riansyah2020empirical} & 2020 & CINT / CDISP & Coupling & N/A \\ \hline
% 68 & Iyad \etal \cite{alazzam2020impact} & 2020 & CK / CC / TLOC / MFA / NBD & Inheritance / Cohesion / Coupling / Complexity / Size & N/A \\
% & & & NSC / CE & &  \\ \hline
% 69 & Hamdi \etal \cite{hamdi2021empirical} & 2021  & CBO / WMC / DIT / RFC & Inheritance / Cohesion / Coupling / Complexity / Size  & N/A\\ 
% & &    & LCOM / TCC / LOC / LCC & &  \\ 
% & & &  NOSI / VQTY & &  \\ \hline
% 70 & AlOmar \etal \cite{alomar2022refactoring} & 2021  & CK /  CC / LOC / NPATH & Inheritance / Cohesion / Coupling / Complexity / Size  & Reusability  \\ 
% & & &  MaxNest / IFANIN / CDL / CLOC & &  \\
% & & &  FANIN / FANOUT / STMTC / NIV & &  \\ \hline 
% 71 & Sellitto \etal \cite{sellittotoward} & 2021 & CIC / CIC\_syc / ITID / NMI / CR & N/A & Readability \\
% & & & NM / TC / NOC / NOC\_norm & & \\ \hline
% 72 & Ouni \etal \cite{ouni2023impact}    & 2023  & LCOM / CBO / NOSI / TCC / NIV / IFANIN& Coupling / Cohesion / Complexity / Inheritance / Size & N/A\\ 
% &  & & RFC / FANIN / WMC / VQYT / NIM & & \\
% &  &  & FANOUT / CC / Evg / MaxNest / DIT & & \\
% & & &  LOC / BLOC / CLOC / STMTC / NOC & & \\
% %\hline


% \bottomrule
% \end{tabular}
% \end{adjustbox}
% %\end{sideways}
% \end{table*}
The prevailing consensus in the software refactoring literature acknowledges its overarching aim of improving software quality and correcting poor design and implementation practices \citep{Fowler:1999:RID:311424}. \textcolor{black}{\textcolor{black}{Tables \ref{Table:Quality Metrics in Related Work} and \ref{Table:Quality Metrics in Related Work-v2}} illustrate two decades of work on a long-standing question within the refactoring community: Does refactoring improve code quality? In recent years, numerous research efforts have been made to examine and explore the influence of refactoring on software quality} \citep{moser2007case,wilking2007empirical,alshayeb2009empirical, shatnawi2011empirical,bavota2015experimental, chavez2017does,mkaouer2017robust, cedrim2016does,hegedHus2010effect}. 
%\ali{some references}.
Most studies have focused on measuring internal and external quality attributes to determine the quality of a software system being refactored. \textcolor{black}{Due to space constraints, this section provides a comprehensive review of some of these studies and a discussion of the pertinent literature on the impact of refactoring on software quality.}

 
Stroulia and Kapoor \citep{stroulia2001metrics} explored the effect of size and coupling measures on software quality after the refactoring application. Their findings indicated that size and coupling metrics decreased after refactorings. Fioravanti \etal \citep{fioravanti2001reengineering} analyzed and described metrics, based on duplication analysis, that
contribute to the process of reengineering analysis of
object-oriented. Antoniol \etal \citep{antoniol2002analyzing} studies cloning evolution in the Linux kernel. Their main result revealed that the Linux system does not contain a relevant fraction of code
duplication.  Kataoka \etal \citep{kataoka2002quantitative} focused solely on coupling measures to study the impact of \textit{Extract Method} and \textit{Extract Class} refactoring operations on the maintainability of a C++ software system. Their study revealed a positive effect of refactoring on system maintainability. Demeyer \citep{demeyer2002maintainability} conducted a comparative study to investigate the impact of refactoring on performance, the results demonstrating an improvement in program performance after refactoring. In addition, Sahraoui \etal \citep{sahraoui2000can} used coupling and inheritance measures to automatically identify potential antipatterns and predict scenarios in which refactoring could enhance software maintainability. The authors found that quality metrics can help bridge the gap between design improvement and automation, but in some situations the process cannot be fully automated, as it requires the programmer's validation through manual inspection. 

Tahvildari \etal \citep{tahvildari2003quality} introduced a software transformation framework that connects software quality requirements, such as performance and maintainability, with the transformation of the program to enhance the targeted qualities. Their results showed that utilizing design patterns increases the system's maintainability and performance. In a related study, Tahvildari and Kontogiannis \citep{tahvildari2003metric} applied the same framework to assess four object-oriented measures (cohesion, coupling, complexity, and inheritance) together with software maintainability. Leitch and Stroulia \citep{leitch2003assessing} utilized dependency graph-based techniques to investigate the impact of two refactorings, namely, \textit{Extract Method} and \textit{Move Method}, on software maintenance using two small systems. Their findings demonstrated that refactoring improved quality by reducing the size of the design, increasing the number of procedures, decreasing data dependencies and minimizing regression testing. Bios and Mens \citep{du2003describing} proposed a framework to analyze the impact of three refactorings on five internal quality attributes (\textit{i.e.,} cohesion, coupling, complexity, inheritance, and size). Their results indicated both positive and negative impacts on the selected measures.  Bios \etal \citep{du2004refactoring} provided a set of guidelines to optimize cohesion and coupling measures. Their study showed that the impact of refactoring on these measures ranged from negative to positive. In a subsequent study,  Bios \etal \citep{du2005does} differentiated between the application of Refactor to Understand and the traditional Read to Understand pattern, demonstrating that refactoring plays a role in improving software understandability. Rieger \etal \citep{rieger2004insights} provided insight into system-wide code duplication. The author proposed a way of grouping the duplication information
into useful abstractions and proposed a number of
polymetric views that structure the data and combine it
with the knowledge about the system that the engineer possesses.

Geppert \etal \citep{geppert2005refactoring} investigated the impact of refactoring on changeability by focusing on three factors: customer-reported defect rates, change effort, and scope of changes. Their findings showed a significant decrease in the first two factors. Ratzinger \etal \citep{ratzinger2005improving} analyzed historical data from a large industrial system, focusing on reducing change couplings. By examining identified change couplings and corresponding code smell changes, they determined efficient areas for applying refactoring, concluding that refactoring has the potential to enhance software evolvability, specifically by reducing change coupling. In an agile development environment, Moser \etal \citep{moser2006does} used internal measures (\textit{i.e.,} CK, MCC, LOC) to explore the effect of refactoring on the reusability of the code using a commercial system. Their study indicated that refactoring could enhance the reusability of classes that initially were difficult to reuse. Wilking \etal \citep{wilking2007empirical} empirically studied the effect of refactoring on non-functional aspects, \textit{i.e.,} the maintainability and modifiability of system systems. They tested maintainability by explicitly adding defects to the code and then measured the time taken to remove them. However, modifiability was examined by adding new functionality and then measuring the LOC metric and the time taken to implement these features. The authors did not find a clear effect of refactoring on these two external attributes. 

Stroggylos and Spinellis \citep{stroggylos2007refactoring} opted for
searching words stemming from the verb ``refactor" such
as \say{refactoring} or \say{refactored} to identify commits related to refactoring to study the impact of refactoring on quality using eight object-oriented metrics. Their results indicated possible negative effects of refactoring on quality, \textit{e.g.,} increased LCOM metric. Moser \etal \citep{moser2007case} investigated the impact of refactoring on productivity within an agile team. Their results indicated that refactoring not only enhanced software developers' productivity but also positively affected various quality aspects, such as maintainability. Alshayeb \citep{alshayeb2009empirical} conducted a study aiming to assess the impact of eight refactorings on five external quality attributes (\textit{i.e.,} adaptability, maintainability, understandability, reusability, and testability). The author found that refactoring could improve the quality in some classes, but could also decrease software quality to some extent in other classes. Hegedus \etal \citep{hegedHus2010effect} examined the effect of singular refactoring techniques on testability, error proneness, and other maintainability attributes. They concluded that refactoring could have undesirable side effects that can degrade the quality of the source code. 

Shatnawi and Li \citep{shatnawi2011empirical} used the hierarchical quality model to assess the impact of refactoring on four quality factors in software, namely reusability, flexibility, extendibility, and effectiveness. The authors found that most of the refactoring operations have a positive impact on quality; however, some operations deteriorated quality. Bavota \etal empirically investigated the developers' perception of coupling, as captured by structural, dynamic, semantic, and logical coupling measures. They found that the semantic coupling measure aligns with developers' perceptions better than the other. Bavota \etal \citep{bavota2015experimental} used \texttt{RefFinder}\footnote{https://github.com/SEAL-UCLA/Ref-Finder}, 
%\citep{5609577}
a version-based refactoring detection tool, to mine the evolution history of three open-source systems. They mainly investigated the relationship between refactoring and quality. The findings of the study indicate that 42\% of the refactorings performed are affected by code smells, and the refactorings were able to eliminate code smells only in 7\% of the cases. 

Cedrim \etal \citep{cedrim2016does} conducted a longitudinal study involving 25 projects to explore the improvement of the structural quality of software. They examined the relationship between refactorings and code smells, categorizing refactorings based on whether they added or removed problematic code structures. The study results indicate that only 2.24\% of the refactorings removed the code smells, while 2.66\% introduced new ones. Chavez \etal \citep{chavez2017does} investigated the impact of refactoring on five internal quality attributes—cohesion, coupling, complexity, inheritance, and size—using 25 quality metrics. Their study indicated that root-canal refactoring-related operations either improved or at least did not worsen internal quality attributes. Furthermore, when floss refactoring-related operations are applied, 55\% of these operations improved these attributes, while only 10\% quality decreased. Pantiuchina \etal \citep{pantiuchina2020developers} investigated the motivation behind refactoring by computing 42 product and process metrics for each of the 213,102 commits in
the studied projects.

%Fakhoury \etal \citep{fakhoury2019improving} have shown that the existing readability models are not able to capture the improvement in readability with minor code changes, and some metrics which can effectively measure the readability improvement are currently not used by readability models. %AlOmar \etal \citep{alomar2019impact} showed a misperception between the state-of-the-art structural metrics widely used as indicators for refactoring and what developers consider to be an improvement in their source code. The research aims to identify (among software quality attributes) the metrics that align with the vision of developers on the quality attribute they explicitly state they want to improve. Their approach entailed mining 322,479 commits from 3,795 open source projects, from which they identified about 1,245 commits based on commit messages that explicitly informed the refactoring towards improving quality attributes. Thereafter, they processed the identified commits by measuring structural metrics before and after the changes. The variations in values were then comparable to distinguish metrics significantly affected by the refactoring, towards better reflecting the intention of developers to improve the corresponding quality attribute. %In follow-up work, AlOmar \etal \citep{alomar2020developers,alomar2022refactoring}  utilized software quality metrics to evaluate the impact of refactoring on reusability.

%Hamdi \etal and Ouni \etal \citep{hamdi2021empirical,ouni2023impact} propose a first analysis on the impact of refactoring on quality metrics in the context of Android applications. They determined the effect each refactoring had upon software quality metrics, and employed the difference-in-differences (DiD) model to determine how much the metric changes brought about by refactoring differ from the metric changes in non-refactoring commits. The results show that metrics can be a strong indicator of refactoring activity, regardless of whether it improves or degrades these metric values.

Two studies, particularly relevant to our work, have delved into comment commits in which developers explicitly aimed to improve software quality \citep{szoke2014bulk,pantiuchina2018improving}. Szoke \etal \citep{szoke2014bulk} studied 198 refactoring commits of five large-scale industrial systems to investigate the effects of these commits on the quality of several revisions over a period of time. To understand the purpose of the applied refactorings, developers were trained and asked to articulate the reason when committing changes to repositories, relating to (1) fixing coding issues, (2) addressing anti-patterns, and (3) improving specific metrics. The results of the study showed that performing a single refactoring could negatively impact the quality, but applying refactorings in blocks (\eg fixing more coding issues or improving more quality metrics) can significantly improve software quality. In a related study, Pantiuchina \etal \citep{pantiuchina2018improving} empirically investigated the correlation between seven code metrics and the quality improvement explicitly reported by developers in 1,282 commit messages. The study showed that quality metrics sometimes do not capture the quality improvement reported by developers. Both studies used quality metrics as a common indicator to assess quality improvements, concluding that minor refactoring changes rarely had a substantial impact on software quality.

All of the aforementioned studies have focused on evaluating the impact of refactorings on quality by examining either internal or external quality attributes through various methodologies. Among them, few studies \citep{ratzinger2005improving, stroggylos2007refactoring, szoke2014bulk, bavota2015experimental, cedrim2016does, chavez2017does, pantiuchina2018improving, hamdi2021empirical,ouni2023impact,alomar2020developers,alomar2022refactoring,fakhoury2019improving,alomar2019impact} mined software repositories to explore the impact on quality. Otherwise, the vast majority of these studies %except Bavota et al. \citep{bavota2015experimental} and Szuke et al. \citep{szoke2014bulk},
used a limited set of projects and mined general commits without applying any form of verification regarding whether refactorings have actually been applied. 

Our work differs from these studies shown \textcolor{black}{in Tables \ref{Table:Quality Metrics in Related Work} and \ref{Table:Quality Metrics in Related Work-v2}}, as our main purpose is to explore whether there is an alignment between quality metrics and the removal of code duplication that developers document in the commit messages. As we summarize these state-of-the-art studies, we identify 5 popular quality attributes, namely \textit{Cohesion}, \textit{Coupling}, \textit{Complexity}, \textit{Inheritance}, and \textit{Design size}. Given the varied metrics advocated by different studies to calculate these quality attributes, we extracted and calculated 32 structural metrics. In a more qualitative sense, we conducted an empirical study using 322 distinct commits that are proven to contain real-world instances of refactoring activities, with the purpose of removing code duplication. To the best of our knowledge, no previous study has empirically investigated, using a curated set of commits, the representativeness of structural design metrics for code duplication. The next section details the steps we took to design our empirical setup.