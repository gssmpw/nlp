\section{Introduction}
\label{Section:Introduction}

%Scope
Duplicating a code fragment involves the process of copying and pasting it, with or without minor modifications, into another section of the codebase. Although this may appear to be an intuitive way to reuse code, the presence of duplicate code introduces a set of challenges in the maintenance and evolution of software systems~\citep{roy2009comparison}. Recent studies emphasize that duplicate code has become a significant issue that affects both developers and researchers. For example, fixing a bug in a duplicate code fragment may require applying the same fix to all instances of that code \citep{thongtanunam2019will}. This can result in a slowdown of maintenance efforts and potentially lead to the widespread propagation of bugs throughout the codebase. Consequently, in response to these challenges, the elimination of duplicate code through refactoring has become a natural and necessary course of action \citep{fanta1999removing}.

Refactoring is the art of remodeling software design without altering its functionality \citep{Fowler:1999:RID:311424}. It is a critical software maintenance activity that developers perform for a variety of purposes: improve program comprehension, eliminate duplicate code, reduce complexity, manage technical debt, and remove code smells~\citep{silva2016we,murphy2008breaking}. Refactoring duplicate code involves the process of taking a code fragment and moving it to create a new method while replacing all instances of that fragment with a call to this newly created method.
%Problem Statement

Despite the increasing emphasis on recommending refactorings through the optimization of structural metrics and the removal of code smells, there is limited evidence on whether developers follow these recommendations when refactoring their code. A study by Pantiuchina \etal \citep{pantiuchina2018improving} has shown that there is a misperception between state-of-the-art structural metrics, widely used as indicators for refactoring, and what developers actually consider to be an improvement in their source code. Furthermore, AlOmar \etal \citep{alomar2019impact} reveals that most metrics that are mapped to the main quality
attributes, \ie cohesion, coupling, and complexity, capture the developer's intentions of quality improvement reported in the
commit messages. In contrast, there is also a case in which
the metrics do not capture quality improvement as perceived
by developers. \textcolor{black}{This paper builds upon our previously published paper \citep{alomar2019impact}. Although earlier research primarily focused on internal quality attributes, placing particular emphasis on whether developer intentions regarding cohesion, coupling, complexity, inheritance, polymorphism, encapsulation, abstraction, and size aligned with their vision of quality optimization, this paper presents a more comprehensive qualitative and quantitative approach with a focus on code duplication. Code duplication is a critical quality issue, especially with recent emphasis on developing tools for this purpose. Furthermore, a study involving GitHub contributors has shown that they are seriously concerned about code duplication \citep{silva2016we}. Specifically, we extend the study by}:
 \begin{itemize}
     \item \textcolor{black}{Examining which metrics are most impacted by refactorings, aiming to identify those that closely capture the developer's intention regarding code duplication rather than internal quality attributes.}
    \item \textcolor{black}{Providing numerous qualitative examples that offer deeper insights into the underlying reasons for instances of alignment and disparity between quality metrics and developers' perception of the removal of duplicate code.}
   \item \textcolor{black}{Offering lessons and insights derived from our experiments to developers, tool builders, and the research community, contributing to the advancement of both the state-of-the-art and state-of-the-practice in code duplication-aware refactoring practices.}
   \item \textcolor{black}{Utilizing an entirely different dataset than the previous work.}
\item \textcolor{black}{Leveraging 32 structural metrics, which encompasses 22 distinct metrics that were not employed in the previous paper.}
     \end{itemize}









%This paper extends our previous work \citep{alomar2019impact}. Although our previous research mainly focused on internal quality attributes, placing particular emphasis on whether developer intentions regarding cohesion, coupling, complexity, inheritance, polymorphism, encapsulation, abstraction, and size aligned with their vision of quality optimization, this paper presents a more comprehensive qualitative and quantitative approach with a focus on code duplication. 


%by identifying, among the various quality models presented in the literature, the ones that are more in line with the developer's vision of code duplicates, when they explicitly state that they are refactoring to improve it.

%Thus, there is a need to distinguish, among all the structural metrics, typically used in the refactoring literature, the particular ones that are of a better representation of the developers' perception of code duplicate removal. 

%This paper aims in identifying, among the various quality models presented in the literature, the ones that are more in line with the developer's vision of code duplicates, when they explicitly state that they are refactoring to improve it. 

%Solution

We start with reviewing studies from the literature that propose quality attributes of software and their corresponding measurement in the source code, in terms of metrics. Software quality attributes are typically characterized by high-level definitions whose interpretations allow multiple ways to calculate them in the source code. Thus, there is little consensus on what would be the optimal match between quality attributes and code-level design metrics. For instance, as shown later in Section \ref{Section:Background}, the notion of complexity was the subject of many studies that proposed several metrics to calculate it. Therefore, we investigate which code-level metrics are more representative of the removal of high-level code duplicates, when their optimization is explicitly stated by the developer when applying refactorings. Furthermore, we investigate the performed refactoring operations, for each explicitly mentioned removal of code duplication. 


%To perform this analysis, we formulate the following research questions:

%\begin{itemize}
   % \item \textbf{RQ$_1$: \textit{Is the developer's perception of code duplicate removal aligned with the quantitative assessment of code quality?}}
   % \item \textbf{RQ$_2$: \textit{What are the refactoring operations that are associated with code duplicate removal?}}
%\end{itemize}


Practically, we have classified 322 commits, as duplicated code removal commits, by manually analyzing their messages and identifying an explicit statement of removing duplicated code, along with detecting their refactoring activities. We use the SmartSHARK dataset \citep{trautsch2021msr}, and refine it by untangling each commit to select only refactored code elements. Afterward, we calculate the values of its corresponding structural metrics, in the files, before and after their refactorings. And finally, we empirically compare the variation of these values, to distinguish the metrics that are significantly impacted by the refactorings, and so they better reflect the developer's intention of removing code duplication. To the best of our knowledge, no previous study has investigated the relationship between the intention of code duplicate removal and their corresponding structural metrics, from the developer's perception. Our key findings show that not all state-of-the-art structural metrics equally represent code duplication. %some quality attributes are being more emphasized than others by developers. %The dataset of refactoring commits, along with structural metrics is available online \hl{\#} for replication and extension purposes.
 In summary, this extended paper makes the following key contributions:

 \begin{itemize}
       
     %\item We extensively review the literature on quality attributes, used in the literature on software quality, and their corresponding possible measurements, in terms of metrics.
     \item In our empirical investigation into the removal of duplicate code, we examine which metrics are most affected by refactorings, aiming to identify those that closely capture the developer's intention. 
\item We provide numerous qualitative examples that offer deeper insights into the underlying reasons for instances of alignment and disparity between quality metrics and developers' perception of the removal of duplicate code.
\item We offer lessons and insights derived from our experiments to developers, tool builders, and the research community, aiming to contribute to the advancement of both the state-of-the-art and state-of-the-practice in code duplication-aware refactoring practices.
\item We provide our  experiment package to further replicate and extend our study. The package contains the raw data, analyzed data, statistical test results, and custom scripts used in our research\footnote{\textcolor{black}{\url{https://smilevo.github.io/self-affirmed-refactoring/}}}. %\citep{Replication-Package}.
 \end{itemize}
%Remainder 
The remainder of this paper is organized as follows. Section \ref{Section:Background} reviews existing studies related to the impact of refactoring on quality. Section \ref{Section:methodology} outlines our empirical setup in terms of data extraction, analysis, and research question. Section \ref{Section:Result} discusses our findings, while the \textcolor{black}{lessons learned and } implications of the research are discussed in Sections \ref{Section:lesson} and \ref{Section:Implication}, respectively. Section \ref{Section:Threats} captures any threats to the validity of our work, before concluding with Section \ref{Section:Conclusion}.