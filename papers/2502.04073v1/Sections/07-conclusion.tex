\section{Conclusion}
\label{Section:Conclusion}

We conducted an empirical study to investigate the alignment between code duplicate removal and software design metrics focusing on 5 internal quality attributes and 32 structural metrics. In particular, we obtained a corpus of more than two million refactoring commits from 128 open-source Java projects. We then extracted 32 structural metrics to identify code duplicate removal commits and the refactoring operations associated with them. \textcolor{black}{ In summary, the main conclusions are:}

\textemdash \textcolor{black}{Our findings show that some state-of-the-art metrics can capture the developer's intention of removing code duplication with different degrees of improvement
and degradation of software quality.}

\textemdash \textcolor{black}{Many metrics associated with key quality attributes, such as cohesion, coupling, complexity, and size, reflect the developers' intentions for duplicate removal mentioned in commit messages. In contrast, there are instances where the metrics do not represent the quality improvements stated by the developers.}

\textemdash \textcolor{black}{As for inheritance, NOC generally decreases as the developer intends to remove code duplication. DIT and NOA exhibit
opposite variations in inheritance, so these findings motivate a deeper investigation to understand the mismatch between theory and practice.}

\textcolor{black}{\textit{Implications.} As most of the mapped metrics associated with the main quality attributes successfully capture developers’ intentions for removing code duplicates, as is evident from the commit messages, we believe our study enables the following novel applications:}

\textemdash \textcolor{black}{Given that features significantly influence the quality of machine learning models, we can help identify which metrics may function as effective features within these algorithms, thus supporting developers in their decision-making. Using the most influential metrics as features to predict whether a particular code fragment should be subject to a specific refactoring operation, we can enhance developers’ confidence in accepting recommended refactorings or selecting the most suitable refactoring candidate.}

\textemdash \textcolor{black}{Empirical researchers can focus on providing a comprehensive taxonomy for code duplication-aware refactoring practices, showing various contexts of code duplicates and refactoring and demonstrating different forms of code reuse.}

\textemdash \textcolor{black}{A qualitative investigation through a developer survey can be conducted to better understand the motivation behind refactoring activities in the context of code duplicate removal to improve software metrics.}

%\textemdash \textcolor{black}{Although this study focuses on the SmartSHARK dataset, we plan to compare results across different sources to assess whether developer behavior introduces variability.}




%In the future, we plan to conduct a qualitative investigation through a developer survey to better understand the motivation behind refactoring activities in the context of code duplicate removal to improve software metrics. 

\section{Acknowledgments}
\noindent\textbf{Declaration of generative AI and AI-assisted technologies in the writing process.}
During the preparation of this work, the author used the ChatGPT web interface and the Wrietfull tool to improve the language and readability of the manuscript. After using this tool, the author reviewed and edited the content as needed and takes full responsibility for the content of the publication.