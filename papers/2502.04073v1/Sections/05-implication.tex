\section{Implications}
\label{Section:Implication}

%The main implications of this study are as follows.

 \noindent{\textbf{ Further advancing quality metrics and duplicate code removal.}} The existing literature discusses various automatic refactoring approaches aimed at assisting practitioners in detecting antipatterns or code smells. Baqais and Alshayeb \citep{baqais2020automatic} have highlighted the growing interest in automatic refactoring studies. The researchers explored the potential of machine learning to identify refactoring opportunities. Since features play a vital role in the quality of machine learning models obtained, this study can contribute to determining which metrics can serve as effective features in machine learning algorithms, facilitating the accurate recommendation of refactoring opportunities at different levels of granularity (\ie class, method, field), which can assist developers in automatically making their decisions. For example, incorporating the most impactful metrics as features in predicting whether a given piece of code should undergo a specific refactoring operation can enhance developers' confidence in accepting recommended refactorings or selecting the most suitable refactoring candidate. This knowledge is needed because, in practice, the built model should require as little data as possible. Furthermore, since we observe that some of the quality metrics did not capture any improvement, we plan to conduct more experiments to validate the effectiveness of these metrics to explore whether the observations are due to the appropriateness of the quality metrics or to the needed validation and clarity of the perception of the developers.

 \noindent{\textbf{ Putting developer in the loop when designing refactoring recommendation systems.}} Based on the findings, it becomes evident that different structural metrics have the capacity to depict code duplication, thereby influencing software quality in diverse ways. Certain metrics improve software quality, whereas others might result in its decline. %This variation highlights that not all metrics align with developers' intentions, as documented in their commit messages. 
 This underscores the importance of involving developers in the design of refactoring recommendation systems, effectively engaging them in the process. This approach emerges as effective in discerning meaningful refactorings that align with the perspectives of developers \citep{hall2012supervised,bavota2012putting,pantiuchina2018improving}.


 \noindent{\textbf{ Examining the code duplicate removal potentials with refactoring.} Our study reveals the context in which developers refactor the code to eliminate code duplicates. Our future research direction can focus on providing a comprehensive taxonomy for code duplication-aware refactoring practices. This taxonomy can show various contexts of code duplicates and refactoring and can demonstrate different forms of code reuse. Thereafter, researchers can build on top of our findings to better understand developer practices and investigate to what extent this taxonomy for refactoring with awareness of duplicate code improves the system's quality.


 \noindent\textbf{ Understanding the completeness of the quality metric capturing duplicate code removal as documented by developers.} We observe that not all quality metrics can capture the improvement in duplicate code removal perceived by developers in their commit messages. Although quality metrics can help pinpoint design flaws for refactoring recommendation systems, such a recommendation would be meaningful if qualitative insights from developers complemented it. Furthermore, the alignment or disparity between the enhancement of software quality as perceived by developers and its evaluation through quality metrics can be attributed to factors such as the focused nature of the duplication removal, the extent of duplication, and the potential compensatory changes. Future research is encouraged to consider the direct effect of duplicate removal and the broader context of code changes and their implications for quality metrics. 

%\faThumbTack \noindent{\textbf{ Lack of clarity of how the code duplicate removal tools leverage metrics and decide the associated threshold to make the decision.}} Existing `Extract Method' refactoring tools such as \texttt{AntiCopyPaster} \citep{alomar2022anticopypaster,alomar2023just} used 78 metrics related to size, complexity, coupling, and keywords to extract duplicate code. On the contrary, \texttt{Aries} \citep{higo2005aries} used six other metrics to identify the removal of the code clone. However, the implementation of these metrics may vary between these tools based on the context. In addition, there may be cases where different metric names are used to improve some quality attributes. This phenomenon might affect the interpretation of the accuracy of the recommended code duplicate removal tools. 

 \noindent{\textbf{ Investigating the characteristics and effects of eliminating code duplication on software quality.} The results advance our understanding of the effects of eliminating code duplication on software quality. It is evident that certain software quality metrics can be used as indicators for code fragments that are more likely to be extracted and identified as problematic and should be removed by refactoring. Consequently, a threshold can be established to show when quality metrics reach a level where duplicate code will have a negative effect on maintenance and need to be refactored.

%he outcome underscores that while code duplication removal can impact various aspects of a codebase, its effects on inheritance metrics might be context-dependent and influenced by the specific structure of the codebase.