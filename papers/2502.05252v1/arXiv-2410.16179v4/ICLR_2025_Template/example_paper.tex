%%%%%%%% ICML 2025 EXAMPLE LATEX SUBMISSION FILE %%%%%%%%%%%%%%%%%

\documentclass{article}

% Recommended, but optional, packages for figures and better typesetting:
\usepackage{microtype}
\usepackage{graphicx}
\usepackage{subfigure}
\usepackage{booktabs} % for professional tables
\usepackage{longtable} 
\usepackage{colortbl} 
\usepackage{multirow} 
\usepackage{xcolor} 


\definecolor{skyblue}{RGB}{135, 206, 235} 
\definecolor{palegreen}{RGB}{152, 251, 152}

% hyperref makes hyperlinks in the resulting PDF.
% If your build breaks (sometimes temporarily if a hyperlink spans a page)
% please comment out the following usepackage line and replace
% \usepackage{icml2025} with \usepackage[nohyperref]{icml2025} above.
\usepackage{hyperref}


% Attempt to make hyperref and algorithmic work together better:
\newcommand{\theHalgorithm}{\arabic{algorithm}}

% Use the following line for the initial blind version submitted for review:
\usepackage{icml2025}

% If accepted, instead use the following line for the camera-ready submission:
%\usepackage[accepted]{icml2025}

% For theorems and such
\usepackage{amsmath}
\usepackage{amssymb}
\usepackage{mathtools}
\usepackage{amsthm} 
\usepackage{xspace}
\usepackage{xcolor} 

% if you use cleveref..
\usepackage[capitalize,noabbrev]{cleveref}

%%%%%%%%%%%%%%%%%%%%%%%%%%%%%%%%
% THEOREMS
%%%%%%%%%%%%%%%%%%%%%%%%%%%%%%%%
\theoremstyle{plain}
\newtheorem{theorem}{Theorem}[section]
\newtheorem{proposition}[theorem]{Proposition}
\newtheorem{lemma}[theorem]{Lemma}
\newtheorem{corollary}[theorem]{Corollary}
\theoremstyle{definition}
\newtheorem{definition}[theorem]{Definition}
\newtheorem{assumption}[theorem]{Assumption}
\theoremstyle{remark}
\newtheorem{remark}[theorem]{Remark}
\usepackage{lipsum}
\usepackage{enumitem}

% Todonotes is useful during development; simply uncomment the next line
%    and comment out the line below the next line to turn off comments
%\usepackage[disable,textsize=tiny]{todonotes}
\usepackage[textsize=tiny]{todonotes}
% \usepackage{subcaption}

% The \icmltitle you define below is probably too long as a header.
% Therefore, a short form for the running title is supplied here:
\icmltitlerunning{Submission and Formatting Instructions for ICML 2025}

\begin{document}

\twocolumn[
\icmltitle{GSM-\Large{$\infty$}: How Do your LLMs Behave over \\Infinitely Increasing Reasoning Complexity and Context Length?} 

% It is OKAY to include author information, even for blind
% submissions: the style file will automatically remove it for you
% unless you've provided the [accepted] option to the icml2025
% package.

% List of affiliations: The first argument should be a (short)
% identifier you will use later to specify author affiliations
% Academic affiliations should list Department, University, City, Region, Country
% Industry affiliations should list Company, City, Region, Country

% You can specify symbols, otherwise they are numbered in order.
% Ideally, you should not use this facility. Affiliations will be numbered
% in order of appearance and this is the preferred way.
\icmlsetsymbol{equal}{*}

\begin{icmlauthorlist}
\icmlauthor{Firstname1 Lastname1}{equal,yyy}
\icmlauthor{Firstname2 Lastname2}{equal,yyy,comp}
\icmlauthor{Firstname3 Lastname3}{comp}
\icmlauthor{Firstname4 Lastname4}{sch}
\icmlauthor{Firstname5 Lastname5}{yyy}
\icmlauthor{Firstname6 Lastname6}{sch,yyy,comp}
\icmlauthor{Firstname7 Lastname7}{comp}
%\icmlauthor{}{sch}
\icmlauthor{Firstname8 Lastname8}{sch}
\icmlauthor{Firstname8 Lastname8}{yyy,comp}
%\icmlauthor{}{sch}
%\icmlauthor{}{sch}
\end{icmlauthorlist}

\icmlaffiliation{yyy}{Department of XXX, University of YYY, Location, Country}
\icmlaffiliation{comp}{Company Name, Location, Country}
\icmlaffiliation{sch}{School of ZZZ, Institute of WWW, Location, Country}

\icmlcorrespondingauthor{Firstname1 Lastname1}{first1.last1@xxx.edu}
\icmlcorrespondingauthor{Firstname2 Lastname2}{first2.last2@www.uk}

% You may provide any keywords that you
% find helpful for describing your paper; these are used to populate
% the "keywords" metadata in the PDF but will not be shown in the document
\icmlkeywords{Machine Learning, ICML}

\vskip 0.3in
]

% this must go after the closing bracket ] following \twocolumn[ ...

% This command actually creates the footnote in the first column
% listing the affiliations and the copyright notice.
% The command takes one argument, which is text to display at the start of the footnote.
% The \icmlEqualContribution command is standard text for equal contribution.
% Remove it (just {}) if you do not need this facility.

%\printAffiliationsAndNotice{}  % leave blank if no need to mention equal contribution

\printAffiliationsAndNotice{\icmlEqualContribution} % otherwise use the standard text. 
\newcommand{\sysb}{\textsc{GSM}-{\large{$\infty$}}\xspace} 

\begin{abstract}
Recently, long-context large language models (LLMs) have shown strong performance in information retrieval and long-document QA. However, to tackle the most challenging intellectual problems, LLMs must reason effectively in long and complex contexts (e.g., frontier mathematical research). Studying how LLMs handle increasing reasoning complexity and context length is essential, yet existing benchmarks lack a solid basis for quantitative evaluation. Inspired by the abstraction of GSM-8K problems as computational graphs—and the ability to introduce noise by adding unnecessary nodes and edges—we develop a grade-school math problem generator capable of producing arithmetic problems with infinite difficulty and context length under fine-grained control. Using our newly synthesized GSM-$\infty$ benchmark, we comprehensively evaluate existing LLMs. We find a consistent sigmoid decline in reasoning performance as complexity increases, along with a systematic inference scaling trend: exponentially increasing inference computation yields only linear performance gains. These findings underscore the fundamental limitations of current long-context LLMs and the key challenges in scaling reasoning capabilities. Our GSM-$\infty$ benchmark provides a scalable and controllable testbed for systematically studying and advancing LLM reasoning in long and complex contexts.

\end{abstract}

\section{Introduction}
\label{sec:introduction}
The business processes of organizations are experiencing ever-increasing complexity due to the large amount of data, high number of users, and high-tech devices involved \cite{martin2021pmopportunitieschallenges, beerepoot2023biggestbpmproblems}. This complexity may cause business processes to deviate from normal control flow due to unforeseen and disruptive anomalies \cite{adams2023proceddsriftdetection}. These control-flow anomalies manifest as unknown, skipped, and wrongly-ordered activities in the traces of event logs monitored from the execution of business processes \cite{ko2023adsystematicreview}. For the sake of clarity, let us consider an illustrative example of such anomalies. Figure \ref{FP_ANOMALIES} shows a so-called event log footprint, which captures the control flow relations of four activities of a hypothetical event log. In particular, this footprint captures the control-flow relations between activities \texttt{a}, \texttt{b}, \texttt{c} and \texttt{d}. These are the causal ($\rightarrow$) relation, concurrent ($\parallel$) relation, and other ($\#$) relations such as exclusivity or non-local dependency \cite{aalst2022pmhandbook}. In addition, on the right are six traces, of which five exhibit skipped, wrongly-ordered and unknown control-flow anomalies. For example, $\langle$\texttt{a b d}$\rangle$ has a skipped activity, which is \texttt{c}. Because of this skipped activity, the control-flow relation \texttt{b}$\,\#\,$\texttt{d} is violated, since \texttt{d} directly follows \texttt{b} in the anomalous trace.
\begin{figure}[!t]
\centering
\includegraphics[width=0.9\columnwidth]{images/FP_ANOMALIES.png}
\caption{An example event log footprint with six traces, of which five exhibit control-flow anomalies.}
\label{FP_ANOMALIES}
\end{figure}

\subsection{Control-flow anomaly detection}
Control-flow anomaly detection techniques aim to characterize the normal control flow from event logs and verify whether these deviations occur in new event logs \cite{ko2023adsystematicreview}. To develop control-flow anomaly detection techniques, \revision{process mining} has seen widespread adoption owing to process discovery and \revision{conformance checking}. On the one hand, process discovery is a set of algorithms that encode control-flow relations as a set of model elements and constraints according to a given modeling formalism \cite{aalst2022pmhandbook}; hereafter, we refer to the Petri net, a widespread modeling formalism. On the other hand, \revision{conformance checking} is an explainable set of algorithms that allows linking any deviations with the reference Petri net and providing the fitness measure, namely a measure of how much the Petri net fits the new event log \cite{aalst2022pmhandbook}. Many control-flow anomaly detection techniques based on \revision{conformance checking} (hereafter, \revision{conformance checking}-based techniques) use the fitness measure to determine whether an event log is anomalous \cite{bezerra2009pmad, bezerra2013adlogspais, myers2018icsadpm, pecchia2020applicationfailuresanalysispm}. 

The scientific literature also includes many \revision{conformance checking}-independent techniques for control-flow anomaly detection that combine specific types of trace encodings with machine/deep learning \cite{ko2023adsystematicreview, tavares2023pmtraceencoding}. Whereas these techniques are very effective, their explainability is challenging due to both the type of trace encoding employed and the machine/deep learning model used \cite{rawal2022trustworthyaiadvances,li2023explainablead}. Hence, in the following, we focus on the shortcomings of \revision{conformance checking}-based techniques to investigate whether it is possible to support the development of competitive control-flow anomaly detection techniques while maintaining the explainable nature of \revision{conformance checking}.
\begin{figure}[!t]
\centering
\includegraphics[width=\columnwidth]{images/HIGH_LEVEL_VIEW.png}
\caption{A high-level view of the proposed framework for combining \revision{process mining}-based feature extraction with dimensionality reduction for control-flow anomaly detection.}
\label{HIGH_LEVEL_VIEW}
\end{figure}

\subsection{Shortcomings of \revision{conformance checking}-based techniques}
Unfortunately, the detection effectiveness of \revision{conformance checking}-based techniques is affected by noisy data and low-quality Petri nets, which may be due to human errors in the modeling process or representational bias of process discovery algorithms \cite{bezerra2013adlogspais, pecchia2020applicationfailuresanalysispm, aalst2016pm}. Specifically, on the one hand, noisy data may introduce infrequent and deceptive control-flow relations that may result in inconsistent fitness measures, whereas, on the other hand, checking event logs against a low-quality Petri net could lead to an unreliable distribution of fitness measures. Nonetheless, such Petri nets can still be used as references to obtain insightful information for \revision{process mining}-based feature extraction, supporting the development of competitive and explainable \revision{conformance checking}-based techniques for control-flow anomaly detection despite the problems above. For example, a few works outline that token-based \revision{conformance checking} can be used for \revision{process mining}-based feature extraction to build tabular data and develop effective \revision{conformance checking}-based techniques for control-flow anomaly detection \cite{singh2022lapmsh, debenedictis2023dtadiiot}. However, to the best of our knowledge, the scientific literature lacks a structured proposal for \revision{process mining}-based feature extraction using the state-of-the-art \revision{conformance checking} variant, namely alignment-based \revision{conformance checking}.

\subsection{Contributions}
We propose a novel \revision{process mining}-based feature extraction approach with alignment-based \revision{conformance checking}. This variant aligns the deviating control flow with a reference Petri net; the resulting alignment can be inspected to extract additional statistics such as the number of times a given activity caused mismatches \cite{aalst2022pmhandbook}. We integrate this approach into a flexible and explainable framework for developing techniques for control-flow anomaly detection. The framework combines \revision{process mining}-based feature extraction and dimensionality reduction to handle high-dimensional feature sets, achieve detection effectiveness, and support explainability. Notably, in addition to our proposed \revision{process mining}-based feature extraction approach, the framework allows employing other approaches, enabling a fair comparison of multiple \revision{conformance checking}-based and \revision{conformance checking}-independent techniques for control-flow anomaly detection. Figure \ref{HIGH_LEVEL_VIEW} shows a high-level view of the framework. Business processes are monitored, and event logs obtained from the database of information systems. Subsequently, \revision{process mining}-based feature extraction is applied to these event logs and tabular data input to dimensionality reduction to identify control-flow anomalies. We apply several \revision{conformance checking}-based and \revision{conformance checking}-independent framework techniques to publicly available datasets, simulated data of a case study from railways, and real-world data of a case study from healthcare. We show that the framework techniques implementing our approach outperform the baseline \revision{conformance checking}-based techniques while maintaining the explainable nature of \revision{conformance checking}.

In summary, the contributions of this paper are as follows.
\begin{itemize}
    \item{
        A novel \revision{process mining}-based feature extraction approach to support the development of competitive and explainable \revision{conformance checking}-based techniques for control-flow anomaly detection.
    }
    \item{
        A flexible and explainable framework for developing techniques for control-flow anomaly detection using \revision{process mining}-based feature extraction and dimensionality reduction.
    }
    \item{
        Application to synthetic and real-world datasets of several \revision{conformance checking}-based and \revision{conformance checking}-independent framework techniques, evaluating their detection effectiveness and explainability.
    }
\end{itemize}

The rest of the paper is organized as follows.
\begin{itemize}
    \item Section \ref{sec:related_work} reviews the existing techniques for control-flow anomaly detection, categorizing them into \revision{conformance checking}-based and \revision{conformance checking}-independent techniques.
    \item Section \ref{sec:abccfe} provides the preliminaries of \revision{process mining} to establish the notation used throughout the paper, and delves into the details of the proposed \revision{process mining}-based feature extraction approach with alignment-based \revision{conformance checking}.
    \item Section \ref{sec:framework} describes the framework for developing \revision{conformance checking}-based and \revision{conformance checking}-independent techniques for control-flow anomaly detection that combine \revision{process mining}-based feature extraction and dimensionality reduction.
    \item Section \ref{sec:evaluation} presents the experiments conducted with multiple framework and baseline techniques using data from publicly available datasets and case studies.
    \item Section \ref{sec:conclusions} draws the conclusions and presents future work.
\end{itemize} 
\section{Related Work and Problem Statement} 
Despite the wide popularity of some existing long-context benchmarks, this section reveals and elaborates on the three key limitations: low complexity, detectable noise, and low resource or limited quantity of test examples. These three limitations make it extremely challenging to measure long-context LLMs' progress toward advanced intellectual agents using existing benchmarks. 

\textbf{Low Complexity.} A significant portion of long-context evaluation datasets, including RULER \citep{hsieh2024rulerwhatsrealcontext}, LongBench \citep{bai2024longbenchbilingualmultitaskbenchmark}, LongBench v2 \citep{bai2025longbenchv2deeperunderstanding}, and LOFT \citep{lee2024longcontextlanguagemodelssubsume}, primarily assess retrieval and summarization rather than complex reasoning. As shown in Figure \ref{passiveplusinteractive}, our experiments demonstrate that RAG systems achieve competitive results with Llama3.1-70B-Instruct across these datasets. Notably, RAG outperforms LLMs in retrieval-focused tasks (e.g., RULER, LOFT) and performs comparably in text summarization and QA (most LongBench tasks), as well as structured reasoning problems such as variable tracking (RULER-vt) and code completion (LongBench-repobench-p). RAG methods provide a strong baseline while being substantially more efficient.

\begin{wrapfigure}{r}{0.5\textwidth} 
    \includegraphics[width=0.5\textwidth]{figures/gsm} 
    \caption{(a) presents a conservative estimate for each problem difficulty in GSM-8K 1.3K test set. We evaluate the difficulty of the problems by the number of operations needed to get to the final answer. The op count ranges from 2 to 12, while most are around 3-4. (b) shows the Llama3.1-8B-Instruct performance across different semantics hierarchies, revealing the hidden reasoning difficulty innate in natural language.} 
    \label{gsm8kvariability} 
    % \vspace{-3mm} 
\end{wrapfigure} 

\textbf{Detectable Noise.} Many long-context benchmarks artificially extend short-context tasks by injecting extraneous text that does not contribute to solving the problem, allowing retrieval-based models to filter out noise effectively. In RULER's variable-tracing task with an 8192-token context, Llama3.1-70B-Instruct achieves 100\%, while OnePassRAG and InteractiveRAG reach 82.4\% and 98.4\%, respectively, despite using only a 2048-token retrieval budget. A detailed breakdown in Figure \ref{mediumandhard} (a) reveals that retrievers consistently identify and prioritize relevant information while disregarding injected noise. These findings indicate that existing long-context benchmarks do not adequately justify the need for expensive long-context LLMs, as RAG systems can effectively mitigate the impact of noise and achieve similar performance. 

\textbf{Low Resource.} Many high-quality reasoning tasks (Math and coding) heavily rely on human annotation and labeling and have test examples in limited quantity. Here we use GSM-8K \citep{cobbe2021trainingverifierssolvemath} as an example, shown in \cref{gsm8kvariability}(a). It is infeasible to extract subsets of examples with exact op at 8 for precise LLM evaluation due to the limited number of available cases—only 26 in total, with even fewer satisfying op $\geq$ 8. This scarcity makes meaningful evaluation impractical. Also, DafnyBench \citep{loughridge2024dafnybenchbenchmarkformalsoftware}, a high-quality long-context coding benchmark only contains 782 verified and deduped examples. 

\textbf{Problem Statement -} \textit{How can we develop a benchmark that contains sufficient problems at every fine-grained level of reasoning difficulty, from easy retrieval tasks to infinitely hard challenges, while providing infinitely customizable context length with high information density?} 
\label{lcllm} 

\begin{figure*}
    \includegraphics[width=\textwidth]{figures/ourlonger} 
    \caption{RAG performance on our proposed long-context benchmarks. (a) studies retriever's behavior on the first 100 chunks of a random problem in vt from RULER with 8192 context length. The chunks that need to be retrieved to solve the problem are labeled in coral, while the noise is in blue. The chunks have retriever scores ranked from large (semantically far) to small (semantically close). Retriever locates the essential chunks with high precision, classifying all necessary chunks with the right side of the spectrum; (b) contrasts vt with our long-context benchmarks, showing that the retriever cannot locate precisely which chunk to retrieve. (c) and (d) display the performance of two RAG systems on our benchmark medium and hard tasks. (Figure best viewed in color)} 
    \label{mediumandhard} 
\end{figure*} 
 
% \input{insights_and_observation} 
\section{Computational Graphs} 

\begin{figure*}
    \includegraphics[width=\textwidth]{figures/computationalgraph} 
    \caption{Computational Graphs Illustration. (a) shows a simple computational graph, where every node and edge can be straightforwardly converted to a statement, the other way conversion works as well. (b) breaks down the essence of implicit operations and the abstract parameter constructions; here, an assumption provided in the system prompt is to assume the animals not mentioned in question don't exist; essentially, problems in natural language omit the two red arrows in the computational graph. (c) provides an example problem when all variables that appeared are ``two-entity" variables, where only implicit addition/subtraction can be generated. (d) contrasts (c) and shows an example that with additional ``three-entity" variables, the computation graph can also generate implicit multiplication/division. Both (c) and (d) also illustrate the design of reverse mode that specifically aims to generate implicit subtraction and division. (Figure best viewed in color)} 
    \label{mediumhard} 
\end{figure*} 

In this section, we detailed the construction of computational graphs, the key construct that enables \sysb generation of infinite quantities of arbitrary context length and reasoning complexity. Specifically, we explain in detail the potential mapping between reasoning problems and computational graphs in \Cref{sec:graphtoreasoning}, and in \Cref{sec:noisecontruction}, we propose to generate indistinguishable noise by strategically extending the computational graph. 

\subsection{Graph Construction to Build Reasoning Problems} 
\label{sec:graphtoreasoning}
After carefully studying GSM-8K problems, we draw the following crucial observations, which allow us to map a randomly generated computational graph to grade-school-level math reasoning problems that cover all possible operations and relationship types. 

\textbf{Mapping Explicit Ops to Computational Graphs - } From Every operation used in the GSM-8K is one of the four ``+”, ``$-$”, ``×”, and ``÷”. Consider the following example when operations are presented explicitly, ``Eggs cost twice as much as tomatoes, while tomatoes cost 1 dollar each.” These statements mention operations (``plus”, ``more”, ``times", etc.) can easily be abstracted out as a computational graph with variables, 
``dollar per egg” and ``dollar per tomato”, as nodes. There are two edges one pointing from "dollar per tomato" to "dollar per egg", while another one from a constant 2 to "dollar per tomato". Another similar example is shown in \cref{mediumhard} (a). 
Therefore, randomly generating a computational graph with different topology of edge connections will
lead to a new reasoning problem once the natural language
context are attached to the nodes of the graph. 

\textbf{Generating Implicit + using Computational Graphs - } On the other hand, the operations can also be presented implicitly hidden in natural language hierarchies. ``Mary earns 20 dollars in the morning, while she earns 25 dollars in the afternoon. How much total she earned that day?” Although the problem doesn’t explicitly mention addition, the solution has to sum up 20 and 25 to get 45. The reason is that natural language assumes a working day consists of morning and afternoon. Similarly, all four operations can be hidden in natural language hierarchies. Inspired by \citet{ye2024physicslanguagemodels21},
we adopt its construct of ``Abstract Parameters” and ``Instance Parameters” to construct computational graphs that facilitate the generation of problem statements containing the hidden operations. Essentially, the newly added constructs can be thought of as adding the ``total money” as a new node to the computational graph, which has two edges coming in, one from node ”Morning money” and the other one from node ``Afternoon money”. But when generating the problem, we omit the description of two edges pointing to the node ``total money on Friday”. We also illustrate it in \cref{mediumhard}(b). Following \cite{ye2024physicslanguagemodels21}, we connect the specific instance parameters to the abstract parameters using red edges. To reiterate, these edges are omitted, forcing the LLMs to use commonsense and inducing implicit operations. 

\textbf{Generating Implicit $\times$ using Computational Graphs -} every variables in the above-mentioned problem with the implicit ``+” operation are ”two-entity variables”. ``Morning Money” contains two entities, ``Morning” and ``Money”, where in the context,
``Money” is an attribute of ``Morning”. Same with ``After-
noon Money”. In fact, out of all the examples we manually examined in GSM-8K, the minimum number of entities in the variable name is two. However, if every variable in the problem are two-entity" variables, only generate implicit operations of + and - will be induced, but not $\times$ and $\div$. For a problem to contain implicit operations $\times$, problems must contain variables with more than two entities in its name. For example, ``Mary works 8 hours on Friday. Her hourly rate on
Friday is 10 dollars. How much she will earn on Friday
in total?” The variable ”money per hour” contains ``money”, ``hour”, and ``Friday” three entities, where
``money” is an attribute of ``hour”, while ``hour” is also an attribute of ``Friday”. We further contrast the above scenario in \cref{mediumhard}(c) and (d), we see that if every variable is two-entity or ``Animal" in ``Location", generating $\times$ isn't straightforward, but once we add in the third entity, or an attribute of animal, generation of $\times$ becomes natural. 

\textbf{Scaling up Reasoning Complexity and maintaining control with Computational Graph -} Our computational graph generator employs the abstract parameter construct to generate implicit operations and three-entity variables to represent multiplication operations. 
% To maximize diversity in reasoning paths, we impose minimal restrictions on graph generation. 
During the generation process of a problem, specifically, a query node is first sampled from the graph. The corresponding topological sort list—ending with the query—ensures the shortest solution path, serving as a measure of the problem's reasoning complexity. 
This approach enables the generation of a vast number of synthetic graphs. Furthermore, adjusting the number of variables provides a coarse control over complexity, which is refined through precise filtering to produce well-defined subsets of graphs that meet specific operation constraints.

\subsection{Noise Construction Using Computational Graphs} 
\label{sec:noisecontruction} 

\begin{wrapfigure}{r}{0.35\textwidth} 
    \includegraphics[width=0.35\textwidth]{figures/noisy.pdf} 
    \caption{The Illustration of Noise as Extension of Core Computational Graph.} 
    \label{noisytwo} 
\end{wrapfigure} 

\textbf{Spider Topology -} We observe that we can view noise as extending the computational graph to incorporate fake and unnecessary parameters and operators. However, two critical questions emerge. First, how to extend the computational graph without contaminating the original graph's solution and contaminations? We found that edges have to point outwards from the nodes in the original graph to the newly added noise nodes, essentially preventing the noise nodes from contributing to the core graph. Second, how to maximize the chance that RAG cannot retrieve the essential graph? It turns out interconnecting edges between newly added noise nodes won't contribute help detering RAG's retriever. We find out a simple trick works well: ensuring the majority of the added edges connect core nodes and the noise nodes contribute to a semantically close noise. We call this design Spider Topology as shown in Figure \ref{mediumhard}(c). 

We evaluated the resulting noise using two RAG systems presented in \ref{lcllm}. The results are shown in Figures \ref{mediumandhard}(c) and (d). Llama3.1-70B-Instruct achieves drastically stronger performance than the two RAG systems on 8K 2-entity problems and 3-entity problems. We also carried out the same study before on our data set setting 2, in Figure \ref{mediumandhard}(b). \textbf{We found that the RAG retriever now completely cannot distinguish which essential chunks from noise chunks, showing a clear contrast with vt tasks of the same context length in (a).}  
 
\section{\sysb} 
\label{gsminfinite} 

In this section, we present key techniques that enable the synthetic dataset to be diverse in operations in \cref{reverseproblem}, LLM-understandable, and enable the evaluation to be free from non-reasoning factors in \cref{attachlanguage}. Then, we present synthetic problem generators capable of generating grade-school math questions with arbitrary reasoning difficulty and context length. Thus, we generate a suite of benchmarks called \sysb detailed in \cref{fullbenchmark}. 

\subsection{Challenge 1: How to Generate Implicit $-$ and $\div$ Operations?} 
\label{reverseproblem} 
\textbf{Firstly, we review why the abstract-instance construct can only generate ``+" and $\times$ but not ``$-$" and $\div$.} 
The key limitation of \citet{ye2024physicslanguagemodels21} abstract parameters and instance parameter design is that it is only able to generate problems with solutions with the ``forward" and constructive ordering. Shown in Figure \ref{mediumhard} (a) and (b), the design dictates that the specific and detailed variables should be defined before a more abstract variable. For example, ``the number of Lions in Zoo" and ``the number of Monkeys in Zoo" have to be defined before ``Total Animal in Zoo" is defined. 
The ``forward" ordering leads to the inability to generate implicit '-' operations for 2-entity problems and implicit ``$\div$" operations for 3-entity problems that require the more abstract variables, e.g. ``Total Animal in Zoo", to be defined before a more specific variable, e.g. ``the number of Monkeys in Zoo". 

\textbf{To generate all four kinds of implicit operations, we introduce a ``reverse mode" to generate the computation graph.} 
Essentially, the graph construction still continues as before: starting with specific detailed variables and growing to incorporate more abstract variables. When it completes and we know all the values of nodes in the graph, we then randomly mask out specific initial low-level variables and force the solution to traverse in the reverse direction as in the "forward" ordering. 
We present the illustration of data generation in Figures \ref{mediumhard} (a) and (b) for the 2-entity and 3-entity, respectively. 
% Now, we have a problem generator capable of generating all the possible operators appearing in GSM-8K. 
However, for 3-entity problems, it can result in quadratic equations leading to multiple possible solutions. We develop some techniques that effectively reduce the probability of the situation. Details are presented in \cref{sometips} of the Appendix. 
\label{reverse_problem} 

\subsection{Challenge 2: How to Ensure LLM-understandable Problem Generation?} 
\label{attachlanguage} 
\begin{wrapfigure}{r}{0.35\textwidth} 
    \includegraphics[width=0.35\textwidth]{figures/illustrationofarea} 
    \caption{Area Under Curve Metrics is Used to Compare between LLM Performance.} 
    \label{areacurve} 
\end{wrapfigure} 
Mapping computation graphs to natural language is critical for evaluating LLMs' reasoning capabilities. \textbf{To automate this process, we develop inter-swappable templates that enhance linguistic diversity while maintaining clarity.} Several key considerations inform our design. 

First, Certain syntactic forms, such as possessive constructions (e.g., A’s B), are straightforward to encode but can mislead LLMs due to their deviation from natural language. For example, South Zoo’s Penguin is restructured as Penguin in South Zoo, and South Zoo’s Adult Penguin’s Average Number of Newborn Children becomes Average Number of Newborn Children per Adult Penguin in South Zoo. Through extended trial and error, variable names with different entity numbers are \textbf{presented naturally} in all prepared templates. 

Second, the number of constraints applied to the random graph generation process is as little as possible, forcing templates to \textbf{enforce unit consistency} across two-entity and three-entity variables to enable assignment between these two. For instance, ``The average number of animal children per penguin in South Zoo" must share a unit with ``The number of penguins in South Zoo" to allow variable assignments.

Third, to ensure real-world knowledge doesn't confuse the LLM's decision, \textbf{we avoid specific real-world locations, people's names, and festival names} from appearing in the template. 
Based on these constricts, we propose three different templates that meet real-world templates: children-animal-zoo, teachers-school-district, and awards-movies-festival. We present in Appendix \ref{templates} an ablation study showing that three templates are consistent in overall performance with only minor fluctuations when evaluated using Llama-3.1-8B-Instruct. At each op, equal problems are tested for both constructive ordering (forward) and reverse ordering (reverse). 

\subsection{Benchmark Details} 
\label{fullbenchmark} 
\begin{table*}[ht] 
\centering 
\footnotesize 
% \caption{Performance Metrics for Various Models Across Subtasks and Detailed Statistics on Hard Subtask} 
\caption{18 selected models are evaluated on \sysb zero-noise benchmarks using Area-Under-Curve (AUC), which is computed by taking the Riemann Sum of accuracy versus op count from 2 to when the model accuracy drops below 5\%. We also present detailed statics of the first op number for the model to have an accuracy lower than 50\%, 10\%, and the average accuracy of the first 30 ops settings. Besides, we also highlight the \textcolor{yellow}{reasoning models}, \textcolor{green}{linear attention hybrid models}, and \textcolor{blue}{SSM hybrid models}. Due to space constraint, ``Mistral-Large-Instruct-2411" is shortened as ``Mistral-Large";``Claude-3.5-Sonnet" and ``Claude-3.5-Haiku" has version number 20241022; ``GPT-4o-2024-11-20" is shortened as ``GPT-4o" and ``GPT-4o-mini-2024-07-18" is shortened as ``GPT-4o-mini".} 
\begin{tabular}{@{}lccc|ccc|c@{}}
\toprule
\multirow{2}{*}{\textbf{Models}} & \multicolumn{3}{c|}{\textbf{Three Subtasks}} & \multicolumn{3}{c|}{\textbf{Detailed Statistics on Hard Subtask}} & \textbf{Score} \\
\cmidrule(lr){2-4} \cmidrule(lr){5-7}
 & \textbf{Symbolic} & \textbf{Medium} & \textbf{Hard} & \textbf{1st$<$50\% op} & \textbf{1st$<$10\% op} & \textbf{Avg. Acc op$\leq$30} & \textbf{Avg.$\uparrow$} \\
\midrule
\rowcolor{yellow!20} 
DeepSeek-R1 & 7280.0 & 9750.85 & 8573.8 & 100 & $>$130 & 0.9427 & 8534.88 \\ 
\rowcolor{yellow!20} 
GPT-o3-mini & 6690.0 & 8335.66 & 5769.96 & 70 & 110 & 0.9423 & 6931.88 \\
\rowcolor{yellow!20} 
GPT-o1-mini & 5060.0 & 6054.91 & 3738.43 & 50 & 90 & 0.8397 & 4951.11 \\
DeepSeek-V3 & 4310.0 & 4100.81 & 2407.86 & 24 & 55 & 0.6669 & 3606.22 \\
\rowcolor{yellow!20} 
QwQ-32B-preview & 3530.0 & 3205.75 & 1846.19 & 21 & 50 & 0.5403 & 2860.65 \\
Gemini-1.5-Pro-002 & 2547.0 & 3659.59 & 2318.28 & 26 & 45 & 0.6924 & 2841.62 \\
Claude-3.5-Sonnet & 2161.0 & 3281.8 & 2115.79 & 26 & 40 & 0.6758 & 2519.53 \\
Mistral-Large & 2332.5 & 2879.92 & 2310.49 & 24 & 50 & 0.6645 & 2507.64 \\
Qwen2.5-72B-Instruct & 2048.0 & 2496.81 & 2016.38 & 21 & 40 & 0.5433 & 2187.06 \\
GPT-4o & 2379.0 & 2457.37 & 1451.54 & 18 & 30 & 0.5064 & 2095.97 \\
Gemini-1.5-Flash-002 & 1970.0 & 1478.75 & 1274.25 & 13 & 30 & 0.4460 & 1574.33 \\
Llama3.1-70B-Instruct & 1769.0 & 1650.25 & 1205.25 & 15 & 30 & 0.4314 & 1541.50 \\
\rowcolor{palegreen!20} 
MiniMax-Text-01 & 1618.5 & 1712.64 & 1178.51 & 14 & 30 & 0.4213 & 1503.22 \\
% llama-3.1-405b-instruct & 1557.0 & 1321.54 & 950.0 & 11 & 20 & 0.3409 & 1276.18 \\ 
GPT-4o-mini & 1389.0 & 1406.5 & 913.89 & 12 & 22 & 0.3094 & 1236.46 \\
Claude-3.5-Haiku & 897.0 & 1053.16 & 784.34 & 10 & 22 & 0.2910 & 911.50 \\
Qwen2.5-7B-Instruct & 786.95 & 886.75 & 618.5 & 7 & 19 & 0.2257 & 764.07 \\
Llama3.1-8B-Instruct & 462.0 & 786.5 & 606.5 & 6 & 17 & 0.2186 & 618.30 \\
\rowcolor{skyblue!20} 
Jamba-1.5-Large & 856.0 & 485.13 & 466.4 & 6 & 26 & 0.1828 & 602.51 \\
\bottomrule
\end{tabular} 
\label{zerocontextleaderb} 
\end{table*} 

With the synthetic problem generators detailed in Section \ref{gsminfinite}, we then use them to generate problems to build a suite of reasoning tasks with increasing complexity. For the brevity of reference, we refer to the generated problems with only explicit operations as ``Easy", the generated problems with 2-entity variables at maximum as ``Medium", and the generated problems with 3-entity variables at maximum as "Hard. 

Ideally, when evaluating an LLM, we want to evaluate all difficulty levels, from the most basic logic complexity to when it completely fails to solve any problem. For the Easy subset of problems, it usually leads to large operation counts for powerful LLMs. 
However, although complexity-wise not challenging, LLMs trained with internal COT tend to generate very long arguments, saturating their API output generation limit (4K for many models). Thus, we observe a sudden decay in accuracy in large ops, not because of LLMs' ability bottlenecks, but because of the above-mentioned nuance. Thus, we make a tweak to its problem: Instead of asking the LLM to find the value of one variable, we ask the LLM to find all the variables that have some value specified, effectively increasing the difficulty of the problem. 

\begin{wraptable}{r}{0.5\textwidth} 
\centering 
\caption{10 selected models are evaluated on \sysb Long Context benchmarks using Average AUC of Symbolic, Medium, and Hard. We evaluated models on 8K, 16K, and 32K context. Although our pipeline is capable of generating longer problems, the resource required to go further for larger models beyond our acceptance, while smaller models effectively has completely failed.} 
\footnotesize 
\begin{tabular}{@{}lcccc@{}}
\toprule
\textbf{Model} & \textbf{8K} & \textbf{16K} & \textbf{32K} & \textbf{Avg.$\uparrow$} \\
\midrule
Gemini-1.5-Pro-002 & 1182.43 & 896.31 & 812.96 & 963.9 \\
Qwen2.5-72B-Instruct & 927.33 & 681.53 & 563.65 & 724.17 \\
Mistral-Large & 914.49 & 563.73 & 319.21 & 599.14 \\
DeepSeek-V3 & 935.10 & 477.02 & 313.66 & 575.2 \\
Gemini-1.5-Flash-002 & 673.88  & 476.72 & 377.38 & 509.3 \\
Llama3.1-70B-Instruct & 479.00 & 394.50 & 355.5 & 409.67 \\
MiniMax-Text-01 & 481.32  & 359.56 & 325.95 & 388.94 \\
GPT-4o-mini & 401.00  & 337.81 & 275.63 & 338.15 \\
Qwen2.5-7B-Instruct & 248.00  & 211.50 & 196.17 & 218.56 \\
Llama3.1-8B-Instruct & 183.67  & 149.50 & 109.45 & 147.54 \\

\bottomrule
\end{tabular} 
\label{longcontextleaderb} 
\vspace{-5mm} 
\end{wraptable} 

For the modified Easy subset, we keep the generated problem in the most basic form: symbolic assignment. The typical problem statement then becomes "v1235 equals v1468 plus 1." Since the modified problem is not easier compared to Medium and Hard, we now call it "Symbolic". For Medium and Hard, we used all three templates and mixed the generated problems together to ensure diversity. For reporting LLM performance, we use \textbf{Area Under Curve (AUC)}, as shown in \cref{areacurve}, which is computing a Riemann sum over the LLM's performance in accuracy versus the number of operations from 2 to when its performance is lower than 5\%. 

We prepare zero-noise, 8K, 16K, and 32K in the benchmarks. The existing generation pipeline is capable of generating in $>$ 16M context, but the smaller 70B level models effectively failed in the 32K context already, while evaluating larger ones brings cost beyond our acceptance.
 

 
\section{Analysis}
\label{sec:analysis}
In the following sections, we will analyze European type approval regulation\footnote{Strictly speaking, the German enabling act (AFGBV) does not regulate type-approval, but how test \& operating permits are issued for SAE-Level-4 systems. Type-approval regulation for SAE-Level-3 systems follows UN Regulation No. 157 (UN-ECE-ALKS) \parencite{un157}.} regarding the underlying notions of ``safety'' and ``risk''.
We will classify these notions according to their absolute or relative character, underlying risk sources, or underlying concepts of harm.

\subsection{Classification of Safety Notions}
\label{sec:safety-notions}
We will refer to \emph{absolute} notions of safety as conceptualizations that assume the complete absence of any kind of risk.
Opposed to this, \emph{relative} notions of safety are based on a conceptualization that specifically includes risk acceptance criteria, e.g., in terms of ``tolerable'' risk or ``sufficient'' safety.

For classifying notions of safety by their underlying risk (or rather ``hazard'') sources, and different concepts of harm, \Cref{fig:hazard-sources} provides an overview of our reasoning, which is closely in line with the argumentation provided by Waymo in \parencite{favaro2023}.
We prefer ``hazard sources'' over ``risk sources'', as a risk must always be related to a \emph{cause} or \emph{source of harm} (i.e., a hazard \parencite[p.~1, def. 3.2]{iso51}).
Without a concrete (scenario) context that the system is operating in, a hazard is \emph{latent}: E.g., when operating in public traffic, there is a fundamental possibility that a \emph{collision with a pedestrian} leads to (physical) harm for that pedestrian. 
However, only if an automated vehicle shows (potentially) hazardous behavior (e.g., not decelerating properly) \emph{and} is located near a pedestrian (context), the hazard is instantiated and leads to a hazardous event.
\begin{figure*}
    \includeimg[width=.9\textwidth]{hazard-sources0.pdf}
    \caption{Graphical summary of a taxonomy of risk related to automated vehicles, extended based on ISO 21448 (\parencite{iso21448}) and \parencite{favaro2023}. Top: Causal chain from hazard sources to actual harm; bottom: summary of the individual elements' contributions to a resulting risk. Graphic translated from \parencite{nolte2024} \label{fig:hazard-sources}}
\end{figure*}
If the hazardous event cannot be mitigated or controlled, we see a loss event in which the pedestrian's health is harmed.
Note that this hypothetical chain of events is summarized in the definition of risk:
The probability of occurrence of harm is determined by a) the frequency with which hazard sources manifest, b) the time for which the system operates in a context that exposes the possibility of harm, and c) by the probability with which a hazardous event can be controlled.
A risk can then be determined as a function of the probability of harm and the severity of the harm potentially inflicted on the pedestrian.

In the following, we will apply this general model to introduce different types of hazard sources and also different types of harm.
\cref{fig:hazard-sources} shows two distinct hazard sources, i.e., functional insufficiencies and E/E-failures that can lead to hazardous behavior.
ISO~21488 \parencite{iso21448} defines functional insufficiencies as insufficiencies that stem from an incomplete or faulty system specification (specification insufficiencies).
In addition, the standard considers insufficiencies that stem from insufficient technical capability to operate inside the targeted Operational Design Domain (performance insufficiencies).
Functional insufficiencies are related to the ``Safety of the Intended Functionality (SOTIF)'' (according to ISO~21448), ``Behavioral Safety'' (according to Waymo \parencite{waymo2018}), or ``Operational Safety'' (according to UN Regulation No. 157 \parencite{un157}).
E/E-Failures are related to classic functional safety and are covered exhaustively by ISO~26262 \parencite{iso2018}.
Additional hazard sources can, e.g., be related to malicious security attacks (ISO~21434), or even to mechanical failures that should be covered (in the US) in the Federal Motor Vehicle Safety Standards (FMVSS).

For the classification of notions of safety by the related harm, in \parencite{salem2024, nolte2024}, we take a different approach compared to \parencite{koopman2024}:
We extend the concept of harm to the violation of stakeholder \emph{values}, where values are considered to be a ``standard of varying importance among other such standards that, when combined, form a value pattern that reduces complexity for stakeholders [\ldots] [and] determines situational actions [\ldots].'' \parencite{albert2008}
In this sense, values are profound, personal determinants for individual or collective behavior.
The notion of values being organized in a weighted value pattern shows that values can be ranked according to importance.
For automated vehicles, \emph{physical wellbeing} and \emph{mobility} can, e.g., be considered values which need to be balanced to achieve societal acceptance, in line with the discussion of required tradeoffs in \cref{sec:terminology}.
For the analysis of the following regulatory frameworks, we will evaluate if the given safety or risk notions allow tradeoffs regarding underlying stakeholder values. 

\subsection{UN Regulation No. 157 \& European Implementing Regulation (EU) 2022/1426}
\label{sec:enabling-act}
UN Regulation No. 157 \parencite{un157} and the European Implementing Regulation 2022/1426 \parencite{eu1426} provide type approval regulation for automated vehicles equipped with SAE-Level-3 (UN Reg. 157) and Level 4 (EU 2022/1426) systems on an international (UN Reg. 157) and European (EU 2022/1426) level.

Generally, EU type approval considers UN ECE regulations mandatory for its member states ((EU) 2018/858, \parencite{eu858}), while the EU largely forgoes implementing EU-specific type approval rules, it maintains the right to alter or to amend UN ECE regulation \parencite{eu858}.

In this respect, the terminology and conceptualizations in the EU Implementing Act closely follow those in UN Reg. No. 157.
The EU Implementing Act gives a clear reference to UN Reg. No. 157 \parencite[][Preamble,  Paragraph 1]{eu1426}.
Hence, the documents can be assessed in parallel.
Differences will be pointed out as necessary.

Both acts are written in rather technical language, including the formulation of technical requirements (e.g., regarding deceleration values or speeds in certain scenarios).
While providing exhaustive definitions and terminology, neither of both documents provide an actual definition of risk or safety.
The definition of ``unreasonable'' risk in both documents does not define risk, but only what is considered \emph{unreasonable}. It states that the ``overall level of risk for [the driver, (only in UN Reg. 157)] vehicle occupants and other road users which is increased compared to a competently and carefully driven manual vehicle.''
The pertaining notions of safety and risk can hence only be derived from the context in which they are used.

\subsubsection{Absolute vs. Relative Notions of Safety}
In line with the technical detail provided in the acts, both clearly imply a \emph{relative} notion of safety and refer to the absence of \emph{unreasonable} risk throughout, which is typical for technical safety definitions.

Both acts require sufficient proof and documentation that the to-be-approved automated driving systems are ``free of unreasonable safety risks to vehicle occupants and other road users'' for type approval.\footnote{As it targets SAE-Level-3 systems, UN Reg. 157 also refers to the driver, where applicable.}
In this respect, both acts demand that the manufacturers perform verification and validation activities for performance requirements that include ``[\ldots] the conclusion that the system is designed in such a way that it is free from unreasonable risks [\ldots]''.
Additionally, \emph{risk minimization} is a recurring theme when it comes to the definition of Minimum Risk Maneuvers (MRM).

Finally, supporting the relative notions of safety and risk, UN Reg. 157 introduces the concept of ``reasonable foreseeable and preventable'' \parencite[Article 1, Clause 5.1.1.]{un157} collisions, which implies that a residual risk will remain with the introduction of automated vehicles.
\parencite[][Appendix 3, Clause 3.1.]{un157} explicitly states that only \emph{some} scenarios that are unpreventable for a competent human driver can actually be prevented by an automated driving system.
While this concept is not applied throughout the EU Implementing Act, both documents explicitly refer to \emph{residual} risks that are related to the operation of automated driving systems (\parencite[][Annex I, Clause 1]{un157}, \parencite[][Annex II, Clause 7.1.1.]{eu1426}).

\subsubsection{Hazard Sources}
Hazard sources that are explicitly differentiated in UN Reg. 157 and (EU) 2022/1426 are E/E-failures that are in scope of functional safety (ISO~26262) and functional insufficiencies that are in scope of behavioral (or ``operational'') safety (ISO~21448).
Both documents consistently differentiate both sources when formulating requirements.

While the acts share a common definition of ``operational'' safety (\parencite[][Article 2, def. 30.]{eu1426}, \parencite[][Annex 4, def. 2.15.]{un157}), the definitions for functional safety differ.
\parencite{un157} defines functional safety as the ``absence of unreasonable risk under the occurrence of hazards caused by a malfunctioning behaviour of electric/electronic systems [\ldots]'', \parencite{eu1426} drops the specification of ``electric/electronic systems'' from the definition.
When taken at face value, this definition would mean that functional safety included all possible hazard sources, regardless of their origin, which is a deviation from the otherwise precise usage of safety-related terminology.

\subsubsection{Harm Types}
As the acts lack explicit definitions of safety and risk, there is no consistent and explicit notion of different harm types that could be differentiated.

\parencite{un157} gives little hints regarding different considered harm types.
``The absence of unreasonable risk'' in terms of human driving performance could hence be related to any chosen performance metric that allows a comparison with a competent careful human driver including, e.g., accident statistics, statistics about rule violations, or changes in traffic flow.

In \parencite{eu1426}, ``safety'' is, implicitly, attributed to the absence of unreasonable risk to life and limb of humans.
This is supported by the performance requirements that are formulated:
\parencite[][Annex II, Clause 1.1.2. (d)]{eu1426} demands that an automated driving system can adapt the vehicle behavior in a way that it minimizes risk and prioritizes the protection of human life.

Both acts demand the adherence to traffic rules (\parencite[][Annex 2, Clause 1.3.]{eu1426}, \parencite[][Clause 5.1.2.]{un157}).
\parencite[][Annex II, Clause 1.1.2. (c)]{eu1426} also demands that an automated driving system shall adapt its behavior to surrounding traffic conditions, such as the current traffic flow.
With the relative notion of risk in both acts, the unspecific clear statement that there may be unpreventable accidents \parencite{un157}, and a demand of prioritization of human life in \parencite{eu1426}, both acts could be interpreted to allow developers to make tradeoffs as discussed in \cref{sec:terminology}.


\subsubsection{Conclusion}
To summarize, the UN Reg. 157 and the (EU) 2022/1426 both clearly support the technical notion of safety as the absence of unreasonable risk.
The notion is used consistently throughout both documents, providing a sufficiently clear terminology for the developers of automated vehicles.
Uncertainty remains when it comes to considered harm types: Both acts do not explicitly allow for broader notions of safety, in the sense of \parencite{koopman2024} or \parencite{salem2024}.
Finally, a minor weak spot can be seen in the definition of risk acceptance criteria: Both acts take the human driving performance as a baseline.
While (EU) 2022/1426 specifies that these criteria are specific to the systems' Operational Design Domain \parencite[][Annex II, Clause 7.1.1.]{eu1426}, the reference to the concrete Operational Design Domain is missing in UN Reg. 157.
Without a clearly defined notion of safety, however, it remains unclear, how aspects beyond net accident statistics (which are given as an example in \parencite[][Annex II, Clause 7.1.1.]{eu1426}), can be addressed practically, as demanded by \parencite{koopman2024}.

\subsection{German Regulation (StVG \& AFGBV)}
\label{sec:afgbv}
The German L3 (Automated Driving Act) and L4 (Act on Autonomous Driving) Acts from 2017 and 2021,\footnote{Formally, these are amendments to the German Road Traffic Act (StVG): 06/21/2017, BGBl. I p. 1648, 07/12/2021 BGBl. I p. 3108.} respectively, provide enabling regulation for the operation of SAE-Level-3 and 4 vehicles on German roads.
The German Implementing Regulation (\parencite{afgbv}, AFGBV) defines how this enabling regulation is to be implemented for granting testing permits for SAE-Level-3 and -4 and driving permits for SAE-Level-3 and -4 automated driving systems.\footnote{Note that these permits do not grant EU-wide type approval, but serve as a special solution for German roads only. At the same time, the AFGBV has the same scope as (EU) 2022/1426.}
With all three acts, Germany was the first country to regulate the approval of automated vehicles for a domestic market.
All acts are subject to (repeated) evaluation until the year 2030 regarding their impact on the development of automated driving technology.
An assessment of the German AFGBV and comparisons to (EU) 2022/1426 have been given in \cite{steininger2022} in German.

Just as for UN Reg. 157 and (EU) 2022/1426, neither the StVG nor the AFGBV provide a clear definition of ``safety'' or ``risk'' -- even though the "safety" of the road traffic is one major goal of the StVG and StVO.
Again, different implicit notions of both concepts can only be interpreted from the context of existing wording.
An additional complication that is related to the German language is that ``safety'' and ``security'' can both be addressed as ``Sicherheit'', adding another potential source of unclarity.
Literal Quotations in this section are our translations from the German act.

\subsubsection{Absolute vs. Relative Notions of Safety}
For assessing absolute vs. relative notions of safety in German regulation, it should be mentioned that the main goal of the German StVO is to ensure the ``safety and ease of traffic flow'' -- an already diametral goal that requires human drivers to make tradeoffs.\footnote{For human drivers, this also creates legal uncertainty which can sometimes only be settled in a-posteriori court cases.}
While UN and EU regulation clearly shows a relative notion of safety\footnote{And even the StVG contains sections that use wording such as ``best possible safety for vehicle occupants'' (§1d (4) StVG) and acknowledges that there are unavoidable hazards to human life (§1e (2) No. 2c)).}, the German AFGBV contains ambiguous statements in this respect:
Several paragraphs contain a demand for a hazard free operation of automated vehicles.
§4 (1) No. 4 AFGBV, e.g., states that ``the operation of vehicles with autonomous driving functions must neither negatively impact road traffic safety or traffic flow, nor endanger the life and limb of persons.''
Additionally, §6 (1) AFGBV states that the permits for testing and operation have to be revoked, if it becomes apparent that a ``negative impact on road traffic safety or traffic flow, or hazards to the life and limb of persons cannot be ruled out''.
The same wording is used for the approval of operational design domains regulated in §10 (1) No. 1.
A particularly misleading statement is made regarding the requirements for technical supervision instances which are regulated in §14 (3) AFGBV which states that an automated vehicle has to be  ``immediately removed from the public traffic space if a risk minimal state leads to hazards to road traffic safety or traffic flow''.
Considering the argumentation in \cref{sec:terminology}, that residual risks related to the operation of automated driving systems are inevitable, these are strong statements which, if taken at face value, technically prohibit the operation of automated vehicles.
It suggests an \emph{absolute} notion of safety that requires the complete absence of risk.  
The last statement above is particularly contradictory in itself, considering that a risk \emph{minimal} state always implies a residual risk.

In addition to these absolute safety notions, there are passages which suggest a relative notion of safety:
The approval for Operational Design Domains is coupled to the proof that the operation of an automated vehicle ``neither negatively impacts road traffic safety or traffic flow, nor significantly endangers the life and limb of persons beyond the general risk of an impact that is typical of local road traffic'' (§9 (2) No. 3 AFGBV).
The addition of a relative risk measure ``beyond the general risk of an impact'' provides a relaxation (cf. also \cite{steininger2022}, who criticizes the aforementioned absolute safety notion) that also yields an implicit acceptance criterion (\emph{statistically as good as} human drivers) similar to the requirements stated in UN Reg. 157 and (EU) 2022/1426.

Additional hints for a relative notion of safety can be found in Annex 1, Part 1, No. 1.1 and Annex 1, Part 2, No. 10.
Part 1, No 1.1 specifies collision-avoidance requirements and acknowledges that not all collisions can be avoided.\footnote{The same is true for Part 2, No. 10, Clause 10.2.5.}
Part 2, No. 10 specifies requirements for test cases.
It demands that test cases are suitable to provide evidence that the ``safety of a vehicle with an autonomous driving function is increased compared to the safety of human-driven vehicles''.
This does not only acknowledge residual risks, but also yields an acceptance criterion (\emph{better} than human drivers) that is different from the implied acceptance criterion given in §9 (2) No. 3 AFGBV.

\subsubsection{Hazard Sources}
Regarding hazard sources, Annex 1 and 3 AFGBV explicitly refer to ISO~26262 and ISO~21448 (or rather its predecessor ISO/PAS~21448:2019).
However, regarding the discussion of actual hazard sources, the context in which both standards are mentioned is partially unclear:
Annex 1, Clause 1.3 discusses requirements for path and speed planning.
Clause 1.3 d) demands that in intersections, a Time to Collision (TTC) greater than 3 seconds must be guaranteed.
If manufacturers deviate from this, it is demanded that ``state-of-the-art, systematic safety evaluations'' are performed.
Fulfillment of the state of the art is assumed if ``the guidelines of ISO~26262:2018-12 Road Vehicles -- Functional Safety are fulfilled''.
Technically, ISO~26262 is not suitable to define the state of the art in this context, as the requirements discussed fall in the scope of operational (or behavioral) safety (ISO~21448).
A hazard source ``violated minimal time to collision'' is clearly a functional insufficiency, not an E/E-failure.

Similar unclarity presents itself in Annex 3, Clause 1 AFGBV: 
Clause 1 specifies the contents of the ``functional specification''.
The ``specification of the functionality'' is an artifact which is demanded in ISO~21448:2022 (Clause 5.3) \parencite{iso21448}.
However, Annex 3, Clause 1 AFGBV states that the ``functional specification'' is considered to comply to the state of the art, if the ``functional specification'' adheres to ISO~26262-3:2018 (Concept Phase).
Again, this assumes SOTIF-related contents as part of ISO~26262, which introduces the ``Item Definition'' as an artifact, which is significantly different from the ``specification of the functionality'' which is demanded by ISO~21448.
Finally, Annex 3, Clause 3 AFGBV demands a ``documentation of the safety concept'' which ``allows a functional safety assessment''.
A safety concept that is related to operational / behavioral safety is not demanded.
Technically, the unclarity with respect to the addressed harm types lead to the fact that the requirements provided by the AFGBV do not comply with the state of the art in the field, providing questionable regulation.

\subsubsection{Harm Types}
Just like UN Reg. 157 and (EU) 2022/1426, the German StVG and AFGBV do not explicitly differentiate concrete harm types for their notions of safety.
However, the AFGBV mentions three main concerns for the operation of automated vehicles which are \emph{traffic flow} (e.g., §4 (1) No. 4 AFGBV), compliance to \emph{traffic law} (e.g., §1e (2) No. 2 StVG), and the \emph{life and limb of humans} (e.g., §4 (1) No. 4 AFGBV).

Again, there is some ambiguity in the chosen wording:
The conflict between traffic flow and safety has already been argued in \cref{sec:terminology}.
The wording given in §4 (1) No. 4 and §6 (1) AFGBV  demand to ensure (absolute) safety \emph{and} traffic flow at the same time, which is impossible (cf. \cref{sec:terminology}) from an engineering perspective.
§1e (2) No. 2 StVG defines that ``vehicles with an autonomous driving function must [\ldots] be capable to comply to [\ldots] traffic rules in a self-contained manner''.
Taken at face value, this wording implies that an automated driving system could lose its testing or operating permit as soon as it violates a traffic rule.
A way out could be provided by §1 of the German Traffic Act (StVO) which demands careful and considerate behavior of all traffic participants and by that allows judgement calls for human drivers.
However, if §1 is applicable in certain situations is often settled in court cases. 
For developers, the application of §1 StVO during system design hence remains a legal risk.

While there are rather absolute statements as mentioned above, sections of the AFGBV and StVG can be interpreted to allow tradeoffs:
§1e (2) No. 2 b) demands that a system,  ``in case of an inevitable, alternative harm to legal objectives, considers the significance of the legal objectives, where the protection of human life has highest priority''.
This exact wording \emph{could} provide some slack for the absolute demands in other parts of the acts, enabling tradeoffs between (tolerable) risk and mobility as discussed in \cref{sec:terminology}.
However, it remains unclear if this interpretation is legally possible.

\subsubsection{Conclusion}
Compared to UN Reg. 157 and (EU) 2022/1426, the German StVG and AFGBV introduce openly inconsistent notions of safety and risk which are partially directly contradictory:
The wording partially implies absolute and relative notions of safety and risk at the same time.
The implied validation targets (``better'' or ``as good as'' human drivers) are equally contradictory. 
The partially implied absolute notions of safety, when taken at face value, prohibit engineers from making the tradeoffs required to develop a system that is safe and provides customer benefit at the same time. 
In consequence, the wording in the acts is prone to introducing legal uncertainty.
This uncertainty creates additional clarification need and effort for manufacturers and engineers who design and develop SAE-Level-3 and -4 automated driving systems. The use of undefined legal terms not only makes it more difficult for engineers to comply with the law, but also complicates the interpretation of the law and leads to legal uncertainty.

\subsection{UK Automated Vehicles Act 2024 (2024 c. 10)}
The UK has issued a national enabling act for regulating the approval of automated vehicles on the roads in the UK.
To the best of our knowledge, concrete implementing regulation has not been issued yet.
Regarding terminology, the act begins with a dedicated terminology section to clarify the terms used in the act \parencite[Part 1, Chapter 1, Section 1]{ukav2024}.
In that regard, the act defines a vehicle to drive ```autonomously' if --- (a)
it is being controlled not by an individual but by equipment of the vehicle, and (b) neither the vehicle nor its surroundings are being monitored by an individual with a view to immediate intervention in the driving of the vehicle.''
The act hence covers SAE-Level-3 to SAE-Level-5 automated driving systems.

\subsubsection{Absolute vs. Relative Notions of Safety}
While not providing an explicit definition of safety and risk, the UK Automated Vehicles Act (``UK AV Act'') \parencite{ukav2024} explicitly refers to a relative notion of safety.
Part~1, Chapter~1, Section~1, Clause (7)~(a) defines that an automated vehicle travels ```safely' if it travels to an acceptably safe standard''.
This clarifies that absolute safety is not achievable and that acceptance criteria to prove the acceptability of residual risk are required, even though a concrete safety definition is not given.
The act explicitly tasks the UK Secretary of State\footnote{Which means, that concrete implementation regulation needs to be enacted.} to install safety principles to determine the ``acceptably safe standard'' in Part~1, Chapter~1, Section~1, Clause (7)~(a).
In this respect, the act also provides one general validation target as it demands that the safety principles must ensure that ``authorized automated vehicles will achieve a level of safety equivalent to, or higher than, that of careful and competent human drivers''.
Hence, the top-level validation risk acceptance criterion assumed for UK regulation is ``\emph{at least as good} as human drivers''.

\subsubsection{Hazard Sources}
The UK AV Act contains no statements that could be directly related to different hazard sources.
Note that, in contrast to the rest of the analyzed documents, the UK AV Act is enabling rather than implementing regulation.
It is hence comparable to the German StVG, which does not refer to concrete hazard sources as well.

\subsubsection{Types of Harm}
Even though providing a clear relative safety notion, the missing definition of risk also implies a lack of explicitly differentiable types of harm.
Implicitly, three different types of harm can be derived from the wording in the act.
This includes the harm to life and limb of humans\footnote{Part~1, Chapter~3, Section~25 defines ``aggravated offence where death or serious injury occurs'' \parencite{ukav2024}.}, the violation of traffic rules\footnote{Part~1, Chapter~1, Clause~(7)~(b) defines that an automated vehicle travels ```legally' if it travels with an acceptably low risk of committing a traffic infraction''}, and the cause of inconvenience to the public \parencite[Part~1, Chapter~1, Section~58, Clause (2)~(d)]{ukav2024}.

The act connects all the aforementioned types of harm to ``risk'' or ``acceptable safety''.
While the act generally defines criminal offenses for providing ``false or misleading information about safety'', it also acknowledges possible defenses if it can be proven that ``reasonable precautions'' were taken and that ``due diligence'' was exercised to ``avoid the commission of the offence''.
This statement could enable tradeoffs within the scope of ``reasonable risk'' to the life and limb of humans, the violation of traffic rules, or to the cause of inconvenience to the public, as we argued in \cref{sec:terminology}.

\subsubsection{Conclusion}
From the set of reviewed documents, the current UK AV Act is the one with the most obvious relative notions of safety and risk and the one that seems to provide a legal framework for permitting tradeoffs.
In our review, we did not spot major inconsistency beyond a missing definitions of safety and risk\footnote{Note that with the Office for Product Safety and Standards (OPSS), there is a British government agency that maintains an exhaustive and widely focussed ``Risk Lexicon'' that provides suitable risk definitions. For us, it remains unclear, to what extent this terminology is assumed general knowledge in British legislation.}.
The general, relative notion of safety and the related alleged ability for designers to argue well-founded development tradeoffs within the legal framework could prove beneficial for the actual implementation of automated driving systems.
While the act thus appears as a solid foundation for the market introduction of automated vehicles, without accompanying implementing regulation, it is too early to draw definite conclusions. 
\section{Conclusion}
In this work, we propose a simple yet effective approach, called SMILE, for graph few-shot learning with fewer tasks. Specifically, we introduce a novel dual-level mixup strategy, including within-task and across-task mixup, for enriching the diversity of nodes within each task and the diversity of tasks. Also, we incorporate the degree-based prior information to learn expressive node embeddings. Theoretically, we prove that SMILE effectively enhances the model's generalization performance. Empirically, we conduct extensive experiments on multiple benchmarks and the results suggest that SMILE significantly outperforms other baselines, including both in-domain and cross-domain few-shot settings. 

% Acknowledgements should only appear in the accepted version.
% \section*{Acknowledgements}

% \textbf{Do not} include acknowledgements in the initial version of
% the paper submitted for blind review.

% If a paper is accepted, the final camera-ready version can (and
% usually should) include acknowledgements.  Such acknowledgements
% should be placed at the end of the section, in an unnumbered section
% that does not count towards the paper page limit. Typically, this will 
% include thanks to reviewers who gave useful comments, to colleagues 
% who contributed to the ideas, and to funding agencies and corporate 
% sponsors that provided financial support.

\section*{Impact Statement}
This paper presents work aimed at advancing the field of Machine Learning, specifically in long-context reasoning for large language models. Our benchmark, \textsc{GSM}{\large{$\infty$}}, provides a scalable and controllable testbed for evaluating and improving LLM reasoning capabilities in complex, extended contexts. While this research contributes to the broader development of AI reasoning, we do not foresee any immediate ethical concerns or societal implications beyond standard considerations in AI research, none which we feel must be specifically highlighted here.
% Authors are \textbf{required} to include a statement of the potential 
% broader impact of their work, including its ethical aspects and future 
% societal consequences. This statement should be in an unnumbered 
% section at the end of the paper (co-located with Acknowledgements -- 
% the two may appear in either order, but both must be before References), 
% and does not count toward the paper page limit. In many cases, where 
% the ethical impacts and expected societal implications are those that 
% are well established when advancing the field of Machine Learning, 
% substantial discussion is not required, and a simple statement such 
% as the following will suffice:

% ``This paper presents work whose goal is to advance the field of 
% Machine Learning. There are many potential societal consequences 
% of our work, none which we feel must be specifically highlighted here.''

% The above statement can be used verbatim in such cases, but we 
% encourage authors to think about whether there is content which does 
% warrant further discussion, as this statement will be apparent if the 
% paper is later flagged for ethics review. 


% In the unusual situation where you want a paper to appear in the
% references without citing it in the main text, use \nocite
\nocite{langley00}

\bibliography{example_paper}
\bibliographystyle{icml2025}


%%%%%%%%%%%%%%%%%%%%%%%%%%%%%%%%%%%%%%%%%%%%%%%%%%%%%%%%%%%%%%%%%%%%%%%%%%%%%%%
%%%%%%%%%%%%%%%%%%%%%%%%%%%%%%%%%%%%%%%%%%%%%%%%%%%%%%%%%%%%%%%%%%%%%%%%%%%%%%%
% APPENDIX
%%%%%%%%%%%%%%%%%%%%%%%%%%%%%%%%%%%%%%%%%%%%%%%%%%%%%%%%%%%%%%%%%%%%%%%%%%%%%%%
%%%%%%%%%%%%%%%%%%%%%%%%%%%%%%%%%%%%%%%%%%%%%%%%%%%%%%%%%%%%%%%%%%%%%%%%%%%%%%%
\newpage
\subsection{Lloyd-Max Algorithm}
\label{subsec:Lloyd-Max}
For a given quantization bitwidth $B$ and an operand $\bm{X}$, the Lloyd-Max algorithm finds $2^B$ quantization levels $\{\hat{x}_i\}_{i=1}^{2^B}$ such that quantizing $\bm{X}$ by rounding each scalar in $\bm{X}$ to the nearest quantization level minimizes the quantization MSE. 

The algorithm starts with an initial guess of quantization levels and then iteratively computes quantization thresholds $\{\tau_i\}_{i=1}^{2^B-1}$ and updates quantization levels $\{\hat{x}_i\}_{i=1}^{2^B}$. Specifically, at iteration $n$, thresholds are set to the midpoints of the previous iteration's levels:
\begin{align*}
    \tau_i^{(n)}=\frac{\hat{x}_i^{(n-1)}+\hat{x}_{i+1}^{(n-1)}}2 \text{ for } i=1\ldots 2^B-1
\end{align*}
Subsequently, the quantization levels are re-computed as conditional means of the data regions defined by the new thresholds:
\begin{align*}
    \hat{x}_i^{(n)}=\mathbb{E}\left[ \bm{X} \big| \bm{X}\in [\tau_{i-1}^{(n)},\tau_i^{(n)}] \right] \text{ for } i=1\ldots 2^B
\end{align*}
where to satisfy boundary conditions we have $\tau_0=-\infty$ and $\tau_{2^B}=\infty$. The algorithm iterates the above steps until convergence.

Figure \ref{fig:lm_quant} compares the quantization levels of a $7$-bit floating point (E3M3) quantizer (left) to a $7$-bit Lloyd-Max quantizer (right) when quantizing a layer of weights from the GPT3-126M model at a per-tensor granularity. As shown, the Lloyd-Max quantizer achieves substantially lower quantization MSE. Further, Table \ref{tab:FP7_vs_LM7} shows the superior perplexity achieved by Lloyd-Max quantizers for bitwidths of $7$, $6$ and $5$. The difference between the quantizers is clear at 5 bits, where per-tensor FP quantization incurs a drastic and unacceptable increase in perplexity, while Lloyd-Max quantization incurs a much smaller increase. Nevertheless, we note that even the optimal Lloyd-Max quantizer incurs a notable ($\sim 1.5$) increase in perplexity due to the coarse granularity of quantization. 

\begin{figure}[h]
  \centering
  \includegraphics[width=0.7\linewidth]{sections/figures/LM7_FP7.pdf}
  \caption{\small Quantization levels and the corresponding quantization MSE of Floating Point (left) vs Lloyd-Max (right) Quantizers for a layer of weights in the GPT3-126M model.}
  \label{fig:lm_quant}
\end{figure}

\begin{table}[h]\scriptsize
\begin{center}
\caption{\label{tab:FP7_vs_LM7} \small Comparing perplexity (lower is better) achieved by floating point quantizers and Lloyd-Max quantizers on a GPT3-126M model for the Wikitext-103 dataset.}
\begin{tabular}{c|cc|c}
\hline
 \multirow{2}{*}{\textbf{Bitwidth}} & \multicolumn{2}{|c|}{\textbf{Floating-Point Quantizer}} & \textbf{Lloyd-Max Quantizer} \\
 & Best Format & Wikitext-103 Perplexity & Wikitext-103 Perplexity \\
\hline
7 & E3M3 & 18.32 & 18.27 \\
6 & E3M2 & 19.07 & 18.51 \\
5 & E4M0 & 43.89 & 19.71 \\
\hline
\end{tabular}
\end{center}
\end{table}

\subsection{Proof of Local Optimality of LO-BCQ}
\label{subsec:lobcq_opt_proof}
For a given block $\bm{b}_j$, the quantization MSE during LO-BCQ can be empirically evaluated as $\frac{1}{L_b}\lVert \bm{b}_j- \bm{\hat{b}}_j\rVert^2_2$ where $\bm{\hat{b}}_j$ is computed from equation (\ref{eq:clustered_quantization_definition}) as $C_{f(\bm{b}_j)}(\bm{b}_j)$. Further, for a given block cluster $\mathcal{B}_i$, we compute the quantization MSE as $\frac{1}{|\mathcal{B}_{i}|}\sum_{\bm{b} \in \mathcal{B}_{i}} \frac{1}{L_b}\lVert \bm{b}- C_i^{(n)}(\bm{b})\rVert^2_2$. Therefore, at the end of iteration $n$, we evaluate the overall quantization MSE $J^{(n)}$ for a given operand $\bm{X}$ composed of $N_c$ block clusters as:
\begin{align*}
    \label{eq:mse_iter_n}
    J^{(n)} = \frac{1}{N_c} \sum_{i=1}^{N_c} \frac{1}{|\mathcal{B}_{i}^{(n)}|}\sum_{\bm{v} \in \mathcal{B}_{i}^{(n)}} \frac{1}{L_b}\lVert \bm{b}- B_i^{(n)}(\bm{b})\rVert^2_2
\end{align*}

At the end of iteration $n$, the codebooks are updated from $\mathcal{C}^{(n-1)}$ to $\mathcal{C}^{(n)}$. However, the mapping of a given vector $\bm{b}_j$ to quantizers $\mathcal{C}^{(n)}$ remains as  $f^{(n)}(\bm{b}_j)$. At the next iteration, during the vector clustering step, $f^{(n+1)}(\bm{b}_j)$ finds new mapping of $\bm{b}_j$ to updated codebooks $\mathcal{C}^{(n)}$ such that the quantization MSE over the candidate codebooks is minimized. Therefore, we obtain the following result for $\bm{b}_j$:
\begin{align*}
\frac{1}{L_b}\lVert \bm{b}_j - C_{f^{(n+1)}(\bm{b}_j)}^{(n)}(\bm{b}_j)\rVert^2_2 \le \frac{1}{L_b}\lVert \bm{b}_j - C_{f^{(n)}(\bm{b}_j)}^{(n)}(\bm{b}_j)\rVert^2_2
\end{align*}

That is, quantizing $\bm{b}_j$ at the end of the block clustering step of iteration $n+1$ results in lower quantization MSE compared to quantizing at the end of iteration $n$. Since this is true for all $\bm{b} \in \bm{X}$, we assert the following:
\begin{equation}
\begin{split}
\label{eq:mse_ineq_1}
    \tilde{J}^{(n+1)} &= \frac{1}{N_c} \sum_{i=1}^{N_c} \frac{1}{|\mathcal{B}_{i}^{(n+1)}|}\sum_{\bm{b} \in \mathcal{B}_{i}^{(n+1)}} \frac{1}{L_b}\lVert \bm{b} - C_i^{(n)}(b)\rVert^2_2 \le J^{(n)}
\end{split}
\end{equation}
where $\tilde{J}^{(n+1)}$ is the the quantization MSE after the vector clustering step at iteration $n+1$.

Next, during the codebook update step (\ref{eq:quantizers_update}) at iteration $n+1$, the per-cluster codebooks $\mathcal{C}^{(n)}$ are updated to $\mathcal{C}^{(n+1)}$ by invoking the Lloyd-Max algorithm \citep{Lloyd}. We know that for any given value distribution, the Lloyd-Max algorithm minimizes the quantization MSE. Therefore, for a given vector cluster $\mathcal{B}_i$ we obtain the following result:

\begin{equation}
    \frac{1}{|\mathcal{B}_{i}^{(n+1)}|}\sum_{\bm{b} \in \mathcal{B}_{i}^{(n+1)}} \frac{1}{L_b}\lVert \bm{b}- C_i^{(n+1)}(\bm{b})\rVert^2_2 \le \frac{1}{|\mathcal{B}_{i}^{(n+1)}|}\sum_{\bm{b} \in \mathcal{B}_{i}^{(n+1)}} \frac{1}{L_b}\lVert \bm{b}- C_i^{(n)}(\bm{b})\rVert^2_2
\end{equation}

The above equation states that quantizing the given block cluster $\mathcal{B}_i$ after updating the associated codebook from $C_i^{(n)}$ to $C_i^{(n+1)}$ results in lower quantization MSE. Since this is true for all the block clusters, we derive the following result: 
\begin{equation}
\begin{split}
\label{eq:mse_ineq_2}
     J^{(n+1)} &= \frac{1}{N_c} \sum_{i=1}^{N_c} \frac{1}{|\mathcal{B}_{i}^{(n+1)}|}\sum_{\bm{b} \in \mathcal{B}_{i}^{(n+1)}} \frac{1}{L_b}\lVert \bm{b}- C_i^{(n+1)}(\bm{b})\rVert^2_2  \le \tilde{J}^{(n+1)}   
\end{split}
\end{equation}

Following (\ref{eq:mse_ineq_1}) and (\ref{eq:mse_ineq_2}), we find that the quantization MSE is non-increasing for each iteration, that is, $J^{(1)} \ge J^{(2)} \ge J^{(3)} \ge \ldots \ge J^{(M)}$ where $M$ is the maximum number of iterations. 
%Therefore, we can say that if the algorithm converges, then it must be that it has converged to a local minimum. 
\hfill $\blacksquare$


\begin{figure}
    \begin{center}
    \includegraphics[width=0.5\textwidth]{sections//figures/mse_vs_iter.pdf}
    \end{center}
    \caption{\small NMSE vs iterations during LO-BCQ compared to other block quantization proposals}
    \label{fig:nmse_vs_iter}
\end{figure}

Figure \ref{fig:nmse_vs_iter} shows the empirical convergence of LO-BCQ across several block lengths and number of codebooks. Also, the MSE achieved by LO-BCQ is compared to baselines such as MXFP and VSQ. As shown, LO-BCQ converges to a lower MSE than the baselines. Further, we achieve better convergence for larger number of codebooks ($N_c$) and for a smaller block length ($L_b$), both of which increase the bitwidth of BCQ (see Eq \ref{eq:bitwidth_bcq}).


\subsection{Additional Accuracy Results}
%Table \ref{tab:lobcq_config} lists the various LOBCQ configurations and their corresponding bitwidths.
\begin{table}
\setlength{\tabcolsep}{4.75pt}
\begin{center}
\caption{\label{tab:lobcq_config} Various LO-BCQ configurations and their bitwidths.}
\begin{tabular}{|c||c|c|c|c||c|c||c|} 
\hline
 & \multicolumn{4}{|c||}{$L_b=8$} & \multicolumn{2}{|c||}{$L_b=4$} & $L_b=2$ \\
 \hline
 \backslashbox{$L_A$\kern-1em}{\kern-1em$N_c$} & 2 & 4 & 8 & 16 & 2 & 4 & 2 \\
 \hline
 64 & 4.25 & 4.375 & 4.5 & 4.625 & 4.375 & 4.625 & 4.625\\
 \hline
 32 & 4.375 & 4.5 & 4.625& 4.75 & 4.5 & 4.75 & 4.75 \\
 \hline
 16 & 4.625 & 4.75& 4.875 & 5 & 4.75 & 5 & 5 \\
 \hline
\end{tabular}
\end{center}
\end{table}

%\subsection{Perplexity achieved by various LO-BCQ configurations on Wikitext-103 dataset}

\begin{table} \centering
\begin{tabular}{|c||c|c|c|c||c|c||c|} 
\hline
 $L_b \rightarrow$& \multicolumn{4}{c||}{8} & \multicolumn{2}{c||}{4} & 2\\
 \hline
 \backslashbox{$L_A$\kern-1em}{\kern-1em$N_c$} & 2 & 4 & 8 & 16 & 2 & 4 & 2  \\
 %$N_c \rightarrow$ & 2 & 4 & 8 & 16 & 2 & 4 & 2 \\
 \hline
 \hline
 \multicolumn{8}{c}{GPT3-1.3B (FP32 PPL = 9.98)} \\ 
 \hline
 \hline
 64 & 10.40 & 10.23 & 10.17 & 10.15 &  10.28 & 10.18 & 10.19 \\
 \hline
 32 & 10.25 & 10.20 & 10.15 & 10.12 &  10.23 & 10.17 & 10.17 \\
 \hline
 16 & 10.22 & 10.16 & 10.10 & 10.09 &  10.21 & 10.14 & 10.16 \\
 \hline
  \hline
 \multicolumn{8}{c}{GPT3-8B (FP32 PPL = 7.38)} \\ 
 \hline
 \hline
 64 & 7.61 & 7.52 & 7.48 &  7.47 &  7.55 &  7.49 & 7.50 \\
 \hline
 32 & 7.52 & 7.50 & 7.46 &  7.45 &  7.52 &  7.48 & 7.48  \\
 \hline
 16 & 7.51 & 7.48 & 7.44 &  7.44 &  7.51 &  7.49 & 7.47  \\
 \hline
\end{tabular}
\caption{\label{tab:ppl_gpt3_abalation} Wikitext-103 perplexity across GPT3-1.3B and 8B models.}
\end{table}

\begin{table} \centering
\begin{tabular}{|c||c|c|c|c||} 
\hline
 $L_b \rightarrow$& \multicolumn{4}{c||}{8}\\
 \hline
 \backslashbox{$L_A$\kern-1em}{\kern-1em$N_c$} & 2 & 4 & 8 & 16 \\
 %$N_c \rightarrow$ & 2 & 4 & 8 & 16 & 2 & 4 & 2 \\
 \hline
 \hline
 \multicolumn{5}{|c|}{Llama2-7B (FP32 PPL = 5.06)} \\ 
 \hline
 \hline
 64 & 5.31 & 5.26 & 5.19 & 5.18  \\
 \hline
 32 & 5.23 & 5.25 & 5.18 & 5.15  \\
 \hline
 16 & 5.23 & 5.19 & 5.16 & 5.14  \\
 \hline
 \multicolumn{5}{|c|}{Nemotron4-15B (FP32 PPL = 5.87)} \\ 
 \hline
 \hline
 64  & 6.3 & 6.20 & 6.13 & 6.08  \\
 \hline
 32  & 6.24 & 6.12 & 6.07 & 6.03  \\
 \hline
 16  & 6.12 & 6.14 & 6.04 & 6.02  \\
 \hline
 \multicolumn{5}{|c|}{Nemotron4-340B (FP32 PPL = 3.48)} \\ 
 \hline
 \hline
 64 & 3.67 & 3.62 & 3.60 & 3.59 \\
 \hline
 32 & 3.63 & 3.61 & 3.59 & 3.56 \\
 \hline
 16 & 3.61 & 3.58 & 3.57 & 3.55 \\
 \hline
\end{tabular}
\caption{\label{tab:ppl_llama7B_nemo15B} Wikitext-103 perplexity compared to FP32 baseline in Llama2-7B and Nemotron4-15B, 340B models}
\end{table}

%\subsection{Perplexity achieved by various LO-BCQ configurations on MMLU dataset}


\begin{table} \centering
\begin{tabular}{|c||c|c|c|c||c|c|c|c|} 
\hline
 $L_b \rightarrow$& \multicolumn{4}{c||}{8} & \multicolumn{4}{c||}{8}\\
 \hline
 \backslashbox{$L_A$\kern-1em}{\kern-1em$N_c$} & 2 & 4 & 8 & 16 & 2 & 4 & 8 & 16  \\
 %$N_c \rightarrow$ & 2 & 4 & 8 & 16 & 2 & 4 & 2 \\
 \hline
 \hline
 \multicolumn{5}{|c|}{Llama2-7B (FP32 Accuracy = 45.8\%)} & \multicolumn{4}{|c|}{Llama2-70B (FP32 Accuracy = 69.12\%)} \\ 
 \hline
 \hline
 64 & 43.9 & 43.4 & 43.9 & 44.9 & 68.07 & 68.27 & 68.17 & 68.75 \\
 \hline
 32 & 44.5 & 43.8 & 44.9 & 44.5 & 68.37 & 68.51 & 68.35 & 68.27  \\
 \hline
 16 & 43.9 & 42.7 & 44.9 & 45 & 68.12 & 68.77 & 68.31 & 68.59  \\
 \hline
 \hline
 \multicolumn{5}{|c|}{GPT3-22B (FP32 Accuracy = 38.75\%)} & \multicolumn{4}{|c|}{Nemotron4-15B (FP32 Accuracy = 64.3\%)} \\ 
 \hline
 \hline
 64 & 36.71 & 38.85 & 38.13 & 38.92 & 63.17 & 62.36 & 63.72 & 64.09 \\
 \hline
 32 & 37.95 & 38.69 & 39.45 & 38.34 & 64.05 & 62.30 & 63.8 & 64.33  \\
 \hline
 16 & 38.88 & 38.80 & 38.31 & 38.92 & 63.22 & 63.51 & 63.93 & 64.43  \\
 \hline
\end{tabular}
\caption{\label{tab:mmlu_abalation} Accuracy on MMLU dataset across GPT3-22B, Llama2-7B, 70B and Nemotron4-15B models.}
\end{table}


%\subsection{Perplexity achieved by various LO-BCQ configurations on LM evaluation harness}

\begin{table} \centering
\begin{tabular}{|c||c|c|c|c||c|c|c|c|} 
\hline
 $L_b \rightarrow$& \multicolumn{4}{c||}{8} & \multicolumn{4}{c||}{8}\\
 \hline
 \backslashbox{$L_A$\kern-1em}{\kern-1em$N_c$} & 2 & 4 & 8 & 16 & 2 & 4 & 8 & 16  \\
 %$N_c \rightarrow$ & 2 & 4 & 8 & 16 & 2 & 4 & 2 \\
 \hline
 \hline
 \multicolumn{5}{|c|}{Race (FP32 Accuracy = 37.51\%)} & \multicolumn{4}{|c|}{Boolq (FP32 Accuracy = 64.62\%)} \\ 
 \hline
 \hline
 64 & 36.94 & 37.13 & 36.27 & 37.13 & 63.73 & 62.26 & 63.49 & 63.36 \\
 \hline
 32 & 37.03 & 36.36 & 36.08 & 37.03 & 62.54 & 63.51 & 63.49 & 63.55  \\
 \hline
 16 & 37.03 & 37.03 & 36.46 & 37.03 & 61.1 & 63.79 & 63.58 & 63.33  \\
 \hline
 \hline
 \multicolumn{5}{|c|}{Winogrande (FP32 Accuracy = 58.01\%)} & \multicolumn{4}{|c|}{Piqa (FP32 Accuracy = 74.21\%)} \\ 
 \hline
 \hline
 64 & 58.17 & 57.22 & 57.85 & 58.33 & 73.01 & 73.07 & 73.07 & 72.80 \\
 \hline
 32 & 59.12 & 58.09 & 57.85 & 58.41 & 73.01 & 73.94 & 72.74 & 73.18  \\
 \hline
 16 & 57.93 & 58.88 & 57.93 & 58.56 & 73.94 & 72.80 & 73.01 & 73.94  \\
 \hline
\end{tabular}
\caption{\label{tab:mmlu_abalation} Accuracy on LM evaluation harness tasks on GPT3-1.3B model.}
\end{table}

\begin{table} \centering
\begin{tabular}{|c||c|c|c|c||c|c|c|c|} 
\hline
 $L_b \rightarrow$& \multicolumn{4}{c||}{8} & \multicolumn{4}{c||}{8}\\
 \hline
 \backslashbox{$L_A$\kern-1em}{\kern-1em$N_c$} & 2 & 4 & 8 & 16 & 2 & 4 & 8 & 16  \\
 %$N_c \rightarrow$ & 2 & 4 & 8 & 16 & 2 & 4 & 2 \\
 \hline
 \hline
 \multicolumn{5}{|c|}{Race (FP32 Accuracy = 41.34\%)} & \multicolumn{4}{|c|}{Boolq (FP32 Accuracy = 68.32\%)} \\ 
 \hline
 \hline
 64 & 40.48 & 40.10 & 39.43 & 39.90 & 69.20 & 68.41 & 69.45 & 68.56 \\
 \hline
 32 & 39.52 & 39.52 & 40.77 & 39.62 & 68.32 & 67.43 & 68.17 & 69.30  \\
 \hline
 16 & 39.81 & 39.71 & 39.90 & 40.38 & 68.10 & 66.33 & 69.51 & 69.42  \\
 \hline
 \hline
 \multicolumn{5}{|c|}{Winogrande (FP32 Accuracy = 67.88\%)} & \multicolumn{4}{|c|}{Piqa (FP32 Accuracy = 78.78\%)} \\ 
 \hline
 \hline
 64 & 66.85 & 66.61 & 67.72 & 67.88 & 77.31 & 77.42 & 77.75 & 77.64 \\
 \hline
 32 & 67.25 & 67.72 & 67.72 & 67.00 & 77.31 & 77.04 & 77.80 & 77.37  \\
 \hline
 16 & 68.11 & 68.90 & 67.88 & 67.48 & 77.37 & 78.13 & 78.13 & 77.69  \\
 \hline
\end{tabular}
\caption{\label{tab:mmlu_abalation} Accuracy on LM evaluation harness tasks on GPT3-8B model.}
\end{table}

\begin{table} \centering
\begin{tabular}{|c||c|c|c|c||c|c|c|c|} 
\hline
 $L_b \rightarrow$& \multicolumn{4}{c||}{8} & \multicolumn{4}{c||}{8}\\
 \hline
 \backslashbox{$L_A$\kern-1em}{\kern-1em$N_c$} & 2 & 4 & 8 & 16 & 2 & 4 & 8 & 16  \\
 %$N_c \rightarrow$ & 2 & 4 & 8 & 16 & 2 & 4 & 2 \\
 \hline
 \hline
 \multicolumn{5}{|c|}{Race (FP32 Accuracy = 40.67\%)} & \multicolumn{4}{|c|}{Boolq (FP32 Accuracy = 76.54\%)} \\ 
 \hline
 \hline
 64 & 40.48 & 40.10 & 39.43 & 39.90 & 75.41 & 75.11 & 77.09 & 75.66 \\
 \hline
 32 & 39.52 & 39.52 & 40.77 & 39.62 & 76.02 & 76.02 & 75.96 & 75.35  \\
 \hline
 16 & 39.81 & 39.71 & 39.90 & 40.38 & 75.05 & 73.82 & 75.72 & 76.09  \\
 \hline
 \hline
 \multicolumn{5}{|c|}{Winogrande (FP32 Accuracy = 70.64\%)} & \multicolumn{4}{|c|}{Piqa (FP32 Accuracy = 79.16\%)} \\ 
 \hline
 \hline
 64 & 69.14 & 70.17 & 70.17 & 70.56 & 78.24 & 79.00 & 78.62 & 78.73 \\
 \hline
 32 & 70.96 & 69.69 & 71.27 & 69.30 & 78.56 & 79.49 & 79.16 & 78.89  \\
 \hline
 16 & 71.03 & 69.53 & 69.69 & 70.40 & 78.13 & 79.16 & 79.00 & 79.00  \\
 \hline
\end{tabular}
\caption{\label{tab:mmlu_abalation} Accuracy on LM evaluation harness tasks on GPT3-22B model.}
\end{table}

\begin{table} \centering
\begin{tabular}{|c||c|c|c|c||c|c|c|c|} 
\hline
 $L_b \rightarrow$& \multicolumn{4}{c||}{8} & \multicolumn{4}{c||}{8}\\
 \hline
 \backslashbox{$L_A$\kern-1em}{\kern-1em$N_c$} & 2 & 4 & 8 & 16 & 2 & 4 & 8 & 16  \\
 %$N_c \rightarrow$ & 2 & 4 & 8 & 16 & 2 & 4 & 2 \\
 \hline
 \hline
 \multicolumn{5}{|c|}{Race (FP32 Accuracy = 44.4\%)} & \multicolumn{4}{|c|}{Boolq (FP32 Accuracy = 79.29\%)} \\ 
 \hline
 \hline
 64 & 42.49 & 42.51 & 42.58 & 43.45 & 77.58 & 77.37 & 77.43 & 78.1 \\
 \hline
 32 & 43.35 & 42.49 & 43.64 & 43.73 & 77.86 & 75.32 & 77.28 & 77.86  \\
 \hline
 16 & 44.21 & 44.21 & 43.64 & 42.97 & 78.65 & 77 & 76.94 & 77.98  \\
 \hline
 \hline
 \multicolumn{5}{|c|}{Winogrande (FP32 Accuracy = 69.38\%)} & \multicolumn{4}{|c|}{Piqa (FP32 Accuracy = 78.07\%)} \\ 
 \hline
 \hline
 64 & 68.9 & 68.43 & 69.77 & 68.19 & 77.09 & 76.82 & 77.09 & 77.86 \\
 \hline
 32 & 69.38 & 68.51 & 68.82 & 68.90 & 78.07 & 76.71 & 78.07 & 77.86  \\
 \hline
 16 & 69.53 & 67.09 & 69.38 & 68.90 & 77.37 & 77.8 & 77.91 & 77.69  \\
 \hline
\end{tabular}
\caption{\label{tab:mmlu_abalation} Accuracy on LM evaluation harness tasks on Llama2-7B model.}
\end{table}

\begin{table} \centering
\begin{tabular}{|c||c|c|c|c||c|c|c|c|} 
\hline
 $L_b \rightarrow$& \multicolumn{4}{c||}{8} & \multicolumn{4}{c||}{8}\\
 \hline
 \backslashbox{$L_A$\kern-1em}{\kern-1em$N_c$} & 2 & 4 & 8 & 16 & 2 & 4 & 8 & 16  \\
 %$N_c \rightarrow$ & 2 & 4 & 8 & 16 & 2 & 4 & 2 \\
 \hline
 \hline
 \multicolumn{5}{|c|}{Race (FP32 Accuracy = 48.8\%)} & \multicolumn{4}{|c|}{Boolq (FP32 Accuracy = 85.23\%)} \\ 
 \hline
 \hline
 64 & 49.00 & 49.00 & 49.28 & 48.71 & 82.82 & 84.28 & 84.03 & 84.25 \\
 \hline
 32 & 49.57 & 48.52 & 48.33 & 49.28 & 83.85 & 84.46 & 84.31 & 84.93  \\
 \hline
 16 & 49.85 & 49.09 & 49.28 & 48.99 & 85.11 & 84.46 & 84.61 & 83.94  \\
 \hline
 \hline
 \multicolumn{5}{|c|}{Winogrande (FP32 Accuracy = 79.95\%)} & \multicolumn{4}{|c|}{Piqa (FP32 Accuracy = 81.56\%)} \\ 
 \hline
 \hline
 64 & 78.77 & 78.45 & 78.37 & 79.16 & 81.45 & 80.69 & 81.45 & 81.5 \\
 \hline
 32 & 78.45 & 79.01 & 78.69 & 80.66 & 81.56 & 80.58 & 81.18 & 81.34  \\
 \hline
 16 & 79.95 & 79.56 & 79.79 & 79.72 & 81.28 & 81.66 & 81.28 & 80.96  \\
 \hline
\end{tabular}
\caption{\label{tab:mmlu_abalation} Accuracy on LM evaluation harness tasks on Llama2-70B model.}
\end{table}

%\section{MSE Studies}
%\textcolor{red}{TODO}


\subsection{Number Formats and Quantization Method}
\label{subsec:numFormats_quantMethod}
\subsubsection{Integer Format}
An $n$-bit signed integer (INT) is typically represented with a 2s-complement format \citep{yao2022zeroquant,xiao2023smoothquant,dai2021vsq}, where the most significant bit denotes the sign.

\subsubsection{Floating Point Format}
An $n$-bit signed floating point (FP) number $x$ comprises of a 1-bit sign ($x_{\mathrm{sign}}$), $B_m$-bit mantissa ($x_{\mathrm{mant}}$) and $B_e$-bit exponent ($x_{\mathrm{exp}}$) such that $B_m+B_e=n-1$. The associated constant exponent bias ($E_{\mathrm{bias}}$) is computed as $(2^{{B_e}-1}-1)$. We denote this format as $E_{B_e}M_{B_m}$.  

\subsubsection{Quantization Scheme}
\label{subsec:quant_method}
A quantization scheme dictates how a given unquantized tensor is converted to its quantized representation. We consider FP formats for the purpose of illustration. Given an unquantized tensor $\bm{X}$ and an FP format $E_{B_e}M_{B_m}$, we first, we compute the quantization scale factor $s_X$ that maps the maximum absolute value of $\bm{X}$ to the maximum quantization level of the $E_{B_e}M_{B_m}$ format as follows:
\begin{align}
\label{eq:sf}
    s_X = \frac{\mathrm{max}(|\bm{X}|)}{\mathrm{max}(E_{B_e}M_{B_m})}
\end{align}
In the above equation, $|\cdot|$ denotes the absolute value function.

Next, we scale $\bm{X}$ by $s_X$ and quantize it to $\hat{\bm{X}}$ by rounding it to the nearest quantization level of $E_{B_e}M_{B_m}$ as:

\begin{align}
\label{eq:tensor_quant}
    \hat{\bm{X}} = \text{round-to-nearest}\left(\frac{\bm{X}}{s_X}, E_{B_e}M_{B_m}\right)
\end{align}

We perform dynamic max-scaled quantization \citep{wu2020integer}, where the scale factor $s$ for activations is dynamically computed during runtime.

\subsection{Vector Scaled Quantization}
\begin{wrapfigure}{r}{0.35\linewidth}
  \centering
  \includegraphics[width=\linewidth]{sections/figures/vsquant.jpg}
  \caption{\small Vectorwise decomposition for per-vector scaled quantization (VSQ \citep{dai2021vsq}).}
  \label{fig:vsquant}
\end{wrapfigure}
During VSQ \citep{dai2021vsq}, the operand tensors are decomposed into 1D vectors in a hardware friendly manner as shown in Figure \ref{fig:vsquant}. Since the decomposed tensors are used as operands in matrix multiplications during inference, it is beneficial to perform this decomposition along the reduction dimension of the multiplication. The vectorwise quantization is performed similar to tensorwise quantization described in Equations \ref{eq:sf} and \ref{eq:tensor_quant}, where a scale factor $s_v$ is required for each vector $\bm{v}$ that maps the maximum absolute value of that vector to the maximum quantization level. While smaller vector lengths can lead to larger accuracy gains, the associated memory and computational overheads due to the per-vector scale factors increases. To alleviate these overheads, VSQ \citep{dai2021vsq} proposed a second level quantization of the per-vector scale factors to unsigned integers, while MX \citep{rouhani2023shared} quantizes them to integer powers of 2 (denoted as $2^{INT}$).

\subsubsection{MX Format}
The MX format proposed in \citep{rouhani2023microscaling} introduces the concept of sub-block shifting. For every two scalar elements of $b$-bits each, there is a shared exponent bit. The value of this exponent bit is determined through an empirical analysis that targets minimizing quantization MSE. We note that the FP format $E_{1}M_{b}$ is strictly better than MX from an accuracy perspective since it allocates a dedicated exponent bit to each scalar as opposed to sharing it across two scalars. Therefore, we conservatively bound the accuracy of a $b+2$-bit signed MX format with that of a $E_{1}M_{b}$ format in our comparisons. For instance, we use E1M2 format as a proxy for MX4.

\begin{figure}
    \centering
    \includegraphics[width=1\linewidth]{sections//figures/BlockFormats.pdf}
    \caption{\small Comparing LO-BCQ to MX format.}
    \label{fig:block_formats}
\end{figure}

Figure \ref{fig:block_formats} compares our $4$-bit LO-BCQ block format to MX \citep{rouhani2023microscaling}. As shown, both LO-BCQ and MX decompose a given operand tensor into block arrays and each block array into blocks. Similar to MX, we find that per-block quantization ($L_b < L_A$) leads to better accuracy due to increased flexibility. While MX achieves this through per-block $1$-bit micro-scales, we associate a dedicated codebook to each block through a per-block codebook selector. Further, MX quantizes the per-block array scale-factor to E8M0 format without per-tensor scaling. In contrast during LO-BCQ, we find that per-tensor scaling combined with quantization of per-block array scale-factor to E4M3 format results in superior inference accuracy across models. 
 


\end{document}


% This document was modified from the file originally made available by
% Pat Langley and Andrea Danyluk for ICML-2K. This version was created
% by Iain Murray in 2018, and modified by Alexandre Bouchard in
% 2019 and 2021 and by Csaba Szepesvari, Gang Niu and Sivan Sabato in 2022.
% Modified again in 2023 and 2024 by Sivan Sabato and Jonathan Scarlett.
% Previous contributors include Dan Roy, Lise Getoor and Tobias
% Scheffer, which was slightly modified from the 2010 version by
% Thorsten Joachims & Johannes Fuernkranz, slightly modified from the
% 2009 version by Kiri Wagstaff and Sam Roweis's 2008 version, which is
% slightly modified from Prasad Tadepalli's 2007 version which is a
% lightly changed version of the previous year's version by Andrew
% Moore, which was in turn edited from those of Kristian Kersting and
% Codrina Lauth. Alex Smola contributed to the algorithmic style files.
