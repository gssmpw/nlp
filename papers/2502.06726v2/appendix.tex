% Appendix

\section{Additional proofs}\label{sec:appendix}

%\scalebox{1.15}{\textbf{Contents}}
\textbf{Contents}
% 
\begin{enumerate}
\item[]%\hspace{-8mm}
\hyperref[apdx:proofs:intro]{\ref{apdx:proofs:intro}\hspace{1mm} Preliminary remarks}
% 
\hspace*{\fill} 
\pageref{apdx:proofs:intro}
% 
% 
\item[]%\hspace{-8mm}
\hyperref[apdx:proofs:preliminary_results]{\ref{apdx:proofs:preliminary_results}\hspace{1mm} Proofs of preliminary results (Section \ref{sec:preliminaries})}
% 
\hspace*{\fill} 
\pageref{apdx:proofs:preliminary_results}
% 
% 
\item[]%\hspace{-8mm}
\hyperref[apdx:proofs:rdes]{\ref{apdx:proofs:rdes}\hspace{1mm} Proofs of rough differential equation results (Section \ref{sec:rdes})}
% 
\hspace*{\fill} 
\pageref{apdx:proofs:rdes}
% 
% 
\item[]%\hspace{-8mm}
\hyperref[apdx:proofs:pmp]{\ref{apdx:proofs:pmp}\hspace{1mm} Additional proofs for the PMP (Section \ref{sec:pmp})}
% 
\hspace*{\fill} 
\pageref{apdx:proofs:pmp}
% 
% 
\item[]%\hspace{-8mm}
\hyperref[apdx:proofs:example]{\ref{apdx:proofs:example}\hspace{1mm} Additional details for the indirect shooting method (Section \ref{sec:example})}
% 
\hspace*{\fill} 
\pageref{apdx:proofs:example}
\end{enumerate}
% 


\subsection{Preliminary remarks}\label{apdx:proofs:intro}
This appendix contains additional proofs for  results in the main manuscript. 
Thoughout, % 
we use the inequalities 
$(|a_1|+|a_2|)^\frac{1}{p}\leq |a_1|^\frac{1}{p}+|a_2|^\frac{1}{p}$ and $|\sum_{i=1}^Na_i|^p\leq N^p\sum_{i=1}^N|a_i|^p$ 
% 
% 
% 
% 
% 
for any $p\geq 1$, $a_i\in\R$, and $N\in\N$. 
% 
% 
% 
% 
% 
% 
% 
% 
% 
Also, any function $f\in C_b^n$ satisfies the mean value theorem $
f(x+h)=f(x)
+
\frac{1}{k!}\sum_{k=1}^{n-1}\nabla^k f(x)h^{\otimes k}
+
\frac{1}{(n-1)!}
\int_0^1
\nabla^nf(x+\theta h)
h^{\otimes n}
(1-\theta)^{n-1}\dd\theta$. 
% 
 

\subsection{Proofs of preliminary results (Section \ref{sec:preliminaries})}\label{apdx:proofs:preliminary_results} 









\subsubsection{Rough paths, controlled rough paths, and rough integration (Section \ref{sec:preliminaries:rough_paths})}
%\subsubsection*{Proof of Lemma \ref{lem:pvariation:inequalities}}
\begin{proof}[Proof of Lemma \ref{lem:pvariation:inequalities}]
To show \eqref{eq:path_finite_var:infty_ineq}, we write
$$
\|X\|_\infty=\sup_{t\in[0,T]}\|X_t\|
\leq\|X_0\|+\sup_{t\in[0,T]}\|X_t-X_0\|^\frac{p}{p}
\leq 
\|X_0\|+\Big(\sup_{t\in[0,T]}\|X_t-X_0\|^p\Big)^\frac{1}{p}
\leq
\|X_0\|+\|X\|_p.
$$

To show \eqref{eq:path_finite_var:sum_p/2vars}, given an arbitrary partition  $\pi$ of $[0,T]$,
 \begin{align*}
 \sum_{[s,t]\in\pi}\|X_{s,t}\|^\frac{p}{2}
 &\leq
 \sum_{[s,t]\in\pi}
 \Big(
 \sum_{i=1}^n\|Y^i_{s,t}\|\|\widetilde{Y}^i_{s,t}\|+
 \sum_{j=1}^m\|Z^j_{s,t}\|
 +
 c|t-s|
 \Big)^\frac{p}{2}
 \\
 &\leq
 (n+m+1)^\frac{p}{2}
 \Big(
 \sum_{i=1}^n
 \sum_{[s,t]\in\pi}
 \|Y^i_{s,t}\|^\frac{p}{2}\|\widetilde{Y}^i_{s,t}\|^\frac{p}{2}
 +
 \sum_{j=1}^m
 \sum_{[s,t]\in\pi}
 \|Z^j_{s,t}\|^\frac{p}{2}
 +
 c^\frac{p}{2}\sum_{[s,t]\in\pi}
 |t-s|^\frac{p}{2}
 \Big)
 \\
 &\leq
 C_p
 \Big(
 \sum_{i=1}^n
 \Big(\sum_{[s,t]\in\pi}
 \|Y^i_{s,t}\|^p
 \Big)^\frac{1}{2}
  \Big(\sum_{[s,t]\in\pi}
 \|\widetilde{Y}^i_{s,t}\|^p
 \Big)^\frac{1}{2}
 +
 \sum_{j=1}^m
 \|Z^j\|_\frac{p}{2}^\frac{p}{2}
 +
 (c\,T)^\frac{p}{2}
 \Big)
 \\
 &\leq
 C_p
 \Big(
 \sum_{i=1}^n
 \|Y^i\|_p^\frac{p}{2}
 \|\widetilde{Y}^i\|_p^\frac{p}{2}
 +
 \sum_{j=1}^m
 \|Z^j\|_\frac{p}{2}^\frac{p}{2}
 +
 (c\,T)^\frac{p}{2}
 \Big),
 \end{align*}
 where we used $|\sum_{i=1}^Na_i|^\frac{p}{2}\leq N^\frac{p}{2}\sum_{i=1}^N|a_i|^\frac{p}{2}$  and H\"older's inequality $\sum_{i=1}^n|a_ib_i|\leq(\sum_{i=1}^n|a_i|^p)^\frac{1}{p}(\sum_{i=1}^n|b_i|^q)^\frac{1}{q}$ for any $a,b\in\R^n$ and $\frac{1}{p}+\frac{1}{q}=1$, so that
% 
 $\sum_{i=1}^n\left(|a_i|^\frac{p}{2}|b_i|^\frac{p}{2}\right)
 \leq
 \left(\sum_{i=1}^n|a_i|^p\right)^\frac{1}{2}\left(\sum_{i=1}^n|b_i|^p\right)^\frac{1}{2}$.
% 
The bound above is independent of the choice of partition $\pi$. 
 Thus, \eqref{eq:path_finite_var:sum_p/2vars} follows after taking the supremum over all partitions $\pi$ of $[0,T]$ and using the inequality $(\sum_i|a_i|)^\frac{2}{p}\leq\sum_i|a_i|^\frac{2}{p}$ for $p\geq 2$.  \eqref{eq:path_finite_var:sum_pvars} is shown similarly.   

To show $\|X\|_p\leq\|X\|_\frac{p}{2}$, we  write 
$
\|X\|_p^p
=\sup_\pi\sum_{[s,t]\in\pi}\left(\|X_{s,t}\|^\frac{p}{2}\right)^2
\leq
(
\sup_\pi\sum_{[s,t]\in\pi}\|X_{s,t}\|^\frac{p}{2})^2
=
\|X\|_\frac{p}{2}^p
$ since $\sum_i a_i^2\leq\left(\sum_i a_i\right)^2$ for $a_i\geq 0$.

To show \eqref{eq:sigma(.,X):pvar}, we use the smoothness of $\sigma$: for all $s,t\in[0,T]$,
\begin{align}\label{eq:sigma(.,Y)_st_bound}
\|\sigma(\cdot,X)_{s,t}\|&\leq\|\sigma(t,X_t)-\sigma(t,X_s)\|+\|\sigma(t,X_s)-\sigma(s,X_s)\|\leq \|\sigma\|_{C_b^1}(
\|X_{s,t}\|
+
|t-s|), 
\end{align}
and we conclude with \eqref{eq:path_finite_var:sum_pvars}.  

To show \eqref{eq:Delta_sigma(.,X):pvar},  
for any $s,t\in[0,T]$, we write
\begin{align}\label{eq:deltaf_timevarying:st:0}
\|(\sigma(\cdot,X)-\sigma(\cdot,\tilde{X}))_{s,t}\|
&\leq
\|
(\sigma(t,X)-\sigma(t,\tilde{X}))_{s,t}
\| 
+
\| 
(\sigma(\cdot,X_s)-\sigma(\cdot,\tilde{X}_s))_{s,t}
\|.
\end{align}
We bound the first term in \eqref{eq:deltaf_timevarying:st:0} next. Denoting $\Delta X = X-\tilde{X}$ and for any $\theta\in\R$,
$$
\frac{\partial\sigma}{\partial x}(t,\tilde{X}_t+\theta\Delta X_t)
=
\frac{\partial\sigma}{\partial x}(t,\tilde{X}_s+\theta\Delta X_s)
+
\int_0^1
\frac{\partial^2\sigma}{\partial x^2}(t,\tilde{X}_s+\theta\Delta X_s+\zeta(\tilde{X}_{s,t}+\theta\Delta X_{s,t}))\dd\zeta
((1-\theta)\tilde{X}_{s,t}+\theta X_{s,t})
$$
so that by denoting $Z_{\theta,\zeta}=\tilde{X}_s+\theta\Delta X_s+\zeta(\tilde{X}_{s,t}+\theta\Delta X_{s,t})$ and using the mean value theorem again,
\begin{align*}
\|\sigma(t,X)_{s,t}-\sigma(t,\tilde{X})_{s,t}\|
&=
\left\|
\int_0^1\nabla \sigma(t,\tilde{X}_t+\theta\Delta X_t)\dd\theta\Delta X_t
-
\int_0^1\nabla \sigma(t,\tilde{X}_s+\theta\Delta X_s)\dd\theta\Delta X_s
\right\|
\\
&\hspace{-3cm}=
\left\|
\int_0^1\nabla \sigma(t,\tilde{X}_s+\theta\Delta X_s)\dd\theta(\Delta X_t-\Delta X_s)
+
\int_0^1\int_0^1
\nabla^2 \sigma(t,Z_{\theta,\zeta})\dd\zeta
((1-\theta)\tilde{X}_{s,t}+\theta X_{s,t})
\dd\theta\Delta X_t 
\right\|
\\
&\leq
\|\sigma\|_{C_b^2}
\left(
\|\Delta X_{s,t}\|
+
(\|X_{s,t}\|+\|\tilde{X}_{s,t}\|)\|\Delta X\|_\infty
\right).
\end{align*}
The second term in \eqref{eq:deltaf_timevarying:st:0} is bounded similarly:
\begin{align}\label{eq:deltaf_timevarying:st:2ndterm}
\| 
(\sigma(\cdot,X_s)-\sigma(\cdot,\tilde{X}_s))_{s,t}
\|
&=
\left\|
\int_0^1\hspace{-1mm}
\Big(\frac{\partial\sigma}{\partial x}(t,\tilde{X}_s+\theta\Delta X_s)- \frac{\partial\sigma}{\partial x}(s,\tilde{X}_s+\theta\Delta X_s)\Big)
\dd\theta\Delta X_s
\right\|
\leq \|\sigma\|_{C_b^2}|t-s|\|\Delta X\|_\infty.
\end{align}
Thus, \eqref{eq:deltaf_timevarying:st:0} can be bounded as
\begin{align}\label{eq:deltaf_timevarying:st}
\|(\sigma(\cdot,X)-\sigma(\cdot,\tilde{X}))_{s,t}\|
&\leq
\|\sigma\|_{C_b^2}
\big(
\|\Delta X_{s,t}\|
+
(\|X_{s,t}\|+\|\tilde{X}_{s,t}\|+
|t-s|)\|\Delta X\|_\infty
\big).
\end{align}
The conclusion follows from  \eqref{eq:path_finite_var:sum_pvars} and $\|\Delta X\|_\infty\leq\|\Delta X_0\|+\|\Delta X\|_p$ in \eqref{eq:path_finite_var:infty_ineq}.
\end{proof}








%\subsubsection*{Proof of Lemma \ref{lem:rough_path:sigma(.,Y):controlled}}

\begin{proof}[Proof of Lemma \ref{lem:rough_path:sigma(.,Y):controlled}]
To show \eqref{eq:controlled_path:pvar_norm} and \eqref{eq:|Y|_p_ineq},   we write
\begin{align*}
\|Y_{s,t}\|
\mathop{\leq}^{\eqref{eq:remainder}}
\|Y'_s\|\|X_{s,t}\|+\|R^Y_{s,t}\|
\leq 
\|Y'\|_\infty\|X_{s,t}\|+\|R^Y_{s,t}\|
\mathop{\leq}^{\eqref{eq:path_finite_var:infty_ineq}}
(\|Y'_0\|+\|Y'\|_p)\|X_{s,t}\|+\|R^Y_{s,t}\|
\end{align*}
for any $s,t\in[0,T]$. % 
Then,  \eqref{eq:controlled_path:pvar_norm} and \eqref{eq:|Y|_p_ineq} follow from \eqref{eq:path_finite_var:sum_pvars}  and 
$\|R^Y\|_p\leq\|R^Y\|_\frac{p}{2}$ in Lemma \ref{lem:pvariation:inequalities}.


To show \eqref{eq:sigma(.,Y):pvar}, for any $s,t\in[0,T]$, we write
\begin{align*}
\|\sigma(\cdot,Y)_{s,t}\|
\mathop{\leq}^{\eqref{eq:sigma(.,Y)_st_bound},\eqref{eq:remainder}}
 \|\sigma\|_{C_b^1}(\|Y'_s\|\|X_{s,t}\|+\|R^Y_{s,t}\|+|t-s|)
\mathop{\leq}^{\eqref{eq:path_finite_var:infty_ineq}} 
\|\sigma\|_{C_b^1}(M_{Y'}\|X_{s,t}\|+\|R^Y_{s,t}\|+|t-s|^\frac{2}{p}),
\end{align*}
where we used % 
$\|Y'\|_\infty\leq\|Y'_0\|+\|Y'\|_p=M_{Y'}$ % 
in the second inequality. Then, \eqref{eq:sigma(.,Y):pvar} follows from \eqref{eq:path_finite_var:sum_pvars}.


To show \eqref{eq:sigma(.,Y)':pvar}, denoting $\nablax\sigma=\frac{\partial\sigma}{\partial x}$, for any $s,t\in[0,T]$, we write
\begin{align*}
&\|\sigma(\cdot,Y)'_{s,t}\|
=
\|
\nablax\sigma(t,Y_t)Y_t'
-\nablax\sigma(t,Y_t)Y_s'+\nablax\sigma(t,Y_t)Y_s'
-\nablax\sigma(s,Y_t)Y_s'+\nablax\sigma(s,Y_t)Y_s'
-\nablax\sigma(s,Y_s)Y_s'
\|
\\
&\hspace{17mm}\leq 
\|\nablax\sigma(t,Y_t)\|\|Y'_t-Y'_s\|+
\|\nablax\sigma(t,Y_t)-\nablax\sigma(s,Y_t)\|\|Y_s'\| +
\|\nablax\sigma(s,Y_t)-\nablax\sigma(s,Y_s)\|\|Y_s'\| 
\\
&\hspace{17mm}\leq 
\|\sigma\|_{C^2_b}(\|Y'_{s,t}\|
+
(|t-s|
+
\|Y_{s,t}\|)
\|Y_s'\|
)
\\
&\hspace{17mm}\leq 
\|\sigma\|_{C^2_b}(
\|Y'_{s,t}\|+
(
|t-s|
+
M_{Y'}\|X_{s,t}\|+\|R^Y_{s,t}\|)
M_{Y'}
),
% 
% 
% 
\\
&\implies
\|\sigma(\cdot,Y)'\|_p
\mathop{\leq}^{\eqref{eq:path_finite_var:sum_pvars}}
C_p\|\sigma\|_{C^2_b}(
\|Y'\|_p
+
M_{Y'}(
T
+
M_{Y'}\|X\|_p+\|R^Y\|_\frac{p}{2}
)
)
\\
&\hspace{27mm}\leq
C_p\|\sigma\|_{C^2_b}M_{Y'}(
1+
T+
M_{Y'}\|X\|_p+\|R^Y\|_\frac{p}{2})
\\
&\hspace{27mm}\leq
C_p\|\sigma\|_{C^2_b}K_Y(1+K_Y+T)(1+\|X\|_p).
\hspace{3.5cm}
(K_Y =  M_{Y'}+\|R^Y\|_\frac{p}{2})
\end{align*}

To show \eqref{eq:RY:p/2var:Y^2+RY+T} and \eqref{eq:RY:p/2var:KYs}, for any $s,t\in[0,T]$,
\begin{align*}
\|R^{\sigma(\cdot,Y)}_{s,t}\|
&\mathop{=}^{\eqref{eq:remainder}}
\|
\sigma(s,Y_t)-\sigma(s,Y_s)-\nablax \sigma(s,Y_s)Y'_sX_{s,t}
+\sigma(t,Y_t)-\sigma(s,Y_t)
\|
\\
&\mathop{=}^{\eqref{eq:remainder}}
\|
\sigma(s,Y_t)-\sigma(s,Y_s)-\nablax \sigma(s,Y_s)Y_{s,t}-\nablax \sigma(s,Y_s)R^Y_{s,t}
+
\sigma(t,Y_t)-\sigma(s,Y_t)\|
\\
&\leq
\left\|
\int_0^1\nablax^2 \sigma(s,Y_s+\theta Y_{s,t})Y_{s,t}^{\otimes 2}(1-\theta)\dd\theta
\right\| 
+\|\nablax \sigma(s,Y_s)R^Y_{s,t}\|
+
\|\sigma(t,Y_t)-\sigma(s,Y_t)\|
\\
&\leq 
\|\sigma\|_{C^2_b}(\|Y_{s,t}\|^2+\|R^Y_{s,t}\|+|t-s|),
% 
% 
% 
\\
\implies
\|R^{\sigma(\cdot,Y)}\|_\frac{p}{2}
&\mathop{\leq}^{\eqref{eq:path_finite_var:sum_p/2vars}}
C_p
\|\sigma\|_{C^2_b}(
\|Y\|_p^2+\|R^Y\|_\frac{p}{2}+T)
\\
&\mathop{\leq}^{\eqref{eq:|Y|_p_ineq}}
C_p\|\sigma\|_{C^2_b}(
(1+\|X\|_p)^2K_Y^2
+
\|R^Y\|_\frac{p}{2}
+
T
)
\\
&\leq
C_p\|\sigma\|_{C^2_b}(
K_Y(K_Y(1+\|X\|_p)^2+1))+T
)
\leq
\eqref{eq:RY:p/2var:KYs}.
\end{align*}

Finally, from the last three inequalities,  $(\sigma(\cdot,Y),\sigma(\cdot,Y)')\in\C^p\times\C^p$ and $\|R^{\sigma(\cdot,Y)}\|_\frac{p}{2}<\infty$, so we conclude that $(\sigma(\cdot,Y),\sigma(\cdot,Y)')\in\sD^p_X$. 
\end{proof}


%\subsubsection*{Proof of Lemma \ref{lem:rough_path:ito_formula}}
\begin{proof}[Proof of Lemma \ref{lem:rough_path:ito_formula}]
This result is a particular case of  \cite[Theorem 7.7]{Friz2020}, assuming that the rough path $\mbX$ is geometric, so that its bracket is zero. The proof below is standard. We may assume that $f\in C^3_b$  since it is only evaluated on the image of the path $Y$ which is bounded.

First,   the rough integral in \eqref{eq:rough_path:ito_formula} is well-defined, since $(\nabla f(Y)Y',(\nabla f(Y)Y')')\in\sD^p_X$ 
with $(\nabla f(Y)Y')'=(\nabla f(Y)Y''+\nabla f^2(Y)(Y'\otimes Y'))$ by Lemma \ref{lem:control_path:product} (note that  Lemma \ref{lem:rough_path:ito_formula} is not used in the proof of Lemma \ref{lem:control_path:product}). The   Young integral in \eqref{eq:rough_path:ito_formula} is also well-defined  since $\nabla f(Y)\in\C^p$ and $\Gamma\in\C^{\frac{p}{2}}$  so that   $\frac{1}{p}+\frac{2}{p}>1$ (see for example  \cite[Theorem 6.8]{Friz2010} or \cite[Proposition 5.2]{Allan2021}). 

Next, we show \eqref{eq:rough_path:ito_formula}. Let $s,t\in[0,T]$. Then, % 
$$
Y_{s,t}
\mathop{=}^{\eqref{eq:rough_int:error_bound}}
Y'_sX_{s,t}+Y''_s\bX_{s,t}+\Gamma_{s,t}+K_{s,t},
\ \text{with }
\|K_{s,t}\|\leq C_p(\|R^Y\|_{\frac{p}{2},[s,t]}\|X\|_{p,[s,t]}+\|Y'\|_{p,[s,t]}\|\bX\|_{\frac{p}{2},[s,t]}),
$$
and by the mean value theorem, with $Z_{s,t}:=\frac{1}{2}\int_0^1\nabla^3f(Y_s+\theta Y_{s,t})Y_{s,t}^{\otimes 3}(1-\theta)^2\dd\theta$,
\begin{align*}
f(Y)_{s,t}
&=
\nabla f(Y_s)Y_{s,t}+\frac{1}{2}\nabla^2f(Y_s)Y_{s,t}^{\otimes 2}
+
Z_{s,t}
% 
\\
&=
\nabla f(Y_s)(Y'_sX_{s,t}+Y''_s\bX_{s,t}+\Gamma_{s,t})
+
\nabla f(Y_s)K_{s,t}
+
\frac{1}{2}\nabla^2f(Y_s)(Y'_sX_{s,t}+R^Y_{s,t})^{\otimes 2}
+
Z_{s,t}
\\
&=
\nabla f(Y_s)(Y'_sX_{s,t}+Y''_s\bX_{s,t})+\nabla f(Y_s)\Gamma_{s,t}
+
\frac{1}{2}\nabla^2f(Y_s)(Y'_s\otimes Y'_s)(X_{s,t}\otimes X_{s,t})
+
\\
&\qquad\quad
\frac{1}{2}\nabla^2f(Y_s)
(Y'_sX_{s,t}\otimes R^Y_{s,t}+R^Y_{s,t}\otimes Y'_sX_{s,t} +R^Y_{s,t}\otimes R^Y_{s,t})
+
\nabla f(Y_s)K_{s,t}
+
Z_{s,t},
\end{align*}
Next, define $\text{Sym}(\bX_{s,t})$ as  the symmetric part of $\bX_{s,t}$, with $\text{Sym}(\bX_{s,t})^{ij}=\frac{1}{2}(\bX_{s,t}^{ij}+\bX_{s,t}^{ji})$. Since $\mbX$ is geometric, $X_{s,t}^{ij}X_{s,t}^{ji}=\bX_{s,t}^{ij}+\bX_{s,t}^{ji}=2\text{Sym}(\bX_{s,t})^{ij}$, so $\frac{1}{2}(X_{s,t}\otimes X_{s,t})=\text{Sym}(\bX_{s,t})$. Also, $\nabla^2f(Y_s)(Y'_s\otimes Y'_s)$ is symmetric, so $\nabla^2f(Y_s)(Y'_s\otimes Y'_s)\bX_{s,t}=\nabla^2f(Y_s)(Y'_s\otimes Y'_s)\text{Sym}(\bX_{s,t})$, since ``the contraction of a symmetric matrix with an antisymmetric matrix is zero'' \cite[Proof of Proposition 6.9]{Allan2021}. Thus, 
\begin{align*}
f(Y)_{s,t} 
&=
\nabla f(Y_s)Y'_sX_{s,t}
+
(\nabla f(Y_s)Y''_s+\nabla f^2(Y_s)(Y'_s\otimes Y'_s))\bX_{s,t}
+
\nabla f(Y_s)\Gamma_{s,t} 
+
\\
&\qquad\quad
\frac{1}{2}\nabla^2f(Y_s)
(Y'_sX_{s,t}\otimes R^Y_{s,t}+R^Y_{s,t}\otimes Y'_sX_{s,t} +R^Y_{s,t}\otimes R^Y_{s,t})
+
\nabla f(Y_s)K_{s,t}
+
Z_{s,t}
\\
&=
(\nabla f(Y)Y')_sX_{s,t}
+
(\nabla f(Y)Y')'_s\bX_{s,t}
+
\nabla f(Y_s)\Gamma_{s,t} 
+
W_{s,t},
\end{align*}
where % 
% 
$W_{s,t}:=\frac{1}{2}\nabla^2f(Y_s)
(Y'_sX_{s,t}\otimes R^Y_{s,t}+R^Y_{s,t}\otimes Y'_sX_{s,t} +R^Y_{s,t}\otimes R^Y_{s,t})
+
\nabla f(Y_s)K_{s,t}
+
Z_{s,t}$ can be bounded as
$$
\|W_{s,t}\|\leq C_{p,f,\|Y\|_\infty,\|Y'\|_\infty}
\left(
\|X\|_{p,[s,t]}\|R^Y\|_{\frac{p}{2},[s,t]}
+
\|R^Y\|_{\frac{p}{2},[s,t]}^2
+
\|Y'\|_{p,[s,t]}\|\bX\|_{\frac{p}{2},[s,t]}
+
\|Y\|_{p,[s,t]}^3
\right),
$$
Since $X,Y,Y'$ have finite $p$-variation and $\bX,R^Y$ have finite $\frac{p}{2}$-variation, up to a time reparameterization, we may assume that $X,Y,Y'$ are $\frac{1}{p}$-H\"older continuous and $\bX,R^Y$ are $\frac{2}{p}$-H\"older continuous \cite[Proposition 5.14]{Friz2010}, so that
$$
\|W_{s,t}\|\leq C_{p,f,X,\mbX,Y,Y',R^Y}|t-s|^\frac{3}{p}.
$$
Thus, for an arbitrary partition $\pi$ of $[0,T]$, we obtain
\begin{align*}
f(Y_T)-f(Y_0)
% 
% 
% 
=
\sum_{[s,t]\in\pi}
(\nabla f(Y)Y')_sX_{s,t}
+
(\nabla f(Y)Y')'_s\bX_{s,t}
+
\sum_{[s,t]\in\pi}
\nabla f(Y_s)\Gamma_{s,t}
+
\sum_{[s,t]\in\pi}
W_{s,t}.
\end{align*}
The conclusion follows after taking the limit over all partitions $\pi$ of $[0,T]$ with vanishing mesh size in the above: % 
% 
% 
% 
% 
% 
% 
% 
% 
% 
% 
% 
% 
% 
% 
% 
% 
% 
% 
The first sum is a rough integral, 
the second is a Young integral, which is  well-defined  by \cite[Theorem 6.8]{Friz2010},  since $\nabla f(Y)\in\C^p$ and $\Gamma\in\C^{\frac{p}{2}}$  so that   $\frac{1}{p}+\frac{2}{p}>1$, and the third   is zero, since 
$$
\sum_{[s,t]\in\pi}
\|W_{s,t}\|
\leq 
C_{p,f,X,\mbX,Y,Y',R^Y} 
\sum_{[s,t]\in\pi}|t-s|^\frac{p}{p}
|\pi|^{\frac{3-p}{p}}
=
C_{p,f,X,\mbX,Y,Y',R^Y}
T
|\pi|^{\frac{3-p}{p}}
\to 0\ \text{as }|\pi|\to0,
$$
so that $\lim_{|\pi|\to0}\sum_{[s,t]\in\pi}
W_{s,t}=0$. 
This concludes the proof of Lemma \ref{lem:rough_path:ito_formula}.
\end{proof}


































\subsubsection{The greedy partition and Gaussian rough paths (Section \ref{sec:preliminaries:greedy_gaussian_paths})}
%\subsubsection*{Proofs of Lemma \ref{lem:Nalpha<=w(0,T)} and of Lemma \ref{lem:sum_Nalpha}}
\begin{proof}[Proof of Lemma \ref{lem:Nalpha<=w(0,T)}] The proof follows \cite[Lemma 4.9]{Cass2013}. Define the greedy partition $\{\tau_i\}_{i=0}^{N_{\alpha,[s,t]}(w)+1}$ of the interval $[0,T]$. Since $w$ is a control, $w(\tau_i,\tau_{i+1})+w(\tau_{i+1},\tau_{i+2})\leq w(\tau_i,\tau_{i+2})$, so 
\begin{align*}
\alpha N_{\alpha,[s,t]}(w)
&=
\sum_{i=0}^{N_{\alpha,[s,t]}(w)-1}
w(\tau_i,\tau_{i+1})
\leq
w(0,\tau_{N_{\alpha,[s,t]}(w)})
\leq
w(0,T),
\end{align*}
and the conclusion follows.
\end{proof}
\begin{proof}[Proof of Lemma \ref{lem:sum_Nalpha}]
The proof is inspired from the proof of \cite[Lemma 3]{Friz2013}. 
First, we define the accumulated $\alpha$-local $w$-variations \cite[Definition 4.1]{Cass2013}
$$
w_\alpha(s,t)=
\sup_{\substack{
\pi=\{t_i\}\subset[s,t]
\\
w(t_i,t_{i+1})\leq\alpha}
}
\sum_{i=0}^{N-1}w(t_i,t_{i+1}),
\qquad
w_{j,\alpha}(s,t)=
\sup_{\substack{
\pi=\{t_i\}\subset[s,t]
\\
w_j(t_i,t_{i+1})\leq\alpha}
}
\sum_{i=0}^{N-1}w_j(t_i,t_{i+1}),
\ \, j=1,\dots,n,
$$
where the supremums are over all partitions $\pi=\{s=t_0<t_1<\dots<t_N=t\}$ of $[s,t]$ such that $w(t_i,t_{i+1})\leq\alpha$  (or $w_j(t_i,t_{i+1})\leq\alpha$, respectively) for all $i=0,\dots,N-1$. We have
\begin{align*}
\frac{w_\alpha(s,t)}{C}
&\leq
\sup_{\substack{
\pi=\{t_i\}\subset[s,t]
\\
w(t_i,t_{i+1})\leq\alpha}
}
\sum_{i=0}^{N-1}\sum_{j=1}^nw_j(t_i,t_{i+1})
% 
% 
% 
% 
% 
% 
% 
% 
% 
% 
% 
\leq
\sum_{j=1}^n
\sup_{\substack{
\pi=\{t_i\}\subset[s,t]
\\
w_j(t_i,t_{i+1})\leq\alpha}
}
\sum_{i=0}^{N-1}w_j(t_i,t_{i+1})
% 
% 
=
\sum_{j=1}^n
w_{j,\alpha}(s,t).
\end{align*}
where the last inequality follows from the fact that $w_j(t_i,t_{i+1})\leq\alpha$ for all $j$ if $w(t_i,t_{i+1})\leq\alpha$. 
 Then,
\begin{align*}
\alpha N_{\alpha,[s,t]}(w)
&=
\sum_{i=0}^{N_{\alpha,[s,t]}(w)-1}
w(\tau_i,\tau_{i+1})
\leq
w_\alpha(s,t)
\leq
C\sum_{j=1}^n
w_{j,\alpha}(s,t).
\end{align*}
Finally, from \cite[Proposition 4.11]{Cass2013} (see also the proof of \cite[Lemma 3]{Friz2013}), $w_{j,\alpha}(s,t)\leq \alpha(2N_{\alpha,[s,t]}(w_j)+1)$ for $j=1,\dots,n$, and the conclusion follows.
\end{proof}

%\subsubsection*{Proof of Corollary \ref{cor:Nalpha:NX_NXtilde_NT:small_intervals}}
\begin{proof}[Proof of Corollary \ref{cor:Nalpha:NX_NXtilde_NT:small_intervals}]
First, $w$ is continuous, $w(t,t)=0$, and $w(s,t)+w(t,u)\leq w(s,u)$ for any $0\leq s\leq t\leq u\leq T$, so $w$ is a control. 
Second, consider the control $w_T$ defined by $w_T(s,t)=|t-s|$, so that $w(s,t)=C_p(w_\mbX(s,t)+w_{\widetilde{\mbX}}(s,t)+w_T(s,t))$. Any interval $[s,t]\subseteq[0,T]$ can be partitioned into intervals of size at most $\alpha$, so $N_{\alpha,[s,t]}(w_T)\leq 1+|t-s|/\alpha$. The inequality \eqref{eq:Nalpha<=3CpNalpha_X_and_time} then follows from \eqref{eq:Nalpha(w)<=sumNalpha(wj)} in Lemma \ref{lem:sum_Nalpha}.


% 
To show \eqref{eq:|X|+|Xtilde|+|dt|<=alpha^p},  let   $0<\alpha\leq 1$  and     $[s,t]\subseteq[0,T]$ be an interval small-enough to satisfy $w(s,t)\leq \alpha$, so that
$$
\|\bX\|_{\frac{p}{2},[s,t]}^\frac{p}{2}\leq w_\mbX(s,t)\leq w(s,t)\leq\alpha\leq 1,
\qquad
|t-s|\leq w(s,t)\leq\alpha\leq 1.
$$
Then, % 
$|t-s|^p\leq |t-s|$ and $\|\bX\|_{\frac{p}{2},[s,t]}^p\leq\|\bX\|_{\frac{p}{2},[s,t]}^\frac{p}{2}$, so that
{\small
\begin{align}\label{eq:mbX^p<=wX}
\|\mbX\|_{p,[s,t]}^p
&=
\big(
\|X\|_{p,[s,t]}+\|\bX\|_{\frac{p}{2},[s,t]}
\big)^p
\leq
2^p\big(
\|X\|_{p,[s,t]}^p+\|\bX\|_{\frac{p}{2},[s,t]}^p
\big)
\leq
2^p\big(
\|X\|_{p,[s,t]}^p+\|\bX\|_{\frac{p}{2},[s,t]}^\frac{p}{2}
\big)
=
2^pw_\mbX(s,t).
\end{align}
}% 
Thus,
{\small
\begin{align*}
(
\|\mbX\|_{p,[s,t]}+\|\widetilde{\mbX}\|_{p,[s,t]}+|t-s|
)^p
&\leq
3^p(
\|\mbX\|_{p,[s,t]}^p+\|\widetilde{\mbX}\|_{p,[s,t]}^p+|t-s|^p
)
\leq
6^p(
w_{\mbX}(s,t)+
w_{\widetilde{\mbX}}(s,t)+
|t-s|
)
= w(s,t),
\end{align*}
}% 
and \eqref{eq:|X|+|Xtilde|+|dt|<=alpha^p} follows with $w(s,t)\leq\alpha$.
% 

% 
% 
% 
% 
% 
% 
% 
% 
% 
% 
% 
% 
% 
% 
% 
% 
% 
% 
% 
% 
% 
% 
% 
% 
% 
% 
% 
% 
% 
% 
% 
% 
% 
% 
% 
% 
% 
% 
% 
% 
% 
% 
% 
% 
% 
% 


To show \eqref{eq:||X||_p<=exp(N_alpha^p)}, consider the greedy partition $\{\tau_i, i=0,1,\dots,N_{\alpha,[s,t]}(w)+1\}$, which is such that \eqref{eq:|X|+|Xtilde|+|dt|<=alpha^p} holds on any subinterval $[\tau_i,\tau_{i+1}]$ of this partition since $w(\tau_i,\tau_{i+1})\leq\alpha$, so that  $\|\mbX\|_{p,[\tau_i,\tau_{i+1}]}\leq\alpha^\frac{1}{p}$, $\|\widetilde{\mbX}\|_{p,[\tau_i,\tau_{i+1}]}\leq\alpha^\frac{1}{p}$, and $|\tau_{i+1}-\tau_i|\leq\alpha^\frac{1}{p}$. Then, by Lemma \ref{lem:pvar:intervals}, with $N:=N_{\alpha,[s,t]}(w)$,
\begin{align*}
\|\mbX\|_{p,[s,t]}+\|\widetilde{\mbX}\|_{p,[s,t]}+|t-s|
&\leq
(N+1)
\bigg(
\bigg(
\sum_{i=0}^N
\|\mbX\|_{p,[\tau_i,\tau_{i+1}]}^p
\bigg)^\frac{1}{p}
+
\bigg(
\sum_{i=0}^N
\|\widetilde{\mbX}\|_{p,[\tau_i,\tau_{i+1}]}^p
\bigg)^\frac{1}{p}
\bigg)
+
\sum_{i=0}^N|\tau_{i+1}-\tau_i|
\\
&% 
\leq
(N+1)
\big(
(N+1)
\alpha^\frac{1}{p}
+
(N+1)
\alpha^\frac{1}{p}
\big)
+
(N+1)\alpha^\frac{1}{p}
% 
% 
\leq
3\alpha^\frac{1}{p}(N+1)^2,
\end{align*}
where we used $p\geq 1$ in the second inequality.  Thus, using $
3\alpha^\frac{1}{p}(N+1)^2\leq
3\alpha^\frac{1}{p}2!\exp(N+1)\leq 6e\alpha^\frac{1}{p}\exp(N)$, we conclude that \eqref{eq:||X||_p<=exp(N_alpha^p)} holds with 
with  $C_{p,\alpha}=6e\alpha^\frac{1}{p}$, which concludes the proof.
\end{proof}


%\subsubsection*{Proof of Theorem \ref{thm:gaussian_rough_paths}}
\begin{proof}[Proof of Theorem \ref{thm:gaussian_rough_paths}]
The first claim follows from \cite[Theorem
15.33]{Friz2010}  (or \cite[Theorem 10.4]{Friz2020}, see also \cite[Corollary 2.3]{Friz2016}). 
Then, following the proof of \cite[Theorem 11]{Bayer2016}, for $\frac{1}{q}=\frac{1}{2\rho}+\frac{1}{2}>\frac{1}{2}$, by  \cite[Lemma 5 and Corollary 2]{Friz2013}, for a constant $C>0$,  
$\Prob(N_{\alpha,[0,T]}(\mbB)\geq r)\leq \exp(
-C\alpha^\frac{2}{p}r^\frac{2}{q}
)$ 
for every $r>0$.  
% 
% 
Thus, for $R>0$,
\begin{align*}
\E\left[\exp\left(DN_{\alpha,[0,T]}(\mbB)\right)\right]
&=
\int_0^\infty\Prob(\exp(DN_{\alpha,[0,T]}(\mbB)) \geq s)\dd s 
\\
&\leq
R+\int_R^\infty\Prob(\exp(DN_{\alpha,[0,T]}(\mbB)) \geq s)\dd s
\\
&=
R+\int_{\frac{\log(R)}{D}}^\infty\Prob(N_{\alpha,[0,T]}(\mbB) \geq r)D\exp(Dr)\dd r
&&\text{($r=\log(s)/D$)}
\\
&\leq
R+D\int_{\frac{\log(R)}{D}}^\infty
\exp\left(
-C\alpha^\frac{2}{p}r^\frac{2}{q}
+
Dr
\right)
\dd r,
\end{align*}
which is bounded since  $\frac{1}{q}>\frac{1}{2}$.
\end{proof}


\subsection{Proofs of rough differential equation results % 
(Section \ref{sec:rdes})}\label{apdx:proofs:rdes}
\subsubsection{Calculus with rough paths: controlled rough paths and rough integration (Section \ref{sec:rdes:calculus})}
%\subsubsection*{Proof of Lemma \ref{lem:control_path:product}}
\begin{proof}[Proof of Lemma \ref{lem:control_path:product}]
The first inequality \eqref{lem:control_path:product:YZ_p} follows from  
$\|(YZ)_{s,t}\|%\leq \|Y_sZ_{s,t}\|+\|Z_sY_{s,t}\|
\leq\|Y\|_\infty\|Z_{s,t}\|+\|Z\|_\infty\|Y_{s,t}\|$ and \eqref{eq:path_finite_var:sum_pvars} in Lemma \ref{lem:pvariation:inequalities}. 
The second inequality \eqref{lem:control_path:product:(YZ)'_p} follows similary from 
\begin{align*}
\|(YZ)'_{s,t}\|
&\leq\|(ZY')_{s,t}\|+\|(YZ')_{s,t}\|
\leq
\|Z\|_\infty\|Y'_{s,t}\|+\|Y'\|_\infty\|Z_{s,t}\|
+
\|Y\|_\infty\|Z'_{s,t}\|+\|Z'\|_\infty\|Y_{s,t}\|
\end{align*}
and \eqref{eq:path_finite_var:sum_pvars} in Lemma \ref{lem:pvariation:inequalities}. 
 To show the third inequality \eqref{lem:control_path:product:R^YZ_p/2}, we write
\begin{align*}
R^{YZ}_{s,t} &= 
(YZ)_{s,t}-(YZ)'_s X_{s,t}
=
Y_tZ_t-Y_sZ_s-(Y_sZ_s'+Z_sY_s') X_{s,t}
\\
&=
Y_sZ_{s,t}+Y_{s,t}Z_s+Y_{s,t}Z_{s,t}-(Y_sZ_s'+Z_sY_s') X_{s,t}
\\
&=Y_sR^Z_{s,t}+R^Y_{s,t}Z_s+Y_{s,t}Z_{s,t}
\end{align*}
so that $\|R^{YZ}_{s,t}\| \leq
\|Y\|_\infty\|R^Z_{s,t}\|+\|R^Y_{s,t}\|\|Z\|_\infty+\|Y_{s,t}\|\|Z_{s,t}\|$, and we conclude with  \eqref{eq:path_finite_var:sum_p/2vars} in  Lemma \ref{lem:pvariation:inequalities}. 
Thus, $(YZ,(YZ)')\in\sD^p_X$ with $(YZ)'=Y'Z+YZ'$. 
\end{proof}






%\subsubsection*{Proof of Lemma \ref{lem:rough_path:f(Y)'-f(tilde(Y))'}}
\begin{proof}[Proof of Lemma \ref{lem:rough_path:f(Y)'-f(tilde(Y))'}]
Throughout the proof, we write $\nabla\sigma:=\frac{\partial\sigma}{\partial x}$ and $\nabla^2\sigma:=\frac{\partial^2\sigma}{\partial x^2}$ for conciseness. 

To show \eqref{eq:controlled_path:Y-Ytilde_p}, we note that
\begin{align*}
\|\Delta Y_{s,t}\|
&=
\|Y'_sX_{s,t}+R_{s,t}^Y-\widetilde{Y}'_s\tilde{X}_{s,t}-R_{s,t}^{\widetilde{Y}}\| 
&&(Y_{s,t}=Y'_sX_{s,t}+R^Y_{s,t})
\\
&\leq
\|\Delta Y'\|_\infty\|X_{s,t}\|+\|\widetilde{Y}'\|_\infty\|\Delta X_{s,t}\|+\|\Delta R_{s,t}\| 
\\
&\leq
\Delta M_{Y'}\|X_{s,t}\|+M_{\widetilde{Y}'}\|\Delta X_{s,t}\|+\|\Delta R_{s,t}\|, 
&&(\|\widetilde{Y}'\|_\infty\leq\|\widetilde{Y}'_0\|+\|\widetilde{Y}'\|_p=M_{\widetilde{Y}'})
\end{align*}
so \eqref{eq:controlled_path:Y-Ytilde_p} follows from applying \eqref{eq:path_finite_var:sum_pvars} in Lemma \ref{lem:pvariation:inequalities}. 


To show \eqref{eq:controlled_path:sigma(Y)-sigma(Ytilde)_p}, we note that
\begin{align*}
\|(\sigma(\cdot,Y)'-\sigma(\cdot,\widetilde{Y})')_{s,t}\|
&=
\|
(\nabla \sigma(\cdot,Y)Y'-\nabla \sigma(\cdot,\widetilde{Y})\widetilde{Y}')_{s,t}
\| 
\\
&\leq
\|
(\nabla \sigma(\cdot,Y)\Delta Y')_{s,t}
\|+
\|((\nabla \sigma(\cdot,Y)-\nabla \sigma(\cdot,\widetilde{Y}))\widetilde{Y}')_{s,t}
\| 
\\
&\hspace{-4cm}\leq
\|
\nabla\sigma(t,Y_t)\Delta Y'_{s,t}
\|
+
\|
\nabla\sigma(\cdot,Y)_{s,t}\Delta Y'_s
\|
+ 
\|(\nabla\sigma(t,Y_t)-\nabla\sigma(t,\widetilde{Y}_t))\widetilde{Y}'_{s,t}
\| 
+
\|(\nabla\sigma(\cdot,Y)-\nabla\sigma(\cdot,\widetilde{Y}))_{s,t}\widetilde{Y}'_s
\|
\\
&\hspace{-4cm}\leq
\|\sigma\|_{C_b^1}\|\Delta Y'_{s,t}\|
+
\big(
\|\sigma\|_{C_b^2}\|Y_{s,t}\|+|t-s|
\big)\|\Delta Y'\|_\infty
+
\|\sigma\|_{C_b^2}\|\Delta Y\|_\infty\|\widetilde{Y}'_{s,t}\|
+
\\
&+
\|\sigma\|_{C_b^3}
\big(
\|\Delta Y_{s,t}\|
+
(
\|Y_{s,t}\|+\|\widetilde{Y}_{s,t}\|
+
|t-s|
)
\|\Delta Y\|_\infty
\big)\|\widetilde{Y}'\|_\infty,
\end{align*}
where in the last inequality, we used  \eqref{eq:sigma(.,Y)_st_bound} to bound  the second term and \eqref{eq:deltaf_timevarying:st} to bound the fourth term. 
Thus,  by \eqref{eq:path_finite_var:sum_pvars} in Lemma \ref{lem:pvariation:inequalities},
\begin{align*}
\|\sigma(\cdot,Y)'-\sigma(\cdot,\widetilde{Y})'\|_p
&\leq
C_p
\|\sigma\|_{C_b^3}
\big(
\|\Delta Y'\|_p+\|Y\|_p\|\Delta Y'\|_\infty+
\|\Delta Y\|_\infty\|\widetilde{Y}'\|_p
\\
&\qquad
+
	(
	\|\Delta Y\|_p
	+
	(\|Y\|_p+\|\widetilde{Y}\|_p)\|\Delta Y\|_\infty
	)
	\|\widetilde{Y}'\|_\infty
+
T(
\|\Delta Y'\|_\infty+\|\Delta Y\|_\infty\|\widetilde{Y}'\|_\infty
)
\big)
\\
&\hspace{-2cm}\leq
C_p\|\sigma\|_{C_b^3}
(1+\|\widetilde{Y}'_0\|+\|Y\|_p+\|\widetilde{Y}\|_p+\|\widetilde{Y}'\|_p
)^2
(
1+T
)
\big(
\|\Delta Y\|_p
+
\|\Delta Y'\|_p
+
\|\Delta Y_0\|
+
\|\Delta Y'_0\|
\big),
\end{align*}
using $\|\Delta Y\|_\infty\leq\|\Delta Y_0\|+\|\Delta Y\|_p$  by \eqref{eq:path_finite_var:infty_ineq} in the last inequality.
Next, 
% 
by \eqref{eq:controlled_path:Y-Ytilde_p}, 
$\|\Delta Y\|_p
\leq 
C_p(\Delta M_{Y'}\|X\|_p+M_{\widetilde{Y}'}\|\Delta X\|_p+\|\Delta R^Y\|_\frac{p}{2})\leq
C_p(1+M_{\widetilde{Y}'})(1+\|X\|_p)(\Delta M_{Y'}+\|\Delta X\|_p+\|\Delta R^Y\|_\frac{p}{2})$ with $\Delta M_{Y'}=\|\Delta Y'_0\|+\|\Delta Y'\|_p$, so 
\begin{align*}
\|\sigma(\cdot,Y)'-\sigma(\cdot,\widetilde{Y})'\|_p
&\leq
C_p\|\sigma\|_{C^3_b}
(1+\|\widetilde{Y}'_0\|+\|Y\|_p+\|\widetilde{Y}\|_p+\|\widetilde{Y}'\|_p)^2
(1+M_{\widetilde{Y}'})(1+\|X\|_p)
(1+T)
\big(
\\
&\qquad\qquad
\|\Delta X\|_p+\|\Delta R^Y\|_\frac{p}{2}
+
\|\Delta Y_0\|
+
\|\Delta Y'_0\|
+
\|\Delta Y'\|_p
\big).
\end{align*}
By \eqref{eq:|Y|_p_ineq}, 
$\|Y\|_p\leq C_p(1+\|X\|_p)K_Y$, so 
\begin{align*}
\|\sigma(\cdot,Y)'-\sigma(\cdot,\widetilde{Y})'\|_p
&\leq
C_p\|\sigma\|_{C^3_b}
(1+K_Y+K_{\widetilde{Y}})^2
(1+M_{\widetilde{Y}'})
(1+\|X\|_p+\|\widetilde{X}\|_p)^3
(1+T)
\big(
\\
&\qquad\qquad
\|\Delta X\|_p+\|\Delta R^Y\|_\frac{p}{2}
+
\|\Delta Y_0\|
+
\|\Delta Y'_0\|
+
\|\Delta Y'\|_p
\big),
\end{align*}
and we obtain   \eqref{eq:controlled_path:sigma(Y)-sigma(Ytilde)_p}.

To show \eqref{eq:controlled_path:R^sigma(Y)-R^sigma(Ytilde)_p/2}, we first decompose  $\|R^{\sigma(\cdot,Y)}_{s,t}-R^{\sigma(\cdot,\widetilde{Y})}_{s,t}\|$ as
\begin{align*}
\|R^{\sigma(\cdot,Y)}_{s,t}-R^{\sigma(\cdot,\widetilde{Y})}_{s,t}\|
&=
\|
\sigma(\cdot,Y)_{s,t}-\nabla \sigma(s,Y_s)Y'_sX_{s,t}-\sigma(\cdot,\widetilde{Y})_{s,t}
+
\nabla \sigma(s,\widetilde{Y}_s)\widetilde{Y}'_s\tilde{X}_{s,t})
\|
\\
&\hspace{-15mm}\leq
\|
\sigma(\cdot,Y)_{s,t}-\nabla \sigma(s,Y_s)Y_{s,t}
-
(\sigma(\cdot,\widetilde{Y})_{s,t}-\nabla \sigma(s,\widetilde{Y}_s)\widetilde{Y}_{s,t})\|
+
\|\nabla \sigma(s,Y_s)R^Y_{s,t}
-
\nabla \sigma(s,\widetilde{Y}_s)R^{\widetilde{Y}}_{s,t}
\|
\\
&\hspace{-15mm}\leq
\|
\sigma(s,Y)_{s,t}-\nabla \sigma(s,Y_s)Y_{s,t}
-
(\sigma(s,\widetilde{Y})_{s,t}-\nabla \sigma(s,\widetilde{Y}_s)\widetilde{Y}_{s,t})\|
+
\|\nabla \sigma(s,Y_s)R^Y_{s,t}
-
\nabla \sigma(s,\widetilde{Y}_s)R^{\widetilde{Y}}_{s,t}
\|
+
\\
&\qquad
\| 
(\sigma(\cdot,Y_t)-\sigma(\cdot,\widetilde{Y}_t))_{s,t}
\|
\\
&\hspace{-15mm}=:\|A_{s,t}\|+\|B_{s,t}\|+\|C_{s,t}\|.
\end{align*}
Next, we bound the three terms.  
First, denoting $\nabla^2\sigma(y)=\nabla^2\sigma(s,y)$ and $Y^\theta_{s,t}=\widetilde{Y}_s+\theta\widetilde{Y}_{s,t}$ for conciseness,
\begin{align*}
\|A_{s,t}\| &=
\left\|
\int_0^1 
\left(
	\nabla^2\sigma(Y_s+\theta Y_{s,t})Y_{s,t}^{\otimes 2}
	- 
	\nabla^2\sigma(\widetilde{Y}_s+\theta\widetilde{Y}_{s,t})\widetilde{Y}_{s,t}^{\otimes 2}
\right)(1-\theta)\dd\theta
\right\|
\\
&=
\left\|
\int_0^1
\left(
	\nabla^2\sigma(Y^\theta_{s,t})(Y_{s,t}^{\otimes 2}
	-
	\widetilde{Y}_{s,t}^{\otimes 2})
	+ 
	(\nabla^2\sigma(Y^\theta_{s,t})-\nabla^2\sigma(\widetilde{Y}^\theta_{s,t}))\widetilde{Y}_{s,t}^{\otimes 2}
\right)(1-\theta)\dd\theta
\right\|
\\
&=
\left\|
\int_0^1
\left(
	\nabla^2\sigma(Y^\theta_{s,t})(Y_{s,t}\otimes\Delta Y_{s,t}
	-
	\Delta Y_{s,t}\otimes\widetilde{Y}_{s,t})
	+ 
	(\nabla^2\sigma(Y^\theta_{s,t})-\nabla^2\sigma(\widetilde{Y}^\theta_{s,t}))\widetilde{Y}_{s,t}^{\otimes 2}
\right)(1-\theta)\dd\theta
\right\|
\\
&\leq
\|\sigma\|_{C^3_b}((\|Y_{s,t}\|+\|\widetilde{Y}_{s,t}\|)
\|\Delta Y_{s,t}\|
+
\|\Delta Y\|_\infty\|\widetilde{Y}_{s,t}\|^2).
\end{align*}

Second,
\begin{align*}
\|B_{s,t}\|
&=
\|(\nabla \sigma(s,Y_s)-\nabla \sigma(s,\widetilde{Y}_s))R^Y_{s,t}
+
\nabla \sigma(s,\widetilde{Y}_s)(R^Y_{s,t}-R^{\widetilde{Y}}_{s,t})
\| 
\leq 
\|\sigma\|_{C^2_b}(\|\Delta Y\|_\infty\|R_{s,t}^Y\|+\|\Delta R^Y_{s,t}\|).
\end{align*}

Third, by  \eqref{eq:deltaf_timevarying:st:2ndterm}, 
$
\| 
C_{s,t}
\|
\leq \|\sigma\|_{C_b^2}|t-s|\|\Delta Y\|_\infty.
$ 
Thus, by \eqref{eq:path_finite_var:sum_p/2vars} in Lemma \ref{lem:pvariation:inequalities},
\begin{align*}
\|R^{\sigma(\cdot,Y)}-R^{\sigma(\cdot,\widetilde{Y})}\|_\frac{p}{2}
&\leq
C_p(
\|A\|_\frac{p}{2}+\|B\|_\frac{p}{2}+\|C\|_\frac{p}{2}
)
\\
&\hspace{-3cm}\leq
C_p\|\sigma\|_{C^3_b}
\left(
(\|Y\|_p+\|\widetilde{Y}\|_p)\|\Delta Y\|_p+\|\Delta Y\|_\infty\|\widetilde{Y}\|^2_p
+
\|R^Y\|_\frac{p}{2}\|\Delta Y\|_\infty+\|\Delta R^Y\|_\frac{p}{2}
+
T\|\Delta Y\|_\infty
\right)
\\
&\hspace{-3cm}\leq
C_p\|\sigma\|_{C^3_b}
\left(
\|Y\|_p+\|\widetilde{Y}\|_p+\|\widetilde{Y}\|^2_p
+
\|R^Y\|_\frac{p}{2}
+
1
+
T
\right)
(\|\Delta Y_0\|+\|\Delta Y\|_p+\|\Delta R^Y\|_\frac{p}{2})
\\
&\hspace{-3cm}\leq
C_p\|\sigma\|_{C^3_b}
(
K_Y
+
K_{\widetilde{Y}}
+
1
)^2
(1+\|X\|_p+\|\tilde{X}\|_p)^2
(1+T)
(\|\Delta Y_0\|+\|\Delta Y\|_p+\|\Delta R^Y\|_\frac{p}{2}).
\end{align*}
where we used $\|Y\|_p\leq C_p(1+\|X\|_p)K_Y$ from \eqref{eq:|Y|_p_ineq} in the last inequality.  
Combining this inequality with  
$$\|\Delta Y\|_p
\mathop{\leq}^{\eqref{eq:controlled_path:Y-Ytilde_p}} 
C_p(\Delta M_{Y'}\|X\|_p+M_{\widetilde{Y}'}\|\Delta X\|_p+\|\Delta R^Y\|_\frac{p}{2})\leq 
C_p(1+K_{\widetilde{Y}})(\Delta M_{Y'}\|X\|_p+\|\Delta X\|_p+\|\Delta R^Y\|_\frac{p}{2}),$$
we obtain the desired inequality \eqref{eq:controlled_path:R^sigma(Y)-R^sigma(Ytilde)_p/2}.
\end{proof}

%\subsubsection*{Proof of Lemma \ref{lem:rough_path:intfdX:error_bounds}}
\begin{proof}[Proof of Lemma \ref{lem:rough_path:intfdX:error_bounds}]
First, by Lemma \ref{lem:rough_path:sigma(.,Y):controlled},  $(\sigma(\cdot,Y),\sigma(\cdot,Y)')=(\sigma(\cdot,Y),\frac{\partial\sigma}{\partial x}(\cdot,Y)Y')$ is a controlled path, so by Proposition \ref{prop:rough_integral_welldefined:error_bound}, the rough integral $\int_s^t \sigma(r,Y_r)\dd\mbX_r$ is well-defined. % 
Thus, for any $s,t\in[0,T]$, 
\begin{align*}
\left\|(R^{\int_0^\cdot \sigma(r,Y_r)\dd\mbX_r})_{s,t}\right\|
&=
\left\|
\int_s^t \sigma(r,Y_r)\dd\mbX_r-\sigma(s,Y_s)X_{s,t}
\right\|
\\
&\mathop{\leq}^{\eqref{eq:rough_int:error_bound}}
C_p(\|\sigma(\cdot,Y)'_s\bX_{s,t}\|+\|R^{\sigma(\cdot,Y)}\|_{\frac{p}{2},[s,t]}\|X\|_{p,[s,t]}+\|\sigma(\cdot,Y)'\|_{p,[s,t]}\|\bX\|_{\frac{p}{2},[s,t]})
\\
&\leq
C_p(\|\sigma\|_{C_b^1}\|Y'\|_\infty\|\bX\|_{\frac{p}{2},[s,t]}+\|R^{\sigma(\cdot,Y)}\|_{\frac{p}{2},[s,t]}\|X\|_{p,[0,T]}+\|\sigma(\cdot,Y)'\|_{p,[0,T]}\|\bX\|_{\frac{p}{2},[s,t]}).
\end{align*}
To continue, we need the following lemma, which is similar to \eqref{eq:path_finite_var:sum_p/2vars} in Lemma \ref{lem:pvariation:inequalities}.
\begin{lemma}\label{lem:sum_p/2vars:p/2vars_subintervals}
Let $p\geq 2$, $T>0$, $c\geq 0$, $X:[0,T]\to\R^d$, $Y^i,\widetilde{Y}^i\in\C^p$ for $i=1,\dots,n$, and $Z^j\in\C^\frac{p}{2}$ for $j=1,\dots,m$. % 
Then, there exists a constant $C_p\geq 1$ such that
\begin{equation}\label{eq:path_finite_var:sum_p/2vars:p/2vars_subintervals}
\begin{split}
&\|X_{s,t}\|\leq 
\sum_{i=1}^n\|Y^i\|_{p,[s,t]}\|\widetilde{Y}^i\|_{p,[s,t]}
+
\sum_{j=1}^m\|Z^j\|_{\frac{p}{2},[s,t]}+c|t-s|
\ \forall s,t\in[0,T]
\\[-2mm]
&\hspace{1cm}\implies 
\|X\|_\frac{p}{2}\leq C_p\bigg(
\sum_{i=1}^n\|Y^i\|_p\|\widetilde{Y}^i\|_p
+
\sum_{j=1}^m\|Z^j\|_\frac{p}{2}
+
c\,T
\bigg).
\end{split}
\end{equation}
\end{lemma}
\begin{proof}[Proof of Lemma \ref{lem:sum_p/2vars:p/2vars_subintervals}]
We prove the particular case $n=0$ (without $(Y^i,\widetilde{Y}^i)$) and $c=0$. The general case follows with minor modifications, see the proof of \eqref{eq:path_finite_var:sum_p/2vars} in Lemma \ref{lem:pvariation:inequalities}. 
Given any partition  $\pi$ of $[0,T]$,
 \begin{align*}
 \sum_{[s,t]\in\pi}\|X_{s,t}\|^\frac{p}{2}
 &\leq
 \sum_{[s,t]\in\pi}
 \Big(
 \sum_{j=1}^m\|Z^j\|_{\frac{p}{2},[s,t]} 
 \Big)^\frac{p}{2}
 \leq
 m^\frac{p}{2}
 \Big( 
 \sum_{j=1}^m
 \sum_{[s,t]\in\pi}
 \|Z^j\|_{\frac{p}{2},[s,t]}^\frac{p}{2}  
 \Big)
 \leq
 C_p 
 \sum_{j=1}^m
 \|Z^j\|_{\frac{p}{2}}^\frac{p}{2}   ,
 \end{align*}
% 
% 
% 
% 
% 
% 
% 
% 
% 
% 
% 
% 
% 
% 
% 
% 
% 
% 
% 
% 
% 
% 
% 
% 
% 
% 
% 
% 
 where we used $|\sum_{i=1}^Na_i|^\frac{p}{2}\leq N^\frac{p}{2}\sum_{i=1}^N|a_i|^\frac{p}{2}$ for $p\geq 2$ in the second inequality, and $\sum_{[s,t]\in\pi}
 \|Z^j\|_{\frac{p}{2},[s,t]}^\frac{p}{2}\leq  
 \|Z^j\|_{\frac{p}{2},[0,T]}^\frac{p}{2}$ in the third inequality, since $w(s,t)=\|Z^j\|_{\frac{p}{2},[s,t]}^\frac{p}{2}$ is a control, so that $\sum_{[s,t]\in\pi}w(s,t)\leq w(0,T)$.
% 
The bound above is independent of the choice of partition $\pi$. 
 Thus, \eqref{eq:path_finite_var:sum_p/2vars:p/2vars_subintervals} follows after taking the supremum over all partitions $\pi$ of $[0,T]$ and using the inequality $(\sum_i|a_i|)^\frac{2}{p}\leq\sum_i|a_i|^\frac{2}{p}$ for $p\geq 2$.  
\end{proof}
\textit{Proof of Lemma \ref{lem:rough_path:intfdX:error_bounds} (continued).}
Thus, by Lemma \ref{lem:sum_p/2vars:p/2vars_subintervals}, 
\begin{align*}
\|R^{\int_0^\cdot \sigma(s,Y_s)\dd\mbX_s}\|_\frac{p}{2}
&\leq
C_p(\|\sigma\|_{C_b^1}\|Y'\|_\infty\|\bX\|_\frac{p}{2}+\|R^{\sigma(\cdot,Y)}\|_\frac{p}{2}\|X\|_p+\|\sigma(\cdot,Y)'\|_p\|\bX\|_\frac{p}{2})
\\
&\hspace{-2cm}\leq
C_p\|\sigma\|_{C_b^2}
\big(
K_Y
\big(
\|\bX\|_\frac{p}{2}
+
(1+K_Y)(1+\|X\|_p)^2
(
\|X\|_p+
\|\bX\|_\frac{p}{2}
)
\big)
+
T(\|X\|_p+K_Y(1+\|X\|_p)\|\bX\|_\frac{p}{2})
\big)
\\
&\leq 
C_p\|\sigma\|_{C_b^2}(1+K_Y)^2(1+\|X\|_p)^2(1+T)(
	\|\bX\|_\frac{p}{2}+\|X\|_p),
\end{align*}
where we used $\|Y'\|_\infty\leq\|Y'_0\|+\|Y'\|_p=M_{Y'}\leq K_Y$, and the inequalities  from Lemma \ref{lem:rough_path:sigma(.,Y):controlled} 
$\|\sigma(\cdot,Y_\cdot)'\|_p
\leq 
C_p
\|\sigma\|_{C^2_b}K_Y(1+K_Y+T)
(1+\|X\|_p)$ 
and 
$\|R^{\sigma(\cdot,Y_\cdot)}\|_\frac{p}{2}
\leq
C_p
\|\sigma\|_{C^2_b}
(
K_Y(1+K_Y)(1+\|X\|_p)^2
+
T
)$. This concludes the proof of \eqref{eq:rough_path:R^int_sig_dX:p/2}. 
Note also that $\|
\int_0^\cdot \sigma(s,Y_r)\dd\mbX_s
\|_p<\infty$, since
{\small
\begin{align*}
\Big\|
\int_s^t \sigma(r,Y_r)\dd\mbX_r
\Big\|
&\leq
\Big\|
\int_s^t \sigma(r,Y_r)\dd\mbX_r-\sigma(s,Y_s)X_{s,t}
\Big\|
+
\|\sigma(s,Y_s)X_{s,t}\|
=
\big\|
R^{\int_0^\cdot \sigma(r,Y_r)\dd\mbX_r}_{s,t}
\big\|
+
\|\sigma(s,Y_s)X_{s,t}\|.
\end{align*}
}% 
Also, $\|\sigma(\cdot,Y)\|_p<\infty$ by \eqref{eq:sigma(.,Y):pvar}. Together, $\|
\int_0^\cdot \sigma(s,Y_s)\dd\mbX_s
\|_p,\|\sigma(\cdot,Y)\|_p,\|R^{\int_0^\cdot \sigma(s,Y_s)\dd\mbX_s}\|_\frac{p}{2}<\infty$ imply that $(Z,Z')\in\sD^p_X$, which concludes the proof. 
\end{proof}



%\subsubsection*{Proof of Lemma \ref{lem:rough_path:intfdX_fY:error_bounds}}
\begin{proof}[Proof of Lemma \ref{lem:rough_path:intfdX_fY:error_bounds}]
To show \eqref{eq:rough_path:intfdX_fY:delta_sigma},
we  combine $\|\sigma(\cdot,Y)-\sigma(\cdot,\widetilde{Y})\|_p\leq C_p\|\sigma\|_{C^2_b}(1 + \|Y\|_p+\|\widetilde{Y}\|_p+T)(\|\Delta Y_0\|+\|\Delta Y\|_p)$ in \eqref{eq:Delta_sigma(.,X):pvar},  
$\|Y\|_p\leq  C_p(1+\|X\|_p)K_Y$  in \eqref{eq:|Y|_p_ineq}, 
  $\|\Delta Y\|_p
 \leq 
C_p\big(\Delta M_{Y'}\|X\|_p+M_{\widetilde{Y}'}\|\Delta X\|_p+\|\Delta R^Y\|_\frac{p}{2}\big)$ in \eqref{eq:controlled_path:Y-Ytilde_p}, 
and  $M_{\widetilde{Y}'}\leq K_{\widetilde{Y}}$, so that
\begin{align*}
\|\sigma(\cdot,Y)-\sigma(\cdot,\widetilde{Y})\|_p
&\mathop{\leq}^{\eqref{eq:Delta_sigma(.,X):pvar}} 
C_p\|\sigma\|_{C^2_b}(1 + \|Y\|_p+\|\widetilde{Y}\|_p+T)
(\|\Delta Y_0\|+\|\Delta Y\|_p)
\\
&\hspace{-3cm}\mathop{\leq}^{\eqref{eq:|Y|_p_ineq},\eqref{eq:controlled_path:Y-Ytilde_p}} 
C_p\|\sigma\|_{C^2_b}
(1+\|X\|_p+\|\tilde{X}\|_p+T)
(1 + K_Y+K_{\widetilde{Y}})
(\|\Delta Y_0\|+\Delta M_{Y'}\|X\|_p+M_{\widetilde{Y}'}\|\Delta X\|_p+\|\Delta R^Y\|_\frac{p}{2})
\\
&\hspace{-30mm}\leq 
C_p\|\sigma\|_{C^2_b}(1\,{+}\,\|X\|_p\,{+}\,\|\tilde{X}\|_p\,{+}\,T)(1 \,{+}\, K_Y\,{+}\,K_{\widetilde{Y}})^2
(\|\Delta Y_0\|\,{+}\,
	(\|\Delta Y_0'\|\,{+}\,\|\Delta Y'\|_p)\|X\|_p\,{+}\,
	\|\Delta X\|_p\,{+}\,\|\Delta R^Y\|_\frac{p}{2})
\end{align*}
which is the desired inequality \eqref{eq:rough_path:intfdX_fY:delta_sigma}. 

To show \eqref{eq:rough_path:intfdX_fY:delta_R}, we observe from \eqref{lem:rough_int_stability:remainder} in Lemma \ref{lem:rough_int_stability} that it suffices to  bound  $K_{\sigma(\cdot,\widetilde{Y})}$ and $\Delta K_{\sigma(\cdot,Y)}$.  
 % 
First,  we bound $K_{\sigma(\cdot,\widetilde{Y})}$ as follows. By \eqref{eq:sigma(.,Y)':pvar}
and \eqref{eq:RY:p/2var:KYs} in Lemma \ref{lem:rough_path:sigma(.,Y):controlled}, $
\|\sigma(\cdot,Y)'\|_p\leq 
C_p\|\sigma\|_{C^2_b}K_Y(1+K_Y+T)(1+\|X\|_p)$ and $\|R^{\sigma(\cdot,Y)}\|_\frac{p}{2}\leq
C_p\|\sigma\|_{C^2_b}(K_Y(1+
K_Y)(1+\|X\|_p)^2+T)$. Also, $\|\sigma(\cdot,\widetilde{Y})'_0\|\leq\|\sigma\|_{C_b^2}\|\widetilde{Y}_0'\|\leq\|\sigma\|_{C_b^2}K_{\widetilde{Y}}$. By combining this inequalities, 
\begin{align}
\nonumber
K_{\sigma(\cdot,\widetilde{Y})}
&= \|\sigma(\cdot,\widetilde{Y})'_0\|+\|\sigma(\cdot,\widetilde{Y})'\|_p+\|R^{\sigma(\cdot,\widetilde{Y})}\|_\frac{p}{2}
\leq
C_p\|\sigma\|_{C^2_b}(1+
K_{\widetilde{Y}}+T)^2(1+\|\tilde{X}\|_p)^2
\\
\label{eq:bound:Ksigma(Y)}
&\leq
C_p\|\sigma\|_{C^2_b}
(1+K_Y+K_{\widetilde{Y}}+T)^3
(1+\|X\|_p+\|\tilde{X}\|_p)^3
(1+T).
\end{align}
Second,  using \eqref{eq:controlled_path:sigma(Y)-sigma(Ytilde)_p} and 
\eqref{eq:controlled_path:R^sigma(Y)-R^sigma(Ytilde)_p/2} in Lemma \ref{lem:rough_path:f(Y)'-f(tilde(Y))'}, we have
\begin{align*}
\|\sigma(\cdot,Y)'-\sigma(\cdot,\widetilde{Y})'\|_p
&\leq
C_p\|\sigma\|_{C^3_b}
(1+K_Y+K_{\widetilde{Y}'})^3(1+\|X\|_p+\|\tilde{X}\|_p)^3
(1+T)
\big(
\\
&\qquad\qquad\qquad
\|\Delta X\|_p+
\|\Delta Y_0\|
+\|\Delta R^Y\|_\frac{p}{2}
+
\|\Delta Y'_0\|
+
\|\Delta Y'\|_p
\big),
\\
\|R^{\sigma(\cdot,Y)}-R^{\sigma(\cdot,\widetilde{Y})}\|_\frac{p}{2}
&\leq 
C_p\|\sigma\|_{C^3_b}(1+K_Y+K_{\widetilde{Y}})^3
	(1+\|X\|_p+\|\tilde{X}\|_p)^3
(1+T)
\big(
\\
&\qquad\qquad\qquad
\|\Delta Y_0\|+\|\Delta R^Y\|_\frac{p}{2}+(\|\Delta Y_0'\|+\|\Delta Y'\|_p)\|X\|_p+\|\Delta X\|_p\big).
\end{align*}
By combining these two inequalities with $\|\Delta \sigma(\cdot,Y)'_0\|\leq\|\sigma\|_{C_b^2}(1+M_{Y'})(\|\Delta Y_0\|+\|\Delta Y_0'\|)$, we get
\begin{align}
\Delta K_{\sigma(\cdot,Y)} &=\|\Delta \sigma(\cdot,Y)'_0\|+\|\Delta \sigma(\cdot,Y)'\|_p+\|\Delta R^{\sigma(\cdot,Y)}\|_\frac{p}{2}
\nonumber
\\
&\leq
C_p\|\sigma\|_{C^3_b}(1+K_Y+K_{\widetilde{Y}})^3
	(1+\|X\|_p+\|\tilde{X}\|_p)^3
	(1+T)
\big(
\nonumber
\\
&\hspace{2cm}
\|\Delta Y_0\|+\|\Delta Y_0'\|+\|\Delta R^Y\|_\frac{p}{2}+\|\Delta Y'\|_p+(\|\Delta Y_0'\|+\|\Delta Y'\|_p)\|X\|_p+\|\Delta X\|_p\big).
\label{eq:bound:DeltaKsigma(Y)}
\end{align}
Finally, combining the last inequalities \eqref{eq:bound:Ksigma(Y)} and \eqref{eq:bound:DeltaKsigma(Y)} using  \eqref{lem:rough_int_stability:remainder} in  Lemma \ref{lem:rough_int_stability}, we get
\begin{align*}
\|R^{\int_0^\cdot \sigma(s,Y_s)\dd\mbX_s}-R^{\int_0^\cdot \sigma(s,\widetilde{Y}_s)\dd\widetilde{\mbX}_s}\|_{\frac{p}{2}}
&\mathop{\leq}^{\eqref{lem:rough_int_stability:remainder} }
C_p
(1+\|\mbX\|_p+\|\widetilde{\mbX}\|_p)
\big(K_{\sigma(\cdot,\widetilde{Y})}\|\Delta\mbX\|_p+\|\mbX\|_p\Delta K_{\sigma(\cdot,Y)}\big)
\\
&\hspace{-5.1cm}\mathop{\leq}^{\eqref{eq:bound:Ksigma(Y)},\eqref{eq:bound:DeltaKsigma(Y)}}
C_p\|\sigma\|_{C_b^3}
(
1+\|\mbX\|_p+\|\widetilde{\mbX}\|_p)^4
(1+K_Y+K_{\widetilde{Y}}+T)^3
(1+T)
\big(
\\
& 
\hspace{-34mm}
\|\Delta\mbX\|_p+
\|\mbX\|_p(\|\Delta Y_0\|+\|\Delta Y_0'\|+\|\Delta R^Y\|_\frac{p}{2}+\|\Delta Y'\|_p+(\|\Delta Y_0'\|+\|\Delta Y'\|_p)\|X\|_p+\|\Delta X\|_p)\big),
\end{align*}
which is the desired inequality \eqref{eq:rough_path:intfdX_fY:delta_R}.
\end{proof}

































































 
\subsubsection{Rough differential equations: existence and unicity  of solutions (Section \ref{sec:rdes:existence_unicity})}

\subsubsection*{Nonlinear rough differential equations (Section \ref{sec:rdes:existence_unicity:nonlinear})}

%\subsubsection*{Proof of Theorem \ref{thm:rdes:existence_unicity}}
\begin{proof}[Proof of Theorem \ref{thm:rdes:existence_unicity}]
% 
% 
First, for any $t\in(0,T]$, we define the map
$$
\M_t:\ \sD^p_X([0,t],\R^n)\to\sD^p_X([0,t],\R^n),\ 
(Y,Y')\mapsto
\left(y+\int_0^\cdot b(s,Y_s,u_s)\dd s+\int_0^\cdot\sigma(s,Y_s)\dd\mbX_s,\sigma(\cdot,Y)\right).
$$
% 
% 
% 
For $\delta\geq 1$, we define 
the ball
$$
\B_t^{(\delta)}=\left\{
(Y,Y')\in\sD^p_X([0,t],\R^n): \, Y_0=y, \, Y'_0=\sigma(0,y), 
\ 
\|Y,Y'\|_{X,p,[0,t]}^{(\delta)}\leq 1
\right\}
\subset\sD^p_X.
$$
with
\begin{equation}\label{eq:controlled_paths:norm}
\|Y,Y'\|_{X,p,[0,t]}^{(\delta)}=\|Y'\|_{p,[0,t]}+\delta\|R^Y\|_{\frac{p}{2},[0,t]}.
\end{equation}
The set $\B_t^{(\delta)}$ is a closed subset of the Banach space $\sD^p_X$ (when equiped with the metric induced by the norm $\|\cdot,\cdot\|_{X,p}^\delta$, and when we restrict  $\sD^p_X$ to controlled paths with fixed initial condition $(Y_0,Y_0')=(y,\sigma(0,y))$), so 
$\B_t^{(\delta)}$ is itself a complete metric space.  
Also, $\B_t^{(\delta)}$ is nonempty, since $s\mapsto(y+\sigma(0,y)X_{0,s},\sigma(0,y))\in\B_t^{(\delta)}$.  






\textbf{Invariance.} We claim that $\M_{t_1}:\B_{t_1}^{(\delta)}\to\B_{t_1}^{(\delta)}$ for $t_1>0$ small enough, i.e., that $\B_{t_1}^{(\delta)}$ is invariant under $\M_{t_1}$. Let $(Y,Y')\in\B_{t_1}^{(\delta)}$. 
Since $\|Y,Y'\|_{X,p,[0,t]}^{(\delta)}\leq 1$,  $K_Y= \|Y_0'\|+\|Y'\|_{p,[0,t]}+\|R^Y\|_{\frac{p}{2},[0,t]}$ and $K_{\widetilde{Y}}$ satisfy $K_Y,K_{\widetilde{Y}}
\leq \|\sigma\|_{C_b^0}+1+\frac{1}{\delta}$. Thus, for $\delta \geq 1$, there exists $C_\sigma\geq 0$ such that
\begin{align}\label{eq:contraction:KY}
K_Y,K_{\widetilde{Y}}\leq \tilde{C}_\sigma(1+\delta^{-1})
\leq C_\sigma,
\   
(1+K_Y+K_{\widetilde{Y}})^2\leq C_\sigma,
\   
(1+K_Y+K_{\widetilde{Y}})^3\leq  C_\sigma.
\end{align}
Next, 
$
R^{y+\int_0^\cdot b(s,Y_s,u_s)\dd s+\int_0^\cdot\sigma(s,Y_s)\dd\mbX_s}_{s,t}
=
\int_s^t b(r,Y_r,u_r)\dd s+\int_s^t\sigma(r,Y_r)\dd\mbX_r
-
\sigma(s,Y_s)X_{s,t}
=
\int_s^t b(r,Y_r,u_r)\dd r
+
R^{\int_0^\cdot\sigma(s,Y_s)\dd\mbX_s}_{s,t}$, so
\begin{align*}
\|\M_{t_1}(Y,Y')\|_{X,p}^{(\delta)}
&=
\|\sigma(\cdot,Y)\|_p+\delta\|R^{y+\int_0^\cdot b(s,Y_s,u_s)\dd s+\int_0^\cdot\sigma(s,Y_s)\dd\mbX_s}\|_{\frac{p}{2}}
\\
&
\hspace{-26mm}\leq
\|\sigma(\cdot,Y)\|_p+C_p\delta\bigg(
\left\|\int_0^\cdot b(s,Y_s,u_s)\dd s\right\|_\frac{p}{2} +
\left\|R^{\int_0^\cdot\sigma(s,Y_s)\dd\mbX_s}\right\|_\frac{p}{2}
\bigg)
&&
\eqref{eq:path_finite_var:sum_p/2vars}
% 
\\
&\hspace{-26mm}\leq\ 
C_p\big(
\|\sigma\|_{C^1_b}
(\|Y\|_p+t_1)
+ 
\delta
(
% 
 C_{p,b}t_1
+
C_p\|\sigma\|_{C_b^2}(1+K_Y)^2(1+\|X\|_p)^2
(1+t_1)\|\mbX\|_p
)
\big)
&&
\eqref{eq:sigma(.,X):pvar},
\eqref{eq:bounds_int_b_ds},
\eqref{eq:rough_path:R^int_sig_dX:p/2}
\\
&
\hspace{-26mm}\leq
C_p
\big(\|\sigma\|_{C^1_b}(\|R^Y\|_\frac{p}{2}+K_Y\|X\|_p+t_1)
+
\delta C_{p,b,\sigma,\|X\|_p,T}
(1+K_Y)^2 
\left(
t_1
+
\|\mbX\|_p
\right)
\big)
&&
\eqref{eq:controlled_path:pvar_norm},t_1\leq T
\\
&\hspace{-26mm}\leq 
C_1
\Big(
\frac{1}{\delta}
+
\|X\|_p
 +t_1+\delta
  \left(
t_1
+
\|\mbX\|_p
\right)
\Big),
&&\hspace{-8mm}\eqref{eq:contraction:KY},\|Y,Y'\|^{(\delta)}_{X,p}\leq 1
\end{align*} 
for a constant $C_1:=C_{p,b,\sigma,\|\mbX\|_p,T}>\frac{1}{2}$. 
% 
Let $\delta=\delta_1:=2C_1\geq 1$. Then, 
$$
\|\M_{t_1}(Y,Y')\|_{X,p}^{(\delta_1)}
\leq 
\frac{1}{2}
+
C_1\left(
\|X\|_{p,[0,t_1]}+t_1+2C_1(t_1+\|\mbX\|_{p,[0,t_1]})
\right).
$$
Then, by taking $t_1$ small-enough,  $\|\M_{t_1}(Y,Y')\|_{X,p}^{(\delta_1)}\leq 1$, so $\M_{t_1}(Y,Y')\in\B_{t_1}^{(\delta_1)}$. Invariance is proved. 

\textbf{Contraction.}  Let $(Y,Y'),(\widetilde{Y},\widetilde{Y}')\in\B_t^{(\delta_1)}$ for some $t\in(0,t_1]$. For any $\delta\geq 1$, by \eqref{eq:path_finite_var:sum_p/2vars},
\begin{align*}
\|\M_t(Y,Y')-\M_t(\widetilde{Y},\widetilde{Y}')\|_{X,p}^{(\delta)}
&\leq 
\\
&\hspace{-37mm}
\|\sigma(\cdot,Y)-\sigma(\cdot,\widetilde{Y})\|_p+
C_p\delta\bigg(
	\left\|\int_0^\cdot (b(s,Y_s,u_s) - b(s,\widetilde{Y}_s,u_s))\dd s\right\|_\frac{p}{2}
	+
\left\|R^{\int_0^\cdot\sigma(s,Y_s)\dd\mbX_s}-R^{\int_0^\cdot\sigma(s,\widetilde{Y}_s)\dd\mbX_s}\right\|_\frac{p}{2}
	\bigg).
\end{align*}
Then, %with $\Delta \widetilde{R}:=R^{\int_0^\cdot \sigma(s,Y_s)\dd\mbX_s}-R^{\int_0^\cdot \sigma(s,\widetilde{Y}_s)\dd\mbX_s}$ and $\int_0^\cdot \Delta b_s\dd s:=\int_0^\cdot (b(s,Y_s,u_s)-b(s,\widetilde{Y}_s,u_s))\dd s$, and 
since $(\Delta X,\Delta Y_0,\Delta Y'_0)=(0,0,0)$, 
\begin{align*}
\|\sigma(\cdot,Y)-\sigma(\cdot,\widetilde{Y})\|_p
&\mathop{\leq}^{\eqref{eq:rough_path:intfdX_fY:delta_sigma}}
C_p\|\sigma\|_{C^2_b}(1 + K_Y+K_{\widetilde{Y}})^2(1+\|X\|_p+t)(
\|\Delta Y'\|_p\|X\|_p+\|\Delta R^Y\|_\frac{p}{2})
\\
&\mathop{\leq}^{\eqref{eq:contraction:KY}}
C_{p,\sigma,\|X\|_p,T} 
(\|\Delta Y'\|_p\|X\|_p+\|\Delta R^Y\|_\frac{p}{2}),
\\
\|R^{\int_0^\cdot \sigma(s,Y_s)\dd\mbX_s}-R^{\int_0^\cdot \sigma(s,\widetilde{Y}_s)\dd\mbX_s}\|_{\frac{p}{2}}
&\mathop{\leq}^{\eqref{eq:rough_path:intfdX_fY:delta_R}} 
C_p\|\sigma\|_{C_b^3}
(1+K_Y+K_{\widetilde{Y}}+t)^3\big(
\\
&\hspace{3cm}
(1+\|\mbX\|_p)^5(1+t)
\|\mbX\|_p(\|\Delta Y'\|_p
+
\|\Delta R^Y\|_\frac{p}{2})
\big)
\\
&\mathop{\leq}^{\eqref{eq:contraction:KY}}
C_{p,\sigma,\|\mbX\|_p,T}
\|\mbX\|_p
(\|\Delta Y'\|_p+\|\Delta R^Y\|_\frac{p}{2}),
\\
\left\|\int_0^\cdot (b(s,Y_s,u_s)-b(s,\widetilde{Y}_s,u_s))\dd s\right\|_\frac{p}{2} 
&\mathop{\leq}^{\eqref{eq:bounds_int_b_ds}} 
C_{p,b}t\|\Delta Y\|_\infty
\mathop{\leq}^{\eqref{eq:path_finite_var:infty_ineq},\eqref{eq:controlled_path:Y-Ytilde_p}}
C_{p,b,\|X\|_p}t(\|\Delta Y'\|_p+\|\Delta R^Y\|_\frac{p}{2}).
\end{align*}
Thus, for $\delta\geq1$ and a constant $C_2:=C_{p,b,\sigma,\|\mbX\|_p,T}>\frac{1}{2}$, we obtain 
\begin{align*}
\|\M_t(Y,Y')-\M_t(\widetilde{Y},\widetilde{Y}')\|_{X,p}^{(\delta)}
&\leq 
C_2
\left(
\|\Delta Y'\|_p\|X\|_p+\|\Delta R^Y\|_\frac{p}{2}
+
\delta(\|\mbX\|_p+t)
\big(
\|\Delta Y'\|_p+\|\Delta R^Y\|_\frac{p}{2}
\big)
\right).
\end{align*}
Next, choose $\delta=\delta_2:=2C_2>1$, so that $C_2=\frac{\delta}{2}$, and choose $t=t_2\leq t_1$ small-enough so that $C_2\delta(\|\mbX\|_p+t)\leq \frac{1}{2}$, so that in particular $C_2\|X\|_p\leq\frac{1}{2\delta}$. Then, 
\begin{align*}
\|\M_{t_2}(Y,Y')-\M_{t_2}(\widetilde{Y},\widetilde{Y}')\|_{X,p}^{(\delta_2)}
&\leq 
\frac{1}{2\delta_2}
\|\Delta Y'\|_p+\frac{\delta_2}{2}\|\Delta R^Y\|_\frac{p}{2}
+
\frac{1}{2}\left(
\|\Delta Y'\|_p+\|\Delta R^Y\|_\frac{p}{2}
\right)
\\
&=
\frac{1+\delta_2}{2\delta_2}
\left(
\|\Delta Y'\|_p+\delta_2\|\Delta R^Y\|_\frac{p}{2}
\right).
\end{align*}
Since $\delta_2>1$, we obtain  
$\|\M_{t_2}(Y,Y')-\M_{t_2}(\widetilde{Y},\widetilde{Y}')\|_{X,p}^{(\delta_2)}<\|\Delta Y'\|_p+\delta_2\|\Delta R^Y\|_\frac{p}{2}=\|Y-\widetilde{Y},Y'-\widetilde{Y}'\|_{X,p}^{(\delta_2)}$, so $\M_{t_2}$ is a contraction.

 To conclude, $\M_{t_2}:\B_{t_2}^{(\delta_2)}\to\B_{t_2}^{(\delta_2)}$ is invariant and a contraction. Thus, there exists a unique fixed point $(Y,Y')\in\sD^p_X([0,t_2],\R^n)$ of the map $\M_{t_2}$, which is the solution to the RDE \eqref{eq:RDE} satisfying $Y'=\sigma(\cdot,Y)$ over the time interval $[0,t_2]$.  Since $t_2$ was chosen independently of the initial conditions, we can stitch together solutions over time intervals $[kt_2,(k+1)t_2]_{k\in\N}$, and deduce that the RDE \eqref{eq:RDE} admits a unique solution over the entire interval $[0,T]$.
\end{proof}

\subsubsection*{Linear rough differential equations (Section \ref{sec:rdes:existence_unicity:linear})}
In this section, we prove existence and unicity of solutions to linear RDEs  with drift \eqref{eq:rde_linear} in Theorem \ref{thm:rde:linear:existence_uniqueness}. 
The proof consists of rewriting the linear RDE with drift \eqref{eq:rde_linear} as a driftless linear RDE with constant coefficients driven by a new geometric rough path by also interpreting the Lebesgue integral $\int A_sV_s\dd s$  as a rough integral, and concluding with \cite[Theorem 10.53]{Friz2010}.  These results %in this section 
may be considered standard, although we could not find them     in the literature.

 
\subsubsection*{Preliminaries for the proof of Theorem \ref{thm:rde:linear:existence_uniqueness}}
% 


% 
% 
% 
% 
% 
% 
% 
% 
% 
\begin{lemma}[Connections between the Lebesgue 
and rough integrals]\label{lem:rough_integral_lebesgue}
Let $p\in[2,3)$, $T>0$, 
$b\in L^\infty([0,T],\R)$, 
$\mbX\in\sC^p([0,T],\R^d)$, % 
and $(Y,Y')\in\sD^p_X([0,T],\R^{1\times d})$. % 
\begin{itemize} 
\item[1)]  
% 
% 
Define $\cT:[0,T]\to\R$ and $\bT:\Delta_{[0,T]}\to\R$   by  the Lebesgue integrals
$$
\cT_t:=
\int_0^t
b_r\dd r,
\quad
\bT_{s,t}:=
\int_s^t\cT_{s,r}b_r\dd r=
\int_s^t\int_s^rb_v\dd v\,b_r\dd r.
$$
Then, $\mbT=(\cT,\bT)\in\sC^1_g([0,T],\R)$, i.e., $\mbT$ is a geometric $1$-rough path. 

Moreover, for any Gubinelli derivative $\hat{Y}'\in \C^p([0,T],\R^{n\times d\times d})$ 
% 
 (in particular, for $\hat{Y}'=0$), the rough integral of $(Y,\hat{Y}')$ % 
against $\mbT$, defined as the limit in \eqref{eq:rough_int}, 
is well-defined, and is equal to the Lebesgue integral $\int Yb\dd t$: for any $t\in[0,T]$,
\begin{equation}\label{eq:rough_integral_lebesgue:dT_bdt}
\int_0^tY_r\dd\mbT_r
= 
\int_0^tY_rb_r\dd r.
\end{equation}
\item[2)] Define $Z:[0,T]\to\R$ and $\bZ:\Delta_{[0,T]}\to\R$  by  the rough integrals
$$
Z_t:=
\int_0^t
Y_r\dd\mbX_r,
\quad
\bZ_{s,t}
% 
% 
:=
\int_s^tZ_{s,r}
Y_r
\dd\mbX_r.
$$
Then, $\mbZ=(Z,\bZ)\in\sC^p([0,T],\R)$, i.e., $\mbZ$ is a $p$-rough path.

Moreover, $\mbZ$ is geometric if  $\mbX$ is geometric, i.e., $\mbX\in\sC^p_g\implies \mbZ\in\sC^p_g$. 

% 
% 
% 
% 
% 
% 
% 
% 
% 
% 
% 
% 
% 
% 
% 
% 
\end{itemize}
\end{lemma}
The pair $(Z,Z'):=(\int_0^\cdot
Y\dd\mbX, Y)$ is a controlled path by Lemma \ref{lem:rough_int_stability}. The rough path $\mbZ$ as defined in Lemma \ref{lem:rough_integral_lebesgue} is called  the canonical rough path lift of $(Z,Z')$, see \cite[Sections 6.1-6.2]{Allan2021}.  
% 
Also, the equality $
\int Yb\dd t
=
\int Y\dd\mbT
$ in \eqref{eq:rough_integral_lebesgue:dT_bdt} (where, on the right hand side,  the Gubinelli derivative $\hat{Y}$  of $Y$ can be defined arbitrarily since $\mbT$ is smooth-enough for $\int Y\dd\mbT$ to be equivalent to a Young integral) will allow us to rewrite the linear RDE with drift \eqref{eq:rde_linear}  as a driftless linear RDE driven by a new rough path. 
\begin{proof}[Proof of Lemma \ref{lem:rough_integral_lebesgue}] 
1) First, 
$
\left|\cT_{s,t}\right|
=
\left|\int_s^tb_r\dd r\right|
\leq 
\left|\int_s^t\|b\|_\infty\dd r\right|
=\|b\|_\infty|t-s|,
$
so $\|\cT\|_1\leq\|b\|_\infty T<\infty$, and 
{\small
\begin{equation}\label{eq:bT_st<=1/2|Z|_infty(t-s)^2}
|\bT_{s,t}|=\left|\int_s^t\int_s^rb_v\dd vb_r\dd r\right|
\leq
\left|\int_s^t\left\|\int_s^\cdot b_v\dd v\right\|_{\infty,[s,r]} b_r\dd r\right|
\leq
\|b\|_\infty
\left|\int_s^t(r-s) b_r\dd r\right|
\leq
\frac{1}{2}\|b\|_\infty^2(t-s)^2,
\end{equation}
}% 
so $\|\bT\|_\frac{1}{2}\leq\|b\|_\infty^2T^2/2<\infty$, and $\|\mbT\|_1<\infty$. Moreover, 
 for any $0\leq s\leq r\leq t\leq T$, 
 {\small
\begin{align*}
\bT_{s,r}
&=
\int_s^r\int_s^ub_v\dd vb_u\dd u
=
\int_s^t\int_s^ub_v\dd vb_u\dd u-
\int_r^t\int_s^ub_v\dd vb_u\dd u
=
\bT_{s,t}-
\left(
\int_r^t
\left(
\int_s^rb_v\dd vb_u\dd u
+
\int_r^ub_v\dd vb_u\dd u
\right)
\right)
\\
&=
\bT_{s,t}-
\left(
\int_s^rb_v\dd v
\int_r^tb_u\dd u
+
\int_r^t
\int_r^ub_v\dd vb_u\dd u
\right)
=
\bT_{s,t}-\bT_{r,t}-\cT_{s,r}\cT_{r,t},
\end{align*}
}% 
so $\mbT$ satisfies Chen's relation \eqref{eq:chen's_relation}. The  condition  in  \eqref{eq:rough_path:integration_by_parts} follows the integration by parts formula
\begin{align*}
% 
% 
\bT_{s,t}
&=
\int_s^t\int_s^rb_v\dd vb_r\dd r
=
% 
% 
\left[
\int_s^\cdot b_v\dd v
\int_s^\cdot b_r\dd r
\right]_s^t
-
\int_s^tb_v\int_s^vb_r\dd r\dd v
=
\cT_{s,t}^2-\bT_{s,t}
\implies\bT_{s,t}=
\frac{1}{2}\cT_{s,t}^2,
\end{align*} 
so we  conclude that $\mbT$ is a geometric $1$-rough path. 

Moreover,  
% 
% 
given any $\hat{Y}'\in \C^p([0,T],\R^{n\times d\times d})$   and any partition $\pi$ of $[0,T]$,
{\small
\begin{equation}\label{eq:smooth_rough_intgral_no_effect}
\Big\|
\sum_{[s,t]\in\pi}\hat{Y}'_s\bT_{s,t}
\Big\|
\mathop{\leq}^{\eqref{eq:bT_st<=1/2|Z|_infty(t-s)^2}}
\frac{1}{2}\|b\|_\infty^2\|\hat{Y}'\|_\infty
\sum_{[s,t]\in\pi}(t-s)^2
\leq
\frac{1}{2}\|b\|_\infty^2\|\hat{Y}'\|_\infty\sup_{[s,t]\in\pi}|t-s|
\sum_{[s,t]\in\pi}|t-s|
\leq
\frac{1}{2}\|b\|_\infty^2\|\hat{Y}'\|_\infty T|\pi|\to 0
\end{equation}
}% 
 as $|\pi|\to0$.  
Thus, 
$$
\int_0^TY_t\dd\mbT_t
\mathop{=}^{\eqref{eq:rough_int}}
\lim_{|\pi|\to0}\sum_{[s,t]\in\pi}Y_s\cT_{s,t}+
\hat{Y}'_s\bT_{s,t}
=
\lim_{|\pi|\to0}\sum_{[s,t]\in\pi}Y_s\cT_{s,t}
=
\int_0^TY_t\dd\cT_t,
$$
% 
% 
% 
% 
% 
% 
% 
% 
% 
% 
% 
% 
% 
% 
% 
% 
where the last integral is a well-defined Young integral since $Y\in\C^p$ and $\cT\in\C^1$ with $\frac{1}{p}+1>1$ \cite[Theorem 6.8]{Friz2010}. 
Thus, the rough integral of $(Y,\hat{Y}')$ against $\mbT$ is well-defined.   
The last integral is  also the Lebesgue integral of $Y$ with respect to  $\cT$. Finally, since
$\cT_t=\cT_s+\int_s^tb_r\dd r$,   $\cT$ is absolutely continuous with respect to the Lebesgue measure, so
$\int_0^TY_t\dd\cT_t
=
\int_0^TY_tb_t\dd t$, 
and we obtain the desired result \eqref{eq:rough_integral_lebesgue:dT_bdt}.



2) Second, $(Z,Z'):=(\int_0^\cdot
Y_s\dd\mbX_s, Y)$ is a controlled path by Lemma \ref{lem:rough_int_stability}. 
$\bZ$ is also well-defined, since $ZY$ is a  controlled path  by Lemma \ref{lem:control_path:product}, and $\|\bZ\|_{\frac{p}{2}}<\infty$ thanks to \eqref{eq:rough_int:error_bound}. 
% 
% 
% 
% 
% 
% 
% 
% 
% 
% 
% 
% 
% 
% 
% 
The pair $\mbZ=(Z,\bZ)$ satisfies Chen's relation \eqref{eq:chen's_relation}, which can be shown via identical computations as for the proof that $\mbT$ satisfies Chen's relation. Thus, $\mbZ$ is a $p$-rough path ($\mbZ\in\sC^p$). 

Moreover, the statement that  $\mbZ$ is geometric if and only if $\mbX$ is geometric 
is a consequence of $[\mbZ]=\int_0^\cdot(Y_r\otimes Y_r)\dd[\mbX]_r$ from \cite[Lemma 6.8]{Allan2021}, where $[\mbZ]_t:=Z_{0,t}^2-2\bZ_{0,t}$  and  $[\mbX]^{ij}_t:=X^i_{0,t} X^j_{0,t}-(\bX^{ij}_{0,t}+\bX^{ji}_{0,t})$ denote the brackets of $\mbZ$ and of $\mbX$ and satisfy $[\mbZ]_{s,t}=Z_{s,t}^2-2\bZ_{s,t}$ and $[\mbX]^{ij}_{s,t}=X^i_{s,t} X^j_{s,t}-(\bX^{ij}_{s,t}+\bX^{ji}_{s,t})$ \cite[Lemma 6.5]{Allan2021},
% 
so that $\{[\mbZ]=0\iff\mbZ\in\sC^p_g\}$ if  $\{[\mbX]=0\iff \mbX\in\sC^p_g\}$. 
% 
% 
% 
% 
% 
% 
% 
% 
% 
% 
% 
% 
% 
\end{proof}


The next result allows us to combine multiple rough paths as a joint rough path. % 
\begin{lemma}[Joint geometric rough path]\label{lem:rough_path:joint_geometric}
Let $p\in[2,3)$, $T>0$, $\mbX\in\sC^p_g([0,T],\R^d)$ be a geometric $p$-rough path, $b^i\in L^\infty([0,T],\R)$ for $i=1,\dots,m$, and $(Y^j,(Y^j)')\in\sD^p_X([0,T],\R^d)$ % 
for $j=1,\dots,n$.  
 % 
For $i=1,\dots,m$, define $\mbT^i=(\cT^i,\bT^i)\in\sC^1_g([0,T],\R)$  with $\cT_t^i=
\int_0^t
b_s^i\dd s$ and $\bT_{s,t}^i=
\int_s^t\cT_{s,r}^ib_r^i\dd r$, and % 
% 
% 
for $j=1,\dots,n$, define $\mbZ^j=(Z^j,\bZ^j)\in\sC^p_g([0,T],\R)$ with $Z_t^j=
\int_0^t
Y_s^j\dd\mbX_s$ and $\bZ_{s,t}^j=
\int_s^tZ_{s,r}^j Y_r^j\dd\mbX_r$,  as in Lemma \ref{lem:rough_integral_lebesgue}. 

Then, 
% 
% 
% 
% 
% 
% 
% 
% 
% 
% 
% 
% 
% 
% 
% 
% 
% 
% 
% 
% 
% 
% 
% 
% 
% 
% 
% 
% 
% 
% 
% 
% 
% 
% 
% 
% 
% 
% 
% 
% 
% 
% 
the pair $\mbJ := (J,\bJ)$ defined with 
$J_t:=(\cT_t^1,\dots,\cT_t^m,Z_t^1,\dots,Z_t^n)$ 
and 
\begin{equation}\label{eq:rough_path:joint_geometric}
% 
% 
% 
% 
% 
% 
% 
% 
% 
% 
\bJ_{s,t}:=
\begin{bmatrix}
\bT^1_{s,t}
& \dots &
% 
\int_s^t \cT^1_{s,r}\dd \mbT^m_r &
\cT^1_{s,t}Z^1_{s,t}-\int_s^t Z^1_{s,r}\dd \mbT^1_r & \dots &
\cT^1_{s,t}Z^n_{s,t}-\int_s^t Z^n_{s,r}\dd \mbT^1_r
\\
\vdots & \ddots & \vdots & 
\vdots & \ddots & \vdots
\\
\int_s^t \cT^m_{s,r}\dd \mbT^1_r & \dots & \bT^m_{s,t}
&
\cT^m_{s,t}Z^1_{s,t}-\int_s^t Z^1_{s,r}\dd \mbT^m_r & \dots & \cT^m_{s,t}Z^n_{s,t}-\int_s^t Z^n_{s,r}\dd \mbT^m_r
\\
\int_s^t Z^1_{s,r}\dd \mbT^1_r & \dots & \int_s^t Z^1_{s,r}\dd \mbT^m_r &  \bZ^1_{s,t} & \dots &  Z^1_{s,t}Z^n_{s,t}-
% 
\int_s^t Z^n_{s,r}Y^1_r\dd\mbX_r
\\
\vdots & \ddots & \vdots & \vdots & \ddots & \vdots & 
\\
\int_s^t Z^n_{s,r}\dd \mbT^1_r & \dots & \int_s^t Z^n_{s,r}\dd \mbT^m_r &  
% 
\int_s^t Z^n_{s,r}Y^1_r\dd\mbX_r
& \dots & \bZ^n_{s,t}
\end{bmatrix}
\end{equation}
is a (joint) geometric $p$-rough path, i.e., $\mbJ=(J,\bJ)\in\sC^p_g([0,T],\R^{m+n})$.
\end{lemma}
\begin{proof}
Assume that $m=n=2$ without loss of generality, and denote $J=(\cT^1,\cT^2,Z^1,Z^2)=(J^1,J^2,J^3,J^4)$, and similarly for $\bJ$. 
First, $\|\mbJ\|_p<\infty$ follows after bounding  each term in $J_{s,t}$ and $\bJ_{s,t}$ and using Lemma \ref{lem:pvariation:inequalities}.
% 
% 
% 
% 
% 
% 
% 
% 
% 
Next, % 
we show Chen's relation \eqref{eq:chen's_relation}:
{\small\allowdisplaybreaks
\begin{align*}
\bJ_{s,r}^{12}
&=
\int_s^r \cT^1_{s,u}\dd \mbT^2_u
\mathop{=}^{\eqref{eq:rough_integral_lebesgue:dT_bdt}}
\int_s^r\int_s^ub_v^1\dd vb_u^2\dd u
=
\int_s^t\int_s^ub_v^1\dd vb_u^2\dd u-
\int_r^t\int_s^ub_v^1\dd vb_u^2\dd u
\\
&=
\bJ_{s,t}^{12}
-
\left(
\int_r^t
\left(
\int_s^rb_v^1\dd vb_u^2\dd u
+
\int_r^ub_v^1\dd vb_u^2\dd u
\right)
\right)
=
\bJ_{s,t}^{12}
-
\left(
\int_s^rb_v^1\dd v
\int_r^tb_u^2\dd u
+
\int_r^t
\int_r^ub_v^1\dd vb_u^2\dd u
\right)
\\
&=
\bJ_{s,t}^{12}-\bJ^{12}_{r,t}-J_{s,r}^1J_{r,t}^2,
% 
% 
% 
% 
% 
\\
% 
% 
% 
% 
% 
\bJ_{s,r}^{31}
&=
\int_s^r Z^1_{s,u}\dd \mbT^1_u
\mathop{=}^{\eqref{eq:rough_integral_lebesgue:dT_bdt}}
\int_s^r\int_s^uY_v^1\dd\mbX_vb_u^1\dd u
=
\int_s^t\int_s^uY_v^1\dd\mbX_vb_u^1\dd u-
\int_r^t\int_s^uY_v^1\dd\mbX_vb_u^1\dd u
\\
&=
\bJ_{s,t}^{31}
-
\left(
\int_r^t
\left(
\int_s^rY_v^1\dd\mbX_vb_u^1\dd u
+
\int_r^uY_v^1\dd\mbX_vb_u^1\dd u
\right)
\right)
=
\bJ_{s,t}^{31}
-
\left(
\int_s^rY_v^1\dd\mbX_v 
\int_r^tb_u^1\dd u
+
\int_r^t
\int_r^uY_v^1\dd\mbX_vb_u^1\dd u
\right)
\\
&=
\bJ_{s,t}^{31}-\bJ^{31}_{r,t}-Z_{s,r}^1\cT_{r,t}^1
\\
&=
\bJ_{s,t}^{31}-\bJ^{31}_{r,t}-J_{s,r}^3J_{r,t}^1,
% 
% 
% 
% 
% 
\\
% 
% 
% 
% 
% 
\bJ_{s,r}^{13}
&=
\cT^1_{s,r}Z^1_{s,r}
-
\int_s^r Z^1_{s,u}\dd \mbT^1_u
=
\cT^1_{s,r}Z^1_{s,r}
-
\bJ_{s,r}^{31}
=
\cT^1_{s,r}Z^1_{s,r}
-
(\bJ_{s,t}^{31}-\bJ_{r,t}^{31}-\cT_{r,t}^1Z_{s,r}^1)
\\
&=
(\cT^1_{s,t}-\cT^1_{r,t})(Z^1_{s,t}-Z^1_{r,t})
-
\bJ_{s,t}^{31}+\bJ_{r,t}^{31}
+
\cT_{r,t}^1Z_{s,r}^1
\\
&=
(\cT^1_{s,t}Z^1_{s,t}
-
\bJ_{s,t}^{31})
+
\cT^1_{r,t}Z^1_{r,t}
+
\bJ_{r,t}^{31}
-\cT^1_{s,t}Z^1_{r,t}
-\cT^1_{r,t}Z^1_{s,t}
+
\cT_{r,t}^1Z_{s,r}^1
\\
&=
(\cT^1_{s,t}Z^1_{s,t}
-
\bJ_{s,t}^{31})
+
\cT^1_{r,t}Z^1_{r,t}
+
\bJ_{r,t}^{31}
-\cT^1_{s,t}Z^1_{r,t}
-\cT^1_{r,t}(Z^1_{s,t}
-
Z_{s,r}^1)
\\
&=
(\cT^1_{s,t}Z^1_{s,t}
-
\bJ_{s,t}^{31})
+
\cT^1_{r,t}Z^1_{r,t}
+
\bJ_{r,t}^{31}
-\cT^1_{s,t}Z^1_{r,t}
-\cT^1_{r,t}Z^1_{r,t}
\\
&=
(\cT^1_{s,t}Z^1_{s,t}
-
\bJ_{s,t}^{31})
+
\bJ_{r,t}^{31}
-(\cT^1_{s,t}-\cT^1_{r,t})Z^1_{r,t}
-\cT^1_{r,t}Z^1_{r,t}
\\
&=
(\cT^1_{s,t}Z^1_{s,t}
-
\bJ_{s,t}^{31})
+
\bJ_{r,t}^{31}
-
\cT_{s,r}^1Z_{r,t}^1
-\cT^1_{r,t}Z^1_{r,t}
\\
&=
(\cT^1_{s,t}Z^1_{s,t}
-
\bJ_{s,t}^{31})
-
(\cT^1_{r,t}Z^1_{r,t}
-
\bJ_{r,t}^{31})
-
\cT_{s,r}^1Z_{r,t}^1
\\
&=
\bJ_{s,t}^{13}-\bJ^{13}_{r,t}-\cT_{s,r}^1Z_{r,t}^1
\\
&=
\bJ_{s,t}^{13}-\bJ^{13}_{r,t}-J_{s,r}^1J_{r,t}^3,
\end{align*}
}% 
and similar derivations show that Chen's relation \eqref{eq:chen's_relation} hold for other pairs of indices $(i,j)$. Thus, $\mbJ\in\sC^p$, i.e., $\mbJ$ is a $p$-rough path.

Finally, to show that $\mbJ$ is geometric, we need to show that $\bJ^{ij}_{s,t}+\bJ^{ji}_{s,t}=J_{s,t}^iJ_{s,t}^j$ in \eqref{eq:rough_path:integration_by_parts} holds for any $i,j\in\{1,2,3,4\}$. For $i=j$, \eqref{eq:rough_path:integration_by_parts} clearly holds since $\mbT$ and $\mbZ$ are geometric. For $1\leq i\neq j\leq 2$, \eqref{eq:rough_path:integration_by_parts} follows from the integration by parts formula that holds for the Lebesgue integral. For other pairs of indices $(i,j)$, \eqref{eq:rough_path:integration_by_parts} holds by definition, e.g., $\bJ^{13}_{s,t}+\bJ^{31}_{s,t}=\cT^1_{s,t}Z_{s,t}^1=J_{s,t}^1J_{s,t}^3$. To conclude, $\mbJ\in\sC^p_g$. % 
\end{proof}


%\subsubsection*{Proof of Theorem \ref{thm:rde:linear:existence_uniqueness}}
 \begin{proof}[Proof of Theorem \ref{thm:rde:linear:existence_uniqueness}]
We rewrite the linear RDE \eqref{eq:rde_linear} as a linear RDE with constant coefficients driven by a new geometric rough path and conclude with \cite[Theorem 10.53]{Friz2010}. 
First, for $1\leq i,j\leq n$, let
\begin{align*}
&\mbT^{ij}=(\cT^{ij},\bT^{ij}),
\ \text{ where}
&&\hspace{-15mm}
\cT^{ij}_t:=
\int_0^t
A_s^{ij}\dd s,
&&&&\hspace{-35mm}
\bT^{ij}_{s,t}:=
\int_s^t
\cT^{ij}_{s,r}
A_r^{ij}
\dd r,
\\
&\mbZ^{ij}=(Z^{ij},\bZ^{ij}), \ 
\ \text{ where}
&&\hspace{-15mm}
Z^{ij}_t:=
	\int_0^t
\Sigma_s^{i\cdot j}
\dd\mbX_s,
&&&&\hspace{-35mm}
\bZ^{ij}_{s,t}
:=
\int_s^tZ_{s,r}^{ij} 
\Sigma_r^{i\cdot j}
\dd\mbX_r.
\end{align*} 
By Lemma \ref{lem:rough_integral_lebesgue}, each $\mbT^{ij}$  and $\mbZ^{ij}$ is a geometric rough path ($\mbT^{ij},\mbZ^{ij}\in\sC^p_g$). 
% 
% 
% 
% 
% 
% 
% 
% 
% 
% 
% 
% 
% 
% 
% 
% 
% 
% 
% 
% 
% 
% 
% 
% 
% 
% 
% 
% 
% 
% 
% 
% 
% 
% 
% 
% 
% 
% 
% 
% 
% 
% 
% 
% 
% 
% 
% 
% 
% 
% 
% 
% 
% 
% 
% 
% 
% 
% 
% 
% 
% 
% 
% 
% 
% 
% 
% 
% 
% 
% 
% 
% 
% 
% 
% 
% 
% 
% 
% 
% 
% 
% 
% 
% 
% 
% 
% 
% 
% 
% 
% 
% 
% 
% 
% 
% 
% 
% 
% 
% 
% 
% 
As in Lemma \ref{lem:rough_path:joint_geometric}, define the joint geometric $p$-rough paths $\mbJ=(J,\bJ)\in\sC^p_g([0,T],\R^{2n^2})$  with % 
\begin{align*}
J
&=
\big(
J^1,
\dots,
J^{n^2},
J^{n^2+1},\dots,
J^{2n^2}
\big)
=
\left(
\cT^{11},
\dots,
\cT^{1n},
\dots,
\cT^{n1},
\dots,
\cT^{nn},
% 
% 
% 
Z^{11},
\dots,
Z^{1n},
\dots,
Z^{n1},
\dots,
Z^{nn}
\right),
\end{align*}
so that  $J^{(i-1)n+j}=\cT^{ij}$ and $J^{n^2+(i-1)n+j}=Z^{ij}$ for any $i,j=1,\dots,n$, and  
define $\bJ:\Delta_{[0,T]}\to\R^{2n^2\times 2n^2}$ as in \eqref{eq:rough_path:joint_geometric} in Lemma \ref{lem:rough_path:joint_geometric}, with in particular $\bJ^{ii}=\bT^{ii}$ for $i\leq n^2$ and $\bJ^{ii}=\bZ^{ii}$ for $i> n^2$. 
 
Then, we define the tensor $F\in\R^{n\times 2n^2\times n}$ with
{\small
$$
F^{i\cdot\cdot}
=
\begin{bmatrix}
0_{(i-1)n\times n} \\
I_{n\times n} \\
0_{(n-1)n\times n} \\
I_{n\times n} \\
0_{(n-i)n \times n}
\end{bmatrix}\hspace{-2pt},
\text{ so that } 
FV_t = \underbrace{
\left[
\begin{array}{cccccccccccccc}
V^1_t&\dots&V^n_t & \dots & 0 &\dots & 0  &
V^1_t&\dots&V^n_t & \dots & 0 &\dots & 0
\\
\vdots &       & \vdots  &\ddots &  
\vdots &       & \vdots   &
\vdots &       & \vdots  &\ddots &  
\vdots &       & \vdots   
\\
0 &\dots & 0  &  \dots & V^1_t&\dots&V^n_t &
0 &\dots & 0  &  \dots & V^1_t&\dots&V^n_t
\end{array}
\right]
}_{
n\times 2n^2
}
$$
}% 
for any $V_t\in\R^n$, and the linear RDE
$V_t =
v
+
\int_0^t
F V_r\dd\mbJ_r$, which is a linear RDE with constant-in-time linear vector fields $F^\ell(V)=F^\ell V$ driven by the geometric rough path $\mbJ$. 
Thus, by \cite[Theorem 10.53]{Friz2010}, there exists a unique solution $(V,FV)\in\sD_J([0,T],\R^n)$ to the linear RDE $V_t =
v
+
\int_0^t
F V_r\dd\mbJ_r$.

This concludes the proof, because $(V,\Sigma V)$ is also a solution to the original linear RDE \eqref{eq:rde_linear}, since
\begin{align}\label{eq:rde_linear:dJ_dX}
\int_s^t
F V_r\dd\mbJ_r
=
\int_s^t
A_rV_r\dd r
+  
\int_s^t
\Sigma_rV_r\dd\mbX_r
\end{align} 
for any $s,t\in[0,T]$. The last result \eqref{eq:rde_linear:dJ_dX} follows from long but straightforward computations with  integrals, noting that 
each component on the left hand side of \eqref{eq:rde_linear:dJ_dX} satisfies
{\small
\begin{align}\label{eq:rde_linear:intFVdJ}
 \bigg(
\int_s^tFV_r\dd\mbJ_r\bigg)^i
&\approx
(FV_s)^iJ_{s,t}+(FFV_s)^i\bJ_{s,t},
\end{align}
}% 
for $i=1,\dots,n$, 
where  $a_{s,t}\approx b_{s,t}$ means $a_{s,t}=b_{s,t}+o(|t-s|^{\frac{3}{p}})$ and the estimate comes from \eqref{eq:rough_int:error_bound} in Proposition \ref{prop:rough_integral_welldefined:error_bound} (up to a time reparameterization \cite[Proposition 5.14]{Friz2010}, we may assume that $J,FV,FFV$ are $\frac{1}{p}$-H\"older continuous and that $\bJ,R^{FV}$ are $\frac{2}{p}$-H\"older continuous, and similarly for $X,\bX$). 

Also, for $i=1,\dots,n$,


{\small
\begin{align} 
(FV_s)^iJ_{s,t}
&=
\sum_{j=1}^j
V^j_s
\left(
J^{(i-1)n+j}_{s,t}
+
J^{n^2+(i-1)n+j}_{s,t}
\right)
=
\Big[
\underbrace{
\ 0 \ \dots \  0_{\phantom{s}} 
}_{(i-1)n} \ 
\underbrace{V^1_t \ \dots \ V^n_t}_{n}  \ 
\underbrace{
\ 0 \ \dots \  0_{\phantom{s}} 
}_{(n-1)n} \ 
\underbrace{V^1_t \ \dots \ V^n_t}_{n}  \ 
\underbrace{
\ 0 \ \dots \  0_{\phantom{s}} 
}_{(n-i)n}
\Big]
J_{s,t},
% 
% 
% 
% 
% 
% 
% 
% 
% 
% 
% 
% 
% 
% 
% 
% 
% 
% 
% 
% 
% 
% 
% 
\nonumber
\\
(FFV_s)^i\bJ_{s,t}
&=
\sum_{k=1}^n
\sum_{j=1}^n
V^j_s
\Big(
\bJ^{    (i-1)n+k,     (k-1)n+j}_{s,t}+
\bJ^{    (i-1)n+k, n^2+(k-1)n+j}_{s,t}+
\bJ^{n^2+(i-1)n+k,     (k-1)n+j}_{s,t}+
\nonumber
\\[-2mm]
&\hspace{22mm}
+
\bJ^{n^2+(i-1)n+k, n^2+(k-1)n+j}_{s,t}
\Big)
\nonumber
\\
&\approx
\sum_{k=1}^n
\sum_{j=1}^n
V^j_s
\bJ^{n^2+(i-1)n+k, n^2+(k-1)n+j}_{s,t}
\label{eq:rde_linear:FVJ}
\end{align} 
}% 
%for $i=1,\dots,n$, 
and, for $i,j,k=1,\dots,n$,
% 
% 
% 
\begin{align}\label{eq:rde_linear:J}
J^{n^2+(i-1)n+j}_{s,t}
&=
% 
% 
\int_s^t
\Sigma^{i\cdot j}_r
\dd\mbX_r
\approx
\Sigma^{i\cdot j}_sX_{s,t}
+
(\Sigma^{i\cdot j})'_s\bX_{s,t},
% 
% 
\quad
\bJ^{n^2+(i-1)n+k,n^2+(k-1)n+j}_{s,t}
\approx
\Sigma^{i\cdot k}_s\Sigma^{k\cdot j}_s\bX_{s,t}.
% 
\end{align}
Additional details on the computation of $(FFV_s)^i\bJ_{s,t}$  and $\bJ^{n^2+(i-1)n+k,n^2+(k-1)n+j}_{s,t}$ are provided at the end of this section.
% 
% 
% 
% 
% 
% 
% 
% 
% 
% 
% 
% 
% 
% 
% 
% 
% 
% 
% 
% 
% 
% 
% 
% 
% 
% 
% 
% 
% 
% 
% 
% 
% 
% 
% 
% 
% 
% 
% 
% 
% 
% 
% 
% 
% 
% 
% 
% 
% 
% 
% 
% 
% 
% 
% 
% 
% 
% 
% 
% 
% 
% 
% 
% 
% 
% 
% 
% 
% 
% 
% 
% 
% 
% 
% 
% 
% 
% 
% 
% 
% 
% 
% 
% 
% 
% 
% 
% 
% 
% 
% 
% 
% 
% 
% 
% 
% 
% 
% 
% 
% 
% 
% 
% 
% 
% 
% 
% 
% 
% 
% 
% 
% 
% 
% 
% 
% 
% 
% 
% 
% 
% 
% 
% 
% 
% 
% 
% 
% 
% 
% 
% 
% 
% 
Then, writing $V'=\Sigma V$, the right  hand side of \eqref{eq:rde_linear:dJ_dX} satisfies
% 
% 
% 
% 
% 
% 
% 
% 
% 
% 
% 
% 
% 
% 
% 
% 
% 
% 
% 
% 
% 
% 
% 
% 
% 
% 
% 
% 
% 
% 
% 
% 
% 
% 
% 
% 
% 
% 
% 
% 
% 
% 
% 
% 
% 
% 
% 
% 
% 
% 
% 
% 
% 
% 
% 
% 
% 
% 
% 
% 
% 
% 
% 
% 
% 
% 
% 
% 
% 
% 
% 
% 
% 
% 
% 
% 
% 
% 
% 
% 
% 
% 
% 
% 
% 
% 
% 
% 
% 
% 
% 
% 
% 
% 
% 
% 
% 
% 
% 
% 
% 
% 
% 
% 
% 
% 
% 
% 
% 
% 
% 
% 
% 
% 
% 
% 
% 
% 
% 
% 
% 
% 
% 
% 
% 
% 
% 
% 
% 
% 
% 
% 
% 
{\small\allowdisplaybreaks
\begin{align}
\int_s^t
\left(A_rV_r\right)^i\dd r
&=
\sum_{j=1}^n
\int_s^t 
A_r^{ij}V_r^j\dd r
\mathop{=}^{\eqref{eq:rough_integral_lebesgue:dT_bdt}}
\sum_{j=1}^n
\int_s^t
V_t^j\dd\boldsymbol{\cT}_t^{ij}
\approx
\sum_{j=1}^n 
\big(
V^j_sJ^{(i-1)n+j}_{s,t}+(V^j)'_s\bJ^{(i-1)n+j,(i-1)n+j}_{s,t}
\big)
\nonumber
\\
\label{eq:rde_linear:int_AVdt}
&\approx
\sum_{j=1}^n 
V^j_sJ^{(i-1)n+j}_{s,t},
\\
\int_s^t
\left(\Sigma_rV_r\right)^i\dd\mbX_r
&\approx
\sum_{j=1}^n
\left(
V^j_s\Sigma^{i\cdot j}_sX_{s,t}+
(\Sigma^{i\cdot j}_s(V^j)'_s+V^j_s(\Sigma^{i\cdot j})'_s)\bX_{s,t}
\right)
\nonumber
\\
&=
\sum_{j=1}^n
\bigg(
V^j_s
\left(
\Sigma^{i\cdot j}_sX_{s,t}+(\Sigma^{i\cdot j})'_s\bX_{s,t}
\right)+
\bigg(
\sum_{k=1}^n
\Sigma^{i\cdot j}_s
\Sigma^{j\cdot k}_sV^k_s\bigg)\bX_{s,t}
\bigg)
\nonumber
\\
&\approx
\sum_{j=1}^n
\bigg(
V^j_s
Z^{ij}_{s,t}+
V^j_s
\bigg(
\sum_{k=1}^n
\Sigma^{i\cdot k}_s\Sigma^{k\cdot j}_s
\bigg)\bX_{s,t}
\bigg)
\nonumber
\\
&\mathop{=}^{\eqref{eq:rde_linear:J}}
\sum_{j=1}^n
\bigg(
V^j_s
J^{n^2+(i-1)n+j}_{s,t}
+
V^j_s
\sum_{k=1}^n
\bJ^{n^2+(i-1)n+k,n^2+(k-1)n+j}_{s,t}
\bigg).
\label{eq:rde_linear:int_SigmaVdX}
\end{align}
}% 
% 
% 
% 
% 
% 
% 
% 
% 
% 
% 
% 
% 
% 
% 
% 
% 
% 
% 
% 
% 
% 
% 
% 
% 
% 
% 
% 
% 
% 
% 
% 
% 
% 
% 
% 
% 
% 
% 
% 
% 
% 
% 
Thus, by comparing \eqref{eq:rde_linear:int_AVdt}+\eqref{eq:rde_linear:int_SigmaVdX}  with \eqref{eq:rde_linear:intFVdJ} using \eqref{eq:rde_linear:FVJ}, and after taking the sum over any partition $\pi\in\mathcal{P}([s,t])$ with vanishing meshsize, we conclude that  \eqref{eq:rde_linear:dJ_dX} holds. 
% 
% 
% 
% 
% 
% 
% 
% 
% 
% 
% 
% 
% 
% 
% 
% 
% 
% 
% 
% 
% 
% 
% 
% 
% 
% 
% 
% 
% 
% 
% 
% 
% 
% 
% 
% 
% 
% 
% 
% 
% 
% 
% 
% 
% 
% 
% 
% 
% 
% 
% 
% 
% 
% 
% 
% 
% 
% 
% 
% 
% 
% 
% 
% 
% 
% 
% 
% 
% 
% 
% 
% 
% 
% 
% 
% 
% 
% 
% 
% 
% 
% 
% 
% 
% 
% 
% 
% 
% 
% 
% 
% 
% 
% 
% 
% 
% 
% 
% 
% 
% 
% 
% 
% 
% 
% 
% 
We conclude using \eqref{eq:rde_linear:dJ_dX} that the RDE  \eqref{eq:rde_linear} can be written as
% 
% 
% 
% 
% 
% 
% 
% 
% 
% 
% 
% 
% 
% 
% 
% 
% 
% 
% 
% 
% 
% 
% 
% 
% 
% 
% 
% 
% 
% 
% 
% 
% 
% 
% 
% 
% 
% 
% 
% 
% 
% 
% 
% 
% 
% 
% 
% 
% 
% 
% 
% 
% 
% 
% 
% 
% 
% 
% 
% 
% 
% 
% 
% 
% 
% 
% 
% 
% 
$
V_t =v +  
\int_0^t
A_tV_t\dd t
+  
\int_0^t
\Sigma_tV_t\dd\mbX_t
=
v
+
\int_0^t
F V_t\dd\mbJ_t$  and the RDE on the right hand side  has a unique solution by \cite[Theorem 10.53]{Friz2010}, so the linear RDE \eqref{eq:rde_linear} has a unique solution.
% 
% 
% 
% 
% 
% 
% 
% 
% 
% 
% 
% 
% 
% 
% 
% 
% 
% 
% 
% 
% 
% 
% 
% 
% 
% 
% 
% 
% 
% 
% 
% 
% 
% 
% 
% 
% 
% 
% 
% 
% 
% 
% 
% 
% 
% 
% 
% 
% 
% 
% 
% 
% 
% 
% 
% 
% 
% 
% 
% 
% 
% 
% 
% 
% 
% 
% 
% 
% 
% 
% 
% 
% 
% 
% 
% 
% 
% 
% 
% 
% 
% 
% 
% 
% 
% 
% 
% 
% 
% 
% 
% 
% 
% 
% 
% 
% 
% 
% 
% 
% 
% 
% 
% 
% 
% 
% 
% 
% 
% 
% 
% 
% 
% 
% 
% 
% 
% 
% 
% 
% 
% 
% 
% 
% 
% 
% 
% 
% 
% 
% 
% 
% 
\end{proof}











\subsubsection*{Additional details on computing $(FFV_s)^i  \bJ_{s,t}$ and $\bJ^{n^2+(i-1)n+k,n^2+(k-1)n+j}_{s,t}$} 
For any $V_s\in\R^n$, %we have 
$F\in\R^{n\times 2n^2\times n}$, $FV_s\in\R^{n\times 2n^2}$, and $FFV_s\in\R^{n\times 2n^2\times 2n^2}$. Then,  denoting by $\bJ^{a:b,c:d}_{s,t}\in\R^{(b-a+1)\times (d-c+1)}$ the block of the matrix  $\bJ_{s,t}\in\R^{2n^2\times 2n^2}$ containing the rows $a$ to $b$ and columns $c$ to $d$, and 
using the identification $(FFV_s)^i\in\R^{2n^2\times 2n^2}\cong\L(\R^{2n^2\times 2n^2},\R)$ with $(FFV_s)^i\bJ=\sum_{a=1}^{2n^2}\sum_{b=1}^{2n^2}((FFV_s)^i)^{a,b}\bJ^{a,b}_{s,t}$, we have 
{\small
\allowdisplaybreaks
\begin{align*}
(FFV_s)^i  \bJ_{s,t}
&= 
F^i FV_s
\bJ_{s,t}
\\
&\hspace{-15mm}=
\begin{bmatrix}
0_{(i-1)n\times n} \\
I_{n\times n} \\
0_{(n-1)n\times n} \\
I_{n\times n} \\
0_{(n-i)n \times n}
\end{bmatrix}
\underbrace{
\left[
\begin{array}{cccccccccccccc}
V^1_s&\dots&V^n_s & \dots & 0 &\dots & 0  &
V^1_s&\dots&V^n_s & \dots & 0 &\dots & 0
\\
\vdots &       & \vdots  &\ddots &  
\vdots &       & \vdots   &
\vdots &       & \vdots  &\ddots &  
\vdots &       & \vdots   
\\
0 &\dots & 0  &  \dots & V^1_s&\dots&V^n_s &
0 &\dots & 0  &  \dots & V^1_s&\dots&V^n_s
\end{array}
\right]
}_{
=FV_s \ (n\times 2n^2 \text{ matrix})
}
\bJ_{s,t}
\\
&\hspace{-15mm}=
\underbrace{
\begin{bmatrix}
0_{(i-1)n\times n^2}
\\
FV_s
\\
0_{(n-1)n\times 2n^2}
\\
	FV_s
\\
0_{(n-i)n \times 2n^2} 
\end{bmatrix}
}_{=F^iFV_s}
\underbrace{
\begin{bmatrix}
\star & \star
\\
\bJ^{(i-1)n+1:in, 1:n^2}_{s,t} 
& 
\bJ^{(i-1)n+1:in, n^2+1:2n^2}_{s,t}
\\
\star & \star
\\
\bJ^{n^2+(i-1)n+1:n^2+in, 1:n^2}_{s,t} 
& 
\bJ^{n^2+(i-1)n+1:n^2+in, n^2+1:2n^2}_{s,t}
\\
\star & \star
\end{bmatrix}
}_{=\bJ_{s,t}}
\\
&\hspace{-15mm}=
	\left[
	\begin{array}{cccccccc}
	V^1_s&\dots&V^n_s & \dots & 0 &\dots & 0
	\\
	\vdots &       & \vdots  &\ddots &  
	\vdots &       & \vdots   
	\\
	0 &\dots & 0  &  \dots & V^1_s&\dots&V^n_s
	\end{array}
	\right]
	\Bigg(
	\bJ^{(i-1)n+1:in, 1:n^2}_{s,t} 
+ 
\bJ^{(i-1)n+1:in, n^2+1:2n^2}_{s,t}
+
\bJ^{n^2+(i-1)n+1:n^2+in, 1:n^2}_{s,t} 
+
	\\
	&\hspace{-5mm}
	\begin{bmatrix}	\bJ^{n^2+(i-1)n+1,n^2+1}_{s,t}&\dots&\bJ^{n^2+(i-1)n+1,n^2+n}_{s,t} & \dots & \dots & \dots & \dots
	\\
	\vdots&\vdots&\vdots& \ddots & \vdots & \vdots & \vdots
	\\
	\dots & \dots & \dots & \dots 
	&
	\bJ^{n^2+(i-1)n+n,n^2+(n-1)n+1}_{s,t}&\dots&\bJ^{n^2+(i-1)n+n,2n^2}_{s,t} 
	\end{bmatrix}
	\Bigg)
\\
&\hspace{-15mm}=
\sum_{k=1}^n
\sum_{j=1}^n
V^j_s
\left(
\bJ^{    (i-1)n+k,     (k-1)n+j}_{s,t}+
\bJ^{    (i-1)n+k, n^2+(k-1)n+j}_{s,t}+
\bJ^{n^2+(i-1)n+k,     (k-1)n+j}_{s,t}+
\bJ^{n^2+(i-1)n+k, n^2+(k-1)n+j}_{s,t}
\right),
\end{align*}
}% 
where  the first three terms involve the Lebesgue integral  (thus, they are $o(|t-s|^{\frac{3}{p}})$) and do not play a role after summing over all partitions $\pi\in\mathcal{P}[s,t]$ and taking the limit as $|\pi|\to 0$ (see for example the computation in  \eqref{eq:smooth_rough_intgral_no_effect}), so we obtain \eqref{eq:rde_linear:FVJ}.


Next, we compute $\bJ^{n^2+(i-1)n+k,n^2+(k-1)n+j}_{s,t}$ for $i,j,k=1,\dots,n$. For the case $(i-1)n+k\geq(k-1)n+j$:
\begin{align*}
\bJ^{n^2+(i-1)n+k,n^2+(k-1)n+j}_{s,t}
&=
\int_s^tZ_{s,r}^{ik} 
\Sigma^{k\cdot j}_r
\dd\mbX_r
=
\int_s^tZ^{ik} _r
\Sigma^{k\cdot j}_r
\dd\mbX_r
-
\int_s^tZ^{ik}_s
\Sigma^{k\cdot j}_r
\dd\mbX_r
\\
&\approx
Z^{ik}_s 
\Sigma^{k\cdot j}_s
X_{s,t}
+
\left(
(Z^{ik})'_s 
\Sigma^{k\cdot j}_s
+
Z^{ik}_s 
(\Sigma^{k\cdot j})'_s
\right)\bX_{s,t}
-
(Z_s^{ik} \Sigma^{k\cdot j}_sX_{s,t}
+
Z_s^{ik} (\Sigma^{k\cdot j})'_s\bX_{s,t})
\\
&=
(Z^{ik})'_s\Sigma^{k\cdot j}_s\bX_{s,t}
\\
&=
\Sigma^{i\cdot k}_s\Sigma^{k\cdot j}_s\bX_{s,t}. 
\end{align*}
Also,  for the case $(i-1)n+k<(k-1)n+j$,
\begin{align*}
\bJ^{n^2+(i-1)n+k,n^2+(k-1)n+j}_{s,t}
&\mathop{=}^{\eqref{eq:rough_path:integration_by_parts}}
J^{n^2+(i-1)n+k}_{s,t}J^{n^2+(k-1)n+j}_{s,t}
-
\bJ^{n^2+(k-1)n+j,n^2+(i-1)n+k}_{s,t}
&&\text{($\mbJ$ is geometric)}
% 
% 
% 
% 
\\
&\approx
\Sigma^{i\cdot k}_s
\Sigma^{k\cdot j}_s(X_{s,t}\otimes X_{s,t})
-
\Sigma^{k\cdot j}_s\Sigma^{i\cdot k}_s\bX_{s,t}
\\
&=
\Sigma^{k\cdot j}_s
\Sigma^{i\cdot k}_s(X_{s,t}\otimes X_{s,t}
-
\bX_{s,t})
\\
&\mathop{=}^{\eqref{eq:rough_path:integration_by_parts}}
\Sigma^{k\cdot j}_s
\Sigma^{i\cdot k}_s\bX_{s,t}^\top
&&
\text{($\mbX$ is geometric)}
\\
&=
\Sigma^{i\cdot k}_s\Sigma^{k\cdot j}_s\bX_{s,t},
\end{align*} 
so we obtain \eqref{eq:rde_linear:J}.
% 
% 
% 
% 
% 
% 
% 
% 
% 
% 
% 
% 
% 
% 
% 
% 
% 
% 
% 
% 
 



% 
% 
% 
% 
% 
% 
% 
% 
% 
% 
% 
% 
% 
% 
% 
% 
% 
% 
% 
% 
% 
% 
% 
% 
% 
% 
% 
% 
% 
% 
% 
% 
% 
% 
% 
% 
% 
% 
% 
% 
% 
% 
% 
% 
% 
% 
% 
% 
% 
% 
% 
% 
% 
% 
% 
% 
% 
% 
% 
% 
% 
% 
% 
% 
% 
% 
% 
% 
% 
% 
% 
% 
% 
% 
% 
% 
% 
% 
% 
% 
% 
% 
% 
% 
% 
% 
% 
% 
% 
% 
% 
% 
% 
% 
% 
% 
% 
% 
% 
% 
% 
% 
% 
% 
% 
% 
% 
% 
% 
% 
% 
% 
% 
% 
% 
% 
% 
% 
% 
% 
% 
% 
% 
% 
% 
% 
% 
% 
% 
% 
% 
% 
% 
% 
% 
% 
% 
% 
% 
% 
% 
% 
% 
% 
% 
% 
% 
% 
% 
% 
% 
% 
% 
% 
% 
% 
% 
% 
% 
% 
% 
% 
% 
% 
% 
% 
% 
% 
% 
% 
% 
% 
% 
% 
% 
% 
% 
% 
% 
% 
% 
% 
% 
% 
% 
% 
% 


% 
% 
% 
% 
% 
% 
% 
% 
% 
% 
% 
% 
% 
% 
% 
% 
% 
% 
% 
% 
% 
% 
% 
% 
% 
% 
% 
% 
% 
% 
% 
% 
% 
% 
% 
% 
% 
% 
% 
% 
% 
% 
% 
% 
% 
% 
% 
% 
% 
% 
% 
% 
% 
% 
% 
% 
% 
% 
% 
% 
% 
% 
% 
% 
% 
% 
% 
% 
% 
% 
% 
% 
% 
% 
% 
% 
% 
% 
% 
% 
% 
% 
% 
% 
% 
% 
% 
% 
% 
% 
% 
% 
% 
% 
% 
% 
% 
% 
% 
% 
% 
% 
% 
% 
% 
% 
% 
% 
% 
% 
% 
% 
% 
% 
% 
% 
% 
% 
% 
% 
% 
% 
% 
% 
% 
% 
% 
% 
% 
% 
% 
% 
% 
% 
% 
% 
% 
% 
% 
% 
% 
% 
% 
% 
% 
% 
% 
% 
% 




 
\subsubsection{Bounds on solutions to nonlinear RDEs (Section \ref{sec:rdes:bounds})}
\subsubsection*{Error bounds on short intervals (Section \ref{sec:rdes:bounds:short})}

%\subsubsection*{Proof of Proposition \ref{prop:bounds_pvars_solutions_RDEs}}
\begin{proof}[Proof of Proposition \ref{prop:bounds_pvars_solutions_RDEs}] 
To show \eqref{eq:bounds_pvars_solutions_RDEs:sigma(Y)'_p}, note that $\sigma(\cdot,Y)'=\frac{\partial\sigma}{\partial x}(\cdot,Y)Y'
=\frac{\partial\sigma}{\partial x}(\cdot,Y)\sigma(\cdot,Y)$, so we obtain
$\|\sigma(\cdot,Y)'\|_p\leq C_p\|\frac{\partial\sigma}{\partial x}\sigma\|_{C_b^1}(\|Y\|_p+T)$ by \eqref{eq:sigma(.,X):pvar}. 
 Next, for $s,t\in I$ with $I\subseteq[0,T]$ an arbitrary interval, 
 {\small
\begin{align*}
\|R_{s,t}^Y\|
&=
\|Y_{s,t}-Y_s'X_{s,t}\| 
\leq
\left\|
\int_s^t\sigma(r,Y_r)\dd\mbX_r-\sigma(s,Y_s)X_{s,t}-\sigma(\cdot,Y)'_s\bX_{s,t}
\right\|
+
\left\|
\int_s^tb(r,Y_r,u_r)\dd r
\right\|
+
\left\|\sigma(\cdot,Y)'_s\bX_{s,t}\right\|
\\
&\mathop{\leq}^{\eqref{eq:rough_int:error_bound},\eqref{eq:bounds_int_b_ds}}
C_p\big(
\|R^{\sigma(\cdot,Y)}\|_{\frac{p}{2},[s,t]}\|X\|_{p,I}+\|\sigma(\cdot,Y)'\|_{p,I}\|\bX\|_{\frac{p}{2},[s,t]}
\big)
+
C_{p,b}|t-s|
+
C_\sigma\|\bX\|_{\frac{p}{2},[s,t]}.
\end{align*}
}% 
where we used $\|\sigma(\cdot,Y)'\|_\infty
=\|\frac{\partial\sigma}{\partial x}(\cdot,Y)\sigma(\cdot,Y)\|_\infty\leq C_\sigma$ in the last inequality. 
% 
% 
% 
% 
% 
% 
% 
% 
Thus, by \eqref{eq:path_finite_var:sum_p/2vars:p/2vars_subintervals} in Lemma \ref{lem:sum_p/2vars:p/2vars_subintervals} (which is similar to \eqref{eq:path_finite_var:sum_p/2vars} in Lemma \ref{lem:pvariation:inequalities}),
% 
\begin{align*}
\|R^Y\|_{\frac{p}{2},I}
&\leq
C_{p,b,\sigma}\left(
\|R^{\sigma(\cdot,Y)}\|_{\frac{p}{2},I}
\|X\|_{p,I}+
\|\sigma(\cdot,Y)'\|_{p,I}\|\bX\|_{\frac{p}{2},I}
+
|I|
+
\|\bX\|_{\frac{p}{2},I}
\right)
\\
&\hspace{-12mm}\mathop{\leq}^{\eqref{eq:RY:p/2var:Y^2+RY+T},\eqref{eq:bounds_pvars_solutions_RDEs:sigma(Y)'_p}
}
C_{p,b,\sigma}
\left(
(\|Y\|_{p,I}^2+\|R^Y\|_{\frac{p}{2},I}
+|I|
)
\|X\|_{p,I}+
(\|Y\|_{p,I}+|I|)\|\bX\|_{\frac{p}{2},I}
+
|I|
+
\|\bX\|_{\frac{p}{2},I}
\right)
\\
&\hspace{-12mm}\leq
C_{p,b,\sigma}
\left(
(\|Y\|_{p,I}^2+\|R^Y\|_{\frac{p}{2},I})
\|X\|_{p,I}+
(1+\|Y\|_{p,I})\|\bX\|_{\frac{p}{2},I}
+
|I|
+
|I|(\|X\|_{p,I}+\|\bX\|_{\frac{p}{2},I})
\right)
\\
&\hspace{-12mm}\leq
C_{p,b,\sigma}
\left(
(\|Y\|_{p,I}^2+\|R^Y\|_{\frac{p}{2},I})
\|X\|_{p,I}+
(1+\|Y\|_{p,I}^2)\|\bX\|_{\frac{p}{2},I}
+
|I|
+
|I|(\|X\|_{p,I}+\|\bX\|_{\frac{p}{2},I})
\right)
\quad\text{($|x|\leq 1+x^2$)}
\\
&\hspace{-12mm}\leq
C_{p,b,\sigma}
\left(
\|R^Y\|_{\frac{p}{2},I}
\|X\|_{p,I}+
\|\bX\|_{\frac{p}{2},I}
+
|I|
+
|I|(\|X\|_{p,I}+\|\bX\|_{\frac{p}{2},I})
+
\|Y\|_{p,I}^2(\|X\|_{p,I}+\|\bX\|_{\frac{p}{2},I})
\right),
\end{align*}
where $C_1:=C_{p,b,\sigma}\geq 1$. 
Let $\alpha_1:=\left(\frac{1}{2C_1}\right)^p$. Then, for $I$ small-enough so that $\|\mbX\|_{p,I}+|I|\leq\alpha_1^\frac{1}{p}$, we obtain 
$$
\|\mbX\|_{p,I}+|I|\leq\alpha_1^\frac{1}{p}=\frac{1}{2C_{p,b,\sigma}}< 1,
$$ 
so that $C_{p,b,\sigma}\|X\|_{p,I}\leq \frac{1}{2}$, 
$\|X\|_{p,I}+\|\bX\|_{\frac{p}{2},I}< 1$, and $|I|+|I|(\|X\|_{p,I}+\|\bX\|_{\frac{p}{2},I})
<2|I|$. Thus, 
\begin{align}
\label{eq:bounds_pvars_solutions_RDEs:R^Y_in_proof}
\|R^Y\|_{\frac{p}{2},I}
&\leq
4C_{p,b,\sigma}
\left(
\|\bX\|_{\frac{p}{2},I}
+
|I|
+
\|Y\|_{p,I}^2
\right).
\end{align}
Thus, since $\|Y\|_p\leq C_{p,\sigma}(\|X\|_p+\|R^Y\|_\frac{p}{2})$ by \eqref{eq:controlled_path:pvar_norm}, we obtain 
\begin{align*}
\|Y\|_{p,I}
&\leq
C_2
\left(
\|X\|_{p,I}
+
\|\bX\|_{\frac{p}{2},I}
+
|I|
+
\|Y\|_{p,I}^2
\right)
\end{align*} 
for a new constant $C_2:=C_{p,b,\sigma}>4C_1>1$. 
Note that if $C_2\|Y\|_{p,I}\leq\frac{1}{2}$, then
\begin{align}\label{eq:||Y||p<=2C||X||}
\|Y\|_{p,I}
&\leq
2C_2
\left(
\|X\|_{p,I}
+
\|\bX\|_{\frac{p}{2},I}
+
|I|
\right)
% 
% 
% 
\leq
 C_{p,b,\sigma},
\end{align}
and \eqref{eq:bounds_pvars_solutions_RDEs:small_intervals:Y}
 is proven. If not and $C_2\|Y\|_{p,I}>\frac{1}{2}$, let $|I|$ be smaller so that  $\|Y\|_{p,I}=\frac{1}{2C_2}$.  
Then,
$$
\frac{1}{2C_2}=\|Y\|_{p,I}
\mathop{\leq}^\eqref{eq:||Y||p<=2C||X||}
2C_2
\left(
\|X\|_{p,I}
+
\|\bX\|_{\frac{p}{2},I}
+
|I|
\right)
\leq
2C_2
\alpha_1^\frac{1}{p}.
$$
Then, we obtain $\alpha_1\geq \left(\frac{1}{4C_2^2}\right)^p=:\alpha_2$. Thus,  it suffices to make $I$ smaller so that 
$
\|\mbX\|_{p,I}+|I|\leq
 \alpha_2^\frac{1}{p},
$ 
and  $\|Y\|_{p,I}=\frac{1}{2C_2}=:C_{p,b,\sigma}$, which concludes the proof of  \eqref{eq:bounds_pvars_solutions_RDEs:small_intervals:Y}.
The inequality 
\eqref{eq:bounds_pvars_solutions_RDEs:small_intervals:R}
follows from  \eqref{eq:bounds_pvars_solutions_RDEs:R^Y_in_proof} and \eqref{eq:bounds_pvars_solutions_RDEs:small_intervals:Y}.
The inequality \eqref{eq:bound:KY}  follows from the previous inequalities. 
\end{proof}







%\subsubsection*{Proof of Proposition \ref{prop:rdes:error_bound}}
\begin{proof}[Proof of Proposition \ref{prop:rdes:error_bound}]  
First, Theorem \ref{thm:rdes:existence_unicity} ensures that there exists two unique solutions $(Y,Y')\in\sD^p_X$ and $(\widetilde{Y},\widetilde{Y}')\in\sD^p_{\tilde{X}}$  to the RDEs. Second, % 
let $I=[t_0,t_1]\subseteq[0,T]$ be an interval, and $C_{p,b,\sigma}\geq1$ and $\alpha_1:=\alpha_{p,b,\sigma}$ be two constants from Proposition \ref{prop:bounds_pvars_solutions_RDEs} 
such that the inequalities in  \eqref{eq:bounds_pvars_solutions_RDEs:small_intervals:combined}
hold and
\begin{equation}\label{eq:rdes:error_bound:small_interval_alpha:proof}
\|\mbX\|_{p,I}+|I|\leq
 \alpha_1^\frac{1}{p},
 \quad
 \|\widetilde{\mbX}\|_{p,I}+|I|\leq
 \alpha_1^\frac{1}{p}.
 % 
% 
% 
% 
\end{equation}
% 
% 
% 
% 
% 
% 
By choosing $I$ small-enough so that $\|\mbX\|_{p,I}+\|\widetilde{\mbX}\|_{p,I}+|I|\leq
 \alpha_1^\frac{1}{p}$ % 
holds, we have shown \eqref{eq:bounds_pvars_solutions_RDEs:small_intervals:combined}. 

Next, we show \eqref{eq:RDE:DY'+dRY:close} (for an interval $I$ that is perhaps shorter).  
First, 
% 
% 
% 
% 
% 
% 
% 
% 
% 
% 
% 
% 
% 
% 
% 
{\small
\begin{align*}
\|\sigma(\cdot,Y)-\sigma(\cdot,\widetilde{Y})\|_{p,I}
&\mathop{\leq}^{\eqref{eq:rough_path:intfdX_fY:delta_sigma}} 
C_p\|\sigma\|_{C^2_b}(1 + K_{Y,I}+K_{\widetilde{Y},I})^2
(1+\|X\|_{p,I}+\|\tilde{X}\|_{p,I}+|I|)
\big(
\\
&\hspace{1cm}
\|\Delta X\|_{p,I}+\|\Delta Y_{t_0}\|+(\|\Delta Y_{t_0}'\|+\|\Delta Y'\|_{p,I})\|X\|_{p,I}+\|\Delta R^Y\|_{\frac{p}{2},I}\big)
\\
&\leq
C_{p,b,\sigma}\big(
\|\Delta\mbX\|_{p,I}+\|\Delta Y_{t_0}\|+\|\Delta Y'\|_{p,I}\|X\|_{p,I}+\|\Delta R^Y\|_{\frac{p}{2},I}\big),
\\
\|R^{\int_0^\cdot \sigma(s,Y_s)\dd\mbX_s}-R^{\int_0^\cdot \sigma(s,\widetilde{Y}_s)\dd\widetilde{\mbX}_s}\|_{\frac{p}{2},I}
&\mathop{\leq}^{\eqref{eq:rough_path:intfdX_fY:delta_R}} 
C_p\|\sigma\|_{C_b^3}
(1+K_{Y,I}+K_{\widetilde{Y},I}+|I|)^3
(1+\|\mbX\|_{p,I}+\|\widetilde{\mbX}\|_{p,I})^4
(1+|I|)
\Big(
\\
&\hspace{-4.cm}
\|\Delta\mbX\|_{p,I}+
\|\mbX\|_{p,I}\big(\|\Delta Y_{t_0}\|+
\|\Delta Y_{t_0}'\|
+
\|\Delta R^Y\|_{\frac{p}{2},I}+\|\Delta Y'\|_{p,I}+
(\|\Delta Y_{t_0}'\|+\|\Delta Y'\|_{p,I})\|X\|_{p,I}+\|\Delta X\|_{p,I}\big)\Big)
\\
&\leq
C_{p,b,\sigma}
\Big(
\|\Delta\mbX\|_{p,I}+\|\Delta Y_{t_0}\|
+
\|\mbX\|_{p,I}\big(\|\Delta Y'\|_{p,I}+\|\Delta R^Y\|_{\frac{p}{2},I}\big)\Big),
\\
\left\|\int_0^\cdot (b(s,Y_s,u_s)-b(s,\widetilde{Y}_s,u_s))\dd s\right\|_{\frac{p}{2},I} 
&\mathop{\leq}^{\eqref{eq:bounds_int_b_ds:dy_du}} 
C_{p,b}
|I|(\|\Delta Y\|_{\infty,I}+\|\Delta u\|_{L^\infty,I}
)
\\[-4mm]
&\mathop{\leq}^{\eqref{eq:path_finite_var:infty_ineq}}
C_{p,b}
|I|(\|\Delta Y_{t_0}\|+\|\Delta Y\|_{p,I}
+
\|\Delta u\|_{L^\infty,I}
)
\\[-1mm]
&\hspace{-5mm}\mathop{\leq}^{\eqref{eq:controlled_path:Y-Ytilde_p}}
C_{p,b}
|I|\big(
\|\Delta Y_{t_0}\|
+
\Delta M_{Y'}\|X\|_{p,I}+M_{\widetilde{Y}'}\|\Delta X\|_{p,I}+\|\Delta R^Y\|_{\frac{p}{2},I}
+
\|\Delta u\|_{L^\infty,I}
\big)
\\
&\hspace{-5mm}\leq
C_{p,b,\sigma}
|I|\big(
\|\Delta Y_{t_0}\|
+
\|\Delta Y_{t_0}'\|+\|\Delta Y'\|_{p,I}+
\|\Delta X\|_{p,I}+\|\Delta R^Y\|_{\frac{p}{2},I}
+
\|\Delta u\|_{L^\infty,I}
\big)
\\
&\hspace{-5mm}\leq
C_{p,b,\sigma}
\big(
\|\Delta\mbX\|_{p,I}
+
\|\Delta Y_{t_0}\|
+
|I|\|\Delta u\|_{L^\infty,I}
+
|I|
(
\|\Delta Y'\|_{p,I}+\|\Delta R^Y\|_{\frac{p}{2},I}
)
\big),
\end{align*}
}% 
where we also used $\|\Delta Y'_{t_0}\|\leq\|\sigma\|_{C_b^1}\|\Delta Y_{t_0}\|$ and $\|\Delta Y'\|_{p,I}=\|\sigma(\cdot,Y)-\sigma(\cdot,\widetilde{Y})\|_{p,I}$. % 

Then, for a constant $C=C_{p,b,\sigma}>1$ and any $\delta>1$, 
\begin{align*}
&\|\Delta Y'\|_{p,I}+\delta\|\Delta R^Y\|_{\frac{p}{2},I} 
\\
&\quad\leq 
\|\sigma(\cdot,Y)-\sigma(\cdot,\widetilde{Y})\|_{p,I}+
\delta
\left\|\int_0^\cdot (b(s,Y_s,u_s) - b(s,\widetilde{Y}_s,\tilde{u}_s))\dd s\right\|_{\frac{p}{2},I}
+
\delta
\left\|R^{\int_0^\cdot\sigma(s,Y_s)\dd\mbX_s}-R^{\int_0^\cdot\sigma(s,\widetilde{Y}_s)\dd\widetilde{\mbX}_s}\right\|_{\frac{p}{2},I}
\\
&\quad\leq 
C
\Big[
\|\Delta\mbX\|_{p,I}+\|\Delta Y_{t_0}\|+
\|\Delta Y'\|_{p,I}\|X\|_{p,I}+\|\Delta R^Y\|_{\frac{p}{2},I}
\,
+
\\
&\quad\qquad\qquad
\delta\left(
|I|\|\Delta u\|_{L^\infty,I}+
\|\Delta\mbX\|_{p,I}+\|\Delta Y_{t_0}\|+
(\|\mbX\|_{p,I}+|I|)\big(
\|\Delta Y'\|_{p,I}+\|\Delta R^Y\|_{\frac{p}{2},I}
\big)
\right)
\Big]
\\
&\quad\leq 
C
\Big[
(1+\delta)(\|\Delta\mbX\|_{p,I}+\|\Delta Y_{t_0}\|+|I|\|\Delta u\|_{L^\infty,I})
+
\\
&\quad\qquad\qquad
\big(\|X\|_{p,I}+\delta(\|\mbX\|_{p,I}+|I|)\big)
\|\Delta Y'\|_{p,I}
+
\big(
1+\delta(\|\mbX\|_{p,I}+|I|)
\big)
\|\Delta R^Y\|_{\frac{p}{2},I}
\Big],
\end{align*}
% 
Next, we  choose $\delta:=2C>2$, so that $C=\frac{\delta}{2}$, and $I$ small-enough so that 
\begin{equation}\label{eq:rdes:error_bound:small_interval_alpha:proof:2}
\|\mbX\|_{p,I}+|I|\leq \frac{1}{4C\delta},
\end{equation}
so that
$C\delta(\|\mbX\|_{p,I}+|I|)\leq\frac{1}{4}$  
% 
and $C\|X\|_p\leq\frac{1}{4\delta}\leq\frac{1}{4}$. Then, after rearranging, the previous inequality becomes
\begin{align*}
\left(1-\frac{1}{2}\right)\|\Delta Y'\|_{p,I}+\left(\delta-\frac{\delta}{2}-\frac{1}{4}\right)\|\Delta R^Y\|_{\frac{p}{2},I}
&\leq 
C
(1+\delta)(\|\Delta\mbX\|_{p,I}+\|\Delta Y_{t_0}\|+|I|\|\Delta u\|_{L^\infty,I}),
\end{align*}
Thus, since $\delta>2$ and $(1+\delta)\leq 2\delta=4C$, we obtain
\begin{align*}
\|\Delta Y'\|_{p,I}+\|\Delta R^Y\|_{\frac{p}{2},I}
&\leq 
\|\Delta Y'\|_{p,I}+\left(\delta-\frac{1}{2}\right)\|\Delta R^Y\|_{\frac{p}{2},I}
\leq 
2C(1+\delta)
(\|\Delta\mbX\|_{p,I}+\|\Delta Y_{t_0}\|+
|I|\|\Delta u\|_{L^\infty,I}
).
\\
&\leq 
8 C^2
(\|\Delta\mbX\|_{p,I}+\|\Delta Y_{t_0}\|+
|I|\|\Delta u\|_{L^\infty,I}
).
\end{align*}
% 
Finally, by choosing $\alpha_{p,b,\sigma}=\min(\alpha_1,1/(4C\delta)^p)$ and $I$ small-enough so that $\|\mbX\|_{p,I}+\|\widetilde{\mbX}\|_{p,I}+|I|\leq
 \alpha_{p,b,\sigma}^\frac{1}{p}$ % 
holds (and in particular, \eqref{eq:rdes:error_bound:small_interval_alpha:proof}
and  \eqref{eq:rdes:error_bound:small_interval_alpha:proof:2} hold), we obtain \eqref{eq:RDE:DY'+dRY:close} and conclude the proof. 
\end{proof}

\subsubsection*{Error bounds on long intervals (Section \ref{sec:rdes:bounds:long})}
%\subsubsection*{Proof of Proposition \ref{prop:rdes:error_bound:entire_interval}}
\begin{proof}[Proof of Proposition \ref{prop:rdes:error_bound:entire_interval}] 
By \eqref{eq:|X|+|Xtilde|+|dt|<=alpha^p} in  Corollary \ref{cor:Nalpha:NX_NXtilde_NT:small_intervals} and by Proposition \ref{prop:rdes:error_bound}, there exists constants $C_{p,b,\sigma}\geq 1$ and $\alpha=\alpha_{p,b,\sigma}>0$ such that   \eqref{eq:bounds_pvars_solutions_RDEs:small_intervals:combined} holds and
\begin{align*}
\|Y'-\widetilde{Y}'\|_{p,[s,t]}+\|R^Y-R^{\widetilde{Y}}\|_{\frac{p}{2},[s,t]}
\mathop{\leq}^{\eqref{eq:RDE:DY'+dRY:close}}
C_{p,b,\sigma}(\|\Delta Y_s\|
+
\|\Delta\mbX\|_{p,[s,t]}
+
|t-s|\|\Delta u\|_{L^\infty,[s,t]}
)
\end{align*}
for any $[s,t]\subseteq[0,T]$ such that $w(s,t)\leq
 \alpha$. Thus, as defined in Definition \ref{def:Nalpha}, the greedy partition 
 $\{\tau_i, i=0,1,\dots,N_{\alpha,I}(w)+1\}$ 
of the interval $I$, which satisfies $w(\tau_i,\tau_{i+1})\leq \alpha$ for all $i$, is such that
\begin{align}
\nonumber
\|\Delta Y'\|_{p,[\tau_i,\tau_{i+1}]}+\|\Delta R^Y\|_{\frac{p}{2},[\tau_i,\tau_{i+1}]}
&\leq 
C_{p,b,\sigma}
(
\|\Delta Y_{\tau_i}\|
+
\|\Delta\mbX\|_{p,[\tau_i,\tau_{i+1}]}
+
|\tau_{i+1}-\tau_i|\|\Delta u\|_{L^\infty,[\tau_i,\tau_{i+1}]}
)
\\
\label{eq:DY'_p_ti_ti+1}
&\leq
C_{p,b,\sigma}
(
\|\Delta Y_{\tau_i}\|
+
\|\Delta\mbX\|_{p,I}
+
|I|\|\Delta u\|_{L^\infty,I}
)
\end{align}
for all $i$, and $\|X\|_{p,[\tau_i,\tau_{i+1}]}\mathop{\leq}\limits^{\eqref{eq:|X|+|Xtilde|+|dt|<=alpha^p}} C_{p,b,\sigma}$, and $M_{Y',[\tau_i,\tau_{i+1}]}=\|Y_{\tau_i}'\|+\|Y'\|_{p,[\tau_i,\tau_{i+1}]}\mathop{\leq}\limits^{\eqref{eq:bounds_pvars_solutions_RDEs:small_intervals:combined}} K_{Y,[\tau_i,\tau_{i+1}]}\leq C_{p,b,\sigma}$.

Next, with $\Delta M_{Y',[\tau_i,\tau_{i+1}]}=\|\Delta Y_{\tau_i}'\|+\|\Delta Y'\|_{p,[\tau_i,\tau_{i+1}]}$, % 
and since $\|\Delta Y'_{\tau_i}\|\leq\|\sigma\|_{C_b^1}\|\Delta Y_{\tau_i}\|$, 
\begin{align}
\nonumber
\|\Delta Y\|_{p,[\tau_i,\tau_{i+1}]}
&\mathop{\leq}^\eqref{eq:controlled_path:Y-Ytilde_p} 
C_p\big(\Delta M_{Y',[\tau_i,\tau_{i+1}]}\|X\|_{p,[\tau_i,\tau_{i+1}]}+M_{\widetilde{Y}',[\tau_i,\tau_{i+1}]}\|\Delta X\|_{p,[\tau_i,\tau_{i+1}]}+\|\Delta R^Y\|_{\frac{p}{2},[\tau_i,\tau_{i+1}]}\big)
\\
\nonumber
&\mathop{\leq}^{\eqref{eq:bounds_pvars_solutions_RDEs:small_intervals:combined}}C_{p,b,\sigma}\big(
\|\Delta Y_{\tau_i}'\|+\|\Delta Y'\|_{p,[\tau_i,\tau_{i+1}]}
+ \|\Delta X\|_{p,[\tau_i,\tau_{i+1}]}+\|\Delta R^Y\|_{\frac{p}{2},[\tau_i,\tau_{i+1}]}\big)
\\
\label{eq:DY_p_ti_ti+1}
&\leq
C_{p,b,\sigma}
\big(
\|\Delta Y_{\tau_i}\|
+ \|\Delta X\|_{p,I}+\|\Delta Y'\|_{p,[\tau_i,\tau_{i+1}]}+\|\Delta R^Y\|_{\frac{p}{2},[\tau_i,\tau_{i+1}]}\big),
\end{align} 
so that, since $\|\Delta Y_{\tau_{i+1}}\|
\leq
\|\Delta Y_{\tau_i}\|+\|\Delta Y\|_{p,[\tau_i,\tau_{i+1}]}$ by \eqref{eq:path_finite_var:infty_ineq},
\begin{align}\label{eq:DeltaYti+1}
\|\Delta Y_{\tau_{i+1}}\|
&\mathop{\leq}^{
\eqref{eq:path_finite_var:infty_ineq},
\eqref{eq:DY_p_ti_ti+1}
}
C_{p,b,\sigma}
\big(
\|\Delta Y_{\tau_i}\|
+ \|\Delta X\|_{p,I}+\|\Delta Y'\|_{p,[\tau_i,\tau_{i+1}]}+\|\Delta R^Y\|_{\frac{p}{2},[\tau_i,\tau_{i+1}]}\big),
% 
% 
% 
\\
\nonumber
\|\Delta Y\|_{\infty,[\tau_i,\tau_{i+1}]}
&\mathop{\leq}^{
\eqref{eq:path_finite_var:infty_ineq},
\eqref{eq:DY_p_ti_ti+1}
}
C_{p,b,\sigma}
\big(
\|\Delta Y_{\tau_i}\|
+ \|\Delta X\|_{p,I}+\|\Delta Y'\|_{p,[\tau_i,\tau_{i+1}]}+\|\Delta R^Y\|_{\frac{p}{2},[\tau_i,\tau_{i+1}]}\big)
\\
\label{eq:DY_inf_ti_ti+1}
&\mathop{\leq}^{\eqref{eq:DY'_p_ti_ti+1}}
C_{p,b,\sigma}
\big(
\|\Delta Y_{\tau_i}\|
+
\|\Delta\mbX\|_{p,I}
+
|I|\|\Delta u\|_{L^\infty,I}
\big).
\end{align} 
By sequentially combining \eqref{eq:DY'_p_ti_ti+1},
\eqref{eq:DeltaYti+1} and \eqref{eq:DY_inf_ti_ti+1} over the intervals $[\tau_i,\tau_{i+1}]$, we obtain 
% 
% 
% 
% 
% 
% 
% 
% 
% 
% 
% 
% 
% 
% 
% 
% 
% 
% 
% 
% 
% 
% 
% 
% 
% 
% 
% 
% 
% 
% 
% 
% 
% 
% 
% 
% 
% 
% 
% 
% 
% 
% 
% 
% 
% 
for all $i$,
\begin{align*}
\|\Delta Y'\|_{p,[\tau_i,\tau_{i+1}]}+\|\Delta R^Y\|_{\frac{p}{2},[\tau_i,\tau_{i+1}]}
% 
&\mathop{\leq}^{
\eqref{eq:DY'_p_ti_ti+1},
\eqref{eq:DeltaYti+1}
} 
(C_{p,b,\sigma})^{i+1}(i+1)
\left(
\|\Delta Y_{t_0}\|
+
\|\Delta\mbX\|_{p,I}
+
|I|\|\Delta u\|_{L^\infty,I}
\right),
\\
\|\Delta Y\|_{\infty,[\tau_i,\tau_{i+1}]}
% 
% 
% 
% 
% 
&\mathop{\leq}^{
\ \ \ \, 
\eqref{eq:DY_inf_ti_ti+1}
\ \ \ \, } 
(C_{p,b,\sigma})^{i+1}(i+1)
\left(
\|\Delta Y_{t_0}\|
+
\|\Delta\mbX\|_{p,I}
+
|I|\|\Delta u\|_{L^\infty,I}
\right).
\end{align*}
Thus, for $N:=N_{\alpha,I}(w)$, using Lemma \ref{lem:pvar:intervals}, % 
\begin{align*}
\|\Delta Y'\|_{p,I} 
+
\|\Delta R^Y\|_{\frac{p}{2},I}
&\leq 
(N+1)
\bigg(
\bigg(
\sum_{j=0}^N\|\Delta Y'\|_{p,[\tau_j,\tau_{j+1}]}^p
\bigg)^\frac{1}{p}
+
\bigg(
\sum_{j=0}^N\|R^Y\|_{\frac{p}{2},[\tau_j,\tau_{j+1}]}^\frac{p}{2}
\bigg)^\frac{2}{p}
\bigg)
\\
&\leq
(N+1)
\bigg(
\sum_{j=0}^N\|\Delta Y'\|_{p,[\tau_j,\tau_{j+1}]}
+
\|R^Y\|_{\frac{p}{2},[\tau_j,\tau_{j+1}]}
\bigg)
\\
&\leq
(C_{p,b,\sigma})^{N+1}(N+1)^3
\left(
\|\Delta Y_{t_0}\|
+
\|\Delta\mbX\|_{p,I}
+
|I|\|\Delta u\|_{L^\infty,I}
\right),
\\
\|\Delta Y\|_{\infty,I}
&\leq
(C_{p,b,\sigma})^{N+1}(N+1)^3
\left(
\|\Delta Y_{t_0}\|
+
\|\Delta\mbX\|_{p,I}
+
|I|\|\Delta u\|_{L^\infty,I}
\right),
\end{align*}
where we used $\|\Delta Y\|_{\infty,I}\leq \max_i\|\Delta Y\|_{\infty,[\tau_i,\tau_{i+1}]}\leq (N+1)\max_i\|\Delta Y\|_{\infty,[\tau_i,\tau_{i+1}]}$ in the last inequality. 
The desired inequalities  \eqref{eq:RDE:DY'+dRY:close:full_interval} and \eqref{eq:RDE:DY:close:full_interval}  follow using $(C_{p,b,\sigma})^{N+1}(N+1)^3\leq \exp((N+1)\log(C_{p,b,\sigma}))3!\exp(N+1)=6eC_{p,b,\sigma}\exp((\log(C_{p,b,\sigma})+1)N)$. 


Finally, if $u$ and $\tilde{u}$ only differ on an interval $J=[s_0,s_1]\subseteq I$, then    \eqref{eq:RDE:DY'+dRY:close:full_interval:u_subinterval} and \eqref{eq:RDE:DY:close:full_interval:u_subinterval} follow  from sequentially applying \eqref{eq:RDE:DY'+dRY:close:full_interval}
 and 
\eqref{eq:RDE:DY:close:full_interval}
on  
$[t_0,s_0]
\cup
J
\cup
[s_1,t_1]=I$, noting that $\Delta u_t=0$ for almost every 
$t\in 
[t_0,s_0]
\cup
[s_1,t_1]$.
\end{proof}


\subsubsection{Bounds on  solutions to linear RDEs and on the Jacobian flow (Section \ref{sec:rdes:bounds:linear})}
%\subsubsection*{Proof of Lemma \ref{lem:rde:linear:bounded_solutions}} 

The proof of Lemma \ref{lem:rde:linear:bounded_solutions} follows similar steps as  the proof of Proposition \ref{prop:rdes:error_bound:entire_interval} and relies on the following   Gr\"onwall Lemma for rough paths.
\begin{lemma}[Rough Gr\"onwall Lemma {\cite[Lemma 2.12]{Deya2019}}]
\label{lem:rough_gronwall}
Let $p\geq 1$, $T>0$, $C_1,\alpha>0$, $Y\in C([0,T],\R^n)$, and $w_1,w_2$  be two controls on $[0,T]$ (see Definition \ref{def:Nalpha}) such that
$$
\|Y_{s,t}\|\leq C_1\|Y\|_{\infty,[0,t]}w_1(s,t)^\frac{1}{p}+w_2(s,t)
\ \ \text{for any } [s,t]\subseteq[0,T] \text{ such that }w_1(s,t)\leq \alpha,
$$
and define $C_2=\min(1, 1/(\alpha(2C_1\exp(2))^p))$. 
Then,
$$
\|Y\|_{\infty,[0,T]}
\leq
2\exp\left(\frac{w_1(0,T)}{C_2\alpha}\right)
\left(
\|Y_0\|
+
\left\|w_2(0,\cdot)\exp\left(-w_1(0,\cdot)/(C_2 \alpha)\right)
\right\|_{\infty,[0,T]}
\right).
$$
\end{lemma} 




\begin{proof}[Proof of Lemma \ref{lem:rde:linear:bounded_solutions}]
% 
 By Theorem \ref{thm:rde:linear:existence_uniqueness}, there exists a unique solution $(V,V')\in\sD^p_X$ with $V'=\Sigma V$ to the linear RDE. Let $I=[s,t]\subseteq[0,T]$ be an interval.  By Lemma \ref{lem:control_path:product}, with $(\Sigma V)'=\Sigma V'+V\Sigma'$, % 
\begin{align}
\label{eq:V'=SigmaV_p<=sum|Sigma|p(|V|inf+|V|p}
\|V'\|_{p,I}% 
&\mathop{\leq}^{\eqref{lem:control_path:product:YZ_p}}
C_p(\|\Sigma\|_{\infty,I}\|V\|_{p,I}+\|V\|_{\infty,I}\|\Sigma \|_{p,I})
\leq
C_p(\|\Sigma \|_{\infty,I}+\|\Sigma \|_{p,I})(\|V\|_{\infty,I}+\|V\|_{p,I}),
\\
\nonumber
\|(\Sigma V)'\|_{p,I}
&\mathop{\leq}^{\eqref{lem:control_path:product:(YZ)'_p}}
C_p(\|V\|_{\infty,I}\|\Sigma '\|_{p,I}+\|\Sigma '\|_{\infty,I}\|V\|_{p,I}+\|\Sigma \|_{\infty,I}\|V'\|_{p,I}+\|V'\|_{\infty,I}\|\Sigma \|_{p,I})
\\
\nonumber
&\mathop{\leq}^{\eqref{eq:V'=SigmaV_p<=sum|Sigma|p(|V|inf+|V|p}}
C_p(\|\Sigma \|_{\infty,I}+\|\Sigma \|_{p,I}+\|\Sigma '\|_{\infty,I}+\|\Sigma '\|_{p,I})^2(\|V\|_{\infty,I}+\|V\|_{p,I}),
\\ 
\nonumber
\|R^{\Sigma V}\|_{\frac{p}{2},I}
&\mathop{\leq}^{\eqref{lem:control_path:product:R^YZ_p/2}}
C_p(\|\Sigma \|_{\infty,I}\|R^V\|_{\frac{p}{2},I}+\|R^\Sigma \|_{\frac{p}{2},I}\|V\|_{\infty,I}+\|\Sigma \|_{p,I}\|V\|_{p,I})
\\
\nonumber
&\leq
C_p(\|\Sigma \|_{\infty,I}+\|\Sigma \|_{p,I}+\|R^\Sigma \|_{\frac{p}{2},I})
(\|V\|_{\infty,I}+\|V\|_{p,I}+\|R^V\|_{\frac{p}{2},I}).
\end{align}
Let $I$ be small-enough to satisfy $\|\mbX\|_{p,I}+|I|\leq
 \alpha_\Sigma ^\frac{1}{p}$. Then, by Proposition \ref{prop:rough_integral_welldefined:error_bound},
\begin{align*}
\|R_{s,t}^V\| 
&=
\|V_{s,t}-\Sigma _sV_sX_{s,t}\|
\leq
\left\|\int_s^tA _rV_r\dd r \right\|
+
\left\|\int_s^t\Sigma _rV_r\dd\mbX_r-\Sigma _sV_sX_{s,t}-(\Sigma V)'_s\bX_{s,t}\right\|+\|(\Sigma V)'_s\bX_{s,t}\|
\\
&\mathop{\leq}^{\eqref{eq:rough_int:error_bound}}
\|A\|_{\infty,I}\|V\|_{\infty,I}|I|
+
C_p(\|R^{\Sigma V}\|_{\frac{p}{2},I}\|X\|_{p,I}+\|(\Sigma V)'\|_{p,I}\|\bX\|_{\frac{p}{2},I})+\|(\Sigma V)'\|_{\infty,I}\|\bX\|_{\frac{p}{2},I}.
\end{align*}
Thus, using  \eqref{eq:path_finite_var:sum_p/2vars:p/2vars_subintervals} in Lemma \ref{lem:sum_p/2vars:p/2vars_subintervals} (which is similar to \eqref{eq:path_finite_var:sum_p/2vars} in Lemma \ref{lem:pvariation:inequalities}) and the previous inequalities,
\begin{align*}
\|R^V\|_{\frac{p}{2},I}  
&\leq 
C_p
(
\|A\|_{\infty,I}+
\|\Sigma \|_{\infty,I}+\|\Sigma \|_{p,I}+\|\Sigma '\|_{\infty,I}+\|\Sigma '\|_{p,I}+\|R^\Sigma \|_{\frac{p}{2},I}
)^2
\big(
\\
&\hspace{2cm}
\|V\|_{\infty,I}|I|
+
(\|V\|_{\infty,I}+\|V\|_{p,I}+\|R^V\|_{\frac{p}{2},I})
\|X\|_{p,I}
+
(\|V\|_{\infty,I}+\|V\|_{p,I})\|\bX\|_{\frac{p}{2},I}
\big)
\\
&\leq
C_{p,A,\Sigma}\big(
\|V\|_{\infty,I}|I|
+
(\|V\|_{\infty,I}+\|V\|_{p,I}+\|R^V\|_{\frac{p}{2},I})
\|X\|_{p,I}+
(\|V\|_{\infty,I}+\|V\|_{p,I})\|\bX\|_{\frac{p}{2},I}
\big).
\end{align*}
Next, we choose $I$ small-enough to remove $\|R^V\|_{\frac{p}{2},I}$ in the inequality above.  
Let $\alpha_1:=\left(1/(2C_{p,A,\Sigma })\right)^p \leq 1$ and choose  $I$ small-enough so that $\|X\|_{p,I}^p\leq\alpha_1$. Then,  $C_{p,A,\Sigma }\|X\|_{p,I}\leq\frac{1}{2}$, 
% 
and we obtain
\begin{align}
\|R^V\|_{\frac{p}{2},I}  
&\leq
% 
% 
% 
% 
C_{p,A,\Sigma } 
(\|V\|_{p,I}+\|V\|_{\infty,I})
(|I|+\|\mbX\|_{p,I})
\label{eq:R^Vp/2_I<=C_Vp_Vinf_I+Xp}
\end{align}
for a new constant $C_{p,A,\Sigma }$. 
Using the equation above, $V_{s,t}=\Sigma _sV_sX_{s,t}+R^V_{s,t}$, and $\|\Sigma \|_{\infty,I}\leq C_\Sigma $, we obtain
$$
\|V\|_{p,I}
\mathop{\leq}^{\eqref{eq:path_finite_var:sum_pvars}}
C_p( 
\|\Sigma\|_{\infty,I}\|V\|_{\infty,I}\|X\|_{p,I}+\|R^V\|_{\frac{p}{2},I}
)
\mathop{\leq}^{\eqref{eq:R^Vp/2_I<=C_Vp_Vinf_I+Xp}} 
C_{p,A,\Sigma } 
(
\|V\|_{p,I}
+
\|V\|_{\infty,I}
)
(
|I|+\|\mbX\|_{p,I}
).
$$
Define the control $w$ by $w(s,t)=C_p(|t-s|+w_{\mbX}(s,t))$ with $w_{\mbX}(s,t)=\|X\|_{p,[s,t]}^p+\|\bX\|_{\frac{p}{2},[s,t]}^{\frac{p}{2}}$ and $C_p=6^p$ as in Corollary \ref{cor:Nalpha:NX_NXtilde_NT:small_intervals}, and let $\alpha:=\left(1/(2C_{p,A,\Sigma})\right)^p\leq \alpha_1\leq 1$. Then, by choosing $I$ small-enough so that $w(s,t)\leq\alpha$, we have $|I|+\|\mbX\|_{p,I}\mathop{\leq}\limits^{\eqref{eq:|X|+|Xtilde|+|dt|<=alpha^p}}\alpha^\frac{1}{p}$, so that    
$C_{p,A,\Sigma }(|I|+\|\mbX\|_{p,I})\leq\frac{1}{2}$. Then,  we obtain
\begin{align}
\nonumber
\|V\|_{p,I}
&\leq  
C_{p,A,\Sigma } 
\|V\|_{\infty,I}
(|I|+\|\mbX\|_{p,I})
\mathop{\leq}^{\eqref{eq:mbX^p<=wX}} 
C_{p,A,\Sigma } 
\|V\|_{\infty,I}
(|I|+w_{\mbX}(s,t)^\frac{1}{p})
\\
\label{eq:Vp_I<=C_Vinf_w(s,t)^1/p}
&\leq
C_{p,A,\Sigma } 
\|V\|_{\infty,I}
(|I|^\frac{1}{p}+w_{\mbX}(s,t)^\frac{1}{p})
\leq
C_{p,A,\Sigma } 
\|V\|_{\infty,I}
w(s,t)^\frac{1}{p}
\end{align}
for a new constant $C_{p,A,\Sigma }\geq 0$, 
where we used $|I|\leq|I|^\frac{1}{p}$ since $|I|\leq 1$ and $p\geq 1$, and $(|a|+|b|)^p\leq 2^p(|a|^p+|b|^p)$ in the last line. 
 % 
Then, by applying the rough Gr\"onwall Lemma (Lemma \ref{lem:rough_gronwall}), we obtain 
\begin{equation}\label{eq:Vinf<=CVs:gronwall}
\|V\|_{\infty,I}
\leq
2\exp\left(w(s,t)/(C\alpha)\right)
\|V_s\|
\leq
2\exp\left(1/C\right)
\|V_s\|,
\end{equation}
for any interval $I\subseteq[0,T]$ such that $w(s,t)\leq\alpha$, and where $C>0$ and $\alpha>0$ only depend on $(p,A,\Sigma)$. 
By sequentially using this inequality on the greedy partition $\{\tau_i\}_{i=1}^{N_{\alpha,[0,T]}(w)+1}$ of $[0,T]$ (which satisfies $w(\tau_i,\tau_{i+1})\leq\alpha$ for all $i$, see Definition \ref{def:Nalpha}) as in the end of the proof of  Proposition \ref{prop:rdes:error_bound:entire_interval}, we obtain
$$
\|V\|_{\infty,[0,T]}
\leq 
\big(2\exp(1/C)\big)^{N_{\alpha,[0,T]}(w)+1}
\|v\|
\leq 
\tilde{C}\exp\big(\tilde{C}N_{\alpha,[0,T]}(w)\big)\|v\|
$$
for a new constant $\tilde{C}>0$. 
Finally,  $N_{\alpha,[s,t]}(w)\leq C_p(
N_{\alpha,[s,t]}(\mbX)+T/\alpha+1
)$ by \eqref{eq:Nalpha<=3CpNalpha_X_and_time} in Corollary \ref{cor:Nalpha:NX_NXtilde_NT:small_intervals}, and the desired inequality \eqref{eq:RDE:linear:V:bound} follows.


To show the inequality \eqref{eq:RDE:linear:Vp_V'p_RVp/2:bound}, from the previous inequalities, % 
% 
% 
we have
\begin{align*}
\|V\|_{p,I}
+
\|V'\|_{p,I}
+
\|R^V\|_{\frac{p}{2},I}
&\mathop{\leq}^{
\eqref{eq:Vp_I<=C_Vinf_w(s,t)^1/p},\eqref{eq:V'=SigmaV_p<=sum|Sigma|p(|V|inf+|V|p},\eqref{eq:R^Vp/2_I<=C_Vp_Vinf_I+Xp}}
% 
C\|V\|_{\infty,I}
\big(
w(s,t)^\frac{1}{p}
+
(1+w(s,t)^\frac{1}{p})
+
(w(s,t)^\frac{1}{p}+1)
\big)
\\
&\mathop{\leq}^{\eqref{eq:Vinf<=CVs:gronwall}}
C\|V_s\|
\end{align*}
for any interval $I\subseteq[0,T]$ with $w(s,t)\leq\alpha$. Again, by sequentially using this inequality on the greedy partition $\{\tau_i\}_{i=1}^{N_{\alpha,[0,T]}(w)+1}$  as in the end of the proof of  Proposition \ref{prop:rdes:error_bound:entire_interval},  the  inequality \eqref{eq:RDE:linear:Vp_V'p_RVp/2:bound}  follows.
\end{proof}


%\subsubsection*{Proof of Lemma \ref{lem:rde:linearized:bounded_solutions}} 
\begin{proof}[Proof of Lemma \ref{lem:rde:linearized:bounded_solutions}]
% 
By Theorem \ref{thm:rdes:existence_unicity} and  Corollary  \ref{cor:rde:linear:existence_uniqueness}, there exists unique solutions to the two RDEs.  
First, by Proposition \ref{prop:bounds_pvars_solutions_RDEs}, there exists two constants $C_{p,b,\sigma}\geq 1$ and $0<\alpha_{p,b,\sigma}<1$ such that 
\begin{align*} 
\|Y\|_{p,I}+\|R^Y\|_{\frac{p}{2},I} + 
\|Y'\|_{\infty,I}+\|Y'\|_{p,I}+\|R^Y\|_{\frac{p}{2},I}
&\leq
C_{p,b,\sigma}.
\end{align*}
for any interval $I=[t_0,t_1]\subseteq[0,T]$ such that $\|\mbX\|_{p,I}+|I|\leq\alpha_{p,b,\sigma}^\frac{1}{p}$, noting that $Y'=\sigma(\cdot,Y_\cdot)$ with $\sigma\in C^3_b$. 

Second, let $A=\frac{\partial b}{\partial x}(\cdot,Y_\cdot,u_\cdot)$, $\Sigma=\frac{\partial\sigma}{\partial x}(\cdot,Y_\cdot)$ and $\Sigma'=\frac{\partial^2\sigma}{\partial x^2}(\cdot,Y_\cdot)Y'_\cdot$. By Lemma \ref{lem:rough_path:sigma(.,Y):controlled}, since $\sigma_x:=\frac{\partial\sigma}{\partial x}\in C^2_b$,  
$(\Sigma,\Sigma')\in\sD^p_X$,  $K_{Y,I}=\|Y'_{t_0}\|+\|Y'\|_{p,I}+\|R^Y\|_{\frac{p}{2},I}\leq C_{p,b,\sigma}$,  $M_{Y',I}\leq K_{Y,I}\leq C_{p,b,\sigma}$, and $\|X\|_{p,I}+|I|\leq\alpha_{p,b,\sigma}^\frac{1}{p}$,
\begin{align*}   
\|\Sigma\|_{p,I}
&\mathop{\leq}^{\eqref{eq:sigma(.,Y):pvar}} 
C_p\|\sigma_x\|_{C^1_b}
(
M_{Y',I}\|X\|_{p,I}+\|R^Y\|_{\frac{p}{2},I}
+
|I|
)
\leq C_{p,b,\sigma},
\\ 
\|\Sigma'\|_{p,I}
&\mathop{\leq}^{\eqref{eq:sigma(.,Y)':pvar}} 
C_p\|\sigma_x\|_{C^2_b}
K_{Y,I}(1+K_{Y,I}+|I|)(1+\|X\|_{p,I})
\leq C_{p,b,\sigma},
\\
\|R^{\Sigma}\|_{\frac{p}{2},I}
&\mathop{\leq}^{\eqref{eq:RY:p/2var:Y^2+RY+T}} 
C_p\|\sigma_x\|_{C^2_b}(
\|Y\|_{p,I}^2
+
\|R^Y\|_{\frac{p}{2},I}
+
|I|
)
\leq C_{p,b,\sigma}.
\end{align*}
Also,  $\|A\|_\infty\leq C_b$ and $\|\Sigma\|_{\infty,I}\leq C_\sigma$. 
Thus, for a constant $C_{p,b,\sigma}\geq 1$,  
\begin{align*} 
\|\Sigma\|_{\infty,I} + \|\Sigma\|_{p,I} +\|\Sigma'\|_{p,I}+\|R^{\Sigma}\|_{\frac{p}{2},I}
&\leq
C_{p,b,\sigma}
\end{align*}
for any interval $I\subseteq[0,T]$ such that $\|\mbX\|_{p,I}+|I|\leq
 \alpha_{p,b,\sigma}^\frac{1}{p}$.   
The conclusion follows from Lemma \ref{lem:rde:linear:bounded_solutions}.
\end{proof}










































\subsection{Additional proofs for the  PMP  (Section \ref{sec:pmp})}\label{apdx:proofs:pmp}
\subsubsection{Needle-like variations (Section \ref{sec:linearization})}
%\subsubsection*{Proof of Corollary \ref{cor:needle_like_error:etas}}

Corollary \ref{cor:needle_like_error:etas} below extends Proposition \ref{prop:linear_variation} to needle-like variations with multiple spikes. 
As in the deterministic case, the proof proceeds by  induction, using Proposition \ref{prop:linear_variation} for the case with one variation $\pi_1=\{t_1,\eta_1,\bar{u}_1\}$.
\begin{corollary}[Needle-like variations]\label{cor:needle_like_error:etas}
Define $p,T,y,U,u,b,\sigma,\mbX,N_{\alpha,[0,T]}(\mbX),Y,Y'$  as in Proposition \ref{prop:linear_variation}. Given $q\in\N$, let $0<t_1<\dots<t_q<T$ be Lebesgue points of $b$ for $u$ (Definition \ref{def:lebesgue_point}), 
$\bar{u}_1,\dots,\bar{u}_q\in U$, 
 $0\leq \eta_i< t_{i+1}-t_i$ for $i=1,\dots, q-1$ and $0\leq\eta_q< T-t_q$, and define the needle-like variation $\pi=\{t_1,\dots,t_q,\eta_1,\dots,\eta_q,\bar{u}_1,\dots,\bar{u}_q\}$ of $u$ as the control $u^\pi$ defined by% 
 $$
 u^\pi_t=\begin{cases}
 \bar{u}_i\quad&\text{if }t\in[t_i,t_i+\eta_i],
 \\
 u_t&\text{otherwise}.
 \end{cases}
 $$ 
Let $(Y^\pi,(Y^\pi)')\in\sD^p_X$ and  $(V^{\pi_i},(V^{\pi_i})')\in\sD^p_X$ for $i=1,\dots,q$ be the unique solutions to the RDEs
\begin{align*}
Y^\pi_t&=y+\int_0^tb(s,Y^\pi_s,u^\pi_s)\dd s+\int_0^t\sigma(s,Y^\pi_s)\dd\mbX_s,
 &&\hspace{-10mm}
 t\in[0,T], 
\\
V^{\pi_i}_t &= 
V^{\pi_i}_{t_i}
+
\int_{t_i}^t
\frac{\partial b}{\partial x}(s,Y_s,u_s)V^{\pi_i}_s\dd s +
\int_{t_i}^t
\frac{\partial\sigma}{\partial x}(s,Y_s)V^{\pi_i}_s\dd\mbX_s,
 &&\hspace{-10mm}
 t\in[t_i,T],
\\
V^{\pi_i}_t&=b(t_i,Y_{t_i},\bar{u}_i)-b(t_i,Y_{t_i},u_{t_i}),
 &&\hspace{-10mm}
 t\in[0,t_i].
\end{align*}
with  
 $(Y^{\pi_1})'=\sigma(\cdot,Y^{\pi_1}_\cdot)$  and   $(V^{\pi_1})'=\frac{\partial\sigma}{\partial x}(\cdot,Y_\cdot)V^{\pi_1}_\cdot$. 
 % 
Then, there exists constants $C_{p,T,b,\sigma}>0$ and $0<\alpha_{p,b,\sigma}<1$ such that % 
\begin{align}
\label{eq:needle_like_deltasols:multi}
\|Y^\pi-Y\|_{\infty,[0,T]}
&\leq 
C_{p,T,b,\sigma}\exp\left(
C_{p,T,b,\sigma}N_{\alpha_{p,b,\sigma},[0,T]}(\mbX)
\right)
\sum_{i=1}^q\eta_i,
\\
\label{eq:needle_like_deltasols_variation:multi}
\Big\|Y^\pi-Y-\sum_{i=1}^q\eta_iV^{\pi_i}\Big\|_{\infty,[t_1,T]}&\leq 
C_{p,T,b,\sigma}\exp\left(
C_{p,T,b,\sigma}N_{\alpha_{p,b,\sigma},[0,T]}(\mbX)
\right)
\sum_{i,j=1}^q\eta_i\eta_j.
\end{align}
\end{corollary}
% 
\begin{proof}[Proof of Corollary \ref{cor:needle_like_error:etas}] 
As in Proposition \ref{prop:linear_variation}, the first inequality  \eqref{eq:needle_like_deltasols:multi}  follows 
from sequentially applying 
% 
% 
% 
\eqref{eq:RDE:DY'+dRY:close:full_interval:u_subinterval}
and 
\eqref{eq:RDE:DY:close:full_interval:u_subinterval} 
on  
$[0,t_1]
\cup
[t_1,t_1+\eta_1]
\cup
[t_1+\eta_1,t_2]
\cup
\dots
\cup
[t_q,t_q+\eta_q]
\cup
[t_q+\eta_q,T]=[0,T]$, and concluding with 
% 
% 
% 
\eqref{eq:Nalpha<=3CpNalpha_X_and_time} in Corollary \ref{cor:Nalpha:NX_NXtilde_NT:small_intervals}.
  

 \noindent\begin{minipage}[c]{0.55\textwidth}
% 
% 
Next, we show the second inequality \eqref{eq:needle_like_deltasols_variation:multi}. 
The case $q=1$ is in Proposition \ref{prop:linear_variation}, so we prove \eqref{eq:needle_like_deltasols_variation:multi}  for $q\geq 2$ by induction,   assuming that it holds for $q$ and proving it for $q+1$ for a needle-like variation $\pi=\pi_{q+1}=\{t_1,\dots,t_{q+1},\eta_1,\dots,\eta_{q+1},\bar{u}_1,\dots,\bar{u}_{q+1}\}$ with associated  control $u^\pi$. Define the needle-like variation 
$$
\pi_q=\{t_1,\dots,t_q,\eta_1,\dots,\eta_q,\bar{u}_1,\dots,\bar{u}_q\}\subset\pi_{q+1}=\pi,
% 
% 
$$  of $u$ as the controls $u^{\pi_q}$  defined by $u^{\pi_q}_t=\bar{u}_i$ if $t\in [t_i,t_i+\eta_i]$ for $i=1,\dots,q$, and $u^{\pi_q}_t=u_t$ otherwise,   and 
let $(Y^{\pi_q},(Y^{\pi_q})')\in\sD^p_X$ be the unique solution to the RDE
{\small
\begin{align*} 
Y^{\pi_q}_t=y+\int_0^tb(s,Y^{\pi_q}_s,u^{\pi_q}_s)\dd s+\int_0^t\sigma(s,Y^{\pi_q}_s)\dd\mbX_s,
\  t\in[0,T].
\end{align*}
}%
Let $(\widetilde{V},(\widetilde{V})')\in\sD^p_X$ be the unique solution to the RDE
\end{minipage}
\hspace{0.007\textwidth}
% 
\begin{minipage}[c]{0.43\textwidth}
\includegraphics[width=1\textwidth]{figs/needle_like_variations_3.jpg}
\captionof{figure}{Needle-like variations with two spikes.}
\label{fig:needlelike}
\end{minipage}
\begin{align*}
\widetilde{V}_t &= 
\widetilde{V}_{t_{q+1}}
+
\int_{t_{q+1}}^t
\frac{\partial b}{\partial x}(s,Y^{\pi_q}_s,u_s)\widetilde{V}_s\dd s +
\int_{t_{q+1}}^t
\frac{\partial\sigma}{\partial x}(s,Y^{\pi_q}_s)\widetilde{V}_s\dd\mbX_s,
 &&t\in[t_{q+1},T],
\\
\widetilde{V}_t
&=
b(t_{q+1},Y^{\pi_q}_{t_{q+1}},\bar{u}_{q+1})
-
b(t_{q+1},Y^{\pi_q}_{t_{q+1}},u_{t_{q+1}}),
&&t\in[0,t_{q+1}],
\end{align*}
see Figure \ref{fig:needlelike} (corresponding to $q+1=2$).
Then,
the error $Y^\pi-Y-\sum_{i=1}^{q+1}\eta_iV^{\pi_i}$ can   be decomposed as
 \begin{align*}
 Y^{\pi}-Y-\sum_{i=1}^{q+1}\eta_iV^{\pi_i}
 &=
 (Y^{\pi}-Y^{\pi_q}-\eta_{q+1}\widetilde{V})
 + 
 \eta_{q+1}(\widetilde{V} -V^{\pi_{q+1}})
 +
 \Big(
 Y^{\pi_q}-Y-\sum_{i=1}^q\eta_iV^{\pi_i}
 \Big)
\end{align*} 
For $C=C_{p,T,b,\sigma}$ and $\alpha=\alpha_{p,b,\sigma}$,
\begin{align*} 
\Big\|Y^\pi-Y^{\pi_q}-\eta_{q+1}\widetilde{V}\Big\|_{\infty,[t_{q+1},T]}
&\mathop{\leq}^{\eqref{eq:needle_like_deltasols_variation}}
C\exp\left(
CN_{\alpha,[0,T]}(\mbX)
\right)
\eta_{q+1}^2 \quad &&(\text{base case }q=1),
% 
% 
% 
% 
% 
% 
% 
% 
\\
\Big\|Y^{\pi_q}-Y-\sum_{i=1}^q\eta_iV^{\pi_i}\Big\|_{\infty,[t_{q+1},T]}
&\mathop{\leq}^{\eqref{eq:needle_like_deltasols_variation:multi}}
C\exp\left(
CN_{\alpha,[0,T]}(\mbX)
\right)
% 
\sum_{i,j=1}^q\eta_i\eta_j 
&&(\text{induction step for }q), 
\end{align*} 
so to conclude, it suffices to show that $\|\widetilde{V} -V^{\pi_{q+1}}\|_{\infty,[t_{q+1},T]}\leq C\exp\left(
CN_{\alpha,[0,T]}(\mbX)
\right)\sum_{i=1}^q\eta_i$.   
Indeed,  for any $t\geq t_{q+1}$, the error can be decomposed as
{\small
\begin{align*}
\widetilde{V}_t-V^{\pi_{q+1}}_t
&=
\left(
\widetilde{V}_{t_{q+1}}-V^{\pi_{q+1}}_{t_{q+1}}
\right)
+
\int_{t_{q+1}}^t
\left(
\frac{\partial b}{\partial x}(s,Y^{\pi_q}_s,u_s)\widetilde{V}_s
-
\frac{\partial b}{\partial x}(s,Y_s,u_s)V^{\pi_{q+1}}_s
\right)
\dd s 
\\
&\quad+
\int_{t_{q+1}}^t
\left(
\frac{\partial\sigma}{\partial x}(s,Y^{\pi_q}_s)\widetilde{V}_s
-
\frac{\partial\sigma}{\partial x}(s,Y_s)V^{\pi_{q+1}}_s
\right)
\dd\mbX_s
\\
&=
\left(\widetilde{V}_{t_{q+1}}-V^{\pi_{q+1}}_{t_{q+1}}\right)
+
\int_{t_{q+1}}^t
\left(
\frac{\partial b}{\partial x}(s,Y^{\pi_q}_s,u_s)(
\widetilde{V}_s-V^{\pi_{q+1}}_s
)
+
\left(\frac{\partial b}{\partial x}(s,Y^{\pi_q}_s,u_s)
-
\frac{\partial b}{\partial x}(s,Y_s,u_s)
\right)
V^{\pi_{q+1}}_s
\right)
\dd s 
\\
&\quad+
\int_{t_{q+1}}^t
\left(
\frac{\partial\sigma}{\partial x}(s,Y^{\pi_q}_s)(\widetilde{V}_s-V^{\pi_{q+1}}_s)
+
\left(
\frac{\partial\sigma}{\partial x}(s,Y^{\pi_q}_s)
-
\frac{\partial\sigma}{\partial x}(s,Y_s)\right)
V^{\pi_{q+1}}_s
\right)
\dd\mbX_s
\\
&=
\left(\widetilde{V}_{t_{q+1}}-V^{\pi_{q+1}}_{t_{q+1}}\right)
+
\int_{t_{q+1}}^t
\left(\frac{\partial b}{\partial x}(s,Y^{\pi_q}_s,u_s)
-
\frac{\partial b}{\partial x}(s,Y_s,u_s)
\right)
V^{\pi_{q+1}}_s
\dd s 
\\
&\quad+ 
\int_{t_{q+1}}^t 
\left(
\frac{\partial\sigma}{\partial x}(s,Y^{\pi_q}_s)
-
\frac{\partial\sigma}{\partial x}(s,Y_s)\right)
V^{\pi_{q+1}}_s 
\dd\mbX_s
\\
&\quad
+
\int_{t_{q+1}}^t
\frac{\partial b}{\partial x}(s,Y^{\pi_q}_s,u_s)(
\widetilde{V}_s-V^{\pi_{q+1}}_s
)
+
\int_{t_{q+1}}^t 
\frac{\partial\sigma}{\partial x}(s,Y^{\pi_q}_s)(\widetilde{V}_s-V^{\pi_{q+1}}_s)
\dd\mbX_s.
% 
% 
% 
% 
% 
% 
% 
% 
% 
% 
% 
% 
% 
% 
% 
% 
% 
% 
% 
% 
% 
% 
% 
% 
% 
% 
% 
% 
% 
% 
% 
% 
% 
% 
% 
\end{align*}
}% 
% 
% 
% 
% 
% 
% 
% 
% 
% 
% 
% 
% 
% 
% 
% 
% 
% 
% 
% 
% 
% 
% 
By following similar arguments as  in  Proposition \ref{prop:linear_variation}, % 
the first three terms in the equation above 
% 
can be bounded by $C\exp\left(
CN_{\alpha,[0,T]}(\mbX)
\right)
\sum_{i=1}^q\eta_i$, for example,
\begin{align*}
\|\widetilde{V}_{t_{q+1}}-V^{\pi_{q+1}}_{t_{q+1}}\|
&=
\|
b(t_{q+1},Y^{\pi_q}_{t_{q+1}},\bar{u}_{q+1})
-
b(t_{q+1},Y^{\pi_q}_{t_{q+1}},u_{t_{q+1}})
-
(b(t_{q+1},Y_{t_{q+1}},\bar{u}_{q+1})
-
b(t_{q+1},Y_{t_{q+1}},u_{t_{q+1}}))
\|
\\
&\leq
\|b(t_{q+1},Y^{\pi_q}_{t_{q+1}},\bar{u}_{q+1})
-
b(t_{q+1},Y_{t_{q+1}},\bar{u}_{q+1})\|
+
\|
b(t_{q+1},Y^{\pi_q}_{t_{q+1}},u_{t_{q+1}})
-
b(t_{q+1},Y_{t_{q+1}},u_{t_{q+1}})\|
\\
&\leq C_b\|Y^{\pi_q}_{t_{q+1}}-Y_{t_{q+1}}\|
\mathop{\leq}^{\eqref{eq:needle_like_deltasols:multi}}
C\exp\left(
CN_{\alpha,[0,T]}(\mbX)
\right)
\sum_{i=1}^q\eta_i,
\end{align*} 
and the other two integrals can be bounded as in the proof of Proposition \ref{prop:linear_variation} using the bounds \eqref{eq:rde:linearized:Vp_V'p_RVp/2:bound}
and \eqref{eq:rde:linearized:bounded_solutions}   for  $\|V\|_\infty,\|V\|_p,\|V'\|_p,\|R^V\|_\frac{p}{2}$ in Lemma \ref{lem:rde:linearized:bounded_solutions}. 
Finally, by defining $\Delta:=\widetilde{V}-V^{\pi_{q+1}}$ and 
looking at the linear RDE $\Delta_t=\Delta_{t_{q+1}}+
\int_{t_{q+1}}^t
\frac{\partial b}{\partial x}(s,Y^{\pi_q}_s,u_s)\Delta_s
+
\int_{t_{q+1}}^t 
\frac{\partial\sigma}{\partial x}(s,Y^{\pi_q}_s)\Delta_s
\dd\mbX_s$ whose initial value satisfies $\|\Delta_{t_{q+1}}\|\leq C\exp\left(
CN_{\alpha,[0,T]}(\mbX)
\right)
\sum_{i=1}^q\eta_i$, we conclude  that  
$\|\Delta\|_{\infty,[t_{q+1},T]}=\|\widetilde{V} -V^{\pi_{q+1}}\|_{\infty,[t_{q+1},T]}\leq C\exp\left(
CN_{\alpha,[0,T]}(\mbX)
\right)
\sum_{i=1}^q\eta_i$ using \eqref{eq:rde:linearized:bounded_solutions} in Lemma  \ref{lem:rde:linearized:bounded_solutions}, and the conclusion follows.
\end{proof}




% 
\subsection{Additional details for the indirect shooting method (Section \ref{sec:example})}\label{apdx:proofs:example}
We provide additional details for the integration schemes used for the Stratonovich SDE and the coupled RDE used in the \texttt{Direct} method and the \texttt{Indirect} shooting method in Section \ref{sec:example}. We only describe the case for the open-loop problem $\olocp$, as the feedback problem $\fbocp$ only has a different drift term.

1) \texttt{Direct} method: We discretize the Stratonovich SDE in the \texttt{Direct} problem using a Milstein scheme. 
Since $\sigma$ is diagonal  and each $\frac{\partial\sigma^{jj}}{\partial x^\ell}=0$ for $\ell\neq j$,   each $j$-th component $[\hat{x}^i]^j$ of $\hat{x}^i$ is approximated as
% 
% 
% 
% 
% 
% 
% 
% 
% 
% 
% 
% 
\begin{align}\label{eq:milstein}
\hspace{-2mm}
[\hat{x}_{k+1}^i]^j
= 
[
\hat{x}_k+ 
(A(\hat{x}_k^i)\hat{x}_k^i+\bar{B}\hat{u}_k)\Delta t + \sigma(\hat{x}_k^i) \Delta B_k^i
]^j + 
\frac{1}{2}
\frac{\partial\sigma^{jj}}{\partial x^j}(\hat{x}_k^i)\sigma^{jj}(\hat{x}_k^i)
([\Delta B_k^i]^j)^2,
\    
j=1,\dots,n,
\end{align}
% 
% 
where  $k=0,\dots,N-1$, 
$\Delta t=\frac{T}{N}$, and 
$\Delta B_k^i=B_{(k+1)\Delta t}^i-B_{k\Delta t}^i$, see \cite[equation (3.12), Chapter 10.3]{Kloeden1992}.  




2) \texttt{Indirect} method: 
To implement the map $F:\R^{Mn}\to\R^{Mn},\  
(p_0^i)_{i=1}^M
\mapsto
(p_T^i)_{i=1}^M$, we numerically integrate  the RDE in \eqref{eq:example:shooting_problem} using the estimate \eqref{eq:rough_int:error_bound} for rough integrals as% 
% 
% 
% 
% 
% 
% 
% 
\begin{align}\label{eq:example:rde_discretization}
\bigg[\begin{matrix}
\hat{x}_{k+1}^i
\\
\hat{p}_{k+1}^i
\end{matrix}
\bigg]
&=
\begin{bmatrix}
\hat{x}_k^i
\\
\hat{p}_k^i
\end{bmatrix}
+ 
\bar{b}\bigg(\hspace{-1mm}\begin{bmatrix}
\hat{x}_k^i
\\
\hat{p}_k^i
\end{bmatrix}\hspace{-1mm},u_k^M
\hspace{-1mm}\bigg)
\Delta t 
+ 
\bar\sigma\bigg(\hspace{-1mm}\begin{bmatrix}
\hat{x}_k^i
\\
\hat{p}_k^i
\end{bmatrix}\hspace{-1mm}\bigg)
B_{k\Delta t,(k+1)\Delta t}^i
+
\nabla\bar\sigma\bigg(\hspace{-1mm}\begin{bmatrix}
\hat{x}_k^i
\\
\hat{p}_k^i
\end{bmatrix}\hspace{-1mm}\bigg) \bar\sigma\bigg(\hspace{-1mm}\begin{bmatrix}
\hat{x}_k^i
\\
\hat{p}_k^i
\end{bmatrix}\hspace{-1mm}\bigg)
\bB^i_{k\Delta t,(k+1)\Delta t},
\end{align}
where % 
$\bar{b}((x,u),u)=(b(x,u),-\frac{\partial H}{\partial x}(x,u,p,\mathfrak{p}_0))$  and  % 
$\bar\sigma(x,p)=(\sigma(x),\frac{\partial\sigma}{\partial x}(x)^\top p)$ denote the augmented drift and diffusion, $k=0,\dots,N-1$, and $\Delta t=\frac{T}{N}$. 
% 
Because $\sigma(x)\propto\textrm{diag}(x)$ is diagonal, only the diagonal elements $[\bB^i_{k\Delta t,(k+1)\Delta t}]^{jj}=\frac{1}{2}([B_{k\Delta t,(k+1)\Delta t}^i]^j)^2$ are required to evaluate  \eqref{eq:example:rde_discretization}, and one observes that the integration rule \eqref{eq:example:rde_discretization} for $\hat{x}^i$ coincides with   
\eqref{eq:milstein}, as we show next. 
% 
% 
% 
% 
% 
% 
% 
% 
% 
% 
% 
% 
% 
% 
% 
% 
% 
% 
% 
% 
% 
% 
% 
% 
% 
% 
% 
% 
% 
% 
% 
% 
% 
% 
% 
% 
% 
% 
% 
% 
% 
% 
% 
% 
% 
% 
% 
% 
% 
The tensor  $\nabla\bar\sigma(x,p) \bar\sigma(x,p) \in\R^{2n\times n\times n}$ is given by
\begin{align*}
\nabla\bar\sigma(x,p) \bar\sigma(x,p) 
=
\begin{bmatrix}
\frac{\partial\sigma}{\partial x}(x)\sigma(x)
+
\frac{\partial\sigma}{\partial p}(x)\frac{\partial\sigma}{\partial x}(x)^\top p
\\
-\frac{\partial}{\partial x}\left(
\frac{\partial\sigma}{\partial x}(x)^\top p
\right)
\sigma(x)
-\frac{\partial}{\partial p}\left(
\frac{\partial\sigma}{\partial x}(x)^\top p
\right)
\frac{\partial\sigma}{\partial x}(x)^\top p
\end{bmatrix}.
\end{align*}
Since $\sigma(x)\propto\textrm{diag}(x)$, 
{\small\allowdisplaybreaks
\begin{align*}
\left[
\frac{\partial\sigma}{\partial x}(x)\sigma(x)
\right]^{ijk}
&=
\sum_{\ell=1}^n
\frac{\partial\sigma^{ij}}{\partial x^\ell}(x)\sigma^{\ell k}(x)
=
\begin{cases}
\frac{\partial\sigma^{ii}}{\partial x^i}(x)\sigma^{ii}(x) & \text{if }i=j=k,
\\
0&\text{otherwise},
\end{cases}
\\
\frac{\partial\sigma}{\partial p}(x)\frac{\partial\sigma}{\partial x}(x)^\top p
&=0.
\\
\left[
\frac{\partial}{\partial x}\left(
\frac{\partial\sigma}{\partial x}(x)^\top p
\right)
\sigma(x)
\right]^{ijk}
&=
\sum_{\ell=1}^n
\frac{\partial}{\partial x^\ell}
\left(
\sum_{q=1}^n
\frac{\partial\sigma^{qj}}{\partial x^i}(x) 
p^q
\right)
\sigma^{\ell k}(x)
= 
\begin{cases}
\frac{\partial^2\sigma^{ii}}{\partial^2 x^i}(x)
p^i
\sigma^{ii}(x) & \text{if }i=j=k,
\\
0&\text{otherwise},
\end{cases},
\\
\left[
\frac{\partial}{\partial p}\left(
\frac{\partial\sigma}{\partial x}(x)^\top p
\right)
\frac{\partial\sigma}{\partial x}(x)^\top p
\right]^{ijk}
&=
\sum_{\ell=1}^n
\frac{\partial}{\partial p^\ell}
\left[
\left(
\frac{\partial\sigma}{\partial x}(x)^\top p
\right)
\right]^{ij}
\left[
\frac{\partial\sigma}{\partial x}(x)^\top p
\right]^{\ell k}
\\
&=
\sum_{\ell=1}^n
\frac{\partial}{\partial p^\ell}
\left(
\sum_{q=1}^n
\frac{\partial\sigma^{qj}}{\partial x^i}(x)  
p^q
\right)
\left(
\sum_{q=1}^n
\frac{\partial\sigma^{qk}}{\partial x^\ell}(x)
p^q
\right) 
\\
&=
\sum_{\ell=1}^n
\frac{\partial}{\partial p^\ell}
\left( 
\frac{\partial\sigma^{jj}}{\partial x^i}(x)
p^j
\right)
\left( 
\frac{\partial\sigma^{kk}}{\partial x^\ell}(x)
p^k
\right) 
= 
\frac{\partial\sigma^{jj}}{\partial x^i}(x)
\frac{\partial\sigma^{kk}}{\partial x^j}(x)
p^k
\\
&=
\begin{cases}
\left(
\frac{\partial\sigma^{ii}}{\partial x^i}(x)
\right)^2
p^i 	& \text{if }i=j=k,
\\
0&\text{otherwise}.
\end{cases}
\end{align*}
}% 
Thus, the tensor  $\nabla\bar\sigma(x,p) \bar\sigma(x,p)$ is such that only the entries $\left[
\nabla\bar\sigma(x,p) \bar\sigma(x,p) \right]^{j,j,j}$ and $\left[
\nabla\bar\sigma(x,p) \bar\sigma(x,p) \right]^{n+j,j,j}$ are non-zero, and each $j$th index of the right hand side of \eqref{eq:example:rde_discretization} is given by
{\small
\begin{align}
\label{eq:example:rough_integral_same_as_milstein}
\left[\nabla\bar\sigma(x,p)\bar\sigma(x,p)\bB^i_{s,t}\right]^j
&=
\left[
\nabla\bar\sigma(x,p) \bar\sigma(x,p)\right]^{j,j,j}
[\bB^i_{s,t}]^{j,j}
=
\frac{1}{2}
\frac{\partial\sigma^{jj}}{\partial x^j}(x) \sigma^{jj}(x)
([B_{s,t}^i]^j)^2,
\\
\left[\nabla\bar\sigma(x,p)\bar\sigma(x,p)\bB^i_{s,t}\right]^{n+j}
&=
\left[
\nabla\bar\sigma(x,p) \bar\sigma(x,p)\right]^{n+j,j,j}
[\bB^i_{s,t}]^{j,j}
=
-\frac{1}{2}
\left(
\frac{\partial^2\sigma^{jj}}{\partial^2 x^j}(x)
\sigma^{jj}(x)
+
\left(
\frac{\partial\sigma^{jj}}{\partial x^j}(x)\right)^2
\right)
p^j\
([B_{s,t}^i]^j)^2,
\nonumber
\end{align}
}% 
where we used $[\bB^i_{s,t}]^{jj}=(\int_s^tB_{s,u}^j\circ\dd B_u^j)^i=\frac{1}{2}([B_{s,t}^i]^j)^2$, where $\int_s^tB_{s,u}^j\circ\dd B_u^j$ is the Stratonovich integral of the $B^j_{s,\cdot}$ against $B^j$.
 
% 
% 
% 
Thus, only the squared increments $[B_{s,t}^i]^j$ are required to evaluate  \eqref{eq:example:rde_discretization}. Using 
\eqref{eq:example:rough_integral_same_as_milstein}, we also observe that the integration rule \eqref{eq:example:rde_discretization}  for $x$ coincides with the Milstein integration scheme in \eqref{eq:milstein}. 

