\section{Conclusion}

\rv{
To sum up, this paper contributes a computational model for simulating errors in touchscreen typing. 
% that incorporates 1) several error-producing cognitive mechanisms 2) a new formulation of the supervisory control problem, and 3) a computational-rationality-based modeling workflow with joint parameter optimization. 
%
By generating realistic distributions for four typographical error types,
covering a wide range of individual differences, 
and adapting to the complicated case of autocorrection during typing, 
\name demonstrated state-of-the-art performance in challenging benchmarking for reproduction of human errors in typing.
%
The model did all this without compromising its strong performance by other important metrics for typing.
We conclude that these results point to great potential in the class of models whereby the prediction of user behavior is rooted in maximizing expected utility under cognitive bounds.
This approach marks a notable divergence from the data-driven approaches so popular today:
in explicitly modeling the \emph{causes} of errors, instead of just ``parroting'' statistically plausible typographical errors in text, 
the model takes a \emph{glass-box} rather than a black-box approach. Every typographical error can be traced to the underlying cognitive events that produced it. 
However, the current version of the model does not fully account for real-world behaviors involving advanced features. Future work should aim to enhance the model to better handle complex dynamics such as gesture-based text input and word prediction.
}