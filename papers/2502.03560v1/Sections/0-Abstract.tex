\begin{abstract}
Empirical evidence shows that typing on touchscreen devices is prone to errors and that correcting them poses a major detriment to users’ performance. Design of text entry systems that better serve users, across their broad capability range, necessitates understanding the cognitive mechanisms that underpin these errors. However, prior models of typing cover only motor slips. The paper reports on extending the scope of computational modeling of typing to cover the cognitive mechanisms behind the three main types of error: slips (inaccurate execution), lapses (forgetting), and mistakes (incorrect knowledge). Given a phrase, a keyboard, and user parameters, ~\name simulates eye and finger movements while making human-like insertion, omission, substitution, and transposition errors. Its main technical contribution is the formulation of a supervisory control problem wherein the controller allocates cognitive resources to detect and fix errors generated by the various mechanisms. The model generates predictions of typing performance that can inform design, for better text entry systems.
\end{abstract} 

% \begin{abstract}
% Human errors frequently occur when typing on touchscreens, which wastes users' time fixing them. Modeling these errors can help better understand how users type and improve the text entry. However, modeling human errors is challenging due to the various factors that can cause them.
% The main contribution of the paper is the formulation of a supervisory control problem over human errors, where the controller allocates resources to minimize the errors and the time budget. 
% We develop a computational model for the case of typing on touchscreens, where we design the error-generating process based on the information-process approach: the human, confronting the touchscreen, may inaccurately perceive the situation, leading to mistakes in their typing; then given an interpretation, they may forget to carry out typing actions due to memory lapses; finally, they may execute the action incorrectly due to slips in motor control.
% Based on the errors, the key feature of the model is its supervisory control of vision and finger to actively detect and correct errors. The model learns to optimize typing behavior to achieve human-like speed and accuracy. By adjusting the error-related parameters of the model, the model has the ability to generate a variety of typing behaviors to cover individual differences.
% We present a benchmark that covers three conditions of error correction: 1) typing errors that cannot be corrected, 2) typing errors that can be manually corrected, and 3) typing errors with auto-correction available. It reflects various types of errors and diverse error-handling strategies from humans, including immediate and delayed corrections. 
% It also shows the model's ability in estimating human error behaviors when using auto-correction.
% Although limited to transcription typing, these results indicate that our approach can produce human error behavior more similar to humans, outperforming previous typing models.
% \end{abstract} 


% \begin{abstract}
% Mitigating human error is a long-standing goal in research on HCI and human factors. The theory of human errors suggests three major categories of errors — mistakes, slips, and lapses, associated with different but partially overlapping generating mechanisms. At the same time, humans can strategically modulate the amount of resources they allocate to prevent errors or affect the probability of errors. However, present computational models do not capture this aspect of strategic control. 
% The main contribution of the paper is the formulation of a supervisory control problem over three categories of human errors, where the controller allocates resources to minimize the errors and the time budget. 
% We develop a computational model for the case of typing on touchscreens, where we design the error-generating mechanism based on the theory: the human, confronting the touchscreen, may inaccurately perceive the situation, leading to mistakes in their typing; then given an interpretation, they may forget to carry out typing actions due to memory lapses; finally, they may execute the action incorrectly due to slips in motor control.
% Based on the errors, the key feature of the model is its supervisory control of vision and finger to actively detect and correct errors. The model learns to optimize typing behavior to achieve human-like speed and accuracy. By adjusting the error-related parameters of the model, the model has the ability to generate a variety of typing behaviors to cover individual differences.
% We present a benchmark that covers various types of errors from diverse user groups in typing, such as elderly users and those with Parkinson's disease. This benchmark reflects diverse error-handling strategies from humans, including immediate and delayed corrections. The results indicate that our approach can produce error-related behavior more similar to humans, outperforming previous typing models.
% \end{abstract}





%%
%% The code below is generated by the tool at http://dl.acm.org/ccs.cfm.
%% Please copy and paste the code instead of the exa1mple below.
%%
\begin{CCSXML}
<ccs2012>
<concept>
<concept_id>10003120.10003121.10003122.10003332</concept_id>
<concept_desc>Human-centered computing~HCI theory, concepts and models</concept_desc>
<concept_significance>500</concept_significance>
</concept>
</ccs2012>
\end{CCSXML}

\ccsdesc[500]{Human-centered computing~HCI theory, concepts and models}

%%
%% Keywords. The author(s) should pick words that accurately describe
%% the work being presented. Separate the keywords with commas.
\keywords{Human errors; User simulation; Mobile typing}