\section{Discussion}

\rv{
\name is the first computational model to accurately simulate a wide range of human errors in a complex, real human--computer interaction task.
Specifically, it simulates omission, transposition, commission, and substitution errors in typing.
%
The model achieves a high level of similarity with human data across multiple conditions and groups, both as judged via aggregate metrics, such as WPM, and when handling trajectory-level predictions.
}

\rv{
What do the results mean for practitioners and for broader understanding of human errors, though, and what work remains to be done?
To tackle these key questions, we discuss the implications and limitations of the results next.
}

\subsection{\rv{Implications}}

\rv{
We see three exciting avenues in applying ~\name: evaluation, user research, and generation of synthetic data.
}

\rv{
Firstly,  
\name makes it possible to evaluate keyboard designs before undertaking an empirical study of them. Compared to CRTypist, \name generates more realistic error patterns and error-handling behavior; hence, it proves more effective for evaluating the fault tolerance of a given design.
It is valuable in covering more errors too, because seemingly innocuous aspects of a design can have surprising effects downstream, on users.
Errors take lots of time to spot and correct during typing; hence, minimizing their occurrence is a major aim in the design of any text entry system.
}

\rv{
Secondly, 
~\name enables practitioners to study individual-level differences in typing. 
The results presented under Level 0 in Table~\ref{tab:benchmark} attest to \name's ability to reproduce diverse error patterns from elderly individuals and users with Parkinson's disease~\cite{nicolau2012elderly, wang2021facilitating}. This is thanks to the explainable modular architecture,
which can support varying the free parameters for vision, motor, and memory that constrain the cognitive capacities of the model. 
We conclude, then, that the architecture design underpinning \name displays potential to generate error behaviors consistent not only with ``average'' users but also with specific target groups with unique characteristics.
}

\rv{
Thirdly,
intelligent text entry (ITE) techniques often rely on supervised learning.
We believe that, on account of the realistic nature of its predictions, ~\name affords new methods of data augmentation,
wherein synthetically produced data serve to complement a dataset, particularly in conditions where empirical data may be hard to collect.
}

\rv{
Looking beyond practical applications, we find the model to hold promise for opening the door to a new way of theorizing about errors in human--computer interaction.
The results of our work stem from a single key assumption behind our model: that users can strategically allocate resources to monitor and correct errors.
This complements the prevailing understanding of human errors, which has focused on the mechanisms that generate errors but not those that fix them. 
%
The underlying principle is aligned with the nascent theory of resource rationality \cite{lieder2020resource}, according to which people adaptively control the way they use their cognition.
From an RL perspective, they learn policies on their cognitive machinery -- and not just for their overt behavior.
%
Our computational implementation lends credence to this idea, as do the results obtained. 
}



\subsection{\rv{Limitations and Future Work}}

\rv{
Much is yet to be done to extend ~\name to support the many types of intelligent features developed for keyboards today.
At present, \name does not completely capture real-world behaviors when autocorrection is involved. We noticed that some errors stem from conflicting correction mechanisms. In this case, the autocorrecting operation may intervene at the very moment the user is trying to correct a mistake. Such simultaneous execution can lead to situations wherein a ``bad correction'' is made, due not to human error but, rather, a misalignment between the user’s act and the automated system’s action. Future efforts must consider dynamic interactions such as these between user inputs and intelligent feedback.
}

\rv{
We readily acknowledge that real-world behavior with ITE techniques is more complex than what our model currently encompasses at ``Level 2.'' 
\name should be extended to handle commonly used techniques for interactively correcting errors, such as selecting text in a modal manner (e.g., with a ``caret''), 
gesture-based text entry~\cite{zhai2003shorthand}, and more advanced techniques \cite{zhang2019type}.
One of the most popular features employed in modern typing is word prediction, which has become integral to the typing process across both mobile and desktop environments.
Word prediction systems allow users to select suggested words, hence bypassing both traditional typing and error correction mechanisms. However, \name does not yet cover how predictive features of such a nature influence typing and its correction. Bridging this gap could be the fruit of future work that implements the latest features in the training environment.
}
