\section{Related Work}

For background, we begin by reviewing related work on how humans make and correct typing errors. We then examine human-error modeling approaches and highlight the research gap.

\subsection{Typographical errors}

\emph{Typographical errors} are errors in typed or printed text. 
Several definitions have been developed to suit text entry contexts,
most of which originate from studies of typing with physical keyboards. 
\rv{
Three fundamental sorts of character-level error take place in typing~\cite{wobbrock2007measures}:
\emph{Insertion errors} occur when an extra character is typed by mistake (e.g., \texttt{typist} becoming \texttt{typ\textbf{o}ist}).
\emph{Omission errors} arise when a character is missing from the word typed (e.g., \texttt{typist} turning into \texttt{tpist}).
\emph{Substitution errors}, also known as \emph{misstrokes}, account for a large proportion of the errors people make when typing a nearby character (e.g., \texttt{typist} ends up as \texttt{typ\textbf{u}st}).
All these types can be identified well through character-level error analysis based on minimum string distance~\cite{mackenzie2002text}. 
Although they together characterize a large percentage of typing errors~\cite{gentner1983glossary}, other errors too appear frequently in typing.
For instance, a \emph{transposition error}~\cite{rumelhart1982simulating} arises from reversing two adjacent characters while typing (e.g., \texttt{should} becomes \texttt{shou\textbf{dl}}); 
a \emph{doubling error} occurs when a word that contains double letters gets the wrong letter doubled (e.g., \texttt{look} turns into \texttt{lo\textbf{kk}}); and there are 
\emph{alternation errors}, similar to doubling errors but with an alternative sequence of characters (e.g., \texttt{these} becomes \texttt{th\textbf{ses}}). 
This paper follows the categorization proposed by~\citet{wang2021facilitating} for computing error metrics, including the error rates of insertion, substitution, omission, and transposition.
}

\subsection{Errors in touchscreen typing}

Typing on touchscreens is known to be error-prone ~\cite{hoggan2008investigating,palin2019people}. 
Accordingly, large datasets have been examined for their distributions of typographical errors \cite{palin2019people}.
Some of the errors are attributed to the difficulty of hitting small keys ~\cite{holz2010generalized, holz2011understanding}, 
connected with the aforementioned fat finger problem~\cite{siek2005fat,baudisch2009back}. 
%
There are also large differences arising from personal characteristics. 
For instance, one study found that omission, primarily with cognitive causes, 
is the most common error type among elderly individuals~\cite{nicolau2012elderly}. 
Another study revealed that users with Parkinson's make significantly more insertion, substitution, and omission errors~\cite{wang2021facilitating}. 
The researchers' analysis attributed this to hand tremors creating a longer distance between adjacent touches~\cite{holz2011understanding}.
Also, people's error rates change situation-specifically. More typos occur when they feel free to leave errors~\cite{palin2019people} as compared to when they are trying to avoid errors~\cite{jiang2020we}.
%
Our research aim was to replicate the tendencies reported in the general population as well as those of two specific user groups.

Importantly for our work, typographical approaches to errors manifest a crucial limitation.
Most prominently, modern keyboards introduce interactive features and, thereby, novel types of errors that cannot be understood as typographical ones.
For example, users making \emph{mode errors} have hit the correct keys but in an inappropriate mode, such as with Caps Lock on ~\cite{norman1988psychology} (e.g., \texttt{hello} becomes \texttt{HELLO}).
Another kind, \emph{autocorrection errors}, occurs especially frequently with mobile devices when a statistical decoder incorrectly assumes the desired word to be one different from that intended. 
When a relatively effective autocorrection feature is in place and the user is less concerned about errors, people engage in faster typing and movement between keys~\cite{banovic2017quantifying, banovic2013effect}.
Therefore, we strove to encompass autocorrection too.

\subsection{How users detect and fix errors}

The ability to detect errors is essential for high performance in typing. 
There are two ways a user can detect that an error has been made:
1) by recognizing it in the text while reading and
2) by noticing an erroneous keypress~\cite{logan2010cognitive}. 
%
When it comes to the former, visual attention has an important role.
Of course, users cannot look at their fingers the whole time, since they need to check the text display also.
%
Researchers found that typists keep their gaze on the keys about 70\% of the time when typing with one finger and about 60\% of the time while using their thumbs ~\cite{jiang2020we}.
% 
They glance at the typed text about four times per sentence (avg. sentence length: 20 characters).
There is a further layer here: errors can occur even with proofreading.
If nothing else, some incorrect text may not be properly detected~\cite{haber1981error}. 
The speed with which users are able to perform proofreading is associated with the accuracy of error detection~\cite{pilotti2009text}.
We have modeled the associated accuracy to cover this.


After detecting an error, the user may choose to fix it.
Evidence suggests that there are two strategies to this end ~\cite{pinet2022correction}: 1) immediate backspacing to correct the error and 2) delayed corrections~\cite{arif2009analysis, jiang2020we}.
%
Furthermore, users can influence the probability of errors strategically.
They might slow down to hit keys more precisely \cite{bi2013ffitts},
one might develop a strategy of moving (two) fingers to minimize the fat finger problem and reduce rapid repetitive motions of any one finger~\cite{cerni2016}, etc.
We aimed to model such strategies. 

\subsection{Models of mechanisms behind typing errors}

Most prior work has focused on errors caused by the limitations of the human motor control system.
%
It has shown that Fitts' law functions well for predicting user performance in pressing keys under various conditions \cite{zhai2004characterizing}.
This model, often employed for examining slips during typing ~\cite{wobbrock2008error},
was extended with Finger Fitts' law~\cite{bi2013ffitts} to consider predictions for touchscreen typing. 
Modeling of this nature is limited to motor slips; however,
%
recent work has started to look at simulation-based approaches that could cover eye--hand coordination during touchscreen typing~\cite{jokinen2021touchscreen}. 
This paves the way toward predicting detail-level gaze and finger movement behavior in proofreading and error correction.

\rv{
CRTypist~\cite{shi2024crtypist} represents the latest model in this area. Its design is based on a supervisory control framework that includes three key modules: vision, finger, and working memory, each trained to replicate human cognitive processes. 
The vision module manages visual attention, enabling the model to shift focus between the keyboard and the text display for proofreading and finger guidance. The finger module simulates motor control, allowing for tapping keys on a touchscreen. Finally, the one for working memory maintains a time-decaying belief about the typed text, which informs decision-making for subsequent actions. Overarching supervisory control coordinates these modules to optimize typing performance, balancing between speed and accuracy. 
This modular, hierarchical design enables effectively predicting typing performance across various designs, tasks, and user groups, but the model focuses mainly on finger movement accuracy, so it addresses just substitution errors related to motor slips. A significant gap remains: simulating other types of errors that can occur during typing.
}
