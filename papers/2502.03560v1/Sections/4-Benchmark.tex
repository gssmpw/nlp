\section{The \textsc{TypingError} benchmark}
\label{Sec:benchmark}

To evaluate ~\name properly, we created a benchmark incorporating datasets that capture several distinct aspects of errors in mobile typing. 
The \benchmark benchmark exhibits some overlap with the  openly available \textsc{MobileTyping} benchmark~\cite{shi2024crtypist}, but the focus here is specifically on errors. 
To that end, new datasets and metrics have been included. 
We have divided the benchmark into three ``levels'', in accordance with the constraints that study conditions may impose on errors:

\begin{itemize}
    \item \textbf{Level 0: Typing errors when errors cannot be corrected}. In this condition, , typing errors cannot be corrected. Users are asked to type as quickly and accurately as possible without making any corrections. This allows researchers to observe the full range of errors that people make.
    \item \textbf{Level 1: Typing errors when errors can be manually corrected}. In this condition, manual error corrections are allowed, with users being asked to type quickly and accurately, correcting errors upon noticing them. Backspacing is the only way of doing so.
    \item \textbf{Level 2: Typing errors when autocorrection is available}. In this condition, autocorrection of mistyped text is available, and manual error corrections are also allowed. Users can decide to correct errors themselves or rely on autocorrection.
\end{itemize}

The benchmarking presentations are arranged by level accordingly, as Table~\ref{tab:benchmark} illustrates, with corresponding datasets (see Subsec.~\ref{sec:datasets}), diverse user groups (see Subsec.~\ref{sec:user-group}), and error-related metrics (see Subsec.~\ref{sec:metrics}).

\subsection{\rv{Datasets}}
\label{sec:datasets}

% Describe how these data collected
\rv{
We collected human typing data from four sources~\cite{nicolau2012elderly, wang2021facilitating, palin2019people, jiang2020we}.
\begin{itemize}
    \item \textit{Parkinson's-affected text entry}~\cite{wang2021facilitating}.
    \rv{
    One dataset is centered on the text entry performance of experiment participants with Parkinson’s disease. The data collection process employed two blocks of text entry tasks, each featuring 25 phrases randomly selected from the phrase sets chosen for evaluating text entry techniques~\cite{mackenzie2003phrase}. Participants were instructed to type quickly and accurately without correcting any errors, thus affording insight into the challenges faced by individuals with motor impairments during text entry. 
    }
    \item \textit{Elderly persons' text entry}~\cite{nicolau2012elderly}.  
    \rv{
    The second dataset aids in exploring text entry performance by elderly persons and how it varies with the type of device used. To help the participants become familiar with touchscreen devices, the researchers asked them to complete tasks that involved entering single letters and copying sentences. Later in the data collection process, they asked participants to perform transcription typing tasks without correcting any errors.
    }
    \item \textit{``How We Type''}~\cite{jiang2020we}.
    \rv{
    Composed of data collected from 30 native Finnish-speakers in a controlled laboratory setting, the third dataset focuses on metrics of typing behavior at detail level. Participants were asked to type quickly and accurately such that no errors remained in the sentence submitted. The project collected eye movement data (by using SMI eye-tracking glasses) and finger motion data (through an OptiTrack Prime 13 motion-capture system).
    }
    \item \textit{``Typing37K''}~\cite{palin2019people}.
    \rv{
    The large-scale online dataset Typing37K captures transcription typing behavior from 37,000 volunteers using a Web-based platform. Participants transcribed 15 sequential sentences. Demographic data (such as age, gender, and language proficiency), typing habits, and the keyboard used were recorded also.
    }
\end{itemize}
}

\begin{table}[t]
    \centering
    \caption{The performance of different pre-trained models on ImageNet and infrared semantic segmentation datasets. The \textit{Scratch} means the performance of randomly initialized models. The \textit{PT Epochs} denotes the pre-training epochs while the \textit{IN1K FT epochs} represents the fine-tuning epochs on ImageNet \citep{imagenet}. $^\dag$ denotes models reproduced using official codes. $^\star$ refers to the effective epochs used in \citet{iBOT}. The top two results are marked in \textbf{bold} and \underline{underlined} format. Supervised and CL methods, MIM methods, and UNIP models are colored in \colorbox{orange!15}{\rule[-0.2ex]{0pt}{1.5ex}orange}, \colorbox{gray!15}{\rule[-0.2ex]{0pt}{1.5ex}gray}, and \colorbox{cyan!15}{\rule[-0.2ex]{0pt}{1.5ex}cyan}, respectively.}
    \label{tab:benchmark}
    \centering
    \scriptsize
    \setlength{\tabcolsep}{1.0mm}{
    \scalebox{1.0}{
    \begin{tabular}{l c c c c  c c c c c c c c}
        \toprule
         \multirow{2}{*}{Methods} & \multirow{2}{*}{\makecell[c]{PT \\ Epochs}} & \multicolumn{2}{c}{IN1K FT} & \multicolumn{4}{c}{Fine-tuning (FT)} & \multicolumn{4}{c}{Linear Probing (LP)} \\
         \cmidrule{3-4} \cmidrule(lr){5-8} \cmidrule(lr){9-12} 
         & & Epochs & Acc & SODA & MFNet-T & SCUT-Seg & Mean & SODA & MFNet-T & SCUT-Seg & Mean \\
         \midrule
         \textcolor{gray}{ViT-Tiny/16} & & &  & & & & & & & & \\
         Scratch & - & - & - & 31.34 & 19.50 & 41.09 & 30.64 & - & - & - & - \\
         \rowcolor{gray!15} MAE$^\dag$ \citep{mae} & 800 & 200 & \underline{71.8} & 52.85 & 35.93 & 51.31 & 46.70 & 23.75 & 15.79 & 27.18 & 22.24 \\
         \rowcolor{orange!15} DeiT \citep{deit} & 300 & - & \textbf{72.2} & 63.14 & 44.60 & 61.36 & 56.37 & 42.29 & 21.78 & 31.96 & 32.01 \\
         \rowcolor{cyan!15} UNIP (MAE-L) & 100 & - & - & \underline{64.83} & \textbf{48.77} & \underline{67.22} & \underline{60.27} & \underline{44.12} & \underline{28.26} & \underline{35.09} & \underline{35.82} \\
         \rowcolor{cyan!15} UNIP (iBOT-L) & 100 & - & - & \textbf{65.54} & \underline{48.45} & \textbf{67.73} & \textbf{60.57} & \textbf{52.95} & \textbf{30.10} & \textbf{40.12} & \textbf{41.06}  \\
         \midrule
         \textcolor{gray}{ViT-Small/16} & & & & & & & & & & & \\
         Scratch & - & - & - & 41.70 & 22.49 & 46.28 & 36.82 & - & - & - & - \\
         \rowcolor{gray!15} MAE$^\dag$ \citep{mae} & 800 & 200 & 80.0 & 63.36 & 42.44 & 60.38 & 55.39 & 38.17 & 21.14 & 34.15 & 31.15 \\
         \rowcolor{gray!15} CrossMAE \citep{crossmae} & 800 & 200 & 80.5 & 63.95 & 43.99 & 63.53 & 57.16 & 39.40 & 23.87 & 34.01 & 32.43 \\
         \rowcolor{orange!15} DeiT \citep{deit} & 300 & - & 79.9 & 68.08 & 45.91 & 66.17 & 60.05 & 44.88 & 28.53 & 38.92 & 37.44 \\
         \rowcolor{orange!15} DeiT III \citep{deit3} & 800 & - & 81.4 & 69.35 & 47.73 & 67.32 & 61.47 & 54.17 & 32.01 & 43.54 & 43.24 \\
         \rowcolor{orange!15} DINO \citep{dino} & 3200$^\star$ & 200 & \underline{82.0} & 68.56 & 47.98 & 68.74 & 61.76 & 56.02 & 32.94 & 45.94 & 44.97 \\
         \rowcolor{orange!15} iBOT \citep{iBOT} & 3200$^\star$ & 200 & \textbf{82.3} & 69.33 & 47.15 & 69.80 & 62.09 & 57.10 & 33.87 & 45.82 & 45.60 \\
         \rowcolor{cyan!15} UNIP (DINO-B) & 100 & - & - & 69.35 & 49.95 & 69.70 & 63.00 & \underline{57.76} & \underline{34.15} & \underline{46.37} & \underline{46.09} \\
         \rowcolor{cyan!15} UNIP (MAE-L) & 100 & - & - & \textbf{70.99} & \underline{51.32} & \underline{70.79} & \underline{64.37} & 55.25 & 33.49 & 43.37 & 44.04 \\
         \rowcolor{cyan!15} UNIP (iBOT-L) & 100 & - & - & \underline{70.75} & \textbf{51.81} & \textbf{71.55} & \textbf{64.70} & \textbf{60.28} & \textbf{37.16} & \textbf{47.68} & \textbf{48.37} \\ 
        \midrule
        \textcolor{gray}{ViT-Base/16} & & & & & & & & & & & \\
        Scratch & - & - & - & 44.25 & 23.72 & 49.44 & 39.14 & - & - & - & - \\
        \rowcolor{gray!15} MAE \citep{mae} & 1600 & 100 & 83.6 & 68.18 & 46.78 & 67.86 & 60.94 & 43.01 & 23.42 & 37.48 & 34.64 \\
        \rowcolor{gray!15} CrossMAE \citep{crossmae} & 800 & 100 & 83.7 & 68.29 & 47.85 & 68.39 & 61.51 & 43.35 & 26.03 & 38.36 & 35.91 \\
        \rowcolor{orange!15} DeiT \citep{deit} & 300 & - & 81.8 & 69.73 & 48.59 & 69.35 & 62.56 & 57.40 & 34.82 & 46.44 & 46.22 \\
        \rowcolor{orange!15} DeiT III \citep{deit3} & 800 & 20 & \underline{83.8} & 71.09 & 49.62 & 70.19 & 63.63 & 59.01 & \underline{35.34} & 48.01 & 47.45 \\
        \rowcolor{orange!15} DINO \citep{dino} & 1600$^\star$ & 100 & 83.6 & 69.79 & 48.54 & 69.82 & 62.72 & 59.33 & 34.86 & 47.23 & 47.14 \\
        \rowcolor{orange!15} iBOT \citep{iBOT} & 1600$^\star$ & 100 & \textbf{84.0} & 71.15 & 48.98 & 71.26 & 63.80 & \underline{60.05} & 34.34 & \underline{49.12} & \underline{47.84} \\
        \rowcolor{cyan!15} UNIP (MAE-L) & 100 & - & - & \underline{71.47} & \textbf{52.55} & \underline{71.82} & \textbf{65.28} & 58.82 & 34.75 & 48.74 & 47.43 \\
        \rowcolor{cyan!15} UNIP (iBOT-L) & 100 & - & - & \textbf{71.75} & \underline{51.46} & \textbf{72.00} & \underline{65.07} & \textbf{63.14} & \textbf{39.08} & \textbf{52.53} & \textbf{51.58} \\
        \midrule
        \textcolor{gray}{ViT-Large/16} & & & & & & & & & & & \\
        Scratch & - & - & - & 44.70 & 23.68 & 49.55 & 39.31 & - & - & - & - \\
        \rowcolor{gray!15} MAE \citep{mae} & 1600 & 50 & \textbf{85.9} & 71.04 & \underline{51.17} & 70.83 & 64.35 & 52.20 & 31.21 & 43.71 & 42.37 \\
        \rowcolor{gray!15} CrossMAE \citep{crossmae} & 800 & 50 & 85.4 & 70.48 & 50.97 & 70.24 & 63.90 & 53.29 & 33.09 & 45.01 & 43.80 \\
        \rowcolor{orange!15} DeiT3 \citep{deit3} & 800 & 20 & \underline{84.9} & \underline{71.67} & 50.78 & \textbf{71.54} & \underline{64.66} & \underline{59.42} & \textbf{37.57} & \textbf{50.27} & \underline{49.09} \\
        \rowcolor{orange!15} iBOT \citep{iBOT} & 1000$^\star$ & 50 & 84.8 & \textbf{71.75} & \textbf{51.66} & \underline{71.49} & \textbf{64.97} & \textbf{61.73} & \underline{36.68} & \underline{50.12} & \textbf{49.51} \\
        \bottomrule
    \end{tabular}}}
    \vspace{-2mm}
\end{table}

\subsection{User groups}
\label{sec:user-group}

\rv{
The user groups were derived from the four datasets, with the data for each group being broken down further by our three levels.
}

At \textbf{Level 0}, the data we have includes the typing activity for individuals who were using a touchscreen without making any corrections. The three sets of users were 
\begin{enumerate}
    \item A group consisting of eight young adults (5 female and 3 male, all right-handed), with an average age of 23.6 years (standard deviation (\emph{SD}) = 3.7)~\cite{wang2021facilitating} 
    \item Eight Parkinson's patients (3 female and 5 male, all right-handed), 60.5 years old on average ({SD} = 9.2, with a range of 47 to 72), from a Parkinson's foundation~\cite{wang2021facilitating}
    \item Fifteen participants (11 female and 4 male), with ages ranging from 67 to 89 and a mean age of 79 (standard deviation = 7.3)~\cite{nicolau2012elderly}
\end{enumerate}

At \textbf{Level 1}, we used data from two separate keyboard layouts: an English and a Finnish one. 
For the Finnish-layout keyboard, we used material from the How We Type dataset~\cite{jiang2020we}, from 30 native Finnish-speakers with normal or corrected vision.
For the English-layout one, we selected a subset from Typing37K~\cite{palin2019people} (5,140 typing trajectories) in which participants were using the Gboard interface and typing without any intelligent features. Since the data were collected from an online-test Web site, participants were more careless but faster than those in the laboratory study.

At \textbf{Level 2}, we further refined the human data from Typing37K by filtering out data with participants using the Gboard interface with \emph{only} autocorrection. This left us with 148 typing trajectories.
% These data from various user groups functioned as ground truth for validating the model's ability to generalize to account for individual-to-individual differences.

\subsection{Error-related metrics}
\label{sec:metrics}

While including general typing metrics such as the commonly used words per minute (WPM) speed measurement, obtained by calculating the number of words divided by the time taken, our benchmark places more emphasis on error-related metrics.

\begin{itemize}
    \item \textit{Uncorrected error rate}~\cite{wobbrock2007measures}: The percentage of non-corrected incorrect keystrokes over the total of incorrect and correct keystrokes.
    \item \textit{Corrected error rate}~\cite{wobbrock2007measures}: Incorrect but rectified keystrokes as a percentage of the  sum of incorrect plus correct keystrokes.
    \item \textit{Keystrokes per character}~\cite{wobbrock2007measures}: The number of keystrokes divided by the number of characters produced (a larger number indicates more corrections).
    \item \textit{Backspaces}~\cite{palin2019people}: The number of \texttt{Backspace} presses for error correction during the typing of the text.
    \item \textit{Immediate error corrections}~\cite{arif2016evaluation}: This refers to the frequency of error correction in which the user immediately identifies and corrects an error with a subsequent Backspace press.
    \item \textit{Delayed error corrections}~\cite{arif2016evaluation}: This denotes the frequency of error correction wherein the user tries to correct previously missed errors in the middle of the text.
    \item \textit{Insertion error rate}~\cite{wang2021facilitating}: The rate of redundant touches that do not correspond to any of the target characters.
    \item \textit{Omission error rate}~\cite{wang2021facilitating}:  The rate of characters that do not correspond to any of the input touch points.
    \item \textit{Substitution error rate}~\cite{wang2021facilitating}: The rate of touches intended for certain characters, but landed on different keys.
    \item \textit{Transposition error rate}~\cite{wang2021facilitating}: The rate of touches resulting in characters being swapped.
\end{itemize}