\section{Results}

This section presents the results of our evaluation via the benchmark, presented in Table~\ref{tab:benchmark}.
In general, our model can synthesize diverse errors based on the cognitive mechanisms we incorporated. The model can closely reproduce human-like behavior at each level of error correction in the benchmark. In the following section, we will analyze the results in detail for each level.

\subsection{Level 0: Typing errors when errors cannot be corrected}

To train a model set-up without error correction, we disabled the \texttt{Backspace} key in the internal environment, so as to simulate typing without the ability to make corrections. The goal of the model remained correct typing of the phrase. However, the model could only control gaze movement to guide finger placement for correct typing and check the text field to determine the next input.

We evaluated the model by considering three user groups. After adjusting cognitive parameters for these groups, we ran simulations with the same number of independent episodes for each group, then compared the results. The model's generation of errors here, shown under Level 0 in Table \ref{tab:benchmark}, can be characterized thus:

\begin{itemize}
    \item[1)] \textit{Young adults}: The model can accurately reproduce the typing errors made by young adults. In typing at a speed of approx. 29~WPM, all types of errors fall within one standard deviation of the data for young adults. The error rates' prevalence order matches that of the humans: substitution errors, insertion errors, omission errors, transposition errors. 
    \item[2)] \textit{Users with Parkinson's}: The model also generates results similar to those of the Parkinson's patients. When typing at a (slow) speed close to that of this group, it yields nearly identical substitution and transposition errors. Also, it accurately reproduces the order of error rates within this group too, with substitution errors being the most common, followed by insertion, omission, and transposition errors. However, the model produces fewer insertion errors than seen in the data from actual users with Parkinson's. In this case, the probability of unintentional double tapping is much higher than the model expects.
    \item[3)] \textit{Elderly users}: Elderly users are the group with the slowest typing. Our model still is able to account for that typing speed within one standard deviation. As for individual error classes, the reference data do not include the standard deviation for each, but if we assume the \emph{SD} to be 1\%, three of the error rates from the model fall within that range, while insertion errors constitute an exception. The most common error type among elderly users is omission errors due to forgetfulness in cognition, and the model can closely match that phenomenon. It also successfully replicates the order of error rates within this group: omission errors dominate, followed by substitution errors, then insertion errors, and finally transposition errors.
\end{itemize}

\subsection{Level 1: Typing errors when errors can be manually corrected}

\begin{figure*}[!t]
\centering
  \includegraphics[width=\textwidth]{Images/corrections_wpm.png}
  \caption{Typing speed vs. error corrections. The figure shows the speed--accuracy tradeoff in both human data and the predictions.
}
  \label{fig:corrections_wpm}
  \vspace{-3mm}
\end{figure*}

In the conditions at this level, users were encouraged to improve the accuracy of the typed text. Therefore, the model's internal environment included the goal for the \texttt{Backspace} key within the action space. The model was trained to type phrases quickly and accurately, and optimization of all human parameters for the model for the target user group was handled by minimizing the differences in typing speed, error rates, and the amount of backspacing.

In the Finnish typing dataset, we sampled 30 independent runs similar to the collected human data. Our model with optimized parameters demonstrated human-like error-handling strategies. Compared to the state-of-the-art baseline approach~\cite{shi2024crtypist}, our model showed similar performance for uncorrected-error rate but proved much closer to humans in its corrected-error rate. Relative to the baseline, \name also uses the \texttt{Backspace} key in a more human manner here, resulting in a similar number of keystrokes per character. As Table~\ref{tab:benchmark} indicates, the baseline model performs better only for the number of delayed corrections, and even for these our model stays within a standard deviation of humans.
\rv{
The model types carefully, in line with human behavior, so produces few errors in the final sentences submitted. In our experiment, the model corrected all omission, substitution, and transposition errors, reaching accuracy levels close to human performance: the average was 0.07\% for omission errors, the rate was 0.11\% for substitution errors, and no transposition errors were observed.
}

We found that the relationship between \name's error corrections and typing speed was consistent with the distinct error correction patterns of users. Since the human parameters are normalized from 0 to 1, we sampled them from a Gaussian distribution, with the mean representing the optimal parameter and a standard deviation of 0.1. We carried out 300 independent simulations using Finnish sentences from the dataset~\cite{jiang2020we} to explore the connection between error correction and typing speed. The results (shown in Figure~\ref{fig:corrections_wpm}) showcase how the model can capture the distribution of errors and replicate the speed--accuracy preference observed in human users. The only difference is that the model made twice as many immediate corrections as the humans did; i.e., it shows a tendency to correct errors immediately.

In our work with the Gboard data, we ran 5,140 independent simulations with the model, matching the number of trials in the human data. Our model can replicate careless behavior, exhibiting a relatively high uncorrected error rate, yet also yields a corrected error rate consistent with human users'. Additionally, it exhibits error correction behavior that is similar to humans' (lying within one standard deviation of the human data) by all metrics, except for the substitution-error rate. The model tends to leave more substitution errors in the text submitted. In a parallel to the test with the Finnish typing dataset, our model showed a slightly elevated number of delayed corrections, but it was still within one standard deviation of the human data.

\rv{
To evaluate how closely the synthesized data from \name and CRTypist align with the human data's distribution, we conducted statistical analysis using the Bayes factor from a \textit{t}-test function~\cite{rouder2009bayesian} utilizing the \texttt{Pingouin} library~\footnote{\textit{\url{https://pingouin-stats.org/}}}. For each model (\name and CRTypist), we simulated 30 data points, thus matching the number of human data. 
We defined the null hypothesis ($H_0$) as no difference between the simulated data and human data, while the alternative hypothesis ($H_1$) asserts that a difference does exist.
The test results suggest that \name aligns more closely with human data than does CRTypist, across most metrics, for both the Finnish typing dataset (\name: 6/7 show support for $H_0$; CRTypist: 2/7 support $H_0$) and the Gboard dataset (\name: 2/7 support $H_0$; CRTypist: 0/7 support $H_0$).
From our tests with the Finnish typing dataset, the results for \name indicate support for $H_0$ by nearly all metrics, including metrics: WPM ($BF_{10} = 0.576$), uncorrected error rate ($BF_{10} = 0.382$), corrected error rate ($BF_{10} = 0.302$), and KSPC ($BF_{10} = 0.593$), delayed corrections ($BF_{10} = 0.529$). The only exception is immediate corrections ($BF_{10} = 81.015$). In contrast, $H_0$ with CRTypist receives support from only two metrics: uncorrected error rate ($BF_{10} = 0.262$) and delayed corrections ($BF_{10} = 0.529$).
With the Gboard dataset, $H_0$ is likewise strongly supported for \name. Alignment is excellent for WPM, corrected-error rate, backspacing, immediate corrections, and delayed corrections. In contrast, the CRTypist evidence supports $H_1$ across all metrics. Detailed analysis results can be found in the supplemental material.
}

% \definecolor{bad}{rgb}{0.99608,0.87843,0.82353}
\definecolor{good}{rgb}{0.89804, 0.96078, 0.87843}
\begin{table}[h!]
\centering
\caption{\rv{Bayesian \textit{t}-tests with Bayesian factors (BF10) and hypothesis support (move to supplemental material)}}
\begin{tabular}{|l|l|l|l|l|}
\hline
\textbf{Model} & \textbf{User Group} & \textbf{Metric} & \textbf{BF10} & \textbf{Support} \\ \hline
\name & Finnish typists & WPM & 0.576 & {\cellcolor{good}} H0 \\ \hline
\name & Finnish typists & Uncorrected error (\%) & 0.382 & {\cellcolor{good}} H0 \\ \hline
\name & Finnish typists & Corrected error (\%) & 0.302 & {\cellcolor{good}} H0 \\ \hline
\name & Finnish typists & KSPC & 0.593 & {\cellcolor{good}} H0 \\ \hline
\name & Finnish typists & Backspaces & 0.332 & {\cellcolor{good}} H0 \\ \hline
\name & Finnish typists & Immediate corrections & 81.015 & {\cellcolor{bad}} H1 \\ \hline
\name & Finnish typists & Delayed corrections & 0.291 & {\cellcolor{good}} H0 \\ \hline
\name & Gboard typists & WPM & 0.324 & {\cellcolor{good}} H0 \\ \hline
\name & Gboard typists & Uncorrected error (\%) & 1.019 & {\cellcolor{bad}} H1 \\ \hline
\name & Gboard typists & Corrected error (\%) & 0.266 & {\cellcolor{good}} H0 \\ \hline
\name & Gboard typists & KSPC & 1.282 & {\cellcolor{bad}} H1 \\ \hline
\name & Gboard typists & Backspaces & 0.336 & {\cellcolor{good}} H0 \\ \hline
\name & Gboard typists & Immediate corrections & 0.263 & {\cellcolor{good}} H0 \\ \hline
\name & Gboard typists & Delayed corrections & 0.41 & {\cellcolor{good}} H0 \\ \hline
CRTypist & Finnish typists & WPM & 2.477 & {\cellcolor{bad}} H1 \\ \hline
CRTypist & Finnish typists & Uncorrected error (\%) & 0.262 & {\cellcolor{good}} H0 \\ \hline
CRTypist & Finnish typists & Corrected error (\%) & 1.86 & {\cellcolor{bad}} H1 \\ \hline
CRTypist & Finnish typists & KSPC & 1.985 & {\cellcolor{bad}} H1 \\ \hline
CRTypist & Finnish typists & Backspaces & 1.184 & {\cellcolor{bad}} H1 \\ \hline
CRTypist & Finnish typists & Immediate corrections & 17170.0 & {\cellcolor{bad}} H1 \\ \hline
CRTypist & Finnish typists & Delayed corrections & 0.529 & {\cellcolor{good}} H0 \\ \hline
CRTypist & Gboard typists & WPM & 1.577e+115 & {\cellcolor{bad}} H1 \\ \hline
CRTypist & Gboard typists & Uncorrected error (\%) & 7.163e+130 & {\cellcolor{bad}} H1 \\ \hline
CRTypist & Gboard typists & Corrected error (\%) & inf & {\cellcolor{bad}} H1 \\ \hline
CRTypist & Gboard typists & KSPC & inf & {\cellcolor{bad}} H1 \\ \hline
CRTypist & Gboard typists & Backspaces & 2.536e+98 & {\cellcolor{bad}} H1 \\ \hline
CRTypist & Gboard typists & Immediate corrections & 2.015e+276 & {\cellcolor{bad}} H1 \\ \hline
CRTypist & Gboard typists & Delayed corrections & 2.534e+59 & {\cellcolor{bad}} H1 \\ \hline
\end{tabular}
\label{tab:typists_metrics}
\end{table}

\subsection{Level 2: Typing errors when autocorrection is available}

For the final level, at which autocorrection is used when the user types text, we improved the external environment by incorporating a rule-based feature that automatically corrects a word if its edit distance from the target word is within two characters. This autocorrection is triggered once the space bar is pressed.

We executed 148 independent runs of the model, which exhibited a slight increase in typing speed when autocorrection is not enabled. This enhancement not only reduced the uncorrected error rate by half but also slightly decreased the corrected error rate. This suggests fewer instances needing manual correction -- a conclusion supported by the reduced backspacing -- while higher accuracy is maintained. Notably, the model favored immediate corrections over delayed ones, relying on the autocorrection feature to rectify earlier mistakes.

\rv{
Research indicates that improvements in typing speed are linked to the accuracy of any autocorrection features~\cite{roy2021typing}. However, contrary to expectations and findings from previous studies~\cite{banovic2019limits}, our source human data demonstrate a decrease in typing speed when autocorrection is enabled. 
}
This discrepancy might be attributable to complexities encountered in real-world typing conditions.
For instance, the simulation of autocorrection might overlook some errors created/exacerbated by autocorrection itself. For instance, in an error type known as ``space key confusion'', users accidentally hit the space bar instead of producing the intended non-space character, thus triggering unintended autocorrection and the insertion of incorrect words.
 

% ``We may want to consider errors that emerge with ITEs''

% ``Shumin gave this example of “bad corrections”: Sometimes user notices an error, corrects it, but the system has already fixed it. Mixed initiative conflict. Or, the ITE “flashes” the user, but the flash disrupts the user.''

% One extreme: User only look at the keyboard “that”, “t” -> “h”, … “t”. Each keystroke is is guided by vision.

% Another extreme (“touch typing on touch keyboards”): Look at the text display, use peripheral vision and proprioception to model where the hand is in relation to the key position. By looking at the output you realize if the key is what you wanted.

