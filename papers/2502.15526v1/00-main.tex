\documentclass[sigconf,review=false,natbib=true,anonymous=false]{acmart}
% \usepackage[a-1b]{pdfx}
\usepackage{subfigure}
\usepackage{algorithm}
\usepackage[noend]{algpseudocode}
\usepackage{multirow} 
\usepackage{rotating}
\usepackage{tikz}
\usepackage{amsmath}
\usepackage{lipsum,adjustbox}
\usepackage{vcell}
\usepackage{xcolor}
\usepackage{xspace}
\usepackage{diagbox}
\usepackage{pifont}
\usepackage{enumitem}
\usepackage{balance}
\usepackage{cleveref}
\usepackage{placeins}
\usepackage{graphicx}
\usepackage{pgfplots}
\usepackage[normalem]{ulem}

\definecolor{darkgreen}{RGB}{0,128,0} % Adjust RGB values as needed
\definecolor{darkred}{RGB}{178,34,34}  % Firebrick (Aesthetic Dark Red)
\newcommand{\framework}{Lion\xspace}
\copyrightyear{2025}
\acmYear{2025}
\setcopyright{rightsretained}

%% These commands are specific for your submission.
% \acmConference[SIGIR '25]{Proceedings of the 48th International ACM SIGIR Conference on Research and Development in Information Retrieval}{July 13--18, 2024}{Padua, Italy.}
% \acmBooktitle{Proceedings of the 48th Int'l ACM SIGIR Conference on Research and Development in Information Retrieval (SIGIR '25), July 13--18, 2024, Padua, Italy}




\makeatother


%
% inline lists.  usage,
%    \begin{inlinelist}
%       \item first item,
%       \item second item, and
%       \item last item.
%    \end{inlinelist}
%
\usepackage{enumitem}
\newlist{inlinelist}{enumerate*}{1}
\setlist*[inlinelist,1]{%
  label=(\roman*),
}
\usepackage{subfigure}

\newcommand{\movie}{Movie\xspace}
\newcommand{\landmark}{Landmark\xspace}
\newcommand{\person}{Person\xspace}


\newcommand{\gptfouro}{GPT-4o\xspace}
\newcommand{\gptfouromini}{GPT-4o-mini\xspace}

% \settopmatter{printacmref=false}



% \title{Scaling Large Language Models for Dense and Sparse Retrieval}
\title{Scaling Sparse and Dense Retrieval in Decoder-Only LLMs}
% \title{LLMs for Sparse and Dense Retrieval: A Comparison of the Scaling Behavior of Different Retrieval Paradigms}

\author{Hansi Zeng}
\affiliation{\institution{University of Massachusetts Amherst}
\country{United States}}
\email{hzeng@cs.umass.edu}

\author{Julian Killingback}
\affiliation{\institution{University of Massachusetts Amherst}
\country{United States}}
\email{jkillingback@cs.umass.com}

\author{Hamed Zamani}
\affiliation{\institution{University of Massachusetts Amherst}
\country{United States}}
\email{zamani@cs.umass.edu}





\begin{document}



% \fancyhead{}

\begin{abstract}
% Recent advances in decoder-only large language models (LLMs) have shown great potential in retrieval tasks. However, the scaling behavior of these models, particularly for sparse retrieval, remains underexplored. In this work, we conduct a comprehensive analysis of scaling decoder-only LLMs (Llama-3 series: 1B, 3B, 8B) for both dense and sparse retrieval. Using MSMARCO passages as the training dataset and a fixed compute budget, we evaluate the impact of different fine-tuning objectives—contrastive loss (CL), knowledge distillation (KD) loss, and their combination—on retrieval performance. 
% Our key findings reveal that: 
Scaling large language models (LLMs) has shown great potential for improving retrieval model performance; however, previous studies have mainly focused on dense retrieval trained with contrastive loss (CL), neglecting the scaling behavior of other retrieval paradigms and optimization techniques, such as sparse retrieval and knowledge distillation (KD).
In this work, we conduct a systematic comparative study on how different retrieval paradigms (sparse vs. dense) and fine-tuning objectives (CL vs. KD vs. their combination) affect retrieval performance across different model scales. Using MSMARCO passages as the training dataset, decoder-only LLMs (Llama-3 series: 1B, 3B, 8B), and a fixed compute budget, we evaluate various training configurations on both in-domain (MSMARCO, TREC DL) and out-of-domain (BEIR) benchmarks.
Our key findings reveal that:
(1) Scaling behaviors emerge clearly only with CL, where larger models achieve significant performance gains, whereas KD-trained models show minimal improvement, performing similarly across the 1B, 3B, and 8B scales.
(2) Sparse retrieval models consistently outperforms dense retrieval across both in-domain (MSMARCO, TREC DL) and out-of-domain (BEIR) benchmarks, and they demonstrate greater robustness to imperfect supervised signals. 
(3) We successfully scale sparse retrieval models with the combination of CL and KD losses at 8B scale, achieving state-of-the-art (SOTA) results in all evaluation sets.
%Our work provides new insights into the impact of retrieval paradigms, fine-tuning objectives, and model scaling on retrieval performance, evaluated across both in-domain and out-of-domain benchmarks.

% (2) Sparse retrieval consistently outperform dense retrieval across both in-domain (MSMARCO, TREC DL) and out-of-domain (BEIR) benchmarks and  and we successfully scale sparse retrieval to achieve state-of-the-art (SOTA) results at the 8B scale. Our work provides a systematic study of scaling up retrieval models with decoder-only LLMs, offering new insights into the interplay between model size, training objectives, and retrieval paradigms.
\end{abstract}

% \keywords{Dense Retrieval; Sparse Retrieval; Scaling LLMs}

% \begin{CCSXML}
% <ccs2012>
% <concept>
% <concept_id>10002951.10003317</concept_id>
% <concept_desc>Information systems~Information retrieval</concept_desc>
% <concept_significance>500</concept_significance>
% </concept>
% <concept>
% <concept_id>10010147.10010257</concept_id>
% <concept_desc>Computing methodologies~Machine learning</concept_desc>
% <concept_significance>500</concept_significance>
% </concept>
% </ccs2012>
% \end{CCSXML}

% \ccsdesc[500]{Information systems~Information retrieval}
% \ccsdesc[500]{Computing methodologies~Machine learning}

\maketitle

\section{Introduction}  
%\hamed{make sure to discuss the scaling law for dense retrieval paper and highlight your contributions compared to that paper.}
% Scaling decoder-only large language models (LLMs) has consistently demonstrated performance improvements across various downstream NLP tasks \cite{}. This scaling benefit also extends to retrieval tasks \cite{}, where decoder-only retrieval models, such as RepLLaMA \cite{} and LLM2Vec \cite{}, parameterized by LLaMA or Mistral at the 7B/8B scale, have proven to significantly outperform traditional BERT-based retrieval models on both in-domain and out-of-domain benchmarks.

% \hansi{Add bridge senstences} In this paper, we systematically investigate how scaling decoder-only LLMs impacts retrieval performance by examining three key dimensions: 1) retrieval paradigms, 2) fine-tuning objectives, and 3) model sizes.


% \hamed{the title needs to change. Use a "scientific" title, not a marketing title.}

Scaling large language models (LLMs) has led to significant improvements across various NLP tasks \cite{gpt-3,chinchilla,mt-scaling-1,mt-scaling-2,emergent-llm}. These scaling benefits also extend to retrieval tasks~\cite{llm2vec,repllama,scale-dr,llama2vec,echo-embed,Tao2024LLMsAA}. Recent studies~\cite{repllama,llama2vec,echo-embed} have demonstrated that dense retrieval models parameterized by large decoder-only LLMs, such as LLaMA2-7B~\cite{llama2}, can significantly outperform traditional BERT-based retrieval models on both in-domain and out-of-domain benchmarks.

% While most studies on scaling up decoder-only LLMs neglect sparse retrieval~\cite{snrm,splade,splade-v2,spade}—a unique single-vector retrieval paradigm proven to be as efficient and effective as dense retrieval~\cite{dpr,ance,lsr-workshop}—this oversight stems from a key challenge: sparse retrieval requires high-quality contextualized token representations, which decoder-only LLMs struggle to generate due to the inherent limitations of causal masking~\cite{ul2,lv-etal-2024-analysis,echo-embed}. To address this, we adopt the approach from~\cite{llm2vec} by replacing the causal mask with a bidirectional mask~\cite{bert} and applying masked next-token prediction (MNTP)~\cite{llm2vec} during pre-training. 
% %This modification enables us to successfully scale up sparse retrieval and compare it with dense retrieval models with minimal architectural changes. 
% This modification minimally alters the decoder-only LLM architecture while making it better suited for sparse retrieval. With this issue resolved, we can now systematically compare sparse and dense retrieval in large LLMs to understand their scaling behaviors and effectiveness.
% \julian{Rewrote, tried to explain what sparse retrieval is and why it matters and make it clear why causal masking is an issue.}
These studies mainly focus on dense retrieval, neglecting sparse retrieval~\cite{snrm,splade,splade-v2,spade}—a single-vector retrieval paradigm that uses high-dimension sparse vectors, where each index represents a vocabulary term. An important feature of sparse retrieval models is that each contextualized token embedding is projected into the vocabulary space, where the model can include information about the original token and do term expansion~\cite{doc-exp,query-exp-llm,deepct,splade}. To do this well, the model needs to know information about each token when it does the projection or it will not be able to include all the relevant information. This is a problem for decoder-only LLMs as their causal attention masks are designed for generation where the next token is not known. Thus, the embedding used for the projection of the $i$th token is not aware of the true $i$th token. To address this shortcoming, we adopt the approach from~\cite{llm2vec} by replacing the causal mask with a bidirectional mask~\cite{bert} and doing a masked next-token prediction (MNTP)~\cite{llm2vec} pre-training step. 
%This modification enables us to successfully scale up sparse retrieval and compare it with dense retrieval models with minimal architectural changes. 
This modification minimally alters the decoder-only LLM architecture while making it better suited for sparse retrieval. With this issue resolved, we can now systematically compare sparse and dense retrieval in LLMs to understand their scaling behaviors and effectiveness.

% \julian{I commented part of the below section out because I think it isn't necessary and saves some space.}
% \citet{scale-dr} demonstrated that the scaling dense retrieval models reduces contrastive entropy and improves test-set ranking metrics when using a contrastive loss (\ContrastiveAcronym). However, the scaling behaviors when employing knowledge distillation (KD)~\cite{colbertv2,rocketqav2,cl-drd,tas-b,margin-mse,tct-colbert,kd}, another widely used and performant retrieval optimization objective, remains underexplored. KD typically employs an external cross-encoder reranker as a teacher model to provide augmented training labels for optimizing student retrieval models. 
% It has proven effective and can close the performance gap between small and large retrieval models~\cite{margin-mse,cl-drd}. For instance, KD-trained DistilBERT-based retrieval models can outperform CL-trained BERT-based models on the in-domain MS MARCO benchmark~\cite{margin-mse,cl-drd}.
% Commented out original here:
% However, it remains unclear whether KD follows the same scaling pattern as contrastive loss (CL) or its benefits diminish as the model size increases. 
% Models trained with KD have been shown to be more effective than those trained with contrastive losses \cite{margin-mse,cl-drd}, but it is unclear whether KD-trained models follows the same scaling pattern as those trained with \ContrastiveAcronym. \hamed{is using KD the only main difference? I think this distinction should be explained better.}

\citet{scale-dr} demonstrated that scaling dense retrieval models reduces contrastive entropy and improves test-set ranking metrics when trained with contrastive loss (CL). However, the scaling behavior of retrieval models trained with knowledge distillation (KD)~\cite{colbertv2,rocketqav2,cl-drd,tas-b,margin-mse,tct-colbert,kd}, another widely used and effective retrieval optimization method, remains underexplored. KD typically employs an external cross-encoder reranker as a teacher model to provide augmented training labels for optimizing student retrieval models~\cite{tas-b}. While KD-trained models have been shown to outperform CL-trained models~\cite{tas-b,cl-drd}, it remains unclear whether this advantage persists as model size increases. As the performance of KD models is inherently tied to the quality of the teacher model, its effectiveness may diminish as models become more capable, making it essential to examine how KD scales with model size.

%Furthermore, \citet{scale-dr} focuses only on in-domain evaluations, neglecting out-of-domain benchmarks~\cite{beir}. 
Furthermore, the scaling behavior of retrieval models in out-of-domain benchmarks was also neglected in previous work~\cite{scale-dr,scale-note,scaling-advertisement-retrieval}. \citet{dragon} has shown that dense retrieval performance in in-domain and out-of-domain settings is often negatively correlated, whereas sparse retrieval exhibits stronger generalization capabilities. This raises important questions: as retrieval models scale, does this trade-off persist or can scaling mitigate the gap, and whether sparse retrieval continues to generalize better than dense retrieval? 
% Additionally, \citet{scale-dr} experiments only with BERT-based dense retrieval models, where the largest models contain 110M parameters. Given recent advancements in large-scale LLMs for retrieval~\cite{repllama,llm2vec}, where even the smallest models exceed 1B parameters, this setting is becoming increasingly less representative of modern retrieval architectures.

Given these open questions, our work conducts a comprehensive study of how different retrieval paradigms (sparse and dense) and fine-tuning objectives (\ContrastiveAcronym and KD) interact with model scaling.
% To answer these questions, we conduct a comprehensive study comparing sparse and dense retrieval models trained with CL, KD, and their combined loss at different model scales. 
Specifically, we use the LLaMA3~\cite{llama3} series (1B, 3B, and 8B) as the backbone for our retrieval models. To ensure a fair comparison, we fix the compute budget in all training configurations. For evaluation, we use MS MARCO Dev \cite{msmarco}, TREC DL 2019 \cite{trec-dl-19}, and TREC DL 2020 \cite{trec-dl-20} as in-domain benchmarks, and the BEIR benchmark \cite{beir} for out-of-domain zero-shot evaluation. We employ parameter-efficient fine-tuning via LoRA \cite{lora} for training. Our experiments reveal several key findings:
\begin{itemize}[leftmargin=*]
    \item Unlike contrastive loss, KD-trained models do not exhibit significant performance gains as model size increases. For instance, KD-trained dense retrieval models overfit and underperform their \ContrastiveAcronym-trained counterparts in BEIR at the 3B and 8B scales.
    % Models trained with CL exhibit a clear positive correlation between model sizes and retrieval performance; however, models trained with KD do not show this correlation.
    %CL loss exhibits a clear scaling pattern, where larger models achieve better performance. In contrast, KD loss does not show this trend.
    %for both in-domain and out-of-domain benchmarks, 1B and 8B retrieval models trained with KD perform similarly, and in some cases, the 8B models perform worse.
    \item Sparse retrieval models consistently outperform dense retrieval models across all evaluation benchmarks with all the fine-tuning objectives. They demonstrate greater robustness as demonstrated by the out-of-domain BEIR benchmark.
    \item The combination of \ContrastiveAcronym and KD loss achieves the best performance, enabling our 8B sparse retrieval model, named \framework-SP-8B, to become the state-of-the-art model in all the evaluation sets. For instance, \framework-SP-8B outperforms ColBERTv2 by $10.6\%$ in BEIR, and RepLlama by $4.1\%$ in TREC DL 19+20.
\end{itemize}
To improve reproducibility, we open-source our codebase in 
\url{https://github.com/HansiZeng/scaling-retriever}.
% \url{https://anonymous.4open.science/r/scaling-retriever-8FD0/}.

% \hamed{the motivation needs to be better. What about starting with the recepie for training retrieval models is generally to use KD for training; we did it for larger models and it didn't work. This inspired us to revisit the role of optimization in training retrieval models at scale.}

% However, these studies have primarily analyzed the scaling patterns of dense retrieval models trained with CL loss, \hamed{sounds like a weird name. use a name that is widely known or explain it.} while overlooking other retrieval paradigms and fine-tuning objectives. \hamed{what do you mean other retrieval paradigms and fine-tuning objectives? the motivation needs to be stronger.} In this paper, we conduct a systematic investigation into \hamed{make sure "into" is the right preposition.} how different retrieval paradigms and fine-tuning objectives affect retrieval performance at various model scales for decoder-only LLMs. \hamed{again, a better motivation needed, you don't do a study just for the sake of doing it. you observe a limitation in the field and you do a study to address that limitation. describe them!}

% For retrieval paradigms, we study both dense and sparse retrieval, as they can be unifiedly categorized as single-vector retrieval models, which are both effective and more efficient than multi-vector retrieval models such as ColBERT~\cite{}. However, they fundamentally differ in their vector representations, making them an ideal pair for comparative studies: sparse retrieval embeds documents (or queries) into high-dimensional, sparse vectors \cite{splade,splade-v2,spade,snrm}, while dense retrieval maps textual information into low-dimensional, continuous vector spaces \cite{dpr,ance, cl-did,repllama}. As a result, sparse retrieval captures expanded lexical representations, whereas dense retrieval focuses on semantic representations. 

% \hamed{also connection to the scaling law of dense retrieval paper from last sigir should be discussed in the beginning of intro.}

% \hamed{I think this paper benefits from the research question writing style. e.g., we need to answer RQ1: xxx and to do that we do this and we find this. then we answer RQ2 ...}
% For retrieval paradigms, we study both dense and sparse retrieval. Both have demonstrated strong retrieval performance~\cite{dpr,ance,splade-v2} and can be unifidely categorized as single-vector retrieval models. However, they fundamentally differ in their vector representations: sparse retrieval~\cite{splade,splade-v2,spade,snrm} embeds documents (or queries) into high-dimensional, sparse vectors, while dense retrieval~\cite{dpr,ance,cl-drd,repllama} maps textual information into low-dimensional, continuous vector spaces. As a result, sparse retrieval captures expanded lexical representations, whereas dense retrieval focuses on semantic representations. The properties make them as an ideal pair for comparative studies. 

% Prior work on scaling retrieval models \cite{repllama,echo-embed,scale-dr} relied on CL loss as the fine-tuning objective, overlooking knowledge distillation (KD) \cite{kd}—a critical branch of retrieval optimization~\cite{cl-drd,rocketqav2,colbertv2,tct-colbert,margin-mse,tas-b,splade-v2}. KD typically employs an external cross-encoder reranker as a teacher model to provide augmented training labels for optimizing student retrieval models. KD has proven effective, particularly in closing the performance gap between small and large retrieval models. For instance, DistilBERT-based retrieval models trained with KD can outperform BERT-based models trained with CL on the MS MARCO benchmark \cite{margin-mse,tas-b}. Therefore, in this work, we explore CL loss, KD loss, and their mixture as fine-tuning objectives.

%We evaluate the performance of different retrieval paradigms and fine-tuning objectives across multiple model scales.
% Specifically, we use the LLaMA3~\cite{llama3} series (1B, 3B, and 8B) as the backbone for our retrieval models. To ensure a fair comparison, we fix the compute budget in all training configurations. We employ parameter-efficient fine-tuning via LoRA \cite{lora} for training. 
% For evaluation, we use MS MARCO Dev \cite{msmarco}, TREC DL 2019 \cite{trec-dl-19}, and TREC DL 2020 \cite{trec-dl-20} as in-domain benchmarks, and the BEIR benchmark \cite{beir} for out-of-domain zero-shot evaluation. Our experiments reveal several key findings:
% \begin{itemize}
%     \item Models trained with CL exhibit a clear positive correlation between model sizes and retrieval performance; however, models trained with KD do not show this correlation.
%     %CL loss exhibits a clear scaling pattern, where larger models achieve better performance. In contrast, KD loss does not show this trend.
%     %for both in-domain and out-of-domain benchmarks, 1B and 8B retrieval models trained with KD perform similarly, and in some cases, the 8B models perform worse.
%     \item Sparse retrieval models consistently outperform dense retrieval models across all evaluation benchmarks with all the fine-tuning objectives. They demonstrate greater robustness as demonstrated by the out-of-domain BEIR benchmark.
%     \item The combination of CL and KD loss achieves the best performance, enabling our 8B sparse retrieval model, named \framework-SP-8B, to become the state-of-the-art model in all the evaluation sets. For instance, \framework-SP-8B outperforms ColBERTv2 by $10.6\%$ in BEIR, and RepLlama by $4.1\%$ in TREC DL 19+20.
% \end{itemize}


\section{Setup}
\begin{table}[t]
    \centering
    \caption{The overview of experimental setup.}
    \vspace{-0.2cm}
    \scalebox{0.85}{
    \begin{tabular}{l!{\color{lightgray}\vrule}l}
    \toprule
        \multicolumn{2}{l}{\textbf{Modeling and Training Objectives: }} \\
        LLM Model Sizes & 1B, 3B, 8B (Llama3 series) \\
        Retrieval Paradigms & dense retrieval, sparse retrieval \\
        Pretraining Objective & MNTP \\
        Finetuning Objectives &  CL, KD, CL+KD \\
        \midrule
        \multicolumn{2}{l}{\textbf{Training and Evaluation Datasets:}} \\       
        Pretrain and Finetuning Corpus & MSMARCO (8.8M docs, 532K queries) \\
        In-domain Evaluation & MSMARCO-Dev, TREC DL 19 \& 20 \\
        Out-of-domain Evaluation & BEIR \\ 
    \bottomrule
    \end{tabular}}
    \label{tab:exp_setup}
    \vspace{-0.5cm}
\end{table}

% Our experiments focus on scaling up LLMs for retrieval. We adopt the Llama 3 series as our backbone models and investigate model sizes ranging from 1B to 8B. Our study examines the impact of retrieval paradigms and fine-tuning objectives on the performance of retrieval models with different sizes across both in-domain and out-of-domain benchmarks. The experimental setup is presented in Table~\ref{tab:exp_setup}.

% \julian{I rewrote the paragraph above LMK what you think}
Our experiments focus on the impact of three variables on in-domain and out-of-domain performance. These variables are: (1) the size of the backbone model. We use the Llama 3 family of models from 1B to 8B (2) the retrieval paradigm including sparse and dense retrieval (3) the fine-tuning objective. The full information on these variables and the training details are presented in Table~\ref{tab:exp_setup}.

% \subsection{Retrieval Paradigm}
% In this work, we focus on two widely-adopted retrieval paradigms: dense retrieval \cite{} and sparse retrieval \cite{}. These approaches are chosen because they have demonstrated strong retrieval performance \cite{} and both fall under the category of single-vector retrieval, where each document (or query) is encoded into a single vector. %This design ensures faster retrieval and more efficient index memory usage compared to multi-vector retrieval paradigms.

% % We assume a large language model (LLM) \( M_\theta \) that encodes any text \( \texttt{T} = [t_1, \ldots, t_L] \) into a contextualized sequence representation with hidden dimension \( D \):
% % \julian{Rewrote this (see commented out one above) feel free to revert}
% Given some text \( \texttt{T} = [t_1, \ldots, t_L] \) and the large language model (LLM) \( M_\theta \) we can get the contextualized sequence representation for the text with hidden dimension \( D \) as follows:
% \[
% M_\theta(\texttt{T}) = \mathbf{H}_T \in \mathbb{R}^{D \times L}.
% \]

% For \textit{dense retrieval}, a pooling method, such as [CLS]-pooling \cite{} or mean-pooling \cite{}, is used to obtain a single vector representation. In our work, we adopt mean-pooling:
% \[
% \text{AvgPool}(\mathbf{H}_T) = \vec{\mathbf{h}_T} \in \mathbb{R}^D.
% \]

% For \textit{sparse retrieval}, we project \( \mathbf{H}_T \) into the high-dimensional subword token space by doing a matrix multiplication with the LLM's embedding table \( \mathbf{E} \in \mathbb{R}^{D \times V} \) (where \( V \) is the vocabulary size), followed by a max-pooling and rescaling operation:
% \[
% \log \big( 1 + \text{ReLU} \big( \text{MaxPool} (\mathbf{E}^T \cdot \mathbf{H}_T) \big) \big) = \vec{\mathbf{c}_T} \in \mathbb{R}^V.
% \]
% In modern LLMs, \( V \gg D \). For instance, in Llama3 8B, \( V = 128,256 \) while \( D = 4,096 \). To sparsify the high-dimensional vector \( \vec{\mathbf{c}_T} \), we apply FLOP regularization \cite{} during training. This ensures that only important tokens are assigned non-zero values. In our experiments, the sparsification rate (zero rate) achieves over \( 98.5\% \).

% % For both dense and sparse retrieval we encode the query \( \texttt{q} \) and document \( \texttt{d} \) into their corresponding vector representations \( \vec{\mathbf{q}} \) and \( \vec{\mathbf{d}} \) and find their relevance score with:

% % \julian{Rewrote the below paragraph (see previous version commented above)}
% For dense and sparse retrieval the relevance score, \( s( \texttt{q},  \texttt{d}) \), between a query \( \texttt{q} \) and a document \( \texttt{d} \) can be computed in a unified way. Given the corresponding vector representations \( \vec{\mathbf{q}} \) and \( \vec{\mathbf{d}} \) the relevance score is defined as:
% \begin{align} \label{eq:rel_score}
%     s( \texttt{q},  \texttt{d}) = \vec{\mathbf{q}} \cdot \vec{\mathbf{d}}
% \end{align}


\subsection{Retrieval Paradigm}
In this work, we focus on two widely-adopted retrieval paradigms: dense retrieval \cite{ance, cl-drd, dpr} and sparse retrieval \cite{splade, splade-v2,snrm}. These approaches are chosen because they have demonstrated strong retrieval performance \cite{cl-drd, splade-v2,dragon} and both fall under the category of single-vector retrieval, where each document (or query) is encoded into a single vector.

Given the text \( \texttt{T} = [t_1, \ldots, t_L] \) with $L$ tokens and the large language model (LLM) \( M_\theta \) we can get the contextualized sequence representation for the text with hidden dimension \( D \) as follows:
\[
\mathbf{H}_T = M_\theta(\texttt{T}) \in \mathbb{R}^{D \times L}.
\]

For our \textit{dense retrieval} we use mean pooling to obtain a single vector representation:
\[
\vec{\mathbf{h}_T} = \text{AvgPool}(\mathbf{H}_T) \in \mathbb{R}^D.
\]

For \textit{sparse retrieval}, we project \( \mathbf{H}_T \) into the high-dimensional subword token space by doing a matrix multiplication with the LLM's embedding table \( \mathbf{E} \in \mathbb{R}^{D \times V} \) where \( V \) is the vocabulary size, followed by a max-pooling and rescaling operation similar to \cite{splade-v2}:
\[
\vec{\mathbf{c}_T} = \log \big( 1 + \text{ReLU} \big( \text{MaxPool} (\mathbf{E}^T \cdot \mathbf{H}_T) \big) \big) \in \mathbb{R}^V.
\]
% In modern LLMs, \( V \gg D \). For instance, in Llama3 8B, \( V = 128,256 \) while \( D = 4,096 \). 
To sparsify the high-dimensional vector \( \vec{\mathbf{c}_T} \), we apply FLOP regularization \cite{flop-reg} during training. 
% This ensures that only important tokens are assigned non-zero values. In our experiments, the sparsification rate (zero rate) achieves over \( 98.5\% \).

For dense and sparse retrieval the relevance score, \( s( \texttt{q},  \texttt{d}) \), between a query \( \texttt{q} \) and a document \( \texttt{d} \) is computed in a unified way. Given the corresponding vector representations \( \vec{\mathbf{q}} \) and \( \vec{\mathbf{d}} \) the relevance score is defined as:
\begin{align} \label{eq:rel_score}
    s( \texttt{q},  \texttt{d}) = \vec{\mathbf{q}} \cdot \vec{\mathbf{d}}
\end{align}


These two paradigms represent textual information differently: sparse retrieval captures \textit{lexical matching signals}, while dense retrieval focuses on \textit{semantic matching signals}. This difference makes them an ideal pair for comparative studies.

\subsection{Pretraining}
To adapt the decoder-only LLMs for the retrieval tasks, we adopt the pretraining strategy from~\cite{llm2vec}, which consists of two key components: (1) enabling bidirectional attention and (2) applying masked next token prediction (MNTP).

\paragraph{Enabling Bidirectional Attention} 
% We first replace the causal attention mask in decoder-only LLMs with a bidirectional attention mask. This allows each token to attend to every token in the sequence, thus improving token representation capability and boosting retrieval performance \cite{}. Therefore, we adopt LLM modified with bidirectional attention masks as the backbone of our retrieval models.

% We first replace the causal attention mask in decoder-only LLMs with a bidirectional attention mask. This allows each token to obtain information from every other token, thus improving the ability to represent the token and increasing the retrieval performance~\cite{echo-embed,llm2vec}. 

We first replace the causal attention mask with a bidirectional attention mask. This allows each token to obtain information from every other token and itself, thus improving the ability to represent the text content and increasing the retrieval performance~\cite{echo-embed,llm2vec}. 

\paragraph{Masked Next Token Prediction (MNTP)}

% We adapt the decoder-only LLMs to bidirectional attention by MNTP \cite{llm2vec}. Specifically, given an arbitrary input sequence $x = (x_1, x_2, \dots, x_N)$, we first mask a fraction of input tokens ($20\%$ in our paper) and then train the model to predict these masked tokens based on the past and future context. Crucially, when predicting a masked token at position $i$, we calculate the loss based on the logits obtained from the token representation at the previous position $i - 1$, instead of from the masked position itself. We use MS MARCO corpus as pre-training corpus \footnote{We do at most 10,000 pre-training steps on MS MARCO, which can be completed in 17 hours on 2 A100 GPUs for Llama3-8B.}.

% \julian{Rewrote}
To adapt the LLMs to bidirectional attention, we train the models with masked next token prediction (MNTP) \cite{llm2vec}. Specifically, given an input sequence $x = (x_1, x_2, \dots, x_N)$, we mask $20\%$ of the input tokens and then train the model to predict these masked tokens based on the past and future context. Crucially, when predicting a masked token at position $i$, we calculate the loss based on the logits obtained from the token representation at the previous position $i - 1$, instead of from the masked position itself. This is done because it aligns with the LLM's existing causal training. We use MS MARCO corpus as pre-training corpus.\footnote{We do at most 10,000 pre-training steps on MS MARCO, which can be completed in 17 hours on 2 A100 GPUs for Llama3-8B.}

\subsection{Finetuning}
We choose (1) Contrastive Loss (\ContrastiveAcronym), (2) Knowledge Distillation (KD), and (3) the combination of \ContrastiveAcronym and KD as our retrieval fine-tuning objectives.

\ContrastiveAcronym is a simple yet effective retrieval training objective. It does not rely on any external models during training, and has been proven effective on both in-domain and out-of-domain evaluation sets \cite{co-condenser}. Formally, given a query $q$, a positive document $d^+$, and a list of negative documents $\mathcal{D}_q^- := [ d^-_1, \ldots , d^-_N]$, 
% we can use either dense or sparse retrieval to obtain their relevance scores using Eq \ref{eq:rel_score}, and CL loss for the query $q$ is:
using Eq \eqref{eq:rel_score}, the \ContrastiveAcronym loss for the query $q$ is:
\[
\mathcal{L}_{\text{\ContrastiveAcronym}} = -\log \frac{\exp(s(q,d^+))}{\exp(s(q,d^+) + \displaystyle\sum_{d^- \in \mathcal{D}_q^-} \exp(s(q, d^-))}
\]

% \hamed{would a reviewer argue that your KD didn't work just because you used a much smaller teacher model than a student model?}

\textit{KD} leverages an external teacher model\footnote{We use \url{https://huggingface.co/cross-encoder/ms-marco-MiniLM-L-6-v2}.} - typically a cross-encoder reranker - to provide more fine-grained soft labels for retrieval model training. 
%Previous work has demonstrated that KD can improve retrieval performance and reduce the performance gap between small and large retrieval models on in-domain evaluation sets. 
%However, its generalization ability to out-of-domain corpora and across different retrieval paradigms remains underexplored, an aspect we aim to investigate in this work. 
% We use MarginMSE \cite{} as the KD loss due to its strong results. 
We use MarginMSE \cite{margin-mse} as the KD loss in our initial comparison due to its strong performance. 
% Formally, given a query $q$, a positive document $d^+$, and a negative document $d^-$, we can use an external reranker to get the teacher scores: $T(q, d^+)$ and $T(q, d^-)$, and the MarginMSE loss is:
Formally, given a query $q$, a positive document $d^+$, a negative document $d^-$, and teacher scores $T(q, d^+)$ and $T(q, d^-)$, the MarginMSE loss is:
% \[ 
%     \mathcal{L}_{KD} = \big((s(q, d^+) - s(q,d^-)) - (T(q,d^+) - T(q,d^-))\big)^2
% \]
\[ 
    \mathcal{L}_{KD} = MSE(s(q, d^+) - s(q,d^-),\, T(q,d^+) - T(q,d^-)\big)
\]

% We also explore combining \textit{\ContrastiveAcronym} and \textit{KD}. To seamlessly integrate into the listwise input style of CL, we use KL-Divergence \cite{} as our KD loss. It minimizes the difference between the listwise score distribution produced by the student retriever and the teacher reranker. The final loss is the balanced weighted sum of CL and KD, which is $\mathcal{L}_{\text{comb}} = 0.5 \cdot (\mathcal{L}_{\text{CL}} + \mathcal{L}_{KD} )$. 


% To seamlessly integrate into the listwise input style of CL, we use KL-Divergence \cite{} as our KD loss. It minimizes the difference between the listwise score distribution produced by the student retriever and the teacher reranker. The final loss is the balanced weighted sum of CL and KD, which is $\mathcal{L}_{\text{comb}} = 0.5 \cdot (\mathcal{L}_{\text{CL}} + \mathcal{L}_{KD} )$. 
% \julian{Rewrote}
We also explore combining \textit{\ContrastiveAcronym} and \textit{KD}. As \ContrastiveAcronym works with  a list of documents and MarginMSE only works for triplets we switch the KD loss to KL-Divergence \cite{colbertv2,rocketqav2}. KL-Divergence minimizes the difference in the listwise distribution of the teacher and student. The final loss is the weighted sum of the \ContrastiveAcronym and KD loss terms which is: $\mathcal{L}_{\text{comb}} = 0.5 \cdot (\mathcal{L}_{\text{CL}} + \mathcal{L}_{KD} )$.

\subsection{Training and Evaluation}
% We use the MS MARCO passage retrieval dataset~\cite{} for training and evaluate the model’s in-domain and out-of-domain zero-shot retrieval performance. For in-domain evaluation, we select MS MARCO Dev\cite{}, TREC DL 19\cite{}, and TREC DL 20~\cite{}, as they share the same distribution as the training data. We follow the official evaluation protocol, reporting MRR@10 for MS MARCO Dev and nDCG@10 for TREC DL 19 and 20.
% For out-of-domain evaluation, we adopt the BEIR benchmark\cite{}. To ensure comparability with existing methods\cite{}, we compute nDCG@10 across the 13 datasets in BEIR, making the results directly comparable.  
% \julian{Rewrote}
We use the MS MARCO passage retrieval dataset~\cite{msmarco} for training. For in-domain evaluation, we select MS MARCO Dev \cite{msmarco}, TREC Deep Learning (DL) Track 19 and 20~\cite{trec-dl-19,trec-dl-20}, as they share the same distribution as the training data. We follow the official evaluation protocol, reporting MRR@10 for MS MARCO Dev and nDCG@10 for TREC DL 19 and 20. For out-of-domain evaluation, we adopt the BEIR benchmark \cite{beir}. Following previous work  \cite{colbertv2,dragon,repllama}, we compute nDCG@10 across 13 datasets in BEIR, making the results directly comparable.  

% For training, we use the MS MARCO passage retrieval dataset \cite{}, because:
% 1) It contains a large-scale and diverse set of web search queries, and models trained on it have been shown to generalize well to zero-shot retrieval benchmarks~\cite{}.  
% 2) The relevance judgments between queries and documents are human-annotated, ensuring high-quality training and evaluation signals. Since the focus in this paper is on the modeling side, we do not include additional retrieval training sets from other domains~\cite{}.
% Also, we use only the MS MARCO dataset and do not incorporate retrieval data from other domains~\cite{}, as our focus in this paper is on the modeling side.
%3) Extensive prior research has been conducted on this dataset, allowing for direct comparison with existing baselines and providing meaningful insights.  

% We choose to use only the MS MARCO dataset for training because our study primarily focuses on the modeling aspects, such as training objectives, retrieval paradigms, and model scaling. To maintain a controlled experimental setup, we minimize variability on the data side by using a single dataset.  

% For evaluation, we assess the model's in-domain and zero-shot out-of-domain retrieval performance. For


% For evaluation, we assess the model’s in-domain and out-of-domain retrieval performance:  

% \begin{itemize}
%     \item \textbf{In-domain evaluation}: We use MS MARCO Dev, TREC DL 2019 \cite{}, and TREC DL 2020\cite{} to measure performance within the same distribution as the training data.  
%     \item \textbf{Out-of-domain evaluation}: We adopt the BEIR benchmark \cite{}, which covers a diverse range of retrieval tasks—including ad-hoc retrieval, fact-checking, and question answering—across multiple domains such as biomedical literature and Wikipedia. This enables a comprehensive assessment of the model's generalization ability.  
% \end{itemize}

% We evaluate the model's in-domain and out-of-domain zero-shot retrieval performance. For in-domain evaluation, we choose MS MARCO Dev~\cite{}, TREC DL 19~\cite{}, and TREC DL 20~\cite{}, as they share the same distribution of training sets.  We use the official metrics MRR@10 for MS MARCO Dev, and nDCG@10 for TREC DL 19 and 20.
% We use the BEIR benchmark~\cite{} for out-of-domain evaluation, which contains diverse retrieval tasks across multiple domains. Specifically, we compute the nDCG@10 over the 13 datasets in BEIR, making the numbers best comparable with other existing methods~\cite{}. 


\section{Analysis}
\begin{figure}
    \centering
    \includegraphics[width=1\linewidth]{figures/sparse_vs_dense.png}
    \caption{Dense and sparse retrieval results on the combined of TREC DL 19 and 20 (in-domain), and BEIR (out-of-domain) datasets. 
    % The fine-tuning objectives are KD and \ContrastiveAcronym, and the model sizes are 1B, 3B, 8B.
    }
    \label{fig:sparse-vs-dense}
    \vspace{-0.5cm}
\end{figure}

% \paragraph{\textbf{Scaling Emerges Significantly with \ContrastiveAcronym, Not with KD}}
\paragraph{\textbf{Scaling Behavior Clearly Emerges with \ContrastiveAcronym, Not with KD}}
From Figure~\ref{fig:sparse-vs-dense}, we observe that when the fine-tuning objective is \ContrastiveAcronym the retrieval model's performance, both for sparse and dense retrieval, increases significantly as the model size grows for both in-domain and out-of-domain evaluation sets. 


% In contrast, when KD loss is used the performance improvement is far less pronounced with a decrease in performance in some cases. On the in-domain evaluation set, the 1B and 8B KD-models perform similarly. Notably, the 1B model is very strong; for example, dense-KD at 1B outperforms dense-CL at 8B on the TREC 19 and 20 evaluation sets. However, when evaluated on the out-of-domain BEIR benchmark, KD-trained models exhibit signs of overfitting: as model size increases, their performance degrades. Specifically, dense-KD models show a decline in performance with larger scales. In addition, KD-trained models tend to perform worse than CL-trained models at larger scales. For example, at the 8B scale, both sparse-KD and dense-KD perform worse than their respective sparse-CL and dense-CL models.
% \julian{Rewrote below}
In contrast, when KD loss is used the performance improvement is far less pronounced with a decrease in performance in some cases. On the in-domain evaluation set, the 1B and 8B KD models perform similarly. Though, compared to \ContrastiveAcronym the KD models perform much better on the in-domain evaluation sets. For example, the 1B KD models outperform the 8B \ContrastiveAcronym models on TREC 19 and 20. However, when evaluated on the out-of-domain BEIR benchmark, KD-models exhibit signs of overfitting. Specifically, dense-KD models show a decline in performance at larger scales. With both sparse and dense KD models performing worse than \ContrastiveAcronym models on out-of-domain evaluation metrics at larger scales. For example, at the 8B scale, both sparse-KD and dense-KD perform worse than their respective sparse-\ContrastiveAcronym and dense-\ContrastiveAcronym models.
% Specifically, we make the following observations: 
% \begin{itemize}
%     \item Retrieval models trained with KD loss achieve strong performance even at the 1B scale, particularly for in-domain evaluation sets. For instance, the dense-KD 1B model outperforms the dense-CL 8B model on the TREC 19+20 set.
%     \item However, this phenomenon does not hold for the out-of-domain benchmark BEIR. Both 1B sparse-KD and 1B dense-KD models underperform their 8B counterparts trained only with CL. Moreover, for dense-KD models, increasing model size can even lead to performance degradation on BEIR.
% \end{itemize}
\paragraph{\textbf{Sparse Retrieval consistently Outperforms Dense Retrieval}}
% From Figure~\ref{fig:sparse-vs-dense}, we observe that sparse retrieval consistently outperforms dense retrieval, regardless of the fine-tuning objective. Specifically, sparse retrieval has stronger zero-shot generalization abilities. Although sparse and dense retrieval achieve similar performance on the in-domain TREC 19+20 evaluation set, sparse retrieval demonstrates significantly better generalization in the zero-shot out-of-domain BEIR benchmark. For example, at the 8B scale, sparse-KD surpasses dense-KD by $10.5\%$, while sparse-CL outperforms dense-CL by $4.3\%$. Moreover, unlike dense-KD which suffers from the overfitting (to the teacher model) issues, sparse-KD's performance monotonically improves when increasing model sizes.

From Figure~\ref{fig:sparse-vs-dense}, we observe that sparse retrieval consistently outperforms dense retrieval with both KD and \ContrastiveAcronym objectives, with a more pronounced increase on the out-of-domain evaluation. For example, at the 8B scale, sparse-KD surpasses dense-KD by $10.5\%$, while sparse-\ContrastiveAcronym outperforms dense-\ContrastiveAcronym by $4.3\%$ in BEIR.

Moreover, unlike dense-KD, which suffers from overfitting to the teacher model, sparse-KD improves steadily as model size increases. While both paradigms perform similarly on in-domain benchmarks, sparse retrieval demonstrates significantly stronger generalization, achieving much better results in zero-shot settings.

\subsection{\ContrastiveAcronym + KD achieves the best Trade-off}
\begin{table}[t]
    \centering
    % \caption{Sparse and dense retrieval performance with \ContrastiveAcronym + KD as the objective. The relative \textcolor{darkgreen}{improvement} (or \textcolor{darkred}{decrease}) of \ContrastiveAcronym+KD w.r.t the objective 1) only KD, 2) only CL in percentile are shown in the tuple. The metric for MSMARCO Dev is MRR@10, and for TREC 19+20 and BEIR is NDCG@10.} 
    \caption{Sparse and dense performance with \ContrastiveAcronym + KD loss. Each decimal is the main metric for the associated evaluation dataset. The first number in parentheses indicates the relative change from only KD while the second is the relative change from only \ContrastiveAcronym.} 
    \scalebox{0.85}{\begin{tabular}{l!{\color{lightgray}\vrule}ccc}
    \toprule
       Model Size  & 1B & 3B & 8B \\
        \midrule
        \multicolumn{2}{l}{\textbf{Sparse Retrieval with \ContrastiveAcronym + KD}} \\
        MSMARCO Dev 
        & .410 (\textcolor{darkgreen}{3.8\%}, \textcolor{darkgreen}{22.0\%}) 
        & .417 (\textcolor{darkgreen}{5.0\%}, \textcolor{darkgreen}{7.5\%}) 
        & .417 (\textcolor{darkgreen}{4.2\%}, \textcolor{darkgreen}{0.5\%})
        \\       
        TREC 19+20 & .749 (\textcolor{darkgreen}{0.1\%}, \textcolor{darkgreen}{18.9\%}) & 
.759 (\textcolor{darkgreen}{0.2\%}, \textcolor{darkgreen}{6.2\%}) & 
.762 (\textcolor{darkred}{-0.1\%}, \textcolor{darkgreen}{3.2\%})
 \\ 
        BEIR & .535 (\textcolor{darkgreen}{3.5\%}, \textcolor{darkgreen}{22.1\%}) & 
        .544 (\textcolor{darkgreen}{3.8\%}, \textcolor{darkgreen}{3.0\%}) & 
        .552 (\textcolor{darkgreen}{3.0\%}, \textcolor{darkred}{-0.9\%}) \\ 
        \midrule
        \multicolumn{2}{l}{\textbf{Dense Retrieval  with \ContrastiveAcronym + KD}} \\
        MSMARCO Dev & .404 (\textcolor{darkgreen}{4.1\%}, \textcolor{darkgreen}{20.2\%}) & 
.414 (\textcolor{darkgreen}{5.6\%}, \textcolor{darkgreen}{5.9\%}) & 
.417 (\textcolor{darkgreen}{5.0\%}, \textcolor{darkred}{-0.7\%})
 \\ 
        TREC 19+20 & .749 (\textcolor{darkgreen}{0.6\%}, \textcolor{darkgreen}{18.8\%}) & 
.760 (\textcolor{darkgreen}{0.8\%}, \textcolor{darkgreen}{7.4\%}) & 
.757 (\textcolor{darkgreen}{0.2\%}, \textcolor{darkgreen}{2.7\%}) \\
        BEIR & .500 (\textcolor{darkgreen}{2.2\%}, \textcolor{darkgreen}{14.2\%}) & 
.496 (\textcolor{darkgreen}{4.0\%}, \textcolor{darkgreen}{1.0\%}) & 
.501 (\textcolor{darkgreen}{3.3\%}, \textcolor{darkred}{-6.2\%})
 \\
 \bottomrule
    \end{tabular}}
    %with their relative \textcolor{darkgreen}{improvement} (or \textcolor{darkred}{decrease}) to the objective CL and KD only.} 
    \label{tab:nce+kd result}
\end{table}
From Figure~\ref{fig:sparse-vs-dense}, we observe that KD loss significantly boosts the performance of small-scale (1B) retrieval models, while \ContrastiveAcronym provides greater benefits for large-scale (8B) models. 
% This phenomenon can be attributed to the nature of MSMARCO's relevance labeling: although accurate, the labels are sparse (not all relevant documents are labeled). 
We hypothesize that small models benefit from the augmented soft labels provided by the teacher model. However, these soft labels are not perfect, and when the model size reaches 8B, the retrieval model's capacity may surpass that of the cross-encoder teacher model. At this point, \ContrastiveAcronym loss becomes more effective.

% Building on these findings, we propose a combination of CL and KD, named CL + KD, as a new training objective that combines the strengths of both, with results shown in Table~\ref{tab:nce+kd result}. We find that CL + KD achieves the best performance balance. For instance, 1B sparse retrieval trained with CL + KD achieves a score of 0.535 on BEIR, outperforming the CL-only counterpart by $22.1\%$ and the KD-only counterpart by $3.5\%$. Looking at the 8B scale, sparse retrieval with the combined loss outperforms the CL-only model by $0.5\%$ on MSMARCO Dev and $3.2\%$ on TREC 19+20, while incurring only a $0.9\%$ loss on BEIR.
% And 8B sparse retrieval with the mixture loss incurs only a $0.9\%$ loss on BEIR compared to the CL-only model, while gaining $0.5\%$ on MSMARCO Dev and $3.2\%$ on TREC 19+20.
% \julian{Rewrote below}
Building on these findings, we investigate a combination of \ContrastiveAcronym and KD, named \ContrastiveAcronym + KD, that combines the strengths of both, with results shown in Table~\ref{tab:nce+kd result}. We find that \ContrastiveAcronym + KD achieves the best performance balance. For all 1B and 3B models and all evaluation datasets \ContrastiveAcronym + KD increases performance over only KD and only \ContrastiveAcronym. For 8B parameter models there are some decreases in performance but these are offset by gains in other domains in most cases. For instance at the 8B scale, sparse retrieval with the combined loss outperforms the \ContrastiveAcronym-only model by $0.5\%$ on MSMARCO Dev and $3.2\%$ on TREC 19+20, while incurring only a $0.9\%$ loss on BEIR.

\subsection{Sparse Retrieval might Beat the Teacher}
Figure~\ref{fig:sparse-vs-dense} shows that sparse-\ContrastiveAcronym and dense-\ContrastiveAcronym outperform their KD counterparts on BEIR at the 3B and 8B scales. We hypothesize that large retrieval models may have greater capacity than their cross-encoder teacher models. To explore this possibility, we used the teacher model to rerank the original rankings of each retrieval model on BEIR and computed the relative performance difference between the original and reranked results, presented in Table~\ref{tab:sparse_dense_wrt_reranker}.

We observe that sparse-\ContrastiveAcronym surpasses the teacher at both the 3B and 8B scales, while dense-\ContrastiveAcronym outperforms the teacher only at the 8B scale. Another finding is that when KD loss is incorporated (KD and \ContrastiveAcronym + KD), sparse retrieval remains highly robust and consistently outperforms the teacher model. However, for dense retrieval trained with the mixture of losses, performance consistently falls behind the teacher reranker by $1.8\%$, $2.4\%$, and $1.4\%$ at the 1B, 3B, and 8B scales, respectively. These findings suggest that sparse retrieval is a more effective and robust retrieval paradigm and has the potential to serve as a stronger teacher model.
\begin{table}
    \centering
    % \caption{Performance of sparse and dense retrieval finetuned with either CL or CL+KD objective in the BEIR (out-of-domain) benchmark. Their relative \textcolor{darkgreen}{improvement} (or \textcolor{darkred}{decrease}) w.r.t the teacher reranker is in bracket in percentile. The evaluation metric is NDDG@10.  }
    \caption{nDCG@10 of sparse and dense retrieval finetuned with \ContrastiveAcronym, KD or \ContrastiveAcronym+KD objective on the BEIR benchmark. The relative change from the teacher reranked results is shown in the parentheses.}
    \begin{tabular}{l!{\color{lightgray}\vrule}ccc}
    \toprule
       Model Size  &  1B & 3B & 8B \\
       \midrule
       \multicolumn{3}{l}{\textbf{Sparse Retrieval in BEIR}} \\
        \ContrastiveAcronym & .496 (\textcolor{darkred}{-8.8\%}) & .528 (\textcolor{darkgreen}{1.3\%}) & .557 (\textcolor{darkgreen}{7.4\%})
 \\
 KD & .517 (\textcolor{darkgreen}{0.9\%}) & .524 (\textcolor{darkgreen}{1.9\%}) & .536 (\textcolor{darkgreen}{4.0\%}) \\
    \ContrastiveAcronym + KD & .535 (\textcolor{darkgreen}{4.3\%}) &  .544 (\textcolor{darkgreen}{6.1\%}) & .552 (\textcolor{darkgreen}{7.6\%}) \\
\midrule
\multicolumn{2}{l}{\textbf{Dense Retrieval in BEIR }}  \\
\ContrastiveAcronym & .435 (\textcolor{darkred}{-14.7\%}) &  .491 (\textcolor{darkred}{-5.1\%}) &  .534 (\textcolor{darkgreen}{3.1\%}) \\
KD & .489 (\textcolor{darkred}{-4.5\%}) & .477 (\textcolor{darkred}{-6.7\%}) & .485 (\textcolor{darkred}{-5.3\%})\\
\ContrastiveAcronym + KD & .500 (\textcolor{darkred}{-1.8\%}) &  .496 (\textcolor{darkred}{-2.4\%}) &  .501 (\textcolor{darkred}{-1.4\%}) \\
\bottomrule
    \end{tabular}
    \label{tab:sparse_dense_wrt_reranker}
    \vspace{-.5cm}
\end{table}

\begin{table}[]
    \centering
    % \caption{Experiment results of the proposed \framework with other SOTA retrieval models. O.O.D represents out-of-domain.}
    \caption{Results of the proposed \framework with other SOTA retrieval models. We use paired t-test with Bonferroni correction with p$\_$value < 0.01. The superscripts refer to significant improvements over CL-DRD($*$), SPLADE++($\dag$), ColBERTv2 ($\ddag$), and RepLlama ($\S$).}
    \scalebox{0.85}{
    \begin{tabular}{l!{\color{lightgray}\vrule}l!{\color{lightgray}\vrule}l!{\color{lightgray}\vrule}l!{\color{black}\vrule}l}
    \toprule
     & \multicolumn{3}{c!{\color{black}\vrule}}{In Domain} & O.O.D \\
     \midrule
    & \textbf{MARCO Dev} & \multicolumn{1}{l}{\textbf{TREC-19}} & \textbf{TREC-20} & \textbf{BEIR-13} \\
    & \multicolumn{1}{c!{\color{lightgray}\vrule}}{ MRR@10} & \multicolumn{2}{c!{\color{black}\vrule}}{ nDCG@10} &  nDCG@10 \\
    \midrule 
    \multicolumn{3}{l}{\textbf{Model size $\leq$ 1B }} \\
    CL-DRD & .381 & .725 & .683 & .448 \\
    SPLADE++ & .389 & .743 & .718 & .503 \\
    ColBERTv2 & .397 & .750 & .746 & .499 \\
    \framework-DS-1B & .404$^{* \dag \ddag}$ & .757$^{* \dag \ddag}$ & .740$^{* \dag}$ & .500$^{* \dag}$ \\
    \framework-SP-1B & .410$^{* \dag \ddag}$ & .747$^{* \dag}$ & .751$^{* \dag}$ & .535$^{* \dag \ddag}$ \\
    \midrule
    \multicolumn{3}{l}{\textbf{Model size > 1B}} \\
    RepLlama & .412 & .743 & .721 & .551 \\
    \framework-DS-8B & \textbf{.417}$^{* \dag \ddag \S}$ & .755$^{* \dag \ddag \S}$ & .759$^{* \dag \ddag \S}$ & .501$^{* \dag \ddag}$ \\
    \framework-SP-8B & \textbf{.417}$^{* \dag \ddag \S}$ & \textbf{.758}$^{* \dag \ddag \S}$ & \textbf{.766}$^{* \dag \ddag \S}$ & \textbf{.552}$^{* \dag \ddag}$ \\
    \bottomrule
    \end{tabular}}
    \label{tab:result-comparison}
    \vspace{-1.cm}
\end{table}

\section{Performance Comparison with Other Retrieval Models}
% We denote the sparse retrieval model proposed in this paper, trained with KD+CL and using LLaMA3-8B as the backbone, as \framework-SP-8B, and its dense retrieval counterpart as \framework-DS-8B. Similarly, their 1B versions are referred to as \framework-SP-1B and \framework-DS-1B, respectively. For retrieval models smaller than 1B, we select state-of-the-art baselines from each retrieval paradigm: CL-DRD~\cite{} (dense retrieval), SPLADE++~\cite{} (sparse retrieval), and ColBERTv2~\cite{} (multi-vector retrieval). For models larger than 1B, we use RepLLaMA~\cite{} as the baseline. The in-domain and out-of-domain performance of all models is shown in Table~\ref{tab:result-comparison}. 
% \julian{Rewrote}
For comparison with our 1B models we select state-of-the-art baselines from each retrieval paradigm that are smaller than 1B parameters: CL-DRD~\cite{cl-drd} (dense retrieval), SPLADE++~\cite{splade-v2} (sparse retrieval), and ColBERTv2~\cite{colbertv2} (multi-vector retrieval). For models larger than 1B, we use RepLLaMA~\cite{repllama} as the baseline. We refer to our models trained with KD + CL as \framework-[TYPE]-[SIZE] where the type SP indicates a sparse model and DS indicates a dense model, e.g. \framework-SP-8B is the sparse model with LLaMA3-8B as the base model. The in-domain and out-of-domain performance of all models is shown in Table~\ref{tab:result-comparison}. 

We observe that \framework-SP-1B and \framework-DS-1B achieve the best overall results among baselines with $\leq$ 1B parameters, demonstrating that recent large decoder-only LLMs serve as a powerful backbone for document retrieval. For instance, \framework-SP-1B surpasses ColBERTv2 by $3.2\%$ on MS MARCO Dev and by $7.2\%$ on BEIR. Moreover, \framework-SP-8B outperforms the state-of-the-art model RepLLaMA across both in-domain and out-of-domain benchmarks, underscoring the effectiveness of the sparse retrieval paradigm and the combination of CL and KD loss. 

Additionally, \framework's performance on individual BEIR datasets is provided in Table~\ref{tab:beir-per-ds-results} in the Appendix for reference.
\vspace{-.25cm}
\section{Related Work}
 Single-vector retrieval models can be broadly categorized into two types: dense retrieval \cite{ance,dpr,cl-drd,rocketqav2,margin-mse,tas-b,repllama,llama2vec,echo-embed} and sparse retrieval \cite{splade,splade-mistral,splade-v2,snrm,spade}. The key distinction lies in their representations: dense retrieval encodes texts into low-dimensional vectors (i.e. 768 dim in BERT) and employs approximate nearest neighbor (ANN)~\cite{ance,faiss-gpu} search for efficient retrieval. In contrast, sparse retrieval represents texts using high-dimensional sparse vectors and relies on an inverted index~\cite{spade,inverted-index} for fast retrieval. 
 %Compared to multi-vector retrieval models~\cite{citadel,colbertv2,colbert}, such as ColBERT~\cite{colbert}, single-vector retrieval is faster, requires less index memory, and is easier to implement.
 
The most straightforward fine-tuning objective for retrieval models is supervised contrastive loss (CL) \cite{repllama,co-condenser}. By incorporating techniques such as hard negative sampling \cite{ance,adore} and retrieval-oriented continued pre-training \cite{b-prop,co-condenser}, the performance of retrieval models can be further improved. 
%However, CL faces several challenges: 1) it requires a large batch size \cite{rocketqa}, 2) it performs poorly on small models \cite{tas-b,margin-mse}, and 3) it is less robust to noisy data \cite{rocketqa}. To address these limitations, 
Knowledge distillation (KD)~\cite{colbertv2,rocketqav2,cl-drd,tas-b,margin-mse,tct-colbert,kd} is another widely used fine-tuning objective. 
%Unlike CL, KD does not require a large batch size \cite{tas-b}, making it more computationally efficient and less demanding on GPU resources. 
It has also been shown to perform comparably to or even better than CL for small-scale retrieval models, particularly on in-domain benchmarks \cite{tas-b,cl-drd}.

Scaling LLMs has consistently led to improved performance across various NLP tasks~\cite{gpt-3,chinchilla,mt-scaling-1,mt-scaling-2,emergent-llm}. This trend also extends to retrieval models. ~\citet{scale-dr} demonstrated that scaling BERT-based dense retrieval models enhances retrieval performance.
Recent works~\cite{llama2vec,echo-embed,splade-mistral,llm2vec} have further explored this trend by training dense retrieval models with larger backbones, such as LLaMA2-7B~\cite{llama2}. These models have achieved state-of-the-art performance, surpassing traditional BERT-based retrieval models.
%However, these studies have exclusively focused on contrastive loss (CL) for dense retrieval, overlooking the scaling behaviors of sparse retrieval and knowledge distillation (KD), which is our focus in the paper.
\vspace{-.25cm}
\section{Conclusions and future work}
We conduct a comprehensive study on the scaling behaviors of different retrieval paradigms and fine-tuning objectives in decoder-only LLMs. One limitation of this work is that we only evaluate knowledge distillation (KD) with a single teacher model. In future work, we aim to expand this analysis by exploring teacher models with varying sizes and different supervision strategies to gain deeper insights into their impact on retrieval performance.

\begin{acks}
    This work was supported in part by the Center for Intelligent Information Retrieval, and in part by the Office of Naval Research contract number N000142412612. Any opinions, findings and conclusions or recommendations expressed in this material are those of the authors and do not necessarily reflect those of the sponsor.
\end{acks}


\bibliographystyle{ACM-Reference-Format}
\bibliography{XX-references}
\clearpage


\appendix
\section{Implementation Details}
\begin{table}[t]
    \centering
    \caption{Training configurations for sparse and dense retrieval models. 
    "Grad. accu." refers to gradient accumulation steps, and "Num. negs" represents the number of negative documents used during training.}

    \scalebox{0.85}{
    \begin{tabular}{l!{\color{lightgray}\vrule}ccccc}
    \toprule
     Loss type & Size & Epochs & Batch size & Grad. accu. & Num. negs \\
        \midrule
        CL, CL+KD & 1B & 7 & 28 & 1 & 16 \\
        CL, CL+KD & 3B & 3 & 16 & 2 & 16 \\
        CL, CL+KD & 8B & 1 & 8 & 4 & 16 \\
        \midrule
        KD & 1B & 56 & 256 & 1 & 1 \\
        KD & 3B & 24 & 128 & 1 & 1 \\
        KD & 8B & 8 & 64 & 2 & 1 \\
        \toprule
    \end{tabular}}
    \label{tab:train-config}
\end{table}
For MNTP pre-training, we use 2×A100 80GB GPUs. The training batch size per device is 32, with a maximum sequence length of 512 and a gradient accumulation step of 1. We apply BERT’s masking strategy with a masking probability of $20\%$ and train all models for 10,000 steps.

For fine-tuning, we use 4×A100 80GB GPUs for all training runs and maintain a fixed compute budget across all configurations. 
Each training run completes within approximately 40–44 hours. 
The key fine-tuning hyperparameters that impact the training time for different model scales and fine-tuning objectives are presented in Table~\ref{tab:train-config}.
As for other hyperparameters, we set the learning rate at 1e-4, the warmup ratio at $4\%$ of all training steps, the maximum query length at 64, the maximum document length at 128. We set LoRA $r=16$ and $\alpha=32$. In sparse retrieval training, the FLOPs regularization coefficient is 0.05 for query vectors and 0.04 for document vectors. We use AdamW as the optimizer. For BEIR evaluation, the maximum sequence length is set to 512.

\section{BEIR results}
\begin{table}[t]
    \centering
    \caption{retrieval performance across datasets in BEIR.}
    \scalebox{0.85}{\begin{tabular}{l!{\color{lightgray}\vrule}cccc}
        \toprule
        Dataset & \framework-DS-1B & \framework-SP-1B & \framework-DS-8B & \framework-SP-8B \\
        \midrule
        ArguAna & 0.476 & 0.488 & 0.494 & 0.484 \\
        FiQA & 0.365 & 0.378 & 0.386 & 0.398 \\
        NFCorpus & 0.341 & 0.365 & 0.363 & 0.370 \\
        Quora & 0.766 & 0.791 & 0.726 & 0.874 \\
        SciDocs & 0.177 & 0.171 & 0.191 & 0.178 \\
        SciFact & 0.691 & 0.732 & 0.687 & 0.730 \\
        TREC-COVID & 0.797 & 0.783 & 0.807 & 0.825 \\
        Webis-Touche2020 & 0.274 & 0.330 & 0.232 & 0.358 \\
        Climate-Fever & 0.274 & 0.304 & 0.283 & 0.281 \\
        DBPedia-Entity & 0.442 & 0.464 & 0.456 & 0.465 \\
        FEVER & 0.691 & 0.870 & 0.627 & 0.874 \\
        HotpotQA & 0.625 & 0.692 & 0.658 & 0.729 \\
        NQ & 0.575 & 0.585 & 0.606 & 0.611 \\
        \midrule
        Avg & 0.500 & 0.535 & 0.501 & 0.552 \\
        \bottomrule
    \end{tabular}}
    \label{tab:beir-per-ds-results}
\end{table}

We also report the detailed BEIR performance per dataset for our proposed \framework models shown in Table~\ref{tab:beir-per-ds-results}.




\end{document}
