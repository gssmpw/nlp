\section{Literature Review}
Understanding collective pedestrian movement has been an active area of research, with various methodologies proposed to detect and analyze groups of individuals moving together. These approaches can broadly be categorized into traditional model-based methods and modern data-driven techniques, with recent advancements exploring hybrid strategies to improve detection accuracy.

\subsection{Traditional Approaches}

Early research in flock detection primarily relied on model-based and clustering techniques. One of the most widely used methods is Density-Based Spatial Clustering of Applications with Noise (DBSCAN), which identifies pedestrian groups based on spatial proximity and density ____. While effective in many cases, DBSCAN and other density-based techniques face challenges in dynamic environments where pedestrian group sizes and movement patterns change over time.

Graph-based methods have also been extensively used, where pedestrians are represented as nodes and edges denote spatial or temporal interactions ____. These methods often employ social force models to simulate pedestrian dynamics, considering factors like inter-agent distance and collision avoidance. While useful for understanding local interactions, such approaches struggle with long-range dependencies and occlusions, limiting their applicability in real-world, large-scale crowd scenarios.

Another line of research involves rule-based and statistical models, such as Hidden Markov Models (HMMs) and Kalman filters, to estimate pedestrian movement patterns. These models typically assume linear motion patterns and rely on handcrafted features, making them less adaptable to complex and unstructured pedestrian flows ____. 

\subsection{Modern Data-Driven Approaches}

With the rise of deep learning, trajectory-based modeling has shifted towards data-driven methods that learn patterns directly from large-scale pedestrian datasets. Recurrent Neural Networks (RNNs) and Long Short-Term Memory (LSTM) networks have been widely used for pedestrian trajectory prediction and group detection due to their ability to capture sequential dependencies ____. These models have demonstrated improved accuracy in detecting pedestrian flocks by learning motion tendencies over time.

More recently, Transformer-based architectures have emerged as state-of-the-art models in sequence learning tasks. Unlike RNNs and LSTMs, Transformers rely on self-attention mechanisms, allowing them to model long-range dependencies more effectively without suffering from vanishing gradient issues ____. This has led to improved performance in large-scale pedestrian behavior analysis, where long-term dependencies and complex interactions are prevalent.

Additionally, attention-based models, such as Social-LSTM and Social-GAN, have been proposed to model social interactions among pedestrians, dynamically adapting their predictions based on surrounding agents ____. These models leverage learned representations to classify pedestrian groups, offering a more flexible alternative to traditional rule-based approaches.

\subsection{Hybrid Approaches}

Recent studies have explored the integration of traditional and deep learning-based methods to enhance flock detection capabilities. Hybrid models, such as DBSCAN combined with LSTMs, have been used to detect groups first through clustering and then refine predictions using sequential learning models ____. Similarly, Graph Neural Networks (GNNs) have been applied to pedestrian movement analysis, capturing both spatial and temporal dependencies by integrating graph structures with deep learning architectures ____.

The combination of these approaches has demonstrated promising results, particularly in dynamic and large-scale scenarios where traditional clustering methods alone may fail. By leveraging both spatial clustering techniques and deep sequence modeling, hybrid models provide a robust framework for real-time multi-agent flock detection.