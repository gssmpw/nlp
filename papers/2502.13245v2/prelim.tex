\section{Preliminaries}\label{sec:prelim}

Here, we define the range search problem and the metric for measuring the accuracy of the search. Range searching, conceptually, can be defined as retrieving results within a ball of radius $r$ for a given metric space. We then define average precision, which indicates how well a search algorithm finds the correct result given a set of queries.


 In this work,
we study a set $\mP\subseteq \mathbb{R}^{d}$ of $n$ points (vectors)
in $d$ dimensions.
We denote the \defn{distance} between two points $p,q\in
\mathbb{R}^{d}$ as $\dist{p}{q}$.  Smaller distance indicates greater
similarity.

Commonly-used distance functions include Euclidean distance ($L_2$
norm), and maximum inner product search, which uses negative inner product as a distance function..


\begin{definition}[Range Search]
	Given a set of points $\mP$ in $d$-dimensions, a query point $q$ and radius $r$, the range search problem finds a set $\mK\subseteq \mP$
	such that $\max_{p\in \mK} \dist{p}{q} \le r$.
\end{definition}


Next we define average precision, a commonly used metric that captures the accuracy of a search algorithm. Note that the measure weights a point's contribution to average precision proportionately to its size. 

\begin{definition}[Average Precision]
	Let $\mP$ be a set of points in $d$-dimensions and $\mQ$ a query point set. For $q \in \mQ$ , let $\mK$ be the true range neighbors of $q$ in $\mP$. Let $\mK' \subset \mP$ be an output of a range search algorithm. Then \defn{Average Precision} is defined as $\frac{\sum\limits_{q \in \mQ}|\mK \cap \mK'|}{\sum\limits_{q \in \mQ}|\mK|}$ 
\end{definition}
