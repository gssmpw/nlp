\section{Outline}

Motivation for range search.   Lack of current algorithms.  Some work
for bucketing based techniques but none for graph based.

Naive approach is just to pick a large Q.

In this paper we present improvements.

In particular we show that can be very effective for some data sets.
Separately handle no points, few points and many points.

\section{Characteristics of Range Search}

Need to figure out what graphs to use.

\begin{itemize}
	\item: Euclidean: sift, gist symsearch, msturing, deep, fashion
	\item: Mips: image2image, wikipedia, msmarco,
	\item nytimes, glove, openai
\end{itemize}

\begin{itemize}
	\item distribution of sizes
	\item plot by radius and percent
	\item plot by step
\end{itemize}

Discuss two types of recall.

Discuss fuzzy recall.

Discuss euclidean vs angular vs mips.


Talk a little bit about prior benchmarks.
Perhaps show that symsearch-net is hard even for k-nearest neighbor.

Possibly give numbers for naive algorithm.

\section{Algorithms}

\begin{itemize}
	\item doubling approach
	\item greedy search
	\item early stopping.  show plots of how well methods separate points
\end{itemize}

\section{Experiments}

\begin{itemize}
	\item qps vs accuracy (both cumulative and pointwise)
	\item recall by category (small, large)
	\item compare to 10-at-10 knn
	\item maybe plot as radius
\end{itemize}