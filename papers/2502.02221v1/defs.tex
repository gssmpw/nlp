% Add yourself here to use color-coded coments
\usepackage{color-edits}
\addauthor{jiri}{red}
\addauthor{illia}{purple}

% \usepackage[capitalise]{cleveref} % to use \cref commands

% \newcommand{\debug}[1]{{\color{purple}#1}}		% for macro coloring
\newcommand{\debug}[1]{#1}		% for removing macro coloring
\newcommand{\newmacro}[2]{\newcommand{#1}{\debug{#2}}}		% for shorthand definitions





\newcommand{\RR}{ \mathbb{R} }
\newcommand{\NN}{\mathbb{N}}
\newcommand{\oneover}[1]{\frac{1}{#1}}
\newcommand{\spaceo}{\hspace{2 mm}}
\newcommand{\setsep}{ \spaceo | \spaceo}
\newcommand{\half}{\frac{1}{2}}
\newcommand{\textgoth}[1]{\mathcal{#1}}
\newcommand{\Prob}[1]{\mathbb{P}\left( #1 \right)}
\newcommand{\Probb}[1]{\mathbb{P}\left[ #1 \right]}
\newcommand{\Probu}[2]{\mathbb{P}_{#1}\left( #2 \right)}
\newcommand{\sym}{~}
\newcommand{\argmax}{\operatornamewithlimits{argmax}}

\newcommand{\Abs}[1]{\left| #1 \right|}
\newcommand{\Set}[1]{\left\{ #1 \right\}}
\newcommand{\Brack}[1]{\left( #1 \right)}
\newcommand{\BBrack}[1]{\left\{ #1 \right\}}
\newcommand{\SqBrack}[1]{\left[ #1 \right]}
\newcommand{\inner}[2]{\left< #1 , #2 \right>}
\newcommand{\tens}{\otimes}
\newcommand{\Exp}[1]{ \mathbb{E} #1}
\newcommand{\Expsubidx}[2]{ \mathbb{E}_{#1} #2}

\newcommand{\norm}[1]{\left\|#1\right\|}
\newcommand{\norms}[1]{\left|#1\right|}

\newcommand{\normsup}[1]{\norm{#1}_{\infty}}
\newcommand{\normLtwo}[1]{\norm{#1}_{L_2}}

\newcommand{\normop}[1]{\left\|#1\right\|_{op}}
\newcommand{\normopnuc}[1]{\left\|#1\right\|_{nuc}}
\newcommand{\normophs}[1]{\left\|#1\right\|_{HS}}

\newcommand{\tr}{tr}
\newcommand{\Ind}[1]{ \mathbbm{1}_{\Set{#1}} }
\newcommand{\eps}{\varepsilon}


%\DeclareMathOperator*{\argmin}{\arg\!\min} 

\newcommand{\quater}{\frac{1}{4}}


\newcommand{\disjcup}{\mathop{\dot{\bigcup}}}

\newcommand{\grad}{\nabla}
\newcommand{\partd}[2]{\frac{\partial #1}{\partial #2}}

\newlength{\dhatheight}
\newcommand{\doublehat}[1]{%
    \settoheight{\dhatheight}{\ensuremath{\hat{#1}}}%
    \addtolength{\dhatheight}{-0.35ex}%
    \hat{\vphantom{\rule{1pt}{\dhatheight}}%
    \smash{\hat{#1}}}}


\newcommand{\mc}[1]{\mathcal{#1}}
\newcommand{\onevec}{\mathbbm{1}}

% ------- PAPER VARIABLES ---------

\newmacro{\ndata}{n}
\newmacro{\nsamples}{\ndata}
\newmacro{\ndims}{d}
\newmacro{\dims}{[\ndims]}
\newmacro{\nprot}{p}
\newmacro{\distance}{\Delta}

\newmacro{\prot}{\mathcal{P}}
\newmacro{\nonprot}{\mathcal{N}}
\newmacro{\target}{Y}

\newmacro{\subg}{S}
\newcommand{\subgs}[1][\prot]{\debug{\mathcal{S}_{#1}}}
\newmacro{\Xsubg}{\mathcal{X}_\subg}

% specific to data
% class
\newmacro{\positive}{+}
\newmacro{\negative}{-}

\newmacro{\datadistr}{D}
\newmacro{\posdistr}{\mu}
\newmacro{\negdistr}{\nu}

\newmacro{\data}{\mathcal{D}}
\newmacro{\datapos}{\mathcal{D}_\positive}
\newmacro{\dataneg}{\mathcal{D}_\negative}

% MIO

\newmacro{\error}{e}
\newcommand{\errvar}[1][i]{\error_{#1}}

\newmacro{\usevariable}{u}
\newcommand{\usevar}[1][j]{\usevariable_{#1}}

\newmacro{\xvariable}{x}
\newcommand{\xvar}[1][i,j]{\xvariable_{#1}}

\newmacro{\minSize}{N_{\text{min}}}

\newmacro{\yhat}{\hat{y}_i}

% \newmacro{\predvariable}{\hat{y}}
% \newcommand{\predvar}[1][i]{\predvariable_{#1}}
\newmacro{\subgvariable}{s}
\newcommand{\subgvar}[1][i]{\subgvariable_{#1}}

\newmacro{\accvariable}{a}
\newcommand{\accvar}[1][i]{\accvariable_{#1}}
\newmacro{\accref}{\accvariable_\text{REF}}

% for outgroup elements
\newcommand{\accvarout}[1][i]{\accvariable^{\prime}_{#1}}
\newmacro{\accrefout}{\accvariable^{\prime}_\text{REF}}

% DNF

\newmacro{\DNF}{DNF}
\newmacro{\nterms}{t}


% ------- end PAPER VARIABLES ---------

\newcommand{\MSD}[4][\distance]{\debug{\textrm{MSD}}_{#1}(#2,#3;#4)}
\newmacro{\MSDdiff}{\MSD[\mathrm{diff}]\posdistr\negdistr\prot}

\newcommand{\MSDc}{\debug{\mathrm{MSD}_\mathrm{diff}}}

\newcommand{\istrue}[1]{\left\llbracket #1 \right\rrbracket}

\newcommand{\bigoh}[1]{\debug{\mathrm{O}}(#1)}
\newcommand{\bigomega}[1]{\debug{\mathrm{\Omega}}(#1)}

\newcommand{\size}[1]{\Abs{#1}}

\newmacro{\cond}{\,\mid\,} 

\newcommand{\negate}[1]{\debug{\overline{#1}}}
\newcommand{\powerset}[1]{\debug{2^{#1}}}
% \newcommand{\powerset}[1]{\debug{\mathscr{P}(#1)}}
