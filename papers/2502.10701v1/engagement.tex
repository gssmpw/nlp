\begin{figure}[t]
    \centering
    \begin{subfigure}{.38\textwidth}
        \centering
        \includegraphics[width=.7\textwidth]{figs/rq_1/num_comments_cdf_wo.pdf}
        \caption{Number of Comments}
        \label{fig:comments}
    \end{subfigure}
    \quad

    \begin{subfigure}{.38\textwidth}
        \centering
        \includegraphics[width=.7\textwidth]{figs/rq_1/upvote_ratio_cdf_wo.pdf}
        \caption{Upvote Ratio}
        \label{fig:upvote}
    \end{subfigure}
    \caption{Cumulative distributions of per-user mean engagement values, per disclosure type. 
    }
    \label{fig:all_engagement_cdf}
\end{figure}



\begin{figure}[t]
    \centering
    \begin{subfigure}{.5\textwidth}
        \centering
        \includegraphics[width=.85\linewidth]{figs/rq_1/numm_comments.pdf}
        \caption{Effect on Number of Comments}
        \label{fig:engagement_num_comments}
    \end{subfigure}
    \quad
    \begin{subfigure}{.48\textwidth}
        \centering
        \includegraphics[width=.85\linewidth]{figs/rq_1/score_ratio.pdf}
        \caption{Effect on Upvote Ratio}
        \label{fig:engagement_upvotes_ratio}
    \end{subfigure}
    \caption{Regression results for engagement. Each panel is a different regression model. The y-axis and x-axis show confounding factors and corresponding estimate, respectively.%, and error bars show the 95\% confidence interval. 
    }
    \label{fig:all_engagement_regression}
\end{figure}


Next, we explore user engagement with self-disclosing posts, focusing on how interactions in self-disclosure communities (\eg, /r/AMA) may encourage others to disclose, potentially creating privacy-compromising attack vectors. 




%%%%%%%%%%%%%%%%%%%%%%%%%%%%%%%%%%%%%%%%%%%
\subsection{Quantifying Engagement with Self-Disclosure} \label{subsec:engagement_diff}
%%%%%%%%%%%%%%%%%%%%%%%%%%%%%%%%%%%%%%%%%%%

From a privacy perspective, higher engagement reflects potentially increased exposure to users' information. It may also create a feedback loop that influences what users post in the future~\cite{haq_short_2022}.
We therefore start by measuring the difference in engagement levels across self-disclosing posts. 



Figure~\ref{fig:all_engagement_cdf} presents the distribution of two engagement metrics: \one number of comments %\two the difference between upvotes and downvotes, 
and \two the ratio of upvotes to downvotes, across all posts containing self-disclosure. 
A non-parametric Kruskal-Wallis test confirms significant differences across the distribution of each disclosure type (test statistics are in Table~\ref{tab:eng_post_hoc_num_comments} and \ref{tab:eng_post_hoc_upvote_ratio} in Appendix). 
For instance, posts with \sexualOrientation have more comments ($\mu = 16.4$) than posts with \job disclosure ($\mu = 11.6$). 

To systematically analyze these differences, we turn to regression analysis with users and time-fixed effects. For each engagement metric (number of comments and upvote ratio) and self-disclosure type, we design a separate engagement prediction task based on whether the posts contain a particular self-disclosure or not (not including other types of self-disclosure). The posts without self-disclosure from all users remain identical across the models, hence providing a common baseline across models to compare the results with non-disclosing posts and within different types.

Figure~\ref{fig:engagement_num_comments} and~\ref{fig:engagement_upvotes_ratio} show the regression results for number of comments and upvote ratio, respectively. The Y-axis shows the corresponding self-disclosure used as a confounding factor. The X-axis shows the regression estimate with a 95\% confidence interval. The values in the blue color are statistically significant ($p<0.05$), and the grey color shows non-significance.

We see that the presence of self-disclosure does have a significant effect on both the number of comments and the upvote ratio. Interestingly, the effect is not similar across all self-disclosure types, and there is also a disparity in each engagement metric for a given self-disclosure. For instance, the presence of \sexualOrientation disclosure increases the number of comments (2.65x); however, the same has a negative effect on the upvote ratio($\approx-.02x$). 

To further see the difference between heterosexual and potentially more vulnerable non-hetero, we do a keyword filtering of the posts (with sexual orientation disclosure) containing the words (\textit{gay, lesbian, bisexual, and straight}), and use a non-parametric Kruskal-Wallis test ($\chi^2 = 251.2, df = 3, p < 0.001 $), followed by Dunn's test, to see the engagement metric distributions difference across each keyword's post. We observe that posts containing words `gay' receive more comments (almost double) ($\mu = 102$) and lower upvote ($\mu = .80$) than other non-hetero (\eg upvote ratio for `lesbian'  $\mu = 0.83$).
However, it receives fewer comments than the posts containing `straight' keywords ($\mu = 119$).
This contrast shows that, although \sexualOrientation increases engagement in terms of comment count, the engagement quality is less positive, on average. 
We posit that such engagement metrics may influence users' future self-disclosure likelihood, as prior Reddit studies find engagement do affect topic choice \cite{haq_short_2022}. Moreover, a lower upvote ratio with higher comments may signify negative discussions and discontent with the original poster~\cite{risch_top_2020}, potentially leading to negative experiences, especially in disclosures like \sexualOrientation and \ethnicity.




\begin{figure}[t]
    \centering
    \includegraphics[width=0.48\textwidth]{figs/rq_1/regression_analysis_m1.pdf}
    \caption{%Regression estimates for 11 models. 
    Self-disclosure types on the y-axis represent one regression model with two independent variables (\#interactions and Interacted). The x-axis shows the $\beta$ estimates.}
    \label{fig:regression_analysis_m1}
\end{figure}



\subsection{Quantifying Impact of Self-Disclosure Communities} \label{subsec:engagement_effect}


There are various Reddit communities directly related to specific forms of self-disclosure, \eg \texttt{r/aznidentity} is related to discussions on ethnicity, \texttt{AskDocs} has health-related discussions, and \texttt{r/AskMen} has gender-related discussions. 
These pertain to \ethnicity, \health, and \gender, respectively. We hypothesize that engaging with members of such communities may increase one's own likelihood of disclosing, even to other communities. This may create attack surface where malicious actors purposefully post (fake) self-disclosure to encourage others to share. We next explore the potential presence of such behaviors, quantifying the impact of receiving comments that contain self-disclosure.




\pb{Self-Disclosure related Subreddits.} 
For the above analysis, we first obtain a set of subreddits dedicated to self-disclosure. The names of the subreddits indicate their association with a focused topic~\cite{adelani_estimating_2020}. We extract the top 50 largest subreddits, in terms of their number of posts classified as containing each type of self disclosures. As some subreddits are associated with multiple self-disclosure types, we end up with 250 unique subreddits out of 550 extracted subreddits. 
We then manually review the name of each subreddit and annotate whether it is associated with one of our self-disclosure types.
Some examples are shown in Table~\ref{tab:sd_association} in Appendix~\ref{app:sd_association}. 



\pb{Regression Task.} We next ask \one what is the effect of a user receiving an interaction from a users in a self-disclosure related subreddit compared to not having received an interaction? 
and \two If a user receives an interaction, what is the impact on the number of such interactions? We model this as a regression task to predict whether the user will have a self-disclosure in future: 

\[ Y_{st} = \alpha + \beta NI_{t-1} + U + T \]
%
$Y_{st}$ is the number of self-disclosures by a user in a period t (1 week). $\beta$ shows the coefficient for number of interactions($N_{t-1}$) at time $t-1$ with the users from selected subreddit communities (SD-specific or general subreddits). $I = [0,1]$ whether a user received an interaction $I = 1$ or not $I = 0$. 
Note, we consider interaction to be any action initiated by users from a selected community. Thus, interaction occurs when a user from a selected subreddit writes a comment (at any level of the post) on a post by our selected users.
Finally, $U$ represent all users, and $T$ represents time (in weekly brackets) to control any user and time-dependent fixed-effect. In total, we run 11 regression models. Each model is specific to the self-disclosure-specific subreddits labeled with that type of disclosure. Note, we consider any self-disclosure as a positive instance of self-disclosure and do not differentiate within different types. 




\pb{Results.} Following these steps, we run our fixed effect regression model while controlling heteroskedasticity~\cite{gujarati_basic_2009}. Figure~\ref{fig:regression_analysis_m1} plots the regression model results. The figure consists of two panels, one for each of the independent variables.
The lower panel shows the binary impact of the user receiving an interaction or not, whereas the upper panel shows the impact of the number of interactions. The x-axis shows the $\beta$ estimates for variables, and the y-axis refers to the regression model depending on the self-disclosure for which the specific subreddits are used. The blue color shows the results that are statistically significant (with $p-values$ being lower than $0.05$). The gray color shows the corresponding estimates are statistically not significant. The missing values are also statistically not significant and the values are less than zero, so they are not shown due to the figure's x-axis scale.  

Confirming our hypothesis, we observe a positive effect ($\beta$ estimates) for future self-disclosure, \ie users are more likely to share a self-disclosing posts if they receive an interaction from a user who has previously posted in self-disclosure-specific subreddits.
Distinct effect sizes for the \textit{Interacted} variable are observed within each model. 
The most significant effect is observed for \job, with the odds of future self-disclosure being 0.34x, followed by \health at 0.26x, and \relationship at 0.24x. Similarly, the \textit{\#interactions} shows the positive effect for each such interaction. 
Given these confounding factors, we hypothesize two scenarios where Reddit users could be exploited: \one A user may be influenced to self-disclose through repeated interactions with other users who engage in self-disclosure, potentially including bots; or \two The creation of subreddits designed to \textit{maliciously} foster a sense of community to increase the likelihood of users engaging in self-disclosure.

