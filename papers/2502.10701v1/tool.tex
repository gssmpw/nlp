To help users control their self-disclosure, we have developed a browser plugin, \textit{InsightWatcher}. The tool automatically scans for self-disclosure, in real time, within any text box that the user loads in their browser. It works across any webpage that contains a text box (including Twitter/X, Facebook, and WhatsApp Web). Whenever a self-disclosure is identified, a small non-invasive popup is raised, notifying the user of the information they are exposing if they proceed.
The plugin achieves this using our classifier. %(\S\ref{sec:classifciation}). 
It does not require any user login and does not record any inputs. 
Note, the tool also allows users to select any text on a web page, and request a list of self-disclosures within the text.
Figure~\ref{fig:sd_active_reminder} shows a screenshot of the active reminder, featuring a pop-up on the right side of the screen in real-time. In Figure~\ref{fig:sd_passive_reminder}, a user selects text from a webpage, and a pop-up displays the self-disclosures from it. The plugin will be open source and available for users install.\footnote{\url{https://github.com/ehsanulhaq1/InsightWatcher}}




\begin{figure}[t]
    \centering
    \begin{subfigure}{.48\textwidth}
        \centering
        \includegraphics[width=\textwidth]{figs/reddit_text_example.pdf}
        \caption{Active Reminder}%: A user is informed about the self-disclosures in the text, when the user stops writing. }
        \label{fig:sd_active_reminder}
    \quad
    \end{subfigure}
    \begin{subfigure}{.48\textwidth}
        \centering
        \includegraphics[width=\textwidth]{figs/selected_text_annotated.pdf}
        \caption{Passive Reminder}%: A user is reminded about the self-disclosure types in their  self-selected text.}
        \label{fig:sd_passive_reminder}
    \end{subfigure}
    \caption{Working Example of Browser Plugin}
    \label{fig:browser_plugin}
\end{figure}
