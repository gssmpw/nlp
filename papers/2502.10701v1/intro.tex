Online self-disclosure is sharing information about oneself with other users in online communities ~\cite{wang_modeling_2016,joinson_oxford_2007}. This may include information on health, gender, sexual orientation, or general opinions, \eg liking or disliking a person~\cite{lee_online_2023}. Reasons for disclosure are diverse, including the perceived sense of anonymity \cite{ma_anonymity_2016}, adherence to community norms \cite{chen_i_2024}, and to improve engagement with community members~\cite{chen_revisiting_2018}. There are many benefits of self-disclosure (such as support seeking~\cite{choudhury_mental_2014}). However, it can also raise privacy concerns. %, particularly among vulnerable populations. 
For example, self-disclosure can lead to serious risks related to mental health~\cite{wood_student_2014,tay_mental_2018,nobels_just_2023} %~\cite{abramova_understanding_2017,balani_detecting_2015,choudhury_mental_2014} 
and intimacy~\cite{lee_effects_2019}.
% ~\cite{laurenceau_intimacy_1998,liang_privacy_2017,kramer_mastering_2020}.
Indeed, there have been numerous cases where accidental self-disclosure has led to physical harms, \eg loss of employment~\cite{ott_reputation_2013},  harassment~\cite{lauckner_catfishing_2019}, and geo-tagging~\cite{harrigian_geocoding_2018}.
Consequently, we argue that it is vital to better understand how such privacy-invasive disclosures occur, and develop tools to mitigate such risks.

Although there have been prior works that study self-disclosure online, these are either specific to a particular type of self-disclosure (\eg gender~\cite{mejova_gender_2023}) or specific to a particular type of community, \eg support seeking subreddits~\cite{chen_i_2024} and mental health forums~\cite{choudhury_mental_2014}.
Hence, we know little about the true scale of self-disclosure across diverse disclosure categories and communities.
Key questions include:
\one~How often do users self-disclose, and what types of self-disclosure are made together? 
\two~How common are high-risk self-disclosures? 
\three~What is the effect of self-disclosure specific communities on other users' self-disclosure propensity? Answering such questions is critical for formulating better user support against associated privacy risks.
Yet, the subtle complexities of online self-disclosure raise a number of challenges that must be overcome to study this. 
Specifically, to date, we lack a methodical way of identifying and classifying forms of self-disclosure in online posts. This is further complicated by the fact that self-disclosures are not always atomic. For instance, a user may disclose more than one type of identifiable information within one post or across multiple (seemingly unrelated) posts. This is particularly common on platforms like Reddit, where users may participate in multiple subreddits~\cite{crowley_expressive_2014}. 
Take the following posts (paraphrased to retain anonymity) as an example: \textit{``I am a 20 years-old male who does not have a good relationship with my parents (mother is 59 and father is 60) and I am suffering from mental health issues. I have an appointment at ER tomorrow, but I have not told my parents.''} The same user in an earlier post wrote -- \textit{``I am 19 year old male, a 19 year old female friend of mine is interested in dating me because I am less masculine as compared to her previous boyfriends.''} 
Here, by combining these posts, a third party can study the user's age, gender, and health, alongside garnering information about the user's parents and potential partner, along with the reason why she wanted to date the user. 



With the above in-mind, this paper characterizes the multifaceted nature of self-disclosure on Reddit, across 11 distinct categories related to identity and sensitive information~\cite{dou_reducing_2024}. These categories encompass age, gender, religion, ethnicity, sexual orientation, and more. 
To study how these patterns vary across communities, we study a diverse pool of Reddit users from the top-10 (by number of users) subreddits. We then use the outcomes to design a tool that can alert users to (potentially unknown) disclosures in their social media posts. Our contributions are:

\begin{itemize}

    \item We design a novel classifier to detect the presence of 11 different types of self-disclosures in a piece of text. We make our classifier open-source to assist other researchers. (\S\ref{sec:sd_types_and_classification})
    
    \item We show that at least 50\% of users self-disclose in at least 10\% of their total posts. Moreover, 50\% of disclosing posts have more than one type of self-disclosure, revealing prior works miss significant volumes of information~\cite{zani_motivating_2022,masur_impact_2023}. We build on this to identify the social norms of self-disclosure by highlighting pairs more likely to co-occur in a post, such as \age and \gender, or \gender and \relationship. (\S\ref{sec:sd_types_co_occurance})
    
    


    \item We show that users' engagement varies significantly with the type of self-disclosure in the posts. For instance, \sexualOrientation gets 2.6x more comments than posts without any disclosure. Moreover, users who have received interactions from self-disclosure-specific community members (such as the LGBT subreddit) are more likely to disclose in the future than those who have not. (\S\ref{sec:engagement_comb})


    \item We embed our work in a browser tool that can automatically alert users to inadvertent online self-disclosure. (\S\ref{sec:tool})

   
\end{itemize}










