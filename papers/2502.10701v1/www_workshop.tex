%%
%% This is file `sample-sigplan.tex',
%% generated with the docstrip utility.
%%
%% The original source files were:
%%
%% samples.dtx  (with options: `sigplan')
%% 
%% IMPORTANT NOTICE:
%% 
%% For the copyright see the source file.
%% 
%% Any modified versions of this file must be renamed
%% with new filenames distinct from sample-sigplan.tex.
%% 
%% For distribution of the original source see the terms
%% for copying and modification in the file samples.dtx.
%% 
%% This generated file may be distributed as long as the
%% original source files, as listed above, are part of the
%% same distribution. (The sources need not necessarily be
%% in the same archive or directory.)
%%
%% Commands for TeXCount
%TC:macro \cite [option:text,text]
%TC:macro \citep [option:text,text]
%TC:macro \citet [option:text,text]
%TC:envir table 0 1
%TC:envir table* 0 1
%TC:envir tabular [ignore] word
%TC:envir displaymath 0 word
%TC:envir math 0 word
%TC:envir comment 0 0
%%
%%
%% The first command in your LaTeX source must be the \documentclass command.
% \documentclass{beamer}
\documentclass[sigconf]{acmart}

% \copyrightyear{2025}
% \acmYear{2025}
% \setcopyright{acmlicensed}
% \acmConference[WWW Companion '25] {Companion of the 16th ACM/SPEC International Conference on Performance Engineering}{April 28-May 2, 2025}{Sydney, NSW, Australia.}
% \acmBooktitle{Companion of the 16th ACM/SPEC International Conference on Performance Engineering (WWW Companion '25), April 28-May 2, 2025, Sydney, NSW, Australia}
% \acmISBN{979-8-4007-1331-6/25/04}
% \acmDOI{10.1145/XXXXXX.XXXXXX}


\copyrightyear{2025}
\acmYear{2025}
\setcopyright{acmlicensed}\acmConference[WWW Companion '25]{Companion Proceedings of the ACM Web Conference 2025}{April 28-May 2, 2025}{Sydney, NSW, Australia}
\acmBooktitle{Companion Proceedings of the ACM Web Conference 2025 (WWW Companion '25), April 28-May 2, 2025, Sydney, NSW, Australia}
\acmDOI{10.1145/3701716.3717740}
\acmISBN{979-8-4007-1331-6/2025/04}
% 1 Authors, replace the red X's with your assigned DOI string during the rightsreview eform process.
% 2 Your DOI link will become active when the proceedings appears in the DL.
% 3 Retain the DOI string between the curly braces for uploading your presentation video.

\settopmatter{printacmref=true}


% \settopmatter{printacmref=false} % Removes citation information below abstract
% \renewcommand\footnotetextcopyrightpermission[1]{} % removes footnote with conference information in first column
% \pagestyle{plain} % removes running headers

% \documentclass[sigconf,authordraft,review,anonymous]{acmart}
%% NOTE that a single column version is required for 
%% submission and peer review. This can be done by changing
%% the \doucmentclass[...]{acmart} in this template to 
%% \documentclass[manuscript,screen,review]{acmart}
%% 
%% To ensure 100% compatibility, please check the white list of
%% approved LaTeX packages to be used with the Master Article Template at
%% https://www.acm.org/publications/taps/whitelist-of-latex-packages 
%% before creating your document. The white list page provides 
%% information on how to submit additional LaTeX packages for 
%% review and adoption.
%% Fonts used in the template cannot be substituted; margin 
%% adjustments are not allowed.
%%
%% \BibTeX command to typeset BibTeX logo in the docs

%%
%% \BibTeX command to typeset BibTeX logo in the docs
\AtBeginDocument{%
  \providecommand\BibTeX{{%
    Bib\TeX}}}


%% Rights management information.  This information is sent to you
%% when you complete the rights form.  These commands have SAMPLE
%% values in them; it is your responsibility as an author to replace
%% the commands and values with those provided to you when you
%% complete the rights form.
% \setcopyright{acmlicensed}
% \copyrightyear{2018}
% \acmYear{2018}
% \acmDOI{XXXXXXX.XXXXXXX}

% % These commands are for a PROCEEDINGS abstract or paper.
% \acmConference[Conference acronym 'XX]{Make sure to enter the correct
%   conference title from your rights confirmation emai}{June 03--05,
%   2018}{Woodstock, NY}
% %
% %  Uncomment \acmBooktitle if th title of the proceedings is different
% %  from ``Proceedings of ...''!
% %
% % \acmBooktitle{Woodstock '18: ACM Symposium on Neural Gaze Detection,
% %  June 03--05, 2018, Woodstock, NY} 
% \acmISBN{978-1-4503-XXXX-X/18/06}



% ======================================
% user packages
\usepackage{xspace}
% \usepackage{xcolor}
\usepackage{booktabs}
\usepackage{todonotes}
\usepackage{tabularx}
\usepackage{subcaption}
\usepackage{colortbl} 
\usepackage{multirow}
\usepackage{color,soul}
\usepackage[normalem]{ulem}

\usepackage{multicol}%set multicolumn features in page 
\usepackage{float}%Places the float at precisely the location in the LaTeX code


% ======================================
% user commands


\usepackage{tabularray}
\usepackage{float}
\usepackage{graphicx}
\usepackage{codehigh}
\usepackage[normalem]{ulem}
\UseTblrLibrary{booktabs}
\UseTblrLibrary{siunitx}
\newcommand{\tinytableTabularrayUnderline}[1]{\underline{#1}}
\newcommand{\tinytableTabularrayStrikeout}[1]{\sout{#1}}
\NewTableCommand{\tinytableDefineColor}[3]{\definecolor{#1}{#2}{#3}}



\newcommand{\one}{({\em i}\/)\xspace}
\newcommand{\two}{({\em ii}\/)\xspace}
\newcommand{\three}{({\em iii}\/)\xspace}
\newcommand{\four}{({\em iv}\/)\xspace}
\newcommand{\five}{({\em v}\/)\xspace}
\newcommand{\six}{({\em vi}\/)\xspace}
\newcommand{\seven}{({\em vii}\/)\xspace}
\newcommand{\eight}{({\em viii}\/)\xspace}
\newcommand{\nine}{({\em ix}\/)\xspace}


\newcommand{\age}{{\textit{Age}}\xspace}
\newcommand{\gender}{{\textit{Gender}}\xspace}
\newcommand{\health}{{\textit{Health}}\xspace}
\newcommand{\ethnicity}{{\textit{Ethnicity}}\xspace}
\newcommand{\sexualOrientation}{{\textit{Sexual Orientation}}\xspace}
\newcommand{\religion}{{\textit{Religion}}\xspace}
\newcommand{\job}{{\textit{Job}}\xspace}
\newcommand{\education}{{\textit{Education}}\xspace}
\newcommand{\location}{{\textit{Location}}\xspace}
\newcommand{\physical}{{\textit{Physical Appearance}}\xspace}
\newcommand{\relationship}{{\textit{Relationship}}\xspace}











\def\eg{\emph{e.g.}\xspace}
\def\etc{\emph{etc.}\xspace}
\def\ie{\emph{i.e. }\xspace}
\def\etal{\emph{et al.}\xspace}
\def\vs{\emph{vs.}\xspace}
\def\cf{\emph{cf.}\xspace}
\newcommand{\pb}[1]{\vspace{0.75ex}\noindent{\bf \em #1}\hspace*{.3em}}
\newcommand{\pbb}[1]{\vspace{0.75ex}\noindent{\bf #1}\hspace*{.3em}}

\newcommand\gareth[1]{\textbf{\textcolor{red}{GT: #1}}	}
\newcommand\ns[1]{\textbf{\textcolor{magenta}{NS: #1}}	}

\newcommand\ehsan[1]{\textcolor{black}{#1}}	

% ======================================




%%
%% Submission ID.
%% Use this when submitting an article to a sponsored event. You'll
%% receive a unique submission ID from the organizers
%% of the event, and this ID should be used as the parameter to this command.
%%\acmSubmissionID{123-A56-BU3}

%%
%% For managing citations, it is recommended to use bibliography
%% files in BibTeX format.
%%
%% You can then either use BibTeX with the ACM-Reference-Format style,
%% or BibLaTeX with the acmnumeric or acmauthoryear sytles, that include
%% support for advanced citation of software artefact from the
%% biblatex-software package, also separately available on CTAN.
%%
%% Look at the sample-*-biblatex.tex files for templates showcasing
%% the biblatex styles.
%%

%%
%% The majority of ACM publications use numbered citations and
%% references.  The command \citestyle{authoryear} switches to the
%% "author year" style.
%%
%% If you are preparing content for an event
%% sponsored by ACM SIGGRAPH, you must use the "author year" style of
%% citations and references.
%% Uncommenting
%% the next command will enable that style.
%%\citestyle{acmauthoryear}

%%
%% end of the preamble, start of the body of the document source.
\begin{document}

%%
%% The "title" command has an optional parameter,
%% allowing the author to define a "short title" to be used in page headers.
% \title{Self-Disclosure Classification and Measurements on Reddit}

\title[Exploring Self-Disclosure Norms, Engagement Dynamics, and Privacy Implications]{Unpacking the Layers: Exploring Self-Disclosure Norms, Engagement Dynamics, and Privacy Implications}

%Unpacking the Layers:





 \author{Ehsan-Ul Haq}
 \email{euhaq@hkust-gz.edu.cn}
 \orcid{1234-5678-9012}
\affiliation{%
 \institution{Hong Kong University of Science and Technology (GZ)}
 % \state{Arunachal Pradesh}
 \country{China}
 }

  \author{Shalini Jangra}
  \email{ s.jangra@surrey.ac.uk}
 \orcid{0000-0002-0265-6770}
\affiliation{%
 \institution{University of Surrey}
 % \streetaddress{Rono-Hills}
 % \state{Arunachal Pradesh}
 \country{UK}
 }

\author{Suparna De}
  \email{ s.de@surrey.ac.uk}
 \orcid{0000-0001-7439-6077}
\affiliation{%
 \institution{University of Surrey}
 % \streetaddress{Rono-Hills}
 % \state{Arunachal Pradesh}
 \country{UK}
 }


   \author{Nishanth Sastry}
  \email{ n.sastry@surrey.ac.uk}
 \orcid{0000-0002-4053-0386}
\affiliation{%
 \institution{University of Surrey}
 % \streetaddress{Rono-Hills}
 % \state{Arunachal Pradesh}
 \country{UK}
 }

 \author{Gareth Tyson}
  \email{gtyson@ust.hk}
 \orcid{0000-0003-3010-791X}
\affiliation{%
 \institution{Hong Kong University of Science and Technology (GZ)}
 % \streetaddress{Rono-Hills}
 % \state{Arunachal Pradesh}
 \country{China}
 }


%%
%% By default, the full list of authors will be used in the page
%% headers. Often, this list is too long, and will overlap
%% other information printed in the page headers. This command allows
%% the author to define a more concise list
%% of authors' names for this purpose.
\renewcommand{\shortauthors}{Haq et. al.}

%%
%% The abstract is a short summary of the work to be presented in the
%% article.
\begin{abstract}
This paper characterizes the self-disclosure behavior of Reddit users across 11 different types of self-disclosure. We find that at least half of the users share some type of disclosure in at least 10\% of their posts, with half of these posts having more than one type of disclosure.
We show that different types of self-disclosure are likely to receive varying levels of engagement. For instance, a \sexualOrientation disclosure garners more comments than other self-disclosures. We also explore confounding factors that affect future self-disclosure. We show that users who receive interactions from (self-disclosure) specific subreddit members are more likely to disclose in the future. We also show that privacy risks due to self-disclosure extend beyond Reddit users themselves to include their close contacts, such as family and friends, as their information is also revealed. We develop a browser plugin for end-users to flag self-disclosure in their content. 
\end{abstract}




\begin{CCSXML}
<ccs2012>
   <concept>
       <concept_id>10003456.10010927</concept_id>
       <concept_desc>Social and professional topics~User characteristics</concept_desc>
       <concept_significance>500</concept_significance>
       </concept>
   <concept>
       <concept_id>10002951.10003260.10003282.10003286</concept_id>
       <concept_desc>Information systems~Internet communications tools</concept_desc>
       <concept_significance>300</concept_significance>
       </concept>
 </ccs2012>
\end{CCSXML}

\ccsdesc[500]{Social and professional topics~User characteristics}
\ccsdesc[300]{Information systems~Internet communications tools}
%%
%% The code below is generated by the tool at http://dl.acm.org/ccs.cfm.
%% Please copy and paste the code instead of the example below.
%%
% \begin{CCSXML}
% <ccs2012>
%  <concept>
%   <concept_id>00000000.0000000.0000000</concept_id>
%   <concept_desc>Do Not Use This Code, Generate the Correct Terms for Your Paper</concept_desc>
%   <concept_significance>500</concept_significance>
%  </concept>
%  <concept>
%   <concept_id>00000000.00000000.00000000</concept_id>
%   <concept_desc>Do Not Use This Code, Generate the Correct Terms for Your Paper</concept_desc>
%   <concept_significance>300</concept_significance>
%  </concept>
%  <concept>
%   <concept_id>00000000.00000000.00000000</concept_id>
%   <concept_desc>Do Not Use This Code, Generate the Correct Terms for Your Paper</concept_desc>
%   <concept_significance>100</concept_significance>
%  </concept>
%  <concept>
%   <concept_id>00000000.00000000.00000000</concept_id>
%   <concept_desc>Do Not Use This Code, Generate the Correct Terms for Your Paper</concept_desc>
%   <concept_significance>100</concept_significance>
%  </concept>
% </ccs2012>
% \end{CCSXML}

% \ccsdesc[500]{Do Not Use This Code~Generate the Correct Terms for Your Paper}
% \ccsdesc[300]{Do Not Use This Code~Generate the Correct Terms for Your Paper}
% \ccsdesc{Do Not Use This Code~Generate the Correct Terms for Your Paper}
% \ccsdesc[100]{Do Not Use This Code~Generate the Correct Terms for Your Paper}

%%
%% Keywords. The author(s) should pick words that accurately describe
%% the work being presented. Separate the keywords with commas.
% \keywords{Self-disclosure, Identity, Engagement, Norms, Privacy, Reddit}



% \received{20 February 2007}
% \received[revised]{12 March 2009}
% \received[accepted]{5 June 2009}

%%
%% This command processes the author and affiliation and title
%% information and builds the first part of the formatted document.
\maketitle

\section{Introduction} \label{sec:introduction}

\section{Introduction}


\begin{figure}[t]
\centering
\includegraphics[width=0.6\columnwidth]{figures/evaluation_desiderata_V5.pdf}
\vspace{-0.5cm}
\caption{\systemName is a platform for conducting realistic evaluations of code LLMs, collecting human preferences of coding models with real users, real tasks, and in realistic environments, aimed at addressing the limitations of existing evaluations.
}
\label{fig:motivation}
\end{figure}

\begin{figure*}[t]
\centering
\includegraphics[width=\textwidth]{figures/system_design_v2.png}
\caption{We introduce \systemName, a VSCode extension to collect human preferences of code directly in a developer's IDE. \systemName enables developers to use code completions from various models. The system comprises a) the interface in the user's IDE which presents paired completions to users (left), b) a sampling strategy that picks model pairs to reduce latency (right, top), and c) a prompting scheme that allows diverse LLMs to perform code completions with high fidelity.
Users can select between the top completion (green box) using \texttt{tab} or the bottom completion (blue box) using \texttt{shift+tab}.}
\label{fig:overview}
\end{figure*}

As model capabilities improve, large language models (LLMs) are increasingly integrated into user environments and workflows.
For example, software developers code with AI in integrated developer environments (IDEs)~\citep{peng2023impact}, doctors rely on notes generated through ambient listening~\citep{oberst2024science}, and lawyers consider case evidence identified by electronic discovery systems~\citep{yang2024beyond}.
Increasing deployment of models in productivity tools demands evaluation that more closely reflects real-world circumstances~\citep{hutchinson2022evaluation, saxon2024benchmarks, kapoor2024ai}.
While newer benchmarks and live platforms incorporate human feedback to capture real-world usage, they almost exclusively focus on evaluating LLMs in chat conversations~\citep{zheng2023judging,dubois2023alpacafarm,chiang2024chatbot, kirk2024the}.
Model evaluation must move beyond chat-based interactions and into specialized user environments.



 

In this work, we focus on evaluating LLM-based coding assistants. 
Despite the popularity of these tools---millions of developers use Github Copilot~\citep{Copilot}---existing
evaluations of the coding capabilities of new models exhibit multiple limitations (Figure~\ref{fig:motivation}, bottom).
Traditional ML benchmarks evaluate LLM capabilities by measuring how well a model can complete static, interview-style coding tasks~\citep{chen2021evaluating,austin2021program,jain2024livecodebench, white2024livebench} and lack \emph{real users}. 
User studies recruit real users to evaluate the effectiveness of LLMs as coding assistants, but are often limited to simple programming tasks as opposed to \emph{real tasks}~\citep{vaithilingam2022expectation,ross2023programmer, mozannar2024realhumaneval}.
Recent efforts to collect human feedback such as Chatbot Arena~\citep{chiang2024chatbot} are still removed from a \emph{realistic environment}, resulting in users and data that deviate from typical software development processes.
We introduce \systemName to address these limitations (Figure~\ref{fig:motivation}, top), and we describe our three main contributions below.


\textbf{We deploy \systemName in-the-wild to collect human preferences on code.} 
\systemName is a Visual Studio Code extension, collecting preferences directly in a developer's IDE within their actual workflow (Figure~\ref{fig:overview}).
\systemName provides developers with code completions, akin to the type of support provided by Github Copilot~\citep{Copilot}. 
Over the past 3 months, \systemName has served over~\completions suggestions from 10 state-of-the-art LLMs, 
gathering \sampleCount~votes from \userCount~users.
To collect user preferences,
\systemName presents a novel interface that shows users paired code completions from two different LLMs, which are determined based on a sampling strategy that aims to 
mitigate latency while preserving coverage across model comparisons.
Additionally, we devise a prompting scheme that allows a diverse set of models to perform code completions with high fidelity.
See Section~\ref{sec:system} and Section~\ref{sec:deployment} for details about system design and deployment respectively.



\textbf{We construct a leaderboard of user preferences and find notable differences from existing static benchmarks and human preference leaderboards.}
In general, we observe that smaller models seem to overperform in static benchmarks compared to our leaderboard, while performance among larger models is mixed (Section~\ref{sec:leaderboard_calculation}).
We attribute these differences to the fact that \systemName is exposed to users and tasks that differ drastically from code evaluations in the past. 
Our data spans 103 programming languages and 24 natural languages as well as a variety of real-world applications and code structures, while static benchmarks tend to focus on a specific programming and natural language and task (e.g. coding competition problems).
Additionally, while all of \systemName interactions contain code contexts and the majority involve infilling tasks, a much smaller fraction of Chatbot Arena's coding tasks contain code context, with infilling tasks appearing even more rarely. 
We analyze our data in depth in Section~\ref{subsec:comparison}.



\textbf{We derive new insights into user preferences of code by analyzing \systemName's diverse and distinct data distribution.}
We compare user preferences across different stratifications of input data (e.g., common versus rare languages) and observe which affect observed preferences most (Section~\ref{sec:analysis}).
For example, while user preferences stay relatively consistent across various programming languages, they differ drastically between different task categories (e.g. frontend/backend versus algorithm design).
We also observe variations in user preference due to different features related to code structure 
(e.g., context length and completion patterns).
We open-source \systemName and release a curated subset of code contexts.
Altogether, our results highlight the necessity of model evaluation in realistic and domain-specific settings.








%%%%%%%%%%%%%%%%%%%%%%%%%%%%%%%%%%%%%%%%%%%
\section{Background}
\label{sec:related_work}
%%%%%%%%%%%%%%%%%%%%%%%%%%%%%%%%%%%%%%%%%%

\pb{Self-Disclosure Types and Detection.} Most of the prior literature on self-disclosure focuses on one type of self-disclosure or focused user groups and communities. For example, disclosure about mental health~\cite{choudhury_mental_2014,balani_detecting_2015} and subreddits related to health support~\cite{balani_detecting_2015}, empathy and intimacy~\cite{reuel_measuring_2022}. Some studies have looked at more than one type of self-disclosure. However, such studies are based on the discourse related to specific events such as the COVID-19 pandemic~\cite{blose_privacy_2020}. Qualitative methods such as manual coding~\cite{chen_i_2024}, surveys, and interviews~\cite{zani_motivating_2022} have been used to detect self-disclosure. 
Other researchers have used quantitative methods, consisting of supervised learning~\cite{balani_detecting_2015,wang_modeling_2016}, unsupervised learning~\cite{blose_privacy_2020}, and large language models (LLMs)~\cite{dou_reducing_2024} for self-disclosure detection.
A key contrast between our work and the above is that we focus on identifying multiple types of self-disclosure. 



\pb{Self-Disclosure Characterization} 
Online self-disclosure characterization is a multidimensional area~\cite{zillich_norms_2019,bertaglia_influencer_2024}.
Several studies focus on self-disclosure as means of users' support~\cite{choudhury_mental_2014,lee_designing_2020,gaur_let_2018}, other have explored linguistic characteristics within self-disclosure~\cite{gaur_let_2018}, and its impact on privacy~\cite{dinev_privacy_2006,yun_chronological_2019,bazarova_self-disclosure_2014}. One particular direction focuses on investigating social and communication norms among users, which can lead to nuances in self-disclosure behavior~\cite{gilroy_digital_2021,dietz-uhler_formation_2005}. For example, establishment of self-disclosure norms in mental health discussions as part of communication reinforces self-disclosure~\cite{dietz-uhler_formation_2005}. However, differences can be observed within demographics; for example, younger people are more open about their sexuality~\cite{gilroy_digital_2021}. Users' gender may play an important role in their disclosure and reaction to the disclosure of others within the blogging community~\cite{jang_non-directed_2011}.


Our work is distinct in that it focuses on characterizing \emph{multiple} types of self-disclosure of a \emph{general} set of users, spanning various communities. Thus, our insights generalize to a wide population of Reddit users, and captures a general scale of disclosure. 



\pb{Self-Disclosure and Privacy.} Another key line of work focuses on building tools for end-users to help control their disclosure~\cite{dou_reducing_2024,guarino_automatic_2022}.
A recent study proposed a task to help users rewrite the disclosure in their social media posts~\cite{dou_reducing_2024}. 
Similarly, Guarino et al. developed a web browser extension to help users control their disclosure based on keywords. However, it does not cover six of our identified disclosure types and relies on multiple classifiers, increasing the computational cost~\cite{guarino_automatic_2022}.
Our work provides data-driven insights and a tool to improve user privacy. Our tool covers 11 types of self-disclosure and does not require maintaining users' profiles. Additionally, our co-occurrence analysis offers insights into self-disclosures likely to occur together, which is useful in downstream research to increase the efficiency of privacy-preserving methods. 


%%%%%%%%%%%%%%%%%%%%%%%%%%%%%%%%%%%%%%%%%%%
\section{Data Collection Methodology} 
\label{sec:dataset}
%%%%%%%%%%%%%%%%%%%%%%%%%%%%%%%%%%%%%%%%%%%

%\pb{Data Collection.}
Our data collection aims to solicit \emph{all} posts from a Reddit user in a given time window. This ensures a holistic view of self-disclosure by a given user. \ehsan{Thus, we use the pushshift data dump of all Reddit from October 2020 to June 2021~\cite{haq_understanding_2023}, making it possible to track a user's all interactions within this time period.} We first gather a seed list of users to start our data collection. For this, we extract all users who have written a post in any of the top 10 largest subreddits\footnote{The subreddits are \textit{funny, AskReddit, gaming, aww, worldnews, todayilearned, Music, movies, science, pics}} (by community size) during two months (Jan. and Feb. 2021). We call these \emph{general users}.
Note that posts collected from these general subreddits are more likely to contain general discourse than topic-specific discourse like health and intimacy~\cite{choudhury_mental_2014,balani_detecting_2015}. 

We then gather all their posts for these general users, from October 2020 to June 2021. The raw dataset contains 16,706,119 posts from 365,385 users. We remove accounts containing any single or combination of the words \textit{`bot', `moderat', or `auto'} in their usernames to filter bot and moderator accounts~\cite{zhu2024study}. This removed 1,103 accounts and 428,275 posts.








\section{Self-Disclosure Classification} \label{sec:sd_types_and_classification}


To analyze self-disclosure at scale, it is first necessary to devise a methodology that can automatically identify the presence of disclosure in user posts. Thus, we first design a classifier.



\subsection{Defining Self-Disclosure Types}

We start by defining a taxonomy of self-disclosure types to form our supervised label set. Given our focus on privacy, we focus on self-disclosures that may reveal one's identity and information that may be used in a potential prejudice, such as health, age, gender, and sexual orientation~\cite{lee_designing_2020,wang_modeling_2016}. A recent study proposed a list of 19 self-disclosure types related to demographics and personal experiences~\cite{dou_reducing_2024}. We take inspiration from these 19 categories and a review of different self-disclosure types studied in prior literature~\cite{zani_motivating_2022}. Based on our analysis, in Table~\ref{tab:all_sd_types} (Appendix~\ref{app:sd_type_table}), we present the high-level categories of self-disclosure. These cover various aspects of one's life, including identity, relationship, work, health, group affiliations, and opinions.
For simplicity, we do not consider the group affiliations and opinion categories. This is because these categories include far more subjective information. Furthermore, it is challenging to judge if this subjective information carries sensitive insight.

Note that we consider relationship, profession/economics, and health as broader categories in this study. For instance, we consider health to be one type of self-disclosure, as our research questions do not strive to differentiate between mental and physical health at a granular scale. Similarly, finance, profession, and job-related self-disclosure are grouped as ``Job.'' These considerations help us improve our classifier's performance and answer our research questions more effectively.
Our final list of self-disclosure types is: \textbf{Age, Education, Ethnicity, Gender, Health, Job, Location, Physical Appearance, Relationship, Religion, and Sexual Orientation.} We acknowledge that our selected self-disclosure types do not form an exhaustive list. However, this list matches commonly studied self-disclosure types~\cite{zani_motivating_2022} and covers the GDPR definition of personal data, except affiliations, opinions, and genetic data~\cite{noauthor_what_2023}.











\subsection{Self-Disclosure Training Data Annotation} \label{sec:classifciation}

It is next necessary to label a training dataset, for which we take a two-step approach. To optimize the training sample, we first use ChatGPT to extract more relevant posts for manual inspection. Then, two researchers label the posts to create the training set. 



\pb{Identification of Self-Disclosure.} 
There are two commonly used approaches for self-disclosure identification: \one A binary coding (\textit{yes, no}) representing the presence or absence of self-disclosure~\cite{kou_what_2018};
and
\two Identifying self-disclosure as an ordinal variable that quantifies the sensitivity of the information provided by a user~\cite{wang_modeling_2016}. Usually, this is in the form of \textit{low, medium, and high}. There is, however, no fixed criteria for using either of the approaches.
We follow the first approach and use a binary variable to indicate the presence or absence of a particular type of self-disclosure. This is because we are interested in the overall presence of self-disclosure instead of a fine-grained analysis within each self-disclosure type. We also note that an ordinal approach will require a more extensive data annotation exercise and will likely result in reduced performance.

\pb{Data Annotation.} We use a two-step data annotation exercise, employing ChatGPT (3.5) followed by manual annotation. The use of ChatGPT is to improve the sample selection, as only the samples with positive outcomes from ChatGPT are subsequently manually analyzed. We randomly select 2000 samples from the dataset for ChatGPT annotation. We then use the standard ChatGPT guidelines on the chain of thought~\cite{wei_chain--thought_2022} along with a role assignment~\cite{imran_analyzing_2023} to design our prompt (prompt in Appendix~\ref{app:chatgpt_annotation}). The GPT annotations are performed in a single session, and 1,764 posts are selected as containing at least one type of disclosure. 
Two researchers then perform manual validation of these annotations. After a briefing session, the two researchers are assigned a pool of 110 posts (taken from the positive ChatGPT annotations), consisting of 10 posts randomly selected for each of the 11 self-disclosure types. 
The results from the two researchers are compared and disagreements are discussed. Inter-rater reliability (IRR) is calculated, and the Cohen's Kappa score is in the range of 0.51 to 0.80, (median 0.70 and mean 0.70) for all labels. In addition, 19 posts are identified to be completely incorrectly classified as self-disclosing by ChatGPT, with another 30\% posts missing at least one label identified in manual inspection (where both raters have agreements). After the disagreements are discussed between the researchers, another round of manual annotations is followed on a sample of 250 posts, followed by the IRR measurements for round 2. Here, the Cohen's Kappa score for each self-disclosure type is in the range of 0.79 to 0.94 (median 0.85, and mean 0.82); showing a high agreement between the two annotators~\cite{hadi_mogavi_student_2021}. After the second debriefing round, following prior work~\cite{mcdonald_reliability_2019}, one researcher continues with the rest of the manual annotations. In total, 1,329 posts are labeled as self-disclosing. One post can have more than one type of self-disclosure. We use this final manually annotated dataset for training the classifier.  









\subsection{Classifier Training}
Fine-tuning of pre-trained large models has shown positive results in downstream research tasks like ours. 
Thus, we use the labeled data to fine-tune pre-trained BERT and Roberta models~\cite{turc_well-read_2019,liu_roberta_2019}. We define this as a multi-label classification task --- the input is a post, and the output is the probability of each type of self-disclosure it contains (if any). We stratify the labeled data into 70-15-15\% for training, testing, and validation.

Roberta performs better than BERT, with an average weighted F1 of 0.83. Table~\ref{tab:classification_details} (Appendix~\ref{sec:classifier_performance}) reports the F1 scores within each self-disclosure type, along with precision and recall scores. We achieve the best F1 scores for Ethnicity (0.95), Religion (0.94), and Age (0.89). All self-disclosure except \gender and \job have >0.8 F1 score. Note that we consider a positive class if the sigmoid function has a value of 0.5 or higher. Any change in this limit can affect the data size; however, 0.5 gives a moderate approach and is commonly used in classification tasks. We use this model to predict the self-disclosure labels of every post in our dataset. We predict labels for the text and title of posts separately due to difference in the length and writing styles. For each post, we then take the union of the self-disclosure labels identified in both the text and title. Our classifier is open source for use by the community.\footnote{\url{https://huggingface.co/euhaq/self_disclosure}}


\begin{figure}[t]
    \centering
    \begin{subfigure}[t]{.2\textwidth}
        \centering
        \includegraphics[width=.85\textwidth]{figs/rq_1/sd_ratio_user_subreddit.png} %sd_ratio_per_user
        \caption{Ratio of self disclosure posts across all posts}
        \label{fig:ratio_sd_posts}
    \end{subfigure}
    \quad
    \begin{subfigure}[t]{.2\textwidth}
        \centering
        \includegraphics[width=.85\textwidth]{figs/rq_1/unique_sd_per_user_reddit_across_dataset.png}
        \caption{Unique self-disclosures per user and per subreddit}
        \label{fig:number_unique_sd_users}
    \end{subfigure}
    \quad
    \begin{subfigure}[t]{.23\textwidth}
        \centering
        \includegraphics[width=.8\textwidth]{figs/rq_1/sd_per_user.png}
        \caption{Unique self-disclosure types per post and per user}
        \label{fig:sd_per_post_user}
    \end{subfigure}    
    \label{fig:rq_1}
    \caption{Cumulative distributions of self-disclosure.\vspace{-10pt}}
\end{figure}






\section{Quantifying Self-Disclosure} \label{sec:sd_types_co_occurance}

Our core goal is to understand the self-disclosure norms of Reddit users.
Thus, exploiting our labels, we now measure the scale of self-disclosures to understand what sort of (privacy-sensitive) information is disclosed.




\subsection{Quantifying the Scale of Self-Disclosure} 
%evaluate the count 
First, we evaluate the count self-disclosures across all users and posts, to quantify potential privacy-compromising exposure.


\pb{Number of Self-Disclosures Per User.}
For each user, we inspect the ratio of posts that contain at least one type of self-disclosure vs.\ the total number of posts from that user. Figure~\ref{fig:ratio_sd_posts} shows the distribution of this ratio for all users in our dataset. The mean of the ratio is 0.15 (median 0.10, 75th percentile 0.23). 
Half of the users in our dataset self-disclose in at least 10\% of their posts, and 25\% of users have self-disclosure in at least 23\% of their posts. This confirms that a significant number of users are involved in self-disclosure.
We emphasize that our dataset is based on users from general purpose subreddits, and we do not focus on specific subreddits that are more likely to have self-disclosure. Thus, this is surprisingly high. 

\pb{Number of Self-Disclosure Types Per User.} 
Whereas the above may raise privacy concerns, it is less problematic if a user discloses the same information repeatedly (rather than many instances of new information). Thus, we inspect the number of \emph{unique} self-disclosure types per user.
For example, if a user shares \age in four posts and \gender in two posts, this user still only has two unique self-disclosure types (\age and \gender). 

Figure~\ref{fig:number_unique_sd_users} plots the number of unique self-disclosure types per user. We see that users share an average of four ($\sigma =  2$) different types of self-disclosure across their timelines. In fact, 50\% of users have at least five types of self-disclosures in their timelines. Thus, it is clear that most users exhibit a diverse range of disclosures. 
In many cases, we find that these multiple disclosures are embedded within individual posts.
To quantify this, Figure~\ref{fig:sd_per_post_user} (orange line) plots the distribution of the number of unique disclosures per post, and the blue line shows the distribution of the per-user average of this measure.
We find that, on average, posts contain close to 2 ($\mu =1.8$, $\sigma = 2$) self-disclosure types.
This confirms that users expose substantial and diverse information. We posit that it is possible for third parties to easily combine such information. This naturally increases users' susceptibility to malicious actors in online spaces. 




\begin{figure}[t]
    \centering
    % \begin{subfigure}{.44\textwidth}
        \centering
        \includegraphics[width=.35\textwidth]{figs/co_occurance_correlation_in_all_posts_users_specific.png}
        \caption{Correlation between different self-disclosure types that occur commonly per user. Most users who share Age also share Gender and Relationship-related self-disclosures.}
        \label{fig:sd_types_correlation_user}
\end{figure}











%Each column represents a model; each row estimates confounding factors for the self-disclosure type in the header. 


\begin{table*}[t]
\tiny
\caption{The likelihood of self-disclosures to co-occur in the same post. Each column and row represent model and confounding factors, respectively. $***$ indicates statistical significance. The top value displays the $\beta$ estimate and the bottom shows the standard error. Blue and Red colors indicate the most positive and negative estimates, respectively.}
\begin{tabular}{|l|l|l|l|l|l|l|l|l|l|l|l|}
\toprule
                              & \textbf{Age}                     & \textbf{Education}               & \textbf{Ethnicity}               & \textbf{Gender}                  & \textbf{Health}                  & \textbf{Job}                     & \textbf{Location}                & \textbf{Phy. Appearance}         & \textbf{Relationship}            & \textbf{Religion}                & \textbf{Sexual Orientation}      \\
\midrule
\textbf{age}                  &                                  & {\color{blue}  0.013***}  & -0.005***                        & {\color{blue}  0.271***}  & {\color{blue}  0.012***}  & {\color{blue}  0.006***}  & 0.002*                           & -0.020***                        & 0.119***                         & -0.008***                        & 0.002***                         \\
                              &                                  & 0.001                           & 0                                & 0.001                           & 0.001                           & 0.001                           & 0.001                           & 0.001                           & 0.001                           & 0.001                           & 0                                \\
\hline
\textbf{education}            & 0.017***                         &                                  & -0.014***                        & -0.053***                        & -0.168***                        & -0.136***                        & -0.112***                        & -0.057***                        & -0.103***                        & -0.045***                        & -0.023***                        \\
                              & 0.001                           &                                  & 0.001                           & 0.001                           & 0.001                           & 0.001                           & 0.001                           & 0.001                           & 0.001                           & 0.001                           & 0.001                           \\
                              
                              \hline

\textbf{ethnicity}            & {\color{red}  -0.019***} & -0.038***                        &                                  & 0.069***                         & -0.166***                        & -0.198***                        & {\color{blue}  0.009***}  & -0.005***                        & -0.087***                        & -0.029***                        & -0.003**                         \\
                              & 0.002                            & 0.002                            &                                  & 0.001                           & 0.002                            & 0.002                            & 0.002                            & 0.001                           & 0.002                            & 0.001                           & 0.001                           \\
                              
                              \hline
\textbf{gender}               & {\color{blue}  0.364***}  & -0.056***                        & {\color{blue}  0.025***}  &                                  & -0.134***                        & -0.141***                        & -0.069***                        & {\color{blue}  0.060***}  & {\color{blue}  0.141***}  & -0.023***                        & {\color{blue}  0.067***}  \\
                              & 0.001                           & 0.001                           & 0.001                           &                                  & 0.001                           & 0.001                           & 0.001                           & 0.001                           & 0.001                           & 0.001                           & 0.001                           \\
                              
                              \hline
\textbf{health}               & 0.007***                         &   -0.080*** & -0.028***                        & -0.061***                        &                                  & -0.307***                        & {\color{red}  -0.208***} & -0.002***                        & -0.147***                        & -0.063***                        & -0.036***                        \\
                              & 0.001                           & 0.001                           & 0                                & 0.001                           &                                  & 0.001                           & 0.001                           & 0.001                           & 0.001                           & 0                                & 0                                \\
                              
                              \hline
\textbf{job}                  & 0.003***                         & -0.063***                        & {\color{red}  -0.032***} & {\color{red}  -0.062***} & -0.294***                        &                                  & -0.188***                        & -0.075***                        & {\color{red}  -0.198***} & {\color{red}  -0.070***} & {\color{red}  -0.037***} \\
                              & 0.001                           & 0.001                           & 0                                & 0.001                           & 0.001                           &                                  & 0.001                           & 0.001                           & 0.001                           & 0                                & 0                                \\
                              
                              \hline
\textbf{location}             & 0.001*                           & -0.058***                        & 0.002***                         & -0.034***                        & -0.223***                        & -0.211***                        &                                  & -0.044***                        & -0.129***                        & -0.043***                        & -0.018***                        \\
                              & 0.001                           & 0.001                           & 0                                & 0.001                           & 0.001                           & 0.001                           &                                  & 0.001                           & 0.001                           & 0                                & 0                                \\
                              
                              \hline
\textbf{physical\_appearance} & -0.034***                        & -0.074***                        & -0.002***                        & 0.074***                         & -0.005***                        & -0.211***                        & -0.110***                        &                                  & -0.105***                        & -0.042***                        & -0.008***                        \\
                              & 0.001                           & 0.001                           & 0.001                           & 0.001                           & 0.001                           & 0.001                           & 0.001                           &                                  & 0.001                           & 0.001                           & 0.001                           \\
                              
                              \hline
\textbf{relationship}         & 0.069***                         & -0.047***                        & -0.014***                        & 0.062***                         & -0.140***                        & -0.197***                        & -0.115***                        & -0.037***                        &                                  & -0.027***                        & 0.002***                         \\
                              & 0.001                           & 0.001                           & 0                                & 0.001                           & 0.001                           & 0.001                           & 0.001                           & 0.001                           &                                  & 0                                & 0                                \\
                              
                              \hline
\textbf{religion}             & -0.024***                        & {\color{red}  -0.107***} & -0.024***                        & -0.051***                        & {\color{red}  -0.307***} & {\color{red}  -0.357***} & -0.198***                        & {\color{red}  -0.076***} & -0.138***                        &                                  & -0.032***                        \\
                              & 0.002                            & 0.001                           & 0.001                           & 0.001                           & 0.002                            & 0.002                            & 0.002                            & 0.001                           & 0.002                            &                                  & 0.001                           \\
                              
                              \hline
\textbf{sexual\_orientation}  & 0.008***                         & -0.065***                        & -0.003**                         & 0.179***                         & -0.210***                        & -0.227***                        & -0.095***                        & -0.017***                        & 0.012***                         & -0.038***                        &                                  \\
                              & 0.002                            & 0.001                           & 0.001                           & 0.001                           & 0.002                            & 0.002                            & 0.002                            & 0.001                           & 0.002                            & 0.001                           &    \\
                              \bottomrule
\end{tabular}
\label{tab:co_occur_regression}
\end{table*}




\begin{figure}[t]
    \centering
    \includegraphics[width=0.3\textwidth]{figs/frequency_sd_histogram.png}
    \caption{Number of users (top x-axis) and posts (bottom x-axis) for each self-disclosure type.\vspace{-9pt}}
    \label{fig:sd_frequency_overall}
\end{figure}










%%%%%%%%%%%%%%%%%%%%%%%%%%
\subsection{Quantifying Types of Self-Disclosure} \label{sec:sd_correlation}
%\subsection{Identifying Self-Disclosures Norms} \label{sec:sd_correlation} 
%%%%%%%%%%%%%%%%%%%%%%%%%%

The above has identified that a large number of self-disclosures are exposed by users. We proceed to study the specific categories of information disclosed and which are co-located within posts.


\pb{Frequency of Self-Disclosure Types.}
Figure~\ref{fig:sd_frequency_overall} %(Appendix~\ref{sec:users_and_posts_number}) 
presents a histogram of the number of posts and users who disclose each type of information. Health, location, and relationship are the most commonly shared. Location is particularly concerning, as this can expose users to physical risks. Equally, health information tends to be particularly sensitive. That said, support-seeking for users with medical issues is commonplace, and some may argue the benefits outweigh the risks in certain scenarios~\cite{silveira_fraga_online_2018,zou_self-disclosure_2024}.
Information about ethnicity, sexual orientation, and religion is the least shared. these disclosures can also be used to harm users, \eg sharing such information can be used for catfishing and cyberbullying~\cite{lauckner_catfishing_2019}. 


\pb{Co-occurrence of Self-Disclosures.}  
Next, we inspect which combinations of disclosure types are exposed by each user. 
We wish to understand if certain combinations of information disclosure are common (\eg revealing \emph{both} age and gender).
For this, we first compute the sum of all types of self-disclosure shared by each user. This generates a matrix of $n x 11$, where $n$ is the total number of users (recall, we cover 11 types of self-disclosure). 
We then calculate the Pearson correlation for each pair of self-disclosure types.
Figure~\ref{fig:sd_types_correlation_user} shows the correlations on a per-user level. Each cell in the figure shows the correlation between a pair of self-disclosure types --- a higher value indicates that the self-disclosure types from the pair have been shared together by more users. We do see certain pairs commonly occurring, suggesting that certain self-disclosure norms have emerged within Reddit.
Importantly, we emphasize that this disclosure behavior is not limited to support-seeking subreddits, but is instead a \textit{general} self-disclosure norm.



\pb{Modeling Self-Disclosure Relationships.}
To systematize the above analysis, we model the precise relationships using 11 different fixed effect regression models, as follows:

\begin{equation}
    s_j =  \sum_i^{11} \beta_is_i + U + T\quad  i,j \in [1,11], i\neq j
\end{equation}
where, $s_j$ is the dependent variable for the $j^{th}$ self-disclosure, and $B_i$ is the estimate (impact) of the remaining self-disclosure types. $U$ and $T$ are the fixed effects for users and time. We develop 11 regression models, one for each self-disclosure type, with the goal of identifying potential patterns of risky co-disclosure. 

Table~\ref{tab:co_occur_regression} summarizes the results, where
each column reports a regression model for a self-disclosure type mentioned in the column header. The rows of the table report the regression estimate of the remaining self-disclosure types. Additionally, each estimate is accompanied by the standard error along with p-values indicated by  $*$. For each model, we highlight the most positive estimate with blue color and the most negative one with red color. 

Confirming our previous results in Figure~\ref{fig:sd_types_correlation_user}, we observe both positive and negative correlations in self-disclosures, reflecting interesting trends.
For instance, if a post adds a self-disclosure about \gender the likelihood of \age disclosure increases by 0.364x. However, disclosure about \ethnicity will increase the likelihood of disclosing \age by just 0.02x. 
Similarly, mentioning \gender increases the likelihood of mentioning a \relationship by 0.14x, while \job reduces the likelihood of mentioning \relationship by -.198x. 
Overall, we find that \age and \gender have the largest effect sizes for most of the regression models. This implies that users most often disclose their age and gender together on Reddit. Such combinations may have benefits for support but also pose risks associated with mental health to certain demographics\eg youth during gender transitions~\cite{haimson_disclosure_2015}.
This combination is also visible in Figure~\ref{fig:sd_types_correlation_user}, which shows a 0.74 correlation between the presence of \age and \gender disclosure. Although this is arguably privacy-sensitive, it does reveal a common norm, whereby users are expected to express such information (\eg ``[30 F]'') as part of daily discussion~\cite{chen_i_2024}.

At the opposite end of the spectrum, \religion and \job generally have the least effect size for most regression models. This suggests these are least likely to co-occur with other self-disclosure types in the same post. For instance, the likelihood of \health disclosure is reduced by 0.357x with \religion.
\religion and \job generally have a negative impact on most of self-disclosure types to co-occur with them. For instance, the likelihood of \health disclosure is reduced by 0.357x with \religion. 
The positive correlation of \gender with \physical and \sexualOrientation shows that the latter two are accompanied mostly by the former. 
We argue that topics related to physical appearance, such as body shaming together with gender, may lead towards negative outcomes such as body shaming and psychological abuse~\cite{mcmahon_body_2022,corradini_dark_2023}, or hate speech concerning the sexuality of the users~\cite{crowley_expressive_2014,lingiardi_mapping_2020}.






%%%%%%%%%%%%%%%%%%%%%%%%%%%%%%%%%%%%%%%%%%%
\section{Self-Disclosure Engagement}
\label{sec:engagement_comb}

\begin{table}[t]
    \centering
    \small
    \begin{tabularx}{\linewidth}{Xrrr}
        \textbf{Feature} & \textbf{$\neg$Und.} & \multicolumn{1}{r}{\textbf{Und:$\neg$Und.}} & \multicolumn{1}{r}{\textbf{Net Und.}} \\         \midrule
        Link (vs. image)        & -3.440\%*          & +0.342\%             & -3.110\%     \\
        Self (vs. image)        & -2.179\%*          & +0.650\%             & -1.543\%     \\
        Video (vs. image)       & 0.924\%*           & +0.006\%             & 0.930\%      \\ \midrule
        Age                     & -13.140\%*         & -0.467\%\enspace     & -13.545\%    \\
        Num. Comments           & 22.331\%*          & -2.601\%*            & 19.149\%     \\
        Rec. Comments           & 61.740\%*          & +1.056\%*            & 63.449\%     \\
        Prop. Undesired         & -14.467\%*         & +115.633\%*          & 84.437\%     \\
        Prop. Rec. Und.         & -1.500\%\enspace   & +8.713\%*            & 7.082\%      \\
        Score                   & -13.992\%*         & +2.062\%*            & -12.218\%    \\
        Rec. Votes              & 13.530\%*          & -3.422\%*            & 9.645\%      \\
        Prop. Upvotes           & 3.519\%*           & -1.077\%*            &  2.405\%     \\
        Num. Subscribers        & 1.414\%*           & -0.344\%*            & 1.066\%      \\ \midrule
        Pseudo $R^2$            & 0.29 \enspace\enspace &                   &
    \end{tabularx}
    \caption{
        Results from the negative binomial regression model predicting the rate of non-undesired and undesired comments (*$p<0.05$, Bonferroni-adjusted). Percentages in ``$\neg$Und.'' denote expected change in non-undesired comments given a unit increase in the respective feature (see Table \ref{tab:descriptive}). Percentages in ``Und:$\neg$Und.'' denote expected change in the ratio of undesired to non-undesired comments gained. Net expected change to undesired comments is shown in the ``Net Und.'' column. For example, a post that is 2 times older than another is expected to have 13.1\% fewer non-undesired comments and a further 0.47\% fewer undesired comments, for a net of 13.5\% fewer undesired comments.
    }
    \label{tab:engagement}
\end{table}

%%%%%%%%%%%%%%%%%%%%%%%%%%%%%%%%%%%%%%%%%%%




\section{InsightWatcher - A Browser Plugin}\label{sec:tool}
To help users control their self-disclosure, we have developed a browser plugin, \textit{InsightWatcher}. The tool automatically scans for self-disclosure, in real time, within any text box that the user loads in their browser. It works across any webpage that contains a text box (including Twitter/X, Facebook, and WhatsApp Web). Whenever a self-disclosure is identified, a small non-invasive popup is raised, notifying the user of the information they are exposing if they proceed.
The plugin achieves this using our classifier. %(\S\ref{sec:classifciation}). 
It does not require any user login and does not record any inputs. 
Note, the tool also allows users to select any text on a web page, and request a list of self-disclosures within the text.
Figure~\ref{fig:sd_active_reminder} shows a screenshot of the active reminder, featuring a pop-up on the right side of the screen in real-time. In Figure~\ref{fig:sd_passive_reminder}, a user selects text from a webpage, and a pop-up displays the self-disclosures from it. The plugin will be open source and available for users install.\footnote{\url{https://github.com/ehsanulhaq1/InsightWatcher}}




\begin{figure}[t]
    \centering
    \begin{subfigure}{.48\textwidth}
        \centering
        \includegraphics[width=\textwidth]{figs/reddit_text_example.pdf}
        \caption{Active Reminder}%: A user is informed about the self-disclosures in the text, when the user stops writing. }
        \label{fig:sd_active_reminder}
    \quad
    \end{subfigure}
    \begin{subfigure}{.48\textwidth}
        \centering
        \includegraphics[width=\textwidth]{figs/selected_text_annotated.pdf}
        \caption{Passive Reminder}%: A user is reminded about the self-disclosure types in their  self-selected text.}
        \label{fig:sd_passive_reminder}
    \end{subfigure}
    \caption{Working Example of Browser Plugin}
    \label{fig:browser_plugin}
\end{figure}




\section{Discussion} \label{sec:discussion}

Our work presents the first large-scale and multi self-disclosure characterization on Reddit based on general discourse.


 
\pb{Privacy Control Measures.} Our work highlights that a large number of posts have self-disclosure; 50\% of users in our dataset have self-disclosure in at least 10\% of their posts. In addition, half of the users disclose more than one type of self-disclosure.  In addition to the empirical insights, there are other latent risks that lie in multiple disclosures from a user. For example, when users self-disclose, they often include information from their past and other people involved around that time. Thus, this can increase users' vulnerability to mal-actors who can combine such information to find more about a user. This behavior increases the need for tools, like our browser plugin InsightWatcher, that help users control their disclosure. \ehsan{However, Self-disclosure moderation should be context-specific, guided by ethical principles. In support-based communities, self-disclosure can enhance support, so strict moderation might lessen their positive impact. In contrast, disclosure in generic communities can be more harmful to users, thus requiring a nudge on self-disclosure.}



\pb{Co-located Self-Disclosure Types.}
Our work highlights the importance of studying multiple self-disclosures together. Users do this to add more information and context while describing their life events or sharing their thoughts. Our work highlights that pairs of self-disclosure types, such as \gender and \relationship, are likely to appear together. 
This has implications when studying such disclosures alone. The occurrence of one particular disclosure may be inspired by another, hence deviating from commonly associated characteristics with certain disclosures. The study of co-located self-disclosure can offer insights into several use-cases, such as uncovering the mechanism of identity establishment, particularly for vulnerable populations. For instance, studying \sexualOrientation together with \age and \gender and \relationship can uncover how different groups open up about their sexuality. 



\pb{Disclosure about Close Contacts.}% Our qualitative analysis \gareth{This has now been removed} shows that
We observe that users often disclose about people around themselves, including immediate family members, friends, and co-workers.  We find that over 88,709 posts mention at least one word from--- \textit{father, mother, sister, brother, boyfriend, girlfriend, bf, gf} in their main body. Thus, privacy risks can extend from one user to their extended social network. 
Reddit users often put such details and other characters to add more contextual information to refer to some event while sharing their experiences or seeking information. Thus, any tools designed must be flexible enough to accommodate these norms while minimizing privacy risk. We conjecture that dynamically rewriting such information, while not compromising the context could be valuable (\eg changing the specific age).



\section{Conclusion}

We have presented the first multiple-type self-disclosure characterization on Reddit. We identified 11 types of self-disclosure associated with a user's identity and demographics, such as age, gender, relationship, and job.  We first developed our open-source multi-label self-disclosure classifier for the 11 different self-disclosure types. Our characterization shows that at least half of users self-disclose in more than 10\% of their posts. We highlight that user posts are not limited to one particular type of self-disclosure. Instead, users share more than one type of information to add more context to posts. Through thematic analysis, we show that user self-disclosure reveals information about themselves and extends to their social connections, such as parents, siblings, and partners. Building on this, we have developed and deployed a browser plugin tool that can automatically notify users when they are self-disclosing. 

We note our study has several limitations, which form the basis of our future work. Most notably, our list of self-disclosure types is not exhaustive, albeit covering key identifiable information. \ehsan{Future work can extend disclosure types, such as disclosure through opinions (the types we have not included in our analysis) or increasing granularity as high, medium, and low.} \ehsan{Moreover, our work is limited to Reddit. However, a similar approach can be extended to other platforms; for example, self-disclosure-related Facebook groups or hashtag-specific discussions can be taken as a proxy of Reddit communities for Facebook and X, respectively.} Furthermore, our work is focused only on Reddit; similar studies are required on other platforms to analyze the generalization of our findings.
Finally, within our work, we consider the presence of self-disclosure but do not quantify whether the self-disclosure is high or low risk. Our future work will focus on better understanding the exact nature of the risks involved and how they can be mitigated.





% ####### ####### ####### ####### ####### 
% ####### ####### ####### ####### ####### 



\section{Acknowledgments}
This work was supported in part by the Guangzhou Science and Technology Bureau (2024A03J0684); the Guangzhou Municipal Science and Technology Project (2023A03J0011); the Guangzhou Municipal Key Laboratory on Future Networked Systems (024A03J0623), the Guangdong Provincial Key Lab of Integrated Communication, Sensing and Computation for Ubiquitous Internet of Things (No. 2023B1212010007); and by AP4L (EP/W032473/1).

%%
%% The next two lines define the bibliography style to be used, and
%% the bibliography file.
\bibliographystyle{ACM-Reference-Format}
\balance
\bibliography{ref}

% \balance







% %%
% %% If your work has an appendix, this is the place to put it.
\appendix


% \newpage



\section{Ethics Statement}
This research complies with the SAGE (Self-Assessment Governance and Ethics Form for Humans and Data Research) self-check process provided by the University of Surrey, UK 
for ethics approval. No governance risks or ethical concerns falling under the higher, medium, or lower risk criteria were identified so  Ethics and Governance Application (EGA) was not required for this study. No unauthorized access or collection of private data has occurred during this research. All datasets created and collected are sourced from publicly available materials. Reddit, a platform used in this project, openly shares its content as free and open data. Since this project does not involve interactions with human subjects, there was no requirement for informed consent. We confirm compliance with the University’s Code on Good Research Practice, Ethics Policy, and all relevant professional and regulatory guidelines. \ehsan{We highlight that we do not use the data to identify any user. The analysis is performed and reported at aggregate levels, thus minimizing the privacy risks. The examples shown in the paper are chosen at random, and the text is paraphrased (e.g., line 75, page 1) so that readers cannot directly search for the same text on Reddit.}



\section{ChatGPT Annotation Prompt}\label{app:chatgpt_annotation}


\fbox{
    \begin{minipage}{.45\textwidth}
    \texttt{
            \{ "Prompt": "You are a highly talented assistant to annotate the text. You will be helping me to identify the self-disclosure in the following prompts. Self-disclosure is defined as revealing personal information to other. Your goal is to analyze the text field and see if there is any self-disclosure and add the response as yes or no in SD field, and if there is any self-disclosure related to any of the labels in the label, list them in label field. You can use more than one label if they match.",\\
            "Text": " \hl{Proud Kashmiri Pandit}",\\
            "Labels": [`age', `education', 1ethnicity', `gender', `health', `job', `location', `physical appearance', `relationship', `religion', `sexual orientation'],\\
            "Desired format": \{\\
            "SD": "",\\
            "labels":[]\}\\
                \textbf{Response} = 
                \{\\
                     "Desired format": \{ \\ 
                      "SD": "yes",\\
                    "labels":  [\hl{`ethnicity}',\hl{`religion'}]\}\\
                \}  
    }
    \end{minipage}
}


% \fbox{
%     \begin{minipage}{.45\textwidth}
%         \begin{verbatim}
%                 \{ "Prompt": "You are a highly talented assistant to annotate the text. You will be helping me to identify the self-disclosure in the following prompts. Self-disclosure is defined as revealing personal information to other. Your goal is to analyze the text field and see if there is any self-disclosure and add the response as yes or no in SD field, and if there is any self-disclosure related to any of the labels in the label, list them in label field. You can use more than one label if they match.",\\
%                 "Text": " \hl{Proud Kashmiri Pandit}",\\
%                 "Labels": [`age', `education', 1ethnicity', `gender', `health', `job', `location', `physical appearance', `relationship', `religion', `sexual orientation'],\\
%                 "Desired format": \{\\
%                 "SD": "",\\
%                 "labels":[]\}\\
%                 \}\\
%                 \textbf{Response} = 
%                 \{\\
%                      "Desired format": \{ \\ 
%                       "SD": "yes",\\
%                     "labels":  [\hl{`ethnicity}',\hl{`religion'}]\}\\
%                 \}   
%         \end{verbatim}
%     \end{minipage}
% }

\section{Self-Disclosure Types}\label{app:sd_type_table}
Table~\ref{tab:all_sd_types} shows the disclosure types identified through the literature review in this paper. The highlighted boxes show the types that are used in this paper.

\begin{table*}[h]
\small
\caption{Self-Dislcosure Types and Categories: Highlighted color shows the types that are used in the paper.}
\begin{tabular}{|l|l|p{1.6cm}|l|p{3.5cm}|p{4.5cm}|}
\toprule
\textbf{Identity} & \textbf{ \cellcolor{yellow}Relationship} &  \textbf{Profession/ Economic} & \textbf{ \cellcolor{yellow}Health} & \textbf{Group Affiliation} & 
\textbf{Opinions/ Interests/ Feelings} \\
\midrule
   \cellcolor{yellow}Birthday/Age & Family  & \cellcolor{yellow}Job/Finance & General Health & \cellcolor{yellow}Religion & Sports \\
  \cellcolor{yellow}Ethnicity & Relations &   \cellcolor{yellow}Education & Physical Health & Politics & Politics \\
  \cellcolor{yellow}Sexual Orientation & Friendship   & & Mental Health & Community (offline vs online) &  Current-Affairs\\
 \cellcolor{yellow}Location &   &  &  &  &Religion  \\
  \cellcolor{yellow}Physical Appearance &    &  &  &  & Administration \\
 % \cellcolor{yellow}Religion &  &    &  &  &  \\
 % Politics &  &  &  &    &  \\
  \cellcolor{yellow}Gender &  &  &  &  &  \\
  % \cellcolor{yellow}Physical Appearance &    &  &  &  &  \\
 \bottomrule
\end{tabular}
\label{tab:all_sd_types}
\end{table*}





\section{Self-Disclosure Specific Communities}\label{app:sd_association}
Table~\ref{tab:sd_association} lists subreddits specific to \sexualOrientation self-disclosure. These subreddits are more likely to have disclosures related to the given type of self-disclosure. In contrast the general category of subreddit are not restricted to any self-disclosure. 
\begin{table}[H]
    \centering
    % \small
        \caption{Examplars of subreddits' association with self-disclosure.}
    \begin{tabular}{|p{2.5cm}|p{.3\textwidth}|}
    \toprule
        \textbf{Self-disclosure Type} &\textbf{Subreddits} \\
        \midrule
Sexual Orientation & `lgbt', `bisexual', `askgaybros', `BisexualTeens', `LGBTeens', `actuallesbians', `gay', `asexuality', `AreTheStraightsOK', `me\_irlgbt', `SuddenlyGay', `comingout', `pansexual', `TwoXChromosomes', `AskGayMen', `gaybros'  \\
\hline
General Subreddits & `AskReddit', `memes', `cats', `Showerthoughts'\\
\bottomrule
\end{tabular}
\label{tab:sd_association}
\end{table}




\section{Classifier Performance}\label{sec:classifier_performance}

Table~\ref{tab:classification_details} shows the performance metrics of the classifier. 


\begin{table}[H]
% \footnotesize
\caption{$F1$ performance for Roberta Fine-tuning.}
\begin{tabular}{l|l|l|l}
\toprule
\textbf{Self-Disclosure}     & \textbf{Precision} & \textbf{Recall} & \textbf{F1} \\
\midrule
\textbf{Age}                 & 0.84               & 0.93            & 0.89        \\
\textbf{Ethnicity}           & 1.00               & 0.90            & 0.95        \\
\textbf{Gender}              & 0.86               & 0.60            & 0.71        \\
\textbf{Education}           & 0.87               & 0.81            & 0.84        \\
\textbf{Health}              & 0.88               & 0.84            & 0.86        \\
\textbf{Job}                 & 0.89               & 0.70            & 0.78        \\
\textbf{Location}            & 0.79               & 0.88            & 0.83        \\
\textbf{Physical Appearance} & 0.81               & 0.87            & 0.84        \\
\textbf{Relationship}        & 0.84               & 0.86            & 0.85        \\
\textbf{Religion}            & 0.88               & 1.00            & 0.94        \\
\textbf{Sexual orientation}  & 0.88               & 0.79            & 0.83 \\
\bottomrule
\end{tabular}
\label{tab:classification_details}
\end{table}




\section{Engagement Statistics}
Table~\ref{tab:eng_post_hoc_num_comments} and~\ref{tab:eng_post_hoc_upvote_ratio} show the pairwise Dunn test results for the engagement parameters. 

\begin{table*}[h]
\small
\caption{Dunn's test for pairwise comparison for the number of comments across self-disclosure types. A positive value indicates that self-disclosure in the row has a higher mean value than the one in the column header. P-values are adjusted according to the Bonferroni method.  Kruskal-Walis Test = ($\chi^2  = 10484, df = 10, ***$), ($* = p <0.05, ** = p <0.01, *** = p <0.001 $) }
\begin{tabular}{p{1.6cm}|llllllp{1.5cm}llp{1.2cm}}
\toprule
\textbf{SD}                  & \textbf{education} & \textbf{ethnicity} & \textbf{gender} & \textbf{health} & \textbf{job} & \textbf{location} & \textbf{phy. appearance} & \textbf{relationship} & \textbf{religion} & \textbf{sexual orient.} \\
\midrule
\textbf{age}                 & 15.8 ***           & 16.6 ***           & 50.4 ***        & 24.8 ***        & 21.8 ***     & 72.9 ***          & 48.3 ***                     & 25.1 ***              & 20.4 ***          & 31.5 ***                    \\
\hline
\textbf{education}           &                    & 3.7 **             & 30.8 ***        & 4.0 **          & 1.6          & 48.1 ***          & 30.1 ***                     & 4.4 ***               & 5.9 ***           & 18.3 ***                    \\
\hline
\textbf{ethnicity}           &                    &                    & 20.5 ***        & -1.3            & -3.1         & 32.1 ***          & 20.7 ***                     & -0.9                  & 1.5               & 12.7 ***                    \\
\hline
\textbf{gender}              &                    &                    &                 & -36.3 ***       & -38.3 ***    & 15.2 ***          & 1.3                          & -35.4 ***             & -21.2 ***         & -4.9 ***                    \\
\hline
\textbf{health}              &                    &                    &                 &                 & -3.6 *       & 66.7 ***          & 34.2 ***                     & 0.7                   & 3.6 *             & 18.0 ***                    \\
\hline
\textbf{job}                 &                    &                    &                 &                 &              & 68.0 ***          & 36.2 ***                     & 4.2 **                & 5.6 ***           & 19.6 ***                    \\
\hline
\textbf{location}            &                    &                    &                 &                 &              &                   & -12.2 ***                    & -65.0 ***             & -34.9 ***         & -15.1 ***                   \\
\hline
\textbf{phy. appearance} &                    &                    &                 &                 &              &                   &                              & -33.5 ***             & -21.3 ***         & -5.7 ***                    \\
\hline
\textbf{relationship}        &                    &                    &                 &                 &              &                   &                              &                       & 3.2               & 17.6 ***                    \\
\hline
\textbf{religion}            &                    &                    &                 &                 &              &                   &                              &                       &                   & 12.2 ***          \\
\bottomrule
\end{tabular}
\label{tab:eng_post_hoc_num_comments}
\end{table*}


\begin{table*}[h]
\small
\caption{Dunn's test for pairwise comparison for the upvote ratio across self-disclosure types. A positive value indicates that self-disclosure in the row has a higher mean value than the one in the column header. P-values are adjusted according to the Bonferroni method.  Kruskal-Walis Test = ($\chi^2  = 4322.7, df = 10, ***$), ($* = p <0.05, ** = p <0.01, *** = p <0.001 $) }
\begin{tabular}{p{1.6cm}|llllllp{1.5cm}llp{1.2cm}}
\toprule
\textbf{SD}                  & \textbf{education} & \textbf{ethnicity} & \textbf{gender} & \textbf{health} & \textbf{job} & \textbf{location} & \textbf{phy. appearance} & \textbf{relationship} & \textbf{religion} & \textbf{sexual orient.} \\
\midrule
\textbf{age}                 & 2.2                & 24.2 ***           & 4.7 ***         & -0.1            & 30.7 ***     & 11.4 ***          & -8.9 ***                     & 9.7 ***               & 28.1 ***          & 26.1 ***                    \\
\hline
\textbf{education}           &                    & 21.6 ***           & 2.1             & -2.7            & 25.5 ***     & 7.8 ***           & -10.6 ***                    & 6.2 ***               & 24.9 ***          & 23.4 ***                    \\
\hline
\textbf{ethnicity}           &                    &                    & -21.3 ***       & -26.9 ***       & -5.2 ***     & -18.9 ***         & -30.7 ***                    & -20.0 ***             & 0.8               & 1.7                         \\
\hline
\textbf{gender}              &                    &                    &                 & -6.0 ***        & 26.8 ***     & 6.3 ***           & -13.9 ***                    & 4.5 ***               & 25.0 ***          & 23.2 ***                    \\
\hline
\textbf{health}              &                    &                    &                 &                 & 42.1 ***     & 15.9 ***          & -10.7 ***                    & 13.5 ***              & 32.2 ***          & 28.9 ***                    \\
\hline
\textbf{job}                 &                    &                    &                 &                 &              & -26.3 ***         & -40.0 ***                    & -28.4 ***             & 7.2 ***           & 7.4 ***                     \\
\hline
\textbf{location}            &                    &                    &                 &                 &              &                   & -21.6 ***                    & -2.3                  & 23.0 ***          & 21.0 ***                    \\
\hline
\textbf{physical appearance} &                    &                    &                 &                 &              &                   &                              & 20.0 ***              & 35.2 ***          & 32.4 ***                    \\
\hline
\textbf{relationship}        &                    &                    &                 &                 &              &                   &                              &                       & 24.2 ***          & 22.1 ***                    \\
\hline
\textbf{religion}            &                    &                    &                 &                 &              &                   &                              &                       &                   & 1       \\
\bottomrule
\end{tabular}
\label{tab:eng_post_hoc_upvote_ratio}
\end{table*}







\end{document}
\endinput
%%
%% End of file `sample-sigplan.tex'.



