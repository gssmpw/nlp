\section{Related Work}
Appraisal theories of emotion describe that emotions are due to our evaluation of an event based on personal experiences, resulting in various emotions evoked for different people \citep{arnold1960emotion, moors2013appraisal,ellsworth2013appraisal,frijda1986emotions,lazarus1991emotion,ortony2022cognitive,roseman2013appraisal,scherer2009dynamic}. The theory of constructed emotions states that they are not hard-wired in the brain or universal, but are rather concepts constructed by the brain \citep{barrett2016,barrett2017emotions}. 

Prior work in NLP has largely focused on \textit{sentiment analysis} -- detecting whether a text expresses positive, negative, or neutral valence \citep{Mohammad2016,muhammad-etal-2023-semeval}. Recent work focus has shifted to a broader form—detecting emotions in text such as anger, fear, joy, sadness, etc. in text which is in line with discrete models of emotions (e.g., Paul Eckman's six emotions \citep{ekman1992there} and Plutchik's Wheel of Emotions \citep{Plutchik1980} for anger, disgust, fear, happiness, sadness, surprise, anticipation and trust).

Several initiatives have created emotion classification datasets for languages other than English (e.g., Italian \citep{bianchi-etal-2021-feel}, Romanian \citep{ciobotaru-etal-2022-red},
% \citep{ciobotaru-dinu-2021-red},
Indonesian \citep{saputri2018emotion}, and Bengali \citep{iqbal2022bemoc}). However, NLP work in the area is predominantly Western-centric, and while multilingual datasets like XED \citep{ohman2020xed} and XLM-EMO \citep{bianchi2022xlm} exist, XLM-EMO's reliance on translated data for over ten languages may not fully capture cultural nuances in emotion expression. Emotions are culture-sensitive and highly contextualized, influenced by cultural values \citep{havaldar-etal-2023-multilingual,mohamed-etal-2024-culture,hershcovich-etal-2022-challenges}. Further, although emotions can co-occur \citep{vishnubhotla-etal-2024-emotion-granularity}, most datasets assume single-label classification. While GoEmotions \citep{demszky-etal-2020-goemotions} addresses multi-label emotion classification, to our knowledge, no multilingual resources capture the overlapping emotions and intensity across languages.
This work aims to push this boundary by presenting emotion-labeled data for 28 languages. Given the lack of unanimity surrounding language categorisation as low-resource, approximately 15 to 17 of these languages could be considered such. 


%Prior work in NLP has largely focused on determining the attitude towards an entity, such as customer reviews for a product or brand; however, sentiment analysis has many use-cases, such as in government (e.g., opinion towards politicians, government policy or initiatives), education (e.g., emotions are integral to learning), and public health (e.g., chatbots). SemEval Tasks have large audiences working on creating more advanced systems for detecting emotions in text. Since 2011, there have been yearly tasks focused on sentiment analysis on customer reviews and tweets, with more diverse topics in recent years \cite{}. Furthermore, focus has shifted to also detecting emotions in text such as anger, fear, joy, sadness, etc. in text.
% SemEval-2014 Task 4, SemEval-2015 Task 12, SemEval-2016 Task 5

% Emotion detection can also focus on determining the attitude of the speaker or the emotional state of the speaker.
%However, recent efforts have developed emotion labeled data for lower-resource languages such African languages (e.g., AfriSenti-SemEval, \citealp{muhammad-etal-2023-semeval}).