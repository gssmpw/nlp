\appendix
\onecolumn

\section{Prompt templates}

\subsection{Hallucination judgement prompt}

\begin{tcolorbox}[title=Prompt Template of Labeling Process for Faithful Hallucination Benchmark, colback=gray!20, colframe=gray!75, rounded corners, sharp corners=northeast, sharp corners=southwest]
\small

You will be provided with a document and a proposed summary. Your task is to determine if the proposed summary can be directly inferred from the document. If the summary contains any information not found in the document, it is considered false. Even if the summary is different from a ground truth summary, it might still be true, as long as it doesn't contain false information. \\

For each proposed summary, explain why it is true or false based on the information from the document. Focus only on the original document's content, disregarding any external context. \\

After your explanation, give your final conclusion as **Conclusion: True** if the proposed summary is completely accurate based on the document, or **Conclusion: False** if it contains any incorrect or unsupported information. If your conclusion is 'False', identify the exact phrases or name entities from the summary that is incorrect by stating **Problematic Spans: [the inaccurate text spans from the summary, in Python list of strings format]**. \\

\# Document \#: \textcolor{orange}{\$document} \\

\# Ground Truth Summary \#: \textcolor{blue}{\$gt\_response} \\

\# Proposed Summary \#: \textcolor{red}{\$response} \\

Write your explanation first, and then give your final conclusion as **Conclusion: True** if the proposed summary is completely accurate based on the document, or **Conclusion: False** if it contains any incorrect or unsupported information. Add **Problematic Spans: [the exact inaccurate text spans from the summary, in a list of strings]** if your conclusion is 'False'.

\end{tcolorbox}

\textcolor{orange}{\$document} is the original document text for the faithful hallucinated question. \textcolor{blue}{\$gt\_response} is the ground-truth response for the question. \textcolor{red}{\$response} is target LLM's response towards the question.



\begin{tcolorbox}[title=Prompt Template of Labeling Process for Factual Hallucination Benchmark, colback=gray!20, colframe=gray!75, rounded corners, sharp corners=northeast, sharp corners=southwest]
\small

You will be provided with a question (along with the groundtruth answer) and a proposed answer. Your task is to determine if the proposed answer align with the groundtruth answer. If the answer is not aligned with the groundtruth answer, it is considered false. Even if the answer is not totally the same with ground truth answer, it might still be true (For example, it is just a difference in expression or an alias for some nouns.). \\

For each proposed answer, explain why it is true or false based on the given question and the groundtruth answer. \\

After your explanation, give your final conclusion as **Conclusion: True** if the proposed answer is completely accurate, or **Conclusion: False** if it contains any incorrect or unsupported information. If your conclusion is 'False', identify the exact phrases or name entities from the answer that is incorrect by stating **Problematic Spans: [the inaccurate text spans from the answer, in Python list of strings format]**. It should only include the exact wrong phrases or name entities but not the full sentence. Avoid to include the whole sentence.\\

\# Question \#: \textcolor{orange}{\$question} \\

\# Ground Truth Summary \#: \textcolor{blue}{\$gt\_response} \\

\# Proposed Summary \#: \textcolor{red}{\$response} \\

Write your explanation first, and then give your final conclusion as **Conclusion: True** if the proposed answer is completely accurate, or **Conclusion: False** if it contains any incorrect or unsupported information. Add **Problematic Spans: [the exact inaccurate text spans from the answer, in a list of strings]** if your conclusion is 'False'.

\end{tcolorbox}

\textcolor{orange}{\$question} is the original question text for the factual hallucinated question. \textcolor{blue}{\$gt\_response} is the ground-truth response for the question. \textcolor{red}{\$response} is target LLM's response towards the question.