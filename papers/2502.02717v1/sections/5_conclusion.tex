\section{Conclusion}\label{sec:conclusion}
This paper presents the updated version of Astromer, a self-supervised model designed to extract general-purpose embeddings from light curve data. We demonstrate that Astromer 2 outperforms its predecessor, Astromer 1, across multiple scenarios, including both the Alcock and ATLAS datasets. The key improvements in classification performance, especially when trained with limited labeled samples, highlight the informative power of the embeddings in discriminating between different classes.

The results show that Astromer 2 achieves significant gains in F1 score, especially with small training sets, underscoring the effectiveness of the weighted per-sample embeddings and the model’s ability to generalize across different datasets. The analysis of attention weights further reveals that intermediate embeddings contribute meaningfully to the model's performance, focusing more on certain parts of the input data during the classification task.

These findings confirm the potential of Astromer as a robust tool for light curve analysis, showing that self-supervised learning can provide valuable insights into astronomical data. Future directions may explore the integration of multi-modal data and more complex attention mechanisms, though the simplicity and efficiency of the current approach remain significant advantages. Future work on Astromer will focus on incorporating multiband data, either as an additional feature embedding or by directly constraining the embedding space. Additionally, we plan to train Astromer on the entire survey, bringing in more data during pretraining and potentially capturing more informative representations. 