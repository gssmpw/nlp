\section{Data Sources}\label{sec:data}
In this section, we introduce our training data, including unlabeled light curves for pretraining and labeled samples for the downstream classification task. 

\subsection{Unlabeled data - MACHO}
The MACHO project \citep{1993Natur.365..621A} aimed to detect Massive Compact Halo Objects (MACHO) to find evidence of dark matter in the Milky Way halo by searching for gravitational microlensing events. Light curves were collected from 1992 to 1999, producing light curves of more than a thousand observations \citep{1999PASP..111.1539A} in bands B and R.
The observed sky was subdivided into 403 fields. Each field was constructed by observing a region of the sky or tile. The resulting data is available in a public repository\footnote{\url{https://macho.nci.org.au/macho_photometry}} which contains millions of light curves in bands B and R. 

We selected a subset of fields 1, 101, 102, 103, and 104 containing \num{1454792} light curves for training. Similarly, we select field 10 for testing, with a total of \num{74594} light curves. MACHO observed in both bands simultaneously, therefore having two magnitudes associated with each MJD. Since we are looking to improve on Astromer 2, we maintain the single band input.  The light curves from this dataset that exhibited Gaussian noise characteristics were removed based on the criteria: $|\text{Kurtosis}| > 10$, $|\text{Skewness}| > 1$, and $\text{Std} > 0.1$. Additionally, we excluded observations with negative uncertainties (indicative of faulty measurements) or uncertainties greater than one (to maintain photometric quality). Outliers were also removed by discarding the 1st and 99th percentiles for each light curve. This additional filtering does not affect the total number of samples but reduces the number of observations when the criteria were applied.

\begin{figure}
    \centering
    \includegraphics[scale=.88]{figures/data/magnitude_datasets.pdf}
    \caption{Magnitude distributions for the MACHO, Alcock, and ATLAS datasets. The plotted magnitudes reflect their original values as reported in the datasets; however, they are normalized during training, eliminating the differences in their mean positions.
    The Alcock catalog exhibits multimodality. In contrast, the ATLAS magnitudes show significant more variation, as they originate from a different survey.}
    \label{fig:macho-alcock-magn}
\end{figure}

\subsection{Labeled data}
To ensure a fair comparison with Astromer 1, we used the same sample selection from the MACHO \citep[hereafter referred to as Alcock; ][]{Alcock2001Variable} and the  Asteroid Terrestrial-impact Last Alert System \citep[hereafter referred to as ATLAS; ][]{heinze2018first} labeled catalogs. The former has a similar magnitude distribution, whereas the latter differs, as shown in Fig. \ref{fig:macho-alcock-magn}.

\subsubsection{Alcock}
For labeled data, we use the catalog of variable stars from \citet{Alcock2001Variable}, which contains labels for a subset of the MACHO light curves originating from 30 fields from the Large Magellanic Cloud. This labeled data will be used to train and evaluate the performance of the different embeddings on the classification task. 

The selected data comprises \num{20894} light curves, which are categorized into six classes: Cepheid variables pulsating in the fundamental (Cep\_0) and first overtone (Cep\_1), Eclipsing Binaries (EC), Long Period Variables (LPV), RR Lyrae ab and c (RRab and RRc, respectively). Table \ref{tab:alcock} summarizes the number of samples per class. We note that the catalog used is an updated version, as described in \cite{astromer}.

\begin{table}
\caption{Alcock catalog distribution.}              
\label{tab:alcock}  
\centering 
% \begin{tabular}{c c c} 
\begin{tabular}{l l r} 
\hline\hline         
Tag & Class Name & \# of sources \\ \hline
 Cep\_0 & Cepheid type I &\num{1182} \\
 Cep\_1 &Cepheid type II & \num{683} \\
 EC &Eclipsing binary & \num{6824} \\
 LPV &Long period variable &  \num{3046} \\
 RRab &RR Lyrae type ab  &  \num{7397} \\
 RRc &RR Lyrae type c &  \num{1762} \\
 Total & & \textbf{\num{20894}} \\
\hline                            
\end{tabular}
\end{table}

Figure \ref{fig:macho-alcock-magn} compares the magnitude distributions between the Alcock and MACHO datasets. The former exhibits a bimodal distribution, which aligns with the fact that it represents a subset of the light curves from MACHO fields, while the latter encompasses light curves from only five fields. 

Similarly, we compare the distribution of time differences between consecutive observations ($\Delta t$). Figure \ref{fig:macho-alcock-mjd} shows similar distributions, with comparable ranges and means of three and four days for MACHO and Alcock, respectively.
\begin{figure}
    % \centering
    \includegraphics[scale=0.7]{figures/data/mjd_datasets.pdf}
    \caption{Distributions of consecutive observation time differences ($\Delta t$) for the Alcock, MACHO, and ATLAS datasets. The boxplots illustrate the variability in observation cadences across the datasets. The Alcock and MACHO datasets show relatively consistent sampling with narrower distributions, while the ATLAS dataset exhibits a broader range of $\Delta t$, reflecting more diverse observation intervals. The y-axis is shown on a logarithmic scale to highlight differences across several orders of magnitude }
    \label{fig:macho-alcock-mjd}
\end{figure}

\subsection{ATLAS}
The Asteroid Terrestrial-impact Last Alert System \citep[ATLAS; ][]{Tonry2018} is a survey developed by the University of Hawaii and funded by NASA. Operating since 2015, ATLAS has a global network telescopes, primarily focused on detecting asteroids and comets that could potentially threaten Earth. Observing in $c$ (blue), $o$ (orange), and $t$ (red) filters.

The variable star dataset used in this work was presented by \citet{heinze2018first} and includes 4.7 million candidate variable objects, included in the labeled and unclassified objects, as well as a dubious class. According to their estimates, this class is predominantly composed of $90\%$ instrumental noise and only $10\%$ genuine variable stars.

We analyze \num{141376} light curves from the ATLAS dataset, as detailed in Table \ref{tab:ATLAS}. These observations, measured in the $o$ passband, have a median cadence of $\sim$15 minutes, which is significantly shorter than the typical cadence in the MACHO dataset. This substantial difference poses a challenge for the model, as it must adapt to such a distinct temporal distribution. 

\begin{table}[h!]
\caption{ATLAS catalog distribution.}              
\label{tab:ATLAS}
\centering 
\begin{tabular}{l l r} 
\hline\hline         
Tag & Class Name & \# of sources \\
\hline
CB & Close Binaries &  \num{80218} \\
DB & Detached Binary &  \num{28767} \\
Mira & Mira &  \num{7370} \\
Pulse &RR Lyrae, $\delta$-Scuti, Cepheids &  \num{25021} \\
Total & & \textbf{\num{141376}}\\
\hline                            
\end{tabular}
\end{table}

As done in \citet{astromer} and to standardize the labels with other datasets, we combine detached eclipsing binaries identified by full or half periods into the close binaries (CB) category and similarly merge detached binaries (DB). However, objects with labels derived from Fourier analysis are excluded, as these classifications do not directly align with astrophysical categories.

\subsection{MACHO vs ATLAS}\label{sec:machovsatlas}
Figures \ref{fig:macho-alcock-magn} and \ref{fig:macho-alcock-mjd} illustrate the distributional differences between the unlabeled MACHO dataset and the labeled subsets discussed earlier. While the magnitudes show a notable shift between MACHO and ATLAS, our training strategy normalizes the light curves to a zero mean. As a result, the relationships between observations take precedence over the raw magnitude values. Consequently, we do not expect a substantial performance drop when transitioning between datasets. However, for $\Delta t$, the smaller values of $\Delta t$ present a significant challenge, as the model must extrapolate and account for fast variations to capture short-time information effectively. We evidence this in our first results from Astromer 2, where the F1 score on the ATLAS dataset was lower compared to MACHO when having fewer labels for classification. 
