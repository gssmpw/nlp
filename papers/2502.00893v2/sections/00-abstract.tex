% Abstracts: single paragraph, between 4--6 sentences long

\twocolumn[{%
\renewcommand\twocolumn[1][]{#1}%
\maketitle
\includegraphics[width=\textwidth]{figures/teaser.pdf}
\vspace{-4mm}
\captionof{figure}{
\textbf{\system} is an open-source, ML-compatible humanoid platform for efficiently collecting large-scale, high-quality training data in both simulation and the real world. We leverage massive parallel environments, an accurate digital twin for simulation, and an intuitive teleoperation device for precise upper and lower body control in real-world data collection. We demonstrate that \system meets these requirements and successfully acquires a diverse set of loco-manipulation skills from both data sources, including walking, push-ups, pull-ups, wagon pushing, bimanual, and full-body manipulation.}
\vspace{3mm}
\label{fig:teaser}
}]


\begin{abstract}

Learning-based robotics research driven by data demands a new approach to robot hardware design—one that serves as both a platform for policy execution and a tool for embodied data collection to train policies. We introduce \system, a low-cost, open-source humanoid robot platform designed for scalable policy learning and research in robotics and AI. \system enables seamless acquisition of high-quality simulation and real-world data. The plug-and-play zero-point calibration and transferable motor system identification ensure a high-fidelity digital twin, enabling zero-shot policy transfer from simulation to the real-world. A user-friendly teleoperation interface facilitates streamlined real-world data collection for learning motor skills from human demonstrations. Utilizing its data collection ability and anthropomorphic design, \system is an ideal platform to perform whole-body loco-manipulation. Additionally, \systems compact size ($0.56~\mathrm{m},\space 3.4~\mathrm{kg}$) ensures safe operation in real-world environments. Reproducibility is achieved with an entirely 3D-printed, open-source design and commercially available components, keeping the total cost under $6000~\mathrm{USD}$. Comprehensive documentation allows assembly and maintenance with basic technical expertise, as validated by a successful independent replication of the system. We demonstrate \systems capabilities through arm span, payload, endurance tests, loco-manipulation tasks, and a collaborative long-horizon scenario where two robots tidy a toy session together. By advancing ML-compatibility, capability, and reproducibility, \system provides a robust platform for scalable learning and dynamic policy execution in robotics research.

% Data-driven robotics relies on high-quality, large-scale datasets. Simulation enables scalable data collection but demands accurate digital twins, while real-world data offers high fidelity but is costly and complex. To bridge this gap, we present \system, an open-source humanoid platform designed to be ML-compatible, capable, and reproducible. 
% To ensure ML-compatibility, \system facilitates efficient data collection in both simulation and the real world. A plug-and-play calibration procedure creates a high-fidelity digital twin, validated through zero-shot sim-to-real transfer on an RL walking policy. Real-world data collection is scaled through an intuitive teleoperation system, enabling simultaneous upper and lower body control, which supports learning visuomotor policies.
% With 30 degrees of freedom, \system excels in complex whole-body control tasks such as push-ups and pull-ups, achieved by directly replaying keyframe animations in the real world.
% To ensure reproducibility, \system is designed to be low-cost, easy to build, and simple to repair. This was verified by having a second \system instance built by another CS Ph.D. student, demonstrating zero-shot loco-manipulation policy transfer between instances. Additionally, we showcased a long-horizon collaborative task involving two \system units.
% \karen{1. Need to address Toddy's capability/dofs for both manipulation and locomotion. 2. Need to make the ML-compatibility argument more crisp. Argue that we are not just designing yet another humanoid. We believe that a ML-compatible humanoid platform needs to provide two critical services, not just the robot hardware itself. First, we provide a procedure to create and calibrate the digital twin of the ToddlerBot, enabling effective learning and validation in simulation. Second, we provide a teleoperation system to facilitate large-scale real-world data collection. 3. Need to bring out the point that robot hardware for research community needs to be self-reliant (i.e. easily repairable and extensible).}
\end{abstract}

% \begin{IEEEkeywords}
% humanoid
% \end{IEEEkeywords}

