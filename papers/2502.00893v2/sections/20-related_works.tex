\section{Related Works}
\label{sec:related_works}

% \subsection{Humanoid Design}
In recent years, numerous humanoid robots have been developed, showcasing diverse designs and capabilities. Industrial humanoids such as Boston Dynamics Atlas~\citep{atlas}, Booster T1~\citep{humanoid}, Cassie~\citep{cassie}, Digit~\citep{agility}, Figure~\citep{figure}, Fourier GR1~\citep{fourierrobotics}, NAO H25~\citep{nao}, Tesla Optimus~\citep{ai}, Robotis OP3~\citep{nameintroduction}, Unitree G1~\citep{unitreea}, and H1~\citep{unitree} demonstrate remarkable technological advancements. 
On the other hand, humanoids from research institutions, including the Berkeley Humanoid~\citep{liao2024berkeley}, BRUCE~\citep{liu2022design}, Duke Humanoid~\citep{xia2024duke}, iCub~\citep{parmiggiani2012design}, and MIT Humanoid~\citep{chignoli2021mit}, also explore the humanoid design space with different emphasis. 
Humanoid robots can be evaluated using various metrics; we prioritize three core attributes: ML-compatibility, capability, and reproducibility. As outlined in Table~\ref{tab:comparison}, we identify nine metrics to compare these attributes across different humanoid platforms.

Size and weight are critical factors when designing humanoid robots. A smaller humanoid is inherently cheaper, easier to build and repair, and safer. Larger, full-size humanoids typically require a substantial engineering team for operation and maintenance, along with specialized facilities like gantry cranes for safety. In contrast, smaller humanoids can be deployed by a small team, often a single person, and operated in constrained environments with simply a laptop.
But what is the trade-off with a smaller size? Reduced size can limit a robot’s ability to manipulate human-scale objects, However, miniature humanoids can still effectively perform meaningful manipulation tasks when paired with appropriately scaled objects. Furthermore, learning to manipulate smaller objects and developing dynamic whole-body control and locomotion techniques are likely transferable to larger humanoids.

We argue that the number of active DoFs is crucial for the research value of a humanoid platform since more DoFs enable more human-like motion. While the human musculoskeletal system employs over 600 muscles working in complex synergies, the primary functional movements of the human body can be approximated using 32 revolute joints in a robotic system: six DoFs per leg, seven DoFs per arm, three DoFs for the waist, and three DoFs for the neck, excluding fingers and toes. Therefore, humanoid designs aim to achieve a DoF count as close to 32 as possible. The passive DoFs are less important as they do not contribute to the action space. We believe that the public perceived limited performance of miniature humanoids is primarily due to fewer DoFs. This limitation often arises from space constraints that restrict the incorporation of many DoFs, a challenge we have successfully addressed in \system. 

To qualitatively assess the capability, we evaluate the humanoid’s ability to perform both manipulation and locomotion tasks. While each type of motion is important, the combination of both is particularly compelling, as it unlocks opportunities for whole-body control research~\citep{he2024omnih2oa, he2024hover, fu2024humanplus, ji2024exbody2, lu2024mobiletelevision}. Furthermore, certain motions, such as push-ups, pull-ups, and cartwheels, go beyond traditional categories of manipulation and locomotion, treating them as prerequisites and requiring coordinated use of both arms and legs.

Recent advances show that large-scale simulation-based data collection is highly effective for locomotion~\citep{rudin2022learning, tan2018simtoreal, lee2020learning}, while real-world data collection is more promising for manipulation~\citep{oneill2024open, khazatsky2024droid, mandlekar2019scaling}. Therefore, an ideal humanoid research platform should facilitate data collection in both simulation and real-world settings, which we defined as \textbf{ML-compatibility}.

Moreover, being open-source and low-cost is essential for others to reproduce. Without these qualities, research in this field would remain restricted to those with specialized expertise and significant resources. While making no compromise in functionality, \system stands out as completely open-source and the most affordable among recent humanoid platforms, making it accessible to a wider range of researchers.

% Consequently, these advancements in robot learning call for new humanoid hardware designs that are low-cost, robust, and easily reproducible to support large-scale data collection and deployment. Therefore, we propose \system as a low-cost, ML-compatible hardware platform designed to leverage recent advancements in both manipulation and locomotion.

