\subsection{Power Factor}
\label{sec:power_metric}
When comparing the performance of humanoids with different scales and weights, it is essential to establish a meaningful metric. Directly having a full-sized $180~\mathrm{cm}$ humanoid and a $50~\mathrm{cm}$ scaled humanoid both jump $50~\mathrm{cm}$ or run at $3~\mathrm{m/s}$ is not a fair comparison. A more reasonable approach is to normalize performance metrics, for example by evaluating a jump at $10\%$ of body height or a running speed of twice the body length per second. Formally, we say two humanoid having the \textbf{same performance} if they execute the same sequence of joint motions over a time span $T$, and their total power consumption is the same fraction of their motors’ maximum power:
\begin{equation}
     \frac{\int_0^T{{p}(t) dt}}{\sum_{i=0}^{N}|{\tau}_{i}^{\text{max}}\dot{q}_i|} \approx
     \frac{\Delta h \cdot mg}{\sum_{i=0}^{N}|{\tau}_{i}^{\text{max}}\dot{q}_i|} \approx   \frac{h \cdot mg}{\sum_{i=0}^{N}|{\tau}_{i}^{\text{max}}\dot{q}_i|},
\label{eqn:power_factor_explained}
\end{equation}
where $p(t)$ is the humanoid's power output at time $t$. $\tau^\text{max}_i$ and $\dot{q}_i$ are the maximum torque and joint velocity of the $i$-th motor in the humanoid respectively. Moreover, when computing the maximum power, we take the absolute value to ensure all motors perform positive work. For the numerator, the integral of the power output over $T$, which is equivalent to the work done by the robot, can be approximated by the gravitational energy gained: $\Delta h \cdot mg$. Since $\Delta h$ is approximately proportional to the height of the humanoid, we further replace $\Delta h $ with just the humanoid's height $h$.

Based on Equation \ref{eqn:power_factor_explained}, we now define a metric, power factor, to measure the performance of a humanoid:
\begin{equation}
    \tilde{\mathit{p}} = \frac{\sum_{i=0}^{\mathit{N}} |\tau_{i}^{\max}|}{h \cdot m g}.
\end{equation}
Note that we flip the fraction from Equation \ref{eqn:power_factor_explained} to make the power factor value increase as the utilized torque ratio decreases. We also drop $\dot{q}$ as the same sequence of joint motion would be executed when using power factor to compare the performance of humanoids.


% And the numerator $\tau^{h1}_{i}\Delta q_i = W_i$ is really just the work done by the joint i. 
% And we rearrange the equation knowing $\sum_{i=0}^{\mathit{N}}W_i = W \approx mg\Delta h$
% \begin{align}
%      \frac{\sum_{i=0}^{\mathit{N}}\tau^{h1}_{i,max}}{\Delta h_{1} \cdot m_{1}g} &= 
%      \frac{\sum_{i=0}^{\mathit{N}}\tau^{h2}_{i,max}}{\Delta h_{2} \cdot m_{2}g}
% \end{align}
% therefore for the purpose of power factor, we can use height and cancel out the constants on both sides. 
 



% When comparing the performance of humanoids with different scales and weights, it is essential to establish a meaningful metric. Directly having a full-sized $180~\mathrm{cm}$ humanoid and a $50~\mathrm{cm}$ scaled humanoid both jump $50~\mathrm{cm}$ or run at $3~\mathrm{m/s}$ is not a fair comparison. A more reasonable approach is to normalize performance metrics, for example by evaluating a jump at $10\%$ of body height or a running speed of twice the body length per second. Formally, we say two humanoid having the \textbf{same performance} if they execute the same sequence of joint motions over a time span $T$, and their total power consumption is the same fraction of their motors’ maximum power. Based on this definition, we can define the following equation:
% \begin{equation}
%      \frac{\int_0^T{{p}_{1}(t) dt}}{\|\bm{p}_1^{\text{max}}\|_1} = 
%      \frac{\int_0^T{{p}_{2}(t) dt}}{\|\bm{p}_2^{\text{max}}\|_1},
% \end{equation}
% where scalar $p(t)$ is the humanoid's total power output at time $t$, $\|\bm{p}_1^{\text{max}}\|_1$ and $\|\bm{p}_2^{\text{max}}\|_1$ are the L1 norm of the maximum motor powers of the two humanoid robots.

% For the numerator, the integral of power is simply the work done by the robot:
% \begin{equation}
%     w = \int_{0}^{T}{{p}(t) dt}.
% \end{equation}
% % =\int{\bm{\tau}\bm{\dot{q}}\ dt}=\bm{\tau} \Delta \bm{q},
% % we can derive that 
% % \begin{equation}
% %      \frac{w_1}{\|\bm{p}_1^{\text{max}}\|_1} = 
% %      \frac{w_2}{\|\bm{p}_2^{\text{max}}\|_1}.
% % \end{equation}
% Moreover, in a quasi-static situation, the robot's work is approximately equal to the gravitational energy gained:
% \begin{equation}
%     w \approx \Delta h \cdot mg.
% \end{equation}
% Therefore, we have 
% \begin{equation}
%      \frac{\Delta h_1 \cdot m_1g}{\|\bm{p}_1^{\text{max}}\|_1} \approx 
%      \frac{\Delta h_2 \cdot m_2g}{\|\bm{p}_2^{\text{max}}\|_1}.
% \end{equation}
% Since $\Delta h$ is proportional to the height of the humanoid, we can use the height instead:
% \begin{equation}
%      \frac{h_1 \cdot m_1g}{\|\bm{p}_1^{\text{max}}\|_1} \approx 
%      \frac{h_2 \cdot m_2g}{\|\bm{p}_2^{\text{max}}\|_1}.
% \end{equation}

% For the denominator, note that 
% \begin{equation}
%     \|\bm{p}^{\text{max}}\|_1 = \sum_{i=0}^N|{\tau}_i^{\text{max}} {\dot{q}}_i|,
% \end{equation}
% Then we have
% \begin{equation}
%     \frac{h_1 \cdot m_1g}{\sum_{i=0}^{N_1}|{\tau}_{1, i}^{\text{max}} {\dot{q}}_{1, i}|} \approx 
%     \frac{h_2 \cdot m_2g}{\sum_{i=0}^{N_2}|{\tau}_{2, i}^{\text{max}} {\dot{q}}_{2, i}|}
% \end{equation}
% Since we are executing the same sequence of joint motion, $\dot{q}$ can be canceled out from the denominators:
% \begin{equation}
%     \frac{h_1 \cdot m_1g}{\sum_{i=0}^{N_1}|{\tau}_{1, i}^{\text{max}}|} \approx
%     \frac{h_2 \cdot m_2g}{\sum_{i=0}^{N_2}|{\tau}_{2, i}^{\text{max}}|}
% \end{equation}
% % And the numerator $\tau^{h1}_{i}\Delta q_i = W_i$ is really just the work done by the joint i. 
% % And we rearrange the equation knowing $\sum_{i=0}^{\mathit{N}}W_i = W \approx mg\Delta h$
% % \begin{align}
% %      \frac{\sum_{i=0}^{\mathit{N}}\tau^{h1}_{i,max}}{\Delta h_{1} \cdot m_{1}g} &= 
% %      \frac{\sum_{i=0}^{\mathit{N}}\tau^{h2}_{i,max}}{\Delta h_{2} \cdot m_{2}g}
% % \end{align}
% % therefore for the purpose of power factor, we can use height and cancel out the constants on both sides. 
 

% Lastly, we flip the fraction to make the power factor value increase as the utilized torque ratio decreases. This leaves us with the final dimensionless, power factor formula to compare the performance of different humanoids consistently:
% \begin{equation}
%     \tilde{\mathit{p}} = \frac{\sum_{i=0}^{\mathit{N}} |\tau_{i}^{\max}|}{h \cdot m g}.
% \end{equation}

% A humanoid’s ability to accelerate ($\mathrm{a}$) and its characteristic velocity ($\mathrm{v}$) define how energetically it can move. However, velocity and acceleration become less directly comparable across humanoids with vastly different heights. For example, a small $50~\mathrm{cm}$ humanoid moving at $3~\mathrm{m/s}$ is not equivalent (in scale-normalized terms) to a $180~\mathrm{cm}$ humanoid moving at the same speed.

% Joint torque $\bm{\tau}$ is the product of force $F$ and moment arm $r$, meaning both force and lever length affect power output. Since joint torque typically overcomes weight in humanoid motion, we normalize the factor by $\mathrm{m}g$ (weight) and $\mathrm{h}$ (characteristic length), since individual link lengths of a humanoid are proportional to its scale. This yields a dimensionless power factor:

% This factor serves as a dimensionless comparison of how quickly a humanoid can impart energy to its own body relative to its size. For example, jumping $40~\mathrm{cm}$ for a toddler-sized ($55~\mathrm{cm}$) humanoid can be much more challenging than jumping $40~\mathrm{cm}$ for a full-scale humanoid—but in absolute scales, these two jumps are considered equivalent. By including $\mathrm{h}$ in the denominator, we ensure that motion is scaled appropriately to each humanoid’s characteristic height.

In summary, our proposed $\tilde{p}$ naturally incorporates the crucial physical parameters: height, mass, and gravity, to give a fair measure of motion capability across humanoids of very different scales. 
% A higher $\tilde{p}$ means the humanoid has excess torque capability to achieve more dynamic motion compared to its own size and weight. 
Achieving or surpassing the human threshold $\tilde{p}$ is indicative of human-like dynamic potential, although practical constraints (e.g., battery efficiency, control complexity, safety) may set an upper bound on how large $\tilde{p}$ should be in a real-world humanoid design.







% \subsection{Power Factor}
% \label{sec:power_metric}

% When comparing the performance between humanoids of different scale and weight, it is important to clarify what we really want to compare. We believe it would not make sense to ask a full size, 180cm humanoid and a 50cm, scaled humanoid to both jump 50cm and run 3m/s. Instead, a more rational approach is to ask both to jump 10\% body height, and run 2 body length per second, equivalently comparing the same joint motion sequence.
% Hence, if two humanoids execute the same sequence of joint motions while consuming the same proportion of their motors’ power, they will have the same power factor.

% Express this formally in math, we have relation:

% \begin{align}
%     \hat{P}^{h1} &= \hat{P}^{h2}, \\
%     \sum_{i=0}^{\mathit{N}} \frac{\int{p_{i}^{h1}(t) dt}}{\int p_{i,max}^{h1}(t) dt} &= 
%     \sum_{i=0}^{\mathit{N}} \frac{\int{p_{i}^{h2}(t) dt}}{\int p_{i,max}^{h2}(t) dt}
% \end{align}

% If we expand the integral, knowing $W=\int{p\ dt}=\int{\tau\dot{q}\ dt}=\tau \Delta q$:
% \begin{align}
%     \sum_{i=0}^{\mathit{N}} \frac{\tau^{h1}_{i}\Delta q_i}{\tau^{h1}_{i,max}\Delta q_i} &= 
%     \sum_{i=0}^{\mathit{N}} \frac{\tau^{h2}_{i}\Delta q_i}{\tau^{h2}_{i,max}\Delta q_i}
% \end{align}

% Since change in joint angle $\Delta\dot{q}$ is the same across embodiment, it can be canceled out from the denominator of both sides. And the numerator $\tau^{h1}_{i}\Delta q_i = W_i$ is really just the work done by the joint i. In a quasi-static situation it approximately equal to the gravitational energy gained $W_i \approx m_i g h_i$
% \begin{align}
%     \sum_{i=0}^{\mathit{N}} \frac{W_i^{h1}}{\tau^{h1}_{i,max}} &= 
%     \sum_{i=0}^{\mathit{N}} \frac{W_i^{h2}}{\tau^{h2}_{i,max}}
% \end{align}

% To make the equation such that the power factor value increases when the utilized torque ratio decreases, we flip the fraction. And we rearrange the equation knowing $\sum_{i=0}^{\mathit{N}}W_i = W \approx mg\Delta h$

% \begin{align}
%      \frac{\sum_{i=0}^{\mathit{N}}\tau^{h1}_{i,max}}{m^{h1}g\Delta h^{h1}} &= 
%      \frac{\sum_{i=0}^{\mathit{N}}\tau^{h2}_{i,max}}{m^{h2}g\Delta h^{h2}}
% \end{align}

% Since $\Delta h$ is proportional to the height of the humanoid, therefore for the purpose of power factor, we can use height and cancel out the constants on both sides. This leaves us the final, dimensionless, power factor formulation to consistently compare the performance of different humanoids. In the previous sections we also calculated the human power factor as a baseline.

% \begin{equation}
%     \hat{P} = \frac{\sum_{i=0}^{\mathit{N}} \bm{\tau}_{\max}^{i}}{\mathrm{h} \cdot \mathrm{m} g},
% \end{equation}

% In summary, our proposed $\hat{\mathit{P}}$ naturally incorporates the crucial physical parameters: mass, gravity, and height, to give a fair measure of motion capability across humanoids of very different scales. A higher $\hat{\mathit{P}}$ means the humanoid have excess torque capability to achieve more dynamic motion compare to its own size and weight. Achieving or surpassing the human threshold $\hat{\mathit{P}}_{\text{human}}$ is indicative of human-like dynamic potential, although practical constraints (e.g., battery efficiency, control complexity, safety) may set an upper bound on how large $\hat{\mathit{P}}$ should be in a real-world humanoid design.







% \subsection{Power Factor}
% \label{sec:power_metric}

% The power factor $\hat{\mathit{P}}$ in Equation~\ref{eq:performance} is derived from peak power considerations. The instantaneous mechanical power at a joint can be written as:

% \begin{equation}
%     P = \sum_{i=0}^{\mathit{N}} \bm{\tau}_{\max}^{i} \cdot \bm{\dot{\mathrm{q}}}_{i}.
% \end{equation}
% where $\bm{\tau}$ is the joint torque and $\bm{\dot{\mathrm{q}}}$ is the joint angular velocity. 

% % We started with using Power Density as our initial metric to compare the humanoids.
% % \begin{equation}
% %     PD = \frac{\bm{\tau} \cdot \bm{\dot{\mathrm{q}}}}{m}, 
% % \end{equation}

% Many humanoids exhibit comparable ranges of joint angular velocities across different scales. Thus, the total torque capacity often dictates how much overall power a humanoid can produce, allowing us to omit $\bm{\dot{\mathrm{q}}}$ from our factor. This simplifies our initial power factor to $\sum_{i=0}^{\mathit{N}} \bm{\tau}_{\max}^{i}$.

% % \textbf{Relation to Force and Velocity.}
% % \haochen{How about focusing on two points: denominator and numerator.}
% % \textbf{Dimensionless Normalization.}
% % From a translational standpoint, power can also be expressed as:

% % \begin{equation}
% %     P = F \cdot \mathrm{v} = \mathrm{m} \cdot \mathrm{a} \cdot \mathrm{v},
% % \end{equation}

% % % implying

% % % \begin{equation}
% % %     Power = m*a*V, 
% % % \end{equation}

% % A humanoid’s ability to accelerate ($\mathrm{a}$) and its characteristic velocity ($\mathrm{v}$) define how energetically it can move. However, velocity and acceleration become less directly comparable across humanoids with vastly different heights. For example, a small $50~\mathrm{cm}$ humanoid moving at $3~\mathrm{m/s}$ is not equivalent (in scale-normalized terms) to a $180~\mathrm{cm}$ humanoid moving at the same speed.

% Joint torque $\bm{\tau}$ is the product of force $F$ and moment arm $r$, meaning both force and lever length affect power output. Since joint torque typically overcomes weight in humanoid motion, we normalize the factor by $\mathrm{m}g$ (weight) and $\mathrm{h}$ (characteristic length), since individual link lengths of a humanoid are proportional to its scale. This yields a dimensionless power factor:

% \begin{equation}
%     \hat{P} = \frac{\sum_{i=0}^{\mathit{N}} \bm{\tau}_{\max}^{i}}{\mathrm{h} \cdot \mathrm{m} g},
% \end{equation}

% This factor serves as a dimensionless comparison of how quickly a humanoid can impart energy to its own body relative to its size. For example, jumping $40~\mathrm{cm}$ for a toddler-sized ($55~\mathrm{cm}$) humanoid can be much more challenging than jumping $40~\mathrm{cm}$ for a full-scale humanoid—but in absolute scales, these two jumps are considered equivalent. By including $\mathrm{h}$ in the denominator, we ensure that motion is scaled appropriately to each humanoid’s characteristic height.

% In summary, our proposed $\hat{\mathit{P}}$ naturally incorporates the crucial physical parameters: mass, gravity, and height, to give a fair measure of motion capability across humanoids of very different scales. A higher $\hat{\mathit{P}}$ means the humanoid can achieve more dynamic motion compare to its own size and weight. Achieving or surpassing the human threshold $\hat{\mathit{P}}_{\text{human}}$ is indicative of human-like dynamic potential, although practical constraints (e.g., battery efficiency, control complexity, safety) may set an upper bound on how large $\hat{\mathit{P}}$ should be in a real-world humanoid design.

% % However, note that it is not normalized to the scale and weight difference.



% always bold face vectors, make it lowercase
% scalar, italic, not boldface
% matrix, capital, italic (must) boldface (maybe)
% set, mathcal
% index, subscript, ijk, limit,sum, t- T, i-N, j-M, k-K, 
% functions, f,g,h
% no unnecessary sub,super scripts, be as concise as possible
% units, SI
% s,a, state action, q, configuration, for range - 1;t, 
% hat and tilda, ground truth is bar, hat is estimated one, tilda is variation of something, flexible, don't use prime, it's reserved for derivative.
% multiplication, if it's clear, don't do anyting, if not, put a cdot.