\subsection{Motor Selection}
\label{sec:motor_selection}

Given \systems size constraint and 30-DoF design, Brushless Direct Drive (BLDC) motors are not a viable option. As BLDC motors shrink, their winding thickness decreases, reducing current capacity and torque constant. Despite their high power density, they still require a high-ratio gearbox, making them less suitable given our limited space budget.
We initially explored electric linear actuators but found them unsuitable due to insufficient power density and low control frequency. 
After a few iterations, we narrowed our choices to servo motors, ultimately selecting Dynamixel motors for their desirable performance and well-documented support. Given that reproducibility is a hard constraint in our system, we believe a pure-Dynamixel design is the most feasible to reproduce, especially for those with limited hardware experience.

\begin{figure}
  \centering
  \includegraphics[width=\linewidth]{figures/pcb.pdf}
  \caption{\textbf{Power Distribution.} We show the power distribution board design, including four XT30 power plugs, an Estop terminal block, seven JST EH TTL communication outlets, and two $12\mathrm{V}$ step-down convertors.}
  \label{fig:pcb}
  \vspace{-3mm}
\end{figure}





% Battery, torque, power, weight co-optimization
For the Dynamixel motors, the smallest units start at approximately $50~\mathrm{g}$. This allows us to estimate the total weight as $3100= 30\times50~(\text{motors}) + 600~(\text{computer, battery, camera}) + 1000~(\text{3D-printed structure and metal hardware})~\mathrm{g}$.

To estimate the torque required for each joint to achieve human-like motions, we followed the reference values from \cite{grimmer2020human} and Equation~\ref{eq:performance} to derive the torque estimation:
% \begin{align}
%     P_{human} &= P_{humanoid}, \\
%     \frac{\bm{\tau}_{human}}{h_{human} W_{human}} &= \frac{\bm{\tau}_{humanoid}}{h_{humanoid} W_{humanoid}}, \\
%     \bm{\tau}_{\text{robot}} &= \cfrac{\mathrm{h}_{\text{robot}} \cdot \mathrm{m}_{\text{robot}}}{\mathrm{h}_{\text{human}} \cdot \mathrm{m}_{\text{human}}}\bm{\tau}_{\text{human}}.
% \end{align}

\begin{equation}
    \bm{\tau}_{\text{robot}} = \frac{\mathrm{h}_{\text{robot}} \cdot \mathrm{m}_{\text{robot}}}{\mathrm{h}_{\text{human}} \cdot \mathrm{m}_{\text{human}}} \cdot \bm{\tau}_{\text{human}}.
\end{equation}

With an estimated height of $0.5~\mathrm{m}$ and weight of $3.1~\mathrm{kg}$, the required lower limb torque for \system to perform the most demanding tasks—assuming an optimal control policy—such as running and slope climbing~\citep{grimmer2020human}, is estimated as follows: ${\tau}_{\text{robot}}^{\text{knee}} = 2.35~\mathrm{Nm},\space  
{\tau}_{\text{robot}}^{\text{ankle\ pitch}} = 2.66~\mathrm{Nm},\space  
{\tau}_{\text{robot}}^{\text{hip\ pitch}} = 1.77~\mathrm{Nm}.$

As shown in Table~\ref{tab:dynamixel}, XM430 is the only option that provides sufficient torque for the knee and ankle pitch joints. Its metal gears, low backdrive resistance, and high torque output make it ideal for these joints which directly impact walking stability.
For the hip, we use 2XC430s for roll and pitch, as they provide sufficient torque while maintaining a compact design, integrating two actuated DoFs in a single housing. This setup allows for a greater range of motion compared to placing two XC430s sequentially.
The XC330 series is the smallest and most cost-effective in the lineup but has higher tracking errors and backlash. Therefore, we use XC330s on joints with lower torque demands or strict weight and space constraints, such as the neck, waist, hip yaw, and parallel jaw gripper. 
2XL430, a lower-cost variant of the 2XC430, is used in the arm to balance performance and cost, ensuring minimal performance loss while maximizing affordability.
For the remaining joints, including shoulder pitch and ankle roll, XC430 is used as a standard choice.

% It would be nice if we considered the speed required and motor torque map as well...

% Battery life
% We were able to find a suitable motor in the XL and XC series which is also cost optimized, for the dofs that is critical to locomotion stability such as knee and ankle, we went with XM series to maximize the torque redundancy. 