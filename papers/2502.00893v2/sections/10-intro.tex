\section{Introduction}
\label{sec:intro}

\begin{table*}[t]
\centering
\resizebox{\textwidth}{!}{
\begin{threeparttable}
\caption{Comparison with Other Humanoid Research Platforms.}
\label{tab:comparison}
% \footnotesize
\setlength{\tabcolsep}{5pt}

\begin{tabular}{@{}l|ccccccccc@{}}
\toprule
\textbf{Humanoid} & \textbf{Size(m)} & \textbf{Weight(kg)} & \textbf{\# Active DoFs}$^{(a)}$ & \textbf{Manipulation} & \textbf{Locomotion} & \textbf{Sim Data} & \textbf{Real Data} & \textbf{Open Source} & \textbf{Price(\$)} \\ 
\midrule
BD Atlas~\citep{atlas}      & 1.50 & 89.0 & 28 & \tick & \tick & -   & -   & \cross        & -     \\ 
Berkeley~\citep{liao2024berkeley}      & 0.85 & 16.0 & 12 & \cross  & \tick & \tick & \cross  & code  & 10K   \\ 
Booster T1~\citep{humanoid} & 1.18 & 30.0 &	23 & \tick  & \tick & \tick & \cross  & code  & 34K \\
BRUCE~\citep{liu2022design}         & 0.70 & 4.8  & 16 & \cross  & \tick & \tick & \cross  & code  & 6.5K  \\ 
Cassie~\citep{cassie}        & 1.15 & 35.0 & 10 & \cross  & \tick & \tick & \cross  & code  & 250K  \\ 
Digit~\citep{agility}         & 1.75 & 65.0 & 16 & \tick & \tick & \tick & \cross  & code  & 250K  \\ 
Duke~\citep{xia2024duke}  & 1.00 & 30.0 & 10 & \cross & \tick & \tick & \cross  & everything  & 16K  \\ 
Figure~\citep{figure}        & 1.68 & 70.0 & 26 & \tick & \tick & -   & -   & \cross        & -     \\ 
Fourier GR1~\citep{fourierrobotics}   & 1.65 & 55.0 & 32 & \tick & \tick & -   & -   & \cross        & 110K  \\ 
iCub~\citep{parmiggiani2012design}          & 1.04 & 24.0 & 32 & \tick & \tick & \tick & \cross  & code  & 300K  \\ 
MIT~\citep{chignoli2021mit}           & 1.04 & 24.0 & 18 & \cross  & \tick & -   & -   & \cross        & -     \\ 
NAO H25~\citep{nao}       & 0.57 & 5.2  & 23 & \tick & \tick & \tick & \cross  & code  & 14K   \\ 
Optimus~\citep{ai}       & 1.73 & 57.0 & 28 & \tick & \tick & -   & -   & \cross        & -     \\ 
Robotis OP3~\citep{nameintroduction}   & 0.51 & 3.5  & 20 & \cross  & \tick & \tick & \cross  & code  & 11K   \\ 
Unitree G1~\citep{unitreea}    & 1.32 & 35.0 & 29 & \tick & \tick & \tick & \tick & code  & 57K   \\ 
Unitree H1~\citep{unitree}    & 1.76 & 47.0 & 19 & \tick & \tick & \tick & \tick & code  & 70K   \\ 
\midrule
Ours                   & 0.56 & 3.4  & 30 & \tick & \tick & \tick & \tick & everything$^{(b)}$ & 6K    \\ 
Average Adult~\citep{grimmer2020human} & 1.73 & 70.9 & 32$^{(c)}$ & \tick & \tick & -   & -   & -        & -     \\ 
\bottomrule
\end{tabular}

\begin{tablenotes}
% \footnotesize
\item[{(a)}] The active degrees of freedom actuated by motors, excluding end effectors such as parallel grippers or dexterous hands.
\item[{(b)}] Everything includes the digital twin, learning algorithms, hardware design, assembly manual, detailed documentation, and tutorials.  
\item[{(c)}] While human body is powered by over 600 muscles in reality, the primary functional movements of the human body can be approximated using 32 revolute joints: six DoFs per leg, seven DoFs per arm, three DoFs for the waist, and three DoFs for the neck, excluding fingers and toes.
\end{tablenotes}

\end{threeparttable}
}
\vspace{-0.5cm}
\end{table*}


Conventional robot design prioritizes factors such as actuator strength, sensor accuracy, mechanical precision, and repeatability---key objectives for developing robust control algorithms. However, these platforms are not inherently aligned with modern robot learning paradigms driven by embodied data. A robot platform compatible with a machine learning (ML) approach must possess the innate ability to collect observation and action data seamlessly, both in simulation and in the real world, as these complementary data sources are essential for scalable policy learning. Simulation enables rapid and scalable data collection but relies heavily on accurate physics models. In contrast, real-world data tend to be more reliable but are often difficult to scale due to cost and safety concerns.

While some recent quadrupeds~\cite{katz2019mini, kau2022stanford} and robotic manipulators~\cite{zhao2023learning, wu2023gello,shaw2023leap, romero2024eyesight, bhirangi2023all} have been designed with ML-compatibility in mind, we introduce \textbf{\system, a humanoid robot platform} for robotics and AI research community (Figure~\ref{fig:teaser}), specifically developed to facilitate policy learning for both locomotion and manipulation skills. \system is designed to maximize ML-compatibility, capability, and reproducibility, while minimizing the expertise and costs required for its construction and maintenance.

To address the need for ML-compatibility, \system is designed not only to execute policies but also to serve as a robust data collection platform. It enables the acquisition of \textbf{high-quality simulation data} through a plug-and-play zero-point calibration procedure and transferable motor system identification (sysID) results. These tools ensure a high-fidelity digital twin without the need for additional tuning. We validate the quality of simulation data with both keyframe-interpolated motions (e.g., push-ups and pull-ups) and reinforcement learning (RL) policies (e.g., walking and turning), demonstrating the capability for zero-shot sim-to-real transfer. \system can also acquire \textbf{scalable real-world data}. We design an intuitive teleoperation interface that allows simultaneous control of \systems upper and lower body to collect whole-body manipulation data and develop effective visuomotor policies. Additionally, \systems small size and weight ($0.56~\mathrm{m}, 3.4~\mathrm{kg}$) ensure safe and accessible operation in real-world environments.

Beyond ML-compatibility, \system is designed with a focus on capability and reproducibility. A humanoid has the potential to utilize scalable human demonstrations by leveraging its anatomical similarity to the human body. As such, \system features an anthropomorphic design with 30 active degrees of freedom (excluding end effectors), powered by carefully selected motors comparable to human muscle strengths normalized by the body size. We demonstrate \systems capabilities through arm span, payload, and endurance tests and various open-loop and closed-loop loco-manipulation tasks.

Reproducibility is achieved through low-cost, open-source designs and readily accessible hardware components. \system uses commercially available motors and is completely 3D-printed, with a total cost under $6,000~\mathrm{USD}$ ($90\%$ of the cost is for motors and computers).  We will release digital twin software, learning algorithms, hardware designs, and comprehensive tutorials to ensure that \system can be built at home with basic knowledge of hardware and software, without requiring specialized equipment for manufacturing or repair. To validate reproducibility, we enlisted a CS-major student who is not involved in this project to independently build another instance of \system with the provided assembly manual and successfully zero-shot transferred loco-manipulation policies between the two instances. Finally, we showcase a collaborative long-horizon task in which two \system robots work together to tidy a room by organizing stuffed toys from a table and the floor.



% Learning-based robotic systems heavily depend on the quality and quantity of data they are trained on.  Robotics researchers typically have two options for data collection—simulation or the real world. Simulation data is scalable and efficient, but its quality relies on accurate digital twins, often requiring extensive calibration and system identification (sysID). In contrast, real-world data are generally of higher quality but come with significant challenges in scaling due to the cost, complexity, and safety concerns associated with robotic systems. Therefore, to advance machine learning in robotics, we need robotic systems capable of collecting high-quality simulation data and scalable real-world data, which we define as \textit{ML-compatible} robots.

 
% A practical evaluation criterion is whether the entire bill of materials (BOM) is available from online vendors and whether individuals can assemble the robot at home without specialized equipment. 


% The different need call for different design
% Machine learning has become a powerful tool to unlock new robot capabilities, yet research efforts have largely focused on developing new algorithms for existing industrial hardware. However, these robots, optimized for open-loop repetitive tasks, may not meet the needs of modern learning systems. 

% For example, with closed-loop sensory feedback, absolute mechanical precision and repeatability become less critical for high-accuracy tasks. 
% On the other hand, learning algorithms impose new requirements on the system. An \textbf{ML-compatible} robotic system must also provide effective data collection mechanisms for policy learning, both in simulated and real-world environments.
%such as the ability to efficiently collect large amounts of high-quality training data for diverse tasks in both simulated and real-world environments, which we define as ``\textbf{ML-compatibility}''. 

% Why humanoid? 
% Recently, various robot embodiments have demonstrated ML-compatible designs, such as quadrupeds~\cite{katz2019mini, kau2022stanford}, arms~\cite{zhao2023learning, wu2023gello}, and dexterous hands~\cite{shaw2023leap, romero2024eyesight, bhirangi2023all}. In this paper, we introduce \textbf{\system}, an ML-compatible humanoid platform for the research community. \system facilitates efficient data collection in both simulation and the real world, as each domain offers distinct advantages and limitations. While simulation enables rapid, scalable data collection, it relies on accurate physics models that are particularly challenging to obtain for diverse manipulation tasks. In contrast, real-world data are generally of higher quality but often difficult to scale due to cost and safety concerns.


% To acquire \textbf{high-quality simulation data}, we provide a plug-and-play zero-point calibration procedure and transferrable motor system identification (sysID) results \karen{what does transferrable sysID results entail?}. These ensure a high-fidelity digital twin and eliminate the need for additional tuning. We validate the quality of simulation data on both keyframe-interpolated motions (e.g., push-ups and pull-ups) and RL policies (e.g., walking and turning), showing the capability to zero-shot sim2real transfer.

% To obtain \textbf{scalable real-world data}, we design an intuitive teleoperation interface to simultaneously control the upper and lower body of \system to collect whole-body manipulation data 
% This interface allows us to collect 60 trajectories in 20 minutes. 
% \ken{maybe time in seconds or percentage of effective collection is better description} 
% and learn effective visuomotor policies. Additionally, we design the robot to have a small height and weight ($0.56~\mathrm{m}, 3.4~\mathrm{kg}$) to ensure safe operations in the real world.

%However, designing ML-compatible humanoids remains challenging due to their complexity. In this paper, we focus on bridging this gap by developing \textbf{\system}, an ML-compatible humanoid platform for research. 

% We recognize that data collection in both simulated and real-world environments is critical, as each domain offers distinct advantages and limitations -- while simulation enables rapid, scalable data collection, it relies on accurate physics models that are particularly challenging to obtain for diverse manipulation tasks. In contrast, real-world data are generally of higher quality but often difficult to scale due to cost and safety concerns. Hence, we designed \system to facilitate efficient data collection in both domains:
% \begin{itemize}[leftmargin=3.5mm]
%     \item To provide \textbf{high-quality simulation data}, we provide a plug-and-play calibration procedure and transferrable motor system identification(sysID) results to ensure a high-fidelity digital twin, removing the need for additional sysID. We validated the quality of simulation data on both keyframe-interpolated motions (e.g., push-ups and pull-ups) and RL policies (e.g., walking and turning), showing the capability to zero-shot sim2real transfer.
    
%     \item To obtain \textbf{scalable real-world data}, we designed an intuitive teleoperation interface to simultaneously control both upper and lower body of \system to collect whole-body manipulation data. This interface allows us to collect over 100 trajectories within 30 minutes \ken{maybe time in seconds or percentage of effective collection is better description} and learn effective visuomotor policies. Additionally, we designed the robot to have a small height and weight (0.56 m, 3.4 kg) to ensure safe operations in the real world. % Additionally, the robot needs to be safe to deploy to make real-world learning possible. With a height of 0.56 m and a weight of 3.4 kg, \system reduces the risk of damage to itself and its environment, ensuring safe data collection in real-world environments.
% \end{itemize}

% In addition to ML-compatibility, our system design also considers \textbf{capability} and \textbf{reproducibility}. 
% % Capability refers to the hardware's adaptability to diverse tasks and its potential to tackle increasingly challenging problems.  
% Humanoid capability typically depends on similarities with human anatomy, such as the number of active degrees of freedom (DoFs) and motor performance. \system features an anthropomorphic design with 30 active DoFs (excluding end effectors), powered by carefully chosen motors. We showcase \systems capability through a whole-body payload test and various open-loop and closed-loop loco-manipulation tasks.

% Reproducibility involves low-cost, open-source designs with accessible hardware components. We keep the total cost under 6,000 USD and will release the digital twin, learning algorithms, hardware design, and detailed tutorials. We ensure that \system can be built at home with basic knowledge of hardware and software, without the need for specialized equipment to manufacture or repair. To validate that \system can transfer between instances, we recruit another CS Ph.D. student to build a second instance based on the assembly manual and zero-shot transfer loco-manipulation policies between instances. Lastly, we showcase a collaborative long-horizon task where two instances of \system work together to tidy up a room with stuffed toys on the table and the ground.


% \shuran{I feel we need to restructure the following paragraph. right now, this feels like you are redefining ML-compatibility again. 
% however, capability and reproducibility are requirements outside ML compatibility. 
% safety can be added to real-world data collection paragraph. To make real-world learning possible the robot needs to be safe to deploy. 
% }  

% \shuran{Maybe start a new paragraph. 
% In addition to ML-compatibility,  our system design also considers capability and reproducibility ... 
% %
% Talk about capability ... 
% %
% Move reproducibility here: 
% Moreover, to validate hardware reliability and scalability, we built a second instance and zero-shot transferred policies between them. Finally, we showcased \systems robustness through a long-horizon task including toy-organizing and cart-pushing, which required chaining multiple loco-manipulation skills.
% } 

% Notes: 
% How we enable sim data collection?  
% - high-fidelity digital twin
% - automatic sysID 
% allows RL training for locomotion tasks

% How we enable realword data collection?   
% - safe and robust
% - teleoperation interface 
% allows imitation learning for manipulation tasks that involve hard-to-simulation objects, e.g., soft/articulated objects. 

% The value of robotics research usually lies not in its immediate commercial viability but in its contribution to knowledge and proof of novel concepts. This reality calls into question the necessity of relying on costly and specialized commercial robots for research. 
% 
% Moreover, the emergence of robot learning methods has highlighted key differences between industry and research objectives for robot hardware. For example, while the industry prioritizes long-lasting systems and specialized hardware, the research community values fast iteration, rapid prototyping, and modular designs adaptable to many tasks. These diverging needs demand new robot platforms tailored for real-world embodied AI research.

% The low-cost yet effective hardware has significantly accelerated the development of loco-manipulation skills, particularly within the robot learning community. For example, ALOHA~\citep{zhao2023learning} facilitates rapid hardware iterations~\citep{aldacoaloha, fu2024mobile} and introduces new capabilities, such as tying shoes, previously deemed extremely challenging~\citep{tanstabilizing}. 

% By analyzing past robot designs for scalable data collection in the real world, we emphasize three key design decisions to ensure ML-compatibility: capability, safety, and reproducibility. 