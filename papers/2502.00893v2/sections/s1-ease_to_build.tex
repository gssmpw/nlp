% \subsection{Preliminary Sizing Methodology}
% \label{sec:prelim_sizing}

% We formulate humanoid design as a multi-objectives optimization problem. In the preliminary design stage, we seek to use a first-order analysis, such as the classical weighted-sum method, to pin down a rough spec to aim for quickly. With the high-level goals of ensuring ML-compatibility, capability, and reproducibility, we define our design objectives and constraints as follows:
% \begin{equation}
% \begin{array}{rl}
% \underset{x}{\text{maximize}}  & \alpha \mathit{N}(x) + \beta \mathit{P}(x) - \gamma \mathit{C}(x)\\[1.5ex]
% \text{subject to} & x \in \mathcal{X} \\
%                   & \mathit{N}(x) \leq \mathit{N}_{\text{human}},\space \mathit{N}(x) \in \mathbb{Z}^+ \\
%                   & \mathit{P}(x) > \mathit{P}_{\text{human}} \\
%                   & \mathit{C}(x) > 0 \\
%                   & \alpha, \beta, \gamma > 0
% \end{array}
% \end{equation}
% $x$ represents the humanoid design, $\mathit{C}(x)$ denotes the cost of the humanoid, and $\mathcal{X}$ is the feasible design space. We seek to find a Pareto optimum within the given constraints.
% There are two key questions to solve this optimization: 
% \begin{enumerate}[label=\textbf{Q\arabic*}:]
%     \item What are the relative scales of $\alpha$, $\beta$, and $\gamma$?
%     \item How do we prune $\mathcal{X}$?
% \end{enumerate}

% Our answer to \textbf{Q1} is $\alpha \gg \gamma \approx \beta$. We consider the number of active DoFs to be the most critical factor for achieving human-like motion. With this guarantee, achieving high performance while minimizing cost is the key objective. As outlined in Equation~\ref{eq:performance}, maximizing performance entails minimizing height and weight while maximizing torque at each joint. These are the underlying principles of our mechanical design and motor selection.

% To address \textbf{Q2}, we prioritize reproducibility, which essentially requires all components to be 3D-printable or standardized products that are easily accessible. 

% Reachability analysis (assembly feasibility)
\subsection{Ease of Assembly and Maintenance}
\label{sec:ease_to_build}
The ease of assembly and maintenance is crucial yet difficult to specify, as it requires mentally simulating the assembly and disassembly process. In early iterations, we explicitly selected screw types and ensured unobstructed tool access, which allowed a clear assembly direction for the screwdriver. We also prioritized modular design, allowing individual parts to be removed independently. These considerations significantly improve maintainability and simplify repairs.
