\subsection{Transmission Mechanisms}
\label{sec:transmission}

Placing motors directly at the joint is often impractical. With carefully designed transmission mechanisms, motors can be relocated outside the interference zone, amplify torque output, and offload mechanical stress to the structure. This section highlights key design features, including spur gears, coupled bevel gears, and parallel linkages, as shown in Figure~\ref{fig:transmission}. 
% We prototyped a lot more mechanisms, and these are the ones to stay in the final design.

Each transmission type offers unique benefits. To start with, spur gears provide three advantages:
% We find module 1, 3D-printed PLA spur gears strong enough

\begin{enumerate}
    \item \textbf{Relocated joint axis:} A 1:1 spur gear set allows repositioning of the joint axis to a more convenient in-plane location without affecting the motor’s range of motion. This is widely used in \systems arm.
    \item \textbf{Torque modification:} A ratioed spur gear set adjusts the final torque output, which is particularly useful for the parallel jaw gripper.
    \item \textbf{Load distribution:} When a motor’s output shaft has significant free play, as in Dynamixel XC330, where it is supported only by a Teflon bushing, using it directly as the joint axis is undesirable. A 1:1 spur gear set enables a reinforced secondary axis with planar bearings and metal shafts to carry the load, protecting the motor from transverse forces. This approach is used in the hip yaw joints, where torque demands are low, but load-bearing capacity is critical.
\end{enumerate}


With precise tolerance tuning, 3D-printed bevel gears provide a highly interlocking design with minimal backlash, while still being structurally strong. They also offer three key advantages:
\begin{enumerate}
    \item \textbf{Rotated joint axis:} A coupled bevel gear set enables a parallel waist mechanism, where two motors in the same orientation drive two perpendicular DoFs.
    \item \textbf{Combined torque output: } On each axis, both motors contribute to the driving torque, enhancing power and efficiency. This is critical, as a single Dynamixel XC330 lacks the power to drive the entire upper body, but two motors combined are sufficient.
    \item \textbf{Compact actuation:} In the waist, where space is highly constrained, a coupled bevel gear set allows the compact integration of two DoFs. 
\end{enumerate}

\begin{figure}
  \centering
  \includegraphics[width=\linewidth]{figures/transmission.pdf}
  \caption{\textbf{Transmission Mechanisms.} We show three design primitives in \systems mechanical design: spur gears, coupled bevel gears, and parallel linkages.}
  \label{fig:transmission}
  \vspace{-1mm}
\end{figure}





\begin{table}[t]
\centering
\caption{Dynamixel Motor Assignments for \system.}
\setlength{\tabcolsep}{0pt}
\begin{threeparttable}
\begin{tabular}{@{}lcc@{}} % Set column widths as desired
\toprule
\textbf{Motor Model} & $\textbf{Stall Torque}^{(a)}$	&\textbf{Assigned DoFs} \\ \midrule
XC330-T288  & 1.0          & Neck PY$^{(b)}$, Waist RY, Hip Y, Gripper \\
XC430-T240BB  & 1.9             & Shoulder P, Ankle R \\
XM430-W210  & 3.0             & Knee P, Ankle P \\ 
2XL430-W250 & 1.5             & Shoulder RY, Elbow RY, Wrist RP \\
2XC430-W250 & 1.8          & Hip RP \\ \bottomrule
\end{tabular}
\begin{tablenotes}
% \footnotesize
\item[{(a)}] The Stall Torque data are measured at $12\mathrm{V}$ as reported on the Dynamixel official website~\citep{nameintroductiona}. The unit is $\mathrm{Nm}.$
\item[{(b)}] R, P, and Y denote roll, pitch, and yaw respectively.
\end{tablenotes}
\end{threeparttable}

\label{tab:dynamixel}
\vspace{-3mm}
\end{table}


Lastly, parallel linkages allow the motor to be positioned away from the joint axis, as seen in the knee and neck pitch. Despite a limited range of motion (usually $<160\degree$), this design is easy to assemble and efficiently transfers high torque when paired with ball bearings. They provide three key benefits:
\begin{enumerate}
    \item \textbf{Compact design:} This enables a cleaner, more compact neck design by placing the motor inside the head.
    \item \textbf{Reduced Inertia:} The knee motor is placed higher to reduce rotational inertia.
    \item \textbf{Structural Efficiency:}
    In the thigh, the knee motor is bolted to a 3D-printed structure for better load distribution, increased rigidity, and reduced weight.
\end{enumerate}

A potential drawback of these transmission mechanisms is their inaccurate simulation modeling. However, in MuJoCo~\citep{todorov2012mujoco}, we mitigate this by using joint equality constraints for spur gears, fixed tendons for coupled bevel gears, and weld constraints for parallel linkages. This approach has empirically shown a small sim2real gap, as demonstrated in Section~\ref{sec:experiments}.

