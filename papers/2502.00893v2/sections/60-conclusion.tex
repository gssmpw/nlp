\section{Limitations and Future Work}
\label{sec:conclusion}

\systems performance in more agile tasks is constrained by the off-the-shelf motors' max speed, max torque, and communication speed. Rather than achieving superhuman capabilities, \system aligns more closely with average human performance in loco-manipulation tasks. Additionally, its scale limits interaction with human-sized objects, though this does not hinder research if appropriately sized objects are used.

To overcome these limitations, we are developing customized communication boards and improving motor system identification to maximize performance. We also aim to improve sensing capabilities, including stereo vision for depth perception, additional IMUs for improved state estimation, and tactile sensors for richer feedback.


\section{Conclusions}

In conclusion, we demonstrate that \system is ML-compatible, capable, and reproducible through a series of tests and loco-manipulation tasks. While humanoid research is often associated with locomotion, \system extends beyond this to support full-body manipulation, character animation, human-robot interaction, and various ML applications, making it a versatile research platform. With \system fully open-source, we hope to empower researchers to explore new directions in humanoid research and encourage open collaboration in the community. 