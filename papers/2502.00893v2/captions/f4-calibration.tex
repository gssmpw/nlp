\begin{figure}[t]
  \centering
  \includegraphics[width=\linewidth]{figures/calibration.pdf}
  \caption{\textbf{Zero-point Calibration.} We 3D-print devices for the plug-and-play zero-point calibration procedure: orange for the arm, yellow for the neck, red for the hip, and beige for the ankle. Arrows indicate the insertion direction, and the zero-point is fixed once the devices click into place.}
  \label{fig:calibration}
  \vspace{-4mm}
\end{figure}
