\begin{table}[ht]
\centering
\caption{\system's Range of Motion vs. Human's.}
\setlength{\tabcolsep}{7pt}
\resizebox{0.8\linewidth}{!}{
\begin{threeparttable}
\begin{tabular}{@{}lcc@{}} % Adjust column width as needed
\toprule
\textbf{DoF Name} & \textbf{\system}~[\degree] & \textbf{Human}~[\degree]~\citep{kurz1991stretching} \\ \midrule
Neck pitch & $[-35, 80]^{(a)}$ & $[-55, 40]$ \\ 
Neck yaw  & $[-150, 150]$ & $[-70, 70]$ \\ 
Waist yaw  & $[-90, 90]$ & $[-45, 45]$ \\ 
Waist roll  & $[-30, 30]$ & $[-20, 20]$ \\ 
Hip pitch  & $[-90, 135]$ & $[-30, 120]$ \\ 
Hip roll  & $[-45, 45]$ & $[-30, 45]$ \\ 
Hip yaw  & $[-90, 90]$ & $[-40, 45]$ \\ 
Knee pitch  & $[0, 120]$ & $[0, 160]$ \\ 
Ankle pitch & $[-100, 45]^{(b)}$ & $[-20, 45]$ \\  
Ankle roll & $[-90, 90]$  & $[-30, 20]$ \\ 
Shoulder pitch & $[-90, 180]$  & $[-47.5, 180]$ \\  
Shoulder roll & $[-20, 90]^{(c)}$  & $[-37.5, 180]$ \\ 
Shoulder yaw & $[-150, 150]$  & $[-110, 80]$ \\  
Elbow roll & $[-110, 140]$  & $[-180, 150]$ \\ 
Elbow yaw & $[-150, 150]$  & $[-82.5, 90]$ \\ 
Wrist pitch & $[-80, 110]$  & $[-15, 40]$ \\ 
Wrist roll & $[-80, 110]$ & $[-70, 85]$ \\ 
\bottomrule
\end{tabular}

\begin{tablenotes}
% \footnotesize
\item[{(a)}] In this table, extensions, adductions, and inversions are represented by negative values, while flexions, abductions, and eversions are positive.
\item[{(b)}] The additional extension partially compensates for the absence of toes on the robot.
\item[{(c)}] Although $90\degree$ of shoulder abduction is limited compared to humans, the same hand-up pose can be achieved by actuating the shoulder pitch joint.
\end{tablenotes}
\end{threeparttable}
}
\label{tab:range_of_motion}
\vspace{-2mm}
\end{table}