\appendix

\section{Details about PII}
\label{sec:pii}
\subsection{PII Definition}
\label{sec:def}
In this section, we follow previous work by categorizing Personally Identifiable Information (PII) into the following two categories(\citealp{elliot2016anonymisation},\citealp{domingo2022database},\citealp{papadopoulou2022neural}):
\begin{itemize}
    \item \textit{Direct identifiers}: Information that can uniquely identify an individual within a dataset(e.g. name, social security number, email address, etc). 
    \item \textit{Quasi identifiers}: Information that cannot uniquely identify an individual on their own but can do so when combined with other quasi-identifiers(e.g. age, gender, occupation, etc.
\end{itemize}

Because of their high sensitivity or the potential to indirectly identify an individual, both direct and quasi-identifiers are governed by strict legal and privacy standards to ensure personal privacy.

\subsection{PII Types}
\label{sec:type}
Unlike the PII types presented by \citeposs{papadopoulou2022neural}, our classification does not include the MISC category. This exclusion is due to the ambiguous definition of the MISC category and its unclear boundaries with other categories.

The definitions of the seven categories are as follows: 

\textbf{PER}: Refers to individuals' names, including full names, aliases, and social media usernames.

\textbf{CODE}: Encompasses identifying numbers and codes like social security numbers, phone numbers, passport numbers, email addresses, etc.

\textbf{LOC}: Covers geographical locations such as home or work addresses, cities, countries, etc.

\textbf{ORG}: Pertains to the names of entities like companies, schools, public institutions, etc.

\textbf{DEM}: Represents demographic information including age, gender, nationality, occupation, education level, etc.

\textbf{DATETIME}: Indicates specific dates, times, or durations, such as birthdates, appointment times, etc.

\textbf{QUANTITY}: Refers to significant numerical data like monthly income, expenditures, loan amount, credit score, etc.

\subsection{Statistics of PII Types}
Figure~\ref{fig:pii_dist} and Table~\ref{tab:pii_stats} present the distribution of PII types across our datasets: PII-single (1,214 samples), PII-multi (1,228 samples), PII-hard (200 samples), and PII-distract (200 samples). 

\begin{itemize}
\item \textbf{Type Frequencies}: Organization (ORG) and Code-based identifiers (CODE) constitute significant portions across all datasets, with 17.09\% and 15.74\% in PII-single, and 13.47\% and 15.31\% in PII-multi, respectively. This distribution reflects the prevalence of institutional affiliations and digital identifiers in real-world scenarios.

\item \textbf{Dataset Composition}: PII-multi contains 16,136 PII entities across all categories, maintaining balanced proportions ranging from 13.47\% to 15.77\% for most types. PII-single follows a similar pattern with 9,303 entities, demonstrating consistent coverage across different PII categories.

\item \textbf{Specialized Test Sets}: PII-distract, despite comprising only 200 samples, contains 10,211 PII entities due to its multi-description design. PII-hard maintains balanced type coverage with 1,834 entities, with proportions varying from 12.10\% to 16.58\%.
\end{itemize}


\section{PII Entity Generation Methods}
\label{sec:ent_gen}
The generation of PII entities requires careful consideration of both structural constraints and semantic plausibility. We employ two complementary approaches for entity generation: rule-based generation for structured PII types and language model-based generation for context-dependent information.
\subsection{Rule-based Generation}
For PII types with well-defined formats or enumerable value sets, we implement deterministic generation methods. These methods encompass both custom rule-based algorithms and the Faker library's standardized functions. The rule-based approach is particularly effective for:

\begin{enumerate}
    \item Identification Numbers: Generating valid formats for social security numbers, passport numbers, and employee IDs while maintaining regional compliance.
    \item Contact Information: Creating syntactically correct email addresses, phone numbers, and IP addresses.
    \item Financial Data: Producing properly formatted credit card numbers, bank account numbers, and other numerical identifiers with appropriate check digits.
    \item Temporal Information: Generating dates, times, and durations within reasonable ranges and formats.
\end{enumerate}

\begin{figure}[t]
  \includegraphics[width=\columnwidth]{fig/pii_dist.pdf}
  \caption{Distribution of PII types across different datasets in PII-Bench.}
  \label{fig:pii_dist}
\end{figure}

% Please add the following required packages to your document preamble:
% \usepackage{multirow}
\begin{table}[htp]
\centering
\resizebox{\columnwidth}{!}{
\begin{tabular}{ccccccccc}
\toprule
\multirow{2}{*}{\textbf{Type}} &
  \multicolumn{2}{c}{\textbf{PII\_single}} &
  \multicolumn{2}{c}{\textbf{PII\_multi}} &
  \multicolumn{2}{c}{\textbf{PII\_hard}} &
  \multicolumn{2}{c}{\textbf{PII\_distract}} \\ \cline{2-9} 
                  & \#    & \%    & \#     & \%    & \#    & \%    & \#     & \%    \\ \hline
\textbf{PER}      & 1,214 & 13.05 & 2,456  & 15.22 & 286   & 15.59 & 1,449  & 14.19 \\
\textbf{DEM}      & 1,220 & 13.12 & 2,450  & 15.18 & 286   & 15.59 & 1,467  & 14.37 \\
\textbf{CODE}     & 1,464 & 15.74 & 2,470  & 15.31 & 222   & 12.10 & 1,605  & 15.72 \\
\textbf{ORG}      & 1,590 & 17.09 & 2,544  & 13.47 & 251   & 13.69 & 1,673  & 16.38 \\
\textbf{LOC}      & 1,053 & 11.32 & 1,516  & 15.77 & 304   & 16.58 & 1,008  & 9.87  \\
\textbf{DATETIME} & 1,368 & 14.70 & 2,526  & 15.65 & 251   & 13.69 & 1,559  & 15.27 \\
\textbf{QUANTITY} & 1,394 & 14.98 & 2,174  & 9.40  & 234   & 12.76 & 1,450  & 14.20 \\ \hline
\textbf{Total}    & 9,303 & 100   & 16,136 & 100   & 1,834 & 100   & 10,211 & 100   \\ \toprule
\end{tabular}}
\caption{Detailed statistics of PII types across datasets. For each dataset, we report both the absolute count (\#) and relative percentage (\%) of each PII type.}
\label{tab:pii_stats}
\end{table}

\subsection{Language Model-based Generation}
For PII types requiring contextual understanding and real-world knowledge, we leverage large language models through carefully designed prompts. This approach is essential for generating:

\begin{enumerate}
    \item Location Information: Coherent and geographically accurate addresses, landmarks, and regional descriptions.
    \item Organizational Entities: Plausible names for educational institutions, companies, and other organizations that reflect real-world naming conventions.
    \item Demographic Attributes: Culturally appropriate and consistent demographic information, including ethnicity, nationality, and educational background.
\end{enumerate}

\subsection{Entity Categories and Generation Methods}
Table~\ref{tab:ent_gen} presents a comprehensive mapping of PII types to their respective generation methods. The table systematically categorizes 55 distinct PII entities across seven main categories: Personal Identifiers (PER), Codes and Numbers (CODE), Location Information (LOC), Organizational Affiliations (ORG), Demographic Information (DEM), Temporal Data (DATETIME), and Quantitative Values (QUANTITY).

\section{Human Evaluation Details}
\label{sec:human_eval}
The human evaluation of PII-Bench was conducted with 25 graduate students specializing in data security and privacy protection. All evaluators were pursuing their Master's or Ph.D. degrees with at least two years of research experience in privacy-preserving machine learning or data protection systems. The evaluation process consisted of three phases: preparation, evaluation, and validation.

During the preparation phase, participants attended a 4-hour training session covering PII taxonomy, recognition guidelines, and query-related detection criteria. The session included hands-on practice with representative cases from each dataset component. Participants then completed a qualification test featuring 20 diverse instances, requiring 90\% agreement with expert assessments to proceed to the formal evaluation.

During the evaluation phase, participants used our specialized platform designed for systematic PII assessment. To maintain consistent performance, we limited evaluation sessions to two hours and distributed instances across a two-week period. The platform automatically tracked assessment time and accuracy metrics while enforcing our evaluation protocol: participants first performed PII detection by marking entity spans, linking them to subjects, and assigning PII types, before proceeding to query-related detection.

Our validation process incorporated both automated and manual checks to ensure assessment quality. The platform automatically verified assessment completeness and format consistency. Cases with substantial disagreement (Fleiss' kappa < 0.6) underwent expert review by two authors with extensive experience in privacy-preserving systems. Evaluators received detailed feedback on their performance and participated in discussion sessions to resolve systematic discrepancies.

Compensation was structured to encourage both accuracy and efficiency, with a base rate of \$30 per hour and performance bonuses based on agreement with other evaluators.

\section{Experiments Details}
\subsection{Evaluation Metrics}
\label{sec:metrics}
The evaluation framework employs distinct metrics for PII detection and query-related detection tasks.

For PII detection, let $\mathcal{P} = \{p_1, ..., p_m\}$ denote the predicted subject set and $\mathcal{G} = \{g_1, ..., g_n\}$ denote the ground truth subject set. Each subject $p_i$ or $g_j$ contains a set of entity-type pairs $\{(e, t)\}$, where $e$ represents the entity span and $t$ represents its PII type.

For each subject pair $(p_i, g_j)$, we compute three types of evaluation metrics:

\noindent\textbf{1. Strict Matching:} Both entity spans and their types must match exactly:
\begin{equation}
    P_{strict}(p_i, g_j) = \frac{|E_{p_i} \cap E_{g_j}|}{|E_{p_i}|}
\end{equation}
\begin{equation}
    R_{strict}(p_i, g_j) = \frac{|E_{p_i} \cap E_{g_j}|}{|E_{g_j}|}
\end{equation}
\begin{equation}
    F1_{strict}(p_i, g_j) = \frac{2 \cdot P_{strict}(p_i, g_j) \cdot R_{strict}(p_i, g_j)}{P_{strict}(p_i, g_j) + R_{strict}(p_i, g_j)}
\end{equation}
where $E_{p_i}$ and $E_{g_j}$ are the sets of entity-type pairs.

\noindent\textbf{2. Entity-only Matching:} Only entity spans need to match:
\begin{equation}
    P_{ent}(p_i, g_j) = \frac{|S_{p_i} \cap S_{g_j}|}{|S_{p_i}|}
\end{equation}
\begin{equation}
    R_{ent}(p_i, g_j) = \frac{|S_{p_i} \cap S_{g_j}|}{|S_{g_j}|}
\end{equation}
\begin{equation}
    F1_{ent}(p_i, g_j) = \frac{2 \cdot P_{ent}(p_i, g_j) \cdot R_{ent}(p_i, g_j)}{P_{ent}(p_i, g_j) + R_{ent}(p_i, g_j)}
\end{equation}
where $S_{p_i}$ and $S_{g_j}$ are the sets of entity spans.

The optimal subject matching $M^*$ is determined by maximizing the strict F1 score:
\begin{equation}
    M^* = \max_{M \in \mathcal{M}} \sum_{(p_i, g_j) \in M} F1_{strict}(p_i, g_j)
\end{equation}
where $\mathcal{M}$ denotes all possible one-to-one mappings between predicted and ground truth subjects.

The final recognition scores are computed over the optimal matching pairs:
\begin{equation}
    P_{strict} = \frac{1}{|\mathcal{P}|} \sum_{(p_i, g_j) \in M^*} P_{strict}(p_i, g_j)
\end{equation}
\begin{equation}
    R_{strict} = \frac{1}{|\mathcal{G}|} \sum_{(p_i, g_j) \in M^*} R_{strict}(p_i, g_j)
\end{equation}
\begin{equation}
    F1_{strict} = \frac{1}{max(\mathcal{P},\mathcal{G})} \sum_{(p_i, g_j) \in M^*} F1_{strict}(p_i, g_j)
\end{equation}

$P_{span}$, $R_{span}$, and $F1_{span}$ are computed analogously.

For query-related detection, given a predicted entity set $\mathcal{E}_p$ and ground truth set $\mathcal{E}_g$, we compute:
\begin{equation}
    P_{query} = \frac{|\mathcal{E}_p \cap \mathcal{E}_g|}{|\mathcal{E}_p|}
\end{equation}
\begin{equation}
    R_{query} = \frac{|\mathcal{E}_p \cap \mathcal{E}_g|}{|\mathcal{E}_g|}
\end{equation}
\begin{equation}
    F1_{query} = \frac{2 \cdot P_{query} \cdot R_{query}}{P_{query} + R_{query}}
\end{equation}

For both PII detection and query-related detection tasks, we additionally employ Rouge-L based fuzzy matching to handle partial matches between entity spans. Instead of using exact set intersection, the Rouge-L score is used to measure textual similarity between entities:
\begin{equation}
    P_{fuzzy} = \frac{1}{|\mathcal{E}_p|} \sum_{e_p \in \mathcal{E}_p} \max_{e_g \in \mathcal{E}_g} Rouge\text{-}L(e_p, e_g)
\end{equation}
\begin{equation}
    R_{fuzzy} = \frac{1}{|\mathcal{E}_g|} \sum_{e_g \in \mathcal{E}_g} \max_{e_p \in \mathcal{E}_p} Rouge\text{-}L(e_p, e_g)
\end{equation}
\begin{equation}
    F1_{fuzzy} = \frac{2 \cdot P_{fuzzy} \cdot R_{fuzzy}}{P_{fuzzy} + R_{fuzzy}}
\end{equation}
where $Rouge\text{-}L(e_p, e_g)$ computes the longest common subsequence-based F-score between predicted entity $e_p$ and ground truth entity $e_g$.

\subsection{Additional Results}
\label{sec:exp_res}
Table \ref{tab:multi} compares different prompting strategies on PII-multi dataset.

\begin{table*}[htp]
\centering
\resizebox{2\columnwidth}{!}{
\begin{tabular}{lcccccccccc}
\toprule
\multicolumn{1}{l}{} & \multicolumn{2}{c}{\textbf{GPT4o}} & \multicolumn{2}{c}{\textbf{Llama3.1}} & \multicolumn{2}{c}{\textbf{Qwen2.5}} & \multicolumn{2}{c}{\textbf{Llama3.1-SLM}} & \multicolumn{2}{c}{\textbf{Qwen2.5-SLM}} \\
\multicolumn{1}{l}{\multirow{-2}{*}{\textbf{Method}}} & F1 & \multicolumn{1}{c}{RougeL-F} & F1 & \multicolumn{1}{c}{RougeL-F} & F1 & \multicolumn{1}{c}{RougeL-F} & F1 & \multicolumn{1}{c}{RougeL-F} & F1 & RougeL-F \\ \hline
\multicolumn{11}{l}{\cellcolor[HTML]{EFEFEF}\textit{Basic Method}} \\ \hline
Naive & 0.600 & 0.602 & 0.611 & 0.614 & 0.596 & 0.603 & 0.240 & 0.333 & 0.405 & 0.413 \\ \hline
\multicolumn{11}{l}{\cellcolor[HTML]{EFEFEF}\textit{Advanced Method}} \\ \hline
Self-CoT & 0.675 & 0.681 & 0.638 & 0.643 & 0.626 & 0.632 & 0.354 & 0.362 & 0.392 & 0.397 \\ \hline
Auto-CoT(3-shot) & 0.629 & 0.640 & \textbf{0.650} & 0.662 & \textbf{0.657} & 0.665 & \textbf{0.393} & 0.402 & 0.391 & 0.394 \\ \hline
Self-Consistency & \textbf{0.685} & 0.692 & 0.602 & 0.605 & 0.614 & 0.620 & 0.263 & 0.269 & 0.288 & 0.293 \\ \hline
PS-CoT & 0.618 & 0.620 & 0.624 & 0.631 & 0.636 & 0.643 & 0.291 & 0.300 & \textbf{0.431} & 0.436 \\ \hline
\multicolumn{11}{l}{\cellcolor[HTML]{EFEFEF}\textit{w/ Extra Information}} \\ \hline
Naive w/ Choice & 0.846 & 0.846 & 0.775 & 0.775 & 0.804 & 0.804 & 0.387 & 0.388 & 0.743 & 0.743 \\ \bottomrule
\end{tabular}}

\caption{Performance comparison on the Query-Related PII Detection task (PII-multi dataset).}
\label{tab:multi}
\end{table*}

\subsection{Prompt Details}
\label{sec:prompt}
This section presents the prompts used throughout our experiments. For the PII detection task, we employ the template shown in Figure~\ref{fig:eval_recog}. For query-related PII detection, we design and evaluate six distinct prompting strategies. Figure~\ref{fig:basic-prompt} displays the \textbf{Naive} prompts, Figure~\ref{fig:choice-prompt} presents the \textbf{Naive w/ Choice} prompts, Figure~\ref{fig:cot-prompt} features the \textbf{Self-CoT} prompts, Figure~\ref{fig:auto-cot-prompt} reveals the \textbf{Auto-CoT} prompts, Figure~\ref{fig:self-consistency-prompt} exhibits the \textbf{Self-Consistency} prompts,and Figure~\ref{fig:plan-solve-prompt} displays the \textbf{PS-CoT} prompts.


\section{PII Annotation System}
\label{sec:annotation}
We developed a specialized web-based annotation platform to facilitate the systematic evaluation of PII detection and query-related detection capabilities. The platform implements a two-stage annotation process, ensuring comprehensive coverage of both fundamental PII entity identification and contextual relevance assessment.
\subsection{PII Detection Interface}
As shown in Figure~\ref{fig:pii_recog}, the PII detection interface enables annotators to identify and categorize PII entities within user descriptions. The interface provides the following key functionalities:
\begin{itemize}
\item Entity Detection: Annotators can highlight text spans containing PII entities directly in the user description.
\item Type Classification: Each identified entity is assigned a specific PII type (e.g., PER for person names, ORG for organizations, LOC for locations).
\item Subject Association: Entities are linked to their corresponding subjects using alphabetical identifiers (e.g., A, B) to maintain relationship clarity in multi-subject scenarios.
\item Span Verification: The interface displays start and end positions for each entity span, ensuring precise boundary detection.
\end{itemize}
\subsection{Query-Related Detection Interface}
Figure~\ref{fig:pii_understand} illustrates the interface for query-related PII detection, which builds upon the recognition results to assess contextual relevance:
\begin{itemize}
\item Query Context: The interface presents both the user description and the associated query, providing complete context for relevance assessment.
\item Entity Selection: Annotators identify PII entities crucial for addressing the query, with the interface highlighting pre-identified entities from the recognition phase.
\item Subject Verification: For selected query-related entities, annotators must verify the subject associations to ensure consistency across tasks.
\item Relevance Validation: The interface includes a review mechanism to confirm that selected entities are both necessary and sufficient for query resolution.
\end{itemize}
\subsection{Query-unrelated PII Masking Visualization}
To validate the effectiveness of privacy protection while maintaining query relevance, we implemented a masking visualization interface (Figure~\ref{fig:pii_mask}):
\begin{itemize}
\item Original Context: Displays the complete user description with all PII entities highlighted.
\item Masked View: Shows the description with non-relevant PII entities replaced by their corresponding type tags (e.g., <Nickname>, <Phone Number>).
\item Key Information Display: Preserves query-related PII entities while maintaining readability and semantic coherence.
\end{itemize}
\subsection{Annotation Guidelines and Quality Control}
To ensure annotation consistency and quality, we established comprehensive guidelines and implemented several control measures:
\begin{itemize}
\item Entity Span Guidelines: Annotators must select the minimal text span that completely captures the PII entity while maintaining semantic integrity.
\item Inter-annotator Agreement: Each sample is independently annotated by multiple annotators, with disagreements resolved through majority voting or expert review.
\item Validation Checks: The platform implements automatic validation rules to detect potential inconsistencies or missing annotations.
\item Iterative Refinement: Regular review sessions are conducted to discuss challenging cases and update guidelines based on annotator feedback.
\end{itemize}
For quality assurance, we randomly sampled 10\% of the annotations for expert review, achieving an inter-annotator agreement of 95.1\% for PII detection and 91.5\% for query-related detection across all annotators.

\begin{table*}[htp]
\centering
\small
\begin{tabular}{llll}
\toprule
\textbf{PII Type} & \textbf{Entity Category} & \textbf{Generation Approach} & \textbf{Format Constraints} \\
\midrule
\multirow{3}{*}{\textbf{PER}} 
    & Full Name & Rule-based & [First Name] [Last Name] \\
    & Social Media Handle & Rule-based & [@][a-zA-Z0-9]{5,15} \\
    & Nickname & LLM-based & - \\
\midrule
\multirow{12}{*}{\textbf{CODE}} 
    & Social Security Number & Rule-based & XXX-XX-XXXX \\
    & Driver's License & Rule-based & [A-Z][0-9]{8} \\
    & Bank Account & Rule-based & [0-9]{10,12} \\
    & Credit Card & Rule-based & [0-9]{16} \\
    & Phone Number & Rule-based & +[0-9]{1,3}-[0-9]{10} \\
    & IP Address & Rule-based & IPv4/IPv6 format \\
    & Email Address & Rule-based & [user]@[domain].[tld] \\
    & Password Hash & Rule-based & SHA-256 \\
    & Passport Number & Rule-based & [A-Z][0-9]{8} \\
    & Tax ID & Rule-based & [0-9]{9} \\
    & Employee ID & Rule-based & [A-Z]{2}[0-9]{6} \\
    & Student ID & Rule-based & [0-9]{8} \\
\midrule
\multirow{3}{*}{\textbf{LOC}} 
    & Street Address & LLM-based & - \\
    & City/Region & LLM-based & - \\
    % & Country & LLM-based & - \\
    & Landmark & LLM-based & - \\
\midrule
\multirow{5}{*}{\textbf{ORG}} 
    & Company Name & LLM-based & - \\
    & Educational Institution & LLM-based & - \\
    & Government Agency & LLM-based & - \\
    & NGO & LLM-based & - \\
    & Healthcare Facility & LLM-based & - \\
\midrule
\multirow{17}{*}{\textbf{DEM}} 
    & Occupation & Rule-based & Predefined list \\
    & Age & Rule-based & [0-9]{1,3} \\
    & Gender & Rule-based & Binary/Non-binary \\
    & Height & Rule-based & [0-9]{3}cm/[0-9]'[0-9]" \\
    & Weight & Rule-based & [0-9]{2,3}kg/lbs \\
    & Blood Type & Rule-based & A/B/O[+-] \\
    & Sexual Orientation & Rule-based & Predefined list \\
    & Nationality & LLM-based & - \\
    & Ethnicity & LLM-based & - \\
    & Race & LLM-based & - \\
    & Religious Belief & LLM-based & - \\
    & Political Affiliation & LLM-based & - \\
    & Education Level & LLM-based & - \\
    & Academic Degree & LLM-based & - \\
    & Physical Features & LLM-based & - \\
    & Medical Condition & LLM-based & - \\
    & Disability Status & LLM-based & - \\
\midrule
\multirow{3}{*}{\textbf{DATETIME}} 
    & Date & Rule-based & YYYY-MM-DD \\
    & Time & Rule-based & HH:MM:SS \\
    & Duration & Rule-based & [0-9]+[dhms] \\
\midrule
\multirow{12}{*}{\textbf{QUANTITY}} 
    & Monthly Income & Rule-based & [Currency][0-9]+ \\
    & Monthly Expenses & Rule-based & [Currency][0-9]+ \\
    & Account Balance & Rule-based & [Currency][0-9]+ \\
    & Loan Amount & Rule-based & [Currency][0-9]+ \\
    & Annual Bonus & Rule-based & [Currency][0-9]+ \\
    & Credit Limit & Rule-based & [Currency][0-9]+ \\
    & Social Security Payment & Rule-based & [Currency][0-9]+ \\
    & Tax Payment & Rule-based & [Currency][0-9]+ \\
    & Debt Ratio & Rule-based & [0-9]{1,2}.[0-9]{2}\% \\
    & Investment Return & Rule-based & [0-9]{1,2}.[0-9]{2}\% \\
    & ROI & Rule-based & [0-9]{1,2}.[0-9]{2}\% \\
    & Credit Score & Rule-based & [300-850] \\
\bottomrule
\end{tabular}
\caption{Comprehensive categorization of PII entities and their generation methods. Rule-based generation follows specific format constraints, while LLM-based generation produces contextually appropriate content without rigid formatting requirements.}
\label{tab:ent_gen}
\end{table*}

\begin{figure*}[t]
  \includegraphics[width=\linewidth]{fig/pii_recog.pdf}
  \caption{Web Demo for the PII Detection Task}
  \label{fig:pii_recog}
\end{figure*}

\begin{figure*}[t]
  \includegraphics[width=\linewidth]{fig/pii_understand.pdf}
  \caption{Web Demo for the Query-Related PII Detection Task}
  \label{fig:pii_understand}
\end{figure*}

\begin{figure*}[htp]
  \includegraphics[width=\linewidth]{fig/pii_mask.pdf}
  \caption{Web Demo for the Query-unrelated PII Masking Method}
  \label{fig:pii_mask}
\end{figure*}


\definecolor{titlecolor}{rgb}{0.9, 0.5, 0.1}
\definecolor{anscolor}{rgb}{0.2, 0.5, 0.8}
\definecolor{labelcolor}{HTML}{48a07e}
\begin{table*}[h]
	\centering
	
 % \vspace{-0.2cm}
	
	\begin{center}
		\begin{tikzpicture}[
				chatbox_inner/.style={rectangle, rounded corners, opacity=0, text opacity=1, font=\sffamily\scriptsize, text width=5in, text height=9pt, inner xsep=6pt, inner ysep=6pt},
				chatbox_prompt_inner/.style={chatbox_inner, align=flush left, xshift=0pt, text height=11pt},
				chatbox_user_inner/.style={chatbox_inner, align=flush left, xshift=0pt},
				chatbox_gpt_inner/.style={chatbox_inner, align=flush left, xshift=0pt},
				chatbox/.style={chatbox_inner, draw=black!25, fill=gray!7, opacity=1, text opacity=0},
				chatbox_prompt/.style={chatbox, align=flush left, fill=gray!1.5, draw=black!30, text height=10pt},
				chatbox_user/.style={chatbox, align=flush left},
				chatbox_gpt/.style={chatbox, align=flush left},
				chatbox2/.style={chatbox_gpt, fill=green!25},
				chatbox3/.style={chatbox_gpt, fill=red!20, draw=black!20},
				chatbox4/.style={chatbox_gpt, fill=yellow!30},
				labelbox/.style={rectangle, rounded corners, draw=black!50, font=\sffamily\scriptsize\bfseries, fill=gray!5, inner sep=3pt},
			]
											
			\node[chatbox_user] (q1) {
				\textbf{System prompt}
				\newline
				\newline
				You are a helpful and precise assistant for segmenting and labeling sentences. We would like to request your help on curating a dataset for entity-level hallucination detection.
				\newline \newline
                We will give you a machine generated biography and a list of checked facts about the biography. Each fact consists of a sentence and a label (True/False). Please do the following process. First, breaking down the biography into words. Second, by referring to the provided list of facts, merging some broken down words in the previous step to form meaningful entities. For example, ``strategic thinking'' should be one entity instead of two. Third, according to the labels in the list of facts, labeling each entity as True or False. Specifically, for facts that share a similar sentence structure (\eg, \textit{``He was born on Mach 9, 1941.''} (\texttt{True}) and \textit{``He was born in Ramos Mejia.''} (\texttt{False})), please first assign labels to entities that differ across atomic facts. For example, first labeling ``Mach 9, 1941'' (\texttt{True}) and ``Ramos Mejia'' (\texttt{False}) in the above case. For those entities that are the same across atomic facts (\eg, ``was born'') or are neutral (\eg, ``he,'' ``in,'' and ``on''), please label them as \texttt{True}. For the cases that there is no atomic fact that shares the same sentence structure, please identify the most informative entities in the sentence and label them with the same label as the atomic fact while treating the rest of the entities as \texttt{True}. In the end, output the entities and labels in the following format:
                \begin{itemize}[nosep]
                    \item Entity 1 (Label 1)
                    \item Entity 2 (Label 2)
                    \item ...
                    \item Entity N (Label N)
                \end{itemize}
                % \newline \newline
                Here are two examples:
                \newline\newline
                \textbf{[Example 1]}
                \newline
                [The start of the biography]
                \newline
                \textcolor{titlecolor}{Marianne McAndrew is an American actress and singer, born on November 21, 1942, in Cleveland, Ohio. She began her acting career in the late 1960s, appearing in various television shows and films.}
                \newline
                [The end of the biography]
                \newline \newline
                [The start of the list of checked facts]
                \newline
                \textcolor{anscolor}{[Marianne McAndrew is an American. (False); Marianne McAndrew is an actress. (True); Marianne McAndrew is a singer. (False); Marianne McAndrew was born on November 21, 1942. (False); Marianne McAndrew was born in Cleveland, Ohio. (False); She began her acting career in the late 1960s. (True); She has appeared in various television shows. (True); She has appeared in various films. (True)]}
                \newline
                [The end of the list of checked facts]
                \newline \newline
                [The start of the ideal output]
                \newline
                \textcolor{labelcolor}{[Marianne McAndrew (True); is (True); an (True); American (False); actress (True); and (True); singer (False); , (True); born (True); on (True); November 21, 1942 (False); , (True); in (True); Cleveland, Ohio (False); . (True); She (True); began (True); her (True); acting career (True); in (True); the late 1960s (True); , (True); appearing (True); in (True); various (True); television shows (True); and (True); films (True); . (True)]}
                \newline
                [The end of the ideal output]
				\newline \newline
                \textbf{[Example 2]}
                \newline
                [The start of the biography]
                \newline
                \textcolor{titlecolor}{Doug Sheehan is an American actor who was born on April 27, 1949, in Santa Monica, California. He is best known for his roles in soap operas, including his portrayal of Joe Kelly on ``General Hospital'' and Ben Gibson on ``Knots Landing.''}
                \newline
                [The end of the biography]
                \newline \newline
                [The start of the list of checked facts]
                \newline
                \textcolor{anscolor}{[Doug Sheehan is an American. (True); Doug Sheehan is an actor. (True); Doug Sheehan was born on April 27, 1949. (True); Doug Sheehan was born in Santa Monica, California. (False); He is best known for his roles in soap operas. (True); He portrayed Joe Kelly. (True); Joe Kelly was in General Hospital. (True); General Hospital is a soap opera. (True); He portrayed Ben Gibson. (True); Ben Gibson was in Knots Landing. (True); Knots Landing is a soap opera. (True)]}
                \newline
                [The end of the list of checked facts]
                \newline \newline
                [The start of the ideal output]
                \newline
                \textcolor{labelcolor}{[Doug Sheehan (True); is (True); an (True); American (True); actor (True); who (True); was born (True); on (True); April 27, 1949 (True); in (True); Santa Monica, California (False); . (True); He (True); is (True); best known (True); for (True); his roles in soap operas (True); , (True); including (True); in (True); his portrayal (True); of (True); Joe Kelly (True); on (True); ``General Hospital'' (True); and (True); Ben Gibson (True); on (True); ``Knots Landing.'' (True)]}
                \newline
                [The end of the ideal output]
				\newline \newline
				\textbf{User prompt}
				\newline
				\newline
				[The start of the biography]
				\newline
				\textcolor{magenta}{\texttt{\{BIOGRAPHY\}}}
				\newline
				[The ebd of the biography]
				\newline \newline
				[The start of the list of checked facts]
				\newline
				\textcolor{magenta}{\texttt{\{LIST OF CHECKED FACTS\}}}
				\newline
				[The end of the list of checked facts]
			};
			\node[chatbox_user_inner] (q1_text) at (q1) {
				\textbf{System prompt}
				\newline
				\newline
				You are a helpful and precise assistant for segmenting and labeling sentences. We would like to request your help on curating a dataset for entity-level hallucination detection.
				\newline \newline
                We will give you a machine generated biography and a list of checked facts about the biography. Each fact consists of a sentence and a label (True/False). Please do the following process. First, breaking down the biography into words. Second, by referring to the provided list of facts, merging some broken down words in the previous step to form meaningful entities. For example, ``strategic thinking'' should be one entity instead of two. Third, according to the labels in the list of facts, labeling each entity as True or False. Specifically, for facts that share a similar sentence structure (\eg, \textit{``He was born on Mach 9, 1941.''} (\texttt{True}) and \textit{``He was born in Ramos Mejia.''} (\texttt{False})), please first assign labels to entities that differ across atomic facts. For example, first labeling ``Mach 9, 1941'' (\texttt{True}) and ``Ramos Mejia'' (\texttt{False}) in the above case. For those entities that are the same across atomic facts (\eg, ``was born'') or are neutral (\eg, ``he,'' ``in,'' and ``on''), please label them as \texttt{True}. For the cases that there is no atomic fact that shares the same sentence structure, please identify the most informative entities in the sentence and label them with the same label as the atomic fact while treating the rest of the entities as \texttt{True}. In the end, output the entities and labels in the following format:
                \begin{itemize}[nosep]
                    \item Entity 1 (Label 1)
                    \item Entity 2 (Label 2)
                    \item ...
                    \item Entity N (Label N)
                \end{itemize}
                % \newline \newline
                Here are two examples:
                \newline\newline
                \textbf{[Example 1]}
                \newline
                [The start of the biography]
                \newline
                \textcolor{titlecolor}{Marianne McAndrew is an American actress and singer, born on November 21, 1942, in Cleveland, Ohio. She began her acting career in the late 1960s, appearing in various television shows and films.}
                \newline
                [The end of the biography]
                \newline \newline
                [The start of the list of checked facts]
                \newline
                \textcolor{anscolor}{[Marianne McAndrew is an American. (False); Marianne McAndrew is an actress. (True); Marianne McAndrew is a singer. (False); Marianne McAndrew was born on November 21, 1942. (False); Marianne McAndrew was born in Cleveland, Ohio. (False); She began her acting career in the late 1960s. (True); She has appeared in various television shows. (True); She has appeared in various films. (True)]}
                \newline
                [The end of the list of checked facts]
                \newline \newline
                [The start of the ideal output]
                \newline
                \textcolor{labelcolor}{[Marianne McAndrew (True); is (True); an (True); American (False); actress (True); and (True); singer (False); , (True); born (True); on (True); November 21, 1942 (False); , (True); in (True); Cleveland, Ohio (False); . (True); She (True); began (True); her (True); acting career (True); in (True); the late 1960s (True); , (True); appearing (True); in (True); various (True); television shows (True); and (True); films (True); . (True)]}
                \newline
                [The end of the ideal output]
				\newline \newline
                \textbf{[Example 2]}
                \newline
                [The start of the biography]
                \newline
                \textcolor{titlecolor}{Doug Sheehan is an American actor who was born on April 27, 1949, in Santa Monica, California. He is best known for his roles in soap operas, including his portrayal of Joe Kelly on ``General Hospital'' and Ben Gibson on ``Knots Landing.''}
                \newline
                [The end of the biography]
                \newline \newline
                [The start of the list of checked facts]
                \newline
                \textcolor{anscolor}{[Doug Sheehan is an American. (True); Doug Sheehan is an actor. (True); Doug Sheehan was born on April 27, 1949. (True); Doug Sheehan was born in Santa Monica, California. (False); He is best known for his roles in soap operas. (True); He portrayed Joe Kelly. (True); Joe Kelly was in General Hospital. (True); General Hospital is a soap opera. (True); He portrayed Ben Gibson. (True); Ben Gibson was in Knots Landing. (True); Knots Landing is a soap opera. (True)]}
                \newline
                [The end of the list of checked facts]
                \newline \newline
                [The start of the ideal output]
                \newline
                \textcolor{labelcolor}{[Doug Sheehan (True); is (True); an (True); American (True); actor (True); who (True); was born (True); on (True); April 27, 1949 (True); in (True); Santa Monica, California (False); . (True); He (True); is (True); best known (True); for (True); his roles in soap operas (True); , (True); including (True); in (True); his portrayal (True); of (True); Joe Kelly (True); on (True); ``General Hospital'' (True); and (True); Ben Gibson (True); on (True); ``Knots Landing.'' (True)]}
                \newline
                [The end of the ideal output]
				\newline \newline
				\textbf{User prompt}
				\newline
				\newline
				[The start of the biography]
				\newline
				\textcolor{magenta}{\texttt{\{BIOGRAPHY\}}}
				\newline
				[The ebd of the biography]
				\newline \newline
				[The start of the list of checked facts]
				\newline
				\textcolor{magenta}{\texttt{\{LIST OF CHECKED FACTS\}}}
				\newline
				[The end of the list of checked facts]
			};
		\end{tikzpicture}
        \caption{GPT-4o prompt for labeling hallucinated entities.}\label{tb:gpt-4-prompt}
	\end{center}
\vspace{-0cm}
\end{table*}