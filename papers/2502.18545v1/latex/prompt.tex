\begin{figure*}[htbp]
\begin{tcolorbox}[colback=white, colframe=black, title=Consistency Optimization Prompt of Single Subject]
You are a character feature selector tasked with identifying and refining logically consistent feature combinations.I will provide you with character features. Your role is to identify any features that have obvious logical conflicts or inconsistencies, and modify them to create a coherent set while preserving their core classifications.
\\

Requirements:\\
1. The selected character features must be logically consistent with real-world expectations, with no obvious conflicts. \\
2. When resolving conflicts, modify only the feature entities while keeping their PII types and classifications unchanged. \\
3. Modified feature entities must remain within the same PII type and classification categories as their originals. \\
4. Aim to maintain as many features as possible, ideally matching the original count or coming as close as feasible. 
\\

\#\# Character Features \\
<PII Type> <Entity Category> <PII Entity>\\
\{usr\_features\}
\\

Please provide your output in the following format: \\
- Under "\#\# Reason:", explain your selection and modification process \\
- Under "\#\# Final Features:", list the final selected features as JSON objects in the format \{\{"label": xxx, "tag": yyy, "entity": zzz\}\} with no additional content or line breaks
where xxx is the PII type, yyy is the entity category, and zzz is the PII entity.
\\

\#\# Reason: [Explain your selection and modification process] \\
\#\# Final Features: [\{\{"label": xxx, "tag": yyy, "entity": zzz\}\}, \{\{"label":xxx, "tag": yyy, "entity": zzz\}\}, ...]"""

\end{tcolorbox}
\caption{Prompt of Consistency Optimization for Single-Subject}
\end{figure*}

\begin{figure*}[htbp]
\begin{tcolorbox}[colback=white, colframe=black, title=Consistency Optimization Prompt of Multi Subject]
You are a character feature selector tasked with identifying and refining logically consistent feature combinations.I will provide you with character features for different subjects and their relationships. Your role is to: \\
1. Identify any logical conflicts or inconsistencies between features \\
2. Modify conflicting features while maintaining their PII types and categories \\
3. Ensure all features align with the given relationship between subjects \\
\\
Requirements:\\
1. Selected features must be logically consistent and align with the relationship between subjects\\
2. For relationships:\\
   - "Intersection" can indicate friends or colleagues\\
   - "Contains" can indicate parent-child relationships\\
   - "No Intersection" indicates strangers\\
3. When modifying conflicting features:\\
   - Maintain the original PII type and category\\
   - Only modify the entity value\\
   - New entity must belong to the same category\\
4. Maximize the number of selected features:\\
   - Aim to keep the original count\\
   - If not possible, get as close as possible\\

\#\# Subject A Features\\
<PII Type> <Entity Category> <PII Entity>\\
\{usr\_features\_a\}
\\

\#\# Subject B Features\\
<PII Type> <Entity Category> <PII Entity>\\
\{usr\_features\_b\}
\\

\#\# Relationship Between Subjects\\
\{rel\}
\\

Please provide your output in the following format:\\
- Under "\#\# Reason:", explain your selection and modification process \\
- Under "\#\# Final Features A:" or "\#\# Final Features B:", list the final selected features as JSON objects in the format \{\{"label": xxx, "tag": yyy, "entity": zzz\}\} with no additional content or line breaks
where xxx is the PII type, yyy is the entity category, and zzz is the PII entity.\\
\\
\#\# Reason: [Explain your selection and modification process] \\
\#\# Final Features A: [\{\{"label":xxx, "tag": yyy, "entity": zzz\}\}, \{\{"label":xxx, "tag": yyy, "entity": zzz\}\}, ...] \\
\#\# Final Features B: [\{\{"label":xxx, "tag": yyy, "entity": zzz\}\}, \{\{"label":xxx, "tag": yyy, "entity": zzz\}\}, ...]

\end{tcolorbox}
\caption{Prompt of Consistency Optimization for Multi-Subject}
\end{figure*}

\begin{figure*}[htbp]
\begin{tcolorbox}[colback=white, colframe=black, title=PII Detection Prompt]
Please identify the PII entities and their corresponding PII types for each distinct individual mentioned in the conversation transcript, including both speakers and referenced individuals. \\
The PII types are defined as follows:\\
\{pii\_definition\}\\
PII types include: ["PER","CODE","LOC","ORG","DEM","DATETIME","QUANTITY"]\\
\\
\#\# Task Description:\\
Your task is to:\\
1. Identify ALL distinct individuals mentioned in the text, including:\\
   - Primary speakers (marked with [PER\_X])\\
   - Individuals mentioned within others' statements\\
   - Referenced colleagues, family members, or associates\\
\\
2. For each identified individual, extract their associated PII entities, ensuring:\\
   - Each entity is in its smallest viable text span\\
   - Entities are correctly categorized by type\\
   - Cross-referenced information is attributed to the correct individual\\
\\
\#\# Important Rules:\\
1. Treat each individual as a separate subject, even if mentioned within another person's statement\\
2. Include both explicitly named individuals and those referenced through relationships\\
3. Maintain clear boundaries between different individuals' information\\
4. Extract exact entity spans without additional context\\
5. Preserve special characters in codes and quantities\\
6. Handle both direct mentions and indirect references\\
\\
\#\# Given conversation transcript:\\
\{user\_desc\}
\\
\#\# Required Output Format:\\
For each identified individual (both speakers and mentioned persons), output:\\
Subject \{\{N\}\} \{\{ent1: type1, ent2: type2, ...\}\}

\#\# Example:\\
Input text: "[PER\_1]: I'm Alex, working at Google. My friend Bob, who is 25 years old, works at Apple."\\
Expected output:\\
Subject \{\{1\}\} \{\{"Alex": "PER", "Google": "ORG"\}\}\\
Subject \{\{2\}\} \{\{"Bob": "PER", "25 years": "DATETIME", "Apple": "ORG"\}\}\\
\\
Begin analysis now:\\
\end{tcolorbox}
\caption{Prompt used for the PII Detection task}
\label{fig:eval_recog}
\end{figure*}

\begin{figure*}[htbp]
\begin{tcolorbox}[colback=white, colframe=black, title=Basic Query-related PII Detection Prompt]
Please identify highly relevant PII (Personally Identifiable Information) entities from the background description PII entities that directly address or relate to the user's query. \\
\\
Rules:\\
- Extract entities in their smallest possible span\\
- Exclude all person names\\
- Focus only on entities crucial for answering the query\\
- Return entities exactly as they appear in the text\\
\\
\#\#\# Background description:\\
\{desc\}\\
\#\#\# Query:\\
\{query\}\\
\\
Your output will contain the following format:\\
\#\#\# Answer: List the relevant PII entities, each enclosed in double quotes (""). Return only the list without explanation. Example: ["key\_pii\_1", ..., "key\_pii\_n"]\\
\\
Please have your output follow the format below: (if there is only one entity, please output ["key\_pii\_1"]):\\
\#\#\# Answer: ["key\_pii\_1", ..., "key\_pii\_n"]\\
\end{tcolorbox}
\caption{Prompt of Naive Method}
\label{fig:basic-prompt}
\end{figure*}

\begin{figure*}[htbp]
\begin{tcolorbox}[colback=white, colframe=black, title=Choice-Based Query-related PII Detection Prompt]
From the following options, Please identify highly relevant PII (Personally Identifiable Information) entities from the background description PII entities that directly address or relate to the user's query.\\
\\
Rules:\\
- Extract entities in their smallest possible span\\
- Exclude all person names\\
- Focus only on entities crucial for answering the query\\
- Return entities exactly as they appear in the text\\
- Select only from the provided options\\

\#\#\# Background description:\\
\{desc\}\\
\#\#\# Query:\\
\{query\}\\
\#\# Options:\\
\{choices\}\\
\\
Your output will contain the following format:\\
\#\#\# Answer: List the relevant PII entities, each enclosed in double quotes (""). Return only the list without explanation. Example: ["key\_pii\_1", ..., "key\_pii\_n"]\\
\\
Please have your output follow the format below: (if there is only one entity, please output ["key\_pii\_1"]):\\
\#\#\# Answer: ["key\_pii\_1", ..., "key\_pii\_n"]
\end{tcolorbox}
\caption{Prompt of Naive /w Choice Method}
\label{fig:choice-prompt}
\end{figure*}

\begin{figure*}[htbp]
\begin{tcolorbox}[colback=white, colframe=black, title=Chain-of-Thought Query-related PII Detection Prompt]
Please identify highly relevant PII (Personally Identifiable Information) entities from the background description PII entities that directly address or relate to the user's query.\\
\\
Rules:\\
- Extract entities in their smallest possible span\\
- Exclude all person names\\
- Focus only on entities crucial for answering the query\\
- Return entities exactly as they appear in the text\\
\\
\#\#\# Background description:\\
\{desc\}\\
\#\#\# Query:\\
\{query\}\\
\\
Your output will contain the following format:\\
\#\#\# Thought: Explain your reasoning step by step for selecting the relevant PII entities.\\
\#\#\# Answer: List the relevant PII entities, each enclosed in double quotes (""). Return only the list without explanation. Example: ["key\_pii\_1", ..., "key\_pii\_n"]\\
\\
Please have your output follow the format below: (if there is only one entity, please output ["key\_pii\_1"]):\\
\#\#\# Thought: xxx\\
\#\#\# Answer: ["key\_pii\_1", ..., "key\_pii\_n"]\\
\end{tcolorbox}
\caption{Prompt of Self-CoT Method}
\label{fig:cot-prompt}
\end{figure*}

\begin{figure*}[htbp]
\begin{tcolorbox}[colback=white, colframe=black, title=Auto Chain-of-Thought Query-related PII Detection Prompt with Examples]
Please identify highly relevant PII (Personally Identifiable Information) entities from the background description PII entities that directly address or relate to the user's query.\\
\\
Rules:\\
- Extract entities in their smallest possible span\\
- Exclude all person names\\
- Focus only on entities crucial for answering the query\\
- Return entities exactly as they appear in the text\\
\\
\#\#\# Background description:\\
\{desc\}\\
\#\#\# Query:\\
\{query\}\\
\\
You will be given 3 examples to help you understand the task.\\
Example 1:\\
\#\# Background: "Hello, I'm Sarah. I work at Microsoft as a junior developer with 2 years of experience. I live in Seattle."\\
\#\# Query: "What skills should I focus on developing in my early tech career at a leading software company to advance from my entry-level programming role?"\\
\#\# Answer: ["Microsoft", "junior developer"]\\

[Additional examples omitted for brevity]\\

Your output will contain the following format:\\
\#\#\# Thought: Explain your reasoning step by step for selecting the relevant PII entities.\\
\#\#\# Answer: List the relevant PII entities, each enclosed in double quotes (""). Return only the list without explanation. Example: ["key\_pii\_1", ..., "key\_pii\_n"]\\
\\
Please have your output follow the format below: (if there is only one entity, please output ["key\_pii\_1"]):\\
\#\#\# Thought: xxx\\
\#\#\# Answer: ["key\_pii\_1", ..., "key\_pii\_n"]\\
\end{tcolorbox}
\caption{Prompt of Auto-CoT Method}
\label{fig:auto-cot-prompt}
\end{figure*}

\begin{figure*}[htbp]
\begin{tcolorbox}[colback=white, colframe=black, title=Self-Consistency Query-related PII Detection Prompt]
Please identify highly relevant PII (Personally Identifiable Information) entities from the background description PII entities that directly address or relate to the user's query.\\
\\
Rules:\\
- Extract entities in their smallest possible span\\
- Exclude all person names\\
- Focus only on entities crucial for answering the query\\
- Return entities exactly as they appear in the text\\
\\
\#\#\# Background description:\\
\{desc\}\\
\#\#\# Query:\\
\{query\}\\
\\
Your output will contain the following format:\\
\#\#\# Thought: Generate 5 completely different perspectives of your reflections for selecting the relevant PII entities.\\
\#\#\# Summary: Output a summary of all your thinking.\\
\#\#\# Answer: List the relevant PII entities, each enclosed in double quotes (""). Return only the list without explanation. Example: ["key\_pii\_1", ..., "key\_pii\_n"]\\
\\
Please have your output follow the format below: (if there is only one entity, please output ["key\_pii\_1"]):\\
\#\#\# Thought:\\
1. xxxxxx\\
2. xxxxxx\\
3. xxxxxx\\
4. xxxxxx\\
5. xxxxxx\\
\\
\#\#\# Summary:\\
xxxxx\\
\\
\#\#\# Answer: ["key\_pii\_1", ..., "key\_pii\_n"]\\
\end{tcolorbox}
\caption{Prompt of Self-Consistency Method}
\label{fig:self-consistency-prompt}
\end{figure*}

\begin{figure*}[htbp]
\begin{tcolorbox}[colback=white, colframe=black, title=Plan-and-Solve Query-related PII Detection Prompt]
Please identify highly relevant PII (Personally Identifiable Information) entities from the background description that directly address or relate to the user's query.\\
\\
Rules:\\
- Extract entities in their smallest possible span\\
- Exclude all person names\\
- Focus only on entities crucial for answering the query\\
- Return entities exactly as they appear in the text\\
\\
\#\#\# Background description:\\
\{desc\}\\
\#\#\# Query:\\
\{query\}\\
\\
Your output will contain the following format:\\
\#\#\# Thought: Please start with a general plan for selecting the relevant PII entities, and then think step-by-step how to solve it based on the plan.\\
\#\#\# Answer: List the relevant PII entities, each enclosed in double quotes (""). Return only the list without explanation. Example: ["key\_pii\_1", ..., "key\_pii\_n"]\\
\\
Please have your output follow the format below: (if there is only one entity, please output ["key\_pii\_1"]):\\
\#\#\# Thought: xxx\\
\#\#\# Answer: ["key\_pii\_1", ..., "key\_pii\_n"]\\
\end{tcolorbox}
\caption{Prompt of Plan and Solve CoT Method}
\label{fig:plan-solve-prompt}
\end{figure*}