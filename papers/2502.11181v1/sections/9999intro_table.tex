
\begin{table}[t]
    \caption{An example of synthetic queries. 
    The queries are generated sequentially from $q^1$ to $q^3$. 
    \proposed is applied after generating $q^1$.
    % Repeated concepts are denoted in red, while newly covered concepts are denoted in blue. 
    Repeated keywords are denoted in \textcolor{red}{\textbf{red}}, while newly covered concepts are denoted in \textcolor{blue}{\textbf{blue}}. 
    Details of the generation process are illustrated in Figure \ref{fig:method}.
    % A detailed setup is provided in \cref{sec:experimentsetup}.
    }\small
    \centering
    \renewcommand{\arraystretch}{0.75}
    \resizebox{1\linewidth}{!}{
    \begin{tabular}{C} \toprule
    \textbf{Document} \\ \midrule
     Automated music playlist generation is a specific form of music recommendation. 
     Collaborative filtering methods can be used to ... 
     % Generally stated, the user receives a set of song suggestions ...  
     However, the scarcity of thoroughly curated playlists and the bias towards popular songs ... we propose an alternative model based on a song-to-playlist classifier, ... while leveraging song features derived from audio, ... robust performance when recommending rare and out-of-set songs. ...
     \\ \midrule
     \textbf{Generated queries for the document} \\ \midrule
    \begin{enumerate}[leftmargin=*, after={\vspace*{-0.7\baselineskip}}]% \vspace{-\topsep}
         \item[$q^1$] How can \textcolor{black}{song-to-playlist classifiers} enhance \textcolor{black}{automated music playlist generation}?  
         \item[$q^{2}$] How can \textcolor{red}{automated playlist creation} be boosted through \textcolor{red}{song-to-playlist classification} and \textcolor{blue}{\textbf{feature exploitation}}?  $\,$ (\textit{w/o any condition})
         \item[$q^3$] How does \textcolor{red}{song-to-playlist classifier} differ from traditional \textcolor{blue}{\textbf{collaborative filtering}} for music recommendation? $\quad\,\,\,\,\,\,\,\,\,\,\,$ (\textit{w/o any condition})
    \end{enumerate}\vspace{-\topsep} \\ \hdashline
    \begin{enumerate}[leftmargin=*, after={\vspace*{-\baselineskip}}]  
      \item[$q^{2'}$] What techniques can be used to \textcolor{blue}{\textbf{overcome filter bubbles}} and \textcolor{blue}{\textbf{recommend out-of-set songs}}?  $\quad\quad\quad\quad\quad\quad\quad\quad\quad\quad\quad\quad$ (\textit{w/ \proposed})
     \item[$q^{3'}$] How to \textcolor{blue}{\textbf{leverage mel-spectrogram features}} to \textcolor{blue}{\textbf{mitigate the cold-start problem}} in \textcolor{black}{playlist recommendation}? $\quad\quad\quad\quad\quad\,$(\textit{w/ \proposed})
    \end{enumerate}\vspace{-\topsep} \\ \bottomrule
    \end{tabular}}
    \label{tab:intro}
    \vspace{0.3cm}
\end{table}
