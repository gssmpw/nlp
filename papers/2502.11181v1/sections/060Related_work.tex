We provide a detailed survey of LLM-based query generation~in~\cref{prelim:qgen}.

\smallsection{PLM-based retrieval models}
The advancement of PLMs has led to significant progress in retrieval.
Recent studies have introduced retrieval-targeted pre-training \cite{CTR, condenser}, distillation from cross-encoders \cite{AR2}, and advanced negative mining methods \cite{zhan2021optimizing, rocketqa_v1}
There is also an increasing emphasis on pre-training methods specifically designed for the scientific domain. 
In addition to pre-training on academic corpora \cite{SCIBERT}, researchers have exploited metadata associated with scientific papers. 
\cite{razdaibiedina2023miread} uses journal class, \cite{SPECTER, SCINCL} use citations, \cite{ASPIRE} uses co-citation contexts, and \cite{OAGBERT} utilizes venues, affiliations, and authors.
\cite{SPECTER2, zhang2023pre} devise multi-task learning of related tasks such as citation prediction~and~paper~classification.
Very recently, \cite{ToTER, taxoindex} leverage corpus-structured knowledge (e.g., core topics and phrases) for academic concept matching.

% These models can be used as a backbone retriever for \proposed framework.


\smallsection{Synthetic query generation}
Earlier studies \cite{nogueira2019document, nogueira2019doc2query, liang2020embedding, ma2021zero, wang2022gpl} have employed dedicated query generation models, trained using massive document-query pairs from general domains.
Recently, there has been a shift towards replacing these generation models with LLMs \cite{dai2022promptagator, inpars, inpars2, pairwise_qgen, saad2023udapdr, sachan2022improving}, as discussed in \cref{prelim:qgen}.
On the other hand, many recent studies have focused on developing query generation tailored to specific retrieval domains.
\cite{synthetic_apple_VA} focuses on entity search for virtual assistants,
\cite{shen2022diversified} improves the diversity of queries for news article searches guided by a knowledge graph,
\cite{controllable_QGen} focuses on enhancing the retrievability of entities on online content (e.g., Podcast) platforms.
However, a dedicated method for scientific document retrieval has not been studied well in the literature.

    


\vspace{-0.3cm}
 


% Synthetic Query Generation using Large Language Models for Virtual Assistants, SIGIR'24
% Diversified Query Generation Guided by Knowledge Graph, WSDM'22