%% The first command in your LaTeX source must be the \documentclass command.
\DocumentMetadata{}
\documentclass[sigconf]{acmart}
%% \\documentclass[10pt,sigconf,letterpaper]{acmart}

\settopmatter{authorsperrow=4}
%%
%% \BibTeX command to typeset BibTeX logo in the docs 
\AtBeginDocument{%  
\providecommand\BibTeX{{%
  Bib\TeX}}}

\copyrightyear{2025}
\acmYear{2025}
%% \setcopyright{cc}
%% \setcctype{CC-BY}
\setcopyright{rightsretained}
\acmConference[WSDM '25]{Proceedings of the Eighteenth ACM International Conference on Web Search and Data Mining}{March 10--14, 2025}{Hannover, Germany}
\acmBooktitle{Proceedings of the Eighteenth ACM International Conference on Web Search and Data Mining (WSDM '25), March 10--14, 2025, Hannover, Germany}
\acmDOI{10.1145/3701551.3703544}
\acmISBN{979-8-4007-1329-3/25/03}

% The following includes the CC license icon appropriate for your paper.
% Download the image from www.scomminc.com/pp/acmsig/4ACM-CC-by-88x31.eps
% and place within your figs or figures folder

\makeatletter
\gdef\@copyrightpermission{
  \begin{minipage}{0.2\columnwidth}
   \href{https://creativecommons.org/licenses/by/4.0/}{\includegraphics[width=0.90\textwidth]{figs/ccs.png}}
  \end{minipage}\hfill
  \begin{minipage}{0.8\columnwidth}
   \href{https://creativecommons.org/licenses/by/4.0/}{This work is licensed under a Creative Commons Attribution International 4.0 License.}
  \end{minipage}
  \vspace{5pt}
}
\makeatother


\usepackage[normalem]{ulem}
\useunder{\uline}{\ul}{}


\usepackage{hyperref}
\usepackage{pifont}
\usepackage{amsfonts, amsthm, array}
\usepackage[ruled,vlined,linesnumbered]{algorithm2e}
\usepackage{enumitem,multirow,graphicx,subcaption,multicol,lipsum,float,adjustbox}
\captionsetup[table]{skip=2pt}
\captionsetup[figure]{skip=2pt}
\newlength{\textfloatsepsave} \setlength{\textfloatsepsave}{\textfloatsep} \setlength{\textfloatsep}{0.5pt}
\usepackage{cleveref}
\usepackage{algorithmicx}
\usepackage[page]{appendix} % print appendices title
\renewcommand{\appendixpagename}{Appendix} % Appendices title
\crefformat{section}{\S#2#1#3}
\crefformat{subsection}{\S#2#1#3}
\crefformat{subsubsection}{\S#2#1#3}
\usepackage{arydshln}
\usepackage{epstopdf}

\newcolumntype{P}{>{\raggedright\arraybackslash}m{0.98\linewidth}}
\newcolumntype{C}{>{\arraybackslash}m{1.\linewidth}}

\usepackage{makecell}

\settopmatter{printacmref=true}
\begin{document}

%% The "title" command has an optional parameter,
%% allowing the author to define a "short title" to be used in page headers.


\title{Improving Scientific Document Retrieval with~Concept~Coverage-based Query Set Generation}



\author{SeongKu Kang}
\affiliation{%
    \institution{Korea University}
    \city{Seoul}
    \country{South Korea}}
\email{seongkukang@korea.ac.kr}

\author{Bowen Jin}
\affiliation{
    \institution{University of Illinois at Urbana-Champaign}
    \city{Champaign}
    \country{United States}
}
\email{bowenj4@illinois.edu}

\author{Wonbin Kweon}
\affiliation{
    \institution{Pohang University of Science and Technology}
        \city{Pohang}
    \country{South Korea}
}
\email{kwb4453@postech.ac.kr}

\author{Yu Zhang}
\affiliation{
    \institution{Texas A\&M University}
    \city{College Station}
    \country{United States}
}
\email{yuzhang@tamu.edu}

\author{Dongha Lee}
\affiliation{
    \institution{Yonsei University}
        \city{Seoul}
    \country{South Korea}
}
\email{donalee@yonsei.ac.kr}


\author{Jiawei Han}
\affiliation{
    \institution{University of Illinois at Urbana-Champaign}
    \city{Champaign}
    \country{United States}
}
\email{hanj@illinois.edu}

\author{Hwanjo Yu}
\affiliation{
    \institution{Pohang University of Science and Technology}
    \city{Pohang}
    \country{South Korea}
}
\authornote{Corresponding author}
\email{hwanjoyu@postech.ac.kr}

\renewcommand{\shortauthors}{SeongKu Kang et al.}
%% No italics, no superscripts
%% Use footnote or author note to identify equal contribution and/or contact author info

\begin{abstract}

Hypotheses are central to information acquisition, decision-making, and discovery. However, many real-world hypotheses are abstract, high-level statements that are difficult to validate directly. 
This challenge is further intensified by the rise of hypothesis generation from Large Language Models (LLMs), which are prone to hallucination and produce hypotheses in volumes that make manual validation impractical. Here we propose \mname, an agentic framework for rigorous automated validation of free-form hypotheses. 
Guided by Karl Popper's principle of falsification, \mname validates a hypothesis using LLM agents that design and execute falsification experiments targeting its measurable implications. A novel sequential testing framework ensures strict Type-I error control while actively gathering evidence from diverse observations, whether drawn from existing data or newly conducted procedures.
We demonstrate \mname on six domains including biology, economics, and sociology. \mname delivers robust error control, high power, and scalability. Furthermore, compared to human scientists, \mname achieved comparable performance in validating complex biological hypotheses while reducing time by 10 folds, providing a scalable, rigorous solution for hypothesis validation. \mname is freely available at \url{https://github.com/snap-stanford/POPPER}.




\end{abstract}

\begin{CCSXML}
<ccs2012>
   <concept>
       <concept_id>10002951.10003317</concept_id>
       <concept_desc>Information systems~Information retrieval</concept_desc>
       <concept_significance>500</concept_significance>
       </concept>
   <concept>
       <concept_id>10002951.10003317.10003318.10003321</concept_id>
       <concept_desc>Information systems~Content analysis and feature selection</concept_desc>
       <concept_significance>500</concept_significance>
       </concept>
   <concept>
       <concept_id>10002951.10003317.10003325</concept_id>
       <concept_desc>Information systems~Information retrieval query processing</concept_desc>
       <concept_significance>300</concept_significance>
       </concept>
 </ccs2012>
\end{CCSXML}

\ccsdesc[500]{Information systems~Information retrieval}
\ccsdesc[500]{Information systems~Content analysis and feature selection}
\ccsdesc[300]{Information systems~Information retrieval query processing}
%%
%% Keywords. The author(s) should pick words that accurately describe
%% the work being presented. Separate the keywords with commas.
\keywords{Information retrieval; Query generation; Scientific document search}
% methods
\newcommand{\proposed}{CCQGen\xspace}
\newcommand{\proposedtwo}{CSR\xspace}

\newcommand{\smallsection}[1]{{\vspace{0.03in} \noindent \bf {#1.}}}

\newcommand{\ctr}{{Contriever-MS}\xspace}
\newcommand{\specter}{{SPECTER-v2}\xspace}
\newcommand{\csfcube}{CSFCube\xspace}
\newcommand{\dorismae}{DORIS-MAE\xspace}



%%
%% This command processes the author and affiliation and title
%% information and builds the first part of the formatted document.
\maketitle

\section{Introduction}
%
% 1. Evaluation of LLMs is challenging, many different benchmarks
One of the most celebrated aspects of state of the art large language models (LLMs) is that they can solve open-ended, complex tasks across many different application domains such as coding, healthcare and scientific discovery~\cite{bubeck2023sparks, mozannar2022reading, haupt2023ai, romera2023mathematical}. 
%
However, this is crucially what also makes the evaluation and comparison of LLMs very challenging---it is very difficult, if not impossible, to create a single benchmark.
%
As a consequence, in recent years, there has been a flurry of papers introducing different benchmarks~\cite{bach2022promptsource,wei2022finetuned,talmor2019commonsense,mishra2022cross,chen2021evaluating,liang2023holistic,longpre2023flan,hendryckstest2021,wang2022self,ouyang2022training,wang2023aligning,chiang2024chatbot,taori2023stanford,zheng2023judging,li2023generative,li2023prd,boubdir2023elo,singhal2023large}.
%
In fact, one of the flagship conferences in machine learning has even created a separate datasets and benchmarks track!

%
% 2. Uncertainty is overlooked
In this context, it is somehow surprising that, in comparison, there has been a paucity of work understanding, measuring or controlling for the different sources of uncertainty present in the evaluations and comparisons of LLMs based on these benchmarks~\cite{miller2024adding,madaan2024quantifying,dubey2024llama,saadfalcon2023ares,boyeau2024autoeval,chatzi2024prediction,dorner2024limits,gera2024justrank}.
%
In our work, we focus on one source of uncertainty that has been particularly overlooked, the uncertainty in the outputs of the LLMs under comparison.

%
% 3. Autoregressive process underpinning LLMs output randoms response
Given an input prompt, LLMs generate a sequence of tokens\footnote{Tokens are the units that make up sentences and paragraphs, \eg, (sub-)words, numbers, and special end-of-sequence tokens.} as output using an autoregressive process~\cite{bengio2000neural,radford2019language}. 
%
At each time step, they first use a neural network to map the prompt and the (partial) sequence of tokens generated so far to a token distribution. 
%
Then, they use a sampler to draw the next token at random from the token distribution.\footnote{If an LLM is forced to output tokens deterministically, multiple lines of evidence suggest that its performance worsens~\citep{holtzman2020the}.}
%
Finally, they append the next token to the (partial) sequence of tokens, and continue until a special end-of-sequence token is sampled.
%
To understand why, in the context of LLM evaluation and ranking, the above autoregressive process may lead to inconsistent conclusions, we will use a stylized example.

%
% 4. Example showing independent noise can lead to counterintuitive results (copy of 
% the same model or very similar models).
Consider we are given three LLMs $m_1$, $m_2$ and $m_3$, and we need to rank them according to their ability to answer correctly two types of input prompts, $q$ and $q'$, picked uniformly at random.
%
Moreover, assume that the true probability that each LLM answers correctly each type of input prompt is given by:
%
\begin{table}[h]
    \centering
    \begin{tabular}{lccc}
        \toprule
         & $m_1$ & $m_2$ & $m_3$ \\
        \midrule
        $q$           & 0.4             & 0.48           & 0.5             \\
        $q'$           & 1           & 0.9         & 0.89              \\
        \bottomrule
 \end{tabular}
 \vspace{-8mm}
 \caption*{}
 \end{table}

Then, one may argue that $m_1$ is the best LLM, followed closely by $m_3$, and $m_2$ is the worst, because the average probabilities that they answer a query picked uniformly at random correctly are $0.7$, $0.695$ and $0.69$, respectively.
%
However, if we conduct pairwise comparisons between outputs by two different LLMs to the same input prompt, as commonly done in practice, we may instead argue that $m_3$ is the best LLM, followed by $m_2$, and $m_1$ is the worst, because the probability that an LLM is preferred over others---the win-rates---are $0.16225$, $0.15675$, and $0.1545$, respectively.\footnote{Refer to Appendix~\ref{app:example-ranking} for the detailed calculation of the average win-rates.}
%
In our work, we argue that controlling for the randomization of the autoregressive processes underpinning the LLMs under comparison can, at least in certain cases, avoid such inconsistencies and lead to more intuitive conclusions.
%
Along the way, we also show that it can reduce the number of samples required to reliably compare the performance of LLMs. 

\xhdr{Our contributions}
%
% 5. Causal model of coupled autoregressive generation
Our key idea is to couple the autoregressive processes underpinning a set of LLMs under comparison, particularly their samplers, by means of sharing the same source of randomness. 
%
To this end, we treat the sampler of each LLM as a causal mechanism that receives as input the distribution of the next token and the same set of noise values, which determine the sampler'{}s (stochastic) state.
%
By doing so, at each time step of the generation, we can expect that, if different LLMs map the prompt and the (partial) sequence of tokens generated so far to the same token distribution, they will sample the same next token.
%
Loosely speaking, in the context of LLM evaluation and ranking, coupled autoregressive generation ensures that no LLM will have better luck than others.
%
% 6. Theoretical results
More formally, on evaluations based on benchmark datasets, we show that the difference in average performance of each pair of LLMs under comparison is asymptotically the same under coupled and vanilla autoregressive generation, but coupled autoregressive generation provably leads to a reduction in the required sample size.
%
On evaluations based on (human) pairwise comparisons, we show that the win-rates of the LLMs under comparison can be asymptotically different under coupled and vanilla autoregressive generation and, perhaps surprisingly, the resulting rankings can actually differ.
%
This suggests that the apparent advantage of an LLM over others in existing evaluation protocols may not be genuine but rather confounded by the randomness inherent to the generation process.

%
% 7. Experiments
To illustrate and complement our theoretical results, we conduct experiments with several LLMs of the \texttt{Llama} family, namely \texttt{Llama-3.1-8B-Instruct}, \texttt{Llama-3.2-\{1B, 3B\}-Instruct}, and \texttt{Llama-3.1-8B-Instruct-\{AWQ-INT4, bnb-4bit, bnb-8bit\}}.
%
% 8. MMLU results
% 
We find that, across multiple knowledge areas from the popular MMLU benchmark dataset, 
% coupled autoregressive generation leads to a reduction of up to $40$\% in the required number of samples.
coupled autoregressive leads to a reduction of up to $40$\% in the required number of samples to reach the same conclusions as vanilla autoregressive generation.
%
% 9. LMSYS results
%
Further, using data from the LMSYS Chatbot Arena platform, we find that the win-rates derived from pairwise comparisons by a strong LLM differ under coupled and vanilla autoregressive generation.
%
We conclude with a comprehensive discussion of the limitations of our theoretical results and experiments, including additional avenues for future work. An open-source implementation of coupled autoregressive generation is available at \url{https://github.com/Networks-Learning/coupled-llm-evaluation}.
% \footnote{We provide the code used in our experiments as supplementary 
% material. We will publicly release it with the final version of the paper.}

% 10. Further related work
\xhdr{Further related work}
%
% Our work builds upon a very recent work on counterfactual token generation by Chatzi et al.~\citep{chatzi2024counterfactual}, which also treats the sampler of an LLM as a causal mechanism.
%
% However, their focus is different to ours, they augment a single LLM with the ability to conduct counterfactual reasoning about alternatives to their own outputs.
%
Our work builds upon a very recent work on counterfactual token generation by~\citet{chatzi2024counterfactual}, which also treats the sampler of an LLM as a causal mechanism.
%
However, their focus is different to ours; they augment a single LLM with the ability to reason counterfactually about alternatives to its own outputs if individual tokens had been different.
% conduct counterfactual reasoning about alternatives to its own outputs. 
% 
Our work also shares technical elements with a recent work by~\citet{ravfogel2024counterfactual}, which develops a causal model to generate counterfactual strings resulting from interventions within (the network of) an LLM. 
% However, their work does not relate to model evaluation and, thus, is complementary to ours.
However, their work does not study counterfactual generation for the purposes of model evaluation.
% and, thus, is complementary to ours.
% 
In this context, it is also worth pointing out that the specific class of causal models used in the aforementioned works and our work, called the Gumbel-max structural causal model~\cite{oberst2019counterfactual}, has also been used to enable counterfactual reasoning in Markov decision processes~\citep{tsirtsis2021counterfactual}, temporal point processes~\citep{noorbakhsh2022counterfactual}, and expert predictions~\citep{benz2022counterfactual}.
% In this context, it is also worth pointing out that the specific class of causal mechanisms used by Chatzi et al. and our work, called Gumbel-max structural causal model~\cite{oberst2019counterfactual}, has been also used to enable counterfactual reasoning in string generation~\cite{ravfogel2024counterfactual}, Markov decision processes~\citep{tsirtsis2021counterfactual}, temporal point processes~\citep{noorbakhsh2022counterfactual}, and expert predictions~\citep{benz2022counterfactual}.

Our work also builds upon the rapidly increasing literature on evaluation and comparison of LLMs~\cite{chang2024asurvey}. 
%
Within this literature, LLMs are evaluated and compared using: 
%
(i) benchmark datasets with manually hand-crafted inputs and ground-truth outputs~\cite{bach2022promptsource,wei2022finetuned,talmor2019commonsense,mishra2022cross,chen2021evaluating,liang2023holistic,longpre2023flan} and (ii) the level of alignment with human preferences, as elicited by means of pairwise comparisons~\cite{taori2023stanford,zheng2023judging,li2023generative,li2023prd,boubdir2023elo,singhal2023large,chiang2024chatbot}.
%
However, it has become increasingly clear that oftentimes rankings derived from benchmark datasets do not match those derived from human preferences~\cite{zheng2023judging,li2023generative,li2023prd,chiang2023vicuna,chiang2024chatbot}.
%
Within the literature on ranking LLMs from pairwise comparisons, most studies use the Elo rating system~\cite{askell2021general,dettmers2024qlora,bai2022training,wu2023chatarena,lin2023llm}, originally introduced for chess tournaments~\cite{elo1966uscf}. 
%
However, Elo-based rankings are sensitive to the order of pairwise comparisons, as newer comparisons have more weight than older ones, which leads to unstable rankings~\cite{boubdir2023elo}.
%
To address this limitation, several studies have instead used the Bradley-Terry model~\cite{chiang2024chatbot,boyeau2024autoeval}, which weighs pairwise comparisons equally regardless of their order. % , thus resulting in more stable rankings.
%
Nevertheless, both the Elo rating system and the Bradley-Terry model have faced criticism, 
%
as pairwise comparisons often fail to satisfy the fundamental axiom of transitivity, 
%
upon which both approaches rely~\cite{boubdir2023elo,bertrand2023limitations}. 
%
Recently, several studies have used the win-rate~\cite{zheng2023judging,chiang2024chatbot,boyeau2024autoeval}, which weighs comparisons equally regardless of their order and does not require the transitivity assumption.
%
In our work, we focus on win-rates. However, we believe that it may be possible to extend our theoretical and empirical results to rankings based on Elo ratings and the Bradley-Terry model.

\section{Preliminaries}
\label{sec:preliminary}
\subsection{Fine-tuning Retrieval Model}
To perform retrieval on a new corpus, a PLM-based retriever is fine-tuned using a training set of annotated query-document pairs.
For each query $q$, the contrastive learning loss is typically applied:
\begin{equation}
    \mathcal{L} = -\log\frac{e^{s_{text}(q,\,\, d^+)}}{e^{s_{text}(q,\,\, d^+)} + \sum_{d^-} e^{s_{text}(q,\,\, d^-)}},
\end{equation}
where $d^+$ and $d^-$ denote the relevant and irrelevant documents. 
$s_{text}(q, d)$ represents the similarity score between the query and a document, computed by the retriever.
For effective fine-tuning, a substantial amount of training data is required. 
However, in specialized domains such as scientific document search, constructing vast human-annotated datasets is challenging due to the need for domain expertise, which remains an obstacle for~applications~\cite{li2023sailer, ToTER}.


\subsection{Prompt-based Query Generation}
\label{prelim:qgen}
Several attempts have been made to generate synthetic queries using LLMs. 
Recent advancements have centered on advancing prompting schemes to enhance the quality of these queries.
We summarize recent methods in terms of their prompting schemes.
% It is worth noting that the following schemes are not mutually exclusive, and they can be combined to form a prompt.

\smallsection{Few-shot examples}
Several methods \cite{inpars, inpars2, dai2022promptagator, pairwise_qgen, label_condition_qgen, saad2023udapdr} incorporate a few examples of relevant query-document pairs in the prompt.
The prompt comprises the following components: $P = \{inst, (d_i, q_i)^k_{i=1}, d_t\}$, 
where $inst$ is the textual instruction\footnote{For example, ``\textit{Given a document, generate five search queries for which the document can be a perfect answer}''. 
The instructions vary slightly across methods, typically in terms of word choice.
In this work, we follow the instructions used in \cite{pairwise_qgen}.
}, 
$(d_i, q_i)^k_{i=1}$ denotes $k$ examples of the document and its relevant query, 
and $d_t$ is the new document we want to generate queries for.
By providing actual examples of the desired outputs, this technique effectively generates queries with distributions similar to actual queries (e.g., expression styles and lengths) \cite{dai2022promptagator}.
It is worth noting that this technique is also utilized in subsequent prompting schemes.

\smallsection{Label-conditioning} 
Relevance label $l$ (e.g., relevant and irrelevant) has been utilized to enhance query generation \cite{label_condition_qgen, saad2023udapdr, inpars}.
The prompt comprises $P = \{inst, (l_i, d_i, q_i)^k_{i=1}, (l_t, d_t)\}$, where $k$ label-document-query triplets are provided as examples.
$l_i$ represents the relevance label for the document $d_i$ and its associated query $q_i$.
To generate queries, the prompt takes the desired relevance label $l_t$ along with the document $d_t$.
This technique incorporates knowledge of different relevance, which aids in improving query quality and allows for generating both relevant and irrelevant queries \cite{label_condition_qgen}.
% For instance, in the case of binary relevance labels, the prompt contains examples for both relevant and irrelevant query-document pairs.

\smallsection{Pair-wise generation}
To further enhance the query quality, the state-of-the-art method \cite{pairwise_qgen} introduces a \textit{pair-wise} generation of relevant and irrelevant queries.
It instructs LLMs to first generate relevant queries and then generate relatively less relevant ones. 
The prompt comprises $P = \{inst, (d_i, q_i, q^-_i)^k_{i=1}, d_t\}$, where $q_i$ and $q^-_i$ denote relevant and irrelevant query for $d_i$, respectively.
The generation of irrelevant queries is conditioned on the previously generated relevant ones, allowing for generating thematically similar rather than completely unrelated queries.
These queries can serve as natural `hard negative' samples for training \cite{pairwise_qgen}.


% LLMs excel at extracting keywords and creating plausible sentences by combining them.
% Recent advanced prompting schemes allow for leveraging this capability to synthesize training data. 
% However, in our attempts to apply them to scientific domain retrieval, we observe that existing techniques do not sufficiently simulate actual queries and show limited effectiveness.



\vspace{0.03in}
\textbf{Remarks.}
% While these advanced schemes effectively leverage the text-generation capabilities of LLMs, there remains substantial room for improvement.
Though effective in generating plausible queries, there remains substantial room for improvement.
We observe that existing techniques often generate queries with limited coverage of the document's concepts.
That is, the queries frequently cover similar aspects of the document, exhibiting high redundancy and failing to add new training signals.
Furthermore, the queries show a high lexical overlap with the document, often repeating a few keywords from the document (\cref{result:query_analysis}).
Considering that the same concepts are expressed using diverse terms in actual user queries, merely repeating a few keywords may limit the efficacy of fine-tuning.

% Ideally, training queries should be complementary to each other, comprehensively covering the document's concepts.
% Ideally, training queries should include relevant concepts expressed in various terms, rather than simply repeating phrases from the document.


\begin{figure*}[t]
\centering
\includegraphics[width=1.0\textwidth]{images/method.pdf}
\caption{The overview of Concept Coverage-based Query set Generation (\proposed) framework.
% The example is taken from \csfcube dataset.
Best viewed in color.
}
\label{fig:method}
\vspace{-0.3cm}
\end{figure*}


% provide no explicit guidance on what content to generate, 

% A document encompasses multiple concepts, and users inquire about various aspects of a document. 




% LLM은 keyword들을 추출하고, 조합하여 plausible한 sentence를 만드는데 탁월하다. 
% 그동안 연구되어온 prompting scheme들은 이러한 capability를 효과적을 활용하여, high-quality training data를 generation 했다. 
% 그러나, scientic domain에서 이러한 방법론을 적용하려는 시도 끝에, 우리는 기존 기법들이 충분히 actual query를 모방하고 있지 않으며, there still room for improvements임을 발견했다.
% To elaborate, 
% LLM-generated query는 keyword들을 반복하여, 특히 높은 lexical overlap을 보인다. 그러나, 유저들은 같은 academic concept을 아주 다양한 term, expression으로 표현한다. Simply repeating phrases in the document is less effective.
% 나아가, a document는 multiple concepts을 포괄하며, users ask about various aspects of a paper. 
% 기존 기법들은 어떤 내용을 generation 할 것인가에 대한 명시적인 고려나 조건을 활용하지 않는다. 
% When generating several queries, they are often redundant, falling short of adding new training signals.


\section{Methodology}
\label{sec:method}
We present \textbf{C}oncept \textbf{C}overage-based \textbf{Q}uery set \textbf{Gen}eration (\proposed) framework, designed to meet two desiderata of training queries:
(1) The concepts covered by queries should be complementary to each other, enabling a comprehensive coverage of the document’s concepts.
(2) The queries should articulate the document concepts in various related terms, rather than merely repeating phrases from the document.
\proposed consists of two major stages:
\begin{itemize}[leftmargin=*]\vspace{-\topsep}
    \item \textbf{Concept identification and enrichment (\cref{subsec:method_a}):}
        We first identify the core academic concepts of each document.
        Then, we enrich the identified concepts by assessing their importance and adding related concepts not explicitly mentioned in the document.
        This information serves as the basis for generating queries.
        % This model is used to predict academic concepts from input text.
        % Then, we train a small model called a \textit{concept extractor}
    \item \textbf{Concept coverage-based query generation (\cref{subsec:method_b}):}
        Given the previously generated queries $Q^{m-1}_d = \{q^1_d, ..., q^{m-1}_d\}$, we compare the concepts of the document $d$ with those covered by $Q^{m-1}_d$ to identify uncovered concepts.
        These uncovered concepts are then leveraged as conditions for generating the subsequent query $q^{m}_d$, allowing $q^{m}_d$ to cover complementary aspects of $Q^{m-1}_d$.
        % the next query $q^{l}_d$ can cover complementary aspects from $Q^{l-1}_d$.
\end{itemize}\vspace{-\topsep}
Moreover, we propose a new technique, concept similarity-enhanced retrieval (\proposedtwo), that leverages the obtained concept information for \textbf{filtering out low-quality queries} and for \textbf{improving retrieval accuracy (\cref{subsub:method_filtering})}.
Figure \ref{fig:method} provides an overview of \proposed.
% Moreover, we propose new techniques that leverage the obtained concept information for \textbf{filtering out low-quality queries} and for \textbf{improving retrieval accuracy (\cref{subsub:method_filtering})}.




% filter out low-quality queries
% We represent the concepts at two different granularities: topic and phrase levels (Fig.\ref{fig:method}a).

\subsection{Concept Identification and Enrichment}
\label{subsec:method_a}
To measure concept coverage, we first identify the core academic concepts of each document.
We represent the concepts using a combination of two different granularities: topic and phrase levels (Figure \ref{fig:method}a).
Topic level provides broader categorizations of research, such as `collaborative filtering' or `machine learning', while phrase level includes specific terms in the document, such as `playlist continuation' or `song-to-playlist classifier', complementarily revealing the document concepts.


A tempting way to obtain these topics and phrases is to simply instruct LLMs to find them in each document.
However, this approach has several limitations:
the results may contain concepts not covered by the document, and there is always a potential risk of hallucination.
As a solution, we propose a new approach that first constructs a candidate set, and then uses LLMs to pinpoint the most relevant ones from the given candidates, instead of directly generating them.
By doing so, the output space is restricted to the predefined candidate space, greatly reducing the risk of hallucinations while effectively leveraging the language-understanding capability of LLMs.





\subsubsection{\textbf{Core topics identification}}
\label{subsub:core_topic}
To identify the core topics of documents, we propose using an \textit{academic topic taxonomy} \cite{MAG_FS}. 
In the scientific domain, academic taxonomies are widely used for categorizing studies in various institutions and can be easily obtained from the web.\footnote{E.g., IEEE Taxonomy (\href{https://www.ieee.org/content/dam/ieee-org/ieee/web/org/pubs/ieee-taxonomy.pdf}{\color{blue} link}), ACM Computing Classification System (\href{https://dl.acm.org/ccs}{\color{blue} link}).}
A taxonomy refers to a hierarchical tree structure outlining academic topics (Figure \ref{fig:method}a).
Each node represents a topic, with child nodes corresponding to its sub-topics.
Leveraging taxonomy allows for exploiting domain knowledge of topic hierarchy and reflecting researchers' tendency to~classify~studies.


\smallsection{Candidate set construction}
One challenge in finding candidate topics is that the taxonomy obtained from the web is often very large and contains many irrelevant topics.% \footnote{In our experiments, it encompasses $19$ disciplines with over $400,000$ topic nodes.}
To effectively narrow down the candidates, we employ a \textit{top-down traversal} technique that \textit{recursively visits} the child nodes with the highest similarities at each level.
For each document, we start from the root node and compute its similarity to each child node.
We then visit child nodes with the highest similarities.\footnote{We visit multiple child nodes and create multiple paths, as a document usually covers various topics. For a node at level $l$, we visit $l+2$ nodes to reflect the increasing number of nodes at deeper levels of the taxonomy. The root~node~is~level~$0$.}
This process recurs until every path reaches leaf nodes, and \textit{all visited nodes} are regarded as candidates.


The document-topic similarity ${s}(d, c)$ can be defined in various ways.
As a topic encompasses its subtopics, we collectively consider the subtopic information for each topic node.
Let $\mathcal{N}_c$ denote the set of nodes in the sub-tree having $c$ as a root node.
We compute the similarity as: ${s}(d, c) = \frac{1}{|\mathcal{N}_c|}\sum_{j \in \mathcal{N}_c} \operatorname{cos}(\mathbf{e}_{d}, \mathbf{e}_{j})$, 
where $\mathbf{e}_d$ and $\mathbf{e}_j$ denote representations from PLM for a document $d$ and the topic name of node $j$, respectively.\footnote{We use BERT with mean pooling as the simplest choice.}


\smallsection{Core topic selection}
We instruct LLMs to select the most relevant topics from the candidates.
An example of an input prompt is:
\begin{table}[h]
\small
    \centering
    \resizebox{1.0\linewidth}{!}{
    \begin{tabular}{|C|}
    \hline
    You will receive a document along with a set of candidate topics. Your task is to select the topics that best align with the core theme of the document. 
    Exclude topics that are too broad or less relevant.
    You may list up to [$k^t$] topics, using only the topic names in the candidate set. \textbf{Document}:~[\textsc{Document}],~\textbf{Candidate~topic~set}:~[\textsc{Candidates}]\\ \hline 
    \end{tabular}}
    \vspace{-0.3cm}
\end{table}

In this work, we set $k^t=10$.
For each document $d$, we obtain core topics as $\mathbf{y}^t_d \in \{0,1\}^{|\mathcal{T}|}$, where $y^t_{di}=1$ indicates $i$ is a core topic of $d$, otherwise $0$.
$\mathcal{T}$ denotes the topic set obtained~from~the~taxonomy.

% $|\mathcal{T}|$ denotes the total number of topics from the taxonomy.

% 

% this can go to implementation detail
% After identifying core topics for all documents, we tailor the taxonomy by only retaining the topics selected as core topics at least once, along with~their~ancestor~nodes.



% \footnote{The phrase set in the corpus is automatically extracted by an off-the-shelf phrase mining tool \cite{autophrase}.}+
% Following \cite{tao2016multi, lee2022taxocom}, we compute the indicativeness of each phrase $p$ based on two criteria:


% These phrases provide fine-grained information not covered by topic levels, playing a critical role in understanding detailed contents and subsequent retrieval.

% 

\subsubsection{\textbf{Core phrases identification}}
\label{method:core_phrase}
From each document, we identify core phrases used to describe its concepts.
These phrases offer fine-grained details not captured at the topic level.
We note that not all phrases in the document are equally important.
Core phrases should describe concepts strongly relevant to the document but \textit{not frequently covered} by other documents with similar topics.
For example, among documents about `recommender system' topic, the phrase `user-item interaction' is very commonly used, and less likely to represent the most important concepts~of~the~document.  


\smallsection{Candidate set construction}
Given the phrase set $\mathcal{P}$ of the corpus\footnote{The phrase set is obtained using an off-the-shelf phrase mining tool \cite{autophrase}.}, we measure the distinctiveness of phrase $p$ in document $d$.
Inspired by recent phrase mining methods \cite{tao2016multi, lee2022taxocom}, we compute the distinctiveness as: $\exp(\operatorname{BM25}(p, d))/\,(1 + \sum_{d'\in\mathcal{D}_{d}}\exp(\operatorname{BM25}(p, d')))$.
This quantifies the relative relevance of $p$ to the document $d$ compared to other topically similar documents $\mathcal{D}_{d}$. 
$\mathcal{D}_{d}$ is simply retrieved using Jaccard similarity of core topics $\mathbf{y}^t_d$.
We set $|\mathcal{D}_{d}|=100$.
We select phrases with top-20\% distinctiveness score~as~candidates. 

\smallsection{Core phrase selection}
We instruct LLMs to select the most relevant phrases (up to $k^p$ phrases) from the candidates, using the same instruction format used for the topic selection.
We set $k^p=15$.
The core phrases are denoted by $\mathbf{y}^p_d \in \{0,1\}^{|\mathcal{P}|}$, where $y^p_{dj}=1$ indicates $j$ is a core phrase of $d$, otherwise $0$.


\subsubsection{\textbf{Enriching concept information}}
\label{method:enrich}
We have identified core topics and phrases representing each document's concepts.
We further enrich this information by (1) measuring their relative importance, and (2) incorporating strongly related concepts (i.e., topics and phrases) not explicitly revealed in the document.
This enriched information serves as the basis for generating queries.


% can aid in finding related phrases
\vspace{0.02in} \noindent
\textbf{Concept extractor.}
We employ a small model called a \textit{concept extractor}.
For a document $d$, the model is trained to predict its core topics $\mathbf{y}^{t}_{d}$ and phrases $\mathbf{y}^{p}_{d}$ from the PLM representation $\mathbf{e}_d$.
We formulate this as a two-level classification task: topic and~phrase~levels.


% We observe that leveraging their complementary nature through multi-task learning consistently enhances both tasks, compared to using two separate classifiers.

% To exploit their complementary nature, we employ 
Topics and phrases represent concepts at different levels of granularity, and learning one task can aid the other by providing a complementary perspective.
To exploit their complementarity, we employ a multi-task learning model with two heads \cite{mmoe}.
Each head has a Softmax output layer, producing probabilities for topics $\hat{\mathbf{y}}^{t}_{d}$ and phrases $\hat{\mathbf{y}}^{p}_{d}$, respectively.
The cross-entropy loss is then applied for classification learning: $-\sum_{i=1}^{|\mathcal{T}|} y^t_{di} \log \hat{y}^{t}_{di} - \sum_{j=1}^{|\mathcal{P}|} y^p_{dj} \log \hat{y}^{p}_{dj}$.


\smallsection{Concept enrichment}
Using the trained concept extractor, we compute $\hat{\mathbf{y}}^{t}_{d}$ and $\hat{\mathbf{y}}^{p}_{d}$, which reveal their importance in describing the document's concepts.
Also, we identify strongly related topics and phrases that are expressed differently or not explicitly mentioned, by incorporating those with the highest prediction probabilities.
For example, in Figure \ref{fig:method}, we identify phrases `cold-start problem', `filter bubble', and `mel-spectrogram', which are strongly relevant to the document's concepts but not explicitly mentioned, along with their detailed importance.
These phrases are used to aid in articulating the document's concepts in various related terms.

% For example, in Figure \ref{fig:method}, we identify related phrases `cold-start problem', `filter bubble', and `mel-spectrogram', which are strongly relevant but not explicitly mentioned in the document, along with their detailed importance.

We obtain $k^{t'}$ enriched topics and $k^{p'}$ enriched phrases for each document with their importance from $\hat{\mathbf{y}}^{t}_{d}$ and $\hat{\mathbf{y}}^{p}_{d}$.
We set the probabilities for the remaining topics and phrases as $0$, and normalize the probabilities for selected topics and phrases, denoted by $\bar{\textbf{y}}^t_d$ and $\bar{\textbf{y}}^p_d$.


% We use $\bar{\textbf{y}}^t_d$ and $\bar{\textbf{y}}^p_d$ to denote the normalized probability distribution of core topics and phrases, respectively.

\subsection{Concept Coverage-based Query Generation}
\label{subsec:method_b}
We present how we generate a set of queries that comprehensively cover the various concepts of a document.
We first identify concepts insufficiently covered by the previously generated queries (\cref{subsub:method_sampling}) and leverage them as conditions for subsequent generation (\cref{subsub:method_condition}).
Then, a filtering step is applied to ensure the query quality~(\cref{subsub:method_filtering}).

This process is repeated until a predefined number ($M$) of queries per document is achieved.
$M$ is empirically determined, considering available training resources such as GPU memory and training time.
For the first query of each document, we impose no conditions, thus it is identical to the results obtained from existing methods.


% 
\subsubsection{\textbf{Concept sampling based on query coverage}}
\label{subsub:method_sampling}
The enriched information $\bar{\textbf{y}}_d$ reveals the core concepts and their importance within the document.
Our key idea is to generate queries that align with this distribution to ensure comprehensive coverage of the document's concepts.
Let $Q^{m-1}_d = \{q^1_d, ..., q^{m-1}_d\}$ denote the previously generated queries.
Using the concept extractor, which is trained to predict core concepts from the text, we identify the concepts covered by the queries, i.e., $\bar{\textbf{y}}^t_Q$ and $\bar{\textbf{y}}^p_Q$.
We use the concatenation of queries as input, denoted as $Q$.
A high value in $\bar{\textbf{y}}_d$ coupled with a low value in $\bar{\textbf{y}}_Q$ indicates that the existing queries do not sufficiently cover the corresponding concepts.

Based on the concept coverage information, we identify concepts that need to be more emphasized in the subsequently generated query.
We opt to leverage phrases as \textit{explicit} conditions for generation, as topics reveal concepts at a broad level, making them less effective for explicit inclusion in the query.
Note that topics are \textit{implicitly} reflected in identifying and enriching core phrases. 
We define a probability distribution to identify less~covered~concepts~as:
\begin{equation}
    \boldsymbol{\pi} = \operatorname{normalize}(\,\max(\bar{\textbf{y}}^p_d - \bar{\textbf{y}}^p_Q, \,\epsilon)\,)
\end{equation}
We set $\epsilon = 10^{-3}$ as a minimal value to the core phrases for numerical stability.
We sample $\lfloor \frac{k^{p'}}{M} \rfloor$ different phrases from $\operatorname{Multinomial}(\boldsymbol{\pi})$, where $M$ is the total number of queries per document.
Note that $\bar{\textbf{y}}^p_Q$ is dynamically adjusted during the construction of the~query~set.

% We define % A higher value in $\pi$ indicates that the concept is not sufficiently covered by the queries.


\subsubsection{\textbf{Concept conditioning for query generation}}
\label{subsub:method_condition}
The sampled phrases are leveraged as conditions for generating the next query $q^m_d$.
There have been active studies to control the generation of LLMs for various tasks. 
Recent methods \cite{control_gen, outline_condition} have specified conditions for the desired outputs, such as sentiment, keywords, and an outline, directly in the prompts.
Following these studies, we impose a condition by adding a simple textual instruction $C$: ``\textit{Generate a relevant query based on the following keywords}: [\textsc{Sampled phrases}]''.
While more sophisticated instruction could be employed, we obtained satisfactory results with~our~choice.


The final prompt is constructed as $[P; C]$, where $P$ is an existing prompting scheme discussed in \cref{prelim:qgen}.
This integration allows us to inherit the benefits of existing techniques (e.g., few-shot examples), while generating queries that comprehensively cover the document's concepts.
For example, in Figure \ref{fig:method}, $C$ includes phrases like `cold-start problem' and `audio features', which are not well covered by the previous queries.
Based on this concept condition, we guide LLMs to generate a query that covers complementary aspects to the previous ones.
It is important to note that $C$ adds an \textit{additional condition} for $P$; the query is still about playlist recommendation, the main task of the document.



\subsubsection{\textbf{Concept coverage-based consistency filtering}}
\label{subsub:method_filtering}
After generating a query, we apply a filtering step to ensure its quality.
A critical criterion for this process is \textit{round-trip consistency} \cite{alberti2019synthetic}; a query should be answerable by the document from which it was generated.
Existing work \cite{dai2022promptagator, label_condition_qgen} employs a retriever to assess this consistency.
Given a generated pair $(q_d, d)$, the retriever retrieves documents for $q_d$. 
Then, $q_d$ is retained only if $d$ ranks within the top-$N$ results.
The accuracy of the retriever is crucial in this step;
a weak retriever may fail to filter out low-quality queries and also only retain queries that are \textit{too easy} (e.g., high lexical overlap with the document), thereby limiting the effectiveness of training.

 


We note that relying on the existing retriever is insufficient for measuring relevance.
While it is effective at capturing similarities of surface texts, the retriever often fails to match underlying concepts.
For example, in Figure \ref{fig:method}, the generated query includes phrases `cold-start problem' and `mel-spectrogram', which are highly pertinent to `data scarcity' and `audio features' discussed in the document.
Nevertheless, as these phrases are not directly used in the document, the retriever struggles to assess the relevance and ranks the document low. 
Consequently, the query is considered unreliable and removed during the filtering process.


\smallsection{Concept similarity-enhanced retrieval (CSR)}
We propose a simple and effective technique to enhance retrieval by using concept information.
For relevance prediction, we consider both textual similarity from the retriever $s_{text}(q,d)$, and concept similarity $s_{concept}(q,d)$.
We measure concept similarity using core phrase distributions, i.e., $s_{concept}(q,d) = sim(\bar{\mathbf{y}}^p_q, \,\bar{\mathbf{y}}^p_d)$, which reveals related concepts at a fine-grained level.\footnote{Here, we compute the similarity for top-10\% phrases (instead of $k^{p'}$) to consider concepts having a certain degree of relevance.
We also tried using core topics. However, it proved less effective as topics reveal concepts only~at~a~broad~level.
}
$sim(\cdot, \cdot)$ is the similarity function, for which use inner-product.
The relevance score~is~defined~as:
\begin{equation}
\begin{aligned}
rel_{CSR}(q,d) = f(s_{text}(q,d), \,s_{concept}(q,d)),
\end{aligned}
\end{equation}
where $f(\cdot, \cdot)$ is a function that combines the two scores. 
We use a simple addition after rescaling them via z-score normalization.
We denote this technique as Concept Similarity-enhanced Retrieval~(\proposedtwo).

For \textbf{filtering process}, we assess the round-trip consistency using \proposedtwo.
By directly matching underlying concepts not apparent from the surface text, we can more accurately measure relevance and distinguish low-quality queries.
Additionally, for \textbf{search with test queries} (i.e., after fine-tuning using the generated data), \proposedtwo can be used as a supplementary technique to further enhance retrieval.
It helps to understand test queries, which contain highly limited contexts and jargon not included in the training queries, by associating them with pre-organized~concept~information.


% \vspace{0.03in}
% \textbf{Remarks on the efficiency of \proposed.}
% \proposed iteratively assesses the generated queries and imposes conditions for subsequent generations. 
% We acknowledge that this approach inevitably incurs additional computations during generation. 
% However, we highlight that no extra costs are incurred during the fine-tuning and inference phases.
% Further, \proposed consistently yields large improvements, even when the number of queries is highly limited, whereas existing methods often fail to improve~the~retriever~(\cref{result:CCQGen}).

\section{Experiments}
% \subsection{Experimental Setup}
% \label{sec:experimentsetup}
\smallsection{\textbf{Datasets}}
We conduct a thorough review of the literature to find retrieval datasets in the scientific domain, specifically those where relevance has been assessed by skilled experts or annotators.
We select two recently published datasets: \textbf{\csfcube} \cite{CSFCube} and \textbf{\dorismae} \cite{DORISMAE}.
They offer test query collections annotated by human experts and LLMs, respectively, and embody two real-world search scenarios: query-by-example and human-written queries.
For both datasets, we conduct retrieval from the entire corpus, including all candidate documents.
\csfcube dataset consists of 50 test queries, with about 120 candidates per query drawn from approximately 800,000 papers in the S2ORC corpus \cite{lo2020s2orc}. 
\dorismae dataset consists of 165,144 test queries, with candidates drawn similarly to \csfcube.
We consider annotation scores above `2', which indicate documents are `nearly identical or similar' (\csfcube) and `directly answer all key components' (\dorismae), as relevant.
Note that training queries are not provided in both datasets.


\smallsection{\textbf{Academic topic taxonomy}}
We utilize the field of study taxonomy from Microsoft Academic \cite{MAG_FS}, which contains $431,416$ nodes with a maximum depth of $4$.
After the concept identification step (\cref{subsec:method_a}), we obtain $1,164$ topics and $18,440$ phrases for \csfcube, and $1,498$ topics and $34,311$ phrases for \dorismae.



\smallsection{\textbf{Metrics}}
Following \cite{mackie2023generative, ToTER}, we employ Recall@$K$ (R@$K$) for a large retrieval size ($K$), and NDCG@$K$ (N@$K$) and MAP@$K$ (M@K) for a smaller $K$ ($\leq 20$).
Recall@$K$ measures the proportion of relevant documents in the top $K$ results, while NDCG@$K$ and MAP@$K$ assign higher weights to relevant documents at higher~ranks.


\smallsection{\textbf{Backbone retrievers}}
We employ two representative models: 
(1) \textbf{\ctr} \cite{CTR} is a widely used retriever fine-tuned using vast labeled data from general domains (i.e., MS MARCO).
(2) \textbf{\specter} \cite{SPECTER2} is a PLM specifically developed for the scientific domain. It is trained using metadata (e.g., citation relations) of scientific papers. 
For both models, we use public checkpoints: \texttt{facebook/contriever-msmarco} and  \texttt{allenai/specter2\_base}.
% \footnote{\texttt{facebook/contriever-msmarco}, \texttt{allenai/specter2\_base}.}

\smallsection{\textbf{Baselines}}
We compare various query generation methods.
For all LLM-based methods, we use \texttt{gpt-3.5-turbo-0125}.
Additionally, we explore the results with a smaller LLM (\texttt{Llama-3-8B}) in \cref{result:Llama}.
For each document, we generate \textbf{five} relevant queries~\cite{BEIR}.
\begin{itemize}[leftmargin=*]\vspace{-0.7\topsep}
    \item \textbf{GenQ} \cite{BEIR} employs a specialized query generation model, trained with massive document-query pairs from the general domains.
    We use T5-base, trained using approximately $500,000$ pairs from MS MARCO dataset \cite{nogueira2019doc2query}: \texttt{BeIR/query-gen-msmarco-t5-base-v1}.
\end{itemize}
\noindent
\proposed can be flexibly integrated with existing LLM-based methods to enhance the concept coverage of the generated queries.
We apply \proposed to two recent approaches,  discussed~in~\cref{prelim:qgen}.
\begin{itemize}[leftmargin=*]\vspace{-0.7\topsep}
    \item \textbf{Promptgator} \cite{dai2022promptagator} is a recent LLM-based query generation method that leverages \textbf{few-shot examples} within the prompt. 
    
    % \item \textbf{Pair-wise generation} \cite{pairwise_qgen} is the state-of-the-art LLM-based query generation method that generates both relevant and irrelevant queries in a \textbf{pair-wise} manner.

    \item \textbf{Pair-wise generation} \cite{pairwise_qgen} is the state-of-the-art method that generates relevant and irrelevant queries in a \textbf{pair-wise}~manner.
\end{itemize}
Additionally, we devise a new competitor that adds more instruction in the prompt to enhance the quality of queries:
\textbf{Promptgator\_{diverse}} is a variant of Promptgator, where we add the instruction ``\textit{use various terms and reduce redundancy among the~queries}''.

\smallsection{\textbf{Implementation details}}
We conduct all experiments using 4 NVIDIA RTX A5000 GPUs, 512 GB memory, and a single Intel Xeon Gold 6226R processor. 
For fine-tuning, we use top-50 BM25 hard negatives for each query \cite{formal2022distillation}.
We use 10\% of training data as a validation set. 
The learning rate is set to $10^{-6}$ for \ctr and $10^{-7}$ for \specter, after searching among $\{10^{-7}, 10^{-6}, ..., 10^{-3}\}$.
We set the batch size as $64$ and the weight decay as $10^{-4}$.
We report the average performance over five independent runs.
For all methods, we generate five synthetic queries for each document ($M=5$).
For the few-shot examples in the prompt, we randomly select five annotated examples, which are then excluded in the evaluation process \cite{dai2022promptagator}.
We follow the textual instruction used in \cite{pairwise_qgen}.
For other baseline-specific setups, we adhere to the configurations described in the original papers.
For the concept extractor, we employ a multi-gate mixture of expert architecture \cite{mmoe}, designed for multi-task learning.
We use three experts, each being a two-layer MLP.
For the consistency filtering, we set $N=5$.
We set the number of enriched topics and phrases to $k^{t'}=15$ and $k^{p'}=20$,~respectively.



% instruction
% promptgator:
% pair-wise: 
% max epochs? 30
% filtering of existing methods?

% Please add the following required packages to your document preamble:
% \usepackage{multirow}
\begin{table*}[t]
\caption{Retrieval performance comparison after fine-tuning with the generated queries. \textcolor{red}{Red} color denotes results that fail to show improvements over no Fine-Tune. $^{\dagger}$ and * indicate a statistically significant difference ($p<0.05$) from no Fine-Tune (one-sample t-test) and the applied query generation method (paired t-test), respectively.}
\centering
\renewcommand{\arraystretch}{0.95}
\resizebox{\linewidth}{!}{
\begin{tabular}{cl llllll  llllll}\toprule
% \multirow{2}{*}{\makecell[c]{\textbf{Backbone}\\ \textbf{retriever}}}
& \multicolumn{1}{c}{\multirow{2}{*}{\textbf{Query generation}}} & \multicolumn{6}{c}{\textbf{CSFCube}} & \multicolumn{6}{c}{\textbf{DORIS-MAE}} \\ \cmidrule(lr){3-8} \cmidrule(lr){9-14}
 &  & \textbf{N@10} & \textbf{N@20} & \textbf{M@10} & \textbf{M@20} & \textbf{R@50} & \textbf{R@100} & \textbf{N@10} & \textbf{N@20} & \textbf{M@10} & \textbf{M@20} & \textbf{R@50} & \textbf{R@100} \\ \midrule
\parbox[t]{2mm}{\multirow{7}{*}{\rotatebox[origin=c]{90}{\ctr}}} & no Fine-Tune & 0.3313 & 0.3604 & 0.1525 & 0.1937 & 0.5783 & 0.7136 & 0.2603 & 0.2707 & 0.1177 & 0.1422 & 0.4509 & 0.5877 \\
 & GenQ & 0.3401 & \textcolor{red}{0.3495} & \textcolor{red}{0.1476} & \textcolor{red}{0.1841} & \textcolor{red}{0.5571} & \textcolor{red}{0.6843} & \textcolor{red}{0.2496} & \textcolor{red}{0.2647} & \textcolor{red}{0.1152} & \textcolor{red}{0.1396} & 0.4598 & \textcolor{red}{0.5805} \\
 & Promptgator\_{diverse} & 0.3539 & 0.3771 & 0.1606 & 0.2029 & 0.5950 & \textcolor{red}{0.7132} & \textcolor{red}{0.2461} & \textcolor{red}{0.2690} & \textcolor{red}{0.1143} & \textcolor{red}{0.1406} & 0.4645 & 0.5951 \\ \cmidrule(lr){2-14}
 & Promptgator & 0.3441 & 0.3670 & 0.1538 & 0.1974 & 0.5928 & 0.7298 & \textcolor{red}{0.2526} & 0.2724 & \textcolor{red}{0.1161} & \textcolor{red}{0.1418} & 0.4718 & 0.5961 \\
 & w/ CCQGen (ours) & \textbf{0.3605}$^{\dagger}$ & \textbf{0.3991}$^{\dagger}$$^*$ & \textbf{0.1614}$^{\dagger}$ & \textbf{0.2194}$^{\dagger}$$^*$ & \textbf{0.6333}$^{\dagger}$$^*$ & \textbf{0.7467}$^{\dagger}$ & \textbf{0.2697}$^*$ & \textbf{0.2883}$^{\dagger}$$^*$ & \textbf{0.1267}$^{\dagger}$$^*$ & \textbf{0.1536}$^{\dagger}$$^*$ & \textbf{0.4983}$^{\dagger}$$^*$ & \textbf{0.6327}$^{\dagger}$$^*$ \\ \cmidrule(lr){2-14}
 & Pair-wise generation & 0.3418 & 0.3686 & \textcolor{red}{0.1522} & 0.1971 & 0.5961 & 0.7225 & \textcolor{red}{0.2541} & 0.2753 & \textcolor{red}{0.1177} & 0.1445 & 0.4809 & 0.5947 \\
 & w/ CCQGen (ours) & \textbf{0.3670}$^{\dagger}$$^*$ & \textbf{0.4063}$^{\dagger}$$^*$ & \textbf{0.1656}$^{\dagger}$$^*$ & \textbf{0.2228}$^{\dagger}$$^*$ & \textbf{0.6362}$^{\dagger}$$^*$ & \textbf{0.7526}$^{\dagger}$$^*$ & \textbf{0.2783}$^{\dagger}$$^*$ & \textbf{0.2943}$^{\dagger}$$^*$ & \textbf{0.1308}$^{\dagger}$$^*$ & \textbf{0.1577}$^{\dagger}$$^*$ & \textbf{0.5089}$^{\dagger}$$^*$ & \textbf{0.6331}$^{\dagger}$$^*$ \\ \midrule
\parbox[t]{2mm}{\multirow{7}{*}{\rotatebox[origin=c]{90}{SPECTER-v2}}} & no Fine-Tune & 0.3503 & 0.3579 & 0.1615 & 0.2043 & 0.5341 & 0.6859 & 0.2121 & 0.2283 & 0.0942 & 0.1147 & 0.4182 & 0.5441 \\
 & GenQ & 0.3658 & 0.3659 & 0.1699 & 0.2083 & 0.5541 & \textcolor{red}{0.6836} & 0.2338 & 0.2525 & 0.1045 & 0.1287 & 0.4412 & 0.5613 \\
 & Promptgator\_diverse & 0.3672 & 0.3801 & 0.1721 & 0.2157 & 0.5687 & 0.6972 & 0.2469 & 0.2733 & 0.1121 & 0.1401 & 0.4843 & 0.6102 \\ \cmidrule(lr){2-14}
 & Promptgator & 0.3766 & 0.3886 & 0.1790 & 0.2245 & 0.5715 & 0.6962 & 0.2479 & 0.2713 & 0.1131 & 0.1398 & 0.4851 & 0.6064 \\
 & w/ CCQGen (ours) & \textbf{0.4105}$^{\dagger}$$^*$ & \textbf{0.4176}$^{\dagger}$$^*$ & \textbf{0.2085}$^{\dagger}$$^*$ & \textbf{0.2549}$^{\dagger}$$^*$ & \textbf{0.5886}$^{\dagger}$ & \textbf{0.7355}$^{\dagger}$$^*$ & \textbf{0.2634}$^{\dagger}$$^*$ & \textbf{0.2891}$^{\dagger}$$^*$ & \textbf{0.1226}$^{\dagger}$ & \textbf{0.1520}$^{\dagger}$$^*$ & \textbf{0.4988}$^{\dagger}$ & \textbf{0.6265}$^{\dagger}$ \\ \cmidrule(lr){2-14}
 & Pair-wise generation & 0.3870 & 0.3999 & 0.1966 & 0.2423 & 0.5722 & 0.6972 & 0.2523 & 0.2782 & 0.1163 & 0.1442 & 0.4885 & 0.6148 \\
 & w/ CCQGen (ours) & \textbf{0.4031}$^{\dagger}$$^*$ & \textbf{0.4150}$^{\dagger}$ & \textbf{0.2040}$^{\dagger}$ & \textbf{0.2534}$^{\dagger}$$^*$ & \textbf{0.5844}$^{\dagger}$ & \textbf{0.7333}$^{\dagger}$$^*$ & \textbf{0.2681}$^{\dagger}$$^*$ & \textbf{0.2932}$^{\dagger}$$^*$ & \textbf{0.1247}$^{\dagger}$ & \textbf{0.1546}$^{\dagger}$$^*$ & \textbf{0.5064}$^{\dagger}$ & \textbf{0.6304}$^{\dagger}$\\ \bottomrule
\end{tabular}}
\label{tab:main}
\vspace{-0.3cm}
\end{table*}

% 

\begin{table*}[t]
\caption{Retrieval performance comparison with various enhancement methods.
* indicates a significant difference (paired t-test, $p<0.05$) from the best baseline (i.e., the combination of the best existing query generation and enhancement methods). }
\renewcommand{\arraystretch}{0.9}
\resizebox{\linewidth}{!}{
\begin{tabular}{llllllllllllll}\toprule
% \multirow{2}{*}{\makecell[c]{\textbf{Backbone}\\ \textbf{retriever}}}
\multicolumn{1}{c}{\multirow{2}{*}{\parbox{1.5cm}{\textbf{Query} \\ \textbf{generation}}}} & \multicolumn{1}{l}{\multirow{2}{*}{\parbox{1.5cm}{\textbf{Retrieval} \\ \textbf{enhancement}}}} & \multicolumn{6}{c}{\textbf{CSFCube}} & \multicolumn{6}{c}{\textbf{DORIS-MAE}} \\ \cmidrule(lr){3-8} \cmidrule(lr){9-14}
 &  & \textbf{N@10} & \textbf{N@20} & \textbf{M@10} & \textbf{M@20} & \textbf{R@50} & \textbf{R@100} & \textbf{N@10} & \textbf{N@20} & \textbf{M@10} & \textbf{M@20} & \textbf{R@50} & \textbf{R@100} \\ \midrule
\multirow{3}{*}{\parbox{1.5cm}{\centering Pair-wise \\generation}} & Retriever & 0.3418 & 0.3686 & 0.1522 & 0.1971 & 0.5961 & 0.7225 & 0.2541 & 0.2753 & 0.1178 & 0.1445 & 0.4809 & 0.5947 \\
 & w/ GRF & 0.3401 & 0.3713 & 0.1540 & 0.2008 & 0.5778 & 0.7151 & 0.2535 & 0.2753 & 0.1147 & 0.1416 & 0.4832 & 0.6159 \\
 & w/ ToTER & \textbf{0.3745} & \textbf{0.4072} & \textbf{0.1719} & \textbf{0.2267} & \textbf{0.6352} & \textbf{0.7606} & \textbf{0.2932} & \textbf{0.3138} & \textbf{0.1381} & \textbf{0.1680} & \textbf{0.5361} & \textbf{0.6579} \\ \midrule
\multirow{4}{*}{\parbox{1.7cm}{\centering w/ CCQGen\\(ours)}} & Retriever & 0.3670 & 0.4063 & 0.1656 & 0.2228 & 0.6362 & 0.7526 & 0.2783 & 0.2943 & 0.1308 & 0.1577 & 0.5089 & 0.6331 \\
 & w/ GRF & 0.3741 & 0.4071 & 0.1715 & 0.2272 & 0.6288 & 0.7490 & 0.2709 & 0.2925 & 0.1262 & 0.1542 & 0.5138 & 0.6384 \\
 & w/ ToTER & 0.4023 & 0.4205 & 0.1844 & 0.2403 & \textbf{0.6441} & 0.7698 & 0.2965 & 0.3159 & 0.1394 & 0.1697 & 0.5391 & 0.6635 \\
 & w/ \proposedtwo (ours) & \textbf{0.4244}$^*$ & \textbf{0.4359}$^*$ & \textbf{0.2029}$^*$ & \textbf{0.2530}$^*$ & 0.6412 & \textbf{0.7792}$^*$ & \textbf{0.3034}$^*$ & \textbf{0.3237}$^*$ & \textbf{0.1438}$^*$ & \textbf{0.1745}$^*$ & \textbf{0.5588}$^*$ & \textbf{0.6818}$^*$ \\ \bottomrule
\end{tabular}}
\label{tab:csr}
\vspace{-0.3cm}
\end{table*}




\subsection{Performance Comparison}
\label{sec:experimentresult}

\subsubsection{\textbf{Effectiveness of \proposed}}
\label{result:CCQGen}
Table \ref{tab:main} presents retrieval performance after fine-tuning with various query generation methods.
\proposed consistently outperforms all baselines, achieving significant improvements across various metrics with both backbone models.
We observe that GenQ underperforms compared to LLM-based methods, showing the advantages of leveraging the text generation capability of LLMs.
Also, existing methods often fail to improve the backbone model (i.e., no Fine-Tune), particularly \ctr.
As it is trained on labeled data from general domains, it already captures overall textual similarities well, making further improvements challenging.
The consistent improvements by \proposed support its efficacy in generating queries that effectively represent the scientific documents.
Notably, Promptgator\_diverse struggles to produce consistent improvements.
We observe that it often generates redundant queries covering similar aspects, despite increased diversity in their expressions (further analysis provided in \cref{result:query_analysis}).
This underscores the importance of proper control over generated content and supports the validity of our~approach.


\smallsection{Impact of amount of training data}
In Figure \ref{fig:amount}, we further explore the retrieval performance by limiting the amount of training data, using \ctr as the backbone model.
The existing LLM-based generation method (i.e., Pair-wise gen.) shows limited performance under restricted data conditions and fails to fully benefit from an increasing volume of training data.
This supports our claim that the generated queries are often redundant and do not effectively introduce new training signals.
Conversely, \proposed consistently delivers considerable improvements, even with a limited number of queries.
\proposed guides each new query to complement the previous ones, allowing for reducing redundancy and fully leveraging the limited number of queries.



\begin{figure}[t]
\centering
\includegraphics[height=3cm]{images/limited_label_CSFCube_NDCG.png}
\includegraphics[height=3cm]{images/limited_label_CSFCube_Recall.png}\hspace{-0.1cm}\\ \hspace{-0.3cm}
\includegraphics[height=3cm]{images/limited_label_DORISMAE_NDCG.png}
\includegraphics[height=3cm]{images/limited_label_DORISMAE_Recall.png}\hspace{-0.1cm}
\caption{Results with varying amounts of training data.
x\% denotes setups using a random x\% of generated queries.
}
\label{fig:amount}
\end{figure}



\subsubsection{\textbf{Effectiveness of \proposed with CSR}}
In \cref{subsub:method_filtering}, we introduce \proposedtwo, designed to enhance retrieval using concept information from \proposed.
This technique aligns with the ongoing research direction of enhancing retrieval by integrating additional context not directly revealed from the queries and document \cite{ToTER, mackie2023generative, BERTQE, DensePRF}.
We compare \proposedtwo with two recent methods:
(1) \textbf{GRF}~\cite{mackie2023generative} generates relevant contexts by LLMs. For a fair comparison, we generate both topics and keywords, as used in \proposed.
(2) \textbf{ToTER}~\cite{ToTER} uses the topic distributions between queries and documents, with topics provided by~the~taxonomy.
\ctr is used as~the~backbone.

Table \ref{tab:csr} presents the retrieval performance with various enhancement methods.
\proposedtwo significantly improves the retrieval performance.
Notably, the combination of proposed concept-based query generation (\proposed) and enhancement (\proposedtwo) methods achieves significant improvements over the best existing solutions (i.e., Pair-wise gen. combined with ToTER).
GRF often degrades performance because the LLM-generated contexts are not tailored to target documents;
these contexts may be related but often not covered by documents in the corpus, potentially causing discrepancies in focused aspects.
Lastly, ToTER only considers topic-level information, which may be insufficient for providing find-grained details necessary to distinguish between topically-similar documents.




\subsection{Study of \proposed}


\subsubsection{\textbf{Analysis of generated queries}}
\label{result:query_analysis}
We analyze whether \proposed indeed reduces redundancy among the queries and includes a variety of related terms.
We introduce two criteria: (1) \textbf{redundancy}, measured as the average cosine similarity of term frequency vectors of queries.\footnote{We use CountVectorizer from the SciKit-Learn library.}
A high redundancy indicates that queries tend to cover similar aspects of the document.
(2) \textbf{lexical overlap}, measured as the average BM25 score between the queries and the document.
A higher lexical overlap indicates that queries tend to reuse~terms~from~the~document.


In Table~\ref{tab:query_analysis}, the generated queries show higher lexical overlap with the document compared to the actual user queries.
This shows that the generated queries tend to use a limited range of terms already present in the document, whereas actual user queries include a broader variety of terms.
With the ‘diverse condition’ (i.e., Promptgator\_diverse), the generated queries exhibit reduced lexical overlap and redundancy. 
However, this does not consistently lead to performance improvements.
The improved term usage often appears in common expressions, not necessarily enhancing concept coverage.
Conversely, \proposed directly guides each new query to complement the previous ones.
Also, \proposed incorporate concept-related terms not explicitly mentioned in the document via enrichment step (\cref{method:enrich}).
This provides more systematic controls over the generation, leading to consistent improvements.
% Furthermore, in Table~\ref{tab:case_study}, we provide a case study that compares two approaches.
% \proposed shows enhanced term usage and concept coverage, which supports its superiority in the previous experiments.



\subsubsection{\textbf{Effectiveness of concept coverage-based filtering}}
Figure~\ref{fig:filtering} presents the improvements achieved through the filtering step, which aims to remove low-quality queries that the document does not answer (\cref{subsub:method_filtering}).
As shown in Table~\ref{tab:csr}, \proposedtwo largely enhances retrieval accuracy by incorporating concept information.
This enhanced accuracy helps to accurately measure round-trip consistency, effectively improving the effects of fine-tuning.



\begin{table}[t]
\caption{Analysis of generated queries.
(a) Statistics of queries generated by different methods.
(b) Retriever performance (\specter on NDCG@10) after fine-tuning using the queries.
The average lexical overlap of actual queries is $13.32$ for \csfcube and $20.42$ for \dorismae.}
\textbf{\csfcube}
\resizebox{1.02\linewidth}{!}{
\begin{tabular}{l l l l} \toprule
\multirow{2}{*}{\textbf{ Query generation}} & \multicolumn{2}{c}{\textbf{(a) Query statistics}} &  \multirow{2}{*}{ \parbox{1.9cm}{\centering \textbf{(b) Retriever} \\\textbf{performance}}}\\ \cmidrule(lr){2-3}
 & \textbf{redundancy} ($\downarrow$) & \textbf{lexical overlap} ($\downarrow$)& \\ \midrule
Promptgator & 0.5072 & 31.51 & 0.3766 \\
w/ diverse condition & 0.4512 (-11.0\%) & \textbf{24.05} \textbf{(-23.7\%)} & 0.3672 (-2.6\%) \\
w/ \proposed & \textbf{0.3997} \textbf{(-21.2\%)} & 24.41 (-22.5\%) & \textbf{0.4105} \textbf{(+9.0\%)} \\ \bottomrule
\end{tabular}}
\vspace{0.03cm}
\textbf{\dorismae} 
\resizebox{\linewidth}{!}{
\begin{tabular}{l l l l} \toprule
\multirow{2}{*}{\textbf{Query generation}} & \multicolumn{2}{c}{\textbf{(a) Query statistics}} &  \multirow{2}{*}{ \parbox{1.9cm}{\centering \textbf{(b) Retriever} \\\textbf{performance}}}\\ \cmidrule(lr){2-3}
 & \textbf{redundancy} ($\downarrow$) & \textbf{lexical overlap} ($\downarrow$) & \\ \midrule
Promptgator	&0.4861 & 53.58&	0.2479 \\
w/ diverse condition&	\textbf{0.3958 (-18.6\%)} & 41.56 (-22.4\%)&	0.2469 (-0.4\%) \\
w/ \proposed & 0.3993 (-17.9\%) & \textbf{40.54 (-24.3\%)}	&	\textbf{0.2634 (+6.2\%)} \\ \bottomrule
\end{tabular}}
\label{tab:query_analysis}
\vspace{-0.4cm}
\end{table}



\begin{figure}[t]
\centering
\includegraphics[height=2.3cm]{images/Delta_NDCG.png}
\includegraphics[height=2.3cm]{images/Delta_Recall.png}\hspace{-0.1cm}
\caption{Improvements by concept coverage-based filtering.}
\label{fig:filtering}
\end{figure}




\subsubsection{\textbf{Results with a smaller LLM}}
\label{result:Llama}
In Table~\ref{tab:Llama}, we explore the effectiveness of the proposed approach using a smaller LLM, Llama-3-8B, with \ctr as the backbone model. 
Consistent with the trends observed in Table~\ref{tab:main} and Table~\ref{tab:csr}, the proposed techniques (\proposed and \proposedtwo) consistently improve the existing method.
We expect \proposed to be effective with existing LLMs that possess a certain degree of capability. 
Since comparing different LLMs is not the focus of this work, we leave further investigation on more various LLMs and their comparison for future study.



% \begin{table}[t]
%     \caption{A case study on synthetic queries on \dorismae (Doc id: 167551).
%     The conditioned aspects are denoted in bold.}
%     \small
%     \centering
%     %\renewcommand{\arraystretch}{0.9}
%     \resizebox{1.02\linewidth}{!}{
%     \begin{tabular}{C} \toprule
%     \textbf{Synthetic queries generated by \proposed} \\ \midrule
%     \begin{itemize}[leftmargin=-0.001pt, after={\vspace*{-\baselineskip}}]
%         \item[1] In what ways does \textbf{implicit differentiation} in \textbf{implicit MAML algorithm} contribute to enhancing efficiency and performance of \textbf{meta-learning}? $\quad\quad\quad$
%         - Condition: MAML algorithm, meta-learning, implicit differentiation 
%         \item[2] How does implicit differentiation in \textbf{scaling meta-learning systems} contribute to \textbf{theoretical proof} of accurate \textbf{meta-parameter computation}? $\quad\quad\quad\quad\quad$
%         - Condition: theoretical proof, meta-parameters, scaling
%         \item[3] How do \textbf{few-shot learning} scenarios benefit from \textbf{meta-learning} techniques through efficient \textbf{meta-gradient computation}? \phantom{This text invisible.}
%         - Condition: few-shot learning, meta-learning, meta-gradient computation
%     \end{itemize}\vspace{-\topsep}\\ \midrule
%     \textbf{Synthetic queries generated by Promptgator\_diverse} \\ \midrule
%     \begin{itemize}[leftmargin=-0.01pt, after={\vspace*{-1\baselineskip}}]
%         \item[1] What are the key advantages of implicit MAML compared to traditional gradient-based meta-learning approaches?
%         \item[2] In what ways does the implicit MAML algorithm demonstrate empirical gains on few-shot image recognition benchmarks?
%         \item[3] How does the implicit MAML algorithm contribute to the scalability of gradient-based meta-learning approaches?
%     \end{itemize}\vspace{-\topsep}\\ \bottomrule
%     \end{tabular}}
%     \label{tab:case_study}
%     \vspace{-0.4cm}
% \end{table}



\begin{table}[t]
\caption{Retrieval performance with Llama-3-8B. 
We report improvements over no Fine-Tune. $^*$ denotes $p < 0.05$ from paired t-test with pair-wise generation.}
\centering
\renewcommand{\arraystretch}{0.85}
\resizebox{\linewidth}{!}{
\begin{tabular}{c l lll} \toprule
\textbf{Dataset} & \textbf{Method} & \textbf{N@10} & \textbf{N@20} & \textbf{R@100} \\ \midrule
\multirow{3}{*}{CSFCube} & Pair-wise generation & +5.25\% & +0.94\% & \textcolor{red}{- 0.21\%} \\
 & w/ CCQGen & +6.55\% & +7.82\%$^*$ & +5.48\%$^*$ \\
 & w/ CCQGen + \proposedtwo & +27.92\%$^*$ & +20.09\%$^*$ & +9.01\%$^*$ \\ \midrule
\multirow{3}{*}{DORIS-MAE} & Pair-wise generation & +0.00\% & +2.92\% & +5.43\% \\
 & w/ CCQGen & +5.19\%$^*$ & +9.57\%$^*$ & +6.69\% \\
 & w/ CCQGen + \proposedtwo & +16.75\%$^*$ & +20.87\%$^*$ & +14.65\%$^*$\\ \bottomrule
\end{tabular}}
\label{tab:Llama}
\end{table}




\section{Related Work}
\label{sec:relatedwork}
We provide a detailed survey of LLM-based query generation~in~\cref{prelim:qgen}.

\smallsection{PLM-based retrieval models}
The advancement of PLMs has led to significant progress in retrieval.
Recent studies have introduced retrieval-targeted pre-training \cite{CTR, condenser}, distillation from cross-encoders \cite{AR2}, and advanced negative mining methods \cite{zhan2021optimizing, rocketqa_v1}
There is also an increasing emphasis on pre-training methods specifically designed for the scientific domain. 
In addition to pre-training on academic corpora \cite{SCIBERT}, researchers have exploited metadata associated with scientific papers. 
\cite{razdaibiedina2023miread} uses journal class, \cite{SPECTER, SCINCL} use citations, \cite{ASPIRE} uses co-citation contexts, and \cite{OAGBERT} utilizes venues, affiliations, and authors.
\cite{SPECTER2, zhang2023pre} devise multi-task learning of related tasks such as citation prediction~and~paper~classification.
Very recently, \cite{ToTER, taxoindex} leverage corpus-structured knowledge (e.g., core topics and phrases) for academic concept matching.

% These models can be used as a backbone retriever for \proposed framework.


\smallsection{Synthetic query generation}
Earlier studies \cite{nogueira2019document, nogueira2019doc2query, liang2020embedding, ma2021zero, wang2022gpl} have employed dedicated query generation models, trained using massive document-query pairs from general domains.
Recently, there has been a shift towards replacing these generation models with LLMs \cite{dai2022promptagator, inpars, inpars2, pairwise_qgen, saad2023udapdr, sachan2022improving}, as discussed in \cref{prelim:qgen}.
On the other hand, many recent studies have focused on developing query generation tailored to specific retrieval domains.
\cite{synthetic_apple_VA} focuses on entity search for virtual assistants,
\cite{shen2022diversified} improves the diversity of queries for news article searches guided by a knowledge graph,
\cite{controllable_QGen} focuses on enhancing the retrievability of entities on online content (e.g., Podcast) platforms.
However, a dedicated method for scientific document retrieval has not been studied well in the literature.

    


\vspace{-0.3cm}
 


% Synthetic Query Generation using Large Language Models for Virtual Assistants, SIGIR'24
% Diversified Query Generation Guided by Knowledge Graph, WSDM'22


\section{Conclusion}
\label{sec:conclusion}
In this paper, we systematically study how to enhance the logical reasoning ability of LLMs by addressing their limitations in reasoning order variations. We introduce an order-centric data augmentation framework based on the principles of independency and commutativity in logical reasoning. Our method involves shuffling independent premises to introduce order variations and constructing directed acyclic graphs (DAGs) to identify valid step reorderings while preserving logical dependency. Extensive experiments across multiple logical reasoning benchmarks demonstrate that our method significantly improves LLMs’ reasoning performance and their adaptability to diverse logical structures.



\noindent
\textbf{Acknowledgements.}
This work was supported IITP grant funded by MSIT (No.2018-0-00584, No.2019-0-01906), NRF grant funded by the MSIT (No.RS-2023-00217286, No.2020R1A2B5B03097210).
It was also in part by US DARPA INCAS Program No. HR0011-21-C0165 and BRIES Program No. HR0011-24-3-0325, National Science Foundation IIS-19-56151, the Molecule Maker Lab Institute: An AI Research Institutes program supported by NSF under Award No. 2019897, and the Institute for Geospatial Understanding through an Integrative Discovery Environment (I-GUIDE) by NSF under Award No. 2118329.


\pagebreak
\newpage
\clearpage

\bibliographystyle{ACM-Reference-Format}         
\bibliography{main}

\end{document}
\endinput
%%
%% End of file `sample-sigconf.tex'.
