%%%% ijcai25.tex

\typeout{IJCAI--25 Instructions for Authors}

% These are the instructions for authors for IJCAI-25.

\documentclass{article}
\pdfpagewidth=8.5in
\pdfpageheight=11in

% The file ijcai25.sty is a copy from ijcai22.sty
% The file ijcai22.sty is NOT the same as previous years'
\usepackage{ijcai25}

% Use the postscript times font!
\usepackage{times}
\usepackage{soul}
\usepackage{url}
\usepackage[hidelinks]{hyperref}
\usepackage[utf8]{inputenc}
\usepackage[small]{caption}
\usepackage{graphicx}
\usepackage{amsmath}
\usepackage{amsthm}
\usepackage{booktabs}
\usepackage{algorithm}
\usepackage{algorithmic}
\usepackage[switch]{lineno}

\usepackage{bm}
\usepackage{multirow}
\usepackage{tabularx}
\usepackage{caption}
\usepackage{subcaption}
\let\Bbbk\relax
\usepackage{amssymb}
\usepackage{balance}

% Comment out this line in the camera-ready submission
% \linenumbers

\urlstyle{same}

% the following package is optional:
%\usepackage{latexsym}

% See https://www.overleaf.com/learn/latex/theorems_and_proofs
% for a nice explanation of how to define new theorems, but keep
% in mind that the amsthm package is already included in this
% template and that you must *not* alter the styling.
% \newtheorem{example}{Example}
% \newtheorem{theorem}{Theorem}

\newcommand{\stab}{\vspace{1.2ex}\noindent}
\newcommand{\sstab}{\rule{0pt}{8pt}\\[-2.2ex]}
\newcommand{\stitle}[1]{\sstab\noindent{\bf #1}}
\newcommand{\etitle}[1]{\vspace{1mm}\noindent{\underline{\em #1}}}
\newcommand{\ie}{{\em i.e.,}\xspace}
\newcommand{\eg}{{\em e.g.,}\xspace}
\newcommand{\wrt}{\emph{w.r.t.}\xspace}
\newcommand{\aka}{\emph{a.k.a.}\xspace}



% Single author syntax
% \author{
%     Anonymous Author(s)
%     % Author Name
%     % \affiliations
%     % Affiliation
%     % \emails
%     % email@example.com
% }

% Multiple author syntax (remove the single-author syntax above and the \iffalse ... \fi here)
% \iffalse
\author{
Xudong Yang$^1$
\and
Yizhang Zhu$^1$\and
Nan Tang$^{1,2}$\And
Yuyu Luo$^{1,2}$\\
\affiliations
\small{$^1$The Hong Kong University of Science and Technology (Guangzhou), Guangzhou, China} \\
\small{$^2$The Hong Kong University of Science and Technology, Hong Kong SAR, China} \\
% $^3$Third Affiliation\\
% $^4$Fourth Affiliation\\
\emails
\small{sootungyoung@gmail.com,
yzhu305@connect.hkust-gz.edu.cn
\{nantang, yuyuluo\}@hkust-gz.edu.cn}
}
% \fi

\begin{document}

\title{RAMer: Reconstruction-based Adversarial Model for Multi-party Multi-modal Multi-label Emotion Recognition}

\maketitle

\begin{abstract}
Conventional Multi-modal multi-label emotion recognition (MMER) from videos typically assumes full availability of visual, textual, and acoustic modalities. However, real-world \textit{multi-party} settings often violate this assumption, as non-speakers frequently lack acoustic and textual inputs, leading to a significant degradation in model performance. Existing approaches also tend to unify heterogeneous modalities into a single representation, overlooking each modality’s unique characteristics.
To address these challenges, we propose \textbf{RAMer} (\textbf{R}econstruction-based \textbf{A}dversarial \textbf{M}odel for \textbf{E}motion \textbf{R}ecognition), which leverages adversarial learning to refine multi-modal representations by exploring both modality commonality and specificity through reconstructed features enhanced by contrastive learning. \textbf{RAMer} also introduces a personality auxiliary task to complement missing modalities using modality-level attention, improving emotion reasoning. To further strengthen the model's ability to capture label and modality interdependency, we propose a stack shuffle strategy to enrich correlations between labels and modality-specific features.
Experiments on three benchmarks, i.e., MEmoR, CMU-MOSEI, and $M^3$ED, demonstrate that \textbf{RAMer} achieves state-of-the-art performance in dyadic and multi-party MMER scenarios. The code will be publicly available at https://github.com/Sootung/RAMer
\end{abstract}


\section{Introduction}
\label{sec:introduction}
The business processes of organizations are experiencing ever-increasing complexity due to the large amount of data, high number of users, and high-tech devices involved \cite{martin2021pmopportunitieschallenges, beerepoot2023biggestbpmproblems}. This complexity may cause business processes to deviate from normal control flow due to unforeseen and disruptive anomalies \cite{adams2023proceddsriftdetection}. These control-flow anomalies manifest as unknown, skipped, and wrongly-ordered activities in the traces of event logs monitored from the execution of business processes \cite{ko2023adsystematicreview}. For the sake of clarity, let us consider an illustrative example of such anomalies. Figure \ref{FP_ANOMALIES} shows a so-called event log footprint, which captures the control flow relations of four activities of a hypothetical event log. In particular, this footprint captures the control-flow relations between activities \texttt{a}, \texttt{b}, \texttt{c} and \texttt{d}. These are the causal ($\rightarrow$) relation, concurrent ($\parallel$) relation, and other ($\#$) relations such as exclusivity or non-local dependency \cite{aalst2022pmhandbook}. In addition, on the right are six traces, of which five exhibit skipped, wrongly-ordered and unknown control-flow anomalies. For example, $\langle$\texttt{a b d}$\rangle$ has a skipped activity, which is \texttt{c}. Because of this skipped activity, the control-flow relation \texttt{b}$\,\#\,$\texttt{d} is violated, since \texttt{d} directly follows \texttt{b} in the anomalous trace.
\begin{figure}[!t]
\centering
\includegraphics[width=0.9\columnwidth]{images/FP_ANOMALIES.png}
\caption{An example event log footprint with six traces, of which five exhibit control-flow anomalies.}
\label{FP_ANOMALIES}
\end{figure}

\subsection{Control-flow anomaly detection}
Control-flow anomaly detection techniques aim to characterize the normal control flow from event logs and verify whether these deviations occur in new event logs \cite{ko2023adsystematicreview}. To develop control-flow anomaly detection techniques, \revision{process mining} has seen widespread adoption owing to process discovery and \revision{conformance checking}. On the one hand, process discovery is a set of algorithms that encode control-flow relations as a set of model elements and constraints according to a given modeling formalism \cite{aalst2022pmhandbook}; hereafter, we refer to the Petri net, a widespread modeling formalism. On the other hand, \revision{conformance checking} is an explainable set of algorithms that allows linking any deviations with the reference Petri net and providing the fitness measure, namely a measure of how much the Petri net fits the new event log \cite{aalst2022pmhandbook}. Many control-flow anomaly detection techniques based on \revision{conformance checking} (hereafter, \revision{conformance checking}-based techniques) use the fitness measure to determine whether an event log is anomalous \cite{bezerra2009pmad, bezerra2013adlogspais, myers2018icsadpm, pecchia2020applicationfailuresanalysispm}. 

The scientific literature also includes many \revision{conformance checking}-independent techniques for control-flow anomaly detection that combine specific types of trace encodings with machine/deep learning \cite{ko2023adsystematicreview, tavares2023pmtraceencoding}. Whereas these techniques are very effective, their explainability is challenging due to both the type of trace encoding employed and the machine/deep learning model used \cite{rawal2022trustworthyaiadvances,li2023explainablead}. Hence, in the following, we focus on the shortcomings of \revision{conformance checking}-based techniques to investigate whether it is possible to support the development of competitive control-flow anomaly detection techniques while maintaining the explainable nature of \revision{conformance checking}.
\begin{figure}[!t]
\centering
\includegraphics[width=\columnwidth]{images/HIGH_LEVEL_VIEW.png}
\caption{A high-level view of the proposed framework for combining \revision{process mining}-based feature extraction with dimensionality reduction for control-flow anomaly detection.}
\label{HIGH_LEVEL_VIEW}
\end{figure}

\subsection{Shortcomings of \revision{conformance checking}-based techniques}
Unfortunately, the detection effectiveness of \revision{conformance checking}-based techniques is affected by noisy data and low-quality Petri nets, which may be due to human errors in the modeling process or representational bias of process discovery algorithms \cite{bezerra2013adlogspais, pecchia2020applicationfailuresanalysispm, aalst2016pm}. Specifically, on the one hand, noisy data may introduce infrequent and deceptive control-flow relations that may result in inconsistent fitness measures, whereas, on the other hand, checking event logs against a low-quality Petri net could lead to an unreliable distribution of fitness measures. Nonetheless, such Petri nets can still be used as references to obtain insightful information for \revision{process mining}-based feature extraction, supporting the development of competitive and explainable \revision{conformance checking}-based techniques for control-flow anomaly detection despite the problems above. For example, a few works outline that token-based \revision{conformance checking} can be used for \revision{process mining}-based feature extraction to build tabular data and develop effective \revision{conformance checking}-based techniques for control-flow anomaly detection \cite{singh2022lapmsh, debenedictis2023dtadiiot}. However, to the best of our knowledge, the scientific literature lacks a structured proposal for \revision{process mining}-based feature extraction using the state-of-the-art \revision{conformance checking} variant, namely alignment-based \revision{conformance checking}.

\subsection{Contributions}
We propose a novel \revision{process mining}-based feature extraction approach with alignment-based \revision{conformance checking}. This variant aligns the deviating control flow with a reference Petri net; the resulting alignment can be inspected to extract additional statistics such as the number of times a given activity caused mismatches \cite{aalst2022pmhandbook}. We integrate this approach into a flexible and explainable framework for developing techniques for control-flow anomaly detection. The framework combines \revision{process mining}-based feature extraction and dimensionality reduction to handle high-dimensional feature sets, achieve detection effectiveness, and support explainability. Notably, in addition to our proposed \revision{process mining}-based feature extraction approach, the framework allows employing other approaches, enabling a fair comparison of multiple \revision{conformance checking}-based and \revision{conformance checking}-independent techniques for control-flow anomaly detection. Figure \ref{HIGH_LEVEL_VIEW} shows a high-level view of the framework. Business processes are monitored, and event logs obtained from the database of information systems. Subsequently, \revision{process mining}-based feature extraction is applied to these event logs and tabular data input to dimensionality reduction to identify control-flow anomalies. We apply several \revision{conformance checking}-based and \revision{conformance checking}-independent framework techniques to publicly available datasets, simulated data of a case study from railways, and real-world data of a case study from healthcare. We show that the framework techniques implementing our approach outperform the baseline \revision{conformance checking}-based techniques while maintaining the explainable nature of \revision{conformance checking}.

In summary, the contributions of this paper are as follows.
\begin{itemize}
    \item{
        A novel \revision{process mining}-based feature extraction approach to support the development of competitive and explainable \revision{conformance checking}-based techniques for control-flow anomaly detection.
    }
    \item{
        A flexible and explainable framework for developing techniques for control-flow anomaly detection using \revision{process mining}-based feature extraction and dimensionality reduction.
    }
    \item{
        Application to synthetic and real-world datasets of several \revision{conformance checking}-based and \revision{conformance checking}-independent framework techniques, evaluating their detection effectiveness and explainability.
    }
\end{itemize}

The rest of the paper is organized as follows.
\begin{itemize}
    \item Section \ref{sec:related_work} reviews the existing techniques for control-flow anomaly detection, categorizing them into \revision{conformance checking}-based and \revision{conformance checking}-independent techniques.
    \item Section \ref{sec:abccfe} provides the preliminaries of \revision{process mining} to establish the notation used throughout the paper, and delves into the details of the proposed \revision{process mining}-based feature extraction approach with alignment-based \revision{conformance checking}.
    \item Section \ref{sec:framework} describes the framework for developing \revision{conformance checking}-based and \revision{conformance checking}-independent techniques for control-flow anomaly detection that combine \revision{process mining}-based feature extraction and dimensionality reduction.
    \item Section \ref{sec:evaluation} presents the experiments conducted with multiple framework and baseline techniques using data from publicly available datasets and case studies.
    \item Section \ref{sec:conclusions} draws the conclusions and presents future work.
\end{itemize}
\section{RELATED WORK}
\label{sec:relatedwork}
In this section, we describe the previous works related to our proposal, which are divided into two parts. In Section~\ref{sec:relatedwork_exoplanet}, we present a review of approaches based on machine learning techniques for the detection of planetary transit signals. Section~\ref{sec:relatedwork_attention} provides an account of the approaches based on attention mechanisms applied in Astronomy.\par

\subsection{Exoplanet detection}
\label{sec:relatedwork_exoplanet}
Machine learning methods have achieved great performance for the automatic selection of exoplanet transit signals. One of the earliest applications of machine learning is a model named Autovetter \citep{MCcauliff}, which is a random forest (RF) model based on characteristics derived from Kepler pipeline statistics to classify exoplanet and false positive signals. Then, other studies emerged that also used supervised learning. \cite{mislis2016sidra} also used a RF, but unlike the work by \citet{MCcauliff}, they used simulated light curves and a box least square \citep[BLS;][]{kovacs2002box}-based periodogram to search for transiting exoplanets. \citet{thompson2015machine} proposed a k-nearest neighbors model for Kepler data to determine if a given signal has similarity to known transits. Unsupervised learning techniques were also applied, such as self-organizing maps (SOM), proposed \citet{armstrong2016transit}; which implements an architecture to segment similar light curves. In the same way, \citet{armstrong2018automatic} developed a combination of supervised and unsupervised learning, including RF and SOM models. In general, these approaches require a previous phase of feature engineering for each light curve. \par

%DL is a modern data-driven technology that automatically extracts characteristics, and that has been successful in classification problems from a variety of application domains. The architecture relies on several layers of NNs of simple interconnected units and uses layers to build increasingly complex and useful features by means of linear and non-linear transformation. This family of models is capable of generating increasingly high-level representations \citep{lecun2015deep}.

The application of DL for exoplanetary signal detection has evolved rapidly in recent years and has become very popular in planetary science.  \citet{pearson2018} and \citet{zucker2018shallow} developed CNN-based algorithms that learn from synthetic data to search for exoplanets. Perhaps one of the most successful applications of the DL models in transit detection was that of \citet{Shallue_2018}; who, in collaboration with Google, proposed a CNN named AstroNet that recognizes exoplanet signals in real data from Kepler. AstroNet uses the training set of labelled TCEs from the Autovetter planet candidate catalog of Q1–Q17 data release 24 (DR24) of the Kepler mission \citep{catanzarite2015autovetter}. AstroNet analyses the data in two views: a ``global view'', and ``local view'' \citep{Shallue_2018}. \par


% The global view shows the characteristics of the light curve over an orbital period, and a local view shows the moment at occurring the transit in detail

%different = space-based

Based on AstroNet, researchers have modified the original AstroNet model to rank candidates from different surveys, specifically for Kepler and TESS missions. \citet{ansdell2018scientific} developed a CNN trained on Kepler data, and included for the first time the information on the centroids, showing that the model improves performance considerably. Then, \citet{osborn2020rapid} and \citet{yu2019identifying} also included the centroids information, but in addition, \citet{osborn2020rapid} included information of the stellar and transit parameters. Finally, \citet{rao2021nigraha} proposed a pipeline that includes a new ``half-phase'' view of the transit signal. This half-phase view represents a transit view with a different time and phase. The purpose of this view is to recover any possible secondary eclipse (the object hiding behind the disk of the primary star).


%last pipeline applies a procedure after the prediction of the model to obtain new candidates, this process is carried out through a series of steps that include the evaluation with Discovery and Validation of Exoplanets (DAVE) \citet{kostov2019discovery} that was adapted for the TESS telescope.\par
%



\subsection{Attention mechanisms in astronomy}
\label{sec:relatedwork_attention}
Despite the remarkable success of attention mechanisms in sequential data, few papers have exploited their advantages in astronomy. In particular, there are no models based on attention mechanisms for detecting planets. Below we present a summary of the main applications of this modeling approach to astronomy, based on two points of view; performance and interpretability of the model.\par
%Attention mechanisms have not yet been explored in all sub-areas of astronomy. However, recent works show a successful application of the mechanism.
%performance

The application of attention mechanisms has shown improvements in the performance of some regression and classification tasks compared to previous approaches. One of the first implementations of the attention mechanism was to find gravitational lenses proposed by \citet{thuruthipilly2021finding}. They designed 21 self-attention-based encoder models, where each model was trained separately with 18,000 simulated images, demonstrating that the model based on the Transformer has a better performance and uses fewer trainable parameters compared to CNN. A novel application was proposed by \citet{lin2021galaxy} for the morphological classification of galaxies, who used an architecture derived from the Transformer, named Vision Transformer (VIT) \citep{dosovitskiy2020image}. \citet{lin2021galaxy} demonstrated competitive results compared to CNNs. Another application with successful results was proposed by \citet{zerveas2021transformer}; which first proposed a transformer-based framework for learning unsupervised representations of multivariate time series. Their methodology takes advantage of unlabeled data to train an encoder and extract dense vector representations of time series. Subsequently, they evaluate the model for regression and classification tasks, demonstrating better performance than other state-of-the-art supervised methods, even with data sets with limited samples.

%interpretation
Regarding the interpretability of the model, a recent contribution that analyses the attention maps was presented by \citet{bowles20212}, which explored the use of group-equivariant self-attention for radio astronomy classification. Compared to other approaches, this model analysed the attention maps of the predictions and showed that the mechanism extracts the brightest spots and jets of the radio source more clearly. This indicates that attention maps for prediction interpretation could help experts see patterns that the human eye often misses. \par

In the field of variable stars, \citet{allam2021paying} employed the mechanism for classifying multivariate time series in variable stars. And additionally, \citet{allam2021paying} showed that the activation weights are accommodated according to the variation in brightness of the star, achieving a more interpretable model. And finally, related to the TESS telescope, \citet{morvan2022don} proposed a model that removes the noise from the light curves through the distribution of attention weights. \citet{morvan2022don} showed that the use of the attention mechanism is excellent for removing noise and outliers in time series datasets compared with other approaches. In addition, the use of attention maps allowed them to show the representations learned from the model. \par

Recent attention mechanism approaches in astronomy demonstrate comparable results with earlier approaches, such as CNNs. At the same time, they offer interpretability of their results, which allows a post-prediction analysis. \par


\section{Research Methodology}~\label{sec:Methodology}

In this section, we discuss the process of conducting our systematic review, e.g., our search strategy for data extraction of relevant studies, based on the guidelines of Kitchenham et al.~\cite{kitchenham2022segress} to conduct SLRs and Petersen et al.~\cite{PETERSEN20151} to conduct systematic mapping studies (SMSs) in Software Engineering. In this systematic review, we divide our work into a four-stage procedure, including planning, conducting, building a taxonomy, and reporting the review, illustrated in Fig.~\ref{fig:search}. The four stages are as follows: (1) the \emph{planning} stage involved identifying research questions (RQs) and specifying the detailed research plan for the study; (2) the \emph{conducting} stage involved analyzing and synthesizing the existing primary studies to answer the research questions; (3) the \emph{taxonomy} stage was introduced to optimize the data extraction results and consolidate a taxonomy schema for REDAST methodology; (4) the \emph{reporting} stage involved the reviewing, concluding and reporting the final result of our study.

\begin{figure}[!t]
    \centering
    \includegraphics[width=1\linewidth]{fig/methodology/searching-process.drawio.pdf}
    \caption{Systematic Literature Review Process}
    \label{fig:search}
\end{figure}

\subsection{Research Questions}
In this study, we developed five research questions (RQs) to identify the input and output, analyze technologies, evaluate metrics, identify challenges, and identify potential opportunities. 

\textbf{RQ1. What are the input configurations, formats, and notations used in the requirements in requirements-driven
automated software testing?} In requirements-driven testing, the input is some form of requirements specification -- which can vary significantly. RQ1 maps the input for REDAST and reports on the comparison among different formats for requirements specification.

\textbf{RQ2. What are the frameworks, tools, processing methods, and transformation techniques used in requirements-driven automated software testing studies?} RQ2 explores the technical solutions from requirements to generated artifacts, e.g., rule-based transformation applying natural language processing (NLP) pipelines and deep learning (DL) techniques, where we additionally discuss the potential intermediate representation and additional input for the transformation process.

\textbf{RQ3. What are the test formats and coverage criteria used in the requirements-driven automated software
testing process?} RQ3 focuses on identifying the formulation of generated artifacts (i.e., the final output). We map the adopted test formats and analyze their characteristics in the REDAST process.

\textbf{RQ4. How do existing studies evaluate the generated test artifacts in the requirements-driven automated software testing process?} RQ4 identifies the evaluation datasets, metrics, and case study methodologies in the selected papers. This aims to understand how researchers assess the effectiveness, accuracy, and practical applicability of the generated test artifacts.

\textbf{RQ5. What are the limitations and challenges of existing requirements-driven automated software testing methods in the current era?} RQ5 addresses the limitations and challenges of existing studies while exploring future directions in the current era of technology development. %It particularly highlights the potential benefits of advanced LLMs and examines their capacity to meet the high expectations placed on these cutting-edge language modeling technologies. %\textcolor{blue}{CA: Do we really need to focus on LLMs? TBD.} \textcolor{orange}{FW: About LLMs, I removed the direct emphase in RQ5 but kept the discussion in RQ5 and the solution section. I think that would be more appropriate.}

\subsection{Searching Strategy}

The overview of the search process is exhibited in Fig. \ref{fig:papers}, which includes all the details of our search steps.
\begin{table}[!ht]
\caption{List of Search Terms}
\label{table:search_term}
\begin{tabularx}{\textwidth}{lX}
\hline
\textbf{Terms Group} & \textbf{Terms} \\ \hline
Test Group & test* \\
Requirement Group & requirement* OR use case* OR user stor* OR specification* \\
Software Group & software* OR system* \\
Method Group & generat* OR deriv* OR map* OR creat* OR extract* OR design* OR priorit* OR construct* OR transform* \\ \hline
\end{tabularx}
\end{table}

\begin{figure}
    \centering
    \includegraphics[width=1\linewidth]{fig/methodology/search-papers.drawio.pdf}
    \caption{Study Search Process}
    \label{fig:papers}
\end{figure}

\subsubsection{Search String Formulation}
Our research questions (RQs) guided the identification of the main search terms. We designed our search string with generic keywords to avoid missing out on any related papers, where four groups of search terms are included, namely ``test group'', ``requirement group'', ``software group'', and ``method group''. In order to capture all the expressions of the search terms, we use wildcards to match the appendix of the word, e.g., ``test*'' can capture ``testing'', ``tests'' and so on. The search terms are listed in Table~\ref{table:search_term}, decided after iterative discussion and refinement among all the authors. As a result, we finally formed the search string as follows:


\hangindent=1.5em
 \textbf{ON ABSTRACT} ((``test*'') \textbf{AND} (``requirement*'' \textbf{OR} ``use case*'' \textbf{OR} ``user stor*'' \textbf{OR} ``specifications'') \textbf{AND} (``software*'' \textbf{OR} ``system*'') \textbf{AND} (``generat*'' \textbf{OR} ``deriv*'' \textbf{OR} ``map*'' \textbf{OR} ``creat*'' \textbf{OR} ``extract*'' \textbf{OR} ``design*'' \textbf{OR} ``priorit*'' \textbf{OR} ``construct*'' \textbf{OR} ``transform*''))

The search process was conducted in September 2024, and therefore, the search results reflect studies available up to that date. We conducted the search process on six online databases: IEEE Xplore, ACM Digital Library, Wiley, Scopus, Web of Science, and Science Direct. However, some databases were incompatible with our default search string in the following situations: (1) unsupported for searching within abstract, such as Scopus, and (2) limited search terms, such as ScienceDirect. Here, for (1) situation, we searched within the title, keyword, and abstract, and for (2) situation, we separately executed the search and removed the duplicate papers in the merging process. 

\subsubsection{Automated Searching and Duplicate Removal}
We used advanced search to execute our search string within our selected databases, following our designed selection criteria in Table \ref{table:selection}. The first search returned 27,333 papers. Specifically for the duplicate removal, we used a Python script to remove (1) overlapped search results among multiple databases and (2) conference or workshop papers, also found with the same title and authors in the other journals. After duplicate removal, we obtained 21,652 papers for further filtering.

\begin{table*}[]
\caption{Selection Criteria}
\label{table:selection}
\begin{tabularx}{\textwidth}{lX}
\hline
\textbf{Criterion ID} & \textbf{Criterion Description} \\ \hline
S01          & Papers written in English. \\
S02-1        & Papers in the subjects of "Computer Science" or "Software Engineering". \\
S02-2        & Papers published on software testing-related issues. \\
S03          & Papers published from 1991 to the present. \\ 
S04          & Papers with accessible full text. \\ \hline
\end{tabularx}
\end{table*}

\begin{table*}[]
\small
\caption{Inclusion and Exclusion Criteria}
\label{table:criteria}
\begin{tabularx}{\textwidth}{lX}
\hline
\textbf{ID}  & \textbf{Description} \\ \hline
\multicolumn{2}{l}{\textbf{Inclusion Criteria}} \\ \hline
I01 & Papers about requirements-driven automated system testing or acceptance testing generation, or studies that generate system-testing-related artifacts. \\
I02 & Peer-reviewed studies that have been used in academia with references from literature. \\ \hline
\multicolumn{2}{l}{\textbf{Exclusion Criteria}} \\ \hline
E01 & Studies that only support automated code generation, but not test-artifact generation. \\
E02 & Studies that do not use requirements-related information as an input. \\
E03 & Papers with fewer than 5 pages (1-4 pages). \\
E04 & Non-primary studies (secondary or tertiary studies). \\
E05 & Vision papers and grey literature (unpublished work), books (chapters), posters, discussions, opinions, keynotes, magazine articles, experience, and comparison papers. \\ \hline
\end{tabularx}
\end{table*}

\subsubsection{Filtering Process}

In this step, we filtered a total of 21,652 papers using the inclusion and exclusion criteria outlined in Table \ref{table:criteria}. This process was primarily carried out by the first and second authors. Our criteria are structured at different levels, facilitating a multi-step filtering process. This approach involves applying various criteria in three distinct phases. We employed a cross-verification method involving (1) the first and second authors and (2) the other authors. Initially, the filtering was conducted separately by the first and second authors. After cross-verifying their results, the results were then reviewed and discussed further by the other authors for final decision-making. We widely adopted this verification strategy within the filtering stages. During the filtering process, we managed our paper list using a BibTeX file and categorized the papers with color-coding through BibTeX management software\footnote{\url{https://bibdesk.sourceforge.io/}}, i.e., “red” for irrelevant papers, “yellow” for potentially relevant papers, and “blue” for relevant papers. This color-coding system facilitated the organization and review of papers according to their relevance.

The screening process is shown below,
\begin{itemize}
    \item \textbf{1st-round Filtering} was based on the title and abstract, using the criteria I01 and E01. At this stage, the number of papers was reduced from 21,652 to 9,071.
    \item \textbf{2nd-round Filtering}. We attempted to include requirements-related papers based on E02 on the title and abstract level, which resulted from 9,071 to 4,071 papers. We excluded all the papers that did not focus on requirements-related information as an input or only mentioned the term ``requirements'' but did not refer to the requirements specification.
    \item \textbf{3rd-round Filtering}. We selectively reviewed the content of papers identified as potentially relevant to requirements-driven automated test generation. This process resulted in 162 papers for further analysis.
\end{itemize}
Note that, especially for third-round filtering, we aimed to include as many relevant papers as possible, even borderline cases, according to our criteria. The results were then discussed iteratively among all the authors to reach a consensus.

\subsubsection{Snowballing}

Snowballing is necessary for identifying papers that may have been missed during the automated search. Following the guidelines by Wohlin~\cite{wohlin2014guidelines}, we conducted both forward and backward snowballing. As a result, we identified 24 additional papers through this process.

\subsubsection{Data Extraction}

Based on the formulated research questions (RQs), we designed 38 data extraction questions\footnote{\url{https://drive.google.com/file/d/1yjy-59Juu9L3WHaOPu-XQo-j-HHGTbx_/view?usp=sharing}} and created a Google Form to collect the required information from the relevant papers. The questions included 30 short-answer questions, six checkbox questions, and two selection questions. The data extraction was organized into five sections: (1) basic information: fundamental details such as title, author, venue, etc.; (2) open information: insights on motivation, limitations, challenges, etc.; (3) requirements: requirements format, notation, and related aspects; (4) methodology: details, including immediate representation and technique support; (5) test-related information: test format(s), coverage, and related elements. Similar to the filtering process, the first and second authors conducted the data extraction and then forwarded the results to the other authors to initiate the review meeting.

\subsubsection{Quality Assessment}

During the data extraction process, we encountered papers with insufficient information. To address this, we conducted a quality assessment in parallel to ensure the relevance of the papers to our objectives. This approach, also adopted in previous secondary studies~\cite{shamsujjoha2021developing, naveed2024model}, involved designing a set of assessment questions based on guidelines by Kitchenham et al.~\cite{kitchenham2022segress}. The quality assessment questions in our study are shown below:
\begin{itemize}
    \item \textbf{QA1}. Does this study clearly state \emph{how} requirements drive automated test generation?
    \item \textbf{QA2}. Does this study clearly state the \emph{aim} of REDAST?
    \item \textbf{QA3}. Does this study enable \emph{automation} in test generation?
    \item \textbf{QA4}. Does this study demonstrate the usability of the method from the perspective of methodology explanation, discussion, case examples, and experiments?
\end{itemize}
QA4 originates from an open perspective in the review process, where we focused on evaluation, discussion, and explanation. Our review also examined the study’s overall structure, including the methodology description, case studies, experiments, and analyses. The detailed results of the quality assessment are provided in the Appendix. Following this assessment, the final data extraction was based on 156 papers.

% \begin{table}[]
% \begin{tabular}{ll}
% \hline
% QA ID & QA Questions                                             \\ \hline
% Q01   & Does this study clearly state its aims?                  \\
% Q02   & Does this study clearly describe its methodology?        \\
% Q03   & Does this study involve automated test generation?       \\
% Q04   & Does this study include a promising evaluation?          \\
% Q05   & Does this study demonstrate the usability of the method? \\ \hline
% \end{tabular}%
% \caption{Questions for Quality Assessment}
% \label{table:qa}
% \end{table}

% automated quality assessment

% \textcolor{blue}{CA: Our search strategy focused on identifying requirements types first. We covered several sources, e.g., ~\cite{Pohl:11,wagner2019status} to identify different formats and notations of specifying requirements. However, this came out to be a long list, e.g., free-form NL requirements, semi-formal UML models, free-from textual use case models, UML class diagrams, UML activity diagrams, and so on. In this paper, we attempted to primarily focus on requirements-related aspects and not design-level information. Hence, we generalised our search string to include generic keywords, e.g., requirement*, use case*, and user stor*. We did so to avoid missing out on any papers, bringing too restrictive in our search strategy, and not creating a too-generic search string with all the aforementioned formats to avoid getting results beyond our review's scope.}


%% Use \subsection commands to start a subsection.



%\subsection{Study Selection}

% In this step, we further looked into the content of searched papers using our search strategy and applied our inclusion and exclusion criteria. Our filtering strategy aimed to pinpoint studies focused on requirements-driven system-level testing. Recognizing the presence of irrelevant papers in our search results, we established detailed selection criteria for preliminary inclusion and exclusion, as shown in Table \ref{table: criteria}. Specifically, we further developed the taxonomy schema to exclude two types of studies that did not meet the requirements for system-level testing: (1) studies supporting specification-driven test generation, such as UML-driven test generation, rather than requirements-driven testing, and (2) studies focusing on code-based test generation, such as requirement-driven code generation for unit testing.




\section{Experiments}
% \label{headings}

\subsection{Experimental setup}

\paragraph{Datasets and Metrics.}
We evaluated RAMer on three multimodal, multi-label benchmark datasets: MEmoR~\cite{MEmoR}, a multi-party conversation dataset that includes personality, and CMU-MOSEI~\cite{Cmu-mosei} and $M^3$ED~\cite{M3ED}, which are dyadic conversation datasets that do not include personality information. The evaluation is conducted under the protocols of these datasets. For CMU-MOSEI and $M^3$ED, we employed four commonly used evaluation metrics: Accuracy (Acc), Micro-F1, Precision (P), and Recall (R). Due to data imbalance in MEmoR, we followed the benchmark's protocol and used Micro-F1, Macro-F1, and Weighted-F1 metrics. 
% Detailed descriptions of the datasets and preprocessing steps are provided in the Appendix.

% MEmor dataset comparison
\begin{table*}[t!]
\begin{center}
\caption{Performance comparison on the MEmoR dataset under primary and fine-grained settings.}
\vspace{-1em}
\parbox{\textwidth}{\centering \small \em With various modality combinations (visual($v$), acoustic($a$)), textual($t$), personality($p$)).}  
\vspace{-1em}
\label{tab:primary} 
\resizebox{0.95\textwidth}{!} % limit the width of the table to  text width
{ 
\begin{tabular}{cc|ccc|ccc}
\hline
\hline
\multirow{2}{*}{Methods} 
& \multirow{2}{*}{Modality} 
& \multicolumn{3}{c|}{Primary}       
& \multicolumn{3}{c}{Fine-grained}  \\ 
\cline{3-8} 

&                           
& Micro-F1 & Macro-F1 & Weighted-F1 
& Micro-F1 & Macro-F1 & Weighted-F1 
\\ 
\hline

MDL with Personality     & $v,a,t,p$                   
& 0.429    & 0.317    & 0.423       
& 0.363    & 0.217    & 0.345       
\\

MDAE                    & $v,a,t,p$                   
& 0.421    & 0.303    & 0.410       
& 0.363    & 0.219    & 0.341       
\\

BiLSTM+TFN             & $v,a,t,p$                   
& 0.470    & 0.310    & 0.454       
& 0.366    & 0.207    & 0.350       
\\

BiLSTM+LMF             & $v,a,t,p$                   
&0.449    & 0.294    & 0.432       
& 0.364    & 0.198    & 0.351       
\\

DialogueGCN            & $v,a,t,p$                   
& 0.441    & 0.310    & 0.425       
& 0.373    & 0.229    & 0.373       
\\

AMER w/o Personality     & $v,a,t$                     
& 0.446    & 0.339    & 0.440       
& 0.401    & 0.246    & 0.379       
\\

AMER                & $v,a,t,p$                   
& 0.477    & 0.353    & 0.465       
& 0.419    & 0.262    & 0.400       
\\ 

DialogueCRN     & $v,a,t,p$
& 0.441    & 0.310    & 0.425 
& 0.373    & 0.229    & 0.373 
\\
TAILOR               & $v,a,t,p$
& 0.341 & 0.287 & 0.326 
& 0.303 & 0.069 & \textbf{0.490} 
\\
CARAT              & $v,a,t,p$
& 0.399 & 0.224 & 0.422 
& 0.346 & 0.090 & 0.483 
\\
\hline
\textbf{RAMer}                   & $v,a,t,p$                  
& \textbf{0.499}    & \textbf{0.402}    & \textbf{0.503}       
& \textbf{0.431}    & \textbf{0.299}   & 0.404       
\\ 
\hline
\hline
\end{tabular}
} 
\end{center}
\end{table*}

% \usepackage{multirow}
\begin{table}[t]
\begin{center}
\caption{Performance Comparison on CMU-MOSEI dataset.} \label{tab:cmu-mosei} 
\vspace{-1em}
\resizebox{\columnwidth}{!}
{ 
\begin{tabular}{c|cccc|cccc}
    \hline
    \hline

    \multirow{2}{*}{Methods} & \multicolumn{4}{c|}{Aligned}      & \multicolumn{4}{c}{Unaligned}    \\ 
    \cline{2-9} 
                                  
                             
    & Acc   & P     & R     & Micro-F1 
    & Acc   & P     & R     & Micro-F1 \\ 
    \hline

    CC                    
    & 0.225 & 0.306 & 0.523 & 0.386    
    & 0.235 & 0.320 & 0.550 & 0.404    
    \\

    ML-GCN                   
    & 0.411 & 0.546 & 0.476 & 0.509    
    & 0.437 & 0.573 & 0.482 & 0.524    
    \\
    
    MulT                    
    & 0.445 & 0.619 & 0.465 & 0.531    
    & 0.423 & 0.636 & 0.445 & 0.523    
    \\
    
                                    
    MISA                    
    & 0.43  & 0.453 & \textbf{0.582} & 0.509    
    & 0.398 & 0.371 & \textbf{0.571} & 0.45     
    \\ 
    
    
    MMS2S                    
    & 0.475 & 0.629 & 0.504 & 0.56     
    & 0.447 & 0.619 & 0.462 & 0.529    
    \\  
    
    HHMPN                  
    & 0.459 & 0.602 & 0.496 & 0.556    
    & 0.434 & 0.591 & 0.476 & 0.528    
    \\ 
    
    TAILOR                  
    & 0.488 & 0.641 & 0.512 & 0.569    
    & 0.46  & 0.639 & 0.452 & 0.529    
    \\ 
    
    AMP                      
    & 0.484 & 0.643 & 0.511 & 0.569    
    & 0.462 & 0.642 & 0.459 & 0.535    
    \\  
    
    CARAT                   
    & 0.494 & 0.661 & 0.518 & 0.581    
    & 0.466 & 0.652 & 0.466 & 0.544    
    \\ 
    \hline 
    
    \textbf{RAMer}                   
    & \textbf{0.505} & \textbf{0.668} & 0.551 & \textbf{0.604}    
    & \textbf{0.469} & \textbf{0.660} & 0.486 & \textbf{0.560}    
    \\ 
    \hline
    \hline
\end{tabular}
}
\vspace{-1em}
\end{center}
\end{table}

\begin{table}[t]
% \vspace{-.3cm}
\begin{center}
\caption{Performance Comparison on the M$^3$ED dataset} \label{tab:m3ed} 
\vspace{-1em}
% \resizebox{\columnwidth}{!}
% { 
\small
\begin{tabular}{ccccc}
\hline
\hline
Methods & Acc   & P     & R     & Micro-F1 \\ 
\hline
MMS2S   & 0.645 & 0.813 & 0.737 & 0.773    \\
HHMPN   & 0.648 & 0.816 & 0.743 & 0.778    \\
TAILOR  & 0.647 & 0.814 & 0.739 & 0.775    \\
AMP     & 0.654 & 0.819 & 0.748 & 0.782    \\
CARAT   & 0.664 & 0.824 & 0.755 & 0.788    \\ 
\hline
\textbf{RAMer}  & \textbf{0.665} & \textbf{0.826} & \textbf{0.759} & \textbf{0.791}     \\ 
\hline
\hline
\end{tabular}
% }
\end{center}
\vspace{-1em}
\end{table}

% \subsubsection{Headings: third level}

\paragraph{Baselines.}
For the MEmoR dataset, we compare RAMer with multi-party conversation baselines, including MDL, MDAE~\cite{MDAE}, BiLSTM+TFN~\cite{TFN}, BiLSTM+LMF~\cite{LMF}, DialogueGCN~\cite{Dialoguegcn}, DialogueCRN~\cite{DialogueCRN}, and AMER~\cite{MEmoR}. We also evaluate its robustness against recent models designed for dyadic conversations, such as CARAT~\cite{CARAT} and TAILOR~\cite{Tailor}. For the CMU-MOSEI and $M^3$ED datasets, we test three categories of methods. 1) Classic methods. CC~\cite{CC}, which concatenates all available modalities as input for binary classifiers. 2) Deep-based methods. ML-GCN~\cite{chen2019multi}, using Graph Convolutional Networks to map label representations and capture label correlations. 3) Multi-modal multi-label methods. These include MulT~\cite{cross_adaptation} for cross-modal interactions, MISA~\cite{MISA} for learning modality-invariant and modality-specific features, and methods like MMS2S~\cite{MMER1}, HHMPN~\cite{MMER2}, TAILOR~\cite{Tailor}, AMP~\cite{adversarial_masking}, and CARAT~\cite{CARAT}.
% 



\begin{figure*}[thb]
    \centering
    \begin{subfigure}[b]{0.2\textwidth}
        \centering
        \includegraphics[width=\textwidth]{images/wo_adv2.png}
        \caption{w/o Adversarial Training}
        \label{fig:top-left}
    \end{subfigure}
    \hspace{0.02\textwidth}
    \begin{subfigure}[b]{0.2\textwidth}
        \centering
        \includegraphics[width=\textwidth]{images/w_adv2.png}
        \caption{w/ Adversarial Training}
        \label{fig:top-right}
    \end{subfigure}
    \hspace{0.02\textwidth} 
    \begin{subfigure}[b]{0.2\textwidth}
        \centering
        \includegraphics[width=\textwidth]{images/wo_rec2.png}
        \caption{w/o RN and CLN}
        \label{fig:bottom-left}
    \end{subfigure}
    \hspace{0.02\textwidth}
    \begin{subfigure}[b]{0.2\textwidth}
        \centering
        \includegraphics[width=\textwidth]{images/w_rec2.png}
        \caption{w/ RN and CLN}
        \label{fig:bottom_right}
    \end{subfigure}
    \vspace{-.5em}
    \caption{(a) and (b), the t-SNE visualization of specific and common embedding without/with adversarial training. The red, green, and blue colors represent textual(t), visual(v), and acoustic(a) modalities respectively. Dark colors correspond to specific parts, and light colors denote common parts. (c) and (d), the t-SNE visualization of reconstruction embedding without/with RN and CLN. Different colors indicate different modalities and different saturation represents different emotions.}
    \label{fig:tsne}
    \vspace{-1em}
\end{figure*}

\subsection{Comparison with the state-of-the-arts}
% main experiments
We present the performance comparisons of RAMer on the MEmoR, CMU-MOSEI, and M3ED datasets in Table~\ref{tab:primary}, Table~\ref{tab:cmu-mosei}, and Table~\ref{tab:m3ed}, respectively, with following observations.

1) On the MEmoR dataset, RAMer outperforms all baselines by a significant margin. While TAILOR achieves a high weighted-F1 score in the fine-grained setting, its overall performance is weaker due to biases toward frequent and easier-to-recognize classes. RAMer consistently delivers strong results across all settings, demonstrating its ability to learn more effective representations. 
2) On the CMU-MOSEI and M3ED datasets, RAMer surpasses state-of-the-art methods on all metrics except recall, which is less critical compared to accuracy and Micro-F1 in these contexts. 
3) Deep-based methods outperform classical ones, highlighting the importance of capturing label correlations for improved classification performance. 
4) Multimodal methods like HHMPN and AMP significantly outperform the unimodal ML-GCN, emphasizing the necessity of multimodal interactions.
5) Models optimized for dyadic conversations, such as CARAT, experience a notable performance drop in multi-party settings with incomplete modalities. In contrast, RAMer excels in both scenarios, achieving substantial improvements in Macro-F1 scores on the MEmoR dataset, outperforming CARAT by 0.178 and 0.209, respectively. 


\subsection{Ablation Study}
To better understand the importance of each component of RAMer, we compared various ablated variants. 

As shown in Table ~\ref{tab:ablation}, we make the following observations:

\begin{itemize}
\item The specificity and commonality enhance MMER performance. Variants (1), (2), and (3) exhibit an approximately 0.05 decrease in Micro-F1 performance compared to variant (11). This indicates that jointly learning specificity and commonalities yields superior performance, underscoring the importance of capturing both modality-specific specificity and shared commonality.

\item  Contrastive learning benefits the MMER. The inclusion of loss functions $\mathcal{L}_{scl}$ in adversarial training leads to progressive performance improvements, as evidenced by the superior results of (4). 
%Moreover, (6) validates the rationality of exploring the intrinsic vectors in the latent space.

% \textbf{3)} Variants (3) and (4) perform worse than (12), indicating that jointly learning modality-specific features and commonalities yields better performance.

\item Feature reconstruction net benefits MMER. Variants (5), (6), (7) are worse than (11), and (8) shows an 0.045 decrease in Micro-F1, which indicates that feature reconstruction can improve model performance. When the entire reconstruction process is omitted, the performance of (8) declines even more compared to (6) and (7), confirming the effectiveness of multi-level feature reconstruction in achieving multi-modal fusion.

% \textbf{4)} The Stack Shuffle strategy is effective in improving model performance. When either sample-wise or modality-wise shuffling is removed, variants (8) and (9) perform worse than (11). The performance further declines when both dimensions of shuffling are removed, indicating that stack shuffle aids in feature enhancement.
\item Changing the fusion order leads to poor performance, variants (9) and (10) perform worse than (11). It validates the rationality and optimality of feature fusion.
\end{itemize}
% \textbf{5)} Personality information benefits both complete and incomplete modalities. As shown in Table 5, incorporating personality traits consistently improves RAMer’s performance across all modalities.
 

\begin{table}[t]
\begin{center} 
\caption{Ablation study on the aligned CMU-MOSEI dataset. $\Lambda$ refers to the fusion order, and ${{L}_{sc}}$ represents the specific and common loss. ``w/o $\varepsilon ^m$, $d^m$'' denotes the removal of the encoding and decoding processes.}\label{tab:ablation} 
\vspace{-1em}
\resizebox{\columnwidth}{!}
{ 
\begin{tabular}{lcccc}
\hline
\hline
Approaches                                                                                   
& Acc   & P     & R     & Micro-F1 
\\ 
\hline

% (1) w/o LA                                                                                       
% & 0.491 & 0.653 & 0.519 & 0.582    
% \\

(1) w/o ${{L}_{sc}}$                                                                                      
& 0.474 & 0.610  & 0.517 & 0.573    
\\

(2) w/o $\bm{C}^{\{v,a,t\}}$                                                             
& 0.467 & 0.612 & 0.501 & 0.552    
\\

(3) w/o $\bm{S}^{\{v,a,t\}}$                                                            
& 0.460  & 0.599 & 0.491 & 0.552    
\\
\hline

(4) w/o ${{L}_{scl}}$                                                                               
& 0.492 & 0.651 & 0.540  & 0.588    
\\

(5) w/o $\varepsilon ^m$, $d^m$                                             
& 0.480  & 0.633 & 0.524 & 0.580     
\\

(6) w/o $\bm{\mathcal{X}}_\beta^m$                                              
& 0.481 & 0.641 & 0.538 & 0.590     
\\

(7) w/o $\bm{\mathcal{X}}_\gamma^m$                                               
& 0.485 & 0.620  & 0.523 & 0.586    
\\

(8) w/o $\bm{\mathcal{X}}_\beta^m$ + $\bm{\mathcal{X}}_\gamma^m$ 
& 0.477 & 0.603 & 0.490  & 0.557    
\\ 

\hline

(9) $\Lambda {\{v,t,a,\bm{C}\}}$                                                               
& 0.489 & 0.603 & 0.514 & 0.564    
\\ 

(10) $\Lambda {\{t,a,v,\bm{C}\}} $                                                               
& 0.494 & 0.650  & 0.525 & 0.582    
\\

\hline

(11) \textbf{RAMer}                                                                                       
&\textbf{0.502} &\textbf{0.672} & \textbf{0.545} & \textbf{0.602}   
\\ 
\hline
\hline
\end{tabular}
}
\end{center}
\vspace{-1em}
\end{table}

\subsection{Qualitative Analysis}
\subsubsection{Visualization of Learned Modality Representations.}
To evaluate the effectiveness of reconstruction-based adversarial training in generating distinguishable representations, we used t-SNE to visualize the commonality representations $\bm{C}_{\left\{v,a,t \right\}}$ and specificity representations $\bm{S}_{\left\{v,a,t \right\}}$ learned in the aligned CMU-MOSEI dataset. In figure \ref{fig:tsne}(a), without adversarial training, specificity and commonality are loosely separated, but their distributions overlap in certain areas, such as the lower-right corner. In contrast, Figure~\ref{fig:tsne}(b) shows that with adversarial training, commonality and specificity are clearly separated in latent subspaces, forming distinct boundaries and effectively distinguishing emotions across modalities. 
%However, the different modalities in both spaces are separated and much easier to distinguish. This indicates that adversarial training is effective in aligning distributions across different modalities. 
Figure \ref{fig:tsne}(c) demonstrates that without the reconstruction net (RN) and contrastive learning net (CLN), representations of different modalities are distinguishable, but emotion labels within the same modality remain intermixed. In contrast, Figure \ref{fig:tsne}(d) shows that embeddings of both different modalities and labels are distinctly separable, highlighting that the reconstruction-based module effectively enhances modal specificity. Overall, RAMer accurately captures both the commonality and specificity of different modalities.

% \begin{figure}[thb]
%     \centering
%     \subfigure{
%         \includegraphics[width=0.45\columnwidth]{images/freq_w_adv.pdf}
%         \label{fig:left}
%     }
%     \hspace{0.05\textwidth} % 控制图片间的距离
%     \subfigure{
%         \includegraphics[width=0.45\columnwidth]{images/freq_wo_adv.pdf}
%         \label{fig:right}
%     }
%     \caption{The visualization of modality-to-label dependencies with/without (left/right) adversarial training, and the rows indicate different emotions and the columns denotes different modalities. Darker colors indicate stronger correlations.}
%     \label{fig:modality-to-label}
% \end{figure}

\subsubsection{Visualization of Modality-to-Label Correlations.}
To explore the relationship between modalities and labels, we visualized the correlation of labels with their most relevant modalities. As shown in Figure \ref{fig:modality-to-label}, regardless of the presence of adversarial training, different emotion label is influenced by different modalities. For instance, surprise is predominantly correlated with the acoustic modality, while anger is primarily associated with the visual modality. This indicates that each modality captures the distinguishable semantic information of the labels from distinct perspectives.

\begin{figure}[t]
    \centering
\includegraphics[width=0.6\linewidth]{images/freq_rec_horizontal_wo_adv.pdf}
    % \caption{With AT}
    \vspace{-1.0em}
    \caption{The correlation of modality-to-label dependencies.}
    \label{fig:modality-to-label}
\vspace{-0.5em}
\end{figure}

\begin{figure}[t!] 
% \vspace{-.3cm}
    \centering % 图片居中
\includegraphics[width=0.98\linewidth]{images/case_small.pdf}
    \vspace{-0.6em}
    \caption{An example of the case study results.}
    \label{fig:case study}
    \vspace{-1em}
\end{figure}

\subsubsection{Case Study}
% 
To demonstrate RAMer’s robustness in complex scenarios, Figure~\ref{fig:case study} shows an example of multi-party emotion recognition on MEmoR dataset where specific target persons have incomplete modality signals. The top three rows display different modalities from a video clip, segmented semantically with aligned multi-modal signals. Key observations include: 1) The target moment requires recognizing emotions for both the speaker (e.g., Howard) and non-speakers (e.g., Penny and Leonard). While the speaker typically has complete multi-modal signals, non-speakers often lack certain modalities. TAILOR, relying on incomplete modalities, produced partial predictions as its self-attention mechanisms struggled to align labels with missing features. 2) Limitations of single or incomplete modalities. A single modality, such as text, is often insufficient for accurate emotion inference (e.g., only Howard’s Joy is detectable from text alone). Although CARAT attempts to reconstruct missing information, it fails to capture cross-modal commonality, leading to incorrect predictions. 3)The importance of inter-person interactions and external knowledge is evident in emotion recognition, where inter-person attention enables individuals with incomplete modalities to gain supplementary information from others. Moreover, integrating external knowledge, such as personality traits, enhances emotion reasoning across participants and contexts, emphasizing the synergy between user profiling and emotion recognition. Experimental results validate that RAMer demonstrates superior robustness and effectiveness in these challenging, real-world scenarios.

\section{Conclusion}
In this work, we propose a simple yet effective approach, called SMILE, for graph few-shot learning with fewer tasks. Specifically, we introduce a novel dual-level mixup strategy, including within-task and across-task mixup, for enriching the diversity of nodes within each task and the diversity of tasks. Also, we incorporate the degree-based prior information to learn expressive node embeddings. Theoretically, we prove that SMILE effectively enhances the model's generalization performance. Empirically, we conduct extensive experiments on multiple benchmarks and the results suggest that SMILE significantly outperforms other baselines, including both in-domain and cross-domain few-shot settings.

\clearpage

\bibliographystyle{named}
\bibliography{references}
\clearpage

\end{document}

