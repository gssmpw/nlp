%%%% ijcai25.tex

\typeout{IJCAI--25 Instructions for Authors}

% These are the instructions for authors for IJCAI-25.

\documentclass{article}
\pdfpagewidth=8.5in
\pdfpageheight=11in

% The file ijcai25.sty is a copy from ijcai22.sty
% The file ijcai22.sty is NOT the same as previous years'
\usepackage{ijcai25}

% Use the postscript times font!
\usepackage{times}
\usepackage{soul}
\usepackage{url}
\usepackage[hidelinks]{hyperref}
\usepackage[utf8]{inputenc}
\usepackage[small]{caption}
\usepackage{graphicx}
\usepackage{amsmath}
\usepackage{amsthm}
\usepackage{booktabs}
\usepackage{algorithm}
\usepackage{algorithmic}
\usepackage[switch]{lineno}

\usepackage{bm}
\usepackage{multirow}
\usepackage{tabularx}
\usepackage{caption}
\usepackage{subcaption}
\let\Bbbk\relax
\usepackage{amssymb}
\usepackage{balance}

% Comment out this line in the camera-ready submission
% \linenumbers

\urlstyle{same}

% the following package is optional:
%\usepackage{latexsym}

% See https://www.overleaf.com/learn/latex/theorems_and_proofs
% for a nice explanation of how to define new theorems, but keep
% in mind that the amsthm package is already included in this
% template and that you must *not* alter the styling.
% \newtheorem{example}{Example}
% \newtheorem{theorem}{Theorem}

\newcommand{\stab}{\vspace{1.2ex}\noindent}
\newcommand{\sstab}{\rule{0pt}{8pt}\\[-2.2ex]}
\newcommand{\stitle}[1]{\sstab\noindent{\bf #1}}
\newcommand{\etitle}[1]{\vspace{1mm}\noindent{\underline{\em #1}}}
\newcommand{\ie}{{\em i.e.,}\xspace}
\newcommand{\eg}{{\em e.g.,}\xspace}
\newcommand{\wrt}{\emph{w.r.t.}\xspace}
\newcommand{\aka}{\emph{a.k.a.}\xspace}



% Single author syntax
% \author{
%     Anonymous Author(s)
%     % Author Name
%     % \affiliations
%     % Affiliation
%     % \emails
%     % email@example.com
% }

% Multiple author syntax (remove the single-author syntax above and the \iffalse ... \fi here)
% \iffalse
\author{
Xudong Yang$^1$
\and
Yizhang Zhu$^1$\and
Nan Tang$^{1,2}$\And
Yuyu Luo$^{1,2}$\\
\affiliations
\small{$^1$The Hong Kong University of Science and Technology (Guangzhou), Guangzhou, China} \\
\small{$^2$The Hong Kong University of Science and Technology, Hong Kong SAR, China} \\
% $^3$Third Affiliation\\
% $^4$Fourth Affiliation\\
\emails
\small{sootungyoung@gmail.com,
yzhu305@connect.hkust-gz.edu.cn
\{nantang, yuyuluo\}@hkust-gz.edu.cn}
}
% \fi

\begin{document}

\title{RAMer: Reconstruction-based Adversarial Model for Multi-party Multi-modal Multi-label Emotion Recognition}

\maketitle

\begin{abstract}
Conventional Multi-modal multi-label emotion recognition (MMER) from videos typically assumes full availability of visual, textual, and acoustic modalities. However, real-world \textit{multi-party} settings often violate this assumption, as non-speakers frequently lack acoustic and textual inputs, leading to a significant degradation in model performance. Existing approaches also tend to unify heterogeneous modalities into a single representation, overlooking each modality’s unique characteristics.
To address these challenges, we propose \textbf{RAMer} (\textbf{R}econstruction-based \textbf{A}dversarial \textbf{M}odel for \textbf{E}motion \textbf{R}ecognition), which leverages adversarial learning to refine multi-modal representations by exploring both modality commonality and specificity through reconstructed features enhanced by contrastive learning. \textbf{RAMer} also introduces a personality auxiliary task to complement missing modalities using modality-level attention, improving emotion reasoning. To further strengthen the model's ability to capture label and modality interdependency, we propose a stack shuffle strategy to enrich correlations between labels and modality-specific features.
Experiments on three benchmarks, i.e., MEmoR, CMU-MOSEI, and $M^3$ED, demonstrate that \textbf{RAMer} achieves state-of-the-art performance in dyadic and multi-party MMER scenarios. The code will be publicly available at https://github.com/Sootung/RAMer
\end{abstract}


\documentclass[../main.tex]{subfiles}
\graphicspath{{../images/}}
\makeatletter
\def\input@path{{../images/}}
\makeatother
\begin{document}
\section{Introduction}
\begin{figure}
\centering
\begin{tikzpicture}
\node[inner sep=0pt] (ws) at (0, 0) {
\includegraphics[height=.4\textwidth, trim={10cm 0 10cm 0},clip]{world_space.png}};
\node[inner sep=0pt] (cs) at (6,0) {\includegraphics[height=.4\textwidth, trim={10cm 1cm 10cm 4cm},clip]{conf_space.png}};
\end{tikzpicture}
\vspace{-5pt}
\label{fig:pbrm_intro}
\caption{\textbf{Left}: Shows world space obstacles as grey spheres. Robots start and goal configuration is colored red and green, respectively. Configurations along the computed path are colored transparent blue. \textbf{Right:} Mapped world space scenario to configuration space. Obstacle region is the grey mesh. Red spheres are collision-free regions computed by the neural SCDF. The optimized shortest path in the convex corridor is the blue curve.}
\vspace{-25pt}
\end{figure}
Motion planning is the problem of finding a collision-free trajectory that connects a given start and goal configuration. The planning takes place in the configuration space of the robot. For single body robots, like mobile robots or drones, the configuration space and the world space are usually the same. This simplifies the planning, since explicit obstacle representations are available which enables geometrical tools like separating hyperplanes, smallest distance to obstacles etc., to be used when designing motion planning algorithms. For multi-body robots like manipulators, the situation is completely different. The world space obstacles are usually mapped to non-convex regions, and to make the problem even harder, the mapping is usually not known. Forming explicit representations of the obstacle region in the configuration space is usually too expensive or intractable. Despite all of this, sampling based planners are used with great success, which mainly is due to their use of implicit representations of the obstacle region. The basic idea is to construct a graph in the configuration space that covers and connects the collision-free region. From this graph, a path can be extracted that connects a given start and goal configuration. The approach is computationally expensive, since the graph is constructed with the smallest geometrical building block available, points, which represents a collision-check. Furthermore, the extracted paths from the graph are non-smooth and jagged due to the stochastic nature of the approach. This adds an additional post-processing step to the process, where the paths are shortcutted and smoothened, before the path can be used for tracking. Clearly a lot of time is invested to form this graph and produce smooth paths. Thus, if the obstacles start to move, then all of this work is done in no use, since all points that make up this graph need to be re-verified, which is simply too time consuming to be done in real time.
\\\\
In this work, we want to address the existing drawbacks of the sampling based planners. Our main contribution is an improved motion planner where each vertex in the graph covers a collision-free region in the form of a sphere instead of a point and where the edges are formed with neighboring intersecting spheres. This representation has the advantage of instead of returning piecewise linear paths, returning a sequence of overlapping spheres, i.e. a convex corridor, that connects a given start and goal configuration, illustrated in Figure \ref{fig:pbrm_intro}. This convex corridor allows us to use convex optimization to produce smooth trajectories, instead of computationally expensive post-processing methods. The representation further allows us to estimate the coverage of the collision-free space, which gives us awareness and feedback in the offline roadmap construction phase. Finally, our representation is simple to adapt to moving obstacles, simply requery for the new radii and recheck for intersections. 
\\\\
The spherical collision-free regions are formed using a signed distance function (SDF), which is a function that returns the smallest distance from an arbitrary point to the boundary of an obstacle. As the name implies, the distance is signed, thus if the point is inside the obstacle it is negative otherwise positive. If the distance is positive, a sphere with radius equal to the distance is guaranteed to cover a collision-free region. Using an SDF in motion planning is not new, but what is novel about our approach is that we express the distance in the configuration space instead of the world space and by doing so allows us to form these convex collision-free regions. We refer to the resulting SDF as a signed configuration distance function (SCDF). Computing an SCDF analytically is non-trivial, our approach is therefore to parameterize the SCDF with a deep neural network and learn the mapping by supervised learning. Our resulting neural SCDF can compute distances for different parameter values of obstacle shapes and we also show how multiple distances can be combined, thus making our approach flexible.
\section{Related work}
Motion planning algorithms can roughly be divided into three families, grid-based, sampling based and optimization based methods. Grid-based methods (GBM) discretize the planning space from which a graph is then compiled. A standard search method is A$^\star$ \citep{a_star}, which is classified as an \textit{informed} search method, since it employs a heuristic function to speed up the search. A$^\star$ guarantees to return an optimal path at the level of discretization used. GBMs usually discretize the planning space by a regular lattice and this limits the GBMs to problems with low dimensionality due to the curse of dimensionality. Thus, GBMs are usually limited to single-body robots where the degrees of freedom (DOF) are low. To overcome the inherent scaling problem with the GBMs, stochastic methods are usually used for multi-body robots. These methods are termed as sampling-based methods (SBM) and core members within this family are the rapidly-exploring random trees (RRT) \citep{rrt} and the probabilistic roadmap (PRM) \citep{prm}. RRT grows a tree from the start configuration and explores the collision-free region in a rapid way until it is able to connect to the goal region. RRT is usually improved by bi-directional planning \citep{rrt_connect}, i.e. an additional tree is grown from the goal configuration and the trees are tested for connection after any tree has been expanded. RRT is a single-query method, thus it searches for a path from scratch each time it is queried. Contrary to this, PRM is a multi-query method, which solves for multiple queries without starting from scratch. PRM does this by creating a roadmap (graph) that covers the collision-free space as an offline step. The graph is then used to solve for multiple queries. PRMs are used in cases where the environment does not change since the extra offline step is too computationally costly and needs to be re-done if the environment is changed. In our work, we address this inherent issue by using a different roadmap representation. Our vertices in the graph cover a collision-free region in the form of spheres and we form the edges by checking for intersecting spheres. If something in the environment changes, we recompute the spheres radii and recheck the intersections, without relying on collision detection. We use a trained neural network to compute the sphere radius, therefore querying for the radius can be done fast, hence our representation enables the PRM for dynamic environments.
\\\\
In the recent decades, optimization based methods (OBM) \citep{chomp, schulman, itomp, stomp} have been introduced as an alternative to SBM for multi-body robots. Like the SBM, the OBMs scale well to higher dimensional problems and produce smoother motion. It is common to use a SDF in the optimization since it is a smooth function, thus enabling gradient-based methods. However, the standard way of expressing the SDF is in world space. The distance therefore needs to be mapped to the configuration space by the forward kinematics. This mapping makes the optimization problem a non-linear program (NLP), which is computationally expensive to solve. Recently, a different approach has been proposed. In \cite{mp_gcs} motion planning is formulated as a convex optimization problem by using the graph of convex sets framework \citep{gcs}. The underlying idea is to decompose the collision-free space into intersecting convex sets from which a convex optimization problem is formulated. In cases where an explicit representation of the obstacles in the configuration space exists, like for single-body robots, creating collision-free convex regions can be done fast \citep{iris}. For multi-body robots, this is non-trivial. Existing work does this successfully \citep{iris_nlp, iris_c} by an optimization based approach, but the methods are still too time consuming to be used in the presence of moving obstacles. Our approach is instead to use deep learning to learn an SDF expressed in the configuration space. With this, we can query for shortest distances to the collision boundary, which allows us to expand spherical regions which are collision-free. Our approach is fast and therefore enables our suggested roadmap planner to be used in dynamic environments.
\\\\
Recent research has focused on learning collision detection \citep{fk_kernel_distance, diffco, graphdistnet} by predicting the signed distance between the robot links and the surrounding obstacles in the world space. The learned SDF is used in trajectory optimization but since the distance is expressed in the world space, the problem becomes an NLP and therefore takes a long time to solve. We take a novel approach and suggest to instead express the signed distance in the configuration space. This allows us to improve the PRM at the same time as it enables convex optimization for trajectory optimization, which runs faster and is more reliable than NLP solvers. In \cite{cspf} a learned signed distance function in the configuration space is proposed similar to our approach. However, their approach is restricted to point cloud representations, while we propose to represent the obstacles as parameterized geometric shapes, e.g. spheres. Furthermore, we also show how to use our learned SCDF to improve an existing roadmap planner.
\section{Problem formulation}
A robot is located in the world space, $\W \subset \R^3 $. The unique location of the robot is given by its configuration $\q \in \C$, where $\C$ is the configuration space. The set of points covered by the robots bodies at a certain configuration is expressed as $\B(\q) \subset \W$. The robot is surrounded by $\NrObst$ obstacles $\O = \bigcup_{i=1}^{\NrObst} \O_i$, where  $\O_i \subset \W$. The representation of the obstacle in the configuration space is the set $\C\O_i = \{\q \in \C \: |\: \B(\q) \cap \O_i \neq \emptyset \}$. The obstacle space is formed as $\Co = \bigcup_{i=1}^{\NrObst} \C \O_i$. The complement is referred to as the free space, $\Cf = \C \setminus \Co$. The path planning problem is a tuple, ($\Cf$, $\qStart$, $\qGoal$), where we want to connect a query pair, consisting of a start, $\qStart$, and goal configuration, $\qGoal$, with a geometric path, $\q(s): [0, 1] \mapsto \Cf$, such that $\q(0)=\qStart$ and $\q(1)=\qGoal$, or report correctly when such a path does not exist.
\end{document}

\section{Related Work}
% \subsection{Vision Language Model}
% 시각장애인에서 상황을 설명할 DB가 없으니 만들었다. 그리고 이를 VLM에 튜닝했다.
\subsection{Technical approaches for assisting the visually-impaired}


\subsection{Datasets for visual instruction tuning}


\section{\label{sec:method}Methodology}

Each SIEM system uses its own RDL to define threat detection rules, and each RDL has its own schema.
For example, the Splunk SIEM uses the SPL to define its threat detection rules.
The task of understanding threat detection rules and recommending relevant MITRE ATT\&CK techniques (or sub-techniques) requires complex reasoning skills.
In the case of LLMs, this can be achieved with a technique called prompt chaining in which each task is divided into multiple sub-tasks in order to understand the complex reasoning behind the task.
Therefore, we employ a multi-phase architecture based on prompt chaining that leverages the power of LLMs to take a SIEM rule defined in any RDL and map it to relevant MITRE ATT\&CK techniques using the power of LLMs.
Our approach is based on the following intuitions:
\begin{itemize}[nosep,leftmargin=*]
    \item \textit{LLMs' implicit knowledge}: LLMs possess deep understanding of diverse RDLs. This enables them to interpret any rule, regardless of the RDL it is defined in, and convert it into comprehensible natural language text.
    \item \textit{LLMs' similarity comparison capability}: LLMs are adept at analyzing and comparing textual descriptions. 
    They can intelligently assess the similarity between two textual inputs to establish a meaningful connection.
\end{itemize}

\methodName has two main phases: (1) the rule to text translation phase, and (2) the MITRE ATT\&CK techniques recommendation phase.
These two phases in the pipeline include six key steps to determine relevant TTPs, as illustrated in Figure~\ref{fig:r2t}.

Although LLMs excel at translating SIEM rules into natural language, they often lack critical domain-specific contextual information related to IoCs in the rules.
To overcome this limitation, the \textit{rule to text translation} phase includes three steps: IoC extraction, contextual information retrieval, and natural language translation.

The workflow begins with the extraction of IoCs from the rules (for example, processes, log source, event codes, and file names) that the rule searches for in the logs (step (1)).In the next sstep a web search agent performs the task of obtaining additional contextual information about the IoCs discovered ((step 2)).
By incorporating this additional domain-specific information, the pipeline enhances the language translation, resulting in a more accurate and meaningful interpretation of SIEM rules.
The rule itself and the IoCs' contextual additional information from the previous stage are then used to translate the rule from RDL to natural language (step (3)).

The \textit{MITRE ATT\&CK techniques} recommendation phase of the pipeline includes the following three steps.
The rule in processed in data source identification step in which the probable origin of the data is identified (step (4)).
The description of the rule is then used to determine probable MITRE ATT\&CK techniques based on the implicit knowledge of the LLM (step (5)).
Finally, using chain-of-thought~\cite{wei2022chain} prompting, the most relevant techniques are extracted from the list of probable techniques (step (6)).
Each step of our method is further described in detail below.


% [bb=0 0 1440 900,width=1.43\linewidth,height=0.9\textwidth]
\begin{figure*}[htbp]
   \includegraphics[width=\textwidth]{Images/stages.jpg}
    
   \caption{An illustration of the different steps in \methodName.}
   \label{fig:stages}
\end{figure*} 

\subsection{IoC Extraction}
The context associated with a SIEM detection rule is crucial for its accurate interpretation and effective application. 
Obtaining this contextual understanding requires comprehensive analysis of the embedded IoCs in the SIEM rule.
In the first step, \methodName systematically identifies and extracts all IoCs, identifying the types of IoCs and their corresponding values that form the foundational elements of the detection rules. 
Leveraging the LLM's inherent understanding of rule structures and IoCs, we employ a zero-shot prompting approach for this task. 
Zero-shot prompting enables the direct extraction of IoCs from the rules without requiring extensive pre-training on specific datasets.

\noindent The result of this stage is a dictionary structure, where:
\begin{itemize}[nosep,leftmargin=*]
    \item Keys represent types of IoC, such as processes, files, IP addresses, and log sources.
    \item Values are lists containing specific IoC details, such as process names, file names, IP addresses, and log source identifiers.
\end{itemize}

In the example depicted in Figure~\ref{fig:stages}(a), the pipeline processes a rule for which relevant MITRE ATT\&CK techniques need to be recommended. 
The IoC extractor LLM produces a dictionary structure as output, organizing the IoCs in a structured format to support subsequent stages in the analysis pipeline. 



\subsection{Contextual Information Retrieval}
In this step, an LLM agent is employed to retrieve relevant information pertaining to the IoCs extracted from the rule.
A REACT agent~\cite{react} was used in this case to generate both reasoning traces and task-specific actions in an interleaved manner.
REACT agents interact with external tools to retrieve additional information that leads to more factual and reliable responses.
The LLM agent conducts a systematic search across web resources to gather additional contextual information for each IoC value present in the rule. 
This step addresses LLMS' lack of up-to-date knowledge or specialized domain expertise (which is critical to understanding the role and significance of the IoCs in the rule), without the need for retraining or fine-tuning.
Figure~\ref{fig:stages}(b) presents an example in which the rule includes the process name \texttt{soaphound.exe} as an IoC.
As can be seen, the web search results indicate that \texttt{soaphound.exe} is being used for active directory (AD) enumeration, which is important for the understanding of the attack. 

\subsection{Natural Language Translation}

The translation of detection rules into natural language textual descriptions fulfills three key objectives:
\begin{enumerate}[nosep,leftmargin=*]
    \item \textbf{Ensures that \methodName is format-agnostic}: It converts rules defined in various RDL formats into a generic, unstructured text format, ensuring compatibility with different SIEM systems, regardless of the specific rule format.
    \item \textbf{Provides contextual explanation}: It includes all relevant contextual information to produce a concise and comprehensible explanation of the rule.
    \item \textbf{Enhances the comprehension for LLMs}: It enables LLMs to more effectively compare the translated rule with descriptions in the MITRE ATT\&CK framework by providing a unified textual representation.
\end{enumerate}
To achieve these objectives, a zero-shot prompting technique is employed. 
The input to the LLM comprises two components:
\begin{itemize}
    \item \textbf{Syntactical information}: The rule itself, providing the structural and operational details.
    \item \textbf{Contextual information}: Details of the IoCs extracted from the rule, providing semantic insights into the rule's intent and function.
\end{itemize}
The LLM utilizes these inputs to generate a natural language textual description of the rule. 
This transformation not only ensures a more interpretable representation but also facilitates further steps of analysis and comparison, particularly in aligning the rule with MITRE ATT\&CK techniques and sub-techniques.



\subsection{Data Source or Mitigation Identification}
Identifying the most relevant data component or mitigation associated with the rule description in this step is critical for filtering out irrelevant MITRE ATT\&CK techniques (or sub-techniques) in subsequent steps of the pipeline.
In the MITRE ATT\&CK framework, data sources represent various categories of information that can be gathered from sensors or logs. 
These data sources include \textit{data components}, which are specific attributes or properties within a data source that are directly relevant to detecting a particular technique or sub-technique~. 
For example, in the context of the rule described in Figure~\ref{fig:stages}(a), the term \texttt{Endpoint.Processes} indicates that the activity is happening on an endpoint. 
Presence of the terms such as, \texttt{soaphound.exe}, \texttt{--buildcache}, \texttt{--certdump} and etc. indicate that the rule searches for command line execution of an executable named \texttt{soaphound.exe} with specific parameters. 
Therefore, the appropriate data source in this example is \textit{Command}, with the corresponding data component being \textit{Command Execution}.
Additionally, \textit{mitigations} are defined as categories of technologies or strategies that can prevent or reduce the impact of specific techniques or sub-techniques. 
The MITRE ATT\&CK framework explicitly establishes relationships between data components, mitigations, and techniques (or sub-techniques), enabling a systematic approach for identifying relevant elements.

To identify the most relevant data component or mitigation associated with a given rule description, we utilize agentic retrieval augmented generation (RAG), which incorporates an AI Agent-based implementation of the RAG framework.
Data from the MITRE ATT\&CK framework, specifically related to data components and mitigations, is stored in a vector database (e.g., ChromaDB). 
The process begins with the rule description from the previous stage, which serves as the input to the AI Agent. 
The LLM-powered agent automatically generates a search query tailored to retrieve relevant information from the RAG database.

For each query, the system retrieves the five most similar documents from the database, each containing contextual information about data components or mitigations. 
These documents are then utilized by the LLM agent to contextualize the rule description. 
By comparing the content of these retrieved documents with the rule description, the LLM agent determines and outputs the most relevant data component or mitigation along with a chain-of-thought as to why the data component or mitigation is related to the rule.


\subsection{Probable Technique Recommendation}

In this step, an LM agent is utilized to propose probable MITRE ATT\&CK techniques (and sub-techniques) that may be relevant to the description of the provided rule.
We used a REACT agent in this step as well to utilize both implicit and explicit knowledge during reasoning.
For explicit knowledge, the agent searches the MITRE ATT\&CK framework to obtain the list of probable techniques (and sub-techniques).
The natural language description of the rule from the previous step serves as input to the LLM agent.
The output of this stage consists of a list of JSON objects, each containing the MITRE technique ID, technique name, and technique description as seen in Figure~\ref{fig:stages}(c).

Throughout our experiments, we observed that as the number of recommendations ($k$) increases, both the framework's average recall and precision initially improve, however beyond a certain threshold of $k$, the %average 
precision begins to decline.
Based on these observations(please refer Table~\ref{tab:results3}), we selected a $k$-value of 11 to ensure a high recall.



\subsection{Relevant Technique Extraction}
In this step, \methodName refines the set of probable MITRE ATT\&CK techniques identified in the previous stage by eliminating irrelevant entries.
This step in the pipeline serves two primary purposes: (1) to enhance precision while maintaining recall achieved in previous step, and (2) to provide a clear rationale for the selection of the labels, ensuring transparency and interpretability of the mapping process.
This refinement process is grounded in the assumption that LLMs are effective for text similarity matching tasks.

The process comprises two key steps:
\begin{itemize}
    \item \textit{Rule-technique comparison}: The description of each technique in the set of probable techniques is compared with the rule description. 
    A chain-of-thought technique is then applied to elucidate the reasoning behind the association of each technique with the rule.
    \item \textit{Confidence calculation}: The generated chain-of-thought rationale for each technique (or sub-technique) is compared with the rule description to compute a relevance (or confidence) score, as done in prior work~\cite{freitas2024ai}.
    % \item \textbf{Reasoning}: \new{Add here the reasoning that it provides - explaining in NLP why it was selected...}
\end{itemize}

Techniques with higher confidence scores are deemed more relevant to the rule. 
Conversely, techniques with scores falling below a predefined threshold are excluded.
The techniques retained after this filtering step represent the most relevant techniques corresponding to the given rule's description. 


The chain-of-thought (CoT) rationale generated during the comparison of each rule to its probable technique is also provided as an output in this step.
This rationale offers a detailed natural language explanation, articulating why a particular technique is relevant to the given rule. 
Such explanations are highly valuable for security analysts, as they provide clear and transparent reasoning behind the mapping, enabling analysts to better understand and validate the association between the rule and the technique.
Other classification models proposed in previous works within this domain also suffer from the limitation of being black-box models, which lack the ability to provide clear reasoning or explanations. 
Unlike \methodName, these models fail to generate transparent, CoT rationales that explain why a particular rule is mapped to a specific technique, making them less interpretable and less useful for security analysts.
\section{Experiments}
% \label{headings}

\subsection{Experimental setup}

\paragraph{Datasets and Metrics.}
We evaluated RAMer on three multimodal, multi-label benchmark datasets: MEmoR~\cite{MEmoR}, a multi-party conversation dataset that includes personality, and CMU-MOSEI~\cite{Cmu-mosei} and $M^3$ED~\cite{M3ED}, which are dyadic conversation datasets that do not include personality information. The evaluation is conducted under the protocols of these datasets. For CMU-MOSEI and $M^3$ED, we employed four commonly used evaluation metrics: Accuracy (Acc), Micro-F1, Precision (P), and Recall (R). Due to data imbalance in MEmoR, we followed the benchmark's protocol and used Micro-F1, Macro-F1, and Weighted-F1 metrics. 
% Detailed descriptions of the datasets and preprocessing steps are provided in the Appendix.

% MEmor dataset comparison
\begin{table*}[t!]
\begin{center}
\caption{Performance comparison on the MEmoR dataset under primary and fine-grained settings.}
\vspace{-1em}
\parbox{\textwidth}{\centering \small \em With various modality combinations (visual($v$), acoustic($a$)), textual($t$), personality($p$)).}  
\vspace{-1em}
\label{tab:primary} 
\resizebox{0.95\textwidth}{!} % limit the width of the table to  text width
{ 
\begin{tabular}{cc|ccc|ccc}
\hline
\hline
\multirow{2}{*}{Methods} 
& \multirow{2}{*}{Modality} 
& \multicolumn{3}{c|}{Primary}       
& \multicolumn{3}{c}{Fine-grained}  \\ 
\cline{3-8} 

&                           
& Micro-F1 & Macro-F1 & Weighted-F1 
& Micro-F1 & Macro-F1 & Weighted-F1 
\\ 
\hline

MDL with Personality     & $v,a,t,p$                   
& 0.429    & 0.317    & 0.423       
& 0.363    & 0.217    & 0.345       
\\

MDAE                    & $v,a,t,p$                   
& 0.421    & 0.303    & 0.410       
& 0.363    & 0.219    & 0.341       
\\

BiLSTM+TFN             & $v,a,t,p$                   
& 0.470    & 0.310    & 0.454       
& 0.366    & 0.207    & 0.350       
\\

BiLSTM+LMF             & $v,a,t,p$                   
&0.449    & 0.294    & 0.432       
& 0.364    & 0.198    & 0.351       
\\

DialogueGCN            & $v,a,t,p$                   
& 0.441    & 0.310    & 0.425       
& 0.373    & 0.229    & 0.373       
\\

AMER w/o Personality     & $v,a,t$                     
& 0.446    & 0.339    & 0.440       
& 0.401    & 0.246    & 0.379       
\\

AMER                & $v,a,t,p$                   
& 0.477    & 0.353    & 0.465       
& 0.419    & 0.262    & 0.400       
\\ 

DialogueCRN     & $v,a,t,p$
& 0.441    & 0.310    & 0.425 
& 0.373    & 0.229    & 0.373 
\\
TAILOR               & $v,a,t,p$
& 0.341 & 0.287 & 0.326 
& 0.303 & 0.069 & \textbf{0.490} 
\\
CARAT              & $v,a,t,p$
& 0.399 & 0.224 & 0.422 
& 0.346 & 0.090 & 0.483 
\\
\hline
\textbf{RAMer}                   & $v,a,t,p$                  
& \textbf{0.499}    & \textbf{0.402}    & \textbf{0.503}       
& \textbf{0.431}    & \textbf{0.299}   & 0.404       
\\ 
\hline
\hline
\end{tabular}
} 
\end{center}
\end{table*}

% \usepackage{multirow}
\begin{table}[t]
\begin{center}
\caption{Performance Comparison on CMU-MOSEI dataset.} \label{tab:cmu-mosei} 
\vspace{-1em}
\resizebox{\columnwidth}{!}
{ 
\begin{tabular}{c|cccc|cccc}
    \hline
    \hline

    \multirow{2}{*}{Methods} & \multicolumn{4}{c|}{Aligned}      & \multicolumn{4}{c}{Unaligned}    \\ 
    \cline{2-9} 
                                  
                             
    & Acc   & P     & R     & Micro-F1 
    & Acc   & P     & R     & Micro-F1 \\ 
    \hline

    CC                    
    & 0.225 & 0.306 & 0.523 & 0.386    
    & 0.235 & 0.320 & 0.550 & 0.404    
    \\

    ML-GCN                   
    & 0.411 & 0.546 & 0.476 & 0.509    
    & 0.437 & 0.573 & 0.482 & 0.524    
    \\
    
    MulT                    
    & 0.445 & 0.619 & 0.465 & 0.531    
    & 0.423 & 0.636 & 0.445 & 0.523    
    \\
    
                                    
    MISA                    
    & 0.43  & 0.453 & \textbf{0.582} & 0.509    
    & 0.398 & 0.371 & \textbf{0.571} & 0.45     
    \\ 
    
    
    MMS2S                    
    & 0.475 & 0.629 & 0.504 & 0.56     
    & 0.447 & 0.619 & 0.462 & 0.529    
    \\  
    
    HHMPN                  
    & 0.459 & 0.602 & 0.496 & 0.556    
    & 0.434 & 0.591 & 0.476 & 0.528    
    \\ 
    
    TAILOR                  
    & 0.488 & 0.641 & 0.512 & 0.569    
    & 0.46  & 0.639 & 0.452 & 0.529    
    \\ 
    
    AMP                      
    & 0.484 & 0.643 & 0.511 & 0.569    
    & 0.462 & 0.642 & 0.459 & 0.535    
    \\  
    
    CARAT                   
    & 0.494 & 0.661 & 0.518 & 0.581    
    & 0.466 & 0.652 & 0.466 & 0.544    
    \\ 
    \hline 
    
    \textbf{RAMer}                   
    & \textbf{0.505} & \textbf{0.668} & 0.551 & \textbf{0.604}    
    & \textbf{0.469} & \textbf{0.660} & 0.486 & \textbf{0.560}    
    \\ 
    \hline
    \hline
\end{tabular}
}
\vspace{-1em}
\end{center}
\end{table}

\begin{table}[t]
% \vspace{-.3cm}
\begin{center}
\caption{Performance Comparison on the M$^3$ED dataset} \label{tab:m3ed} 
\vspace{-1em}
% \resizebox{\columnwidth}{!}
% { 
\small
\begin{tabular}{ccccc}
\hline
\hline
Methods & Acc   & P     & R     & Micro-F1 \\ 
\hline
MMS2S   & 0.645 & 0.813 & 0.737 & 0.773    \\
HHMPN   & 0.648 & 0.816 & 0.743 & 0.778    \\
TAILOR  & 0.647 & 0.814 & 0.739 & 0.775    \\
AMP     & 0.654 & 0.819 & 0.748 & 0.782    \\
CARAT   & 0.664 & 0.824 & 0.755 & 0.788    \\ 
\hline
\textbf{RAMer}  & \textbf{0.665} & \textbf{0.826} & \textbf{0.759} & \textbf{0.791}     \\ 
\hline
\hline
\end{tabular}
% }
\end{center}
\vspace{-1em}
\end{table}

% \subsubsection{Headings: third level}

\paragraph{Baselines.}
For the MEmoR dataset, we compare RAMer with multi-party conversation baselines, including MDL, MDAE~\cite{MDAE}, BiLSTM+TFN~\cite{TFN}, BiLSTM+LMF~\cite{LMF}, DialogueGCN~\cite{Dialoguegcn}, DialogueCRN~\cite{DialogueCRN}, and AMER~\cite{MEmoR}. We also evaluate its robustness against recent models designed for dyadic conversations, such as CARAT~\cite{CARAT} and TAILOR~\cite{Tailor}. For the CMU-MOSEI and $M^3$ED datasets, we test three categories of methods. 1) Classic methods. CC~\cite{CC}, which concatenates all available modalities as input for binary classifiers. 2) Deep-based methods. ML-GCN~\cite{chen2019multi}, using Graph Convolutional Networks to map label representations and capture label correlations. 3) Multi-modal multi-label methods. These include MulT~\cite{cross_adaptation} for cross-modal interactions, MISA~\cite{MISA} for learning modality-invariant and modality-specific features, and methods like MMS2S~\cite{MMER1}, HHMPN~\cite{MMER2}, TAILOR~\cite{Tailor}, AMP~\cite{adversarial_masking}, and CARAT~\cite{CARAT}.
% 



\begin{figure*}[thb]
    \centering
    \begin{subfigure}[b]{0.2\textwidth}
        \centering
        \includegraphics[width=\textwidth]{images/wo_adv2.png}
        \caption{w/o Adversarial Training}
        \label{fig:top-left}
    \end{subfigure}
    \hspace{0.02\textwidth}
    \begin{subfigure}[b]{0.2\textwidth}
        \centering
        \includegraphics[width=\textwidth]{images/w_adv2.png}
        \caption{w/ Adversarial Training}
        \label{fig:top-right}
    \end{subfigure}
    \hspace{0.02\textwidth} 
    \begin{subfigure}[b]{0.2\textwidth}
        \centering
        \includegraphics[width=\textwidth]{images/wo_rec2.png}
        \caption{w/o RN and CLN}
        \label{fig:bottom-left}
    \end{subfigure}
    \hspace{0.02\textwidth}
    \begin{subfigure}[b]{0.2\textwidth}
        \centering
        \includegraphics[width=\textwidth]{images/w_rec2.png}
        \caption{w/ RN and CLN}
        \label{fig:bottom_right}
    \end{subfigure}
    \vspace{-.5em}
    \caption{(a) and (b), the t-SNE visualization of specific and common embedding without/with adversarial training. The red, green, and blue colors represent textual(t), visual(v), and acoustic(a) modalities respectively. Dark colors correspond to specific parts, and light colors denote common parts. (c) and (d), the t-SNE visualization of reconstruction embedding without/with RN and CLN. Different colors indicate different modalities and different saturation represents different emotions.}
    \label{fig:tsne}
    \vspace{-1em}
\end{figure*}

\subsection{Comparison with the state-of-the-arts}
% main experiments
We present the performance comparisons of RAMer on the MEmoR, CMU-MOSEI, and M3ED datasets in Table~\ref{tab:primary}, Table~\ref{tab:cmu-mosei}, and Table~\ref{tab:m3ed}, respectively, with following observations.

1) On the MEmoR dataset, RAMer outperforms all baselines by a significant margin. While TAILOR achieves a high weighted-F1 score in the fine-grained setting, its overall performance is weaker due to biases toward frequent and easier-to-recognize classes. RAMer consistently delivers strong results across all settings, demonstrating its ability to learn more effective representations. 
2) On the CMU-MOSEI and M3ED datasets, RAMer surpasses state-of-the-art methods on all metrics except recall, which is less critical compared to accuracy and Micro-F1 in these contexts. 
3) Deep-based methods outperform classical ones, highlighting the importance of capturing label correlations for improved classification performance. 
4) Multimodal methods like HHMPN and AMP significantly outperform the unimodal ML-GCN, emphasizing the necessity of multimodal interactions.
5) Models optimized for dyadic conversations, such as CARAT, experience a notable performance drop in multi-party settings with incomplete modalities. In contrast, RAMer excels in both scenarios, achieving substantial improvements in Macro-F1 scores on the MEmoR dataset, outperforming CARAT by 0.178 and 0.209, respectively. 


\subsection{Ablation Study}
To better understand the importance of each component of RAMer, we compared various ablated variants. 

As shown in Table ~\ref{tab:ablation}, we make the following observations:

\begin{itemize}
\item The specificity and commonality enhance MMER performance. Variants (1), (2), and (3) exhibit an approximately 0.05 decrease in Micro-F1 performance compared to variant (11). This indicates that jointly learning specificity and commonalities yields superior performance, underscoring the importance of capturing both modality-specific specificity and shared commonality.

\item  Contrastive learning benefits the MMER. The inclusion of loss functions $\mathcal{L}_{scl}$ in adversarial training leads to progressive performance improvements, as evidenced by the superior results of (4). 
%Moreover, (6) validates the rationality of exploring the intrinsic vectors in the latent space.

% \textbf{3)} Variants (3) and (4) perform worse than (12), indicating that jointly learning modality-specific features and commonalities yields better performance.

\item Feature reconstruction net benefits MMER. Variants (5), (6), (7) are worse than (11), and (8) shows an 0.045 decrease in Micro-F1, which indicates that feature reconstruction can improve model performance. When the entire reconstruction process is omitted, the performance of (8) declines even more compared to (6) and (7), confirming the effectiveness of multi-level feature reconstruction in achieving multi-modal fusion.

% \textbf{4)} The Stack Shuffle strategy is effective in improving model performance. When either sample-wise or modality-wise shuffling is removed, variants (8) and (9) perform worse than (11). The performance further declines when both dimensions of shuffling are removed, indicating that stack shuffle aids in feature enhancement.
\item Changing the fusion order leads to poor performance, variants (9) and (10) perform worse than (11). It validates the rationality and optimality of feature fusion.
\end{itemize}
% \textbf{5)} Personality information benefits both complete and incomplete modalities. As shown in Table 5, incorporating personality traits consistently improves RAMer’s performance across all modalities.
 

\begin{table}[t]
\begin{center} 
\caption{Ablation study on the aligned CMU-MOSEI dataset. $\Lambda$ refers to the fusion order, and ${{L}_{sc}}$ represents the specific and common loss. ``w/o $\varepsilon ^m$, $d^m$'' denotes the removal of the encoding and decoding processes.}\label{tab:ablation} 
\vspace{-1em}
\resizebox{\columnwidth}{!}
{ 
\begin{tabular}{lcccc}
\hline
\hline
Approaches                                                                                   
& Acc   & P     & R     & Micro-F1 
\\ 
\hline

% (1) w/o LA                                                                                       
% & 0.491 & 0.653 & 0.519 & 0.582    
% \\

(1) w/o ${{L}_{sc}}$                                                                                      
& 0.474 & 0.610  & 0.517 & 0.573    
\\

(2) w/o $\bm{C}^{\{v,a,t\}}$                                                             
& 0.467 & 0.612 & 0.501 & 0.552    
\\

(3) w/o $\bm{S}^{\{v,a,t\}}$                                                            
& 0.460  & 0.599 & 0.491 & 0.552    
\\
\hline

(4) w/o ${{L}_{scl}}$                                                                               
& 0.492 & 0.651 & 0.540  & 0.588    
\\

(5) w/o $\varepsilon ^m$, $d^m$                                             
& 0.480  & 0.633 & 0.524 & 0.580     
\\

(6) w/o $\bm{\mathcal{X}}_\beta^m$                                              
& 0.481 & 0.641 & 0.538 & 0.590     
\\

(7) w/o $\bm{\mathcal{X}}_\gamma^m$                                               
& 0.485 & 0.620  & 0.523 & 0.586    
\\

(8) w/o $\bm{\mathcal{X}}_\beta^m$ + $\bm{\mathcal{X}}_\gamma^m$ 
& 0.477 & 0.603 & 0.490  & 0.557    
\\ 

\hline

(9) $\Lambda {\{v,t,a,\bm{C}\}}$                                                               
& 0.489 & 0.603 & 0.514 & 0.564    
\\ 

(10) $\Lambda {\{t,a,v,\bm{C}\}} $                                                               
& 0.494 & 0.650  & 0.525 & 0.582    
\\

\hline

(11) \textbf{RAMer}                                                                                       
&\textbf{0.502} &\textbf{0.672} & \textbf{0.545} & \textbf{0.602}   
\\ 
\hline
\hline
\end{tabular}
}
\end{center}
\vspace{-1em}
\end{table}

\subsection{Qualitative Analysis}
\subsubsection{Visualization of Learned Modality Representations.}
To evaluate the effectiveness of reconstruction-based adversarial training in generating distinguishable representations, we used t-SNE to visualize the commonality representations $\bm{C}_{\left\{v,a,t \right\}}$ and specificity representations $\bm{S}_{\left\{v,a,t \right\}}$ learned in the aligned CMU-MOSEI dataset. In figure \ref{fig:tsne}(a), without adversarial training, specificity and commonality are loosely separated, but their distributions overlap in certain areas, such as the lower-right corner. In contrast, Figure~\ref{fig:tsne}(b) shows that with adversarial training, commonality and specificity are clearly separated in latent subspaces, forming distinct boundaries and effectively distinguishing emotions across modalities. 
%However, the different modalities in both spaces are separated and much easier to distinguish. This indicates that adversarial training is effective in aligning distributions across different modalities. 
Figure \ref{fig:tsne}(c) demonstrates that without the reconstruction net (RN) and contrastive learning net (CLN), representations of different modalities are distinguishable, but emotion labels within the same modality remain intermixed. In contrast, Figure \ref{fig:tsne}(d) shows that embeddings of both different modalities and labels are distinctly separable, highlighting that the reconstruction-based module effectively enhances modal specificity. Overall, RAMer accurately captures both the commonality and specificity of different modalities.

% \begin{figure}[thb]
%     \centering
%     \subfigure{
%         \includegraphics[width=0.45\columnwidth]{images/freq_w_adv.pdf}
%         \label{fig:left}
%     }
%     \hspace{0.05\textwidth} % 控制图片间的距离
%     \subfigure{
%         \includegraphics[width=0.45\columnwidth]{images/freq_wo_adv.pdf}
%         \label{fig:right}
%     }
%     \caption{The visualization of modality-to-label dependencies with/without (left/right) adversarial training, and the rows indicate different emotions and the columns denotes different modalities. Darker colors indicate stronger correlations.}
%     \label{fig:modality-to-label}
% \end{figure}

\subsubsection{Visualization of Modality-to-Label Correlations.}
To explore the relationship between modalities and labels, we visualized the correlation of labels with their most relevant modalities. As shown in Figure \ref{fig:modality-to-label}, regardless of the presence of adversarial training, different emotion label is influenced by different modalities. For instance, surprise is predominantly correlated with the acoustic modality, while anger is primarily associated with the visual modality. This indicates that each modality captures the distinguishable semantic information of the labels from distinct perspectives.

\begin{figure}[t]
    \centering
\includegraphics[width=0.6\linewidth]{images/freq_rec_horizontal_wo_adv.pdf}
    % \caption{With AT}
    \vspace{-1.0em}
    \caption{The correlation of modality-to-label dependencies.}
    \label{fig:modality-to-label}
\vspace{-0.5em}
\end{figure}

\begin{figure}[t!] 
% \vspace{-.3cm}
    \centering % 图片居中
\includegraphics[width=0.98\linewidth]{images/case_small.pdf}
    \vspace{-0.6em}
    \caption{An example of the case study results.}
    \label{fig:case study}
    \vspace{-1em}
\end{figure}

\subsubsection{Case Study}
% 
To demonstrate RAMer’s robustness in complex scenarios, Figure~\ref{fig:case study} shows an example of multi-party emotion recognition on MEmoR dataset where specific target persons have incomplete modality signals. The top three rows display different modalities from a video clip, segmented semantically with aligned multi-modal signals. Key observations include: 1) The target moment requires recognizing emotions for both the speaker (e.g., Howard) and non-speakers (e.g., Penny and Leonard). While the speaker typically has complete multi-modal signals, non-speakers often lack certain modalities. TAILOR, relying on incomplete modalities, produced partial predictions as its self-attention mechanisms struggled to align labels with missing features. 2) Limitations of single or incomplete modalities. A single modality, such as text, is often insufficient for accurate emotion inference (e.g., only Howard’s Joy is detectable from text alone). Although CARAT attempts to reconstruct missing information, it fails to capture cross-modal commonality, leading to incorrect predictions. 3)The importance of inter-person interactions and external knowledge is evident in emotion recognition, where inter-person attention enables individuals with incomplete modalities to gain supplementary information from others. Moreover, integrating external knowledge, such as personality traits, enhances emotion reasoning across participants and contexts, emphasizing the synergy between user profiling and emotion recognition. Experimental results validate that RAMer demonstrates superior robustness and effectiveness in these challenging, real-world scenarios.

\section*{Conclusion}
This paper aims to enhance our understanding of the computational complexity of computing various Shapley value variants. We found that for various ML models --- including decision trees, regression tree ensembles, weighted automata, and linear regression --- both local and global interventional and baseline SHAP can be computed in polynomial time under HMM modeled distributions. This extends popular algorithms, such as TreeSHAP, beyond their empirical distributional scope. We also establish strict complexity gaps between the various SHAP variants (baseline, interventional, and conditional) and prove the intractability of computing SHAP for tree ensembles and neural networks in simplified scenarios. Overall, we present SHAP as a versatile framework whose complexity depends on four key factors: \begin{inparaenum}[(i)] \item model type, \item SHAP variant, \item distribution modeling approach, \item and local vs. global explanations\end{inparaenum}. We believe this perspective provides deeper insight into the computational complexity of SHAP, paving the way for future work.




%We believe that our framework provides a more intricate understanding of SHAP computation complexity across different models, distributions, and variants, paving the way for further research.

Our work opens promising directions for future research. First, expanding our computational analysis to other SHAP-related metrics, such as asymmetric SHAP~\citep{frye20} and SAGE~\citep{covert2020understanding}, would be valuable. Additionally, we aim to explore more expressive distribution classes and relaxed assumptions beyond those in Section \ref{sec:tractable} while maintaining tractable SHAP computation. Finally, when exact computation is intractable (Section \ref{sec:intractable}), investigating the approximability of SHAP metrics through approximation and parameterized complexity theory~\citep{downey2012parameterized} is an important direction.

%Our work opens several promising avenues for future research on the computational properties of explainable AI methods, with a particular focus on SHAP. First, it would be interesting to broaden the computational analysis conducted in this work to include other popular SHAP-related metrics in the literature, such as asymmetric SHAP \cite{frye20} and SAGE \cite{covert2020understanding}. Also, in the future, we aim to explore more expressive distribution classes and relaxed distributional assumptions—extending beyond those examined in Section \ref{sec:tractable} —that still yield tractable SHAP computation. Finally, when exact computation proves intractable (Section \ref{sec:intractable}), it is worthwhile to theoretically investigate the question of the approximability of computing the SHAP metrics across various configurations, through the lens of approximation and parametrized complexity theory \cite{arora2009computational}.

%This paper aims to deepen our understanding of the computational complexity involved in obtaining different Shapley value variants. We found that for a variety of ML models, including decision trees, tree ensembles for regression, weighted automata, and linear regression models — computing both local and global interventional and baseline SHAP can be done in polynomial time when distributions are modeled by HMMs. This extends the distributional scope of popular algorithms like TreeSHAP, which is limited to empirical distributions. Additionally, we demonstrate a strict complexity gap between SHAP variants, showing that interventional and baseline SHAP can be strictly easier to compute than conditional SHAP. Despite these positive results, we uncovered intractability for various SHAP variants in neural networks and tree ensembles. Finally, we provided generalized complexity relations across SHAP variants. We believe that our framework offers a deeper understanding of the complexity involved in computing SHAP across various variants, models, distributions, as well as in both local and global computations, laying the groundwork for future research.

\clearpage

\bibliographystyle{named}
\bibliography{references}
\clearpage

\end{document}

