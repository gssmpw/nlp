% This must be in the first 5 lines to tell arXiv to use pdfLaTeX, which is strongly recommended.
\pdfoutput=1
% In particular, the hyperref package requires pdfLaTeX in order to break URLs across lines.

\documentclass[11pt]{article}

% Change "review" to "final" to generate the final (sometimes called camera-ready) version.
% Change to "preprint" to generate a non-anonymous version with page numbers.
\usepackage[preprint]{acl}

% Standard package includes
\usepackage{times}
\usepackage{latexsym}
\usepackage{amsmath} 
% For proper rendering and hyphenation of words containing Latin characters (including in bib files)
\usepackage[T1]{fontenc}
% For Vietnamese characters
% \usepackage[T5]{fontenc}
% See https://www.latex-project.org/help/documentation/encguide.pdf for other character sets

% This assumes your files are encoded as UTF8
\usepackage[utf8]{inputenc}

% This is not strictly necessary, and may be commented out,
% but it will improve the layout of the manuscript,
% and will typically save some space.
\usepackage{microtype}

% This is also not strictly necessary, and may be commented out.
% However, it will improve the aesthetics of text in
% the typewriter font.
\usepackage{inconsolata}

%Including images in your LaTeX document requires adding
%additional package(s)
\usepackage{graphicx}
\usepackage{multirow}
\usepackage{booktabs}
\usepackage{hyperref}
\usepackage{inconsolata}
\usepackage{pifont}% http://ctan.org/pkg/pifont
\newcommand{\cmark}{\ding{51}}
\newcommand{\xmark}{\ding{55}}
\usepackage{tabu}
\usepackage{fontawesome}
\usepackage{amsmath}
\usepackage{amsfonts}
\usepackage{amssymb}
\usepackage{enumitem}
\usepackage{listings}
\usepackage{xstring}
\usepackage{graphicx}
\usepackage{pbox}
\usepackage{subcaption}
\usepackage{epstopdf}
\usepackage{xstring}
\usepackage{multirow}
\newcommand{\sq}{\faCheckSquare}
\newcommand{\cq}{\faCheck}
\usepackage{soul}
\usepackage{color}
\usepackage{graphicx}
\usepackage{multirow}
\usepackage{comment}
\usepackage{todonotes}
\usepackage{tcolorbox}

% \usepackage{polyglossia}
\usepackage{listings} % Add this for using the listings package
\usepackage{graphicx}
\usepackage{subcaption}
\usepackage{verbatim}
\usepackage{stfloats}
\renewcommand{\UrlFont}{\ttfamily\small}

% \usepackage{xltxtra}  % Extra features for XeLaTeX
% \usepackage{polyglossia}
% \setmainlanguage{english}  % Set main language as English
% \setotherlanguage{bengali}  % Enable Bengali language
% \newfontfamily\bengalifont[Script=Bengali]{Noto Sans Bengali} % Set Bengali font


\lstset{
  basicstyle=\ttfamily, %\small,
  breaklines=true,
  aboveskip=4mm,
  belowskip=4mm,  
  captionpos=b, % Position the caption at the bottom
  %frame=single, % Adds a frame around the listing
  columns=fullflexible
}

\usepackage{xcolor} % For coloring text
\definecolor{lightblue}{rgb}{.50,.90,0.51}
\definecolor{tri}{rgb}{.25,.88,.82}
\definecolor{lilac}{rgb}{0.85,0.64,0.85}
\definecolor{atomictangerine}{rgb}{1.0, 0.6, 0.4}
% \newcommand\firoj[1]{\hlc[blue]{{\bf Firoj}: #1}}
\newcommand{\firoj}[1]{{\color{blue}\textbf{Firoj}: #1}}
\newcommand{\com}[1]{{\color{red}\textbf{Firoj}: #1}}
\newcommand{\titu}{\emph{TituLLMs}}

% If the title and author information does not fit in the area allocated, uncomment the following
%
%\setlength\titlebox{<dim>}
%
% and set <dim> to something 5cm or larger.

\title{TituLLMs: A Family of Bangla LLMs with Comprehensive Benchmarking}

% Author information can be set in various styles:
% For several authors from the same institution:
% \author{Author 1 \and ... \and Author n \\
%         Address line \\ ... \\ Address line}
% if the names do not fit well on one line use
%         Author 1 \\ {\bf Author 2} \\ ... \\ {\bf Author n} \\
% For authors from different institutions:
% \author{Author 1 \\ Address line \\  ... \\ Address line
%         \And  ... \And
%         Author n \\ Address line \\ ... \\ Address line}
% To start a separate ``row'' of authors use \AND, as in
% \author{Author 1 \\ Address line \\  ... \\ Address line
%         \AND
%         Author 2 \\ Address line \\ ... \\ Address line \And
%         Author 3 \\ Address line \\ ... \\ Address line}


\author{
\textbf{Shahriar Kabir Nahin}\textsuperscript{1},
\textbf{Rabindra Nath Nandi}\textsuperscript{1},
\textbf{Sagor Sarker}\textsuperscript{1}, \\
\textbf{Quazi Sarwar Muhtaseem}\textsuperscript{1}, 
\textbf{Md Kowsher}\textsuperscript{2},
\textbf{Apu Chandraw Shill}\textsuperscript{1},
\textbf{Md Ibrahim}\textsuperscript{1}, \\
\textbf{Mehadi Hasan Menon}\textsuperscript{1},
\textbf{Tareq Al Muntasir}\textsuperscript{1}, 
\textbf{Firoj Alam}\textsuperscript{3}, \\
\textsuperscript{1}Hishab Singapore Pte. Ltd, Singapore, 
\textsuperscript{2}University of Central Florida, USA \\
\textsuperscript{3}Qatar Computing Research Institute, Qatar
}




\begin{document}
\maketitle
\begin{abstract}
In this paper, we present \titu{}, the \textit{first} large pretrained Bangla LLMs, available in 1b and 3b parameter sizes. Due to computational constraints during both training and inference, we focused on smaller models. To train \titu{}, we collected a pretraining dataset of approximately $\sim37$ billion tokens. We extended the Llama-3.2 tokenizer to incorporate language- and culture-specific knowledge, which also enables faster training and inference.
There was a lack of benchmarking datasets to benchmark LLMs for Bangla. To address this gap, we developed \textit{five benchmarking datasets}. We benchmarked various LLMs, including \titu{}, and demonstrated that \titu{} outperforms its initial multilingual versions. However, this is not always the case, highlighting the complexities of language adaptation. Our work lays the groundwork for adapting existing multilingual open models to other low-resource languages. To facilitate broader adoption and further research, we have made the \titu{} models and benchmarking datasets publicly available.\footnote{\url{https://huggingface.co/collections/hishab/titulm-llama-family-6718d31fc1b83529276f490a}}


% Significant progress has been made in large language models (LLMs), leading to their widespread adoption across various disciplines and applications. However, a substantial gap remains in their capabilities for low-resource languages. One approach to addressing this issue is to adapt existing open LLMs for new languages. This, however, presents several challenges, including the need for vast amounts of training data, limitations in computational resources, and the difficulty of collecting benchmarking datasets.

% To benchmark the \titu{} including vario 
% explore the development of language-specific LLMs for Bangla and provide a comprehensive benchmarking analysis of several multilingual LLMs. 

% and develop \textit{five benchmarking datasets}. We train a family of models, \titu{}, based on Llama, by extending the tokenizer, which is the \textit{first} set of Bangla LLMs trained with a largest dataset. Our findings show that in several benchmarking datasets, the adapted LLMs outperform their initial multilingual versions. However, this is not always the case, highlighting the complexities of language adaptation.
% Our work lays the groundwork for adapting existing multilingual open models to other low-resource languages. To facilitate broader adoption and further research, we have made the \titu{} models and benchmarking datasets publicly available.\footnote{\url{anonymous.com}}  
% The widespread adoption of large language models (LLMs) for both research and industrial applications necessitates a comprehensive understanding of their performance across various languages, especially low-resource languages like Bangla. As LLMs such as ChatGPT, LLaMA, and others are increasingly employed in multilingual contexts, it is critical to evaluate their capabilities in these underrepresented languages. In this work, we focus on benchmarking popular multilingual LLMs on Bangla, a low-resource language, by creating diverse test sets comprising multiple-choice questions (MCQ), question-answer (QA) pairs, and true/false statements. Our aim is to assess the competency, strengths, and limitations of these models when handling Bangla-specific tasks. This benchmark evaluation will provide valuable insights into the performance gaps of current LLMs in Bangla and highlight potential areas for improvement, paving the way for more equitable advancements in natural language processing across different languages. The experimental resources will be publicly available.\footnote{\url{}}
\end{abstract}

\section{Introduction}
\label{sec:introduction}
The business processes of organizations are experiencing ever-increasing complexity due to the large amount of data, high number of users, and high-tech devices involved \cite{martin2021pmopportunitieschallenges, beerepoot2023biggestbpmproblems}. This complexity may cause business processes to deviate from normal control flow due to unforeseen and disruptive anomalies \cite{adams2023proceddsriftdetection}. These control-flow anomalies manifest as unknown, skipped, and wrongly-ordered activities in the traces of event logs monitored from the execution of business processes \cite{ko2023adsystematicreview}. For the sake of clarity, let us consider an illustrative example of such anomalies. Figure \ref{FP_ANOMALIES} shows a so-called event log footprint, which captures the control flow relations of four activities of a hypothetical event log. In particular, this footprint captures the control-flow relations between activities \texttt{a}, \texttt{b}, \texttt{c} and \texttt{d}. These are the causal ($\rightarrow$) relation, concurrent ($\parallel$) relation, and other ($\#$) relations such as exclusivity or non-local dependency \cite{aalst2022pmhandbook}. In addition, on the right are six traces, of which five exhibit skipped, wrongly-ordered and unknown control-flow anomalies. For example, $\langle$\texttt{a b d}$\rangle$ has a skipped activity, which is \texttt{c}. Because of this skipped activity, the control-flow relation \texttt{b}$\,\#\,$\texttt{d} is violated, since \texttt{d} directly follows \texttt{b} in the anomalous trace.
\begin{figure}[!t]
\centering
\includegraphics[width=0.9\columnwidth]{images/FP_ANOMALIES.png}
\caption{An example event log footprint with six traces, of which five exhibit control-flow anomalies.}
\label{FP_ANOMALIES}
\end{figure}

\subsection{Control-flow anomaly detection}
Control-flow anomaly detection techniques aim to characterize the normal control flow from event logs and verify whether these deviations occur in new event logs \cite{ko2023adsystematicreview}. To develop control-flow anomaly detection techniques, \revision{process mining} has seen widespread adoption owing to process discovery and \revision{conformance checking}. On the one hand, process discovery is a set of algorithms that encode control-flow relations as a set of model elements and constraints according to a given modeling formalism \cite{aalst2022pmhandbook}; hereafter, we refer to the Petri net, a widespread modeling formalism. On the other hand, \revision{conformance checking} is an explainable set of algorithms that allows linking any deviations with the reference Petri net and providing the fitness measure, namely a measure of how much the Petri net fits the new event log \cite{aalst2022pmhandbook}. Many control-flow anomaly detection techniques based on \revision{conformance checking} (hereafter, \revision{conformance checking}-based techniques) use the fitness measure to determine whether an event log is anomalous \cite{bezerra2009pmad, bezerra2013adlogspais, myers2018icsadpm, pecchia2020applicationfailuresanalysispm}. 

The scientific literature also includes many \revision{conformance checking}-independent techniques for control-flow anomaly detection that combine specific types of trace encodings with machine/deep learning \cite{ko2023adsystematicreview, tavares2023pmtraceencoding}. Whereas these techniques are very effective, their explainability is challenging due to both the type of trace encoding employed and the machine/deep learning model used \cite{rawal2022trustworthyaiadvances,li2023explainablead}. Hence, in the following, we focus on the shortcomings of \revision{conformance checking}-based techniques to investigate whether it is possible to support the development of competitive control-flow anomaly detection techniques while maintaining the explainable nature of \revision{conformance checking}.
\begin{figure}[!t]
\centering
\includegraphics[width=\columnwidth]{images/HIGH_LEVEL_VIEW.png}
\caption{A high-level view of the proposed framework for combining \revision{process mining}-based feature extraction with dimensionality reduction for control-flow anomaly detection.}
\label{HIGH_LEVEL_VIEW}
\end{figure}

\subsection{Shortcomings of \revision{conformance checking}-based techniques}
Unfortunately, the detection effectiveness of \revision{conformance checking}-based techniques is affected by noisy data and low-quality Petri nets, which may be due to human errors in the modeling process or representational bias of process discovery algorithms \cite{bezerra2013adlogspais, pecchia2020applicationfailuresanalysispm, aalst2016pm}. Specifically, on the one hand, noisy data may introduce infrequent and deceptive control-flow relations that may result in inconsistent fitness measures, whereas, on the other hand, checking event logs against a low-quality Petri net could lead to an unreliable distribution of fitness measures. Nonetheless, such Petri nets can still be used as references to obtain insightful information for \revision{process mining}-based feature extraction, supporting the development of competitive and explainable \revision{conformance checking}-based techniques for control-flow anomaly detection despite the problems above. For example, a few works outline that token-based \revision{conformance checking} can be used for \revision{process mining}-based feature extraction to build tabular data and develop effective \revision{conformance checking}-based techniques for control-flow anomaly detection \cite{singh2022lapmsh, debenedictis2023dtadiiot}. However, to the best of our knowledge, the scientific literature lacks a structured proposal for \revision{process mining}-based feature extraction using the state-of-the-art \revision{conformance checking} variant, namely alignment-based \revision{conformance checking}.

\subsection{Contributions}
We propose a novel \revision{process mining}-based feature extraction approach with alignment-based \revision{conformance checking}. This variant aligns the deviating control flow with a reference Petri net; the resulting alignment can be inspected to extract additional statistics such as the number of times a given activity caused mismatches \cite{aalst2022pmhandbook}. We integrate this approach into a flexible and explainable framework for developing techniques for control-flow anomaly detection. The framework combines \revision{process mining}-based feature extraction and dimensionality reduction to handle high-dimensional feature sets, achieve detection effectiveness, and support explainability. Notably, in addition to our proposed \revision{process mining}-based feature extraction approach, the framework allows employing other approaches, enabling a fair comparison of multiple \revision{conformance checking}-based and \revision{conformance checking}-independent techniques for control-flow anomaly detection. Figure \ref{HIGH_LEVEL_VIEW} shows a high-level view of the framework. Business processes are monitored, and event logs obtained from the database of information systems. Subsequently, \revision{process mining}-based feature extraction is applied to these event logs and tabular data input to dimensionality reduction to identify control-flow anomalies. We apply several \revision{conformance checking}-based and \revision{conformance checking}-independent framework techniques to publicly available datasets, simulated data of a case study from railways, and real-world data of a case study from healthcare. We show that the framework techniques implementing our approach outperform the baseline \revision{conformance checking}-based techniques while maintaining the explainable nature of \revision{conformance checking}.

In summary, the contributions of this paper are as follows.
\begin{itemize}
    \item{
        A novel \revision{process mining}-based feature extraction approach to support the development of competitive and explainable \revision{conformance checking}-based techniques for control-flow anomaly detection.
    }
    \item{
        A flexible and explainable framework for developing techniques for control-flow anomaly detection using \revision{process mining}-based feature extraction and dimensionality reduction.
    }
    \item{
        Application to synthetic and real-world datasets of several \revision{conformance checking}-based and \revision{conformance checking}-independent framework techniques, evaluating their detection effectiveness and explainability.
    }
\end{itemize}

The rest of the paper is organized as follows.
\begin{itemize}
    \item Section \ref{sec:related_work} reviews the existing techniques for control-flow anomaly detection, categorizing them into \revision{conformance checking}-based and \revision{conformance checking}-independent techniques.
    \item Section \ref{sec:abccfe} provides the preliminaries of \revision{process mining} to establish the notation used throughout the paper, and delves into the details of the proposed \revision{process mining}-based feature extraction approach with alignment-based \revision{conformance checking}.
    \item Section \ref{sec:framework} describes the framework for developing \revision{conformance checking}-based and \revision{conformance checking}-independent techniques for control-flow anomaly detection that combine \revision{process mining}-based feature extraction and dimensionality reduction.
    \item Section \ref{sec:evaluation} presents the experiments conducted with multiple framework and baseline techniques using data from publicly available datasets and case studies.
    \item Section \ref{sec:conclusions} draws the conclusions and presents future work.
\end{itemize}
%\newcommand{\CompCertUrl}[0]{\href{https://github.com/AbsInt/CompCert}{https://github.com/AbsInt/CompCert}}
\newcommand{\MathCompUrl}[0]{\href{https://github.com/math-comp/math-comp}{https://github.com/math-comp/math-comp}}
\newcommand{\GeoCoqUrl}[0]{\href{https://github.com/GeoCoq/GeoCoq}{https://github.com/GeoCoq/GeoCoq}}
\newcommand{\CategoryTheoryUrl}[0]{\href{https://github.com/jwiegley/category-theory}{https://github.com/jwiegley/category-theory}}
\newcommand{\LeanUrl}[0]{\href{https://github.com/leanprover-community/mathlib4}{https://github.com/leanprover-community/mathlib4}}
\newcommand{\CodeTFiveModelSize}[0]{\texttt{60 M}}
\newcommand{\CatTheory}[0]{\textsc{Cat-Theory}}

In this section, we explain our dataset construction and model training choices towards our demonstration of positive transfer from multilingual training as well as adaptability to new domains via further fine-tuning.
% We also investigate further fine-tuning them on held-out datasets to see if our models can adapt to on a different domains following the primary training. 

\subsection{Dataset Details}
We collect datasets across multiple languages and language versions of Coq and Lean 4, sourcing data from existing repositories. Our data collection approach involves collecting proof states from the ITP through tactic execution. We construct several data-mixes, of different subsets of the accumulated data, to train various monolingual and multilingual \proofwala{} models to perform proof step prediction. The training data is formatted into prompts as shown in \Cref{fig:prompt-format} (in \Cref{app:training-data}). We collect proof-step data for the various data mixtures as shown in \Cref{tab:data-mix}. 

\renewcommand\theadfont{}
\begin{table}[t!]
    \centering
    \scalebox{0.7}{
    \begin{tabular}{llll}
    \toprule
    \multicolumn{4}{c}{\thead{\textbf{Initial Fine-tuning}}}\\
    \hline
    \thead{\textbf{Data-mix}}  & \thead{\textbf{Data-mix Source}}  & \thead{\textbf{\name\;}\\\textbf{Models Trained}} & \thead{\textbf{Token Count}}\\
    \toprule
    1. CompCert\footref{fnote:CompCert-url} & CompCert Repo\footref{fnote:same-as-proverbot} & - & 61.6 M \\
    2. MathComp\footref{fnote:MathComp-url} & MathComp Repo & - & 18.2 M\\
    3. GeoCoq\footref{fnote:GeoCoq-url} & GeoCoq Repo & - & 91.2 M\\
    \renewcommand\theadfont{}
    4. \coq & {\textbf{Data-Mixes:} 1-3} & \thead[l]{\coq} & 171 M \\
    \renewcommand\theadfont{}
    5. \lean\footref{fnote:Lean-url} & Mathlib Repo\footref{fnote:same-as-reprover} & \thead[l]{\lean} & 99 M \\
    \renewcommand\theadfont{}
    6. \multi & {\textbf{Data-Mixes:} 4-5} & \thead[l]{\multi} & 270 M \\
    \hline
     \multicolumn{4}{c}{\thead{\textbf{Further Fine-tuning}}}\\
    \hline
    \renewcommand\theadfont{}
    7. CategoryTheory\footref{fnote:CategoryTheory-url} & CategoryTheory Repo & \thead[l]{\multi-\\\CatTheory\;\;\&\;\\\coq-\\\CatTheory} & 1.7 M\\
    \bottomrule
    \end{tabular}}
    \caption{Different data-mixes used to extract proof-step and proof state pair data. Various \proofwala\; models trained on these data mixes.}
    % \\\textsuperscript{*}\small{Same as Proverbot split \citep{sanchez2020generating}}
    % \\\textsuperscript{**}\small{Same as random split in ReProver \citep{yang2023leandojo}}}
    \label{tab:data-mix}
\end{table}
\footnotetext{\label{fnote:same-as-proverbot}Same as Proverbot split \citep{sanchez2020generating}}
\addtocounter{footnote}{+1}\footnotetext{\label{fnote:same-as-reprover}Same as random split in ReProver \citep{yang2023leandojo}}

We use different Coq and Lean repositories to generate this proof-step data. We use well-known repositories, namely CompCert,\footnote{\label{fnote:CompCert-url}\CompCertUrl} Mathlib,\footnote{\label{fnote:Lean-url}\LeanUrl} MathComp,\footnote{\label{fnote:MathComp-url}\MathCompUrl} GeoCoq,\footnote{\label{fnote:GeoCoq-url}\GeoCoqUrl} and CategoryTheory,\footnote{\label{fnote:CategoryTheory-url}\CategoryTheoryUrl} to generate the proof-step data. For CompCert we used the train-test split proposed by \citet{sanchez2020generating}, and for Mathlib we used the split proposed by \citet{yang2023leandojo}. Together we have \texttt{442607} proof-step pairs derived from over \texttt{76997} theorems across Lean and Coq (details of the split shown in \Cref{tab:data-mix-size}). We hold out the CategoryTheory dataset from initial training data-mixes for experimentation with further fine-tuning for our novel domain adaptation experiment.
% We also introduce some new 
% \george{I'm not sure what you mean by new, Coq-specific data collection have used these for sure. I would rephrase the beginning of this paragraph as (loosely) "We collect from Mathlib, Compcert, GeoCoq, CategoryTheory repositories which constritute some of the largest and most popular repositories in both ITPs"} proof-step data from MathComp\footnote{\label{fnote:MathComp-url}\MathCompUrl}, GeoCoq\footnote{\label{fnote:GeoCoq-url}\GeoCoqUrl}, and CategoryTheory\footnote{\label{fnote:CategoryTheory-url}\CategoryTheoryUrl} repositories in Coq.
% \gd{you need much more info here. how were these converted to Coq? what previous work does this build on? you need citations here and description inlined here, then you can remove them from the table, and ideally examples of the datasets in the appendix} 
 %We trained each of our models over \texttt{2.71 B} tokens. \amit{Change all the entries in \Cref{tab:data-mix} once the models are re-trained}

% \begin{table}[ht]
%     \centering
%     \footnotesize
%     \begin{tabular}{l|l|l|l|l|l|l}
%     \hline
%     \thead{\textbf{Data-mix}\\\textbf{Name}}  & \multicolumn{3}{c|}{\thead{\textbf{Proof-Step \&}\\\textbf{State Pair Count}}}  & \multicolumn{3}{c}{\thead{\textbf{Theorem}\\\textbf{Count}}}\\
%     \hline
%     & \textbf{Train} & \textbf{Test} & \textbf{Val} & \textbf{Train} & \textbf{Test} & \textbf{Val}\\
%     \hline
%     1. CompCert & 80288 & 6199 & - & 5440 & 501 & - \\
%     2. MathComp & 34543 & 711 & 720 & 11385 & 220 & 221 \\
%     3. GeoCoq & 104045 & 1866 & 2351 & 4537 & 89 & 88 \\
%     4. CategoryTheory & 4763 & 82 & 48 & 690 & 14 & 13 \\
%     5. \coq & 223639 & 8858 & 3119 & 22152 & 824 & 322 \\
%     6. \lean & 237003 & 4323 & 4220 & 56140 & 991 & 1035 \\
%     7. \multi & 460642 & 13181 & 7339 & 78292 & 1815 & 1357 \\
%     \end{tabular}
%     \caption{Size of different data-mixes. The \proofwala\;models were trained on the training split of \coq, \lean, and \multi\;data-mixes. After extracting proof-step and state pair data, the training, validation, and test split are randomly decided. For \lean\;and CompCert data-mix we used the same split as proposed by \citet{yang2023leandojo} and \citet{sanchez2020generating} respectively.}
%     \label{tab:data-mix-size}
% \end{table}

\begin{table}[ht]
    \centering
    % \footnotesize
    \scalebox{0.75}{
    \begin{tabular}{lrrrrrr}
    \hline
    & \multicolumn{3}{c}{\thead{\textbf{\# Proof-Step \& State Pairs}}}  & \multicolumn{3}{c}{\thead{\textbf{Theorem Count}}}\\
    \cmidrule(lr){2-4}\cmidrule(lr){5-7}
    \thead{\textbf{Data-mix}} & \textbf{Train} & \textbf{Test} & \textbf{Val} & \textbf{Train} & \textbf{Test} & \textbf{Val}\\
    \toprule
    1. CompCert & 80288 & 6199 & - & 5440 & 501 & - \\
    2. MathComp & 34196 & 1378 & 2285 & 11381 & 536 & 729 \\
    3. GeoCoq & 91120 & 12495 & 4928 & 4036 & 505 & 208 \\
    4. \coq & 205604 & 20072 & 7213 & 20857 & 1542 & 937 \\ %Update the sum
    5. \lean & 237003 & 4323 & 4220 & 56140\textsuperscript{\footref{fnote:lean-test-set-size}} & 991\textsuperscript{\footref{fnote:lean-test-set-size}} & 1035\textsuperscript{\footref{fnote:lean-test-set-size}} \\
    6. \multi & 442607 & 24395 & 11433 & 76997 & 2533 & 1972 \\
    7. CategoryTheory & 4114 & 610 & 208 & 573 & 101 & 43 \\
    \bottomrule
    \end{tabular}}
    \caption{Size of different data-mixes. The \proofwala\;models were trained on the training split of \coq, \lean, and \multi\;data-mixes. After extracting proof-step and state pair data, the training, validation, and test split are randomly decided so as to include at least 500 testing theorems, except in the case of CategoryTheory where the overall dataset is relatively small. For the \lean\;and CompCert data-mix we used the same split as proposed by \citet{yang2023leandojo}\footref{fnote:lean-test-set-size} and \citet{sanchez2020generating} respectively.
    }
    \label{tab:data-mix-size}
\end{table}
\addtocounter{footnote}{+1}
\footnotetext{\label{fnote:lean-test-set-size}While the LeanDojo dataset \citep{yang2023leandojo} officially has 2000 test theorems, only 991 of these are proved using tactics and have their tactics extracted in the dataset. Since our approach involves generating only tactic-based proofs, our Lean dataset is collected from those theorems with tactic-based proofs.}


\subsection{Model Details}
\label{sec:model-details}
We used the \codeTFive-\base\;\citep{wang2021codet5} pretrained model---which has \texttt{220} million parameters---to fine-tune models on the different data-mixes as described in \Cref{tab:data-mix}.
We trained three models \proofwala-\{\multi, \coq, \lean\} with the same step count and batch sizes for all settings. Training the models with the same number of steps aligns with recent work on training models for multilingual autoformalization \citep{jiang2023multilingualmathematicalautoformalization} which ensures that each model has the same number of gradient updates. Our models are initially trained on CompCert, Mathlib, MathComp, and GeoCoq. The hyperparameters used for training are described in \Cref{tab:hyperparams} in \Cref{app:hyperparams}. \Cref{app:hyperparams} also describes the amount of computing we used to train our models. 

To demonstrate the usefulness of our models on subsequent theorem-proving tasks, we perform further fine-tune of our \proofwala-\{\multi, \coq\} models on CategoryTheory\footref{fnote:CategoryTheory-url} theory data. We used the same hyperparameters as \Cref{tab:hyperparams} (in \Cref{app:hyperparams}) but we reduce the number of training steps to 1200 and batch size to 8.

\begin{table*}[ht]
    \centering
    \scalebox{0.8}{
    \begin{tabular}{lclllllll}
    \toprule
    \multicolumn{2}{c}{\textbf{Data-Mix}} &  
     & 
    \multicolumn{5}{c}{\textbf{Pass-at-$k$} \%} &
    \\
    \cmidrule(lr){1-2}\cmidrule(lr){4-8}
    \textbf{Name} & 
    \textbf{\# Theorems} & 
    \thead[c]{\textbf{Proof Step Model}} &
    \textbf{Pass@1} & 
    \textbf{Pass@2} & 
    \textbf{Pass@3} & 
    \textbf{Pass@4} & 
    \textbf{Pass@5} & 
    \thead[c]{\textbf{$p_{\mathrm{value}}$}\\(\textbf{$\alpha$}: 0.05)\footref{fnote:p-value}}\\
    \toprule
    \textbf{\lean} & 
    991 &
    \proofwala-\lean & 
    24.92 & 
    26.64 & 
    27.54 &
    28.05 &
    28.25 &
    \\
     & 
     & 
     \proofwala-\multi & 
     \textbf{26.84} & 
     \textbf{28.56} & 
     \textbf{29.67} &
     \textbf{29.97} &
     \textbf{30.58} &
     \textbf{0.018} \\
    \hline
    \textbf{MathComp} & 
    536 & 
    \proofwala-\coq & 
    \textbf{28.28} & 
    28.65 & 
    29.4 &
    29.59 &
    30.15 &
    \\
     &
     &
     \proofwala-\multi & 
     27.9 & 
     \textbf{29.21} & 
     \textbf{29.59} &
     \textbf{30.15} &
     \textbf{30.52} &
     0.355
     \\
    \hline
    \textbf{GeoCoq} & 
    505 & 
    \proofwala-\coq & 
    \textbf{32.87} & 
    \textbf{33.66} & 
    33.86 &
    34.06 &
    34.46 &
    \\
    & 
    & 
    \proofwala-\multi & 
    30.89 & 
    \textbf{33.66} & 
    \textbf{34.65} &
    \textbf{35.64} &
    \textbf{35.84} &
    0.135
    \\
    \hline
    \textbf{CompCert} &
    501 & 
    \proofwala-\coq &  
     17.56 & 
     18.76 & 
     19.16 &
     19.76 &
     20.76 &
     \\
     &
     & 
     \proofwala-\multi & 
     \textbf{17.96} & 
     \textbf{19.76} & 
     \textbf{20.56} &
     \textbf{21.16} &
     \textbf{21.96} &
     0.191 \\
    \hline
    \textbf{CategoryTheory} & 
    101 & 
    \proofwala-\coq-\CatTheory & 
    36.63 & 
    42.57 & 
    44.55 & 
    44.55 & 
    45.54 & \\
    & 
    & 
    \proofwala-\multi-\CatTheory & 
    \textbf{44.55} & 
    \textbf{51.49} & 
    \textbf{52.48} & 
    \textbf{53.47} & 
    \textbf{53.47} & 
    \textbf{0.008} \\
    \bottomrule
    \end{tabular}}
    \caption{Comparison between various \proofwala$\;$models and the \proofwala-\multi\; model on different data-mixes. We can see that transfer happening between Lean and Coq on all data-mixes from various domains in math and software verification. We observe that the \multi\; model outperforms the \lean\; and \coq\; models on all data mixes. The performance improvement is also statistically significant on the biggest data-mix \lean\; (Mathlib). We also observe that after further fine-tuning, the \multi\; model significantly outperforms the \coq\; model on the CategoryTheory dataset.
    }
    \label{tab:all-experiments}
\end{table*}
\addtocounter{footnote}{+1}
\footnotetext{\label{fnote:p-value}The results are statistically significant using a paired bootstrap test if $p_{\mathrm{value}} < 0.05$.}
 \section{Dataset}
\label{sec:dataset}

\subsection{Data Collection}

To analyze political discussions on Discord, we followed the methodology in \cite{singh2024Cross-Platform}, collecting messages from politically-oriented public servers in compliance with Discord's platform policies.

Using Discord's Discovery feature, we employed a web scraper to extract server invitation links, names, and descriptions, focusing on public servers accessible without participation. Invitation links were used to access data via the Discord API. To ensure relevance, we filtered servers using keywords related to the 2024 U.S. elections (e.g., Trump, Kamala, MAGA), as outlined in \cite{balasubramanian2024publicdatasettrackingsocial}. This resulted in 302 server links, further narrowed to 81 English-speaking, politics-focused servers based on their names and descriptions.

Public messages were retrieved from these servers using the Discord API, collecting metadata such as \textit{content}, \textit{user ID}, \textit{username}, \textit{timestamp}, \textit{bot flag}, \textit{mentions}, and \textit{interactions}. Through this process, we gathered \textbf{33,373,229 messages} from \textbf{82,109 users} across \textbf{81 servers}, including \textbf{1,912,750 messages} from \textbf{633 bots}. Data collection occurred between November 13th and 15th, covering messages sent from January 1st to November 12th, just after the 2024 U.S. election.

\subsection{Characterizing the Political Spectrum}
\label{sec:timeline}

A key aspect of our research is distinguishing between Republican- and Democratic-aligned Discord servers. To categorize their political alignment, we relied on server names and self-descriptions, which often include rules, community guidelines, and references to key ideologies or figures. Each server's name and description were manually reviewed based on predefined, objective criteria, focusing on explicit political themes or mentions of prominent figures. This process allowed us to classify servers into three categories, ensuring a systematic and unbiased alignment determination.

\begin{itemize}
    \item \textbf{Republican-aligned}: Servers referencing Republican and right-wing and ideologies, movements, or figures (e.g., MAGA, Conservative, Traditional, Trump).  
    \item \textbf{Democratic-aligned}: Servers mentioning Democratic and left-wing ideologies, movements, or figures (e.g., Progressive, Liberal, Socialist, Biden, Kamala).  
    \item \textbf{Unaligned}: Servers with no defined spectrum and ideologies or opened to general political debate from all orientations.
\end{itemize}

To ensure the reliability and consistency of our classification, three independent reviewers assessed the classification following the specified set of criteria. The inter-rater agreement of their classifications was evaluated using Fleiss' Kappa \cite{fleiss1971measuring}, with a resulting Kappa value of \( 0.8191 \), indicating an almost perfect agreement among the reviewers. Disagreements were resolved by adopting the majority classification, as there were no instances where a server received different classifications from all three reviewers. This process guaranteed the consistency and accuracy of the final categorization.

Through this process, we identified \textbf{7 Republican-aligned servers}, \textbf{9 Democratic-aligned servers}, and \textbf{65 unaligned servers}.

Table \ref{tab:statistics} shows the statistics of the collected data. Notably, while Democratic- and Republican-aligned servers had a comparable number of user messages, users in the latter servers were significantly more active, posting more than double the number of messages per user compared to their Democratic counterparts. 
This suggests that, in our sample, Democratic-aligned servers attract more users, but these users were less engaged in text-based discussions. Additionally, around 10\% of the messages across all server categories were posted by bots. 

\subsection{Temporal Data} 

Throughout this paper, we refer to the election candidates using the names adopted by their respective campaigns: \textit{Kamala}, \textit{Biden}, and \textit{Trump}. To examine how the content of text messages evolves based on the political alignment of servers, we divided the 2024 election year into three periods: \textbf{Biden vs Trump} (January 1 to July 21), \textbf{Kamala vs Trump} (July 21 to September 20), and the \textbf{Voting Period} (after September 20). These periods reflect key phases of the election: the early campaign dominated by Biden and Trump, the shift in dynamics with Kamala Harris replacing Joe Biden as the Democratic candidate, and the final voting stage focused on electoral outcomes and their implications. This segmentation enables an analysis of how discourse responds to pivotal electoral moments.

Figure \ref{fig:line-plot} illustrates the distribution of messages over time, highlighting trends in total messages volume and mentions of each candidate. Prior to Biden's withdrawal on July 21, mentions of Biden and Trump were relatively balanced. However, following Kamala's entry into the race, mentions of Trump surged significantly, a trend further amplified by an assassination attempt on him, solidifying his dominance in the discourse. The only instance where Trump’s mentions were exceeded occurred during the first debate, as concerns about Biden’s age and cognitive abilities temporarily shifted the focus. In the final stages of the election, mentions of all three candidates rose, with Trump’s mentions peaking as he emerged as the victor.

% \input{sections/EST}
\section{Experiments}
\label{sec:experiments}
The experiments are designed to address two key research questions.
First, \textbf{RQ1} evaluates whether the average $L_2$-norm of the counterfactual perturbation vectors ($\overline{||\perturb||}$) decreases as the model overfits the data, thereby providing further empirical validation for our hypothesis.
Second, \textbf{RQ2} evaluates the ability of the proposed counterfactual regularized loss, as defined in (\ref{eq:regularized_loss2}), to mitigate overfitting when compared to existing regularization techniques.

% The experiments are designed to address three key research questions. First, \textbf{RQ1} investigates whether the mean perturbation vector norm decreases as the model overfits the data, aiming to further validate our intuition. Second, \textbf{RQ2} explores whether the mean perturbation vector norm can be effectively leveraged as a regularization term during training, offering insights into its potential role in mitigating overfitting. Finally, \textbf{RQ3} examines whether our counterfactual regularizer enables the model to achieve superior performance compared to existing regularization methods, thus highlighting its practical advantage.

\subsection{Experimental Setup}
\textbf{\textit{Datasets, Models, and Tasks.}}
The experiments are conducted on three datasets: \textit{Water Potability}~\cite{kadiwal2020waterpotability}, \textit{Phomene}~\cite{phomene}, and \textit{CIFAR-10}~\cite{krizhevsky2009learning}. For \textit{Water Potability} and \textit{Phomene}, we randomly select $80\%$ of the samples for the training set, and the remaining $20\%$ for the test set, \textit{CIFAR-10} comes already split. Furthermore, we consider the following models: Logistic Regression, Multi-Layer Perceptron (MLP) with 100 and 30 neurons on each hidden layer, and PreactResNet-18~\cite{he2016cvecvv} as a Convolutional Neural Network (CNN) architecture.
We focus on binary classification tasks and leave the extension to multiclass scenarios for future work. However, for datasets that are inherently multiclass, we transform the problem into a binary classification task by selecting two classes, aligning with our assumption.

\smallskip
\noindent\textbf{\textit{Evaluation Measures.}} To characterize the degree of overfitting, we use the test loss, as it serves as a reliable indicator of the model's generalization capability to unseen data. Additionally, we evaluate the predictive performance of each model using the test accuracy.

\smallskip
\noindent\textbf{\textit{Baselines.}} We compare CF-Reg with the following regularization techniques: L1 (``Lasso''), L2 (``Ridge''), and Dropout.

\smallskip
\noindent\textbf{\textit{Configurations.}}
For each model, we adopt specific configurations as follows.
\begin{itemize}
\item \textit{Logistic Regression:} To induce overfitting in the model, we artificially increase the dimensionality of the data beyond the number of training samples by applying a polynomial feature expansion. This approach ensures that the model has enough capacity to overfit the training data, allowing us to analyze the impact of our counterfactual regularizer. The degree of the polynomial is chosen as the smallest degree that makes the number of features greater than the number of data.
\item \textit{Neural Networks (MLP and CNN):} To take advantage of the closed-form solution for computing the optimal perturbation vector as defined in (\ref{eq:opt-delta}), we use a local linear approximation of the neural network models. Hence, given an instance $\inst_i$, we consider the (optimal) counterfactual not with respect to $\model$ but with respect to:
\begin{equation}
\label{eq:taylor}
    \model^{lin}(\inst) = \model(\inst_i) + \nabla_{\inst}\model(\inst_i)(\inst - \inst_i),
\end{equation}
where $\model^{lin}$ represents the first-order Taylor approximation of $\model$ at $\inst_i$.
Note that this step is unnecessary for Logistic Regression, as it is inherently a linear model.
\end{itemize}

\smallskip
\noindent \textbf{\textit{Implementation Details.}} We run all experiments on a machine equipped with an AMD Ryzen 9 7900 12-Core Processor and an NVIDIA GeForce RTX 4090 GPU. Our implementation is based on the PyTorch Lightning framework. We use stochastic gradient descent as the optimizer with a learning rate of $\eta = 0.001$ and no weight decay. We use a batch size of $128$. The training and test steps are conducted for $6000$ epochs on the \textit{Water Potability} and \textit{Phoneme} datasets, while for the \textit{CIFAR-10} dataset, they are performed for $200$ epochs.
Finally, the contribution $w_i^{\varepsilon}$ of each training point $\inst_i$ is uniformly set as $w_i^{\varepsilon} = 1~\forall i\in \{1,\ldots,m\}$.

The source code implementation for our experiments is available at the following GitHub repository: \url{https://anonymous.4open.science/r/COCE-80B4/README.md} 

\subsection{RQ1: Counterfactual Perturbation vs. Overfitting}
To address \textbf{RQ1}, we analyze the relationship between the test loss and the average $L_2$-norm of the counterfactual perturbation vectors ($\overline{||\perturb||}$) over training epochs.

In particular, Figure~\ref{fig:delta_loss_epochs} depicts the evolution of $\overline{||\perturb||}$ alongside the test loss for an MLP trained \textit{without} regularization on the \textit{Water Potability} dataset. 
\begin{figure}[ht]
    \centering
    \includegraphics[width=0.85\linewidth]{img/delta_loss_epochs.png}
    \caption{The average counterfactual perturbation vector $\overline{||\perturb||}$ (left $y$-axis) and the cross-entropy test loss (right $y$-axis) over training epochs ($x$-axis) for an MLP trained on the \textit{Water Potability} dataset \textit{without} regularization.}
    \label{fig:delta_loss_epochs}
\end{figure}

The plot shows a clear trend as the model starts to overfit the data (evidenced by an increase in test loss). 
Notably, $\overline{||\perturb||}$ begins to decrease, which aligns with the hypothesis that the average distance to the optimal counterfactual example gets smaller as the model's decision boundary becomes increasingly adherent to the training data.

It is worth noting that this trend is heavily influenced by the choice of the counterfactual generator model. In particular, the relationship between $\overline{||\perturb||}$ and the degree of overfitting may become even more pronounced when leveraging more accurate counterfactual generators. However, these models often come at the cost of higher computational complexity, and their exploration is left to future work.

Nonetheless, we expect that $\overline{||\perturb||}$ will eventually stabilize at a plateau, as the average $L_2$-norm of the optimal counterfactual perturbations cannot vanish to zero.

% Additionally, the choice of employing the score-based counterfactual explanation framework to generate counterfactuals was driven to promote computational efficiency.

% Future enhancements to the framework may involve adopting models capable of generating more precise counterfactuals. While such approaches may yield to performance improvements, they are likely to come at the cost of increased computational complexity.


\subsection{RQ2: Counterfactual Regularization Performance}
To answer \textbf{RQ2}, we evaluate the effectiveness of the proposed counterfactual regularization (CF-Reg) by comparing its performance against existing baselines: unregularized training loss (No-Reg), L1 regularization (L1-Reg), L2 regularization (L2-Reg), and Dropout.
Specifically, for each model and dataset combination, Table~\ref{tab:regularization_comparison} presents the mean value and standard deviation of test accuracy achieved by each method across 5 random initialization. 

The table illustrates that our regularization technique consistently delivers better results than existing methods across all evaluated scenarios, except for one case -- i.e., Logistic Regression on the \textit{Phomene} dataset. 
However, this setting exhibits an unusual pattern, as the highest model accuracy is achieved without any regularization. Even in this case, CF-Reg still surpasses other regularization baselines.

From the results above, we derive the following key insights. First, CF-Reg proves to be effective across various model types, ranging from simple linear models (Logistic Regression) to deep architectures like MLPs and CNNs, and across diverse datasets, including both tabular and image data. 
Second, CF-Reg's strong performance on the \textit{Water} dataset with Logistic Regression suggests that its benefits may be more pronounced when applied to simpler models. However, the unexpected outcome on the \textit{Phoneme} dataset calls for further investigation into this phenomenon.


\begin{table*}[h!]
    \centering
    \caption{Mean value and standard deviation of test accuracy across 5 random initializations for different model, dataset, and regularization method. The best results are highlighted in \textbf{bold}.}
    \label{tab:regularization_comparison}
    \begin{tabular}{|c|c|c|c|c|c|c|}
        \hline
        \textbf{Model} & \textbf{Dataset} & \textbf{No-Reg} & \textbf{L1-Reg} & \textbf{L2-Reg} & \textbf{Dropout} & \textbf{CF-Reg (ours)} \\ \hline
        Logistic Regression   & \textit{Water}   & $0.6595 \pm 0.0038$   & $0.6729 \pm 0.0056$   & $0.6756 \pm 0.0046$  & N/A    & $\mathbf{0.6918 \pm 0.0036}$                     \\ \hline
        MLP   & \textit{Water}   & $0.6756 \pm 0.0042$   & $0.6790 \pm 0.0058$   & $0.6790 \pm 0.0023$  & $0.6750 \pm 0.0036$    & $\mathbf{0.6802 \pm 0.0046}$                    \\ \hline
%        MLP   & \textit{Adult}   & $0.8404 \pm 0.0010$   & $\mathbf{0.8495 \pm 0.0007}$   & $0.8489 \pm 0.0014$  & $\mathbf{0.8495 \pm 0.0016}$     & $0.8449 \pm 0.0019$                    \\ \hline
        Logistic Regression   & \textit{Phomene}   & $\mathbf{0.8148 \pm 0.0020}$   & $0.8041 \pm 0.0028$   & $0.7835 \pm 0.0176$  & N/A    & $0.8098 \pm 0.0055$                     \\ \hline
        MLP   & \textit{Phomene}   & $0.8677 \pm 0.0033$   & $0.8374 \pm 0.0080$   & $0.8673 \pm 0.0045$  & $0.8672 \pm 0.0042$     & $\mathbf{0.8718 \pm 0.0040}$                    \\ \hline
        CNN   & \textit{CIFAR-10} & $0.6670 \pm 0.0233$   & $0.6229 \pm 0.0850$   & $0.7348 \pm 0.0365$   & N/A    & $\mathbf{0.7427 \pm 0.0571}$                     \\ \hline
    \end{tabular}
\end{table*}

\begin{table*}[htb!]
    \centering
    \caption{Hyperparameter configurations utilized for the generation of Table \ref{tab:regularization_comparison}. For our regularization the hyperparameters are reported as $\mathbf{\alpha/\beta}$.}
    \label{tab:performance_parameters}
    \begin{tabular}{|c|c|c|c|c|c|c|}
        \hline
        \textbf{Model} & \textbf{Dataset} & \textbf{No-Reg} & \textbf{L1-Reg} & \textbf{L2-Reg} & \textbf{Dropout} & \textbf{CF-Reg (ours)} \\ \hline
        Logistic Regression   & \textit{Water}   & N/A   & $0.0093$   & $0.6927$  & N/A    & $0.3791/1.0355$                     \\ \hline
        MLP   & \textit{Water}   & N/A   & $0.0007$   & $0.0022$  & $0.0002$    & $0.2567/1.9775$                    \\ \hline
        Logistic Regression   &
        \textit{Phomene}   & N/A   & $0.0097$   & $0.7979$  & N/A    & $0.0571/1.8516$                     \\ \hline
        MLP   & \textit{Phomene}   & N/A   & $0.0007$   & $4.24\cdot10^{-5}$  & $0.0015$    & $0.0516/2.2700$                    \\ \hline
       % MLP   & \textit{Adult}   & N/A   & $0.0018$   & $0.0018$  & $0.0601$     & $0.0764/2.2068$                    \\ \hline
        CNN   & \textit{CIFAR-10} & N/A   & $0.0050$   & $0.0864$ & N/A    & $0.3018/
        2.1502$                     \\ \hline
    \end{tabular}
\end{table*}

\begin{table*}[htb!]
    \centering
    \caption{Mean value and standard deviation of training time across 5 different runs. The reported time (in seconds) corresponds to the generation of each entry in Table \ref{tab:regularization_comparison}. Times are }
    \label{tab:times}
    \begin{tabular}{|c|c|c|c|c|c|c|}
        \hline
        \textbf{Model} & \textbf{Dataset} & \textbf{No-Reg} & \textbf{L1-Reg} & \textbf{L2-Reg} & \textbf{Dropout} & \textbf{CF-Reg (ours)} \\ \hline
        Logistic Regression   & \textit{Water}   & $222.98 \pm 1.07$   & $239.94 \pm 2.59$   & $241.60 \pm 1.88$  & N/A    & $251.50 \pm 1.93$                     \\ \hline
        MLP   & \textit{Water}   & $225.71 \pm 3.85$   & $250.13 \pm 4.44$   & $255.78 \pm 2.38$  & $237.83 \pm 3.45$    & $266.48 \pm 3.46$                    \\ \hline
        Logistic Regression   & \textit{Phomene}   & $266.39 \pm 0.82$ & $367.52 \pm 6.85$   & $361.69 \pm 4.04$  & N/A   & $310.48 \pm 0.76$                    \\ \hline
        MLP   &
        \textit{Phomene} & $335.62 \pm 1.77$   & $390.86 \pm 2.11$   & $393.96 \pm 1.95$ & $363.51 \pm 5.07$    & $403.14 \pm 1.92$                     \\ \hline
       % MLP   & \textit{Adult}   & N/A   & $0.0018$   & $0.0018$  & $0.0601$     & $0.0764/2.2068$                    \\ \hline
        CNN   & \textit{CIFAR-10} & $370.09 \pm 0.18$   & $395.71 \pm 0.55$   & $401.38 \pm 0.16$ & N/A    & $1287.8 \pm 0.26$                     \\ \hline
    \end{tabular}
\end{table*}

\subsection{Feasibility of our Method}
A crucial requirement for any regularization technique is that it should impose minimal impact on the overall training process.
In this respect, CF-Reg introduces an overhead that depends on the time required to find the optimal counterfactual example for each training instance. 
As such, the more sophisticated the counterfactual generator model probed during training the higher would be the time required. However, a more advanced counterfactual generator might provide a more effective regularization. We discuss this trade-off in more details in Section~\ref{sec:discussion}.

Table~\ref{tab:times} presents the average training time ($\pm$ standard deviation) for each model and dataset combination listed in Table~\ref{tab:regularization_comparison}.
We can observe that the higher accuracy achieved by CF-Reg using the score-based counterfactual generator comes with only minimal overhead. However, when applied to deep neural networks with many hidden layers, such as \textit{PreactResNet-18}, the forward derivative computation required for the linearization of the network introduces a more noticeable computational cost, explaining the longer training times in the table.

\subsection{Hyperparameter Sensitivity Analysis}
The proposed counterfactual regularization technique relies on two key hyperparameters: $\alpha$ and $\beta$. The former is intrinsic to the loss formulation defined in (\ref{eq:cf-train}), while the latter is closely tied to the choice of the score-based counterfactual explanation method used.

Figure~\ref{fig:test_alpha_beta} illustrates how the test accuracy of an MLP trained on the \textit{Water Potability} dataset changes for different combinations of $\alpha$ and $\beta$.

\begin{figure}[ht]
    \centering
    \includegraphics[width=0.85\linewidth]{img/test_acc_alpha_beta.png}
    \caption{The test accuracy of an MLP trained on the \textit{Water Potability} dataset, evaluated while varying the weight of our counterfactual regularizer ($\alpha$) for different values of $\beta$.}
    \label{fig:test_alpha_beta}
\end{figure}

We observe that, for a fixed $\beta$, increasing the weight of our counterfactual regularizer ($\alpha$) can slightly improve test accuracy until a sudden drop is noticed for $\alpha > 0.1$.
This behavior was expected, as the impact of our penalty, like any regularization term, can be disruptive if not properly controlled.

Moreover, this finding further demonstrates that our regularization method, CF-Reg, is inherently data-driven. Therefore, it requires specific fine-tuning based on the combination of the model and dataset at hand.
\newcommand{\cods}[1]{\textcolor[RGB]{51,68,103}{#1}}
\newcommand{\codt}[1]{\textcolor[RGB]{156,77,93}{#1}}

\newcommand{\e}[2]{{#1}}
\newcommand{\tl}[2]{\begin{tabular}[c]{@{}c@{}} \cods{#1}\\\codt{#2}\end{tabular}}
\newcommand{\tln}[2]{\begin{tabular}[c]{@{}c@{}} #1\\#2\end{tabular}}

\newcommand{\tc}[2]{\cods{#1}&\codt{#2}}

\newcommand{\tlds}[2]{\begin{tabular}[c]{@{}c@{}} \cods{#1}\\\cods{(#2)}\end{tabular}}
\newcommand{\tldt}[2]{\begin{tabular}[c]{@{}c@{}} \codt{#1}\\\codt{(#2)}\end{tabular}}

\newcommand{\bestoo}[0]{\cellcolor[HTML]{FFEEEB}}
\newcommand{\besto}[0]{\cellcolor[HTML]{E7F2F5}}
\newcommand{\bestt}[0]{\cellcolor[HTML]{faf1d8}}


\setlength{\tabcolsep}{3pt}
\setlength{\fboxsep}{0pt}

\begin{table*}[ht!]
    % \scriptsize
    \tiny
    % \small
    \centering
    \begin{tabular}{c|cc|cc|cc|cc|cc|cc|cc|cc|cc||cc}
      \toprule
      & 
      \multicolumn{2}{c|}{\textbf{Digits}} &
      \multicolumn{2}{c|}{\textbf{RMNIST}} &
      \multicolumn{2}{c|}{\textbf{CIFAR \& STL}} &
      \multicolumn{2}{c|}{\textbf{VisDA}} &
      \multicolumn{2}{c|}{\textbf{Office-Home}} &
      \multicolumn{2}{c|}{\textbf{DomainNet}} &
      \multicolumn{2}{c|}{\textbf{VLCS}} &
      \multicolumn{2}{c|}{\textbf{PCAS}} &
      \multicolumn{2}{c||}{\textbf{TerraInc}} &
      \multicolumn{2}{c}{\textbf{Avg.}} 
      \\

      \cmidrule(lr){2-21}  

      &
      \textbf{SA} $\uparrow$ & \textbf{TA} $\downarrow$ & 
      \textbf{SA} $\uparrow$ & \textbf{TA} $\downarrow$ & 
      \textbf{SA} $\uparrow$ & \textbf{TA} $\downarrow$ & 
      \textbf{SA} $\uparrow$ & \textbf{TA} $\downarrow$ & 
      \textbf{SA} $\uparrow$ & \textbf{TA} $\downarrow$ & 
      \textbf{SA} $\uparrow$ & \textbf{TA} $\downarrow$ & 
      \textbf{SA} $\uparrow$ & \textbf{TA} $\downarrow$ & 
      \textbf{SA} $\uparrow$ & \textbf{TA} $\downarrow$ & 
      \textbf{SA} $\uparrow$ & \textbf{TA} $\downarrow$ & 
      \textbf{SA} $\uparrow$ & \textbf{TA} $\downarrow$ 
      \\
        
      \midrule
      \midrule
      SL & 
      \tc{\e{97.7}{?}}{\e{56.0}{?}} &
      \tc{\e{99.2}{?}}{\e{62.4}{?}} &
      \tc{\e{88.2}{?}}{\e{65.7}{?}} &
      \tc{\e{86.8}{?}}{\e{37.7}{?}} &
      \tc{\e{66.4}{?}}{\e{36.9}{?}} &
      \tc{\e{45.6}{?}}{\e{ 9.9}{?}} &
      \tc{\e{79.9}{?}}{\e{56.9}{?}} &
      \tc{\e{89.5}{?}}{\e{47.3}{?}} &
      \tc{\e{93.6}{?}}{\e{14.9}{?}} &
      \tc{\e{83.0}{?}}{\e{43.0}{?}} \\

      \cmidrule(lr){1-21}
      \tln{NTL}{\cite{wang2021non}} & 
      \tc{\tlds{95.6}{-2.1}}{\tldt{12.2}{-43.8}} &
      \tc{\tlds{98.7}{-0.5}}{\tldt{12.3}{-50.1}} &
      \tc{\tlds{83.9}{-4.4}}{\tldt{ 9.9}{-55.8}} &
      \tc{\tlds{82.0}{-4.8}}{\tldt{10.9}{-26.8}} &
      \tc{\tlds{64.8}{-1.6}}{\tldt{32.4}{-4.5}} &
      \tc{\tlds{ 7.6}{-38.0}}{\tldt{ 1.4}{-8.6}} &
      \tc{\tlds{78.0}{-1.9}}{\tldt{27.1}{-29.8}} &
      \tc{\tlds{85.8}{-3.7}}{\tldt{18.0}{-29.2}} &
      \tc{\tlds{90.0}{-3.6}}{\tldt{ 8.8}{-6.1}} &
      \tc{\tlds{76.3}{-6.7}}{\tldt{14.8}{-28.3}} \\

      \cmidrule(lr){2-21}
      \tln{CUTI-domain}{\cite{wang2023model}} & 
      \tc{\tlds{97.0}{-0.8}}{\tldt{ 9.5}{-46.5}} &
      \tc{\tlds{99.2}{-0.1}}{\tldt{15.5}{-46.9}} &
      \tc{\tlds{85.1}{-3.2}}{\tldt{10.7}{-55.0}} &
      \tc{\tlds{85.3}{-1.5}}{\tldt{ 8.9}{-28.8}} &
      \tc{\besto\tlds{56.7}{-9.7}}{\besto\tldt{17.8}{-19.1}} &
      \tc{\tlds{14.0}{-31.7}}{\tldt{ 2.0}{-7.9}} &
      \tc{\besto\tlds{78.3}{-1.6}}{\besto\tldt{26.7}{-30.1}} &
      \tc{\tlds{88.4}{-1.1}}{\tldt{18.3}{-28.9}} &
      \tc{\besto\tlds{87.9}{-5.7}}{\besto\tldt{ 0.8}{-14.1}} &
      \tc{\besto\tlds{76.9}{-6.1}}{\besto\tldt{12.2}{-30.8}} \\

      \cmidrule(lr){2-21}
      \tln{H-NTL}{\cite{hong2024improving}} & 
      \tc{\tlds{97.5}{-0.2}}{\tldt{ 9.6}{-46.4}} &
      \tc{\besto\tlds{99.0}{-0.2}}{\besto\tldt{10.8}{-51.5}} &
      \tc{\besto\tlds{87.2}{-1.0}}{\besto\tldt{ 9.9}{-55.8}} &
      \tc{\besto\tlds{86.5}{-0.3}}{\besto\tldt{ 8.6}{-29.0}} &
      \tc{\tlds{51.1}{-15.2}}{\tldt{17.0}{-19.8}} &
      \tc{\besto\tlds{33.3}{-12.3}}{\besto\tldt{ 2.1}{-7.8}} &
      \tc{\tlds{79.2}{-0.8}}{\tldt{42.7}{-14.2}} &
      \tc{\tlds{89.1}{-0.3}}{\tldt{22.1}{-25.1}} &
      \tc{\tlds{88.4}{-5.2}}{\tldt{14.6}{-0.2}} &
      \tc{\tlds{79.0}{-4.0}}{\tldt{15.3}{-27.8}} \\

      \cmidrule(lr){2-21}
      \tln{SOPHON}{\cite{deng2024sophon}} & 
      \tc{\tlds{95.2}{-2.5}}{\tldt{ 9.9}{-46.1}} &
      \tc{\tlds{96.6}{-2.6}}{\tldt{38.8}{-23.6}} &
      \tc{\tlds{69.5}{-18.7}}{\tldt{24.8}{-40.9}} &
      \tc{\tlds{77.3}{-9.5}}{\tldt{10.9}{-26.8}} &
      \tc{\tlds{45.9}{-20.4}}{\tldt{17.6}{-19.3}} &
      \tc{\tlds{30.1}{-15.6}}{\tldt{ 2.5}{-7.4}} &
      \tc{\tlds{79.4}{-0.6}}{\tldt{29.5}{-27.4}} &
      \tc{\tlds{86.7}{-2.8}}{\tldt{21.6}{-25.7}} &
      \tc{\tlds{88.8}{-4.8}}{\tldt{ 7.1}{-7.7}} &
      \tc{\tlds{74.4}{-8.6}}{\tldt{18.1}{-25.0}} \\

      \cmidrule(lr){2-21}
      \tln{CUPI-domain}{\cite{wang2024say}} &  
      \tc{\besto\tlds{96.7}{-1.0}}{\besto\tldt{ 8.8}{-47.2}} &
      \tc{\tlds{98.8}{-0.4}}{\tldt{21.0}{-41.3}} &
      \tc{\tlds{86.0}{-2.3}}{\tldt{11.3}{-54.4}} &
      \tc{\tlds{84.6}{-2.2}}{\tldt{ 8.2}{-29.5}} &
      \tc{\tlds{11.6}{-54.7}}{\tldt{ 2.3}{-34.6}} &
      \tc{\tlds{ 0.8}{-44.9}}{\tldt{ 0.3}{-9.7}} &
      \tc{\tlds{77.5}{-2.5}}{\tldt{29.5}{-27.4}} &
      \tc{\besto\tlds{87.8}{-1.7}}{\besto\tldt{11.5}{-35.8}} &
      \tc{\tlds{82.4}{-11.1}}{\tldt{ 1.3}{-13.6}} &
      \tc{\tlds{69.6}{-13.4}}{\tldt{10.4}{-32.6}} \\

      \bottomrule
    \end{tabular}
    \vspace{-3mm}
    \caption{Comparison of SL and 5 NTL methods on multiple datasets. We report the \cods{source-domain accuracy} (\textbf{SA}) (\%) in \cods{blue} and \codt{target-domain accuracy} (\textbf{TA}) (\%) in \codt{red}. The best results of overall performance (OA) are highlighted in \colorbox[HTML]{E7F2F5}{blue background}. The accuracy drop compared to the pre-trained model is shown in brackets. The average performance of 9 datasets are shown in the last column (\textbf{Avg.}).}
    \label{tab:tgt-spec}
    \vspace{-4mm}
  \end{table*}
  
  
  


% \vspace{-1mm}
\section{Benchmarking NTL}
\label{sec:exp}

The post-training robustness has not been well-evaluated in NTL, which motivates us to build a comprehensive benchmark.
In this section, we first demonstrate the framework of our \texttt{NTLBench} (\Cref{sec:ntlbench}). Then, we present main results by conducting our \texttt{NTLBench} (\Cref{sec:ntlbenchresults}), including the pretrained NTL performance on multiple datasets, and the robustness of NTL baselines against different attacks. 


\subsection{\texttt{NTLBench}}
\label{sec:ntlbench}

We propose the first NTL benchmark (\texttt{NTLBench}), which contains a standard and unified training and evaluation process. \texttt{NTLBench} supports 5 SOTA NTL methods, 9 datasets (more than 116 domain pairs), 5 network architectures families, and 15 post-training attacks from 3 attack settings, providing more than 40,000 experimental configurations. 

\paragraph{Datasets.} Our \texttt{NTLBench} is compatible with: Digits (5 domains)~\cite{deng2012mnist,hull1994database,netzer2011reading,ganin2016domain,roy2018effects}, RotatedMNIST (3 domains)~\cite{ghifary2015domain}, CIFAR and STL (2 domains)~\cite{krizhevsky2009learning,coates2011analysis}, VisDA (2 domains)~\cite{peng2017visda}, Office-Home (4 domains)~\cite{venkateswara2017deep}, DomainNet (6 domains)~\cite{peng2019moment}, VLCS (4 domains)~\cite{fang2013unbiased}, PCAS (4 domains)~\cite{li2017deeper}, and TerraInc (5 domains)~\cite{beery2018recognition}. Different domains in any dataset share the same label space, but have distribution shifts, thus being suitable for evaluating NTL methods.

\paragraph{NTL baselines.}
\texttt{NTLBench} involves all open-source NTL methods: NTL~\cite{wang2021non}, CUTI-domain~\cite{wang2023model}, H-NTL~\cite{hong2024improving}, SOPHON~\cite{deng2024sophon}, CUPI-domain~\cite{wang2024say}. Besides, we also add a vanilla supervised learning (SL) as a reference.

\paragraph{Network architecture.}
The proposed \texttt{NTLBench} is compatible with multiple network architectures, including but not limited to: 
VGG~\cite{simonyan2014very}, ResNet~\cite{he2016deep}, WideResNet~\cite{zagoruyko2016wide}, ViT~\cite{dosovitskiy2020image}, SwinT~\cite{liu2021swin}.

\paragraph{Threat I: source domain fine-tuning (SourceFT).} \textit{Attacking goal}: SourceFT tries to destroy the non-transferability by fine-tuning the NTL model using a small set of source domain data. \textit{Attacking method}: \texttt{NTLBench} involves 5 methods, including four basic fine-tuning strategies\footnote{\label{initfc}initFC: re-initialize the last full-connect (FC) layer. direct: no re-initialize. all: fine-tune the whole model. FC: fine-tune last FC.}: initFC-all, initFC-FC, direct-FC, direct-all~\cite{deng2024sophon} and the special designed attack for NTL: TransNTL~\cite{hong2024your}. 


\paragraph{Threat II: target domain fine-tuning (TargetFT).} \textit{Attacking goal}: TargetFT tries to directly use labeled target domain data to fine-tune the NTL model, thus recovering target domain performance. \textit{Attacking method}: \texttt{NTLBench} use 4 basic fine-tuning strategies\textsuperscript{\ref{initfc}} leveraged in~\cite{deng2024sophon} as attack methods: initFC-all, initFC-FC, direct-FC, direct-all.

\paragraph{Threat III: source-free domain adaptation (SFDA).} \textit{Attacking goal}: We introduce SFDA to test whether using unlabeled target domain data poses a threat to NTL. \textit{Attacking method}: \texttt{NTLBench} involves 6 SOTA SFDA methods: SHOT~\cite{liang2020we}, CoWA~\cite{lee2022confidence}, NRC~\cite{yang2021exploiting}, PLUE~\cite{litrico2023guiding}, AdaContrast~\cite{chen2022contrastive}, and DIFO~\cite{tang2024source}.

\paragraph{Evaluation metric.} For source domain, we use source domain accuracy (\textbf{SA}) to evaluate the performance. Higher SA means lower influence of non-transferability to the source domain utility.
For target domain, we use target domain accuracy (\textbf{TA}) to evaluate the performance. Lower TA means better performance of non-transferability.
Besides, we calculate the overall performance (denoted as \textbf{OA}) of an NTL method as: $\text{OA}=(\text{SA}+(100\%-\text{TA}))/2$, with higher OA representing better overall performance of an NTL method. These evaluation metrics are applicable for both non-transferability performance and robustness against different attacks.


\subsection{Main Results and Analysis.}
\label{sec:ntlbenchresults}

Due to the limited space, we present main results obtained from our \texttt{NTLBench}. 
We first show the key implementation details, and then we present and analyse of our results.

\paragraph{Implementation details.}
Briefly, in pre-training stage, we sequentially pair $i$-th and ($i$+1)-th domains within a dataset for training. Each domain is randomly split into 8:1:1 for training, validation, and testing. The results for each dataset are averaged across domain pairs. NTL methods and the reference SL method are pretrained by up to 50 epochs. We search suitable hyper-parameters for each method by setting 5 values around their original value and choose the best value according to the best OA on validation set. All the batch size, learning rate, and optimizer are follow their original implementations. Following the original NTL paper~\cite{wang2021non}, we use VGG-13 without batch-normalization. All input images are resize to 64$\times$64. 
In attack stage, we use 10\% amount of the training set to perform attack. All attack results we reported are run on CIFAR \& STL. Attack training is up to 50 epochs.
We run all experiments on RTX 4090 (24G).


\paragraph{Non-transferability performance.} The non-transferability performance are shown in \Cref{tab:tgt-spec}, where we compare 5 NTL methods and SL on 9 datasets. From the results, all NTL methods generally effectively degrade source-to-target generalization, leading to a significant drop in TA compared to SL. However, in more complex datasets such as Office-Home and DomainNet, existing NTL methods fail to achieve a satisfactory balance between maintaining SA and degrading TA, highlighting their limitations. From the \textbf{Avg.} column, CUTI-domain reaches the overall best performance.
% \label{sec:ntlbench1}

\paragraph{Post-training robustness.} 
For \textbf{SourceFT} attack (\Cref{tab:atk_src}), fine-tuning each NTL model by using basic fine-tuning strategies on 10\% source domain data cannot directly recover the source-to-target generalization. However, all NTL methods are fragile when facing the TransNTL attack. For \textbf{TargetFT} attack (\Cref{tab:atk_tgt_label}), all NTL methods cannot fully resist supervised fine-tuning attack by using target domain data. In particular, fine-tuning all parameters usually results in better attack effectiveness. For \textbf{SFDA} (\Cref{tab:atk_tgt_sfda}), although the target domain data are unlabeled, advanced source-free unsupervised domain adaptation, leveraging self-supervised strategies, can still partially recover target domain performance. All these results verify the fragility of existing NTL methods. 




\begin{table}[t!]
    % \scriptsize
    \tiny
    % \small
    \centering
    % \hspace{-2mm}
    \begin{tabular}{@{\hspace{4pt}}c@{\hspace{3pt}}|c@{\hspace{2pt}}c@{\hspace{3pt}}|@{\hspace{3pt}}c@{\hspace{2pt}}c@{\hspace{3pt}}|@{\hspace{3pt}}c@{\hspace{3pt}}c@{\hspace{3pt}}|@{\hspace{3pt}}c@{\hspace{3pt}}c@{\hspace{3pt}}|@{\hspace{3pt}}c@{\hspace{3pt}}c@{\hspace{3pt}}}

        
    \toprule
  
      & 
      \multicolumn{2}{c|@{\hspace{3pt}}}{\textbf{NTL}} &
      \multicolumn{2}{c|@{\hspace{3pt}}}{\textbf{CUTI}} &
      \multicolumn{2}{c|@{\hspace{3pt}}}{\textbf{H-NTL}} &
      \multicolumn{2}{c|@{\hspace{3pt}}}{\textbf{SOPHON}} &
      \multicolumn{2}{@{\hspace{3pt}}c}{\textbf{CUPI}}
      \\
  
      \cmidrule(lr){2-11}
  
      &
      \textbf{SA} $\uparrow$ & \textbf{TA} $\downarrow$ & 
      \textbf{SA} $\uparrow$ & \textbf{TA} $\downarrow$ & 
      \textbf{SA} $\uparrow$ & \textbf{TA} $\downarrow$ & 
      \textbf{SA} $\uparrow$ & \textbf{TA} $\downarrow$ & 
      \textbf{SA} $\uparrow$ & \textbf{TA} $\downarrow$ 
      
      \\
      \midrule
      \midrule
      Pre-train & 
      \tc{\e{83.9}{?}}{\e{ 9.9}{?}} &
      \tc{\e{85.1}{?}}{\e{10.6}{?}} &
      \tc{\e{87.2}{?}}{\e{ 9.9}{?}} &
      \tc{\e{69.5}{?}}{\e{24.8}{?}} &
      \tc{\e{86.0}{?}}{\e{11.3}{?}} \\
      \cmidrule(lr){1-11}
  
      initFC-all & 
      \tc{\tlds{84.0}{+0.2}}{\tldt{ 9.8}{-0.1}} &
      \tc{\tlds{84.2}{-0.9}}{\tldt{10.6}{+0.0}} &
      \tc{\tlds{87.8}{+0.6}}{\tldt{16.2}{+6.3}} &
      \tc{\tlds{82.2}{+12.7}}{\tldt{38.1}{+13.3}} &
      \tc{\tlds{85.3}{-0.7}}{\tldt{11.4}{+0.1}} \\
      \cmidrule(lr){2-11}
  
      initFC-FC & 
      \tc{\tlds{84.2}{+0.3}}{\tldt{10.0}{+0.1}} &
      \tc{\tlds{85.4}{+0.3}}{\tldt{10.6}{+0.0}} &
      \tc{\tlds{87.2}{-0.1}}{\tldt{10.2}{+0.3}} &
      \tc{\tlds{71.9}{+2.4}}{\tldt{23.3}{-1.6}} &
      \tc{\tlds{85.9}{-0.1}}{\tldt{11.3}{+0.0}} \\
      \cmidrule(lr){2-11}
  
      direct-FC & 
      \tc{\tlds{84.0}{+0.2}}{\tldt{ 9.9}{+0.0}} &
      \tc{\tlds{85.2}{+0.2}}{\tldt{10.6}{+0.0}} &
      \tc{\tlds{87.3}{+0.1}}{\tldt{ 9.9}{+0.0}} &
      \tc{\tlds{74.3}{+4.8}}{\tldt{23.8}{-1.1}} &
      \tc{\tlds{86.1}{+0.1}}{\tldt{11.3}{+0.0}} \\
      \cmidrule(lr){2-11}
  
      direct-all & 
      \tc{\tlds{84.7}{+0.8}}{\tldt{ 9.8}{-0.1}} &
      \tc{\tlds{85.3}{+0.3}}{\tldt{10.9}{+0.3}} &
      \tc{\tlds{88.0}{+0.8}}{\tldt{10.1}{+0.2}} &
      \tc{\tlds{83.4}{+13.9}}{\tldt{32.2}{+7.4}} &
      \tc{\tlds{85.5}{-0.5}}{\tldt{11.3}{+0.0}} \\
      \cmidrule(lr){2-11}
  
      TransNTL & 
      \tc{\bestoo \tlds{81.7}{-2.2}}{\bestoo \tldt{44.3}{+34.4}} &
      \tc{\bestoo \tlds{81.3}{-3.8}}{\bestoo \tldt{61.0}{+50.3}} &
      \tc{\bestoo \tlds{86.3}{-1.0}}{\bestoo \tldt{63.7}{+53.8}} &
      \tc{\bestoo \tlds{83.8}{+14.3}}{\bestoo \tldt{60.1}{+35.3}} &
      \tc{\bestoo \tlds{83.1}{-2.9}}{\bestoo \tldt{60.6}{+49.3}} \\
  
      \bottomrule
    \end{tabular}
    \vspace{-3mm}
    \caption{NTL robustness against source domain fine-tuning (Source-\\FT). We show \cods{source-domain accuracy} (\textbf{SA}) (\%) and \codt{target-domain accuracy} (\textbf{TA}) (\%). The most serious threat (worst OA) to each NTL is marked as\colorbox[HTML]{fee8e4}{ red.} Accuracy drop from the pre-trained model is in ($\cdot$).}
    \label{tab:atk_src}
    \vspace{1mm}
    \begin{tabular}{@{\hspace{4pt}}c@{\hspace{3pt}}|c@{\hspace{2pt}}c@{\hspace{3pt}}|@{\hspace{3pt}}c@{\hspace{2pt}}c@{\hspace{3pt}}|@{\hspace{3pt}}c@{\hspace{3pt}}c@{\hspace{3pt}}|@{\hspace{3pt}}c@{\hspace{3pt}}c@{\hspace{3pt}}|@{\hspace{3pt}}c@{\hspace{3pt}}c@{\hspace{3pt}}}
      \toprule
    
        & 
        \multicolumn{2}{c|@{\hspace{3pt}}}{\textbf{NTL}} &
        \multicolumn{2}{c|@{\hspace{3pt}}}{\textbf{CUTI}} &
        \multicolumn{2}{c|@{\hspace{3pt}}}{\textbf{H-NTL}} &
        \multicolumn{2}{c|@{\hspace{3pt}}}{\textbf{SOPHON}} &
        \multicolumn{2}{@{\hspace{3pt}}c}{\textbf{CUPI}}
        \\
    
        \cmidrule(lr){2-11}
    
        &
        \textbf{SA} $\uparrow$ & \textbf{TA} $\downarrow$ & 
        \textbf{SA} $\uparrow$ & \textbf{TA} $\downarrow$ & 
        \textbf{SA} $\uparrow$ & \textbf{TA} $\downarrow$ & 
        \textbf{SA} $\uparrow$ & \textbf{TA} $\downarrow$ & 
        \textbf{SA} $\uparrow$ & \textbf{TA} $\downarrow$ 
        
        \\
        \midrule
        \midrule
        Pre-train & 
        \tc{\e{83.9}{?}}{\e{ 9.9}{?}} &
        \tc{\e{85.1}{?}}{\e{10.7}{?}} &
        \tc{\e{87.2}{?}}{\e{ 9.9}{?}} &
        \tc{\e{69.5}{?}}{\e{24.8}{?}} &
        \tc{\e{86.0}{?}}{\e{11.3}{?}} \\
        \cmidrule(lr){1-11}
    
        initFC-all & 
        \tc{\bestoo \tlds{23.9}{-60.0}}{\bestoo \tldt{37.8}{+27.9}} &
        \tc{\bestoo \tlds{13.3}{-71.8}}{\bestoo \tldt{15.9}{+5.3}} &
        \tc{\tlds{19.0}{-68.3}}{\tldt{10.4}{+0.5}} &
        \tc{\tlds{59.0}{-10.5}}{\tldt{68.5}{+43.7}} &
        \tc{\tlds{41.2}{-44.8}}{\tldt{53.1}{+41.8}} \\
        \cmidrule(lr){2-11}
    
        initFC-FC & 
        \tc{\tlds{33.9}{-50.0}}{\tldt{ 9.6}{-0.4}} &
        \tc{\tlds{30.2}{-54.9}}{\tldt{ 9.7}{-1.0}} &
        \tc{\tlds{19.1}{-68.1}}{\tldt{ 9.7}{-0.2}} &
        \tc{\tlds{21.6}{-48.0}}{\tldt{16.8}{-8.1}} &
        \tc{\tlds{21.8}{-64.2}}{\tldt{12.1}{+0.8}} \\
        \cmidrule(lr){2-11}
    
        direct-FC & 
        \tc{\tlds{64.2}{-19.7}}{\tldt{10.2}{+0.3}} &
        \tc{\tlds{38.0}{-47.1}}{\tldt{10.6}{-0.1}} &
        \tc{\tlds{87.1}{-0.1}}{\tldt{10.0}{+0.1}} &
        \tc{\tlds{70.5}{+1.0}}{\tldt{24.5}{-0.4}} &
        \tc{\tlds{78.6}{-7.4}}{\tldt{11.0}{-0.4}} \\
        \cmidrule(lr){2-11}
    
        direct-all & 
        \tc{\tlds{13.9}{-70.0}}{\tldt{17.6}{+7.7}} &
        \tc{\tlds{10.1}{-75.0}}{\tldt{ 8.8}{-1.9}} &
        \tc{\bestoo \tlds{84.7}{-2.5}}{\bestoo \tldt{53.3}{+43.4}} &
        \tc{\bestoo \tlds{68.0}{-1.6}}{\bestoo \tldt{72.9}{+48.1}} &
        \tc{\bestoo \tlds{51.9}{-34.1}}{\bestoo \tldt{58.4}{+47.1}} \\
    
        \bottomrule
      \end{tabular}
      \vspace{-3mm}
      \caption{NTL robustness against target domain fine-tuning (Target-\\FT).  We report \cods{source-domain accuracy} (\textbf{SA}) (\%) and \codt{target-domain accuracy} (\textbf{TA}) (\%). The most serious threat (best TA) to each NTL is marked as\colorbox[HTML]{fee8e4}{ red.} Accuracy drop from the pre-trained model is in ($\cdot$).}
      \vspace{-3mm}
      \label{tab:atk_tgt_label}
  \end{table}
  
\paragraph{More results.} 

Additional results and analysis on: various architectures, attack using different data amount, cross-domain/task, and visualizations (e.g., feature activation, t-SNE \cite{van2008visualizing}, GradCAM \cite{selvaraju2017grad}) will be released soon at our online page.
  
  
  \begin{table}[t!]
    % \scriptsize
    \tiny
    % \small
    \centering
    % \hspace{-2mm}
    \begin{tabular}{@{\hspace{4pt}}c@{\hspace{3pt}}|c@{\hspace{2pt}}c@{\hspace{3pt}}|@{\hspace{3pt}}c@{\hspace{2pt}}c@{\hspace{3pt}}|@{\hspace{3pt}}c@{\hspace{3pt}}c@{\hspace{3pt}}|@{\hspace{3pt}}c@{\hspace{3pt}}c@{\hspace{3pt}}|@{\hspace{3pt}}c@{\hspace{3pt}}c@{\hspace{3pt}}}
    \toprule
  
      & 
      \multicolumn{2}{c|@{\hspace{3pt}}}{\textbf{NTL}} &
      \multicolumn{2}{c|@{\hspace{3pt}}}{\textbf{CUTI}} &
      \multicolumn{2}{c|@{\hspace{3pt}}}{\textbf{H-NTL}} &
      \multicolumn{2}{c|@{\hspace{3pt}}}{\textbf{SOPHON}} &
      \multicolumn{2}{@{\hspace{3pt}}c}{\textbf{CUPI}}
      \\
  
      \cmidrule(lr){2-11}
  
      &
      \textbf{SA} $\uparrow$ & \textbf{TA} $\downarrow$ & 
      \textbf{SA} $\uparrow$ & \textbf{TA} $\downarrow$ & 
      \textbf{SA} $\uparrow$ & \textbf{TA} $\downarrow$ & 
      \textbf{SA} $\uparrow$ & \textbf{TA} $\downarrow$ & 
      \textbf{SA} $\uparrow$ & \textbf{TA} $\downarrow$ 
      
      \\
      \midrule
      \midrule
      Pre-train & 
      \tc{\e{83.9}{?}}{\e{ 9.9}{?}} &
      \tc{\e{85.1}{?}}{\e{10.7}{?}} &
      \tc{\e{87.2}{?}}{\e{ 9.9}{?}} &
      \tc{\e{69.5}{?}}{\e{24.8}{?}} &
      \tc{\e{85.5}{?}}{\e{11.3}{?}} \\
      \cmidrule(lr){1-11}
  
      SHOT & 
      \tc{\tlds{63.0}{-20.9}}{\tldt{29.6}{+19.7}} &
      \tc{\tlds{35.3}{-49.8}}{\tldt{34.7}{+24.0}} &
      \tc{\tlds{86.6}{-0.6}}{\tldt{41.9}{+32.0}} &
      \tc{\bestoo \tlds{64.8}{-4.8}}{\bestoo \tldt{56.7}{+31.9}} &
      \tc{\tlds{85.8}{+0.3}}{\tldt{11.3}{+0.0}} \\
      \cmidrule(lr){2-11}
  
      CoWA & 
      \tc{\tlds{81.1}{-2.8}}{\tldt{12.4}{+2.5}} &
      \tc{\tlds{84.0}{-1.1}}{\tldt{12.7}{+2.1}} &
      \tc{\tlds{87.2}{+0.0}}{\tldt{10.1}{+0.2}} &
      \tc{\tlds{69.2}{-0.4}}{\tldt{26.1}{+1.3}} &
      \tc{\tlds{85.7}{+0.2}}{\tldt{11.3}{+0.0}} \\
      \cmidrule(lr){2-11}
  
      NRC & 
      \tc{\tlds{57.7}{-26.2}}{\tldt{19.8}{+9.9}} &
      \tc{\tlds{39.4}{-45.7}}{\tldt{35.5}{+24.8}} &
      \tc{\tlds{87.3}{+0.1}}{\tldt{12.1}{+2.2}} &
      \tc{\tlds{66.6}{-3.0}}{\tldt{55.6}{+30.8}} &
      \tc{\tlds{86.0}{+0.5}}{\tldt{12.2}{+0.9}} \\
      \cmidrule(lr){2-11}
  
      PLUE &  
      \tc{\bestoo \tlds{71.5}{-12.4}}{\bestoo \tldt{52.8}{+42.9}} &
      \tc{\bestoo \tlds{76.1}{-9.0}}{\bestoo \tldt{63.8}{+53.1}} &
      \tc{\tlds{85.5}{-1.8}}{\tldt{20.1}{+10.2}} &
      \tc{\tlds{75.5}{+6.0}}{\tldt{41.1}{+16.3}} &
      \tc{\bestoo \tlds{82.4}{-3.2}}{\bestoo \tldt{43.6}{+32.3}} \\
      \cmidrule(lr){2-11}
  
      \tln{Ada-}{Contrast} & 
      \tc{\tlds{ 9.4}{-74.5}}{\tldt{ 9.8}{-0.1}} &
      \tc{\tlds{ 9.3}{-75.8}}{\tldt{10.0}{-0.7}} &
      \tc{\tlds{86.3}{-1.0}}{\tldt{12.1}{+2.2}} &
      \tc{\tlds{64.5}{-5.1}}{\tldt{33.4}{+8.6}} &
      \tc{\tlds{47.2}{-38.3}}{\tldt{11.3}{+0.0}} \\
      \cmidrule(lr){2-11}
  
      DIFO & 
      \tc{\tlds{ 9.2}{-74.7}}{\tldt{ 9.2}{-0.7}} &
      \tc{\tlds{ 9.2}{-75.9}}{\tldt{ 9.2}{-1.5}} &
      \tc{\bestoo \tlds{85.0}{-2.2}}{\bestoo \tldt{42.1}{+32.2}} &
      \tc{\tlds{56.3}{-13.2}}{\tldt{51.3}{+26.5}} &
      \tc{\tlds{48.4}{-37.1}}{\tldt{10.4}{-1.0}} \\
  
      \bottomrule
    \end{tabular}
    \vspace{-3mm}
    \caption{NTL robustness against source-free domain adaptation (SFDA). We show \cods{source-domain accuracy} (\textbf{SA}) (\%), \codt{target-domain accuracy} (\textbf{TA}) (\%), and accuracy drop from the pre-trained model is in ($\cdot$). The most serious threat (highest TA) to each NTL is in\colorbox[HTML]{fee8e4}{ red.}}
    \vspace{-2.5mm}
    \label{tab:atk_tgt_sfda}
  \end{table}
  
  
%\section{Experiments}
\label{sec:experiments}
The experiments are designed to address two key research questions.
First, \textbf{RQ1} evaluates whether the average $L_2$-norm of the counterfactual perturbation vectors ($\overline{||\perturb||}$) decreases as the model overfits the data, thereby providing further empirical validation for our hypothesis.
Second, \textbf{RQ2} evaluates the ability of the proposed counterfactual regularized loss, as defined in (\ref{eq:regularized_loss2}), to mitigate overfitting when compared to existing regularization techniques.

% The experiments are designed to address three key research questions. First, \textbf{RQ1} investigates whether the mean perturbation vector norm decreases as the model overfits the data, aiming to further validate our intuition. Second, \textbf{RQ2} explores whether the mean perturbation vector norm can be effectively leveraged as a regularization term during training, offering insights into its potential role in mitigating overfitting. Finally, \textbf{RQ3} examines whether our counterfactual regularizer enables the model to achieve superior performance compared to existing regularization methods, thus highlighting its practical advantage.

\subsection{Experimental Setup}
\textbf{\textit{Datasets, Models, and Tasks.}}
The experiments are conducted on three datasets: \textit{Water Potability}~\cite{kadiwal2020waterpotability}, \textit{Phomene}~\cite{phomene}, and \textit{CIFAR-10}~\cite{krizhevsky2009learning}. For \textit{Water Potability} and \textit{Phomene}, we randomly select $80\%$ of the samples for the training set, and the remaining $20\%$ for the test set, \textit{CIFAR-10} comes already split. Furthermore, we consider the following models: Logistic Regression, Multi-Layer Perceptron (MLP) with 100 and 30 neurons on each hidden layer, and PreactResNet-18~\cite{he2016cvecvv} as a Convolutional Neural Network (CNN) architecture.
We focus on binary classification tasks and leave the extension to multiclass scenarios for future work. However, for datasets that are inherently multiclass, we transform the problem into a binary classification task by selecting two classes, aligning with our assumption.

\smallskip
\noindent\textbf{\textit{Evaluation Measures.}} To characterize the degree of overfitting, we use the test loss, as it serves as a reliable indicator of the model's generalization capability to unseen data. Additionally, we evaluate the predictive performance of each model using the test accuracy.

\smallskip
\noindent\textbf{\textit{Baselines.}} We compare CF-Reg with the following regularization techniques: L1 (``Lasso''), L2 (``Ridge''), and Dropout.

\smallskip
\noindent\textbf{\textit{Configurations.}}
For each model, we adopt specific configurations as follows.
\begin{itemize}
\item \textit{Logistic Regression:} To induce overfitting in the model, we artificially increase the dimensionality of the data beyond the number of training samples by applying a polynomial feature expansion. This approach ensures that the model has enough capacity to overfit the training data, allowing us to analyze the impact of our counterfactual regularizer. The degree of the polynomial is chosen as the smallest degree that makes the number of features greater than the number of data.
\item \textit{Neural Networks (MLP and CNN):} To take advantage of the closed-form solution for computing the optimal perturbation vector as defined in (\ref{eq:opt-delta}), we use a local linear approximation of the neural network models. Hence, given an instance $\inst_i$, we consider the (optimal) counterfactual not with respect to $\model$ but with respect to:
\begin{equation}
\label{eq:taylor}
    \model^{lin}(\inst) = \model(\inst_i) + \nabla_{\inst}\model(\inst_i)(\inst - \inst_i),
\end{equation}
where $\model^{lin}$ represents the first-order Taylor approximation of $\model$ at $\inst_i$.
Note that this step is unnecessary for Logistic Regression, as it is inherently a linear model.
\end{itemize}

\smallskip
\noindent \textbf{\textit{Implementation Details.}} We run all experiments on a machine equipped with an AMD Ryzen 9 7900 12-Core Processor and an NVIDIA GeForce RTX 4090 GPU. Our implementation is based on the PyTorch Lightning framework. We use stochastic gradient descent as the optimizer with a learning rate of $\eta = 0.001$ and no weight decay. We use a batch size of $128$. The training and test steps are conducted for $6000$ epochs on the \textit{Water Potability} and \textit{Phoneme} datasets, while for the \textit{CIFAR-10} dataset, they are performed for $200$ epochs.
Finally, the contribution $w_i^{\varepsilon}$ of each training point $\inst_i$ is uniformly set as $w_i^{\varepsilon} = 1~\forall i\in \{1,\ldots,m\}$.

The source code implementation for our experiments is available at the following GitHub repository: \url{https://anonymous.4open.science/r/COCE-80B4/README.md} 

\subsection{RQ1: Counterfactual Perturbation vs. Overfitting}
To address \textbf{RQ1}, we analyze the relationship between the test loss and the average $L_2$-norm of the counterfactual perturbation vectors ($\overline{||\perturb||}$) over training epochs.

In particular, Figure~\ref{fig:delta_loss_epochs} depicts the evolution of $\overline{||\perturb||}$ alongside the test loss for an MLP trained \textit{without} regularization on the \textit{Water Potability} dataset. 
\begin{figure}[ht]
    \centering
    \includegraphics[width=0.85\linewidth]{img/delta_loss_epochs.png}
    \caption{The average counterfactual perturbation vector $\overline{||\perturb||}$ (left $y$-axis) and the cross-entropy test loss (right $y$-axis) over training epochs ($x$-axis) for an MLP trained on the \textit{Water Potability} dataset \textit{without} regularization.}
    \label{fig:delta_loss_epochs}
\end{figure}

The plot shows a clear trend as the model starts to overfit the data (evidenced by an increase in test loss). 
Notably, $\overline{||\perturb||}$ begins to decrease, which aligns with the hypothesis that the average distance to the optimal counterfactual example gets smaller as the model's decision boundary becomes increasingly adherent to the training data.

It is worth noting that this trend is heavily influenced by the choice of the counterfactual generator model. In particular, the relationship between $\overline{||\perturb||}$ and the degree of overfitting may become even more pronounced when leveraging more accurate counterfactual generators. However, these models often come at the cost of higher computational complexity, and their exploration is left to future work.

Nonetheless, we expect that $\overline{||\perturb||}$ will eventually stabilize at a plateau, as the average $L_2$-norm of the optimal counterfactual perturbations cannot vanish to zero.

% Additionally, the choice of employing the score-based counterfactual explanation framework to generate counterfactuals was driven to promote computational efficiency.

% Future enhancements to the framework may involve adopting models capable of generating more precise counterfactuals. While such approaches may yield to performance improvements, they are likely to come at the cost of increased computational complexity.


\subsection{RQ2: Counterfactual Regularization Performance}
To answer \textbf{RQ2}, we evaluate the effectiveness of the proposed counterfactual regularization (CF-Reg) by comparing its performance against existing baselines: unregularized training loss (No-Reg), L1 regularization (L1-Reg), L2 regularization (L2-Reg), and Dropout.
Specifically, for each model and dataset combination, Table~\ref{tab:regularization_comparison} presents the mean value and standard deviation of test accuracy achieved by each method across 5 random initialization. 

The table illustrates that our regularization technique consistently delivers better results than existing methods across all evaluated scenarios, except for one case -- i.e., Logistic Regression on the \textit{Phomene} dataset. 
However, this setting exhibits an unusual pattern, as the highest model accuracy is achieved without any regularization. Even in this case, CF-Reg still surpasses other regularization baselines.

From the results above, we derive the following key insights. First, CF-Reg proves to be effective across various model types, ranging from simple linear models (Logistic Regression) to deep architectures like MLPs and CNNs, and across diverse datasets, including both tabular and image data. 
Second, CF-Reg's strong performance on the \textit{Water} dataset with Logistic Regression suggests that its benefits may be more pronounced when applied to simpler models. However, the unexpected outcome on the \textit{Phoneme} dataset calls for further investigation into this phenomenon.


\begin{table*}[h!]
    \centering
    \caption{Mean value and standard deviation of test accuracy across 5 random initializations for different model, dataset, and regularization method. The best results are highlighted in \textbf{bold}.}
    \label{tab:regularization_comparison}
    \begin{tabular}{|c|c|c|c|c|c|c|}
        \hline
        \textbf{Model} & \textbf{Dataset} & \textbf{No-Reg} & \textbf{L1-Reg} & \textbf{L2-Reg} & \textbf{Dropout} & \textbf{CF-Reg (ours)} \\ \hline
        Logistic Regression   & \textit{Water}   & $0.6595 \pm 0.0038$   & $0.6729 \pm 0.0056$   & $0.6756 \pm 0.0046$  & N/A    & $\mathbf{0.6918 \pm 0.0036}$                     \\ \hline
        MLP   & \textit{Water}   & $0.6756 \pm 0.0042$   & $0.6790 \pm 0.0058$   & $0.6790 \pm 0.0023$  & $0.6750 \pm 0.0036$    & $\mathbf{0.6802 \pm 0.0046}$                    \\ \hline
%        MLP   & \textit{Adult}   & $0.8404 \pm 0.0010$   & $\mathbf{0.8495 \pm 0.0007}$   & $0.8489 \pm 0.0014$  & $\mathbf{0.8495 \pm 0.0016}$     & $0.8449 \pm 0.0019$                    \\ \hline
        Logistic Regression   & \textit{Phomene}   & $\mathbf{0.8148 \pm 0.0020}$   & $0.8041 \pm 0.0028$   & $0.7835 \pm 0.0176$  & N/A    & $0.8098 \pm 0.0055$                     \\ \hline
        MLP   & \textit{Phomene}   & $0.8677 \pm 0.0033$   & $0.8374 \pm 0.0080$   & $0.8673 \pm 0.0045$  & $0.8672 \pm 0.0042$     & $\mathbf{0.8718 \pm 0.0040}$                    \\ \hline
        CNN   & \textit{CIFAR-10} & $0.6670 \pm 0.0233$   & $0.6229 \pm 0.0850$   & $0.7348 \pm 0.0365$   & N/A    & $\mathbf{0.7427 \pm 0.0571}$                     \\ \hline
    \end{tabular}
\end{table*}

\begin{table*}[htb!]
    \centering
    \caption{Hyperparameter configurations utilized for the generation of Table \ref{tab:regularization_comparison}. For our regularization the hyperparameters are reported as $\mathbf{\alpha/\beta}$.}
    \label{tab:performance_parameters}
    \begin{tabular}{|c|c|c|c|c|c|c|}
        \hline
        \textbf{Model} & \textbf{Dataset} & \textbf{No-Reg} & \textbf{L1-Reg} & \textbf{L2-Reg} & \textbf{Dropout} & \textbf{CF-Reg (ours)} \\ \hline
        Logistic Regression   & \textit{Water}   & N/A   & $0.0093$   & $0.6927$  & N/A    & $0.3791/1.0355$                     \\ \hline
        MLP   & \textit{Water}   & N/A   & $0.0007$   & $0.0022$  & $0.0002$    & $0.2567/1.9775$                    \\ \hline
        Logistic Regression   &
        \textit{Phomene}   & N/A   & $0.0097$   & $0.7979$  & N/A    & $0.0571/1.8516$                     \\ \hline
        MLP   & \textit{Phomene}   & N/A   & $0.0007$   & $4.24\cdot10^{-5}$  & $0.0015$    & $0.0516/2.2700$                    \\ \hline
       % MLP   & \textit{Adult}   & N/A   & $0.0018$   & $0.0018$  & $0.0601$     & $0.0764/2.2068$                    \\ \hline
        CNN   & \textit{CIFAR-10} & N/A   & $0.0050$   & $0.0864$ & N/A    & $0.3018/
        2.1502$                     \\ \hline
    \end{tabular}
\end{table*}

\begin{table*}[htb!]
    \centering
    \caption{Mean value and standard deviation of training time across 5 different runs. The reported time (in seconds) corresponds to the generation of each entry in Table \ref{tab:regularization_comparison}. Times are }
    \label{tab:times}
    \begin{tabular}{|c|c|c|c|c|c|c|}
        \hline
        \textbf{Model} & \textbf{Dataset} & \textbf{No-Reg} & \textbf{L1-Reg} & \textbf{L2-Reg} & \textbf{Dropout} & \textbf{CF-Reg (ours)} \\ \hline
        Logistic Regression   & \textit{Water}   & $222.98 \pm 1.07$   & $239.94 \pm 2.59$   & $241.60 \pm 1.88$  & N/A    & $251.50 \pm 1.93$                     \\ \hline
        MLP   & \textit{Water}   & $225.71 \pm 3.85$   & $250.13 \pm 4.44$   & $255.78 \pm 2.38$  & $237.83 \pm 3.45$    & $266.48 \pm 3.46$                    \\ \hline
        Logistic Regression   & \textit{Phomene}   & $266.39 \pm 0.82$ & $367.52 \pm 6.85$   & $361.69 \pm 4.04$  & N/A   & $310.48 \pm 0.76$                    \\ \hline
        MLP   &
        \textit{Phomene} & $335.62 \pm 1.77$   & $390.86 \pm 2.11$   & $393.96 \pm 1.95$ & $363.51 \pm 5.07$    & $403.14 \pm 1.92$                     \\ \hline
       % MLP   & \textit{Adult}   & N/A   & $0.0018$   & $0.0018$  & $0.0601$     & $0.0764/2.2068$                    \\ \hline
        CNN   & \textit{CIFAR-10} & $370.09 \pm 0.18$   & $395.71 \pm 0.55$   & $401.38 \pm 0.16$ & N/A    & $1287.8 \pm 0.26$                     \\ \hline
    \end{tabular}
\end{table*}

\subsection{Feasibility of our Method}
A crucial requirement for any regularization technique is that it should impose minimal impact on the overall training process.
In this respect, CF-Reg introduces an overhead that depends on the time required to find the optimal counterfactual example for each training instance. 
As such, the more sophisticated the counterfactual generator model probed during training the higher would be the time required. However, a more advanced counterfactual generator might provide a more effective regularization. We discuss this trade-off in more details in Section~\ref{sec:discussion}.

Table~\ref{tab:times} presents the average training time ($\pm$ standard deviation) for each model and dataset combination listed in Table~\ref{tab:regularization_comparison}.
We can observe that the higher accuracy achieved by CF-Reg using the score-based counterfactual generator comes with only minimal overhead. However, when applied to deep neural networks with many hidden layers, such as \textit{PreactResNet-18}, the forward derivative computation required for the linearization of the network introduces a more noticeable computational cost, explaining the longer training times in the table.

\subsection{Hyperparameter Sensitivity Analysis}
The proposed counterfactual regularization technique relies on two key hyperparameters: $\alpha$ and $\beta$. The former is intrinsic to the loss formulation defined in (\ref{eq:cf-train}), while the latter is closely tied to the choice of the score-based counterfactual explanation method used.

Figure~\ref{fig:test_alpha_beta} illustrates how the test accuracy of an MLP trained on the \textit{Water Potability} dataset changes for different combinations of $\alpha$ and $\beta$.

\begin{figure}[ht]
    \centering
    \includegraphics[width=0.85\linewidth]{img/test_acc_alpha_beta.png}
    \caption{The test accuracy of an MLP trained on the \textit{Water Potability} dataset, evaluated while varying the weight of our counterfactual regularizer ($\alpha$) for different values of $\beta$.}
    \label{fig:test_alpha_beta}
\end{figure}

We observe that, for a fixed $\beta$, increasing the weight of our counterfactual regularizer ($\alpha$) can slightly improve test accuracy until a sudden drop is noticed for $\alpha > 0.1$.
This behavior was expected, as the impact of our penalty, like any regularization term, can be disruptive if not properly controlled.

Moreover, this finding further demonstrates that our regularization method, CF-Reg, is inherently data-driven. Therefore, it requires specific fine-tuning based on the combination of the model and dataset at hand.
\subsection{Simulation Results}
We use three metrics to evaluate our method in simulation experiments.
First, we evaluate grasp traversal success, which indicates whether the end effector can move to the grasp pose without colliding with the object of interest.
When evaluating this, we set an initial pose with a 0.2 m offset from the proposed grasp pose.
We count a successful grasp traversal if no collisions occur when moving between the initial and final grasp poses.
Secondly, we evaluate whether the gripper can close without collision after reaching the proposed grasp pose. 
We count a success if the gripper successfully closes with part of the object inside the convex hull of the closed gripper.
Lastly, we evaluate whether, after closing the gripper, the arm can pick up the object 0.1 m above the position and hold it there for 5 seconds.
% We note that these metrics are cascading: If the first metric fails, we also count the second and third as failures.

Table \ref{tab:quant_res_counting} shows the simulation experiments' results.
We point to the results in the partial and noisy reconstruction test setup.
TSGrasp consistently fails for partial reconstructions because the proposed grasp region is in an area with incomplete measurements.
This results in grasp candidates that consistently collide with the object.
Conversely, PUGS can determine that the edges of partially reconstructed geometry are unfavorable and guide the grasp to more reliable locations.
We show this occurring for the kettlebell and coffee mug in Fig. \ref{fig:sideways_fail}.
For the complete reconstruction, TSGrasp consistently outperforms PUGS.
This is expected, as the role of uncertainty as a signal to pull towards informative regions becomes less impactful when a complete reconstruction is obtained.

\subsection{Real World Results}
For the real-world results, we qualitatively evaluate the success of the grasp outputs for a kettlebell captured from different partial views.
Qualitative results from the grasp selection in real-world experiments are shown in Fig. \ref{fig:qualitative_results}.
We highlight that PUGS consistently leads the gripper pose to areas with more observations while retaining the grasping area's geometric feasibility.

The leftmost result in Fig. \ref{fig:qualitative_results} highlights the robustness that PUGS introduces when reasoning over very noisy pointclouds.
For this test, we tune the pointcloud filtering to be less aggressive by setting the outlier removal threshold to $\sigma_\text{thresh}=0.1$.
The proposed grasp pose from TSGrasp~\cite{player_real-time_2023} outputs the most confident grasp pose to be area captured due to noise, while PUGS successfully recovers to the kettlebell handle. 
This recovery of grasp success is shown with the green arrow in Fig. \ref{fig:qualitative_results} to indicate the correction.

\section{RELATED WORK}
\label{sec:relatedwork}
In this section, we describe the previous works related to our proposal, which are divided into two parts. In Section~\ref{sec:relatedwork_exoplanet}, we present a review of approaches based on machine learning techniques for the detection of planetary transit signals. Section~\ref{sec:relatedwork_attention} provides an account of the approaches based on attention mechanisms applied in Astronomy.\par

\subsection{Exoplanet detection}
\label{sec:relatedwork_exoplanet}
Machine learning methods have achieved great performance for the automatic selection of exoplanet transit signals. One of the earliest applications of machine learning is a model named Autovetter \citep{MCcauliff}, which is a random forest (RF) model based on characteristics derived from Kepler pipeline statistics to classify exoplanet and false positive signals. Then, other studies emerged that also used supervised learning. \cite{mislis2016sidra} also used a RF, but unlike the work by \citet{MCcauliff}, they used simulated light curves and a box least square \citep[BLS;][]{kovacs2002box}-based periodogram to search for transiting exoplanets. \citet{thompson2015machine} proposed a k-nearest neighbors model for Kepler data to determine if a given signal has similarity to known transits. Unsupervised learning techniques were also applied, such as self-organizing maps (SOM), proposed \citet{armstrong2016transit}; which implements an architecture to segment similar light curves. In the same way, \citet{armstrong2018automatic} developed a combination of supervised and unsupervised learning, including RF and SOM models. In general, these approaches require a previous phase of feature engineering for each light curve. \par

%DL is a modern data-driven technology that automatically extracts characteristics, and that has been successful in classification problems from a variety of application domains. The architecture relies on several layers of NNs of simple interconnected units and uses layers to build increasingly complex and useful features by means of linear and non-linear transformation. This family of models is capable of generating increasingly high-level representations \citep{lecun2015deep}.

The application of DL for exoplanetary signal detection has evolved rapidly in recent years and has become very popular in planetary science.  \citet{pearson2018} and \citet{zucker2018shallow} developed CNN-based algorithms that learn from synthetic data to search for exoplanets. Perhaps one of the most successful applications of the DL models in transit detection was that of \citet{Shallue_2018}; who, in collaboration with Google, proposed a CNN named AstroNet that recognizes exoplanet signals in real data from Kepler. AstroNet uses the training set of labelled TCEs from the Autovetter planet candidate catalog of Q1–Q17 data release 24 (DR24) of the Kepler mission \citep{catanzarite2015autovetter}. AstroNet analyses the data in two views: a ``global view'', and ``local view'' \citep{Shallue_2018}. \par


% The global view shows the characteristics of the light curve over an orbital period, and a local view shows the moment at occurring the transit in detail

%different = space-based

Based on AstroNet, researchers have modified the original AstroNet model to rank candidates from different surveys, specifically for Kepler and TESS missions. \citet{ansdell2018scientific} developed a CNN trained on Kepler data, and included for the first time the information on the centroids, showing that the model improves performance considerably. Then, \citet{osborn2020rapid} and \citet{yu2019identifying} also included the centroids information, but in addition, \citet{osborn2020rapid} included information of the stellar and transit parameters. Finally, \citet{rao2021nigraha} proposed a pipeline that includes a new ``half-phase'' view of the transit signal. This half-phase view represents a transit view with a different time and phase. The purpose of this view is to recover any possible secondary eclipse (the object hiding behind the disk of the primary star).


%last pipeline applies a procedure after the prediction of the model to obtain new candidates, this process is carried out through a series of steps that include the evaluation with Discovery and Validation of Exoplanets (DAVE) \citet{kostov2019discovery} that was adapted for the TESS telescope.\par
%



\subsection{Attention mechanisms in astronomy}
\label{sec:relatedwork_attention}
Despite the remarkable success of attention mechanisms in sequential data, few papers have exploited their advantages in astronomy. In particular, there are no models based on attention mechanisms for detecting planets. Below we present a summary of the main applications of this modeling approach to astronomy, based on two points of view; performance and interpretability of the model.\par
%Attention mechanisms have not yet been explored in all sub-areas of astronomy. However, recent works show a successful application of the mechanism.
%performance

The application of attention mechanisms has shown improvements in the performance of some regression and classification tasks compared to previous approaches. One of the first implementations of the attention mechanism was to find gravitational lenses proposed by \citet{thuruthipilly2021finding}. They designed 21 self-attention-based encoder models, where each model was trained separately with 18,000 simulated images, demonstrating that the model based on the Transformer has a better performance and uses fewer trainable parameters compared to CNN. A novel application was proposed by \citet{lin2021galaxy} for the morphological classification of galaxies, who used an architecture derived from the Transformer, named Vision Transformer (VIT) \citep{dosovitskiy2020image}. \citet{lin2021galaxy} demonstrated competitive results compared to CNNs. Another application with successful results was proposed by \citet{zerveas2021transformer}; which first proposed a transformer-based framework for learning unsupervised representations of multivariate time series. Their methodology takes advantage of unlabeled data to train an encoder and extract dense vector representations of time series. Subsequently, they evaluate the model for regression and classification tasks, demonstrating better performance than other state-of-the-art supervised methods, even with data sets with limited samples.

%interpretation
Regarding the interpretability of the model, a recent contribution that analyses the attention maps was presented by \citet{bowles20212}, which explored the use of group-equivariant self-attention for radio astronomy classification. Compared to other approaches, this model analysed the attention maps of the predictions and showed that the mechanism extracts the brightest spots and jets of the radio source more clearly. This indicates that attention maps for prediction interpretation could help experts see patterns that the human eye often misses. \par

In the field of variable stars, \citet{allam2021paying} employed the mechanism for classifying multivariate time series in variable stars. And additionally, \citet{allam2021paying} showed that the activation weights are accommodated according to the variation in brightness of the star, achieving a more interpretable model. And finally, related to the TESS telescope, \citet{morvan2022don} proposed a model that removes the noise from the light curves through the distribution of attention weights. \citet{morvan2022don} showed that the use of the attention mechanism is excellent for removing noise and outliers in time series datasets compared with other approaches. In addition, the use of attention maps allowed them to show the representations learned from the model. \par

Recent attention mechanism approaches in astronomy demonstrate comparable results with earlier approaches, such as CNNs. At the same time, they offer interpretability of their results, which allows a post-prediction analysis. \par



% \section{Discussion}



% \section{Conclusion and Future Work}
% \label{sec:conclusion}

% The \textit{Titulm} series demonstrates competitive performance among models of comparable scale on Bengali tasks, with the 3B variant consistently outscoring its 1B counterpart. Moving from 1B to 3B parameters yields more robust improvements in tasks that involve commonsense or specialized reasoning. Additional Bengali tokens in the tokenizer have helped the model to understand the relation between words better. However, as a result of limited training the model performs less for long contexts. The model also shows minimal improvement in the Bangla MMLU dataset, highlighting potential directions for more specialized domain adaptation and training. Another drawback of our training is that the models are trained with only Bengali text data. To increase the training data and ensure more knowledge leveraging larger English-centric datasets can boost low-resource language performance. Such investigations could guide the development of more capable \textit{Titulm} models for under-resourced languages like Bengali, ultimately improving generalization and knowledge retrieval across a diverse range of tasks. Additionally, in future, these base models can be used for making generalized Bengali LLMs. However, with the lack of enough instruction tuning data available in Bengali, this task is beyond this research work for now.

\section{Conclusion}
\label{sec:conclusion}
% In this study, we present the \textit{first} pretrained Bangla LLMs (\textit{Titulm}) with $\sim37b$ tokens by adapting Llama-3.2 models. We have extended tokenizer to enrich language and culture specific knowledge, which also enables faster training and inference. Pretraining data collection is challenging for languages with low digital representation. We provide a complete recipe including raw web data collection, translation and synthetic data generation. Given that there was a lack of LLM based benchmakring dataset, therefore, we have developed five datasets consisting of $137k$ samples, covering knowledge and reasoning. The benchmakring dataset include both manual and our novel translation based approach. Using these dataset, we benchmarked different LLMs including \textit{Titulm}. We demonstrate that performs better in reasoning tasks without any instruction tuning. Future work include collecting large pretraining  datasets, and fine-tuning with instruction datasets. We have made our models publicly available for the community and for the reproduciblity, we aim to release training receipes and benchmarking datasets.
In this study, we present the \textit{first} pretrained Bangla LLMs, \textit{Titulm}, trained on $\sim37b$ tokens by adapting Llama-3.2 models. We extended the tokenizer to incorporate language- and culture-specific knowledge, which also enable faster training and inference. Pretraining data collection remains challenging for languages with low digital representation. To address this, we provide a comprehensive approach, including raw web data collection, translation, and synthetic data generation. Given the lack of LLM-based benchmarking datasets, we developed \textit{five datasets} comprising $137k$ samples, covering both knowledge and reasoning. The benchmarking dataset includes manually curated samples as well as a novel translation-based (EST) approach. Using these datasets, we benchmarked various LLMs, including \textit{Titulm}, demonstrating its superior performance in reasoning tasks without instruction tuning. Future work includes collecting larger pretraining datasets and fine-tuning with instruction-based datasets. We have made our models publicly available, and to support reproducibility, we plan to release training recipes and benchmarking datasets.


% \textit{Titulm} series, 
% This paper presents significant advancements in developing monolingual Bangla language models through the \textit{Titulm} series.  
% By extending the tokenizer and leveraging a large-scale Bangla MMLU dataset in our benchmarking, we demonstrate the effectiveness of these models in enhancing word relation understanding and improving performance on commonsense and specialized reasoning tasks.  
% This work lays a strong foundation for the development of more capable models for Bangla and other low-resource languages, offering valuable resources for further research and application in multilingual LLMs. Additionally, we introduce a novel English to Bangla translation technique to enhance the translation data quality, contributing to more accurate and context-aware translations.

% Additionally, it highlights the potential for domain adaptation and the importance of creating high-quality instruction tuning datasets to further improve performance in specialized tasks.
 
\section{Limitations}
\label{sec:limitations}
There are two limitations in this work. Firstly, despite the improvements observed in the 3b variant, the model's performance on long contexts remains suboptimal, suggesting the need for further enhancement in handling extended sequences. Secondly, while the current models are trained solely on Bangla text, their performance could benefit from incorporating larger, English-centric datasets. This could facilitate better knowledge leveraging and potentially improve low-resource language performance, indicating a direction for future research. Since there is a lack of instruction tuning data in Bangla, we do not explore the full potential of instruction tuning, which could have further improved the models' performance on specialized tasks and domain adaptation.

\section*{Ethical Consideration}
\label{sec:ethics}
We do not anticipate any ethical concerns in this study. All datasets used were collected from publicly available sources, ensuring compliance with ethical research standards. No personally identifiable information (PII) was gathered or utilized in the development of our models. While we do not foresee any potential risks arising from the outcomes of this study, we strongly encourage users of the released models to adhere to responsible AI usage guidelines. This includes avoiding the generation or dissemination of harmful, misleading, or biased content and ensuring that the models are employed in ethical and socially beneficial applications.


% \section*{Acknowledgments}

% Acknowledgments section here 


\bibliography{bibliography}
% \bibliographystyle{acl_natbib}

\appendix

\subsection{Lloyd-Max Algorithm}
\label{subsec:Lloyd-Max}
For a given quantization bitwidth $B$ and an operand $\bm{X}$, the Lloyd-Max algorithm finds $2^B$ quantization levels $\{\hat{x}_i\}_{i=1}^{2^B}$ such that quantizing $\bm{X}$ by rounding each scalar in $\bm{X}$ to the nearest quantization level minimizes the quantization MSE. 

The algorithm starts with an initial guess of quantization levels and then iteratively computes quantization thresholds $\{\tau_i\}_{i=1}^{2^B-1}$ and updates quantization levels $\{\hat{x}_i\}_{i=1}^{2^B}$. Specifically, at iteration $n$, thresholds are set to the midpoints of the previous iteration's levels:
\begin{align*}
    \tau_i^{(n)}=\frac{\hat{x}_i^{(n-1)}+\hat{x}_{i+1}^{(n-1)}}2 \text{ for } i=1\ldots 2^B-1
\end{align*}
Subsequently, the quantization levels are re-computed as conditional means of the data regions defined by the new thresholds:
\begin{align*}
    \hat{x}_i^{(n)}=\mathbb{E}\left[ \bm{X} \big| \bm{X}\in [\tau_{i-1}^{(n)},\tau_i^{(n)}] \right] \text{ for } i=1\ldots 2^B
\end{align*}
where to satisfy boundary conditions we have $\tau_0=-\infty$ and $\tau_{2^B}=\infty$. The algorithm iterates the above steps until convergence.

Figure \ref{fig:lm_quant} compares the quantization levels of a $7$-bit floating point (E3M3) quantizer (left) to a $7$-bit Lloyd-Max quantizer (right) when quantizing a layer of weights from the GPT3-126M model at a per-tensor granularity. As shown, the Lloyd-Max quantizer achieves substantially lower quantization MSE. Further, Table \ref{tab:FP7_vs_LM7} shows the superior perplexity achieved by Lloyd-Max quantizers for bitwidths of $7$, $6$ and $5$. The difference between the quantizers is clear at 5 bits, where per-tensor FP quantization incurs a drastic and unacceptable increase in perplexity, while Lloyd-Max quantization incurs a much smaller increase. Nevertheless, we note that even the optimal Lloyd-Max quantizer incurs a notable ($\sim 1.5$) increase in perplexity due to the coarse granularity of quantization. 

\begin{figure}[h]
  \centering
  \includegraphics[width=0.7\linewidth]{sections/figures/LM7_FP7.pdf}
  \caption{\small Quantization levels and the corresponding quantization MSE of Floating Point (left) vs Lloyd-Max (right) Quantizers for a layer of weights in the GPT3-126M model.}
  \label{fig:lm_quant}
\end{figure}

\begin{table}[h]\scriptsize
\begin{center}
\caption{\label{tab:FP7_vs_LM7} \small Comparing perplexity (lower is better) achieved by floating point quantizers and Lloyd-Max quantizers on a GPT3-126M model for the Wikitext-103 dataset.}
\begin{tabular}{c|cc|c}
\hline
 \multirow{2}{*}{\textbf{Bitwidth}} & \multicolumn{2}{|c|}{\textbf{Floating-Point Quantizer}} & \textbf{Lloyd-Max Quantizer} \\
 & Best Format & Wikitext-103 Perplexity & Wikitext-103 Perplexity \\
\hline
7 & E3M3 & 18.32 & 18.27 \\
6 & E3M2 & 19.07 & 18.51 \\
5 & E4M0 & 43.89 & 19.71 \\
\hline
\end{tabular}
\end{center}
\end{table}

\subsection{Proof of Local Optimality of LO-BCQ}
\label{subsec:lobcq_opt_proof}
For a given block $\bm{b}_j$, the quantization MSE during LO-BCQ can be empirically evaluated as $\frac{1}{L_b}\lVert \bm{b}_j- \bm{\hat{b}}_j\rVert^2_2$ where $\bm{\hat{b}}_j$ is computed from equation (\ref{eq:clustered_quantization_definition}) as $C_{f(\bm{b}_j)}(\bm{b}_j)$. Further, for a given block cluster $\mathcal{B}_i$, we compute the quantization MSE as $\frac{1}{|\mathcal{B}_{i}|}\sum_{\bm{b} \in \mathcal{B}_{i}} \frac{1}{L_b}\lVert \bm{b}- C_i^{(n)}(\bm{b})\rVert^2_2$. Therefore, at the end of iteration $n$, we evaluate the overall quantization MSE $J^{(n)}$ for a given operand $\bm{X}$ composed of $N_c$ block clusters as:
\begin{align*}
    \label{eq:mse_iter_n}
    J^{(n)} = \frac{1}{N_c} \sum_{i=1}^{N_c} \frac{1}{|\mathcal{B}_{i}^{(n)}|}\sum_{\bm{v} \in \mathcal{B}_{i}^{(n)}} \frac{1}{L_b}\lVert \bm{b}- B_i^{(n)}(\bm{b})\rVert^2_2
\end{align*}

At the end of iteration $n$, the codebooks are updated from $\mathcal{C}^{(n-1)}$ to $\mathcal{C}^{(n)}$. However, the mapping of a given vector $\bm{b}_j$ to quantizers $\mathcal{C}^{(n)}$ remains as  $f^{(n)}(\bm{b}_j)$. At the next iteration, during the vector clustering step, $f^{(n+1)}(\bm{b}_j)$ finds new mapping of $\bm{b}_j$ to updated codebooks $\mathcal{C}^{(n)}$ such that the quantization MSE over the candidate codebooks is minimized. Therefore, we obtain the following result for $\bm{b}_j$:
\begin{align*}
\frac{1}{L_b}\lVert \bm{b}_j - C_{f^{(n+1)}(\bm{b}_j)}^{(n)}(\bm{b}_j)\rVert^2_2 \le \frac{1}{L_b}\lVert \bm{b}_j - C_{f^{(n)}(\bm{b}_j)}^{(n)}(\bm{b}_j)\rVert^2_2
\end{align*}

That is, quantizing $\bm{b}_j$ at the end of the block clustering step of iteration $n+1$ results in lower quantization MSE compared to quantizing at the end of iteration $n$. Since this is true for all $\bm{b} \in \bm{X}$, we assert the following:
\begin{equation}
\begin{split}
\label{eq:mse_ineq_1}
    \tilde{J}^{(n+1)} &= \frac{1}{N_c} \sum_{i=1}^{N_c} \frac{1}{|\mathcal{B}_{i}^{(n+1)}|}\sum_{\bm{b} \in \mathcal{B}_{i}^{(n+1)}} \frac{1}{L_b}\lVert \bm{b} - C_i^{(n)}(b)\rVert^2_2 \le J^{(n)}
\end{split}
\end{equation}
where $\tilde{J}^{(n+1)}$ is the the quantization MSE after the vector clustering step at iteration $n+1$.

Next, during the codebook update step (\ref{eq:quantizers_update}) at iteration $n+1$, the per-cluster codebooks $\mathcal{C}^{(n)}$ are updated to $\mathcal{C}^{(n+1)}$ by invoking the Lloyd-Max algorithm \citep{Lloyd}. We know that for any given value distribution, the Lloyd-Max algorithm minimizes the quantization MSE. Therefore, for a given vector cluster $\mathcal{B}_i$ we obtain the following result:

\begin{equation}
    \frac{1}{|\mathcal{B}_{i}^{(n+1)}|}\sum_{\bm{b} \in \mathcal{B}_{i}^{(n+1)}} \frac{1}{L_b}\lVert \bm{b}- C_i^{(n+1)}(\bm{b})\rVert^2_2 \le \frac{1}{|\mathcal{B}_{i}^{(n+1)}|}\sum_{\bm{b} \in \mathcal{B}_{i}^{(n+1)}} \frac{1}{L_b}\lVert \bm{b}- C_i^{(n)}(\bm{b})\rVert^2_2
\end{equation}

The above equation states that quantizing the given block cluster $\mathcal{B}_i$ after updating the associated codebook from $C_i^{(n)}$ to $C_i^{(n+1)}$ results in lower quantization MSE. Since this is true for all the block clusters, we derive the following result: 
\begin{equation}
\begin{split}
\label{eq:mse_ineq_2}
     J^{(n+1)} &= \frac{1}{N_c} \sum_{i=1}^{N_c} \frac{1}{|\mathcal{B}_{i}^{(n+1)}|}\sum_{\bm{b} \in \mathcal{B}_{i}^{(n+1)}} \frac{1}{L_b}\lVert \bm{b}- C_i^{(n+1)}(\bm{b})\rVert^2_2  \le \tilde{J}^{(n+1)}   
\end{split}
\end{equation}

Following (\ref{eq:mse_ineq_1}) and (\ref{eq:mse_ineq_2}), we find that the quantization MSE is non-increasing for each iteration, that is, $J^{(1)} \ge J^{(2)} \ge J^{(3)} \ge \ldots \ge J^{(M)}$ where $M$ is the maximum number of iterations. 
%Therefore, we can say that if the algorithm converges, then it must be that it has converged to a local minimum. 
\hfill $\blacksquare$


\begin{figure}
    \begin{center}
    \includegraphics[width=0.5\textwidth]{sections//figures/mse_vs_iter.pdf}
    \end{center}
    \caption{\small NMSE vs iterations during LO-BCQ compared to other block quantization proposals}
    \label{fig:nmse_vs_iter}
\end{figure}

Figure \ref{fig:nmse_vs_iter} shows the empirical convergence of LO-BCQ across several block lengths and number of codebooks. Also, the MSE achieved by LO-BCQ is compared to baselines such as MXFP and VSQ. As shown, LO-BCQ converges to a lower MSE than the baselines. Further, we achieve better convergence for larger number of codebooks ($N_c$) and for a smaller block length ($L_b$), both of which increase the bitwidth of BCQ (see Eq \ref{eq:bitwidth_bcq}).


\subsection{Additional Accuracy Results}
%Table \ref{tab:lobcq_config} lists the various LOBCQ configurations and their corresponding bitwidths.
\begin{table}
\setlength{\tabcolsep}{4.75pt}
\begin{center}
\caption{\label{tab:lobcq_config} Various LO-BCQ configurations and their bitwidths.}
\begin{tabular}{|c||c|c|c|c||c|c||c|} 
\hline
 & \multicolumn{4}{|c||}{$L_b=8$} & \multicolumn{2}{|c||}{$L_b=4$} & $L_b=2$ \\
 \hline
 \backslashbox{$L_A$\kern-1em}{\kern-1em$N_c$} & 2 & 4 & 8 & 16 & 2 & 4 & 2 \\
 \hline
 64 & 4.25 & 4.375 & 4.5 & 4.625 & 4.375 & 4.625 & 4.625\\
 \hline
 32 & 4.375 & 4.5 & 4.625& 4.75 & 4.5 & 4.75 & 4.75 \\
 \hline
 16 & 4.625 & 4.75& 4.875 & 5 & 4.75 & 5 & 5 \\
 \hline
\end{tabular}
\end{center}
\end{table}

%\subsection{Perplexity achieved by various LO-BCQ configurations on Wikitext-103 dataset}

\begin{table} \centering
\begin{tabular}{|c||c|c|c|c||c|c||c|} 
\hline
 $L_b \rightarrow$& \multicolumn{4}{c||}{8} & \multicolumn{2}{c||}{4} & 2\\
 \hline
 \backslashbox{$L_A$\kern-1em}{\kern-1em$N_c$} & 2 & 4 & 8 & 16 & 2 & 4 & 2  \\
 %$N_c \rightarrow$ & 2 & 4 & 8 & 16 & 2 & 4 & 2 \\
 \hline
 \hline
 \multicolumn{8}{c}{GPT3-1.3B (FP32 PPL = 9.98)} \\ 
 \hline
 \hline
 64 & 10.40 & 10.23 & 10.17 & 10.15 &  10.28 & 10.18 & 10.19 \\
 \hline
 32 & 10.25 & 10.20 & 10.15 & 10.12 &  10.23 & 10.17 & 10.17 \\
 \hline
 16 & 10.22 & 10.16 & 10.10 & 10.09 &  10.21 & 10.14 & 10.16 \\
 \hline
  \hline
 \multicolumn{8}{c}{GPT3-8B (FP32 PPL = 7.38)} \\ 
 \hline
 \hline
 64 & 7.61 & 7.52 & 7.48 &  7.47 &  7.55 &  7.49 & 7.50 \\
 \hline
 32 & 7.52 & 7.50 & 7.46 &  7.45 &  7.52 &  7.48 & 7.48  \\
 \hline
 16 & 7.51 & 7.48 & 7.44 &  7.44 &  7.51 &  7.49 & 7.47  \\
 \hline
\end{tabular}
\caption{\label{tab:ppl_gpt3_abalation} Wikitext-103 perplexity across GPT3-1.3B and 8B models.}
\end{table}

\begin{table} \centering
\begin{tabular}{|c||c|c|c|c||} 
\hline
 $L_b \rightarrow$& \multicolumn{4}{c||}{8}\\
 \hline
 \backslashbox{$L_A$\kern-1em}{\kern-1em$N_c$} & 2 & 4 & 8 & 16 \\
 %$N_c \rightarrow$ & 2 & 4 & 8 & 16 & 2 & 4 & 2 \\
 \hline
 \hline
 \multicolumn{5}{|c|}{Llama2-7B (FP32 PPL = 5.06)} \\ 
 \hline
 \hline
 64 & 5.31 & 5.26 & 5.19 & 5.18  \\
 \hline
 32 & 5.23 & 5.25 & 5.18 & 5.15  \\
 \hline
 16 & 5.23 & 5.19 & 5.16 & 5.14  \\
 \hline
 \multicolumn{5}{|c|}{Nemotron4-15B (FP32 PPL = 5.87)} \\ 
 \hline
 \hline
 64  & 6.3 & 6.20 & 6.13 & 6.08  \\
 \hline
 32  & 6.24 & 6.12 & 6.07 & 6.03  \\
 \hline
 16  & 6.12 & 6.14 & 6.04 & 6.02  \\
 \hline
 \multicolumn{5}{|c|}{Nemotron4-340B (FP32 PPL = 3.48)} \\ 
 \hline
 \hline
 64 & 3.67 & 3.62 & 3.60 & 3.59 \\
 \hline
 32 & 3.63 & 3.61 & 3.59 & 3.56 \\
 \hline
 16 & 3.61 & 3.58 & 3.57 & 3.55 \\
 \hline
\end{tabular}
\caption{\label{tab:ppl_llama7B_nemo15B} Wikitext-103 perplexity compared to FP32 baseline in Llama2-7B and Nemotron4-15B, 340B models}
\end{table}

%\subsection{Perplexity achieved by various LO-BCQ configurations on MMLU dataset}


\begin{table} \centering
\begin{tabular}{|c||c|c|c|c||c|c|c|c|} 
\hline
 $L_b \rightarrow$& \multicolumn{4}{c||}{8} & \multicolumn{4}{c||}{8}\\
 \hline
 \backslashbox{$L_A$\kern-1em}{\kern-1em$N_c$} & 2 & 4 & 8 & 16 & 2 & 4 & 8 & 16  \\
 %$N_c \rightarrow$ & 2 & 4 & 8 & 16 & 2 & 4 & 2 \\
 \hline
 \hline
 \multicolumn{5}{|c|}{Llama2-7B (FP32 Accuracy = 45.8\%)} & \multicolumn{4}{|c|}{Llama2-70B (FP32 Accuracy = 69.12\%)} \\ 
 \hline
 \hline
 64 & 43.9 & 43.4 & 43.9 & 44.9 & 68.07 & 68.27 & 68.17 & 68.75 \\
 \hline
 32 & 44.5 & 43.8 & 44.9 & 44.5 & 68.37 & 68.51 & 68.35 & 68.27  \\
 \hline
 16 & 43.9 & 42.7 & 44.9 & 45 & 68.12 & 68.77 & 68.31 & 68.59  \\
 \hline
 \hline
 \multicolumn{5}{|c|}{GPT3-22B (FP32 Accuracy = 38.75\%)} & \multicolumn{4}{|c|}{Nemotron4-15B (FP32 Accuracy = 64.3\%)} \\ 
 \hline
 \hline
 64 & 36.71 & 38.85 & 38.13 & 38.92 & 63.17 & 62.36 & 63.72 & 64.09 \\
 \hline
 32 & 37.95 & 38.69 & 39.45 & 38.34 & 64.05 & 62.30 & 63.8 & 64.33  \\
 \hline
 16 & 38.88 & 38.80 & 38.31 & 38.92 & 63.22 & 63.51 & 63.93 & 64.43  \\
 \hline
\end{tabular}
\caption{\label{tab:mmlu_abalation} Accuracy on MMLU dataset across GPT3-22B, Llama2-7B, 70B and Nemotron4-15B models.}
\end{table}


%\subsection{Perplexity achieved by various LO-BCQ configurations on LM evaluation harness}

\begin{table} \centering
\begin{tabular}{|c||c|c|c|c||c|c|c|c|} 
\hline
 $L_b \rightarrow$& \multicolumn{4}{c||}{8} & \multicolumn{4}{c||}{8}\\
 \hline
 \backslashbox{$L_A$\kern-1em}{\kern-1em$N_c$} & 2 & 4 & 8 & 16 & 2 & 4 & 8 & 16  \\
 %$N_c \rightarrow$ & 2 & 4 & 8 & 16 & 2 & 4 & 2 \\
 \hline
 \hline
 \multicolumn{5}{|c|}{Race (FP32 Accuracy = 37.51\%)} & \multicolumn{4}{|c|}{Boolq (FP32 Accuracy = 64.62\%)} \\ 
 \hline
 \hline
 64 & 36.94 & 37.13 & 36.27 & 37.13 & 63.73 & 62.26 & 63.49 & 63.36 \\
 \hline
 32 & 37.03 & 36.36 & 36.08 & 37.03 & 62.54 & 63.51 & 63.49 & 63.55  \\
 \hline
 16 & 37.03 & 37.03 & 36.46 & 37.03 & 61.1 & 63.79 & 63.58 & 63.33  \\
 \hline
 \hline
 \multicolumn{5}{|c|}{Winogrande (FP32 Accuracy = 58.01\%)} & \multicolumn{4}{|c|}{Piqa (FP32 Accuracy = 74.21\%)} \\ 
 \hline
 \hline
 64 & 58.17 & 57.22 & 57.85 & 58.33 & 73.01 & 73.07 & 73.07 & 72.80 \\
 \hline
 32 & 59.12 & 58.09 & 57.85 & 58.41 & 73.01 & 73.94 & 72.74 & 73.18  \\
 \hline
 16 & 57.93 & 58.88 & 57.93 & 58.56 & 73.94 & 72.80 & 73.01 & 73.94  \\
 \hline
\end{tabular}
\caption{\label{tab:mmlu_abalation} Accuracy on LM evaluation harness tasks on GPT3-1.3B model.}
\end{table}

\begin{table} \centering
\begin{tabular}{|c||c|c|c|c||c|c|c|c|} 
\hline
 $L_b \rightarrow$& \multicolumn{4}{c||}{8} & \multicolumn{4}{c||}{8}\\
 \hline
 \backslashbox{$L_A$\kern-1em}{\kern-1em$N_c$} & 2 & 4 & 8 & 16 & 2 & 4 & 8 & 16  \\
 %$N_c \rightarrow$ & 2 & 4 & 8 & 16 & 2 & 4 & 2 \\
 \hline
 \hline
 \multicolumn{5}{|c|}{Race (FP32 Accuracy = 41.34\%)} & \multicolumn{4}{|c|}{Boolq (FP32 Accuracy = 68.32\%)} \\ 
 \hline
 \hline
 64 & 40.48 & 40.10 & 39.43 & 39.90 & 69.20 & 68.41 & 69.45 & 68.56 \\
 \hline
 32 & 39.52 & 39.52 & 40.77 & 39.62 & 68.32 & 67.43 & 68.17 & 69.30  \\
 \hline
 16 & 39.81 & 39.71 & 39.90 & 40.38 & 68.10 & 66.33 & 69.51 & 69.42  \\
 \hline
 \hline
 \multicolumn{5}{|c|}{Winogrande (FP32 Accuracy = 67.88\%)} & \multicolumn{4}{|c|}{Piqa (FP32 Accuracy = 78.78\%)} \\ 
 \hline
 \hline
 64 & 66.85 & 66.61 & 67.72 & 67.88 & 77.31 & 77.42 & 77.75 & 77.64 \\
 \hline
 32 & 67.25 & 67.72 & 67.72 & 67.00 & 77.31 & 77.04 & 77.80 & 77.37  \\
 \hline
 16 & 68.11 & 68.90 & 67.88 & 67.48 & 77.37 & 78.13 & 78.13 & 77.69  \\
 \hline
\end{tabular}
\caption{\label{tab:mmlu_abalation} Accuracy on LM evaluation harness tasks on GPT3-8B model.}
\end{table}

\begin{table} \centering
\begin{tabular}{|c||c|c|c|c||c|c|c|c|} 
\hline
 $L_b \rightarrow$& \multicolumn{4}{c||}{8} & \multicolumn{4}{c||}{8}\\
 \hline
 \backslashbox{$L_A$\kern-1em}{\kern-1em$N_c$} & 2 & 4 & 8 & 16 & 2 & 4 & 8 & 16  \\
 %$N_c \rightarrow$ & 2 & 4 & 8 & 16 & 2 & 4 & 2 \\
 \hline
 \hline
 \multicolumn{5}{|c|}{Race (FP32 Accuracy = 40.67\%)} & \multicolumn{4}{|c|}{Boolq (FP32 Accuracy = 76.54\%)} \\ 
 \hline
 \hline
 64 & 40.48 & 40.10 & 39.43 & 39.90 & 75.41 & 75.11 & 77.09 & 75.66 \\
 \hline
 32 & 39.52 & 39.52 & 40.77 & 39.62 & 76.02 & 76.02 & 75.96 & 75.35  \\
 \hline
 16 & 39.81 & 39.71 & 39.90 & 40.38 & 75.05 & 73.82 & 75.72 & 76.09  \\
 \hline
 \hline
 \multicolumn{5}{|c|}{Winogrande (FP32 Accuracy = 70.64\%)} & \multicolumn{4}{|c|}{Piqa (FP32 Accuracy = 79.16\%)} \\ 
 \hline
 \hline
 64 & 69.14 & 70.17 & 70.17 & 70.56 & 78.24 & 79.00 & 78.62 & 78.73 \\
 \hline
 32 & 70.96 & 69.69 & 71.27 & 69.30 & 78.56 & 79.49 & 79.16 & 78.89  \\
 \hline
 16 & 71.03 & 69.53 & 69.69 & 70.40 & 78.13 & 79.16 & 79.00 & 79.00  \\
 \hline
\end{tabular}
\caption{\label{tab:mmlu_abalation} Accuracy on LM evaluation harness tasks on GPT3-22B model.}
\end{table}

\begin{table} \centering
\begin{tabular}{|c||c|c|c|c||c|c|c|c|} 
\hline
 $L_b \rightarrow$& \multicolumn{4}{c||}{8} & \multicolumn{4}{c||}{8}\\
 \hline
 \backslashbox{$L_A$\kern-1em}{\kern-1em$N_c$} & 2 & 4 & 8 & 16 & 2 & 4 & 8 & 16  \\
 %$N_c \rightarrow$ & 2 & 4 & 8 & 16 & 2 & 4 & 2 \\
 \hline
 \hline
 \multicolumn{5}{|c|}{Race (FP32 Accuracy = 44.4\%)} & \multicolumn{4}{|c|}{Boolq (FP32 Accuracy = 79.29\%)} \\ 
 \hline
 \hline
 64 & 42.49 & 42.51 & 42.58 & 43.45 & 77.58 & 77.37 & 77.43 & 78.1 \\
 \hline
 32 & 43.35 & 42.49 & 43.64 & 43.73 & 77.86 & 75.32 & 77.28 & 77.86  \\
 \hline
 16 & 44.21 & 44.21 & 43.64 & 42.97 & 78.65 & 77 & 76.94 & 77.98  \\
 \hline
 \hline
 \multicolumn{5}{|c|}{Winogrande (FP32 Accuracy = 69.38\%)} & \multicolumn{4}{|c|}{Piqa (FP32 Accuracy = 78.07\%)} \\ 
 \hline
 \hline
 64 & 68.9 & 68.43 & 69.77 & 68.19 & 77.09 & 76.82 & 77.09 & 77.86 \\
 \hline
 32 & 69.38 & 68.51 & 68.82 & 68.90 & 78.07 & 76.71 & 78.07 & 77.86  \\
 \hline
 16 & 69.53 & 67.09 & 69.38 & 68.90 & 77.37 & 77.8 & 77.91 & 77.69  \\
 \hline
\end{tabular}
\caption{\label{tab:mmlu_abalation} Accuracy on LM evaluation harness tasks on Llama2-7B model.}
\end{table}

\begin{table} \centering
\begin{tabular}{|c||c|c|c|c||c|c|c|c|} 
\hline
 $L_b \rightarrow$& \multicolumn{4}{c||}{8} & \multicolumn{4}{c||}{8}\\
 \hline
 \backslashbox{$L_A$\kern-1em}{\kern-1em$N_c$} & 2 & 4 & 8 & 16 & 2 & 4 & 8 & 16  \\
 %$N_c \rightarrow$ & 2 & 4 & 8 & 16 & 2 & 4 & 2 \\
 \hline
 \hline
 \multicolumn{5}{|c|}{Race (FP32 Accuracy = 48.8\%)} & \multicolumn{4}{|c|}{Boolq (FP32 Accuracy = 85.23\%)} \\ 
 \hline
 \hline
 64 & 49.00 & 49.00 & 49.28 & 48.71 & 82.82 & 84.28 & 84.03 & 84.25 \\
 \hline
 32 & 49.57 & 48.52 & 48.33 & 49.28 & 83.85 & 84.46 & 84.31 & 84.93  \\
 \hline
 16 & 49.85 & 49.09 & 49.28 & 48.99 & 85.11 & 84.46 & 84.61 & 83.94  \\
 \hline
 \hline
 \multicolumn{5}{|c|}{Winogrande (FP32 Accuracy = 79.95\%)} & \multicolumn{4}{|c|}{Piqa (FP32 Accuracy = 81.56\%)} \\ 
 \hline
 \hline
 64 & 78.77 & 78.45 & 78.37 & 79.16 & 81.45 & 80.69 & 81.45 & 81.5 \\
 \hline
 32 & 78.45 & 79.01 & 78.69 & 80.66 & 81.56 & 80.58 & 81.18 & 81.34  \\
 \hline
 16 & 79.95 & 79.56 & 79.79 & 79.72 & 81.28 & 81.66 & 81.28 & 80.96  \\
 \hline
\end{tabular}
\caption{\label{tab:mmlu_abalation} Accuracy on LM evaluation harness tasks on Llama2-70B model.}
\end{table}

%\section{MSE Studies}
%\textcolor{red}{TODO}


\subsection{Number Formats and Quantization Method}
\label{subsec:numFormats_quantMethod}
\subsubsection{Integer Format}
An $n$-bit signed integer (INT) is typically represented with a 2s-complement format \citep{yao2022zeroquant,xiao2023smoothquant,dai2021vsq}, where the most significant bit denotes the sign.

\subsubsection{Floating Point Format}
An $n$-bit signed floating point (FP) number $x$ comprises of a 1-bit sign ($x_{\mathrm{sign}}$), $B_m$-bit mantissa ($x_{\mathrm{mant}}$) and $B_e$-bit exponent ($x_{\mathrm{exp}}$) such that $B_m+B_e=n-1$. The associated constant exponent bias ($E_{\mathrm{bias}}$) is computed as $(2^{{B_e}-1}-1)$. We denote this format as $E_{B_e}M_{B_m}$.  

\subsubsection{Quantization Scheme}
\label{subsec:quant_method}
A quantization scheme dictates how a given unquantized tensor is converted to its quantized representation. We consider FP formats for the purpose of illustration. Given an unquantized tensor $\bm{X}$ and an FP format $E_{B_e}M_{B_m}$, we first, we compute the quantization scale factor $s_X$ that maps the maximum absolute value of $\bm{X}$ to the maximum quantization level of the $E_{B_e}M_{B_m}$ format as follows:
\begin{align}
\label{eq:sf}
    s_X = \frac{\mathrm{max}(|\bm{X}|)}{\mathrm{max}(E_{B_e}M_{B_m})}
\end{align}
In the above equation, $|\cdot|$ denotes the absolute value function.

Next, we scale $\bm{X}$ by $s_X$ and quantize it to $\hat{\bm{X}}$ by rounding it to the nearest quantization level of $E_{B_e}M_{B_m}$ as:

\begin{align}
\label{eq:tensor_quant}
    \hat{\bm{X}} = \text{round-to-nearest}\left(\frac{\bm{X}}{s_X}, E_{B_e}M_{B_m}\right)
\end{align}

We perform dynamic max-scaled quantization \citep{wu2020integer}, where the scale factor $s$ for activations is dynamically computed during runtime.

\subsection{Vector Scaled Quantization}
\begin{wrapfigure}{r}{0.35\linewidth}
  \centering
  \includegraphics[width=\linewidth]{sections/figures/vsquant.jpg}
  \caption{\small Vectorwise decomposition for per-vector scaled quantization (VSQ \citep{dai2021vsq}).}
  \label{fig:vsquant}
\end{wrapfigure}
During VSQ \citep{dai2021vsq}, the operand tensors are decomposed into 1D vectors in a hardware friendly manner as shown in Figure \ref{fig:vsquant}. Since the decomposed tensors are used as operands in matrix multiplications during inference, it is beneficial to perform this decomposition along the reduction dimension of the multiplication. The vectorwise quantization is performed similar to tensorwise quantization described in Equations \ref{eq:sf} and \ref{eq:tensor_quant}, where a scale factor $s_v$ is required for each vector $\bm{v}$ that maps the maximum absolute value of that vector to the maximum quantization level. While smaller vector lengths can lead to larger accuracy gains, the associated memory and computational overheads due to the per-vector scale factors increases. To alleviate these overheads, VSQ \citep{dai2021vsq} proposed a second level quantization of the per-vector scale factors to unsigned integers, while MX \citep{rouhani2023shared} quantizes them to integer powers of 2 (denoted as $2^{INT}$).

\subsubsection{MX Format}
The MX format proposed in \citep{rouhani2023microscaling} introduces the concept of sub-block shifting. For every two scalar elements of $b$-bits each, there is a shared exponent bit. The value of this exponent bit is determined through an empirical analysis that targets minimizing quantization MSE. We note that the FP format $E_{1}M_{b}$ is strictly better than MX from an accuracy perspective since it allocates a dedicated exponent bit to each scalar as opposed to sharing it across two scalars. Therefore, we conservatively bound the accuracy of a $b+2$-bit signed MX format with that of a $E_{1}M_{b}$ format in our comparisons. For instance, we use E1M2 format as a proxy for MX4.

\begin{figure}
    \centering
    \includegraphics[width=1\linewidth]{sections//figures/BlockFormats.pdf}
    \caption{\small Comparing LO-BCQ to MX format.}
    \label{fig:block_formats}
\end{figure}

Figure \ref{fig:block_formats} compares our $4$-bit LO-BCQ block format to MX \citep{rouhani2023microscaling}. As shown, both LO-BCQ and MX decompose a given operand tensor into block arrays and each block array into blocks. Similar to MX, we find that per-block quantization ($L_b < L_A$) leads to better accuracy due to increased flexibility. While MX achieves this through per-block $1$-bit micro-scales, we associate a dedicated codebook to each block through a per-block codebook selector. Further, MX quantizes the per-block array scale-factor to E8M0 format without per-tensor scaling. In contrast during LO-BCQ, we find that per-tensor scaling combined with quantization of per-block array scale-factor to E4M3 format results in superior inference accuracy across models. 


\label{sec:appendix}


\end{document}
