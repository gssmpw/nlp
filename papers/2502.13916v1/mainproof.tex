
% !TEX root = main.tex

\section{Proof of Lemma~\ref{lem:main}}\label{sec:mainproof}


In this section we prove Lemma \ref{lem:main}, 
%namely  we show that $k$-component \tvass are \lb by $\kbound {(\_)} {\OO(k)}$,
by induction on $k$.
The base of induction, when $k=1$, follows by Lemma \ref{lem:1comp}:
 1-component \tvass are \plb. % by $(\_)^{\OO(1)}$.
Before engaging in the induction step
we need to generalise wideness, defined up to now for 1-component \tvass only,
to all sequential \tvass.


\para{Sequential cones}

%%Let $E_i$ be the set of effects of simple cycles in $V_i$.
%We define the \emph{sequential cone} of a 
Consider a $k$-component \tvass $V=\ktvass$.
By a \emph{cascade} we mean a tuple of $k$ vectors $(v_1, \ldots, v_k)$  
such that the partial sum $v_1 + \ldots + v_i \in (\Qpos)^3$ for every $i\in\setto k$.
Then the \emph{sequential cone} of $V$, denoted $\seqcone V$, is the set of 
sums of all cascades $(v_1, \ldots, v_k)$ 
whose every $i$th vector $v_i$ belongs to $\cone{V_i}$: 
\[
\seqcone V = \{v_1 + \ldots + v_k \mid (v_1, \ldots, v_k) \in \cone{V_1} \times \ldots \times \cone{V_k} 
\text{ is a cascade}\}.
\]
\vspace{-0.7cm}
%
\begin{claim} \label{claim:seqcone}
If $\Lin V$ is 3-dimensional then $\seqcone V$ is a finitely generated open cone.
\end{claim}
%
\begin{appendixproof}[Proof of Claim \ref{claim:seqcone}]
$\seqcone V$ is equivalently definable as the last element $C_k$ of the sequence
of (rational) open cones $C_1, \ldots, C_k$, defined as follows.
We put $C_1 := \cone{V_1}\cap (\Qpos)^3$,
and for $i > 1$ we define inductively:
\[
C_{i} \ := \ \big(C_{i-1}\  + \ \cone{V_{i}}\big) \cap (\Qpos)^3,
\]
where the addition is Minkowski sum $X+Y = \setof{x+y}{x\in X, y\in Y}$.
Then all $C_1, \ldots, C_k$ are finitely generated open cones, as $(\Qpos)^3$ is such
a cone,  
and Minkowski sum and intersection preserve finitely generated open cones.
\end{appendixproof}
%
We prove the fundamental property:
all reachable configurations are at close distance to the sequential cones.
We focus on \tvass, but actually the same proof works for \vass in any other fixed dimension.
Below, let $d(x,y)$ denote Euclidean distance between $x$ and $y$, and let $d(x,S)$ denote 
the distance between $x$ and a set $S$, that is $d(x,S) = \inf \setof{d(x,t)}{y \in S}$. 
%
\begin{lemma}\label{lem:not_far_from_cone}
%For each $d \in \N$ t
There exists a nondecreasing polynomial $P$ such that 
each reachable configuration $q(w)$
in a forward-diagonal sequential \tvass $(V,s)$,  satisfies
$d(w, \seqcone V) \leq P(\size(V,s))$.
\end{lemma}

\begin{proof}
Let $V$ be a $k$-component sequential \tvass, and $M:=\size(V, s)$.
Let $s=p_1(w)$ and suppose $s \trans{\pi} q(x)$.
\Wlog we assume that $q(x)$ is in the last component $V_k$; indeed, if $q(x)$ is in $V_\ell$ for some
$\ell < k$, we prove that claim for $V'=\ltvass$ and rely on the inclusion
$\seqcone {V'} \subseteq \seqcone V$.
Our aim is to define a polynomial $P$ and a point $y \in \seqcone{V}$ such that $d(x,y) \leq P(M)$.
The path $\pi$ decomposes into $\pi = \pi_1;\, u_1;\, \ldots; \, \pi_{k-1}; \, u_{k-1}; \, \pi_k$ where 
$u_1, u_2, \ldots, u_{k-1}$ are bridge transitions.
Let $p_i$ be the source state of $\pi_i$ and $p'_i$ be the target state of $\pi_i$.
Let $\sigma_i$ be the shortest path from $p'_i$ to $p_i$ 
(there is such a path because they are in the same strongly connected component $V_i$). 
Let $v_i \in \Z^3$ be the effect of $\pi_i;\, \sigma_i$, which is a cycle in $V_i$. 

As $\cone{V_i}$ is an open cone, we are not guaranteed that $v_i \in \cone{V_i}$. 
In order to ensure this property, we add to $v_i$ some small multiples of all the simple cycles in $V_i$.
For each $i \in \setto k$, let $c_i\in\Z^3$ 
be the sum of effects of all simple cycles in $V_i$ and let 
$\eps \in \Q_{>0}$ be a small positive rational such that for all $i \in \setto k$ we have $\norm(\eps \cdot c_i) < 1$. 
Then $v_i + \eps \cdot c_i \in \cone{V_i}$.

By forward-diagonality, the configuration $p_1(w+\vec 1)$ is coverable in $V_1$ from $s$.
Due to the upper bound of Rackoff~\cite[Lemma 3.4]{DBLP:journals/tcs/Rackoff78}, instantiated to the
fixed dimension 3,
for some nondecreasing polynomial $R$ the length of such a covering path, and the
norm of $\Delta$, are both at most $R(M)$.
For some $m \in \N$, the vector $m \cdot \Delta + v_1 + \eps \cdot c_1$ 
belongs to $\cone{V_1}$ and the $k$-tuple
\begin{align} \label{eq:cascade}
(m \cdot \Delta + v_1 + \eps \cdot c_1, \ v_2 + \eps \cdot c_2, \ \ldots, \ v_k + \eps \cdot c_k)
\end{align}
is a cascade.
Therefore the sum $y = m \cdot \Delta + \Sigma_{j=1}^k (v_j + \eps\cdot c_j) \in \seqcone V$.
%
We argue that is is enough to use polynomially bounded value of $m$.
Since $\pi$ starts in $s=p_1(w)$, each its prefix can drop by at most $\norm(w) \leq M$ on any coordinate.
Thus, for each $j \in \setto k$ we have 
$\eff(\pi_1;\, u_1;\, \ldots; \, \pi_{k-1}; \, u_{j-1}; \, \pi_j) \geq -\vec{M}$.
As $u_i$ are single transitions, we also have
$\eff(u_1) + \ldots + \eff(u_{j-1}) \leq \vec{M}$. 
In consequence, $\eff(\pi_1) + \ldots + \eff(\pi_j) \geq -2\vec{M}$.
Moreover, the sum of norms of effects of all the paths $\sigma_j$ is at most $-M$, 
which implies that 
$v_1 + \ldots + v_{j} \geq -3\vec{M}$. 
Finally,
for each $j \in \setto k$ we have 
$\norm(\eps \cdot (c_1 + \ldots + c_j)) \leq j \leq M$. 
Therefore, setting $m = 4M$ guarantees that
the tuple \eqref{eq:cascade} is a cascade.

%Let us set $P_d(M) = 4M + 3M \cdot R_d(M)$. 
Let $C = \eps \cdot (c_1 + \ldots + c_k)$, 
$S = \eff(\sigma_1) + \ldots + \eff(\sigma_k)$ and $U=\eff(u_1) + \ldots + \eff(u_k)$. 
We have $\norm(C), \norm(S), \norm(U) \leq M$, and
$y = x + m\cdot \Delta + S + C - U$.
Therefore $\delta = y - x$ satisfies $\norm(\delta) \leq P(M) := 4M\cdot R(M) + 3M$, and we get the bound
\begin{align*}
d(x, y)^2 \ = \ d(x, x + \delta)^2 \ = \ \innprod \delta \delta \ \leq \ \norm(\delta)^2  
% & = d(x, \eff(\pi) - U + S + C + m \cdot \Delta) = d(x,x-w-U+S+C+m\Delta) \\
%& \leq \norm(w) + \norm(U) + \norm(S) + \norm(C) + m \cdot \norm(\Delta) \\
%& \leq 4M + 3M \cdot R_d(M) \leq P_d(M).
\end{align*}
that implies $d(x,y) \leq \norm(\delta)$, and hence $d(x,y) \leq P(M)$ as required.
\end{proof}
%
We say that a $k$-component \tvass $V$ is \emph{wide} if $(\Qpos)^3 \subseteq \seqcone V$
or $(\Qpos)^3 \subseteq \seqcone {\rev V}$.
For $k=1$, the definition relaxes the definition of Section \ref{sec:1comp}. 



\begin{proof}[Proof of Lemma \ref{lem:main}]
%
Lemma \ref{lem:trueind}, formulated below, is a refinement of Lemma \ref{lem:main} suitable for 
inductive reasoning.
When proving the lemma, we distinguish 2 cases, depending on whether
a \tvass  is diagonal and wide (we call the \tvass \emph{easy} in this case), 
or not (we call it \emph{non-easy} then), and rely on the following two facts:
%diagonal and non-wide, or non-diagonal.
%
\begin{lemma} \label{lem:dw}
Easy sequential \tvass are \lb by $P_k(M) = M^{\OO(k)}$, where $k$ is the number of components.
\end{lemma}
%
%On the other hand, the induction step is due to:
%
\begin{lemma} \label{lem:ne}
There is a nondecreasing polynomial $F$ such that for every $k > 1$,
if $(k-1)$-component \tvass are \lb by a function $h$ then
non-easy $k$-component \tvass are \lb by the function $H \circ h \circ H \circ h \circ H$.
\end{lemma}
%
For stating Lemma \ref{lem:trueind},
we define inductively a sequence of polynomials $(h_i)_{i\in\N}$, where
$h_1$ is the polynomial witnessing Lemma \ref{lem:1comp}
(thus 1-component \tvass are \lb by $h_1$), and
\[
h_{j+1} = H \circ h_j \circ H \circ h_j \circ H.
\]
\Wlog we also assume that $h_j$ dominates the polynomial $P_j$ of Lemma \ref{lem:dw}, namely
$P_j(m) \leq h_j(m)$ for every $j,m\geq 1$.

\begin{lemma} \label{lem:trueind}
For every $k\geq 1$, 
$k$-component \tvass are \lb by $h_k$.
\end{lemma}
%
\begin{proof}
The induction base is given by Lemma \ref{lem:1comp}.
For the induction step, suppose $(k-1)$-component \tvass are \lb by $h_{k-1}$.
Easy $k$-component \tvass are \lb by $h_k$ due to Lemma \ref{lem:dw} and the above-assumed domination, 
without referring to induction assumption,
while non-easy $k$-component \tvass are \lb by $h_k$ due to Lemma \ref{lem:ne}.
\end{proof}
%
%\Wlog we assume that the first and the last component are non-trivial:
%%
%\begin{lemma} \label{lem:wlog}
%\Wlog we may assume that $T_1 \neq \emptyset$ and $T_k\neq \emptyset$.
%\end{lemma}
%%
%\begin{appendixproof}[Proof of Lemma \ref{lem:wlog}]
%Indeed, if $T_1 = \emptyset$ then one can remove the first component and the bridge transition 
%$u_1 = (p'_1, \delta_1, p_2)$,
%by setting the source configuration to $p_2(w+\delta_1)$ in $V_2$. 
%If $T_k = \emptyset$, one proceeds similarly.
%\end{appendixproof}
%
%By $(k-1)$-fold application of Lemma \ref{lem:ne}, the $k$-component \tvass are \lb by $h_k$.
Let $c\in\N$ be large enough so that $H(m), h_1(m) \leq m^{c}$ for every $m>1$,
and let $C=c^3$.
%
\begin{claim} \label{claim:kkk}
$h_{k}(m) \leq {m}^{C^{2^{k} -1}}$ for all $m > 1$.
\end{claim}
%
\begin{appendixproof}[Proof of Claim \ref{claim:kkk}]
The inequality is shown easily by induction on $k$.
When $k=1$, by definition of $c$ we have $h_1(m) \leq m^c\leq m^C$, as required.
Furthermore, assuming
$h_{k-1}(m) \leq {m}^{C^{2^{k-1} -1}}$ for all $m>1$, we get
\[
h_k(m) \ \leq \ {m}^{C\cdot (C^{2^{k-1} -1})^2} 
\ = \ m^{C^{2^k-1}},
\]
as required.
\end{appendixproof}
%
Lemma \ref{lem:trueind} implies Lemma \ref{lem:main}, 
as the right-hand side of the inequality in Claim \ref{claim:kkk} is bounded by $\kbound {m} {\OO(k)}$.
Lemma \ref{lem:main} is thus proved (once we prove Lemmas \ref{lem:dw} and \ref{lem:ne}).
\end{proof}
%\medskip

The proof of Lemma \ref{lem:dw} generalises Case 1 of the proof of Lemma \ref{lem:1comp}.
The proof of Lemma \ref{lem:ne} makes crucial use of \sandwich sets introduced
in Section \ref{sec:tools}, and builds on 
Lemma \ref{lem:len-eq}, stated below, whose proof 
generalises Cases 2 and 3 of the proof of Lemma \ref{lem:1comp}.


\begin{appendixproof}[Proof of Lemma \ref{lem:dw}]
%
%\slawek{could go to appendix}
Consider a $k$-component \tvass  $V = \ktvass$,
where $V_i = (Q_i, T_i)$ and $u_i = (p'_i, \delta_i, p_{i+1})$,
together with source and target configurations: 
$s=p_1(w)$ in $V_1$ and $t=p'_k(w')$ in $V_k$.
Let $M = \size(V, s, t) = \size(V) + \norm(s) + \norm(t)$.

Suppose $(V, s, t)$ is diagonal and wide, say $(\Qpos)^3 \subseteq \seqcone{V}$.
%
We have
$p_1(w) \trans\pi p_1(w+\Delta)$ and
$p'_k(w'+\Delta')\trans{\pi'} p'_k(w')$  for some $\Delta,\Delta'\in(\Npos)^3$, and 
$(\Qpos)^3 \subseteq \seqcone V$.
%

Let $P$ be a nondecreasing polynomial witnessing Lemma \ref{lem:zvass-plb}, i.e.,
\tzvass are \lb by $P$.
As $V$ has a path $s \tran t$, it also has a $\Z$-path $s\tran t$.
By Lemma \ref{lem:zvass-plb}, $V$ has a $\Z$-path $s \trans{\sigma} t$ of length at most $P(M)$.
The $\Z$-path factorises into components:
%
\begin{align} \label{eq:Zpath}
\sigma \ = \ 
\sigma_1;\, u_1; \, \ldots; \, \sigma_{k-1}; \, u_{k-1}; \, \sigma_k.
\end{align}
%
As in Case 1 of the proof of Lemma \ref{lem:1comp},
let $R$ be a nondecreasing polynomial  such that in every \tvass of size $m$, the length of a covering
path is at most $R(m)$~\cite[Lemma~3.4]{DBLP:journals/tcs/Rackoff78}.
%
We generalise Lemma \ref{lem:FP} and prove that certain multiplicity of $\Delta'$
may be obtained by executing first a cycle in $V_1$, then a cycle in $V_2$, and so on,
and finally a cycle in $V_k$, so that the total effect of the first $j$ cycles is in $(\Npos)^3$,
for every $j\in\setto k$,
and the lengths of all the cycles are bounded by a polynomial of degree $\OO(k)$:
%
\begin{lemma} \label{lem:FPgen}
There is an integer cascade $(\Delta'_1, \ldots, \Delta'_k)$ and $\ell\in\Npos$ such that
$\Delta'_1 + \ldots + \Delta'_k = \ell\cdot\Delta'$, 
and for $j\in\setto k$ there are paths
\[
p_j(w+\Delta'_1 + \ldots + \Delta'_{j-1}) \trans {\pi_j}  p_j(w+ \Delta'_1 + \ldots + \Delta'_{j})
\]
in $V_j$
of length $R(M)^{\OO(k)}$.
\end{lemma}
%
\begin{proof} %[Proof of Lemma \ref{lem:FPgen}]
%By forward diagonality, $p(w) \trans{\pi} p(w+\Delta)$ for some $\Delta\in(\Npos)^3$.
Let $\rho_j$ be a cycle in $V_j$ that visits all states of $V_j$, and
let $\Delta_j\in\Z^3$ be its effect, for $j\in\setto k$.
We have thus $\Z$-paths:
\[
p_j(\Delta_1 + \ldots + \Delta_{j-1}) \trans{\rho_j} p_j(\Delta_1 + \ldots + \Delta_j).
\]
%
%There is such a cycle of length at most $\OO(M^2)$.
%
%For $m\in \Npos$, let $\pi^m$ denote $m$-fold concatenation of $\pi$:
%$p(w) \trans{\pi^m} p(w+m\Delta)$.
%
Relying on $\Delta\in(\Npos)^3$,
take a sufficiently large multiplicity $m\in\Npos$ so that 
the $\Z$-paths become paths:
%
\begin{align} \label{eq:mDe}
p_j(m\cdot \Delta + \Delta_1 + \ldots + \Delta_{j-1}) \trans{\rho_j} p_j(m\cdot \Delta + \Delta_1 + \ldots + \Delta_j).
\end{align}
%
In particular, the tuple
$(m\cdot\Delta + \Delta_1, \Delta_2, \ldots, \Delta_k)$ becomes a cascade.
Let 
$\widetilde \Delta = m\cdot\Delta + \Delta_1 + \Delta_2 + \ldots + \Delta_k \in \N^3$
be the sum of the cascade.
%is a nonnegative integer combination
%of effects of simple cycles:
%\[
%\Delta_2 = s_1 \cdot e_1 + \ldots + s_n \cdot e_n \in \cone V \cap \N^3,
%\] 
%where $s_1, \ldots, s_n \in \N$.
%
As $\Delta'\in(\Npos)^3$, there is $\ell'\in\Npos$ such that 
$\ell'\cdot\Delta'-\widetilde \Delta \in (\Qpos)^3$,
and hence, by wideness of $(V, s)$, we have
$\ell'\cdot\Delta'-\widetilde \Delta \in \seqcone V$, namely
\[
\ell'\cdot\Delta' - \widetilde \Delta \ = \ s_1  +  \ldots  +  s_k
\]
is the sum of a cascade $(s_1, \ldots, s_k)$, where
\begin{align} \label{eq:presystk} 
\begin{aligned}
s_1 \ =  \ & \ r_{1,1} \cdot e_{1,1} + \ldots + r_{1,n_1} \cdot e_{1,n_1} \\
 & \ \ldots  \\
s_k \ = \ & \ r_{k,1} \cdot e_{k,1} + \ldots + r_{k,n_k} \cdot e_{k,n_k}.
\end{aligned}
\end{align}
for some positive rational coefficients $r_{1,1}, \ldots, r_{1,n_1}, \ldots, r_{k, 1}, \ldots, r_{k,n_k}\in \Qpos$,
where vectors $e_{j,1}, \ldots, e_{j,n_j}\in\Z^3$ are effects of simple cycles in $V_j$, for $j\in\setto k$. Denote by $\rhs_j$ the $j$th right-hand side expression in \eqref{eq:presystk}.
%
Therefore, the system $\cal S$ consisting of the equation
\begin{align} \label{eq:eqk}
\ell \cdot (\ell'\cdot\Delta'-\widetilde\Delta)  \ = \ \rhs_1  +  \ldots +  \rhs_k
\end{align}
together with the $k$ equations
\begin{align} \label{eq:systk}
\begin{aligned}
\ell_1 \ = \ & \ \rhs_1 \\
\ell_2 \ = \ & \ \rhs_1  +  \rhs_2 \\
& \ldots \\
\ell_k \ = \ & \ \rhs_1  +  \ldots  +  \rhs_k,
\end{aligned}
\end{align}
with unknowns $\ell, \ell_1, \ldots, \ell_k, r_{1,1}, \ldots, r_{k,n_k}$, has a positive integer solution. We rewrite the equation \eqref{eq:eqk} to:
\begin{align} 
\begin{aligned}
\label{eq:syst}
\ell \ell'\cdot \Delta'  \ = \ \ \ell m \cdot \Delta +   (\ell\cdot \Delta_1 + \rhs_1) + \ldots + (\ell \cdot \Delta_k + \rhs_k).
\end{aligned}
\end{align}
%
%For sufficiently large $\ell\in\Npos$,
% enough so that the multiplicites of coefficients $\ell_1\cdot r_i$ are integers,
%for $i\in\setto n$.
%Furthermore, picking up a sufficiently large multiplicity $\ell_1$ yields
%$\ell_1 \cdot r_i \geq s_i$ for every $i\in\setto n$.
%Summing up, for some $r_1, \ldots, r_n \in \N$ we have
%\[
%\ell_1\cdot v = \Delta_2 + r_1 \cdot e_1 + \ldots + r_n \cdot e_n \in \cone{V} \cap (\Npos)^3.
%\]
Irrespectively of the value of $\ell$,
the tuple 
$(\ell m\cdot\Delta + \ell\cdot\Delta_1, \ell\cdot\Delta_2, \ldots, \ell\cdot\Delta_k)$ is still a cascade,
and due to $\ell_j>0$ in \eqref{eq:systk} the tuple
\[
(\widetilde\Delta_1, \ldots, \widetilde\Delta_k) \ := \ (\ell m\cdot\Delta + \ell\cdot\Delta_1 + \rhs_1, \ell\cdot\Delta_2 + \rhs_2, \ldots, \ell\cdot\Delta_k+\rhs_k)
\]
is a cascade as well.
Let $r_j = r_0 + \ell\cdot(\Delta_1 + \ldots + \Delta_j) + \rhs_1 + \ldots + \rhs_j$
denote its $j$th partial sum, for $j\in\setto k$, where $r_0 = \ell m \cdot \Delta$.
%
%and therefore
%the $\ell$-fold iteration of each path \eqref{eq:mDe} is still a path:
%%
%\begin{align} \label{eq:lmDe}
%p_j(\ell m\cdot \Delta + \ell\cdot(\Delta_1 + \ldots + \Delta_{j-1})) \trans{(\rho_j)^\ell} 
%p_j(\ell m\cdot \Delta + \ell\cdot(\Delta_1 + \ldots + \Delta_j)).
%\end{align}
%Indeed, the first and the last iteration of $\rho_j$ is a lifting of the path \eqref{eq:mDe},
%and all points visited in the inner iterations of $\rho_j$ are bounded 
%from both sides by corresponding points visited in the first and the last iteration.
%
Let $\sigma_{j,i}$ be a simple cycle of effect $e_{j,i}$ in $V_j$.
Let $\sigma_j$
be a $\Z$-path in $V_j$ that starts (and ends) in state $p_j$ and consists of the  
$\ell$-fold concatenation of the cycle $\rho_j$, with attached 
$(r_{j,i})$-fold concatenation of each $\sigma_{j,i}$, for $i\in \setto{n_j}$
(since $\rho_j$ visits all states, this is possible):
%
\begin{align} \label{eq:sigmajgen}
p_j(r_{j-1}) \trans{\sigma_j} p_j(r_j).
\end{align}
%
The sum of effect of $\sigma_1, \ldots, \sigma_k$, plus $r_0$, 
yields the right-hand side of \eqref{eq:syst}.
% and therefore
%\[
%p(w + \ell m \cdot \Delta) \trans{\sigma} p(w+\ell \ell'\cdot \Delta').
%\]
Each of the paths starts and ends in $(\Npos)^3$ but may pass through non-positive points, and therefore it
needs not be a path.
%
Let $k\in\Npos$ be a multiplicity large enough so that for every $j\in \setto k$, 
the $\Z$-path \eqref{eq:sigmajgen} becomes a path
when lifted by $(k-1) \cdot r_{j-1}$,
i.e., when starting in $p_j(k \cdot r_{j-1})$, 
and also becomes a path when lifted by $(k-1)\cdot r_j$, i.e., when
ending in 
$p_j(k \cdot r_j)$.
%
In this case, the $k$-fold concatenation of each $\sigma_j$ is also a path:
%
\[
p_j(k\cdot r_{j-1}) \trans{(\sigma_j)^k} p_j(k \cdot r_j),
\]
since all points visited in the inner iterations of $\sigma_j$ are bounded 
from both sides by corresponding points visited in the first and the last iteration of $\sigma_j$.
Therefore, the lifting of $(\sigma_j)^k$ by the source vector $w$ is also a path:
%
\begin{align} \label{eq:thepathgen}
p_j(w+k\cdot r_{j-1}) \trans{(\sigma_j)^k} p_j(w+k \cdot r_j).
\end{align}
%
Relying on \eqref{eq:thepathgen},
we define the cascade
\[
(\Delta'_1, \ldots, \Delta'_k) \ := \ (k \cdot \widetilde\Delta_1, \ldots, k \cdot \widetilde\Delta_k),
\]
and cycles $\pi_1, \ldots, \pi_k$:
the cycle $\pi_1$ is  $(\sigma_1)^k$ precomposed with the $(k\ell m)$-fold iteration
of $\pi$,
\[
p_1(w) \trans{\pi^{k\ell m}} p_1(w + k\cdot r_0) \trans{(\sigma_1)^k} p_1(w + \Delta'_1),
\]
%\[
%p_1(w) \trans{\pi_1} p_1(w + \Delta'_1),
%\]
and for $j\in\setfromto 2 k$, the cycle $\pi_j$ is $(\sigma_j)^k$,
\[
p_j(w + \Delta'_1 + \ldots + \Delta'_{j-1}) \trans{(\sigma_j)^k}
p_j(w + \Delta'_1 + \ldots + \Delta'_{j}),
\]
as required.


The estimations of the length of the paths $\pi_j$ are similar as in the proof of Lemma \ref{lem:FP},
so we focus only on new aspects.
The number of different effects of cycles in each component is at most $(2M+1)^3\leq \OO(M^3)$, 
and therefore
the number of unknowns $r_{j,i}$ in $\cal S$
is at most $k\cdot (2M+1)^3 \leq \OO(M^4)$, and consequently so is the 
total number of unknowns in $\cal S$.
The norm of a solution of the system $\cal S$ can now be bounded, due to Lemma \ref{lem:taming},
by $D = R(M)^{\OO(k)}$.
In consequence, we get the same bound $R(M)^{\OO(k)}$ on lengths of paths $\pi_j$.
This completes
the proof of Lemma  \ref{lem:FPgen}.
\end{proof}

We now use Lemma \ref{lem:FPgen} to complete the proof of Lemma \ref{lem:dw}.
Note that $\ell$ in Lemma \ref{lem:FPgen} is necessarily also bounded by $R(M)^{\OO(k)}$, and that
for every $m\in\Npos$ the $m$-fold iteration of the cycle $\pi_j$ is also a path:
%
\begin{align}\label{eq:mfold}
p_j(w+m\cdot(\Delta'_1 + \ldots + \Delta'_{j-1})) \trans {(\pi_j)^m} p_j(w+ m\cdot(\Delta'_1 + \ldots + \Delta'_{j})).
\end{align}
%
We pick an $m\in \Npos$ and 
build a path $\rho$
by interleaving the $\Z$-path \eqref{eq:Zpath} with
$m$-fold iterations of the cycles $\pi_1, \ldots, \pi_k$ of Lemma \ref{lem:FPgen}:
% $p'_k(w'+\Delta')\trans{\pi'} p'_k(w')$:
\[
\rho \ = \ (\pi_1)^m;\, \sigma_1;\ u_1; \ \ldots; \ \ (\pi_{k-1})^m;\, \sigma_{k-1};\ u_{k-1}; \ (\pi_k)^m;\, \sigma_k;  %\, (\pi')^{m\ell}.
\]
As the effect of $(\pi_j)^m$ is $m\cdot \Delta'_j$, and the effect of $\sigma$ is $w'-w$,
we have:
\[
p_1(w) \trans{\rho} p'_k(w'+m\ell\cdot \Delta').
\]
We choose $m$ sufficiently large to enforce that each of $\Z$-paths $\sigma_j$
becomes a path, and hence the whole $\rho$ is a path as well.
It is enough to take 
 $m = M\cdot P(M)$, which
makes the length of $\rho$ bounded by $P(M) \cdot R(M)^{\OO(k)}$.
%
%The multiplicity guarantees that the $\Z$-path 
%$
%p(w) \trans{\sigma} p'(w'), 
%$
%lifted by $\ell \cdot \Delta'$, becomes a path:
%\[
%p(w+\ell \cdot \Delta') \trans{\sigma} p'(w' + \ell\cdot\Delta'), 
%\]
%The length of $\sigma$ is bounded by $P(M)$.
%Finally, let $\delta' = (\pi')^\ell$ be the $\ell$-fold concatenation of
%the cycle $\pi'$:
%\[
%p'(w'+\ell\cdot\Delta') \trans{\delta'} p'(w').
%\]
%The length of this path is bounded by $\ell\cdot R(M) \leq P(M)\cdot R(M)^{\OO(1)}$.
%We concatenate the three paths, $\tau := \delta; \, \sigma;\, \delta'$, to get a required path
%\[
%p(w) \trans{\tau} p'(w')
%\]
%of length bounded by $P(M) \cdot R(M)^{\OO(1)}$.
%It thus remains to prove Lemma \ref{lem:FP} in order to comgenplete Case 1.
%
Finally, we concatenate $\rho$ with the $m\ell$-fold iteration of the path $\pi'$,
\[
p'_k(w' + m\ell\cdot \Delta') \trans{(\pi')^{m\ell}} p'_k(w'),
\]
to get the required path $\rho;\, (\pi')^{m\ell}$
from $p_1(w)$ to $p'_k(w')$
of length bounded by $M\cdot P(M) \cdot R(M)^{\OO(k)} \leq
(M\cdot P(M)\cdot R(M)^{\OO(1)})^k \leq M^{\OO(k)}$.
%
This completes the proof of Lemma \ref{lem:dw}. %, once we prove Lemma \ref{lem:FPgen}.
\end{appendixproof}


For stating and proving Lemma \ref{lem:len-eq} we need a variant of sequential \tvass:
a \emph{good-for-induction} $k$-component \tvass $V=\ktvass$ is defined exactly like $k$-component
sequential \tvass, except that the first component $V_1$ is an arbitrary \geomvass,
not necessarily being strongly connected.
%
\begin{lemma} \label{lem:len-eq}
There is a nondecreasing polynomial $R$ such that 
every non-easy $k$-component \tvass $(V, s, t)$ is length-equivalent to a finite set $S$ of good-for-induction
$k$-component \tvass
% or reverses of such \tvass, 
of size at most $R(\size(V,s,t))$, namely
$\Len V s t = \bigcup_{(V', s', t')\in S} \Len {V'} {s'} {t'}$.
\end{lemma}
%

\begin{appendixproof}[Proof of Lemma \ref{lem:len-eq}]
Consider a non-easy $k$-component \tvass $V = \ktvass$, together with
source and target configurations $s = p_1(w)$ and $t=p'_k(w')$.
If $V_1$  is a \geomvass, there is nothing to prove as $V$ is good-for-induction.
If $\rev{(V_k)}$ is a \geomvass then we are done too, as $\rev V$ is good for induction and 
$\Len V s t = \Len{\rev V} t s$.
Therefore we assume from now on that $V_1$ and $\rev{(V_k)}$ are of geometric dimension 3. 
In consequence, by Claim \ref{claim:seqcone}, all of $\cone {V_1}, \seqcone V$, $\cone{\rev{(V_k)}}$, $\seqcone{\rev V}$ are
3-dimensional open cones.
We distinguish two cases, and hence the polynomial $R$ is the sum of polynomials claimed
in the respective cases.

\para{Case I: $(V, s, t)$ is non-diagonal}
%
We may assume \mywlog that $V$ is non-forward-diagonal (otherwise replace $V$ by $\rev V$),
and therefore $V_1$ is so.
Exactly as in Case 3 of the proof of Lemma \ref{lem:1comp}, we transform
$(V_1,s)$ into
three \geomvass $(\essdvass V_1, s_1), (\essdvass V_2, s_2), (\essdvass V_3, s_3)$.
In each $(\essdvass V_i, s_i)$, we replace the target state $p'_1$ by
$\pair {(p'_1)} b$, for an arbitrarily chosen value $b\in\setfromto 0 B$ of the bounded coordinate,
and modify accordingly the first bridge transition $u_1 = (p'_1, \delta_1, p_2)$ to 
$\essdvass u_{i, b} = (\pair{(p'_1)} b, \delta_1, p_2)$.
This yields a set $S$ of $3(B+1)$ good-for-induction $k$-component \tvass  
\[
S \ = \ \setof{\big(\ktvassmod, s_i, t \big)}{i\in\setto 3, \ b\in \setfromto 0 B},
\]
which is length-equivalent to $(V, s, t)$, as required. 
The size of each of these \tvass is
at most $R(M)$, as in Claim \ref{claim:nondiagsize} in Case 3 of the proof of Lemma \ref{lem:1comp}.

\para{Case II: $(V, s, t)$ is non-wide}
%
We proceed similarly to Case 2 of the proof of Lemma \ref{lem:1comp}, and transform
$(V_1,s)$ into a \geomvass $(\essdvass V, \essdvass s)$, defined as a \mytrim {$a$} {$B$} of $V_1$
for some vector $a\in\Z^3$ and $B\in\N$.
To this aim we need an analog of Claim \ref{claim:ax} (Claim \ref{claim:axgen} below).
%
\begin{claim}\label{claim:empty_intersection}
$C_1 :=  \cone{V_1}$ and $S := \seqcone{\rev V}$ are disjoint.
\end{claim}

\begin{proof}
Let $S'=\seqcone {\rev{(V')}}$, where $V'=\ktvasstwo$ is $V$ without the first component and the first bridge.
By definition, 
$
S = (S' + \cone{\rev{(V_1)}})\cap(\Qpos)^3,
$
but since $\cone{\rev{(V_1)}} = -C_1$, we get
\[
S = (S' - C_1)\cap(\Qpos)^3.
\]
By Claim \ref{claim:seqcone} the set $S'$ is an open cone, and hence
$S' - C_1$ is also so, as Minkowski sum preserves such cones.
By definition $S' - C_1$ contains, for every vector $v\in C_1$, 
a vector $\varepsilon-v$ for some vector $\varepsilon\in S'$ of arbitrarily small norm $(*)$.

Towards a contradiction, suppose $C_1 \cap S$ is nonempty.
Therefore, $S$, and hence also $S'-C_1$ contains some vector $v\in C_1$.
Being an open cone, it also contains $v - \varepsilon$ for every vector $\varepsilon$ of sufficiently small norm
$(**)$.
The two properties $(*)$ and $(**)$,
\[
(*) \ \ \ \prettyforall {v\in C_1} \prettyforall {N\in\Q} \prettyexists {\varepsilon, \norm(\varepsilon)<N} 
(\varepsilon - v \in S'-C_1)
\qquad
(**) \ \ \ \prettyexists {v\in C_1} \prettyexists{N\in\Q} \prettyforall {\varepsilon, \norm(\varepsilon)<N} (v- \varepsilon  \in S'-C_1),
\]
imply that $S'-C_1$ contains both $v-\varepsilon$ and $\varepsilon-v$, for some
$v\in C_1$ and $\varepsilon\in\Q^3$,
and hence $S'-C_1$
includes a line.
As $S-C_1$ is an open cone, we deduce $S'-C_1=\Q^3$ and  $S=(\Qpos)^3$, which means that 
$V$ is wide, a contradiction.
%
%$v \in C_1 \cap \seqcone{\rev V}$. 
%As both cones are open three dimensional cones there are three linearly independent vectors 
%$\Delta_1, \Delta_2, \Delta_3 \in \Q^3$ such that for $i \in \setto 3$ it holds 
%$v + \Delta_i \in \cone{V_1} \cap \seqcone{\rev V}$. 
%
% be sequential cone of $\rev V$ without the last component (that is also first component of $V$). There exists $\epsilon \in C$ such that also for $i \in \set{1,2,3}$ we have $v + \Delta_i + \epsilon \in \cone{V_1} \cap{\rev V}$. Observe, that $\seqcone{\rev V} = (C + \cone{\rev V_1}) \cap \Q^3_{>0} = (C - \cone{V_1}) \cap \Q^3_{>0}$.
%Hence also for each $i \in \set{1,2,3}$ we have that $\epsilon - v - \Delta_i - \epsilon = -v - \Delta_i \in C - \cone{V_1}$.
%Observe, that we $\seqcone{\rev V} \subset C - \cone{V_1}$. We have thus for $i \in \set{1,2,3}$ vectors
%$\alpha_i = v + \Delta_i$, which are linearly independent and for which both $\alpha_i$ and $-\alpha_i$ belongs to $C-\cone{V_1}$ and hence $C-\cone{V_1} = \Q^3$, which implies $\seqcone{\rev V}$ being wide, which contradicts the assumption of it being non-wide.  
\end{proof}
%
%By disjointness of $C_1$ and $S$, due to Claim \ref{claim:empty_intersection},  we deduce:
%
\begin{claim} \label{claim:onefacet}
Let $C\subseteq \Q^3$, $C'\subseteq (\Qpos)^3$ be 3-dimensional disjoint open cones, and $D\in\Qpos$. 
%For some two adjacent facets $F_1$, $F_2$ of $C$, 
All points whose distance to both cones is at most $D$,
are at distance at most $3D$ to one of facet planes of $C$.
%$F_1$ or $F_2$.
\end{claim}
%
\begin{proof}%[Proof (sketch)]
We give a geometric argument.
Let $S$ be any plane \emph{separating} $C$ and $C'$, namely the two cones are on the opposite sides of $S$.
Since $C'\subseteq(\Qpos)^3$ is included the positive quadrant,
the plane $S$ may be chosen to be adjacent to $C$, namely to satisfy one of the following conditions:

\begin{itemize}
\item[(1)] $S$ includes a facet $F$ of $C$, or
\item[(2)] $S$ includes an edge of $C$ (adjacent to two facets $F_1$ and $F_2$).
\end{itemize}

\noindent
Consider an arbitrary point $x\in\Q^3$ such that $d(x, C)\leq D$ and $d(x, C')\leq D$.
Therefore $d(x,S) \leq D$, as $C$ and $C'$ are on opposite sides of $S$.
Let $x'\in S$ be the point in $S$ which is the closest to $x$.
%
In case (1), we have $d(x,S)\leq D\leq 3D$, i..e, $x$ is at distance at most $D$ to the facet plane $S$.
In case (2), let $H_1, H_2$ be the planes including $F_1$, $F_2$, respectively.
Since $d(x, C)\leq D$ and $d(x,S) \leq D$, we deduce that $d(x,H_1)\leq D$ or $d(x,H_2)\leq D$.
\Wlog we assume that the angle between $H_1$ and $S$ is at most as large as the angle between $H_2$ and $S$,
and aim at showing $d(x, H_1) \leq 3D$.
If $d(x, H_1)\leq D$, we are done.
Otherwise $d(x, H_2) \leq D$, and hence by the triangle inequality we get
$d(x', H_2) \leq d(x', x) + d(x, H_2) \leq 2D$.
Since the angle between $H_1$ and $S$ is not larger than the angle between $H_2$ and $S$, 
we deduce $d(x', H_1) \leq d(x', H_2) \leq 2D$, which implies
$d(x, H_1)\leq d(x, x') + d(x', H_1) \leq 3D$, as required.
%
%taking as $H_a$ the hyperplane including either the facet of $C$, or one 
%then 
%
%all points in $Y_C$ satisfy one of inequalies \eqref{eq:D'}, say the first one, 
%and all points in $Y_{C'}$ satisfy the other one. 
%Therefore, all points in $Y_C \cap Y_{C'}$ are at distance at most $D$ to $H_a$.
%
%If $C$ and $C'$ are not adjacent, transform $C'$ isometrically into $C''$ adjacent to $C$, while assuring
%$d(x, C'') \leq D$.
%%by applying an isometry that does not increase the distance to $H$ (in the former two cases of adjacency) or 
%%$H_1$ (in the latter two cases).
%Therefore $x\in X_{C,C''}$, and the claim holds as in the case of adjacent cones.
%
%As before, all points in $Y_C$ satisfy the first inequality in \eqref{eq:D'}.
%Furthermore, all points in $Y_{C'}$ belong to $Y_{C''}$ since the distance of $\iota(x)$ to $H_a$
%is not larger than the distance of $x$ to $H_a$, for every $x\in C'$, and therefore satisfy the second
%inequality \eqref{eq:D'}.
As $x$ was chosen arbitrarily, this completes the proof.
\end{proof}
%
Let $B:=9\cdot D^2 \cdot P(M)^2$, where $P$ comes from Lemma \ref{lem:not_far_from_cone} and
$D\leq \OO(M^2)$ from Claim \ref{claim:a}.
%
\begin{claim} \label{claim:axgen}
There is a vector $a\in\Z^3$ of $\norm(a)\leq D$ such that
all configurations $q(x)$ in $V_1$ appearing on a path $s\tran t$ in $V$ satisfy
$-B \leq \innprod a x \leq B$.
%where $B = \OO(M\cdot D)$. 
\end{claim}
%
\begin{proof}
Consider a path $s \trans{\pi} t$ and let $\pi_1$ be its prefix in $V_1$.
By Lemma~\ref{lem:not_far_from_cone}, all the vectors appearing in $\pi_1$ are not further than $P(M)$ from 
$S=\seqcone{\rev V}$, but also not further than $P(M)$ from $C_1 = \cone{V_1}$.
By disjointness of $C_1$ and $S$, due to Claim \ref{claim:empty_intersection}, 
and by Claim \ref{claim:onefacet},
there is a facet $F$ of $C$ such that
all the vectors $x$ appearing in $\pi_1$ are at distance at most  $3\cdot P(M)$ to
the hyperplane $H$ including $F$.
%
Due to Claim~\ref{claim:a}, $H=\setof{y}{\innprod {a} y = 0}$ for some
vector $a \in \Z^3$ of $\norm(a) \leq D=\OO(M^2)$.

%Let $i\in\{1,2\}$.
In order to bound the value of $\innprod {a} x$ for an arbitrarily vector $x$ appearing in $\pi_1$,
split $x$ into $x = x_1 + x_2$, where $x_1$ is orthogonal to $H_i$, i.e., $x_1 = \ell \cdot a$ for some $\ell\in\Q$.
\Wlog assume that the length of $a$ is at least 1.
The length of $x$  is $\innprod {x_1}{x_1}\leq 9\cdot P(M)^2$, and hence
$x_1 = \ell \cdot a$ for some $\ell$ satisfying $\absv \ell \leq 9\cdot P(M)^2$.
In consequence,  $\innprod {a} x = \innprod {a} {x_1} = \ell \cdot \innprod {a} {a}$,
and hence $\absv {\innprod {a} x} \leq \ell \cdot \norm(a)^2 \leq B= 9\cdot D^2 \cdot P(M)^2$, as required.
%
%\Wlog assume that
%the effect $e$ of every simple cycle satisfies $\innprod {a_i} e\leq 0$ for every $i\in\{1,2\}$.
%In consequence, if $\innprod {a_i} x < - E - M$ for some $x$ appearing on $\pi_1$, then
%$\innprod {a_i} {x'} < - M$ for all subsequent points $x'$ on $\pi_1$, and therefore
%$\innprod {a_{3-i}} {x'} \geq - M$ for all subsequent points $x'$ on $\pi_1$.
%This observation implies that for some $a\in\{a_1,a_2\}$, all the points $x$ appearing on $\pi_1$
%satisfy $\innprod {a} x \geq - E - M = -B$, and therefore also
%$-B \leq \innprod a x \leq B$, as required.
\end{proof}
%
We complete the proof of Case II as in Case 2 of the proof of Lemma \ref{lem:1comp}.
We replace the first component $V_1$ by the \geomvass $\essdvass V$, as defined there,
of size $E = \OO(M\cdot B)$ (as stated in Claim \ref{claim:sizeVV}),
and the source configuration $s$ by $\essdvass s$.
We also replace the first bridge transition $u_1 = (p'_1, \delta_1, p_2)$ by
$\essdvass u_{b} = (\pair{(p'_1)} b, \delta_1, p_2)$, for any $b\in\setfromto{-B} B$.
This yields a set $S$ of $2B+1$ good-for-induction $k$-component \tvass
\[
S \ = \ \setof{\big(\ktvassmodmod, \essdvass s_a, t \big)}{a\in A, \ b\in \setfromto {-B} B},
\]
which is length-equivalent to $(V, s, t)$, as required. 
The size of each of these \tvass is at most $R(M) = E+M \leq \OO(M \cdot B)$.
%
\end{appendixproof}

%By a \emph{lollypop} \tvass we mean a 2-component \tvass $(V_1)u_1(V_2)$ whose second component
%$V_2 = (Q_2, T_2)$ is trivial, namely $Q_2$ is a singleton and $T_2 = \emptyset$. 
%A lollypop is thus a 1-component \tvass with a single bridge transition going out the component. 

\begin{proof}[Proof of Lemma \ref{lem:ne}]
Relying on Lemma \ref{lem:len-eq}, assume \mywlog that
$(V, s, t)$ is a 
good-for-induction $k$-component \tvass of size
$\size(V, s, t) = R(M)$, where $R$ is a polynomial of Lemma \ref{lem:len-eq}.
Let $V = \ktvass$, $u_i = (p'_i, \delta_i, p_{i+1})$,
$s=p_1(w)$ in $V_1$ and $t=p'_k(w')$ in $V_k$.
%
Let $F$ be a nondecreasing polynomial witnessing 
Lemma \ref{lem:geom-sandwich}, and
let $f(x) = 2R(x) \cdot F(R(x))$. 
Let $G$ be a nondecreasing polynomial witnessing Lemma  \ref{lem:geom-plb}, and
let $g(x) = G(2x^2) + x$.


Let $B := h(f(M))$, namely the length-bound for $(k-1)$-component \vass of size $f(M)$.
Assuming $s\trans{\pi} t$ in $ V$, we aim at proving that there is such a path of length at most
$g(h(f(h(f(M)))))$.
%Let $p$ be the first state of $ V_2$ on $\pi$, and 
Decompose the path $s\tran t$ into
%
\begin{align} \label{eq:pathV}
s \trans{\pi_1} p'_1(v') \trans{u_1} p_2(v) \trans{\pi'} t,
\end{align}
%
where $p_2(v)$ is the first configuration of $V_2$ appearing on $\pi$.
As $V_1$ is a \geomvass, so is the \emph{lollypop} \tvass $V'_1$ 
obtained by adding to $V_1$ the first bridge transition $u_1$
(indeed, adding a bridge transition does not create any new cycles).
As $\size(V,s,t) \leq R(M)$ and $F$ is the polynomial witnessing Lemma~\ref{lem:geom-sandwich},
by Lemma~\ref{lem:geom-sandwich} we get:
%
\begin{claim}
The set $\reach_{p_2}(V,s)$ is  \kanapka {$F(R(M))$} {$B$}.
\end{claim}

Thus $v \in L = a + P^*$, where $L$ is a linear set ($*$) or a $B$-approximation thereof ($**$).
In both cases  $v = a + r$, for $r\in P^*$.
%
We construct a $(k-1)$-component \tvass $(V', s', t)$ as follows.
%We drop the first component $V_1$ and the first bridge $u_1$, thus getting:
%$\essdvass V = (V_2) u_2 \ldots u_{k-1} (V_k)$.
$V'$ is obtained from $\ktvasstwo$ by adding,
for every period vector $r\in P$, a self-looping transition
$(p, r, p)$ to $ V_2$.
As the source configuration we take $s' = p_2(a)$, and keep $t$ as the target configuration.

As $R$ is nondecreasing, $B = h(f(M))$ and $f \geq R$ we have $M \leq R(M) \leq B$.
As $(V',s',t')$ is obtained from of $(V,s,t)$ by removing from $V$ the part $V_1$,
adding transitions of total norm at most equal $\norm(P)$ and setting $s' = p_2(a)$ 
therefore we estimate  $M' = \size(V', s', t)$ as 
$
M' \ \leq \ R(M) + \norm(a) + \norm(P).
$
Recall that in the case when $L$ is a linear set ($*$) we have $\norm(a) \leq B \cdot A$, $\norm(P) \leq A$,
while in the case when $L$ is a $B$-approximation ($**$) we have $\norm(a), \norm(P) \leq A$,
in our case $A = F(R(M))$.
Thus the estimations
%in the cases when $L$ is a linear set ($*$) and in case when $L$ is a $B$-approximation ($**$)
look as follows:
%
\begin{align*}
(*) \quad M' & \leq R(M) + (B+1) \cdot F(R(M)) \leq  2\cdot R(B) \cdot F(R(B)) = f(B)\\
(**) \quad M' & \leq R(M) + 2 \cdot F(R(M)) \leq 2\cdot R(M) \cdot F(R(M)) = f(M),
\end{align*}
as $f$ was defined exactly as $f(x) = 2R(x) \cdot F(R(x))$.
Note that the latter bound is dominated by the former one, thus in both cases $M' \leq f(B) = f(h(f(M)))$.

\smallskip

There is a path $s' \tran t$ in $V'$, namely the concatenation of  
a path $p_2(a) \tran p_2(v)$ using the 
just added self-looping transition in the state $p$, with the suffix $\pi'$ of \eqref{eq:pathV}:
\[
p_2(a) \tran p_2(v) \trans{\pi'} t.
\]
By induction assumption, there is a path $s' \tran t$ in $V'$ of length at most
\begin{align*}
(*) \quad & \ell \ = \ h(M') \ \leq \ h(f(B)) \ = \ h(f(h(f(M))))\\
(**) \quad &  \ell \ = \ h(M') \ \leq \ h(f(M)).
%\HH {k-1} {2\cdot F(M) \cdot B} \ = \ 
%\HH {k-1} {F'(M) \cdot {\HH {k-1} {M}}}.
%
%\kbound {(M')} {C\cdot(k-1)} \ \leq \ 
%\kbound {(P(M) \cdot 2B)} {C\cdot(k-1)} \ = \ 
%\kbound {(P(M) \cdot 2 \kbound{M}{C\cdot (k-1)})} {C\cdot(k-1)}.
\end{align*}
\Wlog we may assume that the path executes all just added self-looping transitions in the beginning,
and therefore it splits into:
\[
p_2(a) \trans{\rho''} p_2(v'') \trans{\rho'} t
\]
such that the suffix $\rho'$ is actually a path in $\ktvasstwo$.
Since the length of the prefix $\rho''$ is at most $\ell$, 
we bound $\norm(v'') \leq M \cdot \ell$.
%
If $L$ is a linear set, the following claim is obvious since $L\subseteq\reach_{p_2}(V,s)$.
On the other hand, the claim requires a proof in case when $L$ is a $B$-approximation of a linear set:
%
\begin{claim}[$**$]
There is a path $s \tran p_2(v'')$ in the lollypop \tvass $V'_1$.
\end{claim}
%
Indeed,  
since the length $\ell$ of the path $s'\tran t$ in $V'$ is at most $h(f(M))$
and $B= h(f(M))$, we deduce that $v'' \in a + P^{\leq B}$, and therefore $s\tran p_2(v'')$ in $ V'_1$.

\smallskip

By Lemma \ref{lem:geom-plb}, there is a path $s\trans{\rho} p_2(v'')$ in $V'_1$ of length at most
$G(M + \norm(v'')) \leq G(M\cdot(\ell+1)) \leq G(2\ell^2)$.
Concatenating the two paths $\rho$ and $\rho'$ we get a path
\[
s\trans{\rho} p_2(v'') \trans{\rho'} t
\]
in $V$, of length at most 
$
G(2\ell^2) + \ell = g(\ell) \leq g(h(f(h(f(M))))).
$
Taking $H = f+g$, the sum of polynomials $f$ and $g$, we get the bound
$H(h(H(h(H(M)))))$,
as required.
\end{proof}
