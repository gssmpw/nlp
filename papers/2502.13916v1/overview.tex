% !TEX root = main.tex

\section{Overview} \label{sec:overview}
In this section we present an overview on the proof of our main result, namely of Lemma~\ref{lem:main}.
The proof proceeds by an induction on the number $k$ of components in a sequential \tvass.
The main idea is that either the situation is easy (a short path can be obtained by lifting up a $\Z$-path)
or the first component can be transformed into a finite union of essentially two-dimensional \vass (more precisely,
finite union of \geomvass), each of size bounded polynomially. This transformation is shown in Lemma~\ref{lem:len-eq}.
The induction base is shown in Section~\ref{sec:1comp}. We present the proof of the one component case in detail,
as it illustrates the main concepts of the proof of Lemma~\ref{lem:len-eq}, but in a much simpler setting.
When the first component is transformed into essentially a \dvass, we can use the fact that
reachability sets in \dvass are semi-linear and the size of the semi-linear representation
is at most exponential~\cite{BlondinFGHM15} (the result is true as well in \geomvass).
This fact can be exploited to reduce reachability for $k$-component \tvass to reachability for $(k-1)$-component \tvass
of exponentially larger size, the details of this reduction are explained in Section~\ref{sec:tools}
in the paragraph about the idea of the proof of Lemma~\ref{lem:main}.
However, if we use semi-linear sets, the exponential blowup in unavoidable and this approach gives us a \tower algorithm
resulting from a linear number of exponential blowups (thus not better than \cite{DBLP:conf/icalp/FuYZ24}).
In order to improve the complexity we introduce a novel notion of suitable over- and under-approximations of semi-linear sets.
One of our key technical contributions is Lemma~\ref{lem:2vass-sandwich} stating that reachability sets
of \dvass can be well approximated. 
Intuitively speaking, the precision of the approximation has to be good enough for correctness of the inductive proof;
the better the precision, the bigger the representation of approximants gets.
This approach allows us to reduce reachability for $k$-component \tvass to reachability in $(k-1)$-component \tvass of size which is not anymore exponential, but is polynomial in $B$,
where $B$ is the minimal length of a path in $(k-1)$-component \tvass.
This means that $n^m$ bound on the minimal path length for $(k-1)$-component \tvass implies roughly a $n^{m^2}$
bound on the minimal path length for $k$-component \tvass. The transformation
$n^m \mapsto n^{m^2}$ applied linear number of times results in triply-exponential upper bound for the minimal length of a path in \tvass.

