% !TEX root = main.tex

\section{1-component \tvass are \plb} \label{sec:1comp}

This section is devoted to the proof of the 
induction base for the proof of Lemma \ref{lem:main}:

\begin{lemma} \label{lem:1comp}
1-component \tvass are \plb.
\end{lemma}

\noindent
We also develop a framework to be exploited in the induction step in 
Section \ref{sec:mainproof}.
We may safely restrict to \tvass of geometric dimension 3,
as otherwise Lemma~\ref{lem:geom-plb} immediately implies Lemma~\ref{lem:1comp}.

\para{Case distinction}
%We distinguish between \emph{easy} and \emph{non-easy} \tvass, as defined below.
%\para{Diagonal \vass}
A \tvass $(V, s, t)$, where $s=p(w)$ and $t=p'(w')$,
is \emph{forward-diagonal} if
$p(w) \trans{*} p(w+\Delta)$ in $V$ for some $\Delta \in (\Npos)^3$.
Symmetrically, $(V, s, t)$ is \emph{backward-diagonal} if $\rev{(V, t, s)}$ is diagonal, i.e., if 
$p'(w'+\Delta') \trans{*} p'(w')$ in $V$ for some $\Delta' \in (\Npos)^3$.
Finally, $V$ is \emph{diagonal} if it is both forward- and backward-diagonal.
Obviously, the vectors $\Delta$ and $\Delta'$ need not be equal in general.

%\para{Cones generated by effects of cycles}

Let  $E=\{e_1, \ldots, e_n\} \subseteq \Z^3$ be the effects of simple cycles of $V$.
%\wojtek{Notation $e_i$ suggests elementary vectors, maybe change to $v_i$.}
We define the (rational) open cone generated by this set
to contain all positive rational combinations of vectors from $E$:
\[
\cone {V} \ = \ \setof{ r_1 \cdot e_1 + \ldots + r_n \cdot e_n }{r_1, \ldots, r_n \in \Qpos} \ \subseteq \ \Lin V.
%\ \subseteq \ (\Qpos)^3.
\]
$\cone V$ is thus an open cone inside $\Lin V$.
A 1-component \tvass $V$ is called \emph{wide} if $(\Qpos)^3 \subseteq \cone V$, i.e.,
if $\cone V$ includes the whole positive orphant.

%We say that a \vass $(V,s,t)$ is \emph{length-equivalent} to a finite set $\{(V_1,s_1,t_1), \ldots, (V_n, s_n, t_n)\}$ 
%of \vass if
%the set of lengths of paths $s\tran t$ in $V$ is the same as the set of lengths
%of all paths $s_i \tran t_i$ in $V_i$ for all $i\in\setfromto 1 n$. 
%
%\begin{proof}


Let $\Len V s t$ denote the set of lengths of paths $s\tran t$ in $V$.
%For proving Lemma \ref{lem:1comp}, 
We need to
argue that there is a nondecreasing polynomial $Q$ 
such that every 1-component \tvass $(V, s, t)$
with a path $s\tran t$, has such path of length at most $Q(M)$, where 
$M=\size(V, s,t)$.
We split the proof into three cases:
\smallskip
\begin{enumerate}
\item
If $(V,s,t)$ is diagonal and wide, we exploit the fact that \tzvass are \plb, and use diagonality and wideness to
lift a short $\Z$-path into a path.
\item
If $(V,s,t)$ is diagonal but non-wide, we show that $(V,s,t)$ is \emph{length-equivalent} to a 
\geomvass $(\essdvass V, \essdvass s, \essdvass t)$ of polynomially larger size, namely 
$\Len V s t = \Len {\essdvass V} {\essdvass s} {\essdvass t}$.
\item
Finally, if $(V, s, t)$ is non-diagonal, we show that $(V,s,t)$ is \emph{length-equivalent} to a set
of three
\geomvass $\{(V_1, s_1, t_1), (V_2, s_2, t_2), (V_3, s_3, t_3)\}$, namely 
$\Len V s t = \Len {V_1} {s_1} {t_1} \cup \Len {V_2} {s_2} {t_2} \cup \Len {V_3} {s_3} {t_3}$
of polynomially larger size.
\end{enumerate}
\smallskip
\noindent
In the two latter cases we rely on the fact that \geomvass are \plb (Lemma \ref{lem:geom-plb}).
%
%We distinguish the following cases, depending on whether $(V, s, t)$ is: 
%
%\begin{enumerate}
%\item diagonal and wide;
%\item diagonal and non-wide;
%\item non-diagonal.
%\end{enumerate}
%
%\begin{lemma} \label{lem:easy-1comp-plb}
%Diagonal and wide 1-component \tvass are \plb.
%\end{lemma}
%
%\begin{lemma} \label{lem:-1comp-plb}
%There is a nondecreasing polynomial $P$ such that every diagonal non-wide 1-component \tvass $(V,s,t)$ is equivalent to a \geomvass of size at most $P(\size(V,s,t))$.
%\end{lemma}
%
In consequence, $Q$ is to be the sum of polynomials claimed in the respective cases.
%In the first case we exploit the fact that \tzvass are \plb
%(Lemma \ref{lem:zvass-plb}), while
%in the two latter cases we rely on the fact that \geomvass are so (Lemma \ref{lem:geom-plb}).
%To this aim, in Cases 2 and 3 we transform a \tvass into a form which is  \emph{essentially} a \dvass,
%namely in every configuration $q(v_1, v_2, v_3)$,
%the value of  one of coordinates, say $v_3$, is uniquely determined
%by state $q$ and values of the other two coordinates through an affine mapping,
%$v_3 = \ell_q(v_1, v_2)$.
%
In the sequel let $(V, s, t)$ be a fixed 1-component \tvass  with $s\tran t$, where
$s=p(w)$ and $t=p'(w')$.

\para{Case 1. $(V, s, t)$ is diagonal and wide}
%
By diagonality, $p(w) \trans{\pi} p(w+\Delta)$ and
$p'(w'+\Delta')\trans{\pi'} p'(w')$  for some $\Delta,\Delta'\in(\Npos)^3$.

Let $P$ be a nondecreasing polynomial witnessing Lemma \ref{lem:zvass-plb}, i.e.,
\tzvass are \lb by $P$.
As there is a path $s \tran t$ in $V$, there is also a $\Z$-path $s\tran t$, and
by Lemma \ref{lem:zvass-plb} there is a $\Z$-path $s \trans{\sigma} t$ of length at most $P(M)$.
The maximal norm $N$ of $\Z$-configurations along $\sigma$ is thus bounded by $M \cdot P(M)$,
as every step may update counters by at most $M$.

By diagonality, the configuration $p(w+\vec 1)$ is coverable in $V$ from $s$,
and symmetrically the configuration $p'(w' + \vec 1)$ is coverable in $\rev V$ from $t$.
Due to the upper bound of 
Rackoff~\cite[Lemma~3.4]{DBLP:journals/tcs/Rackoff78},
there is a nondecreasing polynomial $R$ such that in every \tvass of size $m$, the length of a covering
path is at most $R(m)$.
Therefore the lengths of 
both paths $p(w) \trans {\pi} p(w+\Delta)$  and $p'(w'+\Delta')\trans {\pi'} p'(w')$ in $V$, 
where $\Delta, \Delta' \in (\Npos)^3$,
may be assumed to be at most $R(M)$.
%
We argue that there is a cycle 
%$p(w) \tran p(w+ \ell\cdot \Delta')$
from the source configuration $p(w)$ 
that increases $w$ by some multiplicity of $\Delta'$:
%
\begin{lemma} \label{lem:FP}
There is a path $p(w) \trans \delta p(w+ \ell \cdot\Delta')$ of length $R(M)^{\OO(1)}$,
for some $\ell\in\Npos$.
% $\ell\leq R(M)^{\OO(1)}$.
\end{lemma}
%
Before proving the lemma we use it to complete Case 1.
We build a path $p(w) \tran p'(w')$ by concatenating $3$ paths given below.
The first one is  $\delta$
%\[
%p(w) \trans{\delta} p(w+\ell\cdot\Delta'),
%\]
given by Lemma \ref{lem:FP}.
Note that $\ell$ is necessarily also bounded by $R(M)^{\OO(1)}$.
We replace $\ell$ by its sufficiently large multiplicity to enforce $\ell \geq M\cdot P(M)$, which
makes the length of $\delta$ and $\ell$ only bounded by $P(M) \cdot R(M)^{\OO(1)}$.
The multiplicity guarantees that the $\Z$-path 
$
p(w) \trans{\sigma} p'(w'), 
$
lifted by $\ell \cdot \Delta'$, becomes a path:
\[
p(w+\ell \cdot \Delta') \trans{\sigma} p'(w' + \ell\cdot\Delta'), 
\]
The length of $\sigma$ is bounded by $P(M)$.
Finally, let $\delta' = (\pi')^\ell$ be the $\ell$-fold concatenation of
the cycle $\pi'$:
\[
p'(w'+\ell\cdot\Delta') \trans{\delta'} p'(w').
\]
The length of this path is bounded by $\ell\cdot R(M) \leq P(M)\cdot R(M)^{\OO(1)}$.
We concatenate the three paths, $\tau := \delta; \, \sigma;\, \delta'$, to get a required path
\[
p(w) \trans{\tau} p'(w')
\]
of length bounded by $P(M) \cdot R(M)^{\OO(1)}$.
It thus remains to prove Lemma \ref{lem:FP} in order to complete Case 1.


\begin{proof}[Proof of Lemma \ref{lem:FP}]
%By forward diagonality, $p(w) \trans{\pi} p(w+\Delta)$ for some $\Delta\in(\Npos)^3$.
Let $\pi_1$ be a cycle that visits all states, and
let $\Delta_1\in\Z^3$ be its effect.
%
%There is such a cycle of length at most $\OO(M^2)$.
%
%For $m\in \Npos$, let $\pi^m$ denote $m$-fold concatenation of $\pi$:
%$p(w) \trans{\pi^m} p(w+m\Delta)$.
%
Relying on $\Delta\in(\Npos)^3$,
take a sufficiently large multiplicity $m\in\Npos$ so that 
$\widetilde \pi = \pi^m;\, \pi_1$ is a path with nonnegative effect.
%It is enough to take $m\leq \OO(M^3)$.
The path $\widetilde \pi$ is a cycle and  its effect is
$\widetilde \Delta = m\cdot\Delta + \Delta_1 \in \N^3$.
%is a nonnegative integer combination
%of effects of simple cycles:
%\[
%\Delta_2 = s_1 \cdot e_1 + \ldots + s_n \cdot e_n \in \cone V \cap \N^3,
%\] 
%where $s_1, \ldots, s_n \in \N$.

As $\Delta'\in(\Npos)^3$, there is $\ell'\in\Npos$ such that 
$\ell'\cdot\Delta'-\widetilde \Delta \in (\Qpos)^3$,
and hence, by wideness of $(V, s)$, we have
$\ell'\cdot\Delta'-\widetilde \Delta \in \cone V$, namely
\[
\ell'\cdot\Delta' - \widetilde \Delta = r_1 \cdot e_1 + \ldots + r_n \cdot e_n
\] 
for some positive rationals $r_1, \ldots, r_n\in \Qpos$.
%
By Carathéodory's Theorem \cite[p.94]{cara}, $\ell'\cdot\Delta'-\widetilde \Delta$ is a combination of some $3$ vectors
among $e_1, \ldots, e_n$, say $e_1, e_2, e_3$:
\[
\ell'\cdot\Delta' - \widetilde \Delta \ = \ r_1 \cdot e_1 + r_2 \cdot e_2 + r_3 \cdot e_3,
\] 
for some positive rationals $r_1, r_2, r_3\in\Qpos$.
Therefore, the system of $3$ equations
\[
\ell \cdot (\ell'\cdot\Delta'-\widetilde\Delta)  \ = \ r_1 \cdot e_1 + r_2 \cdot e_2 + r_3 \cdot e_3,
\]
with unknowns $\ell, r_1, r_2, r_3$, has a positive integer solution.
We rewrite the system to:
\begin{align} \label{eq:syst_1comp}
\ell \ell'\cdot \Delta' - \ell m \cdot \Delta \ = \  \ell\cdot\Delta_1 + r_1 \cdot e_1 + r_2 \cdot e_2 + r_3 \cdot e_3.
\end{align}
%
%For sufficiently large $\ell\in\Npos$,
% enough so that the multiplicites of coefficients $\ell_1\cdot r_i$ are integers,
%for $i\in\setto n$.
%Furthermore, picking up a sufficiently large multiplicity $\ell_1$ yields
%$\ell_1 \cdot r_i \geq s_i$ for every $i\in\setto n$.
%Summing up, for some $r_1, \ldots, r_n \in \N$ we have
%\[
%\ell_1\cdot v = \Delta_2 + r_1 \cdot e_1 + \ldots + r_n \cdot e_n \in \cone{V} \cap (\Npos)^3.
%\]
Let $\sigma_i$ be simple cycle of effect $e_i$, for $i\in\setto 3$.
Let $
\sigma
$
be a $\Z$-path that starts (and ends) in state $p$ and consists of 
$\ell$-fold concatenation of the cycle $\pi_1$, with attached 
$(r_1)$-fold concatenation of $\sigma_1$,
$(r_2)$-fold concatenation of $\sigma_2$,
and
$(r_3)$-fold concatenation of $\sigma_3$
(since $\pi_1$ visits all states, this is possible).
The effect of $\sigma$ is the right-hand side of \eqref{eq:syst_1comp}, and therefore
$\sigma$ is a $\Z$-path from $p(w+\ell m \cdot \Delta)$ to
$p(w + \ell \ell'\cdot \Delta')$:
\[
p(w + \ell m \cdot \Delta) \trans{\sigma} p(w+\ell \ell'\cdot \Delta').
\]
It needs not be a path in general, and therefore we are going to lift it.
Let $k\in\Npos$ be a multiplicity large enough so that $\sigma$ becomes a path
when lifted by $(k-1)\ell m \cdot \Delta$, i.e., when starting in $p(w+k \ell m \cdot \Delta)$, 
and also becomes a path when lifted by $(k-1)\ell\ell'\cdot \Delta'$, i.e., when
ending in 
$p(w+k \ell\ell' \cdot \Delta')$.
In this case, the $k$-fold concatenation of $\sigma$ is also a path:
%
\begin{align} \label{eq:thepath}
p(w+k\ell m\cdot \Delta) \trans{\sigma^{k}} 
p(w+k \ell\ell'\cdot \Delta'),
\end{align}
%
since all points visited in the inner iterations of $\sigma$ are bounded 
from both sides by corresponding points visited in the first and the last iteration of $\sigma$.
Precomposing this path with $p(w)\trans{\pi^{k\ell m}} p(w + k\ell m\cdot \Delta)$ yields a path
$p(w) \tran p(w+ k \ell \ell'\cdot \Delta')$, as required.

\smallskip

We now (roughly) bound the magnitudes of all items involved in the above reasoning
by a constant power of $R(M)$.
\Wlog we may assume that the cycle $\pi_1$ uses every transition at most $\card Q\leq M$ times,
and thus both the length and norm of the effect of $\pi_1$ are bounded by $M^2$.
In consequence, $\pi_1$ may decrease counters by at most $M^2$.
Therefore $m\leq M^2$, and
norms of vectors $\Delta, \Delta', \widetilde\Delta$ are all bounded by 
$\OO(M^3\cdot R(M))$. % \leq R(M)^{\OO(1)}$.
The effects $e_1, e_2, e_3$ of simple cycles $\sigma_1, \sigma_2, \sigma_3$ are at most $M$, 
as no transition repeats along a simple cycle.
Therefore by Lemma \ref{lem:taming}, the system \eqref{eq:syst_1comp} has a solution $(\ell, r_1, r_2, r_3)$ 
of norm at most $D=\OO(M^3\cdot R(M))^3 = \OO(M^9 \cdot R(M)^{3})$, 
and $\sigma$ has length at most $M^2 \cdot D$
(since $\pi_1$ has length at most $M^2$). %\OO(M^{11}\cdot R(M)^{3})$.
Therefore we deduce $k\leq M^3 \cdot D$, %\OO(M^{12} \cdot R(M)^{3})$, 
and hence
the path \eqref{eq:thepath} has length at most $M^5 \cdot D^2$. % $\OO(M^{23} \cdot R(M)^{6})$.
In consequence of the above bounds, $k \ell m \leq M^5 \cdot D^2$, %\OO(M^{23} \cdot R(M)^{6})$,
and the final path $p(w) \tran p(w+ k \ell v)$ has length at most 
$R(M) \cdot M^5 \cdot D^2
%$\OO(M^{23} \cdot R(M)^{7}) 
\leq \OO(R(M)^{30})\leq R(M)^{\OO(1)}$.
\end{proof}



\para{Case 2. $(V, s, t)$ is non-wide}

%(We do not use diagonality of $(V, s, t)$ in the sequel.)
%
Every non-zero vector $a=(a_1, a_2, a_3)\in\Q^3$ defines an open half-space
\[
%H \ = \ \setof{(x_1, x_2, x_3)\in\Q^3}{a_1 \cdot x_1 + a_2 \cdot x_2 + a_3 \cdot x_3 = 0}.
H_a \ = \ \setof{x\in\Q^3}{\innprod a x > 0},
\]
where $\innprod a x = a_1 x_1 + a_2 x_2 + a_3 x_3$ stands for the inner product of 
$x =(x_1, x_2, x_3)$ and $a$.
As $V$ is assumed to be of geometric dimension 3,
$\cone V$  is an intersection of open half-spaces:
%
\begin{claim} \label{claim:a}
%If vectors generating $\cone V$ spans three dimensional vector space (possibly by using negative coefficients) then 
$\cone V$ is an intersection of %$\Lin V$ with 
finitely many open half-spaces $H_a$, with $\norm(a) \leq D := \OO(M^2)$.
\end{claim}
%
%\wojtek{Maybe introduce some notion of polynomially-bounded (wrt. to $M$)}
%As $V$ is assumed to be of geometric dimension 3, $\Lin V = \Q^3$ may be omitted.
%\slawek{remove $\Lin V$ from the claim?}

\begin{proof}
Norms of vectors generating $\cone V$ --- i.e., effects of simple cycles --- are at most $M$, 
as no transition repeats along a simple cycle.
Consider vectors $a$ orthogonal to some of the facets of $\cone V$, i.e.,
orthogonal to two of the vectors generating $\cone V$. 
The vector $a$ is thus an integer solution of a system of 2 linear equations with 3 unknowns,
where absolute values of coefficients are bounded by $M$.
By Lemma \ref{lem:taming}, there is such an integer solution with $\norm(a)\leq \OO(M^2)$.
This completes the proof.
\end{proof}

%As $(V, s, t)$ is forward-diagonal, we know that there is a positive vector $\Delta \geq \vec 1$
%such that $\Delta \in \cone V$.
As $V$ is non-wide, 
%there is a positive vector $\Delta'\geq \vec 1$ such that
%$\Delta'\notin \cone V$, and,
due to Claim~\ref{claim:a} we know that $\cone V$ is a \emph{non-empty} intersection of half-spaces $H_a$.
Therefore for some of these $H_a$ we have
% $\Delta' \not\in H_a$. On the other hand, 
$\cone V \subseteq H_a$, i.e.,
%Therefore $\innprod \Delta a \geq 0$ and $\innprod {(\Delta')} a \leq 0$.
%As $a\neq(0,0,0)$, 
%this implies that among $a_1, a_2, a_3$ there is both a non-negative and a negative number.
%\Wlog we assume $a_1, a_2\geq 0$, $(a_1,a_2)\neq (0,0)$ and $a_3 < 0$
%(the symmetric case is dealt with by changing inequality sign in Claim~\ref{clm:inner_with_effect} and 
%adapting accordingly the proof of Claim~\ref{claim:ax}).
%
%As $\cone V \subseteq H_a$, 
%
all points $x\in\cone V$ have positive inner product $\innprod a x > 0$.
% (otherwise we replace $a$ by $-a$).
%(while all points in the other cone $-C$ have nonpositive inner product $\innprod a x\leq 0$).
%
This implies that
the value of  inner product with $a$ may not decrease along any cycle in $V$:
%
\begin{claim}\label{clm:inner_with_effect}
The effect $\delta\in\Z^3$ of every simple cycle has nonnegative inner product $\innprod a \delta \geq 0$.
\end{claim}
%
In consequence, on every path $s \tran t$ the value of inner product with $a$ is polynomially
bounded:
%
\begin{claim} \label{claim:ax}
Every configuration $q(x)$ on a path from $s$ to $t$ satisfies
$-B \leq \innprod a x \leq B$, where $B := \OO(M\cdot D)$. 
\end{claim}
%
\begin{proof}
Let $s=p(w)$ and $t=p'(w')$, and let $b = \innprod a w$ and $b' = \innprod a {w'}$.
Every path $s \tran t$ may be decomposed into simple cycles, whose effect may only preserve or increase
the inner product with $a$, plus a short path without cycles, and hence without repetitions of a transitions.
The effect of the latter path is thus in $[-M,M]$.
Therefore, as inner product may at most multiply norms, 
every configuration $q(x)$ on a path from $s$ to $t$ satisfies
$b-M \cdot D \leq \innprod a x \leq b'+M \cdot D$.
Knowing that $\norm(a)\leq D$, $\norm(w), \norm(w')\leq M$, 
and that inner product may at most multiply norms, 
by Claim \ref{claim:a} we deduce 
\[
-M\cdot D \leq b, b' \leq M\cdot D,
\]
and therefore
$-2\cdot M\cdot D \leq \innprod a x \leq 2\cdot M \cdot D$,
which implies the claim.
\end{proof}

%Let $N_b$ be the set of point-wise minimal pairs $(x_1, x_2)$ satisfying 
%the inequality $a_1 x_1 + a_2 x_2 \geq b$:
%\[
%N_b : = \min \setof{(x_1, x_2) \in \N^2}{a_1 x_1 + a_2 x_2 \geq b}.
%\]
%\lukasz{Explain that for checking whether we are in $N_b$ it is enough to subtract and add}
We define a \geomvass $\essdvass V = (\essdvass Q, \essdvass T)$ 
by extending states with the possible values of inner product with $a$
(bounded polynomially by Claim \ref{claim:ax}).
We call $\essdvass V$ the \emph{\mytrim {$a$} {$B$}} of $V$. 
%\footnote{This construction is inspired by \cite[Section 5]{Zhang-geom}.}
The set of states $\essdvass Q$ contains states of the form $\pair q b$, where $q\in Q$ and $-B \leq b \leq B$,
%that correspond to states of $V$, plus a number
%of auxiliary states:
%\[
%\essdvass Q = \setof{\pair q b, \ \pair{\essdvass q} b, \ \triple {\essdvass q} b n}{q\in Q, \ b\in [-B,B], 
%\ n\in N_b}.
%\]
with the intention that every configuration
$c = q(x)$ of $V$ has a corresponding configuration 
$\essdvass c = \pair q b (x)$ in $\essdvass V$, 
where $\innprod a x = b$.
%
%The corresponding configuration of $V$ is unique, as $\innprod a x = b$ implies
%\begin{align} \label{eq:x3}
%x_3 = (b - a_1 x_1 - a_2 x_2)/{a_3}.
%\end{align}
Therefore, for each transition $(q, v, q') \in T$
% where $v=(v_1, v_2, v_3)$,
and for all $b, b'\in \setfromto{-B} B$
such that $b + \innprod a v = b'$,
we add to $\essdvass T$ the transition
\begin{align} \label{eq:tranV}
\big(\pair q b, v, \pair {q'} {b'}\big).
\end{align}
%Moreover, we add to $\essdvass V$ transitions that check whether a configuration 
%$\pair {\essdvass{q}} {b}(x_1, x_2)$ corresponds to a configuration of $V$, i.e.,
%whether $x_3\geq 0$ in \eqref{eq:x3}, which is equivalent to
%$a_1 x_1 + a_2 x_2 \geq b$ (as $a_3<0$).
%To this aim for each $b\in \setfromto{-B}B$ and each $n\in N_b$ we add to $\essdvass T$ two
%auxiliary transitions:
%%
%\begin{align} \label{eq:checkV}
%(\pair {\essdvass{q}} {b}, -n, \triple {\essdvass q} b n)
%\qquad
%(\triple {\essdvass q} b n, n, \pair {q} {b}).
%\end{align}
%%
%%Consider any configurations 
%%$p(w)$, $p'(w')$ of $V$, where $w = (w_1, w_2, w_3)$ and $w' = (w'_1, w'_2, w'_3)$,
%%and let $b=\innprod a w$, $b' = \innprod a {w'}$. 
%%

\begin{claim}\label{clm:trim_makes_geom}
$\essdvass V$ is a \geomvass.
\end{claim}

\begin{proof}
By construction, the effect of each cycle in $\essdvass V$ is orthogonal to $a$, and therefore 
$\Lin {\essdvass V}$ is included in a $2$-dimensional vector space.
\end{proof}
%
Relying on Claim \ref{claim:ax}, paths $s\tran t$ in $V$ have corresponding paths in $\essdvass V$,
and hence we get:
\begin{claim} \label{claim:lenlen}
$\Len V s t = \Len {\essdvass V} {\essdvass s} {\essdvass t}$.
\end{claim}
%The tight correspondence between paths $V$ and $\essdvass V$,
% given in Claims \ref{claim:V31} and
%\ref{claim:V32} below, is essentially 
%Lemma 5.1 of \cite{Zhang-geom}.
%%
%The first one follows directly by our construction:
%%
%\begin{claim} \label{claim:V31}
%For every path $s \tran u$ in $V$ of length $\ell$,
%there is a path $\essdvass s \tran \essdvass u$ in $\essdvass V$ of length $3\ell$.
%%Up to multiplicative constant 3, the lengths of paths 
%%$s \tran t$ in $V$ are the same as
%%the lenghts of paths
%%$\essdvass s \tran \essdvass t$ in $\essdvass V$.
%\end{claim}
%%
%Also by the construction, every path $\essdvass s \tran u'$ in $\essdvass V$
%ending in a non-auxiliary state, i.e.~$u' = \pair q b(v')$ for some $v'\in\N^2$, 
%has length divisible by 3.
%Furthermore, the transitions \eqref{eq:checkV} ensure that the target
%configuration $u'$ is indeed corresponding to a configuration of $V$.
%Therefore, one can reconstructs a corresponding path in $V$:
%
%%
%\begin{claim} \label{claim:V32}
%For every path $\essdvass s \tran u'$ in $\essdvass V$ of length $3\ell$,
%there is a path $s \tran u$ in $V$ with $\essdvass u = u'$, of length $\ell$.
%\end{claim}
%
Finally, we argue that the size of $\essdvass V$ is bounded polynomially with respect to the size of $V$:
%
\begin{claim} \label{claim:sizeVV}
$\size(\essdvass V) \leq R(M) = \OO(M\cdot B).$
\end{claim}
%
\begin{proof}
Recalling Claims \ref{claim:a} and \ref{claim:ax}, namely
$\norm(a)\leq D$, $B=\OO(M\cdot D)$, we deduce that 
transitions \eqref{eq:tranV} contribute at most $(2B+1)M \leq \OO(M\cdot B)$ to the size of $\essdvass V$.
%Estimating roughly, for every $b\in\setfromto {-B} B$,
%the set $N_b$ contains vectors of norm at most $|b|$ (because $a_1$ and $a_2$ are nonnegative) and therefore there at most $b^2+1$ such vectors. 
%Therefore transitions \eqref{eq:checkV} contribute at most 
%$2(2B+1)b^2(|b|+1) \leq 2(2B+1)(B^3+B) \leq \OO(M^4\cdot D^4)$ to the size of $\essdvass V$.
%In total, $\size(\essdvass V) \leq \OO(M^4\cdot D^4)$.
\end{proof}
We are now prepared to complete Case 2.
Let $P$ be the polynomial witnessing Lemma \ref{lem:geom-plb}, i.e.,
\geomvass are \lb by $P$.
As $V$ has a path $s\tran t$, 
By Claim \ref{claim:lenlen}, $\essdvass V$ has a path $\essdvass s \tran \essdvass t$.
By Lemma \ref{lem:geom-plb}, $\essdvass V$ has thus a path $\essdvass s \tran \essdvass t$
of length at most $P(\size(\essdvass V))$, i.e., relying on Claim \ref{claim:sizeVV},
of length at most $P(\OO(M^2\cdot D)) = \OO(P(M^{4}))$.
By Claim \ref{claim:lenlen} again, we get a path $s\tran t$ in $V$ of length
$\OO(P(M^{4}))$.
This completes Case 2.


\para{Case 3. $(V, s, t)$ is non-diagonal}

\Wlog assume that $(V, s)$ is not forward-diagonal (otherwise replace $V$ by $\rev V$).
Therefore for all states $q$ the configuration $s'=q(w+(\vec{M{+}1}))$ is not
coverable from $s=p(w)$. 
Indeed, using strong-connectedness of $V$, if $s'$ were coverable from $s$ then
the configuration $p(w + \vec 1)$ would be coverable form $p(w)$, by 
extending the covering path of $s'$ with an arbitrary shortest path back to state $p$ 
(that cannot decrease a counter by more than $M$), which would contradict forward non-diagonality.

By Lemma \ref{lem:sim-high-d},
in every path from $s$, some coordinate
$j\in\setto 3$ is bounded by $B:=P_3(M, M, M+1)$ (we take $M$ as an upper bound for $n$ and $N$,
relying on $P_3$ being nondecreasing, and take $U=M+1$).
This property allows us, intuitively speaking, to describe all the paths of $V$ by paths of three
\geomvass $V_j$, for $j\in\setto 3$, where $V_j$ behaves exactly like $V$ except that dimension $j$ is additionally kept in state.
%For $w\in\N^3$ and $j\in\setto 3$, let $\dropped w j\in\N^2$ denote $w$ without $j$th coordinate.
%
Formally, let $V_j := (Q_j, T_j)$, where 
\begin{align*}
Q_j = & \setof{\pair q b}{q\in Q, \ b \in \setfromto 0 B} \\
T_j =  & \setof{(\pair q b, v, \pair {q'}{b'})}{(q, v, q')\in T, \ b' = b + v_j}.
\end{align*}
%
The source and target configurations in $V_j$ are $s_j = \pair p {w_j}(w)$ and $t_j = \pair {p'}{w'_j}(w')$,
and there is a tight correspondence between paths in $V$ and paths in $V_1, V_2, V_3$:
%
\begin{claim} \label{claim:nondiag2vass}
$\Len V s t = \Len {V_1} {s_1} {t_1} \cup \Len {V_2} {s_2} {t_2} \cup \Len {V_3} {s_3} {t_3}$.
%$p(w) \tran p'(w')$ in $V$ if and only if
%$\pair p {w_j}(\dropped w j) \tran \pair{p'}{w'_j}(\dropped {w'} j)$ in $V_j$ for some $j\in\setto 3$, by a path
%of the same length.
\end{claim}
%
The size of each of $V_j$ is bounded polynomially with respect to the size of $V$:
%
\begin{claim} \label{claim:nondiagsize}
The size of each of $V_j$ is at most $R(M)=\OO(M\cdot B)$.
\end{claim}
%
Let $P$ be the polynomial witnessing Lemma \ref{lem:geom-plb}.
As $p(w) \tran p'(w')$ in $V$,
by Claim \ref{claim:nondiag2vass}
there is a path $\pair p {w_j}(w) \tran \pair{p'}{w'_j}({w'})$ in $V_j$
for some $j\in\setto 3$.
Therefore, by Lemma \ref{lem:geom-plb} there is such a path of length 
at most $P(R(M))$ which,
again using Claim \ref{claim:nondiag2vass}, implies a path $p(w) \tran p'(w')$ in $V$ of the same length.
%
This polynomial bound completes Case 3, and hence also the proof of
Lemma \ref{lem:1comp}.
%
%\end{proof}
