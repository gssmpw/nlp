% !TEX root = main.tex

\section{Preliminaries}\label{sec:def}

Let $\Q, \Qnonneg, \Qpos$ denote the set of all, nonnegative, and positive rationals, respectively,
and likewise let $\Z, \N, \Npos$ denote the respective sets of integers.
For $a, b \in \Z$, $a \leq b$, let $\setfromto a b$ denote the set $\{x \in \Z \mid a \leq x \leq b\}$.
The $j$th coordinate of a vector $w\in\Q^d$ we write as $w_j$.
Thus $w = (w_1, \ldots, w_d)$.
%By $\dropped w j\in\Q^{d-1}$ we denote $w$ \emph{without} $j$th coordinate, e.g.,
%$\dropped w 1 = (w_2, \ldots, w_d)$.
For $q\in \Q$,
by $\vec q$ we denote the constant vector $(q, \ldots, q) \in \Q^d$.

\para{Vector addition systems with states}

Let $d\in\N\setminus \{0\}$.
A $d$-dimensional \emph{vector addition system with states} (\parvass d in short) $V=(Q,T)$ consists of
a finite set $Q$ of states, and a finite set of transitions $T\subseteq Q\times \Z^d \times Q$.
A \emph{configuration} $c$ of $V$ consists of a state $q\in Q$ and a nonnegative vector $w\in\N^d$,
and is written as $c=q(w)$.
A transition $u=(q,v,q')$ induces \emph{steps} $q(w) \trans{u} q'(w')$ between configurations, where $w' = w+v$.
We refer to the vector $v\in\Z^d$ as the \emph{effect} of the transition $(q,v,q')$ or of an induced step.
A \emph{path} $\pi$ in $V$ is a sequence of steps with the proviso that the target configuration of every step matches
the source configuration of the next one:
\begin{align} \label{eq:path}
\pi \ = \ c_0 \trans{u_1} c_1 \trans{} \ldots \trans{u_n} c_n.
\end{align}
The \emph{effect} $\eff(\pi)\in\Z^d$
of a path is the sum of effect of all steps, and its \emph{length} is the number $n$ of steps.
We say that the path is \emph{from} $c_0$ \emph{to} $c_n$, call $c_0, c_n$ source and target 
configuration, respectively, of the path, and write $c_0\trans{\pi}c_n$.
We also write $c \tran c'$ if there is some path from $c$ to $c'$.
A path $q(v) \tran q(v')$ in $V$ with the same source and target state
we call a \emph{cycle}.
A cycle is \emph{simple} if the only equality of states along the cycle is the equality of source and target states.
When dimension $d$ is irrelevant, we write \vass instead of \parvass d.

By  \emph{geometric} dimension of a \parvass d we mean the dimension of the vector
space $\Lin V \subseteq \Q^d$ spanned by effects of all its simple cycles.
%We say that \tvass is geometrically $2$-dimensional if linear space spanned by effects of its simple cycles is at most $2$ dimensional.
%
In the sequel we most often consider \dvass and \tvass, but also \emph{\geomvass}, i.e.,
\tvass of geometric dimension at most 2.

Two paths $\pi$ and $\pi'$ can be \emph{concatenated} (composed), written $\pi; \, \pi'$, if
the target configuration of $\pi$ equals the source one of $\pi'$.
As long as it does not lead to confusion, 
we adopt a convention that when concatenating paths $\pi;\,\pi'$, 
the latter path $\pi'$ is silently \emph{moved}
so that its source matches the target of $\pi$, 
under assumption that the source state of $\pi'$ is the same as the target state of $\pi$.
%we sometimes adopt terminology that does not distinguish a path \eqref{eq:path} from a 
%formally different path that executes 
%the same transitions $u_1, \ldots, u_n$ but from a different source configuration 
%(with the same state).
For instance, we write $\pi^m$ to denote the $m$-fold concatenation of a cycle $\pi$, even if $\eff(\pi)\neq\vec 0$.
%

We use the following notation for measuring size of representation of a \vass.
By \emph{norm} of a vector $v=(v_1, \ldots, v_d) \in \Q^d$, denoted $\norm(v)$, 
we mean the sum of absolute values of all numbers appearing in it:
$\norm(v) = \absv{v_1} + \ldots + \absv{v_d}$; and 
by norm of a set of vectors $P$ we mean the sum of norms of its element:
$\norm(P) = \sum\setof{\norm(v)}{v\in P}$. 
By \emph{norm} of a configuration $q(w)$, or of a transition $(q, w, q')$,
we mean the norm of its vector $w$.
By \emph{size} of a \vass $V$, denoted $\size(V)$, we mean
the sum of norms of all its transitions, 
plus the number of transitions $\card T$.
%\slawek{changed def of norm of a set, removed def. of norm of a vass}

We often implicitly extend a \vass $V$ with source configuration $s$, or with
a pair of source and target configurations $s, t$.
Slightly overloading terminology, a pair $(V, s)$ and a triple $(V, s, t)$ we call a \vass too.
For convenience we overload further and put $\size(V,s) = \size(V) + \norm(s)$ and 
$\size(V, s, t) = \size(V) + \norm(s) + \norm(t)$.
The \emph{reverse} of a \vass $V = (Q, T)$ is defined as $\rev V = (Q, T')$,
where $T'$ is obtained by reversing all transitions in $T$:
$T' = \setof{(q', -v, q)}{(q,v,q')\in T}$.
Overloading the notation again, we put $\rev{(V, s, t)} := (\rev V, t, s)$.

Given a \vass together with an initial configuration $(V, s)$, we write 
$\reach(V,s)$ to denote the set of configurations $t$ such that $V$ has a path from $s$ to $t$.
For every state $q\in Q$ we write $\reach_q(V,s)$ to denote the
set of vectors $w\in\N^d$ such that $q(w) \in \reach(V,s)$.
We write shortly $\reach(s)$ and $\reach_q(s)$ if the \vass $V$
is clear from the context.
If $t\in\reach(s)$ we say that $t$ is \emph{reachable} from $s$.
If $t+\Delta\in\reach(s)$ for some $\Delta\in\N^d$, we say that $t$ is \emph{coverable} from $s$.

We consider a variant of \vass, called \zvass, where the nonnegativeness constraint is dropped.
Syntactically, \zvass is the same as \vass, namely consists of a finite set of states and a finite set of transitions
$(Q, T)$. 
Semantically, configurations of a \zvass are $Q \times \Z^d$, while
all definitions (path, reachability set, etc.) are the same as in case of \vass.
Equivalently, we may also speak of $\Z$-configurations and $\Z$-paths of a \vass, i.e., 
configurations and paths where the nonnegativeness constraint is dropped.
Note that every $\Z$-path $q(w) \trans{\pi} q'(w')$ may be \emph{lifted} to become a path
$q(w+\Delta) \trans{\pi} q'(w'+\Delta)$, for some $\Delta \in \N^d$.
%We keep the notation light and allow ourselves to use the same symbol $\pi$ for both
%a $\Z$-path and a lifted path.
%Intuitively, in case of $\Z$-paths the exact source and target vectors are not that relevant,
%only their difference is so.


\para{Sequential \vass}

We define the state graph of a \vass $V = (Q, T)$: nodes are states $Q$, and there is an edge
$(q,q')$ if $T$ contains a transition $(q, v, q')$ for some $v\in\Z^d$.
A \vass is called \emph{strongly connected} if its state graph is so.
A \vass $V = (Q, T)$ is called \emph{sequential},
if it can be partitioned into a number of strongly connected 
\vass $V_1=(Q_1, T_1), \ldots, V_k=(Q_k, T_k)$ with pairwise disjoint state spaces, 
and $k-1$ transitions
$u_i=(q_i, v_i, q'_i)$, for $i\in\setto{k-1}$, where $q_i\in Q_i$ and $q'_i \in Q_{i+1}$
(recall Figure \ref{fig:kcompvass}).
%
Thus $Q=Q_1\cup\ldots\cup Q_k$ and 
$T=T_1 \cup \ldots \cup T_k \cup \{u_1, \ldots, u_{k-1}\}$.
We call $V$ a \emph{$k$-component} sequential \vass,
or $k$-component \vass in short,
and write down succinctly as 
$
%\begin{align} \label{eq:kvass}
V  =  \ktvass.
%\end{align}
$
The \vass $V_1, \ldots, V_k$ are called \emph{components}, and transitions
$u_1, \ldots, u_{k-1}$  \emph{bridges}.
By definition, a $1$-component sequential \vass is just a strongly connected \vass.


\para{Integer solutions of linear systems}
We will intensively use the following immediate corollary of \cite{Pottier91}
(see also \cite[Prop.~4]{taming}):
%
\begin{lemma}%[\cite{Pottier91}\slawek{zacytowac konkretne tw.}]
[\cite{taming}, Prop.~4]
\label{lem:taming}
Consider a system $A\cdot x = b$ of $m$ Diophantine linear equations with $n$ unknowns,
where absolute values of coefficients are bounded by $N$.
Every pointwise minimal nonnegative integer solution has norm at most
$\OO(nN)^m$.
\end{lemma}


\para{Diagonal property}

We prove that
if there is a path in a \vass achieving a large value on every coordinate,
then there is a path of bounded length that achieves \emph{simultaneously} large values on all coordinates.%
\footnote{Lemma \ref{lem:sim-high-d} is inspired by \cite[Lemma 4.13]{LS19}, 
but the statement and the proof are different.}
We consider a general case of arbitrary dimension,
as we believe it is of an independent interest, but in the sequel we will use it only for dimension $d = 3$.

\begin{lemma}\label{lem:sim-high-d}
For every $d \in \N$ there are nondecreasing polynomials $P_d, R_d$ such that 
for every \parvass d $(V, s)$ of norm $N$, with $n$ states, and for every $U\in\N$,
if $V$ has a path from $s$ that for every $i\in\setto d$ contains a configuration 
$q(w_1, \ldots, w_d)$ with $w_i \geq P_d(n, N, U)$,
then $V$ has also a path $s \tran q(w_1, \ldots, w_d)$ of length at most $R_d(n, N, U)$
such that $w_i \geq U$ for every $i\in\setto d$.
\end{lemma}

\begin{appendixproof}[Proof of Lemma \ref{lem:sim-high-d}]
Below, we will refer to two inductively defined sequences $(H_i)_{i\in\Npos}$, $(L_i)_{i\in\Npos}$,
(implicitly) parametrised
by numbers $n, N, U \in \N$:
let $H_1 := U$, $L_1 = n U$, and for  $i > 1$ let $H_i = U + N L_{i-1}$ and  
$L_i = n (H_i)^i + L_{i-1}$.

\begin{lemma}\label{lem:sim-high}
Let $d \in \N$, and let $(V, s)$ be a \parvass d of norm $N$, with $n$ states.
If $V$ has a path from $s$ that for every $i\in\setto d$ contains a configuration 
$q(w_1, \ldots, w_d)$ with $w_i \geq H_d$,
then $V$ has also a path $s \tran q(w_1, \ldots, w_d)$ of length at most $L_d$
such that $w_i \geq U$ for every $i\in\setto d$.
\end{lemma}


\begin{proof}%[Proof of Lemma \ref{lem:sim-high}]
The proof proceeds by induction over the dimension $d$, and 
follows the idea of Rackoff~\cite[Lemma 3.4]{DBLP:journals/tcs/Rackoff78}.
Let $V$ be a fixed \parvass d.

When $d = 1$, as $H_1 = U$ the first claim is immediate.
The bound on length $L_1 = n U$ is also obtained immediately, by removing repetitions of
configurations.

For the induction step, we assume that Lemma~\ref{lem:sim-high} holds for dimension $d-1$,
and show it for dimension $d$.
Consider the  shortest 
path $s\trans{\rho} u$ from $s$ in $V$ such that for every $i\in\setto d$, the path contains a configuration 
$q(w_{1}, \ldots, w_{d})$ with $w_{i} \geq H_d$.
Let $q(w_1, \ldots, w_d)$ be the first  configuration  on $\rho$ with
$(w_1, \ldots, w_d) \notin {\setfromto 0 {H_d-1}}^d$:
\[
s \trans {\rho_1} q(w_1, \ldots, w_d) \trans{\rho_2} u.
\]
\Wlog we may assume that $w_d \geq H_d$. 
The length of $\rho_1$ is at most $n (H_d)^d$, as the 
configurations along $\rho_1$ are bounded by $H_d -1$ and do not repeat.

Let $\essdvass V$ denote the \parvass {(d-1)} obtained by dropping the $d$th coordinate of $V$.
By the induction assumption, $\essdvass V$ has a path 
\begin{align*}%\label{eq:indass}
q(w_1, \ldots, w_{d-1}) \trans{\rho_3} p(v_1, \ldots, v_{d-1})
\end{align*}
of length at most $L_{d-1}$, whose target vector satisfies $v_i \geq U$ for all $i\in\setto {d-1}$.
Steps of $\essdvass V$ are steps of $V$ where the $d$th coordinate is dropped,
and each such step may decrease the value on $d$th coordinate by at most $N$.
Therefore $w_d\geq H_d=U+N L_{d-1}$ is large enough so that $L_{d-1}$ steps of 
$\rho_3$ yield at least $U$ on the $d$th coordinate.
Therefore $\rho_3$  can be traced back to a path 
\[
q(w_1, \ldots, w_{d}) \trans{\rho_4} p(v_1, \ldots, v_{d})
\]
in $V$, of the same length as $\rho_3$, where $v_d\geq U$.
The concatenated path $\rho_1; \, \rho_4$ has length at most
$L_d = n (H_d)^d + L_{d-1}$ and hence satisfies the claim of Lemma~\ref{lem:sim-high}.
\end{proof}

We show that $H_i$ and $L_i$ are bounded by a polynomial in $n, N, U$, of
degree doubly exponential in dimension $d$.
We concentrate on $H_i$, as
$H_i \leq L_i \leq H_{i+1}$.

\begin{proposition}\label{prop:hd-bound}
$H_i \leq (4Nn)^{2i!} U^{i!}$.
\end{proposition}

%\begin{proof}
\begin{proof}%[Proof of Proposition \ref{prop:hd-bound}]
Let $C = 4Nn$.
As $H_i \leq 2 N L_{i-1}$ and $L_i \leq 2n (H_i)^i$, we have:
\begin{equation}\label{eq:hd-bound}
H_i \leq C \cdot (H_{i-1})^{i-1}.
\end{equation}
Instead of showing  $H_i \leq C^{2i!} U^{i!}$, we
prove a slightly stronger inequality:
\[
H_i \leq C^{2(i-1)! - 1} U^{(i-1)!},
\]
by induction on $i$.
When $i = 1$ we have $H_1 = U \leq C U$, and when
$i = 2$, by \eqref{eq:hd-bound}
we have $H_2 \leq C H_1 = C U$, as required.
The induction step follows by:
\[
H_{i+1} \leq C \cdot (H_{i})^{i} \leq C \cdot (C^{2(i-1)! -1} U^{(i-1)!})^{i} \leq C^{2 i! - 1} U^{i!},
\]
where first inequality is \eqref{eq:hd-bound}, 
the second one is the inductive assumption, and the latter one follows
since $1 + (2(i-1)! -1)\cdot i \leq 2i! -1$ when $i\geq 2$.
This completes the proof.
%\end{proof}
\end{proof}
%
Lemma~\ref{lem:sim-high}, Proposition~\ref{prop:hd-bound} and the inequality $L_i \leq H_{i+1}$
entail Lemma \ref{lem:sim-high-d}.
\end{appendixproof}


\para{Length-bound on shortest path}

A function $f:\N\to\N$ is called \emph{nondecreasing} if $f(n) \geq n$ for every $n \in \N$,
and $f(n)< f(m)$ for all $n<m$.
Functions used in the sequel are most often nondecreasing.
%
We say that a class $\C$ of \vass of \zvass is \emph{\lb} by a 
non-decreasing function $f:\N\to\N$ if 
for every $(V, s, t)$ in $\C$, if $s \tran t$ then 
$s \trans{\pi} t$ for some path $\pi$ of length at most $f(\size(V, s, t))$.
A class $\C$ which is \lb  by some nondecreasing
polynomial we call \emph{\plb}.
It is known that \dvass  have this property:
%
\begin{lemma}[\cite{DBLP:journals/jacm/BlondinEFGHLMT21}, Theorem 3.2]\label{lem:2vass-plb}
\dvass are \plb.%
\footnote{\cite{DBLP:journals/jacm/BlondinEFGHLMT21} adopts a slightly different, but equivalent
up to a constant multiplicative factor, definition of norm and 
size.
}
\end{lemma}
%
As a corollary of Lemmas \ref{lem:2vass-plb} and \ref{lem:taming}, respectively,
we derive the property for \geomvass 
(using \cite[Lemma 5.1]{Zhang-geom}) and \tzvass, respectively:
%
\begin{lemma}%[\cite{Zhang-geom}, Lemma 5.1]
\label{lem:geom-plb}
\Geomvass are \plb.
\end{lemma}
%
\begin{appendixproof}[Proof of Lemma \ref{lem:geom-plb}]
We rely on the construction of \cite[Lemma 5.1]{Zhang-geom}, which transforms a given 
\geomvass $(V, s, t)$ into a \dvass $(\essdvass V, \essdvass s, \essdvass t)$ such that%
\footnote{
In \cite{Zhang-geom} only zero source and target vectors are considered, but the construction
routinely extends to arbitrary such vectors. 
}
\[
\Len {\essdvass V} {\essdvass s} {\essdvass t} \ = \ 3 \cdot \Len V s t,
\]
and the size of the \dvass is only polynomially larger than the size of the original \geomvass. 
Therefore, as \dvass are \plb due to Lemma \ref{lem:2vass-plb}, so are also \geomvass.
\end{appendixproof}
%
\begin{lemma} \label{lem:zvass-plb}
\tzvass are \plb.
\end{lemma}
%
%\begin{proof}
\begin{appendixproof}[Proof of Lemma \ref{lem:zvass-plb}]
%\slawek{could go to appendix}
Let $(V, s, t)$ be a \tzvass such that $s\trans{\sigma} t$.
Let $V=(Q,T)$, $s=q(w)$ and $t=q'(w')$.
Let $M = \size(V, s, t) = \size(V) + \norm(s) + \norm(t)$.
We express a path as a solution of a Diophantine system of linear equations, and rely
on Lemma \ref{lem:taming}.

Let $Q'\subseteq Q$ and $T'\subseteq T$ be the subsets of states and transitions that appear in $\sigma$.
Simple cycles that use only transitions from $T'$ we call $T'$-cycles.
%The (undirected) state graph restricted to $T'$ is thus connected.
The $\Z$-path $s\trans{\sigma} t$ decomposes into a $\Z$-path $\sigma_0$ that visits all states of $Q'$,
plus a number of $T'$-cycles.
Choose the shortest such $\sigma_0$.
The $\Z$-path $\sigma_0$ visits each state at most $\card Q$ times, as otherwise it could be shortened,
and therefore its effect has norm at most $M^2$.
The effect $\Delta$ of each $T'$-cycle has norm at most $M$, as it contains no repetition of a transition.
We choose, for each such vector $\Delta$, one of $T'$-cycles with effect $\Delta$.
Let $\cal C$ be the set of chosen $T'$-cycles. 
Its size is at most $(2M+1)^3 \leq \OO(M^3)$.
We define a system $\cal U$
of 3 linear equations (one for each dimension), whose unknowns $x_\delta$ correspond to $T'$-cycles $\delta$ from $\cal C$:
\[
\sum_{\delta\in \mathcal{C}} x_{\delta} \cdot e_\delta \ + \ e_{\sigma_0} \ = \ t - s,
\]
where $e_\delta\in\Z^3$ is the effect of $\delta$, 
and $e_{\sigma_0}\in\Z^3$ is the effect of $\sigma_0$.
The system has a nonnegative integer solution, namely the one obtained from decomposition of $\sigma$
into $\sigma_0$ and simple cycles.
As all coefficients of $\cal U$ are bounded by $M^2$,
by Lemma \ref{lem:taming} the system has a solution of norm $\OO(M^3\cdot M^2)^3=\OO(M^{15})$.
The solution $(x_\delta)_\delta$ 
yields a $\Z$-path $s\tran t$ of length $\OO(M^{16})$, consisting of $\sigma_0$ with
attached all cycles $\delta\in\mathcal{C}$
(this is possible, as $\sigma_0$ visits all states used by the cycles),
each $\delta$ iterated $x_\delta$ times.
This completes the proof.
\end{appendixproof}
%
Lemma \ref{lem:main} states that there exists a constant $C\in\N$ such that
for every $k\in\N$, the $k$-component \tvass are 
length-bounded by the function $M\mapsto \kbound {M} {C\cdot k}$.
Therefore for every fixed $k\in\N$, the $k$-component \tvass are \plb,
even if the degree of polynomial grows doubly exponentially in $k$.

