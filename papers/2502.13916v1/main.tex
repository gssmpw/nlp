\documentclass[a4paper, UKenglish, cleveref, autoref, thm-restate]{lipics-v2021}

\usepackage{cite}
%\usepackage{amsmath,amssymb,amsfonts}
%\usepackage{algorithmic}
\usepackage{graphicx}
\usepackage{textcomp}
\usepackage{xcolor}

\usepackage{xspace}
\usepackage{bbm}
\usepackage{comment}
\usepackage{amsmath}
\usepackage{booktabs}
\usepackage{amsfonts}
\usepackage{amssymb}
\usepackage{amsthm}
\usepackage{thmtools,thm-restate}
\usepackage[textsize=small]{todonotes}
\usepackage{tikz}
\usetikzlibrary{arrows,arrows.meta,automata,positioning} 
\usepackage{stmaryrd}
\usepackage{mathdots}
\usepackage{algorithm}
\usepackage[noend]{algpseudocode}
\usepackage{thmtools}
\usepackage{lineno}
\usepackage{enumerate}
\usepackage{apxproof}
\usepackage{hyperref}
\usepackage{url}
%\usepackage[bookmarks,bookmarksopen,bookmarksdepth=2]{hyperref}
%\usepackage[appendix=strip]{apxproof}
\renewcommand{\appendixprelim}{\newpage}

\addtolength{\textwidth}{1cm}
\addtolength{\oddsidemargin}{-0.5cm}
\addtolength{\evensidemargin}{-0.5cm}

\addtolength{\textheight}{0.5cm}
\addtolength{\topmargin}{0.5cm}


\title{Reachability in 3-VASS is Elementary}

% anonymised


\author{Wojciech Czerwi\'nski}
{University of Warsaw}
{wczerwin@mimuw.edu.pl}
{https://orcid.org/0000-0002-6169-868X}
{Supported by the ERC grant INFSYS, agreement no. 950398.}

\author{Isma{\"e}l Jecker}
{FEMTO-ST, CNRS, Univ. Franche-Comté, France}
{ismael.jecker@gmail.com}
{}
{}

\author{Sławomir Lasota}
{University of Warsaw}
{s.lasota@uw.edu.pl}
{https://orcid.org/0000-0001-8674-4470}
{Supported by the ERC grant INFSYS, agreement no. 950398 and by the NCN grant 2021/41/B/ST6/00535.}

\author{Łukasz Orlikowski}
{University of Warsaw}
{l.orlikowski@.mimuw.edu.pl}
{https://orcid.org/0009-0001-4727-2068}
{Supported by the ERC grant INFSYS, agreement no. 950398.}


\authorrunning{W. Czerwi{\'n}ski, I. Jecker, S. Lasota, and {\L}. Orlikowski} 
\Copyright{Wojciech Czerwi{\'n}ski, Isma{\"e}l Jecker, S{\l}awomir Lasota, and {\L}ukasz Orlikowski}

\ccsdesc[500]{Theory of computation~Parallel computing models}
\keywords{vector addition systems, Petri nets, reachability problem, dimension three, doubly exponential space, length of shortest path} 

\pagestyle{plain}
\pagenumbering{arabic}

\newcommand{\thought}[1]{{\color[rgb]{0.2,0.39,0.66}(#1)}}
\newcommand{\todo}[1]{{\color[rgb]{1.0,0.0,0.0}(#1)}}
\newcommand{\hsh}[1]{{\color{green!50!black} Henrik: #1}}
\newcommand{\st}[1]{{\color{red!50!black} Sebastian: #1}}

\newcommand{\ulm}[1]{_{\scaleto{\mathrm{#1}}{3pt}}}
\newcommand\at[2]{\left.#1\right|_{#2}}











\newtheorem{assumption}{Assumption}

\DeclareMathOperator*{\argmax}{arg\,max}
\DeclareMathOperator*{\argmin}{arg\,min}

\newcommand{\swname}[1]{\texttt{#1}}
\newcommand{\ie}{i\/.\/e\/.,\/~}
\newcommand{\eg}{e\/.\/g\/.,\/~}
\newcommand{\cf}{cf\/.\/~}

\newcommand{\fig}{Fig\/.\/~}
\newcommand{\defn}{Def\/.\/~}
\newcommand{\sect}{Sec\/.\/~}
\newcommand{\tabl}{Tab\/.\/~}
\newcommand{\algo}{Algorithm~}
\newcommand{\theo}{Theorem~}

\newcommand{\bnnl}{3 hidden layers}
\newcommand{\bnnn}{50 neurons}
\newcommand{\bnna}{tanh activations}

\newcommand{\capt}[1]{\mdseries{\emph{#1}}}

\newcommand{\videolink}{at \url{https://youtu.be/_d7AqTRjz6g}}
\newcommand{\codelink}{\url{https://github.com/wheelbot/mini-wheelbot}}

\newcommand{\fakepar}[1]{\vspace{0mm}\noindent\textbf{#1.}}

\newcommand{\needref}{\textcolor{red}{[REF]}}

\newcommand{\plotfontsize}{9pt}

% !TEX root = main.tex

%\newtheorem{theorem}{Theorem}
%\newtheorem{corollary}[theorem]{Corollary}
%\newtheorem{lemma}[theorem]{Lemma}
%\newtheorem{proposition}[theorem]{Proposition}
%\newtheorem{claim}[theorem]{Claim}
%\newtheorem{definition}[theorem]{Definition}
%\newtheorem{example}[theorem]{Example}

\renewcommand{\vec}[1]{\overrightarrow{#1}}

\newcommand{\subseteqfin}{\subseteq_{\fin}}
\newcommand{\LPS}{\lps}
\newcommand{\SLPS}{\slps}
\newcommand{\kanapka}[2]{(#1, #2)-approximately semi-linear}
\newcommand{\lb}{length-bounded\xspace}
\newcommand{\plb}{polynomially length-bounded\xspace}

\newcommand{\kbound}[2]{{#1}^{2^{2^{#2}}}}
\newcommand{\HH}[2]{H_{#1}(#2)}

%\newcommand{\slawek}[1]{\todo[color=yellow]{S: #1}}
%\newcommand{\slawekinline}[1]{\todo[color=yellow,inline]{S: #1}}
%\newcommand{\sltodo}{\todo[color=yellow,inline]{TODO}}
\newcommand{\slawek}[1]{\textcolor{blue}{\bf S: #1}}
\newcommand{\slawekinline}[1]{\slawek{#1}}
\newcommand{\sltodo}{\slawek{TODO}}
\newcommand{\Lin}[1]{\textsc{Lin}(#1)}
\newcommand{\Len}[3]{\textsc{Len}(#1, #2, #3)}
\newcommand{\OO}{{\cal O}}
\newcommand{\trim}{$(a, B, c)$-trim }
\newcommand{\mytrim}[2]{(#1, #2)-trim}
\newcommand{\vass}{{\sc vass}\xspace}
\newcommand{\zvass}{$\Z$-{\sc vass}\xspace}
\newcommand{\lps}{\textsc{lps}\xspace}
\newcommand{\dlps}{2-\lps}
\newcommand{\dslps}{2-\slps}
\newcommand{\dvass}{\parvass 2}
\newcommand{\tvass}{\parvass 3}
\newcommand{\geomvass}{geometrically 2-dimensional \tvass}
\newcommand{\Geomvass}{Geometrically 2-dimensional \tvass}
\newcommand{\tzvass}{\parzvass 3}
\newcommand{\parvass}[1]{{$#1$-\vass}\xspace}
\newcommand{\parzvass}[1]{{$#1$-$\Z$-\vass}\xspace}
\newcommand{\ltvass}{(V_1) u_1 (V_2) u_2 \ldots u_{\ell-1} (V_\ell)}
\newcommand{\ktvass}{(V_1) u_1 (V_2) u_2 \ldots u_{k-1} (V_k)}
\newcommand{\ktvasstwo}{(V_2) u_2 \ldots u_{k-1} (V_k)}
\newcommand{\ktvassmod}{(\essdvass V_i) \essdvass u_{i, b} (V_2) u_2 \ldots u_{k-1} (V_k)}
\newcommand{\ktvassmodmod}{(\essdvass V) \essdvass u_{b} (V_2) u_2 \ldots u_{k-1} (V_k)}
\renewcommand{\C}{\mathcal{C}}
\newcommand{\para}[1]{\vspace{-3mm}\subparagraph*{\bf #1.}}
\newcommand{\setfromto}[2]{[#1, #2]}
\newcommand{\setto}[1]{\setfromto 1 {#1}}
\newcommand{\Wlog}{W.l.o.g.~}
\newcommand{\mywlog}{w.l.o.g.~}
\newcommand{\rev}[1]{{#1}^{\text{rev}}}
\newcommand{\DI}{diagonal intersecting}
\newcommand{\DNI}{diagonal non-intersecting}
\newcommand{\ND}{non-diagonal}
%\newcommand{\DI}{(\textsc{DI})}
%\newcommand{\DNI}{(\textsc{DNI})}
%\newcommand{\ND}{(\textsc{ND})}
\newcommand{\cone}[1]{\textsc{Cone}(#1)}
\newcommand{\seqcone}[1]{\textsc{SeqCone}(#1)}
\newcommand{\Qnonneg}{\Q_{\geq 0}}
\newcommand{\Npos}{\N_{>0}}
\newcommand{\inter}[1]{\textsc{Int}(#1)}
\newcommand{\essdvass}[1]{\overline{#1}}
\newcommand{\innprod}[2]{#1 \diamond #2} %\stackrel{{\centerdot}}{} #2}
\newcommand{\pair}[2]{#1_{#2}}
\newcommand{\triple}[3]{#1_{#2, #3}}
%\newcommand{\pair}[2]{\langle #1, #2 \rangle}
%\newcommand{\triple}[3]{\langle #1, #2, #3 \rangle}
\newcommand{\pumps}[1]{\textsc{FP}(#1)}
\newcommand{\absv}[1]{|#1|}
%\newcommand{\dropped}[2]{#1_{\setminus{#2}}}
\newcommand{\rhs}{\textsc{rhs}}

\newcommand{\probdef}[3]{
\begin{itemize} 
    %[leftmargin=2.0cm,labelsep=0.5cm]
    %\setlength\itemsep{0.05cm}
    \item[] \textsc{#1}
    \item[] \textit{Input:} \quad\, #2.
    \item[] \textit{Question:} #3?
\end{itemize}
}



\nolinenumbers

\begin{document}

\maketitle

\begin{abstract}
The reachability problem in 3-dimensional vector addition systems with states (3-VASS) is known to
be PSpace-hard, and to belong to Tower.
We significantly narrow down the complexity gap by proving the problem to be solvable 
in doubly-exponential space.
The result follows from a new upper bound on the length of the shortest path: 
if there is a path between two configurations of a 3-VASS then there is also
one of at most triply-exponential length.
We show it by introducing a novel technique of approximating the reachability sets of 2-VASS by small semi-linear sets.
\end{abstract}


\maketitle

\section{Introduction}
\label{sec:intro}

\begin{figure*}[tb]
    \centering
    \includegraphics[width=0.848\linewidth]{figs/circuitnn.pdf} 
    \caption{Illustration of differentiable CircuitNN. CircuitNN is designed based on differentiable NAND gates. After DAS is guided by PI and PO pairs of the truth table, CircuitNN can get the precise circuit architecture logic equivalent to the truth table.}
    \label{fig:circuitnn}
\end{figure*}

% 1. Describe the importance of logic synthesis
% 2. Existing Problems
% (a) Neural Architecture Search: Unstable, Predefined Setting, etc.
% (b) Circuit Generation: Probabilistic Model, Logic Equivalence

With the rapid advancement of technology, the scale of integrated circuits (ICs) has expanded exponentially. 
This expansion has introduced significant challenges in chip manufacturing, particularly concerning power and area metrics.
A primary objective in IC design is achieving the same circuit function with fewer transistors, thereby reducing power usage and area occupancy.

Logic synthesis~\cite{hachtel2005logicsynth}, a critical step in electronic design automation (EDA), transforms behavioral-level circuit designs into optimized gate-level circuits, ultimately yielding the final IC layout. 
The primary goal of logic synthesis is to identify the physical implementation with the fewest gates for a given circuit function. 
This task constitutes a challenging NP-hard combinatorial optimization problem. 
Current logic synthesis tools~\cite{brayton2010abc, wolf2013yosys} rely on human-designed heuristics, often leading to sub-optimal outcomes.

Differentiable architecture search (DAS) techniques~\cite{liu2018darts, chu2020darts} offer novel perspectives on addressing challenges in this problem.
Circuit functions can be represented through truth tables, which map binary inputs to their corresponding outputs. 
Truth tables provide a precise representation of input-output relationships, ensuring the design of functionally equivalent circuits.
Inspired by this, researchers~\cite{deepmind2024ai4sys, wang2024tnet} have begun exploring the application of DAS to synthesize circuits directly from truth tables.
Specifically, \citet{deepmind2024ai4sys} proposed CircuitNN, a framework that learns differentiable connection structures with logic gates, enabling the automatic generation of logic circuits from truth tables.
This approach significantly reduces the complexity of traditional circuit generation. 
Building on this, \citet{wang2024tnet} introduced T-Net, a triangle-shaped variant of CircuitNN, incorporating regularization techniques to enhance the efficiency of DAS.

Despite these advancements, several challenges remain. 
The computational complexity of DAS grows quadratically with the number of gates, posing scalability issues.
Although triangle-shaped architecture~\cite{wang2024tnet} partially mitigates this problem, redundancy persists. 
%Additionally, DAS is susceptible to converging to local optima, limiting the ability to search architectures that satisfy the given truth tables~\cite{liu2018darts}. 
%Furthermore, hyperparameters (network depth and layer width) require extensive searches, introducing complexity and prolonging the synthesis process. 
Additionally, DAS is susceptible to converging to local optima~\cite{liu2018darts} and hyperparameters (network depth and layer width) require extensive searches. 
The challenges arise from the vast search space in DAS. 
% Even with predefined settings for CircuitNN, finding a configuration that meets the truth table requires extensive trial and error during the DAS process. 
Intuitively, limiting the search space through predefined parameters (network depth, gates per layer, and connection probabilities) can significantly reduce the complexity.

Recent advances~\cite{openai2023gpt4, abramson2024alphafold3, esser2024sd3, li2024mar} in conditional generative models have demonstrated remarkable performance across language, vision, and graph generation tasks. 
Motivated by these developments, we propose a novel approach to circuit generation that generates preliminary circuit structures to guide DAS in generating refined circuits matching specified truth tables. 
Firstly, we introduce CircuitVQ, a tokenizer with a discrete codebook for circuit tokenization. 
Built upon our Circuit AutoEncoder framework~\cite{hou2022graphmae,li2023maskgae,wu2025mgvga}, CircuitVQ is trained through a circuit reconstruction task. 
Specifically, the CircuitVQ encoder encodes input circuits into discrete tokens using a learnable codebook, while the decoder reconstructs the circuit adjacency matrix based on these tokens.
Subsequently, the CircuitVQ encoder serves as a circuit tokenizer for CircuitAR pretraining, which employs a masked autoregressive modeling paradigm~\cite{chang2022maskgit, li2023mage}. 
In this process, the discrete codes function as supervision signals. 
After training, CircuitAR can generate discrete tokens progressively, which can be decoded into initial circuit structures by the decoder of the CircuitVQ. 
These prior insights can guide DAS in producing refined circuits that match the target truth tables precisely.

Our key contributions can be summarized as follows:
\begin{itemize}
\item We introduce CircuitVQ, a circuit tokenizer that facilitates graph autoregressive modeling for circuit generation, based on our Circuit AutoEncoder framework;
\item Develop CircuitAR, a model trained using masked autoregressive modeling, which generates initial circuit structures conditioned on given truth tables;
\item Propose a refinement framework that integrates differentiable architecture search to produce functionally equivalent circuits guided by target truth tables;
\item Comprehensive experiments demonstrating the scalability and capability emergence of our CircuitAR and the superior performance of the proposed circuit generation approach.
\end{itemize}

% Motivation
% (a) Diffusion (Vision, Graph), Autoregressive (Language, Vision)
% (b) Circuit Generation for Predefined Setting
% (c) Neural Architecture Search for Strict Logic Equivalence

% Contribution
% (a) Circuit Tokenizer (new transformer arch, training strategy)
% (b) CircuitAR (train and gen strategies, post-ar strategy)
% (c) Extensive Evaluation including BitD (Bit Distance) for Scalability

\newcommand{\ones}{\mathbf 1}
\newcommand{\reals}{{\mbox{\bf R}}}
\newcommand{\integers}{{\mbox{\bf Z}}}
\newcommand{\symm}{{\mbox{\bf S}}}  % symmetric matrices

\newcommand{\nullspace}{{\mathcal N}}
\newcommand{\range}{{\mathcal R}}
\newcommand{\Rank}{\mathop{\bf Rank}}
%\newcommand{\Tr}{\mathop{\bf Tr}}
\newcommand{\diag}{\mathop{\bf diag}}
\newcommand{\card}{\mathop{\bf card}}
\newcommand{\rank}{\mathop{\bf rank}}
\newcommand{\conv}{\mathop{\bf conv}}
\newcommand{\prox}{\mathbf{prox}}

\newcommand{\Expect}{\mathop{\bf E{}}}
\newcommand{\var}{\mathop{\bf var{}}}
\newcommand{\Prob}{\mathop{\bf Prob}}
\newcommand{\Co}{{\mathop {\bf Co}}} % convex hull
\newcommand{\dist}{\mathop{\bf dist{}}}
%\newcommand{\argmin}{\mathop{\rm argmin}}
%\newcommand{\argmax}{\mathop{\rm argmax}}
\newcommand{\epi}{\mathop{\bf epi}} % epigraph
\newcommand{\Vol}{\mathop{\bf vol}}
\newcommand{\dom}{\mathop{\bf dom}} % domain
\newcommand{\intr}{\mathop{\bf int}}
%\newcommand{\sign}{\mathop{\bf sign}}

\newcommand{\cf}{{\it cf.}}
\newcommand{\eg}{{\it e.g.}}
\newcommand{\ie}{{\it i.e.}}
\newcommand{\etc}{{\it etc.}}

\newcommand{\todo}{{\bf TODO}}

\newcommand{\bone}{\boldsymbol{1}}
\newcommand{\balpha}{\boldsymbol{\alpha}}
\newcommand{\bbeta}{\boldsymbol{\beta}}
\newcommand{\bdelta}{\boldsymbol{\delta}}
\newcommand{\bepsilon}{\boldsymbol{\epsilon}}
\newcommand{\blambda}{\boldsymbol{\lambda}}
\newcommand{\bomega}{\boldsymbol{\omega}}
\newcommand{\bpi}{\boldsymbol{\pi}}
\newcommand{\bnu}{\boldsymbol{\nu}}
\newcommand{\bphi}{\boldsymbol{\phi}}
\newcommand{\bvphi}{\boldsymbol{\varphi}}
\newcommand{\bpsi}{\boldsymbol{\psi}}
\newcommand{\bsigma}{\boldsymbol{\sigma}}
\newcommand{\btheta}{\boldsymbol{\theta}}
\newcommand{\bzeta}{\boldsymbol{\zeta}}
\newcommand{\bxi}{\boldsymbol{\xi}}
\newcommand{\ba}{\boldsymbol{a}}
\newcommand{\bb}{\boldsymbol{b}}
\newcommand{\bc}{\boldsymbol{c}}
\newcommand{\bd}{\boldsymbol{d}}
\newcommand{\be}{\boldsymbol{e}}
\newcommand{\boldf}{\boldsymbol{f}}
\newcommand{\bg}{\boldsymbol{g}}
\newcommand{\bh}{\boldsymbol{h}}
\newcommand{\bi}{\boldsymbol{i}}
\newcommand{\bj}{\boldsymbol{j}}
\newcommand{\bk}{\boldsymbol{k}}
\newcommand{\bell}{\boldsymbol{\ell}}
\newcommand{\bp}{\boldsymbol{p}}
\newcommand{\br}{\boldsymbol{r}}
\newcommand{\bs}{\boldsymbol{s}}
\newcommand{\bt}{\boldsymbol{t}}
\newcommand{\bu}{\boldsymbol{u}}
\newcommand{\bv}{\boldsymbol{v}}
\newcommand{\bw}{\boldsymbol{w}}
\newcommand{\bx}{{\boldsymbol{x}}}
\newcommand{\by}{\boldsymbol{y}}
\newcommand{\bz}{\boldsymbol{z}}
\newcommand{\bA}{\boldsymbol{A}}
\newcommand{\bB}{\boldsymbol{B}}
\newcommand{\bC}{\boldsymbol{C}}
\newcommand{\bD}{\boldsymbol{D}}
\newcommand{\bE}{\boldsymbol{E}}
\newcommand{\bF}{\boldsymbol{F}}
\newcommand{\bG}{\boldsymbol{G}}
\newcommand{\bH}{\boldsymbol{H}}
\newcommand{\bI}{\boldsymbol{I}}
\newcommand{\bJ}{\boldsymbol{J}}
\newcommand{\bL}{\boldsymbol{L}}
\newcommand{\bM}{\boldsymbol{M}}
\newcommand{\bP}{\boldsymbol{P}}
\newcommand{\bQ}{\boldsymbol{Q}}
\newcommand{\bR}{\boldsymbol{R}}
\newcommand{\bS}{\boldsymbol{S}}
\newcommand{\bT}{\boldsymbol{T}}
\newcommand{\bU}{\boldsymbol{U}}
\newcommand{\bV}{\boldsymbol{V}}
\newcommand{\bW}{\boldsymbol{W}}
\newcommand{\bX}{\boldsymbol{X}}
\newcommand{\bY}{\boldsymbol{Y}}
\newcommand{\bZ}{\boldsymbol{Z}}

% new theorems
% \newtheorem{theorem}{Theorem}
%\newtheorem*{proof}{Proof}

% usepackages
\usepackage{amsmath}
\usepackage{amsfonts}
\usepackage{textcomp} % for \textlangle and \textrangle macros
\newcommand{\qdist}[1]{\ifmmode\langle#1\rangle\else\textlangle#1\textrangle\fi}
\usepackage{xcolor}
\usepackage{algorithm} % for algorithms
\usepackage{algpseudocode} % for pseudocode
\usepackage{comment} % for large comments
\usepackage{bbm}
\usepackage{dsfont}
\usepackage{subfigure}
\usepackage{bm}
\usepackage{booktabs} % For better table lines
\usepackage{array} % For better column formatting
%\usepackage{appendix}
%\usepackage[english]{babel}
%\usepackage{amsthm}
\usepackage{graphicx} % for graphs




\begin{figure*}[t]
\begin{center}
\includegraphics[width=.85\linewidth]{fig_overview_v3.pdf}
\end{center}
\caption{
FastAtlas Overview: In each frame, we compute charts spanning fully or partially visible triangles (a), determine texture space bounding boxes for the visible portions of the view-space projections of each chart, and tightly pack these boxes into atlases (b, here $2K \times 2K$). We simultaneously bijectively parameterize and shade the charts into their atlas boxes, obtaining high quality texture space shading (c), and use this shading to render the shaded frames (d).}
\label{fig:overview}
\label{fig:alg_overview}
\end{figure*}

\section{Overview}
\label{sec:overview}
Our work has two core contributions: a real-time, GPU-based algorithm for tight packing of general parameterized charts into compact atlases; and a real-time TSS method that
utilizes this packing.  

\paragraph*{FastAtlas Packing.}
FastAtlas runs entirely on the GPU as a series of compute shaders. It takes the bounding boxes of parameterized charts as input, and packs them into an atlas (Fig~\ref{fig:overview}b, Sec.~\ref{sec:pack}). As such, the only input it requires are the dimensions of the bounding boxes.
Its outputs are deterministic; identical input charts are packed into identical atlases. This is critical for TSS and similar applications, as it ensures that consecutive frames taken from the same camera view have the same shading. Even minute shading differences across such frames can cause sampling jitter, leading to undesirable flicker \cite{baker2012rock}. 
While prior methods such as \cite{mueller2018shading,hladky2019tessellated,hladky2021snakebinning,Neff2022MSA} cap the dimensions of the charts that can be packed as-is for a given atlas size, and scale down all charts that exceed these dimensions, we scale all charts by the same factor, and do so only when strictly necessary to achieve packing success (Figs~\ref{fig:atlas},~\ref{fig:sas_issues}). 

\paragraph*{TSS using FastAtlas.}
Our end-to-end TSS atlas generation method combines the packing method above with a novel approach for computing seamless per-frame charts. 
We define our charts as the connected components of the visible surfaces in each frame (Fig.~\ref{fig:overview}a), and efficiently compute them using a parallel union-find algorithm (Sec.~\ref{sec:visible}). Since the boundaries of these charts coincide with the contours of the rendered surface, they are {\em invisible} to the viewer. This approach 
eliminates the artifacts caused by shading discontinuities along visible seams (Fig.~\ref{fig:seams}). 

\begin{parWithWrapFigure}
\begin{wrapfigure}{l}{.27\columnwidth}%
\includegraphics[width=\linewidth]{fig_inset_view_plane.pdf}%
\end{wrapfigure}
We bijectively parametrize the {\em visible portions} of our charts by projecting them to view space (inset). This maps a constant number of texels to each pixel in the final rendered output, evenly distributing residual undersampling error across all image pixels. While conceptually straightforward, efficiently parameterizing charts containing partially visible triangles using viewspace projection is non-trivial, as the visible portions may no longer be triangular (e.g. green triangle in the inset); applying naive projection to triangles with vertices behind the camera may produce ill-posed results. Clipping triangles before projection is both computationally expensive and significantly complicates downstream operations. We avoid explicit clipping by observing that all that is required for atlas packing is the dimensions of, potentially conservative, bounding boxes of these projected visible portions. We compute such bounding boxes without explicit chart clipping by adapting a conservative screen coverage estimator \shortcite{Blinn:CalculatingScreenCoverage} (Sec.~\ref{sec:box}). We then pack the computed boxes using FastAtlas. 
\end{parWithWrapFigure}

Finally, we shade the visible portion of each chart into its corresponding atlas bounding box (Fig~\ref{fig:overview}c). 
The resulting texture is then used during rasterization as a standard texture map (Fig. ~\ref{fig:overview}d). 
Our framework is compatible with all existing approaches for texture space shading, including forward shading, raytraced illumination, or deferred shading in texture space \cite{baker:2016}. In the examples shown, we use the standard forward shading based rendering pipeline included in the G3D Innovation Engine \cite{G3D17}, a commercial grade renderer.

% !TEX root = main.tex

\section{1-component \tvass are \plb} \label{sec:1comp}

This section is devoted to the proof of the 
induction base for the proof of Lemma \ref{lem:main}:

\begin{lemma} \label{lem:1comp}
1-component \tvass are \plb.
\end{lemma}

\noindent
We also develop a framework to be exploited in the induction step in 
Section \ref{sec:mainproof}.
We may safely restrict to \tvass of geometric dimension 3,
as otherwise Lemma~\ref{lem:geom-plb} immediately implies Lemma~\ref{lem:1comp}.

\para{Case distinction}
%We distinguish between \emph{easy} and \emph{non-easy} \tvass, as defined below.
%\para{Diagonal \vass}
A \tvass $(V, s, t)$, where $s=p(w)$ and $t=p'(w')$,
is \emph{forward-diagonal} if
$p(w) \trans{*} p(w+\Delta)$ in $V$ for some $\Delta \in (\Npos)^3$.
Symmetrically, $(V, s, t)$ is \emph{backward-diagonal} if $\rev{(V, t, s)}$ is diagonal, i.e., if 
$p'(w'+\Delta') \trans{*} p'(w')$ in $V$ for some $\Delta' \in (\Npos)^3$.
Finally, $V$ is \emph{diagonal} if it is both forward- and backward-diagonal.
Obviously, the vectors $\Delta$ and $\Delta'$ need not be equal in general.

%\para{Cones generated by effects of cycles}

Let  $E=\{e_1, \ldots, e_n\} \subseteq \Z^3$ be the effects of simple cycles of $V$.
%\wojtek{Notation $e_i$ suggests elementary vectors, maybe change to $v_i$.}
We define the (rational) open cone generated by this set
to contain all positive rational combinations of vectors from $E$:
\[
\cone {V} \ = \ \setof{ r_1 \cdot e_1 + \ldots + r_n \cdot e_n }{r_1, \ldots, r_n \in \Qpos} \ \subseteq \ \Lin V.
%\ \subseteq \ (\Qpos)^3.
\]
$\cone V$ is thus an open cone inside $\Lin V$.
A 1-component \tvass $V$ is called \emph{wide} if $(\Qpos)^3 \subseteq \cone V$, i.e.,
if $\cone V$ includes the whole positive orphant.

%We say that a \vass $(V,s,t)$ is \emph{length-equivalent} to a finite set $\{(V_1,s_1,t_1), \ldots, (V_n, s_n, t_n)\}$ 
%of \vass if
%the set of lengths of paths $s\tran t$ in $V$ is the same as the set of lengths
%of all paths $s_i \tran t_i$ in $V_i$ for all $i\in\setfromto 1 n$. 
%
%\begin{proof}


Let $\Len V s t$ denote the set of lengths of paths $s\tran t$ in $V$.
%For proving Lemma \ref{lem:1comp}, 
We need to
argue that there is a nondecreasing polynomial $Q$ 
such that every 1-component \tvass $(V, s, t)$
with a path $s\tran t$, has such path of length at most $Q(M)$, where 
$M=\size(V, s,t)$.
We split the proof into three cases:
\smallskip
\begin{enumerate}
\item
If $(V,s,t)$ is diagonal and wide, we exploit the fact that \tzvass are \plb, and use diagonality and wideness to
lift a short $\Z$-path into a path.
\item
If $(V,s,t)$ is diagonal but non-wide, we show that $(V,s,t)$ is \emph{length-equivalent} to a 
\geomvass $(\essdvass V, \essdvass s, \essdvass t)$ of polynomially larger size, namely 
$\Len V s t = \Len {\essdvass V} {\essdvass s} {\essdvass t}$.
\item
Finally, if $(V, s, t)$ is non-diagonal, we show that $(V,s,t)$ is \emph{length-equivalent} to a set
of three
\geomvass $\{(V_1, s_1, t_1), (V_2, s_2, t_2), (V_3, s_3, t_3)\}$, namely 
$\Len V s t = \Len {V_1} {s_1} {t_1} \cup \Len {V_2} {s_2} {t_2} \cup \Len {V_3} {s_3} {t_3}$
of polynomially larger size.
\end{enumerate}
\smallskip
\noindent
In the two latter cases we rely on the fact that \geomvass are \plb (Lemma \ref{lem:geom-plb}).
%
%We distinguish the following cases, depending on whether $(V, s, t)$ is: 
%
%\begin{enumerate}
%\item diagonal and wide;
%\item diagonal and non-wide;
%\item non-diagonal.
%\end{enumerate}
%
%\begin{lemma} \label{lem:easy-1comp-plb}
%Diagonal and wide 1-component \tvass are \plb.
%\end{lemma}
%
%\begin{lemma} \label{lem:-1comp-plb}
%There is a nondecreasing polynomial $P$ such that every diagonal non-wide 1-component \tvass $(V,s,t)$ is equivalent to a \geomvass of size at most $P(\size(V,s,t))$.
%\end{lemma}
%
In consequence, $Q$ is to be the sum of polynomials claimed in the respective cases.
%In the first case we exploit the fact that \tzvass are \plb
%(Lemma \ref{lem:zvass-plb}), while
%in the two latter cases we rely on the fact that \geomvass are so (Lemma \ref{lem:geom-plb}).
%To this aim, in Cases 2 and 3 we transform a \tvass into a form which is  \emph{essentially} a \dvass,
%namely in every configuration $q(v_1, v_2, v_3)$,
%the value of  one of coordinates, say $v_3$, is uniquely determined
%by state $q$ and values of the other two coordinates through an affine mapping,
%$v_3 = \ell_q(v_1, v_2)$.
%
In the sequel let $(V, s, t)$ be a fixed 1-component \tvass  with $s\tran t$, where
$s=p(w)$ and $t=p'(w')$.

\para{Case 1. $(V, s, t)$ is diagonal and wide}
%
By diagonality, $p(w) \trans{\pi} p(w+\Delta)$ and
$p'(w'+\Delta')\trans{\pi'} p'(w')$  for some $\Delta,\Delta'\in(\Npos)^3$.

Let $P$ be a nondecreasing polynomial witnessing Lemma \ref{lem:zvass-plb}, i.e.,
\tzvass are \lb by $P$.
As there is a path $s \tran t$ in $V$, there is also a $\Z$-path $s\tran t$, and
by Lemma \ref{lem:zvass-plb} there is a $\Z$-path $s \trans{\sigma} t$ of length at most $P(M)$.
The maximal norm $N$ of $\Z$-configurations along $\sigma$ is thus bounded by $M \cdot P(M)$,
as every step may update counters by at most $M$.

By diagonality, the configuration $p(w+\vec 1)$ is coverable in $V$ from $s$,
and symmetrically the configuration $p'(w' + \vec 1)$ is coverable in $\rev V$ from $t$.
Due to the upper bound of 
Rackoff~\cite[Lemma~3.4]{DBLP:journals/tcs/Rackoff78},
there is a nondecreasing polynomial $R$ such that in every \tvass of size $m$, the length of a covering
path is at most $R(m)$.
Therefore the lengths of 
both paths $p(w) \trans {\pi} p(w+\Delta)$  and $p'(w'+\Delta')\trans {\pi'} p'(w')$ in $V$, 
where $\Delta, \Delta' \in (\Npos)^3$,
may be assumed to be at most $R(M)$.
%
We argue that there is a cycle 
%$p(w) \tran p(w+ \ell\cdot \Delta')$
from the source configuration $p(w)$ 
that increases $w$ by some multiplicity of $\Delta'$:
%
\begin{lemma} \label{lem:FP}
There is a path $p(w) \trans \delta p(w+ \ell \cdot\Delta')$ of length $R(M)^{\OO(1)}$,
for some $\ell\in\Npos$.
% $\ell\leq R(M)^{\OO(1)}$.
\end{lemma}
%
Before proving the lemma we use it to complete Case 1.
We build a path $p(w) \tran p'(w')$ by concatenating $3$ paths given below.
The first one is  $\delta$
%\[
%p(w) \trans{\delta} p(w+\ell\cdot\Delta'),
%\]
given by Lemma \ref{lem:FP}.
Note that $\ell$ is necessarily also bounded by $R(M)^{\OO(1)}$.
We replace $\ell$ by its sufficiently large multiplicity to enforce $\ell \geq M\cdot P(M)$, which
makes the length of $\delta$ and $\ell$ only bounded by $P(M) \cdot R(M)^{\OO(1)}$.
The multiplicity guarantees that the $\Z$-path 
$
p(w) \trans{\sigma} p'(w'), 
$
lifted by $\ell \cdot \Delta'$, becomes a path:
\[
p(w+\ell \cdot \Delta') \trans{\sigma} p'(w' + \ell\cdot\Delta'), 
\]
The length of $\sigma$ is bounded by $P(M)$.
Finally, let $\delta' = (\pi')^\ell$ be the $\ell$-fold concatenation of
the cycle $\pi'$:
\[
p'(w'+\ell\cdot\Delta') \trans{\delta'} p'(w').
\]
The length of this path is bounded by $\ell\cdot R(M) \leq P(M)\cdot R(M)^{\OO(1)}$.
We concatenate the three paths, $\tau := \delta; \, \sigma;\, \delta'$, to get a required path
\[
p(w) \trans{\tau} p'(w')
\]
of length bounded by $P(M) \cdot R(M)^{\OO(1)}$.
It thus remains to prove Lemma \ref{lem:FP} in order to complete Case 1.


\begin{proof}[Proof of Lemma \ref{lem:FP}]
%By forward diagonality, $p(w) \trans{\pi} p(w+\Delta)$ for some $\Delta\in(\Npos)^3$.
Let $\pi_1$ be a cycle that visits all states, and
let $\Delta_1\in\Z^3$ be its effect.
%
%There is such a cycle of length at most $\OO(M^2)$.
%
%For $m\in \Npos$, let $\pi^m$ denote $m$-fold concatenation of $\pi$:
%$p(w) \trans{\pi^m} p(w+m\Delta)$.
%
Relying on $\Delta\in(\Npos)^3$,
take a sufficiently large multiplicity $m\in\Npos$ so that 
$\widetilde \pi = \pi^m;\, \pi_1$ is a path with nonnegative effect.
%It is enough to take $m\leq \OO(M^3)$.
The path $\widetilde \pi$ is a cycle and  its effect is
$\widetilde \Delta = m\cdot\Delta + \Delta_1 \in \N^3$.
%is a nonnegative integer combination
%of effects of simple cycles:
%\[
%\Delta_2 = s_1 \cdot e_1 + \ldots + s_n \cdot e_n \in \cone V \cap \N^3,
%\] 
%where $s_1, \ldots, s_n \in \N$.

As $\Delta'\in(\Npos)^3$, there is $\ell'\in\Npos$ such that 
$\ell'\cdot\Delta'-\widetilde \Delta \in (\Qpos)^3$,
and hence, by wideness of $(V, s)$, we have
$\ell'\cdot\Delta'-\widetilde \Delta \in \cone V$, namely
\[
\ell'\cdot\Delta' - \widetilde \Delta = r_1 \cdot e_1 + \ldots + r_n \cdot e_n
\] 
for some positive rationals $r_1, \ldots, r_n\in \Qpos$.
%
By Carathéodory's Theorem \cite[p.94]{cara}, $\ell'\cdot\Delta'-\widetilde \Delta$ is a combination of some $3$ vectors
among $e_1, \ldots, e_n$, say $e_1, e_2, e_3$:
\[
\ell'\cdot\Delta' - \widetilde \Delta \ = \ r_1 \cdot e_1 + r_2 \cdot e_2 + r_3 \cdot e_3,
\] 
for some positive rationals $r_1, r_2, r_3\in\Qpos$.
Therefore, the system of $3$ equations
\[
\ell \cdot (\ell'\cdot\Delta'-\widetilde\Delta)  \ = \ r_1 \cdot e_1 + r_2 \cdot e_2 + r_3 \cdot e_3,
\]
with unknowns $\ell, r_1, r_2, r_3$, has a positive integer solution.
We rewrite the system to:
\begin{align} \label{eq:syst_1comp}
\ell \ell'\cdot \Delta' - \ell m \cdot \Delta \ = \  \ell\cdot\Delta_1 + r_1 \cdot e_1 + r_2 \cdot e_2 + r_3 \cdot e_3.
\end{align}
%
%For sufficiently large $\ell\in\Npos$,
% enough so that the multiplicites of coefficients $\ell_1\cdot r_i$ are integers,
%for $i\in\setto n$.
%Furthermore, picking up a sufficiently large multiplicity $\ell_1$ yields
%$\ell_1 \cdot r_i \geq s_i$ for every $i\in\setto n$.
%Summing up, for some $r_1, \ldots, r_n \in \N$ we have
%\[
%\ell_1\cdot v = \Delta_2 + r_1 \cdot e_1 + \ldots + r_n \cdot e_n \in \cone{V} \cap (\Npos)^3.
%\]
Let $\sigma_i$ be simple cycle of effect $e_i$, for $i\in\setto 3$.
Let $
\sigma
$
be a $\Z$-path that starts (and ends) in state $p$ and consists of 
$\ell$-fold concatenation of the cycle $\pi_1$, with attached 
$(r_1)$-fold concatenation of $\sigma_1$,
$(r_2)$-fold concatenation of $\sigma_2$,
and
$(r_3)$-fold concatenation of $\sigma_3$
(since $\pi_1$ visits all states, this is possible).
The effect of $\sigma$ is the right-hand side of \eqref{eq:syst_1comp}, and therefore
$\sigma$ is a $\Z$-path from $p(w+\ell m \cdot \Delta)$ to
$p(w + \ell \ell'\cdot \Delta')$:
\[
p(w + \ell m \cdot \Delta) \trans{\sigma} p(w+\ell \ell'\cdot \Delta').
\]
It needs not be a path in general, and therefore we are going to lift it.
Let $k\in\Npos$ be a multiplicity large enough so that $\sigma$ becomes a path
when lifted by $(k-1)\ell m \cdot \Delta$, i.e., when starting in $p(w+k \ell m \cdot \Delta)$, 
and also becomes a path when lifted by $(k-1)\ell\ell'\cdot \Delta'$, i.e., when
ending in 
$p(w+k \ell\ell' \cdot \Delta')$.
In this case, the $k$-fold concatenation of $\sigma$ is also a path:
%
\begin{align} \label{eq:thepath}
p(w+k\ell m\cdot \Delta) \trans{\sigma^{k}} 
p(w+k \ell\ell'\cdot \Delta'),
\end{align}
%
since all points visited in the inner iterations of $\sigma$ are bounded 
from both sides by corresponding points visited in the first and the last iteration of $\sigma$.
Precomposing this path with $p(w)\trans{\pi^{k\ell m}} p(w + k\ell m\cdot \Delta)$ yields a path
$p(w) \tran p(w+ k \ell \ell'\cdot \Delta')$, as required.

\smallskip

We now (roughly) bound the magnitudes of all items involved in the above reasoning
by a constant power of $R(M)$.
\Wlog we may assume that the cycle $\pi_1$ uses every transition at most $\card Q\leq M$ times,
and thus both the length and norm of the effect of $\pi_1$ are bounded by $M^2$.
In consequence, $\pi_1$ may decrease counters by at most $M^2$.
Therefore $m\leq M^2$, and
norms of vectors $\Delta, \Delta', \widetilde\Delta$ are all bounded by 
$\OO(M^3\cdot R(M))$. % \leq R(M)^{\OO(1)}$.
The effects $e_1, e_2, e_3$ of simple cycles $\sigma_1, \sigma_2, \sigma_3$ are at most $M$, 
as no transition repeats along a simple cycle.
Therefore by Lemma \ref{lem:taming}, the system \eqref{eq:syst_1comp} has a solution $(\ell, r_1, r_2, r_3)$ 
of norm at most $D=\OO(M^3\cdot R(M))^3 = \OO(M^9 \cdot R(M)^{3})$, 
and $\sigma$ has length at most $M^2 \cdot D$
(since $\pi_1$ has length at most $M^2$). %\OO(M^{11}\cdot R(M)^{3})$.
Therefore we deduce $k\leq M^3 \cdot D$, %\OO(M^{12} \cdot R(M)^{3})$, 
and hence
the path \eqref{eq:thepath} has length at most $M^5 \cdot D^2$. % $\OO(M^{23} \cdot R(M)^{6})$.
In consequence of the above bounds, $k \ell m \leq M^5 \cdot D^2$, %\OO(M^{23} \cdot R(M)^{6})$,
and the final path $p(w) \tran p(w+ k \ell v)$ has length at most 
$R(M) \cdot M^5 \cdot D^2
%$\OO(M^{23} \cdot R(M)^{7}) 
\leq \OO(R(M)^{30})\leq R(M)^{\OO(1)}$.
\end{proof}



\para{Case 2. $(V, s, t)$ is non-wide}

%(We do not use diagonality of $(V, s, t)$ in the sequel.)
%
Every non-zero vector $a=(a_1, a_2, a_3)\in\Q^3$ defines an open half-space
\[
%H \ = \ \setof{(x_1, x_2, x_3)\in\Q^3}{a_1 \cdot x_1 + a_2 \cdot x_2 + a_3 \cdot x_3 = 0}.
H_a \ = \ \setof{x\in\Q^3}{\innprod a x > 0},
\]
where $\innprod a x = a_1 x_1 + a_2 x_2 + a_3 x_3$ stands for the inner product of 
$x =(x_1, x_2, x_3)$ and $a$.
As $V$ is assumed to be of geometric dimension 3,
$\cone V$  is an intersection of open half-spaces:
%
\begin{claim} \label{claim:a}
%If vectors generating $\cone V$ spans three dimensional vector space (possibly by using negative coefficients) then 
$\cone V$ is an intersection of %$\Lin V$ with 
finitely many open half-spaces $H_a$, with $\norm(a) \leq D := \OO(M^2)$.
\end{claim}
%
%\wojtek{Maybe introduce some notion of polynomially-bounded (wrt. to $M$)}
%As $V$ is assumed to be of geometric dimension 3, $\Lin V = \Q^3$ may be omitted.
%\slawek{remove $\Lin V$ from the claim?}

\begin{proof}
Norms of vectors generating $\cone V$ --- i.e., effects of simple cycles --- are at most $M$, 
as no transition repeats along a simple cycle.
Consider vectors $a$ orthogonal to some of the facets of $\cone V$, i.e.,
orthogonal to two of the vectors generating $\cone V$. 
The vector $a$ is thus an integer solution of a system of 2 linear equations with 3 unknowns,
where absolute values of coefficients are bounded by $M$.
By Lemma \ref{lem:taming}, there is such an integer solution with $\norm(a)\leq \OO(M^2)$.
This completes the proof.
\end{proof}

%As $(V, s, t)$ is forward-diagonal, we know that there is a positive vector $\Delta \geq \vec 1$
%such that $\Delta \in \cone V$.
As $V$ is non-wide, 
%there is a positive vector $\Delta'\geq \vec 1$ such that
%$\Delta'\notin \cone V$, and,
due to Claim~\ref{claim:a} we know that $\cone V$ is a \emph{non-empty} intersection of half-spaces $H_a$.
Therefore for some of these $H_a$ we have
% $\Delta' \not\in H_a$. On the other hand, 
$\cone V \subseteq H_a$, i.e.,
%Therefore $\innprod \Delta a \geq 0$ and $\innprod {(\Delta')} a \leq 0$.
%As $a\neq(0,0,0)$, 
%this implies that among $a_1, a_2, a_3$ there is both a non-negative and a negative number.
%\Wlog we assume $a_1, a_2\geq 0$, $(a_1,a_2)\neq (0,0)$ and $a_3 < 0$
%(the symmetric case is dealt with by changing inequality sign in Claim~\ref{clm:inner_with_effect} and 
%adapting accordingly the proof of Claim~\ref{claim:ax}).
%
%As $\cone V \subseteq H_a$, 
%
all points $x\in\cone V$ have positive inner product $\innprod a x > 0$.
% (otherwise we replace $a$ by $-a$).
%(while all points in the other cone $-C$ have nonpositive inner product $\innprod a x\leq 0$).
%
This implies that
the value of  inner product with $a$ may not decrease along any cycle in $V$:
%
\begin{claim}\label{clm:inner_with_effect}
The effect $\delta\in\Z^3$ of every simple cycle has nonnegative inner product $\innprod a \delta \geq 0$.
\end{claim}
%
In consequence, on every path $s \tran t$ the value of inner product with $a$ is polynomially
bounded:
%
\begin{claim} \label{claim:ax}
Every configuration $q(x)$ on a path from $s$ to $t$ satisfies
$-B \leq \innprod a x \leq B$, where $B := \OO(M\cdot D)$. 
\end{claim}
%
\begin{proof}
Let $s=p(w)$ and $t=p'(w')$, and let $b = \innprod a w$ and $b' = \innprod a {w'}$.
Every path $s \tran t$ may be decomposed into simple cycles, whose effect may only preserve or increase
the inner product with $a$, plus a short path without cycles, and hence without repetitions of a transitions.
The effect of the latter path is thus in $[-M,M]$.
Therefore, as inner product may at most multiply norms, 
every configuration $q(x)$ on a path from $s$ to $t$ satisfies
$b-M \cdot D \leq \innprod a x \leq b'+M \cdot D$.
Knowing that $\norm(a)\leq D$, $\norm(w), \norm(w')\leq M$, 
and that inner product may at most multiply norms, 
by Claim \ref{claim:a} we deduce 
\[
-M\cdot D \leq b, b' \leq M\cdot D,
\]
and therefore
$-2\cdot M\cdot D \leq \innprod a x \leq 2\cdot M \cdot D$,
which implies the claim.
\end{proof}

%Let $N_b$ be the set of point-wise minimal pairs $(x_1, x_2)$ satisfying 
%the inequality $a_1 x_1 + a_2 x_2 \geq b$:
%\[
%N_b : = \min \setof{(x_1, x_2) \in \N^2}{a_1 x_1 + a_2 x_2 \geq b}.
%\]
%\lukasz{Explain that for checking whether we are in $N_b$ it is enough to subtract and add}
We define a \geomvass $\essdvass V = (\essdvass Q, \essdvass T)$ 
by extending states with the possible values of inner product with $a$
(bounded polynomially by Claim \ref{claim:ax}).
We call $\essdvass V$ the \emph{\mytrim {$a$} {$B$}} of $V$. 
%\footnote{This construction is inspired by \cite[Section 5]{Zhang-geom}.}
The set of states $\essdvass Q$ contains states of the form $\pair q b$, where $q\in Q$ and $-B \leq b \leq B$,
%that correspond to states of $V$, plus a number
%of auxiliary states:
%\[
%\essdvass Q = \setof{\pair q b, \ \pair{\essdvass q} b, \ \triple {\essdvass q} b n}{q\in Q, \ b\in [-B,B], 
%\ n\in N_b}.
%\]
with the intention that every configuration
$c = q(x)$ of $V$ has a corresponding configuration 
$\essdvass c = \pair q b (x)$ in $\essdvass V$, 
where $\innprod a x = b$.
%
%The corresponding configuration of $V$ is unique, as $\innprod a x = b$ implies
%\begin{align} \label{eq:x3}
%x_3 = (b - a_1 x_1 - a_2 x_2)/{a_3}.
%\end{align}
Therefore, for each transition $(q, v, q') \in T$
% where $v=(v_1, v_2, v_3)$,
and for all $b, b'\in \setfromto{-B} B$
such that $b + \innprod a v = b'$,
we add to $\essdvass T$ the transition
\begin{align} \label{eq:tranV}
\big(\pair q b, v, \pair {q'} {b'}\big).
\end{align}
%Moreover, we add to $\essdvass V$ transitions that check whether a configuration 
%$\pair {\essdvass{q}} {b}(x_1, x_2)$ corresponds to a configuration of $V$, i.e.,
%whether $x_3\geq 0$ in \eqref{eq:x3}, which is equivalent to
%$a_1 x_1 + a_2 x_2 \geq b$ (as $a_3<0$).
%To this aim for each $b\in \setfromto{-B}B$ and each $n\in N_b$ we add to $\essdvass T$ two
%auxiliary transitions:
%%
%\begin{align} \label{eq:checkV}
%(\pair {\essdvass{q}} {b}, -n, \triple {\essdvass q} b n)
%\qquad
%(\triple {\essdvass q} b n, n, \pair {q} {b}).
%\end{align}
%%
%%Consider any configurations 
%%$p(w)$, $p'(w')$ of $V$, where $w = (w_1, w_2, w_3)$ and $w' = (w'_1, w'_2, w'_3)$,
%%and let $b=\innprod a w$, $b' = \innprod a {w'}$. 
%%

\begin{claim}\label{clm:trim_makes_geom}
$\essdvass V$ is a \geomvass.
\end{claim}

\begin{proof}
By construction, the effect of each cycle in $\essdvass V$ is orthogonal to $a$, and therefore 
$\Lin {\essdvass V}$ is included in a $2$-dimensional vector space.
\end{proof}
%
Relying on Claim \ref{claim:ax}, paths $s\tran t$ in $V$ have corresponding paths in $\essdvass V$,
and hence we get:
\begin{claim} \label{claim:lenlen}
$\Len V s t = \Len {\essdvass V} {\essdvass s} {\essdvass t}$.
\end{claim}
%The tight correspondence between paths $V$ and $\essdvass V$,
% given in Claims \ref{claim:V31} and
%\ref{claim:V32} below, is essentially 
%Lemma 5.1 of \cite{Zhang-geom}.
%%
%The first one follows directly by our construction:
%%
%\begin{claim} \label{claim:V31}
%For every path $s \tran u$ in $V$ of length $\ell$,
%there is a path $\essdvass s \tran \essdvass u$ in $\essdvass V$ of length $3\ell$.
%%Up to multiplicative constant 3, the lengths of paths 
%%$s \tran t$ in $V$ are the same as
%%the lenghts of paths
%%$\essdvass s \tran \essdvass t$ in $\essdvass V$.
%\end{claim}
%%
%Also by the construction, every path $\essdvass s \tran u'$ in $\essdvass V$
%ending in a non-auxiliary state, i.e.~$u' = \pair q b(v')$ for some $v'\in\N^2$, 
%has length divisible by 3.
%Furthermore, the transitions \eqref{eq:checkV} ensure that the target
%configuration $u'$ is indeed corresponding to a configuration of $V$.
%Therefore, one can reconstructs a corresponding path in $V$:
%
%%
%\begin{claim} \label{claim:V32}
%For every path $\essdvass s \tran u'$ in $\essdvass V$ of length $3\ell$,
%there is a path $s \tran u$ in $V$ with $\essdvass u = u'$, of length $\ell$.
%\end{claim}
%
Finally, we argue that the size of $\essdvass V$ is bounded polynomially with respect to the size of $V$:
%
\begin{claim} \label{claim:sizeVV}
$\size(\essdvass V) \leq R(M) = \OO(M\cdot B).$
\end{claim}
%
\begin{proof}
Recalling Claims \ref{claim:a} and \ref{claim:ax}, namely
$\norm(a)\leq D$, $B=\OO(M\cdot D)$, we deduce that 
transitions \eqref{eq:tranV} contribute at most $(2B+1)M \leq \OO(M\cdot B)$ to the size of $\essdvass V$.
%Estimating roughly, for every $b\in\setfromto {-B} B$,
%the set $N_b$ contains vectors of norm at most $|b|$ (because $a_1$ and $a_2$ are nonnegative) and therefore there at most $b^2+1$ such vectors. 
%Therefore transitions \eqref{eq:checkV} contribute at most 
%$2(2B+1)b^2(|b|+1) \leq 2(2B+1)(B^3+B) \leq \OO(M^4\cdot D^4)$ to the size of $\essdvass V$.
%In total, $\size(\essdvass V) \leq \OO(M^4\cdot D^4)$.
\end{proof}
We are now prepared to complete Case 2.
Let $P$ be the polynomial witnessing Lemma \ref{lem:geom-plb}, i.e.,
\geomvass are \lb by $P$.
As $V$ has a path $s\tran t$, 
By Claim \ref{claim:lenlen}, $\essdvass V$ has a path $\essdvass s \tran \essdvass t$.
By Lemma \ref{lem:geom-plb}, $\essdvass V$ has thus a path $\essdvass s \tran \essdvass t$
of length at most $P(\size(\essdvass V))$, i.e., relying on Claim \ref{claim:sizeVV},
of length at most $P(\OO(M^2\cdot D)) = \OO(P(M^{4}))$.
By Claim \ref{claim:lenlen} again, we get a path $s\tran t$ in $V$ of length
$\OO(P(M^{4}))$.
This completes Case 2.


\para{Case 3. $(V, s, t)$ is non-diagonal}

\Wlog assume that $(V, s)$ is not forward-diagonal (otherwise replace $V$ by $\rev V$).
Therefore for all states $q$ the configuration $s'=q(w+(\vec{M{+}1}))$ is not
coverable from $s=p(w)$. 
Indeed, using strong-connectedness of $V$, if $s'$ were coverable from $s$ then
the configuration $p(w + \vec 1)$ would be coverable form $p(w)$, by 
extending the covering path of $s'$ with an arbitrary shortest path back to state $p$ 
(that cannot decrease a counter by more than $M$), which would contradict forward non-diagonality.

By Lemma \ref{lem:sim-high-d},
in every path from $s$, some coordinate
$j\in\setto 3$ is bounded by $B:=P_3(M, M, M+1)$ (we take $M$ as an upper bound for $n$ and $N$,
relying on $P_3$ being nondecreasing, and take $U=M+1$).
This property allows us, intuitively speaking, to describe all the paths of $V$ by paths of three
\geomvass $V_j$, for $j\in\setto 3$, where $V_j$ behaves exactly like $V$ except that dimension $j$ is additionally kept in state.
%For $w\in\N^3$ and $j\in\setto 3$, let $\dropped w j\in\N^2$ denote $w$ without $j$th coordinate.
%
Formally, let $V_j := (Q_j, T_j)$, where 
\begin{align*}
Q_j = & \setof{\pair q b}{q\in Q, \ b \in \setfromto 0 B} \\
T_j =  & \setof{(\pair q b, v, \pair {q'}{b'})}{(q, v, q')\in T, \ b' = b + v_j}.
\end{align*}
%
The source and target configurations in $V_j$ are $s_j = \pair p {w_j}(w)$ and $t_j = \pair {p'}{w'_j}(w')$,
and there is a tight correspondence between paths in $V$ and paths in $V_1, V_2, V_3$:
%
\begin{claim} \label{claim:nondiag2vass}
$\Len V s t = \Len {V_1} {s_1} {t_1} \cup \Len {V_2} {s_2} {t_2} \cup \Len {V_3} {s_3} {t_3}$.
%$p(w) \tran p'(w')$ in $V$ if and only if
%$\pair p {w_j}(\dropped w j) \tran \pair{p'}{w'_j}(\dropped {w'} j)$ in $V_j$ for some $j\in\setto 3$, by a path
%of the same length.
\end{claim}
%
The size of each of $V_j$ is bounded polynomially with respect to the size of $V$:
%
\begin{claim} \label{claim:nondiagsize}
The size of each of $V_j$ is at most $R(M)=\OO(M\cdot B)$.
\end{claim}
%
Let $P$ be the polynomial witnessing Lemma \ref{lem:geom-plb}.
As $p(w) \tran p'(w')$ in $V$,
by Claim \ref{claim:nondiag2vass}
there is a path $\pair p {w_j}(w) \tran \pair{p'}{w'_j}({w'})$ in $V_j$
for some $j\in\setto 3$.
Therefore, by Lemma \ref{lem:geom-plb} there is such a path of length 
at most $P(R(M))$ which,
again using Claim \ref{claim:nondiag2vass}, implies a path $p(w) \tran p'(w')$ in $V$ of the same length.
%
This polynomial bound completes Case 3, and hence also the proof of
Lemma \ref{lem:1comp}.
%
%\end{proof}

% !TEX root = main.tex

\section{\Sandwich reachability sets} \label{sec:tools}
In this section we introduce the crucial concept of \emph{\sandwich} sets.
In order to motivate it, we start by sketching the overall idea of the
proof of Lemma \ref{lem:main} (given in Section \ref{sec:mainproof}). 

Given a finite set $P\subseteq \N^d$ and $B\in\N$, we set:
%
\begin{align*}
P^* \ =  \ & \setof{p_1 + \ldots + p_k}{k\geq 0, \ p_1, \ldots, p_k \in P} \\
P^{\leq B} \ =  \ & \setof{p_1 + \ldots + p_k}{B\geq k\geq 0, \ p_1, \ldots, p_k \in P}.
\end{align*}
%
Sets of the form $b + P^* = \setof{b+p}{p\in P^*}$, for $b\in\N^d$ and finite $P\subseteq\N^d$,
are called \emph{linear},
and finite unions of linear sets are called \emph{semi-linear}.


\para{Idea of the proof of Lemma \ref{lem:main}}

Let $V=\ktvass$ be a $k$-component \tvass  that has a path $s=q(w) \tran q'(w')=t$.
%
If $V$ is diagonal and wide,
%, namely $V_1$ is forward-diagonal and $V_k$ is backward-diagonal, and also wide,
we use the pumping cycles $q(w) \tran q(w+\Delta)$ in $V_1$ and
$q'(w'+\Delta') \tran q'(w')$ in $V_k$ to lift a $\Z$-path $s\tran t$, \plb
due to Lemma \ref{lem:zvass-plb}, until it becomes a path
(as in Case 1 in the proof of Lemma \ref{lem:1comp}).
%in Lemma \ref{lem:dw} in Section \ref{sec:mainproof}).

On the other hand, if $V$ is non-diagonal or non-wide, our strategy is to reduce the number of components
by 1, and to rely on the induction assumption for $k-1$,
by replacing the first component $V_1$ 
by one of finitely many \geomvass 
(as in Cases 2 and 3 of the proof of Lemma \ref{lem:1comp}).
Relying on the fact that the reachability sets in a \geomvass are semi-linear \cite{DBLP:conf/icalp/FuYZ24},
the proof could go as follows (yielding however only the
already known \tower upper bound \cite{DBLP:conf/icalp/FuYZ24}).
%
Using any of the linear sets $L = a +P^*$
describing the set $\reach_{q_2}(V,s)$, where $q_2$
is the source state of the second component $V_2$,
transform $V$ into a $(k-1)$-component \tvass $V'$ by dropping the first component $V_1$
and the first bridge $u_1$, and by adding to the remaining $(k-1)$-component
\tvass $(V_2)u_2 \ldots u_{k-1}(V_k)$ the self-looping transitions
$
(q_2, r, q_2),
$
one for every period $r\in P$.
The source configuration of $V'$ is $s' = q_2(a)$, i.e., its vector is the base of $L$.
The transformation preserves behaviour of $V$.
In one direction, a path $s\tran q_2(x) \tran t$ in $V$ crossing through $q_2(x)$ for 
some  $x\in L$ has a corresponding path $p_2(a) \tran p_2(x)\tran t$ in $V'$.
Conversely, each path $q_2(a) \tran q_2(x) \tran t$ in $V'$
gives rise to a path $s\tran t$ in $V$, by replacing executions of the self-looping transitions
$q_2(a) \tran q_2(x)$
(\mywlog executed in the beginning), by a path $s\tran q_2(x)$ in $V_1$,
bounded polynomially due to Lemma \ref{lem:1comp}.
However, $\size(V')$ may blow-up exponentially with respect to $\size(V)$, as bases and periods
of $L$ areonly bounded exponentially, 
and therefore
this approach could only yield a $k$-fold exponential bound on the length of the shortest path
in $k$-component \tvass.

%\smallskip

\Sandwich sets are designed as a remedy against the $k$-fold exponential blowup.
The idea is to measure the norms of base and periods of a semi-linear set $L$ 
parametrically with respect to, intuitively speaking, 
the prospective length $B$ of a path $s'\tran t$ in $V'$.
This allows us to control the blow-up of size of $V'$,
also parametrically with $B$, but requires
going outside of semi-linear sets and considering their \emph{$B$-approximations},
namely sets sandwiched between $a+P^{\leq B}$ and $a+P^*$,
good enough for correctness of the above-described transformation
of $V$ to $V'$.
As the outcome, the exponent of our bound on length of the shortest path in $k$-component \tvass
is, roughly speaking, square of the exponent of the respective bound in $(k-1)$-component \tvass. 
For $k$-component \tvass this yields exponent doubly-exponential in $k$, and hence the bound triply-exponential is $k$.
The rigorous reasoning is given in the proof of Lemma \ref{lem:ne} in Section \ref{sec:mainproof}.


\para{\Sandwich sets}

Let $B\in \N$.
By a $B$-\emph{approximation}
of a linear set $a + P^*\subseteq \N^d$ we mean any set $S\subseteq \N^d$ satisfying 
$
a + P^{\leq B} \subseteq S \subseteq a+P^*.
$
%
%\begin{definition}  \rm
A set $X\subseteq \N^d$ is \emph{\kanapka {$A$} {$B$}}
if it is a finite union of:

\begin{itemize}
  \item linear sets $a + P^* \subseteq \N^d$ with $\norm(a) \leq B \cdot A$ and
  $\norm(P) \leq A$; and
  \item $B$-approximations of linear sets $a + P^* \subseteq \N^d$ with $\norm(a) \leq A$ and
  $\norm(P) \leq A$.
\end{itemize}
%\end{definition}
%

\noindent
Thus $X$ either includes 
$B$-approximation of a linear set $a+P^*$, whose norm of base is bounded by $A$,
or $X$ includes 
a whole linear set $a+P^*$, whose norm of base is only bounded by $B\cdot A$.
In both cases, norms of periods are bounded by $A$.

%\begin{definition} \rm
%Let $d\in\{2,3\}$.
We say that a class $\C$ of \parvass d is \emph{\parsandwich{$F$}} 
if for every \vass $(V, s)$ in $\C$, its state $q$,
and $B\in\N$,
the set $\reach_q(V,s)$ is \kanapka {$F(M)$} {$B$}, where $M=\size(V,s)$.
The class $\C$ is \emph{\sandwich} if it is \parsandwich{$F$}, for some nondecreasing 
polynomial $F$.
%
%a finite union of:
%
%\begin{itemize}
%  \item linear sets $a + P^* \subseteq \N^d$ with $\norm(a) \leq B \cdot F(M)$ and
%  $\norm(P) \leq F(M)$; and
%  \item $B$-approximations of linear sets $a + P^* \subseteq \N^d$ with $\norm(a) \leq F(M)$ and
%  $\norm(P) \leq F(M)$.
%\end{itemize}
%\end{definition}


\section{Example}
\label{app:example}

Table~\ref{tab:example} shows a counseling session between a client agent and CAMI, with the client's profile shown in Table~\ref{tab:example profile}. During the session, CAMI explores Health and Relationships topics in the first 6 turns.  It subsequently performs Step Into operation from Relationships to Parenting topics in Turn 7, and from Parenting to Paternal Bond topics in Turn 9, before a Switch operation to Habituation topic in Turn 11.  The full exploration path is visualized in Figure~\ref{fig:exploration path}.

\begin{table*}[ht]
\centering
\begin{tabularx}{\textwidth}{c X}
\toprule
        {\bf Turn ID} & {\bf Utterances}                     \\ \midrule
1       & Counselor: Hello. How are you?                                                                                                                                                                                                                                                                                                                     \\ \hline
2       & Client: I am good. What about you?                                                                                                                                                                                                                                                                                                                 \\ \hline
3       & Counselor: I'm doing well, thanks for asking! Can you share how you've been feeling about your health lately, especially in relation to smoking?                                                                                                                                                                                                   \\ \hline
4       & Client: I've been pretty busy with work and juggling family stuff. Sometimes I just think about how nice it is to take a break with my coworkers outside, chatting and during those moments, life feels a bit more relaxed. You know how it is—it’s nice to unwind.                                                                                \\ \hline
5       & Counselor: It sounds like smoking is a comforting routine for you amidst your busy life with work and your daughter. I wonder how it might impact the time you spend with your family.                                                                                                                                                             \\ \hline
6       & Client: I get what you mean, but I really don't see my smoking as an issue. It helps me manage stress, and I don’t think it affects my time with my daughter. We still have good moments together, and I wouldn't want to change that.                                                                                                             \\ \hline
7       & Counselor: It sounds like you find comfort in smoking to manage stress, and that feels crucial to your parenting now. I wonder, though, how you might feel if your daughter started to see smoking as a way to unwind, too. What impact would that have on your relationship with her?                                                             \\ \hline
8       & Client: That’s a tough thought. I remember starting young and how hard it was to quit later. I wouldn’t want that for her. It makes me reflect on what I’m modeling.                                                                                                   \\ \hline
9       & Counselor: It's great that you're reflecting on your modeling for your daughter. Perhaps seeing smoking as something that may weaken your bond could help reshape how you approach stress—considering healthier ways to connect can actually strengthen your relationship and set a positive example for her.                                      \\ \hline
10      & Client: I appreciate that perspective. I do want to set a strong example for her. My experience with my own family and their struggles makes me cautious. I see the patterns and don’t want her to feel that smoking is a go-to solution for stress, like I did.                                                                                   \\ \hline
11      & Counselor: It sounds like your family’s struggles have really shaped your perspective on smoking. You want to protect your daughter from those same patterns, knowing that smoking can not only impact her stress management but also her overall learning and adaptation to challenges. That’s a powerful motivation.                             \\ \hline
12      & Client: Exactly. I want her to learn healthier coping strategies instead of following in my footsteps. I realize that I need to make a change for both our sakes. It’s not just about me; it’s about creating an environment where she can thrive without those habits. It's time to find better ways to manage stress that we can share together.   
\\ \bottomrule
\end{tabularx}
\caption{Counseling Session between CAMI and a Simulated Client.}
\label{tab:example}
\end{table*}



\begin{table*}[ht]
\centering
\begin{tabularx}{\textwidth}{l X}
\toprule
{\bf Profile Item}        & {\bf Description}                     \\ \midrule
Behavioral Problem  & Smoking \\ \hline
State of Mind       & Precontemplation \\ \hline
Personas            & You smoked when you were young.", "You have a 12-year-old daughter. \newline The friends of the your daughter have tried smoking. \newline When you are at work, you all go out and smoke and chit chat and talk about things. \newline You like to have a cigarette after eating. \\ \hline
Motivation          & You are motivated because of the risk of smoking in habituation for your daughter, as you don't want your daughter to think smoking is good thing.    \\ \hline
Beliefs             & You feel smoking enjoyable and relaxing. \newline You would be left out if you would hate to not smoke. \newline You use smoking as a way to manage weight and control eating habits. \newline You have been smoking for a long time and feels it's almost impossible to quit. \\ \hline
Motivation Topics          & Interpersonal Relationships, Parenting, Habituation
\\ \bottomrule
\end{tabularx}
\caption{Client's Profile in the Example.}
\label{tab:example profile}
\end{table*}


\begin{figure*}
    \centering
    \includegraphics[width=\linewidth]{figs/example_explore_path.pdf}
    \caption{Topic Exploration Path by the Counselor in the Example.}
    \label{fig:exploration path}
\end{figure*}



In our subsequent reasoning we rely on the core technical fact:
%proved in Section \ref{sec:sandwich}:
%  
\begin{lemma}\label{lem:2vass-sandwich}
\dvass are \sandwich.
\end{lemma}

The proof of Lemma~\ref{lem:2vass-sandwich} is moved to the Appendix. We sketch here the intuition behind it, since
it is one of our main technical contributions. We first show that it is enough to show Lemma~\ref{lem:2vass-sandwich}
for a simple class of \dvass, called \dslps, which are of the form $\alpha_0 \beta_1^* \alpha_2 \ldots \alpha_{k-1} \beta_k^* \alpha_k$,
where $\alpha_i$ are fixed sequences of transitions and the loops $\beta_i$ are single transitions. This reduction uses standard techniques,
namely Theorem 1 in~\cite{BlondinFGHM15} stating that the reachability relation of a \dvass can be expressed as a union of reachability relations of a \dlps (\dlps are \dslps without the assumption that $\beta_i$ are single transitions) and Theorem 15 in~\cite{DBLP:conf/lics/EnglertLT16} providing the reduction from \dlps to \dslps. Next, we simplify the \dslps even more, using 
%Filip's lemma - to arXiv
Theorem 4.16 in~\cite{DBLP:conf/focs/0001CMOSW24}, which states that any two vectors reachable by an \dslps can be reached
also by a path of the \dslps of a special form: except a short prefix and suffix it zigzags all the time between configurations close to vertical axis to configurations close to horizontal axis. Thus, to prove Lemma~\ref{lem:2vass-sandwich}
it essentially remains to show polynomial approximability for zigzagging paths.
To achieve that, we roughly speaking
investigate how application of two consecutive loops $\beta \in \N_+ \times \N_-$ and $\beta' \in \N_- \times \N_+$
affects the set of reachable configurations on some vertical line close to the vertical axis. We show that arithmetic sequence
is transformed into a finite union of arithmetic sequences such that the difference is kept at most polynomial in size of the \dslps
and the first term grows additively by at most a polynomial value. All that allows us to conclude that the set of vectors
reachable by zigzagging paths is a union of sets of a form similar to $a + Q^* + P^{\leq T}$.
This quite easily implies polynomial approximability.

% !TEX root = main.tex

%\section{Proof of Lemma~\ref{lem:2vass-sandwich}: \dvass are \sandwich}\label{sec:sandwich}

\begin{appendixproof}[Proof of Lemma~\ref{lem:2vass-sandwich}]
We extend the definition of nondecreasing functions to many-argument ones:
a function $f: \N^k \to \N$ is \emph{nondecreasing} if it is monotonic
in every argument and $f(n_1, \ldots, n_k) \geq n_1 + \ldots + n_k$.
In the sequel we often bound certain quantities polynomially, but an exact polynomial is irrelevant. 
We thus say that a value $n$ is \emph{polynomially bounded} in $n_1, \ldots, n_k$
if there exists a nondecreasing polynomial $P: \N^k \to \N$ such that $n \leq P(n_1, \ldots, n_k)$
for all $n_1, \ldots, n_k \in \N$. 
We also write $n \leq \poly(n_1, \ldots, n_k)$.

\para{Linear path schemes}
A $d$-dimensional \emph{linear path scheme} ($d$-\LPS in short, or \LPS if dimension is irrelevant) 
is a sequential \vass where every component is either trivial (a singleton) or a simple cycle,
i.e., a \vass whose control graph is a simple path with disjoint simple cycles attached to some states of the path.
We write down \LPS in the following form 
\[
%\Lambda \ = \ 
\alpha_0\beta_1^* \alpha_1 \cdots \alpha_{k-1} \beta_{k}^* \alpha_k,
\]
where each $\alpha_i$ and $\beta_i$ is  a fixed sequence of transitions. 
Thus the cycles $\beta_i$ of an \lps may be repeated arbitrarily many times (possibly zero). 
An \LPS is \emph{simple} (\SLPS) when all $\beta_i$ are single transitions, i.e.,
each component is either trivial or a single self-loop transition.
When considering the reachability relation in a $d$-\LPS, we often implicitly take the first and the last state
of the \LPS as the source and target state, respectively, and consider the reachability $w\tran w'$ between
vectors $w,w'\in\N^d$ only.
% for two vectors $s, t \in \N^d$, denoted $s \trans{\Lambda} t$, instead of reachability from one configuration to another. This is just a shortening of notation meaning that we consider reachability from $p(s)$ to $q(t)$ where $p$ is the first state of $\Lambda$ (before $\alpha_1)$ and $q$ is the last state (after $\alpha_n$) of $\Lambda$.

%\begin{lemma}\label{lem:two-slps}
%There is a nondecreasing polynomial $P$ such that for every \dvass $V$ together with its two states $p$ and $q$,
%such that $\size(V) = M$ there is a finite set $\Gamma$ of \dslps, each of size at most $P(M)$,
%such $p(u) \tran q(v)$ in $V$ if and only if $u \trans{\Lambda} v$ for some \dslps $\Lambda \in \Gamma$.
%\end{lemma}

%\wojtek{Maybe remove Lemma~\ref{lem:two-slps} and move its proof directly to the proof of Lemma~\ref{lem:2vass-sandwich}.}

\begin{lemma}\label{lem:two-slps}
For every \dvass $V$ and two its states $q$, $q'$,
there is a finite set $\Gamma$ of \dslps of size polynomially bounded in $\size(V)$,
such that for all $w,w'\in\N^2$,
\begin{align*} %\label{eq:slps}
 \ \ q(w) \tran q'(w') \text{ in } V \ \iff \ w \tran w' \text{ in some  \dslps in } \Gamma.
\end{align*}
\end{lemma}

\begin{proof}
The claim follows by combination of
Theorem 3.1 from~\cite{DBLP:journals/jacm/BlondinEFGHLMT21}, due to which we get
a finite set $\Gamma$ of \dlps of size polynomially bounded in $\size(V)$ that satisfies the claim,
with Lemma 5.2 from~\cite{DBLP:journals/jacm/BlondinEFGHLMT21} 
(or Theorem 15 from~\cite{DBLP:conf/lics/EnglertLT16}), due to which
we can transform every \dlps $\lambda$ into an \dslps of size polynomially bounded in $\size(\Lambda)$.
%We get that the size of the obtained \dslps are polynomially bounded in the size of the input \dlps.
%Therefore, the resulting \dslps have size at most polynomial also wrt. the size of the considered \dvass, as required.
%Finally, notice that one can upper-bound any polynomial by a nondecreasing polynomial.
%Indeed, it is sufficient to take absolute values of the coefficients, and possibly add the $P(x) = x$, if the starting polynomial is a constant.
\end{proof}

%Here is the previous formulation of the main sandwich lemma, probably not useful  now.
%
%\begin{lemma}\label{lem:two-sandwich}
%There exists a polynomial $R$ such that for any $2$-VASS of size $M$,
%any configuration $s$, any state $q$ and any numbers $N, K, N_t \in \N$
%the set $\reach_q(s)$ is a finite union of:
%\begin{itemize}
%  \item linear sets $a_i + Q_i^* \subseteq \N^2$ such that $\norm(a_i) \leq B_\bg$, $\norm(Q_i) \leq R(M)$, and
%  \item sandwich sets $S_i \subseteq \N^2$ such that there are $b_i \in \N^2$, $P_i \subseteq_{\fin} \N^2$ with $\norm(b_i) \leq B_\sml$,
%  $\norm(P_i) \leq R(M)$ and $b_i + P_i^{\leq B_\bg} \subseteq S_i \subseteq b_i + P_i^*$,
%\end{itemize}
%where $N_s = \norm(s)$, $B_\sml = (N_s + N_t) \cdot (2N \cdot R(M))^K$ and $B_\bg = (N_s + N_t) \cdot (2 N \cdot R(M))^{2K}$. 
%\end{lemma}
%
%\vskip 5cm

%The following lemma, together with Lemma \ref{lem:two-slps}, immediately imply Lemma~\ref{lem:2vass-sandwich}:

%\begin{lemma}\label{lem:slps-sandwich}
%There exists a polynomial $R$ such that for any $2$-SLPS with of norm $N$ and with $n$ states,
%and any vector $s \in \N^2$ with $N_s = \norm(s)$ and any numbers $K, N_t, N_R \in \N$
%the set $\reach(s)$ is a finite union of:
%\begin{itemize}
%  \item linear sets $a_i + Q_i^* \subseteq \N^2$ such that $\norm(a_i) \leq 2R(N) \cdot B_\bg$, $\norm(Q_i) \leq R(N)$, and
%  \item sandwich sets $S_i \subseteq \N^2$ such that there are $b_i \in \N^2$, $P_i \subseteq_{\fin} \N^2$ with $\norm(b_i) \leq B_\sml$,
%  $\norm(P_i) \leq R(N)$ and $b_i + P_i^{\leq B_\bg} \subseteq S_i \subseteq b_i + P_i^*$,
%\end{itemize}
%where
%%$B_\sml = (N_s + N_t + nN) \cdot (2 N_R \cdot R(N))^K$ and $B_\bg = (N_s + N_t + nN) \cdot (2 N_R \cdot R(N))^{4K}$. 
%$B_\sml = (N_s + N_t + n \cdot R(N))$ and $B_\bg = (N_s + N_t + nN) \cdot (2 (N_R + R(N)))^{K+1}$. 
%\end{lemma}

%\begin{lemma}\label{lem:slps-sandwich}
%There exists a polynomial $Q$ such that for every $2$-SLPS $\Lambda$,
%any vector $s \in \N^2$ such that $M = \max(\size(\Lambda), \norm(s))$ and any number $B \in \N$
%the set $\reach(s)$ is a finite union of:
%\begin{itemize}
%  \item linear sets $a_i + Q_i^* \subseteq \N^2$ such that $\norm(a_i) \leq B \cdot Q(M)$, $\norm(Q_i) \leq Q(M)$, and
%  \item sandwich sets $S_i \subseteq \N^2$ such that there are $b_i \in \N^2$, $P_i \subseteq_{\fin} \N^2$ with $\norm(b_i) \leq Q(M)$,
%  $\norm(P_i) \leq Q(M)$ and $b_i + P_i^{\leq B} \subseteq S_i \subseteq b_i + P_i^*$.
%\end{itemize}
%\end{lemma}

\begin{lemma}\label{lem:slps-sandwich}
\dslps are \sandwich.
\end{lemma}

Before proving Lemma~\ref{lem:slps-sandwich}, we use it together with Lemma~\ref{lem:two-slps}
to prove Lemma~\ref{lem:2vass-sandwich}.
%
To this aim consider a \dvass $(V, s)$ where $s=p(w)$, and its state $q$, and let $M=\size(V,s)$.
%$M = \max(\size(V), \norm(s))$.
By Lemma~\ref{lem:two-slps} we get a finite set $\Gamma$ of \dslps such that:
\[
\reach_q(V,s) \ = \ \bigcup_{\Lambda \in \Gamma} \reach(\Lambda, w).
\]
Moreover, for some nondecreasing polynomial $F$, every $\Lambda \in \Gamma$ satisfies
$\size(\Lambda) \leq F(M)$.
By Lemma~\ref{lem:slps-sandwich}, there is a nondecreasing polynomial $G$
such that for every $\Lambda \in \Gamma$ and $B\in\N$, 
the set $\reach(\Lambda,w)$
is 
\kanapka {$G(\size(\Lambda,w))$} {$B$}.
Combining the last two statements, we deduce that $\reach_q(V, s)$ is
\kanapka {$G(F(M))$} {$B$} for every $B\in\N$, as required.
%
Lemma~\ref{lem:2vass-sandwich} is thus proved (once we prove
Lemma~\ref{lem:slps-sandwich}).
\end{appendixproof}

\begin{appendixproof}[Proof of Lemma~\ref{lem:slps-sandwich}]
%
We generalise finite prefixes $P^{\leq B}$ of $P^*$ as follows.
Let $P = \set{p_1, \ldots, p_m}$. 
For a positive vector $c \in (\Npos)^m$ and $T \in \N$ we define
\[
P^{x \cdot c \leq T} := \setof{\Sigma_{i=1}^m n_i \cdot p_i}{\Sigma_{i=1}^m n_i \cdot c_i \leq T}.
\]
In particular, $P^{\leq T} = P^{x\cdot \vec 1  \leq T}$.
In the sequel, sets of the form
\begin{align} \label{eq:hybrid}
a + P^{x \cdot c \leq T} + Q^*,
\end{align}
for $a\in\N^2$, $c \in (\Npos)^{\card P}$ and $P,Q\subseteqfin \N^2$, 
we call \emph{hybrid} sets.

%\begin{lemma}\label{lem:sandwich-raw}
%There exists a nondecreasing polynomial $R$ such that for every $2$-SLPS $\Lambda$,
%any vector $s \in \N^2$ and $M = \max(\size(\Lambda), \norm(s))$ the set of vectors reachable by $\Lambda$ from $s$ is a finite union of sets $S$ of the following form.
%For each $S$ there exists a base vector $b \in \N^2$, sets of periods vectors $P, Q \subseteq_\fin \N^2$, vector $c \in \N^{|P|}$ and threshold $T \in \N$
%such that
%\[
%S = b + P^{c \cdot x \leq T} + Q^*,
%\]
%where $\norm(b), \norm(P \cup Q), \norm(c) \leq R(M)$ and $c_i \geq 1$ for each $i \in [1, |P|]$.
%\end{lemma}

\begin{lemma}\label{lem:sandwich-raw}
For every \dslps $(\Lambda,s)$,
% any vector $s \in \N^2$ and $M = \size(\Lambda, s)$
the set $\reach(\Lambda,s)$ is a finite union of hybrid sets \eqref{eq:hybrid}, 
% of vectors reachable by $\Lambda$ from $s$ is a finite union of sets $S$ of the following form.
%For each $S$ there exists a base vector $b \in \N^2$, sets of periods vectors $P, Q \subseteq_\fin \N^2$, vector $c \in \N^{|P|}$ and threshold $T \in \N$ such that
%\[
%S = b + P^{c \cdot x \leq T} + Q^*,
%\]
where $\norm(a),  \norm(c), \norm(P \cup Q)$ are bounded polynomially in $\size(\Lambda,s)$.
% and $c_i \geq 1$ for each $i \in [1, |P|]$.
\end{lemma}

Before proving Lemma~\ref{lem:sandwich-raw} we use it to prove Lemma~\ref{lem:slps-sandwich}.
We need to argue that there is a nondecreasing polynomial $F$ such that 
for every \dslps $(\Lambda, s)$ and $B\in\N$, the set $\reach(\Lambda,s)$ is
\kanapka {$F(M)$} {$B$}, where $M=\size(\Lambda,s)$.
%
%Recall that according to the definition of polynomial approximability
%we need to prove that for each $B \in \N$ there is a polynomial $F$ such that for $M = \size(\Lambda, s)$
%the set of points reachable from $s$ via $\Lambda$ is a finite union of:
%\begin{itemize}
%  \item linear sets $S = a + Q^* \subseteq \N^2$ such that $\norm(a) \leq B \cdot F(M)$, $\norm(Q) \leq F(M)$, and
%  \item $B$-approximations of linear sets, namely $S \subseteq \N^2$ such that there are $a \in \N^2$, $Q \subseteq_{\fin} \N^2$ with $\norm(a), \norm(Q) \leq F(M)$, and $a + Q^{\leq B} \subseteq S \subseteq a + Q^*$.
%\end{itemize}
%By Lemma~\ref{lem:sandwich-raw} we have that the considered set is a finite union of sets
%$S$ of the form
%\[
%S = b + P^{c \cdot x \leq T} + Q^*
%\]
%with $\norm(b), \norm(P \cup Q), \norm(c) \leq R(M)$ for some nondecreasing polynomial $R$
%and $c_i \geq 1$ for each $i \in [1, |P|]$.
We fix the nondecreasing polynomial $F(x) = R(x) + R^2(x)$, where $R$ is a polynomial witnessing 
Lemma~\ref{lem:sandwich-raw}, and some arbitrary $B \in \N$,
and prove that each hybrid set $H$ \eqref{eq:hybrid} 
of Lemma~\ref{lem:sandwich-raw} is
\kanapka {$F(M)$} {$B$}.
% 
%It is enough to check each set $S$ separately, namely to check that for each $B \in \N$
%the set $S$ is $(F(M), B)$-approximately semi-linear.
%Fix $B \in \N$.
% and consider the set $b + P^{c \cdot x \leq T} + Q^*$ for some $T \in \N$.
%
We distinguish two cases.
If $T \geq R(M) \cdot B$ then, since $\norm(c)\leq R(M)$, we have
%because each entry of $c$ is at most $R(M)$, we have that
\[
a + (P \cup Q)^{\leq B} \ \subseteq \ H \ \subseteq \ a + (P \cup Q)^*,
\]
and $\norm(a), \norm(P \cup Q) \leq R(M) \leq F(M)$, as required. 
%
On the other hand, if  $T < R(M) \cdot B$ then $H$ is a finite union
of linear sets of the form $u + Q^*$ for $u \in b + P^{c \cdot x \leq T}$, where
\begin{align*}
\norm(u) \ \leq \ \norm(a) + \norm(P) \cdot T 
\ \leq \ R(M) + R(M)^2 \cdot B \leq  F(M) \cdot B,
\end{align*}
as required. 
As before,  $\norm(Q) \leq R(M) \leq F(M)$.
This completes the proof of Lemma~\ref{lem:slps-sandwich} (once
we prove Lemma~\ref{lem:sandwich-raw}). 
\end{appendixproof}
%\begin{align*}
%\norm(u) & \leq \norm(b) + \norm(P) \cdot T \\
%& \leq \norm(s) + n \cdot R(N) + R(N) \cdot B 
%& \leq B + B + R(N) \cdot B = B \cdot (R(N) + 2)  \\
%& \leq B \cdot 2 R'(N)
%\end{align*}
%for $R'(x) = R(x) + 2$. The second to last inequality holds as clearly $\norm(s) \leq B$ and $n \cdot R(N) \leq B$.
%Thus we get Lemma~\ref{lem:slps-sandwich} for the polynomial $R'$.


\begin{appendixproof}[Proof of Lemma~\ref{lem:sandwich-raw}]
The proof occupies the rest of this section.
%The rest of this subsection is devoted to the proof of Lemma~\ref{lem:sandwich-raw}.
We rely on an insightful characterisation of paths of \slps \cite[Theorem 4.16]{DBLP:conf/focs/0001CMOSW24},
which we state below using a slightly different terminology.
%We formulate an easy consequence of this theorem here, in a slightly different and simpler phrasing.
%Before we need to define a few notions.
%
Speaking informally, a \emph{detailing} of an \slps 
$\Lambda = \alpha_0 \beta_1^* \alpha_1 \ldots \alpha_{k-1} \beta_k^* \alpha_k$
is any \slps obtained by fixing exponents of some of the cycles of $\Lambda$.
Formally, a detailing of $\Lambda$ is any $\Lambda'$ obtained by choosing a subset $S \in [1,k]$ and,
for all $i \in S$, by replacing the cycle $\beta_i$ by a path $\beta_i^{n_i}$, for some $n_i\in\N$, 
which becomes an infix of the simple path of $\Lambda'$.
The number of cycles of $\Lambda'$ is thus $k-\card{S}$.
%
An \dslps is \emph{zigzagging} if the effect of its every cycle $\beta_i$ 
belongs either to the quadrant $\Npos \times (-\Npos)$,
or to the quadrant $(-\Npos) \times \Npos$, and additionally effects of every two consecutive cycles belong 
to different quadrants
(the effects of cycles $\beta_1, \ldots, \beta_k$ thus alternate between quadrants), and the effect of the first cycle belongs to $\Npos \times (-\Npos)$ and the effect of the last cycle belongs to $(-\Npos) \times \Npos$.
Finally, an \slps is \emph{short} if it contains at most three cycles, $k\leq 3$.
For $B\in\N$,
a path 
\[
s_0 \trans{\alpha_0} t_0 \trans{\sigma_1} s_1 \trans{\alpha_1} t_1 \ \cdots \ s_{k-1} \trans{\alpha_{k-1}} t_{k-1} \trans{\sigma_{k}} s_k \trans{\alpha_k} t_k
\]
of an \dslps,
where $\sigma_i \in\beta^*$ for $i\in\setto k$, is called \emph{$B$-close} if all the vectors $x\in \{s_0, t_0, s_1, t_1, \ldots, s_k, t_k\}$,
called \emph{midpoints} below,
are \emph{$B$-close} to some axis, namely either $x\in \setfromto 0 B \times \N$ or 
$x\in \N\times\setfromto 0 B$.

\begin{theorem}[Thm 4.16 in~\cite{DBLP:conf/focs/0001CMOSW24}]\label{thm:2slps-zigzag}
For every \dslps $\Lambda$ there is $B\leq \poly(\size(\Lambda))$
such that for every path $s \tran t$ in $\Lambda$ 
there is a detailing $\Lambda' = \Lambda_1 \Lambda_2 \Lambda_3$ of $\Lambda$  
of $\size(\Lambda')\leq \poly(\size(\Lambda))$ and $u, u' \in \setfromto 0 B \times \N$ such that
\begin{enumerate}
%  \item $\size(\Lambda') \leq \poly(M)$,
 %$P(\size(\Lambda))$,
  \item $\Lambda_1$ and $\Lambda_3$ are short, 
  \item $\Lambda_2$ is zigzagging,
  \item there are paths $s \tran u$ in $\Lambda_1$, 
  a $B$-close path $u \tran u'$ in $\Lambda_2$, and 
  $u'\tran t$ in $\Lambda_3$.
\end{enumerate}
%For any $2$-SLPS $\Lambda$, vectors $s, t \in \N^2$ such that $s \trans{\Lambda} t$ and $M = \max(\norm(s), \size(\Lambda))$
%there exists a bound $B = \poly(M)$ and a detailing $\Lambda'$ of $\Lambda$ such that $\Lambda' = \Lambda_1 \Lambda_2 \Lambda_3$
%with the following properties
%\begin{itemize}
%  \item $\size(\Lambda') \leq \poly(M)$,
% %$P(\size(\Lambda))$,
%  \item $\Lambda_1$ and $\Lambda_3$ are short, while $\Lambda_2$ is zigzagging
%  \item there are $u_1, u_2 \in [0,B] \times \N$ such that $s \trans{\Lambda_1} u_1 \trans{\pi} u_2 \trans{\Lambda_3} t$
%  for some $B$-close $\Lambda_2$-path $\pi$.
%\end{itemize}
\end{theorem}

%Due to Theorem~\ref{thm:2slps-zigzag}, in the proof of 
%Lemma~\ref{lem:sandwich-raw} we may safely restrict to paths of the above form. 
Fix in the sequel $B$ given by Theorem~\ref{thm:2slps-zigzag}.
There are only finitely many detailings $\Lambda'$ of $\Lambda$ of a bounded size,
only finitely many possible decompositions of $\Lambda'$ into $\Lambda_1$, $\Lambda_2$ and $\Lambda_3$,
and only finitely many values of $u_1, u'_1\in\setfromto 0 B$.
By Theorem~\ref{thm:2slps-zigzag}, vectors $t$ reachable from $s$ in $\Lambda$ are exactly those
reachable from $s$ in some of detailing $\Lambda'$.
Therefore it is enough
to show Lemma~\ref{lem:sandwich-raw} for the set of vectors $t\in\N^2$ reachable by paths as in point 3 above,
in a fixed \slps $(\Lambda',s)$, where $\Lambda' = \Lambda_1 \Lambda_2 \Lambda_3$ satisfies points 1, 2 above,
and where $u_1 = b$ and $u'_1 = b'$ for some fixed $b,b'\in\setfromto 0 B$.

In a path $u\tran u'$, every second midpoint is $B$-close to one axis,
say $x\in \setfromto 0 B \times \N$,
while the remaining midpoints are $B$-close to the other one.
We relax this requirement slightly, by dropping the latter condition.
A path $u \tran u'$  in the zigzagging \slps $\Lambda_2$
is \emph{$B$-vertical-close} if $u$, $u'$ and every second midpoint $x$
are $B$-close to the vertical axis, namely $x \in \setfromto 0 B \times \N$
(thus the remaining endpoints on the path do not have to be $B$-close to the horizontal axis).
In order to have more flexibility in the proof of Lemma~\ref{lem:sandwich-raw},
in the sequel we consider those paths in $\Lambda'$, as in point 3 in Theorem~\ref{thm:2slps-zigzag}, where the infix
$u\tran u'$ is $B$-vertical-close but not necessarily $B$-close.
Notice that by relaxing the condition to $B$-vertical-closeness
we enlarge the set of considered paths, but do not enlarge the set of reachable points,
as every $t$ such that $s \tran t$ is already reachable by the paths with the middle path being even $B$-close.
We denote by $\reach(\Lambda', s)$ the set of all vectors $t\in\N^2$ reachable by such a path $s\tran t$ with
the middle part being $B$-vertical-close.

We formulate below Claims \ref{cl:prefix}, \ref{cl:infix} and \ref{cl:suffix}
(taking care of a prefix, infix, and suffix, respectively, of a path $s\tran t$), 
show how they imply Lemma~\ref{lem:sandwich-raw}, and finally proceed with the proofs of the three claims.
To this aim we introduce some more notation.
Given a start $a \in \N$, a difference $r \in \N$, and a bound $T \in \N_\infty = \N\cup\{\infty\}$,
the set $a + \{r\}^{\leq T}$ is called \emph{$(a, r, T)$-arithmetic}. 
We omit brackets and write $a + r^{\leq T}$.
In particular, $a + r^{\leq 0} = \{a\}$.
%The first claim essentially says that it is enough to consider only a finite number
%of arithmetic sequences for vectors reachable from $s$ by a short $2$-SLPS.
A \dslps  $\alpha_1 \beta_1^* \alpha_2 \beta_2^*$ is \emph{one-turn} if 
$\eff(\beta_1) \in \Npos  \times -\Npos$ and $\eff(\beta_2) \in -\Npos  \times \Npos$
(it is thus a special case of short zigzagging \dslps).
%
For a set $S\subseteq \N^2$, we use the notation $\reach(\Lambda, S) = \bigcup_{s\in S} \reach(\Lambda, s)$.

In Claims \ref{cl:prefix}, \ref{cl:infix} and \ref{cl:suffix}, we focus on source/target vectors
in $\setfromto 0 B \times \N$ and, intuitively speaking, on arithmetic subsets of each 'line' $\{b\}\times \N$.
First, Claim \ref{cl:prefix} states that the reachability set of a short \dslps, intersected with each line,
is a finite union of arithmetic sets.
Second, Claim \ref{cl:infix} states that the reachability set of a one-turn \dslps from 
an arithmetic set inside a line, intersected with another line, is a finite union of arithmetic sets.
Importantly, the starting point grows only additively, by a polynomially bounded amount, 
as we will apply Claim \ref{cl:infix} $\OO(k)$ times.
Finally, Claim \ref{cl:suffix} states that the reachability set of a short \dslps, from an arithmetic set
inside a line, is a finite union of hybrid sets.
All quantities in the claims are bounded polynomially.

\begin{claim}\label{cl:prefix}
For every short \dslps $(\Lambda,s)$ 
%$M = \size(\Lambda, s)$  
and $u_1 \in [0,B]$,
the set $\{u_2 \mid (u_1, u_2) \in \reach(\Lambda,s)\}$ is a finite union of $(a, r, T)$-arithmetic sets, where
$a \leq \poly(B,M)$, $r \leq \poly(M)$, and $M=\size(\Lambda,s)$.
\end{claim}
%
\begin{claim}\label{cl:infix}
Let $\Lambda$ be a one-turn \dslps,
and $S_1 = a + r^{\leq K}$ for some $a, r, K \in \N$.
Let $u_1, v_1 \in [0, B]$.
The set $R(S_1) := \{v_2 \mid \exists_{u_2 \in S_1} (u_1, u_2) \tran (v_1, v_2) \text{ in } \Lambda\}$
is a finite union of $(a', r', T')$-arithmetic sets with
$a' \leq a + \poly(B,M,r)$ and $r' \leq \max(\poly(M),r)$, where $M=\size(\Lambda)$.
\end{claim}

%
%The last claim takes care about the short suffix of the $2$-SLPS.
%
%\begin{claim}\label{cl:suffix}
%There is a polynomial $R$ such that for every short $2$-SLPS $\Lambda$ such that $M=\size(\Lambda)$, $S_1 = b_1 + \set{p_1}^{\leq T}$
%and $N = \norm(p_1)$ the set $S_2 = \reach_{\Lambda}(S_1)$
%is a finite union of sets $b_2 + P_2^{a \cdot x \leq T} + Q_2^*$ for $\norm(P_2 \cup Q_2), \norm(a) \leq R(N) \cdot R(M)$, $\norm(b_2) \leq (\norm(b_1)+1) \cdot R(N) \cdot R(M)$ and additionally $a_i \geq 1$ for each $i$. 
%\end{claim}
%
\begin{claim}\label{cl:suffix}
For every short \dslps $\Lambda$ and $u= (u_1, u_2), p = (p_1, p_2)\in\N^2$, 
the set $\reach(\Lambda, u + \set{p}^{\leq T})$
is a finite union of hybrid sets \eqref{eq:hybrid},
where $\norm(a), \norm(c), \norm(P), \norm(Q) \leq \poly(\size(\Lambda), \norm(u),\norm(p))$.
\end{claim}

We use the three  claims to derive Lemma~\ref{lem:sandwich-raw}.
As said above,
we consider a fixed \slps $\Lambda' = \Lambda_1 \Lambda_2 \Lambda_3$ and source $s\in\N^2$,
and focus on the set $\reach(\Lambda', s)$ of vectors $t\in\N^2$ such that
there are paths 
%
\begin{align} \label{eq:paths}
\text{$s \tran u$ in $\Lambda_1$,  \ \  
a $B$-vertically close path $u \tran u'$ in $\Lambda_2$, \ \ 
and   $u'\tran t$ in $\Lambda_3$,}
\end{align}
%
for some $u,u'\in \setfromto 0 B \times \N$, where 
$u_1 = b$ and $u'_1 = b'$ are fixed.
Let $M':=\size(\Lambda',s) \leq \poly(M)$.
%
%we aim at characterising the set of vectors $t \in \N^2$ such that
%there is $B \leq \poly(M)$, $u_1, u_2 \in [0, B] \times \N$ and paths $s \trans{\Lambda_1} u_1 \trans{\pi} u_2 \trans{\Lambda_3} t$
%where $\pi$ is a $B$-vertical-close $\Lambda_2$-path. Let us fix some $x_1, x_2 \in [0,B]$ such that $u_1[1] = x_1$ and $u_2[1] = x_2$.
%
%Recall that $M = \size(\Lambda, s)$ and by Theorem~\ref{thm:2slps-zigzag} we have that $\size(\Lambda') \leq \poly(M)$.
First, by Claim~\ref{cl:prefix}, the set
$\{u_2 \mid (u_1, u_2) \in \reach(\Lambda_1,s)\}$ is a finite union of $(a, r, T)$-arithmetic sets,
where $a, r$ are bounded polynomially in $M'$ and $B$.
%
Second,
a path $u\tran u'$ is a concatenation of $\ell\leq \size(\Lambda_2) \leq \poly(M')$
paths of one-turn \slps. 
By $\ell$-fold application of Claim~\ref{cl:infix}, the set
$\{u'_2 \mid (u'_1, u'_2) \in \reach(\Lambda_1 \Lambda_2,s)\}$
is a finite union of arithmetic sets
$a' + (r')^{\leq T'}$, where
%\begin{equation}\label{eq:rprim}
$a', r' \leq \poly(\size(\Lambda_2)) \leq \poly(M')$.
%\end{equation}
Indeed, the bound on $a'$ comes from $\ell$-fold addition of values bounded by 
$\poly(B, \poly(M'), r)\leq \poly(B, \poly(M'), \poly(M'))$,
itself bounded by $\poly(M')$:
%
\begin{equation}\label{eq:aprim}
a' \leq a + \ell \cdot \poly(M') \leq a + \poly(M') \leq \poly(M').
\end{equation}
%
Finally, by Claim~\ref{cl:suffix} the set $\reach(\Lambda',s)$ is a finite union of hybrid sets \eqref{eq:hybrid},
where
$
\norm(a)$, $\norm(c)$, $\norm(P \cup Q) \leq \poly(M', B) \leq \poly(M).
$ 
%By equations~\eqref{eq:rprim}~and~\eqref{eq:aprim} we have $N = \max(a', r') \leq \poly(M)$.
%Therefore
%\[
%\norm(c), \norm(b), \norm(P \cup Q) \leq \poly(M),
%\]
We conclude the proof of Lemma~\ref{lem:sandwich-raw}, keeping in mind that
it still remains to demonstrate Claims~\ref{cl:prefix},~\ref{cl:infix},~and~\ref{cl:suffix}.

\medskip

Here is  a corollary of Lemma~\ref{lem:taming}, useful in the proofs of the three claims:
%
\begin{lemma}\label{lem:solutions}
Consider a system $A\cdot x = b$ of $m$ Diophantine linear equations with $n$ unknowns,
where absolute values of coefficients are bounded by $N$.
Then, the set of solutions is of a form $U+P^*$, where
$\norm(U \cup P) \leq \poly(nN)^{\poly(n,m)}$.
\end{lemma}

\begin{proof}
The solution set is of the form $U+P^*$,
where $U$ is the set of pointwise minimal nonnegative integer solutions of the system,
and $P$ is the set of pointwise minimal nonnegative integer solutions of its homogeneous version
$A\cdot x = \vec 0$.
By Lemma~\ref{lem:taming} each element of $U \cup P$ has norm at most $M= \Oo(nN)^m$.
Therefore, the number of different solutions is at most $(M+1)^n$, which implies 
$\norm(U \cup P) \leq \poly(nN)^{\poly(n,m)}$.
\end{proof}
%
In the sequel we apply Lemma~\ref{lem:solutions} in case when $n$ and $m$ are constants,
in which case Lemma \ref{lem:solutions} yields the bound $\norm(U+P) \leq \poly(N)$.

\begin{proof}[Proof of Claim~\ref{cl:prefix}]
For $d \in \setto 2$ and
a path $\gamma$, let $\eff_j(\gamma)$ denote the $j$-th coordinate of $\eff(\gamma)$, and
let $\drop_j(\gamma)\in\N$ be the maximal value of $-\eff_j(\delta)$, where $\delta$ ranges over prefixes of $\gamma$,
that is  the maximal amount by which the $j$-th coordinate can be decreased along $\gamma$.

\Wlog we assume that $\Lambda$ has exactly three loops, 
$\Lambda = \alpha_1\beta_1^{*}\alpha_2\beta_2^{*}\alpha_3\beta_3^{*}\alpha_4$. 
We describe paths $s = (s_1, s_2) \tran (u_1, u_2)$ in $\Lambda$,
\begin{align*}
(s_1, s_2) = \ & 
(a^1_1, a^1_2) \trans{\alpha_1} (b^1_1, b^1_2) \trans{\beta_1^{n_1}}
(a^2_1, a^2_2) \trans{\alpha_2} (b^2_1, b^2_2) \trans{\beta_2^{n_2}} \\
& (a^3_1, a^3_2) \trans{\alpha_3} (b^3_1, b^3_2) \trans{\beta_3^{n_3}}
(a^4_1, a^4_2) \trans{\alpha_4} (b^4_1, b^4_2) = (u_1, u_2),
\end{align*}
by the following system of linear Diophantine inequalities, with unknowns 
$a^{i}_1, a^{i}_2, b^{i}_1, b^{i}_2$, for $i\in\setto 4$, and $n_i$, for $i\in\setto 3$,
ensuring that the effects of $\alpha_i$ and $\beta_i^{n_i}$ are respected, 
and that all points along $\alpha_i$ remain nonnegative 
:
%
%\begin{itemize} 
%\item $a^{i}_1, a^{i}_2$, for $i\in\setto 4$, denoting the midpoint just before the execution of $\alpha_{i}$ 
%(for $i=1$, the source);
%\item $b^{i}_1, b^{i}_2$, for $i\in\setto 4$, denoting the midpoint just after  the execution of $\alpha_{i}$
%(for $i=4$, the target);
%\item $n_i$, for $i\in\setto 3$, denoting the number of times $\beta_i$ is executed.
%\end{itemize}
%
%\noindent
%Therefore the path from $s$ to $t$ has the following form:
%
%The following system of inequalities $\U$ guarantees that the path of $\Lambda$ is inside the positive quadrant.
%
\begin{align*}
a^{i}_j + \eff_j(\alpha_i) \ = \ & \ b^{i}_j  & a^1_j \ = \ & \ s_j \\
b^{i}_j + n_i \cdot \eff_j(\beta_i) \ = \ & \ a^{i+1}_j & b^4_1 \ = \ & \ u_1 \\
a^i_j - \drop_j(\alpha_i) \ \geq \ & \ 0 
\end{align*}
%
Notice that each $\beta_i$ is a single transition, so nonnegativity of $(b_i^1, b_i^2)$ and $(a_{i+1}^1, a_{i+1}^2)$
implies that all the vectors along  $\beta_i^{n_i}$ are also nonnegative. Therefore, we do not need to
add an analog of $a^i_j - \drop_j(\alpha_i) \geq 0$ for $\beta_i$ to the above system.
%
%\begin{enumerate}[(1)]
%\item For each $i \in [1,4]$ and $d \in [1,2]$ we take care of the effect of $\alpha_i$:\label{eq:LPS_eff_1}
%$$a_{i}^d + \eff_d(\alpha_i) = b_{i}^d$$
%\item For each $i \in [1,3]$ and $d \in [1,2]$ we take care of the effect of $\beta_i$:\label{eq:LPS_eff_2}
%$$b_{i}^d + n_i \cdot \eff_d(\beta_i) = a_{i+1}^d$$
%\item For each $i \in [1,4]$ and $d \in [1,2]$ we take care that counter remains nonnegative along $\alpha_i$
%$$a_i^d + \drop_d(\alpha_i) \geq 0$$ \label{eq:LPS_non}
%\item For each $d \in [1,2]$ we take care that a path starts in $s$.
%$$a_1^d = s_d$$\label{eq:LPS_start}
%\item We take care that the first coordinate of the target is equal to $x$.
%$$b_4^1 = x$$\label{eq:LPS_end}
%\end{enumerate}
%Notice that . 
By adding dummy variables we change inequalities into equations,
thus obtaining a system $\U$ of linear Diophantine equations.
All the coefficients in $\U$ are bounded by $\max(B, M)$, 
and all coefficients in its homogeneous version are bounded by $M$.
Therefore, by Lemma~\ref{lem:solutions} the solution set of $\U$ is $U+P^*$,
where $\norm(U) \leq \poly(B,M)$ and $\norm(P) \leq \poly(M)$.
By projecting the solution set to the variable $b^4_2$, we deduce that
the set $S:=\{u_2 \mid (u_1, u_2) \in \reach(\Lambda,s)\}$ is a finite union of 
linear sets $a+X^*$, where $a \leq P_1(B,M)$ and $X \subseteq [0,P_2(M)]$,
for some nondecreasing polynomials $P_1, P_2$.
%
% $a \leq \poly(B,M)$ and $X \subseteqfin \N$, such that
%for each $x \in X$ we have $x \leq \poly(M)$.
%
%$(a, r, T)$-arithmetic sets, where
%$a \leq \poly(B,M)$, $r \leq \poly(M)$, and $M=\size(\Lambda,s)$.
%
\begin{claim}\label{cl:linear-1dim}
For every $a,b \in \N$ and $B \subseteq [1,b]$, the linear set $a+B^*$ is a finite union
of arithmetic sets $c + d^*$, where $c \leq a+b^3$ and $d \leq b$.
\end{claim}
\begin{proof}
Let $B = \set{b_1, \ldots, b_m}$.
Let $n \in a + B^*$, and let $(k_1, \ldots, k_m) \in \N^m$ be the lexicographically smallest vector such that
$n = a + \sum_{i=1}^m k_i \cdot b_i$.
We observe that $k_i < b$ for all $i\in\setto {m-1}$ since,
% may hold for at most one $i\in\setto m$.
supposing $k_i \geq b$ for
$i < m$, we would get a lexicographically smaller  vector
\[
(k_1, \ldots, k_{i-1}, k_i - b_m, k_{i+1}, \ldots, k_m+ b_i) \in \N^m
\]
with the same property. 
Thus $n \in a + r + b_m^*$,
where $r = \sum_{i<m} k_i \cdot b_i \leq b^3$. 
As $n\in a+ B^*$ was chosen arbitrarily, we deduce
%This means any $n \in a+B^*$ satisfies $n \in c+b_i^*$, where $c \leq a+b^3$. In consequence
\[
a+B^* \ = \bigcup_{c \leq a+b^3, c \in a+B^*} c + b_m^*,
\]
which concludes the proof of Claim~\ref{cl:linear-1dim}.
\end{proof}
%
By Claim~\ref{cl:linear-1dim}, the set $S$ is a finite union of arithmetic sets  $a+x^*$, where 
$a \leq P_1(B,M) + P_2(M)^3$ and $x \leq P_2(M)$,
which concludes the proof of Claim~\ref{cl:prefix}.
Notice that we actually get $T = \infty$ or $T=0$ in all the arithmetic sets, but that is not needed for our considerations.
\end{proof}

% !TEX root = main.tex

\begin{proof}[Proof of Claim~\ref{cl:infix}]
Fix a  one-turn \slps $\Lambda=\alpha_1\beta_1^*\alpha_2\beta_2^*$, and
let $M:=\size(\Lambda)$ and $S_2 := R(S_1)$.
Our goal is to show that the set $S_2 = \bigcup_{c\in S_1} R(\{c\})$
%For $c \in S_1$ let us denote $R(c) = \set{v_2 \in (u_1, c) \tran (v_1, v_2)}$.
%Let $R(S_1) = \bigcup_{c \in S_1} R(c)$. Our goal is to show that $R(S_1)$ 
is a finite union
of $(a', r', T')$-arithmetic sequences for $a' \leq a + \poly(B, M, r)$ and $r' \leq \poly(M)$.
We write $R(c)$ instead of $R(\{c\})$.

Let $\eff(\beta_1) = (x_1, -y_1)$ and $\eff(\beta_2) = (-x_2, y_2)$, for some $x_1, x_2, y_1, y_2 \in \Npos$.
% (recall that $\beta_1$
%and $\beta_2$ belong to the corresponding quadrants by assumptions of Claim~\ref{cl:infix}).
Every path $\rho = \alpha_1\beta_1^{n_1}\alpha_2\beta_2^{n_2}$ is determined by 
a pair $(n_1,n_2) \in \N^2$,
and hence when $(u_1, u_2) \tran (v_1, v_2)$, we may also write $(u_1, u_2) \trans {(n_1,n_2)} (v_1, v_2)$
for $(n_1, n_2)\in\N^2$, or 
$(u_1, u_2) \trans {(n_1,n_2)}$ when the target vector is not relevant.
%and we say that $\rho = \rho(n_1, n_2)$.
Whenever this happens, we necessarily have the equality
%We are interested in possible $c_2 \in \N$ such that there is $c_1 \in S_1$ such that $(b_1, c_1) \trans{(n_1, n_2)} (b_2, c_2)$
%for some $n_1, n_2 \in \N$. Notice that looking at the first coordinate gives the following equation
$
u_1 + \eff_1(\alpha_1 \alpha_2) + n_1 x_1 - n_2 x_2 = v_1
$
of effects on the first coordinate,
which we transform to an equation with two unknowns $n_1,  n_2$:
\begin{equation}\label{eq:xy}
n_1 x_1 - n_2 x_2 = v_1 - u_1 - \eff_1(\alpha_1 \alpha_2).
\end{equation}
%
The set of solutions $(n_1, n_2)$ of~\eqref{eq:xy} is of the form $w+p^*$, 
where $w = (w_1, w_2) \in \N^2$ is the minimal solution of~\eqref{eq:xy}
and $p = (p_1, p_2) \in \N^2$ is the minimal solution of its uniform version, $n_1 x_1 - n_2 x_2 = 0$.
By Lemma~\ref{lem:solutions} we get $\norm(w) \leq \poly(B, M)$. One can easily observe that $\norm(p) \leq 2M$
($(x_1, x_2) = (n_2, n_1)$ is a solution).
%
Let $\eff_2(w) = -w_1 y_1 + w_2 y_2$ and $\eff_2(p) = -p_1 y_1 + p_2 y_2\in \N$ be 
the effects induced by $w$ and $p$, respectively, on the second coordinate.

\begin{example}\label{ex:infix}
\begin{figure}[t]
\vspace{-2cm}
\begin{minipage}{0.5\textwidth}
\begin{tikzpicture}[scale=0.35]
\scalebox{0.7}{
\draw[->, thick] (0, 0) -- (20, 0) node[right] {$x$};
\draw[->, thick] (0, 0) -- (0, 18) node[above] {$y$};

\fill[red] (0,6) circle (7pt);

\draw[->] (0,6) -> (2,5);
\draw[->] (2,5) -> (4,4);
\draw[very thick,->] (4,4) -> (6,3);
\draw[very thick,->] (6,3) -> (8,2);
\draw[very thick,->] (8,2) -> (10,1);
\draw[->] (10,1) -> (12,0);

\draw[->] (4,4) -> (1,7);
\fill[green] (1,7) circle (7pt);

\draw[very thick,->] (10,1) -> (7,4);
\draw[very thick,->] (7,4) -> (4,7);
\draw[->] (4,7) -> (1,10);
\fill[green] (1,10) circle (7pt);

\draw[step=1, gray, thin] (0, 0) grid (19, 17);
}
\end{tikzpicture}
\end{minipage}
\begin{minipage}{0.5\textwidth}
\begin{tikzpicture}[scale=0.35]
\scalebox{0.7}{
\draw[->, thick] (0, 0) -- (20, 0) node[right] {$x$};
\draw[->, thick] (0, 0) -- (0, 18) node[above] {$y$};

\fill[red] (0,9) circle (7pt);

\draw[->] (0,9) -> (2,8);
\draw[->] (2,8) -> (4,7);
\draw[very thick,->] (4,7) -> (6,6);
\draw[very thick,->] (6,6) -> (8,5);
\draw[very thick,->] (8,5) -> (10,4);
\draw[->] (10,4) -> (12,3);
\draw[->] (12,3) -> (14,2);
\draw[->] (14,2) -> (16,1);
\draw[->] (16,1) -> (18,0);

\draw[->] (4,7) -> (1,10);
\fill[green] (1,10) circle (7pt);

\draw[very thick,->] (10,4) -> (7,7);
\draw[very thick,->] (7,7) -> (4,10);
\draw[->] (4,10) -> (1,13);
\fill[green] (1,13) circle (7pt);

\draw[->] (16,1) -> (13,4);
\draw[->] (13,4) -> (10,7);
\draw[->] (10,7) -> (7,10);
\draw[->] (7,10) -> (4,13);
\draw[->] (4,13) -> (1,16);
\fill[green] (1,16) circle (7pt);

\draw[step=1, gray, thin] (0, 0) grid (19, 17);
}
\end{tikzpicture}
\end{minipage}

\caption{Left: $u_2 = 6$, $S_2 = \set{7,10}$. Right: $u_2 = 9$, $S_2 = \set{10,13,16}$.
Thick vectors add up to $p$.}
\label{fig:infix}
\end{figure}

Let $\alpha_1$ and $\alpha_2$ be empty sequences, $\eff(\beta_1) = (2, -1)$ and $\eff(\beta_2) = (-3, 3)$.
Let $u_1 = 0$, $v_1 = 1$. 
The set of solutions $(n_1, n_2)$ of~\eqref{eq:xy} is of the form $w+p^*$, where
$w = (2,1)$ and $p = (3,2)$, and
$\eff_2(p) = 3 \cdot (-1) + 2 \cdot 3 = 3$.
Figure~\ref{fig:infix} shows $R(u_2) = \set{7,10}$ when $u_2 = 6$, and $R(u_2)  = \set{10,13,16}$ when $u_2 = 9$.
In the latter case, elements of $R(u_2)$ correspond to the solutions $w$, $w+p$ and $w+2p$ of~\eqref{eq:xy},
i.e., to $w + kp$ for $k$ in an interval $[0,2]$.
According to Claim~\ref{cl:interval}, this is true in general.
%$S_2 = \setof{u + kp}{k\in I}$, for some interval $I$.
\end{example}

%The following claim states that the number of possible periods $p$ used in the run is an interval.

\begin{claim}\label{cl:interval}
For each $c \in S_1$ there exists an interval $I_{c} = [k_1, k_2]$, where $k_1\in \N$ and $k_2 \in \N_{\infty}$,
such that
$R(c) = \setof{\eff_2(w) + k \cdot \eff_2(p)}{k\in I_{c}}$.
% and path $\rho(b+kp)$ starting from $(b_1, c)$ is a valid if and only if $k \in I_c$. 
\end{claim}

\begin{proof}
It is sufficient to show that whenever $(u_1, c) \trans{w + k_1 \cdot p}$ and
$(u_1, c) \trans{w + k_2 \cdot p}$ for some
$k_1 < k_2 \in \N$,
then
$(u_1, c) \trans{w + k \cdot p}$ for all $k \in [k_1, k_2]$.
Fix $k \in [k_1, k_2]$.
Every point $x$ on the (possibly $\Z$-)path $(u_1, c) \trans{w + k \cdot p}$ is actually on a straight line between some two points,
one on the path $(u_1, c) \trans{w + k_1 \cdot p}$, and the other on the path $(u_1, c) \trans{w + k_2 \cdot p}$.
In consequence $x$, being a weighted average of the two points in $\N^2$, necessarily belongs to $\N^2$. 
Therefore $(u_1, c) \trans{w + k \cdot p}$ is a path.
\end{proof}

We notice that $\eff_2(p)$ can be negative, but this is irrelevant for our arguments.
%
By Claim~\ref{cl:interval}, for each $c \in S_1$ the set $R(c)$ is
is an arithmetic sequence of difference $\essdvass r := \absv{\eff_2(p)}$ and length equal to the cardinality
of the interval $I_{c}$.
%Thus $R(c) = b + (\essdvass r)^{\card{I_c}}$, for some $b\in\N$. 
Let $\spn(c)\in\N_\infty$ be the difference between the supremum of $R(c)$ and the minimal element in $R(c)$.
We say that $\spn(c) = \infty$ if $R(c)$ is infinite.
We split the proof into cases, depending on
whether $\essdvass r$ divides the difference $r$ of the sequence $S_1$, or not.
Additionally we have a case when $\essdvass r = 0$, in other cases we silently assume that $\essdvass r \neq 0$.


\para{Case I: $r$ is divisible by $\essdvass r$}

Therefore, if $\spn(c) \geq r$ then the sequence $R(c)$ actually touches the sequence $R(c+r)$,
i.e., their union is a larger arithmetic sequence of difference $\essdvass r$:

\begin{claim}\label{cl:merged}
If $R(c+r^*) \neq \emptyset$ and $c \geq D:= 3(M+r) \cdot M^2 + M \cdot \norm(w)$,
then $R(c+r^{\leq T}) = b+(\essdvass r)^{\leq T'}$ for some $b \leq c + \eff_2(w) + M \cdot \eff_2(p)$ and $T' \in \N_\infty$.
\end{claim}

\begin{proof}
We first show that $\spn(c) \geq r$.
Suppose $R(c+r^*) \neq \emptyset$ and $c \geq D$.
Due to the first assumption,  for some $n\in\N$ there is a path $(u_1, c+nr) \trans{(n_1, n_2)}$
of the form
\begin{align} \label{eq:Mp}
(u_1, c+nr) \trans{\alpha_1} (\essdvass{x}_1, \essdvass{y}_1) \trans{\beta_1^{n_1}} (\essdvass{x}_2, \essdvass{y}_2) \trans{\alpha_2} (\essdvass{x}_3, \essdvass{y}_3) \trans{\beta_2^{n_2}} (v_1, v_2).
\end{align}
In particular,  $u_1 \geq \drop_1(\alpha_1)$
and therefore $\essdvass x_1\geq 0$.
It is enough to take $(n_1, n_2) := w + Mp$ in \eqref{eq:Mp}.
Then $n_1 \geq M$, so $\essdvass x_2 \geq \essdvass x_1 + M \geq M \geq \drop_1(\alpha_2)$.
Therefore $\essdvass x_3\geq 0$.
Now we show that $c$ is large enough such that $(u_1, c) \trans{w+Mp}$ is nonnegative on the second coordinate as well.
As $\norm(p) \leq 2M$ then $n_1 + n_2 \leq M \cdot \norm(p) + \norm(w) \leq 2M^2 + \norm(w)$. Therefore
$\beta_1^{n_1}$ and $\beta_2^{n_2}$ can in total decrease the second coordinate by at most $M \cdot (2M^2 + \norm(w))$.
As $\alpha_1 + \alpha_2$ can in total decrease the second coordinate by at most $M$ and
$D \geq M + M \cdot (2M^2 + \norm(w))$ we conclude that indeed the path $(u_1, c) \trans{w+Mp}$ is valid.
For the same reasons, for every $m\in [M, M+r]$ there is a path $(u_1, c) \trans{w+mp}$, which guarantees $\spn(c) \geq r$. 

Now we use that fact that $\spn(c) \geq r$.
By monotonicity of \vass, if $R(c) = b+(\essdvass r)^{\leq {T'}}$
then $R(c+r)$ necessarily includes $R(c) + r= b+r+(\essdvass r)^{\leq {T'}}$.
Since $\spn(c) \geq r$, we have $b+r \in R(c)$, but also $b+r \in R(c+r)$,
and therefore the union $R(c) \cup R(c + r)$ forms one arithmetic sequence 
$b+(\essdvass r)^{\leq T'}$, for some $b\in\N$ and $T'$.
The similar reasoning applies to any finite union, namely to $R(c+r^{\leq m})$ for $m\in\N$.
In consequence, for every $T\in \N_\infty$ we have $R(c+r^{\leq T}) = b + (\essdvass r)^{\leq T'}$,
for some $b\in\N$ and $T'\in \N_\infty$,
and since $c+ \eff_2(w) + M \cdot \eff_2(p) \in R(c)$ we get the inequality $b \leq c + \eff_2(w) + M \cdot \eff_2(p)$, as required.
\end{proof}

We are ready for concluding Case I.
%Let $D = 3(M+r) \cdot M^2 + M \cdot \norm(w)$, as in the Claim~\ref{cl:bigspan}.
As $\norm(w) \leq \poly(B, M)$ we have $D \leq \poly(B, M, r)$.
We  partition $S_1 = a + r^{\leq T}$ into two subsets: $S'_1 = S_1 \cap [0,D)$
and $S''_1 = S_1 \cap [D,\infty)$, both being arithmetic sequences of difference $r$,
and consider $S'_1$ and $S''_1$ separately.
 
Concerning $S'_2:= R(S'_1)$,
as $\max(S'_1) \leq D$, all
elements of $S'_2$ are upper-bounded by a polynomial in $M$ and $r$, namely
$\max(S'_2) \leq M^2 \cdot (2M + D)$. 
Thus $S'_2$ can be seen as a finite sum of singletons, each of which
being an $(a', r', T')$-arithmetic sequence with $a' \leq M^2 \cdot (2M + D) \leq \poly(B,M,r)$,
$r' = 1$ and $T' = 0$. 
Clearly $r' = 1 \leq \poly(M)$, and hence $S'_2$ is of the required form.

Now we consider $S''_2:= R(S''_1)$.
If $S''_2 = \emptyset$ we are done.
Otherwise, let $c:= \min(S''_1)$.
Thus $S''_1 = c+r^{\leq T}$ for some $T \in \N_\infty$, 
and $D\leq c \leq \max(a, D+r)$.
%By Claim~\ref{cl:bigspan}, $\spn(c) \geq r$, and therefore using Claim~\ref{cl:neighbours} 
By Claim~\ref{cl:merged} we deduce that
$S''_2 = b + (\essdvass r)^{\leq T'}$ for some 
$b \leq c + \eff_2(w) + M \cdot \eff_2(p)$ and $T' \in \N_\infty'$. 
As $\eff_2(w) \leq \poly(B,M)$, $\eff_2(p) \leq \poly(M)$ and $c \leq a + \poly(B,M, r)$ we get $b \leq a + \poly(B, M, r)$.
We also have $\essdvass r \leq \poly(M)$, and hence
$S''_2$ is of the required form.


\para{Case II: $r$ is not divisible by $\essdvass r$}
%Assume now that $r$ does not divide $\eff_2(p)$. 

In that case we split $S_1 = a + r^{\leq T}$ into several
arithmetic sequences of difference $r \cdot \essdvass r$, namely into sequences of a form 
$(a + m \cdot r) + (r \cdot \essdvass r)^{\leq T'}$, where $m < \essdvass r$, 
and apply the above reasoning to each of this sequences separately. 
As $\essdvass r \leq \poly(M)$
we get also a finite set of arithmetic sequences with the base bounded by $a + \poly(B, M,r)$ and difference bounded by $\poly(M)$,
as required.

\para{Case III: $\essdvass r = 0$}
\Wlog we assume $R(S_1) \neq \emptyset$. For every $c \in S_1$ we have that either $R(c) = c + \eff_2(w)$ or $R(c) = \emptyset$. By monotonicity of VASS we have that if $R(c) \neq \emptyset$ then $R(c+r) \neq \emptyset$. Let $D := 3(M+r) \cdot M^2 + M \cdot \norm(w)$. Similarly as in the proof of Claim~\ref{cl:merged} we observe that if $c \geq D$ then for some $k \in N$ there is a run $(u_1, c) \trans{w+kp}$. Hence we have $R(S_1) = c + \eff_2(w) + r^{\leq T'}$ for some $T' \in \N_\infty$ and some $c \in S_1$ such that $c \leq \max(a, D+r)$. Therefore $R(S_1) = b + r^{\leq T'}$ for some $b \leq a + \poly(B,M,r)$ as required.

\end{proof}



%
%Observe, that if $(b_1, s) \trans \rho (b_2, t)$ for $s \in S_1$ and $t \in S_2$ then $x,y$ satisfy the following equation:
%$$\eff_1(\alpha_1\alpha_2) + x\eff_1(\beta_1) + y\eff_1(\beta_2) = b_2-b_1$$
%Hence we have that the set $S$ of such  pairs $(x,y)$ associated is included in set of solutions of the above equation, which can be represented by Lemma~\ref{lem:taming} as $L(B,P)$ for some sets $B, P \subseteq \N^2$ such that $\norm(B) \leq c(M+B)$ and $\norm(P) \leq cM$ for some constant $c \in \N_+$. We show, that $P$ has only one element $p$. This is because the set of solutions of the following equations over $\Q$ is one dimensional vector space and there is nonnegative solution, because of different signs of $\eff_1(\beta_1)$ and $\eff_1(\beta_2)$:
%$$x\eff_1(\beta_1) + y\eff_1(\beta_2) = 0$$
%Hence each path can be represented as $b+kp$ for some $k \in \N$ and $b \in B$ (recall that with each path we associated pair $(x,y)$). Because $B$ is finite and we are interested in a finite union representation it is enough to show that there exists polynomial $R(M)$ such that the set $S_2' =   \{y_2 \mid \exists_{y_1 \in S_1} (b_1, y_1) \trans{b+kp} (b_2, y_2), k \in \N \}$ is a finite union of $(a', r', T')$-arithmetic sets with $a' \leq a + (B+r) \cdot R(M)$ and
%$r' \leq R(M)$.
%
%We have $p=(p_1, p_2)$ and we write $\eff(p)$ for $\eff(\beta_1^{p_1}\beta_2^{p_2})$. Moreover $b = (b_1, b_2)$ and we write $\eff(b+kp)$ for $\eff(\alpha_1\beta_1^{b_1}\alpha_2\beta_2^{b_2}) + k \eff(p)$. First we solve a case when $|\eff_2(p)|$ divides $r$, to which we will reduce the other case. Let $m = \frac{r}{|\eff_2(p)|}$. Moreover, \mywlog we can assume that $S_2' \neq \emptyset$.
%The intuition is that the set $S_2'$ will be a condensing of $S_1$ due to repetitions of $p$ with base point $a'$ moved a bit. In order to conclude the prove we need a few claims. 
%\begin{claim}
%For each $s \in S_1$ there exists interval $I = [k_1, k_2]$ such that $k_1, k_2 \in \N_{\infty}$ and path $b+kp$ starting from $s$ is a valid path if and only if $k \in I$. 
%\end{claim}
%\begin{proof}
%It is enough to show, that if for $k_1 < k_2 \in \N$ we know, that $b+k_1p$ and $b+k_2p$ are valid paths starting from $s$ then for all $k_3 \in [k_1, k_2]$ we have that path $b+k_3p$ is a valid path starting from $s$. Let us fix $k_3 \in [k_1, k_2]$ and let $b=(b_1, b_2), p=(p_1, p_2) \in \N^2$. Recall, that $\eff(\beta_1) \in \N_+ \times \N_-$ and $\eff(\beta_2) \in \N_- \times \N_+$. Hence it is enough to show the following inequalities:
%$$\eff_1(\alpha_1\beta_1^{b_1+k_3p_1}\alpha_2\beta_2^{b_2+k_3p_2}) = \eff_1(\alpha_1\beta_1^{b_1+k_1p_1}\alpha_2\beta_2^{b_2+k_1p_2})$$
%$$\eff_2(\alpha_1\beta_1^{b_1+k_3p_1}) \geq \eff_2(\alpha_1\beta_1^{b_1+k_2p_1})$$
%For the first equality observe, that:
%\begin{align*}
%\eff_1(\alpha_1\beta_1^{b_1+k_3p_1}\alpha_2\beta_2^{b_2+k_3p_2}) & = \eff_1(\alpha_1\beta_1^{b_1+k_1p_1}\alpha_2\beta_2^{b_2+k_1p_2}) 
%+ (k_3-k_1)(p_1\eff_1(\beta_1) + p_2\eff_1(\beta_2)) \\
%& = \eff_1(\alpha_1\beta_1^{b_1+k_1p_1}\alpha_2\beta_2^{b_2+k_1p_2})
%\end{align*}
%The second inequality follows from the fact that:
%$$b_1+k_3p_1 \leq b_1+k_2p_1$$
%\end{proof}
%Let us by $I_x = [k_1^s, k_2^s]$ denote the interval for $x \in S_1$ and by $|I_x| = k_2^s - k_1^s$. Now we split elements of $S_1$ into three groups depending on the properties of the points in $S_1$.
%\begin{claim}\label{clm:split}
%For each $x \in S_1$ at least one of the following holds:
%\begin{itemize}
%\item $x \leq a + 3cM^2(B+r)$
%\item $x$ is the maximal element of $S_1$ and $[0,m] \subseteq I_x$
%\item $[0, m] \subseteq I_x$ and if $b+kp$ is executable from $(b_1,x)$ for $k > m$ than $b+(k-m)p$ is executable from $(b_1,x+m\eff_2(p))$ and $x+m\eff_2(p) \in S_1$. 
%\end{itemize}
%\end{claim}
%\begin{proof}
%Let us take $x \in S_1$ such that  $x > a + 3cM^2(B+r)$. Firstly, we show that $[0, m] \subseteq I_x$. Recall, that $\Lambda$ is one-turn and observe, that for all $k \in \N$ we have:
%$$\eff_1(\alpha_1\beta_1^{b_1}\alpha_2\beta_2^{b_2}) = \eff_1(\alpha_1\beta_1^{b_1}\alpha_2\beta_2^{b_2}) + k(p_1\eff_1(\beta_1) + p_2\eff_1(\beta_2))$$
%Hence, because $S_2$ is not empty we have that all paths $b+kp$ starting from $(b_1, x)$ are valid on the first counter. Hence we only need to deal with the second counter. Now observe, that for each $k \in [0,m]$ we have that maximal decrease of the second counter along path $b+kp$ is at most :
%\begin{align*}
%M + (b_1+kp_1)\eff_2(\beta_1) & \leq M + (c(M+B)+mcM)M \\
%& \leq 3cM^2(B+m) \leq 3CM^2(B+r) < x
%\end{align*}
%Hence $[0, m] \subseteq I_x$.
%If $x$ is the maximal element we are done. Otherwise,
%we show, that if $b+kp$ is executable from $(b_1,x)$ for $k > m$ than $b+(k-m)p$ is executable from $(b_1,x+m\eff_2(p))$ and $x+m\eff_2(p) \in S_1$. Observe, that $x+m\eff_2(p) = x+r$ if $\eff_2(p) > 0$ and $x+m\eff_2(p) = x-r$ otherwise. Because $S_1 =  a + r^{\leq K}$ and $x$ is not the maximal element of $S_1$ then in the first case $x+m\eff_2(p) \in S_1$. In the second case we have $x -  r \geq a+r-r \geq a$ and hence $x-r \in S_1$.
%
%Now it is left, that $b+(k-m)p$ is executable from $(b_1,x+m\eff_2(p))$. Similarly as before we only need to take care about the second counter. For this it is enough to observe, that:
%\begin{align*}
%x + m\eff_2(p) + \eff_2(\alpha_1\beta_1^{b_1+(k-m)p_1}) 
%& = x + \eff_2(\alpha_1\beta_1^{b_1+kp_1}) + mp_2\eff_2(\beta_2) \\
%& \geq x + \eff_2(\alpha_1\beta_1^{b_1+kp_1})
%\end{align*}
%This is due to $\eff_2(p) = p_1\eff_2(\beta_1) + p_2\eff_2(\beta_2)$ and $\eff_2(\beta_2) \geq 0$.
%Moreover:
%$$x+m\eff_2(p) \geq a+3cM^2(B+r)-r \geq a + M \geq |\eff_2(\alpha_1)|$$
%\end{proof}
%Now we can split $S_1$ into three parts $A_1, A_2, A_3$ such that elements from $A_1$ are the one with $s \leq a + 3cM^2(B+r)$, $A_3$ is the singleton or empty set containing the maximal element and $A_2$ is the rest of $S_1$. Observe, that $A_2 = a_2+r^{\leq K}$ for $a_2 \leq 3cM^2(B+r) + r, a_2 = a + kr$ for some $k \in \N$ and $K = T - k - 1$. Moreover observe, that if path start from $s \in A_2$ and is of the form $b+kp$ for $k > m$ than the same point is reachable from point $(b_1,s+m\eff_2(p))$ by path $b+(k-m)p$ due to Claim \ref{clm:split}. Hence, because we can iterate the argument at some point the starting point will be in $A_1 \cup A_3$ or $k \leq m$. Hence for points from $A_2$ we only need to consider paths $b+kp$ for $k \leq m$. Let us define $B_1, B_2$ and $B_3$.  
%$$B_1 = \bigcup_{x \in A_1, k \in I_x} \set{x + \eff_2(b+kp)}$$
%$$B_2 = \bigcup_{x \in A_2, k \in [0,m]} \set{x + \eff_2(b+kp)} 
%= a_2 + r^{\leq T} + \eff_2(b) + \eff_2(p)^{\leq m}$$
%$$B_3 = \bigcup_{x \in A_3, k \in I_s} \set{x + \eff_2(b+kp)}$$ 
%Which if $A_3 \neq \emptyset$ can be represented as
%$$B_3= a + (T+1)r  + \eff_2(b) + \eff_2(p)^{\leq |I_{x_{max}}|}$$
%where $x_{max}$ is the element of $A_3$. From now on we fix this maximal element and put $|I_{x_max}| = 0$ if it does not exist. 
%Let us observe:
%$$S_2 = B_1 \cup B_2 \cup B_3$$
%We deal separately with $B_1$ and $B_2 \cup B_3$. Because $a_2 + \eff_2(b) \leq 3cM^2(B+r) + r $ clearly, in order to show, that $B_2 \cup B_3$ is  of the form we want it is enough to show the following claim (recall that $\eff_2(p)$ can not be zero due to divisibility condition):
%\begin{claim}
%$(B_2 \cup B_3) \cap [a_2 + \eff_2(b), \infty] = a_2 + \eff_2(b) + |\eff_2(p)|^{\leq T'}$ where $T' = m(T+1) + |I_{x_{max}}|$ if $\eff_2(p) > 0$ and $T' = m(T+1)$ if $\eff_2(p) < 0$ .  
%\end{claim}
%\begin{proof}
%Firstly we show $(B_2 \cup B_3) \cap [a_2 + \eff_2(b), \infty] \subseteq a_2 + \eff_2(b) + |\eff_2(p)|^{\leq T'}$. Let us take any $x \in (B_2 \cup B_3) \cap [a_2 + \eff_2(b), \infty]$. The first case is that $x \in B_2$ and hence 
%$$x = a_2 + \eff_2(b) + k_1r + k_2\eff_2(p)$$
%for $k_1 \leq T$ and $k_2 \leq m$. We have 
%$$x = a_2 + \eff_2(b) + k_1m|\eff_2(p)| + k_2\eff_2(p)$$
%and hence if $\eff_2(p) > 0$ we have 
%$$x = a_2 + \eff_2(b) + (k_1m + k_2)|\eff_2(p)| \in a_2 + \eff_2(b) + |\eff_2(p)|^{\leq T'}$$
%and if $\eff_2(p) < 0$ we have 
%$$x =  a_2 + \eff_2(b) + |\eff_2(p)|^{k_1m-k_2}$$
%Because $x \geq a_2 + \eff_2(b)$ we have $k_1m-k_2 \geq 0$ and hence $x \in a_2 + \eff_2(b) + |\eff_2(p)|^{\leq T'}$ .
%
%Now we show $a_2 + \eff_2(b) + |\eff_2(p)|^{\leq T'} \subseteq (B_2 \cup B_3) \cap [a_2 + \eff_2(b), \infty]$. Let us take $x = a_2 + \eff_2(b) + k|\eff_2(p)|$ for some $k \leq T'$. Firstly let us consider case when $\eff_2(p) > 0$. 
%\begin{align*}
%x & = a_2 + \eff_2(b) + \floor{\frac{k}{m}}m|\eff_2(p)|+ (k\mod m)|\eff_2(p)| \\
%& = a_2 + \eff_2(b) + \floor{\frac{k}{m}}r + (k \mod m)\eff_2(p)
%\end{align*}
%Observe, that if $k \leq m(T+1)$ we have
%$$\floor{\frac{k}{m}} \leq \floor{\frac{m(T+1)}{m}} = T+1$$
%Hence $x \in a_2 + \eff_2(b) + r^{\leq T} + |\eff_2(p)|^{\leq m} = B_2$
%If $k > m(T+1)$ then observe, that $k - m(T+1) \leq |I_{x_{max}}|$ and see, that
%$$x = a_2 + \eff_2(b) + r(T+1) + \eff_2(p)^{k-m(T+1)} \in B_3 $$
%
%Now we consider case when $\eff_2(p) < 0$. Then:
%$$x=a_2 + \eff_2(b) + \ceil{\frac{k}{m}}m|\eff_2(p)|-(m - (k \mod m))|\eff_2(p)|$$
%Then if $k \leq mT$ we have:
%$$x \in a_2 + \eff_2(b) + r^{\leq T}+\eff_2(p)^{\leq m} = B_2 $$
%And if $k > mT$ we have:
%$$x \in a_2 + \eff_2(b) + (T+1)r + \eff_2(p)^{\leq m} \subseteq B_3$$
%
%\end{proof}
%For the set $B_1$ it is enough to prove the following Claim:
%\begin{claim}
%$B_1$ can be represented as a finite union of $(a', r',T')$-arithmetic sets with $a' \leq a + 10cM^4(B+r)$ and $r' \leq cM$.
%\end{claim}
%\begin{proof}
%The first case is that the second counter is at most $3cM^2(B+r)$ at the beginning. The first counter is at most $B$ at the beginning. Then after $\alpha_1$ they are at most $3cM^2(B+r) + M \leq 4cM^2(B+r)$. Then after all execution of the $\beta_1$ the sum of the counters is at most $8cM^3(B+r)$ then after $\alpha_2$ the sum of counters is at most $8cM^3(B+r) + 2M$ and finally after the last execution of $\beta_2$ the sum of counters and hence also the second counter is at most $8cM^4(B+r) + 2M^2 \leq 10cM^4(B+r)$ as needed. Hence there is a finite number of such elements and they can be represented as a singleton arithmetic sets.
%The second case is that the second counter is at least $3cM^2(B+r)$ at the beginning. Let the beginning point be $x$. We show, that elements of $B_1$ obtained by the paths starting in $x$ can be represented as an arithmetic-set. Therefore assume this set is not empty (otherwise it is trivial). Recall, that $\Lambda$ is one-turn and observe, that for all $k \in \N$ we have:
%$$\eff_1(\alpha_1\beta_1^{b_1}\alpha_2\beta_2^{b_2}) = \eff_1(\alpha_1\beta_1^{b_1}\alpha_2\beta_2^{b_2}) + k(p_1\eff_1(\beta_1) + p_2\eff_1(\beta_2))$$
%All paths $b+kp$ starting from $x$ are valid on the first counter. Hence we only need to deal with the second counter. Now observe, that  we have that maximal decrease of the second counter along path $b$ is at most :
%$$M + b_1\eff_2(\beta_1) \leq M + c(M+B)M \leq 3cM^2B$$ Hence path $b$ is also valid on the second counter. Hence $x_2 + \eff_2(b) \in B_1$ and $x_2 + \eff_2(b) \leq a + 3cM^2(B+r) + \eff_2(\alpha_1\alpha_2) + b_1\eff_2(\beta_1) + b_2\eff_2(\beta_2) \leq a + 3cM^2(B+r) + M +  2c(M+B)M \leq a + 10cM^2(B+r)$. Now if $\eff_2(p) > 0$ then the set can be represented as $(x_2 + \eff_2(b), \eff_2(p), T')$-arithmetic set, which satisfies the conditions. Otherwise all elements are at most $a + 10cM^2(B+r)$ and can be represented as singleton arithmetic sets, which satisfies the conditions.
%\end{proof}
%By previous claims we know, that $S_2$ is a finite union of $(a',r', T')$-arithmetic sets for $a' \leq a + 10cM^4(B+r)$ and $r' \leq cM$.
%
%
%Now we come back to the case when $r$ is not divisible by $|\eff_2(p)|$. The first subcase is that $\eff_2(p) = 0$. Then all paths $b+kp$ have the same effect so we only need to consider path $b$. The lowest $x \in S_1$ from which the path $b$ is executable is at most $M + b_1\eff_2(\beta_1) + a + r \leq M + c(M+B)M + a+r \leq a + 3M^2(B+r)$. Hence set $S_2$ can be represented as $(a', r, T')$-arithmetic set for $a' \leq a+ 3M^2(B+r) + \eff_2(b) \leq a+ 3M^2(B+r) + M+2c(M+B)M \leq a + 10M^2(B+r)$. The second subcase is that $\eff_2(p) \neq 0$. Then we can split $S_1$ into several $(a'', \eff_2(p)r, T'')$-arithmetic sets with $a'' \leq a + \eff_2(p)r \leq a + (2cM^2+M)r$ then we can apply the first case and get representation of $S_2$ as a finite union of $(a',r', T')$-arithmetic sets with $r' \leq cM$ and $a' \leq a + (2cM^2+M)r + 10cM^4(B+r) \leq a+ 13M^4(B+r)$. Thus by setting $R(M) = 13cM^4$ we conclude the proof. 
%\end{proof}



\begin{proof}[Proof of Claim~\ref{cl:suffix}]
\Wlog we assume $\Lambda = \alpha_1\beta_1^*\alpha_2\beta_2^*\alpha_3\beta_3^*\alpha_4$.
Let $S_1 = b + \set{p}^{\leq T}$  and $S_2 = \reach_{\Lambda}(S_1)$. Recall, that $\eff_j(\gamma)$ denotes the $j$-th coordinate of $\eff(\gamma)$, and
$\drop_j(\gamma)\in\N$ is the maximal value of $-\eff_j(\delta)$, where $\delta$ ranges over prefixes of $\gamma$,
that is  the maximal amount by which the $j$-th coordinate can be decreased along $\gamma$.

Similarly as in the proof of Claim~\ref{cl:prefix} we describe paths $s = (s_1,s_2) \tran (t_1, t_2) = t$ in $\Lambda$,
\begin{align*}
(s_1, s_2) = &
(a_1^1, a_2^1) \trans{\alpha_1} (b_1^1, b_2^1) \trans{\beta_1^{n_1}}
(a_1^2, a_2^2) \trans{\alpha_2} (b_1^2, b_2^2) \trans{\beta_2^{n_2}} \\
& (a_1^3, a_2^3) \trans{\alpha_3} (b_1^3, b_2^3) \trans{\beta_3^{n_3}}
(a_1^4, a_2^4) \trans{\alpha_4} (b_1^4, b_2^4) = (t_1, t_2).
\end{align*}
Notice that $(s_1, s_2) = (a_1^1, a_2^1) \in S_1$, so $(a_1^1, a_2^1) = u + p^n$ for some $n \in \N$. The following system of linear Diophantine inequalities $\U$ with unknowns $a_1^i, a_2^i, b_1^i, b_2^i$ for $i \in [1,4]$, and $n_i$, for $i \in [1,3]$, and unknown $n$, ensures that the effects of $\alpha_i$ and $\beta_i^{n_i}$ are respected, and that all points along $\alpha_i$ remain nonnegative and that $s \in S_1$:

\begin{align*}
a^{i}_j + \eff_j(\alpha_i) \ = \ & \ b^{i}_j  & a^1_j \ = \ & \ u_j + n \cdot p_j \\
b^{i}_j + n_i \cdot \eff_j(\beta_i) \ = \ & \ a^{i+1}_j & n \ \leq \ & \ T \label{eq:LPS_bound} \\
a^i_j - \drop_j(\alpha_i) \ \geq \ & \ 0 
\end{align*}
Notice that each $\beta_i$ is a single transition, so nonnegativity of $(b_i^1, b_i^2)$ and $(a_{i+1}^1, a_{i+1}^2)$
implies that all the vectors along  $\beta_i^{n_i}$ are also nonnegative. Therefore, we do not need to
add an analog of $a^i_j - \drop_j(\alpha_i) \geq 0$ for $\beta_i$ to the above system. We first focus on the solutions of $\U$ without the equation $n \leq T$, 
and inequalities $a^i_j - \drop_j(\alpha_i) \geq 0$ transformed into equations with dummy variables on the right, similarly as in the proof of Claim~\ref{cl:prefix}. Let us call such system $\U'$. All the coefficients of $\U'$ are bounded by $\max(M, \norm(u), \norm(p))$. By Lemma~\ref{lem:solutions} set of solutions of $\U'$ can be described as $L(U, V)$ for $\norm(U \cup V) \leq \poly(M, \norm(u), \norm(p))$.

%$\norm(U \cup P) \leq R(N) \cdot R(M)$ for some polynomial $R$. After adding equalities~\eqref{eq:LPS_end2} by Proposition~\ref{prop:fix_variables} we get solutions of the form $L(U', P')$ for $\norm(U') \leq R(N) \cdot R(M) \cdot (\norm(b_1) + 1)$ and $\norm(P') \leq R(N) \cdot R(M)$.

Now we have to care about the last inequality, namely $n \leq T$. If for some $a \in U$ we have $n > T$ then we can remove it from $U$. Let $U'$ be the set $U$ without the removed elements.
As the set $U'$ is finite it is enough to prove the conclusion of Claim~\ref{cl:suffix} separately for each $a \in U'$. Fix $a \in U'$.
We have $\norm(a), \norm(V) \leq \poly(M, \norm(u), \norm(p))$.
It is enough to prove that elements of the set $L(a, V)$ that additionally satisfy $n \leq T$, projected to the variables
$b^4_1$ and $b^4_2$ can be described as a finite union of the sets of the form we need.

Let us consider all the elements of set $V$. Let $Q \subseteq V$ be the set of these elements $v \in V$,
for which $v[n] = 0$ (that means that unknown $n$ is equal to $0$ in elements $v$),
while $P = V \setminus Q$ be the set of the other elements $v \in V$,
so that for which $v[n] > 0$.
Notice that using elements in $Q$ does not influence satisfying $n \leq T$, therefore they
can be used unbounded number of times.
Let $P = \set{p_1, \ldots, p_\ell}$ and let $p_i[n] = c_i$. For each $x \in L(a, V)$ we have $x = a + q + \Sigma_{i=1}^{\ell}k_i \cdot p_i$ where $q \in Q^*$. Hence, in order to satisfy $n \leq T$ we have to satisfy $\innprod c k \leq (T - a[n])$, where $c = (c_1, \ldots, c_\ell) \in \N_{>0}^\ell$ and $k = (k_1, \ldots, k_\ell) \in \N^\ell$.
As $c \in \N_{>0}^\ell$, if $T - a[n] < 0$ then the set of solutions is empty. Otherwise, the set of solutions can be represented as $a+P^{c \cdot x \leq T - a[n]}+Q^*$. Recall that $\norm(a), \norm(P \cup Q) \leq \poly(M, \norm(u), \norm(p))$, and additionally $c \in \N_{>0}^{\ell}$. Additionally $\norm(c) \leq \ell \cdot \norm(P)$, where
$\ell$ is the number of elements in $P$, thus $\ell \leq (\norm(P) + 1)^k$, where $k$ is the number of unknowns in $\mathcal U$
(so $k$ is a constant).
In consequence $\norm(c) \leq \poly(M, \norm(u), \norm(p))$. Summarising, the projection
of $a+P^{c \cdot x \leq T - a[n]}+Q^*$ into variables $b^4_1$ and $b^4_2$ is of the required form.
\end{proof}
%
Claims \ref{cl:prefix}, \ref{cl:infix} and \ref{cl:suffix} are thus shown, and hence so is Lemma~\ref{lem:sandwich-raw}.
%
\end{appendixproof}


In the proof of Lemma~\ref{lem:main} we actually need polynomial approximability not only for \dvass, but also for
its generalisation, \geomvass. It is stated below and shown in the Appendix using Lemma~\ref{lem:2vass-sandwich}

\begin{lemma} \label{lem:geom-sandwich}
\Geomvass are \sandwich.
\end{lemma}
%
\begin{appendixproof}[Proof of Lemma~\ref{lem:geom-sandwich}]
Fix an arbitrary \geomvass $(V, s)$ and let $M = \size(V,s)$.
Norms of vectors generating $\cone V$ --- i.e., effects of simple cycles --- are at most $M$, 
as no transition repeats along a simple cycle.
The effect $\delta \in \Z^3$  of each simple cycle
satisfies $\innprod a \delta = 0$, where $a\in\Z^3$ is a vector orthogonal to $\Lin V$,
or equivalently, orthogonal to some two effects of simple cycles.
The vector $a$ is thus an integer solution of a system of 2 linear equations with 3 unknowns,
where absolute values of coefficients are bounded by $M$.
By Lemma \ref{lem:taming}, there is such an integer solution 
$a=(a_1, a_2, a_3)$ with $\norm(a)\leq D=\OO(M^2)$.


In consequence, on every path $s \tran t$ the value of inner product with $a$ is bounded polynomially
with respect to $M$:
%
\begin{claim} \label{claim:axx}
Every configuration $q(x)\in \reach(V,s)$  satisfies
$-C \leq \innprod a x \leq C$, where $C = \OO(M\cdot D)$. 
\end{claim}
%

We rely on the construction of \cite[Lemma 5.1]{Zhang-geom}, which transforms a 
\geomvass $(V,s )$ into a \dvass $(\essdvass V, \essdvass s)$ of size at most $R(M)$
for some polynomial $R$, by dropping on of dimensions of $V$.

\para{Case I: $a$ contains both positive and negative numbers}
\Wlog assume that $a_1, a_2 \geq 0$ and $a_3<0$,
in which case it is the third coordinate which is dropped 
by the construction of \cite[Lemma 5.1]{Zhang-geom}.
States of $\essdvass V$ are of the form $\pair q c$, where $q\in Q$ and $c\in\setfromto {-C} C$, 
plus some further auxiliary states, omitted here.
Due to Claim \ref{claim:axx},
there is a one-to-one correspondence between
reachable configurations in $V$ and reachable configurations in $\essdvass V$:
\[
e = q(x_1, x_2, x_3) \quad \longmapsto \quad
\essdvass e = \pair q c(x_1, x_2), 
\qquad
\text{ where } c = \innprod a x.
\]
%
The tight correspondence between paths of $V$ and $\essdvass V$,
 given in Claim \ref{claim:V321} below, is essentially 
Lemma 5.1 of \cite{Zhang-geom}:
%
\begin{claim} \label{claim:V321}
For every configurations $s,u$,
here is a path $s \tran u$ in $\essdvass V$
if, and only if,
there is a path $\essdvass s \tran \essdvass u$ in $\essdvass V$.
\end{claim}
%
By Lemma \ref{lem:2vass-sandwich}, there is a polynomial $F$ such that 
for every $B\in\N$,
in the \dvass
$(\essdvass V, \essdvass s)$ obtained by the above construction, 
for every its state $\pair q c$, the set 
$\reach_{\pair q c}(\essdvass V, \essdvass s)$ is \kanapka {$F(M')$} {$B$}, where
$M'=\size(\essdvass V, \essdvass s) \leq R(M)$, and hence also
\kanapka {$F(R(M))$} {$B$}.
We claim that for every state $q\in Q$, for every $B\in\N$, 
the set $\reach_q(V,s)$ is \kanapka {$G(F(R(M)))$} {$B$}, for some nondecreasing polynomial $G$.
Indeed, for any $B\in\N$, any ($B$-approximation of) a linear set 
$L = w + P^*\subseteq \N^2$, where $w=(w_1, w_2)$, witnessing that
$\reach_{\pair q c}(\essdvass V,\essdvass s)$  is \kanapka {$F(R(M))$} {$B$}
is transformed to a ($B$-approximation of) linear set $L'$ witnessing that
$\reach_{q}(V,s)$  is \kanapka {$G(F(R(M)))$} {$B$}, as follows.
Take as base the unique vector $w'=(w_1, w_2, w_3)$ such that $a_1 w_1 + a_2 w_2 + a_3 w_3 = c$.
For every period $p=(p_1, p_2) \in P$, take into the set $P'$  the unique vector
$p'=(p_1, p_2, p_3)$ such that $a_1 p_1 + a_2 p_2 + a_3 p_3 = 0$.
Since $a_3>0$, it is guaranteed that $p_3 \geq 0$, and therefore $p'\in\N^3$.
Let the polynomial $G$ bound the blowup of $\norm(b')$ with respect to $\norm(b)$, and 
$\norm(p')$ with respect to $\norm(p)$, for instance
$G(x) = M \cdot x + B$.
The union of all ($B$-approximations of) 
so described sets $L' = w' + (P')^*$, for all $c\in\setfromto {-C} C$, provides the witness that
$\reach_{q}(V,s)$ is \kanapka {$G(F(R(M)))$} {$B$}.

\para{Case II: $a$ is non-negative or non-positive}
\Wlog assume $a\geq \vec 0$ and $a_3 > 0$.
By Claim \ref{claim:axx}, for each $q(x) \in \reach(V,s)$
we thus have $x_3 \leq C$.
%
We transform $(V,s)$ into $(\essdvass V, \essdvass s)$ 
with states of the form $\pair q c$, where $q\in Q$ and $c\in\setfromto {0} C$, 
by storing the third coordinate in state:
\[
e = q(x_1, x_2, x_3) \quad \longmapsto \quad
\essdvass e = \pair q c(x_1, x_2), 
\qquad
\text{ where } c = x_3.
\]
As above, $\size(\essdvass V, \essdvass s) \leq R(M)$, for a polynomial $R$.
The argument  that for every $B\in\N$,
the set $\reach_{q}(V,s)$ is \kanapka {$G(F(R(M)))$} {$B$}, is similar to Case I (but simpler).
\end{appendixproof}

%
%\begin{lemma} \label{lem:nondiag-sandwich}
%Non-forward-diagonal lollypop \tvass are \sandwich.
%%\slawek{uwaga na $\norm(a) \leq 2R(M) \cdot B$}
%\end{lemma}
%%
%%\begin{proof}
%\begin{proof}[Proof of Lemma \ref{lem:nondiag-sandwich}]
%%\slawek{could go to appendix}
%Exactly as in Case 3 of the proof of Lemma \ref{lem:1comp}, one can transform
%a non-forward-diagonal lollypop \tvass $(V, s)$ into
%three \geomvass $(V_1, s_1), (V_2, s_2), (V_3, s_3)$.
%In each configuration of the three \dvass, the value of the
%missing coordinate is determined by state.
%By Claim \ref{claim:nondiag2vass}, the lengths of paths in $V$ are the same as (jointly) the 
%paths in the three \geomvass, and
%by Claim \ref{claim:nondiagsize} the size of each $(V_j, s_j)$  is bounded by
%$R(M)$, where $M = \size(V,s)$.
%The blowup of the transformation is thus bounded by a nondecreasing polynomial $R$.
%We combine the transformation with Lemma \ref{lem:geom-sandwich},
%according to which there is a nondecreasing polynomial $F$
%such that the class of \geomvass is \parsandwich{$F$},
%to deduce that
%the class of non-forward-diagonal 1-component \tvass is \parsandwich{$(F(R(\_))$}.
%Indeed, all ($B$-approximations of) linear sets witnessing that
%$\reach_{\pair q b}(s_j)$ in $V_j$ is \kanapka {$F(R(M))$} {$B$}, fro all $b$ and $j$,
%jointly witness that
%$\reach_{q}(s)$ in $V$ 
%is \kanapka {$F(R(M))$} {$B$}.
%%by adding the missing coordinate as follows:
%%add the missing coordinate in the base vector $b$ as determined by the target state $q$;
%%and in each period vector $v\in P$, add the missing coordinate 0.
%%\end{proof}
%\end{proof}
%

\section{Main Proof}
\label{sec:mainproof}
We now set out to prove Theorem~\ref{thm:main} following the proof outline sketched in Section~\ref{sec:overview}. We start by a series of reductions that allow us to focus on a simpler task of establishing Theorem~\ref{thm:main} only for a small range of $\gamma$ and $\Loss_{\rS}^\gamma(w)$. We describe these reductions in Section~\ref{sec:reduct} and then proceed to the main arguments in Section~\ref{sec:mainargs}.

\iffalse
\begin{table*}[htbp]
\tiny
\begin{center}
\begin{tabular}{lccccccccccccc}\toprule
Model, ft setting & mem & \#param & ARC-c & ARC-e & BoolQ & HS & OBQA & PIQA & rte & SIQA & WG & Avg
%\\\cmidrule(lr){2-3}\cmidrule(lr){4-5} \cmidrule(lr){6-7} \cmidrule(lr){8-9}\cmidrule(lr){10-11} \cmidrule(lr){12-13} \cmidrule(lr){14-15} \cmidrule(lr){16-17} 
\\\cmidrule(lr){1-13}
Llama2(7B), LoRA, $r=64$ & 23.46GB & 159.9M(2.37\%) & \textbf{44.97} & 77.02 & 77.43 & \textbf{57.75} & 32.0 & \textbf{78.45} & 62.09 & \textbf{47.75} & 68.75 & 60.69\\
Llama2(7B), SPruFT, $r=128$ & \textbf{17.62GB} & 145.8M(2.16\%) & 43.60 & \textbf{77.26} & \textbf{77.77} & 57.47 & \textbf{32.6} & 78.07 & \textbf{64.98} & 46.67 & \textbf{69.30} & \textbf{60.86} \\\cmidrule(lr){2-13}
Llama2(7B), FA-LoRA, $r=64$ & 17.25GB & 92.8M(1.38\%) & 43.77 & \textbf{77.57} & 77.74 & \textbf{57.45} & 31.0 & 77.86 & \textbf{66.06} & \textbf{47.13} & 69.06 & 60.85\\
Llama2(7B), FA-SPruFT, $r=128$ & \textbf{15.21GB} & 78.6M(1.17\%) & \textbf{43.94} & 77.22 & \textbf{77.83} & 57.11 & \textbf{32.0} & \textbf{78.18} & 65.70 & 46.47 & \textbf{69.38} & \textbf{60.87}\\\midrule
Llama3(8B), LoRA, $r=64$ & 30.37GB & 167.8M(2.09\%) & \textbf{53.07} & \textbf{81.40} & \textbf{82.32} & \textbf{60.67} & 34.2 & \textbf{79.98} & 69.68 & \textbf{48.52} & \textbf{73.56} & \textbf{64.82}\\
Llama3(8B), SPruFT, $r=128$ & \textbf{24.49GB} & 159.4M(1.98\%) & 52.47 & 81.10 & 81.28 & 60.29 & \textbf{34.6} & 79.76 & \textbf{70.04} & 47.75 & 73.24 & 64.50 \\\cmidrule(lr){2-13}
Llama3(8B), FA-LoRA, $r=64$ & 24.55GB & 113.2M(1.41\%) & \textbf{52.47} & \textbf{81.36} & \textbf{82.23} & 60.17 & \textbf{35.0} & \textbf{79.76} & \textbf{70.04} & \textbf{48.31} & \textbf{73.56} & \textbf{64.77}\\
Llama3(8B), FA-SPruFT, $r=128$ & \textbf{22.41GB} & 92.3M(1.15\%) & 52.22 & 81.19 & 81.35 & \textbf{60.20} & 34.2 & 79.71 & 69.31 & 47.13 & 73.01 & 64.26 \\\bottomrule
\end{tabular}
%\vspace{-0.2cm}
\caption{Fine-tuning Llama on Alpaca dataset for 5 epochs and evaluating on 9 tasks from EleutherAI LM Harness. "mem" represents the memory usage, with further details provided in Appendix~\ref{apdx:measure}. \#param is the number of trainable parameters, where the difference of \#param between the two approaches depends on the architecture of Llama, as some layers have $d_{in} \neq d_{out}$. Note that 10 million trainable parameters only account for less than 0.15GB of memory requirement. FA indicates that we freeze attention layers, but not including MLP layers followed by attention blocks. HS, OBQA, and WG represent HellaSwag, OpenBookQA, and WinoGrande datasets. More details of datasets can be found in Appendix~\ref{apdx:data}. The ablation study for different $r$ and the comparison with other LoRA variants can be found in Appendix~\ref{apdx:ablation}. All reported results are accuracies on the corresponding tasks. \textbf{Bold} indicates the best results of two approaches on the same task.} \label{tab:llm} 
\end{center}
\end{table*}
\fi

\begin{table*}[htbp]
\tiny
\begin{center}
\begin{tabular}{lccccccccccccc}\toprule
Model, ft setting & mem & \#param & ARC-c & ARC-e & BoolQ & HS & OBQA & PIQA & rte & SIQA & WG & Avg
\\\cmidrule(lr){1-13}
Llama2(7B)\\ \cmidrule(lr){1-1} 
LoRA, $r=64$ & 23.46GB & 159.9M(2.37\%) & \textbf{44.97} & 77.02 & 77.43 & 57.75 & 32.0 & \textbf{78.45} & 62.09 & 47.75 & 68.75 & 60.69\\
VeRA, $r=64$ & 22.97GB & 1.374M(0.02\%) & 43.26 & 76.43 & 77.40 & 57.26 & 31.6 & 78.02 & 62.09 & 45.85 & 68.75 & 60.07\\
DoRA, $r=64$ & 44.85GB & 161.3M(2.39\%) & 44.71 & 77.02 & 77.55 & \textbf{57.79} & 32.4 & 78.29 & 61.73 & \textbf{47.90} & 68.98 & 60.71\\
RoSA, $r=32, d=1.2\%$ & 44.69GB & 157.7M(2.34\%) & 43.86 & \textbf{77.48} & \textbf{77.86} & 57.42 & 32.2 & 77.97 & 63.90 &  47.29 & 69.06 & 60.78\\
SPruFT, $r=128$ & \textbf{17.62GB} & 145.8M(2.16\%) & 43.60 & 77.26 & 77.77 & 57.47 & \textbf{32.6} & 78.07 & \textbf{64.98} & 46.67 & \textbf{69.30} & \textbf{60.86} %\\\cmidrule(lr){2-13}
%FA-LoRA, $r=64$ & 17.25GB & 92.8M(1.38\%) & 43.77 & \textbf{77.57} & 77.74 & \textbf{57.45} & 31.0 & 77.86 & 66.06 & \textbf{47.13} & 69.06 & 60.85\\
%FA-DoRA, $r=64$ & 30.61GB & 93.6M(1.39\%) & 43.94 & 77.44 & 77.49 & 57.44 & 31.0 & 77.86 & \textbf{66.43} & 46.98 & 69.14 & 60.86\\
%FA-RoSA, $r=32, d=1.2\%$ & 38.34GB & 98.3M(1.46\%) & \textbf{44.28} & 77.02 & 77.68 & 57.22 & 31.0 & 77.97 & 64.26 & 46.32 & 69.22 & 60.55\\
%FA-SPruFT, $r=128$ & \textbf{15.21GB} & 78.6M(1.17\%) & 43.94 & 77.22 & \textbf{77.83} & 57.11 & \textbf{32.0} & \textbf{78.18} & 65.70 & 46.47 & \textbf{69.38} & \textbf{60.87}
\\\midrule
Llama3(8B)\\ \cmidrule(lr){1-1} 
LoRA, $r=64$ & 30.37GB & 167.8M(2.09\%) & 53.07 & 81.40 & 82.32 & 60.67 & 34.2 & 79.98 & 69.68 & 48.52 & 73.56 & 64.82\\
VeRA, $r=64$ & 29.49GB & 1.391M(0.02\%) & 50.26 & 80.30 & 81.41 & 60.16 & 34.4 & 79.60 & 69.31 & 46.93 & 72.77 & 63.90\\
DoRA, $r=64$ & 51.45GB & 169.1M(2.11\%) & \textbf{53.33} & \textbf{81.57} & \textbf{82.45} & \textbf{60.71} & 34.2 & \textbf{80.09} & 69.31 & \textbf{48.67} & \textbf{73.64} & \textbf{64.88}\\
RoSA, $r=32, d=1.2\%$ & 48.40GB & 167.6M(2.09\%) & 51.28 & 81.27 & 81.80 & 60.18 & 34.4 & 79.87 & 69.31 & 47.95 & 73.16 & 64.36\\
SPruFT, $r=128$ & \textbf{24.49GB} & 159.4M(1.98\%) & 52.47 & 81.10 & 81.28 & 60.29 & \textbf{34.6} & 79.76 & \textbf{70.04} & 47.75 & 73.24 & 64.50 %\\\cmidrule(lr){2-13}
%FA-LoRA, $r=64$ & 24.55GB & 113.2M(1.41\%) & 52.47 & 81.36 & 82.23 & 60.17 & \textbf{35.0} & 79.76 & 70.04 & 48.31 & \textbf{73.56} & 64.77\\
%FA-DoRA, $r=64$ & 40.62GB & 114.3M(1.42\%) & \textbf{52.56} & \textbf{81.69} & \textbf{82.26} & \textbf{60.20} & 34.4 & \textbf{79.82} & \textbf{70.40} & \textbf{48.46} & 73.40 & \textbf{64.80}\\
%FA-RoSA, $r=32, d=1.2\%$ & 42.31GB & 124.3M(1.55\%) & 52.22 & 81.19 & 82.05 & 60.11 & 34.4 & 79.76 & 69.31 & 47.70 & 73.16 & 64.43\\
%FA-SPruFT, $r=128$ & \textbf{22.41GB} & 92.3M(1.15\%) & 52.22 & 81.19 & 81.35 & \textbf{60.20} & 34.2 & 79.71 & 69.31 & 47.13 & 73.01 & 64.26 
\\\bottomrule
\end{tabular}
%\vspace{-0.2cm}
\caption{Fine-tuning Llama on Alpaca dataset for 5 epochs and evaluating on 9 tasks from EleutherAI LM Harness. ``mem" represents the memory usage, with further details provided in Appendix~\ref{apdx:measure}. \#param is the number of trainable parameters, where the difference of \#param between the two approaches depends on the architecture of Llama, as some layers have $d_{in} \neq d_{out}$. %FA indicates that we freeze attention layers, but not including MLP layers followed by attention blocks. 
HS, OBQA, and WG represent HellaSwag, OpenBookQA, and WinoGrande datasets. %More details of datasets can be found in Appendix~\ref{apdx:data}. 
The ablation study for different $r$ can be found in Appendix~\ref{apdx:ranks}. All reported results are accuracies on the corresponding tasks. \textbf{Bold} indicates the best result on the same task. } \label{tab:llm} 
\end{center}
\end{table*}

\section{Experimental Setup}\label{sec:setup}

%(0.5 page)
%Why the chosen framework?
%Some prior approaches

%- parameter settings
%- uniform across layers vs greedy ... 
%- potential transformer-specific details

%Equations about what these methods do.. 

%(0.5 page)
%Which NN architectures are used, why?
%Number of parameters, layers, ...

%(Potential prior work on compression -- )

\subsection{Datasets} \label{subsec:dataset}
We use multiple datasets for different tasks. For image classification, we fine-tune models on the training split and evaluate it on the validation split of Tiny-ImageNet~\citep{tavanaei2020embedded}, CIFAR100~\citep{alex2009learning}, and Caltech101~\citep{li_andreeto_ranzato_perona_2022}. For text generation, we fine-tune LLMs on 256 samples from Stanford-Alpaca~\citep{alpaca} and assess zero-shot performance on nine EleutherAI LM Harness tasks~\citep{gao2021framework}. See Appendix~\ref{apdx:data} for details.

\subsection{Models and Baselines} \label{subsec:models}

We fine-tune full-precision Llama-2-7B and Llama-3-8B (float32) using our SPruFT, LoRA~\citep{hulora}, VeRA~\citep{kopiczko2024vera}, DoRA~\citep{liu2024dora}, and RoSA~\citep{nikdan2024rosa}. RoSA is chosen as the representative SFT method and is the only SFT due to the high memory demands of other SFT approaches, while full fine-tuning is excluded for the same reason. We freeze Llama’s classification layers and fine-tune only the linear layers in attention and MLP blocks.

Next, we evaluate importance metrics by fine-tuning Llamas and image models, including DeiT~\citep{touvron2021training}, ViT~\citep{dosovitskiy2020image}, ResNet101~\citep{he2016deep}, and ResNeXt101~\citep{xie2017aggregated} on CIFAR100, Caltech101, and Tiny-ImageNet. For image tasks, we set the fine-tuning ratio at 5\%, meaning the trainable parameters are a total of 5\% of the backbone plus classification layers.

\subsection{Training Details} \label{subsec:training}
Our fine-tuning framework is built on torch-pruning\footnote{Torch-pruning is not required, all their implementations are based on PyTorch.}~\citep{fang2023depgraph}, PyTorch~\citep{paszke2019pytorch}, PyTorch-Image-Models~\citep{rw2019timm}, and HuggingFace Transformers~\citep{wolf2020transformers}. Most experiments run on a single A100-80GB GPU, while DoRA and RoSA use an H100-96GB GPU. We use the Adam optimizer~\citep{KingBa15} and fine-tune all models for a fixed number of epochs without validation-based model selection.

%Structured pruning often considers parameter dependencies in importance evaluation~\citep{liu2021group, fang2023depgraph, ma2023llmpruner}. This becomes the following process in our work: first, searching for dependencies by tracing the computation graph of gradient; next, evaluating the importance of parameter groups; and finally, fine-tuning the parameters within those important groups collectively. For instance, if $\W^{a}_{\cdot j}$ and $\W^{b}_{i\cdot}$ are dependent, where $\W^{a}_{\cdot j}$ is the $j$-th column in parameter matrix (or the $j$-th input channels/features) of layer $a$ and $\W^{b}_{i\cdot}$ is the $i$-th row in parameter matrix (or the $i$-th output channels/features) of layer $b$, then $\W^{a}_{\cdot j}$ and $\W^{b}_{i\cdot}$ will be fine-tuned simultaneously while the corresponding $\M^{a}_{dep}$ for $\W^{a}_{\cdot j}$ becomes column selection matrix and $\W^a_s$ becomes $\W^a_{f,dep}\M^a_{dep}$. Consequently, fine-tuning $2.5\%$ output channels for layer $b$ will result in fine-tuning additional $2.5\%$ input channels in each dependent layer. Therefore, for the $5\%$ of desired fine-tuning ratio, the fine-tuning ratio with considering dependencies is set to $2.5\%$\footnote{In some complex models, considering dependencies results in slightly more than twice the number of trainable parameters. However, in most cases, the factor is 2.} for the approach that includes dependencies. More details for dependencies of NN can be found in Appendix~\ref{apdx:dep}. 

\textbf{Image models}: The learning rate is set to $10^{-4}$ with cosine annealing decay~\citep{loshchilov2017sgdr}, where the minimum learning rate is $10^{-9}$. All image models used in this study are pre-trained on ImageNet. 

\textbf{Llama}: For LoRA and DoRA, we set $\alpha = 16$, a dropout rate of $0.1$, and a learning rate of $10^{-4}$  with linear decay (
$0.01$ decay rate). For SPruFT, we control trainable parameters using rank instead of fine-tuning ratio for direct comparison. The learning rate is $2 \cdot 10^{-5}$ with the same decay settings. Linear decay is applied after a warmup over the first $3$\% of training steps. The maximum sequence length is $2048$, with truncation for longer inputs and padding for shorter ones.



\subsection{Random Discretization}
\label{sec:mainargs}
We now set out to prove Lemma~\ref{lem:subgoal}. So let $0 < \delta < 1$, and fix a pair $(\Gamma_i, L_j)$. Following the proof outline in Section~\ref{sec:overview}, we now consider the following random discretization of hypotheses in $\subH(\Gamma_i, L_j)$: Let $k= k(i,j)$ be an integer parameter to be determined. Sample a random $k \times d$ matrix $\rA$ with each entry $\Norm(0,1/k)$ distributed as well as $k$ random offsets $\rt = (\rt_1,\dots,\rt_k)$ all independent and uniformly distributed in $[0,1]$.

Let $\Disc$ be the set of all vectors in $\R^k$ with coordinates in 
\[
\{(1/2)(10 \sqrt{k})^{-1} + z (10 \sqrt{k})^{-1}  \mid z \in \Ints\}.
\]
For $w \in \finalH$ and an outcome $(A,t)$ of $(\rA,\rt)$, define $h_{A,t}(w) \in \Disc$ as the vector obtained as follows: Consider each coordinate $(Aw)_i$ and let $z_i$ denote the integer such that 
\[
(1/2)(10 \sqrt{k})^{-1}  + z_i (10 \sqrt{k})^{-1} \leq (Aw)_i < (1/2)(10 \sqrt{k})^{-1}  + (z_i+1) (10\sqrt{k})^{-1}.
\]
Let $(h_{A,t}(w))_i$ equal $(1/2)(10 \sqrt{k})^{-1} + z_i (10\sqrt{k})^{-1}$ if $t_i \leq p(z_i)$ ($(Aw)_i$ rounded down) and otherwise let it equal $(1/2)(10 \sqrt{k})^{-1}  + (z_i + 1)(10\sqrt{k})^{-1}$. We choose $p(z_i) \in [0,1]$ such that 
\begin{align}
(Aw)_i &= 
p(z_i)\left(\frac{1}{2 \cdot 10 \sqrt{k}} +\frac{z_i}{10\sqrt{k}}\right) + (1-p(z_i))\left(\frac{1}{2 \cdot 10 \sqrt{k}} +\frac{z_i+1}{10\sqrt{k}}\right) \label{eq:expectround}
\end{align}
i.e.\ for fixed $A$, the expected value of the coordinates satisfy $\E_{\rt}[(h_{A,\rt}(w))_i]=(Aw)_i$. 
\begin{remark}
\label{rmk:pIsProb}
The value $p(z_i)$ satisfying~\eqref{eq:expectround} has $p(z_i) \in [0,1]$.
\end{remark}
We thus have that $p(z_i)$ is a well-defined probability. We prove Remark~\ref{rmk:pIsProb} in Appendix~\ref{sec:aux}. The random discretization has the desirable property that it approximately preserves margins/inner products as stated in the following
\begin{lemma}
\label{lem:concdiscretize}
    There is a constant $c>0$, such that for any integer $k \geq 1$, $w \in \finalH, x \in \finalX$ and any $\gamma \in (0,1]$, it holds that 
    $
    \Pr_{\rA,\rt}[|\langle h_{\rA,\rt}(w),\rA x\rangle - \langle w, x\rangle| > \gamma] < c\exp(-\gamma^2 k/c)
    $.
\end{lemma}
The proof of Lemma~\ref{lem:concdiscretize} follows the work by~\cite{AK17} in their work on lower bounds for the Johnson-Lindenstrauss transform, and has thus been deferred to Appendix~\ref{sec:aux}. We now observe that
\begin{align*}
    \Loss_{\cD}(w) &= \Loss^{\gamma_i/2}_{\rA \cD}(h_{\rA,\rt}(w)) + \Pr_{(\rx,\ry) \sim \cD}[y \langle h_{\rA,\rt}(w), \rA \rx\rangle > \gamma_i/2 \wedge \ry \langle w, \rx\rangle \leq 0] \\
    &- \Pr_{(\rx,\ry) \sim \cD}[\ry \langle h_{\rA,\rt}(w), \rA \rx\rangle \leq \gamma_i/2 \wedge \ry \langle w, \rx\rangle > 0].
\end{align*}
Similarly, we have for $\gamma \in \Gamma_i$ and any training set $S$ that
\begin{align*}
    \Loss_{S}^\gamma(w) &= \Loss^{\gamma_i/2}_{\rA S}(h_{\rA,\rt}(w)) + \Pr_{(\rx,\ry) \sim S}[\ry \langle h_{\rA,\rt}(w), \rA \rx\rangle > \gamma_i/2 \wedge \ry \langle w, \rx\rangle \leq \gamma] \\
    &- \Pr_{(\rx,\ry) \sim S}[\ry \langle h_{\rA,\rt}(w), \rA\rx\rangle \leq \gamma_i/2 \wedge \ry \langle w, \rx\rangle > \gamma].
\end{align*}
We now have for any $\gamma \in \Gamma_i$ that
\begin{align}
\sup_{w \in \subH(\Gamma_i,L_j)} \Loss_\cD(w) - \Loss^\gamma_S(w) =& \nonumber \\ 
\sup_{w \in \subH(\Gamma_i,L_j)} \big(\E_{\rA,\rt} [\Loss^{\gamma_i/2}_{\rA \cD}(h_{\rA,\rt}(w)) - \Loss^{\gamma_i/2}_{\rA S}(h_{\rA,\rt}(w))] +& \nonumber\\ \E_{\rA,\rt} [\Pr_{\cD}[\ry \langle h_{\rA,\rt}(w), \rA \rx\rangle > \gamma_i/2 \wedge \ry \langle w, \rx\rangle \leq 0] -\Pr_{S}[\ry \langle h_{\rA,\rt}(w), \rA \rx\rangle > \gamma_i/2 \wedge \ry \langle w, \rx\rangle \leq \gamma] ] +& \nonumber\\
\E_{\rA,\rt} [\Pr_{S}[\ry \langle h_{\rA,\rt}(w), \rA \rx\rangle \leq \gamma_i/2 \wedge \ry \langle w, \rx\rangle > \gamma] -\Pr_{\cD}[\ry \langle h_{\rA,\rt}(w), \rA \rx\rangle \leq \gamma_i/2 \wedge \ry \langle w, \rx\rangle > 0] ]\big).\label{eq:3terms}
\end{align}
A critical observation is that the distribution of $y \langle h_{\rA,\rt}(w),\rA x \rangle$ depends only on $y \langle w, x \rangle$. 
\begin{claim}
\label{clm:distDeterm}
For any $(x,y) \in \finalX \times \{-1,1\}$ and any $w \in \finalH$, the distribution of $y \langle h_{\rA,\rt}(w),\rA x \rangle$ is completely determined from $y \langle w, x \rangle$.
\end{claim}
We prove Claim~\ref{clm:distDeterm} in Section~\ref{sec:lip} by exploiting that the entries of $\rA$ are i.i.d.\ $\Norm(0,1/k)$ distributed and using the rotational invariance of the Gaussian distribution.

As outlined in the proof overview in Section~\ref{sec:overview}, we can now use Claim~\ref{clm:distDeterm} together with the contraction inequality of Rademacher complexity to bound several of the terms in~\eqref{eq:3terms}. Similarly to the introduction of the ramp loss in classic proofs of generalization for large-margin halfspaces, we need to introduce a continuous function upper bounding the probabilities above. With this in mind, we now define the following functions $\phi$ and $\rho$:
\[
\phi(\alpha) = \begin{cases} \Pr_{\rA, \rt}[y \langle h_{\rA,\rt}(w), \rA x\rangle > \gamma_i/2 \mid y \langle w, x \rangle = \alpha] & \text{if } -c_\gamma \leq \alpha \leq 0 \\
                      \frac{(\gamma_i-\alpha)}{\gamma_i}\Pr_{\rA, \rt}[y \langle h_{\rA,\rt}(w), \rA x\rangle > \gamma_i/2 \mid y \langle w, x \rangle = 0]                                    & \text{if } 0 < \alpha \leq \gamma_i      \\
                      0 & \text{if } \gamma_i < \alpha \leq c_\gamma
        \end{cases}
\]
\[
\rho(\alpha) = \begin{cases} \Pr_{\rA, \rt}[y \langle h_{\rA,\rt}(w), \rA x\rangle \leq \gamma_i/2 \mid y \langle w, x \rangle = \alpha] & \text{if } \gamma_i < \alpha \leq c_\gamma\\
                      \frac{\alpha}{\gamma_i}\Pr_{\rA, \rt}[y \langle h_{\rA,\rt}(w), \rA x\rangle \leq \gamma_i/2 \mid y \langle w, x \rangle = \gamma_i]                                    & \text{if } 0 < \alpha \leq \gamma_i      \\
                      0 & \text{if } -c_\gamma \leq \alpha \leq 0
        \end{cases}
\]
Here we slightly abuse notation and write $\Pr_{\rA, \rt}[y \langle h_{\rA,\rt}(w), \rA x\rangle > \gamma_i/2 \mid y \langle w, x \rangle = \alpha]$ to denote the probability $\Pr_{\rA, \rt}[y \langle h_{\rA,\rt}(w), \rA x\rangle > \gamma_i/2]$ for an arbitrary $w \in \finalH, (x,y) \in \finalX \times \{-1,1\}$ with $y \ipr{w,x}=\alpha$ and remark that this probability is the same for all such $w,x,y$ by Claim~\ref{clm:distDeterm}.

We now observe that $\phi$ and $\rho$ upper and lower bounds the terms in~\eqref{eq:3terms}
\begin{remark}
\label{rmk:phirho}
For any training set $S$ and distribution $\cD$ over $\finalX \times \{-1,1\}$, we have
\begin{align*}
    \E_{\rA,\rt} [\Pr_{(\rx,\ry) \sim \cD}[\ry \langle h_{\rA,\rt}(w), \rA \rx\rangle > \gamma_i/2 \wedge \ry \langle w, \rx\rangle \leq 0]] &\leq\E_{(\rx,\ry) \sim \cD}[\phi(\ry \langle w, \rx \rangle)] \\
    \E_{\rA,\rt} [\Pr_{(\rx,\ry) \sim S}[\ry \langle h_{\rA,\rt}(w), \rA \rx\rangle > \gamma_i/2 \wedge \ry \langle w, \rx\rangle \leq \gamma]] &\geq \E_{(\rx,\ry) \sim S}[\phi(\ry \langle w, \rx \rangle)] \\
    \E_{\rA,\rt} [\Pr_{(\rx,\ry) \sim S}[\ry \langle h_{\rA,\rt}(w), \rA \rx\rangle \leq \gamma_i/2 \wedge \ry \langle w, \rx\rangle > \gamma]] &\leq \E_{(\rx, \ry) \sim S}[\rho(\ry \langle w, \rx \rangle)]\\
    \E_{\rA,\rt} [\Pr_{(\rx,\ry) \sim \cD}[\ry \langle h_{\rA,\rt}(w), \rA \rx\rangle \leq \gamma_i/2 \wedge \ry \langle w, \rx\rangle > 0]] &\geq \E_{(\rx, \ry) \sim \cD}[\rho(\ry\langle w, \rx \rangle)].
\end{align*}
\end{remark}
The proof of Remark~\ref{rmk:phirho} follows from the definition of $\phi$ and $\rho$, along with monotonicity of $\Pr_{\rA,\rt}[y \ipr{h_{\rA,\rt}(w),\rA x} > \gamma_i \mid y\ipr{w,x}=\alpha]$ as a function of $\alpha$. The proofs have been deferred to Appendix~\ref{sec:aux}.
Continuing from~\eqref{eq:3terms} using Remark~\ref{rmk:phirho}, linearity of expectation and the triangle inequality, we have for any $\gamma \in \Gamma_i$ that
\begin{align}
\sup_{w \in \subH(\Gamma_i, L_j)} \Loss_\cD(w) - \Loss^\gamma_S(w) \leq&
\sup_{w \in \subH(\Gamma_i, L_j)} \left|\E_{\rA,\rt} [\Loss^{\gamma_i/2}_\cD(h_{\rA,\rt}(w)) - \Loss^{\gamma_i/2}_S(h_{\rA,\rt}(w))]\right| \label{eq:middle}\\
+&\sup_{w \in \subH(\Gamma_i, L_j)} \left| \E_{(\rx,\ry) \sim \cD}[\phi(\ry \langle w, \rx \rangle)] -\E_{(\rx,\ry) \sim S}[\phi(\ry \langle w, \rx \rangle)]\right| \label{eq:phi}\\
+&\sup_{w \in \subH(\Gamma_i, L_j)} \left|\E_{(\rx,\ry) \sim \cD}[\rho(\ry \langle w, \rx \rangle)] -\E_{(\rx,\ry) \sim S}[\rho(\ry \langle w, \rx \rangle)] \right|.\label{eq:rho}
\end{align}
In Section~\ref{sec:meetinmid}, we carefully use Bernstein's plus a (highly non-trivial) union bound over infinitely many grids of increasing size to bound~\eqref{eq:middle} as follows
\begin{lemma}
\label{lem:supW}
There is a constant $c>0$ such that with probability at least $1-\delta$ over $\rS \sim \cD^n$ we have
\begin{align*}
    \eqref{eq:middle} &\leq c \left(\sqrt{\frac{(\ell_{j+1}+ \exp(-\gamma_{i+1}^2k/c)) (k + \ln(e/\delta))}{n}} + \frac{(k + \ln(e/\delta))}{n} \right).
\end{align*}
\end{lemma}
In Section~\ref{sec:rademacher}, we then use Rademacher complexity and a bound on the Lipschitz constants of $\phi$ and $\rho$ to bound~\eqref{eq:phi} and~\eqref{eq:rho} as follows
\begin{lemma}
\label{lem:supPhi}
There are constants $c,c'>0$ such that with probability at least $1-\delta$ over $\rS \sim \cD^n$ we have
\begin{align*}
\max\{\eqref{eq:phi}, \eqref{eq:rho} \}\leq
c\exp(-\gamma_{i+1}^2k/c) \cdot \sqrt{(k + \gamma_{i+1}^{-2} + \ln(e/\delta))/n} .
\end{align*}
provided that $k \geq c' \gamma_{i+1}^{-2}$. 
\end{lemma}
To balance the expressions in Lemma~\ref{lem:supW} and Lemma~\ref{lem:supPhi}, we now set $k = c \gamma_{i+1}^{-2} \ln(e/\ell_{j+1})$ for a sufficiently large constant $c>0$ so that $\exp(-\gamma_{i+1}^2 k/c) \leq \ell_{j+1}/e$ and $k \geq c'\gamma_{i+1}^{-2}$. Combining Lemma~\ref{lem:supW} and Lemma~\ref{lem:supPhi} via a union bound with $\delta'=\delta/2$ and inserting into~\eqref{eq:middle},~\eqref{eq:phi} and~\eqref{eq:rho} gives
\begin{align*}
\sup_{w \in \subH(\Gamma_i, L_j)} \Loss_\cD(w) - \Loss^\gamma_S(w) \leq& \\c\left(\sqrt{\frac{\ell_{j+1}(\gamma_{i+1}^{-2}\ln(e/\ell_{j+1}) + \ln(e/\delta))}{n}}  + \frac{\gamma_{i+1}^{-2} \ln(e/\ell_{j+1})+\ln(e/\delta)}{n}\right) &+\\
c \left( \ell_{j+1} \sqrt{(\gamma_{i+1}^{-2} \ln(e/\ell_{j+1}) + \ln(e/\delta))/n} \right),
\end{align*}
for a constant $c>0$. This completes the proof of Lemma~\ref{lem:subgoal}, which together with Lemma~\ref{lem:subgoal2} completes the proof of our main result, Theorem~\ref{thm:main}.


%% !TEX root = main.tex

%\section{Proof of Lemma~\ref{lem:2vass-sandwich}: \dvass are \sandwich}\label{sec:sandwich}

\begin{appendixproof}[Proof of Lemma~\ref{lem:2vass-sandwich}]
We extend the definition of nondecreasing functions to many-argument ones:
a function $f: \N^k \to \N$ is \emph{nondecreasing} if it is monotonic
in every argument and $f(n_1, \ldots, n_k) \geq n_1 + \ldots + n_k$.
In the sequel we often bound certain quantities polynomially, but an exact polynomial is irrelevant. 
We thus say that a value $n$ is \emph{polynomially bounded} in $n_1, \ldots, n_k$
if there exists a nondecreasing polynomial $P: \N^k \to \N$ such that $n \leq P(n_1, \ldots, n_k)$
for all $n_1, \ldots, n_k \in \N$. 
We also write $n \leq \poly(n_1, \ldots, n_k)$.

\para{Linear path schemes}
A $d$-dimensional \emph{linear path scheme} ($d$-\LPS in short, or \LPS if dimension is irrelevant) 
is a sequential \vass where every component is either trivial (a singleton) or a simple cycle,
i.e., a \vass whose control graph is a simple path with disjoint simple cycles attached to some states of the path.
We write down \LPS in the following form 
\[
%\Lambda \ = \ 
\alpha_0\beta_1^* \alpha_1 \cdots \alpha_{k-1} \beta_{k}^* \alpha_k,
\]
where each $\alpha_i$ and $\beta_i$ is  a fixed sequence of transitions. 
Thus the cycles $\beta_i$ of an \lps may be repeated arbitrarily many times (possibly zero). 
An \LPS is \emph{simple} (\SLPS) when all $\beta_i$ are single transitions, i.e.,
each component is either trivial or a single self-loop transition.
When considering the reachability relation in a $d$-\LPS, we often implicitly take the first and the last state
of the \LPS as the source and target state, respectively, and consider the reachability $w\tran w'$ between
vectors $w,w'\in\N^d$ only.
% for two vectors $s, t \in \N^d$, denoted $s \trans{\Lambda} t$, instead of reachability from one configuration to another. This is just a shortening of notation meaning that we consider reachability from $p(s)$ to $q(t)$ where $p$ is the first state of $\Lambda$ (before $\alpha_1)$ and $q$ is the last state (after $\alpha_n$) of $\Lambda$.

%\begin{lemma}\label{lem:two-slps}
%There is a nondecreasing polynomial $P$ such that for every \dvass $V$ together with its two states $p$ and $q$,
%such that $\size(V) = M$ there is a finite set $\Gamma$ of \dslps, each of size at most $P(M)$,
%such $p(u) \tran q(v)$ in $V$ if and only if $u \trans{\Lambda} v$ for some \dslps $\Lambda \in \Gamma$.
%\end{lemma}

%\wojtek{Maybe remove Lemma~\ref{lem:two-slps} and move its proof directly to the proof of Lemma~\ref{lem:2vass-sandwich}.}

\begin{lemma}\label{lem:two-slps}
For every \dvass $V$ and two its states $q$, $q'$,
there is a finite set $\Gamma$ of \dslps of size polynomially bounded in $\size(V)$,
such that for all $w,w'\in\N^2$,
\begin{align*} %\label{eq:slps}
 \ \ q(w) \tran q'(w') \text{ in } V \ \iff \ w \tran w' \text{ in some  \dslps in } \Gamma.
\end{align*}
\end{lemma}

\begin{proof}
The claim follows by combination of
Theorem 3.1 from~\cite{DBLP:journals/jacm/BlondinEFGHLMT21}, due to which we get
a finite set $\Gamma$ of \dlps of size polynomially bounded in $\size(V)$ that satisfies the claim,
with Lemma 5.2 from~\cite{DBLP:journals/jacm/BlondinEFGHLMT21} 
(or Theorem 15 from~\cite{DBLP:conf/lics/EnglertLT16}), due to which
we can transform every \dlps $\lambda$ into an \dslps of size polynomially bounded in $\size(\Lambda)$.
%We get that the size of the obtained \dslps are polynomially bounded in the size of the input \dlps.
%Therefore, the resulting \dslps have size at most polynomial also wrt. the size of the considered \dvass, as required.
%Finally, notice that one can upper-bound any polynomial by a nondecreasing polynomial.
%Indeed, it is sufficient to take absolute values of the coefficients, and possibly add the $P(x) = x$, if the starting polynomial is a constant.
\end{proof}

%Here is the previous formulation of the main sandwich lemma, probably not useful  now.
%
%\begin{lemma}\label{lem:two-sandwich}
%There exists a polynomial $R$ such that for any $2$-VASS of size $M$,
%any configuration $s$, any state $q$ and any numbers $N, K, N_t \in \N$
%the set $\reach_q(s)$ is a finite union of:
%\begin{itemize}
%  \item linear sets $a_i + Q_i^* \subseteq \N^2$ such that $\norm(a_i) \leq B_\bg$, $\norm(Q_i) \leq R(M)$, and
%  \item sandwich sets $S_i \subseteq \N^2$ such that there are $b_i \in \N^2$, $P_i \subseteq_{\fin} \N^2$ with $\norm(b_i) \leq B_\sml$,
%  $\norm(P_i) \leq R(M)$ and $b_i + P_i^{\leq B_\bg} \subseteq S_i \subseteq b_i + P_i^*$,
%\end{itemize}
%where $N_s = \norm(s)$, $B_\sml = (N_s + N_t) \cdot (2N \cdot R(M))^K$ and $B_\bg = (N_s + N_t) \cdot (2 N \cdot R(M))^{2K}$. 
%\end{lemma}
%
%\vskip 5cm

%The following lemma, together with Lemma \ref{lem:two-slps}, immediately imply Lemma~\ref{lem:2vass-sandwich}:

%\begin{lemma}\label{lem:slps-sandwich}
%There exists a polynomial $R$ such that for any $2$-SLPS with of norm $N$ and with $n$ states,
%and any vector $s \in \N^2$ with $N_s = \norm(s)$ and any numbers $K, N_t, N_R \in \N$
%the set $\reach(s)$ is a finite union of:
%\begin{itemize}
%  \item linear sets $a_i + Q_i^* \subseteq \N^2$ such that $\norm(a_i) \leq 2R(N) \cdot B_\bg$, $\norm(Q_i) \leq R(N)$, and
%  \item sandwich sets $S_i \subseteq \N^2$ such that there are $b_i \in \N^2$, $P_i \subseteq_{\fin} \N^2$ with $\norm(b_i) \leq B_\sml$,
%  $\norm(P_i) \leq R(N)$ and $b_i + P_i^{\leq B_\bg} \subseteq S_i \subseteq b_i + P_i^*$,
%\end{itemize}
%where
%%$B_\sml = (N_s + N_t + nN) \cdot (2 N_R \cdot R(N))^K$ and $B_\bg = (N_s + N_t + nN) \cdot (2 N_R \cdot R(N))^{4K}$. 
%$B_\sml = (N_s + N_t + n \cdot R(N))$ and $B_\bg = (N_s + N_t + nN) \cdot (2 (N_R + R(N)))^{K+1}$. 
%\end{lemma}

%\begin{lemma}\label{lem:slps-sandwich}
%There exists a polynomial $Q$ such that for every $2$-SLPS $\Lambda$,
%any vector $s \in \N^2$ such that $M = \max(\size(\Lambda), \norm(s))$ and any number $B \in \N$
%the set $\reach(s)$ is a finite union of:
%\begin{itemize}
%  \item linear sets $a_i + Q_i^* \subseteq \N^2$ such that $\norm(a_i) \leq B \cdot Q(M)$, $\norm(Q_i) \leq Q(M)$, and
%  \item sandwich sets $S_i \subseteq \N^2$ such that there are $b_i \in \N^2$, $P_i \subseteq_{\fin} \N^2$ with $\norm(b_i) \leq Q(M)$,
%  $\norm(P_i) \leq Q(M)$ and $b_i + P_i^{\leq B} \subseteq S_i \subseteq b_i + P_i^*$.
%\end{itemize}
%\end{lemma}

\begin{lemma}\label{lem:slps-sandwich}
\dslps are \sandwich.
\end{lemma}

Before proving Lemma~\ref{lem:slps-sandwich}, we use it together with Lemma~\ref{lem:two-slps}
to prove Lemma~\ref{lem:2vass-sandwich}.
%
To this aim consider a \dvass $(V, s)$ where $s=p(w)$, and its state $q$, and let $M=\size(V,s)$.
%$M = \max(\size(V), \norm(s))$.
By Lemma~\ref{lem:two-slps} we get a finite set $\Gamma$ of \dslps such that:
\[
\reach_q(V,s) \ = \ \bigcup_{\Lambda \in \Gamma} \reach(\Lambda, w).
\]
Moreover, for some nondecreasing polynomial $F$, every $\Lambda \in \Gamma$ satisfies
$\size(\Lambda) \leq F(M)$.
By Lemma~\ref{lem:slps-sandwich}, there is a nondecreasing polynomial $G$
such that for every $\Lambda \in \Gamma$ and $B\in\N$, 
the set $\reach(\Lambda,w)$
is 
\kanapka {$G(\size(\Lambda,w))$} {$B$}.
Combining the last two statements, we deduce that $\reach_q(V, s)$ is
\kanapka {$G(F(M))$} {$B$} for every $B\in\N$, as required.
%
Lemma~\ref{lem:2vass-sandwich} is thus proved (once we prove
Lemma~\ref{lem:slps-sandwich}).
\end{appendixproof}

\begin{appendixproof}[Proof of Lemma~\ref{lem:slps-sandwich}]
%
We generalise finite prefixes $P^{\leq B}$ of $P^*$ as follows.
Let $P = \set{p_1, \ldots, p_m}$. 
For a positive vector $c \in (\Npos)^m$ and $T \in \N$ we define
\[
P^{x \cdot c \leq T} := \setof{\Sigma_{i=1}^m n_i \cdot p_i}{\Sigma_{i=1}^m n_i \cdot c_i \leq T}.
\]
In particular, $P^{\leq T} = P^{x\cdot \vec 1  \leq T}$.
In the sequel, sets of the form
\begin{align} \label{eq:hybrid}
a + P^{x \cdot c \leq T} + Q^*,
\end{align}
for $a\in\N^2$, $c \in (\Npos)^{\card P}$ and $P,Q\subseteqfin \N^2$, 
we call \emph{hybrid} sets.

%\begin{lemma}\label{lem:sandwich-raw}
%There exists a nondecreasing polynomial $R$ such that for every $2$-SLPS $\Lambda$,
%any vector $s \in \N^2$ and $M = \max(\size(\Lambda), \norm(s))$ the set of vectors reachable by $\Lambda$ from $s$ is a finite union of sets $S$ of the following form.
%For each $S$ there exists a base vector $b \in \N^2$, sets of periods vectors $P, Q \subseteq_\fin \N^2$, vector $c \in \N^{|P|}$ and threshold $T \in \N$
%such that
%\[
%S = b + P^{c \cdot x \leq T} + Q^*,
%\]
%where $\norm(b), \norm(P \cup Q), \norm(c) \leq R(M)$ and $c_i \geq 1$ for each $i \in [1, |P|]$.
%\end{lemma}

\begin{lemma}\label{lem:sandwich-raw}
For every \dslps $(\Lambda,s)$,
% any vector $s \in \N^2$ and $M = \size(\Lambda, s)$
the set $\reach(\Lambda,s)$ is a finite union of hybrid sets \eqref{eq:hybrid}, 
% of vectors reachable by $\Lambda$ from $s$ is a finite union of sets $S$ of the following form.
%For each $S$ there exists a base vector $b \in \N^2$, sets of periods vectors $P, Q \subseteq_\fin \N^2$, vector $c \in \N^{|P|}$ and threshold $T \in \N$ such that
%\[
%S = b + P^{c \cdot x \leq T} + Q^*,
%\]
where $\norm(a),  \norm(c), \norm(P \cup Q)$ are bounded polynomially in $\size(\Lambda,s)$.
% and $c_i \geq 1$ for each $i \in [1, |P|]$.
\end{lemma}

Before proving Lemma~\ref{lem:sandwich-raw} we use it to prove Lemma~\ref{lem:slps-sandwich}.
We need to argue that there is a nondecreasing polynomial $F$ such that 
for every \dslps $(\Lambda, s)$ and $B\in\N$, the set $\reach(\Lambda,s)$ is
\kanapka {$F(M)$} {$B$}, where $M=\size(\Lambda,s)$.
%
%Recall that according to the definition of polynomial approximability
%we need to prove that for each $B \in \N$ there is a polynomial $F$ such that for $M = \size(\Lambda, s)$
%the set of points reachable from $s$ via $\Lambda$ is a finite union of:
%\begin{itemize}
%  \item linear sets $S = a + Q^* \subseteq \N^2$ such that $\norm(a) \leq B \cdot F(M)$, $\norm(Q) \leq F(M)$, and
%  \item $B$-approximations of linear sets, namely $S \subseteq \N^2$ such that there are $a \in \N^2$, $Q \subseteq_{\fin} \N^2$ with $\norm(a), \norm(Q) \leq F(M)$, and $a + Q^{\leq B} \subseteq S \subseteq a + Q^*$.
%\end{itemize}
%By Lemma~\ref{lem:sandwich-raw} we have that the considered set is a finite union of sets
%$S$ of the form
%\[
%S = b + P^{c \cdot x \leq T} + Q^*
%\]
%with $\norm(b), \norm(P \cup Q), \norm(c) \leq R(M)$ for some nondecreasing polynomial $R$
%and $c_i \geq 1$ for each $i \in [1, |P|]$.
We fix the nondecreasing polynomial $F(x) = R(x) + R^2(x)$, where $R$ is a polynomial witnessing 
Lemma~\ref{lem:sandwich-raw}, and some arbitrary $B \in \N$,
and prove that each hybrid set $H$ \eqref{eq:hybrid} 
of Lemma~\ref{lem:sandwich-raw} is
\kanapka {$F(M)$} {$B$}.
% 
%It is enough to check each set $S$ separately, namely to check that for each $B \in \N$
%the set $S$ is $(F(M), B)$-approximately semi-linear.
%Fix $B \in \N$.
% and consider the set $b + P^{c \cdot x \leq T} + Q^*$ for some $T \in \N$.
%
We distinguish two cases.
If $T \geq R(M) \cdot B$ then, since $\norm(c)\leq R(M)$, we have
%because each entry of $c$ is at most $R(M)$, we have that
\[
a + (P \cup Q)^{\leq B} \ \subseteq \ H \ \subseteq \ a + (P \cup Q)^*,
\]
and $\norm(a), \norm(P \cup Q) \leq R(M) \leq F(M)$, as required. 
%
On the other hand, if  $T < R(M) \cdot B$ then $H$ is a finite union
of linear sets of the form $u + Q^*$ for $u \in b + P^{c \cdot x \leq T}$, where
\begin{align*}
\norm(u) \ \leq \ \norm(a) + \norm(P) \cdot T 
\ \leq \ R(M) + R(M)^2 \cdot B \leq  F(M) \cdot B,
\end{align*}
as required. 
As before,  $\norm(Q) \leq R(M) \leq F(M)$.
This completes the proof of Lemma~\ref{lem:slps-sandwich} (once
we prove Lemma~\ref{lem:sandwich-raw}). 
\end{appendixproof}
%\begin{align*}
%\norm(u) & \leq \norm(b) + \norm(P) \cdot T \\
%& \leq \norm(s) + n \cdot R(N) + R(N) \cdot B 
%& \leq B + B + R(N) \cdot B = B \cdot (R(N) + 2)  \\
%& \leq B \cdot 2 R'(N)
%\end{align*}
%for $R'(x) = R(x) + 2$. The second to last inequality holds as clearly $\norm(s) \leq B$ and $n \cdot R(N) \leq B$.
%Thus we get Lemma~\ref{lem:slps-sandwich} for the polynomial $R'$.


\begin{appendixproof}[Proof of Lemma~\ref{lem:sandwich-raw}]
The proof occupies the rest of this section.
%The rest of this subsection is devoted to the proof of Lemma~\ref{lem:sandwich-raw}.
We rely on an insightful characterisation of paths of \slps \cite[Theorem 4.16]{DBLP:conf/focs/0001CMOSW24},
which we state below using a slightly different terminology.
%We formulate an easy consequence of this theorem here, in a slightly different and simpler phrasing.
%Before we need to define a few notions.
%
Speaking informally, a \emph{detailing} of an \slps 
$\Lambda = \alpha_0 \beta_1^* \alpha_1 \ldots \alpha_{k-1} \beta_k^* \alpha_k$
is any \slps obtained by fixing exponents of some of the cycles of $\Lambda$.
Formally, a detailing of $\Lambda$ is any $\Lambda'$ obtained by choosing a subset $S \in [1,k]$ and,
for all $i \in S$, by replacing the cycle $\beta_i$ by a path $\beta_i^{n_i}$, for some $n_i\in\N$, 
which becomes an infix of the simple path of $\Lambda'$.
The number of cycles of $\Lambda'$ is thus $k-\card{S}$.
%
An \dslps is \emph{zigzagging} if the effect of its every cycle $\beta_i$ 
belongs either to the quadrant $\Npos \times (-\Npos)$,
or to the quadrant $(-\Npos) \times \Npos$, and additionally effects of every two consecutive cycles belong 
to different quadrants
(the effects of cycles $\beta_1, \ldots, \beta_k$ thus alternate between quadrants), and the effect of the first cycle belongs to $\Npos \times (-\Npos)$ and the effect of the last cycle belongs to $(-\Npos) \times \Npos$.
Finally, an \slps is \emph{short} if it contains at most three cycles, $k\leq 3$.
For $B\in\N$,
a path 
\[
s_0 \trans{\alpha_0} t_0 \trans{\sigma_1} s_1 \trans{\alpha_1} t_1 \ \cdots \ s_{k-1} \trans{\alpha_{k-1}} t_{k-1} \trans{\sigma_{k}} s_k \trans{\alpha_k} t_k
\]
of an \dslps,
where $\sigma_i \in\beta^*$ for $i\in\setto k$, is called \emph{$B$-close} if all the vectors $x\in \{s_0, t_0, s_1, t_1, \ldots, s_k, t_k\}$,
called \emph{midpoints} below,
are \emph{$B$-close} to some axis, namely either $x\in \setfromto 0 B \times \N$ or 
$x\in \N\times\setfromto 0 B$.

\begin{theorem}[Thm 4.16 in~\cite{DBLP:conf/focs/0001CMOSW24}]\label{thm:2slps-zigzag}
For every \dslps $\Lambda$ there is $B\leq \poly(\size(\Lambda))$
such that for every path $s \tran t$ in $\Lambda$ 
there is a detailing $\Lambda' = \Lambda_1 \Lambda_2 \Lambda_3$ of $\Lambda$  
of $\size(\Lambda')\leq \poly(\size(\Lambda))$ and $u, u' \in \setfromto 0 B \times \N$ such that
\begin{enumerate}
%  \item $\size(\Lambda') \leq \poly(M)$,
 %$P(\size(\Lambda))$,
  \item $\Lambda_1$ and $\Lambda_3$ are short, 
  \item $\Lambda_2$ is zigzagging,
  \item there are paths $s \tran u$ in $\Lambda_1$, 
  a $B$-close path $u \tran u'$ in $\Lambda_2$, and 
  $u'\tran t$ in $\Lambda_3$.
\end{enumerate}
%For any $2$-SLPS $\Lambda$, vectors $s, t \in \N^2$ such that $s \trans{\Lambda} t$ and $M = \max(\norm(s), \size(\Lambda))$
%there exists a bound $B = \poly(M)$ and a detailing $\Lambda'$ of $\Lambda$ such that $\Lambda' = \Lambda_1 \Lambda_2 \Lambda_3$
%with the following properties
%\begin{itemize}
%  \item $\size(\Lambda') \leq \poly(M)$,
% %$P(\size(\Lambda))$,
%  \item $\Lambda_1$ and $\Lambda_3$ are short, while $\Lambda_2$ is zigzagging
%  \item there are $u_1, u_2 \in [0,B] \times \N$ such that $s \trans{\Lambda_1} u_1 \trans{\pi} u_2 \trans{\Lambda_3} t$
%  for some $B$-close $\Lambda_2$-path $\pi$.
%\end{itemize}
\end{theorem}

%Due to Theorem~\ref{thm:2slps-zigzag}, in the proof of 
%Lemma~\ref{lem:sandwich-raw} we may safely restrict to paths of the above form. 
Fix in the sequel $B$ given by Theorem~\ref{thm:2slps-zigzag}.
There are only finitely many detailings $\Lambda'$ of $\Lambda$ of a bounded size,
only finitely many possible decompositions of $\Lambda'$ into $\Lambda_1$, $\Lambda_2$ and $\Lambda_3$,
and only finitely many values of $u_1, u'_1\in\setfromto 0 B$.
By Theorem~\ref{thm:2slps-zigzag}, vectors $t$ reachable from $s$ in $\Lambda$ are exactly those
reachable from $s$ in some of detailing $\Lambda'$.
Therefore it is enough
to show Lemma~\ref{lem:sandwich-raw} for the set of vectors $t\in\N^2$ reachable by paths as in point 3 above,
in a fixed \slps $(\Lambda',s)$, where $\Lambda' = \Lambda_1 \Lambda_2 \Lambda_3$ satisfies points 1, 2 above,
and where $u_1 = b$ and $u'_1 = b'$ for some fixed $b,b'\in\setfromto 0 B$.

In a path $u\tran u'$, every second midpoint is $B$-close to one axis,
say $x\in \setfromto 0 B \times \N$,
while the remaining midpoints are $B$-close to the other one.
We relax this requirement slightly, by dropping the latter condition.
A path $u \tran u'$  in the zigzagging \slps $\Lambda_2$
is \emph{$B$-vertical-close} if $u$, $u'$ and every second midpoint $x$
are $B$-close to the vertical axis, namely $x \in \setfromto 0 B \times \N$
(thus the remaining endpoints on the path do not have to be $B$-close to the horizontal axis).
In order to have more flexibility in the proof of Lemma~\ref{lem:sandwich-raw},
in the sequel we consider those paths in $\Lambda'$, as in point 3 in Theorem~\ref{thm:2slps-zigzag}, where the infix
$u\tran u'$ is $B$-vertical-close but not necessarily $B$-close.
Notice that by relaxing the condition to $B$-vertical-closeness
we enlarge the set of considered paths, but do not enlarge the set of reachable points,
as every $t$ such that $s \tran t$ is already reachable by the paths with the middle path being even $B$-close.
We denote by $\reach(\Lambda', s)$ the set of all vectors $t\in\N^2$ reachable by such a path $s\tran t$ with
the middle part being $B$-vertical-close.

We formulate below Claims \ref{cl:prefix}, \ref{cl:infix} and \ref{cl:suffix}
(taking care of a prefix, infix, and suffix, respectively, of a path $s\tran t$), 
show how they imply Lemma~\ref{lem:sandwich-raw}, and finally proceed with the proofs of the three claims.
To this aim we introduce some more notation.
Given a start $a \in \N$, a difference $r \in \N$, and a bound $T \in \N_\infty = \N\cup\{\infty\}$,
the set $a + \{r\}^{\leq T}$ is called \emph{$(a, r, T)$-arithmetic}. 
We omit brackets and write $a + r^{\leq T}$.
In particular, $a + r^{\leq 0} = \{a\}$.
%The first claim essentially says that it is enough to consider only a finite number
%of arithmetic sequences for vectors reachable from $s$ by a short $2$-SLPS.
A \dslps  $\alpha_1 \beta_1^* \alpha_2 \beta_2^*$ is \emph{one-turn} if 
$\eff(\beta_1) \in \Npos  \times -\Npos$ and $\eff(\beta_2) \in -\Npos  \times \Npos$
(it is thus a special case of short zigzagging \dslps).
%
For a set $S\subseteq \N^2$, we use the notation $\reach(\Lambda, S) = \bigcup_{s\in S} \reach(\Lambda, s)$.

In Claims \ref{cl:prefix}, \ref{cl:infix} and \ref{cl:suffix}, we focus on source/target vectors
in $\setfromto 0 B \times \N$ and, intuitively speaking, on arithmetic subsets of each 'line' $\{b\}\times \N$.
First, Claim \ref{cl:prefix} states that the reachability set of a short \dslps, intersected with each line,
is a finite union of arithmetic sets.
Second, Claim \ref{cl:infix} states that the reachability set of a one-turn \dslps from 
an arithmetic set inside a line, intersected with another line, is a finite union of arithmetic sets.
Importantly, the starting point grows only additively, by a polynomially bounded amount, 
as we will apply Claim \ref{cl:infix} $\OO(k)$ times.
Finally, Claim \ref{cl:suffix} states that the reachability set of a short \dslps, from an arithmetic set
inside a line, is a finite union of hybrid sets.
All quantities in the claims are bounded polynomially.

\begin{claim}\label{cl:prefix}
For every short \dslps $(\Lambda,s)$ 
%$M = \size(\Lambda, s)$  
and $u_1 \in [0,B]$,
the set $\{u_2 \mid (u_1, u_2) \in \reach(\Lambda,s)\}$ is a finite union of $(a, r, T)$-arithmetic sets, where
$a \leq \poly(B,M)$, $r \leq \poly(M)$, and $M=\size(\Lambda,s)$.
\end{claim}
%
\begin{claim}\label{cl:infix}
Let $\Lambda$ be a one-turn \dslps,
and $S_1 = a + r^{\leq K}$ for some $a, r, K \in \N$.
Let $u_1, v_1 \in [0, B]$.
The set $R(S_1) := \{v_2 \mid \exists_{u_2 \in S_1} (u_1, u_2) \tran (v_1, v_2) \text{ in } \Lambda\}$
is a finite union of $(a', r', T')$-arithmetic sets with
$a' \leq a + \poly(B,M,r)$ and $r' \leq \max(\poly(M),r)$, where $M=\size(\Lambda)$.
\end{claim}

%
%The last claim takes care about the short suffix of the $2$-SLPS.
%
%\begin{claim}\label{cl:suffix}
%There is a polynomial $R$ such that for every short $2$-SLPS $\Lambda$ such that $M=\size(\Lambda)$, $S_1 = b_1 + \set{p_1}^{\leq T}$
%and $N = \norm(p_1)$ the set $S_2 = \reach_{\Lambda}(S_1)$
%is a finite union of sets $b_2 + P_2^{a \cdot x \leq T} + Q_2^*$ for $\norm(P_2 \cup Q_2), \norm(a) \leq R(N) \cdot R(M)$, $\norm(b_2) \leq (\norm(b_1)+1) \cdot R(N) \cdot R(M)$ and additionally $a_i \geq 1$ for each $i$. 
%\end{claim}
%
\begin{claim}\label{cl:suffix}
For every short \dslps $\Lambda$ and $u= (u_1, u_2), p = (p_1, p_2)\in\N^2$, 
the set $\reach(\Lambda, u + \set{p}^{\leq T})$
is a finite union of hybrid sets \eqref{eq:hybrid},
where $\norm(a), \norm(c), \norm(P), \norm(Q) \leq \poly(\size(\Lambda), \norm(u),\norm(p))$.
\end{claim}

We use the three  claims to derive Lemma~\ref{lem:sandwich-raw}.
As said above,
we consider a fixed \slps $\Lambda' = \Lambda_1 \Lambda_2 \Lambda_3$ and source $s\in\N^2$,
and focus on the set $\reach(\Lambda', s)$ of vectors $t\in\N^2$ such that
there are paths 
%
\begin{align} \label{eq:paths}
\text{$s \tran u$ in $\Lambda_1$,  \ \  
a $B$-vertically close path $u \tran u'$ in $\Lambda_2$, \ \ 
and   $u'\tran t$ in $\Lambda_3$,}
\end{align}
%
for some $u,u'\in \setfromto 0 B \times \N$, where 
$u_1 = b$ and $u'_1 = b'$ are fixed.
Let $M':=\size(\Lambda',s) \leq \poly(M)$.
%
%we aim at characterising the set of vectors $t \in \N^2$ such that
%there is $B \leq \poly(M)$, $u_1, u_2 \in [0, B] \times \N$ and paths $s \trans{\Lambda_1} u_1 \trans{\pi} u_2 \trans{\Lambda_3} t$
%where $\pi$ is a $B$-vertical-close $\Lambda_2$-path. Let us fix some $x_1, x_2 \in [0,B]$ such that $u_1[1] = x_1$ and $u_2[1] = x_2$.
%
%Recall that $M = \size(\Lambda, s)$ and by Theorem~\ref{thm:2slps-zigzag} we have that $\size(\Lambda') \leq \poly(M)$.
First, by Claim~\ref{cl:prefix}, the set
$\{u_2 \mid (u_1, u_2) \in \reach(\Lambda_1,s)\}$ is a finite union of $(a, r, T)$-arithmetic sets,
where $a, r$ are bounded polynomially in $M'$ and $B$.
%
Second,
a path $u\tran u'$ is a concatenation of $\ell\leq \size(\Lambda_2) \leq \poly(M')$
paths of one-turn \slps. 
By $\ell$-fold application of Claim~\ref{cl:infix}, the set
$\{u'_2 \mid (u'_1, u'_2) \in \reach(\Lambda_1 \Lambda_2,s)\}$
is a finite union of arithmetic sets
$a' + (r')^{\leq T'}$, where
%\begin{equation}\label{eq:rprim}
$a', r' \leq \poly(\size(\Lambda_2)) \leq \poly(M')$.
%\end{equation}
Indeed, the bound on $a'$ comes from $\ell$-fold addition of values bounded by 
$\poly(B, \poly(M'), r)\leq \poly(B, \poly(M'), \poly(M'))$,
itself bounded by $\poly(M')$:
%
\begin{equation}\label{eq:aprim}
a' \leq a + \ell \cdot \poly(M') \leq a + \poly(M') \leq \poly(M').
\end{equation}
%
Finally, by Claim~\ref{cl:suffix} the set $\reach(\Lambda',s)$ is a finite union of hybrid sets \eqref{eq:hybrid},
where
$
\norm(a)$, $\norm(c)$, $\norm(P \cup Q) \leq \poly(M', B) \leq \poly(M).
$ 
%By equations~\eqref{eq:rprim}~and~\eqref{eq:aprim} we have $N = \max(a', r') \leq \poly(M)$.
%Therefore
%\[
%\norm(c), \norm(b), \norm(P \cup Q) \leq \poly(M),
%\]
We conclude the proof of Lemma~\ref{lem:sandwich-raw}, keeping in mind that
it still remains to demonstrate Claims~\ref{cl:prefix},~\ref{cl:infix},~and~\ref{cl:suffix}.

\medskip

Here is  a corollary of Lemma~\ref{lem:taming}, useful in the proofs of the three claims:
%
\begin{lemma}\label{lem:solutions}
Consider a system $A\cdot x = b$ of $m$ Diophantine linear equations with $n$ unknowns,
where absolute values of coefficients are bounded by $N$.
Then, the set of solutions is of a form $U+P^*$, where
$\norm(U \cup P) \leq \poly(nN)^{\poly(n,m)}$.
\end{lemma}

\begin{proof}
The solution set is of the form $U+P^*$,
where $U$ is the set of pointwise minimal nonnegative integer solutions of the system,
and $P$ is the set of pointwise minimal nonnegative integer solutions of its homogeneous version
$A\cdot x = \vec 0$.
By Lemma~\ref{lem:taming} each element of $U \cup P$ has norm at most $M= \Oo(nN)^m$.
Therefore, the number of different solutions is at most $(M+1)^n$, which implies 
$\norm(U \cup P) \leq \poly(nN)^{\poly(n,m)}$.
\end{proof}
%
In the sequel we apply Lemma~\ref{lem:solutions} in case when $n$ and $m$ are constants,
in which case Lemma \ref{lem:solutions} yields the bound $\norm(U+P) \leq \poly(N)$.

\begin{proof}[Proof of Claim~\ref{cl:prefix}]
For $d \in \setto 2$ and
a path $\gamma$, let $\eff_j(\gamma)$ denote the $j$-th coordinate of $\eff(\gamma)$, and
let $\drop_j(\gamma)\in\N$ be the maximal value of $-\eff_j(\delta)$, where $\delta$ ranges over prefixes of $\gamma$,
that is  the maximal amount by which the $j$-th coordinate can be decreased along $\gamma$.

\Wlog we assume that $\Lambda$ has exactly three loops, 
$\Lambda = \alpha_1\beta_1^{*}\alpha_2\beta_2^{*}\alpha_3\beta_3^{*}\alpha_4$. 
We describe paths $s = (s_1, s_2) \tran (u_1, u_2)$ in $\Lambda$,
\begin{align*}
(s_1, s_2) = \ & 
(a^1_1, a^1_2) \trans{\alpha_1} (b^1_1, b^1_2) \trans{\beta_1^{n_1}}
(a^2_1, a^2_2) \trans{\alpha_2} (b^2_1, b^2_2) \trans{\beta_2^{n_2}} \\
& (a^3_1, a^3_2) \trans{\alpha_3} (b^3_1, b^3_2) \trans{\beta_3^{n_3}}
(a^4_1, a^4_2) \trans{\alpha_4} (b^4_1, b^4_2) = (u_1, u_2),
\end{align*}
by the following system of linear Diophantine inequalities, with unknowns 
$a^{i}_1, a^{i}_2, b^{i}_1, b^{i}_2$, for $i\in\setto 4$, and $n_i$, for $i\in\setto 3$,
ensuring that the effects of $\alpha_i$ and $\beta_i^{n_i}$ are respected, 
and that all points along $\alpha_i$ remain nonnegative 
:
%
%\begin{itemize} 
%\item $a^{i}_1, a^{i}_2$, for $i\in\setto 4$, denoting the midpoint just before the execution of $\alpha_{i}$ 
%(for $i=1$, the source);
%\item $b^{i}_1, b^{i}_2$, for $i\in\setto 4$, denoting the midpoint just after  the execution of $\alpha_{i}$
%(for $i=4$, the target);
%\item $n_i$, for $i\in\setto 3$, denoting the number of times $\beta_i$ is executed.
%\end{itemize}
%
%\noindent
%Therefore the path from $s$ to $t$ has the following form:
%
%The following system of inequalities $\U$ guarantees that the path of $\Lambda$ is inside the positive quadrant.
%
\begin{align*}
a^{i}_j + \eff_j(\alpha_i) \ = \ & \ b^{i}_j  & a^1_j \ = \ & \ s_j \\
b^{i}_j + n_i \cdot \eff_j(\beta_i) \ = \ & \ a^{i+1}_j & b^4_1 \ = \ & \ u_1 \\
a^i_j - \drop_j(\alpha_i) \ \geq \ & \ 0 
\end{align*}
%
Notice that each $\beta_i$ is a single transition, so nonnegativity of $(b_i^1, b_i^2)$ and $(a_{i+1}^1, a_{i+1}^2)$
implies that all the vectors along  $\beta_i^{n_i}$ are also nonnegative. Therefore, we do not need to
add an analog of $a^i_j - \drop_j(\alpha_i) \geq 0$ for $\beta_i$ to the above system.
%
%\begin{enumerate}[(1)]
%\item For each $i \in [1,4]$ and $d \in [1,2]$ we take care of the effect of $\alpha_i$:\label{eq:LPS_eff_1}
%$$a_{i}^d + \eff_d(\alpha_i) = b_{i}^d$$
%\item For each $i \in [1,3]$ and $d \in [1,2]$ we take care of the effect of $\beta_i$:\label{eq:LPS_eff_2}
%$$b_{i}^d + n_i \cdot \eff_d(\beta_i) = a_{i+1}^d$$
%\item For each $i \in [1,4]$ and $d \in [1,2]$ we take care that counter remains nonnegative along $\alpha_i$
%$$a_i^d + \drop_d(\alpha_i) \geq 0$$ \label{eq:LPS_non}
%\item For each $d \in [1,2]$ we take care that a path starts in $s$.
%$$a_1^d = s_d$$\label{eq:LPS_start}
%\item We take care that the first coordinate of the target is equal to $x$.
%$$b_4^1 = x$$\label{eq:LPS_end}
%\end{enumerate}
%Notice that . 
By adding dummy variables we change inequalities into equations,
thus obtaining a system $\U$ of linear Diophantine equations.
All the coefficients in $\U$ are bounded by $\max(B, M)$, 
and all coefficients in its homogeneous version are bounded by $M$.
Therefore, by Lemma~\ref{lem:solutions} the solution set of $\U$ is $U+P^*$,
where $\norm(U) \leq \poly(B,M)$ and $\norm(P) \leq \poly(M)$.
By projecting the solution set to the variable $b^4_2$, we deduce that
the set $S:=\{u_2 \mid (u_1, u_2) \in \reach(\Lambda,s)\}$ is a finite union of 
linear sets $a+X^*$, where $a \leq P_1(B,M)$ and $X \subseteq [0,P_2(M)]$,
for some nondecreasing polynomials $P_1, P_2$.
%
% $a \leq \poly(B,M)$ and $X \subseteqfin \N$, such that
%for each $x \in X$ we have $x \leq \poly(M)$.
%
%$(a, r, T)$-arithmetic sets, where
%$a \leq \poly(B,M)$, $r \leq \poly(M)$, and $M=\size(\Lambda,s)$.
%
\begin{claim}\label{cl:linear-1dim}
For every $a,b \in \N$ and $B \subseteq [1,b]$, the linear set $a+B^*$ is a finite union
of arithmetic sets $c + d^*$, where $c \leq a+b^3$ and $d \leq b$.
\end{claim}
\begin{proof}
Let $B = \set{b_1, \ldots, b_m}$.
Let $n \in a + B^*$, and let $(k_1, \ldots, k_m) \in \N^m$ be the lexicographically smallest vector such that
$n = a + \sum_{i=1}^m k_i \cdot b_i$.
We observe that $k_i < b$ for all $i\in\setto {m-1}$ since,
% may hold for at most one $i\in\setto m$.
supposing $k_i \geq b$ for
$i < m$, we would get a lexicographically smaller  vector
\[
(k_1, \ldots, k_{i-1}, k_i - b_m, k_{i+1}, \ldots, k_m+ b_i) \in \N^m
\]
with the same property. 
Thus $n \in a + r + b_m^*$,
where $r = \sum_{i<m} k_i \cdot b_i \leq b^3$. 
As $n\in a+ B^*$ was chosen arbitrarily, we deduce
%This means any $n \in a+B^*$ satisfies $n \in c+b_i^*$, where $c \leq a+b^3$. In consequence
\[
a+B^* \ = \bigcup_{c \leq a+b^3, c \in a+B^*} c + b_m^*,
\]
which concludes the proof of Claim~\ref{cl:linear-1dim}.
\end{proof}
%
By Claim~\ref{cl:linear-1dim}, the set $S$ is a finite union of arithmetic sets  $a+x^*$, where 
$a \leq P_1(B,M) + P_2(M)^3$ and $x \leq P_2(M)$,
which concludes the proof of Claim~\ref{cl:prefix}.
Notice that we actually get $T = \infty$ or $T=0$ in all the arithmetic sets, but that is not needed for our considerations.
\end{proof}

% !TEX root = main.tex

\begin{proof}[Proof of Claim~\ref{cl:infix}]
Fix a  one-turn \slps $\Lambda=\alpha_1\beta_1^*\alpha_2\beta_2^*$, and
let $M:=\size(\Lambda)$ and $S_2 := R(S_1)$.
Our goal is to show that the set $S_2 = \bigcup_{c\in S_1} R(\{c\})$
%For $c \in S_1$ let us denote $R(c) = \set{v_2 \in (u_1, c) \tran (v_1, v_2)}$.
%Let $R(S_1) = \bigcup_{c \in S_1} R(c)$. Our goal is to show that $R(S_1)$ 
is a finite union
of $(a', r', T')$-arithmetic sequences for $a' \leq a + \poly(B, M, r)$ and $r' \leq \poly(M)$.
We write $R(c)$ instead of $R(\{c\})$.

Let $\eff(\beta_1) = (x_1, -y_1)$ and $\eff(\beta_2) = (-x_2, y_2)$, for some $x_1, x_2, y_1, y_2 \in \Npos$.
% (recall that $\beta_1$
%and $\beta_2$ belong to the corresponding quadrants by assumptions of Claim~\ref{cl:infix}).
Every path $\rho = \alpha_1\beta_1^{n_1}\alpha_2\beta_2^{n_2}$ is determined by 
a pair $(n_1,n_2) \in \N^2$,
and hence when $(u_1, u_2) \tran (v_1, v_2)$, we may also write $(u_1, u_2) \trans {(n_1,n_2)} (v_1, v_2)$
for $(n_1, n_2)\in\N^2$, or 
$(u_1, u_2) \trans {(n_1,n_2)}$ when the target vector is not relevant.
%and we say that $\rho = \rho(n_1, n_2)$.
Whenever this happens, we necessarily have the equality
%We are interested in possible $c_2 \in \N$ such that there is $c_1 \in S_1$ such that $(b_1, c_1) \trans{(n_1, n_2)} (b_2, c_2)$
%for some $n_1, n_2 \in \N$. Notice that looking at the first coordinate gives the following equation
$
u_1 + \eff_1(\alpha_1 \alpha_2) + n_1 x_1 - n_2 x_2 = v_1
$
of effects on the first coordinate,
which we transform to an equation with two unknowns $n_1,  n_2$:
\begin{equation}\label{eq:xy}
n_1 x_1 - n_2 x_2 = v_1 - u_1 - \eff_1(\alpha_1 \alpha_2).
\end{equation}
%
The set of solutions $(n_1, n_2)$ of~\eqref{eq:xy} is of the form $w+p^*$, 
where $w = (w_1, w_2) \in \N^2$ is the minimal solution of~\eqref{eq:xy}
and $p = (p_1, p_2) \in \N^2$ is the minimal solution of its uniform version, $n_1 x_1 - n_2 x_2 = 0$.
By Lemma~\ref{lem:solutions} we get $\norm(w) \leq \poly(B, M)$. One can easily observe that $\norm(p) \leq 2M$
($(x_1, x_2) = (n_2, n_1)$ is a solution).
%
Let $\eff_2(w) = -w_1 y_1 + w_2 y_2$ and $\eff_2(p) = -p_1 y_1 + p_2 y_2\in \N$ be 
the effects induced by $w$ and $p$, respectively, on the second coordinate.

\begin{example}\label{ex:infix}
\begin{figure}[t]
\vspace{-2cm}
\begin{minipage}{0.5\textwidth}
\begin{tikzpicture}[scale=0.35]
\scalebox{0.7}{
\draw[->, thick] (0, 0) -- (20, 0) node[right] {$x$};
\draw[->, thick] (0, 0) -- (0, 18) node[above] {$y$};

\fill[red] (0,6) circle (7pt);

\draw[->] (0,6) -> (2,5);
\draw[->] (2,5) -> (4,4);
\draw[very thick,->] (4,4) -> (6,3);
\draw[very thick,->] (6,3) -> (8,2);
\draw[very thick,->] (8,2) -> (10,1);
\draw[->] (10,1) -> (12,0);

\draw[->] (4,4) -> (1,7);
\fill[green] (1,7) circle (7pt);

\draw[very thick,->] (10,1) -> (7,4);
\draw[very thick,->] (7,4) -> (4,7);
\draw[->] (4,7) -> (1,10);
\fill[green] (1,10) circle (7pt);

\draw[step=1, gray, thin] (0, 0) grid (19, 17);
}
\end{tikzpicture}
\end{minipage}
\begin{minipage}{0.5\textwidth}
\begin{tikzpicture}[scale=0.35]
\scalebox{0.7}{
\draw[->, thick] (0, 0) -- (20, 0) node[right] {$x$};
\draw[->, thick] (0, 0) -- (0, 18) node[above] {$y$};

\fill[red] (0,9) circle (7pt);

\draw[->] (0,9) -> (2,8);
\draw[->] (2,8) -> (4,7);
\draw[very thick,->] (4,7) -> (6,6);
\draw[very thick,->] (6,6) -> (8,5);
\draw[very thick,->] (8,5) -> (10,4);
\draw[->] (10,4) -> (12,3);
\draw[->] (12,3) -> (14,2);
\draw[->] (14,2) -> (16,1);
\draw[->] (16,1) -> (18,0);

\draw[->] (4,7) -> (1,10);
\fill[green] (1,10) circle (7pt);

\draw[very thick,->] (10,4) -> (7,7);
\draw[very thick,->] (7,7) -> (4,10);
\draw[->] (4,10) -> (1,13);
\fill[green] (1,13) circle (7pt);

\draw[->] (16,1) -> (13,4);
\draw[->] (13,4) -> (10,7);
\draw[->] (10,7) -> (7,10);
\draw[->] (7,10) -> (4,13);
\draw[->] (4,13) -> (1,16);
\fill[green] (1,16) circle (7pt);

\draw[step=1, gray, thin] (0, 0) grid (19, 17);
}
\end{tikzpicture}
\end{minipage}

\caption{Left: $u_2 = 6$, $S_2 = \set{7,10}$. Right: $u_2 = 9$, $S_2 = \set{10,13,16}$.
Thick vectors add up to $p$.}
\label{fig:infix}
\end{figure}

Let $\alpha_1$ and $\alpha_2$ be empty sequences, $\eff(\beta_1) = (2, -1)$ and $\eff(\beta_2) = (-3, 3)$.
Let $u_1 = 0$, $v_1 = 1$. 
The set of solutions $(n_1, n_2)$ of~\eqref{eq:xy} is of the form $w+p^*$, where
$w = (2,1)$ and $p = (3,2)$, and
$\eff_2(p) = 3 \cdot (-1) + 2 \cdot 3 = 3$.
Figure~\ref{fig:infix} shows $R(u_2) = \set{7,10}$ when $u_2 = 6$, and $R(u_2)  = \set{10,13,16}$ when $u_2 = 9$.
In the latter case, elements of $R(u_2)$ correspond to the solutions $w$, $w+p$ and $w+2p$ of~\eqref{eq:xy},
i.e., to $w + kp$ for $k$ in an interval $[0,2]$.
According to Claim~\ref{cl:interval}, this is true in general.
%$S_2 = \setof{u + kp}{k\in I}$, for some interval $I$.
\end{example}

%The following claim states that the number of possible periods $p$ used in the run is an interval.

\begin{claim}\label{cl:interval}
For each $c \in S_1$ there exists an interval $I_{c} = [k_1, k_2]$, where $k_1\in \N$ and $k_2 \in \N_{\infty}$,
such that
$R(c) = \setof{\eff_2(w) + k \cdot \eff_2(p)}{k\in I_{c}}$.
% and path $\rho(b+kp)$ starting from $(b_1, c)$ is a valid if and only if $k \in I_c$. 
\end{claim}

\begin{proof}
It is sufficient to show that whenever $(u_1, c) \trans{w + k_1 \cdot p}$ and
$(u_1, c) \trans{w + k_2 \cdot p}$ for some
$k_1 < k_2 \in \N$,
then
$(u_1, c) \trans{w + k \cdot p}$ for all $k \in [k_1, k_2]$.
Fix $k \in [k_1, k_2]$.
Every point $x$ on the (possibly $\Z$-)path $(u_1, c) \trans{w + k \cdot p}$ is actually on a straight line between some two points,
one on the path $(u_1, c) \trans{w + k_1 \cdot p}$, and the other on the path $(u_1, c) \trans{w + k_2 \cdot p}$.
In consequence $x$, being a weighted average of the two points in $\N^2$, necessarily belongs to $\N^2$. 
Therefore $(u_1, c) \trans{w + k \cdot p}$ is a path.
\end{proof}

We notice that $\eff_2(p)$ can be negative, but this is irrelevant for our arguments.
%
By Claim~\ref{cl:interval}, for each $c \in S_1$ the set $R(c)$ is
is an arithmetic sequence of difference $\essdvass r := \absv{\eff_2(p)}$ and length equal to the cardinality
of the interval $I_{c}$.
%Thus $R(c) = b + (\essdvass r)^{\card{I_c}}$, for some $b\in\N$. 
Let $\spn(c)\in\N_\infty$ be the difference between the supremum of $R(c)$ and the minimal element in $R(c)$.
We say that $\spn(c) = \infty$ if $R(c)$ is infinite.
We split the proof into cases, depending on
whether $\essdvass r$ divides the difference $r$ of the sequence $S_1$, or not.
Additionally we have a case when $\essdvass r = 0$, in other cases we silently assume that $\essdvass r \neq 0$.


\para{Case I: $r$ is divisible by $\essdvass r$}

Therefore, if $\spn(c) \geq r$ then the sequence $R(c)$ actually touches the sequence $R(c+r)$,
i.e., their union is a larger arithmetic sequence of difference $\essdvass r$:

\begin{claim}\label{cl:merged}
If $R(c+r^*) \neq \emptyset$ and $c \geq D:= 3(M+r) \cdot M^2 + M \cdot \norm(w)$,
then $R(c+r^{\leq T}) = b+(\essdvass r)^{\leq T'}$ for some $b \leq c + \eff_2(w) + M \cdot \eff_2(p)$ and $T' \in \N_\infty$.
\end{claim}

\begin{proof}
We first show that $\spn(c) \geq r$.
Suppose $R(c+r^*) \neq \emptyset$ and $c \geq D$.
Due to the first assumption,  for some $n\in\N$ there is a path $(u_1, c+nr) \trans{(n_1, n_2)}$
of the form
\begin{align} \label{eq:Mp}
(u_1, c+nr) \trans{\alpha_1} (\essdvass{x}_1, \essdvass{y}_1) \trans{\beta_1^{n_1}} (\essdvass{x}_2, \essdvass{y}_2) \trans{\alpha_2} (\essdvass{x}_3, \essdvass{y}_3) \trans{\beta_2^{n_2}} (v_1, v_2).
\end{align}
In particular,  $u_1 \geq \drop_1(\alpha_1)$
and therefore $\essdvass x_1\geq 0$.
It is enough to take $(n_1, n_2) := w + Mp$ in \eqref{eq:Mp}.
Then $n_1 \geq M$, so $\essdvass x_2 \geq \essdvass x_1 + M \geq M \geq \drop_1(\alpha_2)$.
Therefore $\essdvass x_3\geq 0$.
Now we show that $c$ is large enough such that $(u_1, c) \trans{w+Mp}$ is nonnegative on the second coordinate as well.
As $\norm(p) \leq 2M$ then $n_1 + n_2 \leq M \cdot \norm(p) + \norm(w) \leq 2M^2 + \norm(w)$. Therefore
$\beta_1^{n_1}$ and $\beta_2^{n_2}$ can in total decrease the second coordinate by at most $M \cdot (2M^2 + \norm(w))$.
As $\alpha_1 + \alpha_2$ can in total decrease the second coordinate by at most $M$ and
$D \geq M + M \cdot (2M^2 + \norm(w))$ we conclude that indeed the path $(u_1, c) \trans{w+Mp}$ is valid.
For the same reasons, for every $m\in [M, M+r]$ there is a path $(u_1, c) \trans{w+mp}$, which guarantees $\spn(c) \geq r$. 

Now we use that fact that $\spn(c) \geq r$.
By monotonicity of \vass, if $R(c) = b+(\essdvass r)^{\leq {T'}}$
then $R(c+r)$ necessarily includes $R(c) + r= b+r+(\essdvass r)^{\leq {T'}}$.
Since $\spn(c) \geq r$, we have $b+r \in R(c)$, but also $b+r \in R(c+r)$,
and therefore the union $R(c) \cup R(c + r)$ forms one arithmetic sequence 
$b+(\essdvass r)^{\leq T'}$, for some $b\in\N$ and $T'$.
The similar reasoning applies to any finite union, namely to $R(c+r^{\leq m})$ for $m\in\N$.
In consequence, for every $T\in \N_\infty$ we have $R(c+r^{\leq T}) = b + (\essdvass r)^{\leq T'}$,
for some $b\in\N$ and $T'\in \N_\infty$,
and since $c+ \eff_2(w) + M \cdot \eff_2(p) \in R(c)$ we get the inequality $b \leq c + \eff_2(w) + M \cdot \eff_2(p)$, as required.
\end{proof}

We are ready for concluding Case I.
%Let $D = 3(M+r) \cdot M^2 + M \cdot \norm(w)$, as in the Claim~\ref{cl:bigspan}.
As $\norm(w) \leq \poly(B, M)$ we have $D \leq \poly(B, M, r)$.
We  partition $S_1 = a + r^{\leq T}$ into two subsets: $S'_1 = S_1 \cap [0,D)$
and $S''_1 = S_1 \cap [D,\infty)$, both being arithmetic sequences of difference $r$,
and consider $S'_1$ and $S''_1$ separately.
 
Concerning $S'_2:= R(S'_1)$,
as $\max(S'_1) \leq D$, all
elements of $S'_2$ are upper-bounded by a polynomial in $M$ and $r$, namely
$\max(S'_2) \leq M^2 \cdot (2M + D)$. 
Thus $S'_2$ can be seen as a finite sum of singletons, each of which
being an $(a', r', T')$-arithmetic sequence with $a' \leq M^2 \cdot (2M + D) \leq \poly(B,M,r)$,
$r' = 1$ and $T' = 0$. 
Clearly $r' = 1 \leq \poly(M)$, and hence $S'_2$ is of the required form.

Now we consider $S''_2:= R(S''_1)$.
If $S''_2 = \emptyset$ we are done.
Otherwise, let $c:= \min(S''_1)$.
Thus $S''_1 = c+r^{\leq T}$ for some $T \in \N_\infty$, 
and $D\leq c \leq \max(a, D+r)$.
%By Claim~\ref{cl:bigspan}, $\spn(c) \geq r$, and therefore using Claim~\ref{cl:neighbours} 
By Claim~\ref{cl:merged} we deduce that
$S''_2 = b + (\essdvass r)^{\leq T'}$ for some 
$b \leq c + \eff_2(w) + M \cdot \eff_2(p)$ and $T' \in \N_\infty'$. 
As $\eff_2(w) \leq \poly(B,M)$, $\eff_2(p) \leq \poly(M)$ and $c \leq a + \poly(B,M, r)$ we get $b \leq a + \poly(B, M, r)$.
We also have $\essdvass r \leq \poly(M)$, and hence
$S''_2$ is of the required form.


\para{Case II: $r$ is not divisible by $\essdvass r$}
%Assume now that $r$ does not divide $\eff_2(p)$. 

In that case we split $S_1 = a + r^{\leq T}$ into several
arithmetic sequences of difference $r \cdot \essdvass r$, namely into sequences of a form 
$(a + m \cdot r) + (r \cdot \essdvass r)^{\leq T'}$, where $m < \essdvass r$, 
and apply the above reasoning to each of this sequences separately. 
As $\essdvass r \leq \poly(M)$
we get also a finite set of arithmetic sequences with the base bounded by $a + \poly(B, M,r)$ and difference bounded by $\poly(M)$,
as required.

\para{Case III: $\essdvass r = 0$}
\Wlog we assume $R(S_1) \neq \emptyset$. For every $c \in S_1$ we have that either $R(c) = c + \eff_2(w)$ or $R(c) = \emptyset$. By monotonicity of VASS we have that if $R(c) \neq \emptyset$ then $R(c+r) \neq \emptyset$. Let $D := 3(M+r) \cdot M^2 + M \cdot \norm(w)$. Similarly as in the proof of Claim~\ref{cl:merged} we observe that if $c \geq D$ then for some $k \in N$ there is a run $(u_1, c) \trans{w+kp}$. Hence we have $R(S_1) = c + \eff_2(w) + r^{\leq T'}$ for some $T' \in \N_\infty$ and some $c \in S_1$ such that $c \leq \max(a, D+r)$. Therefore $R(S_1) = b + r^{\leq T'}$ for some $b \leq a + \poly(B,M,r)$ as required.

\end{proof}



%
%Observe, that if $(b_1, s) \trans \rho (b_2, t)$ for $s \in S_1$ and $t \in S_2$ then $x,y$ satisfy the following equation:
%$$\eff_1(\alpha_1\alpha_2) + x\eff_1(\beta_1) + y\eff_1(\beta_2) = b_2-b_1$$
%Hence we have that the set $S$ of such  pairs $(x,y)$ associated is included in set of solutions of the above equation, which can be represented by Lemma~\ref{lem:taming} as $L(B,P)$ for some sets $B, P \subseteq \N^2$ such that $\norm(B) \leq c(M+B)$ and $\norm(P) \leq cM$ for some constant $c \in \N_+$. We show, that $P$ has only one element $p$. This is because the set of solutions of the following equations over $\Q$ is one dimensional vector space and there is nonnegative solution, because of different signs of $\eff_1(\beta_1)$ and $\eff_1(\beta_2)$:
%$$x\eff_1(\beta_1) + y\eff_1(\beta_2) = 0$$
%Hence each path can be represented as $b+kp$ for some $k \in \N$ and $b \in B$ (recall that with each path we associated pair $(x,y)$). Because $B$ is finite and we are interested in a finite union representation it is enough to show that there exists polynomial $R(M)$ such that the set $S_2' =   \{y_2 \mid \exists_{y_1 \in S_1} (b_1, y_1) \trans{b+kp} (b_2, y_2), k \in \N \}$ is a finite union of $(a', r', T')$-arithmetic sets with $a' \leq a + (B+r) \cdot R(M)$ and
%$r' \leq R(M)$.
%
%We have $p=(p_1, p_2)$ and we write $\eff(p)$ for $\eff(\beta_1^{p_1}\beta_2^{p_2})$. Moreover $b = (b_1, b_2)$ and we write $\eff(b+kp)$ for $\eff(\alpha_1\beta_1^{b_1}\alpha_2\beta_2^{b_2}) + k \eff(p)$. First we solve a case when $|\eff_2(p)|$ divides $r$, to which we will reduce the other case. Let $m = \frac{r}{|\eff_2(p)|}$. Moreover, \mywlog we can assume that $S_2' \neq \emptyset$.
%The intuition is that the set $S_2'$ will be a condensing of $S_1$ due to repetitions of $p$ with base point $a'$ moved a bit. In order to conclude the prove we need a few claims. 
%\begin{claim}
%For each $s \in S_1$ there exists interval $I = [k_1, k_2]$ such that $k_1, k_2 \in \N_{\infty}$ and path $b+kp$ starting from $s$ is a valid path if and only if $k \in I$. 
%\end{claim}
%\begin{proof}
%It is enough to show, that if for $k_1 < k_2 \in \N$ we know, that $b+k_1p$ and $b+k_2p$ are valid paths starting from $s$ then for all $k_3 \in [k_1, k_2]$ we have that path $b+k_3p$ is a valid path starting from $s$. Let us fix $k_3 \in [k_1, k_2]$ and let $b=(b_1, b_2), p=(p_1, p_2) \in \N^2$. Recall, that $\eff(\beta_1) \in \N_+ \times \N_-$ and $\eff(\beta_2) \in \N_- \times \N_+$. Hence it is enough to show the following inequalities:
%$$\eff_1(\alpha_1\beta_1^{b_1+k_3p_1}\alpha_2\beta_2^{b_2+k_3p_2}) = \eff_1(\alpha_1\beta_1^{b_1+k_1p_1}\alpha_2\beta_2^{b_2+k_1p_2})$$
%$$\eff_2(\alpha_1\beta_1^{b_1+k_3p_1}) \geq \eff_2(\alpha_1\beta_1^{b_1+k_2p_1})$$
%For the first equality observe, that:
%\begin{align*}
%\eff_1(\alpha_1\beta_1^{b_1+k_3p_1}\alpha_2\beta_2^{b_2+k_3p_2}) & = \eff_1(\alpha_1\beta_1^{b_1+k_1p_1}\alpha_2\beta_2^{b_2+k_1p_2}) 
%+ (k_3-k_1)(p_1\eff_1(\beta_1) + p_2\eff_1(\beta_2)) \\
%& = \eff_1(\alpha_1\beta_1^{b_1+k_1p_1}\alpha_2\beta_2^{b_2+k_1p_2})
%\end{align*}
%The second inequality follows from the fact that:
%$$b_1+k_3p_1 \leq b_1+k_2p_1$$
%\end{proof}
%Let us by $I_x = [k_1^s, k_2^s]$ denote the interval for $x \in S_1$ and by $|I_x| = k_2^s - k_1^s$. Now we split elements of $S_1$ into three groups depending on the properties of the points in $S_1$.
%\begin{claim}\label{clm:split}
%For each $x \in S_1$ at least one of the following holds:
%\begin{itemize}
%\item $x \leq a + 3cM^2(B+r)$
%\item $x$ is the maximal element of $S_1$ and $[0,m] \subseteq I_x$
%\item $[0, m] \subseteq I_x$ and if $b+kp$ is executable from $(b_1,x)$ for $k > m$ than $b+(k-m)p$ is executable from $(b_1,x+m\eff_2(p))$ and $x+m\eff_2(p) \in S_1$. 
%\end{itemize}
%\end{claim}
%\begin{proof}
%Let us take $x \in S_1$ such that  $x > a + 3cM^2(B+r)$. Firstly, we show that $[0, m] \subseteq I_x$. Recall, that $\Lambda$ is one-turn and observe, that for all $k \in \N$ we have:
%$$\eff_1(\alpha_1\beta_1^{b_1}\alpha_2\beta_2^{b_2}) = \eff_1(\alpha_1\beta_1^{b_1}\alpha_2\beta_2^{b_2}) + k(p_1\eff_1(\beta_1) + p_2\eff_1(\beta_2))$$
%Hence, because $S_2$ is not empty we have that all paths $b+kp$ starting from $(b_1, x)$ are valid on the first counter. Hence we only need to deal with the second counter. Now observe, that for each $k \in [0,m]$ we have that maximal decrease of the second counter along path $b+kp$ is at most :
%\begin{align*}
%M + (b_1+kp_1)\eff_2(\beta_1) & \leq M + (c(M+B)+mcM)M \\
%& \leq 3cM^2(B+m) \leq 3CM^2(B+r) < x
%\end{align*}
%Hence $[0, m] \subseteq I_x$.
%If $x$ is the maximal element we are done. Otherwise,
%we show, that if $b+kp$ is executable from $(b_1,x)$ for $k > m$ than $b+(k-m)p$ is executable from $(b_1,x+m\eff_2(p))$ and $x+m\eff_2(p) \in S_1$. Observe, that $x+m\eff_2(p) = x+r$ if $\eff_2(p) > 0$ and $x+m\eff_2(p) = x-r$ otherwise. Because $S_1 =  a + r^{\leq K}$ and $x$ is not the maximal element of $S_1$ then in the first case $x+m\eff_2(p) \in S_1$. In the second case we have $x -  r \geq a+r-r \geq a$ and hence $x-r \in S_1$.
%
%Now it is left, that $b+(k-m)p$ is executable from $(b_1,x+m\eff_2(p))$. Similarly as before we only need to take care about the second counter. For this it is enough to observe, that:
%\begin{align*}
%x + m\eff_2(p) + \eff_2(\alpha_1\beta_1^{b_1+(k-m)p_1}) 
%& = x + \eff_2(\alpha_1\beta_1^{b_1+kp_1}) + mp_2\eff_2(\beta_2) \\
%& \geq x + \eff_2(\alpha_1\beta_1^{b_1+kp_1})
%\end{align*}
%This is due to $\eff_2(p) = p_1\eff_2(\beta_1) + p_2\eff_2(\beta_2)$ and $\eff_2(\beta_2) \geq 0$.
%Moreover:
%$$x+m\eff_2(p) \geq a+3cM^2(B+r)-r \geq a + M \geq |\eff_2(\alpha_1)|$$
%\end{proof}
%Now we can split $S_1$ into three parts $A_1, A_2, A_3$ such that elements from $A_1$ are the one with $s \leq a + 3cM^2(B+r)$, $A_3$ is the singleton or empty set containing the maximal element and $A_2$ is the rest of $S_1$. Observe, that $A_2 = a_2+r^{\leq K}$ for $a_2 \leq 3cM^2(B+r) + r, a_2 = a + kr$ for some $k \in \N$ and $K = T - k - 1$. Moreover observe, that if path start from $s \in A_2$ and is of the form $b+kp$ for $k > m$ than the same point is reachable from point $(b_1,s+m\eff_2(p))$ by path $b+(k-m)p$ due to Claim \ref{clm:split}. Hence, because we can iterate the argument at some point the starting point will be in $A_1 \cup A_3$ or $k \leq m$. Hence for points from $A_2$ we only need to consider paths $b+kp$ for $k \leq m$. Let us define $B_1, B_2$ and $B_3$.  
%$$B_1 = \bigcup_{x \in A_1, k \in I_x} \set{x + \eff_2(b+kp)}$$
%$$B_2 = \bigcup_{x \in A_2, k \in [0,m]} \set{x + \eff_2(b+kp)} 
%= a_2 + r^{\leq T} + \eff_2(b) + \eff_2(p)^{\leq m}$$
%$$B_3 = \bigcup_{x \in A_3, k \in I_s} \set{x + \eff_2(b+kp)}$$ 
%Which if $A_3 \neq \emptyset$ can be represented as
%$$B_3= a + (T+1)r  + \eff_2(b) + \eff_2(p)^{\leq |I_{x_{max}}|}$$
%where $x_{max}$ is the element of $A_3$. From now on we fix this maximal element and put $|I_{x_max}| = 0$ if it does not exist. 
%Let us observe:
%$$S_2 = B_1 \cup B_2 \cup B_3$$
%We deal separately with $B_1$ and $B_2 \cup B_3$. Because $a_2 + \eff_2(b) \leq 3cM^2(B+r) + r $ clearly, in order to show, that $B_2 \cup B_3$ is  of the form we want it is enough to show the following claim (recall that $\eff_2(p)$ can not be zero due to divisibility condition):
%\begin{claim}
%$(B_2 \cup B_3) \cap [a_2 + \eff_2(b), \infty] = a_2 + \eff_2(b) + |\eff_2(p)|^{\leq T'}$ where $T' = m(T+1) + |I_{x_{max}}|$ if $\eff_2(p) > 0$ and $T' = m(T+1)$ if $\eff_2(p) < 0$ .  
%\end{claim}
%\begin{proof}
%Firstly we show $(B_2 \cup B_3) \cap [a_2 + \eff_2(b), \infty] \subseteq a_2 + \eff_2(b) + |\eff_2(p)|^{\leq T'}$. Let us take any $x \in (B_2 \cup B_3) \cap [a_2 + \eff_2(b), \infty]$. The first case is that $x \in B_2$ and hence 
%$$x = a_2 + \eff_2(b) + k_1r + k_2\eff_2(p)$$
%for $k_1 \leq T$ and $k_2 \leq m$. We have 
%$$x = a_2 + \eff_2(b) + k_1m|\eff_2(p)| + k_2\eff_2(p)$$
%and hence if $\eff_2(p) > 0$ we have 
%$$x = a_2 + \eff_2(b) + (k_1m + k_2)|\eff_2(p)| \in a_2 + \eff_2(b) + |\eff_2(p)|^{\leq T'}$$
%and if $\eff_2(p) < 0$ we have 
%$$x =  a_2 + \eff_2(b) + |\eff_2(p)|^{k_1m-k_2}$$
%Because $x \geq a_2 + \eff_2(b)$ we have $k_1m-k_2 \geq 0$ and hence $x \in a_2 + \eff_2(b) + |\eff_2(p)|^{\leq T'}$ .
%
%Now we show $a_2 + \eff_2(b) + |\eff_2(p)|^{\leq T'} \subseteq (B_2 \cup B_3) \cap [a_2 + \eff_2(b), \infty]$. Let us take $x = a_2 + \eff_2(b) + k|\eff_2(p)|$ for some $k \leq T'$. Firstly let us consider case when $\eff_2(p) > 0$. 
%\begin{align*}
%x & = a_2 + \eff_2(b) + \floor{\frac{k}{m}}m|\eff_2(p)|+ (k\mod m)|\eff_2(p)| \\
%& = a_2 + \eff_2(b) + \floor{\frac{k}{m}}r + (k \mod m)\eff_2(p)
%\end{align*}
%Observe, that if $k \leq m(T+1)$ we have
%$$\floor{\frac{k}{m}} \leq \floor{\frac{m(T+1)}{m}} = T+1$$
%Hence $x \in a_2 + \eff_2(b) + r^{\leq T} + |\eff_2(p)|^{\leq m} = B_2$
%If $k > m(T+1)$ then observe, that $k - m(T+1) \leq |I_{x_{max}}|$ and see, that
%$$x = a_2 + \eff_2(b) + r(T+1) + \eff_2(p)^{k-m(T+1)} \in B_3 $$
%
%Now we consider case when $\eff_2(p) < 0$. Then:
%$$x=a_2 + \eff_2(b) + \ceil{\frac{k}{m}}m|\eff_2(p)|-(m - (k \mod m))|\eff_2(p)|$$
%Then if $k \leq mT$ we have:
%$$x \in a_2 + \eff_2(b) + r^{\leq T}+\eff_2(p)^{\leq m} = B_2 $$
%And if $k > mT$ we have:
%$$x \in a_2 + \eff_2(b) + (T+1)r + \eff_2(p)^{\leq m} \subseteq B_3$$
%
%\end{proof}
%For the set $B_1$ it is enough to prove the following Claim:
%\begin{claim}
%$B_1$ can be represented as a finite union of $(a', r',T')$-arithmetic sets with $a' \leq a + 10cM^4(B+r)$ and $r' \leq cM$.
%\end{claim}
%\begin{proof}
%The first case is that the second counter is at most $3cM^2(B+r)$ at the beginning. The first counter is at most $B$ at the beginning. Then after $\alpha_1$ they are at most $3cM^2(B+r) + M \leq 4cM^2(B+r)$. Then after all execution of the $\beta_1$ the sum of the counters is at most $8cM^3(B+r)$ then after $\alpha_2$ the sum of counters is at most $8cM^3(B+r) + 2M$ and finally after the last execution of $\beta_2$ the sum of counters and hence also the second counter is at most $8cM^4(B+r) + 2M^2 \leq 10cM^4(B+r)$ as needed. Hence there is a finite number of such elements and they can be represented as a singleton arithmetic sets.
%The second case is that the second counter is at least $3cM^2(B+r)$ at the beginning. Let the beginning point be $x$. We show, that elements of $B_1$ obtained by the paths starting in $x$ can be represented as an arithmetic-set. Therefore assume this set is not empty (otherwise it is trivial). Recall, that $\Lambda$ is one-turn and observe, that for all $k \in \N$ we have:
%$$\eff_1(\alpha_1\beta_1^{b_1}\alpha_2\beta_2^{b_2}) = \eff_1(\alpha_1\beta_1^{b_1}\alpha_2\beta_2^{b_2}) + k(p_1\eff_1(\beta_1) + p_2\eff_1(\beta_2))$$
%All paths $b+kp$ starting from $x$ are valid on the first counter. Hence we only need to deal with the second counter. Now observe, that  we have that maximal decrease of the second counter along path $b$ is at most :
%$$M + b_1\eff_2(\beta_1) \leq M + c(M+B)M \leq 3cM^2B$$ Hence path $b$ is also valid on the second counter. Hence $x_2 + \eff_2(b) \in B_1$ and $x_2 + \eff_2(b) \leq a + 3cM^2(B+r) + \eff_2(\alpha_1\alpha_2) + b_1\eff_2(\beta_1) + b_2\eff_2(\beta_2) \leq a + 3cM^2(B+r) + M +  2c(M+B)M \leq a + 10cM^2(B+r)$. Now if $\eff_2(p) > 0$ then the set can be represented as $(x_2 + \eff_2(b), \eff_2(p), T')$-arithmetic set, which satisfies the conditions. Otherwise all elements are at most $a + 10cM^2(B+r)$ and can be represented as singleton arithmetic sets, which satisfies the conditions.
%\end{proof}
%By previous claims we know, that $S_2$ is a finite union of $(a',r', T')$-arithmetic sets for $a' \leq a + 10cM^4(B+r)$ and $r' \leq cM$.
%
%
%Now we come back to the case when $r$ is not divisible by $|\eff_2(p)|$. The first subcase is that $\eff_2(p) = 0$. Then all paths $b+kp$ have the same effect so we only need to consider path $b$. The lowest $x \in S_1$ from which the path $b$ is executable is at most $M + b_1\eff_2(\beta_1) + a + r \leq M + c(M+B)M + a+r \leq a + 3M^2(B+r)$. Hence set $S_2$ can be represented as $(a', r, T')$-arithmetic set for $a' \leq a+ 3M^2(B+r) + \eff_2(b) \leq a+ 3M^2(B+r) + M+2c(M+B)M \leq a + 10M^2(B+r)$. The second subcase is that $\eff_2(p) \neq 0$. Then we can split $S_1$ into several $(a'', \eff_2(p)r, T'')$-arithmetic sets with $a'' \leq a + \eff_2(p)r \leq a + (2cM^2+M)r$ then we can apply the first case and get representation of $S_2$ as a finite union of $(a',r', T')$-arithmetic sets with $r' \leq cM$ and $a' \leq a + (2cM^2+M)r + 10cM^4(B+r) \leq a+ 13M^4(B+r)$. Thus by setting $R(M) = 13cM^4$ we conclude the proof. 
%\end{proof}



\begin{proof}[Proof of Claim~\ref{cl:suffix}]
\Wlog we assume $\Lambda = \alpha_1\beta_1^*\alpha_2\beta_2^*\alpha_3\beta_3^*\alpha_4$.
Let $S_1 = b + \set{p}^{\leq T}$  and $S_2 = \reach_{\Lambda}(S_1)$. Recall, that $\eff_j(\gamma)$ denotes the $j$-th coordinate of $\eff(\gamma)$, and
$\drop_j(\gamma)\in\N$ is the maximal value of $-\eff_j(\delta)$, where $\delta$ ranges over prefixes of $\gamma$,
that is  the maximal amount by which the $j$-th coordinate can be decreased along $\gamma$.

Similarly as in the proof of Claim~\ref{cl:prefix} we describe paths $s = (s_1,s_2) \tran (t_1, t_2) = t$ in $\Lambda$,
\begin{align*}
(s_1, s_2) = &
(a_1^1, a_2^1) \trans{\alpha_1} (b_1^1, b_2^1) \trans{\beta_1^{n_1}}
(a_1^2, a_2^2) \trans{\alpha_2} (b_1^2, b_2^2) \trans{\beta_2^{n_2}} \\
& (a_1^3, a_2^3) \trans{\alpha_3} (b_1^3, b_2^3) \trans{\beta_3^{n_3}}
(a_1^4, a_2^4) \trans{\alpha_4} (b_1^4, b_2^4) = (t_1, t_2).
\end{align*}
Notice that $(s_1, s_2) = (a_1^1, a_2^1) \in S_1$, so $(a_1^1, a_2^1) = u + p^n$ for some $n \in \N$. The following system of linear Diophantine inequalities $\U$ with unknowns $a_1^i, a_2^i, b_1^i, b_2^i$ for $i \in [1,4]$, and $n_i$, for $i \in [1,3]$, and unknown $n$, ensures that the effects of $\alpha_i$ and $\beta_i^{n_i}$ are respected, and that all points along $\alpha_i$ remain nonnegative and that $s \in S_1$:

\begin{align*}
a^{i}_j + \eff_j(\alpha_i) \ = \ & \ b^{i}_j  & a^1_j \ = \ & \ u_j + n \cdot p_j \\
b^{i}_j + n_i \cdot \eff_j(\beta_i) \ = \ & \ a^{i+1}_j & n \ \leq \ & \ T \label{eq:LPS_bound} \\
a^i_j - \drop_j(\alpha_i) \ \geq \ & \ 0 
\end{align*}
Notice that each $\beta_i$ is a single transition, so nonnegativity of $(b_i^1, b_i^2)$ and $(a_{i+1}^1, a_{i+1}^2)$
implies that all the vectors along  $\beta_i^{n_i}$ are also nonnegative. Therefore, we do not need to
add an analog of $a^i_j - \drop_j(\alpha_i) \geq 0$ for $\beta_i$ to the above system. We first focus on the solutions of $\U$ without the equation $n \leq T$, 
and inequalities $a^i_j - \drop_j(\alpha_i) \geq 0$ transformed into equations with dummy variables on the right, similarly as in the proof of Claim~\ref{cl:prefix}. Let us call such system $\U'$. All the coefficients of $\U'$ are bounded by $\max(M, \norm(u), \norm(p))$. By Lemma~\ref{lem:solutions} set of solutions of $\U'$ can be described as $L(U, V)$ for $\norm(U \cup V) \leq \poly(M, \norm(u), \norm(p))$.

%$\norm(U \cup P) \leq R(N) \cdot R(M)$ for some polynomial $R$. After adding equalities~\eqref{eq:LPS_end2} by Proposition~\ref{prop:fix_variables} we get solutions of the form $L(U', P')$ for $\norm(U') \leq R(N) \cdot R(M) \cdot (\norm(b_1) + 1)$ and $\norm(P') \leq R(N) \cdot R(M)$.

Now we have to care about the last inequality, namely $n \leq T$. If for some $a \in U$ we have $n > T$ then we can remove it from $U$. Let $U'$ be the set $U$ without the removed elements.
As the set $U'$ is finite it is enough to prove the conclusion of Claim~\ref{cl:suffix} separately for each $a \in U'$. Fix $a \in U'$.
We have $\norm(a), \norm(V) \leq \poly(M, \norm(u), \norm(p))$.
It is enough to prove that elements of the set $L(a, V)$ that additionally satisfy $n \leq T$, projected to the variables
$b^4_1$ and $b^4_2$ can be described as a finite union of the sets of the form we need.

Let us consider all the elements of set $V$. Let $Q \subseteq V$ be the set of these elements $v \in V$,
for which $v[n] = 0$ (that means that unknown $n$ is equal to $0$ in elements $v$),
while $P = V \setminus Q$ be the set of the other elements $v \in V$,
so that for which $v[n] > 0$.
Notice that using elements in $Q$ does not influence satisfying $n \leq T$, therefore they
can be used unbounded number of times.
Let $P = \set{p_1, \ldots, p_\ell}$ and let $p_i[n] = c_i$. For each $x \in L(a, V)$ we have $x = a + q + \Sigma_{i=1}^{\ell}k_i \cdot p_i$ where $q \in Q^*$. Hence, in order to satisfy $n \leq T$ we have to satisfy $\innprod c k \leq (T - a[n])$, where $c = (c_1, \ldots, c_\ell) \in \N_{>0}^\ell$ and $k = (k_1, \ldots, k_\ell) \in \N^\ell$.
As $c \in \N_{>0}^\ell$, if $T - a[n] < 0$ then the set of solutions is empty. Otherwise, the set of solutions can be represented as $a+P^{c \cdot x \leq T - a[n]}+Q^*$. Recall that $\norm(a), \norm(P \cup Q) \leq \poly(M, \norm(u), \norm(p))$, and additionally $c \in \N_{>0}^{\ell}$. Additionally $\norm(c) \leq \ell \cdot \norm(P)$, where
$\ell$ is the number of elements in $P$, thus $\ell \leq (\norm(P) + 1)^k$, where $k$ is the number of unknowns in $\mathcal U$
(so $k$ is a constant).
In consequence $\norm(c) \leq \poly(M, \norm(u), \norm(p))$. Summarising, the projection
of $a+P^{c \cdot x \leq T - a[n]}+Q^*$ into variables $b^4_1$ and $b^4_2$ is of the required form.
\end{proof}
%
Claims \ref{cl:prefix}, \ref{cl:infix} and \ref{cl:suffix} are thus shown, and hence so is Lemma~\ref{lem:sandwich-raw}.
%
\end{appendixproof}

% !TEX root = main.tex

\section{Future research}\label{sec:future}
Below we list a few research questions, which we find interesting and
particularly promising directions after our contribution.

\para{Exact complexity for $3$-VASS}
We have shown that shortest paths in binary $3$-VASS are of at most triply-exponential length.
It is tempting to conjecture that actually the upper bound for the length of the paths is shorter,
at most doubly-exponential. We conjecture so and leave this conjecture to the future research.

\para{Example of a $3$-VASS with doubly-exponential path}
We have shown that shortest paths in binary $3$-VASS are of at most triple-exponential length.
However, currently we still do not know any example in which even a path of doubly-exponential length is needed,
it might be that paths of exponential length are sufficient leading to \pspace-completeness for binary $3$-VASS.
It would be very interesting to find an example of a binary $3$-VASS with shortest path between two configurations
being doubly exponential. An example of binary $4$-VASS of doubly-exponential shortest path is known (see Section 5 in~\cite{DBLP:conf/concur/Czerwinski0LLM20}). Maybe some modification of this $4$-VASS would allow to design a $3$-VASS with similar properties.

\para{Reachability for $d$-VASS with $d \geq 4$}
It is a natural question whether our techniques extend to higher dimensions.
The answer is: possibly yes, but we would need a few other structural results for $3$-VASS
to make a similar approach to $4$-VASS possible. In the proof of Lemma~\ref{lem:main} we do not only
use $2$-VASS reachability as a black box, but we use a deep understanding of the reachability relation in $2$-VASS
from~\cite{DBLP:conf/focs/0001CMOSW24}. Probably a similar understanding of the reachability relation for $3$-VASS would be needed
to advance understanding of $4$-VASS along our lines. 

In general it is very interesting to determine the complexity of the reachability problem for $d$-VASS.
We have excluded that for each $d \geq 3$ the problem is $\F_d$-completely, but it is still possible that
the problem is $\F_{d-C}$-complete for some constant $C \in \N$ and $d$ big enough.
Recall that in~\cite{DBLP:conf/fsttcs/CzerwinskiJ0LO23}
it was shown that the reachability problem for $(2d+4)$-VASS is $\F_d$-hard for any $d \geq 3$ and this
is the best currently known lower bound for arbitrary dimension.
Therefore the other natural possibility is that the reachability problem for $(2d+C)$-VASS is $\F_d$-complete for some
constant $C \in \N$. 

\para{Applications of the approximation technique}
Another natural research direction is to search for other applications of the technique of approximating the reachability sets,
which allows to lower the complexity down, below the size of the reachability set.
One particular case, which seems to be prone to such techniques is the $2$-VASS with some number of $\Z$-counters, namely counters, which can take values below zero.
The best complexity lower bound for the reachability problem in this model is \pspace-hardness inherited from~\cite{BlondinFGHM15},
while the best upper bound is Ackermann membership inherited from VASS reachability~\cite{LS19}.
The reachability sets for that systems are not necessarily semilinear.
This disqualifies most of the techniques relying on the semilinearity of reachability sets, but our techniques
seem to be promising for that model.




%\end{document}

% anonymised
%
\paragraph*{Acknowledgements}
We thank Radosław Piórkowski for many inspiring discussions about cones in VASS a few years ago
and Filip Mazowiecki for the discussions about the reachability problem for \tvass.
We thank also Henry Sinclair-Banks for helpful discussions about \cite[Thm 4.16]{DBLP:conf/focs/0001CMOSW24}.
We thank Qizhe Yang and Yangluo Zheng for the discussions on $3$-VASS, which are geometrically $2$-dimensional.

\newpage

\bibliographystyle{plain}
\bibliography{citat}

\newpage


\appendix

%\input{ATTIC/1comp-copy}

\end{document}
