% !TEX root = main.tex

\section{Introduction}\label{sec:intro}

Petri nets are an established model of concurrent systems with extensive applications in various
areas of theoretical computer science.
For most of algorithmic questions, the model is equivalent to
\emph{vector addition systems} with states (\vass in short).
A $d$-dimensional \vass (\parvass d in short) is a finite automaton equipped additionally
with a finite number $d$ of nonnegative integer counters that are updated by
transitions, under the proviso that the counter values can not drop below zero.
Importantly, \vass have no capability to zero-test counters, and hence the model is not Turing-complete. 

One of the central algorithmic problems for VASS is the \emph{reachability problem} which asks,
if in a given \vass there is a path (a sequence of executions of transitions) 
from a given source configuration (consisting of a 
state together with counter values) to a given target configuration:

\probdef{\vass reachability problem}
{\vass $V$, source and target configurations $s, t$}
{Is there a path from $s$ to $t$}

\noindent
Already in 1976, the problem was shown to be \expspace-hard by Lipton~\cite{Lipton76}, and
few years later decidability was shown by Mayr~\cite{Mayr81}.
Later improvements~\cite{DBLP:conf/stoc/Kosaraju82,DBLP:journals/tcs/Lambert92} simplified the construction, 
but no complexity upper bound was given until Leroux and Schmitz showed that the problem 
is solvable in Ackermannian complexity~\cite{LerouxS15,LS19}.
At the same time the lower bound was lifted to \tower-hardness~\cite{CzerwinskiLLLM21}, and soon
afterwards, in 2021, improved to \ackermann-hardness~\cite{DBLP:conf/focs/Leroux21,DBLP:conf/focs/CzerwinskiO21}.

Although the complexity of the reachability problem is now settled to be \ackermann-complete, 
various complexity questions remain still widely open.
%
One of them is the complexity of the reachability problem parametrised by dimension, namely in
$d$-dimensional \vass (\parvass d) for fixed $d \in \N$. 
Despite that the question is investigated since a few decades, 
exact bounds are only known for dimensions 1 and 2
(both for unary or binary representations of numbers in counter updates).
In case of binary encoding, the reachability problem is \np-complete
for \parvass 1~\cite{DBLP:conf/concur/HaaseKOW09} and \pspace-complete 
for \dvass~\cite{BlondinFGHM15}, and in case of
unary encoding the problem is \nl-complete both for \parvass 1 (folklore) and for 
\dvass~\cite{DBLP:conf/lics/EnglertLT16}.
For higher dimension almost nothing is  known.

All the upper complexity bounds for dimension 1 or 2 were obtained by estimating
the length of the shortest path, or of its representation.
%
For unary \parvass 1, it is a folklore that if there is a path between two configurations then there is 
also one of polynomial length (see~\cite{DBLP:journals/lmcs/ChistikovCHPW19} for more demanding 
quadratic upper bound), which implies \nl-completness.
For binary \parvass 1, a polynomial-size representation of the shortest path was provided
by~\cite{DBLP:conf/concur/HaaseKOW09}.
%
Concering binary \dvass,
already in 1979 Hopcroft and Pansiot showed that the reachability sets of \dvass
are effectively semi-linear, and therefore the reachability problem is 
decidable~\cite{DBLP:journals/tcs/HopcroftP79}.
Subsequently, \twoexptime upper complexity bound for binary \dvass
was established by~\cite{DBLP:journals/tcs/HowellRHY86}.
In~\cite{DBLP:conf/concur/LerouxS04} Leroux and Sutre showed that even the reachability relation 
is semi-linear, and that the relation
is flattable, namely there it can be described by a finite number of simple expressions called 
\emph{linear path schemes}.
Only in 2015, careful examination of these linear path schemes led to
exponential upper bound on the length of the shortest path, and 
consequently to \pspace upper bound~\cite{BlondinFGHM15}.
%(\pspace-hardness easily follows by reduction from
%\emph{bounded} binary \parvass 1~\cite{DBLP:conf/icalp/FearnleyJ13}).
%
Concerning unary \dvass, polynomial upper bound on the length of the shortest path was shown
\cite{DBLP:conf/lics/EnglertLT16,DBLP:journals/jacm/BlondinEFGHLMT21},
thus yielding \nl-completeness. 

Our understanding of the model drops drastically for dimensions larger than 2,
as most of good properties admitted in dimension 1 or 2 vanish.
For instance, already since the seminal paper~\cite{DBLP:journals/tcs/HopcroftP79} it is known that
reachability sets of \tvass are not necessarily semi-linear.
Investigation of \tvass was advocated by many papers cited above, 
e.g.~\cite{DBLP:journals/tcs/HopcroftP79,BlondinFGHM15,DBLP:journals/jacm/BlondinEFGHLMT21},
but until now no specific complexity bounds for \tvass are known, except for generic 
parametric bounds known for \parvass d in arbitrary fixed dimension $d\geq 3$.
By~\cite{LS19}, the reachability problem in \parvass d
is in  $\F_{d+4}$,%
\footnote{The complexity class $\F_i$ correspons to the $i$-th level 
of Grzegorczyk's fast-growing hierarchy~\cite{DBLP:journals/toct/Schmitz16}.}
later improved to $\F_d$~\cite{DBLP:conf/icalp/FuYZ24}.
In case of \tvass, this yields membership in $\F_3 = \tower$.
On the other hand, no lower bound is known for binary \tvass except for the \pspace lower bound 
inherited from binary \dvass
(for unary \tvass, \np-hardness has been recently shown by~\cite{DBLP:conf/focs/0001CMOSW24}).
The complexity gaps remains thus huge, namely between \pspace and \tower.
As our main result, we narrow down this gap significantly.

%For completeness, we mention few other known lower bounds in fixed dimensions
%\cite{DBLP:conf/lics/CzerwinskiO22}: \pspace-hardness for unary \parvass 5,
%\expspace-hardness for binary \parvass 6,
%and \tower-hardness for \parvass 8. 


\para{Contribution}

In this paper we investigate complexity of the reachability problem in \tvass.
%
Our main result is the first elementary upper complexity bound for the problem:

\begin{theorem}\label{thm:expspace}
The reachability problem in \tvass is in \twoexpspace, under binary encoding.
\end{theorem}

\noindent
In particular, this refutes the natural conjecture that for every $d \geq 3$, the reachability problem for 
\parvass d is $\F_d$-complete and provides the first algorithm, which solves the problem
for \vass with finite reachability sets faster than the exhaustive search.

%\smallskip

Our way to prove Theorem~\ref{thm:expspace} is by bounding triply-exponentially 
the length of the shortest path between given source and target configurations.
This main technical result, formulated in Lemma~\ref{lem:main} below,
applies to \emph{sequential} \tvass, being a sequence of strongly connected components
$V_1, \ldots, V_k$ linked by single transitions $u_1, \ldots, u_{k-1}$ 
(see Figure \ref{fig:kcompvass}; rigorous definition is given in Section \ref{sec:def}).

\newcommand{\labelscvass}[1]{{\tiny\begin{tabular}{c}\tiny strongly\\ \tiny connected\\ \tiny \tvass $V_{#1}$\end{tabular}}}
%
\begin{figure}
\scalebox{0.75}{
\begin{tikzpicture}[shorten >=1pt,node distance=3cm,on grid,>={Stealth[round]},
    every state/.style={draw=blue!50,thick,fill=blue!20},scale=0.25]

  \node[state]          (q_0)                      {\labelscvass 1};
  \node[state]          (q_1) [right=of q_0] {\labelscvass 2};
  \node[]                  (q_ldots) [right=of q_1] {\ \ \ \ \ \ ... \ \ \ \ \ \ };
  \node[state]          (q_k) [right=of q_ldots] {\labelscvass k};

  \path[->] 
            (q_0) edge              node [above]  {$u_1$} (q_1)
            (q_1) edge              node [above]  {$u_2$} (q_ldots)
            (q_ldots) edge              node [above]  {$u_{k-1}$} (q_k);
\end{tikzpicture}
}
\caption{A sequential \tvass.}
\label{fig:kcompvass}
\end{figure}

%Below, $\norm(v)$ stands for the sum of absolute values of all numbers
%appaering in a vector $v \in \N^3$.
Given a \vass $V$ and source and target configurations $s, t$, by
%\emph{size} of a \vass $V$, denoted 
$\size(V,s,t)$ we mean the sum of absolute values of all the numbers occurring in transitions of $V$, $s$ and $t$, 
plus the number thereof.

\begin{lemma}\label{lem:main}
%Let $k\in\N\setminus\{0\}$.
If there is a path from $s$ to $t$ in a sequential \tvass $V$,
then there is one of length at most 
$\kbound {\size(V,s,t)} {\OO(k)}$, where $k$ is the number of components of $V$.
\end{lemma}

%\wojtek{Maybe change the $\OO$ notation.}

\noindent
Therefore the length of the shortest path in a $k$-component \tvass is bounded by $\kbound M {\OO(k)}$,
where $M = \size(V,s,t)$ is the size of input, under unary encoding.
This is the first bound on the shortest path in \vass of dimension higher than 2, that does not base on
the size of (finite) reachability sets.
Indeed, in a \tvass of size $M$, the size of finite reachability sets may be an arbitrary high tower of exponentials.
%and in \parvass d it may be arbitrarily close to $F_d(M)$.%
%\footnote{Function $F_i$ is the $i$-th level of Grzegorczyk's fast-growing hierarchy.}

In consequence of Lemma \ref{lem:main}, 
Theorem~\ref{thm:expspace} follows immediately: 
the upper bound of Lemma \ref{lem:main} 
is triple-exponential in the size of input, irrespectively whether unary or  binary encoding is used, 
which implies the same bound on norms of configurations
along the shortest path.
This yields a nondeterministic double-exponential space algorithm that
first guesses a sequence of components leading from the source state to the target one,
and then searches for a witnessing path.
Note that the complexity bound under unary encoding is not better than under binary encoding. 


%\smallskip

Lemma \ref{lem:main} immediately yields further upper bounds for the reachability problem,
when the number of components is fixed:

\begin{corollary}\label{cor:fixed-components}
For every fixed $k \geq 1$, the reachability problem in $k$-component \tvass is:
\begin{itemize}
  \item in \nl, under unary encoding,
  \item in \pspace, under binary encoding.
\end{itemize}
\end{corollary}

\noindent
Indeed, for every fixed $k\geq 1$, the bound of Lemma~\ref{lem:main} is polynomial with respect to
unary input size.
Therefore the length of the shortest path, as well as the norm of configurations along this path, are 
polynomially bounded in case of unary encoding,
and exponentially bounded in case of binary encoding.
This bounds yield membership in \nl and \pspace, respectively.
Thus for every fixed $k\geq 1$, the complexity of $k$-component \tvass matches the complexity of \dvass.

%The proof of Lemma \ref{lem:main} proceeds by induction on the number $k$ of components,
%i.e., we infer a bound $B_k$ on the length of a shortest path for $k$-component \tvass
%from a bound $B_{k-1}$ for $(k-1)$-component \tvass.
%%In some cases, when a \tvass is \emph{diagonal}, i.e., when it can simultaneously increase all the coordinates unboundedly,
%%and additionally the set of effects of cycles is rich enough, then one can show that the shortest path is of at most exponential length, without
%%the use of induction.
%Our principal strategy is to replace the first component by a \tvass that is essentially 
%2-dimensional,
%and to exploit semi-linear reachability sets of \dvass.
%This sole strategy is however not sufficient: 
%since bases and periods of these semi-linear sets may be exponential, 
%$B_k$ would be exponentially larger than $B_{k-1}$.
%In consequence, with
%$B_k$ as large as a $k$-fold tower of exponentials,
%the upper complexity bound would be only \tower
%(thus not better than \cite{DBLP:conf/icalp/FuYZ24}).
%
%A crucial innovatory ingredient of our approach is resigning from considering the whole reachability sets 
%in \dvass,
%and considering their suitable over- and under-approximations instead.
%%It is known that the reachability set of a \dvass is semi-linear, i.e., a finite union of linear sets,
%%but this is no more true in \tvass, even in 1-component \tvass.
%For proving Lemma \ref{lem:main} we invent a novel concept of \emph{\sandwich} sets,
%which are approximated from above by a semi-linear set $L$, from below
%by a sufficiently large finite prefix of $L$, and bases and periods of $L$ are only polynomial.
%Intuitively speaking, the idea amounts to trading precision for polynomial size,
%while  keeping the approximations good enough for correctness of the inductive proof.
%Our core technical result states that reachability sets of \dvass are \sandwich,
%which allows us to proceed with the induction step.

\para{Organisation of the paper}
We start by introducing notation and basic facts in Section~\ref{sec:def}.
Overview of the proof of our main result, Lemma \ref{lem:main}, is presented in Section~\ref{sec:overview}.
In Section \ref{sec:1comp} we focus on 1-component \tvass, thus establishing the base of induction
for Lemma \ref{lem:main}.
Next, in Section \ref{sec:tools} we introduce the fundamental concept of \sandwich sets,
and formulate our core technical result: reachability sets of \dvass
are \sandwich.
The result is then applied in the inductive proof of Lemma \ref{lem:main} in Section \ref{sec:mainproof}.
We conclude in Section~\ref{sec:future}.
All the missing proofs are delegated to Appendix.

%~\ref{sec:overview} we present an extensive overview of our approach and formulate main technical statements.
%We show the main Lemma~\ref{lem:main} in Section~\ref{sec:mainproof} using Lemmas~\ref{lem:diagonal}~and~\ref{lem:sandwich}.
%Then we prove Lemmas~\ref{lem:diagonal}~and~\ref{lem:sandwich} respectively in Sections~\ref{sec:diagonal}~and~\ref{sec:sandwich}.
%Finally, we remark on possible interesting future directions in Section~\ref{sec:future}.













