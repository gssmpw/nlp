\documentclass{article}

\usepackage{arxiv}
\usepackage{papercontent}
\usepackage[utf8]{inputenc} % allow utf-8 input
\usepackage[T1]{fontenc}    % use 8-bit T1 fonts
\usepackage{hyperref}       % hyperlinks
\usepackage{url}            % simple URL typesetting
\usepackage{booktabs}       % professional-quality tables
\usepackage{amsfonts}       % blackboard math symbols
\usepackage{nicefrac}       % compact symbols for 1/2, etc.
\usepackage{microtype}      % microtypography
\usepackage{cleveref}       % smart cross-referencing
\usepackage{lipsum}         % Can be removed after putting your text content
\usepackage{graphicx}
\usepackage{natbib}
\usepackage{doi}
\usepackage{tcolorbox}

\title{\papertitle}

% Here you can change the date presented in the paper title
%\date{September 9, 1985}
% Or remove it
\date{}

\newif\ifuniqueAffiliation
% Comment to use multiple affiliations variant of author block 
\uniqueAffiliationtrue

\ifuniqueAffiliation % Standard variant of author block
\author{ 
	{\hspace{1mm}\paperauthor} \\
	%%\texttt{\paperauthoremail} \\
	%%\AND
	%%{\hspace{1mm}\paperauthorsec} \\
	%%\texttt{\paperauthoremailsec} \\
}
\else
% Multiple affiliations variant of author block
\usepackage{authblk}
\fi

% Uncomment to override  the `A preprint' in the header
%\renewcommand{\headeright}{Technical Report}
%\renewcommand{\undertitle}{Technical Report}
%\renewcommand{\shorttitle}{\textit{arXiv} Template}

%%% Add PDF metadata to help others organize their library
%%% Once the PDF is generated, you can check the metadata with
%%% $ pdfinfo template.pdf
\hypersetup{
pdftitle={\papertitle},
pdfsubject={cs},
pdfauthor={\paperauthor},
pdfkeywords={\paperkeywordone, \paperkeywordtwo, \paperkeywordthird, \paperkeywordfour, \paperkeywordfive},
}

\begin{document}
\maketitle

%% abstract 
\begin{abstract}
	\paperabstract
\end{abstract}


% keywords can be removed
\keywords{\paperkeywordone \and \paperkeywordtwo \and \paperkeywordthird \and \paperkeywordfour \and \paperkeywordfive}

\section{Introduction}
\paperintroduction

\section{Related work}
\paperrelatedwork

\begin{figure}
	\centering
	\includegraphics[width=0.5\textwidth]{pictures/evidence.png}
	\caption{Time of using the built-in chain of thought.}
	\label{fig:fig1}
\end{figure}

\section{Method and experiment details}
\subsection{Overview}
\papermethodandexperimentoverview

\subsection{Thinking method based on built-in chain of thought}
\subsubsection{Thinking method}
% \label{sec:thinkingmethod}
\papermethodandexperimentthinkingmethod
\begin{figure}
	\centering
	\includegraphics[width=0.5\textwidth]{pictures/thinking}
	\caption{The thinking process of the thinking method.}
	\label{fig:fig2}
\end{figure}

\papermethodandexperimentthinkingprocess

\papermethodandexperimentbuiltincots

\papermethodandexperimentthinkingelements

\papermethodandexperimentthinkingdetails

\subsubsection{Action calls and Built-in Action calls}
\label{sec:actioncalls}
\papermethodandexperimentactioncalls

\begin{figure}
	\centering
	\includegraphics[width=\textwidth]{pictures/global_local_thinking}
	\caption{Global thinking context and Local thinking context example.}
	\label{fig:fig3}
\end{figure}

\subsubsection{Global thinking context and Local thinking context}
\label{sec:thinkingcontext}
\papermethodandexperimentthinkcontext


\subsubsection{Security}
\papermethodandexperimentsecurity

\begin{figure}
	\centering
	\includegraphics[width=\textwidth]{pictures/dialog_cake.eps}
	\caption{Example of Action tasks dataset.}
	\label{fig:fig4}
\end{figure}
\subsection{Action tasks dataset collection}
\papermethodandexperimentactiondataset

\subsection{Supervised Learning fine-tuning}
\papermethodandexperimentsft

\subsection{Reinforcement Learning fine-tuning}
\subsubsection{Collect samples from policies}
\label{sec:collectconsistencysamples}
\papermethodandexperimentcollectconsistencysamples

\subsubsection{Train a consistency reward model}
\papermethodandexperimenttrainrewardmodel

\subsubsection{Optimize the policy according to the consistency reward model}
\papermethodandexperimentoptimizepolicy

\subsection{Reinforcement Learning for all Tasks}
\papermethodandexperimentalltasks

\begin{table}
	\caption{Tasks and Reward function}
	\centering
    \begin{tabular}{|l|c|}
        \hline
        Tasks & Reward function \\
        \hline
        Action tasks & format reward + consistency reward \\
        \hline
        Reasoning tasks & format reward + rule-based reward \\
        \hline
        Other tasks & format reward + human preference reward \\
        \hline
    \end{tabular}
    \label{tab:tasksandrewardfunction}
\end{table}

\section{Results}
\paperresults

\section{Discussion}
\paperdiscussion

%%% 引用
\bibliographystyle{unsrtnat}
\bibliography{references} 

\end{document}
