\documentclass[11pt,a4paper,english]{article}
\usepackage[utf8]{inputenc} % allow utf-8 input
\usepackage[T1]{fontenc}    % use 8-bit T1 fonts
\usepackage{babel}
\usepackage{blindtext}
\usepackage{hyperref}       % hyperlinks
\usepackage{url}            % simple URL typesetting
\usepackage{booktabs}       % professional-quality tables
\usepackage{amsfonts}       % blackboard math symbols
\usepackage{nicefrac}       % compact symbols for 1/2, etc.
\usepackage{microtype}      % microtypography
\usepackage{xcolor}         % colors
\usepackage{amsmath,amssymb}
\usepackage{wrapfig}
\usepackage{stfloats}
\usepackage{graphicx}
\usepackage{placeins}  
\usepackage{float}    
\usepackage{caption}
\usepackage{subcaption}
\usepackage{bm}
\usepackage{listings}
\usepackage{multicol}
\usepackage{multirow} 
\usepackage{pgffor}
\usepackage{cleveref}
\usepackage{geometry}
%\usepackage{ulem}

\newcommand{\e}{\mathbb{E}}
\newcommand{\lk}{\left[ }
\newcommand{\rk}{\right] }
\newcommand{\lc}{\left(}
\newcommand{\rc}{\right)}
\newcommand{\Eb}{\mathbb{E}}
\newcommand{\Cb}{\mathbb{C}}
\newcommand{\Rb}{\mathbb{R}}
\newcommand{\Zb}{\mathbb{Z}}
\newcommand{\Pb}{\mathbb{P}}
\newcommand{\Qb}{\mathbb{Q}}
\newcommand{\Nb}{\mathbb{N}}
\newcommand{\Fc}{\mathcal{F}}
\newcommand{\mc}{\mathcal{m}}
\newcommand{\Hc}{\mathcal{H}}
\newcommand{\id}{\mathbf{1}}
\newcommand{\veps}{\mathbf{\epsilon}}
\newcommand{\bb}{\mathbf{b}}
\newcommand{\zz}{\mathbf{z}}
\newcommand{\vv}{\mathbf{v}}
\newcommand{\xx}{\mathbf{x}}
\newcommand{\Ab}{\mathbf{A}}
\newcommand{\FF}{\mathbf{F}}
\newcommand{\Nc}{\mathcal{N}}
\newcommand{\Sc}{\mathcal{S}}
\newcommand{\Uc}{\mathcal{U}}
\newcommand{\Lc}{\mathcal{L}}

\newcommand{\myparagraph}[1]{\paragraph{#1}\mbox{}\\}

\NewDocumentCommand{\codeword}{v}{%
\texttt{\textcolor{blue}{#1}}%
}


\title{Recurrent Neural Networks for Dynamic VWAP Execution: Adaptive Trading Strategies with Temporal Kolmogorov-Arnold Networks}

\author{%
    Rémi Genet \\
    \small DRM, Université Paris Dauphine - PSL \\
    \small Aplo \\
    \small remi.genet@dauphine.psl.eu \\
}

\begin{document}

\maketitle

\begin{abstract}
The execution of Volume Weighted Average Price (VWAP) orders remains a critical challenge in modern financial markets, particularly as trading volumes and market complexity continue to increase. In my previous work \cite{genet2025staticvwap}, I introduced a novel deep learning approach that demonstrated significant improvements over traditional VWAP execution methods by directly optimizing the execution problem rather than relying on volume curve predictions. However, that model was static because it employed the fully linear approach described in \cite{genet2024tln}, which is not designed for dynamic adjustment. This paper extends that foundation by developing a dynamic neural VWAP framework that adapts to evolving market conditions in real time. We introduce two key innovations: first, the integration of recurrent neural networks to capture complex temporal dependencies in market dynamics, and second, a sophisticated dynamic adjustment mechanism that continuously optimizes execution decisions based on market feedback. The empirical analysis, conducted across five major cryptocurrency markets, demonstrates that this dynamic approach achieves substantial improvements over both traditional methods and our previous static implementation, with execution performance gains of 10–15\% in liquid markets and consistent outperformance across varying conditions. These results suggest that adaptive neural architectures can effectively address the challenges of modern VWAP execution while maintaining computational efficiency suitable for practical deployment.
\end{abstract}

\newpage

\section{Introduction}
The execution of large trading orders in financial markets is a complex and strategic challenge, with the potential to significantly impact both transaction costs and overall market dynamics. In this context, the concept of Volume Weighted Average Price (VWAP) has emerged as a key benchmark and execution strategy, offering market participants a robust framework for minimizing market impact while closely tracking average traded prices over a specified period of time. The practical and theoretical importance of VWAP execution has grown with the rapid evolution of electronic and algorithmic trading platforms. As documented by Mackenzie \cite{Mackenzie}, algorithmic trading now accounts for a substantial majority of institutional order flow. VWAP strategies represent a key component of this automated execution landscape. Despite this practical prominence, however, the academic literature has historically placed greater emphasis on alternative execution benchmarks such as Implementation Shortfall (IS) \cite{perold1988implementation}, opening a door for research in the design and performance of VWAP strategies. Initially, VWAP execution aims to address two fundamental objectives. First, by distributing an order over time and closely tracking the average traded price, VWAP strategies seek to minimize the market impact of large trades - a key component of overall transaction costs as established by Berkowitz et al. \cite{TotalCostOfTransactions}. Second, by targeting a pre-defined benchmark, VWAP provides a transparent and objective measure of the execution quality. This metric is crucial  for institutional investors who must justify their trading performance to stakeholders.

\subsection{VWAP Definition and Discretization}

Following the seminal work of Konishi \cite{Konishi}, the Volume Weighted Average Price (VWAP) over a time period \([0,T]\) is defined as:
\begin{equation}
    \text{VWAP}_{[0,T]} = \frac{\int_0^T P(t)V(t)\,dt}{\int_0^T V(t)\,dt},
\end{equation}
where \(P(t)\) and \(V(t)\) denote the price and volume at time \(t\), respectively. As noted by McCulloch and Kazakov \cite{Culoch2007}, financial markets operate in discrete time intervals, which leads to the discretized form:
\begin{equation}
    \text{VWAP}_{[0,T]} = \frac{\sum_{t=1}^{T} P_t\,V_t}{\sum_{t=1}^{T} V_t},
\end{equation}
where \(P_t\) and \(V_t\) represent the price and volume in the \(t\)-th time interval.

For a trader executing a large order of total size \(Q\), Humphery-Jenner \cite{Humphery} shows that the objective is to minimize the difference between the achieved execution price and the market VWAP. Let \(q_t\) denote the quantity traded in interval \(t\), so that:
\begin{equation}
\sum_{t=1}^{T} q_t = Q.
\end{equation}
The execution price is then given by:
\begin{equation}
P_{\text{exec}} = \frac{\sum_{t=1}^{T} P_t\,q_t}{Q}.
\end{equation}

Following Bialkowski et al. \cite{LeFol2006}, the VWAP execution problem can be formulated as minimizing the slippage:
\begin{equation}
\min_{q_1,\ldots,q_T} \left|\frac{\sum_{t=1}^{T} P_t\,q_t}{Q} - \frac{\sum_{t=1}^{T} P_t\,V_t}{\sum_{t=1}^{T} V_t}\right|.
\end{equation}

For clarity, the normalized order allocation is defined as \(\tilde{q}_t = \frac{q_t}{Q}\) (so that \(\sum_{t=1}^{T} \tilde{q}_t = 1\)) and the normalized market volume profile as \(\tilde{V}_t = \frac{V_t}{\sum_{t=1}^{T} V_t}\) (with \(\sum_{t=1}^{T} \tilde{V}_t = 1\)). With these definitions, the execution price becomes:
\begin{equation}
P_{\text{exec}} = \sum_{t=1}^{T} P_t\,\tilde{q}_t,
\end{equation}
and the market VWAP is:
\begin{equation}
\text{VWAP} = \sum_{t=1}^{T} P_t\,\tilde{V}_t.
\end{equation}

Genet \cite{genet2025staticvwap} reformulates the slippage as a bound:\begin{equation}
    S_T \le \sum_{t=1}^{T} \left| \bigl(P_t-\text{VWAP}_t\bigr)\tilde{q}_t \right| + \sum_{t=1}^{T} \left| \text{VWAP}_t\Bigl(\tilde{q}_t-\tilde{V}_t\Bigr) \right|,
\end{equation}
where \(\text{VWAP}_t\) denotes the market VWAP computed over interval \(t\). The first term quantifies the impact of price deviations weighted by the trader's participation rate, while the second term captures the error due to discrepancies between the trader's normalized allocation \(\tilde{q}_t\) and the market's volume fraction \(\tilde{V}_t\). Under the volume conservation constraint and non-negativity constraints \(\tilde{q}_t \geq 0\), this decomposition separates the overall slippage \(S_T\) into a price deviation component and a volume allocation error component. This optimization problem is particularly challenging because future prices and volumes are unknown at execution time, which necessitates accurate predictions of market dynamics while managing execution risk \cite{frei}. Furthermore, the increasing sophistication of market participants and the rising prominence of transaction cost analysis (TCA) in institutional trading \cite{Madhavan2002} have amplified the importance of VWAP execution strategies. As markets evolve, executing large orders while minimizing market impact becomes increasingly complex, thereby requiring more advanced approaches to VWAP execution.

\subsection{Classical VWAP Approaches}

The theoretical foundation of VWAP execution strategies emerged from a series of seminal works that established the mathematical framework for optimal order execution. Konishi \cite{Konishi} provided one of the first comprehensive analyses of VWAP strategies, demonstrating that in markets where volume and volatility are uncorrelated, the optimal execution curve mirrors the expected relative market volume curve. He further extended his analysis to cover the case where volume and volatility are correlated. This breakthrough provided mathematical validation for practitioners' empirical observations and established a theoretical benchmark for subsequent research in the field. Building on this foundation, McCulloch and Kazakov \cite{Culoch2007} developed a more sophisticated model that incorporated practical constraints and information asymmetries. Their work introduced two crucial elements: constrained trading rates and potential information advantages, acknowledging that traders or brokers might possess sensitive information that could influence their attempts to outperform the VWAP benchmark. Their research also revealed important stylized facts about expected relative volume patterns—most notably, the characteristic S-shape observed in equity markets and the finding that higher-turnover stocks exhibit less variation in their expected relative volume. These patterns echoed the well-documented "U" shape effect in equity market trading activity, thereby providing a crucial link between microstructure theory and practical execution strategies.

\medskip

The relationship between volumes and other market variables has been extensively studied in the literature. Notable contributions include the work of Easley and O'Hara \cite{Easley1987}, Viswanathan and Foster \cite{Foster}, Tauchen and Pitts \cite{Tauchen1983}, and Karpoff \cite{Karpoff1987}, who examined volumes as covariates in analyzing and explaining target variables such as price and volatility, though primarily focusing on low-frequency data rather than intraday patterns. Gourieroux et al. \cite{Gourieroux} made significant contributions to the measurement of market trading activity, providing a theoretical framework for understanding trading volume dynamics. McCulloch and Kazakov \cite{Culoch2012} further extended this line of research by transforming Konishi's fixed model into a continuous dynamic framework. This work established the crucial connection between optimal VWAP trading strategies and accurate intraday volume estimation, demonstrating that successful execution depends fundamentally on the ability to anticipate and adapt to volume patterns throughout the trading day.

\medskip

The evolution of these classical approaches reflects a growing recognition of the complexity inherent in VWAP execution. While these models provided valuable insights and theoretical foundations, they also revealed the limitations of purely static approaches in capturing the dynamic nature of modern markets. This recognition would eventually lead to the development of more sophisticated dynamic approaches and, ultimately, to the application of machine learning techniques in VWAP execution strategies.

\subsection{Dynamic Volume Approaches}

A significant paradigm shift in VWAP execution strategies occurred with the introduction of dynamic volume estimation approaches. Bialkowski et al. \cite{LeFol2006} pioneered this advancement by proposing a novel method for estimating intraday volumes through component decomposition. Their work, later refined in Bialkowski et al. \cite{LeFol2012}, separated volume patterns into two distinct components: one reflecting broader market evolution and another capturing stock-specific patterns. This decomposition enabled more accurate volume predictions by modeling the dynamic component using ARMA and SETAR models, demonstrating substantially improved accuracy compared to traditional static approaches. However, transitioning from simplistic volume modeling to these more advanced methods comes at a cost: such approaches no longer explicitly account for the volume–volatility relationship, as it becomes much more challenging to realistically incorporate both components simultaneously. The shift from static to dynamic approaches was further advanced by Humphery-Jenner \cite{Humphery}, who introduced the concept of Dynamic VWAP (DVWAP) in contrast to the traditional Historical VWAP (HVWAP). Their research highlighted a crucial limitation of historical approaches—their inability to incorporate real-time market information during execution. By developing a framework that adapts to incoming news and market developments, they demonstrated significant improvements over historical methods in both basic VWAP tracking and the management of market dynamics.

\medskip

Alternative theoretical perspectives emerged through the work of Bouchard and Dang \cite{bouchard} and Frei and Westray \cite{frei}, who approached VWAP execution through the lens of stochastic analysis. As Frei and Westray \cite{frei} noted, their derived optimal trading rates depended primarily on volume curves rather than price processes, reflecting the assumption of uncorrelated Brownian motion in price movements. This theoretical framework provided valuable insights into the relationship between volume patterns and execution strategy, even as it highlighted the limitations of purely stochastic approaches. A significant contribution to the practical aspects of VWAP execution came from Carmona and Li \cite{Tianhui}, who examined the strategic considerations at both macro and micro scales. Their research was particularly notable for addressing the practical dilemma faced by brokers in choosing between aggressive and passive orders at the high-frequency level, bringing theoretical insights to bear on practical execution decisions. Guéant and Royer \cite{Gueant} made two crucial contributions that addressed previously understudied aspects of VWAP execution. First, they incorporated a comprehensive market impact model that considered both temporary and permanent effects, addressing a critical concern for institutional investors using VWAP orders to manage large positions. Second, they developed a framework for pricing guaranteed VWAP services using CARA utility functions and indifference pricing. This work represented a significant shift from traditional approaches focused solely on benchmark tracking, introducing a more nuanced understanding of risk-adjusted optimal execution. These dynamic approaches collectively highlighted a crucial insight: while modeling market volumes is important, the assumption of independence between prices and volumes often fails to reflect market reality. This recognition, combined with the increasing availability of computational power and market data, set the stage for the application of more sophisticated analytical techniques, particularly in the domain of machine learning and artificial intelligence.

\subsection{The Rise of Deep Learning in Financial Time Series}
In parallel with these theoretical advances, the field of machine learning has witnessed a rapid development of powerful techniques and architectures, particularly in the domain of deep learning. The field of time series analysis and prediction has been fundamentally transformed by developments in deep learning, particularly in the domain of neural networks. As documented by Sezer et al. \cite{sezer2020financial} in their comprehensive review, deep learning models have increasingly outperformed traditional machine learning approaches across various financial forecasting tasks. The evolution of deep learning architectures for financial applications has been marked by several key innovations. The introduction of Long Short-Term Memory (LSTM) networks by Hochreiter and Schmidhuber \cite{hochreiter1997} addressed the vanishing gradient problem that had limited traditional recurrent neural networks, enabling effective learning of long-term dependencies in sequential data. This was followed by the development of Gated Recurrent Units (GRU) by Cho et al. \cite{cho2014}, offering comparable performance with a more streamlined architecture. A revolutionary step forward came with the introduction of attention mechanisms Bahdanau et al. \cite{bahdanau2014neural}, culminating in the Transformer architecture Vaswani et al. \cite{vaswani2017attention}. While initially developed for natural language processing, these architectures' ability to capture both local and global dependencies in sequential data made them particularly suitable for financial time series analysis.

\medskip

During the last decade, we have seen an explosion of deep learning applications in finance, with researchers tackling increasingly complex challenges. Ackerer et al. \cite{ackerer2020deep} demonstrated the power of neural networks in fitting and predicting implied volatility surfaces, while Horvath et al. \cite{horvath2019deep} showed how deep learning could revolutionize pricing and calibration in volatility models. As highlighted by Zhang et al. \cite{zhang2023deep} in their recent review, deep learning models are gradually replacing traditional statistical and machine learning models as the preferred choice for price forecasting tasks. In the specific domain of trading volume prediction, significant advances have been made through the development of specialized architectures such as Temporal Kolmogorov-Arnold Networks (TKAN) \cite{genet2024tkan}, Signature-Weighted Kolmogorov-Arnold Networks (SigKAN) \cite{inzirillo2024sigkan}, Temporal Kolmogorov-Arnold Transformers (TKAT) \cite{genet2024tkat}, Kolmogorov-Arnold Mixture of Experts (KAMoE) \cite{inzirillo2024kamoe} and Recurrent Neural Networks with Signature-Based Gating Mechanisms (SigGate) \cite{genet2025siggate}. 

\subsection{Deep Learning Approaches to Market Execution}

The application of deep learning to market execution problems has evolved significantly in recent years. Early approaches focused primarily on using neural networks for price prediction or simple trading signals. However, the complexity of VWAP execution, with its intricate relationship between volume patterns, price impact, and timing decisions, presents unique challenges that require more sophisticated approaches. Recent research has begun to explore more advanced applications of deep learning to execution problems. Papanicolaou et al. \cite{papanicolaou2023optimal} demonstrated the effectiveness of using LSTMs for large order execution within the Almgren and Chriss framework, showing how deep learning models could capture cross-sectional relationships between different stocks' execution characteristics. While not specifically focused on VWAP execution, this work highlighted the potential for neural networks to learn complex relationships in market impact and execution timing. One particularly promising development has been the recent introduction of Temporal Kolmogorov-Arnold Networks (TKAN) Genet and Inzirillo \cite{genet2024tkan}. This architecture combines the representational power of Kolmogorov-Arnold Networks with sophisticated temporal processing capabilities, demonstrating exceptional performance specifically in cryptocurrency volume prediction tasks. The success of TKANs in volume prediction suggests their potential applicability to the broader challenge of VWAP execution optimization.

\subsection{From Static to Dynamic Neural VWAP}
My previous research Genet \cite{genet2025staticvwap} established a novel approach to VWAP execution by leveraging deep learning techniques in a fundamentally different way from existing methods. Rather than focusing on volume curve prediction like traditional approaches, I demonstrated that directly optimizing the execution strategy through neural networks could significantly improve performance. Using a Temporal Linear Network (TLN), this static approach showed particular effectiveness in handling market uncertainty and extreme events, consistently outperforming conventional methods across various market conditions. However, the inherent limitations of static approaches become particularly apparent in highly volatile markets such as cryptocurrencies, where market conditions can change dramatically within a single execution window. The success of TKANs in cryptocurrency volume prediction, combined with these limitations of static approaches, suggests a natural evolution toward a more dynamic framework. This observation aligns with earlier findings from Bialkowski et al. \cite{LeFol2012} and Humphery-Jenner \cite{Humphery} about the importance of adapting to changing market conditions, but approaches the problem with the enhanced capabilities offered by modern deep learning architectures.

\medskip

In this paper, I propose a dynamic VWAP execution framework that represents a significant advancement in several key aspects:
First, it maintains the robust foundation of our static approach while incorporating adaptive capabilities through recurrent neural networks. This design choice allows our model to preserve the reliable performance characteristics that made the static approach successful while adding the flexibility to adjust execution strategies based on evolving market conditions. Second, our framework leverages the TKAN architecture, which has demonstrated superior performance in volume prediction tasks Genet and Inzirillo \cite{genet2024tkan}. By applying this architecture to execution strategy rather than just volume prediction, I extend its capabilities to address the full complexity of VWAP execution, including market impact considerations and optimal timing decisions. Third, I address a fundamental limitation of existing approaches by directly incorporating market feedback into our execution decisions. This design creates a more responsive system that can adapt to changing market conditions while maintaining the execution constraints necessary for effective VWAP targeting.

\section{Dynamic VWAP Architecture}

The proposed dynamic VWAP execution framework extends previous work on static neural VWAP optimization. By integrating recurrent neural networks and introducing a novel sequential optimization mechanism, the framework adapts to evolving market conditions while preserving the key principles of VWAP benchmarking. This section provides a detailed overview of the model architecture, including its main components, design considerations, and implementation details.



\begin{figure}[H]
    \centering
    \includegraphics[width=\linewidth]{figures/dynamic_vwap.drawio.jpg}
    \caption{Overview of the dynamic VWAP execution architecture.}
    \label{fig:dynamic_model_overview}
\end{figure}

\subsection{Model Overview}

At its core, the dynamic VWAP framework processes a sequence of market observations to generate optimal execution trajectories in real time. Let \(x_t \in \mathbb{R}^d\) denote the input features at time \(t\), where \(d\) is the feature dimension. Given a lookback period \(l\) and a prediction horizon \(h\), the model operates on input sequences of length \(l + h - 1\), thereby incorporating both historical context and market information received during execution. The architecture consists of three key components: a learnable base volume curve, a recurrent neural network for dynamic adaptation, and a volume adjustment mechanism. These components work together to generate execution trajectories that minimize VWAP slippage while satisfying practical trading constraints.

\subsection{Base Volume Curve}
The foundation of the model is a learnable base volume curve \(v_b \in \mathbb{R}^h\), initialized uniformly as:
\begin{equation}
    v_b^{(0)} = \left\{\frac{1}{h}, \frac{1}{h}, \dots, \frac{1}{h}\right\}.
\end{equation}
This curve is constrained to satisfy two key properties:
\begin{equation}
\sum_{i=1}^h v_b^{(i)} = 1, \quad v_b^{(i)} \geq 0 \quad \forall i \in \{1, \dots, h\}.
\end{equation}

The inclusion of a learnable base curve is motivated by the findings of the static VWAP research Genet \cite{genet2025staticvwap}, which demonstrated the robustness and effectiveness of fixed volume profiles in minimizing execution slippage. By allowing the base curve to adapt during training, the model can potentially discover superior static profiles that serve as the foundation for further dynamic adjustments.

\subsection{Dynamic Volume Adjustment}

The dynamic component begins with a recurrent neural network $\mathcal{R}$ that processes the input sequence to produce hidden states:
\begin{equation}
    h_t = \mathcal{R}(x_t, h_{t-1}),
\end{equation}
where $h_t \in \mathbb{R}^m$ represents the hidden state at time $t$ with dimension $m$.
Note that $h_t$ depends solely on information available up to time $t$, ensuring causality by using both the previous state and the new information at time $t$.

\subsubsection{RNN Architecture Selection}

The recurrent component \(\mathcal{R}\) can be implemented using different RNN variants. Two approaches are considered: Long Short-Term Memory (LSTM) and Temporal Kolmogorov-Arnold Networks (TKAN). LSTM serves as a well-established baseline with proven capability in handling temporal dependencies, while TKAN provides specialized memory management particularly suited for volume forecasting tasks.

\subsection*{LSTM Architecture}

The LSTM unit processes input vector $x_t \in \mathbb{R}^d$ through a series of gates that control information flow. The architecture maintains two types of state: the cell state $c_t$ and the hidden state $h_t$. The key components are the forget gate, which determines what information to discard from the previous state:
\begin{equation}
    f_t = \sigma(W_f x_t + U_f h_{t-1} + b_f),
\end{equation}
The input gate, controlling the integration of new information:
\begin{equation}
    i_t = \sigma(W_i x_t + U_i h_{t-1} + b_i),
\end{equation}
The cell state update, combining previous and new information:
\begin{equation}
    c_t = f_t \odot c_{t-1} + i_t \odot \tilde{c}_t,
\end{equation}
where $\tilde{c}_t = \tanh(W_c x_t + U_c h_{t-1} + b_c)$ represents candidate cell state values.

\subsection*{TKAN Architecture}
The TKAN architecture represents a more specialized approach, developed specifically for volume forecasting tasks. It implements a dual memory mechanism by combining Recurring Kolmogorov-Arnold Networks (RKAN) with additional memory management components. This structure enables sophisticated temporal pattern recognition through hierarchical processing.

At its core, RKAN layers process the input through layer-specific memory states. For each layer $l$, the input transformation is:
\begin{equation}
    s_{l,t}=W_{l,\tilde{x}} x_t + W_{l,\tilde{h}} \tilde{h}_{l,t-1}
\end{equation}
where $W_{l,\tilde{x}} \in \mathbb{R}^{d \times \text{KAN}_{in}}$ and $W_{l,\tilde{h}} \in \mathbb{R}^{\text{KAN}_{out} \times \text{KAN}_{in}}$ transform the current input and the previous sub-state, respectively. The RKAN layer output is computed as:

\begin{equation}
    \tilde{o}_{t} = \phi_{l}(s_{l,t}),
\end{equation}
where $\phi_l$ represents the KAN layer transformation.
The sub-layer memory state is updated according to:
\begin{equation}
    \tilde{h}_{l,t} = W_{hh} \tilde{h}_{l,t-1} + W_{hz} \tilde{o}_{t}.
\end{equation}
The full TKAN combines these RKAN outputs through:
\begin{equation}
    r_t = \text{Concat}[\phi_1(s_{1,t}),\phi_2(s_{2,t}),...,\phi_L(s_{L,t})].
\end{equation}
The output gate then processes this concatenated representation:
\begin{equation}
    o_t = \sigma(W_{o}r_t + b_o),
\end{equation}
where $W_{o} \in \mathbb{R}^{(\text{KAN}_{out} \cdot L,out)}$. The final hidden state update is given by:
\begin{equation}
    h_t = o_t \odot \tanh(c_t).
\end{equation}

This architecture implements two levels of memory management: the RKAN sub-layer memory states ($\tilde{h}_{l,t}$) capture local temporal patterns within each layer, while the global cell state ($c_t$) and hidden state ($h_t$) maintain longer-term dependencies across the entire network. This dual memory mechanism enables our model to effectively capture both fine-grained market dynamics and longer-term trading patterns, providing a rich representation for subsequent volume adjustments.

\subsection{Dynamic Volume Adjustment Network}

The core innovation of the dynamic VWAP framework is the sequential volume adjustment mechanism, which translates temporal patterns from the recurrent network into optimal execution trajectories. For each time step in the prediction horizon, a separate transformation function \(f_i\) (with \(i \in \{1, \dots, h-1\}\)) is employed, implemented as a multi-layer perceptron. For \(t > 0\), the function processes an augmented input that concatenates the RNN hidden state with the history of previous volume decisions:
\begin{equation}
    f_i(h_t, v_{1:t-1}) = W^{(3)}_i \cdot \text{ReLU}(W^{(2)}_i \cdot \text{ReLU}(W^{(1)}_i [h_t; v_{1:t-1}] + b^{(1)}_i) + b^{(2)}_i) + b^{(3)}_i,
\end{equation}
where $[h_t; v_{1:t-1}]$ represents the concatenation of the hidden state $h_t$ with the sequence of previously allocated volumes $v_{1:t-1}$. For the initial timestep ($t=0$), only the hidden state is used as input:
\begin{equation}
    f_0(h_t) = W^{(3)}_0 \cdot \text{ReLU}(W^{(2)}_0 \cdot \text{ReLU}(W^{(1)}_0 h_t + b^{(1)}_0) + b^{(2)}_0) + b^{(3)}_0.
\end{equation}
The output of these functions is then used to compute the volume adjustment factor:
\begin{equation}
    \alpha_i = 1 + \tanh(f_i(h_{l+i-1}, v_{1:i-1})).
\end{equation}
This architectural design enables each adjustment decision to be informed not only by the market conditions (captured in the hidden state) but also by the actual execution trajectory. The use of separate transformation functions for each timestep allows the model to learn time-specific responses that consider both the current market state and the accumulated execution history. This is particularly important as trading decisions often need to balance immediate market conditions with the execution progress achieved so far.

\subsection{Sequential Volume Allocation}

The volume allocation process implements a sequential mechanism that ensures both causality and volume conservation. The process can be broken down into three key steps:
\subsubsection{Base Volume Initialization}
The base volume curve $v_b$ is initialized uniformly and maintained as a learnable parameter with positivity and sum-to-one constraints:
\begin{equation}
    v_b^{(0)} = \{\frac{1}{h}, \frac{1}{h}, ..., \frac{1}{h}\}, \quad \sum_{i=1}^h v_b^{(i)} = 1, \quad v_b^{(i)} \geq 0.
\end{equation}
\subsubsection{Sequential Adjustment Process}
For timesteps $i = 1$ to $h-1$, the allocated volume is computed via a constrained adjustment of the base curve:
\begin{equation}
    v_i = \operatorname{clip}\bigl(\alpha_i\, v_b^{(i)}, 0, 1 - \sum_{j=1}^{i-1} v_j\bigr).
\end{equation}
The clip function enforces three essential constraints:
1. Non-negativity: $v_i \geq 0$
2. Running sum constraint: $\sum_{j=1}^i v_j \leq 1$
3. Volume conservation: ensures sufficient volume remains for future timesteps
Each decision incorporates both the current market state through the RNN hidden state and the execution history through the accumulated volume curve, enabling more informed and context-aware trading decisions.

\subsubsection{Final Timestep Allocation}
For the final timestep $h$, the remaining volume is allocated to ensure complete volume conservation:
\begin{equation}
    v_h = 1 - \sum_{i=1}^{h-1} v_i.
\end{equation}
This construction guarantees that the resulting volume curve satisfies all required constraints:
\begin{equation}
    \sum_{i=1}^h v_i = 1, \quad v_i \geq 0 \quad \forall i \in \{1,...,h\}
\end{equation}
This sequential adjustment mechanism has several important properties. First, it ensures that the executed volume curve always satisfies the VWAP benchmarking constraints. Second, by conditioning each adjustment on the partial volume curve, it allows the model to dynamically adapt to its own prior decisions. Finally, by preserving the base volume curve as a foundation, it strikes a balance between adaptability and robustness, mitigating the risk of extreme adjustments that could lead to poor execution quality.

\subsection{From Training to Real-Time Deployment}

A notable characteristic of the model architecture is the seamless transition from training to real-time deployment. During training, the input sequence includes complete information for the entire period, including future timesteps. This complete sequence is essential for both parameter optimization and loss function computation, as VWAP performance can only be evaluated over the full execution period.

In real-time deployment, the sequential nature of the architecture, where predictions at time $t$ depend solely on information up to that point, permits straightforward adaptation. For deployment, the unknown future portion of the input sequence is padded with zeros:


\begin{equation}
    \tilde{x}_t = \begin{cases}
        x_t & \text{if } t \leq t_{\text{current}} \\
        0 & \text{if } t > t_{\text{current}}
    \end{cases}
\end{equation}

This padding maintains the required input dimensionality while ensuring that future information does not affect current predictions. The causality of the architecture guarantees that these padded values do not influence the predictions for the current timestep, allowing the same model architecture to be used for both training and deployment without structural modifications or performance degradation.

\section{Empirical Results}

This section presents the key findings from the empirical evaluation of the dynamic VWAP framework across a range of market conditions and benchmarks. An overview of the dataset and experimental setup is provided, followed by a detailed analysis of model performance in terms of VWAP slippage, volume curve fit, and execution stability. The approach is then compared to several baseline models, and the implications of the findings for practical VWAP execution are discussed.

\subsection{Experimental Setup}

\subsubsection{Dataset and Preprocessing}

The empirical analysis utilizes hourly trading data from five major cryptocurrencies (BTC, ETH, BNB, ADA, XRP) traded on Binance perpetual futures contracts. The dataset spans from the inception of each contract until July 1, 2024, offering a comprehensive view across different market regimes and conditions. A robust data partitioning strategy is employed to ensure realistic evaluation of model performance. The final 20\% of each cryptocurrency's data is reserved for testing, providing a true out-of-sample evaluation that mirrors real-world deployment scenarios. From the remaining 80\%, the last 20\% is allocated for validation, with the remainder used for training. Due to the temporal nature of the model's lookback window, randomization in dataset splitting is avoided to prevent unwanted overlap between training and validation data.

\subsubsection{Feature Engineering}

The model processes market data using a lookback window of 120 time steps to generate predictions for the subsequent 12 periods. The feature engineering approach emphasizes stationarity and careful handling of temporal dependencies to avoid forward-looking bias. The feature set comprises:

1. Volume Features: Raw volumes are normalized by dividing by a two-week rolling average, with the averaging window shifted by the combined lookback and prediction horizon to prevent information leakage. This normalization helps capture relative volume patterns while maintaining stationarity.

2. Temporal Indicators: Cyclic patterns are incorporated through hour-of-day and day-of-week indicators, enabling the model to learn time-dependent trading patterns characteristic of cryptocurrency markets.

3. Market Metrics: The feature set includes returns computed from bin VWAP prices. Periods without trading activity are assigned zero returns, ensuring continuous representation of market state while accurately reflecting periods of market inactivity.


\subsubsection{Objective}
Given the use of hourly data, the model's objective is to predict trading proportions for each hour. Three different optimization functions are compared. The first two compute deviations from market VWAP using absolute and quadratic terms, respectively. For these calculations, the price of each bin is considered to be its VWAP, allowing for a comparison of the achieved price with the market price by weighting the bin prices using either the predicted volume curve or the market volume curve over the execution period. In addition, the optimization of the quadratic distance to the market volume curve is evaluated, enabling a comparison between direct VWAP optimization and volume curve optimization, which is more common in the literature.


\subsubsection{Model Training Configuration}

A rigorous training protocol is implemented to ensure model stability and convergence. The optimization process employs the Adam optimizer with an initial learning rate of 0.001, along with two essential callback mechanisms:

1. Early Stopping: Training terminates automatically if the validation loss shows no improvement exceeding 0.00001 over a 10-epoch window. This mechanism helps prevent overfitting while ensuring sufficient model convergence.

2. Learning Rate Adaptation: A learning rate reduction callback monitors validation loss and applies a reduction factor of 0.25 after 5 epochs without improvement. The learning rate has a lower bound of 0.000025 to maintain stable optimization.

\medskip

Training proceeds with a batch size of 128 samples and can continue up to 1000 epochs, although early stopping typically results in earlier convergence. To ensure robust evaluation of model performance and stability, 5 independent training runs are conducted for each model configuration, each with different random initializations.

\subsubsection{Implementation Details}

All models are implemented using the Keras 3 framework with a Jax backend. To ensure reproducibility, a global random seed (seed = 1) is set at the framework level. The implementation leverages TKAN package version 0.4.3 and temporal\_linear\_network version 0.1.2 (for the TLN components in the static VWAP implementation). 

\newpage

\subsection{Results and Discussion}

\begin{table*}[ht]
    \centering
    \caption{Comparison of different algorithms across four datasets based on Test MSE and MAE.\@ 
            $\dagger$ indicates that results are reported from \citeauthor{Zhang.Irregular}~\cite{Zhang.Irregular}.
            Only tPatchGNN, GraFITi and \model{} use tuned hyperparameters. 
            For tPatchGNN we report both our own results as well the (untuned) results from \citeauthor{Zhang.Irregular}
            For the comparison, we focus on our own results, which is why the previously reported performance of tPatchGNN is in small font and parentheses.
            The best model is highlighted in \large \textbf{bold} \normalsize and the second best is \underline{underlined}.  
            }\label{tab:main} 
    \begin{tabular}{l cc cc cc cc}
        \toprule
        Algorithm & \multicolumn{2}{c}{PhysioNet} & \multicolumn{2}{c}{MIMIC} & \multicolumn{2}{c}{Human Activity} & \multicolumn{2}{c}{USHCN} \\ 
        \midrule
        & MSE$\times 10^{-3}$ & MAE$\times 10^{-2}$ & MSE$\times 10^{-2}$ & MAE$\times 10^{-2}$ & MSE$\times 10^{-3}$ & MAE$\times 10^{-2}$ & MSE$\times 10^{-1}$ & MAE$\times 10^{-1}$ \\ 
        \midrule
        DLinear$^\dagger$ & 41.86 ± 0.05 & 15.52 ± 0.03 & 4.90 ± 0.00 & 16.29 ± 0.05 & 4.03 ± 0.01 & 4.21 ± 0.01 & 6.21 ± 0.00 & 3.88 ± 0.02 \\ 
        TimesNet$^\dagger$ & 16.48 ± 0.11 & 6.14 ± 0.03 & 5.88 ± 0.08 & 13.62 ± 0.07 & 3.12 ± 0.01 & 3.56 ± 0.02 & 5.58 ± 0.05 & 3.60 ± 0.04 \\ 
        PatchTST$^\dagger$ & 12.00 ± 0.23 & 6.02 ± 0.14 & 3.78 ± 0.03 & 12.43 ± 0.10 & 4.29 ± 0.14 & 4.80 ± 0.09 & 5.75 ± 0.01 & 3.57 ± 0.02 \\ 
        Crossformer$^\dagger$ & 6.66 ± 0.11 & 4.81 ± 0.11 & 2.65 ± 0.10 & 9.56 ± 0.29 & 4.29 ± 0.20 & 4.89 ± 0.17 & 5.25 ± 0.04 & 3.27 ± 0.09 \\
        Graph Wavenet$^\dagger$  & 6.04 ± 0.28 & 4.41 ± 0.11 & 2.93 ± 0.09 & 10.50 ± 0.15 & 2.89 ± 0.03 & 3.40 ± 0.05 & 5.29 ± 0.04 & 3.16 ± 0.09 \\ 
        MTGNN$^\dagger$ & 6.26 ± 0.18 & 4.46 ± 0.07 & 2.71 ± 0.23 & 9.55 ± 0.65 & 3.03 ± 0.03 & 3.53 ± 0.03 & 5.39 ± 0.05 & 3.34 ± 0.02 \\ 
        StemGNN$^\dagger$ & 6.86 ± 0.28 & 4.76 ± 0.19 & 1.73 ± 0.02 & 7.71 ± 0.11 & 8.81 ± 0.37 & 6.90 ± 0.02 & 5.75 ± 0.09 & 3.40 ± 0.09 \\ 
        CrossGNN$^\dagger$ & 7.22 ± 0.36 & 4.96 ± 0.12 & 2.95 ± 0.16 & 10.82 ± 0.21 & 3.03 ± 0.10 & 3.48 ± 0.08 & 5.66 ± 0.04 & 3.53 ± 0.05 \\ 
        FourierGNN$^\dagger$ & 6.84 ± 0.35 & 4.65 ± 0.12 & 2.55 ± 0.03 & 10.22 ± 0.08 & 2.99 ± 0.02 & 3.42 ± 0.02 & 5.82 ± 0.06 & 3.62 ± 0.07 \\ 
        GRU-D$^\dagger$ & {5.59 ± 0.09} & 4.08 ± 0.05 & 1.76 ± 0.03 & 7.53 ± 0.09 & 2.94 ± 0.05 & 3.53 ± 0.06 & 5.54 ± 0.38 & 3.40 ± 0.28 \\ 
        SeFT$^\dagger$ & 9.22 ± 0.18 & 5.40 ± 0.08 & 1.87 ± 0.01 & 7.84 ± 0.08 & 12.20 ± 0.17 & 8.43 ± 0.07 & 5.80 ± 0.19 & 3.70 ± 0.11 \\ 
        RainDrop$^\dagger$ & 9.82 ± 0.08 & 5.57 ± 0.06 & 1.99 ± 0.03 & 8.27 ± 0.07 & 14.92 ± 0.14 & 9.45 ± 0.05 & 5.78 ± 0.22 & 3.67 ± 0.17 \\ 
        Warpformer$^\dagger$ & 5.94 ± 0.35 & 4.21 ± 0.12 & 1.73 ± 0.04 & 7.58 ± 0.13 & 2.79 ± 0.04 & 3.39 ± 0.03 & 5.25 ± 0.05 & 3.23 ± 0.05 \\ 
        mTAND$^\dagger$ & 6.23 ± 0.24 & 4.51 ± 0.17 & 1.85 ± 0.06 & 7.73 ± 0.13 & 3.22 ± 0.07 & 3.81 ± 0.07 & 5.33 ± 0.05 & 3.26 ± 0.10 \\ 
        Latent-ODE$^\dagger$ & 6.05 ± 0.57 & 4.23 ± 0.26 & 1.89 ± 0.19 & 8.11 ± 0.52 & 3.34 ± 0.11 & 3.94 ± 0.12 & 5.62 ± 0.03 & 3.60 ± 0.12 \\ 
        CRU$^\dagger$ & 8.56 ± 0.26 & 5.16 ± 0.09 & 1.97 ± 0.02 & 7.93 ± 0.19 & 6.97 ± 0.78 & 6.30 ± 0.47 & 6.09 ± 0.17 & 3.54 ± 0.18 \\ 
        Neural Flow$^\dagger$ & 7.20 ± 0.07 & 4.67 ± 0.04 & 1.87 ± 0.05 & 8.03 ± 0.19 & 4.05 ± 0.13 & 4.46 ± 0.09 & 5.35 ± 0.05 & 3.25 ± 0.05 \\ 
        \footnotesize (tPatchGNN$^\dagger$)  & \footnotesize({4.98 ± 0.08}) & \footnotesize(3.72 ± 0.03) & \footnotesize(1.69 ± 0.03) & \footnotesize(7.22 ± 0.09) & \footnotesize({2.66 ± 0.03}) & \footnotesize({3.15 ± 0.02}) & \footnotesize(5.00 ± 0.04) & \footnotesize({3.08 ± 0.04}) \\
        \midrule       
        tPatchGNN & 5.44 ± 0.14 & 3.85 ± 0.24   &  1.33 ± 0.02 & 6.58 ± 0.11 &  2.70  ± 0.06 & 3.18 ± 0.06 & \UL{5.06 ± 0.02} &  \UL{3.11 ± 0.05} \\ 
        GraFITi & \UL{4.91 ± 0.05} & \UL{3.57 ± 0.03} & \BF{1.21 ± 0.01} & \BF{6.19 ± 0.07}& \UL{2.64 ± 0.05} & \UL{3.08 ± 0.01} & {5.17 ± 0.07} & {3.19 ± 0.19}\\ 
        \model{} (ours) & \BF{4.88 ± 0.03} & \BF{3.47 ± 0.01} & \UL{1.25 ± 0.02} & \UL{6.20 ± 0.05} & \BF{2.49 ± 0.01} & \BF{3.06  ± 0.01} & \BF{5.01  ± 0.08} & \BF{3.05 ± 0.03} \\ 
        \bottomrule
    \end{tabular}
\end{table*}

The empirical analysis in table \ref{tab:dynamic_vwap_results} reveals significant improvements in VWAP execution performance through dynamic modeling approaches. The models are evaluated using three key metrics: absolute VWAP loss (measured in $10^{-2}$), quadratic VWAP loss (measured in $10^{-4}$), and the \(R^2\) score for volume curve prediction. The \(R^2\) score is noteworthy as it represents the traditional optimization target for volume-centric methodologies and serves as an important benchmark for comparing the approach with conventional methods.


\subsubsection{Performance Across Model Architectures}

The analysis shows a clear progression of performance improvements as the models evolve from simple linear models to more sophisticated architectures. The original static VWAP implementation using Temporal Linear Networks (TLN) demonstrates significant improvements over naive baselines, with reductions in absolute VWAP loss ranging from 20–25\% across all assets.


\medskip

The transition from TLN to recurrent architectures in the static framework yields substantial additional improvements. Looking at ETH, the static LSTM reduces absolute VWAP loss to 0.139 compared to TLN's 0.144, representing a 3.5\% improvement. The TKAN architecture further enhances these results, with the static implementation achieving 0.136, a 5.6\% improvement over TLN. Similar patterns emerge for BNB, where static LSTM and TKAN achieve losses of 0.137 and 0.134 respectively, compared to TLN's 0.142, representing improvements of 3.5\% and 5.6\%. This consistent pattern suggests that recurrent architectures' ability to capture temporal dependencies provides meaningful advantages even in static implementations. The transition to dynamic architectures marks an even more significant advancement in performance. For ETH, the DynamicVWAP with TKAN achieves an absolute VWAP loss of 0.121, an 11.0\% improvement over its static counterpart (0.136) and a 16.0\% improvement over the static TLN baseline (0.144). In BNB markets, the dynamic TKAN implementation reduces absolute VWAP loss to 0.129, representing improvements of 3.7\% over static TKAN (0.134) and 9.2\% over static TLN (0.142). The dynamic LSTM shows similar patterns but consistently falls slightly behind TKAN, with losses of 0.123 for ETH and 0.131 for BNB. The superiority of TKAN over LSTM becomes particularly pronounced in the dynamic setting. For ADA, dynamic TKAN achieves an absolute loss of 0.178 compared to dynamic LSTM's 0.184, a 3.3\% improvement. This advantage extends to quadratic loss, where dynamic TKAN's 0.127 represents a 4.8\% improvement over dynamic LSTM's 0.133. XRP shows similar patterns, with dynamic TKAN outperforming dynamic LSTM by 3.1\% in absolute loss (0.174 versus 0.179) and 3.2\% in quadratic loss (0.177 versus 0.183).

\medskip

These improvements are particularly noteworthy as they scale with market liquidity. In highly liquid markets such as BTC and ETH, the transition from static to dynamic architectures yields the largest relative improvements, often exceeding 10\%. For less liquid assets such as ADA and XRP, while the absolute improvements are smaller, the relative advantage of dynamic over static implementations remains consistent, indicating that the approach successfully adapts to varying market conditions. The enhanced performance of TKAN in both static and dynamic settings, particularly its superior ability to capture complex temporal patterns in real time, establishes a clear hierarchy of performance. Dynamic TKAN consistently leads, followed by dynamic LSTM, then static TKAN, static LSTM, and finally static TLN, with this ordering remaining stable across all assets and market conditions.

\subsubsection{Impact of Optimization Functions}

In many optimization scenarios, quadratic loss is favored because its gradient decreases with the magnitude of the error, providing smoother parameter updates and facilitating convergence. In contrast, absolute loss maintains a constant gradient regardless of the error size, which may lead to instability in the updates as the same magnitude of adjustment is applied for both small and large errors. Surprisingly, experimental results indicate that models trained with absolute VWAP loss consistently outperform those trained with quadratic loss in out-of-sample testing, for both static and dynamic implementations. For example, in BTC experiments, dynamic TKAN models exhibited a 10.2\% improvement in absolute VWAP loss when trained with absolute loss rather than quadratic loss. Moreover, models trained with absolute loss not only achieved lower absolute VWAP loss but also produced better performance when evaluated using quadratic metrics.

\medskip

Dynamic models optimized for volume curve prediction achieved remarkable improvements in \(R^2\) scores compared to their static counterparts. For instance, with ETH, the dynamic LSTM attained an \(R^2\) of 0.56, compared to 0.13 for the static version—an improvement ranging from 200\% to 400\% across different assets. Despite this enhanced predictive power for the volume curve, such improvement does not directly translate into superior VWAP execution performance. In fact, models that directly optimize for the VWAP execution target, particularly when using absolute loss, yield the best overall execution results—even if their \(R^2\) scores for volume prediction are lower or negative (for example, an \(R^2\) of -0.25 for BTC in the dynamic TKAN model).

\medskip

It is possible that quadratic loss, by emphasizing larger errors, may overweight extreme events during training, leading to an excessive drift in the internal base volume curve and reduced robustness under market stress. In contrast, the constant gradient provided by absolute loss appears to offer a more stable training regime, which is reflected in better generalization to out-of-sample data. In summary, while quadratic loss typically offers smoother optimization dynamics, the robustness achieved by directly targeting absolute execution error appears to be more beneficial in the context of VWAP execution.


\subsubsection{Asset-Specific Performance}

Market structure and liquidity significantly influence the relative advantages of dynamic over static models. More liquid assets like BTC and ETH show the largest relative improvements when moving from static to dynamic approaches, with the performance enhancement being particularly stable across different market conditions. For less liquid assets like ADA and XRP, while the absolute performance remains weaker than for major assets, the relative improvement from static to dynamic models remains consistent, suggesting the proposed approach successfully adapts to varying market conditions.

\subsubsection{Computational Considerations}

The computational requirements of the different architectures warrant careful consideration, particularly with respect to their performance benefits. The transition from static to dynamic models introduces a modest increase in computational overhead, primarily because both approaches already incorporate sequential processing of the full lookback window through recurrent neural network components. For TKAN implementations, dynamic models require 150–220 seconds of training time compared to 115–170 seconds for static counterparts, representing an approximate 30\% increase. Similarly, dynamic LSTM models require 50–85 seconds compared to 35–65 seconds for static versions.

\medskip
A more significant computational distinction is observed when comparing recurrent architectures with the simple linear approach of TLN, which exhibits training times of only 5–8 seconds. This order-of-magnitude difference in computational requirements between TLN and recurrent implementations (both static and dynamic) represents the primary computational trade-off. However, the increased computational cost is justified by substantial performance benefits, as recurrent architectures reduce VWAP tracking errors by 15–25\% compared to TLN approaches.

\medskip

These computational considerations should be evaluated in the context of practical VWAP execution, where even marginal improvements in execution quality can result in significant financial benefits given typical transaction volumes in cryptocurrency markets. Furthermore, the observed training times remain within practical limits for regular model updates, even accounting for the increased complexity of recurrent architectures. The moderate additional overhead introduced by dynamic implementations over static ones appears justified given the resulting improvements in execution quality.

\subsubsection{Visual Analysis of Execution Performance}
Graphical analysis provides additional insights into the comparative performance of different VWAP execution approaches. Figures \ref{fig:dynamic_slippage_full} and \ref{fig:dynamic_slippage_subset} present a comparison of slippage between naive, static, and dynamic implementations across different time horizons. Figures \ref{fig:dynamic_slippage_diff_full} and \ref{fig:dynamic_slippage_diff_subset} highlight the relative performance improvements over the naive baseline.

\medskip

In Figure \ref{fig:dynamic_slippage_full}, examining the full sample period, a notable pattern emerges during periods of sharp price movements. The magnitude of slippage spikes demonstrates a clear hierarchy of performance: dynamic implementations consistently exhibit smaller deviations compared to static models, which in turn show reduced slippage relative to the naive approach. This pattern is particularly evident during significant market events, suggesting that more sophisticated models better maintain execution quality during challenging conditions.

\begin{figure}[H]
    \centering
    \includegraphics[width=\columnwidth]{figures/vwap_analysis_full_sample_comparison.jpg}
    \caption{Slippage between approaches on the full out-of-sample set}
    \label{fig:dynamic_slippage_full}
\end{figure}
However, the density of information in the full sample analysis can obscure performance differences during more stable market periods. To address this, Figure \ref{fig:dynamic_slippage_subset} provides a more granular view of execution performance. This analysis reveals a consistent pattern in which dynamic implementations not only achieve lower absolute slippage but also maintain this advantage consistently. The visualization clearly demonstrates that improvements in slippage reduction from dynamic models (compared to naive) consistently exceed those achieved by static implementations, providing visual confirmation of the quantitative metrics presented in the earlier tables.

To address potential obscuring of performance differences during more stable market periods, Figure \ref{fig:dynamic_slippage_subset} provides a more granular view of execution performance. This analysis reveals a consistent pattern in which dynamic implementations not only achieve lower absolute slippage but also maintain this advantage consistently. The visualization confirms that improvements in slippage reduction from dynamic models exceed those achieved by static implementations, aligning with the quantitative metrics presented earlier.


\begin{figure}[H]
    \centering
    \includegraphics[width=\columnwidth]{figures/vwap_analysis_2000_sample_comparison.jpg}
    \caption{Slippage between approaches on a subsample of the out-of-sample set}
    \label{fig:dynamic_slippage_subset}
\end{figure}
Figures \ref{fig:dynamic_slippage_diff_full} and \ref{fig:dynamic_slippage_diff_subset} quantify these improvements by showing the difference in absolute slippage versus the naive approach. The predominantly negative values for dynamic implementations (shown in red and blue) confirm systematic outperformance, with the magnitude of improvement often exceeding 0.005 (50 basis points) during challenging market conditions.

\begin{figure}[H]
    \centering
    \includegraphics[width=\columnwidth]{figures/vwap_analysis_full_sample_slippage.jpg}
    \caption{Difference in absolute slippage versus naive approach on the full out-of-sample set. Negative values indicate improved performance over the naive approach.}
    \label{fig:dynamic_slippage_diff_full}
\end{figure}

\begin{figure}[H]
    \centering
    \includegraphics[width=\columnwidth]{figures/vwap_analysis_2000_sample_slippage.jpg}
    \caption{Difference in absolute slippage versus naive approach on the subsample set. Negative values indicate improved performance over the naive approach.}
    \label{fig:dynamic_slippage_diff_subset}
\end{figure}

\subsection{Analysis of Extended Time Horizons}
The appendix Tables \ref{table:dynamic_48_step} and \ref{table:dynamic_6_step} provide insights into model performance across different execution horizons, comparing 48-step and 6-step ahead predictions with the baseline 12-step model. Several key patterns emerge from this analysis. The short-horizon results presented in Table \ref{table:dynamic_6_step} demonstrate notably improved performance across all metrics compared to longer horizons, with absolute VWAP losses typically 30-40\% lower than in the 48-step scenario shown in Table \ref{table:dynamic_48_step}. This suggests that shorter execution windows benefit more significantly from dynamic adaptation. The relative advantage of dynamic over static implementations remains consistent across time horizons, though the absolute magnitude of improvement decreases with longer horizons. This pattern indicates that while dynamic benefits persist, the inherent difficulty of longer-term prediction partially offsets the advantages of adaptive execution. Particularly noteworthy is the stability of model performance, as evidenced by the standard deviations reported in both tables. These remain proportionally consistent across horizons, suggesting that the dynamic implementations maintain reliable execution quality regardless of the prediction timeframe. This stability is particularly important for practical applications where execution horizons may vary based on order characteristics.

\subsubsection{Visual Analysis of Execution Performance}

The graphical analysis reveals several key patterns in the execution behavior across different time horizons and model architectures. For shorter prediction windows (6 and 12 steps ahead), static VWAP implementations using recurrent neural networks demonstrate notable variability around their mean execution curves. This flexibility in predictions appears to contribute to their improved performance compared to traditional approaches. This variability becomes even more pronounced in dynamic implementations, particularly in models optimized for volume curve prediction. The enhanced adaptability manifests as wider prediction bands around the mean execution trajectory, suggesting these models can better respond to changing market conditions while maintaining overall execution objectives.
A striking observation across all architectures is the consistency of average execution curves (shown in red). Despite the significant differences in model complexity and implementation approach, these curves maintain remarkably similar patterns. This convergence suggests that the dynamic approaches preserve the fundamental principles of optimal VWAP execution while enabling better adaptation to market conditions. The analysis of longer-horizon predictions (48 steps ahead) reveals an interesting constraint effect in dynamic models. Significant contraction of prediction bands around the mean can be observed, a direct consequence of the model formulation. The use of scaling factors bounded between 0 and 2 effectively limits the variability of predictions at longer horizons. This observation suggests that while the current model specification performs well for shorter time horizons, modifications may be necessary for optimal long-horizon performance. Specifically, the constraint mechanism might benefit from horizon-dependent scaling to maintain appropriate levels of adaptability across different execution timeframes. These visual patterns provide important insights into both the strengths of the dynamic approach and potential areas for future refinement, particularly in the handling of longer execution horizons. The consistency of mean execution patterns across architectures, combined with the enhanced adaptability of dynamic models at shorter horizons, supports the quantitative findings regarding the advantages of dynamic implementation while highlighting specific areas for potential improvement in model design.

\section{Conclusion}

This paper demonstrates that dynamic neural approaches represent a significant advancement in VWAP execution methodology, particularly in cryptocurrency markets characterized by high volatility and evolving microstructure. While my framework achieves substantial improvements without explicitly modeling market impact, its modular design allows for straightforward integration of impact considerations through modification of the loss function, opening interesting avenues for future research. The empirical results reveal several key insights about the role of recurrent neural architectures in execution problems. While the initial application of RNNs to static VWAP models showed only modest improvements over traditional approaches (3-5\% reduction in tracking error), their true potential emerges through dynamic implementation. By leveraging RNNs' inherent ability to maintain and update state information, the proposed dynamic framework achieves substantial performance gains, reducing absolute VWAP tracking error by 10-15\% in liquid markets and maintaining consistent outperformance across varying market conditions. This dramatic improvement in performance when moving from static to dynamic implementations suggests that the key to successful RNN application lies not in their raw predictive power, but in their ability to adapt to evolving market conditions. The computational considerations reinforce this insight about effective architectural choices. When applied directly, RNNs introduce substantial computational overhead compared to simple linear models like TLN - training times increase by an order of magnitude, from 5-8 seconds to over 150 seconds for TKAN implementations. However, these architectures prove essential as building blocks for our dynamic framework. The additional 30\% computational overhead introduced by the dynamic implementation represents a reasonable trade-off given the resulting 10-15\% improvement in execution quality. This demonstrates that the key to achieving favorable performance-to-cost ratios lies in designing frameworks that effectively leverage the unique capabilities of sophisticated neural architectures.

\medskip

The success of this dynamic framework opens several promising directions for future research. The integration of market impact modeling through modified loss functions could provide more realistic execution strategies for large orders. Additionally, extending the model to incorporate cross-asset dependencies and exploring alternative neural architectures could further enhance performance. Significant improvements could be achieved through more sophisticated feature engineering, moving beyond our intentionally basic approach used to demonstrate the framework's potential. Development of horizon-dependent scaling factors could also enhance the model's effectiveness across different execution timeframes. These results suggest that adaptive neural architectures can effectively address the challenges of modern VWAP execution while maintaining computational efficiency suitable for practical deployment. The framework's ability to balance execution quality with computational constraints makes it particularly valuable for institutional investors managing large orders in cryptocurrency markets. By implementing our model as a standard Keras architecture and making all code publicly available, I provide a foundation that practitioners can readily deploy and researchers can build upon. This research thus contributes to both the theoretical understanding of optimal execution and practical implementation of VWAP strategies, bridging the gap between academic research and industry practice in an increasingly important area of market microstructure.

\section*{Code Availability}
The source code used for all experiments and analyses in this paper is available at \url{https://github.com/remigenet/DeepDynamicVWAP}.

\newpage
\bibliographystyle{IEEEtran}
\bibliography{bib}

\newpage
\begin{appendix}
\documentclass{MITstyle}

%\usepackage[table]{xcolor}
\usepackage{chngcntr}
\usepackage{hyperref}
\usepackage{microtype}

\title{A Lightweight and Extensible Cell Segmentation and Classification Model for Whole Slide Images}

\author{Nikita Shvetsov~$^{1, }$\footnote{Correspondence e-mail: nikita.shvetsov@uit.no}, Thomas K. Kilvaer~$^{2, 3}$, Masoud Tafavvoghi~$^{4}$, Anders Sildnes~$^{1}$, \\ Kajsa Møllersen~$^{4}$, Lill-Tove Rasmussen Busund~$^{5, 6}$, Lars Ailo Bongo~$^{1}$ \\
%
\vspace{1em} % Space between authors and afilliations
%
\normalfont{\small $^{1}$Department of Computer Science, UiT The Arctic University of Norway}\\
\normalfont{\small $^{2}$Department of Oncology, University Hospital of North Norway}\\
\normalfont{\small $^{3}$Department of Clinical Medicine, UiT The Arctic University of Norway}\\
\normalfont{\small $^{4}$Department of Community Medicine, UiT The Arctic University of Norway}\\
\normalfont{\small $^{5}$Department of Medical Biology, UiT The Arctic University of Norway} \\
\normalfont{\small $^{6}$Department of Clinical Pathology, University Hospital of North Norway} %\vspace{2em}
}

\begin{document}
\maketitle

\section*{Abstract}

% \begin{abstract}
% Developing clinically useful cell-level analysis tools in digital pathology remains challenging due to limitations in dataset granularity, inconsistent annotations, computational demands of advanced models, and difficulties in integrating new technologies into clinical workflows. To address these challenges, we propose a multi-faceted solution that enhances data quality, model performance, and usability to create a lightweight and extensible cell segmentation and classification model.

% First, we update data labels by employing a cross-relabeling process that refines the labels of two existing datasets, PanNuke and MoNuSAC, to create a new unified dataset with enhanced granularity, encompassing seven distinct cell types. Second, we leverage the H-Optimus foundation model as a fixed encoder to improve feature representation for simultaneous cell segmentation and classification tasks. Third, to address the computational demands of foundation models, we employ knowledge distillation to reduce model size and complexity while maintaining comparable performance. Finally, to facilitate integration into clinical workflows, we integrate the distilled model into the QuPath software, a widely used open-source platform in digital pathology.

% Our results demonstrate improvements in cell segmentation and classification performance using the H‑Optimus-based model compared to a CNN-based model. Specifically, the average $R^2$ improved from 0.575 to 0.871, and the average $PQ$ score improved from 0.450 to 0.492, indicating better alignment with actual cell counts and enhanced segmentation and classification quality. Furthermore, the distilled student model maintains performance comparable to the larger foundation model while reducing the parameter count by a factor of 48.
% Overall, by reducing computational complexity and integrating it into existing workflows, the proposed approach may significantly impact diagnostic processes, reduce the workload of pathologists, and contribute to improved patient outcomes. Though our approach shows potential enhancements in efficiency and usability of cell segmentation and classification models in digital pathology, extensive validation is needed to deploy these models in clinical practice.
% \end{abstract}

%%% shortened abstract
\begin{abstract}
Developing clinically useful cell-level analysis tools in digital pathology remains challenging due to limitations in dataset granularity, inconsistent annotations, high computational demands, and difficulties integrating new technologies into workflows. To address these issues, we propose a solution that enhances data quality, model performance, and usability by creating a lightweight, extensible cell segmentation and classification model. 

First, we update data labels through cross-relabeling to refine annotations of PanNuke and MoNuSAC, producing a unified dataset with seven distinct cell types. Second, we leverage the H-Optimus foundation model as a fixed encoder to improve feature representation for simultaneous segmentation and classification tasks. Third, to address foundation models' computational demands, we distill knowledge to reduce model size and complexity while maintaining comparable performance. Finally, we integrate the distilled model into QuPath, a widely used open-source digital pathology platform. 

Results demonstrate improved segmentation and classification performance using the H-Optimus-based model compared to a CNN-based model. Specifically, average $R^2$ improved from 0.575 to 0.871, and average $PQ$ score improved from 0.450 to 0.492, indicating better alignment with actual cell counts and enhanced segmentation quality. The distilled model maintains comparable performance while reducing parameter count by a factor of 48. By reducing computational complexity and integrating into workflows, this approach may significantly impact diagnostics, reduce pathologist workload, and improve outcomes. Although the method shows promise, extensive validation is necessary prior to clinical deployment.
\end{abstract}
\clearpage

\section{Introduction}
In digital pathology, accurate segmentation and classification of cells are crucial for many diagnostic, prognostic, and predictive analyses \cite{Jaber_Beziaeva_etal._2019,Lin_Pan_etal._2022,Park_Ock_etal._2022,Shen_Choi_etal._2024}. Nowadays, developments in computational pathology offer multiple solutions \cite{H._Qu_P._Wu_etal._2020,Javed_Mahmood_etal._2020} to utilize cell-level datasets to train machine learning models that solve these problems. The quality and specificity of training datasets are critical for robust and accurate models. Adhering to the principle of "garbage in, garbage out", it is essential to ensure that these datasets are extensively and accurately labeled with distinct classes that reflect the diverse biological characteristics of different cell types. Unfortunately, the number of open-source datasets comprising such high-quality annotations is limited. Existing cell segmentation datasets \cite{Gamper_Koohbanani_etal._2019,Graham_Vu_etal._2019,Verma_Kumar_etal._2021} may offer extensive annotations for certain cell types while providing more general labels for others. For example, in PanNuke, which is one of the largest open-source datasets comprising labeled cells, various types of morphologically and functionally different inflammatory cells like macrophages and lymphocytes are clustered in a broad "inflammatory" class. Consequently, these classes are frequently omitted from analyses or aggregated into broader meta-classes \cite{Gamper_Koohbanani_etal._2020} and likely interfere with other cell classes included in the dataset. This and similar inconsistencies in annotation granularity limit the ability of machine learning models to learn the comprehensive and nuanced features necessary for accurate cell segmentation and classification. To address these challenges, methods for refining and standardizing dataset annotations are essential to enhance the quality of training data.

A complementary approach to mitigate the absence of high-quality training data is the use of foundation models. Foundation models as encoders are defined as large-scale, versatile networks pre-trained on vast, diverse datasets using self-supervised learning, contrasting with convolutional neural network (CNN) pre-trained encoders that rely on supervised learning with labeled data. In practice, foundation models leverage enormous amounts of weakly or unlabeled data from millions of whole slide images (WSIs) and employ self-attention mechanisms to capture long-range dependencies and global context \cite{Chen_Ding_etal._2024,Saillard_Jenatton_etal._2024,Vorontsov_Bozkurt_etal._2024,Xu_Usuyama_etal._2024}. As a consequence, foundation models are able to produce transferable feature representations across different cell types and tissue environments. The feature representations can be leveraged by decoder networks to produce segmentation masks and pixel-level classifications. Because foundation models have comprehensive feature representations, they can be effectively fine-tuned using much smaller amounts of cell-level data compared to the large datasets needed to train models from scratch. Furthermore, foundation models incorporate adversarial training elements or contrastive learning \cite{Chen_Ding_etal._2024,Xu_Usuyama_etal._2024}, enhancing their resilience and adaptability by exposing them to challenging and varied scenarios during training. This may result in more generalizable models, often making them well-suited for diverse and complex tasks in digital pathology.

Despite the inherent advantages of foundation models, their deployment for practical use faces its own obstacles. In particular, they require substantial computational power, financial investments and rigorous testing to ensure reliability and efficacy for a given task \cite{Akkus_Dangott_etal._2022,Dragomir_Cocuz_etal._2022,Go_2022,Jafri_Farooqui_etal._2024}. Moreover, while foundation models enhance feature representation and performance, they depend on the quality of available annotations for decoder fine-tuning and, like any other model, cannot resolve existing inconsistencies or ambiguities in data labels. Therefore, there remains a critical need for solutions that address both data quality and practical deployment considerations.
Further, integrating new technologies into existing clinical workflows often encounters resistance, as it necessitates adjustments to established diagnostic processes. So, there is a need to develop solutions that could be integrated into current practices, minimizing the burden on medical professionals to adopt new tools \cite{King_Williams_etal._2023}.

Existing solutions \cite{Goldsborough_Philps_etal._2024,Hörst_Rempe_etal._2024}, while addressing some aspects of these challenges, fall short in providing a comprehensive approach. To address the data quality and clinical deployment issues, we propose a multi-faceted solution that encompasses data refinement, model optimization, and integration with existing pathology tools (\hyperref[fig:fig1]{Figure 1}). The outcome is a lightweight cell segmentation and classification model that can be integrated into digital pathology workflows for practical clinical use.

\begin{figure}[h!]
    \centering
    \includegraphics[width=\textwidth, height=0.82\textheight, keepaspectratio]{images/Figure_1.pdf}
    \caption{Overview of the proposed solution, including 1) Data refinement using cross-relabeling, 2) Teacher model development and fine tuning, 3) Student model optimization with knowledge distillation and 4) Student model and QuPath integration}
    \label{fig:fig1}
\end{figure}
\clearpage

Our approach begins with preparing the data for the fine-tuning and training of the machine learning models. We create a refined dataset, acquired via cross-relabeling two cell-level datasets, enhancing annotation specificity and consistency of the labeled data. Subsequently, we create a cell segmentation and classification model based on the foundation model. We leverage the foundation model as a fixed encoder and fine-tune a decoder using the refined dataset to improve generalization across diverse tissue- and cell types.
To ensure that the model remains lightweight and deployable in a possibly resource-constrained environment, we employ knowledge distillation to approximate the functionality of the foundation model. Finally, to facilitate the practical application of our model in digital pathology workflows, we integrate it with the QuPath \cite{Bankhead_Loughrey_etal._2017} application. Each methodological component contributes to the overarching goal of enhancing model performance, generalizability, and usability in clinical settings.

The primary contributions of this paper are:
\begin{enumerate}
    \item \textit{Data labels refinement through cross-relabeling:}
    
    We propose a new method for refining labels of cell-level datasets through cross-relabeling. This method employs classification models to re-label broad and ambiguous instances, resulting in a more diverse dataset. Our evaluation demonstrates that these classification models achieve high accuracy on test subsets, indicating the reliability of the method for label refinement.

    \item \textit{Enhanced model performance via foundation models:}
    
    We employ a foundation model as a feature extractor for the cell segmentation and classification task. In comparison with training a CNN model from scratch, the foundation model backbone only needs fine-tuning, which significantly reduces training time, computational resources and data requirements. We show that using a foundation model encoder leads to better performance in cell segmentation and classification networks than using a CNN-based encoder. This improvement may enable the model to generalize more effectively across various tissue types and imaging methods.
    
    \item \textit{Model optimization through knowledge distillation:}
    
    We show that a smaller student model trained using knowledge distillation on the refined dataset obtained via our cross-relabeling approach from a foundation model achieves comparable performance in cell segmentation and quantification tasks. As a result, this model is more suitable for deployment in environments without high-performance computing resources.
    
    \item \textit{Integration with QuPath:}
    
    We integrate the distilled cell segmentation and classification model into QuPath, a widely used open-source digital pathology platform, to accelerate clinical adaptation by enabling pathologists to more easily incorporate advanced computational tools into their existing workflows.
\end{enumerate}

Through these methodological steps, we aim to bridge the gap between advanced machine learning techniques and practical clinical applications, making accurate and efficient digital pathology accessible in a broader range of healthcare settings.

\section{Refining Existing Datasets Using Cross-Relabeling}
To address the limitations of sparse and ambiguous labeling of cell-level datasets, we propose a generalizable cross-relabeling strategy that can be applied to any dataset containing broadly categorized or imprecisely labeled cell types. This approach involves training and subsequently leveraging classification models to refine broad categories into more specific or biologically relevant classes.
When applied to cell-level data, the methodology includes extracting individual cell images from the dataset patches, preprocessing these images to standardize the size and accommodate partial cells, and then training deep learning classifiers capable of distinguishing between the finer cell subtypes within the coarser categories. 
To illustrate our approach, we focus on the PanNuke \cite{Gamper_Koohbanani_etal._2020, Gamper_Koohbanani_etal._2019} and MoNuSAC \cite{Verma_Kumar_etal._2021} datasets that we have used to train models for cell quantification in our previous works \cite{Shvetsov_Grønnesby_etal._2022,Shvetsov_Sildnes_etal._2024}. We find that for better cell differentiation we have to introduce more granular labels. PanNuke includes a broad classification of "inflammatory" cells, encompassing lymphocytes, macrophages, and neutrophils. Each cell type differs significantly in structure, function, and clinical relevance. Conversely, MoNuSAC uses the label "epithelial" for a class that comprises both benign epithelial cells and malignant neoplastic cells. This practice makes it challenging to differentiate between benign and malignant epithelial cells in the dataset, which is a critical distinction when identifying tumor areas within tissue samples. To address these issues, we implement a cross-relabeling strategy as shown in \hyperref[fig:fig2]{Figure 2}. The key components are two classification models: one is trained on singular cell images from PanNuke data to classify the epithelial meta-class into epithelial and neoplastic classes. The other is trained on MoNuSAC to refine the inflammatory class into lymphocytes, neutrophils, and macrophages.

\begin{figure}[h!]
    \centering
    \includegraphics[width=\textwidth]{images/Figure_2.pdf}
    \caption{Refined dataset generation via cross relabeling}
    \label{fig:fig2}
\end{figure}

The refining approach consists of three consecutive steps. The first is the preprocessing step, in which we extract individual cells from both datasets (\hyperref[fig:fig3]{Figure 3}). The specifics of PanNuke and MoNuSAC patch preparation before cell preprocessing are provided in \hyperref[chap:S1]{Appendix S1}.

\begin{figure}[h!]
    \centering
    \includegraphics[width=\textwidth]{images/Figure_3.pdf}
    \caption{Cell instances preprocessing including (1) cell map extraction, (2) bounding box delineation, (3) adjusting cell boxes and (4) cropping and resizing of cell images}
    \label{fig:fig3}
\end{figure}

During preprocessing, we extract cell type maps from the ground truth label mask and calculate bounding boxes around each cell instance. To accommodate partial cells at patch borders, a common issue in cropped patch images, we employ mirror padding and extend the field of view of the cell label by 15 pixels to capture adjacent cells. We then crop and resize the identified regions to $64 \times 64$ pixels using bicubic interpolation.

The preprocessed PanNuke dataset comprises 68,031 neoplastic and 23,207 epithelial cell images, while MoNuSAC comprises  33,104 lymphocytes, 1,252 neutrophils, and 1,695 macrophages, which we subsequently use in training cell classification models and classifying the cell image data \hyperref[fig:S2]{Appendix Figure S2 (1)}. 

The next step is to train two distinct ResNet50-based classifiers tailored to address the specific labeling challenges inherent in each dataset. We use ResNet50 for classification models due to its proven effectiveness for image classification tasks in histopathology \cite{pan2022reviewmachinelearningapproaches}, and its compatibility with small images. For the PanNuke dataset, we design the classifier, trained on MoNuSAC data, to disaggregate the heterogeneous "inflammatory" cell category into distinct subtypes: lymphocytes, macrophages, and neutrophils. Similarly, for the MoNuSAC dataset, the classifier is trained on PanNuke data and distinguishes between benign and malignant epithelial cells within the overarching "epithelial" label. By applying these targeted classifiers to their respective datasets, we assign more specific labels to individual cell instances, thus enabling us to create a unified dataset.
To ensure a balanced representation of classes, we train both models on datasets that had been equalized to match the size of the least represented class. Thus, we obtain datasets comprising 23,207 samples per class for PanNuke and 1,252 samples per class for MoNuSAC data. Next, we partition both of them into training (70\%), validation (20\%), and testing (10\%) subsets. To mitigate the risk of overfitting, we use a single dropout layer with a rate of p=0.5 in both models and data augmentation using randomized color perturbations, rotation, and horizontal and vertical flipping. We employ AdamW optimizer and the cross-entropy loss function for the training criterion.

To evaluate the two trained models, we measure the classification accuracy on the respective test subsets. The accuracies on the test subset for both classifiers are presented in \hyperref[tab:1]{Table 1}. The PanNuke model achieves an average accuracy of 93.57\%, with higher accuracy for neoplastic cells (96.06\%) compared to epithelial cells (86.26\%). The confusion matrix in Figure A3.1 shows that the model predominantly distinguishes accurately between epithelial and neoplastic tissues, with a substantial number of correct classifications and relatively few misclassifications. The MoNuSAC model demonstrates an average accuracy of 98.92\%, excelling in classifying lymphocytes (99.67\%) and macrophages (94.12\%), with lower performance for neutrophils (85.71\%). The confusion matrix in Figure A3.2 shows that the model identifies lymphocytes and performs reasonably well with macrophages and neutrophils.

\begin{table}[h!]
\renewcommand{\arraystretch}{1.5}
  \centering
  \caption{Cell classification results for PanNuke and MoNuSAC trained models (CI 95\%).}
  \label{tab:1}
  \begin{tabular}{|l|c|c|}
   \hline
   %\rowcolor{gray!30}
    Accuracy               & PanNuke model              & MoNuSAC model              \\
    \hline
    Average      & 0.936 (0.931--0.941)         & 0.989 (0.986--0.993)        \\
    \hline
    Neoplastic   & 0.961 (0.956--0.965)         & -                          \\
    \hline
    Epithelial   & 0.863 (0.849--0.877)         & -                          \\
    \hline
    Lymphocytes  & -                          & 0.997 (0.995--0.999)        \\
    \hline
    Neutrophils  & -                          & 0.857 (0.796--0.918)        \\
    \hline
    Macrophages  & -                          & 0.941 (0.906--0.976)        \\
    \hline
  \end{tabular}
\end{table}

Finally, during the last step, we use the model trained on PanNuke data for epithelial cells in MoNuSAC and the model trained on MoNuSAC for the inflammatory cells class in PanNuke. Specifically, we use classifier models to relabel epithelial cells in MoNuSAC and inflammatory cells in PanNuke data. Then we combine cells with refined labels and the rest of the cells in both datasets to create a refined dataset (\hyperref[fig:S2]{Appendix Figure S2 (2)}). The process of relabeling cells and visualizing them on a patch is shown in \hyperref[fig:fig4]{Figure 4}. The cell counts in the refined dataset are provided in \hyperref[tab:S4]{Appendix Table S4}.

\begin{figure}[h!]
    \centering
    \includegraphics[width=\textwidth, height=0.42\textheight, keepaspectratio]{images/Figure_4.pdf}
    \caption{Cell relabeling procedure for epithelial and inflammatory cell classes}
    \label{fig:fig4}
\end{figure}

%\hfill

Relabeling and combining datasets have been explored in a prior study \cite{Parulekar_Kanwat_etal._2023}, where consecutive fine-tuning on multiple datasets was employed to account for hierarchical class label structures. While the method presented in \cite{Parulekar_Kanwat_etal._2023} is intuitive, it often lacks consistency and requires multiple fine-tuning runs, which can be cumbersome and time-consuming. 
In contrast, cross-relabeling simplifies this process by using specialized classification models tailored to each dataset's specific labeling challenges. This approach provides better transparency and produces a unified dataset encompassing seven distinct cell types across multiple tissue samples, enhancing data diversity for further model training or fine-tuning.

Despite these improvements, cross-relabeling does not entirely resolve issues related to poor labeling quality or the amount of labeled data. Specifically, our results show lower accuracies persist for underrepresented classes, such as macrophages, which may stem from a limited sample availability and intrinsic challenges in distinguishing these cells based solely on H\&E staining. Furthermore, while our method enhances label specificity, it relies on the initial quality of the broad labels; thus, any fundamental inaccuracies in the original annotations can propagate through the relabeling process. Addressing the overall problem of limited data labels may require integrating additional data sources or utilizing complementary immunohistochemical staining methods.
Although the reported performance metrics are obtained from evaluations on the native test sets of each dataset, it is important to note that the primary application of these classifiers is to perform cross-relabeling, where a model trained on one dataset (e.g., PanNuke) is applied to another (e.g., MoNuSAC) and vice versa. We acknowledge that a more systematic evaluation of cross-dataset generalization is needed and could be performed in future work.

Overall, the refined dataset produced by our approach can enhance the supervised training or fine-tuning of cell segmentation and classification models, especially those that utilize pre-trained foundation models to improve feature extraction robustness. In addition, these models can detect nuanced classes that enable researchers to conduct more detailed analyses of biological processes in computational pathology.

\section{Foundation models for robust cell segmentation and classification}

Accurate cell segmentation and classification in digital pathology are hindered by limited labeled data and the fact that conventional CNNs are unable to capture global contextual information due to their local receptive field constraints \cite{Gheflati_Rivaz_2022,Yang_Marcus_etal.}. Traditional approaches in cell quantification have predominantly relied on CNN encoders, such as ResNet50, given their proven effectiveness in semantic segmentation tasks \cite{Deshmane_2023,Graham_Vu_etal._2019,Mukasheva_Koishiyeva_etal._2024,Stringer_Wang_etal._2021}. However, approaches that include fine-tuning of pretrained CNNs, data augmentation, and stain normalization to partially increase data variability and address staining differences often fail to achieve the necessary generalization and robustness across diverse tissue types and staining conditions \cite{G._Wang_W._Li_etal._2018,Gao_Bagci_etal._2018,Karim_El_Khoury_Martin_Fockedey_etal._2021}.

To overcome these challenges, we leverage an encoder-decoder network that uses a foundation model as the encoder and a CNN upsampling decoder (\hyperref[fig:fig5]{Figure 5}) for simultaneous cell segmentation and classification in 2D patches extracted from WSIs. Foundation models with transformer-based architectures are viable alternatives to CNN-based encoders \cite{Shamshad_Khan_etal._2023,Sourget_2023}. They enable the creation of more advanced architectures that can decode or transform learned features more effectively \cite{Chen_Duan_etal._2023,Cheng_Misra_etal._2022,Xie_Wang_etal._2021}.

\begin{figure}[h!]
    \centering
    \includegraphics[width=\textwidth]{images/Figure_5.pdf}
    \caption{UNETR-like model with foundational model as backbone}
    \label{fig:fig5}
\end{figure}

By utilizing a transformer-based encoder, we incorporate global contextual information into the feature extraction process, which is a key advantage of such architectures \cite{Chen_Lu_etal._2021}. This foundation model integration facilitates accurate pixel-wise segmentation and classification without the need for extensive encoder training, thereby potentially improving generalization across varied cellular structures and tissue types.
In our implementation, we employ a modified UNETR \cite{Hatamizadeh_Tang_etal._2021} architecture that combines a vision transformer (ViT) \cite{Dosovitskiy_Beyer_etal._2021} encoder with a CNN-based decoder. The encoder utilizes the pretrained H-Optimus foundation model, which contains 1.1 billion parameters and is trained on over 500,000 H\&E stained WSIs \cite{Saillard_Jenatton_etal._2024}. We extract outputs from four evenly spaced transformer blocks $Z_i$, where $i \in [1, 14, 26, 38]$, to serve as residual connections for the CNN decoder. We select these blocks based on our observation that features from non-adjacent levels of the encoder lead to better overall performance on the test subset.

The CNN decoder upsamples the feature representations, acquired from the transformer blocks, to generate an intermediate vector that is handled by two task-specific layers that generate cell segmentation and classification masks. The first task-specific layer is the ‘Cellpose head’,  which is used to delineate cell instances. The layer generates horizontal and vertical gradient maps to form vector fields that are refined through gradient tracking in a post-processing step using the Cellpose algorithm \cite{Stringer_Wang_etal._2021}, known for its efficacy in cell segmentation tasks and generalizability across multiple domains \cite{Pachitariu_Stringer_2022,Stringer_Pachitariu_2024}. The second task-specific layer is the "Cell type head", which assigns labels to individual pixels. In the post-processing step, we determine the output classification label of each segmented cell instance by majority voting over the labeled pixels that comprise the cell in the segmentation map.

To evaluate model performance and measure the impact of adding a foundation model as backbone, we compare it to a ResNet50-based model. ResNet50 is a widely used solution for encoders in segmentation architectures in the medical domain \cite{Deshmane_2023,Graham_Vu_etal._2019,Mukasheva_Koishiyeva_etal._2024,Stringer_Wang_etal._2021}. For the H-Optimus-based model, we utilize frozen weights for the encoder and only fine-tune the decoder to take advantage of the extensive pre-training of the foundation model. For the ResNet50-based model we start with ImageNet \cite{Deng_Dong_etal.} weights and train both encoder and decoder parts. Hyperparameters for the training step are set to be identical, where possible, for comparable evaluation. 
For this evaluation, we deliberately use the PanNuke dataset to provide a standardized and controlled comparison between the H‑Optimus and ResNet50-based models (\hyperref[fig:S2]{Appendix Figure S2 (3)}). Specifically, we use two of the default PanNuke dataset splits (66\%) for training and validation, and reserve the third split (33\%) for testing.

To address the challenge of cell class imbalance in the PanNuke dataset, which is a common characteristic in most cell-level H\&E patch datasets, both models’ training processes employ a weighted loss function comprising cross-entropy and focal loss \cite{Lin_Goyal_etal._2018}. The focal loss component is adjusted with coefficients derived from each cell class' instance frequency, emphasizing learning from underrepresented classes and enhancing the model's sensitivity to rare but significant cellular patterns. The cross-entropy loss is augmented with spectral decoupling regularization \cite{Pezeshki_Kaba_etal._2021,Pohjonen_Stürenberg_etal._2022} and spatially varying label smoothing \cite{Islam_Glocker_2021}, which potentially stabilizes training and improves generalization in case of complex tissue morphologies. For optimization, we employ the \textit{AdamW} \cite{Loshchilov_Hutter_2019} to counter unbalanced class scenarios, with cosine annealing learning rate scheduler.

We utilize the scikit-learn library \cite{Van_der_Walt_Schönberger_etal._2014} and HoVer-Net \cite{Graham_Vu_etal._2019} implementations of $R^2$ (the coefficient of determination) and $PQ$ (panoptic quality) to evaluate our experiments. Complete mathematical formulations and detailed explanations of these metrics are provided in \hyperref[chap:S5]{Appendix S5}. To compute confidence intervals, we use nonparametric bootstrapping, where after calculating the metric on the full sample, we generated 1000 bootstrap replicates by resampling with replacement and then determined the 95\% confidence intervals as the 2.5th and 97.5th percentiles of the resulting empirical distribution.

%\hfill

The model comparisons are summarized in \hyperref[tab:2]{Table 2}. The H‑Optimus-based model achieves higher $R^2$ across all cell classes compared to the ResNet50-based model, which means that its predictions are more closely aligned with the PanNuke cell counts, indicating a stronger correlation with the observed data. Notably, the improvement of $R^2_{dead}$ may be an indicator of better global contextual representations provided by the foundation model backbone. In terms of segmentation and classification quality combined, measured by the PQ score, the H‑Optimus-based model demonstrates notable improvements across most cell classes. Overall, the average $R^2$ improved from 0.575 to 0.871, while the average $PQ$ score improved from 0.450 to 0.492, demonstrating better performance of the H-Optimus-based model.

\begin{table}[h!]
\renewcommand{\arraystretch}{1.5}
  \centering
  \caption{Cell quantification metrics for baseline and proposed models (CI 95\%).}
  \label{tab:2}
  \begin{tabular}{|l|c|c|}
    \hline
    %\rowcolor{gray!30}
    Metric             & Resnet50-based            & H-optimus-based              \\
    \hline
    $R^2_{neoplastic}$    & 0.681 (0.576--0.769)       & \textbf{0.941 (0.917--0.960)} \\
    \hline
    $R^2_{inflammatory}$  & 0.863 (0.778--0.903)       & \textbf{0.949 (0.918--0.966)} \\
    \hline
    $R^2_{connective}$    & 0.600 (0.488--0.698)       & 0.609 (0.436--0.772)          \\
    \hline
    $R^2_{dead}$          & 0.097 (-11.389--0.669)     & 0.925 (0.404--0.982)          \\
    \hline
    $R^2_{epithelial}$    & 0.635 (0.490--0.747)       & \textbf{0.930 (0.886--0.964)} \\
    \hline
    $PQ_{neoplastic}$       & 0.517 (0.499--0.535)       & \textbf{0.589 (0.575--0.604)} \\
    \hline
    $PQ_{inflammatory}$     & 0.455 (0.429--0.482)       & \textbf{0.528 (0.507--0.549)} \\
    \hline
    $PQ_{connective}$       & 0.416 (0.400--0.431)       & \textbf{0.451 (0.436--0.465)} \\
    \hline
    $PQ_{dead}$             & 0.374 (0.342--0.408)       & 0.292 (0.209--0.365)          \\
    \hline
    $PQ_{epithelial}$       & 0.488 (0.460--0.519)       & \textbf{0.599 (0.579--0.618)} \\
    \hline
  \end{tabular}
\end{table}

Our results  show that integrating the H‑Optimus foundation model within the UNETR architecture enhances the model's ability to segment and classify cells across diverse tissues from PanNuke data. The pretrained transformer encoder provides robust feature representations, resulting in higher average $R^2$ and $PQ$ scores compared to the CNN-based model. This leads to more reliable cell quantification and more accurate downstream analysis. Additionally, the streamlined fine-tuning process reduces computational overhead and training time, making the model more adaptable for new data.

Despite these advancements, the foundation model-based approach does not fully resolve all challenges related to cell segmentation and classification. We observe lower metric scores for underrepresented classes in the training data. Furthermore, foundation models typically encompass billions of parameters, resulting in substantial computational and memory requirements. It therefore poses challenges for deployment in resource-constrained environments, limiting their practical applicability in certain clinical settings.

\section{Model optimization via Knowledge Distillation}

To address the limitations posed by the extensive size of foundation models, we implement knowledge distillation — a model compression technique that leverages the teacher-student paradigm \cite{Hinton_Vinyals_etal._2015}. By training a smaller, more efficient student model to replicate the output of a larger, pre-trained teacher model, we retain performance while significantly reducing the model's complexity and resource requirements (\hyperref[fig:fig6]{Figure 6}).

\begin{figure}[h!]
    \centering
    \includegraphics[width=\textwidth, height=0.45\textheight, keepaspectratio]{images/Figure_6.pdf}
    \caption{Knowledge distillation framework for training a student model using a pre-trained teacher}
    \label{fig:fig6}
\end{figure}

We employ knowledge distillation to compress the H‑Optimus-based teacher model into a more efficient student model. The teacher model is the modified UNETR architecture with the H‑Optimus foundation model described in the previous chapter. The student model is based on a UNet architecture augmented with residual connections and incorporates a smaller ViT encoder with 9 million parameters \cite{Steiner_Kolesnikov_etal._2022,Wightman_2019}. 

First, we fine-tune the teacher model using the refined dataset from the cross-relabeling procedure (Section 2). Initially we train the decoder of the teacher model while keeping the encoder weights frozen. We split the refined dataset into train (70\%), validation (20\%) and test (10\%) subsets (\hyperref[fig:S2]{Appendix Figure S2 (4)}). During fine-tuning, we use the train and validation subsets, while leaving the test subset for model evaluation. We set the training procedure and model hyperparameters to be identical to those that were used to demonstrate the utility of foundation models for the simultaneous cell segmentation and classification task.

Next, we perform knowledge distillation from teacher to student using the refined dataset used to fine-tune the teacher model. The student model is trained to replicate the teacher model's outputs. We utilize a specialized loss function that aligns the student's predicted probability distribution with the teacher's, incorporating the teacher's class probability distribution derived from the output. Following the methodology of Hinton et al. \cite{Hinton_Vinyals_etal._2015}, we experiment with various hyperparameter settings for the temperature ($T$) and the balancing coefficients ($\alpha$ and $\beta$) in the loss function. We vary $T$ from 1 to 20 and adjust $\alpha$ and $\beta$ to balance the distillation and student losses. Through iterative tuning and evaluation, we identify that setting $T=14$, $\alpha=0.3$, and $\beta=0.7$ yields a configuration that converges and closely approximates the teacher model's performance during training.

Finally, we assess the performance of both models using the $R^2$ and $PQ$ (defined in \hyperref[chap:S5]{Appendix S5}) on the test set of the refined dataset (\hyperref[tab:3]{Table 3}). We observe that the 95\% confidence intervals overlap for most cell types, so we cannot claim statistically significant performance differences between the teacher and student models. One exception appears in the neoplastic class. The teacher model produces an $R^2$ of 0.919, while the student model shows an $R^2$ of 0.852. In addition, the student model achieves higher $PQ$ values for the neoplastic and connective classes, though the confidence intervals show overlap.

\begin{table}[h!]
\renewcommand{\arraystretch}{1.5}
  \centering
  \caption{Cell quantification metrics for teacher and distilled student models (CI 95\%).}
  \label{tab:3}
  \begin{tabular}{|l|c|c|}
    \hline
    %\rowcolor{gray!30}
    Metric & Teacher & Student \\
    \hline
    $R^2_{neoplastic}$    & \textbf{0.919} (0.898--0.939) & 0.852 (0.800--0.891) \\
    \hline
    $R^2_{lymphocyte}$    & 0.969 (0.956--0.977)         & 0.969 (0.956--0.978) \\
    \hline
    $R^2_{connective}$    & 0.694 (0.548--0.809)         & 0.618 (0.469--0.741) \\
    \hline
    $R^2_{dead}$          & 0.755 (0.400--0.908)         & 0.424 (0.100--0.731) \\
    \hline
    $R^2_{epithelial}$    & 0.922 (0.870--0.958)         & 0.843 (0.738--0.917) \\
    \hline
    $R^2_{macrophage}$    & 0.384 (-0.369--0.724)        & 0.704 (0.352--0.859) \\
    \hline
    $R^2_{neutrofil}$     & 0.854 (0.578--0.929)         & 0.833 (0.502--0.925) \\
    \hline
    $PQ_{neoplastic}$       & 0.581 (0.569--0.593)         & 0.601 (0.588--0.613) \\
    \hline
    $PQ_{lymphocyte}$       & 0.536 (0.520--0.553)         & 0.563 (0.544--0.579) \\
    \hline
    $PQ_{connective}$       & 0.436 (0.421--0.451)         & 0.457 (0.441--0.474) \\
    \hline
    $PQ_{dead}$             & 0.272 (0.235--0.315)         & 0.279 (0.201--0.369) \\
    \hline
    $PQ_{epithelial}$       & 0.522 (0.500--0.545)         & 0.530 (0.506--0.555) \\
    \hline
    $PQ_{macrophage}$       & 0.524 (0.459--0.588)         & 0.474 (0.405--0.543) \\
    \hline
    $PQ_{neutrofil}$        & 0.541 (0.490--0.592)         & 0.565 (0.522--0.607) \\
    \hline
  \end{tabular}
\end{table}


We further decompose the $PQ$ metric into its $SQ$ and $DQ$ components (\hyperref[tab:S6]{Appendix Table S6}). Both models produce nearly identical $SQ$ values, which indicates that they predict instance boundaries with similar precision. Although the student model shows some improvement in $DQ$ scores for certain classes, the confidence intervals overlap and do not confirm a statistically significant difference.

We observe that the student and teacher models yield comparable detection performance despite the student model using a much smaller and simpler architecture. A model with fewer parameters reduces the risk of overfitting when training data are scarce relative to the model’s complexity \cite{Farias_Ludermir_etal._2022}. The knowledge distillation process also encourages the student model to focus on the most generalizable detection features learned from the teacher. These factors enable the student model to achieve similar detection performance across different cell types.

Additionally, considering the model sizes reported in \hyperref[tab:4]{Table 4}, the distilled model achieves a significant reduction compared to the teacher model, with a 48-fold decrease in parameter count and a 5.5-fold reduction in on-disk size. In inference mode, the teacher model requires 16 GB of VRAM for a batch size of 32, while the distilled model only needs 3 GB of VRAM for the same batch size. These reductions make the distilled model significantly more practical for fine-tuning and deployment in resource-constrained environments.

\begin{table}[h!]
\renewcommand{\arraystretch}{1.5}
  \centering
  \caption{Parameter counts and size of teacher and distilled model}
  \label{tab:4}
  \adjustbox{max width=\textwidth}{%
  \begin{tabular}{|l|c|c|c|}
    \hline
    %\rowcolor{gray!30}
    Metric & H-optimus-based (Teacher) & mobileViT-based (Student) & Magnitude of difference \\
    \hline
    Parameters count       & 1,158,917,906   & \textbf{24,093,393}   & \textbf{48x}  \\
    \hline
    Estimated Total Size (MB) & 87,912       & \textbf{15,935}    & \textbf{5.5x} \\
    \hline
  \end{tabular}%
}
\end{table}

%\hfill

With recent advancements in complex network architectures and the use of pretrained encoders to achieve state-of-the-art performance \cite{Baumann_Dislich_etal._2024,Hörst_Rempe_etal._2024} in cell segmentation and classification tasks, model size, computational complexity, and processing times have increased. This limits the scalability and accessibility of these models. As we demonstrate, this may be mitigated using knowledge distillation. Studies in the field of natural language processing have demonstrated the efficacy of knowledge distillation in retaining the capabilities of the teacher model while achieving significant reductions in size and complexity \cite{Huangpu_Gao_2024,Sun_Yu_etal.}. 

We demonstrate the feasibility of knowledge distillation in digital pathology, specifically for cell segmentation and classification tasks. Moreover, we achieve this performance while also significantly reducing the parameter count. In addressing the challenge of knowledge transfer, we found that distillation from a transformer-based model to a smaller transformer is more straightforward than attempting to map transformer features to CNN blocks. In our experiments, using a CNN-based network as a student results in worse cell quantification performance due to the structural constraints of CNN feature space dimensions. 

Although our primary approach relies on a transformer-based student model that performs well, it can be further optimized to incorporate advantages from CNN architectures. For example, employing alternative techniques such as using ViT adapters \cite{Chen_Duan_etal._2023} or $1 \times 1$ convolutions to adjust feature map sizes may be beneficial for harnessing CNN advantages like enhanced local feature extraction. Moreover, if additional performance improvements are desired, the process can be further enhanced by applying supplementary knowledge distillation techniques, such as self-distillation \cite{Zhang_Song_etal._2019} or online distillation \cite{Houyon_Cioppa_etal._2023}.

Despite these promising results, further validation on independent datasets is necessary to fully understand the model's limitations. Underrepresented classes may pose challenges when addressing complex cases. Pathologists need to validate these models to adopt them in clinical settings. While the distilled models are smaller and more deployable, a technological gap persists because pathologists traditionally rely on established methods for inspecting WSIs and diagnosing diseases. Addressing the complexities involved in deploying models for inference and supporting pathologists in adopting new tools is essential for integrating these models into clinical workflows.

\section{Model integration with QuPath}
Digital pathology tools with graphical user interfaces are essential for visualizing and analyzing WSIs. To make our student model useful in clinical pathology workflows, it needs to be integrated into a tool that enables inspecting regions, creating annotations, and providing quantitative analyses of biomarkers. Therefore, we integrate the trained student model from the previous chapter into the QuPath open‑source platform \cite{Bankhead_Loughrey_etal._2017}. QuPath provides the required annotation, visualization, and analysis tools to interpret complex histological data, including workflows for cell segmentation, classification, and quantification (\hyperref[fig:fig7]{Figure 7}). 

\begin{figure}[h!]
    \centering
    \includegraphics[width=\textwidth]{images/Figure_7.pdf}
    \caption{Visualization of model-generated cell quantification annotations (left) and the corresponding unannotated slide (right) in QuPath}
    \label{fig:fig7}
\end{figure}

To identify the regions in a WSI critical for prognosticating tumor development, such as specific tumor areas or border regions without overlapping healthy tissue, the pathologist uses QuPath to outline these regions. Then, the pathologist initiates a cell segmentation and classification script through the QuPath interface for the selected regions. The resulting annotations and quantified cell information are then directly overlaid onto the WSI in the QuPath interface. Additional design and implementation details are in \hyperref[chap:S7]{Appendix S7}. 

Two common approaches for integrating deep learning models into QuPath are Java‑based native QuPath extensions \cite{Goldsborough_Philps_etal._2024} and the execution of RESTful API requests to a model server coupled with handling the response via an extension, as demonstrated in the application of cell segmentation models applied to immunofluorescence images \cite{Sugawara_2023}. While the community is actively working on these integration strategies, there is currently no universal solution that fully addresses all integration and performance requirements.

Extensions may offer better integration with QuPath, allowing slightly improved performance and more widespread usage of the built-in QuPath models, but they lack the flexibility to customize models and modify their behavior. For example, the newest version of QuPath includes models such as StarDist \cite{Weigert_Schmidt} and InstanSeg \cite{Goldsborough_Philps_etal._2024} that can perform cell segmentation. Both models pose limitations when applied to simultaneous cell segmentation and classification. StarDist performs well only on convex, round shapes by design, whereas some neoplastic, inflammatory, and connective cells exhibit complex and non-convex shapes. InstanSeg provides only semantic segmentation without assigning classes to the segmented cells.

%\hfill

In contrast, our approach offers an alternative integration strategy. It utilizes the paquo library to directly interact with QuPath’s internal application programming interface from within Python. This enables data exchange and processing without the need for intermediate conversion steps and provides greater control over model customization, retraining, and the incorporation of custom processing steps.

The integration of our custom model with QuPath underscores its potential to significantly enhance the diagnostic process by reducing the time burden on pathologists and enabling them to focus on more complex interpretative tasks using familiar software. Leveraging a tool that is already well-established among pathologists increases the likelihood of its adoption into daily clinical workflows. The quantitative data generated through the automated workflow is critical for both clinical decision-making and research, facilitating more accurate biomarker analysis, enabling robust statistical evaluations, and supporting hypothesis generation and testing. Additionally, by streamlining cell segmentation and classification, the tool enhances the scalability and reproducibility of pathological assessments, ultimately contributing to improved diagnostic accuracy and patient outcomes.

\section{Conclusion and future work}

In this study, we address critical challenges in digital pathology and tackle the usability and deployment issues of the developed models in standard computing environments without the need for high-performance computing systems. Our multi-faceted approach encompasses data refinement through cross-relabeling, leveraging foundation models for robust cell segmentation and classification, optimizing model performance via knowledge distillation, and integrating the optimized model into the QuPath software for practical application. This approach is used to construct a capable, versatile, and adjustable model for cell segmentation and classification, with enhanced performance and usability.

\begin{sloppypar}
While our approach shows potential in the field of computational pathology, certain limitations persist. 
For example, our implementation currently exhibits lower performance in detecting macrophages. 
This serves as an instance of the broader challenge of accurately identifying complex cell types. In order to address this issue, extending our approach to incorporate additional data sources, exploring alternative modeling approaches, and integrating other imaging modalities such as immunohistochemical staining may help improve detection accuracy. Moreover, although the distilled model reduces computational demands, integrating advanced deep learning models into clinical practice requires addressing technological gaps and potential resistance to adopting new tools within established diagnostic processes.
\end{sloppypar}

Future work could focus on several key areas to refine the proposed approach and facilitate its adoption in clinical environments. Enhancing the cell-relabeling process with additional datasets \cite{Graham_Jahanifar_etal._2021} could improve the representation of underrepresented cell types and enhance overall model performance. Also, incorporating additional data sources, such as multi-modal imaging or complementary staining methods, may address limitations related to cell type differentiation and class imbalance. Exploring other foundation models \cite{Vorontsov_Bozkurt_etal._2024,Zimmermann_Vorontsov_etal._2024} or introducing additional modalities \cite{Ding_Wagner_etal._2024,Vaidya_Zhang_etal._2025} may provide alternative architectures better suited to specific tasks or offer improved efficiency. Implementing more complex knowledge distillation techniques \cite{Houyon_Cioppa_etal._2023,Zhang_Song_etal._2019} could further optimize the model's performance and adaptability. Additionally, deeper integration with QuPath or other digital pathology software could provide pathologists more control over cell quantification analysis directly within the QuPath interface, thereby increasing accessibility and usability. Such enhancements would not only refine model performance but also ensure greater adaptability and scalability within various clinical environments. Finally, extensive validation of the model by pathologists and benchmarking against independent datasets are essential steps toward establishing the model's reliability and fostering confidence in its clinical utility.

\section*{Acknowledgments} 
This work was funded in part by the Research Council of Norway grant no. 309439 SFI Visual Intelligence, and the North Norwegian Health Authority grant no. HNF1521-20.

\bibliographystyle{IEEEtran}
\begin{sloppypar}
\begin{thebibliography}{99}

\bibitem{chaplot2020neural} Chaplot, Devendra Singh, et al. "Neural topological slam for visual navigation." Proceedings of the IEEE/CVF conference on computer vision and pattern recognition. 2020.

\bibitem{maksymets2021thda} Maksymets, Oleksandr, et al. "Thda: Treasure hunt data augmentation for semantic navigation." Proceedings of the IEEE/CVF International Conference on Computer Vision. 2021.

\bibitem{mezghan2022memory} Mezghan, Lina, et al. "Memory-augmented reinforcement learning for image-goal navigation." 2022 IEEE/RSJ International Conference on Intelligent Robots and Systems (IROS). IEEE, 2022.

\bibitem{al2022zero} Al-Halah, Ziad, Santhosh Kumar Ramakrishnan, and Kristen Grauman. "Zero experience required: Plug \& play modular transfer learning for semantic visual navigation." Proceedings of the IEEE/CVF Conference on Computer Vision and Pattern Recognition. 2022.

\bibitem{ye2021auxiliary} Ye, Joel, et al. "Auxiliary tasks and exploration enable objectgoal navigation." Proceedings of the IEEE/CVF international conference on computer vision. 2021.

\bibitem{chaplot2020object} Chaplot, Devendra Singh, et al. "Object goal navigation using goal-oriented semantic exploration." Advances in Neural Information Processing Systems 33 (2020)

\bibitem{ramakrishnan2022poni} Ramakrishnan, Santhosh Kumar, et al. "Poni: Potential functions for objectgoal navigation with interaction-free learning." Proceedings of the IEEE/CVF Conference on Computer Vision and Pattern Recognition. 2022.

\bibitem{ramrakhya2022habitat} Ramrakhya, Ram, et al. "Habitat-web: Learning embodied object-search strategies from human demonstrations at scale." Proceedings of the IEEE/CVF Conference on Computer Vision and Pattern Recognition. 2022.

\bibitem{mousavian2019visual} Mousavian, Arsalan, et al. "Visual representations for semantic target driven navigation." 2019 International Conference on Robotics and Automation (ICRA). IEEE, 2019.

\bibitem{dhariwal2021diffusion} Dhariwal, Prafulla, and Alexander Nichol. "Diffusion models beat gans on image synthesis." Advances in neural information processing systems 34 (2021)

\bibitem{ho2022classifier} Ho, Jonathan, and Tim Salimans. "Classifier-free diffusion guidance." arXiv preprint arXiv:2207.12598 (2022).

\bibitem{nichol2021glide} Nichol, Alex, et al. "Glide: Towards photorealistic image generation and editing with text-guided diffusion models." arXiv preprint arXiv:2112.10741 (2021)

\bibitem{brooks2023instructpix2pix} Brooks, Tim, Aleksander Holynski, and Alexei A. Efros. "Instructpix2pix: Learning to follow image editing instructions." Proceedings of the IEEE/CVF Conference on Computer Vision and Pattern Recognition. 2023.

\bibitem{fu2023guiding} Fu, Tsu-Jui, et al. "Guiding instruction-based image editing via multimodal large language models." arXiv preprint arXiv:2309.17102 (2023).

\bibitem{geng2024instructdiffusion} Geng, Zigang, et al. "Instructdiffusion: A generalist modeling interface for vision tasks." Proceedings of the IEEE/CVF Conference on Computer Vision and Pattern Recognition. 2024.

\bibitem{zhou2024minedreamer} Zhou, Enshen, et al. "Minedreamer: Learning to follow instructions via chain-of-imagination for simulated-world control." arXiv preprint arXiv:2403.12037 (2024).

\bibitem{zhou2023esc} Zhou, Kaiwen, et al. "Esc: Exploration with soft commonsense constraints for zero-shot object navigation." International Conference on Machine Learning. PMLR, 2023.

\bibitem{yu2023l3mvn} Yu, Bangguo, Hamidreza Kasaei, and Ming Cao. "L3mvn: Leveraging large language models for visual target navigation." 2023 IEEE/RSJ International Conference on Intelligent Robots and Systems (IROS). IEEE, 2023.

\bibitem{gadre2023cows} Gadre, Samir Yitzhak, et al. "Cows on pasture: Baselines and benchmarks for language-driven zero-shot object navigation." Proceedings of the IEEE/CVF Conference on Computer Vision and Pattern Recognition. 2023.

\bibitem{shah2023navigation} Shah, Dhruv, et al. "Navigation with large language models: Semantic guesswork as a heuristic for planning." Conference on Robot Learning. PMLR, 2023.

\bibitem{cai2024bridging} Cai, Wenzhe, et al. "Bridging zero-shot object navigation and foundation models through pixel-guided navigation skill." 2024 IEEE International Conference on Robotics and Automation (ICRA). IEEE, 2024.

\bibitem{yu2023co} Yu, Bangguo, Hamidreza Kasaei, and Ming Cao. "Co-NavGPT: Multi-robot cooperative visual semantic navigation using large language models." arXiv preprint arXiv:2310.07937 (2023).

\bibitem{wu2024voronav} Wu, Pengying, et al. "Voronav: Voronoi-based zero-shot object navigation with large language model." arXiv preprint arXiv:2401.02695 (2024).

\bibitem{qin2023mp5} Qin, Yiran, et al. "Mp5: A multi-modal open-ended embodied system in minecraft via active perception." arXiv preprint arXiv:2312.07472 (2023).

\bibitem{du2024learning} Du, Yilun, et al. "Learning universal policies via text-guided video generation." Advances in Neural Information Processing Systems 36 (2024).

\bibitem{ajay2024compositional} Ajay, Anurag, et al. "Compositional foundation models for hierarchical planning." Advances in Neural Information Processing Systems 36 (2024).

\bibitem{liang2024skilldiffuser} Liang, Zhixuan, et al. "Skilldiffuser: Interpretable hierarchical planning via skill abstractions in diffusion-based task execution." Proceedings of the IEEE/CVF Conference on Computer Vision and Pattern Recognition. 2024.

\bibitem{heusel2017gans} Heusel, Martin, et al. "Gans trained by a two time-scale update rule converge to a local nash equilibrium." Advances in neural information processing systems 30 (2017).

\bibitem{zhang2018unreasonable} Zhang, Richard, et al. "The unreasonable effectiveness of deep features as a perceptual metric." Proceedings of the IEEE conference on computer vision and pattern recognition. 2018.

\bibitem{brown2020language} Brown, Tom B. "Language models are few-shot learners." arXiv preprint arXiv:2005.14165 (2020).

\bibitem{podell2023sdxl} Podell, Dustin, et al. "Sdxl: Improving latent diffusion models for high-resolution image synthesis." arXiv preprint arXiv:2307.01952 (2023).

\bibitem{brohan2022rt} Brohan, Anthony, et al. "Rt-1: Robotics transformer for real-world control at scale." arXiv preprint arXiv:2212.06817 (2022).

\bibitem{brohan2023rt} Brohan, Anthony, et al. "Rt-2: Vision-language-action models transfer web knowledge to robotic control." arXiv preprint arXiv:2307.15818 (2023).

\bibitem{li2024manipllm} Li, Xiaoqi, et al. "Manipllm: Embodied multimodal large language model for object-centric robotic manipulation." Proceedings of the IEEE/CVF Conference on Computer Vision and Pattern Recognition. 2024.

\bibitem{shah2023vint} Shah, Dhruv, et al. "ViNT: A foundation model for visual navigation." arXiv preprint arXiv:2306.14846 (2023).

\bibitem{liu2024visual} Liu, Haotian, et al. "Visual instruction tuning." Advances in neural information processing systems 36 (2024).

\bibitem{hu2021lora} Hu, Edward J., et al. "Lora: Low-rank adaptation of large language models." arXiv preprint arXiv:2106.09685 (2021).

\bibitem{qin2023supfusion} Qin, Yiran, et al. "SupFusion: Supervised LiDAR-camera fusion for 3D object detection." Proceedings of the IEEE/CVF International Conference on Computer Vision. 2023.

\bibitem{qin2024worldsimbench} Qin, Yiran, et al. "Worldsimbench: Towards video generation models as world simulators." arXiv preprint arXiv:2410.18072 (2024).

\bibitem{yu2025gamefactory} Yu, Jiwen, et al. "GameFactory: Creating New Games with Generative Interactive Videos." arXiv preprint arXiv:2501.08325 (2025).

\bibitem{zhou2024code} Zhou, Enshen, et al. "Code-as-Monitor: Constraint-aware Visual Programming for Reactive and Proactive Robotic Failure Detection." arXiv preprint arXiv:2412.04455 (2024).

\bibitem{zhang2024ad} Zhang, Zaibin, et al. "AD-H: Autonomous Driving with Hierarchical Agents." arXiv preprint arXiv:2406.03474 (2024).

\bibitem{wang2024toward} Wang, Chaoqun, et al. "Toward Accurate Camera-based 3D Object Detection via Cascade Depth Estimation and Calibration." arXiv preprint arXiv:2402.04883 (2024).

\bibitem{huang2024story3d} Huang, Yuzhou, et al. "Story3d-agent: Exploring 3d storytelling visualization with large language models." arXiv preprint arXiv:2408.11801 (2024).

\bibitem{savinov2018semi} Savinov, Nikolay, Alexey Dosovitskiy, and Vladlen Koltun. "Semi-parametric topological memory for navigation." arXiv preprint arXiv:1803.00653 (2018).

\bibitem{majumdar2022zson} Majumdar, Arjun, et al. "Zson: Zero-shot object-goal navigation using multimodal goal embeddings." Advances in Neural Information Processing Systems 35 (2022): 32340-32352.

\bibitem{yadav2023offline} Yadav, Karmesh, et al. "Offline visual representation learning for embodied navigation." Workshop on Reincarnating Reinforcement Learning at ICLR 2023. 2023.

\bibitem{yadav2023ovrl} Yadav, Karmesh, et al. "Ovrl-v2: A simple state-of-art baseline for imagenav and objectnav." arXiv preprint arXiv:2303.07798 (2023).

\bibitem{sun2024fgprompt} Sun, Xinyu, et al. "FGPrompt: fine-grained goal prompting for image-goal navigation." Advances in Neural Information Processing Systems 36 (2024).

\bibitem{zhu2017target} Zhu, Yuke, et al. "Target-driven visual navigation in indoor scenes using deep reinforcement learning." 2017 IEEE international conference on robotics and automation (ICRA). IEEE, 2017.

\bibitem{koh2024generating} Koh, Jing Yu, Daniel Fried, and Russ R. Salakhutdinov. "Generating images with multimodal language models." Advances in Neural Information Processing Systems 36 (2024).

\bibitem{krantz2022instance} Krantz, Jacob, et al. "Instance-specific image goal navigation: Training embodied agents to find object instances." arXiv preprint arXiv:2211.15876 (2022).

\bibitem{schulman2017proximal} Schulman, John, et al. "Proximal policy optimization algorithms." arXiv preprint arXiv:1707.06347 (2017).

\bibitem{anderson2018evaluation} Anderson, Peter, et al. "On evaluation of embodied navigation agents." arXiv preprint arXiv:1807.06757 (2018).

\bibitem{lin2024navcot} Lin, Bingqian, et al. "NavCoT: Boosting LLM-Based Vision-and-Language Navigation via Learning Disentangled Reasoning." arXiv preprint arXiv:2403.07376 (2024).

\bibitem{NavGPT} Zhou, Gengze, Yicong Hong, and Qi Wu. "Navgpt: Explicit reasoning in vision-and-language navigation with large language models." Proceedings of the AAAI Conference on Artificial Intelligence.

\bibitem{hahn2021no} Hahn, Meera, et al. "No rl, no simulation: Learning to navigate without navigating." Advances in Neural Information Processing Systems 34 (2021): 26661-26673.

\bibitem{li2025t2isafety} Li, Lijun, et al. "T2ISafety: Benchmark for Assessing Fairness, Toxicity, and Privacy in Image Generation." arXiv preprint arXiv:2501.12612 (2025).

\bibitem{an2024agfsync} An, Jingkun, et al. "AGFSync: Leveraging AI-Generated Feedback for Preference Optimization in Text-to-Image Generation." arXiv preprint arXiv:2403.13352 (2024).


\end{thebibliography}
\end{sloppypar}

\clearpage
\beginsupplement
\section*{Appendix}
\renewcommand{\thesubsection}{S\arabic{subsection}}

\subsection{\label{chap:S1}PanNuke and MoNuSAC preprocessing}
The PanNuke dataset comprises a set of 7,901 RGB patches, each with dimensions of $256 \times 256$ pixels, which we set as the standard patch size for our analysis. In contrast, the MoNuSAC dataset encompasses 294 images of heterogeneous dimensions. To standardize the MoNuSAC images with our experiments, we implement a standardization protocol. Specifically, for images exceeding the dimensions of $256 \times 256$ pixels, we segment them into equal-sized patches and apply mirror padding to the remaining portions to avoid information loss at the peripherals. Patches with dimensions less than $128 \times 128$ pixels are excluded from the dataset due to the insufficient resolution to capture relevant cellular details. For patches where either dimension falls between 128 and 256 pixels, we employ upsampling to achieve the standard patch size. As a result, we obtain a total of 2,823 RGB patches derived from the MoNuSAC dataset for subsequent analysis. For additional details on the MoNuSAC data preparation process, refer to the source code \cite{Shvetsov_2025a}.
\clearpage

\subsection{\label{chap:S2}Data usage for the methodology}

\counterwithin{figure}{subsection}
\renewcommand{\thefigure}{S\arabic{subsection}}

\begin{figure}[h!]
    \centering
    \includegraphics[width=\textwidth, height=0.85\textheight, keepaspectratio]{images/A2.pdf}
    \caption{Overview of the methodology for cross-labeling, dataset refinement, and model comparison. (1) Cross-relabeling - training and testing cell classification models, (2) Cross-relabeling - using cell classification models to create refined dataset, (3) Fine-tuning and training models for comparison, (4) Student knowledge distillation with refined dataset}
    \label{fig:S2}
\end{figure}
\clearpage

\subsection{\label{chap:S3}Confusion matrices for classification models}
\counterwithin{figure}{subsection}
\renewcommand{\thefigure}{S\arabic{subsection}.\arabic{figure}}

\begin{figure}[h!]
    \centering
    \includegraphics[width=\textwidth, height=0.4\textheight, keepaspectratio]{images/A3_1.pdf}
    \caption{Confusion matrix for PanNuke trained model}
    \label{fig:S3.1}
\end{figure}

\begin{figure}[h!]
    \centering
    \includegraphics[width=\textwidth, height=0.4\textheight, keepaspectratio]{images/A3_2.pdf}
    \caption{Confusion matrix for MoNuSAC trained model}
    \label{fig:S3.2}
\end{figure}

\clearpage

\subsection{\label{chap:S4}Datasets cell counts}

\counterwithin{table}{subsection}
\renewcommand{\thetable}{S\arabic{subsection}}

\begin{table}[h!]
\renewcommand{\arraystretch}{2.0}
\centering
\caption{\label{tab:S4}Cell counts for PanNuke, MoNuSAC and refined datasets. Numbers in parentheses indicate preprocessed cell counts for cell classifier models training and testing.}
%\adjustbox{max width=\textwidth}{%
\begin{tabular}{|l|c|c|c|}
\hline
%\rowcolor{gray!30}
Cell type & PanNuke & MoNuSAC & Refined \\
\hline
Neoplastic & 77,403 (68,031) & - & 105,451 \\
\hline
Epithelial & 26,572 (23,207) & - & 29,926 \\
\hline
Epithelial (benign and malignant) & - & 31,402 & - \\
\hline
Inflammatory & 32,276 & - & - \\
\hline
Lymphocytes & - & 37,045 (33,104) & 65,275 \\
\hline
Neutrophils & - & 1,355 (1,252) & 3,833 \\
\hline
Macrophage & - & 1,842 (1,695) & 3,410 \\
\hline
Dead & 2,908 & - & 2,908 \\
\hline
Connective & 50,585 & - & 50,585 \\
\hline
\end{tabular}
%
%}
\end{table}



\clearpage

\subsection{\label{chap:S5}Definition of validation metrics}
\counterwithin{equation}{subsection}
\renewcommand{\theequation}{\arabic{equation}}

\subsubsection{\label{chap:S5.1}R\textsuperscript{2}}
The coefficient of determination, denoted as $R^2$, is a statistical measure that represents the proportion of variance in the dependent variable that is predictable from the independent variables. In the context of cell quantification in pathology, $R^2$ is used to assess how well the predicted quantities of different cell types in a patch align with the actual quantities observed in the ground truth data, with higher values representing more accurate quantification. $R^2$ is defined as
\begin{equation*}
R^2 = 1 - \frac{\sum_{i=1}^n (y_i - \hat{y}_i)^2}{\sum_{i=1}^n (y_i - \bar{y})^2},
\end{equation*}
where $y_i$ represents the actual number of cells of a specific type in the $i$-th image, $\hat{y}_i$ represents the predicted number of cells of that type in the $i$-th image, $\bar{y}$ is the mean of the actual numbers across all images, and $n$ is the total number of images in the dataset.

The $R^2$ metric has a range of $(-\infty, 1]$. An $R^2$ of 1 indicates perfect prediction, where all predicted values exactly match the actual values. An $R^2$ of 0 suggests that the model explains none of the variability of the response data around its mean. If $R^2$ is negative, it indicates that the model performs worse than a model that simply predicts the mean of the actual values for all observations.

\subsubsection{\label{chap:S5.2}PQ}
Panoptic Quality ($PQ$) is a comprehensive metric used to evaluate the performance of segmentation models in tasks that require both instance segmentation and classification. $PQ$ provides a single score that encapsulates both the detection accuracy (i.e., how many objects were correctly identified) and the segmentation quality (i.e., how accurately the objects' boundaries were delineated). This metric is particularly useful in multiclass scenarios where each pixel is classified into distinct categories, such as different cell types in pathology images.

$PQ$ is calculated as the product of two terms: Detection Quality ($DQ$) and Segmentation Quality ($SQ$). It can be expressed as
\begin{equation*}
PQ = DQ \cdot SQ,
\end{equation*}
where
\begin{equation*}
DQ = \frac{TP}{TP + 0.5\, FP + 0.5\, FN},
\end{equation*}
\begin{equation*}
SQ = \frac{\sum_{(p, g) \in \mathcal{M}} IoU(p, g)}{TP}.
\end{equation*}
In these formulas, $TP$ denotes the number of correctly matched instances between ground truth and prediction, $FP$ denotes the predicted instances that have no corresponding ground truth, $FN$ denotes the ground truth instances that were not detected, $IoU(p, g)$ is the Intersection over Union for a pair of matched instances $p$ (prediction) and $g$ (ground truth), and $\mathcal{M}$ is the set of matched pairs.

The $PQ$ metric is calculated for each class and is averaged across classes to provide a global performance measure.

The $PQ$ score has a range of $[0, 1.0]$, where a higher score indicates better performance in both detecting and segmenting the instances correctly. A $PQ$ of 1 signifies perfect identification and segmentation of all instances, whereas a $PQ$ of 0 indicates that no instances were correctly identified and segmented.

\clearpage

\subsection{\label{chap:S6}Segmentation and Detection quality metrics for teacher and student models}

\begin{table}[h!]
\renewcommand{\arraystretch}{2.0}
\centering
\caption{Segmentation and detection quality for student and teacher models (CI 95\%)}
\label{tab:S6}
%\adjustbox{max width=\textwidth}{%
\begin{tabular}{|l|c|c|}
\hline
%\rowcolor{gray!30}
Metric & Teacher & Student \\
\hline
$SQ_{neoplastic}$ & 0.819 (0.815--0.823) & 0.824 (0.819--0.828) \\
\hline
$SQ_{lymphocyte}$ & 0.795 (0.788--0.802) & 0.790 (0.783--0.796) \\
\hline
$SQ_{connective}$ & 0.770 (0.762--0.776) & 0.780 (0.772--0.786) \\
\hline
$SQ_{dead}$ & 0.659 (0.623--0.688) & 0.657 (0.624--0.695) \\
\hline
$SQ_{epithelial}$ & 0.780 (0.770--0.790) & 0.788 (0.779--0.797) \\
\hline
$SQ_{macrophage}$ & 0.788 (0.760--0.810) & 0.757 (0.730--0.783) \\
\hline
$SQ_{neutrofil}$ & 0.782 (0.761--0.801) & 0.775 (0.759--0.792) \\
\hline
$DQ_{neoplastic}$ & 0.706 (0.692--0.719) & 0.727 (0.712--0.741) \\
\hline
$DQ_{lymphocyte}$ & 0.675 (0.656--0.698) & 0.713 (0.691--0.734) \\
\hline
$DQ_{connective}$ & 0.566 (0.546--0.584) & 0.583 (0.565--0.602) \\
\hline
$DQ_{dead}$ & 0.410 (0.361--0.465) & 0.435 (0.306--0.561) \\
\hline
$DQ_{epithelial}$ & 0.668 (0.639--0.694) & 0.673 (0.644--0.702) \\
\hline
$DQ_{macrophage}$ & 0.657 (0.583--0.727) & 0.615 (0.531--0.703) \\
\hline
$DQ_{neutrofil}$ & 0.691 (0.625--0.753) & 0.729 (0.679--0.778) \\
\hline
\end{tabular}
%
%}
\end{table}

\clearpage

\subsection{\label{chap:S7}QuPath integration method}
We adopt an integration strategy leveraging the paquo \cite{Bayer_AG} library, a Python package that enables direct interaction with QuPath’s internal API, thereby facilitating seamless data exchange without intermediate conversion steps. The data processing pipeline (\hyperref[fig:S7]{Appendix Figure S7}) begins with the acquisition of WSIs and their associated annotations from QuPath, which are represented as Shapely \cite{Gillies_Wel_etal._2024} polygons. Utilizing paquo, we directly read, create, and modify these annotations and detections within a QuPath project in the Python environment. Images are then cropped using these polygons and processed by cell segmentation and classification models employing standard vision processing toolkits such as OpenCV, pyvips, and PyTorch. Additionally, QuPath employs Groovy scripts to initiate a Python process that starts the entire pipeline from QuPath graphical interface: fetching polygons, extracting images from them, and running deep learning model inference on the cropped images. 
The results are returned to QuPath, leveraging paquo's Python bindings to manipulate QuPath data while minimizing the computational overhead typically associated with cross-environment communication.

\counterwithin{figure}{subsection}
\renewcommand{\thefigure}{S\arabic{subsection}}

\begin{figure}[h!]
    \centering
    \includegraphics[width=\textwidth]{images/A7.pdf}
    \caption{QuPath integration workflow using Python environment}
    \label{fig:S7}
\end{figure}

Compared to traditional workflows that involve exporting annotations as GeoJSON, classifying them in Python, and reimporting them into QuPath, our approach offers several advantages. We eliminate the need to switch between programming languages, providing a cohesive and streamlined development process entirely within QuPath software and removing the necessity to use other tools. Meanwhile, we avoid storing annotations as intermediate JSON files unless required for external use or archiving. By conducting the entire inference and post-processing workflow within the Python environment, we leverage the power and flexibility of Python libraries for image processing and machine learning. This approach also enables adjustments to any set of labels and models, thereby improving its applicability.

%\hfill

The distilled model and QuPath integration code are packaged into a Docker container, enabling streamlined execution with the Docker engine. Detailed integration code and deployment instructions can be found in the GitHub repository \cite{Shvetsov_2025b}.

Despite these benefits, we acknowledge that the paquo library is a proof‑of‑concept project in its early development stage and has not been tested across all versions of QuPath.

\clearpage

\subsection{\label{chap:S8}Data and code availability statement}
All datasets, models, and code used in this study are publicly available and can be obtained from the repositories listed below. 
The PanNuke \cite{Gamper_Koohbanani_etal._2019} and MoNuSAC \cite{Verma_Kumar_etal._2021} datasets are publicly accessible, and download information along with detailed descriptions can be found in their respective articles. Preprocessing scripts for PanNuke and MoNuSAC data, as well as individual cell extraction scripts, are available on GitHub \cite{Shvetsov_2025a}. The H-Optimus foundation model used in our experiments can be downloaded from the HuggingFace repository \cite{hoptimus2024}, and model information is available on GitHub \cite{Saillard_Jenatton_etal._2024}. In addition, the integration code for QuPath and the distilled model packaged in a Docker container are provided in the repository \cite{Shvetsov_2025b}, and paquo Python library is available from the authors GitHub repository \cite{Bayer_AG}.
\clearpage

\end{document}

\end{appendix}

\end{document}
