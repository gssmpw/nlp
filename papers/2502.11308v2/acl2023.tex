% This must be in the first five lines to tell arXiv to use pdfLaTeX, which is strongly recommended.
\pdfoutput=1
% In particular, the hyperref package requires pdfLaTeX in order to break URLs across lines.

\documentclass[11pt]{article}

% Remove the "review" option to generate the final version.
% \usepackage[review]{ACL2023}
\usepackage{ACL2023}

% Standard package includes
\usepackage{times}
\usepackage{latexsym}
\usepackage{graphicx}
\usepackage{fancybox}
\usepackage{amsmath}
\usepackage{amssymb}
\usepackage{booktabs}
\usepackage{multirow}
\usepackage{booktabs} % For better table lines



% For proper rendering and hyphenation of words containing Latin characters (including in bib files)
\usepackage[T1]{fontenc}
% For Vietnamese characters
% \usepackage[T5]{fontenc}
% See https://www.latex-project.org/help/documentation/encguide.pdf for other character sets

% This assumes your files are encoded as UTF8
\usepackage[utf8]{inputenc}

% This is not strictly necessary, and may be commented out.
% However, it will improve the layout of the manuscript,
% and will typically save some space.
\usepackage{microtype}

% This is also not strictly necessary, and may be commented out.
% However, it will improve the aesthetics of text in
% the typewriter font.
\usepackage{inconsolata}

% % definition.
% \theoremstyle{definition}
% \newtheorem{definition}{Definition}[section]


% \theoremstyle{remark}
% \newtheorem*{remark}{Remark}

% \input{math_common}
\newcommand{\stap}{\bS_{\rm TAP}}
\newcommand{\slamp}{\bS_{\rm LAMP}}
\newcommand{\gout}{\bg_{\rm out}}

\newcommand{\Py}{\mathsf{Z}}
\newcommand{\I}{\mathbb{I}}
\newcommand{\Zout}{\Py}
\newcommand{\dgout}{\bG}

\newcommand{\bSigma}{\boldsymbol{\Sigma}}

% Probability
\renewcommand{\P}{\mathbb{P}}
\newcommand{\E}{\mathbb{E}}
\newcommand{\Var}{\text{Var}}
\newcommand{\Cov}{\mathrm{Cov}}
\newcommand{\cN}{\mathcal{N}}

% Sets
\newcommand{\Z}{\mathbb{Z}}
\newcommand{\R}{\mathbb{R}}
\newcommand{\C}{\mathbb{C}}
\newcommand{\N}{\mathbb{N}}
\renewcommand{\S}{\mathbb{S}}
\def\ball{{\mathsf B}}

% Variables
\newcommand{\eps}{\varepsilon} 
\newcommand{\vphi}{\varphi}
\def\id{{\mathbf I}}


% Math
\renewcommand{\d}{\textup{d}}
\renewcommand{\l}{\vert}
\newcommand{\dl}{\Vert}
\newcommand{\<}{\langle}
\renewcommand{\>}{\rangle}
\newcommand{\sign}{\text{sign}}
\newcommand{\diag}{\text{diag}}
%\newcommand{\tr}{\text{tr}}
%\newcommand{\op}{{\rm op}}
\newcommand{\ones}{\bm{1}}
\newcommand{\what}{\widehat}
%\newcommand{\grad}{\boldsymbol{\nabla}}
\def\sT{{\mathsf T}}
\def\bzero{{\boldsymbol 0}}
\newcommand{\bomega}{\boldsymbol{\omega}}
\newcommand{\bOmega}{\boldsymbol{\Omega}}
\newcommand{\flatten}{\operatorname{flat}}
\newcommand{\bcT}{\boldsymbol{\mathcal{T}}}


\DeclareMathOperator*{\argmin}{arg\,min}
\DeclareMathOperator*{\argmax}{arg\,max}
\DeclareMathOperator*{\argsup}{arg\,sup}
\DeclareMathOperator*{\arginf}{arg\,inf}
\newcommand{\eqnd}{\, {\buildrel d \over =} \,} 
\newcommand{\eqndef}{\mathrel{\mathop:}=}
\def\doteq{{\stackrel{\cdot}{=}}}
\newcommand{\goto}{\longrightarrow}
\newcommand{\gotod}{\buildrel d \over \longrightarrow} 
\newcommand{\gotoas}{\buildrel a.s. \over \longrightarrow} 
\def\simiid{{\stackrel{i.i.d.}{\sim}}}


% Notations 
\newcommand{\notate}[1]{\textcolor{red}{\textbf{[#1]}}}
\newcommand{\cc}[1]{\textcolor{blue}{\textbf{[CC:#1]}}}
\newcommand{\yw}[1]{\textcolor{pink}{\textbf{[YW:#1]}}}
\newcommand{\mc}[1]{\mathcal{#1}}
\newcommand{\mb}[1]{\mathbf{#1}}


% Theorem
\newtheorem{question}{Question}
\newtheorem{property}{Property}
\newtheorem{objective}{Objective}
\newtheorem{claim}{Claim}
\newtheorem{example}{Example}



%\usepackage[inline]{showlabels}

\DeclareSymbolFont{rsfs}{U}{rsfs}{m}{n}
\DeclareSymbolFontAlphabet{\mathscrsfs}{rsfs}



% Bold symbols
\def\bA{{\boldsymbol A}}
\def\bB{{\boldsymbol B}}
\def\bC{{\boldsymbol C}}
\def\bD{{\boldsymbol D}}
\def\bE{{\boldsymbol E}}
\def\bF{{\boldsymbol F}}
\def\bG{{\boldsymbol G}}

\def\bH{{\boldsymbol H}}
\def\bI{{\boldsymbol I}}
\def\bJ{{\boldsymbol J}}
\def\bK{{\boldsymbol K}}
\def\bL{{\boldsymbol L}}
\def\bM{{\boldsymbol M}}
\def\bN{{\boldsymbol N}}
\def\bO{{\boldsymbol O}}
\def\bP{{\boldsymbol P}}
\def\bQ{{\boldsymbol Q}}
\def\bR{{\boldsymbol R}}
\def\bS{{\boldsymbol S}}
\def\bT{{\boldsymbol T}}
\def\bU{{\boldsymbol U}}
\def\bV{{\boldsymbol V}}
\def\bW{{\boldsymbol W}}
\def\bX{{\boldsymbol X}}
\def\bY{{\boldsymbol Y}}
\def\bZ{{\boldsymbol Z}}

\def\ba{{\boldsymbol a}}
\def\bb{{\boldsymbol b}}
\def\be{{\boldsymbol e}}
\def\boldf{{\boldsymbol f}}
\def\bg{{\boldsymbol g}}
\def\bh{{\boldsymbol h}}
\def\bi{{\boldsymbol i}}
\def\bj{{\boldsymbol j}}
\def\bk{{\boldsymbol k}}
\def\bt{{\boldsymbol t}}
\def\bu{{\boldsymbol u}}
\def\bv{{\boldsymbol v}}
\def\bw{{\boldsymbol w}}
\def\bx{{\boldsymbol x}}
\def\by{{\boldsymbol y}}
\def\bz{{\boldsymbol z}}

\def\bmu{{\boldsymbol \mu}}
\def\bbeta{{\boldsymbol \beta}}
\def\bdelta{{\boldsymbol\delta}}
\def\beps{{\boldsymbol \eps}}
\def\blambda{{\boldsymbol \lambda}}
\def\bpsi{{\boldsymbol \psi}}
\def\bphi{{\boldsymbol \phi}}
\def\btheta{{\boldsymbol \theta}}
\def\bvphi{{\boldsymbol \vphi}}
\def\bxi{{\boldsymbol \xi}}

\def\bDelta{{\boldsymbol \Delta}}
\def\bLambda{{\boldsymbol \Lambda}}
\def\bPsi{{\boldsymbol \Psi}}
\def\bPhi{{\boldsymbol \Phi}}
\def\bSigma{{\boldsymbol \Sigma}}
\def\bTheta{{\boldsymbol \Theta}}

\def\bfzero{{\boldsymbol 0}}
\def\bfone{{\boldsymbol 1}}
\def\bPi{{\boldsymbol \Pi}}


% Symbols with hat
\def\hba{{\hat {\boldsymbol a}}}
\def\hf{{\hat f}}
\def\ha{{\hat a}}
\def\tcT{\widetilde{\mathcal T}}
\def\tK{\widetilde{K}}


\def\cR{\mathcal{R}}
\def\test{{\rm test}}
\def\train{{\rm train}}
\def\CV{\text{CV}}
\def\GCV{\text{GCV}}
\def\sfs{{\sf s}}

% rm symbols
\def\spn{{\rm span}}
\def\supp{{\rm supp}}
\def\Easy{{\rm E}}
\def\Hard{{\rm H}}
\def\post{{\rm post}}
\def\pre{{\rm pre}}
\def\Rot{{\rm Rot}}
\def\Sft{{\rm Sft}}
\def\endd{{\rm end}}
\def\KR{{\rm KR}}
\def\bbHe{{\rm He}}
\def\sk{{\rm sk}}
\def\de{{\rm d}}
\def\Tr{{\rm Tr}}
\def\lin{{\rm lin}}
\def\res{{\rm res}}
\def\degzero{{\rm deg0}}
\def\degone{{\rm deg1}}
\def\Poly{{\rm Poly}}
\def\Poly{{\rm Poly}}
\def\Coeff{{\rm Coeff}}
\def\de{{\rm d}}
\def\Unif{{\rm Unif}}
\def\lin{{\rm lin}}
\def\res{{\rm res}}
\def\RF{{\rm RF}}
\def\NT{{\rm NT}}
\def\Cyc{{\rm Cyc}}
\def\RC{{\rm RC}}
\def\kernel{\rm Ker}
\def\image{{\rm Im}}
\def\Easy{{\rm E}}
\def\Hard{{\rm H}}
\def\post{{\rm post}}
\def\pre{{\rm pre}}
\def\Rot{{\rm Rot}}
\def\Sft{{\rm Sft}}
\def\ddiag{{\rm ddiag}}
\def\KR{{\rm KR}}
\def\RR{{\rm RR}}
\def\bbHe{{\rm He}}
\def\eff{{\rm eff}}

\def\spn{{\rm span}}


%mathcal symbols
\def\cV{{\mathcal V}}
\def\cG{{\mathcal G}}
\def\cO{{\mathcal O}}
\def\cP{{\mathcal P}}
\def\cW{{\mathcal W}}
\def\cT{{\mathcal T}}
\def\cC{{\mathcal C}}
\def\cQ{{\mathcal Q}}
\def\cL{{\mathcal L}}
\def\cF{{\mathcal F}}
\def\cE{{\mathcal E}}
\def\cS{{\mathcal S}}
\def\cI{{\mathcal I}}
\def\cV{{\mathcal V}}
\def\cG{{\mathcal G}}
\def\cO{{\mathcal O}}
\def\cP{{\mathcal P}}
\def\cW{{\mathcal W}}
\def\cT{{\mathcal T}}
\def\cH{{\mathcal H}}
\def\cA{{\mathcal A}}


\def\tbA{\Tilde \bA}

%mathbb mathsf sf symbols
\def\K{{\mathbb K}}
\def\H{{\mathbb H}}
\def\T{{\mathbb T}}
\def\bbV{{\mathbb V}}
\def\W{{\mathbb W}}
\def\sM{{\mathsf M}}
\def\sW{{\mathsf W}}
\def\Unif{{\sf Unif}}
\def\normal{{\sf N}}
\def\proj{{\mathsf P}}
\def\ik{{\mathsf k}}
\def\il{{\mathsf l}}
\def\sM{{\sf M}}
\def\RKHS{{\sf RKHS}}
\def\RF{{\sf RF}}
\def\NT{{\sf NT}}
\def\NN{{\sf NN}}
\def\reals{{\mathbb R}}
\def\integers{{\mathbb Z}}
\def\naturals{{\mathbb N}}
\def\Top{{\mathbb T}}
\def\Kop{{\mathbb K}}
\def\Aop{{\mathbb A}}
\def\normal{{\sf N}}
\def\proj{{\mathsf P}}
\def\bbV{{\mathbb V}}
\def\sW{{\mathsf W}}
\def\sM{{\mathsf M}}
\def\T{{\mathbb T}}
\def\K{{\mathbb K}}
\def\H{{\mathbb H}}
\def\Unif{{\sf Unif}}
\def\normal{{\sf N}}
\def\Uop{{\mathbb U}}
\def\Hop{{\mathbb H}}
\def\Sop{{\mathbb S}}
\def\proj{{\mathsf P}}
\def\ik{{\mathsf k}}
\def\il{{\mathsf l}}
\def\sM{{\sf M}}
\def\RKHS{{\sf RKHS}}
\def\RF{{\sf RF}}
\def\NT{{\sf NT}}
\def\NN{{\sf NN}}
\def\reals{{\mathbb R}}
\def\integers{{\mathbb Z}}
\def\naturals{{\mathbb N}}
\def\proj{{\mathsf P}}
\def\Hop{{\mathbb H}}
\def\Uop{{\mathbb U}}
\def\App{{\rm App}}
\def\sU{{\sf U}}
\def\sV{{\sf V}}
\def\sfp{{\sf p}}
\def\tcE{\widetilde{\cE}}
\def\tmu{\widetilde  \mu}
\def\tbD{\widetilde{\bD}}




\def\stest{\mbox{\tiny\rm test}}

\def\seff{\mbox{\tiny\rm eff}}

\def\Ker{K}
\def\tKer{\tilde{K}}
\def\oKop{\overline{{\mathbb K}}}
\def\oKer{\overline{K}}
\def\ocV{\overline{{\mathcal V}}}

\def\th{\tilde{h}}
\def\tQ{\tilde{Q}}
\def\tsigma{\Tilde{\sigma}}


\def\hba{{\hat {\boldsymbol a}}}
\def\hf{{\hat f}}
\def\hy{{\hat y}}
\def\hU{\widehat{U}}
\def\hUop{\widehat{\mathbb U}}
\def\tbDelta{\widetilde{\bDelta}}


\def\tcT{\widetilde{\mathcal T}}

\def\Cyc{{\rm Cyc}}
\def\inv{{\rm inv}}


\def\cE{{\mathcal E}}
\def\cD{{\mathcal D}}
\def\cX{{\mathcal X}}
\def\cF{{\mathcal F}}
\def\cS{{\mathcal S}}
\def\cI{{\mathcal I}}



\def\He{{\rm He}}
\def\lin{{\rm lin}}
\def\res{{\rm res}}
\def\degzero{{\rm deg0}}
\def\degone{{\rm deg1}}
\def\Poly{{\rm Poly}}
\def\Coeff{{\rm Coeff}}
\def\de{{\rm d}}
\def\Unif{{\rm Unif}}
\def\RF{{\rm RF}}
\def\NT{{\rm NT}}
\def\Cyc{{\rm Cyc}}
\def\RC{{\rm RC}}

\def\tK{\widetilde{K}}
\def\stest{\mbox{\tiny\rm test}}


\def\ttau{\tilde{\tau}}


\def\cE{{\mathcal E}}
\def\bt{{\boldsymbol t}}
\def\normal{{\sf N}}

\def\bDelta{{\boldsymbol \Delta}}










\def\cX{{\mathcal X}}
\def\CKR{{\rm CKR}}
\def\bproj{{\overline \proj}}
\def\quadratic{{\rm quad}}
\def\cube{{\rm cube}}
\def\Cube{{\mathscrsfs Q}}

\def\Poly{{\rm Poly}}
\def\Coeff{{\rm Coeff}}
\def\RF{{\rm RF}}
\def\NT{{\rm NT}}
\def\bA{{\boldsymbol A}}
\def\btheta{{\boldsymbol \theta}}
\def\bTheta{{\boldsymbol \Theta}}
\def\bLambda{{\boldsymbol \Lambda}}
\def\blambda{{\boldsymbol \lambda}}

\def\cM{{\mathcal M}}

\def\cT{{\mathcal T}}
\def\cV{{\mathcal V}}
\def\bP{{\boldsymbol P}}
\def\diag{{\rm diag}}
\def\bS{{\boldsymbol S}}
\def\bO{{\boldsymbol O}}
\def\bD{{\boldsymbol D}}
\def\bPsi{{\boldsymbol \Psi}}
\def\bsh{{\boldsymbol h}}
\def\bL{{\boldsymbol L}}



\def\osigma{\overline{\sigma}}
\def\tbu{\Tilde \bu}
\def\tbZ{\Tilde \bZ}
\def\tbphi{\Tilde \bphi}
\def\tbpsi{\Tilde \bpsi}

\def\tbf{\Tilde \boldf}
\def\hbU{\hat{{\boldsymbol U}}_\lambda }
\def\hbUi{\hat{{\boldsymbol U}}_\lambda^{-1} }
\def\bb{{\boldsymbol b}}
\def\bsigma{{\boldsymbol \sigma}}

\def\hf{\hat f}
\def\hbf{\hat \boldf}
\def\bR{{\boldsymbol R}}
\def\bpsi{{\boldsymbol \psi}}
\def\cuH{\mathscrsfs{H}}

\def\noisestd{\sigma_{\varepsilon}}

\def\evn{{\mathsf m}}
\def\evN{{\mathsf M}}

\def\lvn{{\mathsf s}}
\def\lvN{{\mathsf S}}

\def\bc{{\boldsymbol c}}
\def\bC{{\boldsymbol C}}
\def\oba{\overline{{\boldsymbol a}}}
\def\uba{\underline{{\boldsymbol a}}}

\def\barsigma{\bar{\sigma}}

\def\tbN{\Tilde \bN}
\def\dv{{D}}

\def\tbV{\Tilde \bV}
\def\hiota{{\hat \iota}}
\def\biota{{\boldsymbol \iota}}
\def\hbiota{{\hat {\boldsymbol \iota}}}

\def\bzeta{{\boldsymbol \zeta}}
\def\hbzeta{{\hat {\boldsymbol \zeta}}}
\def\oproj{{\overline \proj}}
\def\barHop{\bar{\Hop}}
\def\barUop{\bar{\Uop}}
\def\barU{\bar{U}}
\def\barH{\bar{H}}
\def\ind{\mathbbm{1}}

\def\tC{\Tilde C}
\def\tQ{\Tilde Q}
\def\balpha{\boldsymbol{\alpha}}
\def\bgamma{\boldsymbol{\gamma}}
\def\cU{\mathcal{U}}
\def\tbC{\Tilde \bC}
\def\tba{\Tilde \ba}
\def\tbeta{\Tilde \beta}
\def\tbbeta{\Tilde \bbeta}
\def\boldf{\boldsymbol{f}}
\def\bXi{\boldsymbol{\Xi}}
\def\cB{\mathcal{B}}
\def\MP{{\rm MP}}
\def\complex{\mathbbm{C}}
\def\Im{{\rm Im}}
\def\tbM{\Tilde \bM}

\def\sR{\mathsf R}
\def\sV{\mathsf V}
\def\sB{\mathsf B}

\def\obR{\overline{\bR}}
\def\obM{\overline{\bM}}
\def\wbM{\widetilde{\bM}}
\def\tbR{\widetilde{\bR}}
\def\tbM{\widetilde{\bM}}

\def\ulambda{\overline{\lambda}}
\def\hbtheta{\hat \btheta}
\def\rr{{\rm r}}

\def\rC{\textcolor{red}{C}}

\def\rSQ{{\rm SQ}}

\def\rdc{{\rm dc}}
\def\rmc{{\rm mc}}
\def\cY{\mathcal{Y}}
\def\cZ{\mathcal{Z}}
\def\rdeg{{\rm deg}}


\def\dom{{\rm dom}}
\def\prox{{\rm prox}}
\def\hE{\widehat{\E}}
\def\okappa{\overline{\kappa}}
\def\otau{\overline{\tau}}
\def\br{{\boldsymbol r}}
\def\bGamma{{\boldsymbol \Gamma}}
\def\cJ{\mathcal{J}}
\def\oxi{\overline{\xi}}
\def\hbalpha{\hat{\balpha}}
\def\sfG{\textsf{G}}
\def\sfMG{\textsf{MG}}
\def\obz{\overline{\bz}}
\def\obZ{\overline{\bZ}}
\def\obg{\overline{\bg}}
\def\obG{\overline{\bG}}
\def\tbU{\Tilde{\bU}}
\def\obx{\overline{\bx}}
\def\ox{\overline{x}}



\def\tC{\Tilde C}
\def\tQ{\Tilde Q}
\def\balpha{\boldsymbol{\alpha}}
\def\bgamma{\boldsymbol{\gamma}}
\def\cU{\mathcal{U}}
\def\tbC{\Tilde \bC}
\def\tba{\Tilde \ba}
\def\tbeta{\Tilde \beta}
\def\tbbeta{\Tilde \bbeta}
\def\boldf{\boldsymbol{f}}
\def\bXi{\boldsymbol{\Xi}}
\def\cB{\mathcal{B}}
\def\MP{{\rm MP}}
\def\complex{\mathbbm{C}}
\def\Im{{\rm Im}}
\def\tbM{\Tilde \bM}

\def\sR{\mathsf R}
\def\sV{\mathsf V}
\def\sB{\mathsf B}

\def\ulambda{\overline{\lambda}}
\def\hbtheta{\hat \btheta}
\def\oPhi{\overline{\Phi}}
\def\sfPhi{\mathsf \Phi}

\def\hbSigma{\hat{\bSigma}}
\def\sfC{{\sf C}}
\def\sfc{{\sf c}}
\def\sfD{{\sf D}}
\def\sfM{{\sf M}}
\def\rmI{{\rm I}}
\def\rmII{{\rm II}}
\def\obQ{\overline{\bQ}}
\def\tS{\widetilde{S}} 
\def\tbS{\widetilde{\bS}}  
\def\obtheta{\overline{\btheta}}
\def\onu{\overline{\nu}}
\def\oT{\overline{T}}
\def\sL{\mathsf{L}}
\def\bq{\boldsymbol{q}}
\def\og{\overline{g}}
\def\oq{\overline{q}}
\def\ske{{\sf ske}}
\def\bs{{\boldsymbol s}}
\def\obD{\overline{\bD}}
\def\osfD{{\overline{{\sf D}}}}
\def\sflf{{\sf leaf}}
\def\sfT{{\sf T}}
\def\sfG{{\sf G}}
\def\bsfT{{\boldsymbol \sfT}}
\def\bsfG{{\boldsymbol \sfG}}
\def\obi{\overline{\bi}}
\def\obsfT{\overline{\bsfT}}
\def\obsfG{\overline{\bsfG}}
\def\oi{\overline{i}}
\def\osfT{\overline{\sfT}}
\def\osfG{\overline{\sfG}}
\def\sfH{{\sf H}}
\def\tbD{\widetilde{\bD}}
\def\polylog{\text{polylog}}
\def\tcL{{\widetilde{\cL}}}
\def\tsL{{\widetilde{\sL}}}

\def\seff{{\sf eff}}
\def\sG{\mathsf{G}}
\def\sKL{\mathsf{KL}}
\def\oevn{\overline{\evn}}
\def\obeta{\overline{\beta}}
\def\oC{\overline{C}}

\def\tnu{\Tilde{\nu}}
\def\hbSigma{\widehat{\bSigma}}
\def\tmu{\Tilde{\mu}}
\def\sK{{\sf K}}
\def\sA{{\sf A}}
\def\tPhi{\widetilde{\Phi}}
\def\obF{\overline{\bF}}
\def\oboldf{\overline{\boldf}}
\def\tr{\widehat{r}}
\def\hxi{\hat{\xi}}
\def\hr{\widehat{r}}
\def\hrho{\widehat{\rho}}
\def\trho{\widetilde{\rho}}
\def\tcA{\widetilde{\cA}}
\def\obv{\overline{\bv}}
\def\tsB{\widetilde{\sB}}
\def\tbG{\widetilde{\bG}}


\newcommand{\G}{\mathbf{G}}
\newcommand{\GT}{\mathbf{G}^\top}
\newcommand{\bet}{\boldsymbol{\beta}}
\newcommand{\U}{\mathbf{U}}
\newcommand{\V}{\mathbf{V}}
\newcommand{\D}{\mathbf{D}}
%\newcommand{\R}{\mathbb{R}}
%\newcommand{\E}{\mathbb{E}}
\newcommand{\Sph}{\mathbb{S}}
%\newcommand{\I}{\mathbb{I}}
%\newcommand{\Pr}{\mathbb{P}}
%\newcommand{\bx}{\boldsymbol{x}}
%\newcommand{\bw}{\boldsymbol{w}}
%\newcommand{\bz}{\boldsymbol{z}}
\newcommand{\bblV}{{\color{blue}\bV}}
\newcommand{\ourmethod}{ALGEN}

% \usepackage{fontspec}
% \usepackage{emoji}




% If the title and author information does not fit in the area allocated, uncomment the following
%
%\setlength\titlebox{<dim>}
%
% and set <dim> to something 5cm or larger.

% \title{Few-shot Textual Embedding Inversion: Cross-model Embedding Alignment}
% \title{Few-shot Textual Embedding Inversion using Victim-Attacker Embedding Alignment}
% Embeddings are Invertible$^{+}$ ?
% \title{ Few-shot Textual Embedding Inversion Attacks using Embedding Alignment}
\title{ ALGEN: Few-shot Inversion Attacks on Textual Embeddings \\using Alignment and Generation}

% Author information can be set in various styles:
% For several authors from the same institution:
% \author{Author 1 \and ... \and Author n \\
%         Address line \\ ... \\ Address line}
% if the names do not fit well on one line use
%         Author 1 \\ {\bf Author 2} \\ ... \\ {\bf Author n} \\
% For authors from different institutions:
% \author{Author 1 \\ Address line \\  ... \\ Address line
%         \And  ... \And
%         Author n \\ Address line \\ ... \\ Address line}
% To start a seperate ``row'' of authors use \AND, as in
% \author{Author 1 \\ Address line \\  ... \\ Address line
%         \AND
%         Author 2 \\ Address line \\ ... \\ Address line \And
%         Author 3 \\ Address line \\ ... \\ Address line}

\author{
  Yiyi Chen\textsuperscript{1},\space
  Qiongkai Xu\textsuperscript{2}\thanks{\ \ Corresponding author.},\space
  Johannes Bjerva\textsuperscript{1} \\
  \textsuperscript{1}Department of Computer Science, Aalborg University, Copenhagen, Denmark \\
  \textsuperscript{2}School of Computing, FSE, Macquarie University, Sydney, Australia \\
  % \textsuperscript{3}School of Computing and Information System, the University of Melbourne, Australia \\
  \texttt{yiyic@cs.aau.dk, qiongkai.xu@mq.edu.au, jbjerva@cs.aau.dk}
}
% \author{Yiyi Chen\\\And Qiongkai Xu \\\And Johannes Bjerva\\

%         Department of Computer Science, Aalborg University, Copenhagen, Denmark \\
%         {yiyic, jbjerva}@cs.aau.dk
%         }

        
% \author{Yiyi Chen \\
%   Affiliation / Address line 1 \\
%   Affiliation / Address line 2 \\
%   Affiliation / Address line 3 \\
%   \texttt{email@domain} \\\And
%   Qiongkai Xu~\thanks{ Corresponding author} \\ % corresponding author.
%   Affiliation / Address line 1 \\
%   Affiliation / Address line 2 \\
%   Affiliation / Address line 3 \\
%   \texttt{email@domain} \\\And
%   Johannes Bjerva \\
%   Affiliation / Address line 1 \\
%   Affiliation / Address line 2 \\
%   Affiliation / Address line 3 \\
%   \texttt{email@domain} \\
% }

\begin{document}

\maketitle
\begin{abstract}
% broader importance of the embeddings inversion attacks.
With the growing popularity of Large Language Models (LLMs) and vector databases, private textual data is increasingly processed and stored as numerical embeddings.
However, recent studies have proven that such embeddings are vulnerable to inversion attacks, where original text is reconstructed to reveal sensitive information.
Previous research has largely assumed access to millions of sentences to train attack models, e.g., through data leakage or nearly unrestricted API access.
With our method, \emph{a single data point} is sufficient for a partially successful inversion attack.
With as little as 1k data samples, performance reaches an optimum across a range of black-box encoders, without training on leaked data.
We present a Few-shot Textual Embedding Inversion Attack using \textbf{AL}ignment and \textbf{GEN}eration (\textbf{\ourmethod}), 
by aligning victim embeddings to the attack space and using a generative model to reconstruct text.
We find that \textbf{\ourmethod} attacks can be effectively transferred across domains and languages, revealing key information.
We further examine a variety of defense mechanisms against \textbf{\ourmethod}, and find that none are effective, highlighting the vulnerabilities posed by inversion attacks.
By significantly lowering the cost of inversion and proving that embedding spaces can be aligned through one-step optimization, we establish a new textual embedding inversion paradigm with broader applications for embedding alignment in NLP.\footnote{We open-source our code \url{https://github.com/siebeniris/ALGEN}.}

% Previous research in this area has largely assumed access to a substantial amount of data, either through leakage or nearly unrestricted access to the model API, and has consequently developed compute-intensive techniques in this attack space.  
%\xqk{Add a sentence on our achievement. The significance of this work. For example, we significantly reduce the requirement on training attack model.}

% by employing one-step linear scaling on victim embeddings to align to the attack embedding space, requiring only minimal data leakage.
% Moreover, we find that \textbf{\ourmethod} attacks 
% can be effectively transferred across different domains and languages.
%We examine a variety of defense mechanisms pertinent to textual embeddings, we find that applying metric-based local differential privacy (DP) proves to effectively counter the inversion attack while preserving embedding utility.
% By significantly lowering the cost of inversion and proving that embedding spaces can be aligned with one-step optimization, this work establishes a new paradigm for textual embedding inversion with broader applications in embedding alignment in natural language processing (NLP).\footnote{We open-source our code.}

\end{abstract}



\begin{figure}[t!]
    \centering
    \includegraphics[width=\linewidth]{images/Figure1_04.pdf}
    \caption{An illustration of inversion attacks on textual embeddings stored in a vector DB, in scenarios where (I) a user exploits API access to extract excessive embeddings to train attack model; (II) a generative AI agent's interaction channel with the DB is compromised; (III) the DB is misconfigured by an insider to expose private embeddings. }
    % \caption{An illustration of inversion attacks on textual embeddings stored in (II) a vector database in (I) a generative AI Agent, when attackers are given access to the communication channel in \textcolor{red}{red} arrows.\xqk{please add `(I)' and `(II)' to the figure and mark the malicious arrows red. I would also suggest to add tags of 1-4 to the figure to make it more informative. Sometimes readers would like to know the key idea by scanning the figure only, although this is not a suggested practice.}}
    \label{fig:schema}
\end{figure}


\section{Introduction}
Large Language Models (LLMs) such as OpenAI's GPT series~\citep{radford2018improving, radford2019language,brown2020languagemodelsfewshotlearners,openai2024gpt4technicalreport} and Claude from Anthropic,\footnote{\url{https://www.anthropic.com/claude}}  have become essential across a wide range of applications, extending far beyond natural language processing (NLP). 
These models are deeply integrated into people's daily lives and business operations, e.g., powering search engines, virtual assistants, and content generation. 
A critical component enabling the efficiency of these applications is vector databases (DB), which allow for fast and scalable retrieval and processing of high-dimensional vector representations. 
Companies such as Pinecone and Weaviate,
provide vector DB services and build AI services on top of them.\footnote{\url{https://www.pinecone.io}, \url{https://weaviate.io}}
% At a global scale, the vector DB market size was expected to grow even further from 2.46 billion US dollars value in 2024 for the next few years~\citep{vector_database_market_2024}.
% Vector databases are essential for efficiently storing high-dimensional embeddings, which enables advanced semantic search, facilitates more accurate context understanding, and ensures better scalability to manage large and ever-expanding datasets.
% One of the most prolific use cases is retrieval-augmented generation (RAG), where user queries are embedded via LLMs and sent to a vector DB for semantic search, from which the retrieved vector index is further leveraged for generating more diverse and factually grounded responses~\citep{rag_lewis_2020}. 
In a recent Google whitepaper on generative AI agents~\citep{agents_google},
vector DBs are considered one of the essential components enabling such agents through external sources. 
Retrieval-augmented generation (RAG) is another common use case in leveraging vector DBs to generate more diverse and factually grounded responses~\citep{rag_lewis_2020}.

While applications such as these benefit from vector DBs, the potential security and privacy risks permeate the process.
Fig.~\ref{fig:schema} illustrates three separate threat scenarios where a vector DB can be exploited to expose private and sensitive information:
(I) a malicious user exploits the model API to extract  embeddings to train an attack model;
(II) when an AI agent interacts with the DB, a malicious attacker can compromise the communication channel to intercept sensitive data;
%or exploit prompt injection to extract sensitive text and embeddings;
(III) a misconfigured vector DB may expose private data, either through access vulnerabilities or insider threats.
% Figure~\ref{fig:schema} illustrates an interaction loop of a RAG-based generative AI agent, where Data Stores are implemented as a vector database via which the agent can access various data (e.g., website content, project code) in runtime. To acquire responses from the AI Agent, each user (1) query is sent to an LLM and the generated embeddings are matched against the vector database; (2) the relevant content is retrieved from the database and sent back to the agent; (3) the agent processes the information to formulate a response or action; (4) the final response is sent to the user~\citep{agents_google}.
% a user query is sent to an LLM to generate embeddings; (2) the query embeddings are matched against a vector database using matching algorithms; (3) relevant content is retrieved from the database as text and sent back to the agent; (4) the agent processes the user query and retrieved content to formulate a response or action; (5) the final response is sent to the user.
% 
% In this process, a malicious attacker could intercept a body of texts and their corresponding embeddings during steps (2) and (3).
The attacker can train an attack model (\eg an embedding-to-text generator) to reconstruct text from intercepted embeddings, which might contain sensitive, private, or proprietary information. 
This so-called \textit{embedding inversion attack} poses significant risks and potential harm.

%the attacker can launch \textit{embedding inversion attacks} to reconstruct texts from the intercepted embeddings, which might contain sensitive, private, or proprietary information, posing significant risks and potential harm. 
%\xqk{Currently, AI Agent scenario does not motivate our attack well. Maybe accessing vector database and attackers collected samples from db or internet communication is enough.}
% % moved from abstract.
% Previous research in this area has largely assumed access to a substantial amount of data, either through leakage or nearly unrestricted access to the model API, and has consequently developed compute-intensive techniques in this attack space.  

Previous work has demonstrated the feasibility and detrimental effects of inversion attacks~\citep{10.1145/3372297.3417270, li-etal-2023-sentence}. %morris2023text,chen2024text}. 
However, either a massive amount of intercepted (victim) embeddings and their texts are required for training an attack model~\citep{ morris2023text}, or the attack is conducted under white-box settings, where the model parameters and architecture are known to the attacker~\citep{10.1145/3372297.3417270}. 
Moreover, inversion attacks have been demonstrated to threaten multiple languages, especially lower-resource ones~\citep{chen2024typ,chen2024text}.
% Moreover, almost all works focus solely on English embeddings except~\citep{chen2024typ,chen2024text}. 

We propose a Few-shot Textual Embedding Inversion Attack using \textbf{AL}ignment and \textbf{GEN}eration (\textbf{\ourmethod}), 
to first align victim embeddings to the attack embedding space, and then reconstruct text from the aligned embeddings using the generative attack model.
In contrast to previous work, we investigate inversion attacks using a small handful of samples -- e.g., a Rouge-L score of 10 can be reached by leveraging \emph{a single leaked data point}. 
%\textbf{ALGEN} can be operated in a realistic setting, where the leaked data is scarce. Notably, our approach can reconstruct texts with RougeL score of 10 using a single leaked data point.
Our work makes the following main contributions:

\begin{itemize}
%\noindent \textbf{i)} 
\item We propose and verify the effectiveness of a novel few-shot inversion attack, which drastically reduces the cost and complexity of such attacks, making them plausible real-world threats. %to plausible real-world threat scenarios.

% \noindent \textbf{ii)} 
\item We demonstrate the transferability of the inversion attack across various languages, models and domains.

% \noindent \textbf{iii)} 
\item We examine several established defense mechanisms, none of which are successful mitigation strategies for this attack, highlighting the new security and privacy vulnerabilities of embeddings in vector databases.

\end{itemize}
% \noindent \textbf{iv)} discuss the potential of the alignment method in broader applications in NLP.


\section{Related Work}

\subsection{Textual Embedding Inversion Attacks}
% Let $x$ and $enc(\cdot)$ be text and an encoder,  $\ve= enc(x)$ is the corresponding embedding vector. 
Textual embedding inversion attack aims to learn the inversion function that reconstructs the original textual inputs given their embeddings. 
%$enc^{-1}(\ve)=x$.
% reconstruct text $\hat{\vx}$ from embeddings $enc(\mathbf{x})$, so that $\mathbf{\hat{x}}\approx \mathbf{x}$.
% Textual Embedding Inverison attacks have advanced rapidly, evolving from reconstructing 50\% to 70\% of tokens~\citep{10.1145/3372297.3417270} to achieving near-exact matches with more recent approaches~\citep{li-etal-2023-sentence, morris2023text}. 
\citet{10.1145/3372297.3417270} demonstrates that it is possible to recover over half of the input words from a text embedding without preserving their order.
\citet{li-etal-2023-sentence} starts to treat the inversion attacks as a generation task, generating coherent and contextually similar sentences compared to the original text.
%train attack models assuming leaked data of more than 100k samples.
% average sentence length is low. 11, and it uses a lot of training data.
~\citet{morris2023text} adopts an iterative approach to train the attack model by parameterizing attack and hypothesis embeddings based on decoded text from the previous step, which results in exact matches between original and reconstructed text in certain settings. 
%However, this approach requires almost unlimited access to the victim encoder. Despite its inversion performance, this approach is computationally costly and assumes an unrealistic setting, involving days of model training and 5 million leaked data.
~\citet{huang_transferable_2024} implements adversarial training to align victim embeddings to attack embeddings, making them not differentiable. 
%Although this work uses significantly fewer leaked data samples than prior works, it still leverages at least 8k leaked samples and the performance is lower than 20 in RougeL across the models.
%8k leaked samples were still needed to achieve 20 in RougeL. 
~\citet{chen2024typ,chen2024text} expand inversion attacks beyond English embeddings to multilingual spaces, leveraging linguistic typology to investigate inversion attack performance, finding that certain languages are particularly vulnerable.

However, all existing works in embedding inversion attacks require an enormous amount of data leakage to train the generative attack models, such as 100k samples for~\citet{li-etal-2023-sentence}, 1-5 million for~\citet{morris2023text, chen2024typ,chen2024text} and 8k for~\citet{huang_transferable_2024}.
In comparison, our proposed approach \textbf{\ourmethod} does not require training an attack model on leaked/private embeddings, and the inversion attack succeeds with few leaked data, we additionally experiment on multiple languages.
% \xqk{main reference and must read (\ie EMNLP'23 one) papers.} # done

% \xqk{Prefer not to highlight the scale of samples for training for each work, which makes the attack implausible (e.g., a monitor of network data transmission can avoid such a scale of data leakage). Instead, highlight All existing work requires an enormous amount of training samples, such as xxx for xxx and xxx for xxx. (Two most representative works here.)}



\subsection{Embedding Alignment}
% Embedding alignment has been a critical area in NLP, primarily due to its role in bridging representation spaces across different languages, domains, and modalities. 
Embedding alignment has continuously progressed in NLP with the development of embedding representations and LLMs.
% Early on, a widely adopted approach in this subfield involves first learning monolingual word vectors independently~\citep{10.5555/2999792.2999959}. Subsequently, a mapping from source language embeddings to target language embeddings is learned with a bilingual dictionary~\citep{mikolov2013exploiting,smith2017offline,artetxe-etal-2017-learning}.
% When this mapping is constrained to be an orthogonal linear transformation,
% the optimal alignment between word pairs can be derived in closed form~\citep{artetxe-etal-2016-learning,schonemann1966generalized}.
% In comparison,~\citet{lample2018word} uses the unsupervised approach to align word embedding spaces with a cross-domain similarity adaption to mitigate the hubness problem.
In the early stages, a common approach involved independently training monolingual word vectors~\citep{10.5555/2999792.2999959} and then learning a mapping between source and target language embeddings using a bilingual dictionary~\citep{mikolov2013exploiting,smith2017offline,artetxe-etal-2017-learning}. 
When this mapping is restricted to an orthogonal linear transformation, the optimal word pair alignment can be computed in closed form~\citep{artetxe-etal-2016-learning,schonemann1966generalized}. 
In contrast,~\citet{lample2018word} introduce an unsupervised method for aligning word embedding spaces, incorporating cross-domain similarity adaptation to address the hubness problem.

With the advancement of contextualized embeddings since the emergence of LLMs such as BERT~\citep{devlin2018bert}, the focus shifted to the alignment of contextual word representations~\citep{schuster-etal-2019-cross, aldarmaki-diab-2019-context,wang-etal-2019-cross,alqahtani-etal-2021-using-optimal,cao2020multilingualalignmentcontextualword, jalili-sabet-etal-2020-simalign}.
Moreover, sentence embedding alignment has been operated in lifelong relation extraction with a linear transformation~\citep{wang-etal-2019-sentence}, aligning encoders in different languages to evaluate crosslingual transfer~\citep{conneau2018xnlievaluatingcrosslingualsentence},  building parallel data for machine translation~\citep{krahn2023sentenceembeddingmodelsancient}. 
In comparison, \textbf{\ourmethod} aligns sentence embeddings from different models to conduct embedding inversion attacks, but it can also be applied in embedding alignment in general. 


\subsection{Mitigating Embedding Inversion Attacks}
Most research on textual embedding inversion focuses on attacks~\cite {li-etal-2023-sentence, huang_transferable_2024,chen2024typ}.
While~\citet{10.1145/3372297.3417270} adopt an adversarial training approach to mitigate the risks of inversion attacks, this method is ineffective for defending textual embeddings in black-box settings.
To defend against inversion attacks while maintaining embedding utility in downstream tasks, \citet{morris2023text} propose inserting Gaussian noise as a defense mechanism.
Expanding inversion attacks into multilingual space using the same method,~\citet{chen2024text} find that Gaussian noise effectively protects monolingual embeddings but is less effective for multilingual ones.

Differential privacy (DP) limits the impact of individual element~\citep{dwork2014algorithmic}, and has been shown to preserve the privacy of the extracted representation from text, when applied during model training~\citep{lyu-etal-2020-differentially}.
To ensure sequence-level metric-based local DP, which can be employed during inference, a sentence embedding sanitization pipeline has been developed, maintaining non-private task accuracy and effectively thwarting privacy threats of membership inference attacks~\citep{du2023sanitizing}.
Pertinent to vector DBs, Watermarking EaaS with Linear Transformation (WET) introduces a method that applies linear transformations to embeddings to implant watermarks to counter paraphrasing vulnerabilities~\citep{shetty2024wet}.

In this work, we examine these defenses against \textbf{\ourmethod}, and discuss the potentials and challenges of defending embeddings from inversion attacks.

\section{Methodology}


\begin{figure}[t!]
    \centering
    \includegraphics[width=\linewidth]{images/Figure2_07.pdf}
    \caption{Three steps for Few-shot Inversion Attack, (1) Train a Local Embedding-to-Text Generation Model; (2) Transform \textcolor{blue}{victim embeddings $\ve_{V}$} to the \textcolor{orange}{attack embeddings space $A$} with matrix $\mW$; and (3) Textual embedding inversion attack. 
    % fire: train, snowflake: inference
    % \xqk{We can align this fig with the structure of Sec 3. One block for training W and one block for training the local generator and one block for inference (actual attacks). The first two blocks can be at the top. When training W we can have $e_V$ and $e_A$ together to get W. Note that the current 2 and 3 are in inference not training which may confuse readers.}
    }
    \label{fig:method}
\end{figure}

% \textcolor{red}{explain the $D_l$ and  $D_v$}

% \textcolor{blue}{highlight how the decoder is the main bottleneck in the work; vec2text essentially train the decoder to adapt to the new embeddings, thus it needs a substantial amount of leaked data; we have a local decoder trained on $D_l$.}

% establish notation here. 
We explore a situation in which a malicious attacker gains access to a \emph{limited set of embeddings}, and attempts to reconstruct private and sensitive text data. 
We propose \textbf{\ourmethod} to circumvent the disadvantage of scarce data and leverage a pretrained encoder-decoder to align the victim embeddings to the attack space and reconstruct the texts.
We note the victim and attack embedding spaces as $V$ and $A$, respectively.
As illustrated in Fig.~\ref{fig:method}, the framework consists of three steps: 
(1) we train an embedding-to-sequence generation model by fine-tuning a pre-trained decoder~$dec_{A}(\cdot)$; (2) we align the embeddings from a black-box victim encoder $\ve_V$  
%$enc_{V}(\cdot)$ 
to attack embeddings in attacker's model space $\ve_A$;
%space $A$ from the attack encoder $enc_{A}(\cdot)$, 
%we note the aligned embeddings as $\mE_{V\rightarrow A}$; 
(3) the attacker leverages the capability of the generation model $dec_{A}(\cdot)$ with the embedding alignment model $\mW$ to reconstruct the original text from $\ve_{V}$ to $\ve_{V \rightarrow A}$ and finally to text.
% \xqk{Match the notations here with new figure. $dec_A$, $\ve_A$, $\ve_V$ and $W$ are easy to follow. $\mE$ has not be properly introduced before, so try to avoid E, but it should be fine to have E in Sec 3.2.}
% Formally, textual embedding inversion attacks involve two distinct embedding spaces, i.e., the victim space, denoted as $V$, where the victim embeddings $\ve_{V}$ reside, and the attack space, denoted as $A$. In these attacks, $\ve_V$ is aligned to the attack space $A$ and subsequently processed to reconstruct their original text with the attack model $dec_{A}(\cdot)$. 



\subsection{Local Embedding-to-Text Generator}
% Train an embedding-to-sequence model.
% experimented training this embedding-to-sequence model with samples from 10k to 1M, the model with 150k has the best rougeL 56.
To train the local embedding-to-text generator, we use a publicly available text corpus, noted as $D_L$. 
Given a sentence $x\in D_{L} $, and attack encoder $enc_{A}(\cdot)$, the token embeddings are obtained $\mH=enc_{A}(x)\in \mathbb{R}^{s\times n}$
where $s$ is the sequence length and $n$ the embedding dimension. 
The sentence embeddings are computed through mean pooling the last hidden embeddings $\ve_{A}= \sum^{s}_{j=1}\vm_j \mH_j/ \sum^{s}_{j=1}\vm_j \in \mathbb{R}^{n}$, where $\vm\in \{0,1\}^{s}$ is the attention mask for the sequence $x$.
Furthermore, L2 normalization is implemented on the sentence embeddings, i.e., ${\ve_{A}}/{\|\ve_{A}\|}$
% $ \frac{\ve_{A}}{\sqrt{\sum_{i=1}^{n}\ve_{A, i}^2}}$,\xqk{check the equation, we usually use ${\ve_{A}}/{\|\ve_{A}\|}$} 
as sentence embedding normalization has proven beneficial in avoiding overfitting and inducing faster convergence in fine-tuning~\citep{aboagye2022normalization}.
The pre-trained decoder $dec_{A}(\cdot)$, parameterized by $\theta$, processes the embeddings to produce an output sequence $\hat{x}=(\hat{x}_1, \hat{x}_{2},\dots, \hat{x}_{s})$, where $\hat{x}_i$ are tokens in the predicted output sequence. We train the embedding-to-text generator by fine-tuning $dec_{A}(\cdot)$, by minimizing the cross-entropy loss:
\begin{equation}
    \mathcal{L}(\theta)= - \sum_{i=1}^{s} \log \mathbb{P}(\hat{x}_i | \hat{x}_{<i}, \ve_{A}; \theta),
\end{equation}
% \xqk{It would be better to use $\ve$ as an embedding vector, $\mE$ as an embedding matrix and $\mW$ as a weight matrix.}
where $\mathbb{P}(\hat{x}_i | \hat{x}_{<i}, \ve_{A}; \theta)$ is the probability of predicting token $\hat{x}_i$ given the previous tokens $\hat{x}_{<i}$ and the input sentence embeddings $\ve_{A}$. 

Notably, attackers can easily retrieve a large corpus $D_L$ to train this generator model, as it operates independently of any victim models.
The challenge of aligning victim semantics to target embedding space is addressed in the next subsection.

% \xqk{It is worth highlighting that attackers can easily retrieve a large corpus to train this generator as it does not involve any information from victim models. We leave the challenge of aligning semantics to target embedding space in the next subsection. }
% The decoder $dec_{A}(\cdot)$ is optimized using AdamW~\citep{loshchilov2017fixing}. 
%, with a learning rate $\eta$ and weight decay $\lambda$:
% \begin{equation}
% \theta_{t+1}    = \theta_{t} - \eta\nabla_{\theta} \mathfrak{L}(\theta_{t}) - \lambda \theta_{t}.
% \end{equation}


\subsection{Embedding Space Alignment}
% Learning the semantic transformation function to transform victim's embedding space to attacker's embedding space.
Suppose there is a leaked data pair $(\mX, \mE_V)$,  given that $\mX \subseteq D_{V}$ is the victim dataset with $b \in \mathbb{N}$ samples, and embedding matrix $\mE_V=enc_{V}(\mathbf{\mX})$, where $enc_{V}$ is the black-box viticm encoder. 
To align the victim embeddings to the attack space $A$, we obtain the embedding matrix $\mE_{A}=enc_{A}(\mathbf{\mX})$ given the leaked text dataset, and seek a solution to solve the system 
\begin{equation}
    \mE_{V}\mW \approx \mE_{A},
\end{equation}
and the best possible $\mW \in \mathbb{R}^{m\times n}$, given $\mE_{V}\in \mathbb{R}^{b\times m}$ and $\mE_{A}\in \mathbb{R}^{b\times n}$, where $n$ and $m$ are the regarding embedding dimensions of victim and attacker embeddings. 
While there is no exact solution to the system, our approach is to minimize their deviation, $e=\mE_{A}-\mE_{V}\mW$. 
Taking the square of the error by each sample, the objective is to minimize the following:
% in the notation of embedding vectors:
\begin{equation}
    \min_{\mW}
    \sum_{i=1}^{b}
    \|\ve^{i}_{A}- \ve^{i}_{V} \cdot \mW\|^{2}.
\end{equation}
% \xqk{Prefer to have a 'sum' of vectors' losses and a matrix version here. To teach readers such a derivation.}
The solution to this least squares loss is:
\begin{equation}
\mW = (\mE^{T}_V \mE_{V})^{-1} \mE_{V}^{T} \mE_{A},
\end{equation}
where $(\mE^{T}_V \mE_V)^{-1} \mE_{V}^{T}$ is the \textit{Moore-Penrose Inverse} of $\mE_V$ (see the detailed derivation in Appendix~\ref{normal_equation}).

The aligned embedding from $V$ to $A$ is thus:
\begin{equation}
    \mE_{V \rightarrow A} = \mE_{V}  \mW,
\end{equation}
where $\mE_{V \rightarrow A}\in \mathbb{R}^{b\times n} $. Implementing this alignment does not require any training; it is a one-step linear scaling. 
Moreover, aligning using $D_{V}$ with a batch size $b$ varying from 1 to 1,000, we observe that even with as few as 30 samples the Rouge-L score exceeds 20, and a reasonably successful attack can be initiated with only a single data point.
% \xqk{1-1000 is a big range. We can highlight two points: i) mostly successful under xxx samples and ii) in xxx setting, only a single example could trigger a reasonably successful attack.} is sufficient for inversion attacks.
%\xqk{May connect to the consistency of weights for each embedding. Discuss it in the \textit{Discussion} Session but highlight it here.}
The density distribution of the alignment transformation matrix $\mW$'s weights of encoders remains consistent across different datasets (see Fig~\ref{fig:W_alignment_density}).
% To evaluate on test data, the transformation $W^{V\rightarrow A}$ is applied directly on unseen $E_V$.

\subsection{Textual Embedding Inversion Attack}
Given the attack model, \ie the local embedding-to-text generator $dec_{A}(\cdot)$ and the $\ve_V$ to $\ve_A$ alignment model, %(cf. Section~\ref{sec:experiment_attack}),
% (trained in Section~\ref{sec:experiment_attack}) and the alignment matrix  (calculated in Section~\ref{todo}), 
and a body of eavesdropped embeddings $\mE_V$, we launch the inversion attack:
\begin{equation}
    \hat{\mX} = dec_A(\mE_V \mW).
\end{equation}


\section{Experimental Setup}

\subsection{LLMs}
We use pretrained~\textsc{FlanT5} as the backbone to launch our attack modules, encoder~$enc_{A}(\cdot)$ and decoder~$dec_{A}(\cdot)$. For victim models, a variety of encoders are experimented on, including ~\textsc{T5},~\textsc{GTR},~\textsc{mT5}, ~\textsc{mBERT} and OpenAI text embedders~\textsc{text-embedding-ada-002} (\textsc{ada-2}) and~\textsc{text-embedding-3-large} (\textsc{3-large}) (see the details of LLMs in Tabel~\ref{tab:llms}).


\subsection{Attack Experimental Setup}

\paragraph{Datasets and Attack Model}
We train the embedding-to-text generator $dec_{A}(\cdot)$ by fine-tuning ~\textsc{FlanT5}-decoder, using the MultiHPLT English dataset~\citep{de-gibert-etal-2024-new} to explore few-shot inversion attacks. 
For multilingual inversion attacks, we utilize English, German, French, and Spanish datasets from mMarco~\citep{bonifacio2021mmarco}. 
In dataset, we split 150k samples ($D_L$) to train $dec_{A}(\cdot)$; and up to 1k samples ($D_V$) to derive the alignment metrix $\mW$ by aligning $\ve_V$ to $\ve_A$, as alignment samples; and 200 for evaluation.
% for $D_L$ to train $dec_{A}(\cdot)$, we curate 150k samples; for $D_V$, we use up to 1k samples for obtaining \textit{alignment matrix} $\mW$ by aligning $\ve_V$ to $\ve_A$, noted as alignment samples, and 200 for evaluation.

\begin{table}[t!]
\centering
\resizebox{\linewidth}{!}{
\begin{tabular}{c|l|c|c|c|c|c}
\toprule
 &\textbf{Model} &  \textbf{BLEU1} & \textbf{BLEU2} & \textbf{Rouge-L} & \textbf{Rouge1} & $\mathbf{COS}$ \\
\midrule

& $enc_{A}(\cdot)$ &  62.27 & 40.68 & 54.16 & 62.07 & - \\
 
\midrule
% \multicolumn{7}{l}{\textbf{Vec2Text}} \\
\multirow{8}{*}{\rotatebox{90}{\textbf{Vec2Text}}}  

% & \textsc{FlanT5-B} (Base)     &  768  & 20.82  & 8.99   & \textbf{17.05} & 18.93  & 0.4131 \\
% & Corrector     &  768  & 18.56 & 7.80 & 15.60 & 17.37 & \underline{0.5131}\\
%  \cmidrule{2-8} 

 & \textsc{T5} (Base)  & 21.47  & 9.07 & \textbf{17.38} & 19.52  & 0.4663 \\
 & Corrector     & 18.35 & 7.60 & 15.81 & 17.76 &  \underline{0.4835}\\

 \cmidrule{2-7} 
 & \textsc{GTR}  (Base)          & 6.70 &  2.31 & 4.70 & 4.82 & 0.1911\\
 & Corrector          & 13.42 & 2.79 & \textbf{10.26} & 12.31 & \underline{0.2725}\\
 \cmidrule{2-7} 

 & \textsc{mT5} (Base)        &22.27 &  9.86 & \textbf{17.21} & 19.28 & \underline{ 0.7118}\\
 & Corrector       &18.73 &  7.79 & 15.98  & 17.82 & 0.6891 \\ 
 \cmidrule{2-7} 
% & \textsc{ME5}(Base)       & 768 & 23.04 & 10.40 &  \textbf{16.34} & 18.43 & \underline{ 0.9011 }\\

% & Corrector     & 768 & 18.50 & 7.68 &  15.97 &18.18 & 0.8582 \\
%  \cmidrule{2-8} 

& \textsc{mBERT} (Base)  &   21.56 &   9.09 &  \textbf{16.97} &18.81 & \underline{0.5335} \\
& Corrector       & 18.45 & 7.48 & 15.99 & 18.10 & 0.5531\\
\midrule

\multirow{5}{*}{\rotatebox{90}{\textbf{ALGEN}}}  

& \textsc{Random} &  11.63 & 0.6 & 7.09 & 8.36 &  -0.0440 \\
\cmidrule{2-7}

% & \textsc{FlanT5-B}      & 768 & 51.44 & 32.57 & 44.14 & 49.75 & 0.9415 \\
& \textsc{T5}                & 52.98 & 33.86 & \textbf{45.75} & 51.56 & \underline{0.9464} \\
& \textsc{GTR}              & 42.59 & 26.17 & 38.27 & 42.32 & 0.8879 \\
 \cmidrule{2-7} 

& \textsc{mT5}              & 49.61 & 31    & \textbf{43.35} & 48.47 & \underline{ 0.9370} \\
% & \textsc{ME5}              & 768 & 46.33 & 27.83 & 40.47 & 45.9  & 0.9243 \\
& \textsc{mBERT}            & 47.06 & 28.66 & 39.9  & 45.04 & 0.9217 \\
%& \textsc{SBERT}            & 384 & 39.35 & 23.62 & 34.91 & 38.86 & 0.8768 \\
 \cmidrule{2-7} 
& \textsc{OpenAI (ada-2)} &  46.7  & 28.67 & \textbf{41.45} & 47.01 & \underline{0.9312} \\  
&  \textsc{OpenAI (3-large)}     & 46.28 & 28.74 & 41.31 & 46.28 & 0.9066 \\

\bottomrule
\end{tabular}}
\caption{Inversion Attack Performances by victim models with 1,000 leaked data samples. The best Rouge-L scores are \textbf{bolded}, and the highest cosine similarities are \underline{underlined}.}
\label{tab:inversion_attack_performance}
\end{table}


\begin{table}[h!]
\centering
\resizebox{\linewidth}{!}{
\begin{tabular}{l|ccrrr}
    \toprule
    Dataset & Clasisification & \#Class  & \#Train & \#Dev & \#Test \\
    \midrule
    SNLI & NLI & 3 & 540,000 & 200 & 200 \\
        SST2 & Sentiment & 2 & 59,560 &  200 & 200 \\
    S140 & Sentiment & 2 & 1,599,798 & 200 &200\\
    \bottomrule
    \end{tabular}
    }
    \caption{Statistics of Utility Task Datasets}
\label{tab:utility_stats}
\end{table}

\paragraph{Embedding-to-Text Generator Training}
We conduct a series of experiments varying the training set size (from 10k to 1M samples), learning rate and weight decay, then select the generator configuration that achieves the best Rouge-L score on the evaluation set.
% We experiment on a range of number of data samples from 10k to 1M, learning rate and weight decay, we select the trained local decoder with the best RougeL score on the dev dataset. 
Eventually, to strike a balance of performance and data usage, we train $dec_{A}(\cdot)$, an embedding-to-sequence decoder, with a learning rate of $1e-4$ and weight decay $1e-4$ on AdamW optimizer~\citep{loshchilov2017fixing}, batch size 128, and 150k data samples performs the best.
We use \text{Cross-entropy Loss} for training the generator. 


%in the trade-off between training data sizes and performances.
%(see details in Appendix~\ref{appendix:attack_model}).


\paragraph{Evaluation Metrics}
%We use \textbf{Cross-entropy Loss} %\footnote{\url{https://pytorch.org/docs/stable/generated/torch.nn.CrossEntropyLoss.html}} 
%for training the local inversion model. 
~\textbf{Rouge-L}~\citep{lin-2004-rouge} is used to measure the accuracy and overlap between ground truth text $x$ and reconstructed text $\hat{x}$ based on n-grams. 
In addition, we report~\textbf{Rouge1},~\textbf{BLEU1} and ~\textbf{BLEU2}.
~\textbf{Cosine Similarity (COS)} between the aligned victim embeddings $\ve_{V\rightarrow A}$ and the attack embeddings $\ve_{A}$ is calculated to evaluate the semantic similarity in the latent embedding space.






\subsection{Defense Experimental Setup}~\label{sec:experiment_utility}
We aim to evaluate how embeddings with defense mechanisms perform in downstream tasks.

\paragraph{Datasets}
We use SST2~\citep{socher-etal-2013-recursive}, sentiment140 (S140)~\citep{go2009twitter} and SNLI~\citep{bowman-etal-2015-large} in our experiments, which are curated to ensure a balanced distribution of labels.
Table~\ref{tab:utility_stats} shows the statistics of datasets.

\paragraph{Utility of Embeddings}
Using the embeddings from the victim encoders, we train multi-layer perception classifiers on the datasets and evaluate the accuracy (ACC) and \fscore (F1) performance. 
We train each classifier 6 epochs, and select the best model with ACC on the dev dataset to evaluate the test dataset.
There would be minimal difference in the utility performance between the protected and original embeddings, if a defense is successful.







\section{Few-shot Inversion Attacks}
Each subsection aims to answer one Research Question (RQ).

\subsection{How few Leaked Data do Attackers Need?}




\begin{figure}[t!]
    \centering
    \includegraphics[width=\linewidth]{images/result1_RougeL.pdf}
    \includegraphics[width=\linewidth]{images/result1_COS.pdf}
    \caption{Inversion Performance in Rouge-L (Top) and Cosine Similarities (Bottom) by Victim Models and Alignment samples. Dashed lines are results of Vec2Text and solid lines are results of \textbf{ALGEN}.}
    \label{fig:inversion_cos_rougel}
\end{figure}




With only a single leaked data sample, our attack model manages to invert the victim embeddings, achieving a Rouge-L score of 10 across the encoders, as shown in Fig.~\ref{fig:inversion_cos_rougel}. 
We use randomly generated embeddings as a baseline to verify that the aligned embeddings from $\textbf{\ourmethod}$ capture meaningful information. 
As shown in Table~\ref{tab:inversion_attack_performance}, all victim embeddings substantially outperform the \textsc{random} across metrics, validating our approach.
% show a substantial jump in performance across metrics compared to $\textsc{random}$, thereby validating our approach.
Furthermore, when the number of leaked data samples increases until 1k, the inversion performance increases sharply, reaching 45.75 in Rouge-L and $0.9464$ for cosine similarity for \textsc{T5} embeddings.
Notably, while \textsc{GTR} and \textsc{T5} share the same tokenizer with the attack model, their inversion performances are not superior to others. 
Moreover, the inversion performance on proprietary \textsc{OpenAI} embeddings also reaches comparable performance, more than 41 in Rouge-L and 0.9 in cosine similarities for both~\textsc{ada-2} and~\textsc{3-large}, highlighting the risks posed by inversion attacks.
As shown in the qualitative analysis in Table~\ref{tab:qualitative}, some sentences can be inverted with almost an exact match.
Furthermore, we conduct an ablation study with the size of alignment data samples and find that 1k alignment samples strike a balance between data size and performances (see  Fig.~\ref{fig:datasize_ablation} in Appendix.~\ref{apendix:datasize}).

We compare our method with Vec2Text~\citep{morris2023text}, which trains two-step models (i.e., Base and Corrector) with iterative access to the victim encoder, and it requires training on embeddings from each encoder to invert the regarding embeddings.
In comparison, our method only requires training one local attack model, and training does not involve specific victim encoders.
Up to 1k samples, Vec2Text performance is much inferior compared to our method both in Rouge-L and cosine similarities, lower than 20 and 0.72, respectively, as detailed in Fig.~\ref{fig:inversion_cos_rougel} and Table~\ref{tab:inversion_attack_performance}.

We generally observe small alignment errors, with the cosine similarities between victim embeddings and attack embeddings consistently higher than 0.8 and near 1.0 when the number of alignment samples increases. 
The bottleneck of the performance of \textbf{ALGEN} is likely to lie in the decoding, as the trained attack decoder can only reach 54.16 in Rouge-L to invert the attack embeddings, as shown in Table~\ref{tab:inversion_attack_performance},
%in Table~\ref{tab:inversion_attack_performance}, 
which is considered to be the upper bound of inversion attack performance.



\subsection{Are Other Languages (More) Vulnerable? }
Building on previous work on multilingual embedding inversion, we also investigate the impact of \textbf{\ourmethod} on languages other than English.
To achieve this, we trained local attack models in English, French, German, and Spanish. 
We further conduct crosslingual embedding inversion attacks, for example, applying the English-trained attack model to invert French textual embeddings. 
Table~\ref{tab:crosslingual_inversion} summarizes the results of multilingual and crosslingual embedding inversion attacks. 
The first row shows the results of monolingual inversion on the attack embeddings, which serve as an upper bound of the inversion performances on this dataset. 
As expected, the monolingual inversion attacks perform better than the crosslingual ones. 



\begin{table}[t!]
    \centering
    \resizebox{\linewidth}{!}{

    \begin{tabular}{c|c|l|l|l|l}

\toprule
      & \multirow{2}{*}{\textbf{Attack Lang.}} & \multicolumn{4}{c}{\textbf{Victim Languages}}\\
      
       &   & \multicolumn{1}{c}{\textbf{English}} &\multicolumn{1}{c}{\textbf{French}} & \multicolumn{1}{c}{\textbf{German}} & \textbf{Spanish} \\ 
       \midrule      
    $enc_{A}(\cdot)$ &    & 54.47 & 52.78 & 24.77 & 53.98 \\
 \midrule
       
       \multirow{4}{*}{ {\textsc{T5}}} 

       &  \textbf{English} & 29.29 & 12.57 (+2.58) & 5.54 (+0.7) & 13.31 (+0.33) \\ 
       &  \textbf{French} & 8.67 (+3.37) & 31.01 & 2.4 (+1.25) & 13.72 (+0.84) \\ 
       &  \textbf{German} & 15.72 (+0.09) & 15.56 (+0.06) & 13.04 & 16.63 (+0.09) \\ 
       &  \textbf{Spanish} & 6.56 (+4.49) & 11.33 (+1.05) & 2.18 (+0.95) & 31.83 \\ 
        \midrule
        
          \multirow{4}{*}{\textsc{GTR}} 

       &  \textbf{English} & 19.22 & 10.45 (+1.34) & 4.27 (+0.57) & 10.17 (+1.09) \\ 
       &  \textbf{French} & 4.78 (+2.85) & 23.98 & 1.95 (+1.14) & 12.16 (+0.42) \\ 
       &  \textbf{German} & 11.13 (+0.48) & 13.37 (-0.15) & 8.48 & 14.32 (-0.09) \\ 
      &   \textbf{Spanish} & 4.4 (+3.74) & 10.5 (+1.98) & 1.9 (+0.95) & 20.88 \\ 
        \midrule
        
 \multirow{4}{*}{\textsc{mT5}} 
 
 &  \textbf{English} & 25.48 & 12.29 (+1.64) & 5.31 (+0.03) & 12.61 (+0.87) \\ 
  &  \textbf{French} & 8.66 (+2.79) & 24.6 & 2.27 (+1.43) & 13.1 (+0.48) \\ 
        &  \textbf{German} & 15.54 (+0.06) & 15.2 (-0.1) & 10.09 & 15.59 (+0.02) \\ 
       &  \textbf{Spanish} & 6.84 (+3.98) & 11.62 (+1.35) & 1.88 (+1.31) & 24.05 \\ 

       \midrule
 \multirow{4}{*}{\textsc{mBERT}} 
       & \textbf{English} & 21.3 & 11.72 (+1.42) & 4.46 (+0.3) & 11.76 (+0.82) \\ 
      &  \textbf{French} & 5.91 (+3.61) & 22.79 & 1.87 (+1.71) & 12.05 (-0.19) \\ 
      &  \textbf{German} & 14.29 (+0.12) & 14.16 (-0.05) & 9.17 & 15.6 (+0.05) \\ 
      &  \textbf{Spanish} & 5.17 (+4.14) & 11.07 (+0.09) & 1.75 (+0.57) & 21.74 \\ 
        \midrule
    
     \multirow{4}{*}{\shortstack{\textsc{OpenAI}\\ \textsc{(ada-2)}}} 

& \textbf{English} & 24.18 & 11.62 (+1.32) & 5 (+0.17) & 11.36 (+0.95) \\ 
      &  \textbf{French} & 6.97 (+3.73) & 22.73 & 1.59 (+1.52) & 12.73 (+0.07) \\ 
      &  \textbf{German} & 14.74 (+0.11) & 13.7 (+0.07) & 9.63 & 14.33 (+0.05) \\ 
     &   \textbf{Spanish} & 7.08 (+3.57) & 9.82 (+1.39) & 1.63 (+1.05) & 21.25 \\ 
        \bottomrule
    \end{tabular}}
    \caption{Crosslingual Embedding Inversion Performance in Rouge-L with $|D_V|=$1k. The results in the brackets are the performance gain after translation.}
    \label{tab:crosslingual_inversion}
\end{table}

Consistent with previous findings~\citep{chen2024text}, crosslingual inversion often reconstructs text in a language other than the intended, usually English (the dominant language in most multilingual LLMs) or trained languages in the attack model, hindering the performance evaluation with string-match metrics.
For example, a French text ``\textit{un composé organique qui ne contient que du carbone}'', is reconstructed into English ``\textit{a chemical compound composed of carbon}''. 
Rouge-L evaluation is more accurate when the inverted text is translated back into the target language (e.g., ``\textit{un composé chimique composé de carbone}'').
To ensure fairness, we translate the inverted texts into their target languages using deep-translator.\footnote{\url{https://github.com/nidhaloff/deep-translator}}
After translation, Rouge-L scores improve across most models, with the most notable gains observed in the French-to-English scenario, as shown in Table~\ref{tab:crosslingual_inversion}.
% From the perspective of an attacker, splitting English words other than the target languages is a beneficial feature, as the victim's language may not be intelligible to the attackers, the proficiency in English can be generally assumed.
From an attacker's perspective, splitting words in English rather than in the victim's language is advantageous. 
Attackers can be assumed to be proficient in English, while the victim's language might be unintelligible to them. This further exacerbates the vulnerabilities of non-English languages.
% This phenomenon also highlights the vulnerabilities of other languages.
% \xqk{Do we have resultl for this?}
% \yiyi{yes, it is shown as the performance gain in the brackets in the table.}
% \xqk{I'd consider this as an advantage. We may add a comment that "We view splitting English words during an attack as a beneficial feature, as attackers may not understand or be familiar with the victim's language, but their proficiency in English can generally be assumed.".}



\subsection{Is Risk Transferable across Domains?}
To evaluate the cross-domain transferability of $\textbf{\ourmethod}$, we attack the embeddings on the mMarco English dataset using an attack model trained on MultiHPLT English data. 
Although the cross-domain inversion performance in Rouge-L is about 25\% lower than that of in-domain attacks on mMarco (see Table~\ref{tab:crosslingual_inversion}), the results remain alarmingly high - with Rouge-L near 20 and BLEU1 near 31 across victim encoders.

\begin{table}[h!]
    \centering
    \resizebox{\linewidth}{!}{
    \begin{tabular}{l|ccccc}
    \toprule
    \textbf{Model} & \textbf{BLEU1} & \textbf{BLEU2} & \textbf{Rouge-L} & \textbf{Rouge1} & \textbf{COS}  \\ 
    % \midrule
     % $enc_{A}(\cdot)$  &  & & & \\
     \midrule
        \textsc{T5} & 31.4 & 5.72 & 21.76 & 29.6 & 0.9278 \\ 
                \textsc{GTR} & 22.04 & 2.26 & 15.27 & 20.45 & 0.8442\\ 

        \textsc{mT5} & 30.08 & 4.82 & 19.33 & 26.94 & 0.9188 \\ 
           \textsc{mBERT} & 26.68 & 3.57 & 17.16 & 23.88 & 0.9033 \\ 
        \shortstack{\textsc{OpenAI (ada-2)}} & 27.62 & 4.63 & 19.07 & 26.40 & 0.9089 \\ 
         \bottomrule
    \end{tabular}}
    \caption{Cross-Domain Inversion Attack with $|D_V|$=1k.}
    \label{tab:ood}
\end{table}



% Compared to in-domain inversion attack with mMarco (shown in Table~\ref{tab:crosslingual_inversion}), 
% the results are 25\% worse across the victim encoders. 
% However, the inversion results are still critical as RougeL and BLEU1 approaching 30 and 20, respectively, posing significant threat to security breach of private information.

% mMarco <-> multiHPLT (English, and crosslingual)




\subsection{Does Inversion Recover Key Information? }

\begin{table}[h!]
    \centering
    \resizebox{\linewidth}{!}{
    \begin{tabular}{l|c|ccccccccc}
    \toprule
   \textbf{Model}  & \textbf{Overall} & \textbf{Product} & \textbf{ORG} & \textbf{GPE} & \textbf{Date} & \textbf{Time} & \textbf{Cardinal} & \textbf{Ordinal} \\ 
     \midrule
              \textsc{T5} & 23.86 & 50 & 32.41 & 21.62 & 14.55 & 16.67 & 28.57 & 40.00 \\ 

               \textsc{ mT5} & 20.2 & 53.33 & 28.04 & 17.5 & 11.11 & 0 & 10.53 & 25.00 \\ 
  \textsc{GTR} & 21.17 & 54.55 & 32.13 & 10.67 & 11.11 & 22.22 & 13.79 & 25.00 \\ 
          \textsc{mBERT} & 19.68 & 49.12 & 28.03 & 12.35 & 15.09 & 33.33 & 6.90 & 33.33 \\ 

        \textsc{OpenAI} & 22.45 & 53.33 & 34.13 & 16.47 & 13.79 & 0 & 14.55 & 37.50 \\ 
         \bottomrule
    \end{tabular}}
    \caption{Named Entity Recognition in F1 scores for overall and top 10 entities in specific. }
    \label{tab:ner}
\end{table}

To examine whether \textbf{ALGEN} attacks real key information, we apply Named Entity Recognition\footnote{\url{https://github.com/explosion/spacy-stanza}} on input and reconstructed test data from MultiHPLT English dataset to calculate the current ratio of Named Entities in the reconstructed texts. 
% We evaluate the performance with F1 scores of the recognized entities between original and reconstructed data. 
Table~\ref{tab:ner} shows the results in F1 scores for overall and individual entities. 
In the qualitative analysis, as shown in Table~\ref{tab:qualitative}, the input and reconstructed texts in the inversion attacks on \textsc{OpenAI (ada-2)} embeddings are compared, with named entities highlighted. 
These attacks reveal sensitive details, such as organization, country, and numbers, %and in some cases, reconstruct entire sentences, 
highlighting the risks of privacy disclosure by embeddings. 






\begin{table}[t!]
    \centering
    \resizebox{0.9\linewidth}{!}{
    \begin{tabular}{c|c|cc|cc}
    \toprule 
        \textbf{Victim} & \textbf{Defense} & \textbf{Rouge-L}$\downarrow$ & \textbf{COS}$\downarrow$ & \textbf{ACC}$\uparrow$ & \textbf{F1}$\uparrow$ \\ 
        \midrule 
    \textsc{Random}	 & -  &	12.42	 &0.0052	 &40.5 &	40.33\\
     % &Shuffling	 &11.11	 &-0.0816	 &40.5 &	40.33 \\
     % &	WET &	12.23 &	-0.0063	 &40.5 &	40.33\\
        \midrule
          & - & 42.89 & 0.9595 & 63 & 62.92 \\ 
          \cmidrule{2-6} 
          & WET & 39.4 & 0.9562 & 63 & 62.92 \\ 
           \cmidrule{2-6} 
          & Shuffling & 35.33 & 0.9599 & 63 & 62.92 \\ 
           \cmidrule{2-6} 
         
        \textsc{T5}  & 0.001 & 43.3 & 0.9595 & 66.5 & 66.38 \\ 
          & 0.005 & 42.88 & 0.9601 & 63 & 62.55 \\ 
         & 0.01 & 40.26 & 0.9537 & 69 & 68.57 \\ 
          & 0.05 & 22.82 & 0.8459 & 59 & 58.61 \\ 
          & 0.1 & 19.32 & 0.7918 & 48 & 47.6 \\ 
          & 0.5 & 14.29 & 0.3853 & 38 & 37.92 \\ 
          & 1 & 12.97 & 0.1543 & 30.5 & 30.54 \\ 
          \midrule
         & - & 37.39 & 0.9284 & 60.5 & 60.53 \\ 
         \cmidrule{2-6} 
          & WET & 35.58 & 0.9289 & 53.5 & 53.1 \\ 
         \cmidrule{2-6} 
          & Shuffling & 31.01 & 0.9280 & 60.5 & 60.58 \\
         \cmidrule{2-6} 
        \textsc{GTR} & 0.001 & 36.52 & 0.9279 & 60 & 60.06 \\ 
          & 0.005 & 36.48 & 0.9299 & 63 & 62.91 \\ 
          & 0.01 & 34.41 & 0.9218 & 60.5 & 60.59 \\ 
          & 0.05 & 22.31 & 0.8160 & 46.5 & 46.42 \\ 
          & 0.1 & 19.39 & 0.7701 & 50.5 & 49.98 \\ 
          & 0.5 & 14.31 & 0.2920 & 30.5 & 29.97 \\ 
          & 1 & 13.48 & 0.1659 & 30.5 & 30.57 \\ 
        \midrule
          & - & 37.98 & 0.9518 & 61 & 60.95 \\ 
         \cmidrule{2-6} 
          & WET & 34.69 & 0.9479 & 55.5 & 55.12 \\ 
         \cmidrule{2-6} 
          & Shuffling & 31.83 & 0.9515 & 60.5 & 60.44 \\ 
         \cmidrule{2-6} 
        \textsc{mT5}  & 0.001 & 37.97 & 0.9519 & 61.5 & 61.47 \\ 
          & 0.005 & 37.84 & 0.9493 & 55.5 & 55.06 \\ 
          & 0.01 & 34.5 & 0.9438 & 60 & 59.96 \\ 
          & 0.05 & 21.58 & 0.8292 & 54 & 53.52 \\ 
          & 0.1 & 17.65 & 0.7578 & 36.5 & 36.45 \\ 
          & 0.5 & 14.8 & 0.4482 & 36 & 35.17 \\ 
          & 1 & 13.31 & 0.1468 & 35.5 & 34.85 \\ 
        \midrule
          & - & 35.47 & 0.9423 & 57 & 57.05 \\ 
         \cmidrule{2-6} 
          & WET & 34.35 & 0.9428 & 53.5 & 53.5 \\ 
         \cmidrule{2-6} 
          & Shuffling & 30.29 & 0.9408 & 52 & 51.93 \\ 
         \cmidrule{2-6} 
       \textsc{mBERT}   & 0.001 & 35.82 & 0.9422 & 51.5 & 51.37 \\ 
          & 0.005 & 36.18 & 0.9409 & 51 & 50.82 \\ 
          & 0.01 & 34.18 & 0.9349 & 57 & 57.01 \\ 
          & 0.05 & 22.29 & 0.8265 & 50.5 & 50.48 \\ 
          & 0.1 & 18.65 & 0.7741 & 43 & 43.03 \\ 
          & 0.5 & 13.76 & 0.3969 & 35 & 33.96 \\ 
          & 1 & 13.11 & 0.1824 & 32.5 & 32.62 \\ 
        \midrule
        & - & 38.15 & 0.9433 & 74.5 & 74.78 \\
         \cmidrule{2-6} 
         & WET & 31.74 & 0.9405 & 64.5 & 64.51 \\ 
         \cmidrule{2-6} 
          & Shuffling & 33.91 & 0.9440 & 72 & 72.31 \\ 
         \cmidrule{2-6} 
      \shortstack{\textsc{OpenAI}}   & 0.001 & 37.94 & 0.9437 & 74.5 & 74.51 \\ 
        \textsc{(ada-2)} & 0.005 & 37.63 & 0.9445 & 69 & 69.09 \\ 
       & 0.01 & 35.59 & 0.9409 & 67 & 66.85 \\ 
          & 0.05 & 20.59 & 0.8426 & 49.5 & 49.39 \\ 
         & 0.1 & 18.72 & 0.8231 & 44 & 42.76 \\ 
         & 0.5 & 14.75 & 0.4483 & 37 & 36.87 \\ 
         & 1 & 11.94 & 0.1097 & 30.5 & 30.49 \\ 
         
        \bottomrule
    \end{tabular}}
     \caption{The Inversion and Utility Performance on Classification Tasks on SNLI dataset with WET, Shuffling, Guassian Noise Insertion. From a defender's perspective, $\uparrow$ means higher are better, $\downarrow$ means lower are better.}
    \label{tab:wet_shuffling_gaussian_snli}
\end{table}


\begin{figure*}[t!]
    \centering
    % \includegraphics[width=\linewidth]{images/dp_privacy/snli_LapMech_ACC.pdf}
    % \caption*{(a) LapMech}
    % \includegraphics[width=\linewidth]{images/dp_privacy/snli_PurMech_ACC.pdf}
    % \caption*{(b) PurMech}
        \includegraphics[width=0.8\linewidth]{images/Figure5.pdf} 

    \caption{The Inversion and Utility Performance in Accuracy on Classification Tasks on SNLI dataset with local DP, across $\epsilon$. The solid lines represent utility performance for non-private embeddings, while the dashed lines are for LDP-guaranteed embeddings. }
    \label{fig:ldp_snli}
\end{figure*}


\section{Defending Textual Embeddings}
To explore defenses against \textbf{\ourmethod}, we evaluate defense mechanisms designed to protect textual embeddings from adversarial attacks.
%, such as WET, Shuffling, Gaussian Noise Insertion, and metric-based Local Differential Privacy (LDP).




\subsection{Defense Methods}
% todo
% Fig.~\ref{fig:distribution_sst2_encoders}
% get also density distribution of the T.r

\paragraph{WET} We implement WET on textual embeddings, to examine whether it makes embeddings robust against inversion attacks, since it is effective in defending paraphrasing attacks~\citep{shetty2024wet}. 
A transformation matrix $\mT$ is generated to transform the $\ve_{V}$ into $\ve_{WET}$ with ${\mT \cdot \ve_{V}} / {\|\mT \cdot \ve_{V}\|} $,
%Eq.~(\ref{eq:wet}), 
to ensure that i) the original elements are discarded and only the transformed ones are retained; and ii) $\mT$ is full-rank and well-conditioned to allow for accurate recovery of the original embeddings (see details of generating $\mT$ in Appendix~\ref{wet_algo}).

% \begin{equation}\label{eq:wet}
%     \ve_{WET} = \text{Norm}(\mT \cdot \ve_{V}) .
%     %= \frac{\mT \cdot \ve_{V}}{||\mT \cdot \ve_{V}||}.
% \end{equation}

\paragraph{Shuffling}

We randomly shuffle the embeddings with $ \ve_{V, \pi(i)}$,
where $\pi$ is a random permutation function that reorders the indices $i$ along the hidden dimension.

% \begin{equation}
%     \ve_{\text{Shuffling}} = \ve_{V, \pi(i)}.
% \end{equation}


\paragraph{Gaussian Noise Insertion}
We add Gaussian noises to $\ve_V$  with ${(\ve_{V}+ \lambda \cdot \epsilon)}/\|{\ve_{V}+ \lambda \cdot \epsilon}\|, \epsilon \sim \mathcal{N}(0,1)$~\citep{morris2023text,chen2024text} with $\lambda\in[0.001,0.005,0.01,0.05,0.1,0.5,1]$.
% \begin{equation}
%     \ve_{noisy} = \ve_{V}+ \lambda \cdot \epsilon, \epsilon \sim \mathcal{N}(0,1).
% \end{equation}


\paragraph{Differential Privacy}
~\citet{du2023sanitizing} adopts Purkayastha Mechanism (PurMech) and Normalized Planer Laplace (LapMech) on sentence embeddings to ensure metric-LDP (see details in Appendix~\ref{ldp_appendix}). 
We adopt the parameters from~\citet{du2023sanitizing} to experiment with defending textual embeddings from inversion attacks. 
The privacy budgets $\epsilon\in [1,4,8,12]$ are selected.


\begin{table}[htb!]
    \centering
     \resizebox{0.9\linewidth}{!}{
    \begin{tabular}{l|c|cc|cc}
    \toprule
   \multirow{2}{*}{\textbf{Victim}} & $\epsilon$ & \textbf{Rouge-L}$\downarrow$ & \textbf{COS}$\downarrow$ & \textbf{Rouge-L}$\downarrow$ & \textbf{COS}$\downarrow$ \\ 
\cmidrule{2-6}   
         &  & \multicolumn{2}{c|}{LapMech} & \multicolumn{2}{c}{PurMech} \\ 
        \midrule
          % & - & 42.89 & 0.9595 & 42.89 & 0.9595 \\ 
          % \cmidrule{2-6}   
          & 1 & 11.58 & 0.0184 & 11.38 & -0.0341 \\ 
        \textsc{T5} & 4 & 12.02 & -0.0171 & 11.52 & -0.0095 \\ 
          & 8 & 11.8 & 0.0510 & 11.96 & 0.0137 \\ 
          & 12 & 11.64 & 0.0438 & 11.48 & -0.0239 \\ 
                \midrule
          % & - & 37.39 & 0.9284 & 37.39 & 0.9284 \\ 
          % \cmidrule{2-6}   
          & 1 & 11.9 & 0.0327 & 11.86 & 0.0183 \\ 
        \textsc{GTR} & 4 & 11.78 & -0.0193 & 11.82 & 0.0073 \\ 
          & 8 & 11.89 & -0.0217 & 11.51 & 0.0226 \\ 
          & 12 & 10.92 & -0.0277 & 11.33 & -0.0314 \\ 
                \midrule
          % & - & 37.98 & 0.9518 & 37.98 & 0.9518 \\ 
          % \cmidrule{2-6}   
          & 1 & 12.3 & 0.0827 & 11.96 & 0.0143 \\ 
        \textsc{mT5} & 4 & 13.15 & 0.0696 & 13.17 & 0.0652 \\ 
          & 8 & 11.89 & -0.0148 & 12.09 & 0.0493 \\ 
          & 12 & 13.44 & 0.1179 & 11.61 & -0.0443 \\ 
                \midrule
          % & - & 35.47 & 0.9423 & 35.47 & 0.9423 \\ 
          % \cmidrule{2-6}   
          & 1 & 12.37 & -0.0026 & 11.33 & 0.0016 \\ 
        \textsc{mBERT} & 4 & 11.84 & -0.0431 & 11.58 & -0.0019 \\ 
          & 8 & 11.58 & -0.0333 & 12.93 & 0.1003 \\ 
          & 12 & 11 & -0.0242 & 11.48 & 0.0271 \\ 
        \midrule
      % & - & 38.15 & 0.9433 &38.15 & 0.9433   \\ 
      % \cmidrule{2-6}   
        & 1 & 12.33 & 0.0198 & 11.15 & -0.0020 \\ 
       \shortstack{\textsc{OpenAI}}& 4 & 12.31 & 0.0133 & 11.87 & -0.0417 \\ 
        \textsc{(ada-2)} & 8 & 10.87 & -0.0659 & 13.13 & 0.1078 \\ 
        & 12 & 11.07 & -0.0050 &12.48 & 0.0447 \\ 
        \bottomrule
    \end{tabular}
    }
     \caption{The Inversion Performance on Classification Tasks on SNLI dataset with Local DP. From a defender's perspective, $\downarrow$ means lower are better.}
         \label{tab:snli_ldp_inversion}

\end{table}




\subsection{Results}
As shown in Table~\ref{tab:wet_shuffling_gaussian_snli},~\ref{tab:wet_shuffling_gaussian_sst2} and~\ref{tab:wet_shuffling_gaussian_s140}, WET and Shuffling have minimal impact on both inversion attack performance and the utility performance across victim models and datasets.
The randomly generated embeddings are also used as a baseline.
With Gaussian noise insertion, the bigger the noise $\lambda$, both performance in inversion and utility decrease.
Using local DP, while the utility performance is preserved almost as the non-private embeddings with $\epsilon=12$, as shown in Fig.~\ref{fig:ldp_snli}, Tabel~\ref{tab:ldp_sst2} and~\ref{tab:s140_ldp}. 
However, the inversion performance still maintains more than 25\% of the non-private embeddings in Rouge-L across encoders for both LapMech and PurMech, as detailed in Table~\ref{tab:snli_ldp_inversion},~\ref{tab:ldp_sst2} and~\ref{tab:s140_ldp}, posing security and privacy risks for the embeddings. 



\section{Discussion and Conclusion}
In this work, we introduce and validate the effectiveness of a novel few-shot inversion attack,~\textbf{\ourmethod}, 
which drastically lowers the cost and complexity of such attacks on widely used vector databases.
Our results show that the attack transfers effectively across domains and languages while revealing critical information.
%In addition, we have demonstrated the transferability of the inversion attack across domains and languages, and also its ability to reveal key information.
Moreover, its ability to align embeddings from different LLMs with minimal loss highlights its broad NLP applications, especially in cross-lingual embedding alignment.
Finally, our evaluation of existing defense mechanisms reveals that none can adequately protect textual embeddings from inversion attacks while maintaining utility, highlighting significant security and privacy vulnerabilities.
% the effectiveness of this inversion attack in aligning embeddings across different LLMs with minimal loss demonstrates its broader applications in NLP. 
% Notably, its strong performance in cross-lingual inversion further supports its potential for tasks such as cross-lingual embedding alignment.
% Finally, we have examined several established defense mechanisms pertinent to textual embeddings. 
% However, none of them are successful in defending the embeddings from inversion attacks while preserving their utility, highlighting the security and privacy vulnerabilities of embeddings in vector databases in this new attack paradigm.




% Future work:
% refine alignment;
% improve decoder.




% \subsection{Appendices}

% Use \verb|\appendix| before any appendix section to switch the section numbering over to letters. See Appendix~\ref{sec:appendix} for an example.

% \section{Bib\TeX{} Files}
% \label{sec:bibtex}

% Unicode cannot be used in Bib\TeX{} entries, and some ways of typing special characters can disrupt Bib\TeX's alphabetization. The recommended way of typing special characters is shown in Table~\ref{tab:accents}.

% Please ensure that Bib\TeX{} records contain DOIs or URLs when possible, and for all the ACL materials that you reference.
% Use the \verb|doi| field for DOIs and the \verb|url| field for URLs.
% If a Bib\TeX{} entry has a URL or DOI field, the paper title in the references section will appear as a hyperlink to the paper, using the hyperref \LaTeX{} package.

\section*{Limitations}
Our work does not propose a sufficient defense mechanism for \textbf{\ourmethod}. 
Although we evaluated a number of existing defense mechanisms for textual embeddings, we found them to be ineffective against the proposed embedding inversion attack. The primary focus of this work is to expose the security vulnerabilities in embedding services and to inspire the development of future defense paradigms.

\section*{Computational Resources}
We conduct experiments and train each text-to-embedding generator model on a single Nvidia A40 GPU, with the training process completing in three hours. 
Beyond this, \textbf{\ourmethod} requires minimal GPU resources, making it a genuinely few-shot experimental setting.

\section*{Ethics Statement}
We comply with the ACL Ethics Policy. 
The inversion attacks implemented in this paper can be misused and potentially harmful to proprietary embeddings.
% However, all the models we experimented on are open-sourced. We discussed mitigation and defense mechanisms.
We discuss and experiment with potential mitigation and defense mechanisms, and we encourage further research in developing effective defenses in this attack space.

\section*{Acknowledgements}
YC and JB are funded by the Carlsberg Foundation, under the Semper Ardens: Accelerate programme (project nr. CF21-0454). We further acknowledge the support of the AAU AI Cloud and express our gratitude to DeiC for providing computing resources on the LUMI cluster (project nr. DeiC-AAU-N5-2024085-H2-2024-28).
YC expresses her gratitude to the Danish Ministry of Higher Education and Science for the EliteForsk travel grant and to the School of Computing at Macquarie University for hosting her research stay.
QK acknowledges support from 24 FSE DDRI Grant and 2024 FSE Strategic Startup.



% Scientific work published at ACL 2023 must comply with the ACL Ethics Policy.\footnote{\url{https://www.aclweb.org/portal/content/acl-code-ethics}} We encourage all authors to include an explicit ethics statement on the broader impact of the work, or other ethical considerations after the conclusion but before the references. The ethics statement will not count toward the page limit (8 pages for long, 4 pages for short papers).


% \section*{Acknowledgements}
% aicloud , lumi supercomputer, 

% Weijun Li for discussion about decoding.


% Entries for the entire Anthology, followed by custom entries
\bibliography{anthology,custom}
\bibliographystyle{acl_natbib}


\appendix


\newpage
\section{Derivation of Normal Equation}~\label{normal_equation}
To determine the optimal transformation matrix $\mW$, we aim to minimize a cost function $J$ that quantifies the discrepancy between the attack embedding matrix $\mE_{A}$ and the victim embeddings $\mE_{V}$: 

\begin{equation}
\begin{aligned}
J(\mW) &= \frac{1}{2} (\mE_A  - \mE_V \mW)^{T} (\mE_A - \mE_V \mW) \\
& = \frac{1}{2}(\mE_A^{T} \mE_A - \mE_A^{T} \mE_V \mW -  (\mE_{V} \mW)^{T} \mE_{A} \\
& + (\mE_{V} \mW)^{T} \mE_V \mW) \\
& = \frac{1}{2}(\mE_A^{T} \mE_A - \mE_A^{T} \mE_V \mW -   \mW^{T}\mE_{V}^{T} \mE_{A} \\
& +  \mW^{T}\mE_{V}^{T} \mE_V \mW)
\end{aligned}
\end{equation}

To compute the derivatives of $J(\mW)$:

\begin{equation}
    \begin{aligned}
        \nabla_{\mW} J(\mW)  & =\frac{1}{2} \nabla_{\mW} (\mE_A^{T} \mE_A - \mE_A^{T} \mE_V \mW \\
        & - \mW^{T}\mE_{V}^{T} \mE_{A} +  \mW^{T}\mE_{V}^{T} \mE_V \mW) \\
        % & = 0- \mE_A^{T} \mE_V - \mE_{A}^{T} \mE_{V} + 2\mE_{V}^{T} \mE_V \mW  \\
        & = 2\mE_{V}^{T} \mE_V \mW -2 \mE_{V}^{T} \mE_{A} .
    \end{aligned}
\end{equation}


To minimize $J$, setting its derivatives to 0, we obtain the normal equation :
\begin{equation}
    \mE^{T}_{V} \mE_{V} \mW = \mE^{T}_{V} \mE_{A}.
\end{equation}

The matrix $\mW$ that minimizes $J(\mW)$ is

\begin{equation}
\mW = (\mE_V^{T}\mE_V)^{-1}\mE^{T}_{V} \mE_{A}.
\end{equation}




\begin{table*}[t!]
    \centering
    \resizebox{\linewidth}{!}{
    % Resize to text width
    \begin{tabular}{l|l|l|l|l}
    \toprule
    Model  &  Huggingface & Architecture & \#Languages &   Dimension\\
    \midrule
       \textsc{Flan-T5}~\citep{chung2022scalinginstructionfinetunedlanguagemodels}  & google/flan-t5-small  & Encoder-Decoder  & 60  & 512 \\
        ~\textsc{GTR}~\citep{ni2021largedualencodersgeneralizable} & sentence-transformers/gtr-t5-base &  Encoder & 1  & 768\\
        % ~\textsc{ME5} & intfloat/multilingual-e5-base &   &  & \\
    % ~\textsc{FlanT5-Base} & google/flan-t5-base &   &  & \\
    ~\textsc{T5}~\citep{raffel2023exploringlimitstransferlearning} & google-t5/t5-base &  Encoder-Decoder & 4 & 768\\
      ~\textsc{mT5}~\citep{xue-etal-2021-mt5} & google/mt5-base &   Encoder-Decoder & 102  &768\\
    ~\textsc{mBERT}~\citep{devlin2019bertpretrainingdeepbidirectional} & google-bert/bert-base-multilingual-cased &  Encoder  &  104 &768 \\
    ~\textsc{text-embedding-ada-002}& OpenAI API &  Encoder & 100+ & 1536 \\
    ~\textsc{text-embedding-3-large} & OpenAI API & Encoder & 100+ & 3072 \\
    \bottomrule  
    \end{tabular}}
    \caption{Details of LLMs and Embeddings.}
    \label{tab:llms}
\end{table*}


% We use pretrained~\textsc{FlanT5}~\citep{chung2022scalinginstructionfinetunedlanguagemodels} as the backbone to launch our attack modules, encoder~$enc_{A}(\cdot)$ and decoder~$dec_{A}(\cdot)$. For victim models, a variety of encoders are experimented on, including ~\textsc{T5}~\citep{raffel2023exploringlimitstransferlearning},~\textsc{GTR}~\citep{ni2021largedualencodersgeneralizable},~\textsc{mT5}~\citep{xue-etal-2021-mt5}, ~\textsc{mBERT}~\citep{devlin2019bertpretrainingdeepbidirectional} and OpenAI text embedders~\textsc{text-embedding-ada-002} (\textsc{ada-2}) and~\textsc{text-embedding-3-large} (\textsc{3-large}) (see the details of LLMs in Tabel~\ref{tab:llms}).


\section{Defense Mechanisms}




\subsection{WET}~\label{wet_algo}
% To investigate whether WET is able to protect embeddings from inversion attacks, we implement the following transformation on the victim embeddings:

% \begin{equation}
%     \ve_{WET} = \text{Norm}(\mT \cdot \ve_{V}) = \frac{\mT \cdot \ve_{V}}{||\mT \cdot \ve_{V}||}
% \end{equation}

$\mT$ is constructed by adopting circulant matrices~\citep{gray2006toeplitz} to ensure that the transformation matrix is both full-rank and well-conditioned to allow for accurate pseudoinverse computation for recovering the original embeddings from watermarked embeddings~\citep{shetty2024wet}, refer to~\citet{shetty2024wet} for the complete algorithm for generating $\mT$.

In detail, WET as a defense is applied to aligned embeddings with the equation~\ref{eq:wet_equation_tranformation}, where $\mW$ is the optimal solution for alignment, and $\mT$ is invertible.


% \;\stackrel{\mT\text{ is invertible}}{=}\;
\begin{equation}\label{eq:wet_equation_tranformation}
    \begin{aligned}
    \text{Norm}(\mT \mE_{V \to A} )
    & =  \text{Norm}(\mT  (\mE_V  \mW))\\
    & = \text{Norm}((\mT\mE_V)\mW )
    \end{aligned}
\end{equation}



\subsection{(Local) Differential Privacy (DP)}~\label{ldp_appendix}

As illustrated in~\citet{du2023sanitizing}, DP ensures that a randomized mechanism $\gM$ behaves similarly on any two neighboring datasets $\gX\simeq \gX'$ differing in only one individual's contribution (e.g., a sequence).
It is formally defined as follows:

\begin{definition}
    Let $\epsilon \geq 0$, $0\leq \delta \leq 1$ be \textit{two privacy parameters}. $\gM$ fulfills $(\epsilon, \delta)-$DP, if $\forall \gX\simeq \gX'$ and any output set $\gO \subseteq Range(\gM) $, 
    $\Pr[\gM(\gX)\in \gO]\leq e^{\epsilon} \cdot \Pr[\gM(\gX')\in \gO] + \delta$.
\end{definition}
If $\delta=0$, then we say that $\gM$ is $\epsilon-$DP or pure DP.

There are two popular DP settings, \textit{central} and \textit{local}.
In central DP, a trusted curator can access the raw data of all individuals, apply a Mechanism $\gM$ with random noise to ensure DP, and then release the perturbed outputs.
Local DP (LDP) is ensured without the curator by letting individuals perturb their data locally before being shared. The local DP~\citep{kasiviswanathan2011can} is defined as follows:

\begin{definition}
    Let $\epsilon \geq 0$ be a \textit{privacy parameter}. $\gM$ is $\epsilon-$LDP, 
    if for any two private inputs $\gX$, $\gX'$ and \textit{any output set} $\gO \subseteq Range(\gM)$,
    $\Pr[\gM(\mX) \in \gO] \leq e^{\epsilon} \cdot \Pr[\gM(\gX')\in \gO]$.
    
\end{definition}
However, $\epsilon-LDP$ offers homogenous protection for all input pairs, which can be too stringent in certain scenarios. When $\epsilon$ is too small, the noisy outputs are useless for utility tasks.

Thus,~\citet{du2023sanitizing} customizes heterogeneous privacy guarantees for different pairs of inputs, so called metric-based LDP, formally defined as follows:

\begin{definition}
    Let $\epsilon\geq 0 $ be the \textit{privacy parameter}, and $d$ be a suitable distance metric for the input space. 
    $\gM$ satisfies $\epsilon d$-LDP, if for any two inputs $\gX$, $\gX'$ and any output set $\gO \subseteq Range(\gM)$, 
    $\Pr[\gM(\mX) \in \gO] \leq e^{\epsilon d(\mX, \mX')} \cdot \Pr[\gM(\mX')\in \gO]$.
\end{definition}

\textit{Purkayastha Mechanism} with Purkayastha distribution and \textit{Planar Lapalace Mechanism} with Euclidean metric, are thus proposed to ensure $\epsilon d$-LDP on embeddings. Refer to~\citet{du2023sanitizing} for details in transforming embeddings with these mechanisms.





%\section{Ablation Study for Aligning Word Embeddings}
\section{Ablation Study of Leakage Data Size}\label{apendix:datasize}

The more data for alignment, the better the performance for \textbf{\ourmethod}. However, after a certain amount of data, i.e., 3K, the increase in data samples does not boost the inversion performance.
We conduct an ablation study of the sizes of leakage data for alignment in terms of inversion performances.
Fig.~\ref{fig:datasize_ablation} shows the inversion performance in Cosine Similarity (Top) and Rouge-L (Bottom) with the leakage data sizes from 1 to 8k.
As shown in Fig.~\ref{fig:datasize_ablation} (Top), embeddings for alignment are the perfect match for cosine similarities from 1 to 100 samples, then they decrease until they converge with the cosine similarities for aligned test embeddings for the corresponding encoder.
In Fig.~\ref{fig:datasize_ablation} (Top), while the inversion performances in Rouge-L increase with more data points for alignment until 8k across the encoders, the performance increases sharply from 1 to 1k. It becomes relatively stagnant from 2k to 8k.
Considering the trade-off between inversion performance and the size of data samples, we choose 1k as the upper bound of the number of data samples for alignment to conduct thorough experimentation in this work.


% Considering a trade-off between performance and data samples and a realistic scenario of data leakage in vector databases, we investigate inversion attacks with data leakage of up to 8k samples.



\begin{figure}[htbp]
    \centering
    % \includegraphics[width=\linewidth]{images/W_plots/sst2.pdf}
    
    \includegraphics[width=\linewidth]{images/W_plots/snli.pdf}
    (a) SNLI
    \includegraphics[width=\linewidth]{images/W_plots/sst2.pdf} 
    (b) SST2
    \includegraphics[width=\linewidth]{images/W_plots/sentiment140.pdf}
    (C) S140
    \caption{The Analysis of Alignment Transformation Weight ($\mW$) on Victim Encoders on different datasets.}
    \label{fig:W_alignment_density}
\end{figure}

\begin{figure*}[t!]
    \centering
    % \includegraphics[width=0.8\linewidth]{images/FlanT5-small_multiHPLT_english_log_3_models_RougeL_and_COS_lines_1_to_8000_log.pdf}
    \includegraphics[width=\linewidth]{images/ablation_datasamples/ablation_datasamples_COS.pdf}
    \includegraphics[width=0.95\linewidth]{images/ablation_datasamples/ablation_datasamples_RougeL.pdf}
    \caption{Inversion Performance vs. Leakage Data (Alignment) Sizes. (Top) shows the cosine similarities for embeddings of alignment data (dashed) and for embeddings of test data by the size of alignment data. (Bottom) shows the Rouge-L scores for the }
    \label{fig:datasize_ablation}
\end{figure*}









% \section{Crosslingual Inversion with attack model \textsc{mt5}}~\label{appendix:mt5_crosslingual}
% Crosslingual Inversion performance with attack model~\textsc{mt5}, as shown in Fig.~\ref{fig:mt5_crosslingual}.
% \begin{figure*}
% \centering
%     \includegraphics[width=\linewidth]{images/mt5-crosslingual.pdf}
%     \caption{Crosslingual Inversion Performance in RougeL with Attack Model \textsc{mt5}.}
%     \label{fig:mt5_crosslingual}
% \end{figure*}

% \begin{table*}[!ht]
%     \centering
%     \resizebox{\linewidth}{!}{
%     \begin{tabular}{l|p{16cm}}
%     \hline
%     \toprule

%     \textsc{T5} & English $\rightarrow$ French \\

%     \midrule
%        Input & Un chlorofluorocarbure (CFC) est un \textcolor{teal}{composé} organique qui ne contient que du \textcolor{teal}{carbone}, du chlor. \\

%    Reconstructed & duodenum (plural duodenums) 1 C carbon monoxide, a chemical compound composed of carbon decomposed \\
%    Translated & duodénum (duodénums pluriel) 1 C monoxyde de \textcolor{red}{carbone}, un \textcolor{red}{composé} chimique \textcolor{red}{composé} de \textcolor{red}{carbone} décomposé \\
% \midrule

%      \bottomrule

%     \end{tabular}
%     }
%     \caption{Quantitative Analysis of Crosslingaul Inversion Results from \textsc{OpenAI (ada-002)} embeddings with 1k alignment data samples and the attack model trained on MultiPHLT English dataset. The matched entities with their entity types are colored and \textbf{bolded} in \textcolor{teal}{\textbf{Input}} and \textcolor{red}{\textbf{Reconstructed}}. The mismatched reconstructed texts are in \colorbox{lightgray}{grey colored box}. 
% }
% \end{table*}

\begin{table*}[!ht]
    \centering
    \resizebox{\linewidth}{!}{
    \begin{tabular}{l|p{14cm}}
    \hline
    \toprule
            Input & This business uses tools provided by \textcolor{teal}{\textbf{TripAdvisor [ORG]}}(or one of its official Review Collection Partners) to encourage and collect guest reviews, including this one.  \\ 

        Reconstructed & This business uses tools provided by \textcolor{red}{\textbf{TripAdvisor [ORG]}} (or one of its official Review Collection Partners) to encourage and collect guest reviews, including this one \\
        \midrule
                Input & Book your flights now from  Hermosillo (\textcolor{teal}{\textbf{Mexico [GPE]}}) to the most important cities in the world. The box below contains flights from Her. \\

        Reconstructed & Book your flights now from  \colorbox{lightgray}{Las Vegas} (\textcolor{red}{\textbf{Mexico [GPE]}}) to the most important cities in the world. The box below contains flights from \colorbox{lightgray}{Las Vegas to} \\
                \midrule

                Input & KOffice Project Reviews Starts: \textcolor{teal}{\textbf{1, 2, 3, 4, 5 [CARDINAL]}} with comment only In chronological order from new to old - KOffice - OSDN Download. \\

        Reconstructed & \colorbox{lightgray}{DevOps} Project Reviews Starts: \textcolor{red}{\textbf{1, 2, 3, 4, 5 [CARDINAL]}} with comment only In chronological order from new to old - \colorbox{lightgray}{DevOp}  \\

        
                       \midrule

        Input & \textcolor{teal}{\textbf{TripAdvisor [ORG]}} is proud to partner with \textcolor{teal}{\textbf{Travelocity, Expedia, Hotels.com, Agoda, Booking.com, Price [ORG]}}. \\

        Reconstructed & \textcolor{red}{\textbf{TripAdvisor [ORG]}} is proud to partner with \textcolor{red}{\textbf{Booking.com, Expedia, Hotels.com, Travelocity, Agoda, Price [ORG]}} \\ 

         \midrule
         Input & Step \textcolor{teal}{\textbf{5 [CARDINAL]}}: Utilize Windows System Restore to """"Undo"""" Recent System Changes Windows System Restore allows you to """"go back in. \\
         
        Reconstructed & Step \textcolor{red}{\textbf{5 [CARDINAL]}}: Utilize Windows System Restore to """"Undo"""" Recent System Changes Windows System Restore allows you to """"go back in \\

        
                \midrule

    
        Input & If you want to ensure you grab a bargain, try to book \textcolor{teal}{\textbf{more than 90 days [DATE]}} before your stay to get the best price for a \textcolor{teal}{\textbf{Paris [GPE]}}.\\
         Reconstructed & If you want to ensure you grab a bargain, try to book \textcolor{red}{\textbf{more than 90 days [DATE]}} before your stay to get the best price for a \textcolor{red}{\textbf{Paris [GPE]}} \\
                       \midrule
        Input  & The Fund’s total amount for the Fund is limited to a maximum of \textcolor{red}{\textbf{\$4,000,000 [MONEY]}}, and the Fund’s total amount for each transaction is\\

        Reconstructed & \colorbox{lightgray}{means an amount up to} \textcolor{teal}{\textbf{USD 4,000,000 [MONEY]}}, \colorbox{lightgray}{for the Winners, the exact amount } \colorbox{lightgray}{is subject to the sole and final discretion of} the Fund;\\
                        \midrule
        Input & \textcolor{teal}{\textbf{Microsoft [ORG]}} is constantly updating and improving \textcolor{teal}{\textbf{Windows [PRODUCT]}} system files that could be associated with 100street\_bkg\_bikini\_bottom.swf.\\
        Reconstructed & \textcolor{red}{\textbf{Microsoft [ORG]}} is constantly updating and improving \textcolor{red}{\textbf{Windows [PRODUCT]}} system files that could be associated with \colorbox{lightgray}{jabber-shp-src.jar. Sometimes} \\

        \bottomrule

    \end{tabular}
    }
    \caption{Qualitative Analysis of In-Domain Inversion Results from \textsc{OpenAI (ada-2)} embeddings with 1k alignment data samples and the attack model trained on MultiPHLT English dataset. The matched entities with their entity types are colored and \textbf{bolded} in \textcolor{teal}{\textbf{Input}} and \textcolor{red}{\textbf{Reconstructed}}. The mismatched reconstructed texts are in \colorbox{lightgray}{grey colored box}. 
    [GPE]: Countries/cities/states; [ORG]:Organization.
    }
    \label{tab:qualitative}
\end{table*}




\begin{table}[htbp]
    \centering
    \resizebox{\linewidth}{!}{
    \begin{tabular}{c|c|cc|cc}
    \toprule 
        \textbf{Victim} & \textbf{Defense} & \textbf{Rouge-L}$\downarrow$ & \textbf{COS}$\downarrow$ & \textbf{ACC}$\uparrow$ & \textbf{F1}$\uparrow$ \\ 
        \midrule
        \textsc{random} &- &  6.57 & -0.0108 & 56.50 & 56.02 \\

        \midrule 
          & - & 25.06 & 0.9359 & 89.5 & 89.5 \\ 
         \cmidrule{2-6} 
          & WET & 24.52 & 0.9344 & 87 & 87 \\ 
         \cmidrule{2-6} 
          & Shuffling & 20.34 & 0.9358 & 89 & 89 \\ 
         \cmidrule{2-6} 
        \textsc{T5} & 0.001 & 25.15 & 0.9355 & 89.5 & 89.5 \\ 
          & 0.005 & 24.59 & 0.9308 & 88.5 & 88.5 \\ 
          & 0.01 & 22.06 & 0.9172 & 88 & 88 \\ 
          & 0.05 & 12.11 & 0.7637 & 87.5 & 87.5 \\ 
          & 0.1 & 9.57 & 0.7092 & 79 & 79 \\ 
          & 0.5 & 7.03 & 0.3169 & 59 & 59 \\ 
          & 1 & 7.05 & 0.1878 & 56.5 & 56.41 \\ 
        \midrule
          & - & 18.14 & 0.8823 & 85.5 & 85.5 \\ 
         \cmidrule{2-6} 
          & WET & 15.69 & 0.9178 & 84 & 83.97 \\ 
         \cmidrule{2-6} 
          & Shuffling & 14.69 & 0.8835 & 85.5 & 85.5 \\ 
         \cmidrule{2-6} 
        \textsc{GTR} & 0.001 & 17.22 & 0.8804 & 85 & 84.99 \\ 
          & 0.005 & 17.01 & 0.8748 & 85 & 84.99 \\ 
          & 0.01 & 15.09 & 0.8489 & 85 & 84.99 \\ 
          & 0.05 & 9.91 & 0.7136 & 83.5 & 83.5 \\ 
          & 0.1 & 9.18 & 0.6505 & 76 & 75.91 \\ 
          & 0.5 & 7.53 & 0.2297 & 55 & 54.93 \\ 
          & 1 & 6.62 & 0.0293 & 49 & 48.87 \\ 
        \midrule
          & - & 21.83 & 0.9320 & 79.5 & 79.47 \\ 
         \cmidrule{2-6} 
          & WET & 21.04 & 0.9307 & 80 & 79.93 \\ 
         \cmidrule{2-6} 
          & Shuffling & 18.19 & 0.9321 & 78.5 & 78.47 \\ 
         \cmidrule{2-6} 
        \textsc{mT5} & 0.001 & 22.13 & 0.9327 & 77.5 & 77.43 \\ 
          & 0.005 & 21.71 & 0.9277 & 80 & 79.99 \\ 
         & 0.01 & 18.97 & 0.9119 & 79.5 & 79.41 \\ 
          & 0.05 & 9.72 & 0.7410 & 77.5 & 77.34 \\ 
          & 0.1 & 8.83 & 0.6924 & 64 & 62.79 \\ 
          & 0.5 & 7.55 & 0.2407 & 51 & 50.98 \\ 
          & 1 & 6.55 & 0.1930 & 49 & 48.95 \\ 
        \midrule
          & - & 20.44 & 0.9211 & 76.5 & 76.06 \\ 
         \cmidrule{2-6} 
          & WET & 20.19 & 0.9261 & 77 & 76.66 \\ 
         \cmidrule{2-6} 
          & Shuffling & 17.07 & 0.9209 & 76 & 75.59 \\ 
         \cmidrule{2-6} 
        \textsc{mBERT} & 0.001 & 20.32 & 0.9209 & 76.5 & 76.06 \\ 
          & 0.005 & 20.01 & 0.9156 & 77 & 76.6 \\ 
          & 0.01 & 18.22 & 0.9018 & 76 & 75.59 \\ 
          & 0.05 & 10.97 & 0.7320 & 69.5 & 68.54 \\ 
          & 0.1 & 8.6 & 0.6851 & 64.5 & 64.02 \\ 
          & 0.5 & 7.06 & 0.3407 & 59.5 & 59.5 \\ 
          & 1 & 6.12 & 0.0838 & 46 & 45.86 \\ 
        \midrule
       & - & 20.15 & 0.9309 & 92 & 92 \\ 
         \cmidrule{2-6} 
          & WET & 15.83 & 0.9258 & 91 & 91 \\ 
         \cmidrule{2-6} 
         & Shuffling & 17.74 & 0.9311 & 91.5 & 91.5 \\ 
         \cmidrule{2-6} 
        \shortstack{\textsc{OpenAI}\\ \textsc{(ada-2)}}
        
        & 0.001 & 20.11 & 0.9308 & 91 & 91 \\ 
         & 0.005 & 20.01 & 0.9300 & 91 & 91 \\ 
         & 0.01 & 17.57 & 0.9179 & 90 & 90 \\ 
         & 0.05 & 9.39 & 0.7787 & 82.5 & 82.5 \\ 
        & 0.1 & 8.07 & 0.7502 & 71 & 70.9 \\ 
        & 0.5 & 7.81 & 0.4743 & 51.5 & 51.4 \\ 
          & 1 & 5.92 & 0.1324 & 53.5 & 52.2 \\ 
        \bottomrule
    \end{tabular}}
     \caption{The Inversion and Utility Performance on Classification Tasks on SST2 dataset with WET, Shuffling, Guassian Noise Insertion. From a defender's perspective, $\uparrow$ means higher are better, $\downarrow$ means lower are better.}
     \label{tab:wet_shuffling_gaussian_sst2}
\end{table}


\begin{table}[htbp]
     \centering
    \resizebox{\linewidth}{!}{
    \begin{tabular}{c|c|cc|cc}
    \toprule 
        \textbf{Victim} & \textbf{Defense} & \textbf{Rouge-L}$\downarrow$ & \textbf{COS}$\downarrow$ & \textbf{ACC}$\uparrow$ & \textbf{F1}$\uparrow$ \\ 
        \midrule 

       \textsc{random} & - & 5.01  & 0.0275  & 48.5  & 48.13  \\

        \midrule
          & - & 23.04 & 0.9219 & 71.5 & 70.37 \\ 
          \cmidrule{2-6} 
          & WET & 19.86 & 0.9186 & 69 & 67.42 \\ 
          \cmidrule{2-6} 
          & Shuffling & 17.56 & 0.9245  & 71 & 70.03 \\ 
          \cmidrule{2-6} 
         \textsc{T5} & 0.001 & 22.05 & 0.9230 & 71.5 & 70.37 \\ 
         & 0.005 & 21.11 & 0.9188 & 70.5 & 69.45 \\ 
          & 0.01 & 18.29 & 0.9078 & 69 & 67.84 \\ 
          & 0.05 & 8.85 & 0.7561 & 68 & 66.8 \\ 
          & 0.1 & 7.6 & 0.6901 & 59.5 & 58.86 \\ 
          & 0.5 & 4.97 & 0.2678 & 46 & 45.95 \\ 
          & 1 & 4.85 & 0.1101 & 50.5 & 50.47 \\ 
          \midrule
          & - & 13.22 & 0.8585 & 70.5 & 68.45 \\
        \cmidrule{2-6} 
          & WET & 12.15 & 0.8861 & 71.5 & 69.35 \\ 
        \cmidrule{2-6} 
          & Shuffling & 11.69 & 0.8599 & 70 & 68 \\ 
        \cmidrule{2-6} 
        \textsc{GTR} & 0.001 & 13.59 & 0.8594 & 69.5 & 67.38 \\ 
          & 0.005 & 12.86 & 0.8515 & 70 & 68.17 \\ 
          & 0.01 & 12.11 & 0.8260 & 71 & 69.23 \\ 
          & 0.05 & 7.92 & 0.7055 & 70.5 & 69.21 \\ 
          & 0.1 & 5.8 & 0.6137 & 64 & 63.28 \\ 
          & 0.5 & 4.6 & 0.3075 & 56.5 & 56.47 \\ 
          & 1 & 4.23 & 0.0582 & 53 & 52.83 \\ 
        \midrule
          & - & 19.82 & 0.9212  & 68 & 66.8 \\ 
                  \cmidrule{2-6} 

          & WET & 18.4 & 0.9219  & 63 & 61.61 \\ 
                  \cmidrule{2-6} 
          & Shuffling & 16.51 & 0.9215  & 66.5 & 65.44 \\ 
                  \cmidrule{2-6} 
        \textsc{mT5} & 0.001 & 20.4 & 0.9217 & 67.5 & 66.47 \\ 
          & 0.005 & 18.82 & 0.9173 & 67 & 65.89 \\ 
          & 0.01 & 16.48 & 0.9037 & 65.5 & 64.28 \\ 
          & 0.05 & 8.09 & 0.7625 & 58.5 & 57.03 \\ 
          & 0.1 & 7.02 & 0.6952 & 60 & 59.6 \\ 
          & 0.5 & 5.13 & 0.3770 & 55.5 & 55.31 \\ 
          & 1 & 4.21 & 0.1505 & 42 & 41.99 \\ 
          \midrule
          & - & 17.88 & 0.9070 & 64.5 & 64.18 \\
          \cmidrule{2-6} 

          & WET & 16.83 & 0.9151  & 63 & 62.63 \\ 
          \cmidrule{2-6} 
          & Shuffling & 13.8 & 0.9062 & 63.5 & 63.29 \\ 
          \cmidrule{2-6} 
        mBERT & 0.001 & 17.83 & 0.9077 & 64 & 63.77 \\ 
          & 0.005 & 17.01 & 0.9020 & 64.5 & 64.24 \\ 
          & 0.01 & 15.44 & 0.8901 & 63.5 & 63.17 \\ 
          & 0.05 & 8.47 & 0.7498 & 60.5 & 60.21 \\ 
          & 0.1 & 7.26 & 0.6876 & 52 & 50.39 \\ 
          & 0.5 & 5.25 & 0.2904 & 50 & 49.68 \\ 
          & 1 & 4.34 & 0.1619 & 55.5 & 55.31 \\ 
          \midrule
       & - & 17.13 & 0.9224 & 71.5 & 70.12 \\ 
        \cmidrule{2-6} 
         & WET & 14.19 & 0.9237 & 69.5 & 67.55 \\ 
         \cmidrule{2-6} 
        & Shuffling & 15.97 & 0.9229 & 71 & 69.53 \\ 
        \cmidrule{2-6} 
        \shortstack{\textsc{OpenAI}\\ \textsc{(ada-2)}} & 0.001 & 17.12 & 0.9225 & 71 & 69.53 \\ 
     & 0.005 & 17.19 & 0.9208 & 72.5 & 71.17 \\ 
          & 0.01 & 14.8 & 0.9100 & 71.5 & 70.12 \\ 
        & 0.05 & 7.4 & 0.7974 & 74 & 73.13 \\ 
         & 0.1 & 6.42 & 0.7691 & 67.5 & 67.4 \\ 
        & 0.5 & 5.36 & 0.4271 & 53.5 & 53.16 \\ 
         & 1 & 5.27 & 0.1753 & 55.5 & 54.09 \\ 
        \bottomrule
    \end{tabular}}
    \caption{The Inversion and Utility Performance on Classification Tasks on S140 dataset with WET, Shuffling, Guassian Noise Insertion. From a defender's perspective, $\uparrow$ means higher are better, $\downarrow$ means lower are better.}
         \label{tab:wet_shuffling_gaussian_s140}

\end{table}



\begin{table*}[htbp]
    \centering
     \resizebox{0.65\linewidth}{!}{
    \begin{tabular}{l|l|cccc|cccc}
            \toprule

        Victim Model & $\epsilon$ & Rouge-L & COS & ACC & F1 & Rouge-L & COS & ACC & F1 \\ 
        \midrule
        ~ & &  \multicolumn{4}{c|}{LapMech} &  \multicolumn{4}{c}{PurMech} \\ 
         % \cmidrule{3-10} 
         \midrule
          & - & 25.06 & 0.9359 & 89.5 & 89.5 & 23.04 & 0.9219 & 71.5 & 70.37 \\ 
      \cmidrule{2-10} 
          & 1 & 6.84 & 0.0103 & 54.5 & 54.5 & 4.19 & -0.0235 & 55 & 54.93 \\ 
        \textsc{T5} & 4 & 5.94 & -0.0660 & 76 & 75.96 & 4.48 & 0.0322 & 63.5 & 63.39 \\ 
    
          & 8 & 6.53 & 0.0526 & 84.5 & 84.5 & 4.29 & -0.1135 & 71 & 70.03 \\ 
          & 12 & 6.78 & 0.0551 & 89.5 & 89.49 & 4.29 & -0.0210 & 71 & 70.03 \\
          \midrule
          & - & 18.14 & 0.8823 & 85.5 & 85.5 & 13.22 & 0.8585 & 70.5 & 68.45   \\       \cmidrule{2-10} 

          & 1 & 5.86 & -0.0213 & 55.5 & 55.5 & 4.29 & -0.0079 & 50 & 49.82 \\ 
        \textsc{GTR} & 4 & 6.64 & 0.0232 & 73 & 72.98 & 4.55 & -0.0066 & 61 & 58.4 \\ 
          & 8 & 6.75 & 0.0172 & 83 & 82.99 & 4.82 & -0.0179 & 67 & 65.16 \\ 
          & 12 & 7.05 & 0.0157 & 84 & 84 & 4.85 & 0.0439 & 70 & 68.75 \\ 
          \midrule
          & - & 21.83 & 0.9320 & 79.5 & 79.47 & 17.13 & 0.9224 & 71.5 & 70.12 \\ 
        \cmidrule{2-10} 
        \textsc{mT5} & 1 & 7.18 & -0.0850 & 54 & 53.98 & 3.89 & -0.0560 & 55 & 54.93 \\ 
          & 4 & 6.78 & 0.0471 & 69.5 & 69.49 & 4.09 & -0.0255 & 66.5 & 65.03 \\ 
          & 8 & 6.75 & -0.0008 & 78 & 78 & 4.9 & -0.0166 & 68.5 & 66.66 \\ 
          & 12 & 6.71 & 0.0638 & 81 & 80.95 & 5.3 & 0.1235 & 69.5 & 67.72 \\ 
          
          \midrule
         & - & 20.44 & 0.9211 & 76.5 & 76.06 & 17.88 & 0.9070 & 64.5 & 64.18 \\ 
         \cmidrule{2-10} 
        \textsc{mBERT} & 1 & 6.59 & 0.0182 & 50.5 & 50.5 & 4.23 & 0.0215 & 49.5 & 49.5 \\ 
          & 4 & 6.11 & -0.1486 & 70 & 69.97 & 4.6 & -0.0190 & 63.5 & 63.46 \\ 
          & 8 & 6.76 & -0.0680 & 76 & 75.8 & 4.26 & 0.0942 & 63 & 62.91 \\ 
          & 12 & 6.55 & 0.0431 & 78 & 77.68 & 4.66 & -0.0340 & 62 & 61.69 \\ 
          
          
          
          \midrule
             & - & 20.15 & 0.9309 & 92 & 92 & 19.82 & 0.9212 & 68 & 66.8 \\
                 \cmidrule{2-10} 

         & 1 & 6.06 & -0.0289 & 50.5 & 50.47 & 4.52 & 0.0507 & 53 & 52.98 \\ 
        \shortstack{\textsc{OpenAI}\\ \textsc{(ada-2)}}   & 4 & 6.56 & 0.0336 & 79 & 79 & 4.14 & -0.0377 & 63.5 & 62.6 \\ 
         & 8 & 6.65 & 0.1389 & 90.5 & 90.5 & 4.31 & 0.0997 & 67 & 65.76 \\ 
          & 12 & 6.01 & -0.0267 & 94.5 & 94.5 & 4.24 & 0.0022 & 66 & 64.99 \\
          \bottomrule
    \end{tabular}}
     \caption{The Inversion and Utility Performance on Classification Tasks on SST2 dataset with local DP. From a defender's perspective, $\uparrow$ means higher are better, $\downarrow$ means lower are better.}
     \label{tab:ldp_sst2}
\end{table*}


\begin{table*}[htbp]
    \centering
    \resizebox{0.65\linewidth}{!}{
    \begin{tabular}{l|l|cccc|cccc}
            \toprule

        Victim Model & $\epsilon$ & Rouge-L & COS & ACC & F1 & Rouge-L & COS & ACC & F1 \\ 
        \midrule
        ~ & &  \multicolumn{4}{c|}{LapMech} &  \multicolumn{4}{c}{PurMech} \\ 
         % \cmidrule{3-10} 
         \midrule
          & - & 23.04 & 0.9219 & 71.5 & 70.37 & 23.04 & 0.9219 & 71.5 & 70.37 \\
          \cmidrule{2-10}
        \textsc{T5} & 1 & 4.23 & 0.0109 & 52.5 & 52.49 & 4.19 & -0.0235 & 55 & 54.93 \\ 
          & 4 & 4.23 & -0.0526 & 61.5 & 60.67 & 4.48 & 0.0322 & 63.5 & 63.39 \\ 
          & 8 & 4.19 & 0.0457 & 68.5 & 67.5 & 4.29 & -0.1135 & 71 & 70.03 \\ 
          & 12 & 4.33 & 0.0082 & 72 & 71.06 & 4.29 & -0.0210 & 71 & 70.03 \\ 
                 \midrule

          & - & 13.22 & 0.8585 & 70.5 & 68.45 & 13.22 & 0.8585 & 70.5 & 68.45 \\ 
          \cmidrule{2-10}
       \textsc{ GTR} & 1 & 4.54 & -0.0107 & 51 & 50.82 & 4.29 & -0.0079 & 50 & 49.82 \\ 
          & 4 & 4.89 & 0.0635 & 61.5 & 59.81 & 4.55 & -0.0066 & 61 & 58.4 \\ 
          & 8 & 4.55 & 0.0415 & 67 & 65.48 & 4.82 & -0.0179 & 67 & 65.16 \\ 
          & 12 & 4.64 & 0.0673 & 70.5 & 68.93 & 4.85 & 0.0439 & 70 & 68.75 \\
                 \midrule

          & - & 19.82 & 0.9212 & 68 & 66.8 & 19.82 & 0.9212 & 68 & 66.8 \\ 
          \cmidrule{2-10}
        \textsc{mT5} & 1 & 4.54 & -0.0650 & 47 & 46.99 & 4.52 & 0.0507 & 53 & 52.98 \\ 
          & 4 & 4.31 & 0.0432 & 61.5 & 60.67 & 4.14 & -0.0377 & 63.5 & 62.6 \\ 
          & 8 & 4.85 & 0.0386 & 65.5 & 64.86 & 4.31 & 0.0997 & 67 & 65.76 \\ 
          & 12 & 4.62 & 0.1038 & 66 & 64.99 & 4.24 & 0.0022 & 66 & 64.99 \\ 
                 \midrule
            & - & 17.88 & 0.9070 & 64.5 & 64.18 & 4.66 & -0.0340 & 62 & 61.69 \\ 
          \cmidrule{2-10}
        \textsc{mBERT} & 1 & 4.6 & 0.0153 & 54 & 54 & 17.88 & 0.9070 & 64.5 & 64.18 \\ 
          
         & 4 & 5 & 0.1196 & 59 & 58.9 & 4.23 & 0.0215 & 49.5 & 49.5 \\ 
          & 8 & 4.4 & 0.0156 & 60.5 & 60.28 & 4.6 & -0.0190 & 63.5 & 63.46 \\ 
          & 12 & 4.76 & 0.0117 & 61.5 & 61.45 & 4.26 & 0.0942 & 63 & 62.91 \\ 
        
                 \midrule

          & 0 & 17.13 & 0.9224 & 71.5 & 70.12 & 17.13 & 0.9224 & 71.5 & 70.12 \\ 
          \cmidrule{2-10}
        \shortstack{\textsc{OpenAI}\\ \textsc{(ada-2)}}   & 1 & 4.45 & 0.0041 & 53 & 52.83 & 3.89 & -0.0560 & 55 & 54.93 \\ 
          & 4 & 4.84 & 0.0572 & 63 & 62.46 & 4.09 & -0.0255 & 66.5 & 65.03 \\ 
         & 8 & 4.66 & -0.0191 & 68.5 & 66.97 & 4.9 & -0.0166 & 68.5 & 66.66 \\ 
        & 12 & 4.84 & 0.0728 & 71.5 & 69.83 & 5.3 & 0.1235 & 69.5 & 67.72 \\ 
                  \bottomrule

    \end{tabular}}
    \caption{The Inversion and Utility Performance on Classification Tasks on S140 dataset with local DP. From a defender's perspective, $\uparrow$ means higher are better, $\downarrow$ means lower are better.}
    \label{tab:s140_ldp}
\end{table*}

\end{document}
