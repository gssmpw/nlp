\section{Related work}
\label{sec:related-work}
Many existing studies have focused on formalizing the definition of ``feature'' across different contexts, providing structured categorizations and guidelines for consistent usage~\cite{clas08,apel09, bane12,berg15,kala18,mole22}. 
They define features from perspectives such as requirements engineering, software product lines, and information science. 
However, these studies primarily address conceptual clarity rather than empirically analyzing how features are developed and managed in practice.

There also exists a vast body of research on mining GitHub data, with many studies focusing on specific aspects relevant to our work, such as contributor behavior, project management, and communication patterns. 

While Chatziasimidis \etal focus on analyzing contributor activity primarily to identify project success factors (\eg user activity and the number of collaborators)\cite{Chat15}, our research aims to understand contributor behavior specifically in the context of feature development. 
The study by Ortu \etal provides relevant insights into contributor roles and communication patterns through comment analysis in OSS projects\cite{Ortu18}. 
However, its focus is limited to comments, whereas our research aims to analyze multiple aspects of contributor behavior, including their activity in feature branches, pull requests, and issue resolution, to gain a deeper understanding of feature-specific development practices.

The study by Onoue \etal explores the diversity of developer behaviors, including coding, commenting, and issue handling patterns, but it is limited to only two OSS projects~\cite{Ono13}. 
Rahman \etal compare successful and unsuccessful pull requests across 78 GitHub projects involving 20,142 developers and 103,192 forks~\cite{Rah14}, while our analysis is broader and targeted toward understanding the lifecycle of feature development. 
Finally, Guzman \etal analyzed 60,425 commit comments from 90 projects to understand developer sentiments~\cite{Guz14}, but again, our scope is broader and more focused on feature-specific interactions.
%Borges \etal studied factors that impact the popularity of GitHub repositories~\cite{Borg16}.
%Qi \etal mined GitHub to collect diverse real-world effort data for effort estimation~\cite{Qi17}.


%Kovalenko \etal evaluated the importance of correctly handling branches when calculating file modification histories~\cite{Kova18}.


%=================================%
% =========== SECTION ============%
%=================================%