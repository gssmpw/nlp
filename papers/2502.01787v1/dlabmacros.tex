%%%%%%%%%%%%%%%%%%%%%%%%%%%%%%%%%%%%%%%%%%%%%%%%%%%%%%%%%%%%%%%%%%%%%%%%%%%%%%%
% Introduction
%%%%%%%%%%%%%%%%%%%%%%%%%%%%%%%%%%%%%%%%%%%%%%%%%%%%%%%%%%%%%%%%%%%%%%%%%%%%%%%

% Include this file in all LaTeX papers that you write at dlab by adding a line
% "\input{dlab_macros}" right after the "\documentclass" command.

%%%%%%%%%%%%%%%%%%%%%%%%%%%%%%%%%%%%%%%%%%%%%%%%%%%%%%%%%%%%%%%%%%%%%%%%%%%%%%%
% Some standard packages
%%%%%%%%%%%%%%%%%%%%%%%%%%%%%%%%%%%%%%%%%%%%%%%%%%%%%%%%%%%%%%%%%%%%%%%%%%%%%%%

% \usepackage[utf8]{inputenc}
% \usepackage[T1]{fontenc}
\usepackage{hyphenat}
\usepackage{xspace}
% \usepackage{amsmath}
% \usepackage{amsfonts}
% \usepackage{hyperref}
% \usepackage{url}
% \usepackage{booktabs}
% \usepackage{multirow}
% \usepackage{makecell}
% \usepackage{caption}
% \usepackage{minibox}
% \usepackage{bbm}
% \usepackage{graphicx}
% \usepackage{balance}
% \usepackage{mathtools}
% \usepackage{color}
% \usepackage{marvosym}
% \usepackage{ifthen}
% \usepackage{textcomp}
% \usepackage{enumitem}
% \usepackage{verbatim}
% \usepackage{algorithm}
% \usepackage{algorithmic}
% \usepackage{numprint}
% \usepackage{balance}

\usepackage{amsthm}
\theoremstyle{plain}
\newtheorem{theorem}{Theorem}
\newtheorem{definition}[theorem]{Definition}
\newtheorem{lemma}[theorem]{Lemma}
\newtheorem{proposition}[theorem]{Proposition}
\newtheorem{example}[theorem]{Example}




%%%%%%%%%%%%%%%%%%%%%%%%%%%%%%%%%%%%%%%%%%%%%%%%%%%%%%%%%%%%%%%%%%%%%%%%%%%%%%%
% How to make your edits conspicuous
%%%%%%%%%%%%%%%%%%%%%%%%%%%%%%%%%%%%%%%%%%%%%%%%%%%%%%%%%%%%%%%%%%%%%%%%%%%%%%%

% In the final stages of editing, it is often useful to mark edits in color, so
% everyone can easily see what was changed. To do so, define a command that has
% the same name as you and use your favorite color.
\newcommand{\bob}[1]{\textcolor{red}{#1}}
\newcommand{\yourname}[1]{\textcolor{blue}{#1}}

%%%%%%%%%%%%%%%%%%%%%%%%%%%%%%%%%%%%%%%%%%%%%%%%%%%%%%%%%%%%%%%%%%%%%%%%%%%%%%%
% Latin abbreviations
%%%%%%%%%%%%%%%%%%%%%%%%%%%%%%%%%%%%%%%%%%%%%%%%%%%%%%%%%%%%%%%%%%%%%%%%%%%%%%%

% Don't use plain text for Latin abbreviations such as "e.g.", "i.e.", etc.
% Use these macros instead. Advantage: you can consistently change their style,
% e.g., if you want to typeset them in italics at some point.

% Latin abbreviations in normal font.
\newcommand{\abbrevStyle}[1]{#1}
% Latin abbreviations in italics.
% \newcommand{\abbrevStyle}[1]{\textit{#1}}

\newcommand{\ie}{\abbrevStyle{i.e.}\xspace}
\newcommand{\eg}{\abbrevStyle{e.g.}\xspace}
\newcommand{\cf}{\abbrevStyle{cf.}\xspace}
\newcommand{\vs}{\abbrevStyle{vs.}\xspace}
\newcommand{\etc}{\abbrevStyle{etc.}\xspace}
\newcommand{\viz}{\abbrevStyle{viz.}\xspace}

%%%%%%%%%%%%%%%%%%%%%%%%%%%%%%%%%%%%%%%%%%%%%%%%%%%%%%%%%%%%%%%%%%%%%%%%%%%%%%%
% Referring to sections, figures, tables, etc.
%%%%%%%%%%%%%%%%%%%%%%%%%%%%%%%%%%%%%%%%%%%%%%%%%%%%%%%%%%%%%%%%%%%%%%%%%%%%%%%

% To refer to sections, figures, tables, etc., use the following macros.
% Don't type "Section~1", "Fig.~1", etc., manually. This way, you can easily
% and consistently switch between styles, e.g., if you want to use "Sec."
% instead of "Section" at some point.

\newcommand{\Secref}[1]{Sec.~\ref{#1}}
\newcommand{\Eqnref}[1]{Eq.~\ref{#1}}
\newcommand{\Dashsecref}[2]{Sec.~\ref{#1}--\ref{#2}}
\newcommand{\Dblsecref}[2]{Sec.~\ref{#1} and \ref{#2}}
\newcommand{\Tabref}[1]{Table~\ref{#1}}
\newcommand{\Figref}[1]{Fig.~\ref{#1}}
\newcommand{\Dashfigref}[2]{Fig.~\ref{#1}--\ref{#2}}
\newcommand{\Appref}[1]{Appendix~\ref{#1}}
\newcommand{\Thmref}[1]{Thm.~\ref{#1}}
\newcommand{\Lemmaref}[1]{Lemma~\ref{#1}}
\newcommand{\Defref}[1]{Def.~\ref{#1}}

%%%%%%%%%%%%%%%%%%%%%%%%%%%%%%%%%%%%%%%%%%%%%%%%%%%%%%%%%%%%%%%%%%%%%%%%%%%%%%%
% Paragraph headings
%%%%%%%%%%%%%%%%%%%%%%%%%%%%%%%%%%%%%%%%%%%%%%%%%%%%%%%%%%%%%%%%%%%%%%%%%%%%%%%

% Academic text is often much more legible if you give important paragraphs a
% concise name that describes what the paragraph is about. Use the \xhdr
% command for this.
\newcommand{\xhdr}[1]{\vspace{0.7mm}\noindent{{\bf #1.}}}

% Same as \xhdr, but without a period after the heading. Use this version if
% the heading is directly integrated into the first sentence of the paragraph;
% e.g., "\xhdrNoPeriod{Results} are shown in \Figref{fig}."
\newcommand{\xhdrNoPeriod}[1]{\vspace{1mm}\noindent{{\bf #1}}}

%%%%%%%%%%%%%%%%%%%%%%%%%%%%%%%%%%%%%%%%%%%%%%%%%%%%%%%%%%%%%%%%%%%%%%%%%%%%%%%
% More compact lists
%%%%%%%%%%%%%%%%%%%%%%%%%%%%%%%%%%%%%%%%%%%%%%%%%%%%%%%%%%%%%%%%%%%%%%%%%%%%%%%

% In some styles, list items are widely spaced. To condense them and save some
% space, you may use this command.
\newcommand{\denselist}{ \itemsep -2pt\topsep-10pt\partopsep-10pt }

% Same, but with slightly different spacing.
\newcommand{\denselistRefs}{ \itemsep -2pt\topsep-5pt\partopsep-7pt }

%%%%%%%%%%%%%%%%%%%%%%%%%%%%%%%%%%%%%%%%%%%%%%%%%%%%%%%%%%%%%%%%%%%%%%%%%%%%%%%
% Miscellaneous useful macros
%%%%%%%%%%%%%%%%%%%%%%%%%%%%%%%%%%%%%%%%%%%%%%%%%%%%%%%%%%%%%%%%%%%%%%%%%%%%%%%

% Some bibliography styles make it hard to typeset references like
% "Einstein et al. (1905)". This command provides a convenient way to do so.
\newcommand{\textcite}[1]{\citeauthor{#1} \shortcite{#1}}

% When you frequently refer to Wikipedia articles, Wikidata entities, etc., it
% may be useful to typeset those in a particular font. Use the \cpt (for
% "concept") command for this purpose.
\newcommand{\cpt}[1]{\textsc{\MakeLowercase{#1}}}

% To exclude a large portion of text from the PDF, wrap it in \hide.
\newcommand{\hide}[1]{}

% Wrap matrix variables in \mtx. Don't make them bold etc. manually. By using
% a macro, you can consistently change the rendering style at any point.
\newcommand{\mtx}[1]{\mathbf{#1}}

% Transpose of a matrix, e.g., $A\trans{}$.
\newcommand{\trans}{^\top}

% \argmin and \argmax.
\DeclareMathOperator*{\argmax}{arg\,max}
\DeclareMathOperator*{\argmin}{arg\,min}

%%%%%%%%%%%%%%%%%%%%%%%%%%%%%%%%%%%%%%%%%%%%%%%%%%%%%%%%%%%%%%%%%%%%%%%%%%%%%%%
% Hyphenation
%%%%%%%%%%%%%%%%%%%%%%%%%%%%%%%%%%%%%%%%%%%%%%%%%%%%%%%%%%%%%%%%%%%%%%%%%%%%%%%

% Some words are ill-hyphenated by default. Here you can define the correct
% hyphenation once, and it is then used consistently.

\hyphenation{
Wi-ki-pe-dia
Wi-ki-me-dia
Wi-ki-da-ta
De-ter-mine
Page-Rank
web-page
web-pages
da-ta-set
}

%%%%%%%%%%%%%%%%%%%%%%%%%%%%%%%%%%%%%%%%%%%%%%%%%%%%%%%%%%%%%%%%%%%%%%%%%%%%%%%
% Avoid widows!
%%%%%%%%%%%%%%%%%%%%%%%%%%%%%%%%%%%%%%%%%%%%%%%%%%%%%%%%%%%%%%%%%%%%%%%%%%%%%%%

% The term "widow" refers to the first line of a paragraph if it is the last
% line on a page, or to the last line of a paragraph if it is the first line on
% a page. Widows are considered a cardinal typesetting sin, so avoid them at
% all cost, via the following commands.

\widowpenalty=10000
\clubpenalty=10000

%%%%%%%%%%%%%%%%%%%%%%%%%%%%%%%%%%%%%%%%%%%%%%%%%%%%%%%%%%%%%%%%%%%%%%%%%%%%%%%
% Enable section numbering in the AAAI style (e.g., used by ICWSM)
%%%%%%%%%%%%%%%%%%%%%%%%%%%%%%%%%%%%%%%%%%%%%%%%%%%%%%%%%%%%%%%%%%%%%%%%%%%%%%%

% In the AAAI style, this enables section numbering.
\setcounter{secnumdepth}{2}

%%%%%%%%%%%%%%%%%%%%%%%%%%%%%%%%%%%%%%%%%%%%%%%%%%%%%%%%%%%%%%%%%%%%%%%%%%%%%%%
% Listing authors in a space-economic way in the ACM style
%%%%%%%%%%%%%%%%%%%%%%%%%%%%%%%%%%%%%%%%%%%%%%%%%%%%%%%%%%%%%%%%%%%%%%%%%%%%%%%

% By default, using "\documentclass[sigconf]{acmart}" will list authors in rows
% of 2, which can take up a lot of space. To get more authors in one row, use
% something like this:
% \author{
%   \authorbox{Author 1}{Affiliation 1}{Email 1}
%   \authorbox{Author 2}{Affiliation 2}{Email 2}
%   ...
% }

% If you use \authorbox, you will also have to suppress the standard reference
% block, by pasting the following row somewhere before "\begin{document}" ...
% \settopmatter{printacmref=false, printfolios=false}

% ... and add the reference block manually after abstract and \maketitle (also
% if you use asterisks and daggers after author names without necessarily using
% \authorbox), like this:
% {\fontsize{8pt}{8pt} \selectfont
% \textbf{ACM Reference Format:}\\
% Roland Aydin, Lars Klein, Arnaud Miribel, and Robert West.
% 2020.
% Broccoli: Sprinkling Lightweight Vocabulary Learning into Everyday Information Diets.
% In \textit{Proceedings of The Web Conference 2020 (WWW '20), April 20--24, 2020, Taipei, Taiwan.}
% ACM, New York, NY, USA, 11 pages. \url{https://doi.org/10.1145/3366423.3380209}}

\newcommand{\affilSize}{9pt}
\newcommand{\authorbox}[3]{
  \minibox[c]{
    #1\\
    {\fontsize{\affilSize}{\affilSize}\selectfont{}#2}\\
    {\fontsize{\affilSize}{\affilSize}\selectfont{}#3}
  }
}

% If you add an asterisk or dagger after an author name, add the corresponding
% footnote using the \blfootnote{} command.
\newcommand\blfootnote[1]{%
  \begingroup
  \renewcommand\thefootnote{}\footnote{#1}%
  \addtocounter{footnote}{-1}%
  \endgroup
}

%%%%%%%%%%%%%%%%%%%%%%%%%%%%%%%%%%%%%%%%%%%%%%%%%%%%%%%%%%%%%%%%%%%%%%%%%%%%%%%
% Some tricks to make papers that use the Times font nicer.
%%%%%%%%%%%%%%%%%%%%%%%%%%%%%%%%%%%%%%%%%%%%%%%%%%%%%%%%%%%%%%%%%%%%%%%%%%%%%%%

\makeatletter
\newcommand{\iffont}[2]{\ifthenelse{\equal{\f@family}{#1}}{#2}{}}
\makeatother

% If the paper font is Times (e.g., in the AAAI style) ...
\iffont{ptm}{
  % ... we also want to typeset math in Times ...
  \usepackage{mathptmx}

  % ... and use a nicer Greek font, since the default is ugly.
  \DeclareSymbolFont{greek}{OML}{cmm}{m}{n}
  \DeclareMathSymbol{\alpha}{\mathalpha}{greek}{"0B}
  \DeclareMathSymbol{\beta}{\mathalpha}{greek}{"0C}
  \DeclareMathSymbol{\gamma}{\mathalpha}{greek}{"0D}
  \DeclareMathSymbol{\delta}{\mathalpha}{greek}{"0E}
  \DeclareMathSymbol{\epsilon}{\mathalpha}{greek}{"0F}
  \DeclareMathSymbol{\zeta}{\mathalpha}{greek}{"10}
  \DeclareMathSymbol{\eta}{\mathalpha}{greek}{"11}
  \DeclareMathSymbol{\theta}{\mathalpha}{greek}{"12}
  \DeclareMathSymbol{\iota}{\mathalpha}{greek}{"13}
  \DeclareMathSymbol{\kappa}{\mathalpha}{greek}{"14}
  \DeclareMathSymbol{\lambda}{\mathalpha}{greek}{"15}
  \DeclareMathSymbol{\mu}{\mathalpha}{greek}{"16}
  \DeclareMathSymbol{\nu}{\mathalpha}{greek}{"17}
  \DeclareMathSymbol{\xi}{\mathalpha}{greek}{"18}
  \DeclareMathSymbol{\pi}{\mathalpha}{greek}{"19}
  \DeclareMathSymbol{\rho}{\mathalpha}{greek}{"1A}
  \DeclareMathSymbol{\sigma}{\mathalpha}{greek}{"1B}
  \DeclareMathSymbol{\tau}{\mathalpha}{greek}{"1C}
  \DeclareMathSymbol{\upsilon}{\mathalpha}{greek}{"1D}
  \DeclareMathSymbol{\phi}{\mathalpha}{greek}{"1E}
  \DeclareMathSymbol{\chi}{\mathalpha}{greek}{"1F}
  \DeclareMathSymbol{\psi}{\mathalpha}{greek}{"20}
  \DeclareMathSymbol{\omega}{\mathalpha}{greek}{"21}
  \DeclareMathSymbol{\varepsilon}{\mathalpha}{greek}{"22}
  \DeclareMathSymbol{\vartheta}{\mathalpha}{greek}{"23}
  \DeclareMathSymbol{\varpi}{\mathalpha}{greek}{"24}
  \DeclareMathSymbol{\varrho}{\mathalpha}{greek}{"25}
  \DeclareMathSymbol{\varsigma}{\mathalpha}{greek}{"26}
  \DeclareMathSymbol{\varphi}{\mathalpha}{greek}{"27}
  \DeclareSymbolFont{otone}{OT1}{cmr}{m}{n}
  \DeclareMathSymbol{\Gamma}{\mathalpha}{otone}{0}
  \DeclareMathSymbol{\Delta}{\mathalpha}{otone}{1}
  \DeclareMathSymbol{\Theta}{\mathalpha}{otone}{2}
  \DeclareMathSymbol{\Lambda}{\mathalpha}{otone}{3}
  \DeclareMathSymbol{\Xi}{\mathalpha}{otone}{4}
  \DeclareMathSymbol{\Pi}{\mathalpha}{otone}{5}
  \DeclareMathSymbol{\Sigma}{\mathalpha}{otone}{6}
  \DeclareMathSymbol{\Upsilon}{\mathalpha}{otone}{7}
  \DeclareMathSymbol{\Phi}{\mathalpha}{otone}{8}
  \DeclareMathSymbol{\Psi}{\mathalpha}{otone}{9}
  \DeclareMathSymbol{\Omega}{\mathalpha}{otone}{10}
  \DeclareSymbolFont{syms}{OML}{cmm}{m}{it}
  \DeclareMathSymbol{\partial}{\mathord}{syms}{"40}
  \DeclareMathAlphabet{\mathbold}{OML}{cmm}{b}{it}
  \DeclareSymbolFont{largesymbols}{OMX}{cmex}{m}{n}

  % If you want to use a less curly \mathcal font than the default one used
  % with Times, uncomment this line.
  %\DeclareMathAlphabet{\mathcal}{OMS}{cmsy}{m}{n}
}