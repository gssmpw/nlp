\section{Background and Related work}
We build upon two bodies of work: research on social technologies in corporate environments and on connectivity in the workplace.

\begin{figure*}[t]
    \centering
    \includegraphics[width=0.9\textwidth]{imgs/VE_screenshot.png}
    \caption{Engage, the ESMP studied here. Engage offers standard social media functionality (algorithmic feeds, posts, reactions) and is integrated within Microsoft Teams, which offers instant messaging functionalities.}
    \label{fig:enter-label}
\end{figure*}

\vspace{0.5mm}
\noindent
\textbf{Enterprise social media.}
With the advent of Web 2.0, emphasizing user-generated content and participatory culture, major companies developed and used various social technologies~\cite{mcafee2006enterprise}.
These ranged from generic software allowing employees to easily deploy blogs on the intranet~\cite{yardi2008pulse, efimova2007crossing} to tailor-made platforms aggregating social media functionality with the organization's employee directory. 
In that context, early papers provide a detailed description of the usage, deployment, and design of early ESMPs~\cite{brzozowski2009watercooler, dimicco2008motivations, efimova2007crossing, kolari2007structure}. 
For example, the WaterCooler system deployed within HP in the late 2000s is described in \citet{brzozowski2009watercooler}. The system allowed employees to create profiles and tag themselves and each other. It also allowed employees 
to write posts and search them with sophisticated filters.
Also, in a similar spirit,  \citet{dimicco2008motivations} describes the social network deployed at IBM around 2008. ``Beehive'' allowed personal and professional sharing---employees could create and share lists and albums, as well as comment on other employees' profiles.

Other work has focused on understanding the motivations behind ESMP use. 
For example, past research found that users' participation in ESMPs was correlated to the involvement of recent managers or coworkers and with receiving feedback, e.g., in the form of comments~\cite{brzozowski2009effects}.
Indeed, employees often described being frustrated when they contributed to blogs, but their expectations of attention were unmet~\cite{yardi2009blogging}. 
It is worth noting that the adoption of social media at work was not without its hiccups. Early adopters of `employee blogs' (the proto-ESMPs) had to navigate obscure guidelines of what was and what was not allowed~\cite{efimova2009passion}---and were sometimes fired for behavior seen as inappropriate, e.g., discussing everyday life at work~\cite{coneRiseBlog2005}. In her book, ``Passion at Work,'' \citet{efimova2009passion} argues employee bloggers flourished (even amidst these setbacks) because blogging creates valuable connections and helps to improve ideas by having to articulate them in public.

Perhaps assuming their inevitable adoption, other research has examined the correlation between ESMP use and a variety of outcomes of interest to companies.
For example, \citet{aboelmaged2018knowledge} suggests that ESMP usage is correlated with employees' productivity;
\citet{pitafi2018investigating} link ESMPs to employees' agility, i.e., their ability to respond to market changes.
\citet{luo2018can} associate ESMP use to affective organization commitment
Yet, evidence of whether and how ESMPs impact companies remains preliminary. 
Relevant research relies on observational cross-sectional designs (often survey-based) that are not able to disentangle confounders associated with both ESMP use and outcomes of interest or fails to consider the full spectrum of communication technologies within a company~\cite{azaizah2018impact}.
Both these limitations threaten the validity of these papers' results. 
On the one hand, ESMP use, as these papers study, is largely impacted by self-selection, and the studies may be measuring characteristics that lead users to use ESMPs.
In other words, it may be that more productive employees adopt ESMPs, not that ESMPs make them more productive.
On the other hand, these studies often consider ESMPs in isolation and ignore that they may replace rather than complement existing communication mediums within the company.
Here, we address these limitations by employing a credible identification strategy to isolate the causal effect of ESMPs on company-level outcomes and by considering the full spectrum of corporate communication technologies. 

\vspace{0.5mm}
\noindent
\textbf{Connectivity on the workplace.}
The field of network science has found that the social structure is consequential in explaining outcomes associated with individuals, companies, and even entire economies~\cite{watts1999small,barabasi2014linked}. 
In that context, a vast, interdisciplinary literature has described and tried to understand the impact of social structure and connectivity in the workplace~\cite{reagans2003network,raub1990reputation,soda2021networks}.
For example, \citet{granovetter1973strength} showed how `weak ties,' casual connections, and loose acquaintances play a disproportionally large role for job-seekers.
\citet{coleman1988social} highlighted the importance of network closure, noting that tightly connected networks increase cooperation and social capital.
\citet{burt_structural_1992} proposed the concept of `structural holes,' arguing that individuals positioned as mediators of two closely connected groups  (often referred to as bridges) gain important competitive advantages within companies (and the most varied scenarios).

Much of the foundational research on connectivity in the workplace (and in general) happened before the popularization of the Web and the widespread adoption of Web-based communication platforms at work~\cite{coleman1988social,granovetter1973strength,burt_structural_1992}. In that context, online communication networks provided a suitable context for systematic, empiric investigation of these theories.
A study of 21 billion Facebook friendships found that connectedness among individuals with varying economic status is critical to upward income mobility~\cite{chetty2022social}.
A large experiment on LinkedIn manipulated the strength ties recommended by the website's recommendation algorithm, finding experimental causal evidence
supporting the `strength of weak ties'~\cite{rajkumar2022causal}.
An extensive analysis of a dynamic (email) social network of a university~\cite{kossinets2006empirical} examined, among other things, the temporal stability of bridges and structural holes, finding them to be largely unstable.
This work draws from these large-scale empirical investigations as we attempt to uncover whether mechanisms idealized by the designers of ESMPs happen in practice.

Concerns about how connectivity impacts the workplace were heightened by the advent of remote work, drastically accelerated by the COVID-19 pandemic~\cite{phillips2020working}.
Recent work has indicated that remote work trends may harm companies by inadvertently altering their internal social structure.
\citet{yang2022effects} found that firm-wide remote work led to more static and siloed collaboration networks.
\citet{yu2023large} found significant gaps in the social networks of workers onboarded remotely during the pandemic, indicating that they may be at a professional disadvantage.
\citet{brown2023effects} found that time spent alone during the pandemic correlates with poorer organizational well-being.
Here, we investigate to which extent ESMPs may counteract some of the challenges associated with transitioning to remote work.

\vspace{0.5mm}
\noindent
\textbf{Engage.} In Figure~\ref{fig:enter-label}, we depict Engage, the ESMP considered in this study. Engage offers features like stories and posts that mirror familiar social media experiences (e.g., Facebook, Twitter) and integrates seamlessly with Microsoft Teams and other Microsoft 365 tools (which we also consider here when looking at companies' communication networks).

\begin{figure}[t]
    \centering
    \includegraphics[width=\linewidth]{imgs/img1.pdf}
    \caption{For a random company in our sample, we show the number of active users across ESM, Email, and Instant Messaging in the 69-week study period considered. A vertical dotted line indicates when the company suddenly adopted Engage, Microsoft's ESMP solution. We use the discontinuity around the adoption of Engage to study the effect of its adoption on the company's communication network (see Sec.~\ref{sec:did} for methodological details).}
    \label{fig:breakingpoint_example}
\end{figure}