\section{Literature review}
____ were the first to formally define max-flow games from the perspective of cooperative game theory. They demonstrated that the core of a flow game has a relatively explicit structure and established that flow games are equivalent to totally balanced games, implying that every subgame of a flow game has a non-empty core. The more general multi-commodity flow games were considered in ____.  Most follow-up studies on max-flow games  focused on the computation and refinement of the core and the nucleolus. For instance, ____ presented an algorithm for computing the nucleolus of simple flow games in a general setting, ____ showed that the nucleolus of a simple flow game can be efficiently computed in polynomial time using a linear programming duality approach, ____ developed algorithms for computing the core and f-nucleon of simple flow games.  

Regarding core refinement, ____ employed the dual linear program of the max-flow linear program to partially characterize the core,  ____ introduced the Dual-Consistent Core (DCC) as a refinement of the core,  offering improved fairness and computational efficiency compared to the core.

While edges in the max-flow problem are viewed  players in the aforementioned literature (including this paper), there are also models that view nodes as players.  Investigating Internet traffic flows using  cooperative game theory is also one of the earliest models in the community of algorithmic game theory, where each node of the network is a server and is viewed as a player  in the model ____. ____ contributed to understanding cooperative behaviors in decentralized flow networks by exploring flow allocation games, where players controlling nodes allocate flows strategically. They analyzed the computational complexity of equilibrium solutions, characterized core allocations, and proposed algorithms to ensure stability in flow distribution. 


To the best of our knowledge, ____ were the first to consider mechanism design problems in the context of max-flow, as an application of the general theory they studied with a single private parameter for each player. In the setting of max-flows,  ____ assumed that the single private parameter of each player is (eventually) the edge capacity, as in this paper. However, their payoff setting differs significantly from ours: they used the inverse of the edge capacity to define the unit transportation cost, and the payoff for each player is the allocated payoff minus the total cost (the inverse of this edge capacity times the flow amount). Moreover, they also assumed that the transferred payoff  and the allocated flow of each player (given other players' reported parameters) are twice differentiable in the player's reported parameter, which is not the case in our paper (they are not even differentiable). 
Mechanism design problems related to max-flows are also considered in ____ and ____, where players are neither edges nor nodes, but source-sink pairs (thus their models are closer to the congestion game). 
All mentioned papers concentrated on truthfulness of the mechanism as well as computational issues, whereas we do not focus on computation but on more desirable properties of the mechanism. To the best of our knowledge, we are the first to consider SIR, SP, MP, and CM in the setting of max-flows.
____ also considered mechanism design in max-flows problems and each player in their model may possess capacities of multiple edges. However, they did not consider truthfulness issues and only core allocations were studied.