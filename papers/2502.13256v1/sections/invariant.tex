\section{Invariant Rules}

\begin{definition}
Invariant rules are defined as physical conditions that must be satisfied for any given state of an Industrial Control System (ICS). These rules are derived from the operational data of the ICS and describe the expected relationships between sensor readings and actuator states. If these conditions are violated, it indicates a deviation from normal operation, which can be used to detect anomalies in the system. The rules help identify whether the system is operating within its normal parameters or if any unusual activities could signify faults or cyber-attacks\cite{21}. Invariants are stable relationships or dependencies between different components of a system that remain consistent under normal operating conditions. Detecting these invariants is crucial for monitoring system health and identifying anomalies when these relationships break\cite{23}.
\end{definition}
The proposed framework introduces a novel, systematic approach to automatically generating invariant rules from data logs of Industrial Control Systems (ICS) to enhance anomaly detection. Invariant rules, which define the normal operational conditions of ICS by specifying physical relationships between sensors and actuators, are essential for identifying deviations that could signify faults or cyber-attacks. Traditionally, these rules are manually defined by system engineers based on design specifications, a process that is both costly and prone to errors, especially in complex systems. To overcome these limitations, our framework leverages machine learning and data mining techniques to automatically derive invariant rules directly from historical operational data logs.

The process begins with data collection, where sensor measurements and actuator states are recorded over time during the normal operation of the ICS. This data is then processed to generate "predicates," or conditions that describe specific states or relationships within the system. Predicates for actuator states are straightforward, while predicates for continuous sensor readings are generated using two strategies: a distribution-driven strategy that models sensor reading updates using Gaussian Mixture Models (GMMs) to identify hidden control states, and an event-driven strategy that identifies critical sensor values that trigger changes in actuator states.

Once the predicates are generated, the framework converts each data log entry into an "itemset"—a collection of satisfied predicates—and applies association rule mining to discover meaningful invariant rules. These rules are defined as those that meet a minimum support condition, ensuring they are statistically significant, and are non-redundant, providing unique information that is not captured by other rules. To optimize the number of generated rules while minimizing false positives, the framework incorporates a parameter tuning process using a validation dataset. This process fine-tunes the parameters to balance rule generation with the expected false positive rate, ensuring the model is both effective and efficient for real-time anomaly detection.

The generated rules are then applied to real-time operational data to detect anomalies, with each data entry checked against the rules. Violations indicate potential anomalies, prompting further investigation or immediate response. This data-driven approach is validated through case studies on two real-world ICS testbeds, demonstrating significantly improved anomaly detection performance over traditional design-based and residual error-based models. The framework not only automates the rule generation process but also provides a robust, scalable solution for enhancing ICS security, capable of adapting to the evolving nature of industrial environments and threats.\cite{21}(invariant and detection)

$$**************************$$

ARTINALI is a dynamic invariant detection approach specifically designed to enhance the security of Cyber-Physical Systems (CPS) by incorporating time as a fundamental property alongside data and events. The process begins with the instrumentation of the CPS source code to monitor and log key events, data variables, and time information. Events, such as system calls or significant operations like reading a sensor value or sending a network packet, are identified and logged during normal (attack-free) operation. The collected logs provide detailed execution traces, including event types, data states, and timestamps, which are essential for the subsequent invariant mining process.

ARTINALI mines three types of invariants from these execution traces: Data per Event (D|E), Event per Time (E|T), and Data per Time (D|T). The D|E invariants capture the relationships between data variables and specific events by grouping the system logs by event and applying frequent item set mining to identify common data patterns associated with each event. For instance, if a variable like \texttt{temperature} always falls within a certain range during a specific event like ``sensor\_read,'' a D|E invariant would specify this range for that event. E|T invariants are then derived to understand the timing relationships between events. This involves identifying sequences of events and calculating the typical time intervals between them. For example, an E|T invariant might state that the event ``send\_data'' should always follow ``sensor\_read'' within a specific time window, such as 5 to 10 seconds. Finally, ARTINALI combines the D|E and E|T invariants to infer D|T invariants, which describe how data values change over time in relation to events. This process involves calculating the conditional probabilities of data values occurring within specific time intervals between events, providing a comprehensive understanding of the temporal data relationships.

The mined invariants are then used to configure an Intrusion Detection System (IDS) that continuously monitors the CPS for deviations from these learned invariants. During runtime, if an event occurs outside its typical time window, if a data variable takes on an unexpected value, or if there is any deviation from the established patterns, the IDS flags a potential intrusion or anomaly. For example, in a smart meter system, if the event ``send\_data'' occurs much later than the expected time following ``process\_data,'' or if the power consumption data suddenly falls outside its normal range, the IDS would detect these deviations as potential security threats. By dynamically detecting invariants that consider the interplay between data, events, and time, ARTINALI significantly reduces false positives and negatives, providing a robust framework for real-time CPS security\cite{22}.

\textbf{Different. }The ARTINALI paper differentiates itself from other research in the field of Cyber-Physical System (CPS) security by introducing a novel approach that incorporates time as a first-class property alongside data and events for dynamic invariant detection. Unlike traditional methods that focus solely on data or event relationships, ARTINALI mines multi-dimensional invariants by considering data, events, and time simultaneously, providing a more comprehensive model of system behavior. This integration significantly reduces false positives and negatives, enhancing the accuracy of intrusion detection systems (IDS) for CPS. The paper demonstrates ARTINALI's effectiveness on real-world CPS platforms, such as smart meters and smart medical devices, showing substantial improvements over existing tools. By effectively capturing the complex interactions within CPS and improving real-time monitoring capabilities, ARTINALI offers a robust framework for detecting sophisticated attacks, laying a foundation for future advancements in CPS security.

$$************************$$

illiad detects invariants in cyber-physical systems (CPS) through a systematic approach that combines data-driven and model-based methodologies to monitor the system's normal operating conditions and identify anomalies. The first step in the process involves collecting and preparing data from various sensors and devices within the CPS. In a typical example, such as a microgrid with a solar panel, battery storage system, and local load, sensors would gather time-series data on variables like solar power output, battery state of charge (SOC), and load power consumption. This data is then preprocessed to address missing values and noise, ensuring it is normalized and consistent for further analysis.

The next step is to define the invariants and dependencies within the system. Invariants are stable relationships between components that remain consistent under normal operating conditions. For instance, in the microgrid example, an invariant might be the relationship between battery SOC, solar power output, and load power. If solar power output is high and load power is low, it is expected that the battery charges, leading to an increase in SOC. illiad aims to detect these kinds of relationships to monitor the system's health.

To detect these invariants, illiad utilizes a hybrid approach combining three primary techniques: Autoregressive Models with Exogenous Inputs (ARX), Latent Factor Analysis (LFA), and Kalman Filter-Based State Estimation. ARX models predict a time series, such as battery SOC, based on its own past values and the past values of other related series, like solar power and load power. If the prediction error of an ARX model remains consistently low, illiad infers a stable invariant relationship between the components. Latent Factor Analysis is then employed to uncover hidden relationships that may not be directly observable but are inferred through latent variables influencing multiple observed variables. For example, a latent factor might represent an underlying environmental condition, such as sunlight availability, affecting both solar power output and load power. Finally, Kalman Filters are used for dynamic state estimation, combining current measurements with previous state estimates to predict the system's state. If Kalman filters accurately estimate the state over time, a strong invariant relationship is inferred.

Once these models have been applied, illiad integrates the results to construct an invariant graph, where each node represents a system component and each edge represents an inferred invariant relationship. The system selects the strongest relationships—those with the least prediction error—from the ARX, LFA, and Kalman Filter models. Thresholds are applied to ensure that only meaningful, stable relationships are included as invariants in the graph. For example, if ARX predicts battery SOC with less than a 5\% error but LFA has a 10\% error, ARX would be selected as the basis for the invariant. The final invariant graph provides a comprehensive view of the system's normal operating conditions.

illiad continuously monitors these invariant relationships in real-time to detect any deviations that might indicate anomalies. An anomaly is detected when a relationship deviates beyond acceptable limits, indicating a possible fault or failure in the system. For example, if the battery SOC decreases unexpectedly despite high solar power and low load power, this could signal a malfunctioning battery. When an anomaly is detected, illiad issues an alert, highlighting the broken invariant and pinpointing the component potentially causing the issue.

To facilitate real-time monitoring and quick response, illiad includes a visual dashboard that displays the invariant graph and highlights any broken invariants as they occur. This interactive dashboard allows operators to drill down into specific components, view historical data, and make informed decisions for troubleshooting and maintenance. By integrating data-driven and model-based approaches, illiad provides a robust framework for detecting invariants, monitoring system health, and promptly identifying anomalies in complex cyber-physical systems\cite{23}.

$$**********************$$

To detect invariants in Cyber-Physical Systems (CPS), they propose a method that combines machine learning with mutation testing. The objective is to learn invariants that define correct CPS behavior and identify deviations indicating faults or malicious activity.

The process begins with collecting two types of sensor data traces: \textit{positive traces}, representing normal behavior, and \textit{negative traces}, representing abnormal behavior. While positive traces are obtained from normal operations, generating negative traces requires simulating faults or security breaches. This is achieved through \textit{mutation testing}, where small changes (mutations) are introduced to the CPS software, simulating faults. Running the mutated software on the CPS or a simulator generates sensor data that serves as negative samples.

Next, they use these traces to train a machine learning model, specifically a Support Vector Machine (SVM), to classify system behavior. The SVM learns to distinguish between normal and abnormal states based on features extracted from the sensor data, such as sensor readings over time. The output is a set of \textit{learned invariants}—conditions that must hold for the system to be considered in a normal state. For example, an invariant might require that a water tank's level remains between specific thresholds.

To ensure these invariants are accurate and reliable, they apply \textit{verification techniques} like statistical model checking and symbolic execution. Statistical model checking provides probabilistic guarantees by analyzing system executions, while symbolic execution verifies invariants against all possible code paths. Together, these techniques validate the learned invariants, ensuring they effectively monitor and control the CPS for safe operation.

This integrated approach leverages the strengths of machine learning, software testing, and formal methods to robustly detect and verify invariants, enhancing the safety and security of CPS in critical applications\cite{24}.

$$*******************************$$

The invariants in the study were detected using \textit{association rule mining}, a data-driven approach that automates the discovery of relationships between system components. The method was applied to data collected from the Secure Water Treatment (SWaT) testbed, a scaled-down version of a water treatment plant designed for cybersecurity research. The dataset contained 11 days of sensor and actuator data, with the first seven days representing normal operations and the remaining days including a series of cyber and physical attacks. This extensive dataset, comprising over 3 million rows, provided the basis for mining relationships between system components.

Before applying the mining algorithm, the data needed to be preprocessed. Sensor and actuator values were numerical and often contained values with several decimal places, making it difficult to generate rules directly. To address this, the researchers categorized the data into discrete ranges, such as "High," "Low," "On," and "Off," based on the operational thresholds of the system. This process transformed the continuous data into categorical data, which is more suitable for association rule mining. By converting the data in this way, the study ensured that the rules generated would be interpretable and aligned with the operational states of the water treatment system.

The researchers then applied the \textit{Apriori algorithm}, a widely used technique in association rule mining, to the preprocessed dataset. The algorithm identifies frequent itemsets—combinations of sensor and actuator states that appear together frequently—and from these itemsets, it generates rules that describe the relationships between different system components. Each rule is evaluated based on its \textit{support}, which measures how often the combination occurs in the dataset, and its \textit{confidence}, which indicates the likelihood that the rule holds true. In total, the algorithm generated approximately 11,500 rules, providing a comprehensive set of constraints that describe the normal behavior of the system.

Once the rules were generated, they were validated by comparing them to known physical invariants derived from the system's design. These known invariants, which were manually defined, served as a benchmark to assess the accuracy of the automatically generated rules. For example, one known invariant is that when a motorized valve (MV101) is open, the flow indicator (FIT101) should display a high value. The Apriori algorithm successfully detected this relationship, with a confidence value of 100\%, indicating that the rule was always observed to be true in the dataset. In contrast, the reverse condition—that FIT101 should not be high when MV101 is closed—had a confidence value of only 0.49\%, further confirming the validity of the automatically generated rules.

While the method successfully detected pairwise relationships between sensors and actuators, it is currently limited in its ability to capture more complex, multi-component interactions that are common in Cyber-Physical Systems (CPS). For example, the behavior of one process may depend on the state of multiple sensors and actuators across different stages of the system. Additionally, the current approach does not account for time-based dependencies, meaning it does not consider how system states evolve over time. Addressing these limitations will be important for future work to improve the accuracy and applicability of the method for detecting cyber-attacks in CPS environments\cite{25}.

$$**************************************$$

The detection of invariants in the proposed approach is carried out using a systematic method called \textit{Design to Invariants (D2I)}. This method derives invariants directly from the design and behavior of the physical process in an Industrial Control System (ICS). The process starts by modeling the ICS using \textit{hybrid automata}, which are formal models capable of representing both the discrete and continuous dynamics of system components. For example, in a water treatment plant, components like pumps and valves are modeled as having discrete states (such as open or closed), while components like tanks are modeled with continuous states (such as varying water levels). The hybrid automaton provides the foundation for understanding how these components behave individually and in combination.

The next step extends the hybrid automata model to better represent the interactions between discrete and continuous states through the use of \textit{state markers} and \textit{invariant multi-graphs}. The state markers define key points in the behavior of continuous components—such as high and low water levels in a tank—that help track the system's evolution. By establishing relationships between these states and the operations of discrete components (such as valves and pumps), \textit{invariants} are derived. These invariants are boolean conditions that must always hold true for the system to be considered operating normally. For instance, an invariant might specify that if a valve is open, the flow rate through a pipe should fall within a particular range. These conditions can be \textit{state-specific}, applying only when the system is in a particular configuration, or \textit{state-agnostic}, applying universally throughout the system's operation.

Once derived, the invariants are \textit{coded and installed in the Programmable Logic Controllers (PLCs)} that manage the physical processes within the ICS. These PLCs continuously monitor the system's state by checking sensor data and actuator operations against the established invariants. When an invariant is violated—meaning the boolean condition returns false—it serves as an indicator of anomalous behavior, which may be due to a cyber attack or a component failure. The invariants act as real-time checks on the system's physical state, ensuring that the ICS behaves as expected according to the system's design.

For example, in a water treatment plant, an invariant might state that when a valve is open and a pump is running, the water level in a connected tank should decrease. If the water level remains constant despite the pump operating, the invariant would be violated, signaling a potential issue. This could indicate a cyber attack, such as a compromised valve that is being falsely reported as open, or a mechanical failure.

The invariants also play a critical role in \textit{detecting cyber attacks}. They capture the expected physical relationships between system components, so if an attacker compromises one or more components, such as by preventing a valve from opening while simultaneously sending false sensor data to the PLC, the violation of an invariant would trigger an alert. Even sophisticated attacks, like \textit{multi-point attacks} that manipulate multiple system elements, can be detected when the combined behavior of the components does not conform to the expected invariants.

In summary, the D2I method systematically derives and installs process-based invariants in the ICS control logic, allowing the system to be continuously monitored for anomalies. By comparing real-time operations against the physical and logical constraints defined by the invariants, the system is able to detect cyber attacks or malfunctions that would otherwise go unnoticed. This approach provides a robust and efficient method for securing ICS operations without relying on statistical anomaly detection or machine learning, making it particularly well-suited for critical infrastructure systems\cite{26}.

$$**************************************$$

Invariants in Cyber-Physical Systems (CPS) are essential to ensuring system correctness, stability, and fault tolerance. The process of detecting and using invariants involves a systematic approach to integrate the cyber, physical, and network components into a cohesive system. This section outlines the step-by-step process of detecting and applying invariants in CPS using a simplified smart grid example, where power management between energy sources and loads is controlled by a cyber algorithm.

The first step is to define the core components of the CPS, which include the cyber, physical, and network domains. In our example, the \textit{cyber component} consists of software algorithms responsible for monitoring power usage and controlling distribution, the \textit{physical component} includes the energy sources (e.g., solar panels) and loads (e.g., homes or devices), and the \textit{network component} represents the communication between these elements. These components need to work together in a coordinated way to maintain system stability and functionality.

The next step is to identify the \textit{cyber invariant}, which ensures the correctness of the software algorithm managing power distribution. A key invariant in this context could be that the total power produced by the energy sources must always be equal to or greater than the total power requested by the loads. Mathematically, this can be expressed as:
\[
P_{\text{produced}} \geq P_{\text{requested}}.
\]
This condition ensures that the system does not attempt to allocate more power than is available, preventing potential instability or power outages.

The \textit{physical invariant} is detected using a Lyapunov-like function to model the stability of the physical components. In our example, the physical system's stability is tracked by monitoring the energy levels and ensuring they remain within a stable range. A Lyapunov-like function, such as 
\[
V(\omega) = \frac{1}{2} J (\omega - \omega_0)^2,
\]
where \( \omega_0 \) is the nominal frequency (e.g., 60 Hz), can represent how far the system deviates from its stable state. For the system to be stable, the derivative of this function must be non-positive, ensuring that the system's energy either remains constant or decreases over time. This helps detect whether the physical system is on the verge of instability.

The \textit{network invariant} ensures that communication delays between components do not disrupt the system's operation. In the smart grid example, an important network invariant is that the time delay between sending power commands and receiving acknowledgments must remain below a threshold \( T_{\text{max}} \). If communication delays exceed this threshold, it could result in outdated or incorrect power commands, potentially leading to instability in the physical system. Therefore, the network invariant is expressed as:
\[
\Delta t_{\text{message}} \leq T_{\text{max}},
\]
and the system must ensure that this condition is met.

After identifying individual invariants for the cyber, physical, and network components, the next step is to \textit{compose these invariants} into a system-wide invariant. This system-wide invariant governs the overall operation of the CPS, ensuring that each component functions within its safe bounds without interfering with others. The combined invariant could be expressed as:
\[
P_{\text{produced}} \geq P_{\text{requested}} \quad \land \quad \frac{dV}{dt} \leq 0 \quad \land \quad \Delta t_{\text{message}} \leq T_{\text{max}}.
\]
This ensures that the cyber system only requests power within the limits of production, the physical system remains stable, and the network delivers commands in a timely manner.

Once the system-wide invariant is defined, the next step involves testing the system under various simulated conditions. For example, the smart grid system might be subjected to power spikes or delays in communication to see if the invariants hold. If any of the invariants are violated during simulation, this would indicate potential instability, requiring corrective actions. For instance, if the cyber invariant is violated (e.g., power demand exceeds supply), the system may reduce power consumption or activate backup generators to restore balance. Similarly, if the physical invariant is violated due to frequency instability, the system could adjust power flows or reduce the load to regain stability.

Finally, if any invariant is violated during real-time operation, the system must take \textit{corrective action} to restore stability. For example, if the cyber invariant fails due to over-demand, the system might reduce power requests or shift to alternative energy sources. If the physical invariant is violated (e.g., frequency instability), the system could temporarily cut power to non-critical loads or adjust power flow. If the network invariant fails due to excessive delays, alternative communication channels might be used, or power dispatch commands could be paused until the network stabilizes.

In conclusion, detecting and applying invariants is crucial for maintaining the stability and correctness of a CPS. Invariants, particularly those based on Lyapunov-like functions for the physical domain and axiomatic proofs for the cyber domain, provide a robust framework for monitoring and ensuring the overall stability of the system. By integrating these invariants, system designers can effectively manage complex CPS environments like smart grids and ensure they operate safely and reliably\cite{27}.


$$**************************************$$

The detection of invariants is a systematic process that ensures stability and correctness in cyber-physical systems (CPS) composed of computational, physical, and networking subsystems. The process begins by analyzing the distinct stability requirements of each subsystem—cyber, physical, and network—since each has its own specific conditions for stability and correctness. The key challenge is to find a unified set of invariants that bridge these subsystems, ensuring that the entire system can function as a cohesive unit. This involves defining logical predicates, or invariants, that remain true throughout the execution of the system, even as it switches between different operating modes.

For the physical subsystem, the paper uses \textit{Lyapunov-like functions} to define stability. Lyapunov functions are traditionally employed to analyze the stability of continuous dynamic systems, and in cases where a true Lyapunov function cannot be found, a Lyapunov-like function is used as a substitute. This function must be positive definite, radially unbounded, and non-increasing to ensure that the system remains stable. The authors apply these functions to switched systems, where each operational mode has its own Lyapunov-like function. The stability of the system is maintained if the value of the function forms a non-increasing series at the switching points between modes.

On the cyber side, the authors define \textit{cyber invariants} that guarantee the correctness of computational processes. These invariants are rooted in the concept of non-interference, which ensures that the actions in one part of the system do not interfere with the stability of other subsystems. Non-interference is crucial for composing the proofs across the cyber, physical, and networking subsystems. By proving that actions in the cyber component do not invalidate the physical system's stability, the authors ensure that the invariants can be applied across the entire CPS, creating a cohesive framework that governs the system's behavior.

The paper then integrates the cyber and physical invariants to create a unified invariant that can govern the entire system. This invariant combines the cyber correctness and physical stability into a common framework, ensuring that the CPS operates as a whole without interference between the subsystems. The authors validate these unified invariants through simulations, which test the system under different operating conditions and verify that the Lyapunov-like functions decrease over time, indicating stability. If the functions show a non-increasing behavior during switching events, the system remains stable.

Finally, to ensure the system-wide stability, the authors rely on \textit{non-interference} between the subsystems. This means that each invariant must hold true independently while ensuring that actions in one subsystem do not interfere with the invariants of another. By ensuring non-interference and composing the invariants from each subsystem, the authors achieve a unified invariant that guarantees overall system stability and correctness, even in complex distributed environments with switching dynamics\cite{29}.

$$**************************************$$

In the approach described in the paper, detecting invariants involves a systematic process that integrates machine learning and statistical validation to model and monitor the behavior of a Cyber-Physical System (CPS). The process begins with the collection of system data through simulations of both normal and abnormal behavior. The normal system behavior is simulated by running the system under normal operating conditions, where sensor data traces, such as water levels in a water treatment system, are recorded. To generate abnormal behavior, mutations are systematically introduced into the software components of the CPS, particularly in the programmable logic controllers (PLCs), simulating subtle faults or cyber-attacks. These mutations result in abnormal system traces, such as an unintended overflow of a water tank. This step produces a large dataset of normal and abnormal traces, which form the foundation for learning invariants.

In the second step, these raw traces are transformed into feature vectors, representing snapshots of sensor readings at two different points in time. For example, in a water tank system, a feature vector may represent the water level at two distinct time intervals. The feature vectors derived from normal system traces are labeled as "normal," while those from mutant-induced abnormal traces are labeled as "abnormal." This conversion of raw data into feature vectors allows the system's behavior to be systematically analyzed. By collecting both normal and abnormal traces, the system now has labeled data that can be used for training a supervised machine learning model.

Next, a supervised learning algorithm, such as a Support Vector Machine (SVM), is applied to the labeled feature vectors to learn a model that can distinguish between normal and abnormal behaviors. The SVM identifies patterns in the feature vectors that characterize normal behavior, such as stable water levels within a safe range. The trained SVM becomes a classifier capable of predicting whether a new feature vector, derived from live sensor data, corresponds to normal or abnormal behavior. For instance, if the water level in a tank rises beyond an expected threshold due to a faulty pump, the SVM classifier would flag this as abnormal. This step allows the system to "learn" the invariant properties of the CPS—those properties that should always hold true under normal conditions.

Once the classifier has been trained, it undergoes validation through statistical model checking to ensure that it accurately captures the system's normal behavior and can be considered a valid invariant. Statistical model checking is used to assess whether the classifier's predictions are statistically reliable by observing additional normal traces and applying a hypothesis test, such as the Sequential Probability Ratio Test (SPRT). If the classifier consistently labels normal behavior correctly with high confidence, it is validated as an invariant—an assertion about the system that holds in all normal operating conditions.

Detect Attack.
Finally, the validated invariant can be deployed to monitor the system in real-time. The CPS is continuously checked against the learned invariant, and any deviation from the expected behavior, such as sensor readings indicating abnormal water levels or pump actions, triggers an alert. In this way, the learned invariant becomes a powerful tool for detecting attacks or faults in the system. For example, if a network attack causes the sensor readings to be manipulated, resulting in an incorrect actuator response, the invariant would detect the inconsistency and flag it for further investigation. This method provides a robust, automated means of monitoring CPS for potential security threats or malfunctions, leveraging machine learning to understand the system's normal behavior and statistical techniques to validate its correctness\cite{28}.

