\section{Introduction}

\IEEEPARstart{I}{n} our connected world, \emph{Cyber-Physical Systems} (CPS) are systems that combine computation and physical processes, changing the way we live and work. These systems integrate digital technology with physical actions, and they are all around us, playing an important role in industries like healthcare, transportation, energy, and manufacturing \cite{117,118}. Imagine a world where driver-less cars make smart decisions on the road, power grids adjust themselves to meet energy demand, and medical devices communicate instantly to save lives \cite{119}. All of these things are possible because of CPS, which link computer systems with physical actions, creating smarter and more adaptive solutions. CPS not only make things more efficient but also add a level of accuracy and resilience that traditional systems cannot match \cite{120}.

The \emph{Internet of Things} (IoT) is an important part of CPS. IoT is a specific type of CPS that involves devices linked together to share data over a network \cite{121,122}. Both involve physical devices connected to computers, but IoT specifically refers to devices that are linked together and share data over a network. Just like CPS, IoT systems can experience anomalies, which are unusual behaviors caused by things like system errors or even attacks, as shown in fig \ref{fig:test_file}, this is a type of anomaly detection \cite{116,123,124}. Some researchers focus on finding and fixing these anomalies in IoT systems to keep them safe and working properly. Similarly, \emph{Industrial Control Systems} (ICS) are critical components that manage and oversee industrial processes. ICS are widely used in sectors such as manufacturing, power generation, and water treatment to ensure that physical processes are controlled efficiently and safely \cite{125,126,127}.

\emph{Anomalies} in CPS are deviations from normal operational behavior that may indicate security threats, system malfunctions, or faults. These deviations can take various forms, including unexpected changes in sensor readings, unusual network traffic patterns, irregular actuator behavior, deviations in control commands, unauthorized access attempts, and anomalous packet structures. Anomalies can broadly be categorized into two types: attacks and faults. Attacks encompass various malicious activities such as denial-of-service (DoS) attacks, man-in-the-middle (MITM) attacks, packet injection, unauthorized protocol use, and dictionary attacks targeting web interfaces. Faults, on the other hand, arise from unexpected issues within the system, such as sensor and actuator malfunctions, which can disrupt normal operations and degrade system performance. Detecting these anomalies is important due to the integration of heterogeneous technologies and the interaction between cyber and physical components in CPS. The challenge lies in identifying these deviations amidst the complex and dynamic nature of these systems. Anomalies can signal a range of issues, from benign system errors to sophisticated cyberattacks, making their timely detection essential for maintaining the integrity, availability, and confidentiality of CPS \cite{116, 16, 128,129,130,131,132,133,134,135,136}.

\emph{Anomaly detection} plays a crucial role in CPS as it helps identify irregular behaviors that deviate from the system's normal operations. Anomalies can often result from malicious attacks targeting the system, and such deviations from expected behavior can have significant consequences. In the context of CPS, anomalies can lead to system failures, financial losses, and even endanger human lives. Therefore, ensuring the safety and security of CPS by effectively detecting and addressing anomalies is vital for system stability and reliability \cite{137,138}.

Most of the research papers in the field of anomaly detection focus on physical-based CPS or physical-based anomaly detection \cite{139}. These systems and methods emphasize the monitoring and analysis of physical components and processes within a CPS, such as sensors, actuators, and mechanical parts \cite{140}. Since these physical elements are directly tied to real-world operations, they play a critical role in identifying and addressing anomalies \cite{141}.

Physical-based anomaly detection relies on understanding the physical behavior of the system. This includes tracking parameters like temperature, pressure, vibration, and flow rates to identify any deviations from normal conditions. For instance, in a manufacturing plant, abnormal vibrations in a motor might signal a mechanical failure, while unusual pressure readings in a pipeline could indicate a potential leak. These approaches typically use sensor data to detect problems in real-time \cite{142,143,144}. Sensors continuously monitor the physical aspects of the system, providing data that can be analyzed for irregular patterns. For example, techniques like threshold-based monitoring can quickly flag values that exceed safe operational limits, while more advanced methods like machine learning models can identify subtler patterns that may indicate early signs of failure \cite{145,146}.

In recent years, many research papers\cite{15,16,17,18} introduce new methods for detecting anomalies in CPS. These methods are diverse, and each paper presents a different approach. For this survey, we have chosen the most important papers in the field. Although each paper uses a unique method, we have classified them into a few main groups.

This paper presents existing solutions by providing a comprehensive survey that seeks to answer the key questions: Which methodologies have been used for anomaly detection in CPS over recent years? 

\begin{figure}[h]
    \centering
    \includegraphics[width=\linewidth]{images/tyoe-of-anomaly.png}
    \caption{A Type of Anomaly Detection}
    \label{fig:test_file}
\end{figure}

To do this, the paper categorizes and compares various anomaly detection methods for CPS, including machine learning, deep learning, mathematical, hybrid, invariant-based, and other approaches. By exploring the strengths and weaknesses of these diverse methods, this survey aims to provide a balanced understanding of their accuracy, efficiency, and suitability for different CPS environments. 
\begin{comment}
Machine learning methods help to identify normal behavior patterns, deep learning techniques offer the ability to uncover complex and non-obvious patterns, mathematical methods provide rigorous modeling and formal verification, hybrid approaches combine strengths from multiple techniques for enhanced performance, and invariant-based methods establish system rules to detect deviations effectively. The system is designed to be flexible, learning from new data as the CPS environment changes, which helps it stay effective against new threats. This approach also focuses on using resources efficiently, which is important given the limited computing power of many CPS devices. By keeping the resource demands low, the system ensures that added security does not hurt overall performance.
\end{comment}
The goal of this survey is to present and analyze existing solutions for creating a safer and more reliable CPS environment. By examining advanced anomaly detection methods that can work in real-time and adapt to changing conditions, this survey aims to highlight the gaps left by current solutions. As CPS continue to grow and change, our security solutions must evolve accordingly. This paper aims to ensure that the systems we depend on are not only smarter but also safer, keeping them secure, efficient, and ready to meet the needs of an increasingly automated world.\\

\noindent{\textit{Roadmap.}} We provide an overview of CPS and related systems, such as ICS and IoT, highlighting their relationships and importance in various sectors in Section \ref{sec:overview}. In Section \ref{sec:challenges}, we discuss the primary security challenges faced by CPS, including system-level vulnerabilities, threats, and mitigation strategies. In Section \ref{sec:anomaly-detection}, we provide a detailed survey of anomaly detection techniques used in CPS, which is divided into multiple categories: machine learning approaches, deep learning methods, hybrid techniques, mathematical approaches, and invariant-based methods. Each of these sections explores specific detection methodologies, their strengths, and their applicability to different CPS environments and in section \ref{sec:general-scheme} we provide a general view of anomaly detection. In Section \ref{sec:realtime}, we talk about the importance of real-time anomaly detection. Finally, we provide the future work in section \ref{sec:future} and the conclusion in section \ref{sec:conclusion}.

