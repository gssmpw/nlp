\section{Anomalies}
\subsection{Definition}
Anomalies in cyber-physical systems (CPS) are deviations from normal operational behavior that may indicate security threats, system malfunctions, or faults. These deviations can take various forms, including unexpected changes in sensor readings, unusual network traffic patterns, irregular actuator behavior, deviations in control commands, unauthorized access attempts, and anomalous packet structures\cite{14,20}.

Anomalies can broadly be categorized into two types: attacks and faults. Attacks encompass various malicious activities such as denial-of-service (DoS) attacks, man-in-the-middle (MITM) attacks, packet injection, unauthorized protocol use, and dictionary attacks targeting web interfaces\cite{13,15}. Faults, on the other hand, arise from unexpected issues within the system, such as sensor and actuator malfunctions, which can disrupt normal operations and degrade system performance\cite{15,18}.

Detecting these anomalies is crucial due to the integration of heterogeneous technologies and the interaction between cyber and physical components in CPS\cite{16}. The challenge lies in identifying these deviations amidst the complex and dynamic nature of these systems. Anomalies can signal a range of issues, from benign system errors to sophisticated cyberattacks, making their timely detection essential for maintaining the integrity, availability, and confidentiality of CPS\cite{17,19}.