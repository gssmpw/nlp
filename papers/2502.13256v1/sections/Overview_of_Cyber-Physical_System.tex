\section{Overview of CPS and ICS}\label{sec:overview}

\subsection{CPS and ICS}
CPS combine computers and physical processes, working together through embedded computers and networks to monitor, control, and integrate these processes smoothly \cite{2,140,147}. CPS tightly link software and hardware, allowing systems to handle tasks like real-time data processing, autonomous control, and quick adjustments\cite{3}. These systems, as described by several scholars, are known for their ability to work reliably and adaptively in various environments, needing new methods that combine physical actions and computer models\cite{1}. CPS uses embedded computing, sensors, and network communication to go beyond traditional control systems, enabling smart and software-driven operations in areas like autonomous vehicles, space exploration, medical devices, and industrial automation\cite{5,208}. 

CPS is vital in developing the Internet of Things (IoT), offering and using online data-accessing and data-processing services \cite{202}. This connection and interaction between physical and cyber parts highlight the importance of their integration, not just their combination\cite{2}. The complex nature of CPS requires a broad approach that combines knowledge from system science, engineering, and computer science to create strong, reliable, and efficient systems\cite{4}. By linking physical processes and computer control, CPS opens new possibilities for improved human-machine interaction and advanced control mechanisms, showing the ongoing progress and integration of technology in various fields\cite{6}.

Industrial Control Systems (ICS), on the other hand, are a collective term that encompasses various types of control systems and associated instrumentation used for industrial process control \cite{148,149}. These systems include Supervisory Control and Data Acquisition (SCADA) systems, Distributed Control Systems (DCS), and other configurations such as Programmable Logic Controllers (PLC) \cite{150,151,152,153,154}. ICS are integral to the operations of critical infrastructures and industrial sectors such as power generation, water treatment, and manufacturing \cite{155,156}. They consist of numerous control loops, human-machine interfaces, and remote diagnostics tools, and are built using an array of network protocols to achieve various industrial objectives like manufacturing and transportation of matter or energy\cite{7}.

An ICS encompasses various control systems and instrumentation used for industrial process control, including SCADA systems, DCS, and PLCs. These systems are designed to manage large-scale, complex industrial processes, ensuring operational efficiency and safety. ICS are often interconnected with corporate and Internet networks, which increases their vulnerability to cyber threats\cite{8,157,158,159}. Typically, ICS involves numerous field devices such as sensors, actuators, Remote Terminal Units (RTUs), and PLCs, which interact with physical processes. Additionally, ICS includes supervisory devices like SCADA servers, Human Machine Interfaces (HMIs), and engineering workstations, all of which facilitate the management and operation of industrial processes\cite{9,160}. The architecture of ICS is often based on the Purdue Model, which divides the network into logical segments with different functionalities, including the Enterprise Zone (IT network), the Demilitarized Zone (DMZ), the Control Zone (OT network), and the Safety Zone\cite{10,161}.

Originally, ICS focused on SCADA and PLC systems, but with technological advancements, they have evolved to incorporate complex computer-based control systems. These systems are now integral in managing processes across various industries such as electricity, water, oil and gas, chemical, transportation, and manufacturing\cite{11,162}.

\subsection{ICS vs. CPS}
While ICS are focused primarily on the control and automation of industrial processes, CPS represent a broader concept that integrates computation with physical processes. CPS involves a tight coupling between computational elements and physical entities, often through embedded systems and sensors, to monitor and control physical environments. Unlike traditional ICS, CPS encompasses a wider range of applications beyond industrial control, including smart grids, autonomous vehicles, and smart buildings. Both ICS and CPS share the goal of enhancing efficiency and functionality, but CPS extends this integration to more diverse and interconnected domains, creating additional cybersecurity challenges due to its broader scope\cite{7, 163,164,165}.
ICS can be considered a subset of CPS, tailored to industrial environments with specific requirements for safety, reliability, and availability. CPS applications extend beyond industrial control to areas such as smart grids, autonomous vehicles, and healthcare systems\cite{8,205}.
While both ICS and CPS involve the interaction of physical and digital components, CPS encompasses a wider range of applications and is characterized by more complex interactions between computational and physical elements. The paper's focus on ICS highlights the specific challenges and security considerations in industrial environments, particularly concerning sequence attacks and the need for specialized intrusion detection mechanisms\cite{9}.
ICS primarily focuses on the management and control of industrial processes through interconnected devices and systems that operate physical processes, emphasizing real-time operational reliability and safety. CPS, on the other hand, extends beyond traditional ICS by integrating computation, networking, and physical processes more cohesively, often incorporating advanced control strategies and data analytics to optimize performance\cite{10}.
CPS extends beyond industrial applications to include areas like smart grids, autonomous automotive systems, medical monitoring, and robotics. While ICSs are a subset of CPS focusing on industrial applications, CPS represents a broader paradigm that combines cyber capabilities with physical processes across diverse fields, emphasizing seamless integration and interaction between the computational and physical elements\cite{11}.
In contrast, CPS focuses on seamless integration and coordination between computational and physical elements across various fields, not limited to industrial processes. CPS applications often involve more complex interactions and require sophisticated algorithms to manage the interplay between digital and physical worlds\cite{12}.