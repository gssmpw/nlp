\section{Real-Time Anomaly Detection}\label{sec:realtime}

Real-time anomaly detection plays a crucial role in ensuring the smooth operation of CPS. These systems, which are deeply embedded in industries such as manufacturing, transportation, and energy, integrate physical processes with digital control \cite{194,195,196,197}. They rely on sensors, networks, and software to make real-time decisions. However, when something goes wrong like a sensor failure, a cyber attack, or a system glitch it is essential to catch the problem quickly to avoid damage. That is where real-time anomaly detection steps in \cite{198,199,200}.

\subsection{Need for Real-Time Detection}

Imagine a factory running 24/7, producing goods non-stop. In such a high-paced environment, every minute counts. If one machine starts to malfunction or an attacker tries to disrupt the system, it needs to be identified instantly, before it impacts the entire production line. A delay in detection could cause equipment breakdowns, safety hazards, or even financial losses. This is especially true in autonomous vehicles or power grids, where a small delay in detecting a malfunction could mean life-threatening accidents or power outages \cite{98,99}.

So, the challenge is clear: detect problems as soon as they occur, with minimal delay. But doing this in real-time is not easy. CPS are complex systems, often generating huge amounts of data from various sensors. Processing this data quickly and accurately requires advanced technology \cite{101,60}.

\subsection{Challenges of Real-Time Detection}

One of the biggest challenges in real-time anomaly detection is latency the system needs to react fast, without causing delays in operations. For instance, a smart grid must continue to provide electricity while monitoring for any anomalies, like unusual power consumption patterns. If the system takes too long to detect the anomaly, it could lead to power failures \cite{50}.

Another challenge is dealing with noisy data. Sensors in CPS can generate a lot of unnecessary or incorrect data, which can confuse the detection system. Imagine a factory where one sensor is faulty and constantly gives wrong readings. If the detection system can not filter out this noise, it might either miss the real problem or give too many false alarms \cite{98,60}.

Additionally, CPS environments often produce imbalanced data, meaning that the system operates normally most of the time, with only a few instances of anomalies. This makes it harder for detection systems to learn from past problems and detect new ones \cite{60,50}.

\subsection{Approaches to Real-Time Anomaly Detection}

To tackle these challenges, researchers have developed several approaches to real-time anomaly detection. These methods range from traditional statistical techniques to cutting-edge machine learning and deep learning models.

\subsubsection{Hybrid Models}
A common solution is to use hybrid models that combine both traditional statistical methods and machine learning. For example, models like \textit{SARIMA} (Seasonal Autoregressive Integrated Moving Average) can predict future sensor readings based on past data. When combined with machine learning models like LSTM (Long Short-Term Memory), which is great for handling time series data, these models can detect anomalies that occur over time \cite{60}. This approach is especially useful in ICS, where monitoring the normal operations of machines over time helps detect subtle changes that might indicate a problem.

\subsubsection{Deep Learning Models}
Deep learning is another powerful tool. Deep learning models, like Convolutional Neural Networks (CNNs) and Recurrent Neural Networks (RNNs), can analyze sensor data in real-time to find patterns that indicate an anomaly. For example, in autonomous vehicles, CNNs have been used to monitor sensor data from various systems, like cameras and radar, to detect any abnormal behavior \cite{97,99}. These models are particularly effective because they can automatically learn from large amounts of data, even when the patterns are complex or hidden from the human eye.

Additionally, autoencoders a type of neural network are becoming popular for real-time anomaly detection in CPS. These models are trained to recreate normal data patterns, and when they encounter something that does not fit the normal pattern, they flag it as an anomaly \cite{15}.

\subsubsection{Human-Cyber-Physical Systems (HCPS)}
With the rise of Industry 5.0, there is a growing recognition that humans should play a central role in these systems. The concept of HCPS brings humans back into the loop by integrating human expertise with the digital system's capabilities. In industries like manufacturing, human workers often have valuable experience that can help in identifying anomalies that machines might miss. For example, in a smart factory, human experts might oversee an anomaly detection system that uses big data analytics and edge computing to monitor real-time production \cite{101}.

In one case study, an HCPS system was deployed to monitor vinyl flooring production. The system combined human knowledge with machine learning to detect quality issues in real-time, sending feedback to adjust machine operations automatically \cite{101}. This approach highlights how human input can enhance the detection system, especially in environments where machine-learning models might struggle with the complexity or variability of the data.

\subsubsection{Edge Computing for Low Latency}
Edge computing is a technology that processes data close to where it is generated, rather than sending it all to a central server. This reduces the time it takes to analyze the data and send back a decision. In a smart factory, edge computing might be used to monitor machines and detect problems as they happen, without the delays that come with cloud computing \cite{100,101}. This is particularly useful in real-time anomaly detection, as decisions need to be made instantly.

\subsubsection{Adversarial Machine Learning and Evasion Attacks}
As CPS systems become smarter, so do the attackers. Adversarial machine learning involves using techniques to fool detection systems into missing an anomaly or incorrectly classifying normal data as a problem. For example, attackers might manipulate sensor readings to look normal, while in reality, they are carrying out an attack. This is known as an evasion attack \cite{96}. To counter this, researchers have developed machine learning models that are trained to recognize these subtle, adversarial changes in the data, making real-time detection systems more resilient to sophisticated attacks \cite{96}.

\subsection{Real-World Applications of Real-Time Anomaly Detection}

Real-time anomaly detection is being applied in various fields:

\begin{itemize}
    \item \textbf{Industrial Manufacturing:} Factories use real-time monitoring systems to ensure that machines operate smoothly. If a machine starts behaving abnormally, the system can immediately alert operators or shut down the machine to prevent further damage \cite{100}.
    \item \textbf{Autonomous Vehicles:} In self-driving cars, real-time anomaly detection ensures that all sensors and systems are working correctly. If a sensor starts malfunctioning, the car can respond by switching to backup systems or pulling over safely \cite{97,98}.
    \item \textbf{Smart Grids:} Power grids use real-time detection to monitor electricity usage and detect any unusual patterns, which could indicate a cyber-attack or system failure \cite{98}.
\end{itemize}

\subsection{Future Directions}

Looking to the future, the focus will be on improving the scalability and resilience of real-time anomaly detection systems. One promising direction is the use of \textit{decentralized detection}, where anomalies are detected locally at different points in the system, rather than relying on a central system. This approach is expected to make systems more robust and less vulnerable to single points of failure \cite{50}.

Another exciting development is the integration of blockchain technology, which could enhance the security and integrity of the data used for anomaly detection. By ensuring that data cannot be tampered with, blockchain could make real-time anomaly detection even more reliable in sensitive environments like power grids and healthcare \cite{101}.
