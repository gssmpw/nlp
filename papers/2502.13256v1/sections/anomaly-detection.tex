\section{Anomaly Detection}\label{sec:anomaly-detection}

Anomaly detection plays a crucial role in CPS as it helps identify irregular behaviors that deviate from the system's normal operations. Anomalies can often be a result of malicious attacks targeting the system, and these deviations from the expected behavior can have significant consequences. In the context of CPS, an anomaly can potentially lead to system failures, financial losses, and even endanger human lives. Therefore, ensuring the safety and security of CPS by effectively detecting and addressing anomalies is vital for the system's stability and reliability.
\begin{definition}
Anomalies in CPS are deviations from normal operational behavior that may indicate security threats, system malfunctions, or faults. These deviations can take various forms, including unexpected changes in sensor readings, unusual network traffic patterns, irregular actuator behavior, deviations in control commands, unauthorized access attempts, and anomalous packet structures\cite{14,20}.
Anomalies can broadly be categorized into two types: attacks and faults. Attacks encompass various malicious activities such as denial-of-service (DoS) attacks, man-in-the-middle (MITM) attacks, packet injection, unauthorized protocol use, and dictionary attacks targeting web interfaces\cite{13,15}. Faults, on the other hand, arise from unexpected issues within the system, such as sensor and actuator malfunctions, which can disrupt normal operations and degrade system performance\cite{15,18}.
Detecting these anomalies is important due to the integration of heterogeneous technologies and the interaction between cyber and physical components in CPS\cite{16}. The challenge lies in identifying these deviations amidst the complex and dynamic nature of these systems. Anomalies can signal a range of issues, from benign system errors to sophisticated cyberattacks, making their timely detection essential for maintaining the integrity, availability, and confidentiality of CPS\cite{17,19}.
\end{definition}
The IoT is an important part of CPS, though many people mistakenly use the two terms as if they are the same. In fact, IoT is a type of CPS. Both involve physical devices connected to computers, but IoT specifically refers to devices that are linked together and share data over a network. Just like CPS, IoT systems can experience anomalies, which are unusual behaviors caused by things like system errors or even attacks. Some researchers focus on finding and fixing these anomalies in IoT systems to keep them safe and working properly.

In recent years, many research papers have introduced new methods for detecting anomalies in CPS. These methods are diverse, and each paper presents a different approach. For this survey, we have chosen the most important papers in the field. Although each paper uses a unique method, we have classified them into a few main groups.

Some papers use machine learning techniques, which rely on algorithms to recognize patterns and detect anomalies. Others use deep learning methods, often based on neural networks, to detect anomalies by learning complex patterns in the data. There are also papers that combine machine learning and deep learning to improve detection accuracy. Another group of papers relies on mathematical approaches, such as statistics, probability, and formal methods, to detect unusual behaviors. Some methods are based on invariants, which identify anomalies by checking whether the system follows certain rules. Additionally, hybrid methods combine multiple approaches from the mentioned groups for better detection.

Most papers fit into one of these groups, but occasionally, researchers propose new methods that do not belong to any of these categories. These papers are placed in the others group, although they are not very common.

\begin{figure*}
    \centering
    \includegraphics[width=1\linewidth]{images/schema2.png}
    \caption{General View of Anomaly Detection}
    \label{fig:general-view}
\end{figure*}
\subsection{Machine Learning Approaches for Anomaly Detection}

Anomaly detection in CPS using machine learning follows several important steps to ensure unusual behaviors are detected accurately. These steps generally involve gathering data, cleaning it, selecting key features, choosing the best machine learning models, and finally testing the models to make sure they can reliably detect anomalies. Each stage is designed to handle the large volumes of data generated by CPS and helps to detect problems early.

\subsubsection{Data Collection and Preprocessing}

The first step in machine learning-based anomaly detection is data collection. This involves gathering data from various CPS components like sensors, logs, or network traffic. For instance, in an energy grid, data might be collected from transformers, power lines, and smart meters to ensure all critical parts of the system are monitored. Having access to this data makes it possible to detect early signs of malfunction or attack, which is crucial for preventing system failures \cite{44}.

In IoT networks, large volumes of data are gathered from interconnected devices such as sensors, cameras, or smart home systems. This data includes network traffic, device activity logs, and sensor readings, all of which are critical for detecting potential security threats. Real-time data collection helps establish a baseline for normal operations, which is essential for distinguishing between typical behaviors and suspicious activities \cite{84}.

In many studies, a systematic approach has been used to detect anomalies in CPS using machine learning techniques. The process begins with a clear definition of potential attack scenarios that could threaten the integrity of CPS, particularly focusing on a water treatment facility. They categorize ten distinct types of attacks, such as inflow manipulations, outflow disruptions, and tank level alterations, each designed to exploit specific vulnerabilities within the system. For instance, one attack scenario involves changing the inflow sensor reading to zero, thereby misleading the system into thinking that no water is entering the facility \cite{92}.

Following the generation of training data, the authors proceed to preprocess the data by normalizing sensor readings to ensure consistency across the dataset. They label the data according to whether the system's state is normal or indicative of an attack, facilitating the application of supervised machine learning techniques.

After collecting the data, preprocessing is performed to clean and prepare the data for analysis. This might involve dealing with missing data, removing outliers, and transforming the data into a common format. Techniques like mean imputation are used to fill missing values, while outliers that could distort the results are removed. For instance, in IoT systems, preprocessing might include converting different types of sensor data into numerical formats to standardize them \cite{84, 85}. Additionally, dimensionality reduction methods like Principal Component Analysis (PCA) are often applied to simplify the dataset while keeping the most critical information \cite{44}.

\subsubsection{Feature Engineering and Time Series Analysis}

Once the data is clean, feature engineering is used to extract key information from the raw data. In CPS, time-series analysis is particularly important since the systems continuously generate data over time. For example, in smart grids, features like average power consumption or voltage spikes over time help distinguish between normal and abnormal behavior \cite{44, 66}. Domain-specific knowledge plays a big role here, as it helps to create features that are especially useful for the system in question.

In ICS, a method combines machine learning and fuzzy logic to detect anomalies. Fuzzy logic helps reduce false alarms by evaluating how severe the anomaly is, ensuring that important issues are flagged while less critical ones are minimized \cite{89}.

\subsubsection{Model Selection: Supervised, Unsupervised, and Semi-Supervised Learning}

Choosing the right machine learning model is essential. If there are labeled datasets available (i.e., when normal and abnormal behaviors are known), supervised models like Support Vector Machines (SVM) or Random Forests are commonly used. These models learn from the labeled data and can effectively classify new data as normal or anomalous \cite{85}.

In IoT networks, Random Forest and Decision Tree algorithms are widely applied for supervised anomaly detection. These models rely on labeled datasets to differentiate between normal and abnormal behaviors in real-time, ensuring quick detection of anomalies such as denial-of-service (DoS) attacks or unauthorized access \cite{84}.

The training phase in the aforementioned CPS attack detection study involves feeding the labeled dataset into nine different classifiers, including Support Vector Machines (SVM), Random Forests (RF), Decision Trees, and Bayesian Networks. These classifiers are trained to detect specific types of attacks, such as inflow manipulation or tank level alteration, by learning the behaviors associated with normal and attack states \cite{92}.

However, in cases where labeled data is limited, unsupervised learning techniques are used. These models, such as K-Means clustering or Gaussian Mixture Models (GMM), can detect anomalies without predefined labels by identifying outliers based on patterns in the data. For example, in smart grids, unsupervised models can help detect irregularities in real-time sensor readings, signaling potential system faults \cite{66}.

Semi-supervised learning is applied when datasets contain a mix of labeled and unlabeled data. Techniques such as self-training or consistency regularization can enhance anomaly detection in complex environments where obtaining labels for all anomalies is challenging \cite{44}.

\subsubsection{Handling Imbalanced Datasets}

An important challenge in CPS is the imbalance between normal data and rare anomalies. Traditional models may fail to detect rare but critical anomalies because they are too focused on the more common normal data. The Causality-Guided Counterfactual Debiasing Framework (CDF) addresses this by using causal graphs to identify and remove bias in model predictions, making anomaly detection more accurate \cite{90}.

\subsubsection{Real-Time Detection and Evaluation}

Once trained, machine learning models are evaluated using metrics like precision, recall, F1 score, and the Area Under the Curve (AUC-ROC) to assess how well they detect anomalies while minimizing false positives. This step ensures the model works reliably before being deployed for real-time monitoring.

For example, logistic regression models have been used to detect faults in smart grids by monitoring data from Phasor Measurement Units (PMUs). This real-time detection helps prevent large-scale failures in power systems by identifying issues early on \cite{66}. Similarly, Random Forest models have been deployed in IoT systems to detect cyberattacks in real-time using a fog computing architecture, which allows for faster anomaly detection \cite{78, 85}.

In Cyber Manufacturing Systems (CMS), machine learning models have been used to detect anomalies in data such as acoustic signals and images collected from machines like 3D printers and CNC mills. These models analyze deviations from normal patterns in physical data, allowing for real-time detection of cyberattacks or system malfunctions \cite{52}.

The models used in the aforementioned CPS attack detection study are deployed for continuous monitoring. Incoming data from sensors and actuators is analyzed in real-time, with significant deviations from normal behavior flagged as potential attacks. Additionally, the classifiers can classify the type of attack, allowing for a more targeted response to the detected threats \cite{92}.

\subsubsection{Improving Robustness}

To ensure machine learning models are reliable, they are tested under both normal and adverse conditions, such as noisy data or deliberate cyberattacks. For instance, in safety-critical CPS such as Artificial Pancreas Systems (APS), adding domain knowledge has been shown to reduce robustness errors by up to 54.2\%. This makes the models more reliable in detecting anomalies, even when the input data is slightly distorted \cite{63}.

By combining machine learning techniques with domain-specific knowledge and rigorous evaluation, these methods offer a reliable approach to safeguarding CPS. Whether applied in smart grids, IoT systems, or industrial settings, machine learning enhances real-time anomaly detection, improving the security and resilience of CPS.


\begin{comment}
\subsection{Deep Learning Approaches}

Deep learning has become a powerful tool for anomaly detection in Cyber-Physical Systems (CPS), enabling the detection of complex patterns in multivariate time-series data. These models can capture both spatial and temporal dependencies, making them ideal for real-time monitoring of highly dynamic environments. Various deep learning methods, such as generative models, autoencoders, graph-based networks, and hybrid approaches, offer significant improvements in detecting subtle or rare anomalies.

\subsubsection{Generative Models for Anomaly Detection}

Generative models are frequently used in anomaly detection due to their ability to model the underlying distribution of normal data and identify deviations from it. One notable approach is the MTS-DVGAN model, which combines deep generative models with contrastive learning to detect anomalies in multivariate time-series data. The model uses an LSTM-based encoder to learn latent representations of CPS data and generates reconstructed versions of the input. By comparing the reconstruction and discrimination losses, the model identifies anomalies when the deviation between real and reconstructed data becomes significant. The integration of contrastive learning further enhances the model's ability to distinguish between normal and abnormal behavior, making it particularly effective in scenarios where anomalies are close to normal data distributions \cite{87}.

Similarly, the ATTAIN system integrates generative modeling with a digital twin for real-time anomaly detection. A digital twin—a virtual representation of the CPS—learns from historical and real-time data to predict system behavior. The system compares predicted values with real-time sensor data, flagging deviations as potential anomalies. A Generative Adversarial Network (GAN) further strengthens this process by generating synthetic attack scenarios, improving the system's ability to detect complex and emerging threats \cite{68}.

\subsubsection{Autoencoders for Real-Time Anomaly Detection}

Autoencoders are widely used for anomaly detection in CPS, particularly because of their ability to learn the normal behavior of a system and detect deviations through reconstruction errors. A prominent example is the 1D Convolutional Autoencoder (1D-ConvAE) used in decentralized real-time anomaly detection systems for CPS production environments. This model is trained on normal system data, learning to recreate typical data patterns. During live operations, if the data significantly deviates from these learned patterns, the system flags it as an anomaly. Decentralizing the detection process ensures faster responses, as each component of the CPS monitors itself without relying on a central unit \cite{50}.

Another example is the RmsAnomaly model, which captures both temporal dependencies and inter-sensor correlations. RmsAnomaly constructs signature matrices from multivariate time-series data, allowing it to capture the relationships between different sensors. A convolutional autoencoder processes this data, comparing the original and reconstructed data to detect anomalies. The model employs an adaptive threshold mechanism that adjusts based on the training data, improving detection accuracy while minimizing false positives \cite{46}.

In complex CPS environments, models such as BiGRU-VAE combine Bidirectional Gated Recurrent Units (BiGRU) with Variational Autoencoders (VAE) to enhance anomaly detection. The BiGRU captures both past and future dependencies, while the VAE maps data into a latent space to learn normal patterns. Reconstruction errors serve as the basis for anomaly detection, with high accuracy demonstrated in critical infrastructure systems like the Secure Water Treatment (SWaT) dataset. This approach is particularly effective for real-time monitoring of time-series data in CPS, where early detection of anomalies is crucial for preventing system failures or cyberattacks \cite{57}.

\subsubsection{Hybrid Deep Learning Approaches}

Hybrid deep learning models combine different architectures and techniques to improve anomaly detection. The Adaptive-Correlation-Aware Unsupervised Deep Learning (ACUDL) model is one such hybrid approach, which combines traditional autoencoders with graph-based learning. ACUDL builds an initial correlation graph using K-Nearest Neighbors (KNN), dynamically updating the graph as the system evolves. The model uses a Dual-Autoencoder (D-AE) structure to extract both non-correlated and correlated features, with a Gaussian Mixture Model (GMM) used for probabilistic anomaly scoring. This dynamic, hybrid approach ensures that the system adapts to changing relationships between sensors and devices, making it highly suited to noisy and evolving environments \cite{91}.

Another example of hybrid models is the method using multipath neural networks to monitor the performance of autoencoders in Intrusion Detection Systems (IDS). This model continuously evaluates the reconstruction errors of autoencoders, using a Wilcoxon-Mann-Whitney test to detect deviations in neural network confidence levels. The combination of multipath neural networks and statistical analysis allows for the detection of subtle anomalies, such as spoofing attacks, that might be missed by more straightforward models \cite{88}.

\subsubsection{Graph-Based Neural Networks for Anomaly Detection}

In scenarios where CPS are structured as interconnected devices or networks, Graph Neural Networks (GNNs) offer a powerful method for detecting anomalies. GNNs excel at capturing both spatial and temporal dependencies between nodes (devices) and their relationships, making them particularly well-suited for Industrial IoT (IIoT) systems.

Graph Convolutional Networks (GCNs) and Graph Attention Networks (GATs) are commonly used in these settings to aggregate data from nodes and their neighbors, detecting point, contextual, and collective anomalies. GNNs enable the detection of complex patterns in IIoT environments, where device behavior may only be abnormal in relation to other devices or in specific contexts. By learning the graph structure, GNNs can identify anomalies in real-time, improving the resilience and security of industrial networks \cite{73}.

\subsubsection{Deep Learning for IoT and Network-Based Anomaly Detection}

Deep learning models are also extensively used in Internet of Things (IoT) and network-based anomaly detection, where real-time monitoring of network traffic is critical for identifying potential cyberattacks. Convolutional Neural Networks (CNNs) have been successfully applied to detect attacks like Distributed Denial of Service (DDoS) or ransomware in IoT environments. These models process large volumes of network traffic data, learning to distinguish normal traffic from anomalous patterns based on features like packet size and protocol types. Once trained, CNNs can continuously monitor network traffic, raising alerts when deviations from normal behavior are detected, ensuring robust security for IoT networks \cite{81}.

The proposed anomaly detection method for IoT networks involves a comprehensive, step-by-step process that leverages the capabilities of deep neural networks (DNN) to accurately identify malicious activity in network traffic. The process starts by capturing network traffic using tools like tcpdump or Wireshark, which intercept and log data flowing through the IoT environment. The traffic data is preprocessed and relevant features, such as IP addresses, packet lengths, and protocols, are extracted and normalized. A DNN model is then trained on the processed data, learning to classify traffic as benign or anomalous. This approach ensures accurate detection of cyber threats in IoT networks, including zero-day attacks and Distributed Denial of Service (DDoS) attacks \cite{80}.

\subsubsection{Long Short-Term Memory (LSTM) Neural Networks for Time-Series Anomaly Detection}

In the methodology, the anomaly detection process for CPS begins with understanding the normal behavior of the system through time-series data. Anomalies, or deviations from this behavior, are critical to detect as they often indicate faults or failures in the system. Given that many CPS have complex internal dynamics, traditional methods may fail to account for time-dependent anomalies. To address this, Long Short-Term Memory (LSTM) neural networks are employed due to their ability to model temporal sequences and capture long-term dependencies. LSTMs excel at learning from past data to make predictions about the future state of the system, allowing for more accurate anomaly detection in systems where current behavior depends on historical states. Once trained, the LSTM network is deployed to monitor CPS in real-time, comparing predicted outputs with actual system observations to identify deviations \cite{47}.

\subsubsection{Few-Shot Learning with Siamese Networks}

The anomaly detection method presented in \cite{58} uses a few-shot learning approach based on Siamese Convolutional Neural Networks (CNN), addressing the challenge of limited labeled data. This method is particularly effective for detecting anomalies in industrial environments where obtaining large labeled datasets of abnormal events is impractical. By focusing on relative-feature representation, the Siamese network identifies novel types of anomalies, even with a few labeled examples. The network calculates the distance between input samples and leverages this information for anomaly detection, making it highly adaptable for real-time industrial monitoring.

\subsubsection{ABATe: Real-Time Contextual Anomaly Detection}

ABATe is another methodology for detecting anomalies in CPS by leveraging neural networks to abstract system behavior into context vectors. These vectors represent the relationships between system states and transitions, enabling the model to detect both point anomalies and contextual anomalies in real-time. By clustering similar system states and calculating context vectors, ABATe can evaluate transitions between states and flag significant deviations from normal behavior \cite{49}. This approach is particularly suited for domain-independent CPS applications, ranging from automotive systems to sewage treatment plants.

\subsubsection{Zone Partitioning and Cross-Zone Neural Networks}

In an industrial CPS, zone partitioning is used to divide the system into multiple zones, each monitoring specific crucial states. Each zone has a neural network trained to predict the normal behavior of these states. The system detects anomalies by cross-referencing data between different zones. This multi-zone approach is highly resilient, as it allows the system to detect anomalies by comparing predictions across different zones, even if one zone is compromised \cite{59}.













\subsection{Machine Learning and Deep Learning Approaches}

In modern Cyber-Physical Systems (CPS) and Internet of Things (IoT) environments, leveraging both machine learning and deep learning approaches has proven to be an effective strategy for anomaly detection. By combining the strengths of these two methodologies, systems can detect a wide range of anomalies, from subtle data deviations to sophisticated cyber-attacks. This section explores hybrid approaches where machine learning and deep learning techniques work in tandem to ensure comprehensive and reliable anomaly detection.

\subsubsection{Semi-Supervised Approaches}

Semi-supervised learning, which leverages both labeled and unlabeled data, is one of the key methods for combining machine learning and deep learning. CPS-GUARD exemplifies this approach by using deep autoencoders trained solely on normal data. The autoencoders learn to reconstruct normal system behavior, minimizing reconstruction error (RE) for typical operations. During deployment, CPS-GUARD employs an outlier-aware thresholding technique, using isolation forests to account for rare but legitimate behaviors, ensuring a dynamic and adaptable RE threshold. Anomalous events—such as system malfunctions or cyber intrusions—are flagged when the RE exceeds this threshold. This method is particularly effective in dynamic CPS and IoT environments, where unseen attack patterns can be detected without needing labeled attack data \cite{86}.

\subsubsection{Integrating Machine Learning with Adversarial Robustness}

Another hybrid approach integrates machine learning anomaly detection with robustness against adversarial attacks, which is increasingly important in security-critical CPS. One example employs Artificial Neural Networks (ANN) trained on datasets like Bot-IoT and Modbus, capturing both normal and attack activities. To evaluate the robustness of the ANN models, adversarial samples are generated using the Fast Gradient Sign Method (FGSM), which introduces small perturbations in the data to trick the model into misclassification. The study highlights that adversarial attacks severely degrade the performance of deep learning models, exposing vulnerabilities in anomaly detection. To counter this, adversarial training is introduced, where the model is retrained using both clean and adversarial samples, significantly enhancing its ability to detect anomalous behavior even in the presence of adversarial manipulation \cite{67}.

\subsubsection{Combining Siamese Networks with Probabilistic Models}

In another innovative approach, deep learning models are combined with probabilistic techniques for anomaly detection in CPS. This method integrates a Siamese Convolutional Neural Network (SCNN) with Kalman Filtering (KF) and Gaussian Mixture Models (GMM) to enhance detection accuracy. The SCNN employs few-shot learning to detect anomalies with limited labeled data by measuring the Euclidean distance between feature vectors of normal and query data. To refine this detection, Kalman Filtering performs state estimation by tracking system dynamics over time, verifying whether detected anomalies are consistent with system trends. This hybrid system demonstrates high accuracy in reducing false positives, especially when detecting subtle or evolving anomalies in multi-domain CPS environments \cite{71}.

\subsubsection{Deep Learning and Machine Learning in IoT Data Streams}

Anomaly detection in IoT environments often requires both machine learning and deep learning to handle the high-dimensional, continuous nature of IoT data streams. These systems typically start with preprocessing steps, such as data normalization and noise reduction, followed by windowing techniques like sliding or fading windows to break the data into manageable segments. Once the data is prepared, models such as Support Vector Machines (SVM), Autoencoders, or Long Short-Term Memory (LSTM) networks are applied to detect anomalous patterns.

Supervised learning models, such as decision trees and SVMs, are effective when labeled data is available, while unsupervised techniques, like K-means clustering or Local Outlier Factor (LOF), work well in environments without labeled data. Deep learning models, especially Autoencoders, excel in reconstructing normal behavior and identifying anomalies based on reconstruction error, while LSTM models capture temporal dependencies to detect unexpected deviations in time-series data. By combining these approaches, IoT systems can monitor and detect anomalies in real-time, minimizing false positives and maintaining high detection accuracy \cite{79}.

\subsubsection{Systematic Machine Learning and Deep Learning Integration}

A systematic integration of machine learning and deep learning is applied to IoT environments through models such as Random Forest (RF) and Artificial Neural Networks (ANN). In a study using the Distributed Smart Space Orchestration System (DS2OS) dataset, various machine learning algorithms were trained to detect anomalies such as Denial of Service (DoS) attacks, Malicious Control, and Data Type Probing. Logistic Regression (LR), Support Vector Machines (SVM), and Decision Trees (DT) were evaluated alongside deep learning methods like ANN. Among the tested models, Random Forest emerged as the most effective, achieving an accuracy of 99.4\% in detecting anomalies. The study demonstrates how machine learning models can be highly effective in identifying both known and previously unseen anomalies in IoT environments, providing strong security measures \cite{76}.

The process typically begins with extensive data collection, followed by preprocessing steps like handling missing values and transforming categorical features into numerical formats. Once the data is ready, a variety of models are applied, and the best-performing model is selected based on metrics such as accuracy, precision, recall, and F1 score. Random Forest, in particular, demonstrated superior performance, with its ability to detect complex anomalies in real-world IoT environments, underscoring the effectiveness of combining traditional machine learning with deep learning methods in IoT anomaly detection \cite{76}.

\subsubsection{Hybrid Few-Shot Learning with Real-Time Anomaly Detection}

For scenarios where labeled data is scarce, hybrid models combining few-shot learning with machine learning are highly effective. In one such method, Siamese Neural Networks are used to detect anomalies with minimal labeled data. This approach reduces the reliance on large labeled datasets by calculating the similarity between new data and reference data. Kalman Filters further enhance accuracy by tracking system state changes over time, providing a probabilistic measure of anomaly likelihood. This hybrid system ensures robust real-time anomaly detection with minimal false positives in highly dynamic CPS environments \cite{71}.

\subsection{Mathematical Approaches}

Mathematical models play a pivotal role in anomaly detection for Cyber-Physical Systems (CPS), enabling the precise identification of abnormal patterns through formal, statistical, and probabilistic methods. These approaches allow for a structured understanding of system behavior, providing clear and interpretable methods for detecting anomalies in real time.

\subsubsection{Graph-Based Methods}

Graph-based approaches are effective in capturing the relationships between interconnected devices and system components. In the paper on **IoT Anomaly Detection via Device Interaction Graph**, the authors propose a Device Interaction Graph (DIG) to model interactions between IoT devices. Each device is represented as a node, and interactions are modeled as directed edges. The system uses Conditional Probability Tables (CPTs) to map normal interaction patterns between devices and detect anomalies by monitoring deviations from these patterns. Real-time events are compared to expected device behavior using the DIG, and abnormal behaviors are identified through significant deviations in the associated probabilities. For instance, if a light turns on without a presence sensor detecting any movement, the system flags this as a contextual anomaly. Additionally, DIG enables the detection of collective anomalies, where multiple abnormal events form a suspicious chain of interactions. This graph-based approach provides a strong foundation for monitoring complex IoT environments by focusing on the relationships between device behaviors\cite{74}.

\subsubsection{Bayesian Inference and Probabilistic Models}

Bayesian inference is a powerful tool for anomaly detection, offering a probabilistic framework to model system behavior under uncertainty. One application of Bayesian methods involves using a **damped harmonic oscillator model** to describe the normal behavior of mechanical systems. In this approach, Bayesian inference is employed to estimate unknown system parameters—such as frequency or damping—from observed data. By combining prior knowledge with real-time sensor data, the system generates a posterior probability distribution for these parameters. If the estimated parameters deviate significantly from their expected values, an anomaly is detected. This probabilistic approach allows for early detection of deviations with minimal data, making it ideal for real-time applications in CPS\cite{51}.

Similarly, **Bayesian networks** have been employed for anomaly detection in CPS by modeling causal relationships between system variables. In this method, physical and cyber data are aligned temporally, and the Bayesian network learns the normal interaction patterns between variables. During operation, anomalies are detected by calculating the probability of the observed state given the states of related variables. If the probability falls below a predefined threshold, the event is flagged as an anomaly. This approach enables the detection of both cyber and physical anomalies without requiring labeled data, making it suitable for unsupervised learning in complex CPS environments\cite{56}.

\subsubsection{Digital Twins and Statistical Methods}

Digital twins provide a virtual replica of physical systems, enabling real-time comparison between expected and actual system behavior. In one framework, the **Digital Twin-based CPS** continuously collects data from both the physical system and its digital twin, comparing real-time sensor data to predicted values generated by the twin. A Gaussian Mixture Model (GMM) captures the multimodal behavior of the system, detecting discrepancies between the physical and digital system outputs. If these discrepancies persist, they are classified as anomalies using conformal prediction methods and Martingale tests. By integrating statistical methods with digital twins, this approach offers a comprehensive mechanism for real-time anomaly detection and classification, allowing for effective responses to system faults or sensor inaccuracies\cite{61}.

\subsubsection{Kalman Filters and Ensemble Methods}

Kalman filters and ensemble methods are well-suited for tracking the dynamic state of CPS, particularly in detecting anomalies in control systems. One approach employs the **Ensemble Kalman Filter (EnKF)** in conjunction with GPU parallel processing to monitor large-scale power grids. The EnKF generates an ensemble of possible system states using historical data, predicting the system's behavior in real time. Anomalies are detected by comparing predicted states with real-time measurements, and significant deviations trigger further analysis. The system differentiates between random errors and False Data Injection (FDI) attacks by applying statistical tests such as the Chi-Square test. This method is highly scalable and provides accurate, real-time detection of cyber-attacks in complex systems\cite{72}.

A multi-sensor fusion strategy also leverages Kalman filters and optimization techniques for anomaly detection in CPS. Multiple sensors monitor system states and generate local estimates, which are then fused at a central monitoring center using convex optimization. By aggregating sensor data, the system improves the accuracy and speed of anomaly detection, especially in bandwidth-limited environments. This multi-sensor approach provides a robust defense against FDI attacks by enabling early detection and mitigation\cite{53}.

\subsubsection{Causality and Information-Theoretic Approaches}

Transfer entropy, an information-theoretic measure, is used to capture causal relationships between system components in CPS. In one method, transfer entropy quantifies the information flow between sensor measurements, identifying stable cause-effect relationships during normal operation. When an attack occurs—such as a Denial of Service (DoS) or data injection attack—these relationships are disrupted, resulting in a significant deviation in transfer entropy values. This method does not require prior knowledge of specific attacks, making it a versatile and robust solution for detecting anomalies across different types of cyber-attacks\cite{69}.

\subsubsection{Formal Methods}

Signal Temporal Logic (STL) provides a formal method for specifying and detecting anomalies based on temporal constraints in CPS. In this approach, the system's normal behavior is modeled by inferring an STL formula from data collected during normal operations. This formula describes the expected relationships between system variables over time, capturing temporal dependencies such as "if a temperature exceeds 70°C for more than 10 seconds, an alarm must be triggered." The system monitors real-time behavior by comparing it to the STL formula, using a robustness metric to measure how closely the current behavior matches the expected behavior. If the robustness metric turns negative, indicating a violation of the STL formula, the system flags the anomaly. This formal method is particularly useful for detecting time-bound anomalies in critical infrastructure systems like train braking systems or power grids\cite{19}.

\subsubsection{Event-Based Anomaly Detection}

Event-based anomaly detection models the relationships between physical events and system control logic. In the **Orpheus framework**, real-time CPS events are monitored and mapped to control program state transitions using an event-aware finite-state automaton (eFSA). The system ensures that each state transition corresponds to a valid physical event, such as turning on a valve when water levels reach a threshold. If the program attempts to execute a transition without the corresponding event, Orpheus flags the anomaly. This method effectively detects data-oriented attacks by ensuring that program control flow remains consistent with real-world events, providing a robust defense against subtle cyber-attacks\cite{65}.




\subsection{Hybrid Approaches}

Hybrid approaches to anomaly detection in Cyber-Physical Systems (CPS) and IoT environments combine multiple methodologies to improve detection accuracy and robustness. These approaches integrate techniques from machine learning, statistical analysis, and threshold-based methods to address the diverse and dynamic nature of CPS, where both cyber and physical anomalies may arise.

\subsubsection{CPS-GUARD: Hybrid Machine Learning-Based Anomaly Detection}

CPS-GUARD uses a hybrid anomaly detection approach, combining deep learning with a threshold-based mechanism for detecting cyber-attacks and physical anomalies in CPS and IoT environments. The system leverages deep autoencoders trained on normal data to identify typical system behavior, creating a baseline for comparison. When new data is introduced, the system calculates a reconstruction error (RE) by comparing the original data with its reconstruction. If the RE exceeds a predefined threshold, the data is flagged as anomalous. This method allows CPS-GUARD to detect both known and unknown attacks by combining a machine learning-based baseline model with a dynamic thresholding mechanism for improved accuracy\cite{86}.

\subsubsection{Ensemble Learning for Hybrid Anomaly Detection}

A robust hybrid method developed for CPS uses Ensemble Learning (EL) to combine signature-based, threshold-based, and behavior-based detection mechanisms. The signature-based component quickly identifies known threats, while threshold-based detection monitors physical parameters like temperature and pressure, triggering alerts when values exceed safe limits. The EL approach uses various machine learning algorithms—such as Logistic Regression, Support Vector Machines (SVM), and K-Nearest Neighbors (KNN)—to detect more subtle anomalies in system behavior. By combining the outputs of these models, EL reduces both false positives and false negatives, enhancing overall detection accuracy in critical CPS environments\cite{70}.

\subsubsection{Cyber-Physical Anomalies Detection Using Bayesian Networks}

This method integrates cyber and physical data to detect anomalies using a hybrid of unsupervised behavior-based detectors and Bayesian Networks. The system collects data from both the network (cyber) and sensors (physical), analyzing them independently before combining their outputs. The Bayesian Network consolidates these results to assess the likelihood of different types of anomalies, whether cyber-attacks, physical faults, or a combination of both. This approach is particularly effective for localizing anomalies within specific subsystems, allowing for targeted interventions in complex industrial environments\cite{48}.

\subsubsection{IoT Anomaly Detection Using Hybrid Approaches}

In IoT environments, hybrid anomaly detection methods combine statistical models with machine learning techniques to handle the complexity and volume of data generated by interconnected devices. For instance, models like Seasonal Autoregressive Integrated Moving Average (SARIMA) capture short-term patterns, while Long Short-Term Memory (LSTM) networks model long-term dependencies. By combining these models, the system dynamically adjusts to both immediate and prolonged deviations from normal behavior. This integration allows for the detection of both cyber-attacks and physical system faults, reducing the likelihood of false alarms in real-time IoT networks\cite{60}.

\subsubsection{Graph-Based and Transformer Hybrid Models for IoT Systems}

Graph Learning with Transformer Anomaly detection (GTA) is a hybrid framework designed for detecting anomalies in IoT environments. It first constructs a graph that captures the dependencies between sensors, followed by the application of a transformer model to forecast normal behavior. By combining graph-based learning with transformers, GTA is highly effective in identifying both subtle and extreme anomalies in IoT systems, especially in scenarios where sensor dependencies are complex and dynamic. This method outperforms traditional models by leveraging both temporal and spatial relationships within the data\cite{75}.

\subsubsection{Neural System Identification with Bayesian Filtering (NSIBF)}

NSIBF is a hybrid method that integrates Neural Networks and Bayesian Filtering for real-time anomaly detection in CPS. The system first learns normal behavior using a neural network that predicts sensor readings based on historical data. Bayesian Filtering is then applied to track deviations from expected behavior by comparing actual sensor readings with predicted values. By combining predictive modeling with uncertainty tracking, NSIBF effectively handles noisy data and provides early detection of anomalies. This method has been shown to outperform traditional approaches in scenarios with high levels of system noise and uncertainty\cite{45}.

\subsubsection{Hybrid Detection in Cyber-Physical Systems}

In industrial CPS, hybrid detection methods combine both machine learning and rule-based models to address diverse anomaly sources. A hybrid approach using Gaussian Mixture Models (GMM) and Kalman Filters (KF) models the distribution of normal system behavior and dynamically adjusts thresholds based on real-time data. This combination allows for precise anomaly detection, even in complex environments where subtle deviations might indicate a potential failure or cyber-attack. By merging statistical models with real-time state estimation, this hybrid method ensures high accuracy in anomaly detection and low false-positive rates\cite{17}.



\subsection{Invariant-Based Anomaly Detection}

\begin{definition}
Invariant rules are defined as physical conditions that must be satisfied for any given state of an Industrial Control System (ICS). These rules, derived from operational data, describe the expected relationships between sensor readings and actuator states. A deviation from these conditions signals an anomaly, which could indicate system faults or cyber-attacks. Invariants represent stable relationships or dependencies between different components of a system that remain consistent under normal operating conditions. Monitoring these invariants is crucial for detecting anomalies when these relationships break\cite{21, 23}.
\end{definition}

The proposed framework introduces a systematic approach to automatically generating invariant rules from the operational data logs of Industrial Control Systems (ICS). These invariants describe normal operational conditions by specifying the expected relationships between sensors and actuators. Traditionally, these rules are manually defined based on system design, a time-consuming and error-prone process, particularly in complex systems. Our framework automates the rule generation using machine learning and data mining techniques, allowing for real-time anomaly detection in evolving industrial environments.

\subsubsection{Invariant Generation from Data Logs}

The process begins with data collection, where sensor readings and actuator states are recorded during normal system operation. Predicates, or conditions, are generated from this data, representing specific relationships within the system. Continuous sensor readings are modeled using techniques like Gaussian Mixture Models (GMMs) to capture hidden control states, while events (e.g., actuator changes) are used to define triggers. These predicates are combined into itemsets, which represent the system's operational state at a given time.

Association rule mining is then applied to these itemsets to discover statistically significant invariant rules. These rules describe relationships that hold true under normal conditions and are optimized to balance rule generation with false-positive reduction. By validating these rules against real-time data, deviations from the established invariants are flagged as anomalies, indicating potential system malfunctions or cyber intrusions.

\subsubsection{Dynamic Invariant Detection with Time Dependencies: ARTINALI}

ARTINALI, a dynamic invariant detection approach, extends the traditional invariant concept by incorporating time as a fundamental dimension, alongside data and events. This approach mines three types of invariants: Data per Event (D|E), Event per Time (E|T), and Data per Time (D|T). By capturing not just the relationships between data and events but also the temporal sequences of events, ARTINALI provides a more comprehensive model of system behavior. For example, an E|T invariant might specify that a sensor reading must be processed within a fixed time window, such as 5-10 seconds. Any deviation from these learned invariants triggers an alert in the Intrusion Detection System (IDS), enabling timely anomaly detection\cite{22}.

\subsubsection{Combining Data-Driven and Model-Based Approaches: illiad Framework}

The illiad framework combines data-driven methods like Autoregressive Models with Exogenous Inputs (ARX) and Kalman Filters to detect invariants in complex CPS environments. It builds an invariant graph, where system components are nodes, and inferred stable relationships (invariants) between them form the edges. Real-time monitoring of these relationships allows illiad to detect anomalies when these relationships break. For example, in a microgrid, if the relationship between solar power output, load consumption, and battery charge deviates from its normal pattern, the system will flag a potential fault or attack\cite{23}.

\subsubsection{Learning and Validating Invariants Using Machine Learning}

In a more automated approach, machine learning techniques like Support Vector Machines (SVM) are employed to learn invariants from both normal and abnormal system traces. Abnormal traces are generated through mutation testing, where simulated faults or cyber-attacks are introduced into the system. These learned invariants, validated through statistical model checking, enable real-time detection of deviations from normal behavior. For instance, if the water level in a tank falls outside an expected range due to a pump malfunction, the system flags the deviation, prompting further investigation\cite{24}.

\subsubsection{Design-to-Invariant Approach in ICS}

In the Design-to-Invariant (D2I) approach, invariants are derived from hybrid automata models that represent both discrete and continuous states of ICS components. These invariants are then coded directly into the Programmable Logic Controllers (PLCs) that control the system, enabling real-time monitoring and anomaly detection. Invariants ensure that physical processes, like fluid levels in a tank, behave as expected. Any violation of these process-based invariants—such as a valve failing to open or close correctly—signals a potential attack or system fault\cite{26}.

\subsubsection{Unified Invariants for Cyber-Physical System Stability}

In complex CPS, integrating cyber, physical, and network invariants is critical to maintaining overall system stability. Physical invariants ensure stability through Lyapunov-like functions, while cyber invariants guarantee computational correctness and prevent interference between subsystems. By combining these invariants into a unified framework, the system ensures both operational stability and security, even in environments with switching dynamics or distributed control\cite{27}.

\subsubsection{Association Rule Mining in CPS Testbeds}

Using association rule mining techniques, invariants can be derived from large-scale testbed data, such as the Secure Water Treatment (SWaT) system. By transforming continuous sensor data into categorical ranges (e.g., high, low), frequent itemsets are identified, representing stable relationships between sensors and actuators. For example, when a motorized valve is open, the flow sensor should detect a high flow rate. These automatically discovered invariants provide a basis for anomaly detection, highlighting deviations from expected behavior, such as when flow remains constant despite a valve opening\cite{25}.

\subsubsection{Real-Time Anomaly Detection Using Learned Invariants}

In a real-time operational context, learned invariants are continuously validated as the system runs. Any deviation from these invariants, such as sensor readings that do not match expected trends, triggers an alert. For example, in a power grid, if communication delays between components exceed a threshold or if power flows deviate from the predicted balance, the system flags these anomalies as potential security threats, ensuring fast intervention to prevent cascading failures\cite{28,29}.



\subsection{Others}

In this study, anomaly detection in Cyber-Physical Production Systems (CPPS) is performed using a bio-inspired host intrusion detection system (A-HIDS) based on the Incremental Dendritic Cell Algorithm (iDCA). The iDCA mimics the human immune system by classifying network traffic and detecting abnormal patterns indicative of attacks. The system captures network traffic using the OPC Unified Architecture (OPC UA) protocol and categorizes it into safe signals (Ss), danger signals (Ds), and pathogenic associated molecular patterns (PAMPs), representing varying levels of risk. The dendritic cells process these antigens (Ag) and classify traffic as normal or abnormal based on the predominance of safe or danger signals. Anomalous traffic is flagged based on a mature context antigen value (mcav), and incidents are logged, blocked, or escalated in real-time to protect CPPS operations from cyber threats\cite{64}.

The anomaly detection process described in another approach monitors CPPS in real time by establishing a baseline of normal behavior through continuous data collection. The ÆCID tool uses a whitelisting approach, learning expected system patterns during a training phase and flagging deviations as anomalies. For example, in a semiconductor manufacturing cooling system, if malicious PLC logic prevents a cooling valve from opening when the temperature exceeds 90°C, the anomaly detection system flags this abnormal behavior and triggers a security alert. Self-adaptive mechanisms are then activated, such as resetting the PLC or activating backup systems to ensure operational safety\cite{55}.

In the context of big data techniques for real-time industrial environments, anomaly detection begins with collecting data from sensors that monitor parameters like spindle speed and power consumption. Clustering algorithms are used to condense this vast data into summaries, creating baseline models of normal operations. New data is compared to these baselines using relevance evaluation, and deviations trigger alerts based on predefined thresholds. This approach helps focus attention on significant anomalies, ensuring early detection of machine faults and minimizing costly downtime\cite{62}.

For power grids, anomaly detection integrates Intrusion Detection Systems (IDS) and Anomaly Detection Systems (ADS) to monitor both network and physical components. Network-Based IDS (NIDS) and Host-Based IDS (HIDS) scan data packets and device behaviors, identifying suspicious patterns. The ADS correlates data to detect coordinated anomalies, such as falsified GOOSE packets that trip circuit breakers without cause. Automated responses are triggered to isolate compromised devices and prevent cascading failures\cite{93}.

In Cyber-Physical Production Systems (CPPS), anomalies are detected by leveraging the timing behavior of system events. Timed Automata models normal event sequences and timing distributions, comparing real-time performance to expected patterns. Deviations from expected timing windows signal anomalies, such as delays caused by worn equipment. Statistical methods define acceptable timing ranges using means and standard deviations, balancing sensitivity to anomalies while reducing false positives. This approach has been tested in real-world production systems to detect mechanical wear and process delays\cite{94}.

The Abnormal Traffic-indexed State Estimation (ATSE) method integrates both cyber and physical components of the Smart Grid for enhanced detection accuracy. The system first detects cyber anomalies using IDS, assigning a Cyber Impact Factor (CIF) to quantify the threat level. The system then recalibrates physical state estimation by down-weighting suspicious data points, ensuring manipulated data has less influence on grid state estimation. A Chi-square test compares the estimated and actual states, flagging anomalies when residuals exceed thresholds. This method improves detection of both cyber and physical attack vectors\cite{95}.
\end{comment}









%$$*/*/*/*//*/*/*/*/*/*/*/*/*/*/*/*/*/*/*/*/*/*/$$


\begin{comment}


\subsection{Machine Learning}
In recent years, many methods have been proposed for anomaly detection in Cyber-Physical Systems (CPS) using machine learning. These methods generally follow a common process, starting with data collection from various system components, such as sensors, network traffic, or system logs. For example, in an energy grid, data from transformers, power lines, and meters is collected to identify potential issues. Once the data is collected, it goes through preprocessing, which includes cleaning the data, handling missing values, removing outliers, and standardizing it to make it suitable for machine learning models. Techniques like Principal Component Analysis (PCA) can also be used to reduce the number of variables and improve the model's efficiency.

After preprocessing, feature extraction is done to create useful inputs for the machine learning model. In some systems, features like power consumption patterns or voltage changes are extracted to help the model distinguish between normal and abnormal behavior. The next step is choosing the appropriate machine learning model. If labeled data is available, supervised learning methods such as Support Vector Machines (SVM) or Random Forest are used. In cases where labeled data is limited, unsupervised learning methods like K-Means or Gaussian Mixture Models (GMM) can be applied to detect anomalies without predefined labels. Semi-supervised methods, like Variational Autoencoders (VAE), are also useful when a small amount of labeled data is available, as they combine the benefits of both supervised and unsupervised learning.

Once the model is trained, its performance is evaluated using metrics like precision, recall, and accuracy to ensure it can detect anomalies accurately and avoid false positives. After tuning, the model is deployed for real-time monitoring of the CPS, alerting operators to any detected anomalies. This process significantly enhances the system's ability to detect problems early and prevent potential failures or cyber-attacks, especially in critical systems like energy grids or healthcare networks \cite{44, 89, 90}.

A framework known as Fuzzy Controller-Empowered Autoencoder (FCAF) is another method that combines machine learning and fuzzy logic to detect anomalies in Industrial Control Systems (ICS). The FCAF collects and preprocesses multivariate time-series data from ICS sensors. A Long Short-Term Memory (LSTM)-based autoencoder is trained on normal data to learn system behavior. During detection, the model flags abnormal inputs that the autoencoder cannot properly reconstruct, leading to a high reconstruction error. To refine this process, a fuzzy logic system applies if-then rules to the reconstruction error values, classifying the anomaly as high, medium, or low priority based on its severity. This dual-layer approach improves both detection accuracy and the reliability of the system \cite{89}.

The Causality-Guided Counterfactual Debiasing Framework (CDF) tackles the challenge of bias in machine learning models trained on imbalanced data, where normal data far outweighs anomalies. This imbalance often results in models failing to detect rare but critical anomalies. CDF uses causal graphs to understand how bias affects model predictions. It generates counterfactual scenarios to simulate how the model would behave without the influence of bias and then adjusts the predictions accordingly, making the model more effective in detecting anomalies \cite{90}.

In some cases, domain knowledge is integrated into machine learning models to improve the detection of unsafe actions in CPS. For example, in systems like Artificial Pancreas Systems (APS), a special "semantic loss" function is used to ensure that control commands, such as insulin injections, do not lead to dangerous conditions like hypoglycemia. Domain experts define safety rules that the model learns during training, and this knowledge helps the model avoid making unsafe decisions, improving both reliability and robustness in real-time applications \cite{63}.

In Internet of Things (IoT) networks, anomaly detection starts with gathering data from devices like sensors and cameras. The data is cleaned and key features, such as network traffic volume and device activity, are extracted. Machine learning models, like Random Forests or unsupervised methods, are then used to detect unusual patterns in the data. Once trained, these models monitor the IoT network in real-time, flagging anomalies like sudden spikes in traffic or suspicious device behavior, which may indicate security threats such as denial-of-service (DoS) attacks \cite{84}.

Anomaly detection in smart grids often involves analyzing data from Phasor Measurement Units (PMUs) using machine learning models like logistic regression. PMUs provide high-frequency data on voltage, current, and frequency, which is crucial for monitoring the grid's health. The model is trained to identify abnormal values and alert operators to potential faults in real-time, helping prevent power outages and ensuring the grid's stability \cite{66}.

In smart city IoT networks, the AD-IoT framework collects and preprocesses network traffic data, such as packet sizes and connection times. Using the Random Forest algorithm, a machine learning model is built to detect abnormal patterns that may signal cyberattacks. The model is deployed in a fog computing system, allowing it to quickly analyze data and flag suspicious activity. This early detection helps protect IoT networks from attacks like distributed denial-of-service (DDoS) before they cause widespread damage \cite{78}.

Anomaly detection in CyberManufacturing Systems (CMS) can involve using machine learning to monitor physical data, like images or acoustic signals from devices like 3D printers. The data is processed, and important features are extracted to train machine learning models to recognize normal system behavior. Once trained, these models detect deviations from expected patterns, such as defects in 3D-printed parts or unusual machine vibrations, flagging them as anomalies. This real-time detection helps prevent defects and maintain the quality of manufacturing processes \cite{52}.

In another approach, machine learning is used to detect attacks in CPS, focusing on specific systems like water treatment facilities. Data from the system's sensors is collected and labeled to represent normal and attack conditions. Several classifiers, including Support Vector Machines (SVM) and Random Forests, are trained to distinguish between normal operations and attacks. These models are then deployed to monitor the system in real-time, flagging suspicious behavior and identifying the specific type of attack, helping operators respond quickly and effectively \cite{92}.
\subsection{Deep Learning}

The MTS-DVGAN model detects anomalies in Cyber-Physical Systems (CPS) by leveraging deep generative models and contrastive learning to analyze multivariate time series data. The process begins by preprocessing data collected from sensors and actuators within the CPS, dividing it into smaller time-based subsequences using a sliding window technique to capture both current and past behaviors. The data is then normalized for efficiency and accuracy. The model consists of two modules: the main module, built on an LSTM-based encoder, learns latent representations of the time series, and the augmented module enhances anomaly detection through contrastive learning. This learning technique helps push abnormal samples away from normal ones in the latent space, making it easier to distinguish between normal and anomalous behaviors. The detection phase combines reconstruction loss and discrimination loss to calculate an anomaly score, which flags anomalies when it exceeds a predefined threshold. Contrastive constraints in the augmented module further improve the model's performance by stabilizing the generator and ensuring accurate reconstructions of data \cite{87}.

The methodology for decentralized real-time anomaly detection in Cyber-Physical Production Systems (CPPS) begins with high-frequency data collection from various CPS within the production environment. Irrelevant features are filtered out, and relevant features like torque and motor speed are identified. 1D Convolutional Autoencoders (1D-ConvAE) are used to model normal behavior, with anomalies detected as deviations from learned normal patterns. Each CPS is equipped with its own anomaly detection model, allowing for decentralized monitoring and reducing the need for centralized systems. This decentralized setup ensures real-time anomaly detection and minimizes system downtime \cite{50}.

Another deep learning approach focuses on enhancing the security of Intrusion Detection Systems (IDS) through multipath neural networks (NN). These NNs continuously monitor the performance of autoencoders to ensure they function correctly. The system uses the reconstruction error from autoencoders to detect anomalies and spoofing attacks. The Wilcoxon-Mann-Whitney (WMW) test is used to identify significant changes in reconstruction error distributions, flagging anomalies and potential attacks. This approach provides real-time monitoring and detection, making the system resilient against both known and unknown threats \cite{88}.

The Adaptive-Correlation-Aware Unsupervised Deep Learning (ACUDL) model uses a dynamic graph update mechanism to refine the correlations between data points over time, ensuring that the model can detect both individual and relational anomalies. The dual-autoencoder (D-AE) structure captures non-correlated and correlated features. Anomalies are detected by comparing the original and reconstructed data, with high reconstruction errors signaling potential anomalies. This probabilistic approach enables robust anomaly detection in dynamic CPS environments \cite{91}.

The RmsAnomaly model addresses anomaly detection in CPS by capturing both temporal dependencies and inter-sensor correlations. The model builds signature matrices to reflect sensor correlations and uses multi-scale windows to track short- and long-term trends. A convolutional autoencoder then reconstructs the data, and anomalies are detected based on the difference between the original and reconstructed data. A novel threshold-setting mechanism automatically determines the optimal threshold, ensuring reliable detection without manual tuning \cite{46}.

In another method, anomaly detection in CPS is achieved through a combination of Bidirectional Gated Recurrent Units (BiGRU) and Variational Autoencoders (VAE). The model is trained on normal data, capturing forward and backward temporal dependencies, and reconstructs input data. Anomalies are detected when the reconstruction error exceeds a predetermined threshold. The sliding window mechanism ensures continuous real-time monitoring. This method, tested on the SWaT dataset, demonstrated high accuracy in detecting subtle system deviations \cite{57}.

The ATTAIN system integrates a digital twin model with Generative Adversarial Networks (GANs) for real-time anomaly detection in CPS. The digital twin continuously compares real-time system data with predicted values to flag anomalies. GANs generate synthetic data to improve the system's ability to detect both real and simulated attack scenarios. This layered approach allows for robust anomaly detection in real-time, ensuring that operators can intervene before significant damage occurs \cite{68}.

For IoT networks, anomaly detection uses deep neural networks (DNN) to identify malicious activity in network traffic. After capturing and preprocessing traffic data, the DNN is trained to classify traffic as normal or anomalous. Mutual information is used for feature selection to improve accuracy and reduce the false alarm rate. The DNN then continuously monitors the network in real-time, flagging potential threats like zero-day attacks or DDoS \cite{80}.

In another IoT approach, convolutional neural networks (CNN) are used to detect anomalies in network traffic. After data preprocessing, features are extracted and normalized for efficient model training. The CNN architecture includes multiple layers to classify traffic, detecting anomalies such as DDoS or ransomware attacks. Transfer learning further improves the model's performance by fine-tuning pre-trained models. The CNN continuously monitors traffic in real-time, ensuring robust detection of both known and novel threats \cite{81}.

Anomaly detection in Industrial Internet of Things (IIoT) systems leverages Graph Neural Networks (GNNs) to model the relationships between IIoT devices, represented as nodes in a graph. GNNs aggregate data from neighboring nodes, capturing spatial and temporal relationships. The model detects point, contextual, and collective anomalies by analyzing deviations in node and edge behavior. This method provides accurate and interpretable anomaly detection in complex, interconnected IIoT environments \cite{73}.

Finally, in another deep learning-based method for CPS, Long Short-Term Memory (LSTM) networks are used to capture temporal dependencies and predict future system behavior. Anomalies are detected when the predicted values deviate from actual values by more than a predefined threshold. LSTM networks enable early detection of system faults or attacks, even in cases where anomalies evolve gradually over time \cite{47}.

The ABATe methodology combines offline learning with real-time monitoring for anomaly detection in CPS. In the offline phase, state vectors representing system snapshots are clustered, and context vectors are generated using neural networks to capture normal system transitions. In the online phase, current data is compared to learned normal behavior. Anomalies are flagged when current states deviate significantly from expected transitions. This method ensures accurate detection of both point and contextual anomalies across various domains \cite{49}.

In the multi-zone approach for Industrial CPS, the system is divided into zones, each monitoring critical variables. A neural network predicts expected behavior for each zone, and anomalies are detected by cross-referencing data between zones using correlation and error metrics. This method enhances real-time anomaly detection and ensures system resilience against sensor tampering and cyber-attacks \cite{59}.

\subsection{Machine Learning and Deep Learning}

CPS-GUARD detects anomalies in cyber-physical systems (CPS) and IoT environments by leveraging a semi-supervised approach that uses deep autoencoders. These autoencoders are trained on normal data, learning to accurately reconstruct normal behavior. When new data is fed into the system, the autoencoder calculates a reconstruction error (RE). If the RE is below a predefined threshold, the data is classified as normal. However, if the RE exceeds the threshold, it is flagged as anomalous, indicating potential system malfunctions or intrusions. CPS-GUARD's dynamic thresholding mechanism, which uses isolation forests to account for uncommon but legitimate behavior, helps reduce false positives. For instance, in a water treatment system, if abnormal sensor readings (such as a combination of high water levels and closed valves) are detected, CPS-GUARD would flag them as anomalies, alerting system operators. This approach allows CPS-GUARD to detect previously unseen attacks without requiring labeled attack data, making it highly adaptable to real-world applications \cite{86}.

The process of anomaly detection in CPS typically involves several steps, starting with data collection from real-world datasets that capture both normal and attack behaviors. In one study, the authors used the Bot-IoT and Modbus datasets to train an Artificial Neural Network (ANN) for detecting attacks such as Distributed Denial of Service (DDoS). After preprocessing the data, including balancing class distributions with techniques like SMOTE, the ANN was trained to classify network traffic as either normal or attack. The model was tested on both clean and adversarial samples, revealing that adversarial attacks significantly reduced its performance. To improve robustness, the authors implemented adversarial training, which involved retraining the model using a mixture of clean and adversarial samples. This method improved the model's ability to resist adversarial attacks and accurately detect anomalies \cite{67}.

Another research approach integrates a Siamese Convolutional Neural Network (SCNN) with Kalman Filtering (KF) and Gaussian Mixture Models (GMM) for detecting anomalies in CPS. The process starts with the acquisition of multi-domain data from the physical, network, and application layers of the CPS. This data is preprocessed using GMM to transform it into a more manageable format. The SCNN then calculates the Euclidean distance between reference and query sets of data to identify potential anomalies. Kalman Filtering refines these results by tracking the system's behavior over time, verifying whether flagged anomalies are genuine deviations. This dual approach reduces false positives and enhances the system's overall accuracy \cite{71}.

Anomaly detection in IoT data streams follows a multi-step process, starting with data collection from sensors generating continuous data streams. After preprocessing the data, techniques like Support Vector Machines (SVM), K-means clustering, and deep learning models like Autoencoders and Long Short-Term Memory (LSTM) networks are applied to detect anomalies. The models generate an anomaly score for each data point, flagging points that exceed a predefined threshold. Performance is evaluated using metrics such as accuracy, precision, recall, and the F1-score. This approach ensures efficient real-time anomaly detection, adapting to evolving data patterns in IoT systems \cite{79}.

In another study, anomalies in IoT systems were detected using machine learning techniques, beginning with the collection and preprocessing of data from a virtual IoT environment. Several machine learning models, including Logistic Regression, Support Vector Machine (SVM), Decision Tree (DT), Random Forest (RF), and Artificial Neural Networks (ANN), were trained to classify instances as normal or anomalous. The Random Forest model emerged as the most effective, achieving an accuracy of 99.4\%. The model successfully detected critical attacks, such as Denial of Service and Malicious Control, while maintaining a low rate of misclassification. By detecting anomalies in IoT behaviors, the system enhances the security and reliability of IoT infrastructures \cite{76}.

\subsection{Mathematics Approaches}

One approach for detecting anomalies in IoT systems uses a Device Interaction Graph (DIG). In this method, devices are represented as nodes, and interactions between them are shown as edges. The system builds a graph based on historical data, which captures normal interactions between devices. Each interaction is assigned a probability table that shows the likelihood of certain actions based on device states. When real-time data comes in, the system compares it with expected behavior. If an interaction differs significantly from what is expected, it is flagged as an anomaly. The system can also detect patterns of unusual events happening in a sequence, which may indicate a more complex problem. This method helps identify both individual and group anomalies in IoT environments \cite{74}.

Another method uses Bayesian inference to detect anomalies in systems by estimating important parameters based on sensor data. For example, a motor's normal behavior might be modeled with parameters like frequency or damping. Using Bayesian inference, the system updates its understanding of these parameters by combining prior knowledge with new sensor data. If the current parameters deviate too much from the expected values, the system flags an anomaly. This approach works well even with noisy data and can detect problems early \cite{51}.

Anomaly detection in systems with digital twins compares real-time data from the physical system with predictions from the digital twin. The system uses statistical models to detect differences between the real and expected behavior. If these differences persist, it flags an anomaly. After identifying an issue, the system classifies the problem to figure out if it is caused by a sensor fault, a problem in the physical system, or a mismatch between the real system and the digital twin model. This helps ensure the system runs smoothly by catching problems before they cause major issues \cite{61}.

In some systems, anomalies are detected using the Ensemble Kalman Filter (EnKF), which predicts future system states based on historical data. Real-time data is compared to these predictions. If the difference between the predicted and actual states is too large, the system flags an anomaly. This method works well for detecting issues in large-scale systems, such as power grids, and can handle both random errors and targeted cyber-attacks. When an anomaly is detected, the system adjusts its predictions to correct any faulty data and stabilize the system \cite{72}.

Multi-sensor fusion is another approach for detecting anomalies in systems like power grids. Multiple sensors monitor the system and send data to a central system. The central system combines the data from all sensors and calculates if the system's behavior is normal. If the combined data indicates something unusual, the system triggers an alert. This method improves detection speed and accuracy by using data from multiple sensors, making it more reliable than single-sensor methods \cite{53}.

Another approach models system behavior by mapping physical events to control actions. The system first learns what normal behavior looks like by observing how different physical events affect the control program. During real-time monitoring, the system checks if the current actions match the expected behaviors. If the system tries to perform an action without the required physical event happening, it is flagged as an anomaly. This method helps detect subtle attacks where internal data is manipulated without directly altering the control flow \cite{65}.

Transfer entropy is used to measure how much one system signal predicts another. Under normal conditions, the relationships between signals are stable. When an attack occurs, such as a replay or data injection attack, these relationships are disrupted, causing a drop in transfer entropy. The system monitors these changes in real-time and flags significant deviations as anomalies. This method works across different types of attacks because it focuses on the disruption of normal information flow, rather than relying on specific attack patterns \cite{69}.

Bayesian networks can also be used to detect anomalies by modeling the relationships between different parts of a system. The system learns the normal behavior by capturing how variables (such as sensor readings and actions) influence each other. During real-time monitoring, the system calculates the likelihood of the current state based on other related variables. If this likelihood is too low, the system flags the state as anomalous. This method is effective for detecting new types of anomalies and does not require pre-labeled data for training \cite{56}.

Another approach uses Signal Temporal Logic (STL) to model how a system should behave over time. The system first learns a formula that describes the normal timing and behavior of key variables. During real-time monitoring, the system continuously compares current data to this formula. If the system's behavior deviates from the expected pattern, it is flagged as an anomaly. This approach provides a clear and interpretable way to understand how the system should operate and is particularly useful for detecting time-dependent anomalies \cite{19}.


\subsection{Hybrid Approaches}

Anomaly detection in Cyber-Physical Systems (CPS) is an important process used to find unusual behaviors that could signal both physical problems and cyber threats. The process starts by collecting data from sensors and actuators that monitor things like temperature, pressure, vibration, and network activity. For example, in a smart greenhouse, sensors might track temperature and soil moisture while actuators control irrigation and ventilation. This data reflects how the system normally operates under different conditions.

After the data is collected, it needs to be preprocessed. This step involves cleaning the data, scaling it, and normalizing it. For example, temperature data might need to be scaled so that it doesn't overshadow other data types. Key features, like typical temperature and soil moisture ranges, are identified to help with anomaly detection. This makes the data easier to work with when training the models.

Next, the data is used to train machine learning models. A common choice is one-class classification models, such as K-Nearest Neighbors (KNN) or Support Vector Machines (SVM), which are trained using only normal data. For instance, a model trained on normal data in a greenhouse would learn that soil moisture should stay within a specific range. When new data is compared to the model, any behavior outside the learned normal range is flagged as an anomaly. This method is effective because anomalies are rare, and it's hard to gather examples of abnormal behavior.

In real-time operations, the system checks incoming data against the trained model. It uses methods like threshold detection (for physical limits) or signature detection (for known cyber threats). For example, if the temperature in the greenhouse rises above 40°C or soil moisture drops too low, an anomaly alert would be triggered. Machine learning models can also detect behavioral anomalies that don't match known patterns.

Once an anomaly is found, it is classified as either benign (like a sensor malfunction) or malicious (like a cyber attack). For example, a sensor malfunction might cause the greenhouse temperature to spike to 45°C, which is considered benign but still needs attention. However, abnormal network traffic would suggest a cyber threat that needs immediate action.

The last step is responding to the anomaly. For physical problems, the system may trigger actions like adjusting the irrigation system in the greenhouse. For cyber anomalies, it might block suspicious traffic or alert operators. This layered hybrid approach ensures that both predictable physical problems and unpredictable cyber threats are detected and managed quickly.

This hybrid approach combines multiple methods—such as signature detection, threshold detection, and machine learning—making it robust. It covers a range of scenarios, from physical issues to unexpected cyber attacks, ensuring that the system can handle both known and new problems effectively\cite{54}.

Another hybrid approach involves gathering data from both the IT and OT (operational technology) parts of a CPS, including network traffic and sensor readings. This data is preprocessed to remove irrelevant information and balance class distributions. The hybrid approach combines signature detection, threshold-based methods, and behavior-based detection using Ensemble Learning (EL). The system checks for known attack patterns, monitors physical metrics like temperature, and uses machine learning to find new threats.

Ensemble Learning combines the strengths of different models by using techniques like voting and boosting to improve accuracy. For example, the system might use models like Logistic Regression, Naïve Bayes, and K-Nearest Neighbors together. This combination leads to higher accuracy in detecting both physical and cyber threats. By using this hybrid approach, systems can handle both known and unknown issues effectively\cite{70}.

In another method, cyber and physical data are combined to detect anomalies in industrial control systems (ICS). This approach breaks the system into smaller parts and monitors both the cyber and physical sides. Unsupervised behavior-based detectors learn what normal behavior looks like and trigger alerts when something unusual happens. For example, if a pump in a water system suddenly changes its behavior or the network shows unusual activity, the system flags it as an anomaly. These outputs are then combined using a Bayesian Network to classify the anomaly as cyber, physical, or both\cite{48}.


\subsection{Invariant Approaches}

\begin{definition}
Invariant rules are defined as stable relationships or physical conditions that must always be met in a system. In the context of Industrial Control Systems (ICS), these invariants describe the normal expected relationships between sensor readings and actuator states. When these conditions are violated, it suggests that the system is no longer operating normally, potentially indicating a fault or cyber-attack. The purpose of detecting invariants is to ensure that the system is running smoothly by identifying when these stable relationships break\cite{21}.
\end{definition}

The framework proposed here introduces a method to automatically generate invariant rules from historical data logs of an Industrial Control System (ICS) to improve anomaly detection. These invariants describe the normal operating conditions by defining expected relationships between sensors and actuators. Traditionally, system engineers manually create these rules based on design specifications, but this approach can be time-consuming and prone to error, especially in complex systems. Instead, this method uses machine learning techniques to automatically derive these rules from the system's operational data.

The process begins with the collection of data from sensors and actuators. For example, sensor readings could include temperature, pressure, or vibration, while actuators might control valves or motors. This data is used to generate "predicates," or conditions, that describe the state of the system. Then, machine learning techniques, like association rule mining, are applied to find the most important relationships between the components, creating invariant rules that represent normal system behavior. If the system violates these rules during real-time operations, it could signal a fault or a cyber-attack, and an alert is triggered for further investigation\cite{21}.

In another approach, ARTINALI, time is treated as a critical factor in defining invariants. By analyzing the timing of events (such as sensor readings and actuator operations) and how they relate to each other over time, ARTINALI can detect deviations from normal timing patterns. For instance, if a sensor usually sends a reading within 5 seconds of an actuator operating, any delay beyond that time could be flagged as an anomaly. This time-based analysis helps reduce false positives and ensures more accurate anomaly detection\cite{22}.

Another method called illiad integrates several techniques to detect stable relationships (invariants) within a system. illiad uses Autoregressive Models, Latent Factor Analysis, and Kalman Filters to predict how different components in a system should behave based on historical data. These techniques work together to build an "invariant graph," which maps out the relationships between system components under normal conditions. If these relationships break, the system detects the anomaly and alerts the operators, allowing them to fix the issue promptly\cite{23}.

Invariants can also be learned using machine learning combined with mutation testing. In this process, a system is intentionally disrupted by introducing small faults, allowing the model to learn what abnormal behavior looks like. By comparing normal and abnormal behavior, the model can identify conditions that represent normal operations. These learned invariants are then used to detect anomalies in real-time, ensuring that any deviation from normal operations is flagged and addressed\cite{24}.

Another approach uses a method called \textit{Design to Invariants (D2I)}, which derives invariants directly from the design of the system. In this method, the physical processes in an ICS are modeled, and from these models, the key relationships between components are identified. These invariants are then programmed into the system's controllers, ensuring that the system remains within its normal operational boundaries. If any invariant is violated, it indicates that something has gone wrong, such as a mechanical failure or a cyber-attack\cite{26}.

Detecting invariants in Cyber-Physical Systems (CPS) often involves monitoring both cyber and physical components to ensure they are working together correctly. In a smart grid, for example, the system might ensure that the total power produced is always equal to or greater than the total power consumed. This invariant ensures the grid remains stable and prevents overloads. By constantly monitoring these relationships, the system can detect potential faults early and take action before they lead to bigger problems\cite{27}.

Finally, machine learning models can be used to learn and validate invariants by analyzing both normal and abnormal behavior. In this approach, a model is trained using data from normal system operations and then validated through statistical methods to ensure it can reliably identify anomalies. Once validated, this model can be used to monitor the system in real-time, quickly identifying any deviations that suggest an attack or malfunction\cite{28}.

\subsection{Others}

In anomaly detection for Cyber-Physical Production Systems (CPPS), a bio-inspired system mimics the human immune system to detect network threats. This system, called A-HIDS, uses the Incremental Dendritic Cell Algorithm (iDCA) to classify network traffic based on different signals, such as safe signals, danger signals, and harmful patterns (PAMPs). It analyzes the traffic between devices and identifies abnormal behavior by looking at how these signals mix. If danger signals are dominant, the system flags the traffic as abnormal, indicating a potential cyber-attack. The system can take action in real-time, like blocking traffic or logging the threat, and it helps protect against attacks like denial-of-service (DoS) or man-in-the-middle attacks \cite{64}.

Another method uses a whitelist-based anomaly detection system to learn normal behaviors during a training phase. Once the system has a clear picture of expected operations, it monitors real-time activities for deviations from this baseline. For example, if a valve in a cooling system doesn't open when it should, this would trigger an alert. The system can automatically respond to the anomaly, like resetting a controller or activating backup systems, keeping operations safe and secure \cite{55}.

In another approach, big data techniques are used for detecting anomalies in industrial environments. Large amounts of sensor data from machines are collected and organized into clusters. These clusters represent normal operation conditions and are used to detect deviations that might indicate a malfunction. When a deviation is found, the system issues alerts based on the severity of the problem, allowing operators to address issues before they cause major disruptions \cite{62}.

For power grids, anomaly detection combines Intrusion Detection Systems (IDS) with Anomaly Detection Systems (ADS) to monitor both network traffic and physical equipment. IDS checks for unusual network activity, while ADS watches for unusual device behavior, such as a circuit breaker opening without a fault. When both systems detect anomalies, the data is analyzed together to determine if there's a coordinated cyber-attack. If an attack is confirmed, the system can isolate compromised devices and prevent further damage \cite{93}.

In another approach, timing behaviors of system events are monitored to detect anomalies. The system learns the normal timing patterns of events, like how long it usually takes to fill a bottle on a production line. If an event takes significantly longer or shorter than usual, the system flags it as abnormal. This method helps identify equipment wear or delays before they cause major production issues \cite{94}.

A method called \textbf{Abnormal Traffic-indexed State Estimation (ATSE)} focuses on detecting anomalies in smart grids. It integrates both cyber and physical data by monitoring communication traffic and adjusting how much influence suspicious data has on the overall grid's state estimation. If the system detects manipulated data, it reduces its impact and checks for anomalies using a statistical test. This method helps improve detection accuracy by combining cyber-attack data with the physical state of the grid \cite{95}.
\end{comment}



%$$*/*/*/*//*/*/*/*/*/*/*/*/*/*/*/*/*/*/*/*/*/*/$$
\begin{comment}
\subsection{Machine Learning}

Machine learning techniques have emerged as powerful tools for detecting anomalies in Cyber-Physical Systems (CPS), offering the ability to identify both known and unknown threats. The process of implementing machine learning for anomaly detection in CPS typically involves several key phases: data collection, preprocessing, feature engineering, model selection, and evaluation.

\subsubsection{Data Collection and Preprocessing}
The first crucial step is gathering comprehensive data from various CPS components, including physical devices, network traffic logs, and system logs. This multi-source approach ensures coverage of all critical elements within the CPS environment. For instance, in an energy grid, data might be collected from transformers, power lines, and smart meters in real-time \cite{44}. Once collected, the data undergoes preprocessing to clean and transform it into a suitable format for machine learning models. This step involves handling missing values, removing outliers, and standardizing the data to a common scale. In some cases, dimensionality reduction techniques like Principal Component Analysis (PCA) may be applied to improve computational efficiency \cite{44}.

\subsubsection{Feature Engineering}
Feature engineering is a critical phase where meaningful features are extracted from the raw data to enhance the model's ability to detect anomalies. Time-series analysis is often employed to capture temporal dependencies, while statistical metrics like mean, variance, and kurtosis are used to describe data distributions. Domain-specific knowledge plays a crucial role in creating relevant features. For example, in a power grid, features such as average power consumption, voltage spike counts, and seasonal variations in electricity demand might be extracted \cite{44}.

\subsubsection{Model Selection and Training}
The choice of machine learning algorithm depends on the nature of the data and the specific characteristics of the CPS environment. Supervised learning models, such as Support Vector Machines (SVM) or Random Forests, are used when labeled data is available. In cases where labeled data is scarce, unsupervised learning algorithms like K-Means or Gaussian Mixture Models (GMM) are preferred. Semi-supervised techniques, such as Variational Autoencoders (VAE) or Generative Adversarial Networks (GANs), can leverage both labeled and unlabeled data \cite{44}.

For instance, in a study focusing on IoT networks, both Logistic Regression and Artificial Neural Networks (ANN) were employed for anomaly detection. These models were trained on a dataset containing various features describing the IoT network, such as source IDs, addresses, operations, and timestamps. The dataset was divided into training (75\%) and test (25\%) sets to evaluate the models' performance \cite{85}.

\subsubsection{Evaluation and Deployment}
The performance of the models is evaluated using metrics such as True Positive Rate (TPR), False Positive Rate (FPR), Precision, F1 Score, and Area Under the Receiver Operating Characteristic Curve (AUC-ROC). These metrics provide a comprehensive understanding of the model's ability to correctly identify anomalies while minimizing false alarms \cite{44}. In the IoT network study, both Logistic Regression and ANN models achieved exceptional accuracy scores of 99.4\% in detecting various types of anomalies, including Denial of Service (DoS) attacks, malicious control, and data type probing \cite{85}.

Once evaluated and fine-tuned, the models are deployed in the CPS environment for real-time anomaly detection. When new data is fed into the system, the model compares it against learned patterns of normal behavior. Significant deviations are flagged as potential anomalies, triggering alerts for further investigation or immediate action \cite{85}.

\subsubsection{Practical Applications}
The application of machine learning for anomaly detection extends to various CPS domains. In smart grids, for example, logistic regression models have been applied to data from Phasor Measurement Units (PMUs) to detect voltage anomalies and other grid faults. The model processes batches of PMU data and identifies patterns indicative of potential faults, allowing for quick intervention to prevent large-scale power outages \cite{66}.

In manufacturing environments, machine learning techniques have been used to analyze physical data, such as images from 3D printing processes or acoustic signals from CNC milling. By establishing a baseline of normal operations and continuously comparing new data against it, these systems can detect anomalies that may indicate cyber-physical attacks or manufacturing defects with high accuracy \cite{52}.

\subsubsection{Conclusion}
Machine learning approaches offer powerful tools for anomaly detection in CPS, capable of processing vast amounts of data and identifying subtle deviations that might indicate threats or system failures. By combining domain knowledge with advanced algorithms, these methods provide a robust framework for enhancing the security and reliability of critical infrastructure systems. As CPS continue to evolve and face new challenges, machine learning-based anomaly detection will play an increasingly important role in safeguarding these complex environments.

\end{comment}
\subsection{Deep Learning Approaches for Anomaly Detection}

Deep learning techniques have revolutionized anomaly detection in CPS by offering sophisticated methods to analyze complex, multivariate time series data. These approaches excel at capturing intricate temporal and spatial dependencies, enabling the detection of both known and novel anomalies.

\subsubsection{Temporal and Spatial Modeling}

Recurrent Neural Networks (RNNs), particularly Long Short-Term Memory (LSTM) networks, have emerged as powerful tools for modeling the temporal aspects of CPS data. LSTMs are effective due to their ability to capture long-term dependencies, making them well-suited for systems where current behavior is influenced by historical states \cite{47}\cite{15}. For example, in ICS monitoring water treatment plants, LSTM models can predict future water-level readings based on past sensor data, flagging significant deviations as potential anomalies \cite{15}.

Building upon this foundation, \cite{57} proposed a Bidirectional GRU (BiGRU) combined with a Variational Autoencoder (VAE), enhancing the model's capacity to detect subtle anomalies by considering both past and future contexts.

Autoencoders have gained prominence in CPS anomaly detection due to their ability to learn compact representations of normal data. The MTS-DVGAN model \cite{87} combines deep generative models with contrastive learning, using an LSTM-based encoder to learn latent representations of multivariate time series data. This approach employs reconstruction loss and discrimination loss to enhance the model's ability to differentiate between normal and anomalous samples in the latent space.

The RmsAnomaly model \cite{46} further advances autoencoder applications by using convolutional autoencoders to capture both temporal dependencies and inter-sensor correlations. This model constructs signature matrices and uses multi-scale windows to analyze different time scales, calculating an anomaly score based on the difference between original and reconstructed data.

\subsubsection{Advanced Architectures and Methodologies}

To address the challenges of real-time anomaly detection in large-scale CPS, decentralized approaches have been developed. \cite{50} proposed a methodology using 1D Convolutional Autoencoders (1D-ConvAE) deployed directly on individual CPS components. This approach allows each component to independently monitor its own data, reducing reliance on centralized systems and enabling faster detection and response to anomalies.

CNNs have shown particular efficacy in analyzing network traffic for IoT anomaly detection. \cite{81} proposed a method utilizing CNN1D, CNN2D, and CNN3D architectures to handle various input data types, demonstrating high accuracy in detecting diverse attack types.

For systems with more complex interconnections, such as Industrial Internet of Things (IIoT) networks, Graph Neural Networks (GNNs) provide a natural framework for modeling device relationships.  \cite{73} presented a GNN-based method that represents IIoT devices as nodes in a graph, excelling in detecting point, contextual, and collective anomalies.

\subsubsection{Enhanced Security and Robustness}

To improve the security and reliability of autoencoder-based Intrusion Detection Systems (IDS), \cite{88} introduced a method using multipath neural networks. This approach continuously monitors and authenticates the performance of autoencoders, analyzing reconstruction errors to detect anomalies and potential spoofing attacks. The use of a Wilcoxon-Mann-Whitney test enhances the system's ability to detect subtle changes and gradual attacks.

In the context of IoT networks, \cite{80} proposed a comprehensive process using Deep Neural Networks (DNNs) to identify malicious activity in network traffic. This approach involves capturing network traffic, extracting and preprocessing relevant features, and training a DNN model to classify traffic as benign or anomalous. The use of mutual information (MI) for feature selection helps minimize complexity while maintaining high detection accuracy.

\subsubsection{Context-Aware and Zone-Based Approaches}

Context-aware approaches have been developed to improve the accuracy and interpretability of anomaly detection in CPS. The ABATe methodology \cite{49} uses neural networks to generate context vectors that encode relationships between different system states. This approach is effective in detecting both point anomalies and contextual anomalies, and is adaptable across various CPS domains.

For industrial CPS, zone-based approaches offer robust and redundant anomaly detection. \cite{59} proposed a method that divides the physical system into multiple zones, each monitored by a neural network model. Anomalies are detected by cross-referencing data between zones, using tendency and error analysis. This approach is particularly effective in detecting both cyber and physical threats, even when individual zones are compromised.

\subsubsection{Hybrid and Adaptive Systems}

To address the multifaceted challenges of CPS anomaly detection, researchers have developed sophisticated hybrid architectures:

\begin{itemize}
    \item The ATTAIN system \cite{68} integrates a digital twin model with a Generative Adversarial Network (GAN), allowing the digital twin to provide ground truth labels while the GAN enhances detection through adversarial learning.
    
    \item The Adaptive-Correlation-Aware Unsupervised Deep Learning (ACUDL) model \cite{91} employs a dynamic graph update mechanism in conjunction with a Dual-Autoencoder (D-AE), adapting to evolving system dynamics.
    
    \item To address the challenge of limited labeled data, \cite{58} proposed a Siamese Convolutional Neural Network, enabling the identification of novel anomalies with minimal labeled examples.
\end{itemize}

These advanced deep learning techniques offer robust frameworks for processing vast amounts of multivariate time series data and identifying subtle deviations that may indicate threats or system failures. As CPS continue to grow in complexity and face evolving challenges, deep learning-based anomaly detection plays an increasingly crucial role in safeguarding these critical systems.



\begin{figure}[h]
    \centering
    \includegraphics[width=1\linewidth]{images/Picture2.png}
    \caption{This visualization compares the percentage of papers across various CPS application areas, demonstrating the focus and trends in anomaly detection research.}
    \label{fig:application-of-cps}
\end{figure}


\begin{table}[h!]
\centering
\caption{Comparison of Machine Learning (ML) and Deep Learning (DL) Approaches}
\label{tab:ml_dl_comparison}
\renewcommand{\arraystretch}{1.5}
    \begin{tabular}{>{\raggedright\arraybackslash}m{2cm}>{\raggedright\arraybackslash}m{3cm}>{\raggedright\arraybackslash}m{3cm}}
        \hline
        \textbf{Aspect} & \textbf{Machine Learning (ML)} & \textbf{Deep Learning (DL)} \\ \hline

        Feature Engineering & \textbf{Manual}: Requires domain expertise for feature extraction. & \textbf{Automatic}: Learns complex features without manual intervention. \\ 
        \rowcolor[HTML]{EFEFEF}
        Data Complexity & Effective for \textbf{tabular} and simple data. & Suitable for \textbf{unstructured data} (e.g., images, raw sensor data). \\ 
        Interpretability & \textbf{More interpretable}: Models like Decision Trees are easier to understand. & \textbf{Black-box}: Hard to interpret due to complex layers. \\ 
        \rowcolor[HTML]{EFEFEF}
        Data Requirements & Works with \textbf{smaller datasets}; needs labeled data. & Needs \textbf{large datasets} for training; can work with unlabeled data (e.g., Autoencoders). \\ 
        Real-Time Use & \textbf{Lightweight and faster} for inference. & \textbf{Computationally intensive}, though suitable for complex environments. \\ 
        \rowcolor[HTML]{EFEFEF}
        Application & Ideal for \textbf{well-defined features} and smaller setups. & Better for \textbf{complex CPS environments} (e.g., IoT, Smart Factories). \\ \hline
    \end{tabular}
\end{table}
Table \ref{tab:ml_dl_comparison} provides a comparison between Machine Learning (ML) and Deep Learning (DL) approaches, highlighting key differences in aspects such as feature engineering, data complexity, interpretability, data requirements, real-time applicability, and suitability for different applications. While ML relies on manual feature engineering and works well with smaller datasets and simpler data structures, DL excels in handling unstructured data, automating feature extraction, and performing in complex environments like IoT and smart factories, albeit with higher computational requirements.
\subsection{Machine Learning with Deep Learning Approaches Together for Anomaly Detection}

The integration of traditional machine learning techniques with deep learning approaches has emerged as a powerful strategy for anomaly detection in CPS and IoT environments. This combination leverages the strengths of both paradigms to create more robust, efficient, and accurate detection systems.

\subsubsection{Hybrid Architectures}

Several studies have proposed hybrid architectures that combine different machine learning and deep learning techniques to enhance anomaly detection capabilities. For instance, CPS-GUARD \cite{86} utilizes deep autoencoders in conjunction with traditional machine learning techniques. The system trains autoencoders on normal data to learn expected behavior patterns, then employs an outlier-aware thresholding technique using the isolation forest method to dynamically set thresholds for anomaly detection. This approach allows CPS-GUARD to identify previously unseen attacks or faults without requiring explicit labeling of attack data during training.

Another notable study \cite{71} proposes a framework that combines a Siamese Convolutional Neural Network (SCNN) with Kalman Filtering (KF) and Gaussian Mixture Models (GMM). In this approach, the GMM preprocesses heterogeneous data from various CPS layers, while the SCNN performs few-shot learning for anomaly detection. The Kalman Filter then refines the results by analyzing the system's behavior over time, minimizing false positives. These hybrid approaches demonstrate how the combination of deep learning models with traditional machine learning techniques can lead to more robust and adaptable anomaly detection systems.

\subsubsection{Ensemble Methods}

Ensemble methods, which combine multiple models to improve overall performance, have shown promise in CPS and IoT anomaly detection. A comprehensive study \cite{76} evaluated various machine learning algorithms, including Logistic Regression, Support Vector Machines, Decision Trees, Random Forests, and Artificial Neural Networks, for IoT anomaly detection. The Random Forest model, an ensemble of decision trees, emerged as the most effective, achieving 99.4\% accuracy in detecting various types of anomalies, including Denial of Service and Malicious Control attacks. This finding highlights the potential of ensemble methods to outperform individual machine learning or deep learning models in certain scenarios.

\subsubsection{Adversarial Training}

The integration of adversarial training techniques with deep learning models has been explored to enhance the robustness of anomaly detection systems. Research by \cite{67} demonstrated the vulnerability of deep learning-based anomaly detection models to adversarial attacks. To address this issue, they proposed a defense strategy that combines adversarial sample generation using the Fast Gradient Sign Method (FGSM) with retraining of the neural network model. This approach significantly improved the model's resilience to adversarial attacks while maintaining high performance on clean data. This work underscores the importance of considering adversarial scenarios when developing anomaly detection systems for CPS and IoT environments.

\subsubsection{Multi-Stage Processing}

Some approaches leverage both machine learning and deep learning in different stages of the anomaly detection process. A framework for IoT data stream anomaly detection \cite{79} employs a multi-stage approach. In this system, traditional machine learning techniques like clustering algorithms (e.g., K-means or Local Outlier Factor) are used for initial anomaly detection when labeled data is scarce. Subsequently, deep learning models, such as Autoencoders and Long Short-Term Memory (LSTM) networks, are applied to handle more complex, high-dimensional, and time-dependent data. This multi-stage approach allows the system to leverage the strengths of both machine learning and deep learning techniques at different points in the anomaly detection pipeline.



The integration of machine learning and deep learning techniques for anomaly detection in CPS and IoT environments offers several advantages. These include improved adaptability to different types of data and anomalies, enhanced robustness against adversarial attacks, better handling of complex, high-dimensional, and time-series data, and the ability to detect both known and unknown anomalies. As CPS and IoT systems continue to evolve and face increasingly sophisticated threats, the combination of machine learning and deep learning approaches will likely play a crucial role in developing more effective and resilient anomaly detection systems. Future research in this area may focus on further optimizing these hybrid approaches, developing more interpretable models, and addressing the challenges of real-time processing in resource-constrained IoT environments.









\subsection{Mathematics Approaches for Anomaly Detection}

Mathematical approaches play a crucial role in anomaly detection for CPS and IoT environments, offering rigorous frameworks for modeling system behavior and identifying deviations. These methods range from probabilistic models to formal logic systems, each providing unique advantages in detecting and classifying anomalies.

\subsubsection{Graph-based Models}

Graph-based models have shown effectiveness in capturing the complex interactions within IoT systems. The Device Interaction Graph (DIG) approach \cite{74} models IoT devices as nodes and their interactions as directed edges. Each edge is associated with a Conditional Probability Table (CPT), defining the likelihood of a device's state based on interacting devices. This model allows for the detection of both contextual and collective anomalies by comparing real-time events against expected behaviors stored in the graph. This method is particularly useful in smart environments where device interactions follow discernible patterns.

\subsubsection{Bayesian Inference}

Bayesian inference provides a probabilistic framework for estimating system parameters and detecting anomalies. \cite{51} applied Bayesian inference to estimate unknown parameters in mechanical systems modeled as damped harmonic oscillators. This approach uses Markov Chain Monte Carlo (MCMC) sampling to generate plausible parameter values, allowing for probabilistic anomaly detection even with noisy or limited data. Another study \cite{56} employed Bayesian networks to model the causal relationships between cyber and physical components in CPS. This method calculates the probability of observed system states given the states of related variables, flagging low-probability states as anomalies. Bayesian methods excel in handling uncertainty and providing probabilistic assessments of anomalies.

\subsubsection{State Estimation, Filtering and Fusion}

Advanced state estimation techniques have been applied to detect anomalies, particularly in the context of False Data Injection (FDI) attacks. The Ensemble Kalman Filter (EnKF) approach \cite{72} generates an ensemble of possible system states using historical data to forecast normal behavior. Anomalies are detected by comparing these predictions with real-time measurements using Euclidean distance. A multi-sensor fusion strategy \cite{53} combines data from multiple sensors using optimized weights to form a fused residual signal. This method employs logarithmic quantization and convex optimization to ensure real-time detection despite bandwidth constraints. These techniques are particularly valuable in large-scale systems like power grids, where rapid and accurate anomaly detection is crucial. Methods that combine cyber and physical data can provide more accurate anomaly detection. The Abnormal Traffic-indexed State Estimation (ATSE) method \cite{95} integrates cyber impact factors from network monitoring with physical state estimation in Smart Grids. This fusion approach down-weights measurements from buses with higher cyber threat levels, enhancing detection accuracy for both cyber and physical attack vectors.

\subsubsection{Information Theory}

Information theoretic approaches offer novel ways to detect anomalies based on the flow of information within a system. Transfer entropy-based causality countermeasures \cite{69} quantify the information flow between different system signals. Anomalies are detected when the transfer entropy deviates significantly from baseline values, indicating disrupted causal relationships. This method is especially effective in detecting a wide range of attacks without requiring prior knowledge of specific attack types.

\subsubsection{Formal Methods}

Formal methods provide rigorous, logic-based approaches to anomaly detection. Signal Temporal Logic (STL) \cite{19} is used to model normal system behavior through time-bound constraints. Anomalies are detected when the system's behavior violates the inferred STL formula, quantified by a robustness metric. This approach offers the advantage of producing human-readable descriptions of normal system behavior and anomalies.

\subsubsection{Automata}

Leveraging the timing behavior of system events for anomaly detection has proven effective in certain contexts. Some papers \cite{94} propose a method that models normal timing behavior using Timed Automata and probability density functions (PDFs). Real-time performance is compared against learned timing distributions, with deviations flagged as potential anomalies. This approach is particularly effective in production systems with variable timing patterns.

\subsubsection{Hybrid Approaches}

Many studies combine multiple mathematical techniques to create more robust anomaly detection systems. A framework for Digital Twin-based CPS \cite{61} integrates Gaussian Mixture Models (GMM) for discrepancy detection, conformal prediction for calculating p-values, and Hidden Markov Models (HMM) for anomaly classification. The Orpheus framework \cite{65} combines finite-state automata with event-aware modeling to detect anomalies by verifying consistency between program actions and physical events. These hybrid approaches leverage the strengths of multiple mathematical techniques to provide comprehensive anomaly detection in complex CPS environments.

Mathematical approaches to anomaly detection in CPS and IoT offer rigorous, interpretable, and often computationally efficient methods for identifying system deviations. These techniques provide a strong foundation for developing robust anomaly detection systems, capable of handling the complex, dynamic nature of modern cyber-physical environments. As CPS and IoT systems continue to evolve, the integration of advanced mathematical methods with machine learning and deep learning approaches is likely to yield even more powerful and adaptable anomaly detection solutions.






\subsection{Hybrid Approaches for Anomaly Detection}

Hybrid approaches to anomaly detection in CPS and IoT environments combine multiple techniques to leverage their respective strengths and overcome individual limitations. These methods often integrate signature-based, threshold-based, and machine learning techniques to provide comprehensive coverage against both known and unknown threats.

\subsubsection{Integration of Multiple Detection Strategies}

Hybrid approaches typically combine various detection strategies to enhance overall performance. \cite{54} proposed a framework that integrates signature-based, threshold-based, and machine-learning techniques. This approach uses one-class classifiers like K-Nearest Neighbors (KNN) or Support Vector Machines (SVM) trained on normal data, combined with threshold-based detection for physical limits and signature-based detection for known cyber threats. In another study, \cite{70} introduced a hybrid structure incorporating signature-based, threshold-based, and behavioral-based detection through Ensemble Learning (EL). The EL component combines multiple machine learning algorithms using techniques like voting, stacking, bagging, and boosting to improve predictive performance. \cite{16} describes a comprehensive approach that combines signature-based, threshold-based, and behavior-based models, emphasizing the importance of establishing a baseline of normal behavior and identifying different types of anomalies (point, contextual, and collective).

\subsubsection{Multi-Step Anomaly Detection Process}

Several studies propose a structured, multi-step process for anomaly detection. \cite{83} outlines a process involving data collection, preprocessing, feature extraction, model selection, thresholding, and decision-making. This approach emphasizes the importance of efficient and scalable data collection in real-time IoT environments. \cite{82} presents a systematic methodology applicable across various IoT domains. The process includes understanding data nature, preprocessing, selecting anomaly types, choosing appropriate detection methods, and deploying the system for real-time or historical data analysis. \cite{77} describes a workflow for IoT time-series data that includes data collection, preprocessing, defining normal behavior, real-time monitoring, and reporting. This approach emphasizes the importance of preprocessing steps like handling missing values and dimensionality reduction.

\subsubsection{Combining Statistical and Machine Learning Techniques}

Many hybrid approaches integrate statistical methods with advanced machine learning techniques. \cite{18} proposes a combination of Long Short-Term Memory Recurrent Neural Networks (LSTM-RNN) and the Cumulative Sum (CUSUM) method. The LSTM-RNN learns temporal patterns and predicts expected sensor values, while CUSUM tracks cumulative deviations to detect small, gradual anomalies. \cite{60} combines Seasonal Autoregressive Integrated Moving Average (SARIMA) with LSTM models. SARIMA captures short-term trends and seasonal patterns, while LSTM recognizes long-term dependencies and recurring patterns. \cite{17} proposes a two-phase approach using Gaussian Mixture Models (GMM) and Kalman Filters (KF). GMM models the distribution of normal behavior, while KF estimates dynamic states and calculates dynamic thresholds.

\subsubsection{Graph-based and Transformer Approaches}

Some hybrid methods leverage graph structures and transformer architectures. The GTA framework \cite{75} combines graph learning with transformer-based models to capture both spatial and temporal dependencies in multivariate time series data from IoT systems. The illiad system \cite{23} integrates model-based predictions using Kalman filters with data-driven methods like autoregression and latent factor analysis, representing the system as an invariant graph.

\subsubsection{Context-Aware and Adaptive Systems}

Hybrid approaches often incorporate context-awareness and adaptability.\cite{48} proposed a methodology that combines unsupervised behavior-based detectors for both cyber (network traffic) and physical (sensor data) domains. The outputs of these detectors are then integrated using a Bayesian Network to calculate the likelihood of different types of anomalies. The NSIBF framework \cite{45} combines Neural System Identification with Bayesian Filtering, learning the system's normal behavior through neural networks and then applying Bayesian filtering to monitor the system's state over time.

Hybrid approaches to anomaly detection in CPS and IoT environments offer several advantages, including improved detection of both known and unknown anomalies, enhanced ability to handle complex, multi-domain data from cyber and physical components, increased robustness against false positives and false negatives, and better adaptability to evolving system dynamics and threat landscapes. As CPS and IoT systems continue to grow in complexity and face increasingly sophisticated threats, hybrid approaches that combine multiple detection strategies, integrate cyber and physical domain analysis, and leverage both statistical and machine learning techniques will likely play a crucial role in developing more effective and resilient anomaly detection systems.




\begin{figure*}
    \centering
    \includegraphics[width=1\linewidth]{images/aass.png}
    \caption{General Workflow for Anomaly Detection}
    \label{fig:general-steps}
\end{figure*}


\subsection{Invariant-based Approaches for Anomaly Detection}

Invariant-based approaches have gained significant attention as a robust technique for anomaly detection in CPS and IoT environments. These methods focus on identifying and monitoring stable relationships or dependencies between different components of a system that remain consistent under normal operating conditions.

\begin{definition}
Invariant rules are defined as physical or logical conditions that must be satisfied for any given state of an ICS. These rules describe the expected relationships between sensor readings and actuator states, and their violation indicates a deviation from normal operation \cite{21,23}.
\end{definition}

Invariant-based approaches rely on defining specific rules or properties that the system must adhere to at all times. Violations of these invariants are indicative of potential anomalies. These invariants can be derived from system design, operational specifications, or learned patterns, ensuring they accurately represent the system's expected behavior. We have three types of invariants:

\begin{itemize}
    \item \textbf{State-Based Invariants}: These define specific states or relationships that must always hold. For instance, a valve's state must correspond to specific sensor readings during normal operations \cite{26}.
    \item \textbf{Temporal Invariants}: These enforce timing constraints, such as specific sequences or delays between events. Temporal invariants are particularly critical for real-time systems where timing consistency is essential \cite{22}.
    \item \textbf{Unified Invariants}: These combine multiple dimensions of CPS (cyber, physical, and network) to create overarching stability rules. Unified invariants often use concepts like Lyapunov-like functions to ensure overall system stability and integrity \cite{29}.
\end{itemize}

These invariant-based methods are particularly effective in systems with well-defined operational rules. They excel at detecting anomalies that may not be evident through purely data-driven approaches. By integrating invariants with data-driven techniques, hybrid models can further enhance anomaly detection capabilities, providing a balanced approach to robustness and adaptability.

The importance of invariants in CPS security lies in their ability to capture the fundamental physical and logical constraints of the system. This makes them a robust framework for detecting anomalies indicative of faults, cyber-attacks, or other security breaches \cite{21,23}.


\subsubsection{Techniques for Extracting Invariants}

There are four primary techniques for extracting invariants in CPS:

\begin{itemize}
    \item \textbf{Design-Based}\cite{26,27,29}: Utilizes the system's design specifications and hybrid automata to derive invariants that must hold for correct operation. An example includes the D2I (Design-to-Invariants) approach.
    \item \textbf{Data-Driven}\cite{24,25,28}: Applies machine learning techniques on historical data to detect patterns and automatically derive invariants. Methods such as association rule mining fall into this category.
    \item \textbf{Mutation-Based}\cite{25}: Involves intentionally introducing faults ("mutants") to explore the system's boundaries between normal and abnormal behaviors, thereby generating invariants.
    \item \textbf{LLM-Based}\cite{116}: Leverages Large Language Models (LLMs) to interpret technical documentation. Techniques like chain-of-thought prompts and Retrieval-Augmented Generation (RAG) workflows are used to propose hypothetical invariants based on semantic relationships among components.
\end{itemize}

These techniques provide diverse ways to derive invariants, enabling tailored approaches for different CPS environments and improving the robustness of anomaly detection \cite{206}.




\subsubsection{Verification and Validation of Invariants}

Ensuring the accuracy and reliability of detected invariants is crucial for effective anomaly detection. Statistical Model Checking provides probabilistic guarantees by analyzing system executions against the learned invariants \cite{24}. Symbolic Execution verifies invariants against all possible code paths, ensuring comprehensive coverage of the system's behavior \cite{24}. The integration of machine learning, software testing, and formal methods enhances the robustness of invariant detection and verification, particularly in critical CPS applications \cite{24}.

\subsubsection{Application in Anomaly Detection}

Once invariants are detected, they are used for real-time monitoring and anomaly detection. Invariants are continuously checked against real-time system data. Any violation of these invariants triggers an alert, indicating a potential anomaly \cite{26}\cite{28}. Some approaches, like illiad \cite{23}, provide visual dashboards displaying the invariant graph and highlighting broken invariants in real-time, facilitating quick response and decision-making. Invariant-based methods are particularly effective in detecting sophisticated multi-point attacks that manipulate multiple system elements \cite{26}.



While invariant-based approaches offer robust anomaly detection, several challenges remain. Current methods often struggle to capture complex, multi-component interactions common in CPS \cite{25}. Many approaches do not account for time-based dependencies, limiting their ability to detect temporal anomalies \cite{25}. As CPS grow in complexity, scalable methods for invariant detection and monitoring are needed. Developing invariant-based methods that can adapt to evolving system dynamics without compromising security remains a challenge.

Future research directions include integrating invariant-based approaches with other anomaly detection techniques, developing methods for handling dynamic invariants in adaptive CPS, and improving the interpretability of learned invariants to aid in system understanding and forensic analysis.






\subsection{Other Approaches for Anomaly Detection}

While machine learning, deep learning, and invariant-based approaches are widely used for anomaly detection in CPS and Smart Grids, several other innovative methodologies have been proposed to address specific challenges in these complex environments.

\subsubsection{Bio-inspired Approaches}

Bio-inspired methods draw inspiration from biological systems to create robust anomaly detection mechanisms. The Incremental Dendritic Cell Algorithm (iDCA) \cite{64} mimics the human immune system, particularly dendritic cells, to classify network traffic and detect abnormal patterns. This approach categorizes traffic based on safe signals, danger signals, and pathogenic associated molecular patterns (PAMPs), providing a scalable and effective method for detecting cyber-attacks in industrial settings.

\subsubsection{Whitelisting and Self-Adaptation}

Some approaches focus on establishing baselines of normal behavior and implementing self-adaptive mechanisms. The \AE CID tool \cite{55} uses a whitelisting approach to learn expected patterns of system operation during a training phase. Any deviation from this baseline triggers an alert, and the system employs self-adaptation policies to mitigate threats, such as resetting PLCs or activating backup systems.


\subsubsection{Big Data Techniques}

Leveraging big data analytics for anomaly detection in industrial environments has shown promising results. A real-time anomaly detection system \cite{62} uses data summarization techniques and clustering algorithms to condense vast amounts of sensor data. Relevance evaluation techniques compare new data clusters to baseline clusters, flagging significant deviations as potential anomalies.

\subsubsection{Multi-layered Detection Systems}

Integrating multiple detection mechanisms provides comprehensive coverage. A combined approach using Network-Based Intrusion Detection Systems (NIDS), Host-Based Intrusion Detection Systems (HIDS), and Anomaly Detection Systems (ADS) \cite{93} monitors both network traffic and physical device behaviors. This layered approach allows for the detection of coordinated anomalies that may signal larger attacks in power grid systems.

The diversity of these methodologies reflects the complex nature of CPS and Smart Grids, where anomalies can manifest in various forms across cyber and physical domains. Future research may focus on integrating these diverse approaches to create more robust, adaptive, and comprehensive anomaly detection systems capable of addressing the evolving threat landscape in critical infrastructure systems.


\begin{table*}[h!]
\centering
\caption{Comparison of Different Approaches for Problem-Solving}
\label{tab:anomaly_comparison}
\renewcommand{\arraystretch}{1.5}
    \begin{tabular}{>{\raggedright\arraybackslash}m{2.5cm}>{\raggedright\arraybackslash}m{3.4cm}>{\raggedright\arraybackslash}m{3.3cm}>{\raggedright\arraybackslash}m{3.4cm}>{\raggedright\arraybackslash}m{3.4cm}}
        \hline
        \textbf{Approach} & \textbf{What It Does} & \textbf{Good Points} & \textbf{Limitations} & \textbf{Example} \\ \hline

        Math Models & Uses formulas to predict normal system behavior. & Simple and clear \newline Needs less data & Hard for complex systems & Predicting room temperature with a formula. \\ 
        \rowcolor[HTML]{EFEFEF}
        Machine Learning (ML) & Finds patterns in data using algorithms. & Good with labeled data \newline Easy to use & Needs training data \newline Can miss hidden issues & Spotting faulty IoT devices. \\ 
        Deep Learning (DL) & Uses neural networks for finding hidden patterns. & Works with complex data \newline Learns by itself & Needs a lot of power \newline Hard to explain & Finding sensor issues in power grids. \\ 
        \rowcolor[HTML]{EFEFEF}
        Invariant-Based & Checks rules that must always be true. & Easy to understand \newline No data needed & Only works for known rules & Ensuring water flow rates in a treatment plant. \\
        Hybrid Approaches & Combines two or more methods for better results. & Finds more problems \newline Flexible & Complicated to set up & Using AI with rules to check smart grid problems. \\ 
        \rowcolor[HTML]{EFEFEF}
        Other Methods & Special tools like Big Data or immune system ideas. & Works for special needs & Hard to use in general & Filtering data in large IoT networks. \\ \hline
    \end{tabular}

\end{table*}
Figure \ref{fig:application-of-cps} presents a visualization of the percentage of research papers focused on anomaly detection across different CPS application areas. The chart highlights that industrial systems dominate the research focus with 41.3\%, followed by general CPS anomaly detection (26.1\%), healthcare systems (10.9\%), critical infrastructure (13.0\%), and transportation systems (8.7\%), showcasing the varying levels of interest and emphasis in anomaly detection research.