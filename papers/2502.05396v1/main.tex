\documentclass{article}
% \documentclass[journal]{IEEEtran}
% \usepackage[backend=biber]{biblatex}




\usepackage{graphicx} % Required for inserting images
\usepackage{geometry}
\geometry{a4paper,left=2cm,right=2cm,top=2cm,bottom=2cm}

\usepackage[colorlinks,bookmarksopen,bookmarksnumbered,citecolor=red,urlcolor=red]{hyperref}

\usepackage{amsmath}
\usepackage{amssymb}
\usepackage{amsfonts}

\title{\textbf{A Novel Convolutional-Free Method for 3D Medical Imaging Segmentation}}
\author{\textbf{Canxuan Gang} \\AI Geeks\\\url{https://aigeeksgroup.github.io}}
\date{}

\begin{document}
\setlength{\parindent}{0pt}
\setlength{\parskip}{10pt}

\maketitle

% Your content here

\begin{abstract}
    Segmentation of 3D medical images is a critical task for accurate diagnosis and treatment planning. Convolutional neural networks (CNNs) have dominated the field, achieving significant success in 3D medical image segmentation. However, CNNs struggle with capturing long-range dependencies and global context, limiting their performance, particularly for fine and complex structures. Recent transformer-based models, such as TransUNet and nnFormer, have demonstrated promise in addressing these limitations, though they still rely on hybrid CNN-transformer architectures. This paper introduces a novel, fully convolutional-free model based on transformer architecture and self-attention mechanisms for 3D medical image segmentation. Our approach focuses on improving multi-semantic segmentation accuracy and addressing domain adaptation challenges between thick and thin slice CT images. We propose a joint loss function that facilitates effective segmentation of thin slices based on thick slice annotations, overcoming limitations in dataset availability. Furthermore, we present a benchmark dataset for multi-semantic segmentation on thin slices, addressing a gap in current medical imaging research. Our experiments demonstrate the superiority of the proposed model over traditional and hybrid architectures, offering new insights into the future of convolution-free medical image segmentation.
\end{abstract}

\section{Introduction}

Segmentation is a key task in computer vision and has been a primary research focus for many years \cite{1}. One of the most common applications of 3D segmentation in voxel representation is medical imaging, as the images obtained from main medical imaging modalities such as CT, PET, and MRI are typically in 3D volume \cite{2,wu2023bhsd,zhang2024segreg,tan2024segstitch}. Therefore, accurate 3D segmentation is essential for medical image analysis, including diagnosis and treatment planning. 



In the field of computer vision, convolutional neural networks (CNNs) have been highly successful in recent decades \cite{3,4,5,6}. Their ability to automatically learn and extract features from image data through hierarchical layers of convolutional and pooling operations has revolutionized many computer vision tasks, including object detection \cite{7,zhang2024meddet,cai2024msdet,cai2024medical}, image classification \cite{3,4,5,6,ji2024sine}, and semantic segmentation \cite{8,ge2024esa,tan2024segkan,zhang2025gamed}. The availability of large datasets, such as ImageNet \cite{9}, and advancements in parallel computing and deep learning frameworks have also contributed to the success of CNNs in computer vision. Thus, CNNs have become the main approach for many computer vision tasks due to their ability to capture complex image features through translation equivariance, local sparse connections, and weight sharing \cite{10}. These properties have also made CNNs highly effective for medical imaging segmentation, where accurate delineation of structures and regions of interest is critical for diagnosis and treatment planning \cite{zhang2024deep,hiwase2025can,zhao2024landmark,qi2025projectedex}. Consequently, the state-of-the-art (SOTA) models for medical imaging segmentation are now largely based on CNN architectures, such as U-Net \cite{11} and U-Net++ \cite{12} in 2D imaging, 3D U-Net \cite{13} and nnUNet \cite{14} in 3D imaging, which have been shown to outperform traditional image processing techniques and other types of neural networks in this domain. 

Despite their success in many computer vision tasks, CNNs have been found to have limitations in accurately segmenting fine and complex structures in medical images. Recent research has shown that CNNs can struggle to capture long-range dependencies and global context information \cite{15}, which are critical for accurate segmentation of medical images. In medical imaging segmentation tasks, these limitations are often reflected in low classification accuracy and sub-optimal segmentation quality. To address these limitations, transformer-based architectures with self-attention mechanisms (which shown in figure \ref{fig:1}) have emerged as a promising alternative in general computer vision tasks \cite{17}. For example, the Vision Transformer (ViT) \cite{15} based architecture has demonstrated state-of-the-art performance on various computer vision tasks, including image classification \cite{18}, object detection \cite{19}, and semantic and instance segmentation \cite{20}. With the recent successful implementation of transformer architecture and self-attention mechanism in the general computer vision field, there has been a growing interest in utilizing these techniques for medical imaging segmentation. Several transformer-based models have been proposed, such as TransUNet \cite{21} for 2D segmentation and nnFormer \cite{23} for 3D segmentation. However, while these models have shown promising results, they still rely on combining transformer architecture with convolutional architecture to either extract features or decode masks, making them a compromised architecture rather than entirely "convolutional-free". Thus, the development of a purely convolutional-free architecture based on transformer and self-attention mechanisms is crucial in improving the performance of semantic segmentation and advancing the field of medical imaging segmentation.

\begin{figure}
    \centering
    \includegraphics[width=0.6\linewidth]{att.PNG} 
    \caption{Self-attention architecture from Transformer \cite{16}}
    \label{fig:1}
\end{figure}


Moreover, among various medical imaging and radiological diagnostic modalities, non-contrast computed tomography (NCCT) is one of the most frequently utilized techniques \cite{24}. In the context of medical imaging, there are two types of images that can be obtained from a CT scan: thick slices and thin slices \cite{25}. Thick slices are obtained from thick scans or retrospective reconstruction \cite{26}, whereas thin slices are acquired from thin scans. Unfortunately, publicly available datasets with pixel-wise ground truth are mostly restricted to thick-slice datasets. As a result, existing medical imaging segmentation methods are primarily designed for thick-slice segmentation. However, thick slices are subject to poor resolution in the depth direction, or z-resolution, and are therefore not isomorphic. Conversely, thin slices contain more volumetric information and are isomorphic, meaning they offer superior segmentation results. Additionally, only thin slices can produce a real 3D volume through volumetric rendering (VR) \cite{27}, while interpolation of thick slices leads to the introduction of fake images. This disparity has resulted in a considerable gap in the quality of segmentation outcomes between thin and thick slices. However, hand-annotating a large number of thin slices by a human expert such as a radiologist is not feasible due to the time-consuming nature of the process. Therefore, the development of a novel architecture that incorporates effective strategies for segmenting thin slices based on thick-slice annotations is crucial for the advancement of diagnostic medicine \cite{zhang2023thinthick}.

In general, our research objectives can be summarized as follows: 
\begin{itemize}
    \item To design a novel and efficient model that can significantly enhance the performance of semantic segmentation on 3D medical images, with a focus on improving the classification accuracy of semantic labels and the quality of segmentation masks.
    \item To develop a new architecture that can effectively address the domain adaptation problem between thick and thin slices, allowing for accurate segmentation of thin slices based on thick annotations.
    \item To contribute a new dataset for multi-semantic segmentation on thin slices, addressing the lack of such datasets in the field of medical imaging segmentation.
\end{itemize}
These objectives are critical to advancing the state-of-the-art in medical image analysis and improving diagnostic medicine.

Furthermore, our research approaches can be summarized as follows:

\begin{itemize}
    \item Develop a novel and efficient architecture that is fully convolutional-free and based on transformers to improve the performance of multi-semantic segmentation on 3D medical images. This involves exploring the use of self-attention mechanisms \cite{16} and other techniques to enhance feature extraction and decoding of masks.
    \item Design a joint loss that can supervise the model to segment thin using thick slices. This involves developing a novel loss function that can effectively utilize both thick and thin slices to improve segmentation accuracy.
    \item Publish a benchmark dataset with thin slice multi-semantic segmentation. This involves collecting a large dataset of NCCT images with thin slices and multi-semantic annotations and making it publicly available for researchers to use and evaluate their models.
\end{itemize}

The report will be divided into distinct sections to ensure a coherent flow of information. These sections are as follows:

\begin{itemize}
    \item \textbf{Research Problems}: This section will entail a general description of existing model on 3D medical imaging segmentation. Furthermore, the section will highlight the significance of the proposed method and its potential to address real-world issues. Additionally, it will underscore the benefits of the proposed method when compared to existing models.
    \item \textbf{Methodology}: In this section, the proposed method used to achieve the research objectives will be described. This will involve a detailed explanation of the fully convolutional-free architecture based on transformers, designed to enhance the performance of multi-semantic segmentations. Additionally, the joint loss that supervises the model to segment thin using thick annotations will be presented.
    \item \textbf{Evaluation Criteria}: This section will outline how the model's performance will be evaluated. It will include details on the dataset to be used for evaluation, the design of the experiment for evaluation, and the specific evaluation metrics that will be employed in medical imaging segmentation.
    \item \textbf{Timeline Schedule}: This section will provide a detailed timeline of the research activities and milestones, indicating the expected duration of each activity and the interdependencies between them.
    \item  \textbf{Limitations and Potential Problems}: This section will discuss the potential limitations and challenges of our proposed method, including but not limited to, (1) the potential of utilizing generative models, such as the diffusion model, to address the issue of the long-tail problem by generating synthetic samples of minority classes, (2) the potential capability of large-scale self-supervised pre-training for feature extraction, and (3) potential the ethics and security problem of medicial imaging datasets. We will also explore potential solutions to these problems and discuss future directions for research in this field.
    \item \textbf{Conclusions}: This section will provide a summary of the article, highlighting the key findings and implications for future research. It will also provide recommendations for potential applications of the proposed model in medical imaging segmentation.
\end{itemize}



\section{Research Problems}
\subsection{Related Works}

Prior to introducing our proposed method, we will provide a review of related literature on existing models used in 3D medical imaging segmentation.

The Hough-CNN, proposed by Milletari et al \cite{28} is a typical convolutional neural network designed for 3D medicial imaging segmentations, especially for MRI and Ultrasound. The approach leverages the abstraction capabilities of CNNs and implements Hough voting, which allows for automatic localization and segmentation of anatomies. 

The 3D U-Net, proposed by Çiçek et al \cite{13}, stands out as the first modern model specifically designed for 3D medical imaging segmentations. The architecture of this model is based on the fully-convolutional U-Net proposed by Ronneberger et al. During its inception, the authors acknowledged the unfeasibility and time-consuming nature of labeling a dense ground truth for a 3D volumetric image. As a result, the method employed dense segmentations based on sparse annotations using weighted loss. It should be noted, however, that the lack of publicly available code associated with this work presents a notable limitation. Despite this limitation, the paper's significant influence on subsequent works remains undeniable and has served as a source of inspiration for our proposed method.

The Residual Symmetric 3D U-Net, proposed by Lee et al. \cite{29}, was inspired by the ResNet architecture proposed by Kaiming He et al. \cite{6} in 2016. The key innovation of this approach is the introduction of residual connections between each convolutional layer in the 3D U-Net. This method allows for the successful implementation of residual connections in a fully convolutional architecture, enabling the design of deeper encoder-decoder networks that can extract more abstract features while avoiding issues such as gradient explosion and vanishing. The use of residual connections ensures that the network can learn residual mappings that facilitate the optimization process and improve performance. Overall, the Residual Symmetric 3D U-Net represents an important advancement in the field of 3D medical image segmentation by leveraging the benefits of both residual connections and fully convolutional networks.

The nnU-Net, proposed by Isensee et al \cite{14}. has made significant advancements in the field of 3D medical imaging segmentation, and is widely regarded as the state-of-the-art model in this area. This is due to its innovative pipeline, which builds upon the well-established 3D U-Net architecture. While the backbone of the model remains the same, the authors have introduced novel pre-processing and post-processing techniques, as well as incorporating additional metadata in the widely-used nifti format for CT images. This incorporation of metadata is a significant contribution to the field, as it allows for a more comprehensive understanding of the image and its features. This, in turn, leads to more accurate and reliable segmentation results. The success of nnU-Net has inspired subsequent works, such as CoTr and nnFormer, both of which have also built upon the 3D U-Net architecture. However, nnU-Net remains the most performant model for medical imaging segmentation, due to its ability to leverage metadata and its incorporation of state-of-the-art techniques.


The TransUNet, proposed by Chen et al. \cite{21}, is a pioneering approach to medical imaging segmentation using transformer architecture. The model leverages the transformer layer at the end of the CNN encoder, creating a CNN-Transformer hybrid encoder. This is combined with a cascaded upsampler as a decoder to generate masks. The transformer layer enhances the model's capability to capture global context information, while the CNN encoder still plays a crucial role in feature extraction. However, it is important to note that TransUNet is essentially a 2D segmentation method that concatenates 2D images into 3D, which can result in inconsistencies in masks between scans. This can be observed from the sagittal and coronal views. Despite this limitation, TransUNet has laid the foundation for future research for transformers in medical imaging segmentation.



The CoTr is a novel approach that effectively combines convolutional neural network (CNN) and transformer for 3D medical image segmentation. Although the model is based on the 3D U-Net architecture and utilizes the same pre-processing and post-processing methods as nnU-Net, it introduces an additional deformable transformer (DeTrans) encoder. This encoder is designed to capture deeper features from the feature maps produced by the CNN encoder, allowing for better representation of complex image patterns. By bridging the gap between CNNs and transformers, CoTr offers improved performance and greater flexibility in 3D medical image segmentation tasks. The results of the study demonstrate that CoTr outperforms several state-of-the-art models in terms of segmentation accuracy, making it a promising direction for future research in this field.

The nnFormer, proposed by Zhou et al. \cite{23}, is a recent addition to the class of CNN-Transformer hybrid models that has been designed as an extension of the nnUNet architecture. The model consists of several components, including a convolutional embedding layer, a Multi-head Self-attention encoder, and a symmetrical transformer block with a deconvolution-based upsampling layer as the decoder. The convolutional embedding layer maps the input image to a feature space, which is then processed by the Multi-head Self-attention encoder to capture global contextual information. The symmetrical transformer block is responsible for capturing long-range dependencies and generating highly contextualized features. Finally, the deconvolution-based upsampling layer is used to reconstruct the output mask. This model has been shown to be highly effective in 3D medical image segmentation tasks and outperforms several state-of-the-art methods.

The paper "Convolution-Free Medical Image Segmentation using Transformers" by Karimi et al \cite{30}. presents a convolutional-free architecture for medical image segmentation using transformers. However, this approach cannot been used for thin segmentation using thick annotations, and the lack of published code and dataset raises concerns about the validity of the reported results.

Recently, the Mamba model \cite{zhang2024infinimotion,zhang2024kmm,zhang2024motion} has demonstrated state-of-the-art efficiency in sequence modeling by leveraging selective state-space layers. Its strong long-range dependency modeling makes it well-suited for medical imaging segmentation, enabling frameworks like UMamba and SegMamba to achieve improved feature extraction and boundary delineation.

In general, the related works mentioned above primarily utilize either convolutional architecture or incorporate transformer layers as part of the encoder-decoder architecture. In contrast, our proposed model deviates significantly from this approach by introducing a convolutional-free architecture. This novel approach aims to explore the potential of non-convolutional architectures for medical image segmentation, potentially enabling the development of more efficient and effective models in this domain.

\subsection{Research Significants}


The field of 3D medical imaging segmentation has significant implications for real-world medical applications. The development of automatic segmentation using computer vision technology \cite{zhang2024motionavatar} can aid radiologists in detecting the focus of the disease and labeling annotations, reducing the time and effort required for manual segmentation. Additionally, segment masks can provide more accurate attributes of diseases, such as the estimation of the volume of brain hemorrhages, aiding in diagnosis and treatment planning. Moreover, the use of segmentation in longitudinal analysis can assist radiologists in predicting the prognosis of certain diseases, such as cancer, and prioritize urgent cases accordingly. The ability to automatically segment 3D medical images has the potential to greatly enhance the accuracy and efficiency of medical diagnosis and treatment, and ultimately improve patient outcomes.

Our proposed convolutional-free method has two significant contributions that outperform existing segmentation models. Firstly, the transformer architecture in our encoder-decoder model can capture global context information better than CNN-based models. The self-attention mechanism in the transformer architecture allows for the model to attend to important regions in the input image, which results in more accurate semantic labeling and generates high-quality masks. Additionally, transformers can capture long-range dependencies in the image, allowing for better modeling of complex relationships between pixels.

Secondly, our method includes domain adaptation between thin slices and thick annotations. This allows us to segment thin slices with high z-resolution and generate an isomorphic mask that accurately represents the volume of focus. This mask can be used in surgical planning and navigation, providing a more precise pre-surgical simulation and direct guidance during surgery. By accurately representing the volume of focus, our method has the potential to improve patient outcomes and reduce surgical complications. Overall, our proposed method represents a significant improvement over existing segmentation models, with potential for real-world impact in medical applications.

\section{Methodology}

Our proposed methodology comprises of four key components: an extract block, 3D patch embeddings, 3D patch encodings, and a Transformer encoder coupled with an MLP decoder, which shows in figure \ref{fig:2}

\begin{figure}
    \centering
    \includegraphics[width=1\linewidth]{pipline.png} 
    \caption{Proposed pipeline}
    \label{fig:2}
\end{figure}

\subsection{Extract Block}

Suppose the shape of the input 3D images is $(channel,height,width,depth)$, we first extract the image into block,

\begin{equation}
  B \in \mathbb{R}^{W \times W \times W \times c}
  \label{eq:1}
\end{equation}

where  $W$ denotes the resolution of the block $B$, and $c$ represents the channel of images.

\subsection{Patch Embedding}

The block $B$ is subdivided into a finite number $ N = n^3$ of contiguous non-overlapping 3D patches,

\begin{equation}
  \{p_i \in \mathbb{R}^{w \times w \times w \times c}\}_{i=1}^N
  \label{eq:2}
\end{equation}

where $n$ represents the number of patches along each dimension of the block, and $ w = W/n $ stands for the resolution of each 3D patch.

The resulting patches are arranged in a specific order to form a sequence. Subsequently, each patch is flattened into a vector of length of $\mathbb{R}^{w^3c}$ and passed through a trainable embedding layer, $E \in \mathbb{R}^{D \times w^3c}$, to yield an embedded representation $E_P \in \mathbb{R}^{D}$.

Note that patches are formed into a sequence, so we can obtained an embedded representation matrix $[E_{P_1};...;E_{P_N}]$ based on the sequence. 

\subsection{Patch Encoding}

We can generate a patch encoding matrix $E_{pos} \in \mathbb{R}^{D \times N}$ using the sine and cosine position encoding functions which demonstrated in Transformer \cite{16}.

Later, we obtained the sequence after patch encoding:

\begin{equation}
  X^0 = [E_{P_1};...;E_{P_N}] + E_{pos}
  \label{eq:3}
\end{equation}

\subsection{Transformer Encoder}

The encoder is composed of a stack of $k$ identical layers (building block), and each building block has two sub-layers:

The first sub-layer is a multi-head self-attention layer consists of $h$ self-attention heads, where the query, key, and value matrices are represented as:

\begin{equation}
  Q = E_QX^k, K = E_KX^k, V = E_VX^k
  \label{eq:4}
\end{equation}

where $E_Q, E_K, E_V \in \mathbb{R}^{D_h \times D}$

Then the attention $A$ of self-attention head is calculated as:

\begin{equation}
  A(Q,K,V) = softmax(\frac{QK^T}{\sqrt{D_h}})V
  \label{eq:5}
\end{equation}

Consequently, the attention $MA$ of multi-head attention layer is calculated as:

\begin{equation}
  MA(Q,K,V) = Concat(Head_1,...,Head_h)W^O
  \label{eq:6}
\end{equation}

where

\begin{equation}
  Head_i = A(QW^Q_i,KW^K_i,VW^V_i)
  \label{eq:7}
\end{equation}

where projections are parameter matrices $W^Q_i, W^K_i, W^V_i$, and $W^O$.

The second sub-layer in the building block is fully connected feed-forward neural network (FFNN), which consists of two linear transformations with a ReLU activation in between.

\begin{equation}
  FFNN(x) = ReLU(xW_1 + b_1)W_2 + b_2
  \label{eq:8}
\end{equation}

The two sub-layer is connected via residual connection, followed by layer normalization. 

\begin{equation}
  Output_{Sublayer} = LayerNorm(x + Sublayer(x))
  \label{eq:9}
\end{equation}

\subsection{MLP Decoder}

The output of the Transformer encoder, $X^k$, will pass through a fully connected linear transformation layer to unify the feature maps into correct channel dimension, and predict mask through a Softmax function for the center patch of the block $B$.

\begin{equation}
  Mask = Softmax(X^kW + b)
  \label{eq:10}
\end{equation}

\subsection{Thin-thick adaptation}

We design a joint weighted loss for the domain adaptation of thin-thick scans, i.e. doing thin-slice segmentation using thick-annotations.

We first generate corresponding thick slices and their ground-truth annotations using given thin slices. The thick slices are obtained from average intensity projection (AIP) \cite{31} and thick anotations are labeled by radiologist. The joint weighted loss for back propagation will be:

\begin{itemize}
    \item Using the ground truth of thick slices for direct supervision, i.e. the prediction of the mask on the thick slices should be the exact ground truth.
    \item The average of the mask intensity of corresponding thin slices should be the ground truth of thick slices.
    \item The average of the feature maps of corresponding thin slices should be the feature map of thick slices.
\end{itemize}

\section{Evaluation Criteria}

\subsection{Evaluation Datasets}

The dataset used for evaluation will be categorized into two categories: (1) public datasets with thick slices and annotations, and (2) a private dataset comprising thin slices and corresponding thick slices and annotations. Notably, there is a lack of publicly available datasets with thin slices and ground truth annotations at present, necessitating the use of a private dataset.

In regards to the thick slice dataset, several publicly available datasets with thick annotations are available. One prominent example is the Medical Segmentation Decathlon \cite{32}, which offers a generalizable 3D semantic segmentation dataset consisting of various medical conditions such as liver tumors, brain tumors, lung tumors, colon cancer, pancreas tumors, and other similar conditions, totaling up to 10 different datasets.

Regarding the private dataset, we specifically aim to collect brain hemorrhage data as this condition is commonly scanned using thin slices. The dataset is also multi-semantic with five distinct labels, namely epidural hemorrhage (EDH), intracerebral hemorrhage (ICH), intraventricular hemorrhage (IVH), subarachnoid hemorrhage (SAH), and subdural hemorrhage (SDH) \cite{33}, making it a challenging segmentation task in the field of medical imaging analysis. The absence of publicly available datasets containing thin slices and corresponding ground truth annotations makes it imperative for us to gather our own data. Additionally, this dataset will be publicly released to address the current lack of public thin-slice data for medical imaging segmentation, and will serve as a new benchmark for multi-semantic segmentation in medical imaging.



\subsection{Experiment Design}

The experimental evaluation of the proposed model will be conducted in two phases: a comparative experiment and an ablation experiment. 

The comparative experiment will compare the performance of different 3D medical imaging segmentation models, including our proposed method, from two perspectives: thick segmentation based on thick annotations using public datasets, and thin segmentation based on thin annotations using private datasets. 

In the subsequent ablation experiment, a systematic analysis of the proposed method's individual components will be conducted to assess its contributions to the field. This will allow us to isolate each variation of the proposed method and provide evidence of their respective impacts, thus contributing to the broader medical imaging segmentation research community.

\subsection{Evaluation Matrices}

In the context of 3D semantic segmentation in medical imaging, two prominent evaluation metrics are used, namely mean Intersection over Union (mIoU) \cite{34} and Dice Similarity Coefficient (DSC) \cite{35}. The former is a measure of the intersection over union between the predicted and ground-truth segmentation masks, while the latter is a measure of the similarity between the predicted and ground-truth masks. These metrics are commonly employed to assess the accuracy and quality of segmentation results in medical imaging, and will be used in our proposed model evaluation as well.

\subsubsection{mIoU}

The mean Intersection over Union (mIoU) is a commonly used evaluation metric in general computer vision segmentation tasks. It measures the similarity between the predicted segmentation mask and the ground truth mask by calculating the intersection over union (IoU) for each class and then taking the average across all classes. The IoU is calculated as the ratio of the intersection between the predicted and ground truth mask to the union of both masks. The mIoU ranges from 0 to 1, where a value of 1 indicates a perfect segmentation and a value of 0 indicates no overlap between the predicted and ground truth masks.

\begin{equation}
  Intersection = GroundTruth \cap Prediction
  \label{eq:11}
\end{equation}

\begin{equation}
  Union = GroundTruth \cup Prediction
  \label{eq:12}
\end{equation}

\begin{equation}
  IoU = Intersection / Union
  \label{eq:13}
\end{equation}

\begin{equation}
  mIoU = mean(IoU_1, ...IoU_n)
  \label{eq:14}
\end{equation}

where $n$ is the number of semantic labels.

\subsubsection{DSC}

Dice Similarity Coefficient (DSC) is a widely used evaluation metric in medical imaging segmentation tasks. It measures the overlap between the predicted segmentation and the ground truth segmentation. The DSC ranges from 0 to 1, where 0 indicates no overlap and 1 indicates a perfect match between the predicted and ground truth segmentations. 

\begin{equation}
  DSC = (2 \times Intersection)/Union
  \label{eq:15}
\end{equation}

Where $Intersection$ represents the intersection of the predicted and ground truth segmentation, and $Union$ represents the Area of the predicted segmentation + Area of the ground truth segmentation.

It is a useful metric for evaluating the accuracy of segmentation models, especially in scenarios where the class imbalance is significant.


\section{Contributions}

\textbf{1: Data acquisition}

\begin{itemize}
    \item Obtain thin-slice data and generate thick slice data based on thin slices
    \item Annotate the thin slice data for multi-semantic segmentation on brain hemorrhage
\end{itemize}

\textbf{2: Model development}

\begin{itemize}
    \item Implement the multi-semantic segmentation model using transformers
    \item Implement the backpropagation of loss for thin-thick adaptation
\end{itemize}

\textbf{3: Experimentation and analysis}

\begin{itemize}
    \item Run comparative experiments on both public thick slice datasets and private thin slice datasets using existing models for comparison
    \item Analyze and evaluate the performance of the model in comparison to existing models
\end{itemize}

\textbf{4: Ablation studies}

\begin{itemize}
    \item Perform ablation studies to demonstrate the contribution of the proposed novel model
\end{itemize}

\textbf{5: Paper writing and review}

\begin{itemize}
    \item Write the paper detailing the work, methodology, results, and findings
    \item Submit the paper to the supervisor for review and feedback
    \item Revise the paper based on supervisor feedback and finalize the paper for submission
\end{itemize}

\section{Limitations and Potential Problems}

One potential limitation in 3D multi-semantic medical imaging segmentation is the long-tail problem \cite{36,37}, which refers to an extreme scenario of unbalanced data in machine learning classification tasks \cite{zhang2024jointvit,qi2025medconv}, especially for multi-label and multi-class classification \cite{wu2024xlip}. An ideal training dataset for a deep learning model should follow a universal distribution, rather than a normal distribution. In the case of mildly unbalanced data, random downsampling \cite{38} of the majority class can be used to balance the data. However, for cases where the gap between the minority and majority class is too large for downsampling, a weighted loss \cite{39} can be designed for backpropagation, where more weight is assigned to the minority class. Despite these approaches, long-tail problems remain challenging. However, the recent development of DiffuMask \cite{40}, a diffusion model for generating mask-image pairs, offers a potential solution for generating long-tail classes and can be employed in medical imaging segmentation for pre-training purposes.

The second potential problem in 3D multi-semantic medical imaging segmentation pertains to the inadequacy of pre-trained models in this domain as compared to general computer vision. Despite a vast collection of unlabeled images in public datasets, the unavailability of suitable pre-trained models poses a challenge. To overcome this, there is a huge potential for developing large-scale self-supervised or contrastive pre-training models based on existing techniques like MoCo \cite{42} or MAE \cite{43}. The pre-trained encoder obtained through this approach can be frozen and leveraged as a feature extractor to achieve effective medical image segmentation.

The third potential limitation and problem in 3D multi-semantic medical imaging segmentation is the ethical and security concerns surrounding data acquisition. Medical imaging data contains sensitive personal information of patients, and therefore, it is crucial to ensure the privacy and confidentiality of the patients' data. The medical community must adhere to strict ethical guidelines and regulations when collecting, storing, and sharing medical data.

One challenge in this area is determining what kind of data can be shared without compromising patients' privacy. There is a need for clear guidelines and policies on the types of medical imaging data that can be shared publicly or among researchers, and under what conditions. Moreover, since medical imaging datasets can be massive, there is a risk of data breaches and hacking, which can lead to the disclosure of patients' sensitive information.

To address these concerns, researchers should take extra precautions to ensure the security and privacy of medical imaging data, such as using secure storage systems, implementing data access controls, and anonymizing patient data. Additionally, obtaining informed consent from patients is essential before collecting and using their medical data for research purposes. By prioritizing the ethical and security concerns surrounding medical imaging data acquisition, researchers can ensure that they are using patient data in a responsible and transparent manner.

\section{Conclusions}

The aim of our research was to improve the accuracy of semantic segmentation on 3D medical images by developing a novel and efficient model that addresses the domain adaptation problem between thick and thin slices. To this end, we proposed a fully convolutional-free architecture based on transformers that can effectively capture global content of images and generate accurate semantic labels and segmentation masks. Our model employs self-attention mechanisms and other techniques to enhance feature extraction and decoding of masks, which makes it more efficient than existing convolutional-based models. Additionally, we designed a joint loss function that supervises the model to segment thin slices based on thick annotations, which improves segmentation accuracy. We also contributed a new benchmark dataset of thin slice multi-semantic segmentation, which is a valuable resource for researchers to evaluate and compare their models. These contributions are significant in advancing the state-of-the-art in medical image analysis and improving diagnostic medicine.

\section{Acknowledgements}
I would like to acknowledge the support of Zeyu Zhang and AI Geeks in contributing to this research project.

% \bibliographystyle{plain}
% \bibliography{main}

\documentclass{MITstyle}

%\usepackage[table]{xcolor}
\usepackage{chngcntr}
\usepackage{hyperref}
\usepackage{microtype}

\title{A Lightweight and Extensible Cell Segmentation and Classification Model for Whole Slide Images}

\author{Nikita Shvetsov~$^{1, }$\footnote{Correspondence e-mail: nikita.shvetsov@uit.no}, Thomas K. Kilvaer~$^{2, 3}$, Masoud Tafavvoghi~$^{4}$, Anders Sildnes~$^{1}$, \\ Kajsa Møllersen~$^{4}$, Lill-Tove Rasmussen Busund~$^{5, 6}$, Lars Ailo Bongo~$^{1}$ \\
%
\vspace{1em} % Space between authors and afilliations
%
\normalfont{\small $^{1}$Department of Computer Science, UiT The Arctic University of Norway}\\
\normalfont{\small $^{2}$Department of Oncology, University Hospital of North Norway}\\
\normalfont{\small $^{3}$Department of Clinical Medicine, UiT The Arctic University of Norway}\\
\normalfont{\small $^{4}$Department of Community Medicine, UiT The Arctic University of Norway}\\
\normalfont{\small $^{5}$Department of Medical Biology, UiT The Arctic University of Norway} \\
\normalfont{\small $^{6}$Department of Clinical Pathology, University Hospital of North Norway} %\vspace{2em}
}

\begin{document}
\maketitle

\section*{Abstract}

% \begin{abstract}
% Developing clinically useful cell-level analysis tools in digital pathology remains challenging due to limitations in dataset granularity, inconsistent annotations, computational demands of advanced models, and difficulties in integrating new technologies into clinical workflows. To address these challenges, we propose a multi-faceted solution that enhances data quality, model performance, and usability to create a lightweight and extensible cell segmentation and classification model.

% First, we update data labels by employing a cross-relabeling process that refines the labels of two existing datasets, PanNuke and MoNuSAC, to create a new unified dataset with enhanced granularity, encompassing seven distinct cell types. Second, we leverage the H-Optimus foundation model as a fixed encoder to improve feature representation for simultaneous cell segmentation and classification tasks. Third, to address the computational demands of foundation models, we employ knowledge distillation to reduce model size and complexity while maintaining comparable performance. Finally, to facilitate integration into clinical workflows, we integrate the distilled model into the QuPath software, a widely used open-source platform in digital pathology.

% Our results demonstrate improvements in cell segmentation and classification performance using the H‑Optimus-based model compared to a CNN-based model. Specifically, the average $R^2$ improved from 0.575 to 0.871, and the average $PQ$ score improved from 0.450 to 0.492, indicating better alignment with actual cell counts and enhanced segmentation and classification quality. Furthermore, the distilled student model maintains performance comparable to the larger foundation model while reducing the parameter count by a factor of 48.
% Overall, by reducing computational complexity and integrating it into existing workflows, the proposed approach may significantly impact diagnostic processes, reduce the workload of pathologists, and contribute to improved patient outcomes. Though our approach shows potential enhancements in efficiency and usability of cell segmentation and classification models in digital pathology, extensive validation is needed to deploy these models in clinical practice.
% \end{abstract}

%%% shortened abstract
\begin{abstract}
Developing clinically useful cell-level analysis tools in digital pathology remains challenging due to limitations in dataset granularity, inconsistent annotations, high computational demands, and difficulties integrating new technologies into workflows. To address these issues, we propose a solution that enhances data quality, model performance, and usability by creating a lightweight, extensible cell segmentation and classification model. 

First, we update data labels through cross-relabeling to refine annotations of PanNuke and MoNuSAC, producing a unified dataset with seven distinct cell types. Second, we leverage the H-Optimus foundation model as a fixed encoder to improve feature representation for simultaneous segmentation and classification tasks. Third, to address foundation models' computational demands, we distill knowledge to reduce model size and complexity while maintaining comparable performance. Finally, we integrate the distilled model into QuPath, a widely used open-source digital pathology platform. 

Results demonstrate improved segmentation and classification performance using the H-Optimus-based model compared to a CNN-based model. Specifically, average $R^2$ improved from 0.575 to 0.871, and average $PQ$ score improved from 0.450 to 0.492, indicating better alignment with actual cell counts and enhanced segmentation quality. The distilled model maintains comparable performance while reducing parameter count by a factor of 48. By reducing computational complexity and integrating into workflows, this approach may significantly impact diagnostics, reduce pathologist workload, and improve outcomes. Although the method shows promise, extensive validation is necessary prior to clinical deployment.
\end{abstract}
\clearpage

\section{Introduction}
In digital pathology, accurate segmentation and classification of cells are crucial for many diagnostic, prognostic, and predictive analyses \cite{Jaber_Beziaeva_etal._2019,Lin_Pan_etal._2022,Park_Ock_etal._2022,Shen_Choi_etal._2024}. Nowadays, developments in computational pathology offer multiple solutions \cite{H._Qu_P._Wu_etal._2020,Javed_Mahmood_etal._2020} to utilize cell-level datasets to train machine learning models that solve these problems. The quality and specificity of training datasets are critical for robust and accurate models. Adhering to the principle of "garbage in, garbage out", it is essential to ensure that these datasets are extensively and accurately labeled with distinct classes that reflect the diverse biological characteristics of different cell types. Unfortunately, the number of open-source datasets comprising such high-quality annotations is limited. Existing cell segmentation datasets \cite{Gamper_Koohbanani_etal._2019,Graham_Vu_etal._2019,Verma_Kumar_etal._2021} may offer extensive annotations for certain cell types while providing more general labels for others. For example, in PanNuke, which is one of the largest open-source datasets comprising labeled cells, various types of morphologically and functionally different inflammatory cells like macrophages and lymphocytes are clustered in a broad "inflammatory" class. Consequently, these classes are frequently omitted from analyses or aggregated into broader meta-classes \cite{Gamper_Koohbanani_etal._2020} and likely interfere with other cell classes included in the dataset. This and similar inconsistencies in annotation granularity limit the ability of machine learning models to learn the comprehensive and nuanced features necessary for accurate cell segmentation and classification. To address these challenges, methods for refining and standardizing dataset annotations are essential to enhance the quality of training data.

A complementary approach to mitigate the absence of high-quality training data is the use of foundation models. Foundation models as encoders are defined as large-scale, versatile networks pre-trained on vast, diverse datasets using self-supervised learning, contrasting with convolutional neural network (CNN) pre-trained encoders that rely on supervised learning with labeled data. In practice, foundation models leverage enormous amounts of weakly or unlabeled data from millions of whole slide images (WSIs) and employ self-attention mechanisms to capture long-range dependencies and global context \cite{Chen_Ding_etal._2024,Saillard_Jenatton_etal._2024,Vorontsov_Bozkurt_etal._2024,Xu_Usuyama_etal._2024}. As a consequence, foundation models are able to produce transferable feature representations across different cell types and tissue environments. The feature representations can be leveraged by decoder networks to produce segmentation masks and pixel-level classifications. Because foundation models have comprehensive feature representations, they can be effectively fine-tuned using much smaller amounts of cell-level data compared to the large datasets needed to train models from scratch. Furthermore, foundation models incorporate adversarial training elements or contrastive learning \cite{Chen_Ding_etal._2024,Xu_Usuyama_etal._2024}, enhancing their resilience and adaptability by exposing them to challenging and varied scenarios during training. This may result in more generalizable models, often making them well-suited for diverse and complex tasks in digital pathology.

Despite the inherent advantages of foundation models, their deployment for practical use faces its own obstacles. In particular, they require substantial computational power, financial investments and rigorous testing to ensure reliability and efficacy for a given task \cite{Akkus_Dangott_etal._2022,Dragomir_Cocuz_etal._2022,Go_2022,Jafri_Farooqui_etal._2024}. Moreover, while foundation models enhance feature representation and performance, they depend on the quality of available annotations for decoder fine-tuning and, like any other model, cannot resolve existing inconsistencies or ambiguities in data labels. Therefore, there remains a critical need for solutions that address both data quality and practical deployment considerations.
Further, integrating new technologies into existing clinical workflows often encounters resistance, as it necessitates adjustments to established diagnostic processes. So, there is a need to develop solutions that could be integrated into current practices, minimizing the burden on medical professionals to adopt new tools \cite{King_Williams_etal._2023}.

Existing solutions \cite{Goldsborough_Philps_etal._2024,Hörst_Rempe_etal._2024}, while addressing some aspects of these challenges, fall short in providing a comprehensive approach. To address the data quality and clinical deployment issues, we propose a multi-faceted solution that encompasses data refinement, model optimization, and integration with existing pathology tools (\hyperref[fig:fig1]{Figure 1}). The outcome is a lightweight cell segmentation and classification model that can be integrated into digital pathology workflows for practical clinical use.

\begin{figure}[h!]
    \centering
    \includegraphics[width=\textwidth, height=0.82\textheight, keepaspectratio]{images/Figure_1.pdf}
    \caption{Overview of the proposed solution, including 1) Data refinement using cross-relabeling, 2) Teacher model development and fine tuning, 3) Student model optimization with knowledge distillation and 4) Student model and QuPath integration}
    \label{fig:fig1}
\end{figure}
\clearpage

Our approach begins with preparing the data for the fine-tuning and training of the machine learning models. We create a refined dataset, acquired via cross-relabeling two cell-level datasets, enhancing annotation specificity and consistency of the labeled data. Subsequently, we create a cell segmentation and classification model based on the foundation model. We leverage the foundation model as a fixed encoder and fine-tune a decoder using the refined dataset to improve generalization across diverse tissue- and cell types.
To ensure that the model remains lightweight and deployable in a possibly resource-constrained environment, we employ knowledge distillation to approximate the functionality of the foundation model. Finally, to facilitate the practical application of our model in digital pathology workflows, we integrate it with the QuPath \cite{Bankhead_Loughrey_etal._2017} application. Each methodological component contributes to the overarching goal of enhancing model performance, generalizability, and usability in clinical settings.

The primary contributions of this paper are:
\begin{enumerate}
    \item \textit{Data labels refinement through cross-relabeling:}
    
    We propose a new method for refining labels of cell-level datasets through cross-relabeling. This method employs classification models to re-label broad and ambiguous instances, resulting in a more diverse dataset. Our evaluation demonstrates that these classification models achieve high accuracy on test subsets, indicating the reliability of the method for label refinement.

    \item \textit{Enhanced model performance via foundation models:}
    
    We employ a foundation model as a feature extractor for the cell segmentation and classification task. In comparison with training a CNN model from scratch, the foundation model backbone only needs fine-tuning, which significantly reduces training time, computational resources and data requirements. We show that using a foundation model encoder leads to better performance in cell segmentation and classification networks than using a CNN-based encoder. This improvement may enable the model to generalize more effectively across various tissue types and imaging methods.
    
    \item \textit{Model optimization through knowledge distillation:}
    
    We show that a smaller student model trained using knowledge distillation on the refined dataset obtained via our cross-relabeling approach from a foundation model achieves comparable performance in cell segmentation and quantification tasks. As a result, this model is more suitable for deployment in environments without high-performance computing resources.
    
    \item \textit{Integration with QuPath:}
    
    We integrate the distilled cell segmentation and classification model into QuPath, a widely used open-source digital pathology platform, to accelerate clinical adaptation by enabling pathologists to more easily incorporate advanced computational tools into their existing workflows.
\end{enumerate}

Through these methodological steps, we aim to bridge the gap between advanced machine learning techniques and practical clinical applications, making accurate and efficient digital pathology accessible in a broader range of healthcare settings.

\section{Refining Existing Datasets Using Cross-Relabeling}
To address the limitations of sparse and ambiguous labeling of cell-level datasets, we propose a generalizable cross-relabeling strategy that can be applied to any dataset containing broadly categorized or imprecisely labeled cell types. This approach involves training and subsequently leveraging classification models to refine broad categories into more specific or biologically relevant classes.
When applied to cell-level data, the methodology includes extracting individual cell images from the dataset patches, preprocessing these images to standardize the size and accommodate partial cells, and then training deep learning classifiers capable of distinguishing between the finer cell subtypes within the coarser categories. 
To illustrate our approach, we focus on the PanNuke \cite{Gamper_Koohbanani_etal._2020, Gamper_Koohbanani_etal._2019} and MoNuSAC \cite{Verma_Kumar_etal._2021} datasets that we have used to train models for cell quantification in our previous works \cite{Shvetsov_Grønnesby_etal._2022,Shvetsov_Sildnes_etal._2024}. We find that for better cell differentiation we have to introduce more granular labels. PanNuke includes a broad classification of "inflammatory" cells, encompassing lymphocytes, macrophages, and neutrophils. Each cell type differs significantly in structure, function, and clinical relevance. Conversely, MoNuSAC uses the label "epithelial" for a class that comprises both benign epithelial cells and malignant neoplastic cells. This practice makes it challenging to differentiate between benign and malignant epithelial cells in the dataset, which is a critical distinction when identifying tumor areas within tissue samples. To address these issues, we implement a cross-relabeling strategy as shown in \hyperref[fig:fig2]{Figure 2}. The key components are two classification models: one is trained on singular cell images from PanNuke data to classify the epithelial meta-class into epithelial and neoplastic classes. The other is trained on MoNuSAC to refine the inflammatory class into lymphocytes, neutrophils, and macrophages.

\begin{figure}[h!]
    \centering
    \includegraphics[width=\textwidth]{images/Figure_2.pdf}
    \caption{Refined dataset generation via cross relabeling}
    \label{fig:fig2}
\end{figure}

The refining approach consists of three consecutive steps. The first is the preprocessing step, in which we extract individual cells from both datasets (\hyperref[fig:fig3]{Figure 3}). The specifics of PanNuke and MoNuSAC patch preparation before cell preprocessing are provided in \hyperref[chap:S1]{Appendix S1}.

\begin{figure}[h!]
    \centering
    \includegraphics[width=\textwidth]{images/Figure_3.pdf}
    \caption{Cell instances preprocessing including (1) cell map extraction, (2) bounding box delineation, (3) adjusting cell boxes and (4) cropping and resizing of cell images}
    \label{fig:fig3}
\end{figure}

During preprocessing, we extract cell type maps from the ground truth label mask and calculate bounding boxes around each cell instance. To accommodate partial cells at patch borders, a common issue in cropped patch images, we employ mirror padding and extend the field of view of the cell label by 15 pixels to capture adjacent cells. We then crop and resize the identified regions to $64 \times 64$ pixels using bicubic interpolation.

The preprocessed PanNuke dataset comprises 68,031 neoplastic and 23,207 epithelial cell images, while MoNuSAC comprises  33,104 lymphocytes, 1,252 neutrophils, and 1,695 macrophages, which we subsequently use in training cell classification models and classifying the cell image data \hyperref[fig:S2]{Appendix Figure S2 (1)}. 

The next step is to train two distinct ResNet50-based classifiers tailored to address the specific labeling challenges inherent in each dataset. We use ResNet50 for classification models due to its proven effectiveness for image classification tasks in histopathology \cite{pan2022reviewmachinelearningapproaches}, and its compatibility with small images. For the PanNuke dataset, we design the classifier, trained on MoNuSAC data, to disaggregate the heterogeneous "inflammatory" cell category into distinct subtypes: lymphocytes, macrophages, and neutrophils. Similarly, for the MoNuSAC dataset, the classifier is trained on PanNuke data and distinguishes between benign and malignant epithelial cells within the overarching "epithelial" label. By applying these targeted classifiers to their respective datasets, we assign more specific labels to individual cell instances, thus enabling us to create a unified dataset.
To ensure a balanced representation of classes, we train both models on datasets that had been equalized to match the size of the least represented class. Thus, we obtain datasets comprising 23,207 samples per class for PanNuke and 1,252 samples per class for MoNuSAC data. Next, we partition both of them into training (70\%), validation (20\%), and testing (10\%) subsets. To mitigate the risk of overfitting, we use a single dropout layer with a rate of p=0.5 in both models and data augmentation using randomized color perturbations, rotation, and horizontal and vertical flipping. We employ AdamW optimizer and the cross-entropy loss function for the training criterion.

To evaluate the two trained models, we measure the classification accuracy on the respective test subsets. The accuracies on the test subset for both classifiers are presented in \hyperref[tab:1]{Table 1}. The PanNuke model achieves an average accuracy of 93.57\%, with higher accuracy for neoplastic cells (96.06\%) compared to epithelial cells (86.26\%). The confusion matrix in Figure A3.1 shows that the model predominantly distinguishes accurately between epithelial and neoplastic tissues, with a substantial number of correct classifications and relatively few misclassifications. The MoNuSAC model demonstrates an average accuracy of 98.92\%, excelling in classifying lymphocytes (99.67\%) and macrophages (94.12\%), with lower performance for neutrophils (85.71\%). The confusion matrix in Figure A3.2 shows that the model identifies lymphocytes and performs reasonably well with macrophages and neutrophils.

\begin{table}[h!]
\renewcommand{\arraystretch}{1.5}
  \centering
  \caption{Cell classification results for PanNuke and MoNuSAC trained models (CI 95\%).}
  \label{tab:1}
  \begin{tabular}{|l|c|c|}
   \hline
   %\rowcolor{gray!30}
    Accuracy               & PanNuke model              & MoNuSAC model              \\
    \hline
    Average      & 0.936 (0.931--0.941)         & 0.989 (0.986--0.993)        \\
    \hline
    Neoplastic   & 0.961 (0.956--0.965)         & -                          \\
    \hline
    Epithelial   & 0.863 (0.849--0.877)         & -                          \\
    \hline
    Lymphocytes  & -                          & 0.997 (0.995--0.999)        \\
    \hline
    Neutrophils  & -                          & 0.857 (0.796--0.918)        \\
    \hline
    Macrophages  & -                          & 0.941 (0.906--0.976)        \\
    \hline
  \end{tabular}
\end{table}

Finally, during the last step, we use the model trained on PanNuke data for epithelial cells in MoNuSAC and the model trained on MoNuSAC for the inflammatory cells class in PanNuke. Specifically, we use classifier models to relabel epithelial cells in MoNuSAC and inflammatory cells in PanNuke data. Then we combine cells with refined labels and the rest of the cells in both datasets to create a refined dataset (\hyperref[fig:S2]{Appendix Figure S2 (2)}). The process of relabeling cells and visualizing them on a patch is shown in \hyperref[fig:fig4]{Figure 4}. The cell counts in the refined dataset are provided in \hyperref[tab:S4]{Appendix Table S4}.

\begin{figure}[h!]
    \centering
    \includegraphics[width=\textwidth, height=0.42\textheight, keepaspectratio]{images/Figure_4.pdf}
    \caption{Cell relabeling procedure for epithelial and inflammatory cell classes}
    \label{fig:fig4}
\end{figure}

%\hfill

Relabeling and combining datasets have been explored in a prior study \cite{Parulekar_Kanwat_etal._2023}, where consecutive fine-tuning on multiple datasets was employed to account for hierarchical class label structures. While the method presented in \cite{Parulekar_Kanwat_etal._2023} is intuitive, it often lacks consistency and requires multiple fine-tuning runs, which can be cumbersome and time-consuming. 
In contrast, cross-relabeling simplifies this process by using specialized classification models tailored to each dataset's specific labeling challenges. This approach provides better transparency and produces a unified dataset encompassing seven distinct cell types across multiple tissue samples, enhancing data diversity for further model training or fine-tuning.

Despite these improvements, cross-relabeling does not entirely resolve issues related to poor labeling quality or the amount of labeled data. Specifically, our results show lower accuracies persist for underrepresented classes, such as macrophages, which may stem from a limited sample availability and intrinsic challenges in distinguishing these cells based solely on H\&E staining. Furthermore, while our method enhances label specificity, it relies on the initial quality of the broad labels; thus, any fundamental inaccuracies in the original annotations can propagate through the relabeling process. Addressing the overall problem of limited data labels may require integrating additional data sources or utilizing complementary immunohistochemical staining methods.
Although the reported performance metrics are obtained from evaluations on the native test sets of each dataset, it is important to note that the primary application of these classifiers is to perform cross-relabeling, where a model trained on one dataset (e.g., PanNuke) is applied to another (e.g., MoNuSAC) and vice versa. We acknowledge that a more systematic evaluation of cross-dataset generalization is needed and could be performed in future work.

Overall, the refined dataset produced by our approach can enhance the supervised training or fine-tuning of cell segmentation and classification models, especially those that utilize pre-trained foundation models to improve feature extraction robustness. In addition, these models can detect nuanced classes that enable researchers to conduct more detailed analyses of biological processes in computational pathology.

\section{Foundation models for robust cell segmentation and classification}

Accurate cell segmentation and classification in digital pathology are hindered by limited labeled data and the fact that conventional CNNs are unable to capture global contextual information due to their local receptive field constraints \cite{Gheflati_Rivaz_2022,Yang_Marcus_etal.}. Traditional approaches in cell quantification have predominantly relied on CNN encoders, such as ResNet50, given their proven effectiveness in semantic segmentation tasks \cite{Deshmane_2023,Graham_Vu_etal._2019,Mukasheva_Koishiyeva_etal._2024,Stringer_Wang_etal._2021}. However, approaches that include fine-tuning of pretrained CNNs, data augmentation, and stain normalization to partially increase data variability and address staining differences often fail to achieve the necessary generalization and robustness across diverse tissue types and staining conditions \cite{G._Wang_W._Li_etal._2018,Gao_Bagci_etal._2018,Karim_El_Khoury_Martin_Fockedey_etal._2021}.

To overcome these challenges, we leverage an encoder-decoder network that uses a foundation model as the encoder and a CNN upsampling decoder (\hyperref[fig:fig5]{Figure 5}) for simultaneous cell segmentation and classification in 2D patches extracted from WSIs. Foundation models with transformer-based architectures are viable alternatives to CNN-based encoders \cite{Shamshad_Khan_etal._2023,Sourget_2023}. They enable the creation of more advanced architectures that can decode or transform learned features more effectively \cite{Chen_Duan_etal._2023,Cheng_Misra_etal._2022,Xie_Wang_etal._2021}.

\begin{figure}[h!]
    \centering
    \includegraphics[width=\textwidth]{images/Figure_5.pdf}
    \caption{UNETR-like model with foundational model as backbone}
    \label{fig:fig5}
\end{figure}

By utilizing a transformer-based encoder, we incorporate global contextual information into the feature extraction process, which is a key advantage of such architectures \cite{Chen_Lu_etal._2021}. This foundation model integration facilitates accurate pixel-wise segmentation and classification without the need for extensive encoder training, thereby potentially improving generalization across varied cellular structures and tissue types.
In our implementation, we employ a modified UNETR \cite{Hatamizadeh_Tang_etal._2021} architecture that combines a vision transformer (ViT) \cite{Dosovitskiy_Beyer_etal._2021} encoder with a CNN-based decoder. The encoder utilizes the pretrained H-Optimus foundation model, which contains 1.1 billion parameters and is trained on over 500,000 H\&E stained WSIs \cite{Saillard_Jenatton_etal._2024}. We extract outputs from four evenly spaced transformer blocks $Z_i$, where $i \in [1, 14, 26, 38]$, to serve as residual connections for the CNN decoder. We select these blocks based on our observation that features from non-adjacent levels of the encoder lead to better overall performance on the test subset.

The CNN decoder upsamples the feature representations, acquired from the transformer blocks, to generate an intermediate vector that is handled by two task-specific layers that generate cell segmentation and classification masks. The first task-specific layer is the ‘Cellpose head’,  which is used to delineate cell instances. The layer generates horizontal and vertical gradient maps to form vector fields that are refined through gradient tracking in a post-processing step using the Cellpose algorithm \cite{Stringer_Wang_etal._2021}, known for its efficacy in cell segmentation tasks and generalizability across multiple domains \cite{Pachitariu_Stringer_2022,Stringer_Pachitariu_2024}. The second task-specific layer is the "Cell type head", which assigns labels to individual pixels. In the post-processing step, we determine the output classification label of each segmented cell instance by majority voting over the labeled pixels that comprise the cell in the segmentation map.

To evaluate model performance and measure the impact of adding a foundation model as backbone, we compare it to a ResNet50-based model. ResNet50 is a widely used solution for encoders in segmentation architectures in the medical domain \cite{Deshmane_2023,Graham_Vu_etal._2019,Mukasheva_Koishiyeva_etal._2024,Stringer_Wang_etal._2021}. For the H-Optimus-based model, we utilize frozen weights for the encoder and only fine-tune the decoder to take advantage of the extensive pre-training of the foundation model. For the ResNet50-based model we start with ImageNet \cite{Deng_Dong_etal.} weights and train both encoder and decoder parts. Hyperparameters for the training step are set to be identical, where possible, for comparable evaluation. 
For this evaluation, we deliberately use the PanNuke dataset to provide a standardized and controlled comparison between the H‑Optimus and ResNet50-based models (\hyperref[fig:S2]{Appendix Figure S2 (3)}). Specifically, we use two of the default PanNuke dataset splits (66\%) for training and validation, and reserve the third split (33\%) for testing.

To address the challenge of cell class imbalance in the PanNuke dataset, which is a common characteristic in most cell-level H\&E patch datasets, both models’ training processes employ a weighted loss function comprising cross-entropy and focal loss \cite{Lin_Goyal_etal._2018}. The focal loss component is adjusted with coefficients derived from each cell class' instance frequency, emphasizing learning from underrepresented classes and enhancing the model's sensitivity to rare but significant cellular patterns. The cross-entropy loss is augmented with spectral decoupling regularization \cite{Pezeshki_Kaba_etal._2021,Pohjonen_Stürenberg_etal._2022} and spatially varying label smoothing \cite{Islam_Glocker_2021}, which potentially stabilizes training and improves generalization in case of complex tissue morphologies. For optimization, we employ the \textit{AdamW} \cite{Loshchilov_Hutter_2019} to counter unbalanced class scenarios, with cosine annealing learning rate scheduler.

We utilize the scikit-learn library \cite{Van_der_Walt_Schönberger_etal._2014} and HoVer-Net \cite{Graham_Vu_etal._2019} implementations of $R^2$ (the coefficient of determination) and $PQ$ (panoptic quality) to evaluate our experiments. Complete mathematical formulations and detailed explanations of these metrics are provided in \hyperref[chap:S5]{Appendix S5}. To compute confidence intervals, we use nonparametric bootstrapping, where after calculating the metric on the full sample, we generated 1000 bootstrap replicates by resampling with replacement and then determined the 95\% confidence intervals as the 2.5th and 97.5th percentiles of the resulting empirical distribution.

%\hfill

The model comparisons are summarized in \hyperref[tab:2]{Table 2}. The H‑Optimus-based model achieves higher $R^2$ across all cell classes compared to the ResNet50-based model, which means that its predictions are more closely aligned with the PanNuke cell counts, indicating a stronger correlation with the observed data. Notably, the improvement of $R^2_{dead}$ may be an indicator of better global contextual representations provided by the foundation model backbone. In terms of segmentation and classification quality combined, measured by the PQ score, the H‑Optimus-based model demonstrates notable improvements across most cell classes. Overall, the average $R^2$ improved from 0.575 to 0.871, while the average $PQ$ score improved from 0.450 to 0.492, demonstrating better performance of the H-Optimus-based model.

\begin{table}[h!]
\renewcommand{\arraystretch}{1.5}
  \centering
  \caption{Cell quantification metrics for baseline and proposed models (CI 95\%).}
  \label{tab:2}
  \begin{tabular}{|l|c|c|}
    \hline
    %\rowcolor{gray!30}
    Metric             & Resnet50-based            & H-optimus-based              \\
    \hline
    $R^2_{neoplastic}$    & 0.681 (0.576--0.769)       & \textbf{0.941 (0.917--0.960)} \\
    \hline
    $R^2_{inflammatory}$  & 0.863 (0.778--0.903)       & \textbf{0.949 (0.918--0.966)} \\
    \hline
    $R^2_{connective}$    & 0.600 (0.488--0.698)       & 0.609 (0.436--0.772)          \\
    \hline
    $R^2_{dead}$          & 0.097 (-11.389--0.669)     & 0.925 (0.404--0.982)          \\
    \hline
    $R^2_{epithelial}$    & 0.635 (0.490--0.747)       & \textbf{0.930 (0.886--0.964)} \\
    \hline
    $PQ_{neoplastic}$       & 0.517 (0.499--0.535)       & \textbf{0.589 (0.575--0.604)} \\
    \hline
    $PQ_{inflammatory}$     & 0.455 (0.429--0.482)       & \textbf{0.528 (0.507--0.549)} \\
    \hline
    $PQ_{connective}$       & 0.416 (0.400--0.431)       & \textbf{0.451 (0.436--0.465)} \\
    \hline
    $PQ_{dead}$             & 0.374 (0.342--0.408)       & 0.292 (0.209--0.365)          \\
    \hline
    $PQ_{epithelial}$       & 0.488 (0.460--0.519)       & \textbf{0.599 (0.579--0.618)} \\
    \hline
  \end{tabular}
\end{table}

Our results  show that integrating the H‑Optimus foundation model within the UNETR architecture enhances the model's ability to segment and classify cells across diverse tissues from PanNuke data. The pretrained transformer encoder provides robust feature representations, resulting in higher average $R^2$ and $PQ$ scores compared to the CNN-based model. This leads to more reliable cell quantification and more accurate downstream analysis. Additionally, the streamlined fine-tuning process reduces computational overhead and training time, making the model more adaptable for new data.

Despite these advancements, the foundation model-based approach does not fully resolve all challenges related to cell segmentation and classification. We observe lower metric scores for underrepresented classes in the training data. Furthermore, foundation models typically encompass billions of parameters, resulting in substantial computational and memory requirements. It therefore poses challenges for deployment in resource-constrained environments, limiting their practical applicability in certain clinical settings.

\section{Model optimization via Knowledge Distillation}

To address the limitations posed by the extensive size of foundation models, we implement knowledge distillation — a model compression technique that leverages the teacher-student paradigm \cite{Hinton_Vinyals_etal._2015}. By training a smaller, more efficient student model to replicate the output of a larger, pre-trained teacher model, we retain performance while significantly reducing the model's complexity and resource requirements (\hyperref[fig:fig6]{Figure 6}).

\begin{figure}[h!]
    \centering
    \includegraphics[width=\textwidth, height=0.45\textheight, keepaspectratio]{images/Figure_6.pdf}
    \caption{Knowledge distillation framework for training a student model using a pre-trained teacher}
    \label{fig:fig6}
\end{figure}

We employ knowledge distillation to compress the H‑Optimus-based teacher model into a more efficient student model. The teacher model is the modified UNETR architecture with the H‑Optimus foundation model described in the previous chapter. The student model is based on a UNet architecture augmented with residual connections and incorporates a smaller ViT encoder with 9 million parameters \cite{Steiner_Kolesnikov_etal._2022,Wightman_2019}. 

First, we fine-tune the teacher model using the refined dataset from the cross-relabeling procedure (Section 2). Initially we train the decoder of the teacher model while keeping the encoder weights frozen. We split the refined dataset into train (70\%), validation (20\%) and test (10\%) subsets (\hyperref[fig:S2]{Appendix Figure S2 (4)}). During fine-tuning, we use the train and validation subsets, while leaving the test subset for model evaluation. We set the training procedure and model hyperparameters to be identical to those that were used to demonstrate the utility of foundation models for the simultaneous cell segmentation and classification task.

Next, we perform knowledge distillation from teacher to student using the refined dataset used to fine-tune the teacher model. The student model is trained to replicate the teacher model's outputs. We utilize a specialized loss function that aligns the student's predicted probability distribution with the teacher's, incorporating the teacher's class probability distribution derived from the output. Following the methodology of Hinton et al. \cite{Hinton_Vinyals_etal._2015}, we experiment with various hyperparameter settings for the temperature ($T$) and the balancing coefficients ($\alpha$ and $\beta$) in the loss function. We vary $T$ from 1 to 20 and adjust $\alpha$ and $\beta$ to balance the distillation and student losses. Through iterative tuning and evaluation, we identify that setting $T=14$, $\alpha=0.3$, and $\beta=0.7$ yields a configuration that converges and closely approximates the teacher model's performance during training.

Finally, we assess the performance of both models using the $R^2$ and $PQ$ (defined in \hyperref[chap:S5]{Appendix S5}) on the test set of the refined dataset (\hyperref[tab:3]{Table 3}). We observe that the 95\% confidence intervals overlap for most cell types, so we cannot claim statistically significant performance differences between the teacher and student models. One exception appears in the neoplastic class. The teacher model produces an $R^2$ of 0.919, while the student model shows an $R^2$ of 0.852. In addition, the student model achieves higher $PQ$ values for the neoplastic and connective classes, though the confidence intervals show overlap.

\begin{table}[h!]
\renewcommand{\arraystretch}{1.5}
  \centering
  \caption{Cell quantification metrics for teacher and distilled student models (CI 95\%).}
  \label{tab:3}
  \begin{tabular}{|l|c|c|}
    \hline
    %\rowcolor{gray!30}
    Metric & Teacher & Student \\
    \hline
    $R^2_{neoplastic}$    & \textbf{0.919} (0.898--0.939) & 0.852 (0.800--0.891) \\
    \hline
    $R^2_{lymphocyte}$    & 0.969 (0.956--0.977)         & 0.969 (0.956--0.978) \\
    \hline
    $R^2_{connective}$    & 0.694 (0.548--0.809)         & 0.618 (0.469--0.741) \\
    \hline
    $R^2_{dead}$          & 0.755 (0.400--0.908)         & 0.424 (0.100--0.731) \\
    \hline
    $R^2_{epithelial}$    & 0.922 (0.870--0.958)         & 0.843 (0.738--0.917) \\
    \hline
    $R^2_{macrophage}$    & 0.384 (-0.369--0.724)        & 0.704 (0.352--0.859) \\
    \hline
    $R^2_{neutrofil}$     & 0.854 (0.578--0.929)         & 0.833 (0.502--0.925) \\
    \hline
    $PQ_{neoplastic}$       & 0.581 (0.569--0.593)         & 0.601 (0.588--0.613) \\
    \hline
    $PQ_{lymphocyte}$       & 0.536 (0.520--0.553)         & 0.563 (0.544--0.579) \\
    \hline
    $PQ_{connective}$       & 0.436 (0.421--0.451)         & 0.457 (0.441--0.474) \\
    \hline
    $PQ_{dead}$             & 0.272 (0.235--0.315)         & 0.279 (0.201--0.369) \\
    \hline
    $PQ_{epithelial}$       & 0.522 (0.500--0.545)         & 0.530 (0.506--0.555) \\
    \hline
    $PQ_{macrophage}$       & 0.524 (0.459--0.588)         & 0.474 (0.405--0.543) \\
    \hline
    $PQ_{neutrofil}$        & 0.541 (0.490--0.592)         & 0.565 (0.522--0.607) \\
    \hline
  \end{tabular}
\end{table}


We further decompose the $PQ$ metric into its $SQ$ and $DQ$ components (\hyperref[tab:S6]{Appendix Table S6}). Both models produce nearly identical $SQ$ values, which indicates that they predict instance boundaries with similar precision. Although the student model shows some improvement in $DQ$ scores for certain classes, the confidence intervals overlap and do not confirm a statistically significant difference.

We observe that the student and teacher models yield comparable detection performance despite the student model using a much smaller and simpler architecture. A model with fewer parameters reduces the risk of overfitting when training data are scarce relative to the model’s complexity \cite{Farias_Ludermir_etal._2022}. The knowledge distillation process also encourages the student model to focus on the most generalizable detection features learned from the teacher. These factors enable the student model to achieve similar detection performance across different cell types.

Additionally, considering the model sizes reported in \hyperref[tab:4]{Table 4}, the distilled model achieves a significant reduction compared to the teacher model, with a 48-fold decrease in parameter count and a 5.5-fold reduction in on-disk size. In inference mode, the teacher model requires 16 GB of VRAM for a batch size of 32, while the distilled model only needs 3 GB of VRAM for the same batch size. These reductions make the distilled model significantly more practical for fine-tuning and deployment in resource-constrained environments.

\begin{table}[h!]
\renewcommand{\arraystretch}{1.5}
  \centering
  \caption{Parameter counts and size of teacher and distilled model}
  \label{tab:4}
  \adjustbox{max width=\textwidth}{%
  \begin{tabular}{|l|c|c|c|}
    \hline
    %\rowcolor{gray!30}
    Metric & H-optimus-based (Teacher) & mobileViT-based (Student) & Magnitude of difference \\
    \hline
    Parameters count       & 1,158,917,906   & \textbf{24,093,393}   & \textbf{48x}  \\
    \hline
    Estimated Total Size (MB) & 87,912       & \textbf{15,935}    & \textbf{5.5x} \\
    \hline
  \end{tabular}%
}
\end{table}

%\hfill

With recent advancements in complex network architectures and the use of pretrained encoders to achieve state-of-the-art performance \cite{Baumann_Dislich_etal._2024,Hörst_Rempe_etal._2024} in cell segmentation and classification tasks, model size, computational complexity, and processing times have increased. This limits the scalability and accessibility of these models. As we demonstrate, this may be mitigated using knowledge distillation. Studies in the field of natural language processing have demonstrated the efficacy of knowledge distillation in retaining the capabilities of the teacher model while achieving significant reductions in size and complexity \cite{Huangpu_Gao_2024,Sun_Yu_etal.}. 

We demonstrate the feasibility of knowledge distillation in digital pathology, specifically for cell segmentation and classification tasks. Moreover, we achieve this performance while also significantly reducing the parameter count. In addressing the challenge of knowledge transfer, we found that distillation from a transformer-based model to a smaller transformer is more straightforward than attempting to map transformer features to CNN blocks. In our experiments, using a CNN-based network as a student results in worse cell quantification performance due to the structural constraints of CNN feature space dimensions. 

Although our primary approach relies on a transformer-based student model that performs well, it can be further optimized to incorporate advantages from CNN architectures. For example, employing alternative techniques such as using ViT adapters \cite{Chen_Duan_etal._2023} or $1 \times 1$ convolutions to adjust feature map sizes may be beneficial for harnessing CNN advantages like enhanced local feature extraction. Moreover, if additional performance improvements are desired, the process can be further enhanced by applying supplementary knowledge distillation techniques, such as self-distillation \cite{Zhang_Song_etal._2019} or online distillation \cite{Houyon_Cioppa_etal._2023}.

Despite these promising results, further validation on independent datasets is necessary to fully understand the model's limitations. Underrepresented classes may pose challenges when addressing complex cases. Pathologists need to validate these models to adopt them in clinical settings. While the distilled models are smaller and more deployable, a technological gap persists because pathologists traditionally rely on established methods for inspecting WSIs and diagnosing diseases. Addressing the complexities involved in deploying models for inference and supporting pathologists in adopting new tools is essential for integrating these models into clinical workflows.

\section{Model integration with QuPath}
Digital pathology tools with graphical user interfaces are essential for visualizing and analyzing WSIs. To make our student model useful in clinical pathology workflows, it needs to be integrated into a tool that enables inspecting regions, creating annotations, and providing quantitative analyses of biomarkers. Therefore, we integrate the trained student model from the previous chapter into the QuPath open‑source platform \cite{Bankhead_Loughrey_etal._2017}. QuPath provides the required annotation, visualization, and analysis tools to interpret complex histological data, including workflows for cell segmentation, classification, and quantification (\hyperref[fig:fig7]{Figure 7}). 

\begin{figure}[h!]
    \centering
    \includegraphics[width=\textwidth]{images/Figure_7.pdf}
    \caption{Visualization of model-generated cell quantification annotations (left) and the corresponding unannotated slide (right) in QuPath}
    \label{fig:fig7}
\end{figure}

To identify the regions in a WSI critical for prognosticating tumor development, such as specific tumor areas or border regions without overlapping healthy tissue, the pathologist uses QuPath to outline these regions. Then, the pathologist initiates a cell segmentation and classification script through the QuPath interface for the selected regions. The resulting annotations and quantified cell information are then directly overlaid onto the WSI in the QuPath interface. Additional design and implementation details are in \hyperref[chap:S7]{Appendix S7}. 

Two common approaches for integrating deep learning models into QuPath are Java‑based native QuPath extensions \cite{Goldsborough_Philps_etal._2024} and the execution of RESTful API requests to a model server coupled with handling the response via an extension, as demonstrated in the application of cell segmentation models applied to immunofluorescence images \cite{Sugawara_2023}. While the community is actively working on these integration strategies, there is currently no universal solution that fully addresses all integration and performance requirements.

Extensions may offer better integration with QuPath, allowing slightly improved performance and more widespread usage of the built-in QuPath models, but they lack the flexibility to customize models and modify their behavior. For example, the newest version of QuPath includes models such as StarDist \cite{Weigert_Schmidt} and InstanSeg \cite{Goldsborough_Philps_etal._2024} that can perform cell segmentation. Both models pose limitations when applied to simultaneous cell segmentation and classification. StarDist performs well only on convex, round shapes by design, whereas some neoplastic, inflammatory, and connective cells exhibit complex and non-convex shapes. InstanSeg provides only semantic segmentation without assigning classes to the segmented cells.

%\hfill

In contrast, our approach offers an alternative integration strategy. It utilizes the paquo library to directly interact with QuPath’s internal application programming interface from within Python. This enables data exchange and processing without the need for intermediate conversion steps and provides greater control over model customization, retraining, and the incorporation of custom processing steps.

The integration of our custom model with QuPath underscores its potential to significantly enhance the diagnostic process by reducing the time burden on pathologists and enabling them to focus on more complex interpretative tasks using familiar software. Leveraging a tool that is already well-established among pathologists increases the likelihood of its adoption into daily clinical workflows. The quantitative data generated through the automated workflow is critical for both clinical decision-making and research, facilitating more accurate biomarker analysis, enabling robust statistical evaluations, and supporting hypothesis generation and testing. Additionally, by streamlining cell segmentation and classification, the tool enhances the scalability and reproducibility of pathological assessments, ultimately contributing to improved diagnostic accuracy and patient outcomes.

\section{Conclusion and future work}

In this study, we address critical challenges in digital pathology and tackle the usability and deployment issues of the developed models in standard computing environments without the need for high-performance computing systems. Our multi-faceted approach encompasses data refinement through cross-relabeling, leveraging foundation models for robust cell segmentation and classification, optimizing model performance via knowledge distillation, and integrating the optimized model into the QuPath software for practical application. This approach is used to construct a capable, versatile, and adjustable model for cell segmentation and classification, with enhanced performance and usability.

\begin{sloppypar}
While our approach shows potential in the field of computational pathology, certain limitations persist. 
For example, our implementation currently exhibits lower performance in detecting macrophages. 
This serves as an instance of the broader challenge of accurately identifying complex cell types. In order to address this issue, extending our approach to incorporate additional data sources, exploring alternative modeling approaches, and integrating other imaging modalities such as immunohistochemical staining may help improve detection accuracy. Moreover, although the distilled model reduces computational demands, integrating advanced deep learning models into clinical practice requires addressing technological gaps and potential resistance to adopting new tools within established diagnostic processes.
\end{sloppypar}

Future work could focus on several key areas to refine the proposed approach and facilitate its adoption in clinical environments. Enhancing the cell-relabeling process with additional datasets \cite{Graham_Jahanifar_etal._2021} could improve the representation of underrepresented cell types and enhance overall model performance. Also, incorporating additional data sources, such as multi-modal imaging or complementary staining methods, may address limitations related to cell type differentiation and class imbalance. Exploring other foundation models \cite{Vorontsov_Bozkurt_etal._2024,Zimmermann_Vorontsov_etal._2024} or introducing additional modalities \cite{Ding_Wagner_etal._2024,Vaidya_Zhang_etal._2025} may provide alternative architectures better suited to specific tasks or offer improved efficiency. Implementing more complex knowledge distillation techniques \cite{Houyon_Cioppa_etal._2023,Zhang_Song_etal._2019} could further optimize the model's performance and adaptability. Additionally, deeper integration with QuPath or other digital pathology software could provide pathologists more control over cell quantification analysis directly within the QuPath interface, thereby increasing accessibility and usability. Such enhancements would not only refine model performance but also ensure greater adaptability and scalability within various clinical environments. Finally, extensive validation of the model by pathologists and benchmarking against independent datasets are essential steps toward establishing the model's reliability and fostering confidence in its clinical utility.

\section*{Acknowledgments} 
This work was funded in part by the Research Council of Norway grant no. 309439 SFI Visual Intelligence, and the North Norwegian Health Authority grant no. HNF1521-20.

\bibliographystyle{IEEEtran}
\begin{sloppypar}
\begin{thebibliography}{99}

\bibitem{chaplot2020neural} Chaplot, Devendra Singh, et al. "Neural topological slam for visual navigation." Proceedings of the IEEE/CVF conference on computer vision and pattern recognition. 2020.

\bibitem{maksymets2021thda} Maksymets, Oleksandr, et al. "Thda: Treasure hunt data augmentation for semantic navigation." Proceedings of the IEEE/CVF International Conference on Computer Vision. 2021.

\bibitem{mezghan2022memory} Mezghan, Lina, et al. "Memory-augmented reinforcement learning for image-goal navigation." 2022 IEEE/RSJ International Conference on Intelligent Robots and Systems (IROS). IEEE, 2022.

\bibitem{al2022zero} Al-Halah, Ziad, Santhosh Kumar Ramakrishnan, and Kristen Grauman. "Zero experience required: Plug \& play modular transfer learning for semantic visual navigation." Proceedings of the IEEE/CVF Conference on Computer Vision and Pattern Recognition. 2022.

\bibitem{ye2021auxiliary} Ye, Joel, et al. "Auxiliary tasks and exploration enable objectgoal navigation." Proceedings of the IEEE/CVF international conference on computer vision. 2021.

\bibitem{chaplot2020object} Chaplot, Devendra Singh, et al. "Object goal navigation using goal-oriented semantic exploration." Advances in Neural Information Processing Systems 33 (2020)

\bibitem{ramakrishnan2022poni} Ramakrishnan, Santhosh Kumar, et al. "Poni: Potential functions for objectgoal navigation with interaction-free learning." Proceedings of the IEEE/CVF Conference on Computer Vision and Pattern Recognition. 2022.

\bibitem{ramrakhya2022habitat} Ramrakhya, Ram, et al. "Habitat-web: Learning embodied object-search strategies from human demonstrations at scale." Proceedings of the IEEE/CVF Conference on Computer Vision and Pattern Recognition. 2022.

\bibitem{mousavian2019visual} Mousavian, Arsalan, et al. "Visual representations for semantic target driven navigation." 2019 International Conference on Robotics and Automation (ICRA). IEEE, 2019.

\bibitem{dhariwal2021diffusion} Dhariwal, Prafulla, and Alexander Nichol. "Diffusion models beat gans on image synthesis." Advances in neural information processing systems 34 (2021)

\bibitem{ho2022classifier} Ho, Jonathan, and Tim Salimans. "Classifier-free diffusion guidance." arXiv preprint arXiv:2207.12598 (2022).

\bibitem{nichol2021glide} Nichol, Alex, et al. "Glide: Towards photorealistic image generation and editing with text-guided diffusion models." arXiv preprint arXiv:2112.10741 (2021)

\bibitem{brooks2023instructpix2pix} Brooks, Tim, Aleksander Holynski, and Alexei A. Efros. "Instructpix2pix: Learning to follow image editing instructions." Proceedings of the IEEE/CVF Conference on Computer Vision and Pattern Recognition. 2023.

\bibitem{fu2023guiding} Fu, Tsu-Jui, et al. "Guiding instruction-based image editing via multimodal large language models." arXiv preprint arXiv:2309.17102 (2023).

\bibitem{geng2024instructdiffusion} Geng, Zigang, et al. "Instructdiffusion: A generalist modeling interface for vision tasks." Proceedings of the IEEE/CVF Conference on Computer Vision and Pattern Recognition. 2024.

\bibitem{zhou2024minedreamer} Zhou, Enshen, et al. "Minedreamer: Learning to follow instructions via chain-of-imagination for simulated-world control." arXiv preprint arXiv:2403.12037 (2024).

\bibitem{zhou2023esc} Zhou, Kaiwen, et al. "Esc: Exploration with soft commonsense constraints for zero-shot object navigation." International Conference on Machine Learning. PMLR, 2023.

\bibitem{yu2023l3mvn} Yu, Bangguo, Hamidreza Kasaei, and Ming Cao. "L3mvn: Leveraging large language models for visual target navigation." 2023 IEEE/RSJ International Conference on Intelligent Robots and Systems (IROS). IEEE, 2023.

\bibitem{gadre2023cows} Gadre, Samir Yitzhak, et al. "Cows on pasture: Baselines and benchmarks for language-driven zero-shot object navigation." Proceedings of the IEEE/CVF Conference on Computer Vision and Pattern Recognition. 2023.

\bibitem{shah2023navigation} Shah, Dhruv, et al. "Navigation with large language models: Semantic guesswork as a heuristic for planning." Conference on Robot Learning. PMLR, 2023.

\bibitem{cai2024bridging} Cai, Wenzhe, et al. "Bridging zero-shot object navigation and foundation models through pixel-guided navigation skill." 2024 IEEE International Conference on Robotics and Automation (ICRA). IEEE, 2024.

\bibitem{yu2023co} Yu, Bangguo, Hamidreza Kasaei, and Ming Cao. "Co-NavGPT: Multi-robot cooperative visual semantic navigation using large language models." arXiv preprint arXiv:2310.07937 (2023).

\bibitem{wu2024voronav} Wu, Pengying, et al. "Voronav: Voronoi-based zero-shot object navigation with large language model." arXiv preprint arXiv:2401.02695 (2024).

\bibitem{qin2023mp5} Qin, Yiran, et al. "Mp5: A multi-modal open-ended embodied system in minecraft via active perception." arXiv preprint arXiv:2312.07472 (2023).

\bibitem{du2024learning} Du, Yilun, et al. "Learning universal policies via text-guided video generation." Advances in Neural Information Processing Systems 36 (2024).

\bibitem{ajay2024compositional} Ajay, Anurag, et al. "Compositional foundation models for hierarchical planning." Advances in Neural Information Processing Systems 36 (2024).

\bibitem{liang2024skilldiffuser} Liang, Zhixuan, et al. "Skilldiffuser: Interpretable hierarchical planning via skill abstractions in diffusion-based task execution." Proceedings of the IEEE/CVF Conference on Computer Vision and Pattern Recognition. 2024.

\bibitem{heusel2017gans} Heusel, Martin, et al. "Gans trained by a two time-scale update rule converge to a local nash equilibrium." Advances in neural information processing systems 30 (2017).

\bibitem{zhang2018unreasonable} Zhang, Richard, et al. "The unreasonable effectiveness of deep features as a perceptual metric." Proceedings of the IEEE conference on computer vision and pattern recognition. 2018.

\bibitem{brown2020language} Brown, Tom B. "Language models are few-shot learners." arXiv preprint arXiv:2005.14165 (2020).

\bibitem{podell2023sdxl} Podell, Dustin, et al. "Sdxl: Improving latent diffusion models for high-resolution image synthesis." arXiv preprint arXiv:2307.01952 (2023).

\bibitem{brohan2022rt} Brohan, Anthony, et al. "Rt-1: Robotics transformer for real-world control at scale." arXiv preprint arXiv:2212.06817 (2022).

\bibitem{brohan2023rt} Brohan, Anthony, et al. "Rt-2: Vision-language-action models transfer web knowledge to robotic control." arXiv preprint arXiv:2307.15818 (2023).

\bibitem{li2024manipllm} Li, Xiaoqi, et al. "Manipllm: Embodied multimodal large language model for object-centric robotic manipulation." Proceedings of the IEEE/CVF Conference on Computer Vision and Pattern Recognition. 2024.

\bibitem{shah2023vint} Shah, Dhruv, et al. "ViNT: A foundation model for visual navigation." arXiv preprint arXiv:2306.14846 (2023).

\bibitem{liu2024visual} Liu, Haotian, et al. "Visual instruction tuning." Advances in neural information processing systems 36 (2024).

\bibitem{hu2021lora} Hu, Edward J., et al. "Lora: Low-rank adaptation of large language models." arXiv preprint arXiv:2106.09685 (2021).

\bibitem{qin2023supfusion} Qin, Yiran, et al. "SupFusion: Supervised LiDAR-camera fusion for 3D object detection." Proceedings of the IEEE/CVF International Conference on Computer Vision. 2023.

\bibitem{qin2024worldsimbench} Qin, Yiran, et al. "Worldsimbench: Towards video generation models as world simulators." arXiv preprint arXiv:2410.18072 (2024).

\bibitem{yu2025gamefactory} Yu, Jiwen, et al. "GameFactory: Creating New Games with Generative Interactive Videos." arXiv preprint arXiv:2501.08325 (2025).

\bibitem{zhou2024code} Zhou, Enshen, et al. "Code-as-Monitor: Constraint-aware Visual Programming for Reactive and Proactive Robotic Failure Detection." arXiv preprint arXiv:2412.04455 (2024).

\bibitem{zhang2024ad} Zhang, Zaibin, et al. "AD-H: Autonomous Driving with Hierarchical Agents." arXiv preprint arXiv:2406.03474 (2024).

\bibitem{wang2024toward} Wang, Chaoqun, et al. "Toward Accurate Camera-based 3D Object Detection via Cascade Depth Estimation and Calibration." arXiv preprint arXiv:2402.04883 (2024).

\bibitem{huang2024story3d} Huang, Yuzhou, et al. "Story3d-agent: Exploring 3d storytelling visualization with large language models." arXiv preprint arXiv:2408.11801 (2024).

\bibitem{savinov2018semi} Savinov, Nikolay, Alexey Dosovitskiy, and Vladlen Koltun. "Semi-parametric topological memory for navigation." arXiv preprint arXiv:1803.00653 (2018).

\bibitem{majumdar2022zson} Majumdar, Arjun, et al. "Zson: Zero-shot object-goal navigation using multimodal goal embeddings." Advances in Neural Information Processing Systems 35 (2022): 32340-32352.

\bibitem{yadav2023offline} Yadav, Karmesh, et al. "Offline visual representation learning for embodied navigation." Workshop on Reincarnating Reinforcement Learning at ICLR 2023. 2023.

\bibitem{yadav2023ovrl} Yadav, Karmesh, et al. "Ovrl-v2: A simple state-of-art baseline for imagenav and objectnav." arXiv preprint arXiv:2303.07798 (2023).

\bibitem{sun2024fgprompt} Sun, Xinyu, et al. "FGPrompt: fine-grained goal prompting for image-goal navigation." Advances in Neural Information Processing Systems 36 (2024).

\bibitem{zhu2017target} Zhu, Yuke, et al. "Target-driven visual navigation in indoor scenes using deep reinforcement learning." 2017 IEEE international conference on robotics and automation (ICRA). IEEE, 2017.

\bibitem{koh2024generating} Koh, Jing Yu, Daniel Fried, and Russ R. Salakhutdinov. "Generating images with multimodal language models." Advances in Neural Information Processing Systems 36 (2024).

\bibitem{krantz2022instance} Krantz, Jacob, et al. "Instance-specific image goal navigation: Training embodied agents to find object instances." arXiv preprint arXiv:2211.15876 (2022).

\bibitem{schulman2017proximal} Schulman, John, et al. "Proximal policy optimization algorithms." arXiv preprint arXiv:1707.06347 (2017).

\bibitem{anderson2018evaluation} Anderson, Peter, et al. "On evaluation of embodied navigation agents." arXiv preprint arXiv:1807.06757 (2018).

\bibitem{lin2024navcot} Lin, Bingqian, et al. "NavCoT: Boosting LLM-Based Vision-and-Language Navigation via Learning Disentangled Reasoning." arXiv preprint arXiv:2403.07376 (2024).

\bibitem{NavGPT} Zhou, Gengze, Yicong Hong, and Qi Wu. "Navgpt: Explicit reasoning in vision-and-language navigation with large language models." Proceedings of the AAAI Conference on Artificial Intelligence.

\bibitem{hahn2021no} Hahn, Meera, et al. "No rl, no simulation: Learning to navigate without navigating." Advances in Neural Information Processing Systems 34 (2021): 26661-26673.

\bibitem{li2025t2isafety} Li, Lijun, et al. "T2ISafety: Benchmark for Assessing Fairness, Toxicity, and Privacy in Image Generation." arXiv preprint arXiv:2501.12612 (2025).

\bibitem{an2024agfsync} An, Jingkun, et al. "AGFSync: Leveraging AI-Generated Feedback for Preference Optimization in Text-to-Image Generation." arXiv preprint arXiv:2403.13352 (2024).


\end{thebibliography}
\end{sloppypar}

\clearpage
\beginsupplement
\section*{Appendix}
\renewcommand{\thesubsection}{S\arabic{subsection}}

\subsection{\label{chap:S1}PanNuke and MoNuSAC preprocessing}
The PanNuke dataset comprises a set of 7,901 RGB patches, each with dimensions of $256 \times 256$ pixels, which we set as the standard patch size for our analysis. In contrast, the MoNuSAC dataset encompasses 294 images of heterogeneous dimensions. To standardize the MoNuSAC images with our experiments, we implement a standardization protocol. Specifically, for images exceeding the dimensions of $256 \times 256$ pixels, we segment them into equal-sized patches and apply mirror padding to the remaining portions to avoid information loss at the peripherals. Patches with dimensions less than $128 \times 128$ pixels are excluded from the dataset due to the insufficient resolution to capture relevant cellular details. For patches where either dimension falls between 128 and 256 pixels, we employ upsampling to achieve the standard patch size. As a result, we obtain a total of 2,823 RGB patches derived from the MoNuSAC dataset for subsequent analysis. For additional details on the MoNuSAC data preparation process, refer to the source code \cite{Shvetsov_2025a}.
\clearpage

\subsection{\label{chap:S2}Data usage for the methodology}

\counterwithin{figure}{subsection}
\renewcommand{\thefigure}{S\arabic{subsection}}

\begin{figure}[h!]
    \centering
    \includegraphics[width=\textwidth, height=0.85\textheight, keepaspectratio]{images/A2.pdf}
    \caption{Overview of the methodology for cross-labeling, dataset refinement, and model comparison. (1) Cross-relabeling - training and testing cell classification models, (2) Cross-relabeling - using cell classification models to create refined dataset, (3) Fine-tuning and training models for comparison, (4) Student knowledge distillation with refined dataset}
    \label{fig:S2}
\end{figure}
\clearpage

\subsection{\label{chap:S3}Confusion matrices for classification models}
\counterwithin{figure}{subsection}
\renewcommand{\thefigure}{S\arabic{subsection}.\arabic{figure}}

\begin{figure}[h!]
    \centering
    \includegraphics[width=\textwidth, height=0.4\textheight, keepaspectratio]{images/A3_1.pdf}
    \caption{Confusion matrix for PanNuke trained model}
    \label{fig:S3.1}
\end{figure}

\begin{figure}[h!]
    \centering
    \includegraphics[width=\textwidth, height=0.4\textheight, keepaspectratio]{images/A3_2.pdf}
    \caption{Confusion matrix for MoNuSAC trained model}
    \label{fig:S3.2}
\end{figure}

\clearpage

\subsection{\label{chap:S4}Datasets cell counts}

\counterwithin{table}{subsection}
\renewcommand{\thetable}{S\arabic{subsection}}

\begin{table}[h!]
\renewcommand{\arraystretch}{2.0}
\centering
\caption{\label{tab:S4}Cell counts for PanNuke, MoNuSAC and refined datasets. Numbers in parentheses indicate preprocessed cell counts for cell classifier models training and testing.}
%\adjustbox{max width=\textwidth}{%
\begin{tabular}{|l|c|c|c|}
\hline
%\rowcolor{gray!30}
Cell type & PanNuke & MoNuSAC & Refined \\
\hline
Neoplastic & 77,403 (68,031) & - & 105,451 \\
\hline
Epithelial & 26,572 (23,207) & - & 29,926 \\
\hline
Epithelial (benign and malignant) & - & 31,402 & - \\
\hline
Inflammatory & 32,276 & - & - \\
\hline
Lymphocytes & - & 37,045 (33,104) & 65,275 \\
\hline
Neutrophils & - & 1,355 (1,252) & 3,833 \\
\hline
Macrophage & - & 1,842 (1,695) & 3,410 \\
\hline
Dead & 2,908 & - & 2,908 \\
\hline
Connective & 50,585 & - & 50,585 \\
\hline
\end{tabular}
%
%}
\end{table}



\clearpage

\subsection{\label{chap:S5}Definition of validation metrics}
\counterwithin{equation}{subsection}
\renewcommand{\theequation}{\arabic{equation}}

\subsubsection{\label{chap:S5.1}R\textsuperscript{2}}
The coefficient of determination, denoted as $R^2$, is a statistical measure that represents the proportion of variance in the dependent variable that is predictable from the independent variables. In the context of cell quantification in pathology, $R^2$ is used to assess how well the predicted quantities of different cell types in a patch align with the actual quantities observed in the ground truth data, with higher values representing more accurate quantification. $R^2$ is defined as
\begin{equation*}
R^2 = 1 - \frac{\sum_{i=1}^n (y_i - \hat{y}_i)^2}{\sum_{i=1}^n (y_i - \bar{y})^2},
\end{equation*}
where $y_i$ represents the actual number of cells of a specific type in the $i$-th image, $\hat{y}_i$ represents the predicted number of cells of that type in the $i$-th image, $\bar{y}$ is the mean of the actual numbers across all images, and $n$ is the total number of images in the dataset.

The $R^2$ metric has a range of $(-\infty, 1]$. An $R^2$ of 1 indicates perfect prediction, where all predicted values exactly match the actual values. An $R^2$ of 0 suggests that the model explains none of the variability of the response data around its mean. If $R^2$ is negative, it indicates that the model performs worse than a model that simply predicts the mean of the actual values for all observations.

\subsubsection{\label{chap:S5.2}PQ}
Panoptic Quality ($PQ$) is a comprehensive metric used to evaluate the performance of segmentation models in tasks that require both instance segmentation and classification. $PQ$ provides a single score that encapsulates both the detection accuracy (i.e., how many objects were correctly identified) and the segmentation quality (i.e., how accurately the objects' boundaries were delineated). This metric is particularly useful in multiclass scenarios where each pixel is classified into distinct categories, such as different cell types in pathology images.

$PQ$ is calculated as the product of two terms: Detection Quality ($DQ$) and Segmentation Quality ($SQ$). It can be expressed as
\begin{equation*}
PQ = DQ \cdot SQ,
\end{equation*}
where
\begin{equation*}
DQ = \frac{TP}{TP + 0.5\, FP + 0.5\, FN},
\end{equation*}
\begin{equation*}
SQ = \frac{\sum_{(p, g) \in \mathcal{M}} IoU(p, g)}{TP}.
\end{equation*}
In these formulas, $TP$ denotes the number of correctly matched instances between ground truth and prediction, $FP$ denotes the predicted instances that have no corresponding ground truth, $FN$ denotes the ground truth instances that were not detected, $IoU(p, g)$ is the Intersection over Union for a pair of matched instances $p$ (prediction) and $g$ (ground truth), and $\mathcal{M}$ is the set of matched pairs.

The $PQ$ metric is calculated for each class and is averaged across classes to provide a global performance measure.

The $PQ$ score has a range of $[0, 1.0]$, where a higher score indicates better performance in both detecting and segmenting the instances correctly. A $PQ$ of 1 signifies perfect identification and segmentation of all instances, whereas a $PQ$ of 0 indicates that no instances were correctly identified and segmented.

\clearpage

\subsection{\label{chap:S6}Segmentation and Detection quality metrics for teacher and student models}

\begin{table}[h!]
\renewcommand{\arraystretch}{2.0}
\centering
\caption{Segmentation and detection quality for student and teacher models (CI 95\%)}
\label{tab:S6}
%\adjustbox{max width=\textwidth}{%
\begin{tabular}{|l|c|c|}
\hline
%\rowcolor{gray!30}
Metric & Teacher & Student \\
\hline
$SQ_{neoplastic}$ & 0.819 (0.815--0.823) & 0.824 (0.819--0.828) \\
\hline
$SQ_{lymphocyte}$ & 0.795 (0.788--0.802) & 0.790 (0.783--0.796) \\
\hline
$SQ_{connective}$ & 0.770 (0.762--0.776) & 0.780 (0.772--0.786) \\
\hline
$SQ_{dead}$ & 0.659 (0.623--0.688) & 0.657 (0.624--0.695) \\
\hline
$SQ_{epithelial}$ & 0.780 (0.770--0.790) & 0.788 (0.779--0.797) \\
\hline
$SQ_{macrophage}$ & 0.788 (0.760--0.810) & 0.757 (0.730--0.783) \\
\hline
$SQ_{neutrofil}$ & 0.782 (0.761--0.801) & 0.775 (0.759--0.792) \\
\hline
$DQ_{neoplastic}$ & 0.706 (0.692--0.719) & 0.727 (0.712--0.741) \\
\hline
$DQ_{lymphocyte}$ & 0.675 (0.656--0.698) & 0.713 (0.691--0.734) \\
\hline
$DQ_{connective}$ & 0.566 (0.546--0.584) & 0.583 (0.565--0.602) \\
\hline
$DQ_{dead}$ & 0.410 (0.361--0.465) & 0.435 (0.306--0.561) \\
\hline
$DQ_{epithelial}$ & 0.668 (0.639--0.694) & 0.673 (0.644--0.702) \\
\hline
$DQ_{macrophage}$ & 0.657 (0.583--0.727) & 0.615 (0.531--0.703) \\
\hline
$DQ_{neutrofil}$ & 0.691 (0.625--0.753) & 0.729 (0.679--0.778) \\
\hline
\end{tabular}
%
%}
\end{table}

\clearpage

\subsection{\label{chap:S7}QuPath integration method}
We adopt an integration strategy leveraging the paquo \cite{Bayer_AG} library, a Python package that enables direct interaction with QuPath’s internal API, thereby facilitating seamless data exchange without intermediate conversion steps. The data processing pipeline (\hyperref[fig:S7]{Appendix Figure S7}) begins with the acquisition of WSIs and their associated annotations from QuPath, which are represented as Shapely \cite{Gillies_Wel_etal._2024} polygons. Utilizing paquo, we directly read, create, and modify these annotations and detections within a QuPath project in the Python environment. Images are then cropped using these polygons and processed by cell segmentation and classification models employing standard vision processing toolkits such as OpenCV, pyvips, and PyTorch. Additionally, QuPath employs Groovy scripts to initiate a Python process that starts the entire pipeline from QuPath graphical interface: fetching polygons, extracting images from them, and running deep learning model inference on the cropped images. 
The results are returned to QuPath, leveraging paquo's Python bindings to manipulate QuPath data while minimizing the computational overhead typically associated with cross-environment communication.

\counterwithin{figure}{subsection}
\renewcommand{\thefigure}{S\arabic{subsection}}

\begin{figure}[h!]
    \centering
    \includegraphics[width=\textwidth]{images/A7.pdf}
    \caption{QuPath integration workflow using Python environment}
    \label{fig:S7}
\end{figure}

Compared to traditional workflows that involve exporting annotations as GeoJSON, classifying them in Python, and reimporting them into QuPath, our approach offers several advantages. We eliminate the need to switch between programming languages, providing a cohesive and streamlined development process entirely within QuPath software and removing the necessity to use other tools. Meanwhile, we avoid storing annotations as intermediate JSON files unless required for external use or archiving. By conducting the entire inference and post-processing workflow within the Python environment, we leverage the power and flexibility of Python libraries for image processing and machine learning. This approach also enables adjustments to any set of labels and models, thereby improving its applicability.

%\hfill

The distilled model and QuPath integration code are packaged into a Docker container, enabling streamlined execution with the Docker engine. Detailed integration code and deployment instructions can be found in the GitHub repository \cite{Shvetsov_2025b}.

Despite these benefits, we acknowledge that the paquo library is a proof‑of‑concept project in its early development stage and has not been tested across all versions of QuPath.

\clearpage

\subsection{\label{chap:S8}Data and code availability statement}
All datasets, models, and code used in this study are publicly available and can be obtained from the repositories listed below. 
The PanNuke \cite{Gamper_Koohbanani_etal._2019} and MoNuSAC \cite{Verma_Kumar_etal._2021} datasets are publicly accessible, and download information along with detailed descriptions can be found in their respective articles. Preprocessing scripts for PanNuke and MoNuSAC data, as well as individual cell extraction scripts, are available on GitHub \cite{Shvetsov_2025a}. The H-Optimus foundation model used in our experiments can be downloaded from the HuggingFace repository \cite{hoptimus2024}, and model information is available on GitHub \cite{Saillard_Jenatton_etal._2024}. In addition, the integration code for QuPath and the distilled model packaged in a Docker container are provided in the repository \cite{Shvetsov_2025b}, and paquo Python library is available from the authors GitHub repository \cite{Bayer_AG}.
\clearpage

\end{document}


\end{document}
