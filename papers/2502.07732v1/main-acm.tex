\pdfoutput=1

%%
%% This is file `sample-sigconf.tex',
%% generated with the docstrip utility.
%%
%% The original source files were:
%%
%% samples.dtx  (with options: `all,proceedings,bibtex,sigconf')
%% 
%% IMPORTANT NOTICE:
%% 
%% For the copyright see the source file.
%% 
%% Any modified versions of this file must be renamed
%% with new filenames distinct from sample-sigconf.tex.
%% 
%% For distribution of the original source see the terms
%% for copying and modification in the file samples.dtx.
%% 
%% This generated file may be distributed as long as the
%% original source files, as listed above, are part of the
%% same distribution. (The sources need not necessarily be
%% in the same archive or directory.)
%%
%%
%% Commands for TeXCount
%TC:macro \cite [option:text,text]
%TC:macro \citep [option:text,text]
%TC:macro \citet [option:text,text]
%TC:envir table 0 1
%TC:envir table* 0 1
%TC:envir tabular [ignore] word
%TC:envir displaymath 0 word
%TC:envir math 0 word
%TC:envir comment 0 0
%%
%%
%% The first command in your LaTeX source must be the \documentclass
%% command.
%%
%% For submission and review of your manuscript please change the
%% command to \documentclass[manuscript, screen, review]{acmart}.
%%
%% When submitting camera ready or to TAPS, please change the command
%% to \documentclass[sigconf]{acmart} or whichever template is required
%% for your publication.
%%
%%
\documentclass[sigconf, nonacm]{acmart}

\usepackage{xcolor,colortbl}
\usepackage{enumitem}
\usepackage{array,multirow,graphicx}
\usepackage{tabularx}
\usepackage{fontawesome}
\usepackage{ulem}
\usepackage{hyperref}



% \usepackage{arydshln}

% \usepackage{tcolorbox}
% % \definecolor{bg}{RGB}{245,245,245}
% \definecolor{bg}{RGB}{255,255,255}
% \definecolor{bgbord}{RGB}{235,235,235}
% % \definecolor{extitle}{HTML}{455a64}
% \definecolor{extitle}{HTML}{000000}
% \newtcolorbox{examplebox}[1]{colback=bgbord,colframe=bgbord,arc=0pt,coltitle=extitle, title=#1,left=4pt, top=-4pt, bottom=4pt, right=4pt, width=\linewidth, before=\par\smallskip\centering, fonttitle=\Large,}

\usepackage{tcolorbox}
\definecolor{bgbord}{RGB}{220,220,220}  % Slightly darker border
\definecolor{bg}{RGB}{250,250,250}  % Soft gray background
\definecolor{extitle}{HTML}{222222}  % Slightly darker title

\newtcolorbox{examplebox}[1]{
    colback=bg, 
    colframe=bgbord, 
    arc=1pt,  % Slightly rounded corners
    coltitle=extitle, 
    title=#1,
    left=6pt, top=4pt, bottom=4pt, right=6pt,
    width=\linewidth,
    before=\par\smallskip\centering,
    fonttitle=\large\bfseries
}


%%
%% \BibTeX command to typeset BibTeX logo in the docs
\AtBeginDocument{%
  \providecommand\BibTeX{{%
    Bib\TeX}}%
  \hypersetup{
    colorlinks=true,
    linkcolor=myorange,
    citecolor=myorange,
    urlcolor=myorange,
    pdfborder={0 0 0}
  }
}

%% Rights management information.  This information is sent to you
%% when you complete the rights form.  These commands have SAMPLE
%% values in them; it is your responsibility as an author to replace
%% the commands and values with those provided to you when you
%% complete the rights form.
% \setcopyright{acmlicensed}
% \copyrightyear{2018}
% \acmYear{2018}
% \acmDOI{XXXXXXX.XXXXXXX}

%% These commands are for a PROCEEDINGS abstract or paper.
% \acmConference[Conference acronym 'XX]{Make sure to enter the correct
%   conference title from your rights confirmation emai}{June 03--05,
%   2018}{Woodstock, NY}
  
%%
%%  Uncomment \acmBooktitle if the title of the proceedings is different
%%  from ``Proceedings of ...''!
%%
%%\acmBooktitle{Woodstock '18: ACM Symposium on Neural Gaze Detection,
%%  June 03--05, 2018, Woodstock, NY}
% \acmISBN{978-1-4503-XXXX-X/18/06}

\settopmatter{printacmref=false} % Removes ACM reference styling


%%
%% Submission ID.
%% Use this when submitting an article to a sponsored event. You'll
%% receive a unique submission ID from the organizers
%% of the event, and this ID should be used as the parameter to this command.
%%\acmSubmissionID{123-A56-BU3}

%%
%% For managing citations, it is recommended to use bibliography
%% files in BibTeX format.
%%
%% You can then either use BibTeX with the ACM-Reference-Format style,
%% or BibLaTeX with the acmnumeric or acmauthoryear sytles, that include
%% support for advanced citation of software artefact from the
%% biblatex-software package, also separately available on CTAN.
%%
%% Look at the sample-*-biblatex.tex files for templates showcasing
%% the biblatex styles.
%%

%%
%% The majority of ACM publications use numbered citations and
%% references.  The command \citestyle{authoryear} switches to the
%% "author year" style.
%%
%% If you are preparing content for an event
%% sponsored by ACM SIGGRAPH, you must use the "author year" style of
%% citations and references.
%% Uncommenting
%% the next command will enable that style.
\citestyle{acmauthoryear}


%%
%% end of the preamble, start of the body of the document source.
 
% \setcitestyle{authoryear,open={(},close={)}}

\setlength{\parindent}{0pt}  % Removes indentation
\setlength{\parskip}{7pt}    % Adds spacing before a new paragraph

\usepackage{titlesec}

% \renewcommand{\thesection}{\Roman{section}}
% \titleformat{\section}{\LARGE\bfseries}{\thesection.}{0.3em}{}
\titleformat{\section}{\fontsize{11.5}{13}\bfseries}{\thesection.}{0.3em}{}

\titlespacing{\section}{0pt}{10pt}{-5pt}  


% \definecolor{myorange}{rgb}{0.8, 0.4, 0}

\definecolor{myorange}{rgb}{0.65, 0.25, 0}


\usepackage{xcolor}
\usepackage{natbib}

\makeatletter
\let\oldcite\cite
\let\oldcitep\citep

% Redefine \cite
\renewcommand{\cite}[1]{[\textcolor{myorange}{\begingroup
    \let\NAT@open\relax
    \let\NAT@close\relax
    \oldcite{#1}\endgroup}]}

\makeatother



% \let\oldcite\cite
% \renewcommand{\cite}[1]{{\color{myorange} \oldcite{#1}}}



\begin{document}

%%
%% The "title" command has an optional parameter,
%% allowing the author to define a "short title" to be used in page headers.
% \title{The Invisible Hand of Human Motivations in AI Data Quality}
\title{Economics of Sourcing Human Data}
% \title{Data Quality in AI: Impact of Human Motivation \\and Incentives on Machine Learning}

%%
%% The "author" command and its associated commands are used to define
%% the authors and their affiliations.
%% Of note is the shared affiliation of the first two authors, and the
%% "authornote" and "authornotemark" commands
%% used to denote shared contribution to the research.

% \author{Sebastin Santy}
% \affiliation{%
%   \institution{University of Washington}
%   \city{Seattle}
%   \country{USA}}
% \email{ssanty@cs.washington.edu}

% \author{Prasanta Bhattacharya}
% \affiliation{%
%   \institution{ASTAR, NUS}
%   \country{Singapore}
% }

% \author{Manoel Ribeiro}
% \affiliation{%
%  \institution{EPFL}
%  \city{Lausanne}
%  \country{Switzerland}}

% \author{Kelsey Allen}
% \affiliation{%
%   \institution{Google Deepmind}
%   \country{USA}}


\author{ Sebastin Santy$^{1}$ \: Prasanta Bhattacharya$^{2}$ \: Manoel Horta Ribeiro$^{3}$ \: Kelsey Allen$^{4}$ \: Sewoong Oh$^{1}$
}

\affiliation{
\institution{$^{1}$University of Washington \: $^{2}$Institute of High Performance Computing (IHPC), A*STAR\\[1.5pt] $^{3}$Princeton University \: $^{4}$University of British Columbia}
\city{}
\state{}
\country{}
}

\thanks{$^{\dagger}$Institute of High Performance Computing (IHPC), Agency for Science, Technology and Research (A*STAR), 1 Fusionopolis Way, \#16-16 Connexis, Singapore 138632, Republic of Singapore}
\thanks{Corresponding author: ssanty@cs.washington.edu}


%%
%% By default, the full list of authors will be used in the page
%% headers. Often, this list is too long, and will overlap
%% other information printed in the page headers. This command allows
%% the author to define a more concise list
%% of authors' names for this purpose.
\renewcommand{\shortauthors}{Santy et al.}

%%
%% The abstract is a short summary of the work to be presented in the
%% article.
% \begin{abstract}
%   A clear and well-documented \LaTeX\ document is presented as an
%   article formatted for publication by ACM in a conference proceedings
%   or journal publication. Based on the ``acmart'' document class, this
%   article presents and explains many of the common variations, as well
%   as many of the formatting elements an author may use in the
%   preparation of the documentation of their work.
% \end{abstract}

During the early stages of interface design, designers need to produce multiple sketches to explore a design space.  Design tools often fail to support this critical stage, because they insist on specifying more details than necessary. Although recent advances in generative AI have raised hopes of solving this issue, in practice they fail because expressing loose ideas in a prompt is impractical. In this paper, we propose a diffusion-based approach to the low-effort generation of interface sketches. It breaks new ground by allowing flexible control of the generation process via three types of inputs: A) prompts, B) wireframes, and C) visual flows. The designer can provide any combination of these as input at any level of detail, and will get a diverse gallery of low-fidelity solutions in response. The unique benefit is that large design spaces can be explored rapidly with very little effort in input-specification. We present qualitative results for various combinations of input specifications. Additionally, we demonstrate that our model aligns more accurately with these specifications than other models. 

% OLD ABSTRACT
%When sketching Graphical User Interfaces (GUIs), designers need to explore several aspects of visual design simultaneously, such as how to guide the user’s attention to the right aspects of the design while making the intended functionality visible. Although current Large Language Models (LLMs) can generate GUIs, they do not offer the finer level of control necessary for this kind of exploration. To address this, we propose a diffusion-based model with multi-modal conditional generation. In practice, our model optionally takes semantic segmentation, prompt guidance, and flow direction to generate multiple GUIs that are aligned with the input design specifications. It produces multiple examples. We demonstrate that our approach outperforms baseline methods in producing desirable GUIs and meets the desired visual flow.

% Designing visually engaging Graphical User Interfaces (GUIs) is a challenge in HCI research. Effective GUI design must balance visual properties, like color and positioning, with user behaviors to ensure GUIs easy to comprehend and guide attention to critical elements. Modern GUIs, with their complex combinations of text, images, and interactive components, make it difficult to maintain a coherent visual flow during design.
% Although current Large Language Models (LLMs) can generate GUIs, they often lack the fine control necessary for ensuring a coherent visual flow. To address this, we propose a diffusion-based model that effectively handles multi-modal conditional generation. Our model takes semantic segmentation, optional prompt guidance, and ordered viewing elements to generate high-fidelity GUIs that are aligned with the input design specifications.
% We demonstrate that our approach outperforms baseline methods in producing desirable GUIs and meets the desired visual flow. Moreover, a user study involving XX designers indicates that our model enhances the efficiency of the GUI design ideation process and provides designers with greater control compared to existing methods.    



% %%%%%%%%%%%%%%%%%%%%%%%%%%%%%%%%%%%%%%%%%%%%%%%%%%%%%%
% % Writing Clinic Comments:
% %%%%%%%%%%%%%%%%%%%%%%%%%%%%%%%%%%%%%%%%%%%%%%%%%%%%%%
% % Define: Effective UI design
% % Motivate GANs and write in full form.
% % LLMs vs ControlNet vs GANs
% % Say something about the Figma plugin?
% % Write the work is novel or what has been done before
% % What is desirable UI and how to evalutate that?
% % Visual Flow - main theme (center around it)
% % Re-Title: use word Flow!
% % Use ControlNet++ & SPADE for abstract.
% % Write about input/output. 
% % Why better than previous work?
% %%%%%%%%%%%%%%%%%%%%%%%%%%%%%%%%%%%%%%%%%%%%%%%%%%%%%

% % v2:
% % \noindent \textcolor{red}{\textbf{NEW Abstract!} (Post Writing Clinic 1 - 25-Jun)}

% % \noindent \textcolor{red}{----------------------------------------------------------------------}

% % \noindent Designing user interfaces (UIs) is a time-consuming process, particularly for novice designers. 
% % Creating UI designs that are effective in market funneling or any other designer defined goal requires a good understanding of the visual flow to guide users' attention to UI elements in the desired order. 
% % While current Large Language Models (LLMs) can generate UIs from just prompts, they often lack finer pixel-precise control and fail to consider visual flow. 
% % In this work, we present a UI synthesis method that incorporates visual flow alongside prompts and semantic layouts. 
% % Our efficient approach uses a carefully designed Generative Adversarial Network (GAN) optimized for scenarios with limited data, making it more suitable than diffusion-based and large vision-language models.
% % We demonstrate that our method produces more "desirable" UIs according to the well-known contrast, repetition, alignment, and proximity principles of design. 
% % We further validate our method through comprehensive automatic non-reference, human-preference aligned network scoring and subjective human evaluations.
% % Finally, an evaluation with xx non-expert designers using our contributed Figma plugin shows that <method-name> improves the time-efficiency as well as the overall quality of the UI design development cycle.

% % \noindent \textcolor{red}{----------------------------------------------------------------------}


% \noindent \textcolor{blue}{\textbf{NEW Abstract!} (Pre Writing Clinic 9-July)}

% \noindent \textcolor{blue}{----------------------------------------------------------------------}

% \noindent Exploring different graphical user interface (GUI) design ideas is time-consuming, particularly for novice designers. 
% Given the segmentation masks, design requirement as prompt, and/or preferred visual flow, we aim to facilitate creative exploration for GUI design and generate different UI designs for inspiration.
% While current Vision Language Models (VLMs) can generate GUIs from just prompts, they often lack control over visual concepts and flow that are difficult to convey through language during the generation process. 
% In this work, we present FlowGenUI, a semantic map-guided GUI synthesis method that optionally incorporates visual flow information based on the user's choice alongside language prompts. 
% We demonstrate that our model not only creates more realistic GUIs but also creates "predictable" (how users pay attention to and order of looking at GUI elements) GUIs.
% Our approach uses Stable Diffusion (SD), a large paired image-text pretrained diffusion model with a rich latent space that we steer toward realistic GUIs using a trainable copy of SD's encoder for every condition (segmentation masks, prompts, and visual flow). 
% We further provide a semantic typography feature to create custom text-fonts and styles while also alleviating SD's inherent limitations in drawing coherent, meaningful and correct aspect-ratio text. 
% Finally, a subjective evaluation study of XX non-expert and expert designers demonstrates the efficiency and fidelity of our method.


% This process encourages creativity and prevents designers from falling into habitual patterns.


% ------------------------------------------------------------------
% Joongi Why is it important to create realistic GUI?
% I do not see how the Visual Flow given on the left hand side is reflected in the results on the right hand side. 
% I’d avoid making unsubstantiated claims about designers (falling into habitual patterns).
% The UIs you generate do not “align with users’ attention patterns” but rather try to control it (that’s what visual flow means)
% ------------------------------------------------------------------
% Comments - Writing Clinic - 9th July:
% Improve title. More names: FlowGen
% Figure 1: Use an inference time hand-drawn mask
% Figure 1: Show both workflows. Add a designer --> Input.
% Figure 1: Make them more diverse
% ------------------------------------------------------------------
% Designing graphical user interfaces (GUIs) requires human creativity and time. Designers often fall into habitual patterns, which can limit the exploration of new ideas. 
% To address this, we introduce FlowGenUI, a method that facilitates creative exploration and generates diverse GUI designs for inspiration. By using segmentation masks, design requirements as prompts, and/or selected visual flows, our approach enhances control over the visual concepts and flows during the generation process, which current Vision Language Models (VLMs) often lack.
% FlowGenUI uses Stable Diffusion (SD), a largely pretrained text-to-image diffusion model, and guides it to create realistic GUIs. 
% We achieve this by using a trainable copy of SD's encoder for each condition (segmentation masks, prompts, and visual flow). 
% This method enables the creation of more realistic and predictable GUIs that align with users' attention patterns and their preferred order of viewing elements.
% We also offer a semantic typography feature that creates custom text fonts and styles while addressing SD's limitations in generating coherent, meaningful, and correctly aspect-ratio text.
% Our approach's efficiency and fidelity are evaluated through a subjective user study involving XX designers. 
% The results demonstrate the effectiveness of FlowGenUI in generating high-quality GUI designs that meet user requirements and visual expectations.

% ---------------------------------------


%A critical and general issue remains while using such deep generative priors: creating coherent, meaningful and correct aspect-ratio text. 
%We tackle this issue within our framework and additionally provide a semantic typography feature to create custom text-fonts and styles. 


% %Creating UI designs that are effective in market funneling or any other designer-defined goal requires a good understanding of the visual flow to guide users' attention to UI elements in the desired order. 
% %While current largely pre-trained Vision Language Models (VLMs) can generate GUIs from just prompts, they often lack finer or pixel-precise control which can be crucial for many easy-to-understand visual concepts but difficult to convey through language. 
% % However, obtaining such pixe-level labels is an extremely expensive so we
% % For example - overlaying text on images with certain aspect ratios and two equally separated buttons 
% Additionally, all prior GUI generation work fails to consider visual flow information during the generation process. 
% We demonstrate that visual flow-informed generation not only creates more realistic and human-friendly GUIs but also creates "predictable" (how users pay attention to and order of looking at GUI elements) UIs that could be beneficial for designers for tasks like creating effective market funnels.
% In this work, we present a semantic map-guided GUI synthesis method that optionally incorporates visual flow information based on the user's choice alongside language prompts. 
% Our approach uses Stable Diffusion, a large (billions) paired image-text pretrained diffusion model with a rich latent space that we steer toward realistic GUIs using an ensemble of ControlNets. 
% % TODO: Mention it in 1 sentence:
% A critical and general issue remains while using such deep generative priors: creating coherent, meaningful and correct aspect-ratio text. 
% We tackle this issue within our framework and additionally provide a semantic typography feature to create custom text-fonts and styles. 
% To evaluate our method, we demonstrate that our method produces more "desirable" UIs according to the well-known contrast, repetition, alignment, and proximity principles of design. 
% % We further validate our method through comprehensive automatic non-reference and human-preference aligned scores. (TODO: Maybe Unskip if we get UIClip from Jason!)
% % TODO: Re-word this and only keep ideation cycles and time-efficiency.
% Finally, a subjective evaluation study of XX non-expert and expert designers demonstrates the efficiency and fidelity of our method.
% % improves the time-efficiency by quick iterations of the UI design ideation process.
% %Finally, an evaluation with xx non-expert designers using our contributed <method-name> improves the time-efficiency by quick iterations of the UI design ideation cycle.

%\noindent \textcolor{blue}{----------------------------------------------------------------------}


%In an evaluation with xx designers, we found that GenerativeLayout: 1) enhances designers' exploration by expanding the coverage of the design space, 2) reduces the time required for exploration, and 3) maintains a perceived level of control similar to that of manual exploration.



% Present-day graphical user interfaces (GUIs) exhibit diverse arrangements of text, graphics, and interactive elements such as buttons and menus, but representations of GUIs have not kept up. They do not encapsulate both semantic and visuo-spatial relationships among elements. %\color{red} 
% To seize machine learning's potential for GUIs more efficiently, \papername~ exploits graph neural networks to capture individual elements' properties and their semantic—visuo-spatial constraints in a layout. The learned representation demonstrated its effectiveness in multiple tasks, especially generating designs in a challenging GUI autocompletion task, which involved predicting the positions of remaining unplaced elements in a partially completed GUI. The new model's suggestions showed alignment and visual appeal superior to the baseline method and received higher subjective ratings for preference. 
% Furthermore, we demonstrate the practical benefits and efficiency advantages designers perceive when utilizing our model as an autocompletion plug-in.


% Overall pipeline: Maybe drop semantic typography / visual flow?

%%
%% The code below is generated by the tool at http://dl.acm.org/ccs.cfm.
%% Please copy and paste the code instead of the example below.
%%
% \begin{CCSXML}
% <ccs2012>
%  <concept>
%   <concept_id>00000000.0000000.0000000</concept_id>
%   <concept_desc>Do Not Use This Code, Generate the Correct Terms for Your Paper</concept_desc>
%   <concept_significance>500</concept_significance>
%  </concept>
%  <concept>
%   <concept_id>00000000.00000000.00000000</concept_id>
%   <concept_desc>Do Not Use This Code, Generate the Correct Terms for Your Paper</concept_desc>
%   <concept_significance>300</concept_significance>
%  </concept>
%  <concept>
%   <concept_id>00000000.00000000.00000000</concept_id>
%   <concept_desc>Do Not Use This Code, Generate the Correct Terms for Your Paper</concept_desc>
%   <concept_significance>100</concept_significance>
%  </concept>
%  <concept>
%   <concept_id>00000000.00000000.00000000</concept_id>
%   <concept_desc>Do Not Use This Code, Generate the Correct Terms for Your Paper</concept_desc>
%   <concept_significance>100</concept_significance>
%  </concept>
% </ccs2012>
% \end{CCSXML}

% \ccsdesc[500]{Do Not Use This Code~Generate the Correct Terms for Your Paper}
% \ccsdesc[300]{Do Not Use This Code~Generate the Correct Terms for Your Paper}
% \ccsdesc{Do Not Use This Code~Generate the Correct Terms for Your Paper}
% \ccsdesc[100]{Do Not Use This Code~Generate the Correct Terms for Your Paper}

%%
%% Keywords. The author(s) should pick words that accurately describe
%% the work being presented. Separate the keywords with commas.
% \keywords{Do, Not, Us, This, Code, Put, the, Correct, Terms, for,
%   Your, Paper}
%% A "teaser" image appears between the author and affiliation
%% information and the body of the document, and typically spans the
%% page.
% \begin{teaserfigure}
%   \includegraphics[width=\textwidth]{sampleteaser}
%   \caption{Seattle Mariners at Spring Training, 2010.}
%   \Description{Enjoying the baseball game from the third-base
%   seats. Ichiro Suzuki preparing to bat.}
%   \label{fig:teaser}
% \end{teaserfigure}

% \received{20 February 2007}
% \received[revised]{12 March 2009}
% \received[accepted]{5 June 2009}

%%
%% This command processes the author and affiliation and title
%% information and builds the first part of the formatted document.
\maketitle

\newcommand{\seb}[1]{\textcolor{orange}{[#1]}}
\newcommand{\todo}[1]{\textcolor{red}{[#1]}}

\definecolor{babyblueeyes}{rgb}{0.63, 0.79, 0.95}

\newcommand{\synopsis}{\noindent\textcolor{blue}{\textbf{Synopsis for section}\newline}}
{\newcommand{\synopsissection}[1]{\textcolor{babyblueeyes}{#1}}
\newcommand{\prose}{\noindent\textcolor{magenta}{\textbf{Prose for section}\newline}}
\newcommand{\references}{\noindent\textcolor{red}{\textbf{References for section}\newline}}

\newcommand{\removed}[1]{}
\newcommand{\added}[1]{#1}

\section{Human Data in Crisis}
Artificial Intelligence relies heavily on human-generated data to develop ever more capable models and systems that can emulate human-like intelligent behavior. The currently used sources of human data include: (1) human annotations, from data collection platforms (e.g., Amazon MTurk) and (2) raw data, from the Internet (e.g., Wikipedia, social media platforms). Our understanding of how to effectively use these two data sources has been a driving force behind the two most prominent eras of artificial intelligence: the \textbf{Deep Learning era} which began with AlexNet~\cite{krizhevsky2012imagenet} in 2012 and facilitated by ImageNet~\cite{deng2009imagenet} collected at scale through MTurk, and the \textbf{Pre-trained Language Models era} ushered in by BERT~\cite{devlin2018bert} in 2018 and enabled by the availability of large-scale Internet data.

However, the emergence of the third \textbf{Generative Large Language Chat models era}, marked by ChatGPT in 2022~\cite{openai2023chatgpt}, has made human-like language generation ubiquitous and accessible to the general public, disrupting the very mechanisms of data sourcing that have historically enabled AI development. The indiscriminate usage of large language models (LLMs) impacts the previous two key sources of human-produced data: participants in existing data collection systems are turning to LLMs to expedite their tasks~\cite{veselovsky2023artificial,veselovsky2023prevalence}, and the Internet is being flooded with LLM-generated content ~\cite{brooks2024rise}.
This makes it increasingly challenging to obtain and discern authentic human-generated data, raising concerns about a looming shortage of human data needed for continued AI progress.  To compensate, machine learning has further leaned into synthetic data\textemdash{}either to mimic human annotations~\cite{dubois2024alpacafarm} or emulate human behavior~\cite{argyle2023out, park2022social, park2023generative}\textemdash{} albeit not yet at the highest quality~\cite{geng2024unmet} and facing other challenges, such as model collapse~\cite{taori2023data, shumailov2024ai}, keeping the ember of human-generated data still alive~\cite{ashok2024little}.


We argue that these flaws have always existed in data collection platforms but have been recently amplified by LLMs to the point where their very existence is being called into question~\cite{pieces2025data}. Specifically,
{\bf
at the heart of this problem lies the issue of human incentives and motivations for contributing data—one that cannot be solved by simply increasing external rewards like pay but requires careful attention to intrinsic motivations that drive people to engage willingly and actively on platforms, which in turn leads to better-quality data.
} Designing new data collection systems necessitates a critical examination of the flaws in existing ones.


% \textcolor{red}{PB: We need a short para here to summarize the key propositions we're making in this paper in 2-3 lines e.g., a focus on intrinsic motivations requires ceding control at the task level, while designing enabling environments. A good example of where this has worked is data collection games. We equally emphasize the value of preserving user trust in sustaining high quality data collection systems etc.}

In this paper, we analyze the current data requirements in machine learning and how existing data collection systems attempt to meet them. We open up the black box of data collection\textemdash{}complex socio-technical systems shaped by human behaviors, idiosyncrasies, and technical constraints\textemdash{}drawing from popular theories and experiments in psychology, sociology, and economics. In doing so, we examine the quantity-quality tradeoff and argue that, while this tradeoff may not be entirely eliminable, the overall quality and quantity of data can still be improved by identifying and removing factors that undermine intrinsic human motivations. Given that data collection systems operate within broader economic and social structures, we also complement academic research with real-world discourse and case studies of different data collection strategies. Finally, we explore novel paradigms, including games, that offer promising directions for the future of human data sourcing.
\section{Characterizing Human Data Needs}
Progress in machine learning depends on the availability of data at a sufficient scale to inductively learn patterns from it~\cite{kaplan2020scaling,hoffmann2022training}. This need for data has grown exponentially as learning algorithms have evolved from statistical to deep learning and pre-trained language models. The \textit{quantity} of data has uncontestedly been the key consideration for the field~\cite{sutton2019bitter}, with any data source that adds several orders of magnitude to the size of existing datasets, such as data from the Internet, being considered indispensable. A general trend in machine learning regarding data sourcing, especially after the advent of pre-training with BERT~\cite{devlin2018bert}, has been to leverage sources of large data wherever they can be found, such as BookCorpus~\cite{zhu2015aligning}, Wikipedia~\cite{raffel2020exploring}, Reddit~\cite{Gokaslan2019OpenWeb}, and CommonCrawl~\cite{commoncrawl}.

Recently, however, as datasets have grown larger, the importance of \textit{quality} has become more apparent~\cite{nguyen2022quality,zhou2024lima,lee2021deduplicating}. While learning algorithms have improved in extracting signal from noise, they still have limits when faced with excessive noise or irrelevant data~\citetext{e.g., DataComp-LM discards 99\% of data and Text-Image DataComp filters out 70\%; \textcolor{myorange}{\citealt{gadre2024datacomp,li2024datacomp}}}. Data quality has long been assumed to matter, but its significance has become clearer than ever as models trained on external proprietary datasets consistently outperform others on benchmarks and in real-world applications~\cite{brown2020language}. This outperformance\textemdash{}often attributed to access to ``high-quality'' proprietary datasets, such as paywalled content or licensed secondary sources~\cite{bommasani2021opportunities}\textemdash{}has pushed the data quality discourse to the forefront and is now a high priority in machine learning.

\begin{tcolorbox}[
    colback=gray!15,   % Uniform gray background
    colframe=gray!15,  % Border matches background for a soft look
    coltitle=black,    % Title in black
    sharp corners,     % No rounded edges
    width=\linewidth,  % Full-width box
    boxrule=0pt,       % Removes harsh border
    left=10pt, right=10pt, top=7pt, bottom=8pt, % Adjust padding inside box
    before=\bigskip, % Space above box
    after=\bigskip,  % Space below box
    title={\fontsize{12}{15} \textbf{Human Data Sourcing Desiderata}}, % Enlarged title
    fonttitle=\bfseries, % Bold title
    before title={\vspace{10pt}}, % This adds extra spacing above the title
    titlerule=0mm, % Ensures no extra rule appears
]
\begin{enumerate}[leftmargin=14pt, label=(\arabic*), itemsep=10pt] % More spacing
    \item \textbf{High Quality} \\[3pt]
    Collecting high-quality data with a strong signal-to-noise ratio for training ML models.

    \item \textbf{High Quantity} \\[3pt]
    Collecting data large enough to satisfy the exponentially increasing complexity of tasks.
\end{enumerate}
\end{tcolorbox}


While both high quality and high quantity are critical for data sourcing, they often come at the expense of each other\textemdash{}improving one typically leads to a decline in the other. However, this trade-off is not an inherent property of data itself but rather a consequence of system design choices. One way to conceptualize this trade-off is as resembling a Pareto frontier, as illustrated in Figure~\ref{fig:quality-quantity}.

\begin{figure}[h]
    \centering
    \includegraphics[width=0.75\linewidth]{illustrations/pareto-frontier.png}
    \caption{Illustration of the quantity-quality trade-off in data collection systems. Platforms like MTurk, Prolific, and UpWork optimize for either scale or quality but struggle to achieve both simultaneously. In contrast, data from sources not explicitly designed for data collection\textemdash{}such as Wikipedia and Reddit\textemdash{}operate outside this trade-off, hinting at potential alternative paradigms.}
    \label{fig:quality-quantity}
\end{figure}

This trade-off explains why data collection systems struggle to balance quality and quantity. Platforms prioritizing quality, like freelance job platforms (e.g., UpWork), tend to be slower with lower output, while high-throughput systems, like rapid crowdwork platforms (e.g., MTurk), scale efficiently but often sacrifice consistency and quality~\cite{douglas2023data}. While this quantity-quality trade-off may never be fully eliminated for any designed data collection system, it is not a fixed constraint\textemdash{}rather than eliminating the trade-off, the key is to expand the frontier by addressing structural inefficiencies in incentive design, annotation methods, and human oversight.

The dynamics of the quantity-quality trade-off are shaped by multiple interacting and, often, latent factors. Untangling these factors requires opening up current data collection systems and examining their trade-offs through the lens of human behavior, organizational processes, and technical constraints. At a system level, quality depends on balancing intrinsic motivation with external incentives, while quantity is largely driven by process efficiency, often through task fragmentation and parallelization. However, excessive fragmentation can erode intrinsic motivation, leading to long-term declines in quality. This self-reinforcing cycle lies at the heart of the quantity-quality trade-off in data collection system design. 

\section{Position vs. Current Stance}
\noindent\textbf{Position:} Sustaining human-generated data for ML requires shifting focus toward intrinsic human motivations.

\noindent\textbf{Majoritarian Stance:} Data quality is a major consideration in machine learning, and many researchers and companies are actively exploring how to best collect high-quality human-produced data. Researchers recognize that incentives influence the level of effort contributed, and as a result, they often rely exclusively on financial incentives to encourage greater effort in data production. While incentives are important, we contend that solely relying on financial rewards\textemdash{}counter to intuition\textemdash{}risks backfiring by reducing the quality of the data collected. Instead, ML researchers should prioritize enhancing the intrinsic motivation of human participants, using external incentives sparingly as supportive nudges rather than primary drivers.
% \section{Defining and Measuring Data Quality}
\section{Understanding Data Quality}
While quantity is easily measurable and increasingly attainable through new data-sourcing methods, quality has become ever more elusive. As data availability has surged, the question of what constitutes ``high-quality'' data is increasingly debated in machine learning. The challenge then is to understand the existing notions of data quality and explore ways to make it more certain, before dissecting data collection systems.

\textbf{Prior Definitions.} Defining ``data quality'' has long been a challenge in machine learning, as it lacks a universal, quantifiable standard. While it is widely acknowledged that human-generated data varies in quality (e.g., curated datasets from specific websites being more reliable than scraped content), there is no single, absolute definition for what makes data ``high-quality''.  Attempts to define quality span both subjective and objective perspectives. Subjectively, quality is often linked to trustworthiness\textemdash{}Wikipedia, for instance, is generally regarded as more reliable than personal blogs~\cite{albalak2024survey,soldaini2024dolma}. Objectively, quality has been measured using statistical metrics (e.g., readability) or modeled metrics, such as GPT-3 Quality Filters~\cite{gururangan2022whose} and DataComp's curated datasets~\cite{li2024datacomp}, which define quality in the context of their downstream use. 


\textbf{Naturalness as the Basis of Data Quality.} Without clarity on data’s intended use, defining quality becomes challenging. We contend that one of the most intuitive ways to conceptualize quality\textemdash{}without presupposing a specific application\textemdash{}is by anchoring it in \textit{naturalness}: how people behave in routine activities, online or offline, in an authentic manner. Unlike other definitions that rely on the perceived reliability of the source or application-specific criteria, naturalness provides an observable and generalizable signal for what constitutes high-quality data. This pattern is evident in organic data sources, where humans naturally generate valuable data through meaningful tasks, such as editing Wikipedia, participating in Reddit discussions, or sharing artwork and photos on platforms like Flickr. In these settings, data is generated naturally, often without direct external incentives, making it more representative of authentic human behavior. %This already hints at the idea that intrinsic motivation is crucial for producing in-situ data that accurately reflects natural behavior. 
% Thus, we argue that naturalness provides the most robust and context-agnostic definition of data quality, as it explains and unifies prior perspectives in the literature.


\textbf{Is There a Case for Naturalness in AI Training?}
Naturalness has already been central to pretraining, where large-scale internet data\textemdash{}capturing naturally occurring human behavior\textemdash{}has been crucial to the success of LLMs. But its importance extends beyond achieving generalization. Even in supervised fine-tuning, where data is tailored for task-specific applications, naturalness matters, as collected data should ideally reflect real task engagement rather than behavior shaped by artificial constraints or incentives.

Interestingly, this divide between pretraining and fine-tuning mirrors a broader debate in AI. \textbf{Artificial Intelligence (AI)}\textemdash{}as envisioned by John McCarthy~\cite{mccarthy1987generality}\textemdash{}aims to achieve human-like general intelligence and, therefore, benefits from diverse, free-flowing human interactions, much like those found in pretraining data. In contrast, \textbf{Intelligence Augmentation (IA)}\textemdash{}as suggested by Douglas Engelbart~\cite{engelbart1962augmenting}\textemdash{}prioritizes enhancing human intelligence through specialized tools, requiring goal-oriented human interactions towards performing a task, similar to fine-tuning data. In both cases, naturalness remains key but plays distinct roles: capturing free-flowing or goal-oriented human interactions, free from artificial constraints or incentives.

To that end, the use of LLMs in data collection is not inherently bad\textemdash{}what matters is \textit{how} they are used. The real risk to naturalness comes from indiscriminate, careless reliance on AI as a shortcut, rather than as a tool for balancing meaningful engagement and productivity. Instead of aggressively policing AI use, the focus should be on designing environments where contributors engage with tasks in ways that make shortcuts feel unnecessary\textemdash{}just as one wouldn't feel compelled to take shortcuts in a personally meaningful hobby. For example, a survey respondent outsourcing an entire essay writing task to AI without any personal input demonstrates an unwillingness to engage meaningfully with the task\textemdash{}this is the kind of AI use that undermines data quality.

% \section{Quality Factors: Motivation \& Incentives}
\section{Human Factors in Quality\textemdash{}\\ Motivation \& Incentives}
Data collection platforms used in machine learning, such as MTurk, Prolific, UpWork, and ScaleAI, are designed in ways that include compensation structures, that directly influence both effort and data quality. While rapid crowdsourcing platforms (e.g., MTurk) favor low-effort and low-pay tasks that can scale easily, the quality of task output remains unreliable. In contrast, freelance job platforms tend to favor high-effort and higher-pay gigs which require more deliberate and engaged participation, often leading to higher quality outputs.

This difference aligns with a straightforward intuition\textemdash{}higher pay leads to greater effort and better-quality contributions~\citetext{e.g., \textcolor{myorange}{\citealt{mason2009financial,ho2015incentivizing,shah2016double,laux2024improving}}}. This assumption drives much of the current incentive-based data collection paradigm, where the goal is to use external compensation as a lever to elicit higher-quality data, when required. However, while external rewards can drive greater effort, evidence suggests that they are not always the primary determinant of high-quality engagement.

For instance, some of the highest-quality human contributions come from platforms where users are not financially compensated at all, such as Wikipedia, Reddit, and open-source communities. Here, participants contribute not because of pay, but because they find the activity meaningful, socially rewarding, or aligned with personal interests~\cite{forte2005wikipedia,lampe2010motivations}. These platforms challenge the idea that quality data generation must always rely on financial incentives, illustrating that intrinsic motivations can sustain long-term engagement without depending on financial rewards as the primary driver.

While many assume that external incentives and intrinsic motivation are correlated, research shows that their relationship is far more complex.

\begin{figure}[h]
    \centering
    \includegraphics[width=0.8\linewidth]{illustrations/motivation.png}
    \label{fig:motivation}
\end{figure}

\noindent\textbf{Overjustification Effect}~\cite{lepper1973undermining} explains how external rewards can diminish intrinsic motivation and affect task performance. In a classic experiment, preschool children who already enjoyed drawing were divided into three groups: (1) those who were promised and received a reward, (2) those who received an unexpected reward, and (3) those who received no reward. This experiment revealed two notable outcomes: first, children in the expected-reward group spent significantly less time drawing voluntarily after the reward was removed, compared to the other groups. Second, the drawings from the no-reward and unexpected-reward groups were rated as slightly higher in quality than those from the expected-reward group. This suggests that when an activity initially driven by intrinsic motivation is externally incentivized, the removal of rewards can lead to a decline in voluntary participation, and rewards may also subtly shift focus from quality to mere completion of the task.

\textbf{So why do external incentives sometimes backfire?} Two key psychological theories help explain why the Overjustification Effect\textemdash{}how extrinsic rewards can sometimes diminish intrinsic motivation\textemdash{}occurs:
\begin{itemize}[left=0cm]
    \item \textbf{Self-Perception Theory (SPT)}~\cite{bem1972self} suggests that individuals infer their own attitudes and motivations by observing their past behaviors. When external rewards are introduced, people may start attributing their participation to the incentive rather than to their original or intrinsic interest. Over time, this shift in self-perception can make them less likely to continue the behavior once the reward is removed.
    \item \textbf{Self-Determination Theory (SDT)}~\cite{deci1971effects} offers a broader framework by focusing on autonomy, competence, and relatedness as key psychological needs for intrinsic motivation. When a task is externally controlled through incentives, individuals may feel a loss of autonomy, making the activity feel like an obligation rather than a choice. This helps explain why highly controlled environments often struggle to sustain long-term engagement.
\end{itemize}
Together, SPT and SDT highlight why financial incentives alone are not a sustainable solution for maintaining high-quality, long-term engagement.

\textbf{If intrinsic motivation is key to sustaining high-quality, long-term contributions, then what role do external incentives play?} While excessive reliance on extrinsic rewards can be detrimental, carefully designed incentives can help initiate engagement in behaviors that might otherwise remain dormant. For example, small, well-calibrated incentives can serve as interventions, bringing attention to valuable behaviors without overwhelming intrinsic motivation~\cite{deci1971effects}. Even in systems designed to favor intrinsically motivated behavior\textemdash{}such as laissez-faire environments \cite{hayek2014road}, where individuals are free to act and bear the consequences of their choices\textemdash{}subtle incentive mechanisms can still be useful to align individual and collective goals, as with \textit{nudging}~\cite{leonard2008richard}. Overpresence of incentives has generally led to unintended consequences\textemdash{}either they are optimized to the point of losing effectiveness~\citetext{Goodhart's Law; \textcolor{myorange}{\citealt{goodhart1984problems}}}, or, worse, they introduce undesirable behaviors counterproductive to desired goals~\citetext{Perverse Incentives; \textcolor{myorange}{\citealt{kerr1975folly}}}\citetext{Cobra Effect; \textcolor{myorange}{\citealt{siebert2001cobra}}}.

\textbf{So, how do these social theories play out in real-world data sources?} Returning to the two sources of data\textemdash{}data collection systems like MTurk vs. naturally occurring data sources (community-based platforms) like Wikipedia\textemdash{}the discussed social theories provide valuable insights into their differing approaches to engagement. Data collection systems, being intentionally designed, are structured around external incentives, with financial rewards and intrinsic motivation naturally coexisting at first. However, over time, a crowding-out effect takes hold: as intrinsic motivation erodes, platforms and data collectors continually increase external incentives and tighten control to sustain participation and quality, creating a vicious cycle where contributors optimize for efficiency rather than genuine engagement. This often leads to over-reliance on shortcuts, such as automating survey responses with AI tools, at the expense of quality.

In contrast, community-based platforms, like Wikipedia or Reddit, primarily rely on intrinsic motivation, with minimal external incentives such as contributor badges, reputation systems, and recognition. These platforms sustain engagement over the long term by creating a sense of social belonging, often intertwined with competence and autonomy, demonstrating that high-quality contributions can be sustained without heavy reliance on financial incentives.

This contrast underscores a crucial insight: designing sustainable data collection systems is not just about offering better incentives\textemdash{}it requires structuring environments that actively sustain and enhance intrinsic motivation.
% \section{Quantity Factors: Efficiency through Fragmentation}
\section{Human Factors in Quantity\textemdash{}\\Efficiency through Fragmentation}
Data collection systems differ not only in how they compensate contributors but also in how they structure tasks for scalability. At one end, rapid crowdwork platforms fragment tasks into micro-tasks (e.g., HITs on MTurk) that take seconds to complete, optimizing for speed and mass throughput~\cite{malsburg2024mturk}. Platforms like Prolific handle slightly larger but still modular tasks, spanning minutes to hours~\cite{prolific2024completion}. On the other end, freelance job platforms (e.g., UpWork) structure work as full projects, lasting days or weeks and offering greater autonomy and depth of engagement~\cite{upwork2024times,fiverr2024comparison,workathomesmart2024lionbridge}.

Fragmentation into repeatable units that can be completed in a consistent and orderly manner allows tasks to be parallelized across multiple workers, replacing the traditionally serial process of creation. Many innovations and processes begin as creative, effortful tasks\textemdash{}akin to System 2 processes, requiring deliberate, conscious effort~\cite{kahneman2013prospect}. However, to scale, they are often refined into a System 1 process, where execution becomes fast, automated, and intuitive. This transformation\textemdash{}breaking down complex, uncertain tasks into simpler, repeatable steps—underpins mass production systems. 

Crucially, this shift from System 2 to System 1 is not just a natural evolution but is actively accelerated by task fragmentation. When work is divided into modular, repetitive tasks, the process of routinization happens more quickly.  A useful analogy here is \textbf{Fordism}~\cite{hounshell1984american}, which introduced the assembly line, a mechanized, repetitive setup where products are built step by step. This structured fragmentation enables processes to be repeated at scale, maximizing efficiency and throughput.

\begin{figure}[h]
    \centering
    \includegraphics[width=0.8\linewidth]{illustrations/fragmentation.png}
    \label{fig:fragmentation}
\end{figure}

However, as tasks become increasingly repetitive and fragmented, they sacrifice the creativity and problem-solving scope that contribute to meaningful engagement. Over time, workers become disconnected from the broader purpose of their efforts~\citetext{Theory of Alienation; \textcolor{myorange}{\citealt{marx2016economic}}}, %as Marx's Alienation Theory~\cite{marx2016economic} suggests, 
shifting their motivation from seeking fulfillment to merely achieving survival goals -- a regression in the hierarchy of needs~\cite{maslow1943theory}. For employers/collectors, this shift manifests as a decline in quality, as contributors disengage from the task itself. Unlike physical labor, which has built-in quality checks (e.g., material standards, product inspections), knowledge-based tasks lack robust safeguards. In data annotation, for example, there is often no immediate way to verify whether a task was completed thoughtfully or rushed~\cite{klie2024analyzing,klie2024efficient}. As a result, quality can quietly degrade, with errors compounding over time\textemdash{}often going unnoticed until the system has deteriorated beyond repair.

\textbf{So, how does task fragmentation impact real-world data sourcing?} Micro-tasking on platforms like MTurk was once hailed as a transformative shift in computer science, enabling large-scale user studies~\cite{bohannon2011social,kittur2013future,bernstein2011crowds} and efficient data collection for machine learning~\cite{deng2009imagenet}. However, over time, research has raised concerns about the reliance on ``piece rate'' or pay-per-task systems, favoring ``quota'' systems instead~\cite{ikeda2016pay,mason2009financial}, which alludes to a reduction in task quality when micro-tasking is pushed too far.

Micro-tasking doesn’t just impact data quality\textemdash{}it also affects the workers behind it. Beyond quality concerns, it has drawn criticism for its effect on worker well-being. Works such as Ghost Work~\cite{gray2019ghost} and Anatomy of AI~\cite{crawford2018anatomy} have illustrated the often invisible and exploitative nature of these atomized tasks. %performed by an  underpaid global workforce. 
The non-physical nature of knowledge labor further exacerbates this issue, making its value difficult to quantify~\cite{martin2016turking}. This issue is taken to the extreme when microtasks are outsourced to developing countries with favorable exchange rates~\cite{dicken2007global} to cut costs\textemdash{}and thus, incentives\textemdash{}even further~\cite{perrigo2023exclusive,microsoft_google_questioned}, often trapping workers in exploitative, sweatshop-like conditions~\cite{williams2022exploited,hao2022ai}.


% This issue is compounded by global labor arbitrage~\cite{dicken2007global}, where favorable exchange rates in developing countries allow platforms to offer low wages, trapping workers in exploitative, sweatshop-like conditions~\cite{perrigo2023exclusive,microsoft_google_questioned}.

\textbf{Then, what happens when tasks become so repetitive and unfulfilling that workers disengage from them entirely?} As discussed earlier, over time, many human-driven processes shift from System 2 (deliberate \& effortful) to System 1 (intuitive \& fast). As tasks become more structured and predictable, they become prime targets for automation. In physical labor, this transition has been gradual\textemdash{}machines take over repetitive, routine tasks, while humans focus on creative and uncertain work~\cite{brynjolfsson2014second}.

A similar shift is occurring in knowledge-based work, where high-quality LLMs give workers the opportunity to offload mundane tasks\textemdash{}such as grammar corrections, spell-checking, and phrasing refinements\textemdash{}to AI. When used judiciously, this assistance promotes meaningful engagement and enhances productivity without compromising data quality. However, the problem arises when workers become over-reliant on LLMs, using them indiscriminately to complete entire tasks without oversight~\cite{veselovsky2023artificial,veselovsky2023prevalence}. Since knowledge-based tasks often lack clear-cut quality standards, it becomes harder to detect when quality has deteriorated, making it easier for such opportunistic behavior to go unchecked.

As a result, the transition to automation in data sourcing has been rather chaotic. While repetitive physical labor was gradually offloaded to machines in structured ways, knowledge work faces conflicting views\textemdash{}some advocate for fully replacing human contributors~\citetext{e.g., \textcolor{myorange}{\citealt{dubois2024alpacafarm}}}, while others for eliminating LLM usage entirely~\citetext{e.g., \textcolor{myorange}{\citealt{thorp2023chatgpt}}}. However, fully relying on synthetic data risks model feedback loops and collapse~\cite{taori2023data,shumailov2024ai}, while a complete ban slows human productivity and sacrifices efficiency~\cite{liao2024llms,kreitmeir2023unintended}. The most effective approach lies somewhere in between\textemdash{}where AI serves as a tool that productively and progressively supports human effort rather than a crutch for task completion~\citetext{e.g., \textcolor{myorange}{\citealt{ashok2024little,qian2024evolution}}}.

In this landscape, intrinsic motivation becomes even more crucial. Workers must make thoughtful decisions about how to incorporate LLMs in ways that enhance rather than replace meaningful engagement. Designing a sustainable data collection system, therefore, is not just about limiting LLM use for workers or maximizing automation with synthetic data\textemdash{}it's about creating an environment where contributors remain actively engaged with the task, rather than optimizing for speed at the cost of quality.
\section{Elevating the Quality-Quantity Trade-off: \\Rethinking Control}
% \section{Uplifting the Trade-off: Rethinking Control}
Data collection systems have long operated under the assumption that control at the task level\textemdash{}through explicit instructions, incentive structures, and quality enforcement mechanisms\textemdash{}is necessary to ensure both high-quality and high-quantity data. However, as we have discussed, these very mechanisms often introduce unintended side effects. External incentives, while effective in driving participation, tend to crowd out intrinsic motivations over time, leading to disengagement and lower-quality contributions. Similarly, excessive task fragmentation, though useful for efficiency and scalability, can erode a sense of purpose and hence leads contributors to disengage from the task itself\textemdash{}resulting in an over-reliance on shortcuts, manifesting in careless task completion and non-judicious use of LLMs.

In contrast, data that we obtain from systems not intentionally designed for data collection\textemdash{}such as Wikipedia, Reddit, and open-source projects\textemdash{}demonstrate an alternative paradigm. These platforms do not enforce control at the task level but instead create environments where intrinsically motivated contributors engage meaningfully. Crucially, this lack of task-level control removes two major pitfalls seen in structured data collection systems: it prevents (a) the crowding-out effect, where external incentives replace intrinsic motivation over time, and (b) excessive fragmentation, ensuring that contributors remain connected to their purpose of participation. Taking inspiration, ceding control at task level could be the key to pushing the pareto-frontier forward.

\textbf{Ceding Control at the Task Level.} Rather than controlling individual (micro) tasks, data collection systems can benefit from structuring conditions that naturally guide contributor engagement. Moving from direct task management to a broader, environment-driven approach discourages individuals from attributing their participation to external rewards, helping preserve intrinsic motivation~\cite{bem1972self}. 

However, this shift presents a new challenge: relinquishing fine-grained control over tasks means that collectors must instead focus on shaping engagement at a more systemic level. Coarse-grained control\textemdash{}where engagement is influenced through platform design, incentives, and structural conditions\textemdash{}takes longer to align with desired outcomes and demands greater up-front effort. But once in place, it can lead to more sustainable data collection, enabling both higher-quality and higher-quantity contributions, as seen in rare but influential examples.

\begin{figure}[h]
    \centering
    \includegraphics[width=0.8\linewidth]{illustrations/environment-control.png}
    \label{fig:environment-control}
\end{figure}

\textbf{Deployed Robots} are a prime example of achieving this delicate balance. For example, robotic vacuum cleaners (e.g., Roomba) provide utility to users, by cleaning their homes, while simultaneously collecting spatial and navigation data that improves future performance~\cite{astor2017your}. Users engage with the system for its primary function, yet their interactions naturally generate high-quality data that feeds back into the AI’s development~\cite{brynjolfsson2014second}. This data collection approach also scales effectively, as data is gathered continuously and passively as a result of users' routine behaviors, without requiring any additional effort towards data contribution. This model extends to more complex, high-stakes deployments, such as electric vehicles equipped with driver-assist and self-driving features. As an example, consider how Tesla uses real-world driving data from its fleet to refine its self-driving AI algorithms, which ultimately benefit car owners by improving the technology~\cite{karpathy2021ai,tesla_autopilot} and can drive future innovation for the company (e.g., driverless Robotaxi). Similarly, Waymo operates self-driving taxis in real-world environments and has gathered large-scale data that has proven valuable for advancing computer vision research~\cite{sun2020scalability}.

However, replicating such large-scale, product-driven data collection systems is exceptionally difficult. They demand massive hardware infrastructure, well-articulated and trusted benefits for both users and companies, and real-world applications with extensive safeguards and privacy protections. For entities whose primary goal is simply to collect human-generated data, establishing such ecosystems solely for this purpose is neither feasible nor sustainable.

More importantly, these symbiotic relationships rely on a delicate balance of implicit or explicit social contracts, mutual trust, and fair distribution of costs and benefits, as explained by theories of social interaction and exchange\textemdash{}such as Social Exchange Theory~\cite{homans1958social} and Social Contract Theory~\cite{rousseau1762social}. When the benefits clearly extend beyond the primary parties\textemdash{}for example, when collected data is repurposed to serve third parties\textemdash{}this balance can be disrupted. If users feel that their contributions are being exploited without fair reciprocity, trust erodes, and they may come to see the system as exploitative or unjust.

These feelings of mistrust and exploitation are already prevalent in AI, particularly in creative and knowledge-sharing communities. Artists have protested against their work being scraped to train AI models without consent or compensation~\cite{jiang2023ai}\footnote{\url{https://www.aitrainingstatement.org/}}, with many calling for stronger protections against AI-generated art~\cite{guardian2025aiart}. Similarly, Stack Overflow users, frustrated by their contributions being used for external profits, have intentionally altered or deleted their posts to hinder AI training\textemdash{}leading to bans, boycott, and subsequent decline in engagement on the platform~\cite{ars2024stackoverflow, tomshardware2024stackoverflow}. These examples highlight the fragility of trust in data collection and the potential consequences when contributors feel that their data is being repurposed beyond its original intent.
\section{Games as a New Pareto-Frontier?}
Games are uniquely positioned at the intersection of structured environments and intrinsic motivation. Designed worlds crafted by game developers set the rules and constraints, yet within them, players engage voluntarily, driven by curiosity, competition, and creativity~\cite{koster2005theory}. Unlike traditional work, where tasks are often externally imposed, games offer a space where challenge and engagement emerge naturally, 
sustaining long-term participation without the need for direct financial incentives.

For most players\textemdash{}aside from professional esports competitors\textemdash{}games are played purely for enjoyment. At the same time, they require diverse forms of reasoning, strategy, and problem-solving, making them a rich ground for capturing complex human behaviors and decision making~\citetext{as already evidenced by their role in evaluating intelligence; \textcolor{myorange}{\citealt{silver2016mastering, vinyals2019grandmaster, berner2019dota, meta2022human}}}. In this way, games hint at a new frontier\textemdash{}one where many structured environments and organic engagements seamlessly coexist. This intersection offers a compelling model for AI training, where intrinsically motivated interactions can yield structured, high-quality data without the pitfalls of external incentive-driven systems.

\textbf{Games as a Tool for Data Collection.} Games have long been explored as a tool for large-scale human annotation and AI training, most notably through von Ahn's Games with a Purpose (GWAP) \cite{von2006games}. One of the earliest and most influential examples was the ESP Game~\cite{von2005esp}, introduced in 2004, which engaged thousands of players in a collaborative tagging game, generating millions of image annotations. While players simply enjoyed the game, their interactions helped bootstrap Google Image Search~\cite{guardian2006esp}, which previously relied only on filenames, as large-scale labeled datasets like ImageNet were not available until 2009~\cite{deng2009imagenet}.

Von Ahn argued that the billions of hours spent on games\textemdash{}such as the 9 billion hours on Solitaire in 2003 alone, enough to build the Empire State Building in 6.8 hours or the Panama Canal in a day\textemdash{}could be repurposed for more meaningful tasks, inspiring the broader Games with a Purpose framework~\cite{law2011human}. Other notable games in this series included Peek-a-boom~\cite{von2006peekaboom}, which collected image segmentation data, and Verbosity~\cite{von2006verbosity}, designed to gather commonsense factual knowledge.

\textbf{Games in Mainstream ML.} While recent efforts towards this in machine learning have not reached the scale of GWAP, they have made notable progress. Examples include Google's QuickDraw~\cite{ha2017neural} and AllenAI's Iconary~\cite{clark2021iconary}, which focus on collecting freehand drawing data, and AI21's HumanOrNot~\cite{jannai2023human}, which gathers conversational data through a gamified Turing Test~\cite{dugan2023real}.

However, developing entirely new games for data collection presents significant challenges. Machine learning researchers often lack expertise in designing engaging gameplay, making it difficult to ensure both high-quality data and sustained participation. To navigate this, some approaches have focused on repurposing existing games rather than building from scratch. For example, Family Feud has been adapted to generate QA pairs~\cite{boratko2020protoqa}, and Minecraft has been used as a platform for collecting conversational dialogues~\cite{narayan2019collaborative}.

On a smaller scale, some efforts have explored gamification, introducing game-like mechanics into traditionally non-game tasks to enhance engagement, without requiring a full game environment. For instance, CommonsenseQA 2.0 incorporates elements such as scoring, competition, and progression to make data contribution more engaging, improving both user participation and data quality~\cite{talmor2022commonsenseqa}.

\textbf{Design Considerations.} Designing data collection games requires solving multiple challenges at once\textemdash{}the most critical challenge is optimizing data utility while preserving intrinsic player motivation. This involves balancing two often competing objectives:
\begin{itemize}[left=0cm]
    \item Ensuring that collected data meets the needs of AI/ML tasks -- requiring structure, reliability, and task relevance.
    \item Designing a player experience that remains engaging over time --avoiding disengagement due to artificial constraints or coercive incentives.
\end{itemize}
Achieving this balance is a multi-objective optimization problem that requires close collaboration between ML researchers and game developers. This design space spans multiple approaches, including creating entirely new games optimized for data collection, leveraging existing games to re-purpose organic player interactions, or introducing game-like elements into existing data collection tasks to enhance engagement. Each of these approaches comes with trade-offs in control, scalability, and long-term sustainability.

While prior efforts have successfully optimized for quality and quantity, sustaining trust has proven far more elusive, leading to short-lived or unsustainable solutions. Historical implementations offer valuable lessons\textemdash{}GWAP demonstrated the potential for high-quality, high-quantity crowdsourced data but did not endure over time. In contrast, ReCAPTCHA~\cite{von2008recaptcha} conceptualized in 2008, used for annotating books and self-driving data~\cite{captcha_google,captcha_nyt}, remains widely used today but has blurred the line between voluntary participation and coercion, with users reporting frustration and annoyance~\cite{searles2023empirical,searles2023dazed}, alluding to lack of trust. These examples illustrate the challenge of designing systems that optimize for all three aspects: \textbf{high-quality}, \textbf{high-quantity}, and \textbf{high-trust} data collection.

% While prior efforts have successfully optimized for quality and quantity, sustaining trust has proven far more elusive, leading to short-lived or unsustainable solutions. Historical implementations offer valuable lessons -- GWAP demonstrated the potential for high-quality crowdsourced data but lacked sustainability, while ReCAPTCHA~\cite{von2008recaptcha} used for annotating books and self-driving data~\cite{captcha_google,captcha_nyt} has been highly sustainable but did so in a way that blurred the line between voluntary participation and coercion, undermining user trust~\cite{searles2023empirical,searles2023dazed}. These examples illustrate the challenge of designing systems that optimize for all three aspects: high-quality, high-quantity, and high-trust data collection.

This challenge becomes even more evident when considering long-term sustainability. While data collection games have proven effective in gathering large-scale, high-quality data, few have endured over time. To our knowledge, no sustained data collection effort through games exists in machine learning. However, other research fields provide examples of systems that have successfully maintained engagement and trust over extended periods. Scientific discovery platforms such as Zooniverse for citizen science in astronomy~\cite{cardamone2009galaxy,lintott2008galaxy}, Lab in the Wild for HCI and psychology user studies~\cite{reinecke2015labinthewild}, FoldIt for protein folding~\cite{khatib2011crystal,cooper2010predicting}, and even Eve Online for epidemiology research~\cite{kafai2016connected} demonstrate how engagement, motivation, and trust can be sustained without relying on short-lived external incentives. Psychology and cognitive science studies~\cite{allen2024using} further reinforce that game-like participation can be structured in ways that maintain engagement over time.

While games and other naturalistic environments show significant potential towards achieving higher quality and quantity at the same time, they come with their own unique benefits and challenges previously not encountered with data collection systems:

\textbf{Benefits} include \textbf{the collection of previously inaccessible data}, such as natural dialogue interactions. Human conversations are often behind privacy walls, and attempting to collect dialogue through traditional data collection platforms is impractical, because unlike simple annotation tasks, dialogues need to be organic with real-time back-and-forth exchanges. In such cases, LLM-generated synthetic data has been a relief, with datasets like SODA~\cite{kim2022soda} playing a pivotal role in advancing dialogue systems. However, games offer an ideal setting where natural dialogues happen organically in non-real-world contexts, often under pseudonyms, making privacy concerns less of an issue. For instance, Minecraft has been leveraged to collect goal-driven player dialogues~\cite{narayan2019collaborative}, demonstrating how game environments facilitate authentic, context-rich communication.

Benefits also arise from the nature of games, including their ability to \textbf{collect data for open-ended tasks}, which has traditionally been challenging~\cite{karpinska2021perils}, especially over the long term. Moreover, games provide a unique advantage by enabling multi-modal data collection\textemdash{}capturing vision, language, and speech for the same actions\textemdash{}thus supporting advancements in embodied AI research. This has already been demonstrated in virtual environments like Habitat~\cite{puig2023habitat} and AI2-THOR~\cite{kolve2017ai2}, albeit without game-like objectives or direct human involvement to learn from yet.


\textbf{Challenges} include \textbf{difficulty in designing meaningful intrinsic rewards}, as motivations and preferences naturally vary across individuals and cultures. Unlike money, which serves as a universal incentive, intrinsic rewards must align with diverse motivational drivers\textemdash{}curiosity, competition, or self-improvement\textemdash{}to sustain engagement~\cite{jun2017types}. This also impacts demographic representation, as cultural and personal factors shape motivation~\cite{deci1985self}. Certain groups may be overrepresented while others disengage based on how well rewards align with their preferences~\cite{triandis1995individualism}. To ensure fairness, intrinsically motivated data collection must be designed to encourage diverse participation rather than benefiting only specific user segments. 

While games are often seen as leisure activities or ``unproductive,'' repurposing them for data collection contributes to value creation, in which case, it is important to recognize and fairly compensate contributors. The challenge is that, unlike traditional annotation tasks, AI data attribution is ambiguous, making it difficult to measure and \textbf{distribute compensation effectively}. One way to address this is by ensuring that contributors have a stake in the value their data generates, potentially through decentralized compensation models~\cite{loyalAI}. However, in general, compensation must be carefully structured to avoid distorting intrinsic motivation. If participants anticipate compensation upfront, their engagement risks becoming extrinsically driven, leading to lower-quality contributions. A potential solution is delayed, post-hoc recognition, where participants are acknowledged after the fact, ensuring they remain intrinsically engaged while still benefiting from their contributions.

Beyond incentive design, another fundamental challenge is ensuring that game-derived data remains both ethical and useful. Not all games inherently separate real-world contexts, which may necessitate distinguishing user-identifiable behavior from game-playing behavior for privacy reasons~\cite{nair2022exploring}. More broadly, extracting useful and generalizable data from inherently noisy and unstructured game records can be a significant challenge, similar to the difficulties faced in filtering large-scale Internet data~\cite{fang2023data}.
\section{Conclusion}

Subgroup analysis is an important, yet under-utilized tool in data science.
Our results suggest that combining algorithm-generated, rule-based insights with human intuition and experimentation in an interactive workflow can help practitioners develop a thorough understanding of complex datasets.
By implementing these interactions in a lightweight notebook-based tool, we hope to lower the barrier for data scientists to try subgroup discovery and to curate unexpected, interesting subpopulations in their data.
Divisi is available as an open-source package so that data scientists and HCI researchers can build on this work, helping to make exploratory subgroup analysis more feasible for a wider range of contexts.

% 
% humans are sensitive to the way information is presented.

% introduce framing as the way we address framing. say something about political views and how information is represented.

% in this paper we explore if models show similar sensitivity.

% why is it important/interesting.



% thought - it would be interesting to test it on real world data, but it would be hard to test humans because they come already biased about real world stuff, so we tested artificial.


% LLMs have recently been shown to mimic cognitive biases, typically associated with human behavior~\citep{ malberg2024comprehensive, itzhak-etal-2024-instructed}. This resemblance has significant implications for how we perceive these models and what we can expect from them in real-world interactions and decisionmaking~\citep{eigner2024determinants, echterhoff-etal-2024-cognitive}.

The \textit{framing effect} is a well-known cognitive phenomenon, where different presentations of the same underlying facts affect human perception towards them~\citep{tversky1981framing}.
For example, presenting an economic policy as only creating 50,000 new jobs, versus also reporting that it would cost 2B USD, can dramatically shift public opinion~\cite{sniderman2004structure}. 
%%%%%%%% 图1:  %%%%%%%%%%%%%%%%
\begin{figure}[t]
    \centering
    \includegraphics[width=\columnwidth]{Figs/01.pdf}
    \caption{Performance comparison (Top-1 Acc (\%)) under various open-vocabulary evaluation settings where the video learners except for CLIP are tuned on Kinetics-400~\cite{k400} with frozen text encoders. The satisfying in-context generalizability on UCF101~\cite{UCF101} (a) can be severely affected by static bias when evaluating on out-of-context SCUBA-UCF101~\cite{li2023mitigating} (b) by replacing the video background with other images.}
    \label{fig:teaser}
\end{figure}


Previous research has shown that LLMs exhibit various cognitive biases, including the framing effect~\cite{lore2024strategic,shaikh2024cbeval,malberg2024comprehensive,echterhoff-etal-2024-cognitive}. However, these either rely on synthetic datasets or evaluate LLMs on different data from what humans were tested on. In addition, comparisons between models and humans typically treat human performance as a baseline rather than comparing patterns in human behavior. 
% \gabis{looks good! what do we mean by ``most studies'' or ``rarely'' can we remove those? or we want to say that we don't know of previous work doing both at the same time?}\gili{yeah the main point is that some work has done each separated, but not all of it together. how about now?}

In this work, we evaluate LLMs on real-world data. Rather than measuring model performance in terms of accuracy, we analyze how closely their responses align with human annotations. Furthermore, while previous studies have examined the effect of framing on decision making, we extend this analysis to sentiment analysis, as sentiment perception plays a key explanatory role in decision-making \cite{lerner2015emotion}. 
%Based on this, we argue that examining sentiment shifts in response to reframing can provide deeper insights into the framing effect. \gabis{I don't understand this last claim. Maybe remove and just say we extend to sentiment analysis?}

% Understanding how LLMs respond to framing is crucial, as they are increasingly integrated into real-world applications~\citep{gan2024application, hurlin2024fairness}.
% In some applications, e.g., in virtual companions, framing can be harnessed to produce human-like behavior leading to better engagement.
% In contrast, in other applications, such as financial or legal advice, mitigating the effect of framing can lead to less biased decisions.
% In both cases, a better understanding of the framing effect on LLMs can help develop strategies to mitigate its negative impacts,
% while utilizing its positive aspects. \gabis{$\leftarrow$ reading this again, maybe this isn't the right place for this paragraph. Consider putting in the conclusion? I think that after we said that people have worked on it, we don't necessarily need this here and will shorten the long intro}


% If framing can influence their outputs, this could have significant societal effects,
% from spreading biases in automated decision-making~\citep{ghasemaghaei2024understanding} to reducing public trust in AI-generated content~\citep{afroogh2024trust}. 
% However, framing is not inherently negative -- understanding how it affects LLM outputs can offer valuable insights into both human and machine cognition.
% By systematically investigating the framing effect,


%It is therefore crucial to systematically investigate the framing effect, to better understand and mitigate its impact. \gabis{This paragraph is important - I think that right now it's saying that we don't want models to be influenced by framing (since we want to mitigate its impact, right?) When we talked I think we had a more nuanced position?}




To better understand the framing effect in LLMs in comparison to human behavior,
we introduce the \name{} dataset (Section~\ref{sec:data}), comprising 1,000 statements, constructed through a three-step process, as shown in Figure~\ref{fig:fig1}.
First, we collect a set of real-world statements that express a clear negative or positive sentiment (e.g., ``I won the highest prize'').
%as exemplified in Figure~\ref{fig:fig1} -- ``I won the highest prize'' positive base statement. (2) next,
Second, we \emph{reframe} the text by adding a prefix or suffix with an opposite sentiment (e.g., ``I won the highest prize, \emph{although I lost all my friends on the way}'').
Finally, we collect human annotations by asking different participants
if they consider the reframed statement to be overall positive or negative.
% \gabist{This allows us to quantify the extent of \textit{sentiment shifts}, which is defined as labeling the sentiment aligning with the opposite framing, rather then the base sentiment -- e.g., voting ``negative'' for the statement ``I won the highest prize, although I lost all my friends on the way'', as it aligns with the opposite framing sentiment.}
We choose to annotate Amazon reviews, where sentiment is more robust, compared to e.g., the news domain which introduces confounding variables such as prior political leaning~\cite{druckman2004political}.


%While the implications of framing on sensitive and controversial topics like politics or economics are highly relevant to real-world applications, testing these subjects in a controlled setting is challenging. Such topics can introduce confounding variables, as annotators might rely on their personal beliefs or emotions rather than focusing solely on the framing, particularly when the content is emotionally charged~\cite{druckman2004political}. To balance real-world relevance with experimental reliability, we chose to focus on statements derived from Amazon reviews. These are naturally occurring, sentiment-rich texts that are less likely to trigger strong preexisting biases or emotional reactions. For instance, a review like ``The book was engaging'' can be framed negatively without invoking specific cultural or political associations. 

 In Section~\ref{sec:results}, we evaluate eight state-of-the-art LLMs
 % including \gpt{}~\cite{openai2024gpt4osystemcard}, \llama{}~\cite{dubey2024llama}, \mistral{}~\cite{jiang2023mistral}, \mixtral{}~\cite{mistral2023mixtral}, and \gemma{}~\cite{team2024gemma}, 
on the \name{} dataset and compare them against human annotations. We find  that LLMs are influenced by framing, somewhat similar to human behavior. All models show a \emph{strong} correlation ($r>0.57$) with human behavior.
%All models show a correlation with human responses of more than $0.55$ in Pearson's $r$ \gabis{@Gili check how people report this?}.
Moreover, we find that both humans and LLMs are more influenced by positive reframing rather than negative reframing. We also find that larger models tend to be more correlated with human behavior. Interestingly, \gpt{} shows the lowest correlation with human behavior. This raises questions about how architectural or training differences might influence susceptibility to framing. 
%\gabis{this last finding about \gpt{} stands in opposition to the start of the statement, right? Even though it's probably one of the largest models, it doesn't correlate with humans? If so, better to state this explicitly}

This work contributes to understanding the parallels between LLM and human cognition, offering insights into how cognitive mechanisms such as the framing effect emerge in LLMs.\footnote{\name{} data available at \url{https://huggingface.co/datasets/gililior/WildFrame}\\Code: ~\url{https://github.com/SLAB-NLP/WildFrame-Eval}}

%\gabist{It also raises fundamental philosophical and practical questions -- should LLMs aim to emulate human-like behavior, even when such behavior is susceptible to harmful cognitive biases? or should they strive to deviate from human tendencies to avoid reproducing these pitfalls?}\gabis{$\leftarrow$ also following Itay's comment, maybe this is better in the dicsussion, since we don't address these questions in the paper.} %\gabis{This last statement brings the nuance back, so I think it contradicts the previous parapgraph where we talked about ``mitigating'' the effect of framing. Also, I think it would be nice to discuss this a bit more in depth, maybe in the discussion section.}






% \input{sections/02-1-desiderata}
% \input{sections/02-2-motivation}
% \input{sections/02-3-fragmentation}
% \input{sections/02-issues}
% \input{sections/03-designspace}
% \section{Background on Causal Inference}
\label{sec:background-causal} 



 \newtextold{In this section, we 
 %formalize the notion of {\em Average Treatment Effect and understand the 
 review the basic concepts and key assumptions for inferring the effects of an intervention on the outcome on collected datasets without performing randomized controlled experiments. 
We use {\em Pearl's graphical causal model} for {\em observational causal analysis} \cite{pearl2009causal} to define these concepts.}


\par
\paratitle{Causal Inference and Causal DAGs} The primary goal of causal inference is to model causal dependencies between attributes and evaluate how changing one variable (referred to as intervention) would affect the other.
Pearl's Probabilistic Graphical Causal Model \cite{pearl2009causal} can be written as a tuple $(\exo, \edvar, Pr_{\exo}, \psi)$, where $\exo$ is a set of {\em exogenous} variables, $\Pr_{\exo}$ is the joint distribution of \exo, and $\edvar$ is a set of observed {\em endogenous variables}.
Here $\psi$ is a set of structural equations that encode dependencies among variables. The equation for $A \in \edvar$ takes the following form:
%that encode the dependencies among the variables.  These equations are of the form 
$$\psi_{A}: 
\dom(Pa_{\exo}(A)) {\times} \dom(Pa_{\edvar}(A)) \to \dom(A)$$
Here $Pa_{\exo}(A) {\subseteq} {\exo}$ and $Pa_{\edvar}(A) {\subseteq} \edvar \setminus \{A\}$ respectively denote the exogenous and endogenous parents of $A$. A causal relational model is associated with a directed acyclic graph ({\em causal DAG}) $G$, whose nodes are the endogenous variables $\edvar$ and there is a directed edge from $X$ to $O$ if  $X {\in} Pa_{\edvar}(O)$. The causal DAG obfuscates exogenous variables as they are unobserved. %Any given set of values for the exogenous variables completely determines the values of the endogenous variables by the structural equations (we do not need any known closed-form expressions of the structural equations in this work). 
The probability distribution $\Pr_{\exo}$ on exogenous variables $\exo$ induces a probability distribution  
on the endogenous variables $\edvar$ by the structural equations $\psi$.  A causal DAG can be constructed by a domain expert as in the above example, or using existing {\em causal discovery} algorithms~\cite{glymour2019review}. 



\begin{figure}
    \centering
    \small
    \begin{tikzpicture}[node distance=0.6cm and 1cm, every node/.style={minimum size=0.5cm}]
        \tikzset{vertex/.style = {draw, circle, align=center}}

        \node[vertex] (Ethnicity) {\bf\scriptsize{{Ethnicity}}};
        \node[vertex, right=0.3cm of Ethnicity] (Gender) {\bf{\scriptsize{Gender}}};
        \node[vertex, right=0.3cm of Gender] (Age) {\bf{\scriptsize{Age}}};
        \node[vertex, below=0.3cm of Gender] (Role) {\bf{\scriptsize{Role}}};
        \node[vertex, right=0.3cm of Role] (Education) {\bf{\small{\scriptsize{Education}}}};
        \node[vertex, below=0.3cm of Role] (Salary) {\bf{\scriptsize{Salary}}};

        \draw[->] (Ethnicity) -- (Salary);
        \draw[->] (Gender) -- (Role);
        \draw[->] (Age) -- (Role);
         \draw[->] (Education) -- (Role);
           \draw[->] (Education) -- (Salary);
             \draw[->] (Ethnicity) -- (Education);
                \draw[->] (Ethnicity) -- (Role);
             \draw[->] (Gender) -- (Education);
               \draw[->] (Age) -- (Education);
                 \draw[->] (Role) -- (Salary);
        \draw[->] (Gender) to[bend right] (Salary);
        \draw[->] (Age) -- (Salary);
    \end{tikzpicture}
    \caption{Partial causal DAG for the Stack Overflow dataset.}
    \label{fig:causal_DAG}
\end{figure}



 \begin{example}
Figure \ref{fig:causal_DAG} depicts a partial causal DAG for the SO dataset over the attributes in Table \ref{tab:data} as endogenous variables (we use a larger causal DAG with all 20 attributes in our experiments). 
  Given this causal DAG, we can observe that the role that a coder has in their company depends on their education, age gender and ethnicity.
\end{example}
\par


\par
\paratitle{Intervention} In Pearl's model, a treatment $T = t$ (on one or more variables) is considered as an {\em intervention} to a causal DAG by mechanically changing the DAG such that the values of node(s) of $T$ in $G$ are set to the value(s) in $t$, which is denoted by $\doop(T = t)$. Following this operation, the probability distribution of the nodes in the graph changes as the treatment nodes no longer depend on the values of their parents. Pearl's model gives an approach to estimate the new probability distribution by identifying the confounding factors $Z$ described earlier using conditions such as {\em d-separation} and {\em backdoor criteria} \cite{pearl2009causal}, which we do not discuss in this paper.


\par
\paratitle{Average Treatment Effect} The effects of an intervention are often measured by evaluating
% \par
% \paratitle{Causal inference, Treatment, ATE, and CATE}
% \newtextold{One of the primary goals  of {\em causal inference} is to estimate the effect of making a change in terms of a {\em treatment} $T$ (often referred to as an intervention)
% on the outcome $O$. 
% %A variable that is modified is often referred to as the treatment variable $T$ and the metric used to captures 
% The effect of treatment $T$ on outcome $O$ is measured by 
% %is known as 
{\em Conditional Average treatment effect (CATE)}, 
%a {\em treatment variable} $T$ on an outcome variable $O$ (e.g., what is the effect of higher \verb|Education| on \verb|Salary|). 
measuring the effect of an intervention on a subset of records~\cite{rubin1971use,holland1986statistics} by calculating the difference in average outcomes between the group that receives the treatment and the group that does not (called the {\em control} group), providing an estimate of how the intervention by $T$ influences an outcome $O$ for a given subpopulation. 
% Mathematically,
% \begin{equation}
%     %{\small ATE(T,O) = \mathbb{E}[O \mid \doop(T=1)] -      \mathbb{E}[O \mid \doop(T=0)]}
%     {\small ATE(T, O) = \mathbb{E}[O \mid \doop(T=1)] -  
%     \mathbb{E}[O \mid \doop(T=0)]}
% \label{eq:ate}
% \end{equation}
% In our work, where the treatment with maximum effect may vary among different subpopulations, we are interested in computing the \emph{Conditional Average Treatment Effect} (CATE), which measures the effect of a treatment on an outcome on \emph{a subset of input units}~\cite{rubin1971use,holland1986statistics}. 
Given a subset of the records defined by (a vector of) attributes $B$ and their values $b$, 
%g {\in} \Qagg(\db)$ defined by a predicate $G {=} g$ 
we can compute $CATE(T,O \mid B = b)$ as:
{
\begin{eqnarray}    
    %CATE(T,O \mid G=g) = \mathbb{E}[O \mid \doop(T=1)&, G=g] -  \mathbb{E}[O \mid \doop(T=0), G=g] 
   % CATE(T,O \mid B = b) = 
    \mathbb{E}[O \mid \doop(T=1), B = b] -  
    \mathbb{E}[O \mid \doop(T=0), B = b]\label{eq:cate}
\end{eqnarray}
}
Setting $B=\phi$ is equivalent to the ATE estimate.
The above definitions assumes that the treatment assigned to one unit does not affect the outcome of another unit (called the {Stable Unit Treatment Value Assumption (SUTVA)) \cite{rubin2005causal}}\footnote{This assumption does not hold for causal inference on multiple tables and even on a single table where tuples depend on each other.}. 


The ideal way of estimating the ATE and CATE is through {\em randomized controlled experiments}, 
where the population is randomly divided into two groups (treated and control, for binary treatments): 
%treated group that receives the treatment and control group that does not (denoted by 
%{the \em treated} group 
denoted by 
$\doop(T = 1)$ 
%for a binary treatment)  (the {\em control} group, 
and $\doop(T = 0)$ resp.)~\cite{pearl2009causal}.
%\sr{edited up to here, going to read the rest first, this section should not look like causumx}
%\par
%\par
However, randomized experiments cannot always be performed due to ethical or feasibility issues. In these scenarios, observational data is used to estimate the treatment effect, which requires the following additional assumptions. 
% {\em Observational Causal Analysis} still allows sound causal inference under additional assumptions. Randomization in controlled trials mitigates the effect of {\em confounding factors}, i.e., attributes that can affect the treatment assignment and outcome. Suppose we want to understand the causal effect of \verb|Education| on \verb|Salary| from the SO dataset.  %in Example~\ref{ex:running_example}. 
% We no longer apply Eq. (\ref{eq:ate}) since the values of \verb|Education| were not assigned at random in this data, and obtaining higher education largely depends on other attributes like \verb|Gender|, \verb|Age|, and \verb|Country|. 
% Pearl's model provides ways to account for these confounding attributes $Z$ to get an unbiased causal estimate from observational data under the following assumptions ($\independent$ denotes independence):
% \vspace{-2mm}
\newtextold{
The first assumption is called {\em unconfoundedness} or {\em strong ignorability}  \cite{rosenbaum1983central} says that the independence of outcome $O$ and treatment $T$ conditioning on a set of confounder variables  (covariates) $Z$, i.e.,
%\begin{eqnarray}
 $    O \independent T | Z {=} z$.
 %\label{eq:unconfoundedness}
%\end{eqnarray}
The second assumption called {\em overlap or positivity} says that there is a chance of observing individuals in both the treatment and control groups for every combination of covariate values, i.e., 
%\begin{eqnarray}
   $ 0 < Pr(T {=} 1 ~~|~~Z {=} z)< 1 $.
   %\label{eq:overlap}
%\end{eqnarray}
}
%\sg{Is this overlap or positivity? maybe both are the same?} \sr{yeah - same - from Google AI - The overlap assumption, also known as the positivity assumption, is a key assumption in causal inference that states that there is a chance of observing individuals in both the treatment and control groups for every combination of covariate values.}
% The above conditions are known as {\em Strong Ignorability} in Rubin's model \cite{rubin2005causal}.
The unconfoundedness assumption requires that the treatment $T$ and the outcome $O$ be independent when conditioned on a set of variables $Z$. In SO, assuming that only $Z$ =\{\verb|Gender|, \verb|Age|, \verb|Country|\} affects $T = $ \verb|Education|, if we condition on a fixed set of values of $Z$, i.e., consider people of a given gender, from a given country, and at a given age, then $T = $ \verb|Education| and $O = $ \verb|Salary| are independent. For such confounding factors $Z$,  Eq. (\ref{eq:cate}) reduces to the following form 
(positivity 
gives the feasibility of the expectation difference): 
 \vspace{-1mm}
{\small
\begin{flalign}    
% \begin{eqnarray}
   % % & ATE(T,O) = \mathbb{E}_Z \left[\mathbb{E}[O \mid T=1, Z = z] -  
   %  \mathbb{E}[O \mid T=0, Z = z] \right] \label{eq:conf-ate}\\
 & CATE(T,O {\mid} B {=} b) {=} \nonumber
    \mathbb{E}_Z \left[\mathbb{E}[O {\mid} T{=}1, B {=} b, Z {=} z] {-}  
    \mathbb{E}[O {\mid} T{=}0, B {=} b, Z {=} z]\right]\label{eq:conf-cate}
\end{flalign}
% \end{eqnarray}
}
% \vspace{-4mm}
This equation contains conditional probabilities and not $\doop(T = b)$, which can be estimated from an observed data. 
Pearl's model gives a systematic way to find such a $Z$ when a causal DAG is available. 




% \input{sections/03-propositions}
% \input{sections/04-theory}
% \input{sections/05-implications}
% \input{sections/06-design-space}
% \input{sections/02-data_quality}
% \input{sections/03-human_factors}
% \input{sections/04-case_studies}
% \input{sections/05-dichotomy}
% \section{Discussion}
\label{sec:discussion}

The results provide valuable insights into the limitations of machine learning (ML) models to support systematic literature review (SLR) updates. In this discussion, we interpret these results in light of the research questions, contextualize their implications, and outline the trade-offs associated with applying ML models in this domain.

\subsection{Effectiveness of ML Models for SLR Study Selection (RQ1)}

The results for RQ1 indicate that our best-performing model, Random Forest (RF), achieved a modest balance between precision and recall with an F-score of 0.33 at the default threshold of 0.5. This result suggests that while the ML model was able to identify some relevant studies, its overall ability to precisely distinguish between relevant and irrelevant studies was limited. Adjusting the threshold improved the F-score to 0.41, highlighting the sensitivity of the model’s performance to the chosen threshold. However, this improvement came at the cost of increasing false negatives (FNs), potentially missing valuable studies. We interpret the RF model’s performance as indicating that ML may assist in informally identifying a subset of relevant studies but is not yet reliable for the selection of studies for SLR updates.

\subsection{Effort Reduction through ML Models (RQ2)}

In answering RQ2, we focused on maximizing recall to avoid FNs. In our investigations, the SVM model was more suitable for focusing on achieving a high recall and demonstrating some potential for reducing human screening efforts. Results demonstrated that with a recall of 100\%, the SVM model could exclude 33.9\% of studies from the review process without missing any relevant studies. This reduction represents a significant decrease in the manual workload, suggesting ML’s potential to assist researchers with the initial screening stage. However, to achieve this high recall, the model produced a high rate of false positives (FPs), still requiring significant human review effort to discard many non-relevant studies.

As shown in Table \ref{tab:effort_reduction}, gradually increasing the inclusion probability threshold reduced the number of FPs at the cost of a minor drop in recall. For instance, at a threshold of 0.75, the model achieved a recall of 97.37\%, with a reduction of 48.3\% in the number of studies needing review. We interpret this result as indicating that, while ML can reduce screening efforts, care must be taken when applying thresholds to avoid introducing a risk of overlooking critical studies.

\subsection{Supporting Human Reviewers (RQ3)}

For RQ3, we evaluated the support ML could provide compared to that of an additional human reviewer. When we treated the RF model as an additional reviewer and calculated Euclidean Distance (ED) to assess alignment with the final inclusion decision, individual human reviewers outperformed the RF model. Furthermore, pairs of human reviewers clearly outperformed human-ML pairs, suggesting that human-only review teams achieve more accurate results.

This finding reinforces the challenges ML models face in fully replicating the nuanced judgment of human reviewers. Hence, ML can not replace additional human reviewers, and ML assistance is not a valid argument for quality in the selection process. Pairs of human reviewers are still highly recommended for selecting studies in SLR updates.
% \section{Conclusion}\label{sec:conclusion}
This work introduces a novel approach to TOT query elicitation, leveraging LLMs and human participants to move beyond the limitations of CQA-based datasets. Through system rank correlation and linguistic similarity validation, we demonstrate that LLM- and human-elicited queries can effectively support the simulated evaluation of TOT retrieval systems. Our findings highlight the potential for expanding TOT retrieval research into underrepresented domains while ensuring scalability and reproducibility. The released datasets and source code provide a foundation for future research, enabling further advancements in TOT retrieval evaluation and system development.


% 
% humans are sensitive to the way information is presented.

% introduce framing as the way we address framing. say something about political views and how information is represented.

% in this paper we explore if models show similar sensitivity.

% why is it important/interesting.



% thought - it would be interesting to test it on real world data, but it would be hard to test humans because they come already biased about real world stuff, so we tested artificial.


% LLMs have recently been shown to mimic cognitive biases, typically associated with human behavior~\citep{ malberg2024comprehensive, itzhak-etal-2024-instructed}. This resemblance has significant implications for how we perceive these models and what we can expect from them in real-world interactions and decisionmaking~\citep{eigner2024determinants, echterhoff-etal-2024-cognitive}.

The \textit{framing effect} is a well-known cognitive phenomenon, where different presentations of the same underlying facts affect human perception towards them~\citep{tversky1981framing}.
For example, presenting an economic policy as only creating 50,000 new jobs, versus also reporting that it would cost 2B USD, can dramatically shift public opinion~\cite{sniderman2004structure}. 
%%%%%%%% 图1:  %%%%%%%%%%%%%%%%
\begin{figure}[t]
    \centering
    \includegraphics[width=\columnwidth]{Figs/01.pdf}
    \caption{Performance comparison (Top-1 Acc (\%)) under various open-vocabulary evaluation settings where the video learners except for CLIP are tuned on Kinetics-400~\cite{k400} with frozen text encoders. The satisfying in-context generalizability on UCF101~\cite{UCF101} (a) can be severely affected by static bias when evaluating on out-of-context SCUBA-UCF101~\cite{li2023mitigating} (b) by replacing the video background with other images.}
    \label{fig:teaser}
\end{figure}


Previous research has shown that LLMs exhibit various cognitive biases, including the framing effect~\cite{lore2024strategic,shaikh2024cbeval,malberg2024comprehensive,echterhoff-etal-2024-cognitive}. However, these either rely on synthetic datasets or evaluate LLMs on different data from what humans were tested on. In addition, comparisons between models and humans typically treat human performance as a baseline rather than comparing patterns in human behavior. 
% \gabis{looks good! what do we mean by ``most studies'' or ``rarely'' can we remove those? or we want to say that we don't know of previous work doing both at the same time?}\gili{yeah the main point is that some work has done each separated, but not all of it together. how about now?}

In this work, we evaluate LLMs on real-world data. Rather than measuring model performance in terms of accuracy, we analyze how closely their responses align with human annotations. Furthermore, while previous studies have examined the effect of framing on decision making, we extend this analysis to sentiment analysis, as sentiment perception plays a key explanatory role in decision-making \cite{lerner2015emotion}. 
%Based on this, we argue that examining sentiment shifts in response to reframing can provide deeper insights into the framing effect. \gabis{I don't understand this last claim. Maybe remove and just say we extend to sentiment analysis?}

% Understanding how LLMs respond to framing is crucial, as they are increasingly integrated into real-world applications~\citep{gan2024application, hurlin2024fairness}.
% In some applications, e.g., in virtual companions, framing can be harnessed to produce human-like behavior leading to better engagement.
% In contrast, in other applications, such as financial or legal advice, mitigating the effect of framing can lead to less biased decisions.
% In both cases, a better understanding of the framing effect on LLMs can help develop strategies to mitigate its negative impacts,
% while utilizing its positive aspects. \gabis{$\leftarrow$ reading this again, maybe this isn't the right place for this paragraph. Consider putting in the conclusion? I think that after we said that people have worked on it, we don't necessarily need this here and will shorten the long intro}


% If framing can influence their outputs, this could have significant societal effects,
% from spreading biases in automated decision-making~\citep{ghasemaghaei2024understanding} to reducing public trust in AI-generated content~\citep{afroogh2024trust}. 
% However, framing is not inherently negative -- understanding how it affects LLM outputs can offer valuable insights into both human and machine cognition.
% By systematically investigating the framing effect,


%It is therefore crucial to systematically investigate the framing effect, to better understand and mitigate its impact. \gabis{This paragraph is important - I think that right now it's saying that we don't want models to be influenced by framing (since we want to mitigate its impact, right?) When we talked I think we had a more nuanced position?}




To better understand the framing effect in LLMs in comparison to human behavior,
we introduce the \name{} dataset (Section~\ref{sec:data}), comprising 1,000 statements, constructed through a three-step process, as shown in Figure~\ref{fig:fig1}.
First, we collect a set of real-world statements that express a clear negative or positive sentiment (e.g., ``I won the highest prize'').
%as exemplified in Figure~\ref{fig:fig1} -- ``I won the highest prize'' positive base statement. (2) next,
Second, we \emph{reframe} the text by adding a prefix or suffix with an opposite sentiment (e.g., ``I won the highest prize, \emph{although I lost all my friends on the way}'').
Finally, we collect human annotations by asking different participants
if they consider the reframed statement to be overall positive or negative.
% \gabist{This allows us to quantify the extent of \textit{sentiment shifts}, which is defined as labeling the sentiment aligning with the opposite framing, rather then the base sentiment -- e.g., voting ``negative'' for the statement ``I won the highest prize, although I lost all my friends on the way'', as it aligns with the opposite framing sentiment.}
We choose to annotate Amazon reviews, where sentiment is more robust, compared to e.g., the news domain which introduces confounding variables such as prior political leaning~\cite{druckman2004political}.


%While the implications of framing on sensitive and controversial topics like politics or economics are highly relevant to real-world applications, testing these subjects in a controlled setting is challenging. Such topics can introduce confounding variables, as annotators might rely on their personal beliefs or emotions rather than focusing solely on the framing, particularly when the content is emotionally charged~\cite{druckman2004political}. To balance real-world relevance with experimental reliability, we chose to focus on statements derived from Amazon reviews. These are naturally occurring, sentiment-rich texts that are less likely to trigger strong preexisting biases or emotional reactions. For instance, a review like ``The book was engaging'' can be framed negatively without invoking specific cultural or political associations. 

 In Section~\ref{sec:results}, we evaluate eight state-of-the-art LLMs
 % including \gpt{}~\cite{openai2024gpt4osystemcard}, \llama{}~\cite{dubey2024llama}, \mistral{}~\cite{jiang2023mistral}, \mixtral{}~\cite{mistral2023mixtral}, and \gemma{}~\cite{team2024gemma}, 
on the \name{} dataset and compare them against human annotations. We find  that LLMs are influenced by framing, somewhat similar to human behavior. All models show a \emph{strong} correlation ($r>0.57$) with human behavior.
%All models show a correlation with human responses of more than $0.55$ in Pearson's $r$ \gabis{@Gili check how people report this?}.
Moreover, we find that both humans and LLMs are more influenced by positive reframing rather than negative reframing. We also find that larger models tend to be more correlated with human behavior. Interestingly, \gpt{} shows the lowest correlation with human behavior. This raises questions about how architectural or training differences might influence susceptibility to framing. 
%\gabis{this last finding about \gpt{} stands in opposition to the start of the statement, right? Even though it's probably one of the largest models, it doesn't correlate with humans? If so, better to state this explicitly}

This work contributes to understanding the parallels between LLM and human cognition, offering insights into how cognitive mechanisms such as the framing effect emerge in LLMs.\footnote{\name{} data available at \url{https://huggingface.co/datasets/gililior/WildFrame}\\Code: ~\url{https://github.com/SLAB-NLP/WildFrame-Eval}}

%\gabist{It also raises fundamental philosophical and practical questions -- should LLMs aim to emulate human-like behavior, even when such behavior is susceptible to harmful cognitive biases? or should they strive to deviate from human tendencies to avoid reproducing these pitfalls?}\gabis{$\leftarrow$ also following Itay's comment, maybe this is better in the dicsussion, since we don't address these questions in the paper.} %\gabis{This last statement brings the nuance back, so I think it contradicts the previous parapgraph where we talked about ``mitigating'' the effect of framing. Also, I think it would be nice to discuss this a bit more in depth, maybe in the discussion section.}






% \input{sections_v1/02-data_quality}
% \input{sections_v1/03-human_factors}
% \input{sections_v1/04-case_studies}
% \input{sections_v1/05-dichotomy}
% \section{Discussion}
\label{sec:discussion}

The results provide valuable insights into the limitations of machine learning (ML) models to support systematic literature review (SLR) updates. In this discussion, we interpret these results in light of the research questions, contextualize their implications, and outline the trade-offs associated with applying ML models in this domain.

\subsection{Effectiveness of ML Models for SLR Study Selection (RQ1)}

The results for RQ1 indicate that our best-performing model, Random Forest (RF), achieved a modest balance between precision and recall with an F-score of 0.33 at the default threshold of 0.5. This result suggests that while the ML model was able to identify some relevant studies, its overall ability to precisely distinguish between relevant and irrelevant studies was limited. Adjusting the threshold improved the F-score to 0.41, highlighting the sensitivity of the model’s performance to the chosen threshold. However, this improvement came at the cost of increasing false negatives (FNs), potentially missing valuable studies. We interpret the RF model’s performance as indicating that ML may assist in informally identifying a subset of relevant studies but is not yet reliable for the selection of studies for SLR updates.

\subsection{Effort Reduction through ML Models (RQ2)}

In answering RQ2, we focused on maximizing recall to avoid FNs. In our investigations, the SVM model was more suitable for focusing on achieving a high recall and demonstrating some potential for reducing human screening efforts. Results demonstrated that with a recall of 100\%, the SVM model could exclude 33.9\% of studies from the review process without missing any relevant studies. This reduction represents a significant decrease in the manual workload, suggesting ML’s potential to assist researchers with the initial screening stage. However, to achieve this high recall, the model produced a high rate of false positives (FPs), still requiring significant human review effort to discard many non-relevant studies.

As shown in Table \ref{tab:effort_reduction}, gradually increasing the inclusion probability threshold reduced the number of FPs at the cost of a minor drop in recall. For instance, at a threshold of 0.75, the model achieved a recall of 97.37\%, with a reduction of 48.3\% in the number of studies needing review. We interpret this result as indicating that, while ML can reduce screening efforts, care must be taken when applying thresholds to avoid introducing a risk of overlooking critical studies.

\subsection{Supporting Human Reviewers (RQ3)}

For RQ3, we evaluated the support ML could provide compared to that of an additional human reviewer. When we treated the RF model as an additional reviewer and calculated Euclidean Distance (ED) to assess alignment with the final inclusion decision, individual human reviewers outperformed the RF model. Furthermore, pairs of human reviewers clearly outperformed human-ML pairs, suggesting that human-only review teams achieve more accurate results.

This finding reinforces the challenges ML models face in fully replicating the nuanced judgment of human reviewers. Hence, ML can not replace additional human reviewers, and ML assistance is not a valid argument for quality in the selection process. Pairs of human reviewers are still highly recommended for selecting studies in SLR updates.
% \section{Conclusion}\label{sec:conclusion}
This work introduces a novel approach to TOT query elicitation, leveraging LLMs and human participants to move beyond the limitations of CQA-based datasets. Through system rank correlation and linguistic similarity validation, we demonstrate that LLM- and human-elicited queries can effectively support the simulated evaluation of TOT retrieval systems. Our findings highlight the potential for expanding TOT retrieval research into underrepresented domains while ensuring scalability and reproducibility. The released datasets and source code provide a foundation for future research, enabling further advancements in TOT retrieval evaluation and system development.

\bibliographystyle{ACM-Reference-Format}
\bibliography{references}


%%
%% If your work has an appendix, this is the place to put it.
\appendix

\end{document}
\endinput
%%
%% End of file `sample-sigconf.tex'.
