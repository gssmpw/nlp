\pdfoutput=1

%%
%% This is file `sample-sigconf.tex',
%% generated with the docstrip utility.
%%
%% The original source files were:
%%
%% samples.dtx  (with options: `all,proceedings,bibtex,sigconf')
%% 
%% IMPORTANT NOTICE:
%% 
%% For the copyright see the source file.
%% 
%% Any modified versions of this file must be renamed
%% with new filenames distinct from sample-sigconf.tex.
%% 
%% For distribution of the original source see the terms
%% for copying and modification in the file samples.dtx.
%% 
%% This generated file may be distributed as long as the
%% original source files, as listed above, are part of the
%% same distribution. (The sources need not necessarily be
%% in the same archive or directory.)
%%
%%
%% Commands for TeXCount
%TC:macro \cite [option:text,text]
%TC:macro \citep [option:text,text]
%TC:macro \citet [option:text,text]
%TC:envir table 0 1
%TC:envir table* 0 1
%TC:envir tabular [ignore] word
%TC:envir displaymath 0 word
%TC:envir math 0 word
%TC:envir comment 0 0
%%
%%
%% The first command in your LaTeX source must be the \documentclass
%% command.
%%
%% For submission and review of your manuscript please change the
%% command to \documentclass[manuscript, screen, review]{acmart}.
%%
%% When submitting camera ready or to TAPS, please change the command
%% to \documentclass[sigconf]{acmart} or whichever template is required
%% for your publication.
%%
%%
\documentclass[sigconf, nonacm]{acmart}

\usepackage{xcolor,colortbl}
\usepackage{enumitem}
\usepackage{array,multirow,graphicx}
\usepackage{tabularx}
\usepackage{fontawesome}
\usepackage{ulem}
\usepackage{hyperref}



% \usepackage{arydshln}

% \usepackage{tcolorbox}
% % \definecolor{bg}{RGB}{245,245,245}
% \definecolor{bg}{RGB}{255,255,255}
% \definecolor{bgbord}{RGB}{235,235,235}
% % \definecolor{extitle}{HTML}{455a64}
% \definecolor{extitle}{HTML}{000000}
% \newtcolorbox{examplebox}[1]{colback=bgbord,colframe=bgbord,arc=0pt,coltitle=extitle, title=#1,left=4pt, top=-4pt, bottom=4pt, right=4pt, width=\linewidth, before=\par\smallskip\centering, fonttitle=\Large,}

\usepackage{tcolorbox}
\definecolor{bgbord}{RGB}{220,220,220}  % Slightly darker border
\definecolor{bg}{RGB}{250,250,250}  % Soft gray background
\definecolor{extitle}{HTML}{222222}  % Slightly darker title

\newtcolorbox{examplebox}[1]{
    colback=bg, 
    colframe=bgbord, 
    arc=1pt,  % Slightly rounded corners
    coltitle=extitle, 
    title=#1,
    left=6pt, top=4pt, bottom=4pt, right=6pt,
    width=\linewidth,
    before=\par\smallskip\centering,
    fonttitle=\large\bfseries
}


%%
%% \BibTeX command to typeset BibTeX logo in the docs
\AtBeginDocument{%
  \providecommand\BibTeX{{%
    Bib\TeX}}%
  \hypersetup{
    colorlinks=true,
    linkcolor=myorange,
    citecolor=myorange,
    urlcolor=myorange,
    pdfborder={0 0 0}
  }
}

%% Rights management information.  This information is sent to you
%% when you complete the rights form.  These commands have SAMPLE
%% values in them; it is your responsibility as an author to replace
%% the commands and values with those provided to you when you
%% complete the rights form.
% \setcopyright{acmlicensed}
% \copyrightyear{2018}
% \acmYear{2018}
% \acmDOI{XXXXXXX.XXXXXXX}

%% These commands are for a PROCEEDINGS abstract or paper.
% \acmConference[Conference acronym 'XX]{Make sure to enter the correct
%   conference title from your rights confirmation emai}{June 03--05,
%   2018}{Woodstock, NY}
  
%%
%%  Uncomment \acmBooktitle if the title of the proceedings is different
%%  from ``Proceedings of ...''!
%%
%%\acmBooktitle{Woodstock '18: ACM Symposium on Neural Gaze Detection,
%%  June 03--05, 2018, Woodstock, NY}
% \acmISBN{978-1-4503-XXXX-X/18/06}

\settopmatter{printacmref=false} % Removes ACM reference styling


%%
%% Submission ID.
%% Use this when submitting an article to a sponsored event. You'll
%% receive a unique submission ID from the organizers
%% of the event, and this ID should be used as the parameter to this command.
%%\acmSubmissionID{123-A56-BU3}

%%
%% For managing citations, it is recommended to use bibliography
%% files in BibTeX format.
%%
%% You can then either use BibTeX with the ACM-Reference-Format style,
%% or BibLaTeX with the acmnumeric or acmauthoryear sytles, that include
%% support for advanced citation of software artefact from the
%% biblatex-software package, also separately available on CTAN.
%%
%% Look at the sample-*-biblatex.tex files for templates showcasing
%% the biblatex styles.
%%

%%
%% The majority of ACM publications use numbered citations and
%% references.  The command \citestyle{authoryear} switches to the
%% "author year" style.
%%
%% If you are preparing content for an event
%% sponsored by ACM SIGGRAPH, you must use the "author year" style of
%% citations and references.
%% Uncommenting
%% the next command will enable that style.
\citestyle{acmauthoryear}


%%
%% end of the preamble, start of the body of the document source.
 
% \setcitestyle{authoryear,open={(},close={)}}

\setlength{\parindent}{0pt}  % Removes indentation
\setlength{\parskip}{7pt}    % Adds spacing before a new paragraph

\usepackage{titlesec}

% \renewcommand{\thesection}{\Roman{section}}
% \titleformat{\section}{\LARGE\bfseries}{\thesection.}{0.3em}{}
\titleformat{\section}{\fontsize{11.5}{13}\bfseries}{\thesection.}{0.3em}{}

\titlespacing{\section}{0pt}{10pt}{-5pt}  


% \definecolor{myorange}{rgb}{0.8, 0.4, 0}

\definecolor{myorange}{rgb}{0.65, 0.25, 0}


\usepackage{xcolor}
\usepackage{natbib}

\makeatletter
\let\oldcite\cite
\let\oldcitep\citep

% Redefine \cite
\renewcommand{\cite}[1]{[\textcolor{myorange}{\begingroup
    \let\NAT@open\relax
    \let\NAT@close\relax
    \oldcite{#1}\endgroup}]}

\makeatother



% \let\oldcite\cite
% \renewcommand{\cite}[1]{{\color{myorange} \oldcite{#1}}}



\begin{document}

%%
%% The "title" command has an optional parameter,
%% allowing the author to define a "short title" to be used in page headers.
% \title{The Invisible Hand of Human Motivations in AI Data Quality}
\title{Economics of Sourcing Human Data}
% \title{Data Quality in AI: Impact of Human Motivation \\and Incentives on Machine Learning}

%%
%% The "author" command and its associated commands are used to define
%% the authors and their affiliations.
%% Of note is the shared affiliation of the first two authors, and the
%% "authornote" and "authornotemark" commands
%% used to denote shared contribution to the research.

% \author{Sebastin Santy}
% \affiliation{%
%   \institution{University of Washington}
%   \city{Seattle}
%   \country{USA}}
% \email{ssanty@cs.washington.edu}

% \author{Prasanta Bhattacharya}
% \affiliation{%
%   \institution{ASTAR, NUS}
%   \country{Singapore}
% }

% \author{Manoel Ribeiro}
% \affiliation{%
%  \institution{EPFL}
%  \city{Lausanne}
%  \country{Switzerland}}

% \author{Kelsey Allen}
% \affiliation{%
%   \institution{Google Deepmind}
%   \country{USA}}


\author{ Sebastin Santy$^{1}$ \: Prasanta Bhattacharya$^{2}$ \: Manoel Horta Ribeiro$^{3}$ \: Kelsey Allen$^{4}$ \: Sewoong Oh$^{1}$
}

\affiliation{
\institution{$^{1}$University of Washington \: $^{2}$Institute of High Performance Computing (IHPC), A*STAR\\[1.5pt] $^{3}$Princeton University \: $^{4}$University of British Columbia}
\city{}
\state{}
\country{}
}

\thanks{$^{\dagger}$Institute of High Performance Computing (IHPC), Agency for Science, Technology and Research (A*STAR), 1 Fusionopolis Way, \#16-16 Connexis, Singapore 138632, Republic of Singapore}
\thanks{Corresponding author: ssanty@cs.washington.edu}


%%
%% By default, the full list of authors will be used in the page
%% headers. Often, this list is too long, and will overlap
%% other information printed in the page headers. This command allows
%% the author to define a more concise list
%% of authors' names for this purpose.
\renewcommand{\shortauthors}{Santy et al.}

%%
%% The abstract is a short summary of the work to be presented in the
%% article.
% \begin{abstract}
%   A clear and well-documented \LaTeX\ document is presented as an
%   article formatted for publication by ACM in a conference proceedings
%   or journal publication. Based on the ``acmart'' document class, this
%   article presents and explains many of the common variations, as well
%   as many of the formatting elements an author may use in the
%   preparation of the documentation of their work.
% \end{abstract}
Humor is a social binding agent. It is an act of creativity that can provoke emotional reactions on a broad range of topics. Humor has long been thought to be “too human” for AI to generate. However, humans are complex, and humor requires our complex set of skills: cognitive reasoning, social understanding, a broad base of knowledge, creative thinking, and audience understanding. We explore whether giving AI such skills enables it to write humor. We target one audience: Gen Z humor fans. We ask people to rate meme caption humor from three sources: highly upvoted human captions, 2) basic LLMs, and 3) LLMs captions with humor skills. We find that users like LLMs captions with humor skills more than basic LLMs and almost on par with top-rated humor written by people. We discuss how giving AI human-like skills can help it generate communication that resonates with people. 


%%
%% The code below is generated by the tool at http://dl.acm.org/ccs.cfm.
%% Please copy and paste the code instead of the example below.
%%
% \begin{CCSXML}
% <ccs2012>
%  <concept>
%   <concept_id>00000000.0000000.0000000</concept_id>
%   <concept_desc>Do Not Use This Code, Generate the Correct Terms for Your Paper</concept_desc>
%   <concept_significance>500</concept_significance>
%  </concept>
%  <concept>
%   <concept_id>00000000.00000000.00000000</concept_id>
%   <concept_desc>Do Not Use This Code, Generate the Correct Terms for Your Paper</concept_desc>
%   <concept_significance>300</concept_significance>
%  </concept>
%  <concept>
%   <concept_id>00000000.00000000.00000000</concept_id>
%   <concept_desc>Do Not Use This Code, Generate the Correct Terms for Your Paper</concept_desc>
%   <concept_significance>100</concept_significance>
%  </concept>
%  <concept>
%   <concept_id>00000000.00000000.00000000</concept_id>
%   <concept_desc>Do Not Use This Code, Generate the Correct Terms for Your Paper</concept_desc>
%   <concept_significance>100</concept_significance>
%  </concept>
% </ccs2012>
% \end{CCSXML}

% \ccsdesc[500]{Do Not Use This Code~Generate the Correct Terms for Your Paper}
% \ccsdesc[300]{Do Not Use This Code~Generate the Correct Terms for Your Paper}
% \ccsdesc{Do Not Use This Code~Generate the Correct Terms for Your Paper}
% \ccsdesc[100]{Do Not Use This Code~Generate the Correct Terms for Your Paper}

%%
%% Keywords. The author(s) should pick words that accurately describe
%% the work being presented. Separate the keywords with commas.
% \keywords{Do, Not, Us, This, Code, Put, the, Correct, Terms, for,
%   Your, Paper}
%% A "teaser" image appears between the author and affiliation
%% information and the body of the document, and typically spans the
%% page.
% \begin{teaserfigure}
%   \includegraphics[width=\textwidth]{sampleteaser}
%   \caption{Seattle Mariners at Spring Training, 2010.}
%   \Description{Enjoying the baseball game from the third-base
%   seats. Ichiro Suzuki preparing to bat.}
%   \label{fig:teaser}
% \end{teaserfigure}

% \received{20 February 2007}
% \received[revised]{12 March 2009}
% \received[accepted]{5 June 2009}

%%
%% This command processes the author and affiliation and title
%% information and builds the first part of the formatted document.
\maketitle

\newcommand{\seb}[1]{\textcolor{orange}{[#1]}}
\newcommand{\todo}[1]{\textcolor{red}{[#1]}}

\definecolor{babyblueeyes}{rgb}{0.63, 0.79, 0.95}

\newcommand{\synopsis}{\noindent\textcolor{blue}{\textbf{Synopsis for section}\newline}}
{\newcommand{\synopsissection}[1]{\textcolor{babyblueeyes}{#1}}
\newcommand{\prose}{\noindent\textcolor{magenta}{\textbf{Prose for section}\newline}}
\newcommand{\references}{\noindent\textcolor{red}{\textbf{References for section}\newline}}

\newcommand{\removed}[1]{}
\newcommand{\added}[1]{#1}

\section{Human Data in Crisis}
Artificial Intelligence relies heavily on human-generated data to develop ever more capable models and systems that can emulate human-like intelligent behavior. The currently used sources of human data include: (1) human annotations, from data collection platforms (e.g., Amazon MTurk) and (2) raw data, from the Internet (e.g., Wikipedia, social media platforms). Our understanding of how to effectively use these two data sources has been a driving force behind the two most prominent eras of artificial intelligence: the \textbf{Deep Learning era} which began with AlexNet~\cite{krizhevsky2012imagenet} in 2012 and facilitated by ImageNet~\cite{deng2009imagenet} collected at scale through MTurk, and the \textbf{Pre-trained Language Models era} ushered in by BERT~\cite{devlin2018bert} in 2018 and enabled by the availability of large-scale Internet data.

However, the emergence of the third \textbf{Generative Large Language Chat models era}, marked by ChatGPT in 2022~\cite{openai2023chatgpt}, has made human-like language generation ubiquitous and accessible to the general public, disrupting the very mechanisms of data sourcing that have historically enabled AI development. The indiscriminate usage of large language models (LLMs) impacts the previous two key sources of human-produced data: participants in existing data collection systems are turning to LLMs to expedite their tasks~\cite{veselovsky2023artificial,veselovsky2023prevalence}, and the Internet is being flooded with LLM-generated content ~\cite{brooks2024rise}.
This makes it increasingly challenging to obtain and discern authentic human-generated data, raising concerns about a looming shortage of human data needed for continued AI progress.  To compensate, machine learning has further leaned into synthetic data\textemdash{}either to mimic human annotations~\cite{dubois2024alpacafarm} or emulate human behavior~\cite{argyle2023out, park2022social, park2023generative}\textemdash{} albeit not yet at the highest quality~\cite{geng2024unmet} and facing other challenges, such as model collapse~\cite{taori2023data, shumailov2024ai}, keeping the ember of human-generated data still alive~\cite{ashok2024little}.


We argue that these flaws have always existed in data collection platforms but have been recently amplified by LLMs to the point where their very existence is being called into question~\cite{pieces2025data}. Specifically,
{\bf
at the heart of this problem lies the issue of human incentives and motivations for contributing data—one that cannot be solved by simply increasing external rewards like pay but requires careful attention to intrinsic motivations that drive people to engage willingly and actively on platforms, which in turn leads to better-quality data.
} Designing new data collection systems necessitates a critical examination of the flaws in existing ones.


% \textcolor{red}{PB: We need a short para here to summarize the key propositions we're making in this paper in 2-3 lines e.g., a focus on intrinsic motivations requires ceding control at the task level, while designing enabling environments. A good example of where this has worked is data collection games. We equally emphasize the value of preserving user trust in sustaining high quality data collection systems etc.}

In this paper, we analyze the current data requirements in machine learning and how existing data collection systems attempt to meet them. We open up the black box of data collection\textemdash{}complex socio-technical systems shaped by human behaviors, idiosyncrasies, and technical constraints\textemdash{}drawing from popular theories and experiments in psychology, sociology, and economics. In doing so, we examine the quantity-quality tradeoff and argue that, while this tradeoff may not be entirely eliminable, the overall quality and quantity of data can still be improved by identifying and removing factors that undermine intrinsic human motivations. Given that data collection systems operate within broader economic and social structures, we also complement academic research with real-world discourse and case studies of different data collection strategies. Finally, we explore novel paradigms, including games, that offer promising directions for the future of human data sourcing.
\section{Characterizing Human Data Needs}
Progress in machine learning depends on the availability of data at a sufficient scale to inductively learn patterns from it~\cite{kaplan2020scaling,hoffmann2022training}. This need for data has grown exponentially as learning algorithms have evolved from statistical to deep learning and pre-trained language models. The \textit{quantity} of data has uncontestedly been the key consideration for the field~\cite{sutton2019bitter}, with any data source that adds several orders of magnitude to the size of existing datasets, such as data from the Internet, being considered indispensable. A general trend in machine learning regarding data sourcing, especially after the advent of pre-training with BERT~\cite{devlin2018bert}, has been to leverage sources of large data wherever they can be found, such as BookCorpus~\cite{zhu2015aligning}, Wikipedia~\cite{raffel2020exploring}, Reddit~\cite{Gokaslan2019OpenWeb}, and CommonCrawl~\cite{commoncrawl}.

Recently, however, as datasets have grown larger, the importance of \textit{quality} has become more apparent~\cite{nguyen2022quality,zhou2024lima,lee2021deduplicating}. While learning algorithms have improved in extracting signal from noise, they still have limits when faced with excessive noise or irrelevant data~\citetext{e.g., DataComp-LM discards 99\% of data and Text-Image DataComp filters out 70\%; \textcolor{myorange}{\citealt{gadre2024datacomp,li2024datacomp}}}. Data quality has long been assumed to matter, but its significance has become clearer than ever as models trained on external proprietary datasets consistently outperform others on benchmarks and in real-world applications~\cite{brown2020language}. This outperformance\textemdash{}often attributed to access to ``high-quality'' proprietary datasets, such as paywalled content or licensed secondary sources~\cite{bommasani2021opportunities}\textemdash{}has pushed the data quality discourse to the forefront and is now a high priority in machine learning.

\begin{tcolorbox}[
    colback=gray!15,   % Uniform gray background
    colframe=gray!15,  % Border matches background for a soft look
    coltitle=black,    % Title in black
    sharp corners,     % No rounded edges
    width=\linewidth,  % Full-width box
    boxrule=0pt,       % Removes harsh border
    left=10pt, right=10pt, top=7pt, bottom=8pt, % Adjust padding inside box
    before=\bigskip, % Space above box
    after=\bigskip,  % Space below box
    title={\fontsize{12}{15} \textbf{Human Data Sourcing Desiderata}}, % Enlarged title
    fonttitle=\bfseries, % Bold title
    before title={\vspace{10pt}}, % This adds extra spacing above the title
    titlerule=0mm, % Ensures no extra rule appears
]
\begin{enumerate}[leftmargin=14pt, label=(\arabic*), itemsep=10pt] % More spacing
    \item \textbf{High Quality} \\[3pt]
    Collecting high-quality data with a strong signal-to-noise ratio for training ML models.

    \item \textbf{High Quantity} \\[3pt]
    Collecting data large enough to satisfy the exponentially increasing complexity of tasks.
\end{enumerate}
\end{tcolorbox}


While both high quality and high quantity are critical for data sourcing, they often come at the expense of each other\textemdash{}improving one typically leads to a decline in the other. However, this trade-off is not an inherent property of data itself but rather a consequence of system design choices. One way to conceptualize this trade-off is as resembling a Pareto frontier, as illustrated in Figure~\ref{fig:quality-quantity}.

\begin{figure}[h]
    \centering
    \includegraphics[width=0.75\linewidth]{illustrations/pareto-frontier.png}
    \caption{Illustration of the quantity-quality trade-off in data collection systems. Platforms like MTurk, Prolific, and UpWork optimize for either scale or quality but struggle to achieve both simultaneously. In contrast, data from sources not explicitly designed for data collection\textemdash{}such as Wikipedia and Reddit\textemdash{}operate outside this trade-off, hinting at potential alternative paradigms.}
    \label{fig:quality-quantity}
\end{figure}

This trade-off explains why data collection systems struggle to balance quality and quantity. Platforms prioritizing quality, like freelance job platforms (e.g., UpWork), tend to be slower with lower output, while high-throughput systems, like rapid crowdwork platforms (e.g., MTurk), scale efficiently but often sacrifice consistency and quality~\cite{douglas2023data}. While this quantity-quality trade-off may never be fully eliminated for any designed data collection system, it is not a fixed constraint\textemdash{}rather than eliminating the trade-off, the key is to expand the frontier by addressing structural inefficiencies in incentive design, annotation methods, and human oversight.

The dynamics of the quantity-quality trade-off are shaped by multiple interacting and, often, latent factors. Untangling these factors requires opening up current data collection systems and examining their trade-offs through the lens of human behavior, organizational processes, and technical constraints. At a system level, quality depends on balancing intrinsic motivation with external incentives, while quantity is largely driven by process efficiency, often through task fragmentation and parallelization. However, excessive fragmentation can erode intrinsic motivation, leading to long-term declines in quality. This self-reinforcing cycle lies at the heart of the quantity-quality trade-off in data collection system design. 

\section{Position vs. Current Stance}
\noindent\textbf{Position:} Sustaining human-generated data for ML requires shifting focus toward intrinsic human motivations.

\noindent\textbf{Majoritarian Stance:} Data quality is a major consideration in machine learning, and many researchers and companies are actively exploring how to best collect high-quality human-produced data. Researchers recognize that incentives influence the level of effort contributed, and as a result, they often rely exclusively on financial incentives to encourage greater effort in data production. While incentives are important, we contend that solely relying on financial rewards\textemdash{}counter to intuition\textemdash{}risks backfiring by reducing the quality of the data collected. Instead, ML researchers should prioritize enhancing the intrinsic motivation of human participants, using external incentives sparingly as supportive nudges rather than primary drivers.
% \section{Defining and Measuring Data Quality}
\section{Understanding Data Quality}
While quantity is easily measurable and increasingly attainable through new data-sourcing methods, quality has become ever more elusive. As data availability has surged, the question of what constitutes ``high-quality'' data is increasingly debated in machine learning. The challenge then is to understand the existing notions of data quality and explore ways to make it more certain, before dissecting data collection systems.

\textbf{Prior Definitions.} Defining ``data quality'' has long been a challenge in machine learning, as it lacks a universal, quantifiable standard. While it is widely acknowledged that human-generated data varies in quality (e.g., curated datasets from specific websites being more reliable than scraped content), there is no single, absolute definition for what makes data ``high-quality''.  Attempts to define quality span both subjective and objective perspectives. Subjectively, quality is often linked to trustworthiness\textemdash{}Wikipedia, for instance, is generally regarded as more reliable than personal blogs~\cite{albalak2024survey,soldaini2024dolma}. Objectively, quality has been measured using statistical metrics (e.g., readability) or modeled metrics, such as GPT-3 Quality Filters~\cite{gururangan2022whose} and DataComp's curated datasets~\cite{li2024datacomp}, which define quality in the context of their downstream use. 


\textbf{Naturalness as the Basis of Data Quality.} Without clarity on data’s intended use, defining quality becomes challenging. We contend that one of the most intuitive ways to conceptualize quality\textemdash{}without presupposing a specific application\textemdash{}is by anchoring it in \textit{naturalness}: how people behave in routine activities, online or offline, in an authentic manner. Unlike other definitions that rely on the perceived reliability of the source or application-specific criteria, naturalness provides an observable and generalizable signal for what constitutes high-quality data. This pattern is evident in organic data sources, where humans naturally generate valuable data through meaningful tasks, such as editing Wikipedia, participating in Reddit discussions, or sharing artwork and photos on platforms like Flickr. In these settings, data is generated naturally, often without direct external incentives, making it more representative of authentic human behavior. %This already hints at the idea that intrinsic motivation is crucial for producing in-situ data that accurately reflects natural behavior. 
% Thus, we argue that naturalness provides the most robust and context-agnostic definition of data quality, as it explains and unifies prior perspectives in the literature.


\textbf{Is There a Case for Naturalness in AI Training?}
Naturalness has already been central to pretraining, where large-scale internet data\textemdash{}capturing naturally occurring human behavior\textemdash{}has been crucial to the success of LLMs. But its importance extends beyond achieving generalization. Even in supervised fine-tuning, where data is tailored for task-specific applications, naturalness matters, as collected data should ideally reflect real task engagement rather than behavior shaped by artificial constraints or incentives.

Interestingly, this divide between pretraining and fine-tuning mirrors a broader debate in AI. \textbf{Artificial Intelligence (AI)}\textemdash{}as envisioned by John McCarthy~\cite{mccarthy1987generality}\textemdash{}aims to achieve human-like general intelligence and, therefore, benefits from diverse, free-flowing human interactions, much like those found in pretraining data. In contrast, \textbf{Intelligence Augmentation (IA)}\textemdash{}as suggested by Douglas Engelbart~\cite{engelbart1962augmenting}\textemdash{}prioritizes enhancing human intelligence through specialized tools, requiring goal-oriented human interactions towards performing a task, similar to fine-tuning data. In both cases, naturalness remains key but plays distinct roles: capturing free-flowing or goal-oriented human interactions, free from artificial constraints or incentives.

To that end, the use of LLMs in data collection is not inherently bad\textemdash{}what matters is \textit{how} they are used. The real risk to naturalness comes from indiscriminate, careless reliance on AI as a shortcut, rather than as a tool for balancing meaningful engagement and productivity. Instead of aggressively policing AI use, the focus should be on designing environments where contributors engage with tasks in ways that make shortcuts feel unnecessary\textemdash{}just as one wouldn't feel compelled to take shortcuts in a personally meaningful hobby. For example, a survey respondent outsourcing an entire essay writing task to AI without any personal input demonstrates an unwillingness to engage meaningfully with the task\textemdash{}this is the kind of AI use that undermines data quality.

% \section{Quality Factors: Motivation \& Incentives}
\section{Human Factors in Quality\textemdash{}\\ Motivation \& Incentives}
Data collection platforms used in machine learning, such as MTurk, Prolific, UpWork, and ScaleAI, are designed in ways that include compensation structures, that directly influence both effort and data quality. While rapid crowdsourcing platforms (e.g., MTurk) favor low-effort and low-pay tasks that can scale easily, the quality of task output remains unreliable. In contrast, freelance job platforms tend to favor high-effort and higher-pay gigs which require more deliberate and engaged participation, often leading to higher quality outputs.

This difference aligns with a straightforward intuition\textemdash{}higher pay leads to greater effort and better-quality contributions~\citetext{e.g., \textcolor{myorange}{\citealt{mason2009financial,ho2015incentivizing,shah2016double,laux2024improving}}}. This assumption drives much of the current incentive-based data collection paradigm, where the goal is to use external compensation as a lever to elicit higher-quality data, when required. However, while external rewards can drive greater effort, evidence suggests that they are not always the primary determinant of high-quality engagement.

For instance, some of the highest-quality human contributions come from platforms where users are not financially compensated at all, such as Wikipedia, Reddit, and open-source communities. Here, participants contribute not because of pay, but because they find the activity meaningful, socially rewarding, or aligned with personal interests~\cite{forte2005wikipedia,lampe2010motivations}. These platforms challenge the idea that quality data generation must always rely on financial incentives, illustrating that intrinsic motivations can sustain long-term engagement without depending on financial rewards as the primary driver.

While many assume that external incentives and intrinsic motivation are correlated, research shows that their relationship is far more complex.

\begin{figure}[h]
    \centering
    \includegraphics[width=0.8\linewidth]{illustrations/motivation.png}
    \label{fig:motivation}
\end{figure}

\noindent\textbf{Overjustification Effect}~\cite{lepper1973undermining} explains how external rewards can diminish intrinsic motivation and affect task performance. In a classic experiment, preschool children who already enjoyed drawing were divided into three groups: (1) those who were promised and received a reward, (2) those who received an unexpected reward, and (3) those who received no reward. This experiment revealed two notable outcomes: first, children in the expected-reward group spent significantly less time drawing voluntarily after the reward was removed, compared to the other groups. Second, the drawings from the no-reward and unexpected-reward groups were rated as slightly higher in quality than those from the expected-reward group. This suggests that when an activity initially driven by intrinsic motivation is externally incentivized, the removal of rewards can lead to a decline in voluntary participation, and rewards may also subtly shift focus from quality to mere completion of the task.

\textbf{So why do external incentives sometimes backfire?} Two key psychological theories help explain why the Overjustification Effect\textemdash{}how extrinsic rewards can sometimes diminish intrinsic motivation\textemdash{}occurs:
\begin{itemize}[left=0cm]
    \item \textbf{Self-Perception Theory (SPT)}~\cite{bem1972self} suggests that individuals infer their own attitudes and motivations by observing their past behaviors. When external rewards are introduced, people may start attributing their participation to the incentive rather than to their original or intrinsic interest. Over time, this shift in self-perception can make them less likely to continue the behavior once the reward is removed.
    \item \textbf{Self-Determination Theory (SDT)}~\cite{deci1971effects} offers a broader framework by focusing on autonomy, competence, and relatedness as key psychological needs for intrinsic motivation. When a task is externally controlled through incentives, individuals may feel a loss of autonomy, making the activity feel like an obligation rather than a choice. This helps explain why highly controlled environments often struggle to sustain long-term engagement.
\end{itemize}
Together, SPT and SDT highlight why financial incentives alone are not a sustainable solution for maintaining high-quality, long-term engagement.

\textbf{If intrinsic motivation is key to sustaining high-quality, long-term contributions, then what role do external incentives play?} While excessive reliance on extrinsic rewards can be detrimental, carefully designed incentives can help initiate engagement in behaviors that might otherwise remain dormant. For example, small, well-calibrated incentives can serve as interventions, bringing attention to valuable behaviors without overwhelming intrinsic motivation~\cite{deci1971effects}. Even in systems designed to favor intrinsically motivated behavior\textemdash{}such as laissez-faire environments \cite{hayek2014road}, where individuals are free to act and bear the consequences of their choices\textemdash{}subtle incentive mechanisms can still be useful to align individual and collective goals, as with \textit{nudging}~\cite{leonard2008richard}. Overpresence of incentives has generally led to unintended consequences\textemdash{}either they are optimized to the point of losing effectiveness~\citetext{Goodhart's Law; \textcolor{myorange}{\citealt{goodhart1984problems}}}, or, worse, they introduce undesirable behaviors counterproductive to desired goals~\citetext{Perverse Incentives; \textcolor{myorange}{\citealt{kerr1975folly}}}\citetext{Cobra Effect; \textcolor{myorange}{\citealt{siebert2001cobra}}}.

\textbf{So, how do these social theories play out in real-world data sources?} Returning to the two sources of data\textemdash{}data collection systems like MTurk vs. naturally occurring data sources (community-based platforms) like Wikipedia\textemdash{}the discussed social theories provide valuable insights into their differing approaches to engagement. Data collection systems, being intentionally designed, are structured around external incentives, with financial rewards and intrinsic motivation naturally coexisting at first. However, over time, a crowding-out effect takes hold: as intrinsic motivation erodes, platforms and data collectors continually increase external incentives and tighten control to sustain participation and quality, creating a vicious cycle where contributors optimize for efficiency rather than genuine engagement. This often leads to over-reliance on shortcuts, such as automating survey responses with AI tools, at the expense of quality.

In contrast, community-based platforms, like Wikipedia or Reddit, primarily rely on intrinsic motivation, with minimal external incentives such as contributor badges, reputation systems, and recognition. These platforms sustain engagement over the long term by creating a sense of social belonging, often intertwined with competence and autonomy, demonstrating that high-quality contributions can be sustained without heavy reliance on financial incentives.

This contrast underscores a crucial insight: designing sustainable data collection systems is not just about offering better incentives\textemdash{}it requires structuring environments that actively sustain and enhance intrinsic motivation.
% \section{Quantity Factors: Efficiency through Fragmentation}
\section{Human Factors in Quantity\textemdash{}\\Efficiency through Fragmentation}
Data collection systems differ not only in how they compensate contributors but also in how they structure tasks for scalability. At one end, rapid crowdwork platforms fragment tasks into micro-tasks (e.g., HITs on MTurk) that take seconds to complete, optimizing for speed and mass throughput~\cite{malsburg2024mturk}. Platforms like Prolific handle slightly larger but still modular tasks, spanning minutes to hours~\cite{prolific2024completion}. On the other end, freelance job platforms (e.g., UpWork) structure work as full projects, lasting days or weeks and offering greater autonomy and depth of engagement~\cite{upwork2024times,fiverr2024comparison,workathomesmart2024lionbridge}.

Fragmentation into repeatable units that can be completed in a consistent and orderly manner allows tasks to be parallelized across multiple workers, replacing the traditionally serial process of creation. Many innovations and processes begin as creative, effortful tasks\textemdash{}akin to System 2 processes, requiring deliberate, conscious effort~\cite{kahneman2013prospect}. However, to scale, they are often refined into a System 1 process, where execution becomes fast, automated, and intuitive. This transformation\textemdash{}breaking down complex, uncertain tasks into simpler, repeatable steps—underpins mass production systems. 

Crucially, this shift from System 2 to System 1 is not just a natural evolution but is actively accelerated by task fragmentation. When work is divided into modular, repetitive tasks, the process of routinization happens more quickly.  A useful analogy here is \textbf{Fordism}~\cite{hounshell1984american}, which introduced the assembly line, a mechanized, repetitive setup where products are built step by step. This structured fragmentation enables processes to be repeated at scale, maximizing efficiency and throughput.

\begin{figure}[h]
    \centering
    \includegraphics[width=0.8\linewidth]{illustrations/fragmentation.png}
    \label{fig:fragmentation}
\end{figure}

However, as tasks become increasingly repetitive and fragmented, they sacrifice the creativity and problem-solving scope that contribute to meaningful engagement. Over time, workers become disconnected from the broader purpose of their efforts~\citetext{Theory of Alienation; \textcolor{myorange}{\citealt{marx2016economic}}}, %as Marx's Alienation Theory~\cite{marx2016economic} suggests, 
shifting their motivation from seeking fulfillment to merely achieving survival goals -- a regression in the hierarchy of needs~\cite{maslow1943theory}. For employers/collectors, this shift manifests as a decline in quality, as contributors disengage from the task itself. Unlike physical labor, which has built-in quality checks (e.g., material standards, product inspections), knowledge-based tasks lack robust safeguards. In data annotation, for example, there is often no immediate way to verify whether a task was completed thoughtfully or rushed~\cite{klie2024analyzing,klie2024efficient}. As a result, quality can quietly degrade, with errors compounding over time\textemdash{}often going unnoticed until the system has deteriorated beyond repair.

\textbf{So, how does task fragmentation impact real-world data sourcing?} Micro-tasking on platforms like MTurk was once hailed as a transformative shift in computer science, enabling large-scale user studies~\cite{bohannon2011social,kittur2013future,bernstein2011crowds} and efficient data collection for machine learning~\cite{deng2009imagenet}. However, over time, research has raised concerns about the reliance on ``piece rate'' or pay-per-task systems, favoring ``quota'' systems instead~\cite{ikeda2016pay,mason2009financial}, which alludes to a reduction in task quality when micro-tasking is pushed too far.

Micro-tasking doesn’t just impact data quality\textemdash{}it also affects the workers behind it. Beyond quality concerns, it has drawn criticism for its effect on worker well-being. Works such as Ghost Work~\cite{gray2019ghost} and Anatomy of AI~\cite{crawford2018anatomy} have illustrated the often invisible and exploitative nature of these atomized tasks. %performed by an  underpaid global workforce. 
The non-physical nature of knowledge labor further exacerbates this issue, making its value difficult to quantify~\cite{martin2016turking}. This issue is taken to the extreme when microtasks are outsourced to developing countries with favorable exchange rates~\cite{dicken2007global} to cut costs\textemdash{}and thus, incentives\textemdash{}even further~\cite{perrigo2023exclusive,microsoft_google_questioned}, often trapping workers in exploitative, sweatshop-like conditions~\cite{williams2022exploited,hao2022ai}.


% This issue is compounded by global labor arbitrage~\cite{dicken2007global}, where favorable exchange rates in developing countries allow platforms to offer low wages, trapping workers in exploitative, sweatshop-like conditions~\cite{perrigo2023exclusive,microsoft_google_questioned}.

\textbf{Then, what happens when tasks become so repetitive and unfulfilling that workers disengage from them entirely?} As discussed earlier, over time, many human-driven processes shift from System 2 (deliberate \& effortful) to System 1 (intuitive \& fast). As tasks become more structured and predictable, they become prime targets for automation. In physical labor, this transition has been gradual\textemdash{}machines take over repetitive, routine tasks, while humans focus on creative and uncertain work~\cite{brynjolfsson2014second}.

A similar shift is occurring in knowledge-based work, where high-quality LLMs give workers the opportunity to offload mundane tasks\textemdash{}such as grammar corrections, spell-checking, and phrasing refinements\textemdash{}to AI. When used judiciously, this assistance promotes meaningful engagement and enhances productivity without compromising data quality. However, the problem arises when workers become over-reliant on LLMs, using them indiscriminately to complete entire tasks without oversight~\cite{veselovsky2023artificial,veselovsky2023prevalence}. Since knowledge-based tasks often lack clear-cut quality standards, it becomes harder to detect when quality has deteriorated, making it easier for such opportunistic behavior to go unchecked.

As a result, the transition to automation in data sourcing has been rather chaotic. While repetitive physical labor was gradually offloaded to machines in structured ways, knowledge work faces conflicting views\textemdash{}some advocate for fully replacing human contributors~\citetext{e.g., \textcolor{myorange}{\citealt{dubois2024alpacafarm}}}, while others for eliminating LLM usage entirely~\citetext{e.g., \textcolor{myorange}{\citealt{thorp2023chatgpt}}}. However, fully relying on synthetic data risks model feedback loops and collapse~\cite{taori2023data,shumailov2024ai}, while a complete ban slows human productivity and sacrifices efficiency~\cite{liao2024llms,kreitmeir2023unintended}. The most effective approach lies somewhere in between\textemdash{}where AI serves as a tool that productively and progressively supports human effort rather than a crutch for task completion~\citetext{e.g., \textcolor{myorange}{\citealt{ashok2024little,qian2024evolution}}}.

In this landscape, intrinsic motivation becomes even more crucial. Workers must make thoughtful decisions about how to incorporate LLMs in ways that enhance rather than replace meaningful engagement. Designing a sustainable data collection system, therefore, is not just about limiting LLM use for workers or maximizing automation with synthetic data\textemdash{}it's about creating an environment where contributors remain actively engaged with the task, rather than optimizing for speed at the cost of quality.
\section{Elevating the Quality-Quantity Trade-off: \\Rethinking Control}
% \section{Uplifting the Trade-off: Rethinking Control}
Data collection systems have long operated under the assumption that control at the task level\textemdash{}through explicit instructions, incentive structures, and quality enforcement mechanisms\textemdash{}is necessary to ensure both high-quality and high-quantity data. However, as we have discussed, these very mechanisms often introduce unintended side effects. External incentives, while effective in driving participation, tend to crowd out intrinsic motivations over time, leading to disengagement and lower-quality contributions. Similarly, excessive task fragmentation, though useful for efficiency and scalability, can erode a sense of purpose and hence leads contributors to disengage from the task itself\textemdash{}resulting in an over-reliance on shortcuts, manifesting in careless task completion and non-judicious use of LLMs.

In contrast, data that we obtain from systems not intentionally designed for data collection\textemdash{}such as Wikipedia, Reddit, and open-source projects\textemdash{}demonstrate an alternative paradigm. These platforms do not enforce control at the task level but instead create environments where intrinsically motivated contributors engage meaningfully. Crucially, this lack of task-level control removes two major pitfalls seen in structured data collection systems: it prevents (a) the crowding-out effect, where external incentives replace intrinsic motivation over time, and (b) excessive fragmentation, ensuring that contributors remain connected to their purpose of participation. Taking inspiration, ceding control at task level could be the key to pushing the pareto-frontier forward.

\textbf{Ceding Control at the Task Level.} Rather than controlling individual (micro) tasks, data collection systems can benefit from structuring conditions that naturally guide contributor engagement. Moving from direct task management to a broader, environment-driven approach discourages individuals from attributing their participation to external rewards, helping preserve intrinsic motivation~\cite{bem1972self}. 

However, this shift presents a new challenge: relinquishing fine-grained control over tasks means that collectors must instead focus on shaping engagement at a more systemic level. Coarse-grained control\textemdash{}where engagement is influenced through platform design, incentives, and structural conditions\textemdash{}takes longer to align with desired outcomes and demands greater up-front effort. But once in place, it can lead to more sustainable data collection, enabling both higher-quality and higher-quantity contributions, as seen in rare but influential examples.

\begin{figure}[h]
    \centering
    \includegraphics[width=0.8\linewidth]{illustrations/environment-control.png}
    \label{fig:environment-control}
\end{figure}

\textbf{Deployed Robots} are a prime example of achieving this delicate balance. For example, robotic vacuum cleaners (e.g., Roomba) provide utility to users, by cleaning their homes, while simultaneously collecting spatial and navigation data that improves future performance~\cite{astor2017your}. Users engage with the system for its primary function, yet their interactions naturally generate high-quality data that feeds back into the AI’s development~\cite{brynjolfsson2014second}. This data collection approach also scales effectively, as data is gathered continuously and passively as a result of users' routine behaviors, without requiring any additional effort towards data contribution. This model extends to more complex, high-stakes deployments, such as electric vehicles equipped with driver-assist and self-driving features. As an example, consider how Tesla uses real-world driving data from its fleet to refine its self-driving AI algorithms, which ultimately benefit car owners by improving the technology~\cite{karpathy2021ai,tesla_autopilot} and can drive future innovation for the company (e.g., driverless Robotaxi). Similarly, Waymo operates self-driving taxis in real-world environments and has gathered large-scale data that has proven valuable for advancing computer vision research~\cite{sun2020scalability}.

However, replicating such large-scale, product-driven data collection systems is exceptionally difficult. They demand massive hardware infrastructure, well-articulated and trusted benefits for both users and companies, and real-world applications with extensive safeguards and privacy protections. For entities whose primary goal is simply to collect human-generated data, establishing such ecosystems solely for this purpose is neither feasible nor sustainable.

More importantly, these symbiotic relationships rely on a delicate balance of implicit or explicit social contracts, mutual trust, and fair distribution of costs and benefits, as explained by theories of social interaction and exchange\textemdash{}such as Social Exchange Theory~\cite{homans1958social} and Social Contract Theory~\cite{rousseau1762social}. When the benefits clearly extend beyond the primary parties\textemdash{}for example, when collected data is repurposed to serve third parties\textemdash{}this balance can be disrupted. If users feel that their contributions are being exploited without fair reciprocity, trust erodes, and they may come to see the system as exploitative or unjust.

These feelings of mistrust and exploitation are already prevalent in AI, particularly in creative and knowledge-sharing communities. Artists have protested against their work being scraped to train AI models without consent or compensation~\cite{jiang2023ai}\footnote{\url{https://www.aitrainingstatement.org/}}, with many calling for stronger protections against AI-generated art~\cite{guardian2025aiart}. Similarly, Stack Overflow users, frustrated by their contributions being used for external profits, have intentionally altered or deleted their posts to hinder AI training\textemdash{}leading to bans, boycott, and subsequent decline in engagement on the platform~\cite{ars2024stackoverflow, tomshardware2024stackoverflow}. These examples highlight the fragility of trust in data collection and the potential consequences when contributors feel that their data is being repurposed beyond its original intent.
\section{Games as a New Pareto-Frontier?}
Games are uniquely positioned at the intersection of structured environments and intrinsic motivation. Designed worlds crafted by game developers set the rules and constraints, yet within them, players engage voluntarily, driven by curiosity, competition, and creativity~\cite{koster2005theory}. Unlike traditional work, where tasks are often externally imposed, games offer a space where challenge and engagement emerge naturally, 
sustaining long-term participation without the need for direct financial incentives.

For most players\textemdash{}aside from professional esports competitors\textemdash{}games are played purely for enjoyment. At the same time, they require diverse forms of reasoning, strategy, and problem-solving, making them a rich ground for capturing complex human behaviors and decision making~\citetext{as already evidenced by their role in evaluating intelligence; \textcolor{myorange}{\citealt{silver2016mastering, vinyals2019grandmaster, berner2019dota, meta2022human}}}. In this way, games hint at a new frontier\textemdash{}one where many structured environments and organic engagements seamlessly coexist. This intersection offers a compelling model for AI training, where intrinsically motivated interactions can yield structured, high-quality data without the pitfalls of external incentive-driven systems.

\textbf{Games as a Tool for Data Collection.} Games have long been explored as a tool for large-scale human annotation and AI training, most notably through von Ahn's Games with a Purpose (GWAP) \cite{von2006games}. One of the earliest and most influential examples was the ESP Game~\cite{von2005esp}, introduced in 2004, which engaged thousands of players in a collaborative tagging game, generating millions of image annotations. While players simply enjoyed the game, their interactions helped bootstrap Google Image Search~\cite{guardian2006esp}, which previously relied only on filenames, as large-scale labeled datasets like ImageNet were not available until 2009~\cite{deng2009imagenet}.

Von Ahn argued that the billions of hours spent on games\textemdash{}such as the 9 billion hours on Solitaire in 2003 alone, enough to build the Empire State Building in 6.8 hours or the Panama Canal in a day\textemdash{}could be repurposed for more meaningful tasks, inspiring the broader Games with a Purpose framework~\cite{law2011human}. Other notable games in this series included Peek-a-boom~\cite{von2006peekaboom}, which collected image segmentation data, and Verbosity~\cite{von2006verbosity}, designed to gather commonsense factual knowledge.

\textbf{Games in Mainstream ML.} While recent efforts towards this in machine learning have not reached the scale of GWAP, they have made notable progress. Examples include Google's QuickDraw~\cite{ha2017neural} and AllenAI's Iconary~\cite{clark2021iconary}, which focus on collecting freehand drawing data, and AI21's HumanOrNot~\cite{jannai2023human}, which gathers conversational data through a gamified Turing Test~\cite{dugan2023real}.

However, developing entirely new games for data collection presents significant challenges. Machine learning researchers often lack expertise in designing engaging gameplay, making it difficult to ensure both high-quality data and sustained participation. To navigate this, some approaches have focused on repurposing existing games rather than building from scratch. For example, Family Feud has been adapted to generate QA pairs~\cite{boratko2020protoqa}, and Minecraft has been used as a platform for collecting conversational dialogues~\cite{narayan2019collaborative}.

On a smaller scale, some efforts have explored gamification, introducing game-like mechanics into traditionally non-game tasks to enhance engagement, without requiring a full game environment. For instance, CommonsenseQA 2.0 incorporates elements such as scoring, competition, and progression to make data contribution more engaging, improving both user participation and data quality~\cite{talmor2022commonsenseqa}.

\textbf{Design Considerations.} Designing data collection games requires solving multiple challenges at once\textemdash{}the most critical challenge is optimizing data utility while preserving intrinsic player motivation. This involves balancing two often competing objectives:
\begin{itemize}[left=0cm]
    \item Ensuring that collected data meets the needs of AI/ML tasks -- requiring structure, reliability, and task relevance.
    \item Designing a player experience that remains engaging over time --avoiding disengagement due to artificial constraints or coercive incentives.
\end{itemize}
Achieving this balance is a multi-objective optimization problem that requires close collaboration between ML researchers and game developers. This design space spans multiple approaches, including creating entirely new games optimized for data collection, leveraging existing games to re-purpose organic player interactions, or introducing game-like elements into existing data collection tasks to enhance engagement. Each of these approaches comes with trade-offs in control, scalability, and long-term sustainability.

While prior efforts have successfully optimized for quality and quantity, sustaining trust has proven far more elusive, leading to short-lived or unsustainable solutions. Historical implementations offer valuable lessons\textemdash{}GWAP demonstrated the potential for high-quality, high-quantity crowdsourced data but did not endure over time. In contrast, ReCAPTCHA~\cite{von2008recaptcha} conceptualized in 2008, used for annotating books and self-driving data~\cite{captcha_google,captcha_nyt}, remains widely used today but has blurred the line between voluntary participation and coercion, with users reporting frustration and annoyance~\cite{searles2023empirical,searles2023dazed}, alluding to lack of trust. These examples illustrate the challenge of designing systems that optimize for all three aspects: \textbf{high-quality}, \textbf{high-quantity}, and \textbf{high-trust} data collection.

% While prior efforts have successfully optimized for quality and quantity, sustaining trust has proven far more elusive, leading to short-lived or unsustainable solutions. Historical implementations offer valuable lessons -- GWAP demonstrated the potential for high-quality crowdsourced data but lacked sustainability, while ReCAPTCHA~\cite{von2008recaptcha} used for annotating books and self-driving data~\cite{captcha_google,captcha_nyt} has been highly sustainable but did so in a way that blurred the line between voluntary participation and coercion, undermining user trust~\cite{searles2023empirical,searles2023dazed}. These examples illustrate the challenge of designing systems that optimize for all three aspects: high-quality, high-quantity, and high-trust data collection.

This challenge becomes even more evident when considering long-term sustainability. While data collection games have proven effective in gathering large-scale, high-quality data, few have endured over time. To our knowledge, no sustained data collection effort through games exists in machine learning. However, other research fields provide examples of systems that have successfully maintained engagement and trust over extended periods. Scientific discovery platforms such as Zooniverse for citizen science in astronomy~\cite{cardamone2009galaxy,lintott2008galaxy}, Lab in the Wild for HCI and psychology user studies~\cite{reinecke2015labinthewild}, FoldIt for protein folding~\cite{khatib2011crystal,cooper2010predicting}, and even Eve Online for epidemiology research~\cite{kafai2016connected} demonstrate how engagement, motivation, and trust can be sustained without relying on short-lived external incentives. Psychology and cognitive science studies~\cite{allen2024using} further reinforce that game-like participation can be structured in ways that maintain engagement over time.

While games and other naturalistic environments show significant potential towards achieving higher quality and quantity at the same time, they come with their own unique benefits and challenges previously not encountered with data collection systems:

\textbf{Benefits} include \textbf{the collection of previously inaccessible data}, such as natural dialogue interactions. Human conversations are often behind privacy walls, and attempting to collect dialogue through traditional data collection platforms is impractical, because unlike simple annotation tasks, dialogues need to be organic with real-time back-and-forth exchanges. In such cases, LLM-generated synthetic data has been a relief, with datasets like SODA~\cite{kim2022soda} playing a pivotal role in advancing dialogue systems. However, games offer an ideal setting where natural dialogues happen organically in non-real-world contexts, often under pseudonyms, making privacy concerns less of an issue. For instance, Minecraft has been leveraged to collect goal-driven player dialogues~\cite{narayan2019collaborative}, demonstrating how game environments facilitate authentic, context-rich communication.

Benefits also arise from the nature of games, including their ability to \textbf{collect data for open-ended tasks}, which has traditionally been challenging~\cite{karpinska2021perils}, especially over the long term. Moreover, games provide a unique advantage by enabling multi-modal data collection\textemdash{}capturing vision, language, and speech for the same actions\textemdash{}thus supporting advancements in embodied AI research. This has already been demonstrated in virtual environments like Habitat~\cite{puig2023habitat} and AI2-THOR~\cite{kolve2017ai2}, albeit without game-like objectives or direct human involvement to learn from yet.


\textbf{Challenges} include \textbf{difficulty in designing meaningful intrinsic rewards}, as motivations and preferences naturally vary across individuals and cultures. Unlike money, which serves as a universal incentive, intrinsic rewards must align with diverse motivational drivers\textemdash{}curiosity, competition, or self-improvement\textemdash{}to sustain engagement~\cite{jun2017types}. This also impacts demographic representation, as cultural and personal factors shape motivation~\cite{deci1985self}. Certain groups may be overrepresented while others disengage based on how well rewards align with their preferences~\cite{triandis1995individualism}. To ensure fairness, intrinsically motivated data collection must be designed to encourage diverse participation rather than benefiting only specific user segments. 

While games are often seen as leisure activities or ``unproductive,'' repurposing them for data collection contributes to value creation, in which case, it is important to recognize and fairly compensate contributors. The challenge is that, unlike traditional annotation tasks, AI data attribution is ambiguous, making it difficult to measure and \textbf{distribute compensation effectively}. One way to address this is by ensuring that contributors have a stake in the value their data generates, potentially through decentralized compensation models~\cite{loyalAI}. However, in general, compensation must be carefully structured to avoid distorting intrinsic motivation. If participants anticipate compensation upfront, their engagement risks becoming extrinsically driven, leading to lower-quality contributions. A potential solution is delayed, post-hoc recognition, where participants are acknowledged after the fact, ensuring they remain intrinsically engaged while still benefiting from their contributions.

Beyond incentive design, another fundamental challenge is ensuring that game-derived data remains both ethical and useful. Not all games inherently separate real-world contexts, which may necessitate distinguishing user-identifiable behavior from game-playing behavior for privacy reasons~\cite{nair2022exploring}. More broadly, extracting useful and generalizable data from inherently noisy and unstructured game records can be a significant challenge, similar to the difficulties faced in filtering large-scale Internet data~\cite{fang2023data}.
\section{Conclusion}



This paper introduces \sysname, an AI-assisted system designed to enhance the process of visual blend ideation by leveraging metaphors. 
%Our system utilizes large language models and commonsense knowledge bases to explore objects and their associated attributes, forming metaphorical connections with abstract concepts.
Our system utilizes LLMs and commonsense knowledge bases to explore objects and their associated attributes, forming metaphorical connections with abstract concepts. 
It offers the capability to automatically generate blending proposals based on user selections, facilitating rapid creative realization for verification through the T2I model.
To evaluate the system, we conducted a user study involving 24 participants who had AI experience. The findings demonstrate that \sysname\ has the potential to enhance the creativity of the generated ideation results and enable the expression of abstract concepts more metaphorically.
Additionally, this research offers insights into user preferences regarding visual blend design and potential future approaches for supporting design with generative AI.



% 
The increasing reliance on LLMs for multimodal tasks across far-reaching sectors such as healthcare, finance, and manufacturing underscores the need to assess the accuracy and reliability of the information they generate. Vision-Language Models (VLM) have achieved state-of-the-art (SoTA) performance on Visual Question-Answering (VQA) benchmarks, and these models often utilize Retrieval-Augmented Generation (RAG) to maintain factual accuracy and relevance in a dynamic information environment. However, this has led to uncertainty in the information the LLM bases its answer on, as it may choose between parametric memory and retrieved sources. When models rely on memorized information instead of dynamically retrieving information, they may inadvertently propagate outdated or incorrect information, causing serious legal and ethical risks and undermining trust and reliability in AI systems \citep{huang2023survey}.
% The ability to strike a balance between generalization and specialization in AI systems is therefore crucial for ensuring the safe, reliable use of these technologies in real-world applications.

Despite these concerns, the way that Vision-Language models (VLMs) memorize and retrieve information, particularly in complex multimodal tasks, remains under-explored. Current research often focuses on either the general capabilities of large language models (LLMs) or the specialized retrieval mechanisms in retrieval augmented generation systems (RAG) \citep{incontext_rag,chen_murag_2022,liu_universal_2023}. Particularly in the context of multimodal retrieval and multihop reasoning, few studies analyze the tradeoff between finetuning for specialized tasks and zero-shot prompting for general-purpose vision-language capabilities. A lack of consensus on how to approach this tradeoff motivates the development of measures to quantify reliance on parametric memory, as well as metrics for quantifying the potential performance impact of extending LLMs with RAG systems.

To address this gap, we investigate how multimodal QA models balance accuracy with memorization on the WebQA benchmark. We compare finetuned multimodal systems against zero-shot VLMs, analyzing how retrieval performance influences QA accuracy. In particular, we focus on cases where retrieval fails, allowing us to measure reliance on parametric memory through two proposed metrics---the \ppr (\PPR) which quantifies how much model accuracy is influenced by retrieval quality, contrasting performance in best-case versus worst-case retrieval scenarios, and the \ucr (\UCR) which measures how often correct QA responses are generated when the retriever fails, providing a proxy for memorization.

To enable this analysis, we make several methodological contributions. For the finetuned QA models, we investigate Vision-Transformer (ViT) architectures, which allow for multihop reasoning over multiple sources. To investigate the impact of retrieval performance on trained LMs, we propose a variable-input Fusion-in-Decoder (FiD) model \cite{tanaka_slidevqa_2023, nlvr2}, building upon the VoLTA architecture \citep{pramanick_volta_2023}. For the zero-shot case, we build upon previous research on In-Context Retrieval \citep{incontext_rag} by demonstrating that LLMs such as GPT-4o are capable of performing the final ranking step of the retrieval process. In doing so, we find that GPT-4o, a general-purpose LLM, achieves SoTA performance on the WebQA task, outperforming existing finetuned RAG models by a significant margin (7\% higher accuracy). 

Crucially, our results reveal that while retrieval-augmented models reduce memorization, the training paradigm plays an important role. Finetuned models exhibit higher reliance on parametric memory, whereas zero-shot RAG approaches have lower memorization scores at the cost of accuracy. This suggests that while retrieval modules may mitigate the risks associated with outdated or incorrect information, SoTA performance requires that they be coupled with specialized QA models. Our memorization measures contribute to the development of transparent and reliable AI systems, particularly in applications where the sourcing of up-to-date, factual information is critical.



% We investigate the impact of question complexity on the ability of these models to integrate multiple data sources—such as images, text, and external retrievers—and produce coherent and accurate answers. We also explore whether in-context retrieval can be a viable alternative to traditional retrieval-augmented systems, offering a more streamlined approach to multimodal QA.

% To achieve this, we first compare zero-shot prompting multimodal LLMs with finetuned multimodal systems. We evaluate both types of models on the WebQA benchmark, a dataset designed for complex question answering that requires reasoning across both image and text sources. For the finetuned models, we use a Fusion-in-Decoder (FiD) architecture, which allows for multihop reasoning over multiple sources. Additionally, we introduce the concept of In-Context Retrieval Language Modeling (RLM), where the LLM itself performs retrieval tasks without the need for external retrievers. This method builds upon existing research in in-context learning  and aims to explore the viability of LLMs retrieving relevant sources and generating accurate answers directly from their context window.

% In order to investigate source utilization in finetuned multimodal models and LLMs, three lines of inquiry are established; 
% \begin{itemize}
%     \item Study 1: retrieval vs QA performance on webQA (motivating example, does QA answer correctly even with incorrect sources?)
%     \item Study 2: performance on adversarial examples where parametric knowledge would be incorrect by design
%     \item Study 3: improving performance on adversarial examples by fine-tuning (i.e model robustness)
% \end{itemize}

% Note, there is one weakness in this plan which is tying in the work we've already done. 
% If we added something from adversarial generation to the retrieval experiment (like a combination of study 1 + 3) it would be complete. So for instance we could try fine-tuning the retriever with adversarial examples (and not just the QA model)

% \begin{figure}
%     \centering
%     \includegraphics[width=0.95\linewidth]{figures/segmentation/webqa_segment_infill.png}
%     \caption{Example of the segmentation substitution pipeline from the WebQA task.}
%     % d5c76d760dba11ecb1e81171463288e9
%     \label{fig:seg_sub_pipeline}
% \end{figure}



% Retrieval augmented generation (RAG) with zero-shot prompting and fine-tuning Large Language Models (LLMs) have become the go-to methods for tasks relying on information retrieval and text generation. In many cases the LLMs parametric memory can sufficiently generalize to answer questions without being provided with retrieval mechanisms for out-of-domain knowledge. However, LLMs often hallucinate and provide wrong information in certain scenarios. This problem is amplified even further on open-domain Question Answering (QA) tasks involving multiple modalities. Grounded text generation using retrieved sources \citep{lewis2021retrievalaugmented} has been extensively studied for text-to-text QA tasks, but its application in multimodal settings has not been studied as much.


% Multimodal reasoning and question answering have gained prominence in recent research endeavors, with an increasing emphasis on handling various forms of data, particularly text and images. In this study, we address a specific gap in the existing literature by focusing on the development of a versatile multihop model capable of accommodating varying numbers of input images.

% Our motivation for this research lies in the growing complexity of answering questions using information on the web, where the challenge of navigating the open-domain setting is further complicated by the presence of multiple modalities and sometimes requires reasoning over multiple sources. WebQA is an ideal dataset on which to compare performance of finetuned RAG systems against general purpose LLMs; it is multimodal, with correct answers requiring reasoning over image and text sources. It is multihop, requiring a complex reasoning process over multiple sources. Finally, WebQA questions from different categories can be broken down into subdomains to analyze performance over domains of varying cardinality.

% Motivated by the real-world challenges of building retrieval and question answering (QA) systems, we design and finetune a closed domain, multimodal, multihop QA model, that is capable of reasoning over a varying number of sources taken as input from an external retriever module. This research contributes to the relatively underexplored domain of multihop reasoning across various input sources and modalities. Our goal is to explore the challenges posed by these scenarios and develop strategies that enable QA models to retrieve relevant information, conduct logical or numerical reasoning across diverse modalities, and generate coherent responses in natural language. To our knowledge, this is the first application of the Fusion-in-Decoder (FiD) architecture \cite{tanaka_slidevqa_2023, nlvr2} that is shown to work with a variable number of inputs, enabling multi-hop reasoning over sources.

% In-Context Learning refers to the ability of LLMs to perform any task by simply providing examples in the input prompt \citep{dong2022survey,min2022rethinking}. Inspired by this research, we propose a method to use the LLM itself as a multimodal retriever, potentially eschewing the requirement of a distinct retrieval module, thereby allowing the design of simpler retrieval-augmented QA systems. We dub this method In-Context Retrieval Language Modeling (RLM). To the best of the authors knowledge, In-Content RLM is disparate from other retrieval augmented approaches which utilize external retrieval modules \citep{incontext_rag,chen_murag_2022,liu_universal_2023}. Despite being a natural extension of In-Context learning, In-Context RLM has not yet been studied empirically.

% To expand on our contribution of In-Context Retrieval, this stems from the well-researched in-context learning of LLMs. In-context learning is the ability of a model to perform any task given a sufficient context window \citep{dong2022survey,min2022rethinking}. Such tasks could include retrieval and ranking, but typically, the go-to solution for tasks requiring retrieval has been RAG. To the best of the authors knowledge, In-Context Retrieval is distinct from In-Context Retrieval Augmented Language Modelling (RALM), and despite being a natural extension of In-Context learning, In-Context Retrieval has not yet been shown empirically.

% Finally, we explore the tradeoff between using zero-shot prompting LLMs and the fine-tuning approach. While we find that, overall, GPT-4o obtains SoTA performance on the WebQA task, outperforming the accuracy of existing finetuned RAG approaches by 7\%, finetuned approaches still perform better on more restricted subdomains\footnote{``In-Context RLM" @ \url{https://eval.ai/web/challenges/challenge-page/1255/leaderboard/3168}}. Finally, we validate that GPT-4o is relying on retrieval abilities to solve the task; we find that GPT-4o is capable of retrieving relevant sources in the presence of distractors and furthermore, when GPT-4o fails to retrieve correct sources, it answers incorrectly 75\% of the time, meaning that it is not relying on parametric memory for this task.

% \paragraph{Contributions}
% Based on our experimentation and analysis on the WebQA benchmark, we make the following contributions:
% \begin{itemize}
%     \item Propose a new architecture for multimodal multihop QA that takes variable number of input sources inspired by the Fusion-in-Decoder method.
%     \item Comparison of general purpose LLMs vs specialized models on the WebQA benchmark.
%     \item Observation of In-Context Multimodal Retrieval abilities of GPT-4o and that it does not rely on parametric memory for multimodal QA.
%     \item Analysis of relationship between retrieval and QA task performance.
%     \item Analysis of task and query complexity on the performance of retrieval and QA tasks.
% \end{itemize}
















% Throughout this paper, we will present our methodology, experiments, and findings, emphasizing our approach to multihop reasoning over varying numbers of input images. We believe that our work contributes to a deeper understanding of multimodal reasoning and has the potential to enhance the capabilities of question-answering systems in the intricate, multimodal landscape of web-based information.
% \input{sections/02-1-desiderata}
% \input{sections/02-2-motivation}
% \input{sections/02-3-fragmentation}
% \input{sections/02-issues}
% \input{sections/03-designspace}
% \section{Background and Motivation}
\label{sec:background}

We introduce the background on serverless workload serving and motivate the use of runtime resource adaptation to address resource inefficiency in existing serverless platforms.

\subsection{Resource Inefficiency with Early Binding}
% In current serverless platforms, developers are required to specify immutable sizes for their deployed functions.
% Then, providers consider functions' runtime workloads  (e.g., concurrency)  and resource usage to scale out/in their instances.
% Moreover, due to high runtime variability, functions must size their functions for worst-case scenarios.
% This, however, incurs considerable resource inefficiency.
Current serverless workflow platforms (e.g., AWS Step Functions~\cite{aws-step-function} and Azure Durable Functions~\cite{azure-durable-function}) offer the opportunity for developers to build various applications with advanced logic like chaining, branching, and parallel execution.
These applications can be defined by JSON-based structured languages (e.g., Amazon States Language) or other programming languages.
Meanwhile, developers require to specify resource configurations, including memory size, CPU cores, and scaling options, for individual functions---an early-binding approach.
The serverless platform is responsible for monitoring the workload intensity and resource usage at runtime and scaling out/in function instances accordingly.
To account for potential runtime variability, developers must size the functions in their application workflow accounting for the worst case in order to provide SLO guarantees over the end-to-end delay of request processing, e.g., the 99th percentile (P99) of the end-to-end delay must be within a given target. 
After deployment, the function sizes become immutable. The worst case is not representative and over-shoots most of the time, leading to resource inefficiency. 


To verify this claim, we conduct a data-driven analysis with a dataset from Microsoft Azure Functions~\cite{azure-dataset} to explicitly demonstrate the resource inefficiency issue. % , deriving from the worst-case based early bind.
To quantify the inefficiency, we define a metric called \emph{slack}---the margin between the actual execution time and the SLO, which is calculated as $1-l/T$ with $l$ and $T$ representing end-to-end latency and SLO, respectively.
Under certain SLO defined with P99 latency as done by existing works (e.g., \cite{osdi22-orion,mac22-wisefuse}),  we can see from Figure \ref{fig:bg:slack} that more than 60\% function invocations have slacks over 60\%.
Particularly, we analyze slacks of the top 100 most popular functions, whose invocations account for 81.6\% of the total function invocations. % (depicted in Figure~\ref{fig:bg:popular_func}) of overall invocations.
The result shows that only 20\% of the invocations of the popular functions (blue line in Figure~\ref{fig:bg:slack}) have slacks less than 40\%.
This means the majority of requests are processed faster than necessary.
Notably, in DAG-based workloads (i.e., Azure Durable Functions), the resource inefficiency further deteriorates wherein the ratio between the 95th percentile and 50th percentile is by up to three times \cite{mac22-wisefuse}.

% \begin{figure}[t!]
% \centering
% \includegraphics[width=0.25\textwidth]{./figure/motivation/Average_P99_cdf_top=100.pdf}
% \vspace{-0.3cm}
% \caption{Sufficient function slacks in production traces.}
% \label{fig:bg:slack}
% \end{figure}

\subsection{Runtime Dynamics}
\label{sec:bg:worst-case}

The resource inefficiency caused by the large slack can be mainly attributed to the over-provisioning of resources by the developer. This is to ensure that the SLO is guaranteed even in the worst case (i.e., P99). However, normal cases deviate from the worst case significantly due to runtime dynamics. 
In particular, we observe that functions face two major dynamic factors at runtime: varying working sets and inevitable performance interference. These two factors contribute significantly to the variance of the function execution time. 
% Functions face two remarkably dynamic factors at runtime: working sets and performance interference, which lead to considerable variance of execution latency.

\begin{figure*}[!t]
	\centering
	\subfloat[]{
		\includegraphics[width=0.24\textwidth]{./figure/motivation/Average_P99_cdf_top=100.pdf}
		\label{fig:bg:slack}
	}
	\hspace{8mm}
	\subfloat[]{
		\includegraphics[width=0.25\textwidth]{./figure/motivation/function-latency-ml-analyze-varying-worksets.pdf}
		\label{fig:bg:ml-func-latency}
	}
	\hspace{8mm}
	\subfloat[]{
	\includegraphics[width=0.28\textwidth]{./figure/motivation/coresident-perf.pdf}   
	\label{fig:bg:perf-inteference}
	}
	%\vspace{-0.1cm}
	\caption{(a) slacks of function invocations in production traces, (b) function latency variance caused by varying input worksets for functions object detection (OD), question answering (QA), and and text-to-speech (TS), respectively,
 (c) performance interference attributed to co-location of homogeneous function with different dominant resource demands.}
 %\vspace{-0.4cm}
\end{figure*}

%'ml-analyze':{'text-to-speech': 'text-to-speech', 'question-answer': 'question answer',
%                      'object-detection': 'object detection'
\textbf{\textit{Varying working sets.}} 
The working set, i.e., input data like videos, audios, and texts, can have varying sizes.
Taking Microsoft Azure Function Blobs (storage service) as an example, their data size difference can be as high as nine orders of magnitude~\cite{azure-function-blob}.
Such a large difference results in substantial variance of the execution time even for the same function~\cite{socc21-faast,eurosys21-ofc}.
Specifically, we measure the execution time of three functions under different working sets (detailed in \S\ref{exp:setup}).
Figure~\ref{fig:bg:ml-func-latency} illustrates the results, where we can observe a variance of up to 3.8 times in function execution caused by varying working set sizes.

% \begin{figure}[t!]
% \centering
% \includegraphics[width=0.25\textwidth]{././figure/motivation/function-latency-ml-analyze-varying-worksets.pdf}
% \vspace{-0.3cm}
% \caption{Function latency variance caused by varying input worksets}
% \label{fig:bg:ml-func-latency}
% \end{figure}	

\textbf{\textit{Performance interference.}}
% On the other hand, function deployment, which decides when and where to deploy functions, is completely undertaken by providers.
For simplicity and security, commercial serverless platforms, such as Alibaba Function Compute, Microsoft Azure, and AWS Lambda, exclusively deploy function instances belonging to the same tenant, or even belonging to the same function, in the same virtual machine~\cite{socc22-owl,atc18-peek-bench}.
For example, the empirical study in~\cite{socc22-owl} shows that in Alibaba Function Compute 65\% of the virtual machines exclusively deploy instances of the same function.
This co-location of homogeneous function instances, however, can incur severe resource contention on the same resource dimensions, particularly for network bandwidth and memory bandwidth of virtual machines~\cite{sc21-gsight,micro19-faaSprofiler,socc22-owl,atc18-peek-bench}.
To verify this observation, we use a virtual machine to run a function increasing the number of co-located instances from one to six while measuring the execution time of four different functions with resource dominance on different dimensions namely computing, I/O, network, and memory, respectively (detailed in \S\ref{exp:setup}). 
As shown in Figure~\ref{fig:bg:perf-inteference}, the co-location of homogeneous functions leads to substantial resource contention and performance interference, prolonging the function execution time up to 8.1 times. The performance interference is often hard to model and predict.

% this co-residency results in substantial increase of execution latency by up to 8.1 times,leading to considerable variance in function execution time.
% when compared with that with concurrency as one.

%for CPU-, IO-, network- and memory-intensive functions as the concurrency rises from one to six.
%Figure shows that significant performance interference can be observed, . 
%compared with the inclusive deployment (concurrency as one), 
% this exclusive deployment (gray bar) results in substantial increase of execution latency by up to 8.1$\times$ for CPU-, IO-, network- and memory-intensive functions as the concurrency rises from one to six.

% this exclusive deployment (gray bar) results in substantial increase of execution latency by up to 8.1$\times$ for CPU-, IO-, network- and memory-intensive functions as the concurrency rises from one to six.
% As depicted in Figure~\ref{fig:bg:concurrent_latency}, with the concurrency rising  from one to six,  the exclusive deployment results in substantial increase of execution latency by up to 8.1$\times$.
% This significantly magnifies execution latency variance.

% \begin{figure}[t!]
% \centering
% \includegraphics[width=0.25\textwidth]{./figure/motivation/coresident-perf.pdf}
% \vspace{-0.3cm}
% \caption{Performance interference attributed to co-residency of homogeneous function.}
% \label{fig:bg:perf-inteference}
% \end{figure}




\subsection{Runtime Resource Adaptation}
\label{sec:bg:adaptive-allocation}
To tackle the aforementioned resource inefficiency issue, we can adopt a late-binding approach through \emph{runtime resource adaptation}, which resizes functions on the fly based on runtime information (e.g., function slacks), achieving higher resource efficiency without violating SLO. For example, given a workflow as a chain of functions, the resource allocation of the downstream functions can be adjusted when the first function finishes execution. This way, the slack from the first function can be exploited to optimize resource efficiency. 

The idea sounds straightforward and has been considered in some existing works \cite{infocom22-stepconf,middleware20-fifer,socc21-llama,socc21-kraken,middleware20-xanadu}.
However, most of these works make an unrealistic assumption that either the developer performs the adaptation decision with access to runtime information or the serverless platform provider performs the adaptation with domain knowledge of the application workflow. These assumptions render these solutions impractical to deploy in real-world serverless systems. The information barrier between the developer and the provider calls for a new solution. 

We identify the following challenges and opportunities for a full-fledged design for runtime resource adaptation. 

\textbf{\textit{Skewed function execution time distribution.}} 
Resource allocation for a serverless workflow is typically done by leveraging performance profiles of all the functions in the workflow. 
During the offline profiling, the execution time distribution for each function is first obtained by running the function with a variety of sample inputs under different resource conditions. Then, given a time budget, existing approaches typically use P99 of the function execution time as a target and calculate the corresponding resource demands. However, due to the high runtime variability, the distribution of the function execution time is highly skewed where the difference between P50 and P99 can be as high as 100 times~\cite{socc23-huawei-cloud}. This means that if only the function execution time at a single percentile (P50 or P99) is used for resource allocation, there will be significant resource under-provisioning and over-provisioning for most requests at runtime. To address this issue, our idea is to allow for the exploration of the function execution time at diverse percentiles during resource allocation. 


% It is a prerequisite to profile execution latency for adaptive resource allocation.  
% As aforementioned, owing to a variety of unexpected runtime dynamics,  execution latency demonstrates skewed distributions, by up to 100$\times$ between 99\% percentile and 50\% percentile on Huawei cloud serverless~\cite{socc23-huawei-cloud} .
% This makes the current a single statistic (e.g., mean) or 99\% percentile distribution based profiling suffer significant under- and over-estimation.
% To fix this issue, our insight is to \textit{introduce more diverse percentiles to profile execution latency}. 

\textbf{\textit{Dependencies of adaptation decisions.}}
As the function execution progresses, a sub-workflow will be generated by removing the finished function(s) from the workflow. Within each sub-workflow, the resource adaptation decisions for remaining functions are dependent on each other due to the constraint imposed by the end-to-end latency SLO. For example, under-provisioning a function will result in a reduction of the time budget for executing its downstream functions, thus calling for more resources for these downstream functions to avoid SLO violations. Meanwhile, the selection of the percentile for the execution time of each function dictates resource-latency tradeoff for that function. For example, a higher percentile means that more resources will be allocated to ensure that more requests processed by the function will finish within the given time budget. On the contrary, a lower percentile means that more requests will risk SLO violation, but at the benefits of reduced resource consumption. To address such complex dependencies, we propose the following ideas: (1) We introduce two metrics (i.e., the timeout metric and the resilience metric detailed in \S\ref{sec:profilier}) to balance the resource adaptation decisions of the head function of the current sub-workflow and those of the remaining downstream functions. These metrics help us connect the decision making across sub-workflows and avoids sub-optimal adaptation decisions in each sub-workflow. 
(2) We explore lower percentiles for the head function and a high percentile (i.e., P99) for other functions in each sub-workflow. Using lower percentiles maximizes the opportunity for resource optimization since any over-time execution of the head function can later be compensated by resource adaptation in the next round. The high percentile ensures that the resource adaptation is not too radical to cause SLO violations. 

% Each workflow generates multiple sub-workflows as the execution moves forwards. 
% Within sub-workflows, the provisioning is inter-corrected.
% For instance, under-provisioning upstream functions may directly shrink the time budget for downstream functions, which dictates more resources required by the latter against (sub-) SLO violation. 
% This makes sub-workflows generally adopted as the basic unit to make adaptation decisions~\cite{socc21-llama,rtas22-fa2}. 
%  Moreover,  due to the high variance of execution performance, runtime adaptation requires to carry out function by function, i.e.,  discrete adaptation.
%  This, however, can easily lead to a sub-optimal (analyzed in~\S~\ref{sec:synthesizer:generate}).
% Our insight is to \emph{introduce a metric (i.e., resilience detailed in \S~\ref{sec:profilier}) to quantify the inter-correlation as well as a heuristic design (i.e., heavier head explained in \S~\ref{sec:synthesizer:generate})  to calibrate the sub-optimal,  such that resource adaptation can explore higher resource efficiency without SLO guarantee}.

% In particular, latency percentiles (introduced by the profiling)  involves resource adaptation as a new knob.
% Specifically, higher percentile earns  stronger guarantees in SLOs but may be highly prone to resource over-allocation because of its latency over-estimation, impairing resource efficiency.
% In contrast, decreasing percentiles offers the opportunity to explore higher resource efficiency, but suffers the risk of timeout, i.e., execution latency beyond specified time budget, and  may thus incur  SLO violations.
% Here, our insight is to \emph{moderately explore percentiles (detailed in~\S~\ref{sec:synthesizer:generate}), where head functions of  (sub-)workflows can explore lower percentiles because this creates the opportunity to reap higher resource efficiency while possible timeout can be recovered by subsequent functions' re-adaptive allocation.
% On the other head, non head functions maintain percentiles as 99\%}.
% This can well keep the trade-off between opportunities of exploring higher resource efficiency and risks of SLO violations. 
% Additionally, it effectively shrinks the searching space, benefiting the adaptation with higher time-efficiency.


\textbf{\textit{Tight resource adaptation window.}}
Runtime resource adaptation requires to calculate a new resource allocation decision for the remaining sub-workflow immediately when a function finishes execution. Since serverless functions are typically short-lived (less than 1s on average)~\cite{atc18-peek-bench,socc22-owl,atc20-serverless-in-the-wild,socc23-huawei-cloud}, the window for resource adaptation is quite tight. Assuming the serverless platform will perform the runtime adaptation on behalf of the developer since the platform has access to full runtime information, the resource adaptation decision making should be fast without involving complex calculations and logic or exploring a large space. As discussed before, the serverless platform provider does not have domain knowledge of the serverless workflow. Hence, the developer must pass the necessary information to the serverless platform for runtime adaptation decision making. Our idea is to let the developer synthesize critical hints containing resource allocation rules and options, which the serverless platform provider utilizes to perform runtime resource adaptation. The hints should be highly condensed so the serverless platform can make adaptation decisions quickly enough. 


% Apart from highly varying execution performance, serverless functions are also short-living (less than 1s on average)~\cite{atc18-peek-bench,socc22-owl,atc20-serverless-in-the-wild,socc23-huawei-cloud}, so is the window for adaptive allocation. 
% This variance and volatility calls for a well-preparation of hints for all possible runtime situations while promising them compact and straightforward enough for providers to easily take action.

% Here, our insight is to \emph{holistically synthesize hints in an offline manner, and then utilize the discreteness of adaptive allocation in both decision-making and decision-executing (detailed in~\S~\ref{sec:synthesizer:condense}) to fully condense the hints.
% Finally, hints are warped into a simple and compact table.
% Base on that, providers can accomplish the runtime adaption promptly and properly}.

To demonstrate the potential of runtime resource adaptation incorporating all the above ideas, we take a real-world serverless workflow (explained in \S\ref{exp:setup}) as an example, and evaluate its end-to-end latency (denoted by E2E) and resource consumption (CPU cores).
As illustrated in Figure~\ref{fig:bg:size}, the late-binding (blue triangle) reduces the resource consumption by up to 42.2\% compared with existing early-binding solutions (orange circle), while ensuring SLO guarantees. This highlights the significant gains from runtime resource adaptation. 


\begin{figure}[t!]
\centering
\includegraphics[width=0.45\textwidth]{./figure/motivation/size_early_bind_vs_ours.pdf}
%\vspace{-0.1cm}
\caption{Performance comparison between early-binding (left)~\cite{eurosys19-grandslam} and late-binding (runtime resource adaptation), where the CPU consumption (right) is normalized by the optimal obtained with exhaustive search.} 
%\vspace{-0.3cm}
\label{fig:bg:size}
\end{figure}

   
	







% \input{sections/03-propositions}
% \input{sections/04-theory}
% \input{sections/05-implications}
% \input{sections/06-design-space}
% \input{sections/02-data_quality}
% \input{sections/03-human_factors}
% \input{sections/04-case_studies}
% \input{sections/05-dichotomy}
% \section{Conclusion and Discussion}
\label{sec:discussion}
We present an analysis of hypothetical and disjunctive syllogisms on propositional and modal logic and systematically analyze the LLM performance on the dataset.
Our analysis provides novel insights on explaining and predicting LLM performance: in addition to the perplexity or probability of the input text, the underlying logic forms play an important role in determining the performance of LLMs.
In addition, we compare the behaviors of LLMs and humans using the same data through human behavioral experiments.
We discuss the implications of our results as follows.

\vspace{2pt}
\noindent\textbf{Probability in language models.}
Probability and perplexity are often used as intrinsic evaluation metrics for language models.
While \citet{gonen-etal-2023-demystifying} and \citet{mccoyEmbersAutoregressionShow2024} show that probability and perplexity correlate well with LLM performance, literature in program synthesis with LLMs shows little correlation between probability and execution-based evaluation results \citep{li2022competition,shi-etal-2022-natural}.
This work does not necessarily contradict either line but rather provides complementary factors for analyzing LLM performance.

We argue that probability may have become an overloaded term in analyzing LLMs.
Low probability may be due to one or more of the following non-exhaustive reasons: (1) out-of-context content, (2) ungrammatical language, or (3) grammatical but semantically awkward content (cf. the mirror dataset in \cref{sec:perplexity}), (4) reasonable but rare content.
We hypothesize that the probability of language models may not be essentially able to capture all these nuanced differences, and call for encoding and decoding algorithms---such as \citet{meister-etal-2023-locally}---that can better decompose the probability into finer-grained and explainable components.

\vspace{2pt}
\noindent\textbf{Comparing humans and LLMs.}
What is our goal for building LLMs?
To achieve better performance on practical tasks or to build a more human-like model?
Our results, together with \citet{eisape-etal-2024-systematic}, suggest that these two goals may not be perfectly aligned by revealing a mixture of similarity and discrepancy between LLMs and humans---for example, while LLMs exhibit higher benchmark performance than humans on our dataset and show the same argument form preferences with humans (\cref{fig:emmeans-lm,fig:emmeans-human}), they also show systematic biases that we do not find significant in human reasoning (e.g., disfavoring the necessity modality, \cref{subsec:affirmation-bias}).
While there has been positive evidence of using LLMs as human models in psycholinguistic studies \interalia{misra-kim-2024-generating}, our results suggest executing such approaches cautiously.

\vspace{2pt}
\noindent\textbf{On the relation between modality and performance.}
Our results show that there is a significant difference in performance between necessity and possibility modalities, with the former much lower than the latter (\cref{tab:softacc-base}).
Part of the reason for this is that LLMs have a significant tendency to say ``No'' to the necessity modality (\cref{fig:affirmation-rejection}).

On the one hand, our results extend the conclusion of \citet{dentella-etal-2023-systematic} that LLMs generally respond positively---LLM behaviors may be significantly affected by finer-grained factors, including but not necessarily limited to the modality involved in the input.
On the other hand, while LLMs systematically tend to answer ``No'' to questions in necessity modality, we do not find related evidence in human experiments, which leads us to hypothesize that such rejection bias comes from either the model architecture or the training strategies, such as the reinforcement learning with human feedback \citep[RLHF;][]{ouyang-etal-2022-training} protocol.
We leave this as an open question for future research.

\vspace{2pt}
\noindent\textbf{Modal logic and theory of mind.}
Modality, in principle, encodes mental states and beliefs.
The reasoning of beliefs also resonates with the theory of mind \interalia{premackDoesChimpanzeeHave1978,baron-cohenDoesAutisticChild1985} and machine theory of mind \interalia{rabinowitzMachineTheoryMind2018, maHolisticLandscapeSituated2023}.
Following the effort by \citet{sileo-lernould-2023-mindgames} that uses epistemic modal logic to model the machine theory of mind, our work assesses the behaviors of LLMs on alethic modal logic, distantly revealing the future potential of LLMs in achieving the theory of mind.

% \section{Conclusion}\label{sec:conclusion}
This work introduces a novel approach to TOT query elicitation, leveraging LLMs and human participants to move beyond the limitations of CQA-based datasets. Through system rank correlation and linguistic similarity validation, we demonstrate that LLM- and human-elicited queries can effectively support the simulated evaluation of TOT retrieval systems. Our findings highlight the potential for expanding TOT retrieval research into underrepresented domains while ensuring scalability and reproducibility. The released datasets and source code provide a foundation for future research, enabling further advancements in TOT retrieval evaluation and system development.


% 
The increasing reliance on LLMs for multimodal tasks across far-reaching sectors such as healthcare, finance, and manufacturing underscores the need to assess the accuracy and reliability of the information they generate. Vision-Language Models (VLM) have achieved state-of-the-art (SoTA) performance on Visual Question-Answering (VQA) benchmarks, and these models often utilize Retrieval-Augmented Generation (RAG) to maintain factual accuracy and relevance in a dynamic information environment. However, this has led to uncertainty in the information the LLM bases its answer on, as it may choose between parametric memory and retrieved sources. When models rely on memorized information instead of dynamically retrieving information, they may inadvertently propagate outdated or incorrect information, causing serious legal and ethical risks and undermining trust and reliability in AI systems \citep{huang2023survey}.
% The ability to strike a balance between generalization and specialization in AI systems is therefore crucial for ensuring the safe, reliable use of these technologies in real-world applications.

Despite these concerns, the way that Vision-Language models (VLMs) memorize and retrieve information, particularly in complex multimodal tasks, remains under-explored. Current research often focuses on either the general capabilities of large language models (LLMs) or the specialized retrieval mechanisms in retrieval augmented generation systems (RAG) \citep{incontext_rag,chen_murag_2022,liu_universal_2023}. Particularly in the context of multimodal retrieval and multihop reasoning, few studies analyze the tradeoff between finetuning for specialized tasks and zero-shot prompting for general-purpose vision-language capabilities. A lack of consensus on how to approach this tradeoff motivates the development of measures to quantify reliance on parametric memory, as well as metrics for quantifying the potential performance impact of extending LLMs with RAG systems.

To address this gap, we investigate how multimodal QA models balance accuracy with memorization on the WebQA benchmark. We compare finetuned multimodal systems against zero-shot VLMs, analyzing how retrieval performance influences QA accuracy. In particular, we focus on cases where retrieval fails, allowing us to measure reliance on parametric memory through two proposed metrics---the \ppr (\PPR) which quantifies how much model accuracy is influenced by retrieval quality, contrasting performance in best-case versus worst-case retrieval scenarios, and the \ucr (\UCR) which measures how often correct QA responses are generated when the retriever fails, providing a proxy for memorization.

To enable this analysis, we make several methodological contributions. For the finetuned QA models, we investigate Vision-Transformer (ViT) architectures, which allow for multihop reasoning over multiple sources. To investigate the impact of retrieval performance on trained LMs, we propose a variable-input Fusion-in-Decoder (FiD) model \cite{tanaka_slidevqa_2023, nlvr2}, building upon the VoLTA architecture \citep{pramanick_volta_2023}. For the zero-shot case, we build upon previous research on In-Context Retrieval \citep{incontext_rag} by demonstrating that LLMs such as GPT-4o are capable of performing the final ranking step of the retrieval process. In doing so, we find that GPT-4o, a general-purpose LLM, achieves SoTA performance on the WebQA task, outperforming existing finetuned RAG models by a significant margin (7\% higher accuracy). 

Crucially, our results reveal that while retrieval-augmented models reduce memorization, the training paradigm plays an important role. Finetuned models exhibit higher reliance on parametric memory, whereas zero-shot RAG approaches have lower memorization scores at the cost of accuracy. This suggests that while retrieval modules may mitigate the risks associated with outdated or incorrect information, SoTA performance requires that they be coupled with specialized QA models. Our memorization measures contribute to the development of transparent and reliable AI systems, particularly in applications where the sourcing of up-to-date, factual information is critical.



% We investigate the impact of question complexity on the ability of these models to integrate multiple data sources—such as images, text, and external retrievers—and produce coherent and accurate answers. We also explore whether in-context retrieval can be a viable alternative to traditional retrieval-augmented systems, offering a more streamlined approach to multimodal QA.

% To achieve this, we first compare zero-shot prompting multimodal LLMs with finetuned multimodal systems. We evaluate both types of models on the WebQA benchmark, a dataset designed for complex question answering that requires reasoning across both image and text sources. For the finetuned models, we use a Fusion-in-Decoder (FiD) architecture, which allows for multihop reasoning over multiple sources. Additionally, we introduce the concept of In-Context Retrieval Language Modeling (RLM), where the LLM itself performs retrieval tasks without the need for external retrievers. This method builds upon existing research in in-context learning  and aims to explore the viability of LLMs retrieving relevant sources and generating accurate answers directly from their context window.

% In order to investigate source utilization in finetuned multimodal models and LLMs, three lines of inquiry are established; 
% \begin{itemize}
%     \item Study 1: retrieval vs QA performance on webQA (motivating example, does QA answer correctly even with incorrect sources?)
%     \item Study 2: performance on adversarial examples where parametric knowledge would be incorrect by design
%     \item Study 3: improving performance on adversarial examples by fine-tuning (i.e model robustness)
% \end{itemize}

% Note, there is one weakness in this plan which is tying in the work we've already done. 
% If we added something from adversarial generation to the retrieval experiment (like a combination of study 1 + 3) it would be complete. So for instance we could try fine-tuning the retriever with adversarial examples (and not just the QA model)

% \begin{figure}
%     \centering
%     \includegraphics[width=0.95\linewidth]{figures/segmentation/webqa_segment_infill.png}
%     \caption{Example of the segmentation substitution pipeline from the WebQA task.}
%     % d5c76d760dba11ecb1e81171463288e9
%     \label{fig:seg_sub_pipeline}
% \end{figure}



% Retrieval augmented generation (RAG) with zero-shot prompting and fine-tuning Large Language Models (LLMs) have become the go-to methods for tasks relying on information retrieval and text generation. In many cases the LLMs parametric memory can sufficiently generalize to answer questions without being provided with retrieval mechanisms for out-of-domain knowledge. However, LLMs often hallucinate and provide wrong information in certain scenarios. This problem is amplified even further on open-domain Question Answering (QA) tasks involving multiple modalities. Grounded text generation using retrieved sources \citep{lewis2021retrievalaugmented} has been extensively studied for text-to-text QA tasks, but its application in multimodal settings has not been studied as much.


% Multimodal reasoning and question answering have gained prominence in recent research endeavors, with an increasing emphasis on handling various forms of data, particularly text and images. In this study, we address a specific gap in the existing literature by focusing on the development of a versatile multihop model capable of accommodating varying numbers of input images.

% Our motivation for this research lies in the growing complexity of answering questions using information on the web, where the challenge of navigating the open-domain setting is further complicated by the presence of multiple modalities and sometimes requires reasoning over multiple sources. WebQA is an ideal dataset on which to compare performance of finetuned RAG systems against general purpose LLMs; it is multimodal, with correct answers requiring reasoning over image and text sources. It is multihop, requiring a complex reasoning process over multiple sources. Finally, WebQA questions from different categories can be broken down into subdomains to analyze performance over domains of varying cardinality.

% Motivated by the real-world challenges of building retrieval and question answering (QA) systems, we design and finetune a closed domain, multimodal, multihop QA model, that is capable of reasoning over a varying number of sources taken as input from an external retriever module. This research contributes to the relatively underexplored domain of multihop reasoning across various input sources and modalities. Our goal is to explore the challenges posed by these scenarios and develop strategies that enable QA models to retrieve relevant information, conduct logical or numerical reasoning across diverse modalities, and generate coherent responses in natural language. To our knowledge, this is the first application of the Fusion-in-Decoder (FiD) architecture \cite{tanaka_slidevqa_2023, nlvr2} that is shown to work with a variable number of inputs, enabling multi-hop reasoning over sources.

% In-Context Learning refers to the ability of LLMs to perform any task by simply providing examples in the input prompt \citep{dong2022survey,min2022rethinking}. Inspired by this research, we propose a method to use the LLM itself as a multimodal retriever, potentially eschewing the requirement of a distinct retrieval module, thereby allowing the design of simpler retrieval-augmented QA systems. We dub this method In-Context Retrieval Language Modeling (RLM). To the best of the authors knowledge, In-Content RLM is disparate from other retrieval augmented approaches which utilize external retrieval modules \citep{incontext_rag,chen_murag_2022,liu_universal_2023}. Despite being a natural extension of In-Context learning, In-Context RLM has not yet been studied empirically.

% To expand on our contribution of In-Context Retrieval, this stems from the well-researched in-context learning of LLMs. In-context learning is the ability of a model to perform any task given a sufficient context window \citep{dong2022survey,min2022rethinking}. Such tasks could include retrieval and ranking, but typically, the go-to solution for tasks requiring retrieval has been RAG. To the best of the authors knowledge, In-Context Retrieval is distinct from In-Context Retrieval Augmented Language Modelling (RALM), and despite being a natural extension of In-Context learning, In-Context Retrieval has not yet been shown empirically.

% Finally, we explore the tradeoff between using zero-shot prompting LLMs and the fine-tuning approach. While we find that, overall, GPT-4o obtains SoTA performance on the WebQA task, outperforming the accuracy of existing finetuned RAG approaches by 7\%, finetuned approaches still perform better on more restricted subdomains\footnote{``In-Context RLM" @ \url{https://eval.ai/web/challenges/challenge-page/1255/leaderboard/3168}}. Finally, we validate that GPT-4o is relying on retrieval abilities to solve the task; we find that GPT-4o is capable of retrieving relevant sources in the presence of distractors and furthermore, when GPT-4o fails to retrieve correct sources, it answers incorrectly 75\% of the time, meaning that it is not relying on parametric memory for this task.

% \paragraph{Contributions}
% Based on our experimentation and analysis on the WebQA benchmark, we make the following contributions:
% \begin{itemize}
%     \item Propose a new architecture for multimodal multihop QA that takes variable number of input sources inspired by the Fusion-in-Decoder method.
%     \item Comparison of general purpose LLMs vs specialized models on the WebQA benchmark.
%     \item Observation of In-Context Multimodal Retrieval abilities of GPT-4o and that it does not rely on parametric memory for multimodal QA.
%     \item Analysis of relationship between retrieval and QA task performance.
%     \item Analysis of task and query complexity on the performance of retrieval and QA tasks.
% \end{itemize}
















% Throughout this paper, we will present our methodology, experiments, and findings, emphasizing our approach to multihop reasoning over varying numbers of input images. We believe that our work contributes to a deeper understanding of multimodal reasoning and has the potential to enhance the capabilities of question-answering systems in the intricate, multimodal landscape of web-based information.
% \input{sections_v1/02-data_quality}
% \input{sections_v1/03-human_factors}
% \input{sections_v1/04-case_studies}
% \input{sections_v1/05-dichotomy}
% \section{Conclusion and Discussion}
\label{sec:discussion}
We present an analysis of hypothetical and disjunctive syllogisms on propositional and modal logic and systematically analyze the LLM performance on the dataset.
Our analysis provides novel insights on explaining and predicting LLM performance: in addition to the perplexity or probability of the input text, the underlying logic forms play an important role in determining the performance of LLMs.
In addition, we compare the behaviors of LLMs and humans using the same data through human behavioral experiments.
We discuss the implications of our results as follows.

\vspace{2pt}
\noindent\textbf{Probability in language models.}
Probability and perplexity are often used as intrinsic evaluation metrics for language models.
While \citet{gonen-etal-2023-demystifying} and \citet{mccoyEmbersAutoregressionShow2024} show that probability and perplexity correlate well with LLM performance, literature in program synthesis with LLMs shows little correlation between probability and execution-based evaluation results \citep{li2022competition,shi-etal-2022-natural}.
This work does not necessarily contradict either line but rather provides complementary factors for analyzing LLM performance.

We argue that probability may have become an overloaded term in analyzing LLMs.
Low probability may be due to one or more of the following non-exhaustive reasons: (1) out-of-context content, (2) ungrammatical language, or (3) grammatical but semantically awkward content (cf. the mirror dataset in \cref{sec:perplexity}), (4) reasonable but rare content.
We hypothesize that the probability of language models may not be essentially able to capture all these nuanced differences, and call for encoding and decoding algorithms---such as \citet{meister-etal-2023-locally}---that can better decompose the probability into finer-grained and explainable components.

\vspace{2pt}
\noindent\textbf{Comparing humans and LLMs.}
What is our goal for building LLMs?
To achieve better performance on practical tasks or to build a more human-like model?
Our results, together with \citet{eisape-etal-2024-systematic}, suggest that these two goals may not be perfectly aligned by revealing a mixture of similarity and discrepancy between LLMs and humans---for example, while LLMs exhibit higher benchmark performance than humans on our dataset and show the same argument form preferences with humans (\cref{fig:emmeans-lm,fig:emmeans-human}), they also show systematic biases that we do not find significant in human reasoning (e.g., disfavoring the necessity modality, \cref{subsec:affirmation-bias}).
While there has been positive evidence of using LLMs as human models in psycholinguistic studies \interalia{misra-kim-2024-generating}, our results suggest executing such approaches cautiously.

\vspace{2pt}
\noindent\textbf{On the relation between modality and performance.}
Our results show that there is a significant difference in performance between necessity and possibility modalities, with the former much lower than the latter (\cref{tab:softacc-base}).
Part of the reason for this is that LLMs have a significant tendency to say ``No'' to the necessity modality (\cref{fig:affirmation-rejection}).

On the one hand, our results extend the conclusion of \citet{dentella-etal-2023-systematic} that LLMs generally respond positively---LLM behaviors may be significantly affected by finer-grained factors, including but not necessarily limited to the modality involved in the input.
On the other hand, while LLMs systematically tend to answer ``No'' to questions in necessity modality, we do not find related evidence in human experiments, which leads us to hypothesize that such rejection bias comes from either the model architecture or the training strategies, such as the reinforcement learning with human feedback \citep[RLHF;][]{ouyang-etal-2022-training} protocol.
We leave this as an open question for future research.

\vspace{2pt}
\noindent\textbf{Modal logic and theory of mind.}
Modality, in principle, encodes mental states and beliefs.
The reasoning of beliefs also resonates with the theory of mind \interalia{premackDoesChimpanzeeHave1978,baron-cohenDoesAutisticChild1985} and machine theory of mind \interalia{rabinowitzMachineTheoryMind2018, maHolisticLandscapeSituated2023}.
Following the effort by \citet{sileo-lernould-2023-mindgames} that uses epistemic modal logic to model the machine theory of mind, our work assesses the behaviors of LLMs on alethic modal logic, distantly revealing the future potential of LLMs in achieving the theory of mind.

% \section{Conclusion}\label{sec:conclusion}
This work introduces a novel approach to TOT query elicitation, leveraging LLMs and human participants to move beyond the limitations of CQA-based datasets. Through system rank correlation and linguistic similarity validation, we demonstrate that LLM- and human-elicited queries can effectively support the simulated evaluation of TOT retrieval systems. Our findings highlight the potential for expanding TOT retrieval research into underrepresented domains while ensuring scalability and reproducibility. The released datasets and source code provide a foundation for future research, enabling further advancements in TOT retrieval evaluation and system development.

\bibliographystyle{ACM-Reference-Format}
\bibliography{references}


%%
%% If your work has an appendix, this is the place to put it.
\appendix

\end{document}
\endinput
%%
%% End of file `sample-sigconf.tex'.
