% \section{Quality Factors: Motivation \& Incentives}
\section{Human Factors in Quality\textemdash{}\\ Motivation \& Incentives}
Data collection platforms used in machine learning, such as MTurk, Prolific, UpWork, and ScaleAI, are designed in ways that include compensation structures, that directly influence both effort and data quality. While rapid crowdsourcing platforms (e.g., MTurk) favor low-effort and low-pay tasks that can scale easily, the quality of task output remains unreliable. In contrast, freelance job platforms tend to favor high-effort and higher-pay gigs which require more deliberate and engaged participation, often leading to higher quality outputs.

This difference aligns with a straightforward intuition\textemdash{}higher pay leads to greater effort and better-quality contributions~\citetext{e.g., \textcolor{myorange}{\citealt{mason2009financial,ho2015incentivizing,shah2016double,laux2024improving}}}. This assumption drives much of the current incentive-based data collection paradigm, where the goal is to use external compensation as a lever to elicit higher-quality data, when required. However, while external rewards can drive greater effort, evidence suggests that they are not always the primary determinant of high-quality engagement.

For instance, some of the highest-quality human contributions come from platforms where users are not financially compensated at all, such as Wikipedia, Reddit, and open-source communities. Here, participants contribute not because of pay, but because they find the activity meaningful, socially rewarding, or aligned with personal interests~\cite{forte2005wikipedia,lampe2010motivations}. These platforms challenge the idea that quality data generation must always rely on financial incentives, illustrating that intrinsic motivations can sustain long-term engagement without depending on financial rewards as the primary driver.

While many assume that external incentives and intrinsic motivation are correlated, research shows that their relationship is far more complex.

\begin{figure}[h]
    \centering
    \includegraphics[width=0.8\linewidth]{illustrations/motivation.png}
    \label{fig:motivation}
\end{figure}

\noindent\textbf{Overjustification Effect}~\cite{lepper1973undermining} explains how external rewards can diminish intrinsic motivation and affect task performance. In a classic experiment, preschool children who already enjoyed drawing were divided into three groups: (1) those who were promised and received a reward, (2) those who received an unexpected reward, and (3) those who received no reward. This experiment revealed two notable outcomes: first, children in the expected-reward group spent significantly less time drawing voluntarily after the reward was removed, compared to the other groups. Second, the drawings from the no-reward and unexpected-reward groups were rated as slightly higher in quality than those from the expected-reward group. This suggests that when an activity initially driven by intrinsic motivation is externally incentivized, the removal of rewards can lead to a decline in voluntary participation, and rewards may also subtly shift focus from quality to mere completion of the task.

\textbf{So why do external incentives sometimes backfire?} Two key psychological theories help explain why the Overjustification Effect\textemdash{}how extrinsic rewards can sometimes diminish intrinsic motivation\textemdash{}occurs:
\begin{itemize}[left=0cm]
    \item \textbf{Self-Perception Theory (SPT)}~\cite{bem1972self} suggests that individuals infer their own attitudes and motivations by observing their past behaviors. When external rewards are introduced, people may start attributing their participation to the incentive rather than to their original or intrinsic interest. Over time, this shift in self-perception can make them less likely to continue the behavior once the reward is removed.
    \item \textbf{Self-Determination Theory (SDT)}~\cite{deci1971effects} offers a broader framework by focusing on autonomy, competence, and relatedness as key psychological needs for intrinsic motivation. When a task is externally controlled through incentives, individuals may feel a loss of autonomy, making the activity feel like an obligation rather than a choice. This helps explain why highly controlled environments often struggle to sustain long-term engagement.
\end{itemize}
Together, SPT and SDT highlight why financial incentives alone are not a sustainable solution for maintaining high-quality, long-term engagement.

\textbf{If intrinsic motivation is key to sustaining high-quality, long-term contributions, then what role do external incentives play?} While excessive reliance on extrinsic rewards can be detrimental, carefully designed incentives can help initiate engagement in behaviors that might otherwise remain dormant. For example, small, well-calibrated incentives can serve as interventions, bringing attention to valuable behaviors without overwhelming intrinsic motivation~\cite{deci1971effects}. Even in systems designed to favor intrinsically motivated behavior\textemdash{}such as laissez-faire environments \cite{hayek2014road}, where individuals are free to act and bear the consequences of their choices\textemdash{}subtle incentive mechanisms can still be useful to align individual and collective goals, as with \textit{nudging}~\cite{leonard2008richard}. Overpresence of incentives has generally led to unintended consequences\textemdash{}either they are optimized to the point of losing effectiveness~\citetext{Goodhart's Law; \textcolor{myorange}{\citealt{goodhart1984problems}}}, or, worse, they introduce undesirable behaviors counterproductive to desired goals~\citetext{Perverse Incentives; \textcolor{myorange}{\citealt{kerr1975folly}}}\citetext{Cobra Effect; \textcolor{myorange}{\citealt{siebert2001cobra}}}.

\textbf{So, how do these social theories play out in real-world data sources?} Returning to the two sources of data\textemdash{}data collection systems like MTurk vs. naturally occurring data sources (community-based platforms) like Wikipedia\textemdash{}the discussed social theories provide valuable insights into their differing approaches to engagement. Data collection systems, being intentionally designed, are structured around external incentives, with financial rewards and intrinsic motivation naturally coexisting at first. However, over time, a crowding-out effect takes hold: as intrinsic motivation erodes, platforms and data collectors continually increase external incentives and tighten control to sustain participation and quality, creating a vicious cycle where contributors optimize for efficiency rather than genuine engagement. This often leads to over-reliance on shortcuts, such as automating survey responses with AI tools, at the expense of quality.

In contrast, community-based platforms, like Wikipedia or Reddit, primarily rely on intrinsic motivation, with minimal external incentives such as contributor badges, reputation systems, and recognition. These platforms sustain engagement over the long term by creating a sense of social belonging, often intertwined with competence and autonomy, demonstrating that high-quality contributions can be sustained without heavy reliance on financial incentives.

This contrast underscores a crucial insight: designing sustainable data collection systems is not just about offering better incentives\textemdash{}it requires structuring environments that actively sustain and enhance intrinsic motivation.