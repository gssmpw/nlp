\section{Reviving Human Data Sourcing: \\A New Path Forward}
In the search for sustainable ways to collect human-generated data for AI, we examined existing data collection systems through the lens of the quantity-quality trade-off which arises as a consequence of system-level design choices that make achieving both difficult. While such trade-offs are inherent to designed systems, they can be mitigated by carefully addressing the factors that shape them.

To unpack the latent factors involved, we examined two core drivers of this trade-off: quality, influenced by motivation and incentives, and quantity, shaped by fragmentation and efficiency. We examined how over-reliance on external incentives and excessive task fragmentation erode intrinsic motivation, ultimately leading to long-term declines in data quality. To counteract this, we propose shifting from controlling individual tasks to designing structured and trustworthy environments that sustain engagement while still allowing data collectors to shape the kind of data they gather.

We explore games as a promising new frontier for data collection. By naturally balancing structure with voluntary participation, they offer a scalable, high-quality, and sustainable alternative to conventional data collection systems\textemdash{}one that is increasingly important as traditional approaches face growing limitations.

% There exists a trade-off between data quantity and quality—not at the data level itself, but at the system level, where structural design choices make it difficult to achieve both simultaneously. While such trade-offs are likely inevitable in any built system, we argue that they can be mitigated by carefully untangling the latent factors that shape them.

% In this paper, we examine two core drivers of this trade-off: quality, governed by motivation and incentives, and quantity, shaped by fragmentation and efficiency. We show how undesirable system-level distortions—such as excessive reliance on external incentives and over-fragmentation of tasks -- erode intrinsic motivation, ultimately degrading quality over time. To counteract this, we propose a shift from controlling individual tasks to designing structured environments that preserve intrinsic engagement while still giving collectors control over the types of data they gather.

% We explore games as a promising frontier for rethinking data collection systems. As environments that inherently balance structure with organic participation, they offer a compelling alternative to existing data collection models -- one that enables the scalable, high-quality, and sustainable sourcing of human data in an era where traditional methods are becoming increasingly strained.
