Our UASTHN framework is illustrated in Fig.~\ref{fig:framework}. The framework consists of two main components: a Deep Homography Estimation (DHE) module and an Uncertainty Estimation (UE) module utilizing Crop-based Test-Time Augmentation (CropTTA) and Deep Ensembles (DE).

\subsection{Deep Homography Estimation (DHE) Module}
The DHE module employs a homography network $F_H$. We denote $W_S$ as the size of the square satellite image $I_S$ and $W_T$ as the size of the square thermal image $I_T$. Both images are resized to $W_R$, yielding $I_{RS}$ and $I_{RT}$. The homography network $F_H$ takes $I_{RS}$ and $I_{RT}$ and outputs the four-corner displacement $D_{RS\rightarrow RT} \in \mathbb{R}^{2 \times 4}$, indicating the displacement from the four corners of $I_{RS}$ to $I_{RT}$. This displacement essentially aligns $I_{RT}$ into $I_{RS}$. Subsequently, Direct Linear Transformation (DLT)~\cite{ABDELAZIZ2015103} utilizes this displacement to compute the homography matrix $H_{RS\rightarrow RT}$.

\subsection{Crop-based Test-Time Augmentation (CropTTA)}

We propose CropTTA as a simple and effective method for measuring data uncertainty in DHE for TG. Our approach involves augmenting $I_T$ by cropping it with a specific crop offset $o_c$. We denote the $i^\text{th}$ augmented thermal image as $I^i_{CT}$ with the size of $W_{CT} = W_T - o_c$, where $i=1,\cdots,N_C-1$ and $N_C-1$ is the number of augmented samples for each $I_T$. We explore two sampling methods: random sampling and grid sampling. In random sampling, the samples consist of the original image and $N_C-1$ randomly cropped images. In contrast, the grid sampling method utilizes the original image and $N_C-1$ cropped images covering four corners.

\begin{figure*}
    \centering
    \includegraphics[width=0.85\linewidth]{fig/framework_UASTHN.pdf}
    \caption{UASTHN framework: CropTTA augments thermal images, and network $F_H$ with an UE module calculates aggregated displacements ($\tilde D_{RS\rightarrow RT}$) and data uncertainty ($U^\textrm{TTA}_{RS\rightarrow RT}$). $U^\textrm{TTA}_{RS\rightarrow RT}$ is used to reject samples with high uncertainty. Optionally, DE estimates model uncertainty ($U^\textrm{DE}_{RS\rightarrow RT}$), which can be combined with CropTTA for comprehensive UE.}
    \label{fig:framework}
    \vspace{-15pt}
\end{figure*}

Next, we resize $I^i_{CT}$ to $I^i_{RCT}$ to $W_R$, and composite the thermal image batch as $\{I_{RT}, I^1_{RCT},\cdots,I^{N_C-1}_{RCT}\}$. The homography network $F_H$ predicts displacements $\{D_{RS\rightarrow RT}, D^1_{RS\rightarrow RCT},\cdots,D^{N_C-1}_{RS\rightarrow RCT}\}$. To recover the displacements $D^i_{RS\rightarrow RT}$ from the cropped displacements $D^i_{RS\rightarrow RCT}$, we apply the following transformations
\begin{equation}
    H^i_{RS\rightarrow RT} = \textrm{DLT}(\mathbf{\hat x}^i_{RCT}, \mathbf{x}^i_{RCT}),
\end{equation}
\begin{equation}
    H^i_{RCT\rightarrow RT} = \textrm{DLT}(\mathbf{\hat x}^i_{RCT}, \mathbf{\hat x}_{RT}),
\end{equation}
\begin{equation}\small
     \begin{bmatrix}
       \bm{x}^i_{RT}\\ \bm{y}^i_{RT} \\ \mathbf{1}
   \end{bmatrix} = H^i_{RS\rightarrow RT}H^i_{RCT\rightarrow RT}(H^i_{RS\rightarrow RT})^{-1}\begin{bmatrix}
       \bm{x}^i_{RCT}\\ \bm{y}^i_{RCT} \\ \mathbf{1}
   \end{bmatrix},
\end{equation}
\begin{equation}\small
   \mathbf{x}^i_{RT} = \begin{bmatrix}
       \bm{x}^i_{RT}\\ \bm{y}^i_{RT}
   \end{bmatrix},
    \mathbf{x}^i_{RCT} = \begin{bmatrix}
       \bm{x}^i_{RCT}\\ \bm{y}^i_{RCT}
   \end{bmatrix} = \mathbf{x}_{RS} + D^i_{RS\rightarrow RCT},
\end{equation}
and we get recovered displacements as
\begin{equation}
        D^i_{RS\rightarrow RT} = \mathbf{x}^i_{RT} - \mathbf{x}_{RS},
\end{equation}
where $\mathbf{x}_{RS},~\mathbf{x}^i_{RT},~\mathbf{x}^i_{RCT}\in \mathbb{R}^{2\times4}$ represent the four-corner coordinates of $I_{RS}$, and the $i^\text{th}$ predicted four-corner coordinates of $I_{RT}$ and $I^i_{RCT}$ respectively. $\mathbf{\hat x}_{RT},~\mathbf{\hat x}^i_{RCT}\in \mathbb{R}^{2\times4}$ denote the four-corner coordinates of $I_{RT}$ and $I^i_{RCT}$ before the predicted homography transformation. Conversely,  $\mathbf{x}^i_{RCT}$ and $\mathbf{x}^i_{RT}$ are the coordinates after transformation. $H^i_{RCT\rightarrow RT}$ and $H^i_{RS\rightarrow RT}$ denote homography matrices from $I^i_{RCT}$ to $I_{RT}$ and from $I_{RS}$ to $I_{RT}$ for the $i^\text{th}$ prediction. $\bm{x}^i_{RT},~\bm{y}^i_{RT},~\bm{x}^i_{RCT},~\bm{y}^i_{RCT} \in \mathbb{R}^{1\times 4}$ are the $x$ and $y$ coordinates of $\mathbf{x}^i_{RT}$ and $\mathbf{x}^i_{RCT}$. We use the standard deviation (std) of displacements as the measurement of data uncertainty

\begin{equation}
    U^\textrm{TTA}_{RS\rightarrow RT} = \textrm{std}(\mathcal{D}_{RS\rightarrow RT}),
\end{equation}
\begin{equation}
\mathcal{D}_{RS\rightarrow RT} = \{D_{RS\rightarrow RT}, D^1_{RS\rightarrow RT},\cdots,D^{N_C-1}_{RS\rightarrow RT}\},
\end{equation} 
where $U^\textrm{TTA}_{RS\rightarrow RT}\in \mathbb{R}^{2\times4}$ represents the standard deviation for the estimated four-corner displacement. We denote the rejection threshold as $s_c$. The estimation results are rejected if $U^\textrm{TTA}_{RS\rightarrow RT} > s_c$, meaning all elements of $U^\textrm{TTA}_{RS\rightarrow RT}$ are larger than $s_c$. The aggregated displacement $\tilde D_{RS\rightarrow RT}$ is calculated using either the average displacements of all samples in $\mathcal{D}_{RS\rightarrow RT}$ or only the original displacement $D_{RS\rightarrow RT}$.

The intuition of CropTTA is that all cropped views share the same homography matrix as the original thermal image. If $I^\prime_T$ is generated from $I_T$ using matrix $H$, then for any cropped views $I_{CT}$, pixel coordinates transform as $x^\prime_{CT} = Hx_{CT}$, allowing calculation of the transformed four-corner coordinates. Additionally, when thermal images are affected by sensor issues or low-contrast inputs, the network $F_H$ predicts similar $D^i_{RS\rightarrow RCT}$. In such cases, $U^i_{RS\rightarrow RT}$ is dominated by $H^i_{RCT\rightarrow RT}$, maintained by sample distribution.

For the model training of the displacement with CropTTA, the loss function $\mathcal{L}_\textrm{CropTTA}$ is
\begin{equation}
\begin{split}
    \mathcal{L}_\textrm{CropTTA} &= \sum_{k=0}^{K_1-1}\gamma^{K_1 -k-1} \left( \Vert D_{k,RS \rightarrow RT} - D^{gt}_{k,RS \rightarrow RT}\Vert_1 \right.\\ & \left. + \sum_{i=1}^{N_C}\Vert D^i_{k,RS \rightarrow RT} -  D^{gt}_{k,RS \rightarrow RT}\Vert_1 \right),
\end{split}
\end{equation}
where $\gamma$ and $K_1$ denote the decay factor and the number of iterations for $F_H$ if $F_H$ is an iterative network; otherwise, $K_1 = 1$ and $\gamma=1.0$. $D^{gt}_{k,RS \rightarrow RT}$ is the ground truth.

\begin{figure*}[]
\begin{subfigure}[b]{0.33\textwidth}
    \includegraphics[width=\textwidth]{fig/extended/grid_random_5_10_val_mct.pdf}\vspace{-10pt}
    \caption{Number of Training Crops}
    
    \label{mct}
\end{subfigure}
\begin{subfigure}[b]{0.33\textwidth}
    \includegraphics[width=\textwidth]{fig/extended/grid_random_5_10_val_croptta.pdf}\vspace{-10pt}
    \caption{Sampling Methods}
    \label{croptta}
\end{subfigure}
\begin{subfigure}[b]{0.33\textwidth}
    \includegraphics[width=\textwidth]{fig/extended/grid_random_5_10_val_agg.pdf}\vspace{-10pt}
    \caption{Aggregation Methods}
    \label{agg}
\end{subfigure}
\begin{subfigure}[b]{0.33\textwidth}
    \includegraphics[width=\textwidth]{fig/extended/grid_random_5_10_val_sample.pdf}\vspace{-10pt}
    \caption{Sample Numbers}
    \label{sample}
\end{subfigure}
\begin{subfigure}[b]{0.33\textwidth}
    \includegraphics[width=\textwidth]{fig/extended/grid_random_5_10_val_cs.pdf}\vspace{-10pt}
    \caption{Early Stopping}
    \label{cs}
\end{subfigure}
\begin{subfigure}[b]{0.33\textwidth}
    \includegraphics[width=\textwidth]{fig/extended/grid_random_5_10_val_merge.pdf}\vspace{-10pt}
    \caption{Merge Function}
    \label{merge}
\end{subfigure}
    \caption{Ablation Study. We use success rate and Validation (Val) MACE metrics to ablate the training and evaluation settings. \textit{Baseline} indicates STHN~\cite{STHN} two-stage baseline performance.}
    \label{ablation}
\vspace{-15pt}
\end{figure*}

\subsection{Deep Ensembles (DE) and Merge Functions} To comprehensively account for both data uncertainty and model uncertainty, we employ Deep Ensembles (DE) for model uncertainty. We train $N_m$ models with different random seeds. During evaluation, we aggregate the predictions and calculate standard deviations as a measure of uncertainty
\begin{equation}
     U^\textrm{DE}_{RS\rightarrow RT} = \textrm{std}(\mathcal{D}^\prime_{RS\rightarrow RT}), 
     \end{equation}
     \begin{equation}\quad\mathcal{D}^\prime_{RS\rightarrow RT} = \{D_{RS\rightarrow RT,m},~m=1,\cdots,N_m\},
\end{equation}
where $U^\textrm{TTA}_{RS\rightarrow RT}\in \mathbb{R}^{2\times4}$ denotes the model uncertainty, and $D_{RS\rightarrow RT,m}$ represents the predicted displacement from the $m^\text{th}$ trained model. Therefore, we obtain data and model uncertainty for each corner and apply a merge function $f(\cdot)$ to evaluate the total uncertainty
\begin{equation}
     U^\textrm{total}_{RS\rightarrow RT} = f(U^\textrm{TTA}_{RS\rightarrow RT}, U^\textrm{DE}_{RS\rightarrow RT}),
\end{equation}
where $U^\textrm{total}_{RS\rightarrow RT}$ represents the total uncertainty, and $f(\cdot)$ is a strategy that utilizes either the minimum/maximum values of data and model uncertainty or their sum. 

To reduce the computational cost of extra samples, we propose an \textit{early stopping} mechanism. It uses all samples for the first $k$ iterations for UE, then switches to a single sample solely for homography estimation, maintaining accuracy while significantly cutting computational overhead.


