
\vspace{-20pt}
\subsection{Ablation Study}\label{sec:ablation}
In this section, we conduct an ablation study (Fig.~\ref{ablation}) to assess the performance impact of our methods' design. We use the STHN two-stage model~\cite{STHN} with $W_S=1536$ and $D_C=512~\si{m}$ as the baseline. We plot the success rate against the Validation (Val) MACE metric with multiple thresholds. 



\textbf{Number of Training Crops.} Fig.~\ref{mct} displays the performance of models trained with different numbers of training crops. The results indicate that models trained with $N_C=5$ achieve a higher success rate and lower error compared to those trained with $N_C=1$ (only the original image without crop augmentation) and $N_C=3$.


\textbf{CropTTA Sampling Method.} Fig.~\ref{croptta} compares different sampling methods and crop offsets for CropTTA. To minimize the randomness of random sampling, we average the results of five random seeds for each random sampling method. The results indicate that random sampling generally outperforms grid sampling, with random sampling using $o_c = 32$ yielding the best performance.

\begin{figure}
    \centering
\includegraphics[width=0.35\textwidth,height=0.35\textwidth]{fig/grid_random_5_10_roc.pdf}\vspace{-10pt}
   \caption{ROC curves for CropTTA with STHN two-stage methods~\cite{STHN} across different $D_C$, with predictions exceeding $25~\si{m}$ MACE considered as expected rejected predictions.}\label{roc}
   \vspace{-15pt}
\end{figure}

\begin{figure}
    \centering
\includegraphics[width=0.35\textwidth,height=0.35\textwidth]{fig/grid_random_5_10_hist.pdf}\vspace{-10pt}
   \caption{MACE histogram for CropTTA with STHN two-stage methods across different $D_C$}\label{hist}
   \vspace{-15pt}
\end{figure}

\begin{figure*}[]
\centering
\begin{subfigure}[b]{0.15\textwidth}
    \includegraphics[width=\textwidth]{fig/vis1.pdf}
    \vspace{-20pt}
    \caption{}
        \vspace{-5pt}
\end{subfigure}
\hspace{0.04em}
\begin{subfigure}[b]{0.15\textwidth}
    \includegraphics[width=\textwidth]{fig/vis3.pdf}
    \vspace{-20pt}
    \caption{}
        \vspace{-5pt}
\end{subfigure}
\hspace{0.04em}
\begin{subfigure}[b]{0.15\textwidth}
    \includegraphics[width=\textwidth]{fig/vis4.pdf}
    \vspace{-20pt}
    \caption{}
        \vspace{-5pt}
\end{subfigure}
\hspace{0.04em}
\begin{subfigure}[b]{0.15\textwidth}
    \includegraphics[width=\textwidth]{fig/vis2.pdf}
    \vspace{-20pt}
    \caption{}
    \vspace{-5pt}
\end{subfigure}
\hspace{0.04em}
\begin{subfigure}[b]{0.15\textwidth}
    \includegraphics[width=\textwidth]{fig/vis5.pdf}
    \vspace{-20pt}
    \caption{}
        \vspace{-5pt}
\end{subfigure}
\hspace{0.04em}
\begin{subfigure}[b]{0.15\textwidth}
    \includegraphics[width=\textwidth]{fig/vis6.pdf}
    \vspace{-20pt}
    \caption{}
    \vspace{-5pt}
\end{subfigure}
\caption{CropTTA detected failure cases with the STHN two-stage method. Thermal images overlap with satellite images, showing \textcolor{green}{ground truth} and \textcolor{red}{predicted} displacements. Thermal images are overlaid on \textcolor{red}{predicted} displacements on the satellite imagery for visualization. Categories from left to right: (a) textureless thermal features, (b) corrupted thermal images, (c) geometric noise, (d) self-similar satellite maps, (e) thermal images exceeding search regions, and (f) outdated satellite maps.}
    \label{vis}
    \vspace{-15pt}
\end{figure*}

\textbf{Aggregation Methods.} Fig.~\ref{agg} compares model performance with different aggregation methods. \textit{Original} uses the original image result, while \textit{mean} averages results from all samples. It shows that DE performs better with the \textit{mean} method, while CropTTA prefers the \textit{original} method, likely due to cropped images containing partial information.

\textbf{Sample Numbers.} Fig.~\ref{sample} illustrates the evaluation results for different sample numbers. The curves indicate that the performance of CropTTA and DE converge when the sample numbers exceed 4 and 3, respectively. This suggests the minimal sample numbers required for optimal performance.

\textbf{Early Stopping.} Fig.~\ref{cs} illustrates the impact of early stopping on UE across different iteration numbers for $F_H$, assuming $F_H$ is an iterative model~\cite{cao2022iterative, STHN}. The results suggest that early stopping can effectively enhance the efficiency of iterative methods without compromising performance.

\textbf{Merge Function.} Fig.~\ref{merge} presents the results of various merge functions used to combine CropTTA and DE. The \textit{max} and \textit{add} methods exhibit similar performance, while the \textit{min} method achieves higher success rates but also higher error. Combining CropTTA and DE allows us to have a comprehensive UE for better performance. We choose \textit{max} as our default merge function.

\subsection{Comparison with Baselines}\label{sec:baseline}
Table~\ref{baseline} evaluates the performance of various UE methods. Our results show that CropTTA enhances alignment accuracy for large $D_C$ values (low-frequency localization). However, for $D_C = 128~\si{m}$ (high-frequency localization), baselines without UE outperform, indicating inherent lower bounds of MACE and CE. Similar errors persist for $D_C = 256~\si{m}$ and $512~\si{m}$ with UE, even after removing high-uncertainty samples. Notably, IHN and STHN achieve superior performance for $D_C = 256~\si{m}$ and $512~\si{m}$, maintaining success rates above 95\%. Figure~\ref{roc} shows the ROC curves for STHN with CropTTA, where $D_C = 256\si{m}$ and $512~\si{m}$ yield higher true positive rates than $D_C = 128~\si{m}$. This is supported by the long-tail error distribution in the MACE histogram (Fig.~\ref{hist}), showing improved UE performance.

Comparing UE methods, CropTTA generally outperforms or matches DE and DM, except when DHN has high errors or IHN under $D_C=128~\si{m}$. DM struggles to detect data uncertainty in low-error results, leading to uncertainty underestimation, while CropTTA effectively handles these cases. Combining CropTTA and DE improves performance when both estimate well but suffers if either has high errors.

\textbf{Inference Time.} Table~\ref{time} compares inference times of various UE methods, showing CropTTA is more efficient than DE in both one-stage and two-stage methods. Early stopping also reduces the inference time with more samples, making it practical for real-time TG applications.



\subsection{Failure Detection}\label{sec:robustness}
Fig.~\ref{vis} provides a qualitative analysis of failures detected by CropTTA, highlighting six categories of high data uncertainty samples. Textureless thermal images show low contrast or flat features without landmarks. Corrupted thermal images had brightness adjusted for overexposure and underexposure, while geometric noise was added by shifting corners by up to 64 pixels. Self-similar satellite maps, like desert dunes, exhibit repetitive patterns. Thermal images extending beyond the search region had displacements partially outside satellite boundaries. The outdated satellite map in 2020 is compared with thermal images captured in 2021, revealing new roads and farms. CropTTA effectively detects failures due to textureless, corrupted, and out-of-range images, self-similar patterns, outdated maps, and geometric noise.


