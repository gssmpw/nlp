\textbf{UAV Thermal Geo-localization (TG).} TG involves using thermal cameras on UAVs in conjunction with satellite maps to extract the locations of thermal images and determine the UAV's position. Existing approaches employ two primary strategies: global matching~\cite{foundloc, stl} and local matching. This work focuses specifically on local matching methods.

% \textit{Global-matching methods}. These methods are an extension of UAV visual geo-localization~\cite{foundloc, vgscience, directalign3, imgregistration}, and follow the image retrieval paradigm. These methods involve creating a satellite imagery database for targeted search areas and matching these images with RGB images captured by UAVs. Adapting these techniques to thermal imagery presents significant challenges, including limited satellite-thermal image pairs and a significant domain gap between satellite and thermal images, which block the advancement of TG. To address these challenges, \cite{stl} proposes to use a generative model to generate synthesized thermal images to build training pairs with extensive satellite images and applies domain adaptation methods to enhance matching accuracy. Although this method has proven effective, the high sampling density of the satellite database enhances localization accuracy but also significantly increases both search time and memory requirements.

Local matching methods use Deep Homography Estimation (DHE)~\cite{detone2016deep, cao2022iterative, nguyen2018unsupervised, shao2021localtrans} or keypoint matching\cite{pmlr-v155-achermann21a} to align thermal and satellite images. Some works~\cite{electronics12040788, electronics12214441} employ conditional GANs for visual-thermal homography, while \cite{STHN} uses two-stage iterative DHE for TG, offering near real-time performance but struggling with challenging alignment such as textureless thermal images. Our method integrates an uncertainty estimation mechanism with CropTTA, improving DHE resilience by flagging low-confidence cases, and enhancing TG system reliability and situational awareness.

\textbf{Uncertainty Estimation for Deep Learning.}
Uncertainty Estimation (UE)\cite{gawlikowski2023survey, kendall2017uncertainties}, also known as uncertainty quantification\cite{ABDAR2021243}, is vital in safety-critical deep-learning applications like UAV localization and navigation. Without accurate UE, a neural network cannot indicate the reliability of its outputs and may show overconfidence in cases of high data uncertainty (aleatoric), such as noisy data or sensor failures, or high model uncertainty (epistemic), like out-of-distribution samples, potentially leading to system failures. We examine the following categories of UE methods:

\textit{Test-Time Augmentation (TTA).} TTA methods~\cite{shanmugam2021better, kimura2021understanding, kim2020learning, zhang2022memo} use data augmentation during evaluation to combine outputs from augmented input samples and measure uncertainty. These methods explore the best augmentation tailored for different tasks and primarily target classification tasks but have limited application to regression tasks like DHE.

\textit{Deep Ensembles (DE).} DE methods~\cite{lakshminarayanan2017simple, NEURIPS2021_a70dc404,fort2019deep,abe2022deep} train multiple models with varying initializations and data orders, then combine their outputs to assess uncertainty. Although greater model diversity often enhances performance, its impact on out-of-distribution samples remains debated~\cite{fort2019deep, abe2022deep}.

For UE in deep homography estimation, existing works mainly use visibility masks~\cite{zhang2022hvc} and pixel-level photometric matching uncertainty~\cite{xu2022cuahn}. In contrast, our CropTTA method leverages the homography consensus of crop-augmented images to measure data uncertainty. This approach can seamlessly integrate with any DHE method and is proven to be both effective and efficient for TG.
\vspace{-5pt}


% \begin{figure*}
%     \centering
%     \includegraphics[width=0.8\linewidth]{fig/framework.pdf}
%     \caption{STHN Framework Overview: For the data preparation phase, TGM produces synthetic thermal images from unpaired satellite images, augmenting the dataset. The deep homography estimation phase employs an iterative update approach, using $F_H$ to predict the displacement $D_{RS\rightarrow RT}$ between thermal images and satellite maps. For refinement, the framework crops and resizes the selected region $B$, utilizing $F^\prime_H$ to fine-tune the four-corner displacement prediction for enhanced accuracy.}
%     \label{fig:framework}
%     \vspace{-10pt}
% \end{figure*}

