Over the past decade, Unmanned Aerial Vehicles (UAVs) have proven highly adaptable and efficient across various tasks, including agriculture~\cite{info10110349}, solar farm inspections~\cite{drones6110347}, search and rescue~\cite{atif2021uav}, power line monitoring~\cite{rao2022quadformer, powerline}, and object tracking~\cite{saviolo2023unifying}. Research has focused on improving UAV localization and navigation to ensure stable flight and accurate trajectory tracking. For long-term outdoor operations, accurate GPS localization is essential to prevent drift~\cite{review_avl}. When GPS is unreliable due to signal loss or interference, visual geo-localization~\cite{foundloc, vgscience, directalign3, imgregistration} aligns aerial images with satellite maps for robust positioning. In low-light conditions, Thermal Geo-localization (TG), using aerial thermal imagery~\cite{lee2024caltech}, image retrieval~\cite{stl}, and deep homography estimation~\cite{STHN}, supports effective navigation.

Despite the promising results of TG in aligning aerial thermal images with satellite imagery using deep learning techniques~\cite{lecun2015deep}, several challenges hinder their practical applications. First, these methods lack mechanisms to indicate low confidence when confronted with textureless or self-similar patterns in thermal or satellite images. Second, the reliance on north alignment through IMU and compass data, with limited tolerance for geometric noise, renders these systems susceptible to large geometric distortions. Third, existing approaches assume the availability of corresponding satellite images for global matching and that the UAV is within the search area, leading to failures if the UAV moves beyond this zone. Hence, incorporating uncertainty measurement is vital for improving inference reliability. 

Fig.~\ref{teaser} highlights key categories of high data uncertainty samples commonly encountered in TG, identified by our method: (a) \textbf{Textureless Features}: Low-contrast, textureless thermal images, especially at nighttime. (b) \textbf{Image Corruption}: Overexposed, underexposed, or noisy thermal data. (c) \textbf{Geometric Noise}: Severe north-alignment errors from inaccurate IMU or compass information causing geometric distortion. (d) \textbf{Self-similar Maps}: Repetitive satellite patterns (e.g., desert dunes) leading to false matches. (e) \textbf{Exceeding Regions}: Thermal images extending beyond satellite map. (f) \textbf{Outdated Maps}: Satellite images not reflecting recent developments, causing inconsistencies with thermal imagery.


\begin{figure*}[]
    \centering
    \includegraphics[width=\textwidth]{fig/teaser.pdf}
    \caption{Data Uncertainty in Thermal Geo-localization (TG): Our approach captures six categories of high data-uncertainty samples leading to TG failure, where \textcolor{red}{predicted} displacements significantly deviate from the \textcolor{green}{ground truth}. Thermal images are overlaid on predicted displacements on the satellite imagery. High-resolution images are available on our project page.}
    \label{teaser}
    \vspace{-15pt}
\end{figure*}

In this study, we introduce \textit{Uncertainty-Aware Satellite-Thermal Homography Network (UASTHN)}, a sample and consensus-based Uncertainty Estimation (UE) framework for Deep Homography Estimation (DHE) in satellite-thermal geo-localization.Our main contributions are as follows:
\begin{itemize}
\item We introduce a CropTTA strategy with a unique homography consensus mechanism for effective data uncertainty measurement, seamlessly integrating with any DHE approach. Our study also comprehensively assesses model uncertainty using Deep Ensembles (DE).
\item We show how the proposed method is the first solution to address the challenge of uncertainty estimation for localization using cross-domain data, achieving superior homography estimation and geo-localization performance compared to baselines.
\item Extensive experiments validate our approach's effectiveness and efficiency on challenging satellite-thermal datasets. Specifically, our method achieves a geo-localization error of $7~\si{m}$ with a $97\%$ success rate for uncertainty estimation within a $512~\si{m}$ search radius.
\end{itemize}







