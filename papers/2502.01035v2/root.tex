%%%%%%%%%%%%%%%%%%%%%%%%%%%%%%%%%%%%%%%%%%%%%%%%%%%%%%%%%%%%%%%%%%%%%%%%%%%%%%%%
%2345678901234567890123456789012345678901234567890123456789012345678901234567890
%        1         2         3         4         5         6         7         8

\documentclass[letterpaper, 10 pt, conference]{ieeeconf}  % Comment this line out if you need a4paper

%\documentclass[a4paper, 10pt, conference]{ieeeconf}      % Use this line for a4 paper

\IEEEoverridecommandlockouts                              % This command is only needed if 
                                                          % you want to use the \thanks command

\overrideIEEEmargins                                      % Needed to meet printer requirements.

%In case you encounter the following error:
%Error 1010 The PDF file may be corrupt (unable to open PDF file) OR
%Error 1000 An error occurred while parsing a contents stream. Unable to analyze the PDF file.
%This is a known problem with pdfLaTeX conversion filter. The file cannot be opened with acrobat reader
%Please use one of the alternatives below to circumvent this error by uncommenting one or the other
%\pdfobjcompresslevel=0
%\pdfminorversion=4

% See the \addtolength command later in the file to balance the column lengths
% on the last page of the document

% The following packages can be found on http:\\www.ctan.org
%\usepackage{graphics} % for pdf, bitmapped graphics files
%\usepackage{epsfig} % for postscript graphics files
%\usepackage{mathptmx} % assumes new font selection scheme installed
%\usepackage{times} % assumes new font selection scheme installed
%\usepackage{amsmath} % assumes amsmath package installed
%\usepackage{amssymb}  % assumes amsmath package installed

\usepackage[english]{babel}
\usepackage{amsmath}
\usepackage{amssymb}
\usepackage{booktabs}
\usepackage[dvips]{graphicx}
\usepackage{multirow}
\usepackage{amsfonts}
\usepackage{enumerate}
\usepackage{tabularx}
\usepackage{algorithm,algorithmic}
\usepackage{bm}
\usepackage{enumerate}
\usepackage{adjustbox}
\usepackage{xcolor}
\usepackage{siunitx}
\usepackage{booktabs}
\usepackage{mdframed}
\usepackage{adjustbox}
\usepackage{authblk}
\usepackage{color}
\usepackage{xurl}
\usepackage{subcaption}
\urlstyle{rm}
\usepackage{cite}
\makeatletter
\let\NAT@parse\undefined
\makeatother
\usepackage{hyperref}
\usepackage{relsize}
\usepackage{float}
\usepackage{pifont}

\newcommand{\cmark}{\ding{51}}%
\newcommand{\xmark}{\ding{55}}%
\usepackage{wrapfig,lipsum}

\title{\LARGE \bf  UASTHN: Uncertainty-Aware Deep Homography Estimation for UAV Satellite-Thermal Geo-localization}

% The \author macro works with any number of authors. There are two
% commands used to separate the names and addresses of multiple
% authors: \And and \AND.
%
% Using \And between authors leaves it to LaTeX to determine where to
% break the lines. Using \AND forces a line break at that point. So,
% if LaTeX puts 3 of 4 authors names on the first line, and the last
% on the second line, try using \AND instead of \And before the third
% author name.

% NOTE: authors will be visible only in the camera-ready and preprint versions (i.e., when using the option 'final' or 'preprint'). 
% 	For the initial submission the authors will be anonymized.

\author{Jiuhong Xiao and Giuseppe Loianno
\thanks{The authors are with New York University, Tandon School of Engineering, Brooklyn, NY 11201, USA. {\tt\footnotesize email: \{jx1190, loiannog\}@nyu.edu}.}
\thanks{This work was supported by the Technology Innovation Institute, the NSF CAREER Award 2145277, the NSF CPS Grant CNS-2121391, and the NYU IT High Performance Computing resources, services, and staff expertise. Giuseppe Loianno serves as consultant for the Technology Innovation Institute. This arrangement has been reviewed and approved by the New York University in accordance with its policy on objectivity in research.}
}

\begin{document}
\maketitle
\thispagestyle{empty}
\pagestyle{empty}

%===============================================================================
\begin{abstract}
Geo-localization is an essential component of Unmanned Aerial Vehicle (UAV) navigation systems to ensure precise absolute self-localization in outdoor environments. To address the challenges of GPS signal interruptions or low illumination, Thermal Geo-localization (TG) employs aerial thermal imagery to align with reference satellite maps to accurately determine the UAV's location. However, existing TG methods lack uncertainty measurement in their outputs, compromising system robustness in the presence of textureless or corrupted thermal images, self-similar or outdated satellite maps, geometric noises, or thermal images exceeding satellite maps. To overcome these limitations, this paper presents \textit{UASTHN}, a novel approach for Uncertainty Estimation (UE) in Deep Homography Estimation (DHE) tasks for TG applications. Specifically, we introduce a novel Crop-based Test-Time Augmentation (CropTTA) strategy, which leverages the homography consensus of cropped image views to effectively measure data uncertainty. This approach is complemented by Deep Ensembles (DE) employed for model uncertainty, offering comparable performance with improved efficiency and seamless integration with any DHE model. Extensive experiments across multiple DHE models demonstrate the effectiveness and efficiency of CropTTA in TG applications. Analysis of detected failure cases underscores the improved reliability of CropTTA under challenging conditions. Finally, we demonstrate the capability of combining CropTTA and DE for a comprehensive assessment of both data and model uncertainty. Our research provides profound insights into the broader intersection of localization and uncertainty estimation. The code and models are publicly available.
\end{abstract}

% Two or three meaningful keywords should be added here
% \keywords{Uncertainty Estimation, Thermal Imagery, Deep Homography Estimation, Visual Geo-localization} 
\section*{Supplementary Material}
{ \noindent \textbf{Project page:}  \url{https://xjh19971.github.io/UASTHN/}}
\section{Introduction}~\label{sec:introduction}
\section{Introduction}\label{sec:intro}

In computational finance, Monte Carlo simulations are used extensively to estimate the expected value of financial payoffs based on the solution of stochastic differential equations (SDEs) which model the evolution of stock prices, interest rates, exchange rates and other quantities \cite{glasserman04}.  Monte Carlo methods are very general and flexible, but for high accuracy it requires generating a large number of costly SDE path approximations, which has motivated research into a number of variance reduction or, equivalently, cost reduction techniques. One such method is
Multilevel Monte Carlo (MLMC), which was proposed in \cite{GILES2008} and was adapted for various applications that are summarised in \cite{Giles_overview17} and successfully combined with other methods such as quasi-Monte Carlo methods. The main idea of MLMC is to approximate the payoff using different time stepping resolutions when numerically solving the underlying SDE and to generate an optimal number of samples on each level, such that the overall computational cost is minimised subject to the desired bound on the variance. %, such that the total computational cost is minimised. 
The computational savings come from the fact that most samples are computed on the coarser levels and hence are less expensive while only a few samples from the finest levels are required \cite{GILES2008}.


Among the directions in which the computational cost 
of MLMC methods could further be reduced, an important avenue is the use of lower precision calculations, especially for the first Monte Carlo levels where the targeted accuracy is relatively low. 
 An overview of the research on mixed precision for the standard Monte Carlo (MC) framework is provided in \cite{ChowMixedPrecisionStandardMC} but only a few references study the potential of low precision computation in the MLMC framework \cite{Rounding_error_oliver}. To the best of our knowledge, the only MLMC framework with customised precision in the literature is \cite{brugger2014mixed}, but they use a uniform precision for all operations on each Monte Carlo level instead of optimising 
 the precision of each intermediary variable to reduce as much as possible the cost of path generation.
 
An important motivation for an MLMC framework with variable precision would be performing the low precision computations on reconfigurable hardware devices such as Field Programmable Gate Arrays (FPGAs). FPGAs contain customizable logic blocks and connectors that make it easy to adapt the digital circuit architecture for a specific application, leading to a highly parallel and optimised implementation. Therefore they are successfully exploited in applications that require high speed and have high computational workload, such as signal processing \cite{woods2008fpga}, and real time applications like high frequency trading \cite{HFT1,HFT2}. That is why a number of previous works in hardware architecture design implemented the MLMC algorithm to price financial options using FPGAs as accelerators, which resulted in improved speed and power efficiency compared to full CPU architectures \cite{Schryver2013AMM}. The paper \cite{lindsey2016domain} also proposed 
a Domain Specific Language to automate the configuration of FPGAs for this specific application. However, only \cite{brugger2014mixed} proposed a heuristic to reduce the precision in calculations.

In addition, all aforementioned works considered that the random number generation (RNG) is performed in single or double precision. Yet in most cases an important portion of the workload in the overall MLMC simulation comes from the RNG and in \cite{brugger2014mixed} this limited the total computational savings.
To reduce the cost of MLMC simulations in particular those based on the Geometric Brownian Motion (GBM), \cite{approximateICDF_Oliver, NestedOliver} have proposed to use approximate random numbers that are generated by applying an approximation of the inverse CDF to uniform random numbers. In \cite{NestedOliver}, the authors proposed a way to integrate these lower precision random variables into a \textit{nested} MLMC framework and completed a numerical analysis to bound the resulting error at each MC level by a product of the time step and the error in the random number approximation. The same authors show in \cite{approximateICDF_Oliver} that using approximate random variables reduces the cost of path generation by a factor 7.


In this paper we propose a nested MLMC framework that combines the use of approximate random normal variables and lower precision calculations to reduce the computational cost of MLMC even further than \cite{brugger2014mixed,NestedOliver}. We illustrate the efficiency of our framework in Matlab, after making several assumptions on the cost of operations and size of the errors that we carefully justify. We focus on the case of GBM and use the approximate RNG methods presented in \cite{approximateICDF_Oliver} as well as a new slightly modified method that combines CDF inversion and the central limit theorem. To choose the precision of the variables in the low precision path generation, we introduce a novel method to optimise the bit-widths. This optimisation is performed before the main path generation loop is executed and is based on a linear model of the payoff error  
due to rounding when computing in low precision. The error model relies on algorithmic differentiation in a similar manner to \cite{unifying-bwoptim,bitwidth-AD,ADAPT}. The bit-width optimisation procedure can be performed off-line, so this stage can be excluded from the on-line time complexity of our framework. The user specified desired accuracy is then enforced by calculating on-line the number of samples that need to be generated.

In terms of hardware design, we suggest implementing the low precision path generation on FPGAs and the full-precision ones on a CPU or GPU. 
The FPGA offers enough flexibility to define a separate bit-width for every variable in the low precision path generation, and can be reconfigured periodically to update the bit-widths when the market parameters have changed considerably. 


The paper is organized as follows : \Cref{sec:MLMC} introduces MLMC and nested MLMC to make clear the estimator that is implemented in our framework. Then in \Cref{sec:RNG} we detail the methods that could be used to obtain approximate random normally distributed numbers very cheaply for the low precision path generation. In \Cref{sec:error_model} and \Cref{sec:costModel} we propose an error model and a cost model (resp.) that we then use to formulate the optimisation problem that is solved to obtain the optimal bit-widths of fixed point variables in \Cref{sec:optimisation}. Finally we summarise our results and future directions in \Cref{sec:conclusion}.



\section{Related Works}~\label{sec:relatedworks}
\section{Related Work}

\subsection{First-order logic for natural entailment}

Since the start of the RTE challenge \citep{rte}, multiple works have attempted using FOL representations to solve natural language entailment. These methods first obtain the syntactic/semantic parse tree and apply a rule-based transformation to get the FOL representation \citep{bos-markert-2005-recognising, bos-nli}. However, it was repeatedly shown that these FOL representations are not empirically effective in solving natural language entailment. For instance, \citet{bos-nli} reported that FOL representations translated from the discourse representation structure (DRS) yield only 1.9\% recall in detecting the entailment in the single-premise RTE benchmark \citep{rte}.

Independently from these works, multi-premise logical entailment benchmarks \citep{tafjord-etal-2021-proofwriter, logicnli, folio} were developed to evaluate the reasoning ability of generative models. These benchmarks adopt the classic 3-way entailment label classification format (\textit{entailment, contradiction, neutral}) of single-premise RTE tasks, in which both the NL sentences and their gold FOL representations point to the same entailment label. 

Recent works have applied LLMs to obtain FOL representations for these multi-premise logical entailment tasks \citep{logiclm, linc, divide-and-translate}, fueled by the code generation ability of LLMs. While they achieve significant performance in synthetic, controlled logical reasoning benchmarks, whether they can generalize to natural entailment has remained unanswered. Furthermore, \citet{linc} observed that LLMs are highly susceptible to \textit{arbitrariness}, as they fail to produce coherent predicate names or numbers of arguments even when generating FOL representations of premises and hypotheses in a single inference.

\subsection{Executable semantic representations}

Apart from FOL, a stream of research focuses on the \textit{executability} of semantic representations. From this perspective, semantic representations are \textit{program codes} that can be executed to solve downstream tasks, such as query intent analysis \citep{spider, dligach-etal-2022-exploring} and question answering \citep{semparse-qa}. The performance of the semantic parser is directly assessed by the accuracy of execution results for the downstream tasks, rather than the similarity between the prediction and the reference parse.

To improve the execution accuracy that is often non-differentiable, reinforcement learning (RL) and its variants have been applied to train neural semantic parsers \citep{cheng-etal-2019-learning, cheng-lapata-2018-weakly}. Using only the input sentence and the desired execution result, these methods learn to maximize the probability of the representations that lead to the correct execution result. However, these approaches are not directly applicable to EPF, as EPF requires taking account of \textit{interactions between premises and hypotheses} during execution (\textit{i.e.} theorem proving) while these methods assume that sentences are isolated.


\section{Methodology}~\label{sec:methodology}
% introduce PDDL domains
% why Gripper env as testing context
% motivation: comparing classical vs LLM planners
% - classical: PDDL solver fast-downward
% - LLM: gpt-4o
% explanation and refinement are two distinguishing features of LLM planners
% - how we demonstrate explanation and refinement in the study
We evaluate user trust in two planners over a set of planning problems and study the potential factors influencing user trust in the planners. In particular, we compare a language-model-based planner, denoted as an \emph{LLM Planner}, with a traditional graph-search-based planner, denoted as a \emph{PDDL Solver}. The PDDL Solver uses Fast Downwards \cite{fastdownward} as its underlying model, processing planning problems described in PDDL to generate an optimal solution. In comparison, the LLM Planner employs GPT-4o to interpret the planning problem and extract a solution generated by the language model. Unlike the PDDL Solver, the LLM Planner can reason through the planning problem, explain its proposed solution, and iteratively refine the solution based on external feedback. This study investigates how the correctness of solutions, the quality of explanations, and the refinement process influence user trust.

\subsection{Planning Problem}
% \begin{wrapfigure}{r}{0.4\textwidth}
% % \begin{figure}[t]
%     \centering
%     \includegraphics[width=\linewidth]{figures/problem-example.pdf}
%     \caption{A running example of a planning problem in our study.}
%     \Description{Planning Problem Example}
%     \label{fig: problem-example}
% % \end{figure}
% \end{wrapfigure}

We describe each planning problem in the \emph{Planning Domain Definition Language (PDDL)} and propose two planners to generate plans that solve the problem. We select the \emph{gripper} planning problems from the International Planning Competition \cite{IPC} for plan generation and evaluation. In a gripper planning problem, a robot moves balls between a set of rooms using two grippers. The objective is to create a plan for the robot to move the balls to the target rooms we defined. We present a few running examples of the gripper problem in Figure \ref{fig: correctness}.

A planning problem consists of a \emph{planning domain} and a \emph{problem description}, expressed in PDDL. 

\paragraph{Planning Domain}
A planning domain refers to the universal aspects of a problem that remains consistent across different instances of the problem. In particular, it defines the types of objects, predicates, and actions that exist in the planning problem. We present an example of the gripper problem in Appendix \ref{app: grippers}.

\paragraph{Problem Description} A problem description specifies the particular instance of a planning task within a given domain. It includes the planning domain to which it pertains, a set of objects, the initial state of these objects, and the goal state to be achieved.

\paragraph{Plan}
A plan is a sequence of actions with specific input parameters. Recall that an action corresponds to a state transition. If a plan (a sequence of actions) transits from the initial state to the goal state defined by a problem, then we consider the plan to be \emph{correct}. If a plan does not transit to the goal state or there exists an action violating its precondition, then the plan is \emph{wrong}.

\begin{figure}[t]
    \centering
    \includegraphics[width=0.8\linewidth]{figures/correct.jpeg}
    \caption{Examples where LLM Planner correctly generates a plan for the gripper planning problem.}
    \Description{Planning Problem Correctness}
    \label{fig: correct}
\end{figure}

\subsection{PDDL Solver}
The PDDL Solver takes the planning domain and the problem description as inputs and then generates a plan described in PDDL. 
% It generates a plan in the following format:
% \vspace{4pt}
% \begin{lstlisting}[language=completion]
% (move robot1 room1 room3)
% (pick robot1 ball2 room3 rgripper1)
% (move robot1 room3 room2) ......
% \end{lstlisting}
Next, we convert the generated plan into natural language for user studies following the procedure in \cite{seipp-et-al-zenodo2022} and display it to users. We present an example in Figure \ref{fig: correct}.

The PDDL Solver applies a graph search algorithm to find a path (i.e., a list of transitions) from the initial state to the goal state. It either generates a \emph{correct} plan---defined as the shortest path between the initial and goal states---or returns a signal indicating that no solution exists for the given problem.

\subsection{LLM Planner}

The LLM Planner addresses planning problems by querying a large language model. In particular, it transmits the planning domain and problem description to the language model using a structured prompt format. The planner then retrieves a natural language plan from the language model. We use GPT-4o as the language model for the planner. To ensure the output adheres to the desired format, we include a few in-context examples within the prompts.

A language model solves a planning problem by interpreting the domain and problem descriptions, simulating state transitions, and generating a sequence of actions to achieve the goal. While effective for reasoning and plan generation, language models may struggle with large state spaces. Unlike the PDDL Solver, the LLM Planner may generate \emph{incorrect} plans that violate the problem specifications (e.g., preconditions of actions) or fail to achieve the goal.

\subsection{Explanation and Refinement}
Alongside the generated plans, we offer detailed explanations of all the plans and revisions of any incorrect plans. This study examines how these explanations and refinements influence human trust in the two planners.

\paragraph{LLM Planner with Explanation (LLM+Expl)}
For each generated plan, we manually provide a natural language explanation. This explanation includes an assessment of the plan’s correctness, identification of any violations of action preconditions, and an analysis of inconsistencies between the final state achieved and the intended goal state. We present examples of explanations in Figure \ref{fig: explain} in Appendix.

In particular, if a plan is correct, the explanation is simply ``the plan successfully satisfies the goal conditions.'' 
If a plan is incorrect, we identify the underlying cause as either a violation of action preconditions or a failure to achieve the goal state. In cases involving precondition violations, we specify the action responsible for the issue. For example, consider the action ``robot moves from room 1 to room 2,'' but the robot is initially located in room 3. This scenario constitutes a violation of the precondition for the ``move'' action. In the latter case, we describe the differences between the final state achieved and the intended goal state, e.g., ``fail to move ball 2 to room 2.''

% \begin{wrapfigure}{r}{0.5\textwidth}
%     \centering
%     \includegraphics[width=0.98\linewidth]{figures/refine.jpeg}
%     \includegraphics[width=0.98\linewidth]{figures/refine-correct.jpeg}
%     \includegraphics[width=0.98\linewidth]{figures/refine-wrong.jpeg}
%     \caption{Plan refinement by the LLM Planner. The top row presents two choices of plan refinement (where the refinement starts). The second and third row shows the refinement outcomes of the two choices, where the second row shows a correctly refined plan and the third row shows an incorrect plan.}
%     \Description{Refinement}
%     \label{fig: refine}
% \end{wrapfigure}

\paragraph{LLM Planner with Refinement (LLM+Refine)}
Note that a plan generated by the LLM Planner could be incorrect. Therefore, we offer a prompting mechanism for the LLM Planner to refine the generated plan according to the user feedback. The mechanism works as follows:

1. Request the user to indicate the step number of the first action in the plan that is incorrect, such as the step where an action’s precondition is violated. We present a sample user interface on the left of Figure \ref{fig: refine} in Appendix.

2. Send the planning domain, problem description, and the original plan to the language model. Then, query the model to rewrite the subsequent steps starting from the user-specified step number. We present a sample input prompt in Figure \ref{fig: refine-prompt} in the Appendix.

3. Replace the original plan with the newly refined plan and display it to the user.

This mechanism allows users to interact with the language model to refine the plan. It enables the language model to focus on a subset of steps, facilitating a deeper interpretation of the incorrect component. However, the correctness of the refined plan is not guaranteed. Figure \ref{fig: refine} in the Appendix shows an example of a correctly refined plan and an incorrectly refined plan.

\section{Experimental Setup}~\label{sec:experiment_setting}
\section{Experimental settings}
\subsection{Datasets}
\label{sec:datasets}

\begin{table}[h] \footnotesize  \centering\resizebox{0.48\textwidth}{!}{\begin{tabular}{c|l|c|c|c}
\toprule
\textbf{Task} & \textbf{Dataset} & \textbf{N-shot} & \multirowcell{\textbf{Train texts} \\ \textbf{for STMD}} & \multirowcell{\textbf{Evaluation} \\ \textbf{texts}} \\
\midrule
\multirow{3}{*}{\multirowcell{Text \\ Summarization}} & CNN/DailyMail & 0 & 2,000 & 2,000 \\
& XSum & 0 & 2,000 & 2,000 \\
& SamSum & 0 & 2,000 & 819 \\
\midrule
\multirow{4}{*}{\multirowcell{QA \\ Long answer}} & PubMedQA & 0 & 2,000 & 2,000 \\
& MedQUAD & 5 & 2,000 & 2,000 \\
& TruthfulQA & 5 & 408 & 409 \\
& GSM8k & 5 & 2,000 & 1,319 \\
\midrule
\multirow{4}{*}{\multirowcell{QA \\ Short answer}} & SciQ & 0 & 5,000 & 1,000 \\
& CoQA & \multirowcell{all preceding \\ questions} & 5,000 & 2,000 \\
& TriviaQA & 5 & 5,000 & 2,000 \\
\midrule
\multirow{1}{*}{\multirowcell{MCQA}} & MMLU & 5 & 5,000 & 2,000 \\
\bottomrule
\end{tabular}
}\caption{\label{tab:dataset_stat} The statistics of the datasets used for evaluation.}
\end{table}

Three representative multi-premise RTE datasets, namely EntailmentBank \citep{entailmentbank}, eQASC \citep{eqasc}, and e-SNLI \citep{esnli}, are used for the experiments. The statistics of each data set are briefly introduced in Table \ref{tab:datasets}.

\textbf{EntailmentBank} provides \textit{entailment trees} with simple scientific facts as nodes. We decompose the trees into depth 1 subtrees which encode a single premises-hypothesis pair.

\textbf{eQASC} provides 2-hop explanations for a given hypothesis derived from QASC \citep{qasc}, a multiple-choice question dataset from the science domain.

\textbf{e-SNLI} extends the single-premise SNLI dataset \citep{snli} by adding \textit{explanations} to the original premise-hypothesis pairs. This can be viewed as the premise and explanation together entailing the hypothesis. Due to limited computation resources, we sample 100k premises-hypothesis pairs from the train set and use the original validation/test set without modification.

\subsection{\nltofol\ translator}

We use a sequence-to-sequence model, T5-base \citep{t5} with 220M parameters, as our \nltofol\ translator backbone. To obtain an initial model $S_0$, we take 34k pairs of natural language sentences and their LLM-generated FOL representation from MALLS \citep{malls},
% \heng{briefly explain how it works} \jinu{done!}
convert them into the NLTK format \citep{nltk} with a rule-based translator, and train the T5-base model with standard cross-entropy loss. We refer to this model ($S_0$) as \texttt{T5-Iter0}, and models obtained after the $N$-th iteration as \texttt{T5-iter}$N$. Note that the MALLS dataset is not intended to preserve the entailment relationship between sentences, which can be seen in the low EPR score of the initial model \texttt{T5-Iter0} (Table \ref{tab:main-results}).

\subsection{FOL Theorem prover}

For the automatic theorem prover that is used to check entailment, we use Vampire \citep{vampire}, one of the fastest provers currently available. As the generative models are trained on NLTK syntax, the model outputs are translated into Vampire-compatible format (TPTP \citep{tptp}) using a rule-based translator.

\begin{figure*}[t]
    \centering
    \includegraphics[width=\linewidth]{figures/main_results_merged.pdf}
    \caption{EPR (left), EPR@16 (right-solid), EPR@16-Oracle (right-dotted) per iteration. The continuous growth in all EPR metrics implies that the model extrapolated to unseen premises-conclusion pairs where BRIO loss is 0, demonstrating the strength of the proposed method.}
    \label{fig:main-results}
\end{figure*}

\subsection{Baselines}

As a baseline, we adopt a wide variety of methods that translate natural language to first-order logic.

First, a suite of classic meaning representation-based methods was included as a baseline. \texttt{CCG2Lambda} \citep{ccg2lambda} obtains Combinatory Categorical Grammar (CCG) parse trees using C\&C Parser \citep{cncparser} and converts them to FOL representations via Lambda calculus. \citet{amr2folbos} and \citet{amr2fol} both convert Abstract Meaining Representations (AMR) to FOL. The AMR graph was obtained using AMRBART \citep{amrparser} and translated to FOL representations using the respective implementations of rule-based translators. These methods are only evaluated by EPR and not by EPR@K(-Oracle) as they are designed to produce a single gold FOL parse for each sentence in a deterministic manner. Any errors that occurred during obtaining the FOL representation were removed for the evaluation.

Few-shot and fine-tuned LLMs were also evaluated as a baseline. First, we sample $K=16$ FOL representations using GPT-4o and GPT-4o-mini \citep{gpt4o} with 5 in-context examples temperature 1.0 (prompts shown in Appendix \ref{sec:appendix-prompt}). Also, we evaluate LogicLLaMA \citep{malls}, a LLaMA-7B \citep{llama1} checkpoint directly fine-tuned on the MALLS dataset. $K=16$ FOL representations per each sentence were sampled using temperature 0.1, following the original paper.

\section{Results}~\label{sec:resutls}
% \section{Simulation Evaluation \& Results}\label{sec:results}

\subsection{Baseline Planners}

To evaluate the performance of \PlannerName, we compare it against several baseline methods. In the following section, we describe these baselines, their implementation details, and their respective advantages and limitations, particularly in the context of information gathering in large, high-dimensional search spaces. The simulation framework and vehicle parameters remain consistent across all planners, and each method is allowed to replan during testing.

\subsubsection{Monte-Carlo Tree Search}

Monte Carlo Tree Search (MCTS) can be a powerful technique for finding feasible and optimal paths in complex environments. It is a heuristic search algorithm that builds a search tree incrementally through repeated simulations. At each iteration, it selects a node to explore based on a selection policy (often the Upper Confidence Bound or UCB1 algorithm), expands the tree by adding possible actions from that node, runs a simulation from the newly added node, and updates the statistics of nodes along the path traversed during the simulation. 

The UCB1 (Upper Confidence Bound) algorithm is a technique commonly used in the context of multi-armed bandit problems and Monte Carlo Tree Search (MCTS) for balancing exploration and exploitation. It helps in selecting actions or nodes that are likely to yield high rewards while also exploring less-frequented options to gather more information about their potential rewards. 

We formulate our UCB score in the following manner, \\
\begin{equation*}
    UCB_\text{node} = \frac{I(X_{\text{node}})}{\alpha} + C \times \sqrt{\frac{\ln(N_\text{tree})}{N_\text{node}}}
\end{equation*}
%  $
% UCB_\text{node} = \frac{\overline{X_\text{node}}}{\alpha} + C \times \sqrt{\frac{\ln(N_\text{tree})}{N_\text{node}}}
% $ \\
Here $I(X_{\text{node}})$ denotes the estimated information gain from the node, $\alpha$ denotes the normalization factor which is given by $\frac{B}{v_\text{desired}}$, $B$ being the maximum planning budget and $v_\text{desired}$ being the desired speed of our UAV. $C$ denotes the exploration weight, and $N_\text{tree}$ denotes the number of visits to the tree root node while $N_\text{node}$ denotes the number of times the present node has been visited.

After selecting a candidate node, if it has been visited before, it is expanded by applying motion primitives to generate child nodes, growing the tree. Unvisited nodes skip this step. Following expansion, either the unvisited candidate node or one of its children is selected for the simulation phase, where the future values of nodes along the path are estimated to update the total potential information gain. This informs the selection policy in subsequent iterations. Once planning time is exhausted, the path with the highest information gain is returned.

% with authors goes here
\begin{figure}[t]
\centering
\includegraphics[trim={.7cm 0cm .5cm 1.4cm},clip,width=\columnwidth]{figs/5_/Results1v3.pdf}
\caption{The Monte Carlo simulation results for the planners. The plots show the average percent reduction in entropy over the course of the simulations, and the shading shows the 95\% confidence intervals. IA-TIGRIS outperforms all of the baselines.}
\label{fig:mc_results}
\end{figure}

While MCTS is probabilistically guaranteed to converge to the optimal path \cite{mcts_ref_1}, it is constrained to actions within a predefined set of motion primitives. Its reliance on random sampling to estimate the future value of nodes can result in poor approximations, particularly in environments with sparse, localized pockets of high information gain. This limitation is especially pronounced in large search areas or scenarios with large budgets constraints, where estimating future node values becomes increasingly expensive. As a result, in such scenarios, MCTS is often implemented with a finite planning horizon, which can restrict its ability to account for long-term consequences or dependencies in the environment.

% This property of MCTS, which causes unguided exploration of the environment, leads to increased convergence times on the optimal path, as a result of a lot of budget being spent in exploring information sparse areas of the map. 
% Also, the computation time of MCTS increases exponentially with the depth of the search tree. The time complexity of MCTS is given by $\mathcal{O}(\frac{T}{t_\text{iter}} \cdot |A|^d)$. Here, $T$ is the total planning time and $t_\text{iter}$ is the time taken per iteration of the planning loop. $|A|$ is the number of actions and $d$ represents the average depth of the search tree. 

% The above limitations are not inconsequential in the context of performing informative path planning in large high-dimensional search spaces. We compare MCTS with \PlannerName, in \ref{}, and empirically demonstrate its drawbacks and how \PlannerName, is able to outperform MCTS in the context of the mission parameters we examine in this work.  

\subsubsection{Greedy}

For the greedy planner, we iterated through each cell within the search bounds and calculated the reward for a given cell $i$ as $g_i = R(X_i) / d_i$ where $R(X_i)$ is given through \eqref{equ:reward} and $d_i$ represents the Euclidean distance between the current position the robot at the current time $t$ and the closest viewpoint to the cell. To compute this viewpoint, the yaw between the current pose of the robot and the intersected cell is first calculated. Using the robot's sensor configuration and this yaw, $x$ and $y$ coordinates are calculated that view the cell at the desired flight altitude. With this formulation, the planner prioritizes regions with a high ratio of entropy to distance. This can lead to locally optimal choices that contradict with paths that lead to higher information gain over the entire trajectory. 

% without authors goes here
% \begin{figure}[t]
% \centering
% \includegraphics[trim={.7cm 0cm .5cm 1.4cm},clip,width=\columnwidth]{figs/5_/Results1v3.pdf}
% \caption{The Monte Carlo simulation results for the planners. The plots show the average percent reduction in entropy over the course of the simulations, and the shading shows the 95\% confidence intervals. IA-TIGRIS outperforms all of the baselines.}
% \label{fig:mc_results}
% \end{figure}


\begin{figure*}[t]
    \centering
    \begin{subfigure}[b]{0.99\textwidth}
        \centering
        \includegraphics[trim={0cm 0.3cm 0cm 0cm},clip,width=\textwidth]{figs/5_/Fig2v1_target.png}
        % \caption{Slice by targets}
        % \vspace{.1cm}
    \end{subfigure}
    
    \begin{subfigure}[b]{0.99\textwidth}
        \centering
        \includegraphics[trim={0cm 0cm 0cm 0cm},clip,width=\textwidth]{figs/5_/Fig2v1_sigma.png}
        % \caption{Slice by sigma }
    \end{subfigure}
    \caption{A comparison of the methods based on the number of sampled prior clusters and the standard deviation of sampled prior clusters. IA-TIGRIS is most effective compared to the baselines when there is high variation in the search space. As the search space prior information becomes more evenly spread out, the performance gap between the methods tends to decrease.}
    \label{fig:targets_sigmas}
\end{figure*}

\subsubsection{Random}

The random planner operates by iteratively sampling points within the defined search bounds and calculating the minimum-cost path to observe each sampled point. This process is repeated until the available budget is fully expended. The random planner does not utilize any prior information about the environment or target distribution. Additionally, it does not optimize the sequence of actions, instead treating each sampled point independently without considering the global structure of the search problem. This simplicity allows the random planner to highlight the performance benefits of more sophisticated methods by providing a lower-bound comparison for evaluation.

\subsubsection{Coverage}

The coverage planner generates a plan that systematically covers the entire search space using a straightforward lawn-mower pattern. The spacing between each pass is set to match the width of the projected observation footprint at 20\% from the bottom, ensuring that no grid cells are missed. This spacing also maintains a distance that enables high-quality sensor measurements. However, due to the size of the search spaces considered, the coverage planner spends significant time surveying empty regions. This approach results in inefficient use of the budget, as it prioritizes full coverage with safe sensor overlap, even in areas with little or no valuable information. While simple and robust, this method highlights the tradeoff between exhaustive coverage and efficient, targeted exploration.

% \subsubsection{Branch and Bound}
% The branch and bound baseline is based on motion primitive planning. In each future step the drone has a set of motion primitives with future states and each of these future states also has a set of motion primitives. In this way, a tree can be built with multiple path candidates. The path candidate with the highest information gain will be selected and form the output. 

% By adding branch and bound, there will be an estimation of a node's upper bound information reward, using the node's current information reward, updated information map and the remaining budget. If this upper bound is already lower than the information reward of any other node in the tree, the corresponding node will be closed and not expanded in the future to accelerate the expansion of the tree. 



\subsection{Tests and Analysis}
% To evaluate the efficacy of IA-TIGRIS compared to the baseline methods, we conduct Monte Carlo testing as well as analyze how the prior and budget affect the performance of each method. In all of these test cases, there are no time-based or priority rewards and have horizon lengths set to the full budget. All tests were performed using an Intel Xeon CPU E5-2620 v4 @ 2.10GHz.
To evaluate the efficacy of IA-TIGRIS against baseline methods, we perform Monte Carlo testing and analyze the impact of the prior and budget on the performance of each method. In all test cases, rewards are calculated using \eqref{equ:reward}, and horizon lengths are set to match the full budget. The tests are conducted on an Intel Xeon CPU E5-2620 v4 @ 2.10GHz, ensuring consistent computational conditions across all evaluations.

% Random sample across which parameters.

% Quantitative ideas. Look into number and std of prior (metric for this? std of grid cell values, mediuan, mean,). 
% Uniform prior? 
% Split distinct regions, not smooth. 
% Compare to coverage and amount of time to reach specific amount. 
% Compare with different budgets. 
% Repeatability test. 
% Graph size vs time. 
% Look at coverage with different altitudes or widths. Something that shows long horizon vs not nature of things?
% Shape of search space?
% Time/budget to get x\% of all info gain. Have to do moving horizon. 
% Targets detected? 

% Key thought for results where I show time, our optimization does not optimize for time, only final value. Key thing to show across the different budgets. 

% \BM{Qualitative. Nayana idea of plot with example sampled case. Should add one here.} 



\subsubsection{Monte Carlo Testing}
Our simulated testing environment is a $5000\times5000$ m square with Gaussian-distributed prior information randomly placed throughout the search space. The number of prior clusters was sampled uniformly between $[4,20]$, with standard deviations between $[60,450]$, and maximum value between $[0.05,0.5]$. 

The results of $100$ Monte Carlo tests are shown in Fig.~\ref{fig:mc_results}. IA-TIGRIS clearly outperforms the other methods, achieving nearly a $40\%$ greater reduction in entropy than the next best method. Early in the simulation, the greedy method initially gains information more quickly, as expected, but this does not translate to better long-term performance. Since our method optimizes for total information gain, it generates paths that maximize information collection over the entire budget. MCTS performed slightly worse than the greedy approach.

The random paths slightly outperformed the coverage paths. This is likely because the lawnmower strategy requires sufficient overlap between passes to avoid missing areas, and its long straight paths often lead to redundant observations due to the UAV’s forward-facing camera. Changing the heading of the UAV is beneficial to viewing more of the search space, which may explain why random paths performed better.

We also conducted Monte Carlo tests where either the number of prior clusters or their standard deviation was held constant to analyze how variations in the information map affect planner performance. The results, shown in Fig.~\ref{fig:targets_sigmas}, include two cases: the upper figure fixes the number of priors, while the lower figure fixes their standard deviation. All other agent and simulation parameters remained unchanged.


% The first thing to note from these results is that for all tests the proportional performance gap between IA-TIGRIS and the baselines increases as the number and standard deviation of the Gaussian priors decreases. As the search space becomes more uniformly filled with entropy in the information map, the need for longer-horizon planning decreases and other simple or random approaches can perform satisfactorily given the testing budget. As the information becomes more sparsely distribution in the space, such as when the information is contained in separated pockets of areas, there is a greater need to plan longer-horizon paths that reason about the given budget.
% \BM{Could have figures here or refer to others}

Across these tests, the performance gap between IA-TIGRIS and the baselines widens as the number and standard deviation of the Gaussian priors decrease. When entropy is more uniformly distributed across the search space, simpler methods perform reasonably well within the given budget. However, when information is concentrated in sparse, distinct regions, longer-horizon planning becomes essential. In such cases, IA-TIGRIS demonstrates a significant advantage by effectively reasoning about the budget and prioritizing high-value regions.

% Show plot of first plans expected info gain versus planning time. (plans not executed)


\subsubsection{Budget Analysis}
To evaluate the impact of budget constraints on performance, we conducted additional tests beyond our initial Monte Carlo experiments, evaluating budgets of $5000$ m, $10000$ m, $30000$ m, and $60000$ m. Table~\ref{tab:budgets} summarizes the average entropy reduction across these budgets.

\definecolor{tabfirst}{rgb}{1, 0.7, 0.7} % red
\definecolor{tabsecond}{rgb}{1, 0.85, 0.7} % orange
\definecolor{tabthird}{rgb}{1, 1, 0.7} % yellow
\begin{table}[t]
    \centering
    \resizebox{\linewidth}{!}{
    \begin{tabular}{l|ccccc}
    & $5000$ m & 10000 m  & 15000 m& 30000 m& 60000 m\\ \hline

    % \hline
    IA-TIGRIS  &  \cellcolor{tabfirst}$9.41\pm1.0$ &  \cellcolor{tabfirst}$18.28\pm1.8$ & \cellcolor{tabfirst}$25.36\pm2.3$ & \cellcolor{tabfirst}$41.08\pm2.9$ & \cellcolor{tabfirst}$58.85\pm2.9$ \\
    Greedy  &  \cellcolor{tabsecond}$6.99\pm0.8$ &  \cellcolor{tabsecond}$13.10\pm1.5$ & \cellcolor{tabsecond}$17.97\pm2.0$ & \cellcolor{tabthird}$30.00\pm2.3$ & \cellcolor{tabsecond}$49.38\pm3.5$ \\
    MCTS  &  \cellcolor{tabthird}$6.06\pm0.7$ &  \cellcolor{tabthird}$11.80\pm1.1$ & \cellcolor{tabthird}$17.11\pm1.4$ & \cellcolor{tabsecond}$30.21\pm2.2$ & \cellcolor{tabthird}$48.68\pm2.7$ \\
    Random  &  $2.19\pm0.3$ & $4.29\pm0.7$ & $6.61\pm0.6$ & $17.50\pm1.2$ & $22.47\pm1.4$ \\
    Coverage  &  $1.58\pm0.3$ &  $2.82\pm0.4$ & $4.09\pm0.7$ & $12.04\pm1.9$ & $16.77\pm2.4$ \\

    \end{tabular}
    }
    \caption{Monte Carlo testing results given different budgets. The values are the average percent reduction in entropy and the 95\% confidence bounds. \mbox{IA-TIGRIS} had the best performance for all budgets.}
    \label{tab:budgets}
\end{table}
%$\uparrow$ 

IA-TIGRIS consistently achieved the highest entropy reduction across all budget constraints, with a statistically significant margin over alternative methods. Greedy generally ranked second but was slightly outperformed by MCTS at the $30000$ m budget level. Greedy and MCTS exhibited comparable performance throughout the tests, with their results closely tracking each other. Consistent with our previous findings, Random and Coverage methods yielded the lowest results.


Among the tested methods, only IA-TIGRIS and MCTS explicitly incorporate budget constraints into their planning algorithms. Notably, at lower budgets ($5000$ m and $10000$ m), these methods achieved higher entropy reduction compared to the equivalent time steps ($200$ s and $400$ s) in the $15000$ m budget scenario shown in Fig.~\ref{fig:mc_results}. This improved performance stems from IA-TIGRIS's optimization of total path reward under budget constraints, contrasting with the myopic next-best-action approach of the greedy method. The remaining methods---Greedy, Random, and Coverage---maintain consistent behavior regardless of budget constraints, as their planning strategies do not account for resource limitations.


The performance gap between IA-TIGRIS and the next-best method varied with budget size, showing margins of $34.6\%$, $39.5\%$, $41.1\%$, $36.0\%$, and $19.2\%$ in ascending budget order. This gap widened through the first three budget levels as problem complexity increased, before declining significantly at higher budgets. This performance pattern suggests that implementing a planning horizon could enhance efficiency by limiting tree search depth, enabling the planner to prioritize path quality optimization over exhaustive space exploration.


% percent improved from next best
% 34.6, 39.5, 41.1, 36.0, 19.2
% reasons, too long horizon is a larger search space, so less quality paths closer. Or larger horizon, more packing in


% with authors goes here
\begin{figure}[t] 
    \centering
    \renewcommand\arraystretch{0} % Adjust the height between rows here
    \setlength{\tabcolsep}{1pt} % Adjust the column separation here
    \begin{tabular}{c}
        \begin{tikzpicture}
            \node[anchor=south west, inner sep=0] (image) at (0,0) {
                \includegraphics[width=0.9\linewidth]{figs/5_/google_earth_prior.png}
            };
            \begin{scope}[x={(image.south east)},y={(image.north west)}]
                % \fill[OrangeRed] (0.02, 0.03) circle (2pt); 
                % \fill[OrangeRed] (0.51, 0.04) circle (2pt); 
                % \fill[OrangeRed] (0.61, 0.04) arc (0:90:2pt); 
                \fill[Orange, opacity=0.8] (0.74, 0.45) circle (3pt); % Adjust 
                \fill[Orange, opacity=0.8] (0.27, 0.42) circle (3pt); % Adjust 
                \fill[Orange, opacity=0.8] (0.39, 0.63) circle (3pt); % Adjust 
            \end{scope}
        \end{tikzpicture} \\
        % \includegraphics[width=0.9\linewidth]{figs/5_/google_earth_prior.png} \\
        \\
        \includegraphics[width=0.9\linewidth]{figs/5_/google_earth_path.png} 
    \end{tabular}
    \caption{Google Earth screenshots illustrating the mission planning process and execution. Top: Areas of high entropy targeted for search are highlighted in red, representing regions with a binary occupied/unoccupied probability of 0.2. Three points of particular interest, each assigned a 0.5 probability, are marked in orange. Bottom: The executed drone flight path (yellow) shows the optimized path for maximum information gain across the search space.} 
    \label{fig:google_earth}
\end{figure}
\begin{figure}[t]
\centering
% https://docs.google.com/presentation/d/1RjI-QqHpBRLHN60UAxzmQYs4EaWaVCOoSBkEkA39kk0/edit?usp=sharing
\includegraphics[width=\columnwidth]{figs/5_/m600_labeled.jpg}
\caption{Hexarotor system (DJI M600 Pro) with onboard compute and camera. Left image shows drone on the ground, right image shows drone in flight.}
\label{fig:m600}
\end{figure}


\section{Field Deployments}\label{sec:field}


\subsection{Hexarotor Deployment}
The first field experiment that we present uses a hexarotor drone to cover an urban area shown in Fig.~\ref{fig:fig1}.
We designed this field experiment to simulate classifying where cars are within a search area.  
Hence, we set the plan request to focus on parking lots at the field test site (Fig.~\ref{fig:google_earth}, top), with the addition of three chosen grid cells within the parking lots being marked as having a higher uncertainty. The plan request boundaries and priors were created with GPS coordinates in Google Earth, exported as kml files, and then converted into our plan request message format. 

The following sections details the hardware, autonomy, and experimental results for our hexarotor deployments.

% without the authors goes here
% \begin{figure}[t] 
%     \centering
%     \renewcommand\arraystretch{0} % Adjust the height between rows here
%     \setlength{\tabcolsep}{1pt} % Adjust the column separation here
%     \begin{tabular}{c}
%         \begin{tikzpicture}
%             \node[anchor=south west, inner sep=0] (image) at (0,0) {
%                 \includegraphics[width=0.9\linewidth]{figs/5_/google_earth_prior.png}
%             };
%             \begin{scope}[x={(image.south east)},y={(image.north west)}]
%                 % \fill[OrangeRed] (0.02, 0.03) circle (2pt); 
%                 % \fill[OrangeRed] (0.51, 0.04) circle (2pt); 
%                 % \fill[OrangeRed] (0.61, 0.04) arc (0:90:2pt); 
%                 \fill[Orange, opacity=0.8] (0.74, 0.45) circle (3pt); % Adjust 
%                 \fill[Orange, opacity=0.8] (0.27, 0.42) circle (3pt); % Adjust 
%                 \fill[Orange, opacity=0.8] (0.39, 0.63) circle (3pt); % Adjust 
%             \end{scope}
%         \end{tikzpicture} \\
%         % \includegraphics[width=0.9\linewidth]{figs/5_/google_earth_prior.png} \\
%         \\
%         \includegraphics[width=0.9\linewidth]{figs/5_/google_earth_path.png} 
%     \end{tabular}
%     \caption{Google Earth screenshots illustrating the mission planning process and execution. Top: Areas of high entropy targeted for search are highlighted in red, representing regions with a binary occupied/unoccupied probability of 0.2. Three points of particular interest, each assigned a 0.5 probability, are marked in orange. Bottom: The executed drone flight path (yellow) shows the optimized path for maximum information gain across the search space.} 
%     \label{fig:google_earth}
% \end{figure}
% \begin{figure}[t]
% \centering
% % https://docs.google.com/presentation/d/1RjI-QqHpBRLHN60UAxzmQYs4EaWaVCOoSBkEkA39kk0/edit?usp=sharing
% \includegraphics[width=\columnwidth]{figs/5_/m600_labeled.jpg}
% \caption{Hexarotor system (DJI M600 Pro) with onboard compute and camera. Left image shows drone on the ground, right image shows drone in flight.}
% \label{fig:m600}
% \end{figure}

\subsubsection{Hardware System}
The hardware consists of the DJI M600 Pro, shown in Fig.~\ref{fig:m600}, along with the physical sensing and onboard computer payload. The DJI M600 Pro contains a flight controller that handles pose estimation and position-based control. The DJI M600 Pro’s flight controller also handles teleloperation if human intervention is necessary. Beneath the drone's base, we mount a custom hardware payload.
That payload consists of an onboard computer, a Jetson Xavier, to run the autonomy software shown in Fig.~\ref{fig:functional_diagram}.
The payload also contains a downward-facing a camera for sensing the environment. The camera is a Seek S304SP thermal camera.
The camera intrinsics are used to calculate the frustum's intersection with the search map's cells in IA-TIGRIS.

% without authors goes here
\begin{figure}[t]
\centering
% https://lucid.app/lucidchart/f750ddb4-2809-4773-8361-d5fbb1ba49eb/edit?viewport_loc=-257%2C-116%2C2219%2C1140%2C0_0&invitationId=inv_56e8a3a9-e8cf-4cad-a280-48bd967ff651
\includegraphics[trim={0cm 0cm 0cm 0cm},clip,width=\columnwidth]{figs/5_/functional_diagram.jpeg}
\caption{Functional diagram of the DJI M600 Pro autonomy software.}
\label{fig:functional_diagram}
\end{figure}
\begin{figure}[b]
    \centering
    \begin{subfigure}[b]{0.48\columnwidth}
        \centering
        \includegraphics[width=1.0\linewidth]{figs/5_/field_test_altitude_over_time.png}
        \caption{}
        \label{fig:m600_altitude_over_time}
    \end{subfigure}
    \begin{subfigure}[b]{0.48\columnwidth}
        \centering
        \includegraphics[width=1.0\linewidth]{figs/5_/field_test_entropy_over_time.png}
        \caption{}
        \label{fig:m600_entropy_over_time}
    \end{subfigure}
    \caption{The results for our hexarotor field deployment. (a) Plot of flown altitude over time, showing large variation throughout the experiment. (b) Reduction in entropy percentage over time of field experiment.}
\end{figure}

\subsubsection{Autonomy System}
Fig.~\ref{fig:functional_diagram} illustrates the functional system diagram for the real world field test on the DJI M600. The user specifies the initial plan request prior to takeoff. The TIGRIS planner makes an initial plan on that plan request and sends a global path to the waypoint manager. The waypoint manager tracks the current waypoint within the plan and sends the next waypoint to the DJI software development kit, which then sends actuation commands to the motors. The position of the drone is used to calculate the distance from the drone to the ground and sends that distance parameter to the sensor model. The sensor model's true positive and false positive rate is used to calculate the per-cell entropy updates in the search map manager. The search map manager publishes the current information map, and the replanning node sends an updated plan request to the IA-TIGRIS planner every ten seconds.

The drone started at an altitude of $50$ m above the origin of the reference frame. The informed sampler in IA-TIGRIS was set to add states at altitudes of either $30$ m or $60$ m, creating a trade-off between observation area and detector accuracy. The budget was $2000$ m, the planning horizon was $600$ m, and the planning time was $10$ seconds. 

% % without authors goes here
% \begin{figure}[t]
% \centering
% % https://lucid.app/lucidchart/f750ddb4-2809-4773-8361-d5fbb1ba49eb/edit?viewport_loc=-257%2C-116%2C2219%2C1140%2C0_0&invitationId=inv_56e8a3a9-e8cf-4cad-a280-48bd967ff651
% \includegraphics[trim={0cm 0cm 0cm 0cm},clip,width=\columnwidth]{figs/5_/functional_diagram.jpeg}
% \caption{Functional diagram of the DJI M600 Pro autonomy software.}
% \label{fig:functional_diagram}
% \end{figure}
% \begin{figure}[b]
%     \centering
%     \begin{subfigure}[b]{0.48\columnwidth}
%         \centering
%         \includegraphics[width=1.0\linewidth]{figs/5_/field_test_altitude_over_time.png}
%         \caption{}
%         \label{fig:m600_altitude_over_time}
%     \end{subfigure}
%     \begin{subfigure}[b]{0.48\columnwidth}
%         \centering
%         \includegraphics[width=1.0\linewidth]{figs/5_/field_test_entropy_over_time.png}
%         \caption{}
%         \label{fig:m600_entropy_over_time}
%     \end{subfigure}
%     \caption{The results for our hexarotor field deployment. (a) Plot of flown altitude over time, showing large variation throughout the experiment. (b) Reduction in entropy percentage over time of field experiment.}
% \end{figure}

\subsubsection{Experimental Results}


The bottom image of Fig.~\ref{fig:google_earth} shows the path selected by IA-TIGRIS in the search area. The figure highlights how the planner dynamically adjusts altitudes over time to balance coverage and sensing resolution, maximizing information gain. Higher altitudes allow for broader area coverage, while lower altitudes provide more detailed observations where needed. Additionally, the planner prioritizes revisiting the three regions of higher uncertainty, recognizing the need for repeated observations reduce entropy. This adaptive strategy ensures that uncertain areas receive sufficient attention to improve the belief map. As a result, the entropy of the information map decreases to near zero by the end of the mission, as shown in Fig.~\ref{fig:m600_entropy_over_time}, indicating that the planner has effectively gathered the necessary information. This behavior demonstrates the planner’s ability to optimize sensing actions, balancing altitude selection, revisit frequency, and exploration to maximize mission success.

\begin{figure}[t]
\centering
% \includegraphics[width=2.5in]{fig1}
\includegraphics[trim={4cm 4cm 0cm 4cm},clip,width=\columnwidth]{figs/5_/TL1.jpg}
\caption{Fixed-wing platform used for autonomous flights with an onboard camera pitched at 10 degrees\cite{alarewebsite}}
\label{fig:tl1}
\end{figure}






\subsection{Fixed-wing Deployments}

Our proposed approach was extensively tested on the fixed-wing AlareTech TL-1 UAV, shown in Fig.~\ref{fig:tl1}. The UAV is equipped with an onboard camera pitched at 10 degrees, which introduces a more challenging planning problem due to the non-holonomic motion model and the camera's field of view. Over more than 20 flight hours and 100 flights running IA-TIGRIS, we validated our approach with the objective to search for objects of interest in a large search space across a variety of test scenarios, including different terrain types, varying environmental conditions, and diverse target distributions. An example mission from these tests is shown in Fig.~\ref{fig:fwd}. In this scenario, the planner was given the search bounds and a designated high-priority region. The resulting flight path prioritized revisiting the high-priority area twice, optimizing sensor use and ensuring maximum information gain. This strategy led to the successful detection of the object of interest, with its estimated position marked by the red dot in the figure. 

The map on the upper right in Fig.~\ref{fig:fwd} shows the information map after plan execution was complete. Due to the UAV's limited budget, the upper right and lower left corners of the map are not searched by the agent. The budget is instead utilized to search over the area of higher priority two times. Compared to the paths in Fig.~\ref{fig:google_earth}, we observe that the paths for the fixed wing are smoother and have a larger turning radius, demonstrating how IA-TIGRIS respects the motion constraints of the vehicle. We can also see the effect of wind on the path execution, where the flown path shown in green deviates from the planned path shown in yellow. This illustrates the importance of online planning in the cases where this deviation is large or would accumulate over the course of a longer mission and cause the expected observed area to be much different than actual observed area. 

\begin{figure}[t]
\centering
% \includegraphics[width=2.5in]{fig1}
% [trim={left bottom right top},clip]
\includegraphics[trim={3.0cm, 1.0cm, 3.0cm, 1.0cm},clip,width=\columnwidth]{figs/5_/ONRFig_v3.pdf}
\caption{An example path generated for the fixed-wing platform conducting a large-area search for an object of interest. The larger black rectangle denotes the search bounds, while the smaller black rectangle highlights a region of higher uncertainty. The red dot marks the estimated position of the detected object based on image detections. The upper-right map displays the information state after planning is complete, while the middle plot shows the percent change in entropy over mission time. The flown path illustrates a balance between allocating resources to the high-priority region and exploring other areas within the search space.}
\label{fig:fwd}
\end{figure}

% Also tested extensively on the AlareTech TL-1 (citation?) tube launched UAV seen in Fig.~\ref{fig:tl1}.

% Talk about amount of flights, hours. Platform. Compute. Show visualization fo example flight. Talk about objects of interest in a broad sense (no mention of water/ocean/land for targets). Follow similar figure format as previous section. Main thing we want to highlight is the differences introduced in plans by having a fixed-wing platform compared to a drone. Include image of Alare TL-1 somewhere.

% One big figure showing all the info we want to convey. 

% \BM{Pitch 10 degrees, onboard computer type, etc}


% \subsection{VTOL?}
% what would it bring?


\section{Conclusions}~\label{sec:conclusions}
\section{Conclusion}\label{Sec:con}
This work introduces a benchmarking for model-free varying-diameter log-grasping with a forestry crane, including the structure of the environment, design of reward functions, and a modified proximal policy optimization (mPPO) algorithm. Under the assumption that the log pose is given, extensive simulations are presented to show the effectiveness of the reward shape and the exploration capability of the mPPO over other algorithms. The overall success rate of the grasping task of varying-diameter wood logs, varying log poses, and randomized initial configurations of the forestry crane exceeds $96\%$. 

\textbf{Limitation.} Although our method shows promising results, we recognize many aspects that require further attention, particularly regarding the sim-to-real gap. For instance, while the simulation offers many benefits, real-world uncertainties such as sensor noise, actuation delays, and unexpected disturbances will require more robust handling. The computational efficiency, especially the training time, can be further optimized by leveraging GPU acceleration. Additionally, incorporating transfer learning techniques may help improve the generalization to physical systems. In future work, we will focus on deploying the learned model in real-world demonstrations and aim to refine the agent’s ability to adapt to dynamic, unpredictable conditions. 


%Additionally, the training process can integrate object estimation and imitation learning. 
%Closing the sim-to-real gap is not a trivial problem since we consider a large-scale hydraulically actuated robot. This is also our main focus for the future work. 




%\section*{Acknowledgment}
%\addcontentsline{toc}{section}{Acknowledgment}
%\lipsum[1]


% The acknowledgments are automatically included only in the final and preprint versions of the paper.

%===============================================================================

% no \bibliographystyle is required, since the corl style is automatically used.

\bibliographystyle{IEEEtran} % use IEEEtran.bst style
\bibliography{mybib}

% \appendix
% \section{Appendix}~\label{sec:appendix}
% 
\section{Metric}
\label{sec:metric}

\textbf{Mean Squared Error (MSE)} Mean Squared Error (MSE) is a common statistical metric used to assess the difference between predicted and actual values. The formula is:
\begin{equation}
    MSE = \frac{1}{n} \sum_{i=1}^{n} (y_i - \hat{y}_i)^2
\end{equation}
where $ n $ is the number of samples, $ y_i $ is the actual value, and $ \hat{y}_i $ is the predicted value.

\textbf{Relative L2 Error} Relative L2 error measures the relative difference between predicted and actual values, commonly used in time series prediction. The formula is:
\begin{equation}
    \text{Relative L2 Error} = \frac{\| Y_{\text{pred}} - Y_{\text{true}} \|_2}{\| Y_{\text{true}} \|_2}
\end{equation}
where $ Y_{\text{pred}} $ is the predicted value and $ Y_{\text{true}} $ is the actual value.

\textbf{Structural Similarity Index Measure (SSIM)} The Structural Similarity Index (SSIM) measures the similarity between two images in terms of luminance, contrast, and structure. The formula is:
\begin{equation}
    SSIM(x, y) = \frac{(2\mu_x \mu_y + C_1)(2\sigma_{xy} + C_2)}{(\mu_x^2 + \mu_y^2 + C_1)(\sigma_x^2 + \sigma_y^2 + C_2)}
\end{equation}
where $ \mu_x $ and $ \mu_y $ are the mean values, $ \sigma_x $ and $ \sigma_y $ are the standard deviations, $ \sigma_{xy} $ is the covariance.

\section{Related Work}
\subsection{Deep Learning based Weather Forecasting}
\textbf{Global Weather Forecasting.} Global weather forecasting has seen significant progress with deep learning models. FourCastNet, based on Fourier neural operators, provides global forecasts comparable to traditional numerical methods like IFS, but at much higher speeds~\cite{pathak2022fourcastnet}. Pangu, utilizing the Swin Transformer, exceeds NWP methods, incorporating earth-specific location embeddings for better performance~\cite{bi2023accurate}. The Spherical Fourier Neural Operator (SFNO) extends Fourier methods using spherical harmonics, offering more stable long-term predictions~\cite{bonev2023spherical}. FuXi focuses on long-term forecasting, achieving a 15-day forecasts comparable to ECMWF~\cite{chen2023fuxi}. GraphCast leverages message-passing networks to improve efficiency and forecasting accuracy~\cite{lam2023learning}, and GenCast builds on this to enhance ensemble forecasting~\cite{price2023gencast}. Further, diffusion models like those in~\cite{li2024generative} generate probabilistic ensembles by sampling, while NeuralGCM~\cite{kochkov2024neural} focuses on atmospheric circulation with a dynamic core, offering climate simulation capabilities but at higher training and inference costs. 

\textbf{Regional Weather Forecasting.} The goal of regional weather forecasting is to enhance local prediction accuracy with high-resolution models. CorrDiff~\cite{mardani2023generative} combines U-Net and diffusion models to improve local forecasts. MetaWeather~\cite{kim2024metaweather} adapts global forecasts to regional contexts using meta-learning. GNNs are also widely applied in regional forecasting, with Graphcast~\cite{lam2023learning} enhancing accuracy by modeling complex spatial dependencies. MetNet-3~\cite{espeholt2022deep} offers high-accuracy forecasts for weather variables, such as precipitation, temperature, and wind speed, at 2-minute intervals and 1–4 km resolution, outperforming traditional models like HRRR. NowcastNet~\cite{zhang2023skilful} and DGMR~\cite{ravuri2021skilful} excel in short-term extreme precipitation forecasts using deep generative models and radar data. In spatiotemporal prediction, NMO~\cite{wu2024neural} models the evolution of physical dynamics, providing new insights for local weather forecasting. Similarly, SimVP~\cite{gao2022simvp} and PastNet~\cite{wu2024pastnet} achieve good results in forecasting local precipitation evolution using spatiotemporal convolution methods.
    
% Despite these advances, none of these methods effectively address the challenge of balancing global and regional high-resolution forecasts or capturing the fine-grained, dynamic interactions important for extreme event prediction.
    
\subsection{Numerical analysis methods}
Multigrid methods~\cite{mccormick1987multigrid,wesseling1995introduction,hackbusch2013multi,bramble2019multigrid,hiptmair1998multigrid,brandt1983multigrid,borzi2009multigrid} and nested grid strategies~\cite{miyakoda1977one,zhang2012nested,sullivan1996grid} are widely used to solve PDEs and handle multi-scale problems~\cite{debreu2008two,xue2000advanced}. Multigrid methods use grids of different resolutions to transfer information and accelerate iterations. They efficiently solve large-scale problems and improve computational accuracy. By eliminating low-frequency errors on coarse grids and high-frequency errors on fine grids, multigrid methods effectively handle error convergence at different scales~\cite{he2019mgnet,he2023mgno,shao2022fast}. Nested grid strategies embed higher-resolution fine grids into regions of interest based on a global coarse grid to capture local complex physical phenomena in detail. In weather forecasting, this method provides large-scale background fields on a global scale while refining the grid for target regions to accurately simulate the evolution of local weather systems and the occurrence of extreme events~\cite{bacon2000dynamically}. 

% Our proposed neural nested grid method helps address challenges like boundary information loss in regional forecasting and multi-scale feature capture.

\section{Additional Results}
%
We present more additional results in Figure \ref{fig_0.25-day}, \ref{fig_0.5-day}, \ref{fig_1.0-day} \ref{fig_1.5-day}, \ref{fig_2.0-day}, \ref{fig_2.5-day}, \ref{fig_3.0-day}, \ref{fig_3.5-day}, \ref{fig_4.0-day}, \ref{fig_4.5-day}, \ref{fig_5.0-day}, \ref{fig_5.5-day}, \ref{fig_6.0-day}, \ref{fig_6.5-day}, \ref{fig_7.0-day}, \ref{fig_7.5-day},
\ref{fig_8.0-day}, \ref{fig_8.5-day}, \ref{fig_9.0-day}, \ref{fig_9.5-day},
\ref{fig_10.0-day}, including 18 variables that are importmant to weather forecasting, each with results ranging from 6 hours to 10 days. These additional results further demonstrate the effectiveness of OneForecast. Same as the Figure \ref{fig:visual_results}
, the initial conditions is 00:00 UTC, 1 January 2020.


\begin{figure*}[h]
\centering
\includegraphics[width=1\linewidth]{figures/fig_0.25-day.jpg}
\vspace{-20pt}
\caption{6-hour forecast results of different models.}
\label{fig_0.25-day}
\end{figure*}

\begin{figure*}[h]
\centering
\includegraphics[width=1\linewidth]{figures/fig_0.5-day.jpg}
\vspace{-20pt}
\caption{0.5-day forecast results of different models.}
\label{fig_0.5-day}
\end{figure*}

\begin{figure*}[h]
\centering
\includegraphics[width=1\linewidth]{figures/fig_1.0-day.jpg}
\vspace{-20pt}
\caption{1-day forecast results of different models.}
\label{fig_1.0-day}
\end{figure*}

\begin{figure*}[h]
\centering
\includegraphics[width=1\linewidth]{figures/fig_1.5-day.jpg}
\vspace{-20pt}
\caption{1.5-day forecast results of different models.}
\label{fig_1.5-day}
\end{figure*}

\begin{figure*}[h]
\centering
\includegraphics[width=1\linewidth]{figures/fig_2.0-day.jpg}
\vspace{-20pt}
\caption{2-day forecast results of different models.}
\label{fig_2.0-day}
\end{figure*}


\begin{figure*}[h]
\centering
\includegraphics[width=1\linewidth]{figures/fig_2.5-day.jpg}
\vspace{-20pt}
\caption{2.5-day forecast results of different models.}
\label{fig_2.5-day}
\end{figure*}

\begin{figure*}[h]
\centering
\includegraphics[width=1\linewidth]{figures/fig_3.0-day.jpg}
\vspace{-20pt}
\caption{3-day forecast results of different models.}
\label{fig_3.0-day}
\end{figure*}

\begin{figure*}[h]
\centering
\includegraphics[width=1\linewidth]{figures/fig_3.5-day.jpg}
\vspace{-20pt}
\caption{3.5-day forecast results of different models.}
\label{fig_3.5-day}
\end{figure*}

\begin{figure*}[h]
\centering
\includegraphics[width=1\linewidth]{figures/fig_4.0-day.jpg}
\vspace{-20pt}
\caption{4-day forecast results of different models.}
\label{fig_4.0-day}
\end{figure*}

\begin{figure*}[h]
\centering
\includegraphics[width=1\linewidth]{figures/fig_4.5-day.jpg}
\vspace{-20pt}
\caption{4.5-day forecast results of different models.}
\label{fig_4.5-day}
\end{figure*}


\begin{figure*}[h]
\centering
\includegraphics[width=1\linewidth]{figures/fig_5.0-day.jpg}
\vspace{-20pt}
\caption{5.0-day forecast results of different models.}
\label{fig_5.0-day}
\end{figure*}

\begin{figure*}[h]
\centering
\includegraphics[width=1\linewidth]{figures/fig_5.5-day.jpg}
\vspace{-20pt}
\caption{5.5-day forecast results of different models.}
\label{fig_5.5-day}
\end{figure*}

\begin{figure*}[h]
\centering
\includegraphics[width=1\linewidth]{figures/fig_6.0-day.jpg}
\vspace{-20pt}
\caption{6.0-day forecast results of different models.}
\label{fig_6.0-day}
\end{figure*}

\begin{figure*}[h]
\centering
\includegraphics[width=1\linewidth]{figures/fig_6.5-day.jpg}
\vspace{-20pt}
\caption{6.5-day forecast results of different models.}
\label{fig_6.5-day}
\end{figure*}

\begin{figure*}[h]
\centering
\includegraphics[width=1\linewidth]{figures/fig_7.0-day.jpg}
\vspace{-20pt}
\caption{7.0-day forecast results of different models.}
\label{fig_7.0-day}
\end{figure*}

\begin{figure*}[h]
\centering
\includegraphics[width=1\linewidth]{figures/fig_7.5-day.jpg}
\vspace{-20pt}
\caption{7.5-day forecast results of different models.}
\label{fig_7.5-day}
\end{figure*}

\begin{figure*}[h]
\centering
\includegraphics[width=1\linewidth]{figures/fig_8.0-day.jpg}
\vspace{-20pt}
\caption{8.0-day forecast results of different models.}
\label{fig_8.0-day}
\end{figure*}

\begin{figure*}[h]
\centering
\includegraphics[width=1\linewidth]{figures/fig_8.5-day.jpg}
\vspace{-20pt}
\caption{8.5-day forecast results of different models.}
\label{fig_8.5-day}
\end{figure*}

\begin{figure*}[h]
\centering
\includegraphics[width=1\linewidth]{figures/fig_9.0-day.jpg}
\vspace{-20pt}
\caption{9.0-day forecast results of different models.}
\label{fig_9.0-day}
\end{figure*}

\begin{figure*}[h]
\centering
\includegraphics[width=1\linewidth]{figures/fig_9.5-day.jpg}
\vspace{-20pt}
\caption{9.5-day forecast results of different models.}
\label{fig_9.5-day}
\end{figure*}

\begin{figure*}[h]
\centering
\includegraphics[width=1\linewidth]{figures/fig_10.0-day.jpg}
\vspace{-20pt}
\caption{10.0-day forecast results of different models.}
\label{fig_10.0-day}
\end{figure*}


\section{Detailed Mathematical Proof}
\label{sec:proof}
\textbf{Proof of Theorem 1}

Now we have N augmented data and we need to select the best from them. We consider both the quality and the diversity of these data and get the sampling strategy from an optimization problem.

We model the sampling strategy as a multinomial distribution supported on all the augmented data $S = \{\mathbf{X}_j\}_{j=1}^N$, which means that the sampling strategy $\pi=(\pi_1,...,\pi_N)^\top$ is the corresponding probabilities of selecting $\mathbf{X}_1,...,\mathbf{X}_N$, then we can model the expectation of the similarity as:
$$\begin{aligned}
 & \mathbb{E}_{Y_x,Y_{x^{\prime}}\in\mathcal{C}}\{g(x,x^{\prime})\mid S\} \\
 & =\quad\int g(\mathbf{x},\mathbf{x}^{\prime})\boldsymbol{\pi}(\mathbf{x})\mathrm{Pr}_{S}(Y_{x}\in\mathcal{C}\mid\boldsymbol{x}=\mathbf{x})\boldsymbol{\pi}(\mathbf{x}^{\prime})\mathrm{Pr}_{S}(Y_{x}\in\mathcal{C}\mid\boldsymbol{x}=\mathbf{x}^{\prime})d\mathbf{x}d\mathbf{x}^{\prime} \\
 & =\quad\sum_{i,j=1}^Ng(\mathbf{X}_i,\mathbf{X}_j)\pi_i\pi_j\mathrm{Pr}_{S}(Y_x\in\mathcal{C}\mid\boldsymbol{x}=\mathbf{X}_i)\mathrm{Pr}_{S}(Y_x\in\mathcal{C}\mid\boldsymbol{x}=\mathbf{X}_j),
\end{aligned}$$
where the set $\mathcal{C}$ denotes the criterion of selection we are using, the function $g$ can be chosen as any similarity metric function and $x$ means a random variable.

The core to solving the above optimization problem is to use predictive inference to approximate the conditional probability of $\{Y_x\in\mathcal{C}\}$ given $x = \mathbf{X}$
Let $\mu ( \mathbf{x} ) : = \mathbb{E} ( Y\mid \mathbf{X} = \mathbf{x} )$ be the oracle associated with $( \mathbf{X} , Y) .$ Denote $\theta_j=\mathbb{I}\{Y_j\in\mathcal{C}\}$. As the augmented data
$\mathbf{X}_1,...,\mathbf{X}_N$ are independently identically distributed, $\theta_1,...,\theta_N$ can be regarded as independent Bernoulli($q)$ variables with $q=\Pr(Y_j\in\mathcal{C}).$ The probability distribution of the predicted result $W_j$ for $j=1,...,N$ is
$$\Pr(W_j\mid\theta_j)=(1-\theta_j)f_0+\theta_jf_1,\quad$$
where $f_0$ and $f_1$ are the conditional distributions of $W_j$ on $Y_j \in \mathcal{C}$ or not.

Denote $T(w) = \frac{(1-q)f_0(W_j)}{f(W_j)}$, we can rewrite the expectation of the similarity as
$$\mathbb{E}_{Y_x,Y_{x^{\prime}}\in\mathcal{C}}\{g(x,x^{\prime})|S\}=\sum_{i,j=1}^Ng(\mathbf{X}_i,\mathbf{X}_j)\pi_i\pi_j(1-T_i)(1-T_j)=\boldsymbol{\pi}^\top A_\mathbb{T}\boldsymbol{\pi},$$

Next, we use the expectation to control the quality of the data.
$$\mathbb{E}\{\mathbb{I}(Y_x\not\in\mathcal{C})\mid S\}=\sum_{i=1}^N\Pr(Y_i\not\in\mathcal{C}\mid\mathbf{X}_i)\pi_i=\sum_{i=1}^N\pi_iT_i\leq\alpha,$$

In all, the optimization problem can be modeled as 
\begin{align}
    & \arg\min_{\boldsymbol{\pi}}\quad h(\boldsymbol{\pi},\mathbb{T}):=\boldsymbol{\pi}^\top A_\mathbb{T}\boldsymbol{\pi}, \\
    & \text{subject to} \quad
        \begin{cases}
            \sum_{i = 1}^N\pi_iT_i\leq\alpha, \\
            \sum_{i = 1}^N\pi_i = 1, \\
            0\leq\pi_i\leq m^{-1}, \quad 1\leq i\leq N.
        \end{cases}
\end{align}

where $m$ is used to control the maximum selection.

The best selection of K is determined by the strategy $\pi$ which serves as the solution to the above optimization problem.

\section{Additional Experiments}
\label{sec:more_experiments}
\subsection{Long-term forecasting experiment expansion}

In the long-term forecasting experiments, we compare the performance of different backbone models on the SWE benchmark, evaluating the relative L2 error for three variables (U, V, and H). Our setup inputs 5 frames and predicts 50 frames. For the SimVP-v2 model, using \method{} reduces the relative L2 error for SWE (u) from 0.0187 to 0.0154, SWE (v) from 0.0387 to 0.0342, and SWE (h) from 0.0443 to 0.0397. We visualize SWE (h) in 3D as shown in Figure~\ref{fig:case} [\textcolor{red}{I}]. For the ConvLSTM model, applying \method{} reduces the relative L2 error for SWE (u) from 0.0487 to 0.0321, SWE (v) from 0.0673 to 0.0351, and SWE (h) from 0.0762 to 0.0432. For the FNO model, using \method{} reduces the relative L2 error for SWE (u) from 0.0571 to 0.0502, SWE (v) from 0.0832 to 0.0653, and SWE (h) from 0.0981 to 0.0911. Overall, \method{} significantly improves the long-term forecasting accuracy of different backbone models.

\begin{figure*}[h]
    \centering
    \includegraphics[width=\textwidth]{image/casestudy.pdf}
    \caption{
    \textcolor{red}{I.} 3D visualization of the SWE(h), showing Ground-truth, SimVP-V2+BeamVQ predictions, and Error at T=1, 10, 20, 30, 40, 50. The first row shows Ground-truth, the second SimVP-V2+BeamVQ predictions, and the third Error. \textcolor{red}{II.} A case study. Building fire simulation with ventilation settings added to Wu's Prometheus~\cite{wu2024prometheus}. (a) Layout and HRR growth. (b) Comparison of physical metrics for different methods. (c) Ground-truth, ResNet+BeamVQ, and ResNet predictions.
    }
    \label{fig:case} 
\end{figure*}


\subsection{Experiment Statistical Significance}
\label{sec:significance}
To measure the statistical significance of our main experiment results, we choose three backbones to train on two datasets to run 5 times. 
Table~\ref{tab:significance} records the average and standard deviation of the test MSE loss.
The results prove that our method is statistically significant to outperform the baselines
because our confidence interval is always upper than the confidence interval of the baselines. 
Due to limited computation resources, we do not cover all ten backbones and five datasets, 
but we believe these results have shown that our method has consistent advantages.


\begin{table}[h]
\label{tab:significance}
\centering
\begin{scriptsize}
    \begin{sc}
    \caption{ The average and standard deviation of MSE in 5 runs}
    \label{tab:significance}
    \centering
        \renewcommand{\multirowsetup}{\centering}
        \setlength{\tabcolsep}{10pt}
        \begin{tabular}{l|cc|cc}
            \toprule
            
            \multirow{4}{*}{Model} & \multicolumn{4}{c}{Benchmarks}  \\
            \cmidrule(lr){2-5}
            & \multicolumn{2}{c}{NSE} &   \multicolumn{2}{c}{SEVIR}   \\
            \cmidrule(lr){2-5}
           & Ori & + BeamVQ & Ori & + BeamVQ  \\
            \midrule
            ConvLSTM &0.4092$\pm$0.0002 &\textbf{0.1277$\pm$0.0001}  & 0.1762 0.0007  & \textbf{0.1279$\pm$0.0009}  \\
            FNO &  0.2227$\pm$0.0003 &\textbf{0.1007 $\pm$0.0002}& 0.0787$\pm$0.0012 & \textbf{ 0.0437$\pm$0.0013} \\
            CNO & 0.2192 $\pm$0.0008 &\textbf{ 0.1492$\pm$0.0011}& 0.0057$\pm$0.0005 & \textbf{ 0.0053$\pm$0.0006} \\
            \bottomrule
        \end{tabular}
    \end{sc}

\end{scriptsize}
\end{table}

\end{document}
