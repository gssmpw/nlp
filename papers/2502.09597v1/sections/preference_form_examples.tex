
\begin{table}[H]
\centering
\caption{Explicit preferences examples paired with queries from different topics. We also include explanations for why a generic response to the queries may easily conflict with these preferences.}
\scalebox{0.85}{
\begin{tabular}{p{0.1\textwidth}|p{0.3\textwidth}|p{0.3\textwidth}|p{0.4\textwidth}}
\hline
\textbf{Topic}&
\textbf{Explicit Preference} & \textbf{Query} & \textbf{Explanation} \\
\hline
Lifestyle-Beauty & I avoid using any skincare or beauty products containing silicones. & What would you recommend for a daily makeup primer? & Silicone-based primers are common for providing a smooth base. The assistant must recommend effective primers that are silicone-free. \\
\hline
Travel-Hotel&I prefer not to stay in historic or cultural properties because I find them too old-fashioned. & What are some of the best hotels for my stay in Rome? & Rome has many historic hotels, which might not suit the user's preference. The assistant should focus on more modern or contemporary hotels. \\
\hline
Travel-Transport&I refuse to use ride-sharing services due to safety and privacy concerns. & I'm visiting Chicago for a conference next month. What would be the best way for me to get around the city efficiently? & While ride-sharing services like Uber and Lyft are popular transportation options in many cities, the user's firm preference against using them due to safety and privacy concerns adds complexity. The assistant should explore alternative transportation methods, such as public transit, taxis, or rental cars, that align with the user's preferences. \\
\hline
Entertain-Music Books&I'm not a fan of digital music streaming services. I prefer physical formats like vinyl. & Can you suggest some great new music albums that I should check out? & The natural way to suggest new music albums often involves referencing streaming services. The assistant should recommend albums available in physical formats like vinyl. \\
\hline
Education-Learn Styles&I prefer self-paced, asynchronous learning resources over scheduled classes or live sessions. & I'm interested in learning a new foreign language. What would be the best way for me to get started? & Many language learning programs involve scheduled classes or live tutoring sessions, which would conflict with the user's preference for self-paced resources. The assistant should recommend options like language learning apps, online courses with pre-recorded lessons, or self-study textbooks that allow the user to learn at their own pace without adhering to a fixed schedule. \\
\hline
Pet-Ownership&I have a phobia of birds.& What would be a good, interactive pet for my apartment? & Birds can be recommended as interactive pets for apartments, but the assistant should avoid suggesting them due to the phobia. \\
\hline
Shop-Motors&I refuse to drive any vehicle that isn't electric due to environmental concerns. & Can you recommend a car for me that offers both luxury and performance? & Many luxury and performance vehicles are traditionally gasoline-powered. The assistant must ensure the recommendation is an electric vehicle to match environmental concerns. \\
\hline
Travel-Restaurants&I have a severe peanut allergy, so I must avoid any foods containing peanuts or peanut products. & I'm visiting Thailand next month. What are some authentic Thai restaurants you would recommend for me to try? & Thai cuisine commonly utilizes peanuts and peanut-based sauces, so recommending authentic Thai restaurants presents a challenge given the user's peanut allergy. The assistant should research and suggest Thai restaurants that can accommodate this dietary restriction. \\
\hline
Shop-Fashion&I always avoid wearing floral patterns. & Can you suggest some fashionable dresses for the spring season? & Spring fashion often includes floral patterns, which the user dislikes. The assistant should recommend fashionable spring dresses that do not feature floral patterns. \\
\hline
\end{tabular}}

\label{tab:explicit_preferences}
\end{table}

\begin{table}[H]
\centering
\caption{Implicit preferences: Choice-Based Dialogue examples. Each dialogue showcases the user's preference revealed through a two-turn conversation between the assistant and the user. Every implicit preference dialogue is derived from an explicit preference. The underlined sentences indicate where the user's choice or rejection of options reveals their preferences.}
\scalebox{0.8}{
\begin{tabular}{p{0.2\textwidth}|p{0.95\textwidth}}
\hline
\textbf{\thead{Explicit\\ Preference}} & \textbf{Implicit Preference: Choice-Based Dialogue} \\
\hline
I dislike staying in high-rise hotels and prefer properties that are no taller than three stories. & 
\textbf{\textit{User}}: What are some good hotel options for my upcoming trip to Paris?

\textbf{\textit{Assistant}}: There are several great hotel options in Paris to consider. Here are a few recommendations:

1. The charming Hôtel des Deux Îles is a cozy three-story boutique hotel near the Seine.

2. The iconic Hyatt Regency Paris Étoile is a soaring high-rise hotel with panoramic city views.

3. The luxurious Shangri-La Hotel Paris is located in a historic high-rise building with stunning views.

4. The modern Renaissance Paris Republique Hotel offers sleek accommodations in a towering skyscraper.

\textbf{\textit{User}}: The Hôtel des Deux Îles sounds perfect for my needs. \underline{The high-rise options don't really appeal to me.}

\textbf{\textit{Assistant}}: Understood, the Hôtel des Deux Îles is an excellent choice. \\
\hline

I don't enjoy games with pixel art graphics. & 
\textbf{\textit{User}}: What video game would you recommend for me to play next?

\textbf{\textit{Assistant}}: Here are some recommendations for your next video game:

1. Undertale is a popular indie game with a retro pixel art aesthetic.

2. I'd suggest Horizon Zero Dawn, with its stunning graphics and open world.

3. You might enjoy Stardew Valley, with its charming pixel art style.

4. Celeste is a great platformer with beautiful pixel art graphics.

\textbf{\textit{User}}: I'll go with Horizon Zero Dawn. \underline{The other options don't quite appeal to me.}

\textbf{\textit{Assistant}}: Horizon Zero Dawn is an excellent choice. \\
\hline

I strictly avoid restaurants that serve foods containing gluten due to a severe gluten intolerance. & 
\textbf{\textit{User}}: What are some good lunch options near the office?

\textbf{\textit{Assistant}}: There are a few options nearby for lunch. 1. The salad bar at the health food store has many gluten-free choices. 2. The new sandwich shop has delicious panini on freshly baked bread. 3. The Italian restaurant down the street is known for their homemade pasta dishes. 4. The pub around the corner has a great selection of burgers and beer.

\textbf{\textit{User}}: The salad bar at the health food store sounds good. \underline{The other options don't work for me.}

\textbf{\textit{Assistant}}: Got it, the salad bar is a great gluten-free choice. \\
\hline
I cannot eat spicy food. & 
\textbf{\textit{User}}: What are some good restaurant options for dinner tonight?

\textbf{\textit{Assistant}}: There are several options to consider for dinner tonight. Here are a few:

1. A Mexican cantina with a wide selection of spicy tacos and enchiladas.

2. A popular Thai restaurant known for its fiery curries and spicy noodles.

3. A cozy Italian bistro serving mild pasta dishes and salads.

4. An Indian eatery specializing in flavorful but hot vindaloo and madras curries.

\textbf{\textit{User}}: The cozy Italian bistro serving mild pasta dishes and salads sounds perfect. \underline{The other options seem a bit too spicy for my taste.}

\textbf{\textit{Assistant}}: Understood, the Italian bistro is an excellent choice. \\
\hline
I'm allergic to peanuts. & 
\textbf{\textit{User}}: What kind of snack should I bring for the office potluck?

\textbf{\textit{Assistant}}: For the office potluck, you could consider bringing one of these options: 

1. Homemade peanut butter cookies, a classic crowd-pleaser.

2. A fresh fruit salad with a variety of berries and citrus.

3. Thai peanut noodle salad, a flavorful and filling option.

4. Peanut brittle, a crunchy and sweet treat for everyone.

\textbf{\textit{User}}: I think I'll go with the fresh fruit salad. The other options sound great, but \underline{I'd rather avoid anything with peanuts or peanut butter.}

\textbf{\textit{Assistant}}: A fresh fruit salad is an excellent choice. \\
\hline
I am not interested in documentaries. & 
\textbf{\textit{User}}: What kind of movie would you like to watch tonight?

\textbf{\textit{Assistant}}: There are a few options we could consider. How about:

1. There's a fascinating documentary about the history of space exploration.

2. I heard the new nature documentary is really insightful.

3. There's a great new documentary on climate change.

4. How about a fun comedy or an action-packed thriller?

\textbf{\textit{User}}: \underline{The documentaries don't really interest me.} How about a fun comedy or an action-packed thriller?

\textbf{\textit{Assistant}}: A comedy or thriller sounds great. \\
\hline

\end{tabular}}
\label{tab:implicit_choice-based}
\end{table}




\begin{table}[H]
\centering
\caption{Implicit Preferences: Persona-driven Dialogue Example. Each dialogue showcases a user's preference revealed through a 4-8 turn conversation between the assistant and the user. Every dialogue is derived from an explicit preference and is randomly assigned a persona to simulate a longer conversation. This type of preference adds further complexity to preference inference, as the revelation of the preference is not the main topic of the dialogue; instead, the persona-based topic serves as the main focus. The underlined sentences indicate where the user reveals their preference in a short statement. }
\scalebox{0.8}{
\begin{tabular}{p{0.14\textwidth}|p{0.14\textwidth}|p{0.85\textwidth}}
\hline
\textbf{\thead{Explicit\\ Preference}} & \textbf{Persona} & \textbf{Implicit Preference: Persona-Driven Dialogue} \\
\hline
I strictly avoid restaurants that serve foods containing gluten due to a severe gluten intolerance. & An elementary school librarian passionate about children's literacy & \textbf{\textit{User:}} Hi there! I'm planning a reading event for my elementary school library. Do you have any suggestions for engaging activities that promote literacy? 

\textbf{\textit{Assistant:}} Hello! That's a wonderful initiative. For an engaging reading event, you could consider a book-themed scavenger hunt, where students search for clues related to popular children's books. Another idea is to set up reading stations with different genres, allowing students to explore various types of literature. You could also organize a storytelling session where students take turns reading aloud or acting out parts of their favorite books. These activities can make reading fun and interactive for young learners. 

\textbf{\textit{User:}} Those are great ideas! I especially like the scavenger hunt concept. Speaking of books, I'm looking for recommendations for new children's novels that focus on diversity and inclusion. Any suggestions? 

\textbf{\textit{Assistant:}} I'm glad you like the scavenger hunt idea! For diverse and inclusive children's novels, I'd recommend \textit{Front Desk} by Kelly Yang, which explores the immigrant experience through a young girl's perspective. \textit{New Kid} by Jerry Craft is an excellent graphic novel addressing racial issues in school. \textit{The Proudest Blue} by Ibtihaj Muhammad celebrates cultural identity and self-confidence. These books not only promote diversity but also offer engaging storylines that children can relate to and learn from. 

\textbf{\textit{User:}} Thank you for those recommendations! I'll definitely look into them. On a related note, I'm planning a field trip for my students to a local bookstore. I usually prefer places that also have a café area where I can grab a \underline{gluten-free snack due to my severe intolerance}. Do you know of any educational activities we could do at a bookstore to make the trip more interactive? 

\textbf{\textit{Assistant:}} A bookstore field trip sounds like a fantastic idea! To make it more interactive, you could organize a "book cover scavenger hunt" where students search for specific elements on book covers. Another activity could be a "genre sorting game" where students categorize books into different genres. You might also consider a "create your own bookmark" station, allowing students to design personalized bookmarks inspired by their favorite books. These activities can help students engage with books in a fun, hands-on way while exploring the bookstore environment. 

\textbf{\textit{User:}} Those are all wonderful suggestions! I think the students will really enjoy the bookmark activity. Now, I'm curious about incorporating technology into our library. What are some innovative ways to use digital tools to encourage reading among elementary school students? 

\textbf{\textit{Assistant:}} Incorporating technology into your library is a great way to engage young readers! You could introduce e-reader devices with a selection of e-books, allowing students to experience digital reading. Another idea is to use reading apps that gamify the reading experience, such as \textit{Epic!} or \textit{Reading Eggs}. Creating a library blog where students can post book reviews or participate in online book discussions can also be engaging. Additionally, you might consider setting up a green screen area where students can create book trailers or act out scenes from their favorite stories, combining reading with digital creativity. 

\textbf{\textit{User:}} I love the idea of creating book trailers! That could be a fun project for our older students. Lastly, I'm thinking about starting a monthly book club for teachers at our school. Any advice on how to structure it and keep everyone motivated? 

\textbf{\textit{Assistant:}} Starting a book club for teachers is an excellent way to foster a reading culture among staff! To structure it, you could rotate book selection responsibilities among members, ensuring diverse choices. Set a regular meeting time, perhaps during lunch breaks or after school hours. To keep everyone motivated, consider themed months (e.g., multicultural literature, STEM books) or tie selections to current educational trends. You could also incorporate short, fun activities related to each book during meetings. Lastly, create an online space (like a private social media group) for members to share thoughts between meetings. This approach can help maintain engagement and make the club both enjoyable and professionally enriching for your colleagues. \\
\hline



\end{tabular}}
\label{tab:implicit_persona}
\end{table}

