Personalization for dialogue systems aims to endow a text
generation system with particular personality
traits.

\citet{zhang-etal-2018-personalizing} condition on user profile information to provide more personalized chit-chat responses. Focuse on more engaging chit-chat dialogue agents by endowing
them with a configurable, but persistent persona. profile can be stored in a memory-augmented neural network. present the PERSONA-CHAT dataset, a new dialogue dataset consisting of 162,064 utterances between crowdworkers who were randomly paired
and each asked to act the part of a given provided
persona (randomly assigned, and created by another set of crowdworkers). conditioning the agent with
persona information gives improved prediction of
the next dialogue utterance. Metric: next utterance prediction conditioned on the persona based dialogue. 

\citet{mazare-etal-2018-training} following the above work, introduce a new dataset providing 5 million personas and 700 million persona-based dialogues to train on. 

instead of using synthetic or labelled personzlaied dialogue, \citet{wu-etal-2021-personalized} collected personalized responses for questions on Reddit by utilizing personalized user profiles and posting histories in a single-turn dialog setting. 

\citet{ao-etal-2021-pens} focuses on personalized news headline generation problem whose goal is to output a user-specific title.


\citet{zhong-etal-2022-less} learn to extract the most informative segments from the user's history for personalization.

Other types of personalization generation tasks:
\citet{salemi2023lamp} personalizing text classification/generation such as citation identification, tweet paragraphing.


personalize large language models themselves to embody a particular persona, where the goal is to generate output that mimics the speaking style and patterns associated with the provided persona metadata \citep{vincent2023reference}.
