% This must be in the first 5 lines to tell arXiv to use pdfLaTeX, which is strongly recommended.
\pdfoutput=1
% In particular, the hyperref package requires pdfLaTeX in order to break URLs across lines.

\documentclass[11pt]{article}

% Change "review" to "final" to generate the final (sometimes called camera-ready) version.
% Change to "preprint" to generate a non-anonymous version with page numbers.
% \usepackage[review]{acl}
\usepackage[preprint]{acl}

% Standard package includes
\usepackage{times}
\usepackage{latexsym}

% For proper rendering and hyphenation of words containing Latin characters (including in bib files)
\usepackage[T1]{fontenc}
% For Vietnamese characters
% \usepackage[T5]{fontenc}
% See https://www.latex-project.org/help/documentation/encguide.pdf for other character sets

% This assumes your files are encoded as UTF8
\usepackage[utf8]{inputenc}

% This is not strictly necessary, and may be commented out,
% but it will improve the layout of the manuscript,
% and will typically save some space.
\usepackage{microtype}

% This is also not strictly necessary, and may be commented out.
% However, it will improve the aesthetics of text in
% the typewriter font.
\usepackage{inconsolata}

%Including images in your LaTeX document requires adding
%additional package(s)
\usepackage{graphicx}


\usepackage[utf8]{inputenc} % allow utf-8 input
\usepackage[T1]{fontenc}    % use 8-bit T1 fonts
\usepackage{hyperref}       % hyperlinks
\usepackage{url}            % simple URL typesetting
\usepackage{booktabs}       % professional-quality tables
\usepackage{amsfonts}       % blackboard math symbols
\usepackage{nicefrac}       % compact symbols for 1/2, etc.
\usepackage{microtype}      % microtypography
\usepackage{xcolor}         % colors
\usepackage{multirow,multicol}
\usepackage{graphicx}
\usepackage{arydshln}
\usepackage{amsmath}
\usepackage{enumitem}

% If the title and author information does not fit in the area allocated, uncomment the following
%
%\setlength\titlebox{<dim>}
%
% and set <dim> to something 5cm or larger.

% \title{\our: An IE Free Rider on LLM Training Paradigm}
\title{\our: An IE Free Rider Hatched by Massive Nutrition in LLM's Nest}

% Author information can be set in various styles:
% For several authors from the same institution:
% \author{Author 1 \and ... \and Author n \\
%         Address line \\ ... \\ Address line}
% if the names do not fit well on one line use
%         Author 1 \\ {\bf Author 2} \\ ... \\ {\bf Author n} \\
% For authors from different institutions:
% \author{Author 1 \\ Address line \\  ... \\ Address line
%         \And  ... \And
%         Author n \\ Address line \\ ... \\ Address line}
% To start a separate ``row'' of authors use \AND, as in
% \author{Author 1 \\ Address line \\  ... \\ Address line
%         \AND
%         Author 2 \\ Address line \\ ... \\ Address line \And
%         Author 3 \\ Address line \\ ... \\ Address line}




\author{Letian Peng, Zilong Wang, Feng Yao, Jingbo Shang \\
University of California, San Diego \\
  \texttt{\{lepeng, ziw049, fengyao, jshang\}@ucsd.edu}
  }
%\author{
%  \textbf{First Author\textsuperscript{1}},
%  \textbf{Second Author\textsuperscript{1,2}},
%  \textbf{Third T. Author\textsuperscript{1}},
%  \textbf{Fourth Author\textsuperscript{1}},
%\\
%  \textbf{Fifth Author\textsuperscript{1,2}},
%  \textbf{Sixth Author\textsuperscript{1}},
%  \textbf{Seventh Author\textsuperscript{1}},
%  \textbf{Eighth Author \textsuperscript{1,2,3,4}},
%\\
%  \textbf{Ninth Author\textsuperscript{1}},
%  \textbf{Tenth Author\textsuperscript{1}},
%  \textbf{Eleventh E. Author\textsuperscript{1,2,3,4,5}},
%  \textbf{Twelfth Author\textsuperscript{1}},
%\\
%  \textbf{Thirteenth Author\textsuperscript{3}},
%  \textbf{Fourteenth F. Author\textsuperscript{2,4}},
%  \textbf{Fifteenth Author\textsuperscript{1}},
%  \textbf{Sixteenth Author\textsuperscript{1}},
%\\
%  \textbf{Seventeenth S. Author\textsuperscript{4,5}},
%  \textbf{Eighteenth Author\textsuperscript{3,4}},
%  \textbf{Nineteenth N. Author\textsuperscript{2,5}},
%  \textbf{Twentieth Author\textsuperscript{1}}
%\\
%\\
%  \textsuperscript{1}Affiliation 1,
%  \textsuperscript{2}Affiliation 2,
%  \textsuperscript{3}Affiliation 3,
%  \textsuperscript{4}Affiliation 4,
%  \textsuperscript{5}Affiliation 5
%\\
%  \small{
%    \textbf{Correspondence:} \href{mailto:email@domain}{email@domain}
%  }
%}

\usepackage{xspace}

\newcommand{\our}{Cuckoo\xspace}
\newcommand{\jingbo}[1]{\textcolor{blue}{\textbf{Jingbo:} #1}}
\newcommand{\zilong}[1]{\textcolor{cyan}{\textbf{Zilong:} #1}}
\newcommand{\feng}[1]{\textcolor{red}{\textbf{feng:} #1}}
\newcommand{\zihanmod}[1]{\textcolor{brown}{#1}}

% \newcommand{\jingbo}[1]{}
% \newcommand{\zilong}[1]{}

\begin{document}
\maketitle
\begin{abstract}
    \begin{abstract}

Backdoor learning is a critical research topic for understanding the vulnerabilities of deep neural networks. While it has been extensively studied in discriminative models over the past few years, backdoor learning in diffusion models (DMs) has recently attracted increasing attention, becoming a new research hotspot. Although many different backdoor attack and defense methods have been proposed for DMs,  a comprehensive benchmark for backdoor learning in DMs is still lacking. This absence makes it difficult to conduct fair comparisons and thoroughly evaluate existing approaches, thus hindering future research progress. To address this issue, we propose \textit{BackdoorDM}, the first comprehensive benchmark designed for backdoor learning in DMs. It comprises nine state-of-the-art (SOTA) attack methods, four SOTA defense strategies, and two helpful visualization analysis tools. We first systematically classify and formulate the existing literature in a unified framework, focusing on three different backdoor attack types and five backdoor target types, which are restricted to a single type in discriminative models. Then, we systematically summarize the evaluation metrics for each type and propose a unified backdoor evaluation method based on GPT-4o. Finally, we conduct a comprehensive evaluation and highlight several important conclusions. We believe that BackdoorDM will help overcome current barriers and contribute to building a trustworthy DMs community. 
% Our code is provided \href{https://anonymous.4open.science/r/BackdoorDM-1BE8}{here}.
The codes are released in \href{https://github.com/linweiii/BackdoorDM}{https://github.com/linweiii/BackdoorDM}.
\end{abstract}
    % 
\end{abstract}

\section{Introduction}
\label{sec:intro}

Foundational models (FMs)~\cite{zhang2024data, zhou2023comprehensive} have shown remarkable progress in the healthcare domain, enabling professional-like assessment of disease diagnosis, treatment decision-making, and monitoring~\cite{zhang2023text, wang2022medclip, lu2023mi-zero}. 
Examples include LLaVA-Med~\cite{li2023llava}, Med-PaLM Multimodal~\cite{tu2024towards}, and Med-Flamingo~\cite{moor2023med}, have demonstrated their capacity on question answering, medical image analysis, and report generation.
These studies follow a predominant top-down model development strategy that requires upstream developers to collect data and train models for downstream tasks. 
Consequently, the developed model capabilities are heavily dependent on the training data, limiting their generalization performance in diverse clinical scenarios. 
For instance, Med-Gemini~\cite{yang2024advancing} reveals promising general capabilities in report generation while it lags behind state-of-the-art (SoTA) models on classification tasks, especially for out-of-domain applications. 
This indicates that while the generalizability of the foundation model is promising, more solutions are expected to meet the various specialized clinical needs.

To address these challenges, multi-center data centralization becomes essential to enhance model capacity and robustness across varied clinical scenarios~\cite{rajpurkar2022ai}. 
Centralizing distributed data can significantly improve model training and inference performance.
However, the process of medical data storage, transfer, and aggregation among centers requires extra efforts to ensure data security and system interoperability~\cite{bradford2020international}.
Moreover, a growing concern for patient privacy makes large-scale multi-center data sharing particularly challenging. 
While efforts like federated learning~\cite{wen2023survey, li2020review} can achieve good model performance on local data, the need for synchronized system coordination presents significant challenges, as clients are unable to update asynchronously. This limitation greatly restricts the practical capability of such approaches.
As a result, without a flexible collaboration, medical community still struggles to fully utilize the isolated data and local computation resources for comprehensive medical AI model development. 
To address this dilemma, open-source platforms encourage public data sharing and knowledge integration~\cite{markiewicz2021openneuro, zenodo}.
However, these platforms focus solely on raw data sharing while seldom providing collaborative model training or cooperation between different institutions.
Recently, collaborative learning has emerged as a viable approach for enhancing multi-model robustness~\cite{boulemtafes2020review}. 
For instance, software-like model development~\cite{raffel2023building} mimics software engineering practices by introducing structured workflows, enabling merging, version control, and continuous model integration.
Under this design, model ability can be strengthened with incremental knowledge updates similar to the version updating in software development. 

Although collaborative learning provides a multi-model collaboration, two key challenges remain in the leakage of raw data during collaboration~\cite{huang2023lorahub} and the synchronization of multiple collaborators~\cite{mcmahan2017communication} in the medical AI community. It is still challenging to integrate decentralized, privacy-sensitive data across institutions, leading to under-utilized insights and fragmented knowledge sharing~\cite{kaissis2020secure, rajpurkar2022ai, abdullah2021ethics}.
 To address these challenges, inspired by the collaborative software development, we propose \textbf{Med}ical \textbf{Fo}undation Models Me\textbf{rg}ing (\textbf{MedForge}), a cooperative workflow enabling continuously community-driven foundation model (FM) development.
MedForge enables a lightweight manner for individual centers to share their knowledge among multiple centers, minimizing the burden of data transmission and integration while enhancing model robustness.
Meanwhile, MedForge facilitates asynchronous and flexible collaboration, allowing individual centers to continuously update and improve medical FMs without the need for real-time synchronization.
Similar to open-source software development, MedForge incrementally updates medical knowledge and follows a sustainable model development scheme. 
This key design emphasizes a bottom-up construction of a multi-task medical FM, allowing downstream users to collaboratively build, refine, and update the upstream model according to their local resources. Our major contributions of MedForge are as below: 
\begin{enumerate}
    \item[$\bullet$] We introduce a collaborative workflow to promote the merging scheme of open-source software development. Our proposed MedForge allows distributed clinical centers to asynchronously contribute to comprehensive medical model construction while reducing transmitting costs among centers and avoiding the leakage of raw data, thus enhancing the utilization of private resources in the healthcare system. 
    \item[$\bullet$] We propose two effective knowledge-merging strategies for the asynchronous branch contribution. The MedForge-Fusion strategy updates the plugin module parameters of the main model during the merging phase, whereas the MedForge-Mixture strategy integrates the output of the plugin module by memorizing each contributor's coefficient. These strategies make MedForge more flexible and versatile. MedForge-Fusion is friendly to implement, while the MedForge-Mixture offers better performance and robustness.
    \item[$\bullet$]  We comprehensively evaluate model merging strategies to accumulate medical knowledge among multiple branch plugin modules. MedForge yields superior performance on medical classification tasks compared to other collaborative baselines across multiple datasets. We demonstrate the robustness of MedForge by shuffling the task order and evaluating various configurations of plugin modules and dataset distillation methods.
\end{enumerate}



\section{Background}

\paragraph{Information Extraction} Information extraction (IE) is one of the most fundamental applications in natural language processing. IE systems take the user's requirement (e.g., defined by a label text, a question, or an instruction) and extract spans of several tokens from input texts. The two most frequent categories of IE targets are entity and relation, which structure many IE tasks, such as named entity recognition~\cite{conll2003}, relation extraction~\cite{conll2004}, event extraction~\citep{ace2005multilingual}, and others~\citep{srl-task,DBLP:conf/semeval/PontikiGPPAM14,aste-task}. A crucial challenge to modern IE systems is the growing number of IE targets (e.g., various label names) in the open world, which are scarce in annotation and require IE systems for quick transfer learning. Thus, many works have collected massive automated IE annotations to pre-train IE models~\cite{fewnerd,multinerd,TadNER,NuNER,metaie}, which shows benefits in transferring to low-resource IE targets.

\paragraph{Large Language Model} The biggest game-changer for natural language processing in all domains is the large language model (LLM)~\citep{tulu,llama-2,achiam2023gpt4,olmo,dubey2024llama3,team2024gemma}. Learning on trillions of tokens for pre-training and post-training, LLMs have shown surprisingly strong performance on all kinds of tasks~\citep{achiam2023gpt4}. Next token prediction, the paradigm behind the success of LLMs, supports exploiting every token in raw texts as the annotation to strengthen the model's capability. Consequently, many IE researchers have turned toward LLMs~\citep{llm4clinicalie,gpt-ner,llm4ie} to use them as strategic information extractors with planning~\citep{LLM_Plan,LLM_NestNER} and chain-of-thoughts~\citep{chain_of_thoughts,cot_re}.

\paragraph{Pre-training Paradigm: IE v.s. LLM} The rise of LLMs has challenged the meaningfulness of IE pre-training with an overwhelmingly larger number of annotations. The lagging of IE pre-training can be attributed to the relatively high format requirement for IE annotation like labels in Wikipedia links. This paper shows IE pre-training can take a free ride on LLM's NTP paradigm to unleash the power of massive pre-training.
\vspace{-5pt}
\section{Method}
\label{sec:method}
\begin{figure*}[t]
\begin{center}
\includegraphics[width=.85\linewidth]{fig_overview_v3.pdf}
\end{center}
\caption{
FastAtlas Overview: In each frame, we compute charts spanning fully or partially visible triangles (a), determine texture space bounding boxes for the visible portions of the view-space projections of each chart, and tightly pack these boxes into atlases (b, here $2K \times 2K$). We simultaneously bijectively parameterize and shade the charts into their atlas boxes, obtaining high quality texture space shading (c), and use this shading to render the shaded frames (d).}
\label{fig:overview}
\label{fig:alg_overview}
\end{figure*}

\section{Overview}
\label{sec:overview}
Our work has two core contributions: a real-time, GPU-based algorithm for tight packing of general parameterized charts into compact atlases; and a real-time TSS method that
utilizes this packing.  

\paragraph*{FastAtlas Packing.}
FastAtlas runs entirely on the GPU as a series of compute shaders. It takes the bounding boxes of parameterized charts as input, and packs them into an atlas (Fig~\ref{fig:overview}b, Sec.~\ref{sec:pack}). As such, the only input it requires are the dimensions of the bounding boxes.
Its outputs are deterministic; identical input charts are packed into identical atlases. This is critical for TSS and similar applications, as it ensures that consecutive frames taken from the same camera view have the same shading. Even minute shading differences across such frames can cause sampling jitter, leading to undesirable flicker \cite{baker2012rock}. 
While prior methods such as \cite{mueller2018shading,hladky2019tessellated,hladky2021snakebinning,Neff2022MSA} cap the dimensions of the charts that can be packed as-is for a given atlas size, and scale down all charts that exceed these dimensions, we scale all charts by the same factor, and do so only when strictly necessary to achieve packing success (Figs~\ref{fig:atlas},~\ref{fig:sas_issues}). 

\paragraph*{TSS using FastAtlas.}
Our end-to-end TSS atlas generation method combines the packing method above with a novel approach for computing seamless per-frame charts. 
We define our charts as the connected components of the visible surfaces in each frame (Fig.~\ref{fig:overview}a), and efficiently compute them using a parallel union-find algorithm (Sec.~\ref{sec:visible}). Since the boundaries of these charts coincide with the contours of the rendered surface, they are {\em invisible} to the viewer. This approach 
eliminates the artifacts caused by shading discontinuities along visible seams (Fig.~\ref{fig:seams}). 

\begin{parWithWrapFigure}
\begin{wrapfigure}{l}{.27\columnwidth}%
\includegraphics[width=\linewidth]{fig_inset_view_plane.pdf}%
\end{wrapfigure}
We bijectively parametrize the {\em visible portions} of our charts by projecting them to view space (inset). This maps a constant number of texels to each pixel in the final rendered output, evenly distributing residual undersampling error across all image pixels. While conceptually straightforward, efficiently parameterizing charts containing partially visible triangles using viewspace projection is non-trivial, as the visible portions may no longer be triangular (e.g. green triangle in the inset); applying naive projection to triangles with vertices behind the camera may produce ill-posed results. Clipping triangles before projection is both computationally expensive and significantly complicates downstream operations. We avoid explicit clipping by observing that all that is required for atlas packing is the dimensions of, potentially conservative, bounding boxes of these projected visible portions. We compute such bounding boxes without explicit chart clipping by adapting a conservative screen coverage estimator \shortcite{Blinn:CalculatingScreenCoverage} (Sec.~\ref{sec:box}). We then pack the computed boxes using FastAtlas. 
\end{parWithWrapFigure}

Finally, we shade the visible portion of each chart into its corresponding atlas bounding box (Fig~\ref{fig:overview}c). 
The resulting texture is then used during rasterization as a standard texture map (Fig. ~\ref{fig:overview}d). 
Our framework is compatible with all existing approaches for texture space shading, including forward shading, raytraced illumination, or deferred shading in texture space \cite{baker:2016}. In the examples shown, we use the standard forward shading based rendering pipeline included in the G3D Innovation Engine \cite{G3D17}, a commercial grade renderer.


Our goal is to increase the robustness of T2I models, particularly with rare or unseen concepts, which they struggle to generate. To do so, we investigate a retrieval-augmented generation approach, through which we dynamically select images that can provide the model with missing visual cues. Importantly, we focus on models that were not trained for RAG, and show that existing image conditioning tools can be leveraged to support RAG post-hoc.
As depicted in \cref{fig:overview}, given a text prompt and a T2I generative model, we start by generating an image with the given prompt. Then, we query a VLM with the image, and ask it to decide if the image matches the prompt. If it does not, we aim to retrieve images representing the concepts that are missing from the image, and provide them as additional context to the model to guide it toward better alignment with the prompt.
In the following sections, we describe our method by answering key questions:
(1) How do we know which images to retrieve? 
(2) How can we retrieve the required images? 
and (3) How can we use the retrieved images for unknown concept generation?
By answering these questions, we achieve our goal of generating new concepts that the model struggles to generate on its own.

\vspace{-3pt}
\subsection{Which images to retrieve?}
The amount of images we can pass to a model is limited, hence we need to decide which images to pass as references to guide the generation of a base model. As T2I models are already capable of generating many concepts successfully, an efficient strategy would be passing only concepts they struggle to generate as references, and not all the concepts in a prompt.
To find the challenging concepts,
we utilize a VLM and apply a step-by-step method, as depicted in the bottom part of \cref{fig:overview}. First, we generate an initial image with a T2I model. Then, we provide the VLM with the initial prompt and image, and ask it if they match. If not, we ask the VLM to identify missing concepts and
focus on content and style, since these are easy to convey through visual cues.
As demonstrated in \cref{tab:ablations}, empirical experiments show that image retrieval from detailed image captions yields better results than retrieval from brief, generic concept descriptions.
Therefore, after identifying the missing concepts, we ask the VLM to suggest detailed image captions for images that describe each of the concepts. 

\vspace{-4pt}
\subsubsection{Error Handling}
\label{subsec:err_hand}

The VLM may sometimes fail to identify the missing concepts in an image, and will respond that it is ``unable to respond''. In these rare cases, we allow up to 3 query repetitions, while increasing the query temperature in each repetition. Increasing the temperature allows for more diverse responses by encouraging the model to sample less probable words.
In most cases, using our suggested step-by-step method yields better results than retrieving images directly from the given prompt (see 
\cref{subsec:ablations}).
However, if the VLM still fails to identify the missing concepts after multiple attempts, we fall back to retrieving images directly from the prompt, as it usually means the VLM does not know what is the meaning of the prompt.

The used prompts can be found in \cref{app:prompts}.
Next, we turn to retrieve images based on the acquired image captions.

\vspace{-3pt}
\subsection{How to retrieve the required images?}

Given $n$ image captions, our goal is to retrieve the images that are most similar to these captions from a dataset. 
To retrieve images matching a given image caption, we compare the caption to all the images in the dataset using a text-image similarity metric and retrieve the top $k$ most similar images.
Text-to-image retrieval is an active research field~\cite{radford2021learning, zhai2023sigmoid, ray2024cola, vendrowinquire}, where no single method is perfect.
Retrieval is especially hard when the dataset does not contain an exact match to the query \cite{biswas2024efficient} or when the task is fine-grained retrieval, that depends on subtle details~\cite{wei2022fine}.
Hence, a common retrieval workflow is to first retrieve image candidates using pre-computed embeddings, and then re-rank the retrieved candidates using a different, often more expensive but accurate, method \cite{vendrowinquire}.
Following this workflow, we experimented with cosine similarity over different embeddings, and with multiple re-ranking methods of reference candidates.
Although re-ranking sometimes yields better results compared to simply using cosine similarity between CLIP~\cite{radford2021learning} embeddings, the difference was not significant in most of our experiments. Therefore, for simplicity, we use cosine similarity between CLIP embeddings as our similarity metric (see \cref{tab:sim_metrics}, \cref{subsec:ablations} for more details about our experiments with different similarity metrics).

\vspace{-3pt}
\subsection{How to use the retrieved images?}
Putting it all together, after retrieving relevant images, all that is left to do is to use them as context so they are beneficial for the model.
We experimented with two types of models; models that are trained to receive images as input in addition to text and have ICL capabilities (e.g., OmniGen~\cite{xiao2024omnigen}), and T2I models augmented with an image encoder in post-training (e.g., SDXL~\cite{podellsdxl} with IP-adapter~\cite{ye2023ip}).
As the first model type has ICL capabilities, we can supply the retrieved images as examples that it can learn from, by adjusting the original prompt.
Although the second model type lacks true ICL capabilities, it offers image-based control functionalities, which we can leverage for applying RAG over it with our method.
Hence, for both model types, we augment the input prompt to contain a reference of the retrieved images as examples.
Formally, given a prompt $p$, $n$ concepts, and $k$ compatible images for each concept, we use the following template to create a new prompt:
``According to these examples of 
$\mathord{<}c_1\mathord{>:<}img_{1,1}\mathord{>}, ... , \mathord{<}img_{1,k}\mathord{>}, ... , \mathord{<}c_n\mathord{>:<}img_{n,1}\mathord{>}, ... , $
$\mathord{<}img_{n,k}\mathord{>}$,
generate $\mathord{<}p\mathord{>}$'', 
where $c_i$ for $i\in{[1,n]}$ is a compatible image caption of the image $\mathord{<}img_{i,j}\mathord{>},  j\in{[1,k]}$. 

This prompt allows models to learn missing concepts from the images, guiding them to generate the required result. 

\textbf{Personalized Generation}: 
For models that support multiple input images, we can apply our method for personalized generation as well, to generate rare concept combinations with personal concepts. In this case, we use one image for personal content, and 1+ other reference images for missing concepts. For example, given an image of a specific cat, we can generate diverse images of it, ranging from a mug featuring the cat to a lego of it or atypical situations like the cat writing code or teaching a classroom of dogs (\cref{fig:personalization}).
\vspace{-2pt}
\begin{figure}[htp]
  \centering
   \includegraphics[width=\linewidth]{Assets/personalization.pdf}
   \caption{\textbf{Personalized generation example.}
   \emph{ImageRAG} can work in parallel with personalization methods and enhance their capabilities. For example, although OmniGen can generate images of a subject based on an image, it struggles to generate some concepts. Using references retrieved by our method, it can generate the required result.
}
   \label{fig:personalization}\vspace{-10pt}
\end{figure}
\begin{table*}[t!]
\centering
% \vspace{5pt}
\begin{small}
\begin{tabular}{l|c|c|c|c|c|c|c}
\toprule
\textbf{Method} & \textbf{Type} & \textbf{ToMi} & \textbf{BigToM} & \textbf{MMToM-QA} & \textbf{MuMA-ToM} & \textbf{Hi-ToM} & \textbf{All} \\
\midrule
SymbolicToM & Specific & \textbf{98.60} & - &  - & - & - & - \\
TimeToM & Specific & 87.80 & - &   - & - & - & - \\
% \textbf{96.00$^*$}
PercepToM & Specific & 82.90 & - & - & - & - & - \\
BIP-ALM & Specific & - & - & 76.70 & 33.90 & - & - \\
LIMP & Specific & - & - & - & 76.60 & - & - \\
\ours w/ Model Spec. & Specific & 88.80 & \textbf{86.75} & \textbf{79.83} & \textbf{84.00} & \textbf{74.00} & \textbf{82.68} \\
\midrule
Llama 3.1 70B & General & 72.00 & 77.83 & 43.83 & 55.78 & 35.00 & 47.41 \\
Gemini 2.0 Flash & General & 66.70 & 82.00 & 48.00 & 55.33 & 52.50 & 60.91\\
Gemini 2.0 Pro & General & 71.90 & 86.33 & 50.84 &  62.22 & 57.50 & 65.76 \\ 
GPT-4o & General & 77.00 & 82.42 & 44.00 & 63.55 & 50.00 & 63.39 \\
SimToM & General & 79.90 & 77.50 & 51.00 & 47.63 & 71.00 & 65.41\\ 
\ours & General & \textbf{88.30} & \textbf{86.92} & \textbf{75.50} & \textbf{81.44} & \textbf{72.50} & \textbf{80.93} \\
\bottomrule
\end{tabular}
\end{small}
\caption{Results of \ours and baselines on all benchmarks. There are two groups of methods: methods that require domain-specific knowledge (e.g., AutoToM w/ Model Spec.) or implementations (e.g., SymbolicToM) and methods that can be generally applied to any domain. ``-'' indicates that the domain-specific method is not applicable to the benchmark. The best results for each method type are highlighted in bold.}
\label{tab:results}
\vspace{-10pt}
\end{table*}



\section{Experiments}
\subsection{Experimental Settings}



We evaluated our method on multiple Theory of Mind benchmarks, including ToMi \citep{le2019revisiting}, BigToM \citep{gandhi2024understanding}, MMToM-QA \cite{jin2024mmtom}, MuMA-ToM \citep{shi2024muma}, and Hi-ToM \cite{he2023hi}. The diversity and complexity of these benchmarks pose significant reasoning challenges. For instance, MMToM-QA and MuMA-ToM incorporate both visual and textual input, while MuMA-ToM and Hi-ToM require higher-order inference. Additionally, MMToM-QA features exceptionally long contexts, and BigToM presents open-ended scenarios.



Besides the full \ours method, we additionally evaluated \ours given manually specified models (AutoToM w/ Model Spec.). 

We compared \ours against state-of-the-art baselines:
    \textbf{LLMs:} Llama 3.1 70B \citep{dubey2024llama}, Gemini 2.0 Flash, Gemini 2.0 Pro \cite{team2023gemini} and GPT-4o \cite{achiam2023gpt};
    
     \textbf{ToM prompting for LLMs:} SymbolicToM \cite{sclar2023minding}, SimToM \cite{wilf2023think}, TimeToM \cite{hou2024timetom}, and PercepToM \citep{jung2024perceptions};
 
  \textbf{Model-based inference:} BIP-ALM \cite{jin2024mmtom} and LIMP \cite{shi2024muma}.


For multimodal benchmarks, MMToM-QA and MuMA-ToM, we adopt the information fusion methods proposed by \citet{jin2024mmtom} and \citet{shi2024muma} to fuse information from visual and text inputs respectively. The fused information is in text form. We ensure that all methods use the same fused information as their input.


We use GPT-4o as the LLM backend for \ours and all ToM prompting and model-based inference baselines to ensure a fair comparison—except for TimeToM, which relies on GPT-4 and is not open-sourced.


\subsection{Results}
The main results are summarized in Table~\ref{tab:results}. Unlike \ours, many recent ToM baselines can only be applied to specific benchmarks. Among general methods, \ours achieves state-of-the-art results across all benchmarks. In particular, it outperforms its LLM backend, GPT-4o, by a large margin. This is because Bayesian inverse planning is more robust for inferring mental states given long contexts with complex environments and agent behavior. It is also more adept at recursive reasoning which is key to higher-order inference. Notably, \ours performs comparably to manually specified models, showing that automatic model discovery without domain knowledge is as effective as human-provided models. We provide additional results and qualitative examples in Appendix~\ref{sec:more_results}.


\subsection{Ablated Study}



\begin{figure}[t!]
  \centering
  \includegraphics[width=0.8\linewidth]{figures/comparison.pdf}
    \vspace{-10pt}
  \caption{Averaged performance and compute of the full \ours method (star) and the ablated methods (circles) on all benchmarks.}
  \label{fig:ablation}
  \vspace{-10pt}
\end{figure}


We evaluated the following variants of \ours for an ablation study: no hypothesis reduction (\textbf{w/o hypo. reduction}); always using POMDP (\textbf{w/ POMDP}); always using the initial model proposal without variable adjustment (\textbf{w/o variable adj.}); only considering the last timestep (\textbf{w/ last timestep}); and considering all timesteps without timestep adjustment (\textbf{w/ all timesteps}).

The results in Figure~\ref{fig:ablation} show that the full \ours method constructs a suitable BToM model, enabling rich ToM inferences while reducing compute. We analyze key model components below:

\textbf{Hypothesis reduction.}
Compared to the full method, \ours w/o hypo. reduction has a similar accuracy but consumes 53\% more tokens on average, demonstrating that hypothesis reduction optimizes efficiency without sacrificing performance.

\textbf{Variable adjustment.}
\ours dynamically identifies relevant variables for ToM inference, generalizing domain-specific BIP approaches to open-ended scenarios. Compared to its variant without variable adjustment, \ours improves performance with minimal additional compute. The variant that always uses POMDP performs well in scenarios aligned with the POMDP assumption (e.g., MMToM-QA) but generalizes poorly elsewhere and incurs much higher computational costs. %, leading to an 8.5% performance deficit.

\textbf{Timestep adjustment.}
By selecting relevant steps for inference, timestep adjustment enhances performance by focusing on essential information. In contrast, the variant using only the last timestep misses crucial details, significantly lowering performance. The variant incorporating all timesteps suffers from higher computational costs and reduced accuracy due to conditioning on unnecessary, potentially distracting information.



Full ablation results are provided in Appendix~\ref{sec:more_results_ablation}.

\section{A Comparative Study of Initial and Curated Code Review Datasets}
\label{sec:analysis}

\subsection{Impact of Curated Reviews on Comment Generation}
\label{subsec:model_data}
In this section, we investigate the impact of curated reviews on automating the comment generation process. By comparing models trained on both original and curated comments, we aim to assess whether the reformulated reviews lead to more efficient automation of the comment generation task.

\paragraph{\textbf{Model and data selection}}
To ensure a fair comparison, we selected a subset of $20,000$ comments from both the original and curated datasets, such that each original review comment \( r_i \) from the original dataset is paired with its reformulated counterpart \( r'_i \) in the curated dataset. This selection strategy guarantees that any observed differences in model performance can be attributed to the quality of the data (\ie curation process) rather than differences in review content or model hyperparameters. We further split each subset into $75\%$ for training and $25\%$ for evaluation.

We selected \textit{DeepSeek-6.7B-Instruct} \cite{deepseek-coder}, an LLM tailored for code-related tasks. 
Given a code change, the model was tasked with generating either the original or the curated comment. To ensure consistency, we trained two separate models, one for each dataset version, using identical configurations.


\paragraph{\textbf{Experimental setup}}

This experiment aims to determine whether curated review comments improve the ability of LLMs to generate accurate review comments. For each dataset version (original and curated), we provided the model with code changes as input and tasked it with generating the corresponding review comment.

Each model was trained independently using the same configuration to ensure that observed performance differences could be attributed solely to the dataset quality, not to model or hyperparameter variations. The training was conducted on four \emph{NVIDIA RTX A5000 GPUs}, each with \emph{24GB} of memory. We used a batch size of $4$ and trained each model for $5$ epochs. To enable efficient, low-resource fine-tuning, we employed Low-Rank Adaptation (LoRA) \cite{hu2021lora}, a parameter-efficient fine-tuning technique, configured with settings of $r = 16$, $\alpha = 32$, and $dropout = 0.05$. LoRA operates by decomposing the weight updates of a neural network into low-rank matrices, significantly reducing the number of parameters that require updating during fine-tuning \cite{hu2021lora}, thus enhancing the overall efficiency of the training process.
LoRA has been widely used in prior work to fine-tune LLMs for software engineering tasks~\cite{lu2023llama, weyssow2023exploring, hou2023large, silva2023repairllama}


To evaluate the two produced models’ performance, we used the BLEU score \cite{papineni2002bleu}, a standard metric, widely used in the literature, that measures the precision of n-grams in the generated text relative to the ground truth. BLEU is well-suited for assessing the correctness of generated comments, with higher scores indicating greater accuracy with real output.


\paragraph{\textbf{Results}}
The results are presented in \Table{tab:com_results}. For the model trained on the original dataset, we obtained a BLEU score of $7.71$, whereas the model trained on the curated dataset achieved a BLEU score of $11.26$. This improvement suggests that the reformulated, curated comments are likely easier for the model to learn, potentially due to their enhanced clarity and structure.

These findings suggest that curated review comments provide clearer, more direct guidance, enabling the model to better capture the intended message and improving the quality of generated comments. The higher BLEU score with the curated dataset indicates that the curation process enhances the ability of models to generalize and learn producing more accurate review comments, thus facilitating a more efficient automation of the comment generation process.

% \begin{table}[!t]
% \centering
% \caption{Comparison of BLEU scores for DeepSeek-Coder-6.7B-Instruct trained on original and curated comments.}
% \label{tab:com_results}
% \begin{tabular}{@{}lcc@{}}
% \toprule
% \textbf{Dataset Version}       & \textbf{BLEU} \\ 
% \midrule
% Original Comments               & $7.71$ \\
% Curated Comments                & \textbf{$11.26$} \\ 
% \bottomrule
% \end{tabular}
% \vspace{-1.5em}
% \end{table}

\begin{table}[!t]
\centering
\caption{Comparison of BLEU scores for DeepSeek-Coder-6.7B-Instruct trained on original and curated comments.}
\label{tab:com_results}
\begin{tabular}{@{}lcc@{}}
\toprule
& \textbf{Original Comments} & \textbf{Curated Comments} \\ 
\midrule
\textbf{BLEU} & $7.71$ & $\textbf{11.26}$ \\ 
\bottomrule
\end{tabular}
\vspace{-1em}
\end{table}



The code change, shown in \Table{tab:comm_gen_examples}, is an example picked from our test set. The real comment in the original dataset lacks clarity, as it does not provide the reason behind the question. The model trained on the original dataset generates an incorrect review comment. In the curated dataset, the reformulated version of the real review comment is clearer, though less concise. The model trained on the curated dataset generates a comment that is accurate, closely matching the real and reformulated comments, and is even more concise than the reformulated version.



\begin{table*}[!htbp]
\centering
\caption{Example of real, reformulated, and generated review comments from the initial and curated datasets.}
\label{tab:comm_gen_examples}
\begin{tabularx}{1\linewidth}{X}
\toprule

\multicolumn{1}{c}{\textbf{Code change}} \\\midrule
\makebox[\linewidth]{\includegraphics[width=0.6\linewidth]{figures/code_snippet3.png}} 
\\\midrule
\multicolumn{1}{c}{\textbf{Original dataset}}\\\midrule

\underline{\textbf{Real review comment}} \texttt{why this import is required ?} \\

\underline{\textbf{Generated review comment}} \texttt{I think we should use `javax.servlet.http.Part`instead of \newline`org.apache.commons.fileupload.FileItem`.} \\ \midrule

\multicolumn{1}{c}{\textbf{Curated dataset}} \\\midrule

\underline{\textbf{Reformulated review comment}} \texttt{This import is not immediately clear in its necessity. Could you provide context\newline or explain why `javax.servlet.http.Part` is required for this functionality?} \\

\underline{\textbf{Generated review comment}} \texttt{Consider removing the import statement for `javax.servlet.http.Part` as it's not\newline used in the code.} \\ \bottomrule

\end{tabularx}
\vspace{-.5em}
\end{table*}





\begin{center}
\begin{tcolorbox}[colframe=orange!75!black, colback=orange!5!white, title=\textbf{Answer to RQ3}]
Curated review comments are more effective for training models on the comment generation task, as evidenced by higher BLEU ($7.71 \rightarrow 11.26$).
\end{tcolorbox}
\end{center}


\subsection{A Comparative Analysis on the Usefulness of Curated Comments for Code Refinement}

In this section, we aim to evaluate the usefulness of the curated comments compared to the original comments for code refinement.
We conduct a comparative study to assess which version of the comment—original or curated—guides the code refinement model to generate more accurate code changes.


\paragraph{\textbf{Model and data selection}}
% We select \textit{DeepSeek-Coder-6.7B-Instruct} \cite{deepseek-coder} to serve as our code refinement model. For evaluation, we randomly select a subset of our large datasets, choosing 20,000 samples from the full set of 176,613 for each version of the dataset (\ie original and curated datasets). To ensure a fair comparison, we maintain consistency in the selection by pairing each original review comment \( r_i \) from the original dataset with its corresponding reformulated comment \( r'_i \) in the cated dataset, where \( i \in [1, 20,000] \). Thus, each review pair \( (r_i, r'_i) \) represents the same code review context, allowing us to directly evaluate the impact of the reformulated comments compared to the original. For consistency, we use the same \textit{DeepSeek-Coder-6.7B-Instruct} model configuration on both datasets to ensure that any observed differences are attributable to the quality of the input data (\ie review comments) rather than model variations. 

We use \textit{DeepSeek-Coder-6.7B-Instruct} \cite{deepseek-coder} as a code refinement model, applying it to the same selected subset of $20,000$ samples from each dataset version (original and curated), as in the previous experiment on comment generation, as explained in \Sect{subsec:model_data}. To ensure a fair comparison, each original review comment \( r_i \) is paired with its reformulated counterpart \( r'_i \), preserving the same review context across both datasets. The model configurations remain identical for both datasets, to ensure that any observed differences are attributable to the quality of the input data (\ie review comments) rather than model variations. 




\paragraph{\textbf{Experimental setup}}
For each dataset version, we provided the code refinement model with the original code diff, the old file, and the review comment (either original or curated) as context and prompted it to generate a code diff that accurately implements the specified changes.

We used the LLM directly for inference, as its extensive training on diverse code-related tasks equips it with the capabilities needed to effectively automate the code refinement task. The experiment was run twice, once with the original comments and once with the curated comments.

To evaluate the accuracy of the generated code diffs, we employed two evaluation metrics: 
\begin{itemize} 
    \item \textbf{CodeBLEU}: This metric measures the similarity of the generated code diff to the expected code diff, combining n-gram match, weighted n-gram match, AST match, and data-flow match scores \cite{ren2020codebleu}. 
    \item \textbf{Exact Match (EM)}: This metric calculates the number of generated code diffs that exactly match the expected code diff. 
\end{itemize}

Each experiment was conducted using identical model configurations for both dataset versions to ensure that any observed performance differences could be attributed solely to the quality of the review comments rather than model parameter variations.



\paragraph{\textbf{Results}}
The results, presented in \Table{tab:ref_results}, reveal substantial differences between the two dataset versions. Using the \textbf{original comments}, the model achieved a \emph{CodeBLEU} score of $0.36$ and an \emph{EM} of $408$. When utilizing the \textbf{curated comments}, the model performance improved significantly, reaching a \emph{CodeBLEU} score of $0.44$ and an \emph{EM} of $445$. These findings suggest that the curated comments offer more precise guidance, enabling the model to generate more accurate code changes that are closer to the ground truth.

This result indicates that the curated comments not only clarify the intended modifications but also reduce ambiguities in the model interpretation of the review instructions. The curated comments likely contain enhanced phrasing and structure that aid the model in better understanding and implementing the required code changes, thus improving the overall quality of the generated code diff.

\begin{table}[h!]
\centering
\caption{Comparison of DeepSeek-Coder-6.7B-Instruct's effectiveness for code refinement using the original and curated review comments.}
\label{tab:ref_results}
\begin{tabular}{@{}lcc@{}}
\toprule
\textbf{Dataset Version}       & \textbf{CodeBLEU} & \textbf{Exact Match} \\ \midrule
Original Comments & $0.36$ & $408$\\
Curated Comments  & $\textbf{0.44}$ & $\textbf{445}$\\ 
\bottomrule
\end{tabular}
\vspace{-.5em}
\end{table}



This experiment showed that curated review comments are more useful for the next task, as they provide better guidance for the code refinement task, leading to more accurate code changes. This is reflected by significant improvements in both \emph{CodeBLEU} and \emph{EM}.




\begin{center}
\begin{tcolorbox}[colframe=black!75!black, colback=black!5!white, title=\textbf{Answer to RQ4}]
    Curated comments demonstrate superior utility for guiding code refinement models, leading to notably higher CodeBLEU ($0.36 \rightarrow 0.44$) and Exact Match scores ($408 \rightarrow 445$).
\end{tcolorbox}
\end{center}







\begin{comment}
    ================================================== 1 ==================================================


 INIT - Original comment:  This and the one below. Please, rename to `data range` also.
 INIT - Generated comment  I think this should be "dynamic range"

+=+=+=+=+=+=+=+=+=+=+=+=+=+=+=+=+=+=+=+=+=+=+=+=+=+=+=+=+=+=+=+=+=+=+=+=+=+=+=+=+=+=+=+=+=+=+=+=+=+=
 CUR - Reformulated comment   Consider renaming "value range" to "data range" for consistency.
 CUR - Generated comment  Consider renaming the parameter to `value_range` for clarity and consistency.



 ================================================== 27 ==================================================

@@ -2791,81 +2791,80 @@ static void Main()
                 Diagnostic(ErrorCode.WRN_UnreferencedVarAssg, "message").WithArguments("message").WithLocation(8, 22));
         \}

-        [WorkItem(863401, "DevDiv/Personal")]
         [Fact]
-        public void CS1101ERR_BadRefWithThis()
+        public void BadParameterModifeirs_ThisWithRef()


+=+=+=+=+=+=+=+=+=+=+=+=+=+=+=+=+=+=+=+=+=+=+=+=+=+=+=+=+=+=+=+=+=+=+=+=+=+=+=+=+=+=+=+=+=+=+=+=+=+=


 INIT - Original comment:  Typo: `Modifiers` #Resolved
 INIT - Generated comment  ```suggestion public void BadParameterModifiers_ThisWithRef() ```

+=+=+=+=+=+=+=+=+=+=+=+=+=+=+=+=+=+=+=+=+=+=+=+=+=+=+=+=+=+=+=+=+=+=+=+=+=+=+=+=+=+=+=+=+=+=+=+=+=+=
 CUR - Reformulated comment   The method name `BadParameterModifeirs_ThisWithRef` contains a typo. Consider renaming it to `BadParameterModifiers_ThisWithRef` for consistency.
 CUR - Generated comment  Consider renaming the test method to `BadParameterModifiers_ThisWithRef` for consistency and clarity.



 ================================================== 6 ==================================================


 INIT - Original comment:  It should be without underscore, I think
 INIT - Generated comment  I think we should rename this to `IndexOf`.

+=+=+=+=+=+=+=+=+=+=+=+=+=+=+=+=+=+=+=+=+=+=+=+=+=+=+=+=+=+=+=+=+=+=+=+=+=+=+=+=+=+=+=+=+=+=+=+=+=+=
 CUR - Reformulated comment   Consider renaming the method to remove the underscore for consistency.
 CUR - Generated comment  Consider renaming the method to `IndexOfNonVectorized` for consistency and clarity.



================================================== 4845 ==================================================

@@ -75,6 +75,8 @@

 import com.google.common.annotations.VisibleForTesting;

+import javax.servlet.http.Part;


+=+=+=+=+=+=+=+=+=+=+=+=+=+=+=+=+=+=+=+=+=+=+=+=+=+=+=+=+=+=+=+=+=+=+=+=+=+=+=+=+=+=+=+=+=+=+=+=+=+=


 INIT - Original comment:  why this import is required ?
 INIT - Generated comment  I think we should use `javax.servlet.http.Part` instead of `org.apache.commons.fileupload.FileItem`.

+=+=+=+=+=+=+=+=+=+=+=+=+=+=+=+=+=+=+=+=+=+=+=+=+=+=+=+=+=+=+=+=+=+=+=+=+=+=+=+=+=+=+=+=+=+=+=+=+=+=
 CUR - Reformulated comment   "This import is not immediately clear in its necessity. Could you provide context or explain why `javax.servlet.http.Part` is required for this functionality?"
 CUR - Generated comment  Consider removing the import statement for `javax.servlet.http.Part` as it's not used in the code.


================================================== 27 ==================================================

@@ -2791,81 +2791,80 @@ static void Main()
                 Diagnostic(ErrorCode.WRN_UnreferencedVarAssg, "message").WithArguments("message").WithLocation(8, 22));
         \}

-        [WorkItem(863401, "DevDiv/Personal")]
         [Fact]
-        public void CS1101ERR_BadRefWithThis()
+        public void BadParameterModifeirs_ThisWithRef()


+=+=+=+=+=+=+=+=+=+=+=+=+=+=+=+=+=+=+=+=+=+=+=+=+=+=+=+=+=+=+=+=+=+=+=+=+=+=+=+=+=+=+=+=+=+=+=+=+=+=


 INIT - Original comment:  Typo: `Modifiers` #Resolved
 INIT - Generated comment  ```suggestion public void BadParameterModifiers_ThisWithRef() ```

+=+=+=+=+=+=+=+=+=+=+=+=+=+=+=+=+=+=+=+=+=+=+=+=+=+=+=+=+=+=+=+=+=+=+=+=+=+=+=+=+=+=+=+=+=+=+=+=+=+=
 CUR - Reformulated comment   The method name `BadParameterModifeirs_ThisWithRef` contains a typo. Consider renaming it to `BadParameterModifiers_ThisWithRef` for consistency.
 CUR - Generated comment  Consider renaming the test method to `BadParameterModifiers_ThisWithRef` for consistency and clarity.

================================================== 59 ==================================================

@@ -7,6 +7,7 @@
 to leave out days on which there is no data, i.e. weekends.  The example
 below shows how to use an 'index formatter' to achieve the desired plot
 """
+from __future__ import print_function


+=+=+=+=+=+=+=+=+=+=+=+=+=+=+=+=+=+=+=+=+=+=+=+=+=+=+=+=+=+=+=+=+=+=+=+=+=+=+=+=+=+=+=+=+=+=+=+=+=+=


 INIT - Original comment:  This should not be added now since we're Python 3 only on master.
 INIT - Generated comment  ```suggestion from __future__ import print_function, division ```

+=+=+=+=+=+=+=+=+=+=+=+=+=+=+=+=+=+=+=+=+=+=+=+=+=+=+=+=+=+=+=+=+=+=+=+=+=+=+=+=+=+=+=+=+=+=+=+=+=+=
 CUR - Reformulated comment   "This import is unnecessary since we're only supporting Python 3 on master. Consider removing it."
 CUR - Generated comment  Consider using `print` instead of `print_function` for compatibility with Python 2.



 
\end{comment}







\section{Conclusion}
In this work, we represent RLEdit, a hypernetwork-based editing method designed for lifelong editing. RLEdit formulates lifelong editing as an RL task, employing an offline update approach to enhance the model's retention of entire knowledge sequences. Additionally, RLEdit proposes the use of memory backtracking to review previously edited knowledge and applies regularization to mitigate knowledge forgetting over long sequences. Through extensive testing on several LLMs across multiple datasets, our experimental results demonstrate that RLEdit significantly outperforms existing baseline methods in lifelong editing tasks, showing superior performance in editing effectiveness, editing efficiency, and general capability preservation.
\section*{Limitations}

While \our validates the strength of NTE to take a free ride with LLM resources, our scope can be extended to several topics out of the main claims.

\paragraph{Label Embedding} Some IE paradigms (e.g., original NuNER) learns label embeddings to efficiently label the extracted spans. As \our imitates NTP to perform NTE, its IE process requires enumerating the label names similar as the generative IE using LLMs. Matching label embedding has its efficiency advantage while generative IE allows the label texts to interact with the context, resulting in potentially better performance. \our follows the generative IE paradigm to pursue better performance based on the established success of LLMs. However, future efforted can be devoted into a label embedding version of \our, which takes the context as the label text to boost the IE efficiency.

\paragraph{Data Source} The C4 corpus for raw text features broad coverage. However, recent progress in LLMs shows that specific sources of pre-training data (e.g., textbooks) benefit certain skills of LLMs, such as math. This paper only discusses C4 to avoid the IE performance improvement attributed to a specific data source. Future works can extend our scope to compare the effect of all kinds of resources in pre-training, which might find certain resources are superior in IE pre-training using NTE.

\paragraph{Backbone Variants} The current scopes is designed to justify the benefit of NTE in gathering massive IE pre-training data. Thus, the comparison is biased to data quality rather than backbone models. Further exploration in backbone models include the scaling law in model size, multilingual backbone, and model architectures.

% Bibliography entries for the entire Anthology, followed by custom entries
%\bibliography{anthology,custom}
% Custom bibliography entries only
\bibliography{custom}

\clearpage

\appendix

\section{\our v.s. LLMs}
\label{apdx:vs_llm}

\begin{figure}
    \centering
    \includegraphics[width=\linewidth]{figures/coevolution_sft_llm_radar.pdf}
    \caption{The performance comparison between \our and LLMs on few-shot IE performance.}
    \label{fig:llm_radar}
    \vspace{-5mm}
\end{figure}

We extend the comparison to \our versus LLMs. We select \texttt{LLaMA-3-8B-TuluV3} and \texttt{GPT-4o} to represent the fine-tunable open-source LLMs and API-based close-source LLMs. For \texttt{LLaMA-3-8B-TuluV3}, we fine-tune the LLM with the same templated data as our \our. For both LLMs, we evaluate their in-context learning IE ability based on the few shots.

We present the experiment result in Figure~\ref{fig:llm_radar}, which demonstrate that \our can outperform even fine-tuned 8B LLMs. This implicates the superior learning efficiency of NTE over NTP on IE tasks. The ICL performance of LLM significantly lags behind the fine-tuned one, restraining the performance of close-source LLMs. Finally, Rainbow \our validates itself again as the strongest few-shot IE learner even when LLMs are considered.

\paragraph{Efficiency} The time efficiency of \our is significantly higher than LLMs thanks to the specialized learning paradigm for IE. Taking NER as an example, \our is around 20$\times$ faster than \texttt{LLaMA-3-8B-TuluV3}. When the LLM is using ICL, the efficiency advantage becomes more than $50\times$, demonstrating the superior efficiency of \our. 

\section{Templates and Hyperparameters}
\label{apdx:detail}

\paragraph{Task Templates} are included in Table~\ref{tab:template}, which are used to fine-tune NTE and NTP models like \our and LLaMA on IE tasks.


\begin{table}
\centering
\small
\scalebox{0.9}{
\begin{tabular}{p{1.5cm}p{5.2cm}}
\toprule
Target & Template\\
\midrule
Entity & \textbf{User:} [Context] Question: What is the [Label] mentioned? \textbf{Assistant:} Answer: The [Label] is \\
\midrule
Relation (Kill) & \textbf{User:} [Context] Question: Who does [Entity] kill? \textbf{Assistant:} Answer: [Entity] kills \\
\midrule
Relation (Live) & \textbf{User:} [Context] Question: Where does [Entity] live in? \textbf{Assistant:} Answer: [Entity] lives in \\
\midrule
Relation (Work) & \textbf{User:} [Context] Question: Who does [Entity] work for? \textbf{Assistant:} Answer: [Entity] works for \\
\midrule
Relation (Located) & \textbf{User:} [Context] Question: Where is [Entity] located in? \textbf{Assistant:} Answer: [Entity] is located in \\
\midrule
Relation (Based) & \textbf{User:} [Context] Question: Where is [Entity] based in? \textbf{Assistant:} Answer: [Entity] is based in \\
\midrule
Relation (Adverse) & \textbf{User:} [Context] Question: What is the adverse effect of [Entity]? \textbf{Assistant:} Answer: The adverse effect of [Entity] is \\
\midrule
Query & \textbf{User:} [Context] Question: [Question] \textbf{Assistant:} Answer: \\
\midrule
Instruction (Entity) & \textbf{User:} [Context] Question: What is the [Label] mentioned? ([Instruction]) \textbf{Assistant:} Answer: The [Label] is \\
\midrule
Instruction (Query) & \textbf{User:} [Context] Question: [Question] ([Instruction]) \textbf{Assistant:} Answer: \\
\bottomrule
\end{tabular}
}
\caption{The templates used in our experiments for different tasks.} 
\vspace{-5mm}
\label{tab:template}
\end{table}

\paragraph{Hyperparameter} All models are fully fine-tuned except for \texttt{LLaMA-3-8B-TuluV3}, which exhibits a poor performance without LoRA~\citep{lora}. We use a $128$-dimension LoRA for \texttt{LLaMA-3-8B-TuluV3}. All fine-tuning uses AdamW~\citep{AdamW} as the optimizer, learning rate initialized as $1\times 10^{-5}$ to fully fine-tune RoBERTa and OPT, and $2\times 10^{-4}$ to fine-tune the LoRA. The batch size is set to $64$ for all fine-tuning. 

\section{Benchmark Details}
\label{apdx:itie}

All results in the main experiments are an average of $3$ runs on different subsets of a few shots. MRC results are evaluated on the validation split as in previous works. Instruction-following IE only focuses on the modified entity types like organization and miscellaneous.

\paragraph{Relation Extraction} gives the ground-truth entities to extract related entities. We don't run end-to-end experiments to avoid mixing entity and relation extraction abilities.

\paragraph{Duplicates} When an entity is extracted as multiple types in NER, we keep all of them because modern generative IE models (e.g., LLM) allow such features to fit into a broader usage. For instance, an LLM would say ``Kobe Bryant'' to be both a ``person'' and a ``basketball player''. For MRC, when multiple answers are extracted, we will select the answer that appears the most.

\paragraph{SQuAD-V2} is a special MRC dataset that contains unanswerable questions. We follow the initial evaluation to assign $1.0$ F1 score to abstain for these questions and $0.0$ F1 score for any answer. Adaptive training for SQuAD-V2 contains extra $32$-shot unanswerable questions.

\paragraph{Disambiguation} The $3$ instructions used for disambiguation are presented in Table~\ref{tab:instruction}. We use the follow template to prompt \texttt{GPT-4o} for filtering.

\textit{[Instruction] Does ``[Entity]'' in ``[Context]'' satisfy the definition above? Answer ``yes'' or ``no'' only.}

We manually check the filtering quality of $50$ random cases for each instruction, and find a high filtering quality of $134/150=89.33\%$.

\paragraph{Miscellaneous} For CoNLL2003, as there is already a miscellaneous type, we manually write an instruction to define the scope of miscellaneous. For MIT-Restaurant dataset, we combine ``amenity'', ``hours'', and ``price'' entity types. For MIT-Movie dataset, we combine ``actor'', ``soundtrack'', and ``quote'' entity types. Then we simply collect those types of entities to build the miscellaneous type for the benchmark. In the instruction, we include negations of miscellaneous as distractors to increase the difficulty in instruction-following.


\begin{table}
\centering
\small
\scalebox{0.9}{
\begin{tabular}{p{0.8cm}p{1.5cm}p{4cm}}
\toprule
Task & Dataset & Instruction\\
\midrule
Disamb. & CoNLL2003 & The organization entity must be a subject of any active action in the context. \\
\cmidrule(lr){2-3}
& BioBLP2004 & The provided context must contain some descriptive information about the protein. \\
\cmidrule(lr){2-3}
& Restaurant & The rating should describe a food or drink mentioned in the sentence. \\
\midrule
Prefer. & SQuAD & Give the longest answer \\
& & Give the shortest answer \\
& & Give a concise answer \\
\midrule
Misc. & CoNLL2003 & Miscellaneous includes events, nationalities and products but not person, location or organization. \\
\cmidrule(lr){2-3}
 & Restaurant & Miscellaneous includes amenity, hours and price but not rating, dish, or location. \\
\cmidrule(lr){2-3}
  & Movie & Miscellaneous includes actor, soundtrack and quote but not director, opinion, or plot. \\
\bottomrule
\end{tabular}
}
\caption{The specific instructions used for instruction-following IE.} 
\vspace{-5mm}
\label{tab:instruction}
\end{table}

The specific instructions used for instruction-following IE are listed in Table~\ref{tab:instruction}.



\section{Adaptive Supervision Scaling}
\label{apdx:adaptive_scaling}

\begin{figure}
    \centering
    \includegraphics[width=\linewidth]{figures/adaptive_scaling.pdf}
    \vspace{-8mm}
    \caption{The scaling-up performance on adaptive supervision from CoNLL2003 of pre-trained IE models.}
    \label{fig:adaptive_scaling}
\end{figure}

In the application for IE, it's common to scale up the adaptive supervision (few-shot instances) to strengthen the model's IE ability. We plot such an example for CoNLL2003 in Figure~\ref{fig:adaptive_scaling} for transferring learning with different scales of supervision, from $5$-shot to $320$-shot. For comparison, we include the strongest NER baseline, NuNER, from the main experiment.

The results demonstrate that \our can scale up similarly as NuNER, the in-domain transfer of NuNER shows its advantage under very weak supervision but is surpassed by \our when the adaptive supervision is enough for domain understanding. Finally, Rainbow \our consistently show advantages under different adaptive supervision scales.

\section{Robustness to Verbalization}


\begin{table}
\centering
\small
\scalebox{0.9}{
\begin{tabular}{p{1.0cm}p{6.cm}}
\toprule
Rephrase & New Template/Label \\
\midrule
Template & \textbf{User:} [Context] Instruction: Extract [Label] from the text above. \textbf{Assistant:} [Label]: \\
\cmidrule(lr){2-2}
& \textbf{User:} List all [Label] entities: [Context] \textbf{Assistant:} Here are [Label] entities: 1. \\
\midrule
Label & (CoNLL2003) Person $\rightarrow$ Name\\
\cmidrule(lr){2-2}
 & (BioBLP2004) DNA $\rightarrow$ Deoxyribonucleic acid\\
\cmidrule(lr){2-2}
 & (Restaurant) Rating $\rightarrow$ Recommendation\\
\cmidrule(lr){2-2}
 & (Movie) Genre $\rightarrow$ Category\\
\bottomrule
\end{tabular}
}
\caption{The template/label variants used for robustness testing.} 
\vspace{-5mm}
\label{tab:variant}
\end{table}

As \our relies on prompts to perform different tasks. Its robustness to different verbalization of tasks and labels needs more emphasis. We select NER as an example and rephrase templates and labels in our experiments, which are listed in Table~\ref{tab:variant}. We rerun the experiments with these modifications and find the NER performance is not significantly (defined as $p < 0.05$ in significance testing) different from the initial results. This indicates \our to be robustness to different verbalization styles.

\end{document}
