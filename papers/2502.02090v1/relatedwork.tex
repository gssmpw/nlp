\section{Related work}
%This article descends from two unpublished manuscripts~\cite{NagyP24,WronaQuasiDirected}.  Indeed, \Cref{thm:kmain} is a reformulation of
%the main result in~\cite{NagyP24} with the notable differences  that  quasi near-unanimity operations  are  replaced with a chain of quasi J\'{o}nsson operations,  and that the main theorem in~\cite{NagyP24} concerns solely $k$-neoliberal structures with $k \geq 3$ (hence excluding the certainly relevant case of graphs). The new ideas that helped generalize the result come from~\cite{WronaQuasiDirected}, which provides a tractability result for quasi directed J\'{o}nsson chains in the % syntactically 
%restricted setting of first-order expansions of symmetric multi-graphs with free amalgamation. 

\Cref{thm:kmain} provides both a tractability result and a 
\mic{bound on the amount of local consistency required to guarantee a solution to an instance.} \cht{Moreover, this bound is easily seen to be optimal for any structure under consideration.} %\michal{why we do not have (4,4) for random 4-uniform hypergraph?} 
%characterization of bounded width of structures preserved by chains of quasi J\'{o}nsson operations. 
\mic{As mentioned above, the only previous general results providing tractability within the Bodirsky-Pinsker conjecture by imposing conditions on polymorphisms were the one on quasi near-unanimity operations~\cite{BodirskyDalmau} (which is a straightforward generalization from the finite) and the one on canonical pseudo-Siggers operations~\cite{ReductsUnary} (which is not purely algebraic in the sense that it requires topology to be stated). %\michal{After directed quasi Jonsson I have received an email from Manuel saying that canonicity is a property of a topological clone so I think that according to him it is an algebraic condition}\micc{Topology is not algebraic; I added a comment above}. 
We thus perceive~\Cref{thm:kmain} as the first non-trivial result on the tractability of infinite-domain CSPs originating from purely algebraic (height~1)  conditions.}
%\chmichal{There is, however, one unpublished result~\cite{WronaQuasiDirected} which gives tractability for quasi directed Jonsson chains in a syntactically restricted setting of first-order expansions of symmetric multi-graphs with free amalgamation. 
%\Cref{thm:kmain} is much more general than the mentioned result. %\michal{I do not know actually if I should or should not DUE TO REQUIREMENTS ON ANONYMITY cite our papers from which this one descends}}

Building on results in~\cite{FederVardi,DalmauPearson,BartoKozikBoundedWidth}, Libor Barto has shown in~\cite{BartoCollapse} that the  relational width of any finite structure is either $(1,1)$ or $(2,3)$. We do not have a full understanding of this phenomenon over infinite structures. There is a plethora of interesting results, though.    
In~\cite{SmoothApproximations}, bounded width was characterized for CSP templates $\sA$ over several ``ground structures'' $\sB$ (in which they are first-order definable), using \emph{weak near-unanimity} identities satisfied by \emph{canonical polymorphisms}; this amounts to assuming weaker identities than quasi near-unanimity identities, but the additional (non-algebraic) property of canonicity.
%under the assumption of their satisfaction by special polymorphisms (which cannot be used in this way over other ground structures such as the order of the rationals). 
The conditions given there were applied  in~\cite{SymmetriesEnough} to obtain a general  upper bound on the relational width of CSP templates satisfying them, and the bound was shown to be optimal for many templates. The bound on the relational width in~\Cref{thm:kmain} coincides with the bound proven in~\cite{SymmetriesEnough} for CSP templates which posses canonical \emph{pseudo-totally symmetric polymorphisms} of all arities $n\geq 3$. %\tomas{Shouldn't we remark somewhere that our bound is optimal for all structures under consideration (i.e. they cannot have lower width for trivial reasons)?}\micc{Yes please go ahead} 
The results in~\cite{SymmetriesEnough} have been extended to CSP templates over finitely bounded homogeneous $k$-uniform hypergraphs for  $k\geq 3$ in~\cite{hypergraphs}.
\mic{The very first bounds as in}
%of the form of
~\Cref{thm:kmain} over infinite structures within Bodirsky-Pinsker conjecture were, however, obtained in~\cite{Wrona:2020a,Wrona:2020b}, where a similar upper bound was given for first-order expansions $\sA$ of certain binary structures $\sB$  under the assumption that $\sA$ is invariant under  a quasi near-unanimity operation.% \chmichal{The latter result is generalized in~\cite{NagyP24}
%to $k$-neoliberal structurs of $k \geq 3$ with finite duality.}
%Bounded width of first-order expansions  of certain binary structures preserved by quasi near-unanimity operations (or equivalently with bounded strict width)
%have been studied in~\cite{Wrona:2020a,Wrona:2020b}, 

%For finite-domain CSP templates, different algebraic conditions which are stronger than the ones characterizing bounded width and weaker than the ones characterizing bounded strict width have been  considered~\cite{Jonssonterms}. These conditions can also lifted to the $\omega$-categorical case similarly as in the case of near-unanimity polymorphisms -- it is therefore natural to ask if a similar bound as in~\Cref{thm:kmain} can be obtained  for templates satisfying them.