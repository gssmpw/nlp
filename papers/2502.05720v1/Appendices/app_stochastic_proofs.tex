\section{Complements to Section \ref{sec: stochastic predictions}}\label{app: sec3}

Recall in this section the notations $\Price\sim\dprice$ for the maximum price, and $\Pred\sim\dpred$ for the prediction. When considered, their coupling is denoted $\truecoupling$. 

\subsection{Complements on Section \ref{subsec: instances}}

\mypar{Stochastic predictions, deterministic prices.}

\DeterministicMarginalBounds*

\begin{proof}
    This is obtained by direct instantiation of \cref{thm: [Stochastic bounds] bound in expectation using true coupling}. In particular, for $\dprice=\delta_{\price}$ the second term of Eq.~\eqref{eq: PR bound in Expectation of ratio using coupling} can be substituted into with
    \begin{align}
        \Eb[\Price\error(\price,\pred)^{\smth}]&=\Eb[\Price\error(\Price,\Pred)^{\smth}\vert\Price=\price] \price=\int \min\left\{\frac{\pred}{\price}\,,\,\frac{\price}{\pred} \right\}^{\smth}\de\dpred(\pred)\,.\notag
    \end{align}
    Computing this integral explicitly reveals it to be $\detpricefun(\price)$.
\end{proof}

Through this section, we will use the following identity
\begin{align}
\detpricefun(\price)&=\int_\lb^{\price}\left(\frac{\pred}{\price}\right)^{\smth}\de \dpred(\pred) + \int_{\price}^\ub \left(\frac{\price}{\pred}\right)^{\smth}\de \dpred(\pred)\,.
\label{eq: [deterministic marginal] detpredfun form 2}
\end{align}
which is easily derived from the proof above.

Inspection of Eq.~\eqref{eq: [deterministic marginal] detpredfun form 2} reveals that $\detpricefun$ contains two different regimes (above and below $\price$). The \emph{mirrored} coefficients of $(\price,\pred)$ in each term reflect the inherent asymmetry of a threshold algorithm: performance is highly sensitive to whether $p^*\le\thresh_\rob^1(Y)$, which transfers to Eq.~\eqref{eq: [deterministic marginal] detpredfun form 2} via the definition of $\error$. 


%\cstalpha

\DeterministicMarginalsUniformError*

\begin{proof}[{Proof of \cref{prop: [deterministic marginal] Additive and multiplicative uniform error}}]\hfill
    \begin{enumerate}
        \item  Consider first $s>1$. Compute $\price\detpricefun$ for this choice of $\dpred$, which yields
        \begin{align}
            \price\detpricefun(\price)&=\frac {{\price}^{1-\smth}}{2\epsilon}\int_{\price-\epsilon}^{\price} \pred^{\smth}\de y + \frac{{\price}^{1+\smth}}{2\epsilon}\int_{\price}^{\price+\epsilon} \pred^{-\smth}\de \pred\notag\\
            &= \frac {{\price}^{1-\smth}}{2\epsilon} \frac{{\price}^{1+\smth}-(\price-\epsilon)^{1+\smth}}{1+\smth} + \frac{{\price}^{1+\smth}}{2\epsilon}\frac{(\price+\epsilon)^{1-\smth}-{\price}^{1-\smth}}{1-\smth}\,.\label{eq: PR [deterministic marginal] Additive and multiplicative uniform error | proof 1}
        \end{align}
    Continuous differentiability of $\price\mapsto {\price}^{1+\smth}$ and $\price\mapsto {\price}^{1-\smth}$, along with Taylor's theorem, implies the existence of $(\hat {\price}_1,\hat{\price}_2)\in[\price-\epsilon,\price]\times[\price,\price+\epsilon]$ such that:
    \begin{align*}
        \frac{{\price}^{1+\smth}-(\price-\epsilon)^{1+\smth}}{1+\smth} &= \epsilon {\price}^{\smth} -\frac{\epsilon^2}2\smth {\price}^{\smth-1} + \frac{\epsilon^3}6\smth(\smth-1){\hat {\price}_1}^{\smth-2}\,,\\
        \frac{(\price+\epsilon)^{1-\smth}-{\price}^{1-\smth}}{1-\smth}&= \epsilon {\price}^{-\smth} - \frac{\epsilon^2}2\smth {\price}^{-1-\smth} -\frac{\epsilon^3}6\smth(\smth+1){\hat{\price}_2}^{-(2+\smth)}
        \,.
    \end{align*}
    Remark that one has the remainder bounds:
    \begin{align*}
        C_1&:= \frac{\smth(\smth-1)}{6}\ub^{\min\{0,2-\smth\}}\le \frac{\smth(\smth-1)}{6}{\hat{\price}_1}^{\cstalpha-3}\\
        C_2&:= \frac{\smth(1+\smth)}{6}\le \frac{\smth(1+\smth)}{6{\hat{\price}_2}^{2+\smth}}\,.
    \end{align*}
    Applying these bounds and the Taylor expansions of $\Phi$ to Eq.~\eqref{eq: PR [deterministic marginal] Additive and multiplicative uniform error | proof 1}, yields
    \begin{align}
             \price\detpricefun(\price)\ge \price-\frac\smth  2\epsilon -C\price\epsilon^2
    \label{eq: PR [deterministic marginal] Additive and multiplicative uniform error | proof 2}  
    \end{align}
    with $C=(C_1\theta^{-s}+C_2)/2$. Finally, injecting Eq.~\eqref{eq: PR [deterministic marginal] Additive and multiplicative uniform error | proof 2}  into \cref{lemma: [deterministic marginal] general bounds lemma} and recalling that $\dpred=\delta_{\price}$ implies $\renorm=\con/\price$ yields Eq.~\eqref{eq: [deterministic marginal] general bound for det price}.

  \item Now, for $s=1$, the computation of $\detpricefun$ reduces to
  \begin{align*}
      \price\detpricefun(\price)&=\frac1{2\epsilon}\frac{{\price}^2-(\price+\epsilon)^2}{2} + \frac{{\price}^2}{2\epsilon}\int_{\price}^{\price+\epsilon} \frac1\pred\de\pred\\
      &=\frac{\price}2 + \frac{\epsilon}{4} + \frac{{\price}^2}{2\epsilon}(\log(\price+\epsilon)-\log(\price))\,.
  \end{align*}
    Using a Taylor expansion on $\log$, for some $t\in[0,1]$, we have 
    \begin{align*}
            \frac{{\price}^2}{2\epsilon}(\log(\price+\epsilon)-\log(\price))&\ge \frac{{\price}^2}{2\epsilon}\left(\frac{\epsilon}{\price}-\frac{\epsilon^2}2\frac{1}{{\price}^2} + \frac{\epsilon^3}{6}\frac{1}{2(\price+\epsilon t)^3}\right)\\
            &\ge \frac{\price}2 + \frac{\epsilon}4 + \frac{\epsilon^2}{24}\,.
    \end{align*}
    and thus, we obtain an overall bound matching Eq.~\eqref{eq: PR [deterministic marginal] Additive and multiplicative uniform error | proof 2} up to modifying $C$.\qedhere
\end{enumerate}
%    \item Likewise, for $s>1$, this choice of $\dpred$ computing $\detpricefun$ yields
%    \begin{align}
%        \detpricefun(\price)&=\frac{1}{2\epsilon\price}\left({\price}^{1-\smth}\int_{\price(1-\epsilon)}^{\price}\pred^{\smth} \de \pred + {\price}^{1+\smth}\int_{\price}^{\price(1+\epsilon)}\pred^{-\smth}\de\pred\right)\notag\\
%        &=\frac{\price}{2\epsilon}\left(\frac{1-(1-\epsilon)^{1+\smth}}{1+\smth} + \frac{(1+\epsilon)^{1-\smth}-1}{1-\smth}\right)\notag%\label{eq: PR integral expression for uniform multiplicative perturbation of deterministic prediction | proof 1}\,.
%    \end{align}
%    By the same expansion as above, we obtain the existence of $(t_1,t_2)\in[0,1]^2$ such that
%        \begin{align*}
%        \frac{1- (1-\epsilon)^{1+\smth}}{1+\smth} &= \epsilon -\frac{\epsilon^2}2\smth + \frac{\epsilon^3}6\smth(\smth-1){(1-t_1\epsilon)}^{\smth-2}\\
%        \frac{{(1+\epsilon)}^{1-\smth}-1}{1-\smth}&= \epsilon -\frac{\epsilon^2}2\smth +\frac{\epsilon^3}6\smth(1+\smth){(1+t_2\epsilon)}^{-(\smth+2)}
%        \,.
%    \end{align*}
%    Wherefrom one may easily choose remainder bounds
%    \begin{align*}
%        C'_1&:= \frac{\smth(\smth-1)}{6}0^{\min\{0,\smth-2\}}\le \frac{\smth(\smth-1)}6{(1-t_1\epsilon)}^{\smth-2}\\
%        C'_2&:= \frac{\smth(1+\smth)}6\ub\le \frac{\smth(1+\smth)}6{(1+t_2\epsilon)}^{-(\smth+2)}
%    \end{align*}
%    (with the convention that $0^0=1$), so that 
%    \begin{align}
%             \detpricefun(\price)\ge \price-\frac{\smth \price}2\epsilon -C'\price\epsilon^2
%    \label{eq: PR [deterministic marginal] Additive and multiplicative uniform error | proof 3}  
%    \end{align}
%    with $C'=(C'_1+C_2)/2$. Finally, injecting \eqref{eq: PR [deterministic marginal] Additive and multiplicative uniform error | proof 3}  into \cref{lemma: [deterministic marginal] general bounds lemma}, as before, yields \eqref{eq: [deterministic marginal] Multiplicative uniform error PR bound}.
    
%    The case $s=1$ follows without difficulty from the same arguments as in case 1., adapted to case 2.
%    \end{enumerate}
\end{proof}

\begin{restatable}{corollary}{DeterministicMarginalsUniformError2}\label{cor: [deterministic marginal] multiplicative uniform error} Let $\dprice=\delta_{\price}$ for some $\price\in\range$ and $\dpred=\Unif([\price(1-\epsilon'),\price(1+\epsilon')])$. There is $C'>0$ dependent only on $(\smth,\ub)$ such that
        \begin{align}
            \frac{\Eb[\algone(\Price,\Pred)]}{\Eb[\Price]}\ge \frac{1}{\rob\ub}\left(1-\frac{\smth}2\epsilon-C'\epsilon^2\right)\label{eq: [deterministic marginal] Multiplicative uniform error PR bound},
        \end{align}
    as soon as $0<\epsilon'\le \min\{1-{\price}^{-1},\ub{\price}^{-1}-1\}$.
\end{restatable}

\begin{proof}
    Follow the proof of \cref{prop: [deterministic marginal] Additive and multiplicative uniform error} with $\epsilon=\epsilon'p^*$.
\end{proof}

\begin{restatable}{corollary}{DeterministicMarginalsMultiInterval}\label{cor: [deterministic marginal] multiple interval predictions}
    Let $(I_k)_{k=1}^K$, $I_k:=[\locus_k-\epsilon_k,\locus_k+\epsilon_k]$ for $k\in[K]$ be a collection of (w.l.o.g. disjoint) sub-intervals of $\range$. Let $\dprice=\delta_{\price}$ and let
    \[\dpred=\sum_{k=1}^{K}\weight_k\Unif(I_k)\]
    be a mixture of Uniforms of the intervals $I_k$, \ie $\weight_k>0$ for all $k\in[K]$ and $\sum_{k}\weight_k=1$.
    Then, there is a constant $C''>0$ dependent only on $(\parr,\ub)$ such that
    \begin{equation}\label{eq: [deterministic marginal/cor multiple interval predictions] bound}
        \frac{\Eb[\algone(\Price,\Pred)]}{\Eb[\Price]}\ge \frac{1}{\rob\ub\Eb[\Price]}\left(\weight_{k^*}\left(1-\frac{\smth }{2\price}\epsilon_k\right)\right.
            + \sum_{k\neq k^*}\weight_k\error(\price,\locus_k)^{\smth}+C''\hspace{-3pt}\sum_{k\in[K]}\weight_k\epsilon_k^2 \Big).\notag
    \end{equation}
    in which $k^*$ denotes the index (if it exists) such that $\price\in I_{k^*}$.
\end{restatable}

\begin{proof}
 Note that $\dpred$ has density
    \begin{align}
        \pred\in\range\mapsto\sum_{k=1}^K\frac{\weight_k}{2\epsilon_k} \1_{\{\pred\in[\locus_k-\epsilon_k,\locus_k+\epsilon_k]\}}
        \notag
    \end{align}
    with respect to the Lebesgue measure. We can thus express $\detpricefun$ as a function of $\detpricefun_k$ its analogues for each $\Unif(I_k)$ as
    \begin{align}
        \detpricefun(\price)=\sum_{k=1}^K \weight_k \detpricefun_k(\price)\,. \label{[deterministic marginal] cor multiinterval predictions | proof 1}
    \end{align}
    We will now be able to proceed on each $\detpricefun_k$ as in the proof of the first part of \cref{prop: [deterministic marginal] Additive and multiplicative uniform error}. 
    
    Let us remark that the case of $k^*$ permits a direct application of \cref{prop: [deterministic marginal] Additive and multiplicative uniform error}.1, as $\price\in I_k$. This readily implies a contribution to the final bound of
    \begin{align}
        \frac{1}{\rob\ub\Eb[\Price]}\left(\weight_{k^*}(1-\frac{\smth}{2\price}\epsilon_k-C\epsilon_k^2)\right)\,,\label{eq: [deterministic marginal] cor multiinterval predictions | proof k star}
    \end{align}
    which is the first term of Eq.~\eqref{eq: [deterministic marginal/cor multiple interval predictions] bound}.

    Let us now fix $k\neq k^*$. Without loss of generality, we will order the intervals, so that $k<k^*$ implies that $x< p^*$ for every $x\in I_k$ and conversely for $k>k^*$. Notice that this implies that exactly one integral in each $\detpricefun_k$, $k\neq k^*$, may be non-zero, which is to say 
    \begin{align}
        \price\detpricefun_k(\price)= \frac{{\price}^{1-\smth}}{2\epsilon_k}\frac{(\locus_k+\epsilon_k)^{1+\smth}-(\locus_k-\epsilon_k)^{1+\smth}}{1+\smth}\1{\{k<k^*\}}+ \frac{{\price}^{1+\smth}}{2\epsilon_k}\frac{(\locus_k+\epsilon_k)^{1-\smth}-(\locus_k-\epsilon_k)^{1-\smth}}{1-\smth}\1{\{k>k^*\}}\notag
    \end{align}
    for any $k\neq k^*$.
    Let us take each term in turn and apply Taylor's theorem following the same methodology as the proof of \cref{prop: [deterministic marginal] Additive and multiplicative uniform error}. By adding and subtracting $\locus_k^{1+\smth}$ (resp. $\locus_k^{1-\smth}$), and using Taylor's theorem yields the existence of $(C_k^{(1)},C_k^{(2)})_{k\neq k^*}$ such that
    \begin{align}
        \frac{(\locus_k+\epsilon_k)^{1+\smth}-(\locus_k-\epsilon_k)^{1+\smth}}{1+\smth}&\ge 
        2\epsilon_k\locus_k^{\smth}+ \frac{\epsilon_k^3}{6}C_k^{(1)}\,,
        \notag \\
        \frac{(\locus_k+\epsilon_k)^{1-\smth}-(\locus_k-\epsilon_k)^{1-\smth}}{1-\smth}&\ge 
        2\epsilon_k\locus_k^{-\smth}+\frac{\epsilon_k^3}{6}C_k^{(2)}\,.
        \notag
    \end{align}
    Combining these over $k\in[K]$ yields
    \begin{align}
        \sum_{k\neq k^*}\weight_k\detpricefun_k(\price)&\ge {\price}^{1-\smth}\sum_{k<k^*}\weight_k\locus_k^{\smth}+{\price}^{1+\smth}\sum_{k>k^*}\weight_k\locus_k^{-\smth} + \frac{1}{12}\left(\sum_{k<k^*}C_k^{(1)}\ub^{1-\smth}\weight_k\epsilon_k^2+\sum_{k>k^*}C_k^{(2)}\weight_k\epsilon_k^2\right)\,. \label{[deterministic marginal] cor multiinterval predictions | proof 2}
    \end{align}    Now, add together Equations \eqref{[deterministic marginal] cor multiinterval predictions | proof 1} and \eqref{eq: [deterministic marginal] cor multiinterval predictions | proof k star}, recalling Eq.~ \eqref{[deterministic marginal] cor multiinterval predictions | proof 2}, and appeal to \cref{lemma: [deterministic marginal] general bounds lemma} to obtain
    \begin{align}
        \frac{\Eb[\algone(\Price,\Pred)]}{\Eb[\Price]}\ge \renorm\left( \weight_{k^*}\left(1-\frac{\smth}{2\price}\epsilon_k\right) + \sum_{k<k^*}\weight_k\left(\frac{\locus_k}{\price}\right)^{\smth}+\sum_{k>k^*}\weight_k\left(\frac{\price}{\locus_k}\right)^{\smth} + C''\sum_{k\in[K]}\epsilon_k^2\right)
    \end{align}
    for a suitably chosen constant $C''\ge 0$. To complete the proof, notice that the two sums can be combined using $\error$ as $\locus_k<\price$ if an only if $k<k^*$. 
\end{proof}

%\subsection{Complements to section \ref{subsec: example models/independent predictions}}\label{subapp: independent marginals}

\mypar{Deterministic predictions, stochastic prices}

Hereafter, we will use the following identity
\begin{align*}
        \detpredfun(\pred):&= \frac{1}{\Eb[\Price]}\price \int_\lb^\pred   \left(\frac{\price}{\pred}\right)^\smth\de \dprice(\price) + \price \int_\pred^\ub  \left(\frac{\pred}{\price}\right)^\smth\de \dprice(p)\,.\notag
\end{align*}

\mypar{Stochastic independent predictions and prices}

\IndependentMarginalsGeneralBounds*
\begin{proof}
    Starting from \cref{thm: [Stochastic bounds] bound in expectation using true coupling}, this follows from
    \[
        \frac{\Eb[\algone(\Price,\Pred)]}{\Eb[\Price]}\ge\frac{1}{\rob\ub\Eb[\Price]}\int\int \price\error(\price,\pred)^{\smth}\de \dpred(\pred)\de\dprice(\price)\,.
    \]
\end{proof}
\begin{corollary}\label{cor:  [independent marginals] general bounds lemma with density}
    Let $\truecoupling=\dprice\otimes\dpred$, with $\dprice,\dpred$ having density with respect to the Lebesgue measure, then the family $\family$ satisfies 
    \begin{align}
    \frac{\Eb[\algone(\Price,\Pred)]}{\Eb[\Price]}&\ge \frac{1}{\rob\ub\Eb[\Price]}\left( \int_\lb^\ub {\price}^{1-\smth}\de \dprice(\price)\int_\lb^\ub \pred^{-\smth}\de\dpred(\pred) + \int_\lb^\ub {\price}^{1+\smth}\de \dprice(\price)\int_\lb^\ub \pred^{\smth}\de\dpred(\pred)\right.\notag\\
    &\quad \left.-\int_\lb^\ub \left(\pred^{\smth}\int_\lb^\pred {\price}^{1-\smth}\de \dprice(\price) + \pred^{-\smth}\int_\pred^\ub{\price}^{1+\smth}\de\dprice(\price)\right)\de\dpred(\pred)\right)\label{eq: [independent marginals] general bounds lemma with density}
    \end{align}
\end{corollary}

\begin{remark}
    The result of \cref{cor:  [independent marginals] general bounds lemma with density} can be slightly tweaked to hold even without densities using Lebesgue-Stieltjes integration by parts. We omit these details for the sake of conciseness.
\end{remark}

\begin{proof}
    Starting with \cref{lemma: [independent marginals] general bounds lemma}, decompose the integral as
    \begin{align}
        \int_\lb^\ub \price\detpricefun(\price)\de\dprice(\price)&= \underbrace{\int_\lb^\ub {\price}^{1-\smth}\int_\lb^{\price}\pred^{\smth}\de\dpred(\pred)\de \dprice(\price)}_{A} + \underbrace{\int_\lb^\ub {\price}^{1+\smth}\int_{\price}^\ub \pred^{-\smth}\de\dpred(\pred)\de\dprice(\price)}_{B}\,.\notag
    \end{align}
    Using integration by parts, in which the parts for A are $x\mapsto\int_\lb^{x}{\price}^{1-\smth}\de\dprice(\price)$ and $x\mapsto\int_\lb^{x} \pred^{\smth}\de \dpred(\pred)$ yields
    \begin{align}
        A=\int_\lb^\ub {\price}^{1-\smth}\de \dprice(\price)\int_\lb^\ub \pred^{-\smth}\de\dpred(\pred) - \int_\lb^\ub \pred^{\smth}\int_\lb^\pred {\price}^{1-\smth}\de \dprice(\price) \de\dpred(\pred)\label{eq: cor:  [independent marginals] general bounds lemma with density | proof 1}\,.
    \end{align}
    Similarly, $B$ can be integrated by parts with parts $x\mapsto -\int_x^\ub {\price}^{1+\smth}\de\dprice(\price)$ and $x\mapsto \int_x^\ub \pred^{-\smth}\de\dpred(y)$, which yields
    \begin{align}
        B= \int_\lb^\ub {\price}^{
    1+\smth}\de \dprice(\price)\int_\lb^\ub \pred^{\smth}\de\dpred(\pred) - \int_\lb^\ub\pred^{-\smth} \int_\pred^\ub {\price}^{1+\smth}\de\dprice(\price)\de\dpred(\pred)\label{eq: cor:  [independent marginals] general bounds lemma with density | proof 2}\,.
    \end{align}
    Combining Equations \eqref{eq: cor:  [independent marginals] general bounds lemma with density | proof 1} with \eqref{eq: cor:  [independent marginals] general bounds lemma with density | proof 2} completes the proof.
\end{proof}

\begin{restatable}{proposition}{IndependentMarginalsUniforms}\label{prop: [independent marginals] uniform marginals case}
    Let $\truecoupling=\dprice\otimes\dpred$ and $\dprice=\dpred=\Unif([c_1,c_2])$ the family $\family$ satisfies
    \begin{align}
        \frac{\Eb[\algone(\Price,\Pred)]}{\Eb[\Price]}&\ge \frac{1}{\rob\ub}\frac{2}{\zeta(1)^3}\Bigg(\zeta(2-\smth)\zeta(1-\smth)  - \frac{c_1^{2-\smth}\zeta(1+\smth)}{2-\smth} \notag\\
        &\qquad+ \zeta(2+\smth)\zeta(1+\smth)- \frac{c_2^{2+\smth}\zeta(1-\smth)}{2+\smth}\notag\\
        &\hspace{6em}- \zeta(3)\left(\frac1{2-\smth}-\frac1{2+\smth}\right)  \Bigg)\notag
    \end{align}
     when $\smth\not\in\{1,2\}$, with $\zeta:\gamma\in(0,+\infty)\mapsto (c_2^\gamma-c_1^\gamma)\gamma^{-1}$.
\end{restatable}

\begin{proof}
    Starting with the decomposition of \cref{cor:  [independent marginals] general bounds lemma with density}, we can compute the terms separately. For the first two, we have
    \begin{align}
        \int_\lb^\ub {\price}^{1-\smth}\de \dprice(\price)\int_\lb^\ub \pred^{-\smth}\de\dpred(\pred)&= C\left(\frac{c_2^{2-\smth}-c_1^{2-\smth} }{2-\smth}\right)\left(\frac{m_2^{1-\smth}-m_1^{1-\smth} }{1-\smth}\right)\label{eq: IndependentMarginalsUniforms | proof 1}\\
        \int_\lb^\ub {\price}^{
    \cstalpha}\de \dprice(\price)\int_\lb^\ub \pred^{\smth}\de\dpred(\pred)&= C\left(\frac{c_2^{2+\smth}-c_1^{2+\smth}}{2+\smth}\right)\left(\frac{m_2^{1+\smth}-m_1^{1+\smth} }{1+\smth}\right)\label{eq: IndependentMarginalsUniforms | proof 2}
    \end{align}
    in which 
    \[
        C:=\frac1{(c_2-c_1)(m_2-m_1)}\,.
    \]
    Turning now to the second term of Eq.~\eqref{eq: [independent marginals] general bounds lemma with density}, we have 
    \begin{align}
        \int_\lb^\pred {\price}^{1-\smth}\de \dprice(\price) &=\frac{1}{c_2-c_1}\left(\frac{y^{2-\smth}-c_1^{2-\smth}}{2-\smth}\1_{\{\pred\in[c_1,c_2]\}} +\frac{c_2^{2-\smth}-c_1^{2-\smth}}{2-\smth}\1_{\{\pred>c_2\}}\right)
        \notag\\
        \intertext{and}
        \int_\pred^\ub{\price}^{\cstalpha}\de\dprice(\price)&= \frac{1}{c_2-c_1}\left( \frac{c_2^{2+\smth} - \pred^{2+\smth}}{2+\smth} \1_{\{\pred\in[c_1,c_2]\}} + \frac{c_2^{2+\smth}-c_1^{2+\smth}}{2+\smth}\1_{\{\pred<c_1\}}
        \right)\,,\notag
    \end{align}
so that, by integrating according to Eq.~\eqref{eq: [independent marginals] general bounds lemma with density} yields
\begin{align}
    \int_{\lb}^\ub \pred^{\smth}\int_\lb^\pred {\price}^{1-\smth}\de \dprice(\price)\de\dpred(\pred)&= C\left(\frac1{2-\smth}\left[\frac{\pred^3}3 - c_1^{2-\smth}\frac{\pred^{1+\smth}}{1+\smth} \right]_{c_1\vee m_1\wedge c_2}^{c_2\wedge m_2\vee c_1} + \frac{c_2^{2-\smth}-c_1^{2-\smth}}{2-\smth}\left[\frac{\pred^{1+\smth}}{\cstalpha}\right]^{c_2\vee m_2}_{c_2\vee m_1}\right)\label{eq: IndependentMarginalsUniforms | proof 3}
    \\
    \int_{\lb}^\ub \pred^{-\smth}\int_\pred^\ub{\price}^{1+\smth}\de\dprice(\price)\de\dpred(\pred) &= C\left(\frac1{2+\smth}\left[c_2^{2+\smth}\frac{\pred^{1-\smth}}{1-\smth}-\frac{\pred^3}{3}\right]_{c_1\vee m_1\wedge c_2}^{c_2\wedge m_2\vee c_1} + \frac{c_2^{2+\smth}-c_1^{2+\smth}}{2+\smth}\left[\frac{\pred^{1-\smth}}{1-\smth}\right]^{c_1\wedge m_2}_{c_1\wedge m_1} \right)\label{eq: IndependentMarginalsUniforms | proof 4}
\end{align}
    Recombining Equations \eqref{eq: IndependentMarginalsUniforms | proof 1}--\eqref{eq: IndependentMarginalsUniforms | proof 4} yields the result, up to simplifying for $m_i=c_i$ $i\in\{1,2\}$.
\end{proof}

\begin{figure}
    \centering
\includegraphics[width=0.5\linewidth]{figures/3d_plot.pdf}
    \caption{Numerical quadrature of \cref{prop: [independent marginals] uniform marginals case} for $c_1=1$, $c_2=\theta$ (for $\theta\in[1,10]$) as a function of $s\in[1,5]$.}
    \label{fig:3D plot}
\end{figure}


\subsection{Complements to section \ref{subsec: OT}}\label{subapp: OT}

Since it holds regardless of the coupling $\truecoupling$, the bound Eq.~\eqref{eq: in expectation of ratio using OT} has two direct benefits. First, it provides a notion of robustness for uncertainty in the coupling which is relevant for risk-assessment in practical applications. Second, it isolates the influence of the marginal distributions on the prediction from the coupling of $\Pred$ and $\Price$. Consequently, we can return to \eqref{eq: PR bound in Expectation of ratio using coupling} and isolate the contribution of the coupling, either through its transport sub-optimality 
\[
    \Eb[\Price\error(\Pred,\Price)^{\alpha-1}]-\inf_{\coupling\in\couplings(\dpred,\dprice)}\Eb[\Price\error(\Pred,\Price)^{\alpha-1}]
\]
or through a multiplicative analogue
\[
    \frac{\Eb[\Price\error(\Pred,\Price)^{\alpha-1}]}{\inf_{\coupling\in\couplings(\dpred,\dprice)}\Eb[\Price\error(\Pred,\Price)^{\alpha-1}]}\,.
\]
The geometry of these objects is highly intricate and unfortunately doesn't appear to have been studied previously. This highlights an interesting direction of research in the competitive analysis of optimal transport. 



\mypar{Duality}. The Kantorovich problem Eq.~\eqref{eq: in expectation of ratio using OT} admits a dual problem under general conditions (see \eg Thm. 5.10 in  \cite{villani_optimal_2009}). In our case, the cost function is $c:= (\price,\pred)\in\range^2\mapsto \price\error(\price,\pred)^{\smth}\in[0,\ub]$.  This duality is strong, meaning that 
\begin{align}
    \inf_{\coupling\in\couplings(\dprice,\dpred)}\int\price\error(\price,\pred)^{\smth}\de\coupling(\price,\pred) = \sup_{(\varphi,\psi)\in\Xi}\int \varphi(\dpred)\de\dpred(\pred) + \int \psi(\price)\de\dprice(\pred)\,,\label{eq: dual problem OT}
\end{align}
in which 
\[
    \Xi:=\{(\varphi,\psi):\range^2\to[0,+\infty)^2 \mbox{ bounded and measurable}: \forall(\price,\pred)\in\range^2\quad \varphi(\price)+\psi(\pred)\le c(\price,\pred) \}\,.
\]
Given a bounded measurable function $f$, let $f^c$ denote its $c$-transform, \ie the operator $\cdot^c$ such that maps $f$ to
\begin{align}
    f^c:\price\mapsto \inf_{\pred\in\range} \price\error(\price,\pred)^{\smth} - \varphi(\pred)\,.\label{eq: def c-trasform}
\end{align}
The definition Eq.~\eqref{eq: def c-trasform} shows that any bounded measurable function $\varphi:\range\to\Rb$ forms an admissible $(\varphi,\varphi^c)\in\Xi$ with its $c$-transform (this is symmetrical in the sense that $(\psi^c,\psi)$ is admissible if $\psi$ is bounded and measurable).

From the perspective of competitive analysis, the dual problem offers an appealing tool, as it suffices to propose a potential $\varphi$, compute its $c$-transform, and integrate it to obtain a bound. Of course, guessing the optimal potential $\varphi$ is as hard as solving the primal, but sub-optimal proposals can effectively leverage insights about the problem. \Cref{prop: [stochastic predictions] OT dual bounds} gives an example of this methodology. Note that it is possible to improve the potential $\varphi$ again by proposing ${(\varphi^c)}^c$, which we omit for brevity. 

\OTdualBoundsOne*

\begin{proof}
We split the proof into two parts, starting with the bound, and then the equality.
\begin{enumerate}
    \item 
    We start with the potential $\psi\equiv0$, whose $c$-transform (see Eq.~\eqref{eq: def c-trasform}) is 
    \begin{align*}
\varphi(\price)=\psi^c(\price)&=\inf_{\pred\in\range}\price\error(\price,\pred)^{\smth} \\
    &=\min\left\{\frac1{\price},\frac{\price}\ub\right\}^{\smth}\price\\
    &={\price}^{1-\smth}\1_{\{\price\ge \sqrt{\ub}\}}+ {\price}^{1+\smth}\1_{\{\price\le \sqrt{\ub}\}}\,.
    \end{align*}
    Inserting into Eq.~\eqref{eq: dual problem OT} yields the result.
    \item By duality of the optimal transport problem, 
\begin{align}
    \inf_{\dpred\in\Ps(\range)}\inf_{\coupling\in\couplings(\dprice,\dpred)}\int c(\price,\pred)\de\coupling(\price,\pred)&= \inf_{\dpred\in\Ps(\range)}\sup_{(\varphi,\psi)\in\Xi}\int\varphi(\price)\de\dprice(\price)+\int \psi(\pred)\de\dpred(\pred)
    \notag\\
    &\ge \sup_{(\varphi,\psi)\in\Xi}\left\{\int\varphi(\price)\de\dprice(\price)+\inf_{\dpred\in\Ps(\range)}\int \psi(\pred)\de\dpred(\pred)\right\}\,.\label{eq: OT duality inf over G | 1}
\end{align}
The inner infimum in Eq.~\eqref{eq: OT duality inf over G | 1} is equal to $\iota:=\inf\{\psi(y):y\in\range\}\in \Rb$. Consequently, the constraint set $\Xi$
can be replaced\footnote{Hereafter, all optimisation problems are over functions $(\varphi,\psi)$ which are bounded and measurable. We omit this line by line to reduce notational clutter.} without changing the value of the problem by 
\begin{align}
    S:=\{(\varphi,\iota)\in[0,+\infty)^{\range} \times\Rb: \quad \forall \price\in\range^2\quad \varphi(\price)\le \inf_{\pred\in\range}c(\price,\pred)-\iota\}
    \notag
\end{align}
so that 
\begin{align}
    \sup_{(\varphi,\psi)\in\Xi}\left\{\int\varphi(\price)\de\dprice(\price)+\inf_{\dpred\in\Ps(\range)}\int \psi(\pred)\de\dpred(\pred)\right\}&= \sup_{(\varphi,\iota)\in S}\left\{ \int \varphi(\price) \de\dprice(\price) +\iota\right\}
    \notag\\
    &= \sup_{\iota\in\Rb}\sup_{\psi_\iota\in S_\iota}\int \varphi_\iota(\price) \de\dprice(\price) +\iota\,,\label{eq: OT duality inf over G | 2}
\end{align}
in which $S_\iota:=\{\psi_\iota: \psi_\iota\le \inf_{\pred\in\range}c(\price,\pred)-\iota\}$.

As $\dprice$ is positive, the inner maximisation over $\psi_\iota$  in Eq.~\eqref{eq: OT duality inf over G | 2} saturates the constraints, whereafter, since $\int\iota\de\dprice=\iota$ and by combining with Eq.~\eqref{eq: OT duality inf over G | 1}, one has the result. \qedhere

\end{enumerate}

\end{proof}