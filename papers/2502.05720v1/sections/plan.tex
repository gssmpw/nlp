\vspace{1cm}



\mypar{Second draft plan -L}

\textit{Motivation}
\begin{enumerate}
    \item Progress of ML -> reconsider many existing decision algorithms.
    \item Side-information allows to improve on the worst-case bounds of classical TCS.
    \item Beyond worst-case analysis + Algorithms with predictions 
    \item So this field appears to have great potential for allowing ML to solve problems indirectly, through the combination of decision theory with existing predictions methods.
\end{enumerate}

\textit{Problem of applicability}
\begin{enumerate}
    \item The framework of algs with predictions remains deterministic + brittleness + risk-management (not uncertainty,deterministic).
    \item However many situations are modeled by stochastics (ML, spatial models/weather, fin Math etc.) which the current brittle algorithms fails to account for.
     i) Creates difficulties quantifying effective performance guarantees.
     \item Non-smooth algorithms prevents risk assessment of algs with predictions, which is essential in stochastic real-world use cases.
     \item This makes them of little practical relevance + which means it is better to not use the prediction but use the more robust method.
\end{enumerate}

\textit{Approach}: because of the diversity of stochastic environments and decision problems, establishing a unifying picture from the get-go is unrealistic. So we will take a stepping stone approach by restricting ourselves to $\OMS{}$ a highly simplified model of financial trading.  More general problems will exhibit the same core difficulties.


\textit{{\OMS{}}}:
\begin{enumerate} 
    \item a selection problem which can be viewed as the simplest representative of OE.
    \item Describe the game.
    \item OE in TCS: optimal selection, secretaries, and prophets. 
    \item OE in Finance: Stochastic modelling and market impact.
    \item Known Pareto optimal alg.
    \item The litterature review Spyros wrote goes well here
\end{enumerate}


\textit{Goal}: Design smooth optimal algorithms for ML-augmented decision-making applicable to real use-cases in stochastic environements like financial markets and introducing tools for their analysis.

\textit{Challenges}:
\begin{enumerate}
    \item performance imperative: Optimal performance (Consistency and robustness)
    \item Smooth error degradation to permit practical assessement of performances enabling practitioners to balance risk-reward with stochastics. 
    \item flexibility to analyse many different models of stochastics through general tools and analysis. Two things must be captured i) the complex relationship between the model's output and reality (prediction "goodness" via  "correlation") ii) the impact of the distributions of the two in-and-of themselves ("useful" but "bad" prediction graph, relation to non-stationarity in historic data models in finance)
\end{enumerate}



%\subsection{{\OMS{}}} 


\textit{Literature review}:
\begin{enumerate}
    \item Seminal papers du domain + diversity of contexts/applications
    \item Optimality et Pareto optimality
    \item Smoothness and brittleness/no error analysis
    \item Error measures 
    \item (?) distributional prediction/analysis in expectation and Wasserstein 
\end{enumerate}

\textit{Contributions}:
\begin{itemize}
    \item We provide a deterministic Pareto-opt smooth algorithmic family which supersedes the work of \cite{sun_online_2024}.
    \item We provide a smooth error analysis of this family as a function of a novel multiplicative error measure. This fine analysis provides a solid foundation on which it is comparatively simple to analyse the stochastic problem. We also derive additive error. [algorithm analysis]
    \item we apply the algorithm to several stochastic models which was not possible with \cite{sun_online_2024}'s algorithm. We provide two illustrative simple models in which closed-form estimates can be derived, and derive error functions as functions of distributions of the price/prediction. We also provide a worst-case analytical framework using Optimal transport.
\end{itemize}

\begin{figure}
    \centering
    \includegraphics[width=0.5\linewidth]{Execution_with_predictions/Submission/figures/optimal threshold.pdf}
    \caption{Schematic representation of random prices and prediction.}
    \label{fig:thresholds and predictions}
\end{figure}

\Cref{fig:thresholds and predictions} depicts a prediction distribution centred at $\phi^*$, the optimal threshold for a \OMS{} problem with the optimal price distributed as the blue distribution. While it is apparent that the two distributions are very different, and that the prediction doesn't predict the price in any meaningful way, an algorithm which blindly follows the prediction will still perform very well. Thus, it is possible for the 