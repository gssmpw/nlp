\section{Stochastic One-Max Search}\label{sec: stochastic predictions}



One-max-search under competitive analysis is a worst-case abstraction of online selection
%decision-making,
which is highly skewed towards pessimistic scenarios. This is an approach rooted in theoretical computer science that has the benefit of worst-case guarantees, but does not capture the stochasticity of real markets, \eg \citep{cont_financial_2004,donnelly_optimal_2022}. In contrast,  in mathematical (and practical) finance, probabilistic analyses such as risk management are preferred, \eg \cite{MERTON1975621}.  While reconciling the two approaches remains a very challenging perspective, we aim to narrow the very large gap between the worst-case and stochastic regimes by leveraging a probabilistic approach. This necessitates algorithms that can be robust to the randomness of the market, and to this end, the established smoothness of our algorithm (\cref{sec: deterministic predictions}) will play a pivotal role, as we will show.  A probabilistic analysis can thus yield two main practical benefits: 1) estimate performance under price distributions obtained from financial modelling; 2) leverage the consistency-robustness trade-off to handle risk.

%the ability to directly manage the risk of using predictions by leveraging the trade-off presented by the consistency-robustness Pareto-front in the deterministic case.

In the stochastic formulation of \OMS{}, we now consider the prices $(\Prices_i)_{i=1}^n$ to be random variables whose maximum is $\Price\sim\dprice$. Since market prices are random, the historical data used to generate a machine-learned prediction should also be random, hence we consider the prediction to be a random variable $\Pred\sim\dpred$. As before, we consider that $\Prices_i$, for $i\in[n]$, and $\Pred$ take value in $\range$. The trading window unfolds as in the classic \OMS{} problem, except that the prices and predictions are now random. 

We will first give, in \cref{subsec: competitive analysis in stochastic framework}, a general probabilistic competitive analysis of the \OMS{} problem which shows that the bounds of \cref{sec: deterministic predictions} transfer naturally by weighting the bounds of \cref{thm: [deterministic Pareto-Optimal smooth Algorithm] consistency-robustness-smoothness complex} according to the coupling of $(\Price,\Pred)$. In order to better understand the intuition behind these results, in \cref{subsec: instances}, we instantiate the analysis with three insightful models. 
Finally, in \cref{subsec: OT}, we show how to isolate the interaction of $\dpred$ and $\dprice$ using analytical tools from optimal transport theory.



%Consistency fails to characterize the quality of an algorithm with predictions as the probability that the prediction $Y$ is exactly the maximal price $P^*$ can be arbitrarily close, or even equal, to $0$. While robustness (in the worst-case) might still hold, a probabilistic analysis based on the smooth error bounds obtained in \cref{thm: [deterministic Pareto-Optimal smooth Algorithm] consistency-robustness-smoothness complex} offers a natural beyond worst-case alternative.  It  has two main practical benefits: 1) the ability to estimate performance under realistic  conditions, \ie for realistic price distributions; 2) the ability to  exploit good predictions through the risk trade-off presented by the consistency-robustness Pareto-front.

%In \cref{subsec: competitive analysis in stochastic framework}, we give a general probabilistic competitive analysis of the \OMS{} problem. \Cref{thm: [Stochastic bounds] bound in expectation using true coupling} shows a natural transfer of our deterministic bounds into expectation. However, since $\dprice$ and $\dpred$ are arbitrary distributions in this setting, this approach yields integral functionals of the coupling of $(\Price,\Pred)$. In order to better understand the intuition behind these complex expression, we detail two practical settings in \cref{subsec: example models/deterministic prices,subsec: example models/independent predictions},\ltodo{explicti the settings} while deferring more technical results based on optimal transport to \cref{subsec: OT}.   

\subsection{Competitive analysis in the stochastic framework}\label{subsec: competitive analysis in stochastic framework}
In the stochastic setting, we will evaluate the performance of the algorithm using the ratio of expectations $\Eb[\Alg(\Price,\Pred)]/\Eb[\Price]$, but our results and arguments transfer readily to $\Eb[\Alg(\Price,\Pred)/\Price]$.


%a performance ratio analogous to $\Alg/\price$ can still capture the performance relative to an optimal algorithm. However, the probabilistic nature of the analysis introduces two possible choices for this ratio: \emph{Ratio of Expectations} (RoE) 
%or \emph{Expectation of Ratio} (EoR) $\Eb\left[{\Alg(\Price,\Pred)}/{\Price}\right]$.
%Both these ratios are considered in the competitive analysis literature, \eg \cite{ezra_prophet_2023}, though RoE is the most common. Hereafter, we state results for RoE, but our arguments and results adapt easily to EoR as well.
%


Because any algorithm must operate on the realisation of $\Pred$, its performance becomes a random variable depending on the specific relationship of $\Price$ and $\Pred$. This is captured the coupling $\truecoupling$ of $(\Price,\Pred)$, yielding
\begin{align}
    \Eb[\Alg(\Price,\Pred)]:=\int \Alg(\price,\pred) \de\truecoupling(\price,\pred)\,. \label{eq: expectation=coupling integral}
\end{align}
In consequence, we can identify $\truecoupling$ and the instance $(\Price,\Pred)\sim\truecoupling$ without loss of generality, as all such instances are indistinguishable to a probabilistic analysis. 


Taking into account the coupling, the bound proved in \cref{thm: [deterministic Pareto-Optimal smooth Algorithm] additive smoothness} adapts to the stochastic setting to yield \cref{thm: [Stochastic bounds] bound in expectation using true coupling} below.


\begin{lemma}\label{thm: [Stochastic bounds] bound in expectation using true coupling}
The family $\family$ satisfies 
    \begin{align}
     \hspace{-1em}\frac{\Eb[\algone(\Price,\Pred)]}{\Eb[\Price]}\ge \max\left\{\parr\,,\, \frac{1}{\rob\ub}\frac{\Eb\left[\Price\error(\Price,\Pred)^{\smth}\right]}{\Eb[\Price]}\right\}\hspace{-1pt}.\label{eq: PR bound in Expectation of ratio using coupling}
    \end{align}
\end{lemma}

\begin{proof}
    Apply Jensen's inequality to \cref{thm: [deterministic Pareto-Optimal smooth Algorithm] consistency-robustness-smoothness complex}.
\end{proof}
As expected,~\eqref{eq: PR bound in Expectation of ratio using coupling} shows that the robustness of $\family$ carries over to the stochastic setting through the $\max\{r,\cdot\}$ term.



%\section{Models of Stochastic Predictions}\label{sec: example models}

%The theorems of \cref{sec: stochastic predictions} provide clear general results which apply across the breadth of instances of the stochastic \OMS{} setting. However, it is difficult to parse practical bounds for specific instances from the general bounds. In this section, we will turn to the deterministic price and independent sampling settings introduced in \cref{sec:introduction} in order to demonstrate how probabilistic tools can be mobilised to obtain ad-hoc detailed bounds in specific instances. 

\subsection{Instantiations of Lemma \ref{thm: [Stochastic bounds] bound in expectation using true coupling}}\label{subsec: instances}


The coupling $\truecoupling$, and Eq.~\eqref{eq: PR bound in Expectation of ratio using coupling} more broadly, encode effects that influence the quality of a prediction from two different sources: the relationship of $\dpred$ and $\dprice$ and the relationship between $\Pred$ and $\Price$ themselves (\eg correlation). In this section, we aim to isolate the effect of $\dpred$ and $\dprice$.

\paragraph{Stochastic predictions, deterministic prices.}
This semi-deterministic model, in which $\dprice=\delta_{\price}$ (which is to say $\Price=\price$ almost surely), isolates the effect of $\dpred$. From a practical standpoint, it can also be used to model predictions which are noisy measurements of deterministic, but unknown, quantities.
%
Its theoretical interest comes from the fact that it simplifies Eq.~\eqref{eq: expectation=coupling integral} into an integral over $\dprice$. This allows us to derive \cref{lemma: [deterministic marginal] general bounds lemma} from \cref{thm: [Stochastic bounds] bound in expectation using true coupling}, in which the function $\detpricefun:\range\to[0,1]$ defined by 
\begin{align}
\detpricefun(\price)&= \Eb\left[\error(\Price,\Pred)^{\smth}\vert \Price=\price\right] \,
\label{eq: [deterministic marginal] detpredfun definition}
\end{align}
for $\price\in\range$, directly quantifies the quality of the prediction in terms of the performance, with respect to the true, realised, maximal price $\price$. Indeed, $\Lambda(\price) \leq 1$ for all $\price$, and the closer to one, the better the prediction.

In particular, if the maximal price is deterministic, but the prediction is stochastic, this yields the following guarantees. For the sake of clarity of the results, we will no longer specify the term coming from the robustness, with the understanding that one can add a maximum with $\parr$ to any bound on the performance of $\family$. 

\begin{restatable}{corollary}{DeterministicMarginalBounds}\label{lemma: [deterministic marginal] general bounds lemma}
    Let $\dprice=\delta_{\price}$ for some $\price\in\range$, then the family $\family$ satisfies
    \begin{align}
        \frac{\Eb[\algone(\Price,\Pred)]}{\Eb[\Price]}\ge \frac{1}{r\theta} \detpricefun(\price)\,.\label{eq: [deterministic marginal] general bound for det price}
    \end{align}
\end{restatable}

Viewing $\detpricefun$ as a map (taking $\dpred$ to a real-valued function on $\range$) reveals that it quantifies the usefulness of $\dpred$ as a prediction distribution at any $\price\in\range$.

As an integral functional of $\dpred$, $\detpricefun$ may not admit a closed form. Nevertheless, it can be estimated to capture subtle stochastic phenomena as demonstrated by \cref{prop: [deterministic marginal] Additive and multiplicative uniform error}. 

\begin{restatable}{example}{DeterministicMarginalsUniformError}\label{prop: [deterministic marginal] Additive and multiplicative uniform error} Let $\dprice=\delta_{\price}$ for some $\price\in\range$ and $\dpred=\Unif([\price-\epsilon,\price+\epsilon])$. There is a constant $C>0$, dependent only on $(\smth,\ub)$, such that
        \begin{align}
            \frac{\Eb[\algone(\Price,\Pred)]}{\Eb[\Price]}\ge  \frac{1}{\rob\ub} \left(1 -\frac{\smth}{2\price}\epsilon- C\epsilon^2\right)%\tconsistency(\parc,\parlambda) \left(\frac{{\price}^{1-\parlambda}+{\price}^{\parlambda-1}}2 - \frac{\epsilon (1-\parlambda)}4\left(p^{-\parlambda}+{\price}^{\parlambda-2}\right)+C\epsilon^{2}\right)
            \label{eq: [deterministic marginal] Additive uniform error PR bound},
        \end{align}
        as soon as $0<\epsilon\le \min\{\ub-\price,\price-\lb\}$.
\end{restatable}

Eq.~\eqref{eq: [deterministic marginal] Additive uniform error PR bound} reveals that the performance of $\family$ decays from consistency at a rate linear in the uncertainty $\epsilon$ determined by the smoothness $s$ of the algorithm. This captures the scale of the effect of smoothness on a practical example. In Eq.~\eqref{eq: [deterministic marginal] Additive uniform error PR bound} we characterised the rate up to the second order ($\epsilon^2$), but higher-order estimates can be obtained similarly.
%

%All distributions with sufficiently regular (\eg Lipschitz) densities can be approximated using mixtures over the model of \cref{prop: [deterministic marginal] Additive and multiplicative uniform error}, to obtain precise bounds, see \cref{cor: [deterministic marginal] multiple interval predictions}.
%\ltodo{I have removed a paragraph here. We will reinstate it in a more or less long form if there is enough space}

Moreover, this shows that all sufficiently regular distributions can be approximated in terms of $\detpricefun$ using mixtures over the model of \cref{prop: [deterministic marginal] Additive and multiplicative uniform error}, \ie $\dpred=\sum_{k=1}^{K}\weight_k\Unif(I_k)$ for $w_i>0$, $\sum_i w_i=1$, and $(I_k)_k$ disjoint subintervals of $\range$ (see \cref{cor: [deterministic marginal] multiple interval predictions}). Numerical integration (\eg Monte-Carlo) offers another alternative method to estimate~$\detpricefun$.\\

%More sophisticated distributions can be approximated by considering mixtures over the model of \cref{prop: [deterministic marginal] Additive and multiplicative uniform error}, \ie 
%\[
%    \dpred=\sum_{k=1}^{K}\weight_k\Unif(I_k)\;,
%\]
%wherein $w_i>0$, $\sum_i w_i=1$, and $(I_k)_k$ are disjoint subintervals of $\range$, it is possible (see \cref{cor: [deterministic marginal] multiple interval predictions}) to construct piece-wise linear approximations of $\dpred$ to estimate $\detpricefun$ when it does not admit a closed form. Numerical integration (\eg Monte-Carlo) offers an alternative method to estimate $\detpricefun$.

\noindent
\mypar{Deterministic predictions, stochastic prices.}
The performances of our family of algorithms can also be computed if the prices are stochastic, but the prediction is deterministic. This model swaps the randomness: now the prices are random so that $\price\sim\dprice$ is generic and it is $\Pred\sim \delta_\pred$ which is deterministic. 

While this setting appears symmetrical to the previous one, this is not the case as the \OMS{} problem itself is highly asymmetrical. Indeed, using a threshold means that predictions too high or too low do not have the same impact. 
By defining
    \begin{align}
        \detpredfun(\pred):&=\frac{\Eb[\Price\error(\Price,\Pred)^{\smth}\vert \Pred=\pred]}{\Eb[\Price]}\quad\mbox{  for } \pred\in\range 
        %{\pred}^{-\smth}\int_\lb^\pred {\price}^{1+\smth}\de \dprice(\price) + {\pred}^{\smth}\int_\pred^\ub {\price}^{1-\smth}\de \dprice(p)
        \,,\notag
\end{align}
we can establish a quality quantification which mirrors $\detpricefun$: this functional of $\dpred$ states how good any unique prediction $\pred$ is at influencing algorithmic performance. This yields the following \cref{rem:  [deterministic marginal] detpredfun}, an analogue of \cref{lemma: [deterministic marginal] general bounds lemma}. Note that $\detpredfun(y)\le 1$ for all $y\in\range$. 

\begin{restatable}{corollary}{RemarkDetPredFun}\label{rem:  [deterministic marginal] detpredfun}
    Let $\dpred=\delta_{\pred}$ for some $\pred\in\range$. The family $\family$ satisfies
\begin{align}
        \frac{\Eb[\algone(\Price,\Pred)]}{\Eb[\Price]}\ge \frac{1}{r\theta}\detpredfun(y)\,.\label{eq: [deterministic marginal] general bound for det pred}
    \end{align}    %\VP{Still not fixed I guess}
\end{restatable}


\paragraph{Stochastic independent predictions and prices}
The theoretical value of the above two models is their isolation of the effect of $\dpred$ into $\detpricefun$ (resp. $\dprice$ into $\detpredfun$). We now turn to a model in which $\Pred$ and $\Price$ are independent (denoted by $\truecoupling=\dprice\otimes\dpred$) which will illustrate that predictions can be useful even without any correlation. The intuition is simple: some inaccurate predictions can still induce (on average) good thresholds because of the algorithm's internal mechanics.
This effect is captured by the interaction between the functional $\detpricefun$ and the distribution of prices $\dprice$ (resp. $\detpredfun$ and $\dpred$), as shown by \cref{lemma: [independent marginals] general bounds lemma}\footnote{The following result also applies to sampling of distribution-valued predictions \citep{angelopoulos_contract_2024,dinitz_binary_2024}.}.


\begin{restatable}{corollary}{IndependentMarginalsGeneralBounds}\label{lemma: [independent marginals] general bounds lemma}
    Let $\truecoupling=\dprice\otimes\dpred$, the family $\family$ satisfies 
    \begin{align}
        \frac{\Eb[\algone(\Price,\Pred)]}{\Eb[\Price]}&\ge \frac{1}{\rob\ub} \int\detpredfun(\pred)\de\dpred(\pred)
        \, \notag\\
        &=
        \frac{1}{\rob\ub} \int\frac{\price\detpricefun(\price)}{\Eb[\Price]}\de\dprice(\price).
        %\notag
        \label{eq: expectation form of detpricefun bound}
         %\label{eq: [independent marginals] general bound}
    \end{align}
    %Alternatively, this rewrites, for some appropriate distribution $H$ on $\range$, as 
    %\begin{align}
        %\frac{\Eb[\algone(\Price,\Pred)]}{\Eb[\Price]} &\ge \frac{1}{\rob\ub}\Eb_{Z\sim H}[\detpricefun(Z)]\,.\label{eq: expectation form of detpricefun bound}
    %\end{align}
\end{restatable}
Since $\detpricefun(z)$ is always smaller than 1 (and again, the closer to one, the better the predictions are), Eq.~\eqref{eq: expectation form of detpricefun bound} gives an intuitive bound on the performance
%\stodo{this should not be consistency, but rather competitive ratio}
of the algorithms.

The theoretical benefit of the model transpires in \cref{lemma: [independent marginals] general bounds lemma}: independence separates the integral against $\truecoupling$ in Eq.~\eqref{eq: expectation=coupling integral} into a double integral revealing $\detpredfun$. Unfortunately, it is often difficult to obtain a closed form for the resulting expression (see, \eg, \cref{prop: [independent marginals] uniform marginals case}), but one can rely on numerical integration instead (see \cref{fig:3D plot} in \cref{app: sec3}). 



\subsection{Dependent predictions and optimal transport}\label{subsec: OT}


The previous models successfully isolated the effect of the distributions $\dprice$ and $\dpred$. Using tools from Optimal Transport (OT) theory, one can generalise this approach. For brevity, we refer simply to \cite{villani_optimal_2009} for the technicalities and background of this field. The key observation is that the right-hand side of Eq.~\eqref{eq: expectation=coupling integral} is a \emph{transport functional} of $\pi^*$, which can be lower bounded uniformly over the set of couplings $\couplings(\dprice,\dpred)$  of $\dprice$ and $\dpred$. This set is exactly the set of joint distributions for $(\Price,\Pred)$ when $\Price\sim\dprice$ and $\Pred\sim\dpred$. Minimising a transport functional over couplings is the classic OT problem \citep{villani_optimal_2009}, hence \cref{thm: [Stochastic bounds] OT bound for coupling robustness}.


\begin{theorem}\label{thm: [Stochastic bounds] OT bound for coupling robustness}
    The family $\family$ satisfies 
        \begin{align}
        \frac{\Eb[\algone(\Price,\Pred)]}{\Eb[\Price]}\ge \frac{1}{r\theta}\frac{\inf\mathlarger{\int} \price\error(\price,\pred)^{\smth}\de\coupling(\price,\pred)}{\Eb[\Price]}\label{eq: in expectation of ratio using OT}\,.
    \end{align}
    where infimum is taken over couplings $\coupling\in\couplings(\dprice,\dpred)$ ; in particular, the numerator is as most $\Eb[\Price]$.
\end{theorem}

\Cref{thm: [Stochastic bounds] OT bound for coupling robustness} highlights a novel connection between (stochastic) competitive analysis and optimal transport.
Contrary to most literature in OT, in which the optimal configuration tries to minimise the distance points $(\price,\pred)$ are moved, the infimum in ~\eqref{eq: in expectation of ratio using OT} tries to push them far apart to induce the algorithm to make mistakes.

Optimal transport tools have been used before in algorithms with predictions, notably in the \emph{distributional predictions} setting in which the algorithm is given $\dpred$ itself \citep{angelopoulos_contract_2024,dinitz_binary_2024}. This analysis, however, is fundamentally different: it uses Wasserstein distances (see \cref{subapp: proba additive analysis}) in place of $\additiveerror$ in quantifying the error of the distributional prediction $\dpred$ of $\dprice$. Our stochastic framework ties its error metric closely to the asymmetric nature of the problem through $\price\error(\price,\pred)^\smth$, which is why our OT problem, \ie Eq.~\eqref{eq: in expectation of ratio using OT}, is \emph{not} symmetric: exchanging the roles of $(\dpred,\dprice)$ cannot be expected to yield the same performance. 
The optimal transport problem in Eq.~\eqref{eq: in expectation of ratio using OT} generally has no closed form, but thanks to its (strong) dual form (see \cref{subapp: OT}), one can use problem-specific knowledge to derive lower bounds, as demonstrated by \cref{prop: [stochastic predictions] OT dual bounds}.

\begin{restatable}{proposition}{OTdualBoundsOne}\label{prop: [stochastic predictions] OT dual bounds}
    The family $\family$ satisfies
    \begin{align}
    \frac{\Eb[\algone(\Price,\Pred)]}{\Eb[\Price]}\ge\frac{1}{\rob\ub} \frac{\displaystyle \int_\lb^{\ub^{\frac12}}\hspace{-11pt} {\price}^{1+\smth}\de\dprice(\price) \hspace{-2pt}+\hspace{-4pt} \int_{\hspace{-1pt}\ub^{\frac12}}^\ub\hspace{-3pt} {\price}^{1-\smth}\de\dprice(\price)}{\Eb[P^*]}\hspace{2pt}.\notag
    \end{align}
    Moreover, the RHS is the infimum over $\dpred$ of Eq.~\eqref{eq: in expectation of ratio using OT}.
\end{restatable}

% \lc{
% \begin{restatable}{proposition}{OTdualBoundsOne}\label{prop: [stochastic predictions] OT dual bounds}
%     The ratio $\dfrac{\Eb[\algone(\Price,\Pred)]}{\Eb[\Price]}$ is at least
%     \begin{align}
%     \frac{1}{\rob\ub} \frac{\displaystyle \int_\lb^{\ub^{\frac12}}\hspace{-11pt} {\price}^{1+\smth}\de\dprice(\price) \hspace{-2pt}+\hspace{-4pt} \int_{\hspace{-1pt}\ub^{\frac12}}^\ub\hspace{-3pt} {\price}^{1-\smth}\de\dprice(\price)}{\Eb[P^*]}\hspace{2pt}.\notag
%     \end{align}
% \end{restatable}
% }
%\VP{both options are equally good}

\Cref{prop: [stochastic predictions] OT dual bounds} once more highlights the asymmetry of the problem through different contributions of the regions above and below $\sqrt{\ub}$, which is the threshold that guarantees $1/\sqrt{\theta}$-robustness. The dual problem provides thus a practical tool for designing lower bounds for the performance of $\family$ in the stochastic \OMS{} setting.