%\section{Models of Stochastic Predictions}\label{sec: example models}

%The theorems of \cref{sec: stochastic predictions} provide clear general results which apply across the breadth of instances of the stochastic \OMS{} setting. However, it is difficult to parse practical bounds for specific instances from the general bounds. In this section, we will turn to the deterministic price and independent sampling settings introduced in \cref{sec:introduction} in order to demonstrate how probabilistic tools can be mobilised to obtain ad-hoc detailed bounds in specific instances. 

\subsection{Instantiations of Lemma \ref{thm: [Stochastic bounds] bound in expectation using true coupling}}\label{subsec: instances}


The coupling $\truecoupling$, and Eq.~\eqref{eq: PR bound in Expectation of ratio using coupling} more broadly, encode effects that influence the quality of a prediction from two different sources: the relationship of $\dpred$ and $\dprice$ and the relationship between $\Pred$ and $\Price$ themselves (\eg correlation). In this section, we aim to isolate the effect of $\dpred$ and $\dprice$.

\paragraph{Stochastic predictions, deterministic prices.}
This semi-deterministic model, in which $\dprice=\delta_{\price}$ (which is to say $\Price=\price$ almost surely), isolates the effect of $\dpred$. From a practical standpoint, it can also be used to model predictions which are noisy measurements of deterministic, but unknown, quantities.
%
Its theoretical interest comes from the fact that it simplifies Eq.~\eqref{eq: expectation=coupling integral} into an integral over $\dprice$. This allows us to derive \cref{lemma: [deterministic marginal] general bounds lemma} from \cref{thm: [Stochastic bounds] bound in expectation using true coupling}, in which the function $\detpricefun:\range\to[0,1]$ defined by 
\begin{align}
\detpricefun(\price)&= \Eb\left[\error(\Price,\Pred)^{\smth}\vert \Price=\price\right] \,
\label{eq: [deterministic marginal] detpredfun definition}
\end{align}
for $\price\in\range$, directly quantifies the quality of the prediction in terms of the performance, with respect to the true, realised, maximal price $\price$. Indeed, $\Lambda(\price) \leq 1$ for all $\price$, and the closer to one, the better the prediction.

In particular, if the maximal price is deterministic, but the prediction is stochastic, this yields the following guarantees. For the sake of clarity of the results, we will no longer specify the term coming from the robustness, with the understanding that one can add a maximum with $\parr$ to any bound on the performance of $\family$. 

\begin{restatable}{corollary}{DeterministicMarginalBounds}\label{lemma: [deterministic marginal] general bounds lemma}
    Let $\dprice=\delta_{\price}$ for some $\price\in\range$, then the family $\family$ satisfies
    \begin{align}
        \frac{\Eb[\algone(\Price,\Pred)]}{\Eb[\Price]}\ge \frac{1}{r\theta} \detpricefun(\price)\,.\label{eq: [deterministic marginal] general bound for det price}
    \end{align}
\end{restatable}

Viewing $\detpricefun$ as a map (taking $\dpred$ to a real-valued function on $\range$) reveals that it quantifies the usefulness of $\dpred$ as a prediction distribution at any $\price\in\range$.

As an integral functional of $\dpred$, $\detpricefun$ may not admit a closed form. Nevertheless, it can be estimated to capture subtle stochastic phenomena as demonstrated by \cref{prop: [deterministic marginal] Additive and multiplicative uniform error}. 

\begin{restatable}{example}{DeterministicMarginalsUniformError}\label{prop: [deterministic marginal] Additive and multiplicative uniform error} Let $\dprice=\delta_{\price}$ for some $\price\in\range$ and $\dpred=\Unif([\price-\epsilon,\price+\epsilon])$. There is a constant $C>0$, dependent only on $(\smth,\ub)$, such that
        \begin{align}
            \frac{\Eb[\algone(\Price,\Pred)]}{\Eb[\Price]}\ge  \frac{1}{\rob\ub} \left(1 -\frac{\smth}{2\price}\epsilon- C\epsilon^2\right)%\tconsistency(\parc,\parlambda) \left(\frac{{\price}^{1-\parlambda}+{\price}^{\parlambda-1}}2 - \frac{\epsilon (1-\parlambda)}4\left(p^{-\parlambda}+{\price}^{\parlambda-2}\right)+C\epsilon^{2}\right)
            \label{eq: [deterministic marginal] Additive uniform error PR bound},
        \end{align}
        as soon as $0<\epsilon\le \min\{\ub-\price,\price-\lb\}$.
\end{restatable}

Eq.~\eqref{eq: [deterministic marginal] Additive uniform error PR bound} reveals that the performance of $\family$ decays from consistency at a rate linear in the uncertainty $\epsilon$ determined by the smoothness $s$ of the algorithm. This captures the scale of the effect of smoothness on a practical example. In Eq.~\eqref{eq: [deterministic marginal] Additive uniform error PR bound} we characterised the rate up to the second order ($\epsilon^2$), but higher-order estimates can be obtained similarly.
%

%All distributions with sufficiently regular (\eg Lipschitz) densities can be approximated using mixtures over the model of \cref{prop: [deterministic marginal] Additive and multiplicative uniform error}, to obtain precise bounds, see \cref{cor: [deterministic marginal] multiple interval predictions}.
%\ltodo{I have removed a paragraph here. We will reinstate it in a more or less long form if there is enough space}

Moreover, this shows that all sufficiently regular distributions can be approximated in terms of $\detpricefun$ using mixtures over the model of \cref{prop: [deterministic marginal] Additive and multiplicative uniform error}, \ie $\dpred=\sum_{k=1}^{K}\weight_k\Unif(I_k)$ for $w_i>0$, $\sum_i w_i=1$, and $(I_k)_k$ disjoint subintervals of $\range$ (see \cref{cor: [deterministic marginal] multiple interval predictions}). Numerical integration (\eg Monte-Carlo) offers another alternative method to estimate~$\detpricefun$.\\

%More sophisticated distributions can be approximated by considering mixtures over the model of \cref{prop: [deterministic marginal] Additive and multiplicative uniform error}, \ie 
%\[
%    \dpred=\sum_{k=1}^{K}\weight_k\Unif(I_k)\;,
%\]
%wherein $w_i>0$, $\sum_i w_i=1$, and $(I_k)_k$ are disjoint subintervals of $\range$, it is possible (see \cref{cor: [deterministic marginal] multiple interval predictions}) to construct piece-wise linear approximations of $\dpred$ to estimate $\detpricefun$ when it does not admit a closed form. Numerical integration (\eg Monte-Carlo) offers an alternative method to estimate $\detpricefun$.

\noindent
\mypar{Deterministic predictions, stochastic prices.}
The performances of our family of algorithms can also be computed if the prices are stochastic, but the prediction is deterministic. This model swaps the randomness: now the prices are random so that $\price\sim\dprice$ is generic and it is $\Pred\sim \delta_\pred$ which is deterministic. 

While this setting appears symmetrical to the previous one, this is not the case as the \OMS{} problem itself is highly asymmetrical. Indeed, using a threshold means that predictions too high or too low do not have the same impact. 
By defining
    \begin{align}
        \detpredfun(\pred):&=\frac{\Eb[\Price\error(\Price,\Pred)^{\smth}\vert \Pred=\pred]}{\Eb[\Price]}\quad\mbox{  for } \pred\in\range 
        %{\pred}^{-\smth}\int_\lb^\pred {\price}^{1+\smth}\de \dprice(\price) + {\pred}^{\smth}\int_\pred^\ub {\price}^{1-\smth}\de \dprice(p)
        \,,\notag
\end{align}
we can establish a quality quantification which mirrors $\detpricefun$: this functional of $\dpred$ states how good any unique prediction $\pred$ is at influencing algorithmic performance. This yields the following \cref{rem:  [deterministic marginal] detpredfun}, an analogue of \cref{lemma: [deterministic marginal] general bounds lemma}. Note that $\detpredfun(y)\le 1$ for all $y\in\range$. 

\begin{restatable}{corollary}{RemarkDetPredFun}\label{rem:  [deterministic marginal] detpredfun}
    Let $\dpred=\delta_{\pred}$ for some $\pred\in\range$. The family $\family$ satisfies
\begin{align}
        \frac{\Eb[\algone(\Price,\Pred)]}{\Eb[\Price]}\ge \frac{1}{r\theta}\detpredfun(y)\,.\label{eq: [deterministic marginal] general bound for det pred}
    \end{align}    %\VP{Still not fixed I guess}
\end{restatable}


\paragraph{Stochastic independent predictions and prices}
The theoretical value of the above two models is their isolation of the effect of $\dpred$ into $\detpricefun$ (resp. $\dprice$ into $\detpredfun$). We now turn to a model in which $\Pred$ and $\Price$ are independent (denoted by $\truecoupling=\dprice\otimes\dpred$) which will illustrate that predictions can be useful even without any correlation. The intuition is simple: some inaccurate predictions can still induce (on average) good thresholds because of the algorithm's internal mechanics.
This effect is captured by the interaction between the functional $\detpricefun$ and the distribution of prices $\dprice$ (resp. $\detpredfun$ and $\dpred$), as shown by \cref{lemma: [independent marginals] general bounds lemma}\footnote{The following result also applies to sampling of distribution-valued predictions \citep{angelopoulos_contract_2024,dinitz_binary_2024}.}.


\begin{restatable}{corollary}{IndependentMarginalsGeneralBounds}\label{lemma: [independent marginals] general bounds lemma}
    Let $\truecoupling=\dprice\otimes\dpred$, the family $\family$ satisfies 
    \begin{align}
        \frac{\Eb[\algone(\Price,\Pred)]}{\Eb[\Price]}&\ge \frac{1}{\rob\ub} \int\detpredfun(\pred)\de\dpred(\pred)
        \, \notag\\
        &=
        \frac{1}{\rob\ub} \int\frac{\price\detpricefun(\price)}{\Eb[\Price]}\de\dprice(\price).
        %\notag
        \label{eq: expectation form of detpricefun bound}
         %\label{eq: [independent marginals] general bound}
    \end{align}
    %Alternatively, this rewrites, for some appropriate distribution $H$ on $\range$, as 
    %\begin{align}
        %\frac{\Eb[\algone(\Price,\Pred)]}{\Eb[\Price]} &\ge \frac{1}{\rob\ub}\Eb_{Z\sim H}[\detpricefun(Z)]\,.\label{eq: expectation form of detpricefun bound}
    %\end{align}
\end{restatable}
Since $\detpricefun(z)$ is always smaller than 1 (and again, the closer to one, the better the predictions are), Eq.~\eqref{eq: expectation form of detpricefun bound} gives an intuitive bound on the performance
%\stodo{this should not be consistency, but rather competitive ratio}
of the algorithms.

The theoretical benefit of the model transpires in \cref{lemma: [independent marginals] general bounds lemma}: independence separates the integral against $\truecoupling$ in Eq.~\eqref{eq: expectation=coupling integral} into a double integral revealing $\detpredfun$. Unfortunately, it is often difficult to obtain a closed form for the resulting expression (see, \eg, \cref{prop: [independent marginals] uniform marginals case}), but one can rely on numerical integration instead (see \cref{fig:3D plot} in \cref{app: sec3}). 
