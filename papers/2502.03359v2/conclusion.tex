\section{Conclusion}
\label{sec:conclusion}
In this paper, we propose GHOST, our Gaussian Hypothesis Open-Set Technique, which follows the formal definition of provable open-set theory for deep networks, based on the theorems by \citet{bendale2016openmax}. 
We hypothesize that using per-class, per-dimension Gaussian models of feature vectors to normalize raw network logits can effectively differentiate unknown samples and improve OSR performance.    
Although this remains a hypothesis, it may be valuable for future work to explore mean-field theory as a means to formally prove it. 
By utilizing Gaussian models, we move away from traditional assumptions that rely on distance metrics in high-dimensional spaces.
Instead, we normalize logits through a sum of z-scores.
These Gaussian models are more robust to outliers, which can significantly affect extreme value-based statistics.
We demonstrate this on two distinct architectures, providing strong support for our assumption.

Our experiments provide compelling evidence, setting a new state-of-the-art performance. 
Using both networks, we achieve superior results in AUOSCR and AUROC with ImageNet-1K as knowns and four datasets (and more in the supplemental) as unknowns. 
In nearly all cases, GHOST outperforms all methods, with performance gains being statistically very significant (shown in the supplemental).
Furthermore, GHOST is computationally efficient and easy to store, requiring only the mean and standard deviation (i.e., two floats per feature per class). 
A pre-trained network requires just one pass over the validation or training data to compute the GHOST model, and its test-time complexity is $\mathcal{O}(1)$. 

We are the first to investigate fairness in OSR by examining class-wise performance differences, and we hope to encourage research that incorporates more fairness-related metrics for OSR.
We have shown that GHOST maintains the closed-set unfairness of the original classifier across most FPRs, whereas other algorithms struggle significantly, increasing unfairness even at moderate FPRs.
