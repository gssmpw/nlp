\documentclass[journal]{IEEEtran}
\usepackage{amsmath}
\usepackage{cite}
\usepackage{amssymb}
\usepackage{array}
\usepackage{amsfonts}
\usepackage{algorithm}
\usepackage{algpseudocode}
\usepackage{comment}
%\usepackage{algpseudocode}
\usepackage{dsfont}
\usepackage{graphicx}
\usepackage{epsfig}
\usepackage{subfigure}
\usepackage{float}
\usepackage{epstopdf}
\usepackage{psfrag}
\usepackage{xcolor}
\usepackage{url}
\usepackage[colorlinks,linkcolor=black,urlcolor=black,anchorcolor=black,citecolor=black,hyperfootnotes=true]{hyperref}
\newtheorem{example}{Example}
\newtheorem{corollary}{Corollary}
\newtheorem{definition}{Definition}
\newtheorem{lemma}{Lemma}
\newtheorem{theorem}{Theorem}
\newtheorem{proposition}{Proposition}
\newtheorem{remark}{Remark}
\newcommand{\mv}[1]{\mbox{\boldmath{$ #1 $}}}
\renewcommand{\algorithmicrequire}{\textbf{Input:}}
\renewcommand{\algorithmicensure}{\textbf{Output:}}

\usepackage{mathtools}
\DeclarePairedDelimiter\ceil{\lceil}{\rceil}
\DeclarePairedDelimiter\floor{\lfloor}{\rfloor}


\setlength{\skip\footins}{7pt}

\title{{Intelligent Reflecting Surface Based Localization of Mixed Near-Field and Far-Field Targets}
\thanks{
This paper was presented in part at the IEEE International Conference on Wireless Communications and Signal Processing (WCSP) 2024 \cite{Wang_2024_WCSP}.}
\thanks{W. Zhu, Q. Wang, S. Zhang, and L. Liu are with the Department of Electrical and Electronic Engineering, The Hong Kong Polytechnic University, Hong Kong SAR, China (e-mails: \{eee-wf.zhu, shuowen.zhang, liang-eie.liu\}@polyu.edu.hk, qipeng.wang@connect.polyu.hk).}
\thanks{B. Di is with the Department of Electronics, Peking University, Beijing 100871, China (e-mail: diboya@pku.edu.cn).}
\thanks{Yonina C. Eldar is with the Faculty of Mathematics and Computer Science, The Weizmann Institute of Science, Rehovot 7610001, Israel (e-mail: yonina.eldar@weizmann.ac.il).}
}
\author{\IEEEauthorblockN{Weifeng Zhu, Qipeng Wang, Shuowen Zhang, Boya Di, Liang Liu, and Yonina C. Eldar}}





\begin{document}
		\maketitle %\thispagestyle{empty} 

%\begin{figure}[ht]
%	\centering
%	\subfigure[An example of the proposed IRS-assisted bistatic ISAC system: The AOAs from the targets to the receive BS cannot be estimated because no LOS paths exist between them. Instead, we aim to estimate the AOAs and ranges of the paths from the targets to the IRS, but based on the signals received by the receive BS.]{\includegraphics[width=9cm]{SystemModel.eps}\label{fig:system model 3d}}
%	\subfigure[A vertical view of the proposed IRS-assisted ISAC system with $2$ targets. First, the distance of the path between the $k$-th target and the reference IRS element is denoted by $d_k^\text{IT}$. Second, the AOA of the path from the $k$-th target to the IRS is denoted by $\theta_k$. Last, the AOD of the path from the IRS to the receive BS and the AOA of the path from the IRS to the receive BS are denoted by $\xi$ and $\varphi$.]{\includegraphics[width=9cm]{vertical_view.pdf}\label{fig:system model 2d}}
%    \caption{Illustrations of the proposed IRS-assisted bi-static ISAC system.} \label{fig:system models}
%\end{figure}

\begin{abstract}
%In conventional bi-static sensing systems where the transmitter and the receiver are located at different locations, e.g., bi-static radars, the receiver can localize each target based on the propagation delay of the transmitter-target-receiver path, as well as the angle-of-arrival (AOA) information from each target to the receiver. This architecture relies heavily on the existence of line-of-sight (LOS) paths between the receiver and the targets. In this paper, we consider a more challenging bi-static sensing setup in the intelligent reflecting surface (IRS) assisted integrated sensing and communication (ISAC) system, where LOS paths between the targets and the receiver do not exist, while the orthogonal frequency division multiplexing (OFDM) signals emitted by the transmitter can arrive at the receiver via the transmitter-targets-IRS-receiver path. The receiver can perform ISAC to its received signals via decoding transmitter messages and localizing the targets simultaneously. Because there are no LOS links between the targets and the receiver, it is impossible to estimate the propagation delay and the AOA of each target to the receiver, as in traditional bi-static sensing systems. Instead, in this paper, we propose to treat the IRS as the \emph{passive anchor} so as to localize targets, because the LOS paths between the targets and the IRS exist. Here, the main challenge is that the passive IRS cannot process the incident signals reflected from the targets for estimating the signal propagation delay and AOA, and we have to estimate the range and AOA information regarding to the IRS using the signals received by the receiver. To tackle this challenge arising from the passive nature of the IRS, this paper proposes a three-phase localization protocol. Specifically, in the first phase, the receiver estimates the multi-path channel based on the received signals. In the second phase, the receiver aims to extract the range and AOA information of the targets with respect to the IRS and/or the transmitter from the estimated channels. In the third phase, the receiver aims to localize the targets based on the above range and AOA information. Numerical results show that the three-phase localization protocol can achieve very high localization accuracy, indicating that the IRS can be leveraged as a powerful passive anchor in future bi-static ISAC networks.
This paper considers an intelligent reflecting surface (IRS)-assisted bi-static localization architecture for the sixth-generation (6G) integrated sensing and communication (ISAC) network. The system consists of a transmit user, a receive base station (BS), an IRS, and multiple targets in either the far-field or near-field region of the IRS. In particular, we focus on the challenging scenario where the line-of-sight (LOS) paths between targets and the BS are blocked, such that the emitted orthogonal frequency division multiplexing (OFDM) signals from the user reach the BS merely via the user-target-IRS-BS path. Based on the signals received by the BS, our goal is to localize the targets by estimating their relative positions to the IRS, instead of to the BS. We show that subspace-based methods, such as the multiple signal classification (MUSIC) algorithm, can be applied onto the BS's received signals to estimate the relative states from the targets to the IRS. To this end, we create a virtual signal via combining user-target-IRS-BS channels over various time slots. By applying MUSIC on such a virtual signal, we are able to detect the far-field targets and the near-field targets, and estimate the angle-of-arrivals (AOAs) and/or ranges from the targets to the IRS. Furthermore, we theoretically verify that the proposed method can perfectly estimate the relative states from the targets to the IRS in the ideal case with infinite coherence blocks. Numerical results verify the effectiveness of our proposed IRS-assisted localization scheme. Our paper demonstrates the potential of employing passive anchors, i.e., IRSs, to improve the sensing coverage of the active anchors, i.e., BSs.

%where a single-antenna user emits orthogonal frequency division multiplexing (OFDM) signals in the uplink, and a multi-antenna base station (BS) aims at localizing multiple remote targets with the assistance of intelligent reflecting surfaces (IRSs). In particular, we focus on the scenario where the line-of-sight (LOS) paths between the targets and the BS are blocked, while an intelligent reflecting surface (IRS) is deployed at a proper site with LOS paths to the targets, such that the BS's received signals are merely via the user-target-IRS-BS path.  
%%However, the number of supported targets is usually limited by the rank of the BS-IRS channel by classic array signal processing techniques. To cope with the limitation, the challengeable rank-deficient case is also investigated as well as the rank-sufficient case. 
%Stemming from the absence of the LOS target-BS paths and the passive nature of the IRS, our goal is to employ the BS's received signals to estimate the targets' range and angle-of-arrival (AOA) information regarding to the IRS, i.e., the reference node is the IRS, but signal processing is performed by the BS. We propose an efficient algorithm that is able to detect the targets that are in the near-field region and far-filed region of the IRS, and localize these two types of targets separately. 
%Numerical results verify the effectiveness of our proposed IRS-aided localization scheme with meter-level precision. Our paper demonstrates the great potential of employing passive anchors, i.e., IRSs, to improve the sensing coverage of the active anchors, i.e., BSs.
\end{abstract} 
\begin{IEEEkeywords}
Integrated sensing and communication (ISAC), intelligent reflecting surface (IRS), orthogonal frequency division multiplexing (OFDM), the sixth-generation (6G) cellular network, mixed near-field and far-field targets, multiple signal classification (MUSIC).
\end{IEEEkeywords}


\section{Introduction}

%\subsection{Motivations}

Integrated sensing and communication (ISAC) is one of the primary use cases of the future sixth-generation (6G) cellular systems \cite{itu} and has attracted tremendous attention in the literature \cite{isac_survey1,isac_survey2,Zhang_2021_JSTSP,Liu_2022_CST}. It is expected that the future 6G network will provide not only high-quality communication services, but also new sensing functionalities such as high-accuracy localization, large-scale tracking, and high-resolution imaging. However, how to effectively integrate the above sensing functionalities into a communication-oriented network remains an open problem. 

Among various sensing services in 6G ISAC systems, localization is considered to be one of the most crucial functionalities. It has wide-ranging applications and also serves as the foundation to other sensing tasks such as tracking and imaging. In 6G ISAC systems, base stations (BSs) are expected to be the primary anchors to perform localization, because they are deployed at known positions and can emit and process radio signals for estimating the range and angle-of-arrival (AOA) information of targets \cite{Zhao_2020_TSP,Shen_2012_TWC}. 
However, in many scenarios, line-of-sight (LOS) paths between BSs and targets are blocked by a variety of obstacles, challenging existing BS-centric localization algorithms \cite{dvc}. % In practice, it is not a feasible solution to densely deploy the BSs to cover all the targets with LOS paths.
%To cope with the issue, the low-cost and widely deployed IRSs have the potential to enlarge the sensing coverage of the BSs via reflecting the signals from the targets to the BSs \cite{Emil_2022_SPM,Alex_2024_VTM,Shao_2022_JSAC}.

\begin{figure}[t]
	\centering
    \includegraphics[width=.45\textwidth]{SystemModel.eps}
    \vspace{-0.3cm}
    \caption{System model for IRS-assisted bi-static localization in a 6G ISAC network, where LOS paths between the targets and the BS are blocked, and the BS's received signals are merely over the user-target-IRS-BS path. Therein, targets are in both of the near-field and far-field regions of the IRS. We utilize the IRS as a passive anchor for localization. %Specifically, the IRS serves as the reference node to localize the targets, but the distance and the AOA from each target to the IRS have to be estimated based on the signals received by the BS, because of the passive nature of the IRS. 
    }\label{fig:system model}
    \vspace{-0.5cm}
\end{figure}

In this paper, we consider the possibility of employing an intelligent reflecting surface (IRS) to localize the targets that are in the non-line-of-sight (NLOS) region of the BS. Specifically, as shown in Fig. \ref{fig:system model}, we study a bi-static localization system in a 6G network, which consists of a single-antenna user as the transmitter, a multi-antenna BS as the receiver, an IRS to assist the BS, and multiple passive targets to be localized. It is assumed that the LOS paths between the targets and the BS are blocked, while the IRS is deployed at a proper site with LOS paths to the targets and the BS. Moreover, because of the large size of the IRS, some targets are in the near-field region of the IRS, while others are in the far-field region of the IRS \cite{Zhang_2023_CM,near_field_survey}. Under this system, the user emits orthogonal frequency division multiplexing (OFDM) signals in the uplink \cite{3gpp_2024_nr}, while the BS receives the echo signals from the targets merely over the user-target-IRS-BS paths. Our goal is to exploit the BS’s received signals to detect the near-field and far-field targets and localize them based on their range and AOA information in reference to the IRS. This is feasible because the target-IRS-BS channels are functions of these AOAs and distances. Since the reference point to localize the targets is the IRS, but the AOAs and distances from the targets to the IRS are estimated by the BS, the IRS plays the role of a \emph{passive anchor} \cite{Liu_2024_CM} in our considered 6G localization system. 

%In the literature, several works have studied the fundamental limits for localization when the IRS plays the role of the passive anchor \cite{assup_known_channel2,Elzanaty_2021_TSP,Buzzi_2022_TSP}.
In a localization system without LOS paths between the targets and the BS as shown in Fig. \ref{fig:system model}, the main challenge for employing the IRS as a passive anchor lies in how to estimate the AOAs and distances from the targets to the IRS based on the signals received by the BS, instead of those received by the IRS (since the passive IRS cannot perform signal processing). A breakthrough was made in our previous works \cite{Wang_2022_GC, Wang_2024_TWC}, which verifies the feasibility to estimate the distances between the passive targets and the IRS for the first time in the literature. Specifically, the BS can estimate the overall propagation delay (distance) from a target to the IRS to the BS. Because the distance between the BS and the IRS is known, the distance between a target and the IRS can be then obtained by subtracting the IRS-BS distance from the target-IRS-BS overall distance. As a result, the remaining scientific problem for IRS-assisted localization as shown in Fig. \ref{fig:system model} is how to estimate the AOAs from the targets to the IRS based on the BS's received signals.

%Different from \cite{Wang_2022_GC, Wang_2024_TWC} where the overall distance of the target-IRS-BS path estimated by the BS is useful to estimate the target-IRS distance, the angle information of the target-IRS-BS path that can be estimated by the BS, i.e., the AOA from the IRS to the BS, is not useful to estimate the AOAs from the targets to the IRS, as shown in Fig. 1. One may argue that we can first utilize the signals received by the BS for recovering the signals received by the IRS, building upon our knowledge about the IRS-BS channel, and then estimate the AOAs from the targets to the IRS based on existing AOA estimation techniques \cite{aoa_survey}. However, the above approach is practically infeasible, because the number of BS antennas is usually much smaller than that of IRS reflecting elements, and we cannot recover the higher-dimension signals based on the lower-dimension signals.

Recently, several early works \cite{Zhang_2024_TWC,Han_2022_JSTSP,Rahal_2024_JSTSP,Teng_2023_JSAC} have targeted at the AOA estimation problem in IRS-assisted localization systems similar to Fig. \ref{fig:system model}. Specifically, \cite{Zhang_2024_TWC,Han_2022_JSTSP,Rahal_2024_JSTSP} focused on a special case with one target in the system. Under the far-field model, \cite{Zhang_2024_TWC} proposed a deep learning (DL) based algorithm for adaptive AOA estimation. Moreover, under the near-field model, maximum likelihood (ML) based methods are developed in \cite{Han_2022_JSTSP,Rahal_2024_JSTSP}. As to AOA estimation with multiple targets, \cite{Teng_2023_JSAC} proposed a message passing-based algorithm that can estimate the AOAs from far-field targets to the IRS. However, there are some key limitations of the above works \cite{Zhang_2024_TWC,Han_2022_JSTSP,Rahal_2024_JSTSP,Teng_2023_JSAC} for AOA estimation in IRS-assisted localization systems.

1. For the DL-based method proposed in \cite{Zhang_2024_TWC}, the AOA estimation performance cannot be guaranteed in theory.

2. For the ML-based single AOA estimation algorithms proposed \cite{Han_2022_JSTSP,Rahal_2024_JSTSP}, their extension to the multi-target case is of extremely high complexity, because a multi-dimension exhaustive search should be done simultaneously to solve the ML problem.

3. For the message passing-based algorithm proposed in \cite{Teng_2023_JSAC}, it does not work when near-field and far-field targets both exist in the system.

%3. In practice,  some targets are in the near-field region of the IRS, while the others are in the far-field region. In such a case, the above algorithms fail to detect the near-field targets and the far-field targets.

In the literature of AOA estimation, the subspace-based method, such as the multiple signal classification (MUSIC) algorithm \cite{Schmidt_1986_TAP}, is the most widely used method, thanks to its superior performance and low complexity. 
% The main idea of the subspace-based method is to identify internal AOA-related signal components by exploiting the orthogonality between the subspace of signal component and noise from the covariance matrix of received signals. For example, in the conventional $K$-AOAs estimation problem, the received signal at time sample $t$ is given by 
% \begin{equation}
%     \boldsymbol{y}^{(t)} = \sum_{k=1}^{K} \boldsymbol{a}(\theta_k) x_{k}^{(t)} + \boldsymbol{n}^{(t)} = \boldsymbol{A}(\boldsymbol{\theta})\boldsymbol{x}^{(t)} + \boldsymbol{n}^{(t)},
% \end{equation}
% where $\boldsymbol{A}(\boldsymbol{\theta}) = [\boldsymbol{a}(\theta_1),\dots,\boldsymbol{a}(\theta_K)] \in \mathbb{C}^{M_{\rm B} \times K}$ with $\boldsymbol{a}(\theta_k)$ denoting the array response vector of the AOA $\theta_{k}$ and $x^{(t)}_k$ denotes the coefficient of $\boldsymbol{a}(\theta_k)$ that varies across time samples. Under MUSIC, the $K$ AOAs can be estimated by detecting array response vectors contributing to peaks in the so-called ``spatial spectrum"
% \begin{align}
%     P(\theta) = \frac{\boldsymbol{a}(\theta)^H\boldsymbol{a}(\theta)}{\boldsymbol{a}(\theta)^H\boldsymbol{U}_{\rm N}\boldsymbol{U}_{\rm N}^{H}\boldsymbol{a}(\theta)},
% \end{align}
% where $\boldsymbol{U}_{\rm N}$ is the noise subspace of the covariance matrix $\mathbb{E}[\boldsymbol{y}^{(t)}(\boldsymbol{y}^{(t)})^H]$.
It estimates the AOAs of multiple targets based on the orthogonality between the signal subspace and the noise subspace in the covariance matrix of received signals. %Specifically, under the MUSIC algorithm, a sample covariance matrix is first computed based on a sequence of time-varying received signals, an angle spectrum is then characterized based on the relation between the signal and noise eigenvectors of the sample covariance matrix, and the angles leading to the peaks of the spectrum are estimated as the AOAs. 
If MUSIC can be applied in our considered system shown in Fig. \ref{fig:system model}, it is appealing to tackle all of the limitations of methods proposed in \cite{Zhang_2024_TWC,Han_2022_JSTSP,Rahal_2024_JSTSP,Teng_2023_JSAC}. First, different from the model-free DL method \cite{Zhang_2024_TWC}, tremendous works have theoretically and numerically verify the performance of MUSIC \cite{Swin_1992_TSP,Porat_1988_TASSP}. 
%Under the subspace-based method, the sample covariance matrix of received signals is first calculated after a sufficient number of time slots, and AOA and/or range estimation can be performed based on the eigenvalue decomposition (EVD) of this covariance matrix. 
Second, compared with ML-based algorithms in \cite{Han_2022_JSTSP,Rahal_2024_JSTSP} that require a multi-dimension exhaustive search in a multi-target setting, the MUSIC algorithm only requires a one-dimension exhaustive search on the spectrum to estimate all the AOAs. %Moreover, \cite{Amini_2005_SPL} has shown that the $K$ AOAs can be perfectly estimated with sufficient samples when the array response vectors of the $K$ targets are linearly independent. 
Last, different from the message passing-based method [20], MUSIC can be applied in the mixed near-field and far-field localization network \cite{Liang_2010_TSP,Jiang_2013_SJ}. %The key observation is that the near-field steering vector reduces to the far-field steering vector if setting the distance to be infinity. Therefore, by applying two dimensional (2D)-MUSIC for both range and AOA estimation, far-field targets can be detected as the ones whose ranges are extremely large \cite{near_field_survey}.

%Intelligent reflective surface (IRS) is recognized as one promising technology to address the issue of the LOS blockage in 6G localization system \cite{Emil_2022_SPM,Alex_2024_VTM,Shao_2022_JSAC}, due to their strong ability for non-line-of-sight (NLOS) link customization and low infrastructure cost. Motivated by these attractive properties, this paper addresses the IRS-assisted localization problem for multiple passive targets with LOS blockage under the bi-static system setup in this work, as shown in Fig. \ref{fig:system model}. Therein, the single-antenna user emits communication signals in the uplink, while the BS receives the echo signals merely over user-targets-IRS-BS paths for localization. Notably, targets can be located in either the far-field or near-field region of the IRS, owing to its large array size. Unfortunately, the LOS blockage hinders us to localize these passive targets by utilizing the BS as the sole anchor. As a solution, the IRS can be additionally identified as the reference anchor, enabling us to localize these mixed near-field and far-field targets by estimating their range and AOA information towards the IRS. While previous works \cite{Zhang_2021_TWC,Dardari_2022_TWC,Zhang_2024_TWC,Teng_2022_JSTSP,Han_2022_JSTSP,Rahal_2024_JSTSP,Yu_2022_TSP,Teng_2023_JSAC} have explored the IRS-assisted localization problem, they concentrate on scenarios with active targets and these solutions are not directly applicable to localize passive targets. In the case with multiple passive targets, two key challenges arise and require to be addressed:
%\begin{itemize}
%  \item[1.] \emph{How to guarantee that all passive targets can be uniquely identified?}
%  \item[2.] \emph{How to simultaneously localize passive targets in mixed far-field and near-field region?}
%\end{itemize}
%The objective of this paper is to provide solutions to these two challenges within our considered IRS-assisted localization system. To the best of knowledge, this work is the first to combat the above two challenges in IRS-assisted localization systems.


%Recently, \cite{music_irs} and \cite{Liu_2024_CM} proposed a novel idea of employing the intelligent reflecting surfaces (IRS) as the passive anchors to enlarge the sensing region of the BSs. Specifically, \cite{music_irs} considered a localization system consisting of a BS, an IRS, and multiple active users, under which the LOS user-BS paths are all blocked, and the BS can receive the users' signals merely via the user-IRS-BS paths. A modified MUltiple SIgnal Classification (MUSIC) algorithm was designed such that the BS can exploit its received signals to estimate the AOAs from the users to the IRS. Therefore, the IRS serves as the reference node to localize the users, but the signal processing job is performed by the BS, instead of the passive IRS. That is why the IRS is defined as a passive anchor. Inspired by the success of employing the passive anchor to localize active targets such as users, this paper aims to study the possibility to utilize the IRS to localize the passive targets that have no LOS paths to the BSs.


%Motivated by the above observation, we considers a bi-static localization system in 6G ISAC network as shown in Fig. 1, where a single-antenna user emits orthogonal frequency division multiplexing (OFDM) ISAC signals in the uplink \cite{3gpp_2024_nr}, %(cite this “NR and NG-RAN overall description; stage-2,” 2021, 3GPP Standard TS 38.300.), 
%while a multi-antenna BS receives the echo signals from the targets merely over non-line-of-sight (NLOS) paths. % due to the lack of the LOS paths between the targets and the BS.
%Due to the lack of the LOS paths between the targets and the BS, it is impossible to leverage the signals received by the BS to estimate the propagation delay and angle-of-arrival (AOA) information of the targets regarding to the BS.
%To tackle this challenge, we propose to deploy an intelligent reflecting surface (IRS) at a proper site that is with LOS paths to the targets, and leverage this IRS as a \emph{passive anchor} to localize the targets. Specifically, the ``passive" nature indicates that the IRS cannot directly process the signals reflected from the targets to localize them. Instead, it reflects its received signals to the BS to perform signal processing. Moreover, the ``anchor" nature indicates that based on the signals received over the user-target-IRS-BS path, the BS can perform localization via estimating the distance and AOA information of the targets regarding to the IRS with known position. 
%Therefore, the objective of this work is to investigate how to localize multiple passive targets with the assistance of the passive anchor of the IRS in the challenging case where the LOS paths between the targets and the BS are blocked.

%We devise a three-phase protocol to enable the IRS-assisted localization. In the first phase, the OFDM channels contributed by the user-target-IRS-BS paths are estimated. In the second phase, we first detect which targets are in the near-field region and which targets are in the far-field region of the IRS, and then estimate the range and AOA information of the near-field and far-field targets regarding to the IRS based on the estimated channels and the MUSIC algorithm. In the third phase, the locations of the targets are estimated based on their distances and AOAs to the IRS. 



%Current ISAC works usually focus on leveraging the powerful BSs as the active anchors to localize the targets by extracting the useful information from the LOS channels between the BS and targets \cite{Mao_2007_wireless}.
%Such position-related information can be roughly classified into range information and direction information, which can usually be measured with the estimated propagation delays and AOAs, respectively. Accurate range and angle estimations are possible in 6G network thanks to its wide signal bandwidth promised by the millimeter wave (mmWave) technique and the large antenna aperture promised by the massive multiple-input multiple-output (MIMO) technique. With the range and angle information, the positions of targets can be finally determined based on the Euclidean geometry rule. However, such BS-only localization systems heavily rely on the existence of the LOS channels between BSs and targets. In practice, it is not a feasible solution to densely deploy the BSs to cover all the targets with LOS paths. To cope with the issue, the low-cost and widely deployed IRSs have the potential to enlarge the sensing coverage of the BSs via reflecting the signals from the targets to the BSs.

%Depending on the placement of the radio emitters and receivers, the generic localization topology can be categorized into mono-static \cite{Xu_2024_JSAC}, bi-static \cite{Fang_2024_TAES}
%, and multi-static \cite{Zhu_2024_GC} settings. Compared with the mono-static and multi-static settings, the bi-static setting is more flexible to various network paradigms and can be applied in broader sensing applications. Localization in bi-static scenario usually requires the receiver to estimate the AOA and angle of departure (AOD) for each target \cite{Zhang_2023_TSP} and then derive the target location based on Euclidean geometry rule. Generally, such localization method relies on the fact that there have at least two BSs with LOS paths to the interested target. If only one BS is available, we can leverage the user equipment (UE) as another anchor to realize target localization in the bi-static sensing systems \cite{Guo_2024_ICASSP}. Since the low-cost UE usually has one single antenna, we can only extract the angular information between the BS and the target, which make us impossible to localize the target. As a solution, wideband signals, such as OFDM signals, can be employed to additionally acquire the range information \cite{bacchielli_2024_arxiv}, which can be utilized with the angular information to solve the task of localization in such bi-static sensing systems. 
%However, the aforementioned works \cite{Zhang_2023_TSP,Guo_2024_ICASSP,bacchielli_2024_arxiv} rely on the existence of the LOS channel between the BSs and the target. 
%%Due to the widely distributed UEs, we can carefully select the UE to ensure the existence of its LOS channel to the targets, while the LOS path between the BS and the targets might be blocked by obstacles like buildings, which makes the existing localization methods fail to work. 
%%As such, how to localize the interested targets that only has NLOS paths to the BS arises as a challenging problem for the ISAC systems. 
%To cope with the issue, the IRS can be deployed at a proper position such that the LOS path from the IRS to the BS and that from the targets can both exist, which gives rise of the IRS-assisted sensing \cite{Emil_2022_SPM,Alex_2024_VTM,Shao_2022_JSAC}. 

%%In the literature, IRS is visioned to provide a new way to alleviate the wireless channel fading impairment and interference, thereby enhancing network capacity and coverage in wireless communication \cite{irs_survey,Di_2020_JSAC,Basar_2019_Access}. From the aspect of localization, IRS is also promising to expand the sensing region of the BS and serves as the reference anchor to provide additional measurements for localization. Building on these, several works have made efforts in the context of localization algorithm design for IRS-assisted systems \cite{Zhang_2021_TWC,Dardari_2022_TWC,Zhang_2024_TWC,Teng_2022_JSTSP,Han_2022_JSTSP,Rahal_2024_JSTSP}.
%%The work \cite{Zhang_2021_TWC} introduces a received signal strength-based method for IRS-assisted multi-user localization, where a reflection pattern optimization algorithm is designed to minimize the false localization probability. Subsequently, the work \cite{Dardari_2022_TWC} proposes to employ an extremely large IRS with multiple tiles and then extract the range information from each tile to the user for localization. Alternatively, the works \cite{Teng_2022_JSTSP,Zhang_2024_TWC} focus on user localization by estimating the AOA information of the user-IRS path. Specifically, the work \cite{Teng_2022_JSTSP} presents a variational Bayesian localization algorithm to estimate the AOAs towards the IRS based on the received downlink signals, while the work \cite{Zhang_2024_TWC} addresses the uplink localization problem with a deep learning-based algorithm, where the BS utilizes deep neural networks to fulfill the tasks of both AOA estimation and localization using the received pilot signals.  
%%%where  the work \cite{Zhang_2024_TWC} considers an uplink localization problem with multiple IRSs and proposes an adaptive beamforming method to gradually refine the estimation of AOAs of the LOS BS-user path and LOS IRS-user paths for user localization. However, the existence of LOS paths between users and the BS is still indispensable to accomplish the localization task in \cite{Zhang_2024_TWC,Zhang_2021_TWC}. In contrast, the localization schemes proposed in works \cite{Dardari_2022_TWC,Teng_2022_JSTSP} can avoid the reliance on the LOS path. The work \cite{Dardari_2022_TWC} considers to utilize a large IRS with multiple tiles to assist the user localization. Then, each user can extract the propagation delays of signal components reflected by different IRS tiles to localize itself by employing IRS tiles as anchors. The authors in \cite{Teng_2022_JSTSP} propose to deploy multiple IRS's to cover targets and then each user can estimate AOAs from different IRS's based on the received signals for localization. 
%%Under the assumption with near-field users, the works \cite{Han_2022_JSTSP,Rahal_2024_JSTSP} propose to estimate both the range and AOA information of users for localization via maximum likelihood estimation (MLE) methods.
%%
%%The aforementioned works \cite{Zhang_2021_TWC,Dardari_2022_TWC,Zhang_2024_TWC,Teng_2022_JSTSP,Han_2022_JSTSP,Rahal_2024_JSTSP} are limited to the case with active targets, which allows to delegate the multi-target problem into multiple single-target localization subproblems. Specifically, the BS employs the simple time-domain de-correlation operation to separate signal components from different targets, facilitating localization for each individual target. In contrast, when facing multiple passive targets, the de-correlation method fails to work. The echo signals from different targets cannot be decoupled by utilizing the similar de-correlation operation, as they all share the same signal component from the emitter. This limitation hinders the unique identification of each target without carefully designed localization methods, leading to degraded localization performance. Although MLE-based methods can also be applied to the case with passive targets, the prohibitive computational complexity associate with the joint multi-dimension search hinders their practical usage. Recently, several works \cite{Yu_2022_TSP,Teng_2023_JSAC} also explore the algorithm design for multi-target localization without relying on de-correlation pre-processing. The \cite{Yu_2022_TSP} leverages the advanced semi-passive IRS with sensing capabilities for localization, which, however, significantly increases the hardware cost and impeds the dense deployment of IRS's. Meanwhile, \cite{Teng_2023_JSAC} proposes a message passing-based algorithm for joint localization and tracking of multiple active targets, while this algorithm is derived under the far-field assumption and cannot be applicable to the scenario with near-field targets. 
%%
%%These above limitations in existing approaches underscore the need for innovative solutions that can effectively address the challenges for localizing multiple passive targets, particularly in mixed near-field and far-field scenarios.







%In \cite{Dardari_2022_TWC}, the single-anchor localization assisted by a large IRS with multiple tiles is considered, where the location of the user is determined based on the estimated ranges from the user to the BS and to all IRS tiles. In contrast, the work \cite{Wang_2024_TWC} focuses on the device-free sensing setup and proposes an efficient algorithm to accurately estimate the distances between targets and the IRS for localization in the heterogeneous networked sensing systems consisting of BSs and IRS's. Alternatively, the work \cite{Zhang_2021_TWC} proposes to derive the range information based on the received signal strength for localization. 
%In contrast to \cite{Dardari_2022_TWC,Wang_2024_TWC,Zhang_2021_TWC}, the work \cite{Zhang_2024_TWC} proposes to localize the targets by estimating the AOAs of the BS-user channel and the IRS-user channel with an active sensing approach, where adaptive beamforming and reflection pattern are designed. Compared with \cite{Dardari_2022_TWC}, the existence of the LOS channel between the target and the BS is indispensable to accomplish the localization task in \cite{Wang_2024_TWC,Zhang_2024_TWC,Zhang_2021_TWC}.
%Then, the works \cite{Han_2022_JSTSP,Rahal_2024_JSTSP} propose to estimate both the range and AOA information to localize the users, while they consider that users are only located in the near-field region of the IRS. 
%In fact, targets are widely distributed in both the near-field and far-field region of the IRS. How to detect both near-field and far-field targets is also challenging in IRS-assisted localization, which has not been studies in the existing IRS-assisted sensing works. 
%Moreover, the works \cite{Zhang_2021_TWC,Zhang_2024_TWC,Han_2022_JSTSP,Rahal_2024_JSTSP} only consider narrowband systems, while the 6G system can also provide large bandwidth to improve the range estimation performance.
%Therefore, this work will address both of these issues.

%In contrast, the work \cite{Wang_2024_TWC} focuses on the device-free sensing setup and propose an algorithm to accurately estimate the distance between the targets and the IRS for localization in the heterogeneous networked sensing systems consisting of BSs and IRS's. However, the LOS paths between the targets and the BS still needs to exist for the proposed methods in \cite{Zhang_2024_TWC,Wang_2024_TWC}. Interestingly, it is revealed in our previous work \cite{music_irs} that the multiple signal classification (MUSIC) algorithm \cite{music} can estimate the angles of arrival (AOAs) from multiple targets to the passive IRS based on the signals received by the BS even if the LOS paths between the BS and the targets are in absence. However, the work \cite{music_irs} considers a narrow band system and just one angle is not enough to localize a target. In 6G networks, OFDM signals are used and the range information can be derived as well along with the AOA information. Intuitively, localization seems to be realized by simply combining both the AOA and range information, but it remains unknown whether the MUSIC algorithm still works in IRS-assisted OFDM localization. Moreover, due to the large size of the IRS, it is likely that some targets lie in the near-field regime of the IRS. How to detect both near-field and far-field targets is also challenging in IRS-assisted localization, which has not been considered in \cite{music_irs}. This paper will address these issues.     



%In this paper, we consider a bi-static localization system in the 6G network as shown in Fig. \ref{fig:system model}, which consists of a single-antenna user as the transmitter, a multi-antenna BS as the receiver, an IRS as the passive anchor, and multiple passive targets to be localized. It is assumed that the LOS paths between the targets and the BS are blocked, while the IRS is deployed at a proper site with LOS paths to the targets. Moreover, because of the large size of the IRS, some targets are in the near-field region of the IRS, while the others are in the far-field region of the IRS. Under this system, the user emits orthogonal frequency division multiplexing (OFDM) signals in the uplink \cite{3gpp_2024_nr}, while the BS receives the echo signals from the targets merely over the user-target-IRS-BS paths. Our goal is to exploit the BS's received signals to detect the near-field targets and the far-field targets and localize them based on their range and AOA information regarding to the IRS.

%This paper addresses the challenging IRS-assisted localization problem with multiple mixed far-field and near-field passive targets. We employ the orthogonal frequency division multiplexing (OFDM) technique to derive the range measurements of targets, which also allows to classify targets into non-overlap subsets based on their range information. As such, we can then perform localization for each subset of targets independently, resulting in improved localization efficiency when there is a large number of targets. In particular, this paper adopts the subspace method to estimate the AOA information of far-field targets, as well as the AOA and range information of near-field targets, which can typically achieves better performance than the conventional compressed sensing (CS)-based methods \cite{Tropp_2006_SP}. Additionally, the subspace method facilitate the theoretical analysis of the conditions for unique identification of multiple targets, which is crucial for designing the IRS reflection pattern.
%The main contributions of this work can be summarized as follows:

Motivated by the above, this paper aims to investigate how to perform the MUSIC algorithm to detect the near-field targets and the far-field targets and estimate the AOAs and/or distances from the targets to the IRS in our considered localization system as shown in Fig. \ref{fig:system model}. However, it is non-trivial to achieve the above goal. Specifically, if we obtain the sample covariance matrix of the signals received by all BS antennas and analyze the so-called ``spatial spectrum" corresponding to this covariance matrix, we can just estimate the AOA from the IRS to the BS. This is because the incident signals of the BS are from the IRS as shown in Fig. \ref{fig:system model}. How to modify the MUSIC algorithm such that it can estimate the AOAs and/or distances from the targets to the IRS is a challenge. Moreover, the MUSIC algorithm is mainly applied in narrowband systems. However, current cellular network is a broadband system. How to apply the MUSIC algorithm in an OFDM-based broadband localization system is a challenge.

In this paper, we manage to tackle all the above challenges in our considered localization system shown in Fig. \ref{fig:system model}. The contributions of this paper are listed as below. 

1) We devise a novel three-phase localization protocol tailored for the considered system. In Phase I, the channel impulse response (CIR) of our considered OFDM system, containing the strength and delay information of each channel tap, is estimated. In Phase II, we first estimate the range information of user-target-IRS-BS paths based on the delay information of non-zero channel taps. Subsequently, because the user-target-IRS-BS channel with a delay tap can be viewed as a narrowband channel, we can apply the subspace-based method to detect the far-field and near-field targets and estimate their AOAs (far-field targets) or AOAs plus ranges (near-field targets) to the IRS. In Phase III, all targets are localized based on the range and AOA estimations obtained in Phase II.

2) The main challenge of our proposed three-phase protocol lies in Phase II - based on the channels associated with the user-target-IRS-BS paths estimated in Phase I, how to utilize the MUSIC algorithm to detect the far-field and near-field targets and estimate their AOAs and/or ranges to the IRS. As mentioned in the above, if we directly apply the MUSIC algorithm to the estimated user-target-IRS-BS channel vectors, the AOA from the IRS to the BS will be estimated, which is not of our interests. We propose a novel approach to tackle this challenge. Specifically, we find a way to design the IRS reflection coefficients over different time slots of the same coherence block such that after combining the user-target-IRS-BS channels over several time slots of a coherence block into a virtual channel vector, this new channel vector can be surprisingly expressed as a function of the virtual steering vectors of the IRS towards the targets. Consequently, we can apply the MUSIC algorithm onto this properly designed virtual vector to accomplish the goals of Phase II under our proposed localization scheme. In particular, the proposed solution can be considered as a generalization of the method proposed in \cite{music_irs}, which considers narrowband systems with far-field targets.
%Our second contribution is to reveal the mathematical relationship between passive targets and the IRS in the received signals and propose a novel IRS reflecting pattern design with a theoretical guarantee for unique identification of targets. It is noted that all targets cannot be surely identified by following the traditional multi-antenna localization methods even if the BS has a large number of antennas. To deal with this issue, we build an augment multi-antenna signal model by concatenating received signals over several OFDM symbols together. Under such augment multi-antenna system, we analyze the sufficient condition for unique identification of targets based on the subspace method and further propose a reflecting pattern design of the IRS to meet this condition.
%The main challenge of our proposed three-phase protocol lies in Phase II, because although the relation between the targets and the IRS is embedded in the IRS's received signals, we have to reveal such relation based on the signals received by the BS. If we directly apply the MUSIC algorithm to the BS's received signals, we can only know the relation between the BS and the IRS. Our second contribution is to propose an innovative method for creating a virtual signal at the BS side, from which the MUSIC algorithm can surprisingly reveal the relation between the targets and the IRS. Therein, the channel vectors over some OFDM symbols in each coherence block are concatenated together into a virtual channel vector for the extraction of the range and AOA information between targets and the IRS.

3) Other than signal processing design, we also provide the theoretical performance analysis for the MUSIC algorithm proposed in Phase II of our scheme. In the ideal case with infinite coherence blocks, we rigorously show that the AOA information of far-field targets and the AOA and range information of near-field targets can be perfectly estimated under the subspace-based method. To our best knowledge, this is the first theoretical result to justify the effectiveness of employing IRS as the passive anchor to localize the targets. Moreover, numerical results also show that our proposed subspace-based localization scheme outperforms other non-subspace-based counterparts.

%3) We propose a customized MUSIC algorithm to perform target detection and estimate the AOA and/or range estimation of each target for each range cluster. Compared with conventional compressed sensing (CS)-based AOA estimation methods, the proposed subspace-based algorithm can achieve significantly enhanced estimation accuracy with low complexity. Meanwhile, the proposed algorithm is able to detect both of the far-field and near-field targets simultaneously. In particular, we also theoretically analyze the sufficient condition for realizing perfect target estimation under the proposed subspace-based algorithm, which is subsequently utilized for the IRS reflecting pattern design in the considered system.
%The third contribution is to propose a customized low-complexity subspace algorithm, named prior information-assisted multiple signal classification (MUSIC) algorithm, to estimate the AOA and range information of targets in Phase II. By leveraging the prior knowledge that each range cluster only occupies a fraction of the entire area, the searching complexity of the MUSIC algorithm can be significantly reduced. Extensive numerical results demonstrate that more than 94\% search grids can be saved by the proposed algorithm when the bandwidth exceeds $100$ MHz. Furthermore, It is also numerically verified that the proposed MUSIC algorithm can achieve significantly superior performance to the conventional CS-based algorithm.

%1) Our first contribution is to devise a novel three-phase localization protocol to enable the IRS to act as a passive anchor for localizing the targets in our considered system. In the first phase, the OFDM channels contributed by the user-target-IRS-BS paths are estimated and the range of each path is estimated. In the second phase, we first detect which targets are in the near-field region and which targets are in the far-field region of the IRS, and then estimate near-field targets' ranges and AOAs to the IRS as well as far-field targets' AOAs to the IRS based on the MUSIC algorithm. In the third phase, the locations of the targets are estimated based on the range and AOA information obtained in the first two phases.
%
%2) The main challenge of our proposed three-phase protocol lies in Phase II, because although the relation between the targets and the IRS is embedded in the IRS's received signals, we have to reveal such relation based on the signals received by the BS. If we directly apply the MUSIC algorithm to the BS's received signals, we can only know the relation between the BS and the IRS. Our second contribution is to propose an innovative method for creating a virtual signal at the BS side, from which the MUSIC algorithm can surprisingly reveal the relation between the targets and the IRS. Therein, the channel vectors over some OFDM symbols in each coherence block are concatenated together into a virtual channel vector for the extraction of the range and AOA information between targets and the IRS.
%
%3) Building upon the new virtual signal, our third contribution is to design a low-complexity MUSIC algorithm in Phase II. We first calculate the one-dimension (1D) and two-dimension (2D) MUSIC spectrums for the far-field and near-field targets, respectively based on the virtual channel vector. Then, we search the peaks over the 1D spectrum to extract the AOAs of far-field targets, and the peaks over the 2D spectrum to obtain the AOAs and ranges of near-field targets. Instead of searching the whole AOA and range space, we propose to leverage the path range information estimated in Phase I to narrow the search space for the MUSIC algorithm, which results in significantly reduced complexity.

%\begin{enumerate}
%  \item Our first contribution is to devise a novel three-phase localization protocol to enable the IRS to act as a passive anchor for localizing the targets in our considered system. In the first phase, the OFDM channels contributed by the user-target-IRS-BS paths are estimated and the range of each path is estimated. In the second phase, we first detect which targets are in the near-field region and which targets are in the far-field region of the IRS, and then estimate near-field targets' ranges and AOAs to the IRS as well as far-field targets' AOAs to the IRS based on the MUSIC algorithm. In the third phase, the locations of the targets are estimated based on the range and AOA information obtained in the first two phases.
%      %and the MUSIC algorithm
%  \item %To deal with the challenging tasks of Phase II, we propose an innovative scheme to extract the AOA and range information of targets based on the MUSIC technique. Specifically, we first build a novel virtual signal model for the user-target-IRS-BS channels and show that the appropriate reflection pattern design of the IRS can guarantee the successful detection of the interested targets via MUSIC. Then, we estimate the target number based on the covariance-based method and leverage the one-dimension (1D) the two-dimension (2D) MUSIC algorithms to estimate the range and/or AOA information of the far-field targets and near-field targets, respectively. 
%      The main challenge of our proposed three-phase protocol lies in Phase II, because although the relation between the targets and the IRS is embedded in the IRS's received signals, we have to reveal such relation based on the signals received by the BS. If we directly apply the MUSIC algorithm to the BS's received signals, we can only know the relation between the BS and the IRS. Our second contribution is to propose an innovative method for creating a virtual signal at the BS side, from which the MUSIC algorithm can surprisingly reveal the relation between the targets and the IRS. Therein, the channel vectors over all the OFDM symbols in each coherence block together into a virtual channel vector for the extraction of the range and AOA information between the targets and the IRS.
%  \item %To reduce the searching complexity of the standard MUSIC algorithm, we resort to the prior information including the range cluster estimation in Phase I and the near-far field region to significantly narrow the searching space. We also give various numerical results to verify the searching efficiency of the proposed prior information-assisted MUSIC algorithm. It is shown that exploiting the prior information can greatly reduce the average number of search grids for the near-field targets, e.g., more than $98$\% search grids can be saved with $400$ MHz, while bring about a minor reduction over the number of search grids for far-field targets.
%      Building upon the new virtual signal, our third contribution is to design a low-complexity MUSIC algorithm in Phase II. We first calculate the 1D and 2D MUSIC spectrums for the far-field and near-field targets, respectively based on the received signals. Then we can search the peaks over the 1D spectrum to extract the AOAs of far-field targets, and the peaks over the 2D spectrum to obtain the AOAs and ranges of near-field targets. In contrast to search whole AOA and range space, we propose to leverage the path range information estimated in Phase I to narrow the search space in the MUSIC algorithm, which results in significantly reduced complexity.
%      %(First introduce how you detect near-field and far-field targets. Second, say you perform 1D music to estimate AOA of far-field targets. you perform 2D music to estimate both AOA and range of near-field targets. you use path range estimated in Phase I to reduce complexity.)
%\end{enumerate}

%Extensive simulation results show that the proposed three-phase localization protocol can accurately localize multiple targets in the meter level for the IRS-assisted sensing systems, which validates the feasibility and the effectiveness of exploiting the IRS as the passive anchor for the localization tasks.


The rest of the paper is organized as follows. Section II introduces the system model of the IRS-assisted sensing system. In Section III, the three-phase localization protocol is presented. Then the specific target detection scheme in Phase II is proposed to extract the AOA and range information of the near-field and far-field targets in Section IV. Finally, Section V evaluates the performance of the proposed localization methods and Section VII concludes this work.

{\it Notations}: In this paper, vectors and matrices are denoted by boldface lower-case letters and boldface upper-case letters, respectively. For a complex vector $\boldsymbol{x}$, $||\boldsymbol{x}||_q$ and $x_n$ denote the $l_q$-norm and the $n$-th element, respectively. For an $M \times N$ matrix $\boldsymbol{X}$, $\boldsymbol{X}^T$ and $\boldsymbol{X}^{H}$, $\mathcal{N}(\boldsymbol{X})$ denote its transpose, conjugate transpose, and null space, respectively. For a complex number $x \in \mathbb{C}$, its phase is denoted as $\angle x$. Further, $\mathbb{E}_{\boldsymbol{a}}[\cdot]$ denotes the expectation operation over random vector $\boldsymbol{a}$.
%For a square matrix $\boldsymbol{A}$, $|\boldsymbol{A}|$, $\text{tr}(\boldsymbol{A})$, and $\boldsymbol{A}^{-1}$ denote its determinant, trace, and inverse, respectively, and $\boldsymbol{A} \succeq \boldsymbol{0}$ means that $\boldsymbol{A}$ is positive semidefinite. The distribution of a circularly symmetric complex Gaussian random vector with mean $\boldsymbol{\mu}$ and covariance $\boldsymbol{\Sigma}$ is denoted by $\mathcal{CN}(\boldsymbol{\mu},\boldsymbol{\Sigma})$. Further, $\mathbb{E}_{\boldsymbol{a}}[\cdot]$ denotes the expectation operation over random vector $\boldsymbol{a}$, operation $\odot$ denotes Hadamard product of two matrices, and $\mathcal{R}\{\cdot\}$ denote the column space of the argument. Letters $\mathbb{R}^{M \times N}$ and $\mathbb{C}^{M \times N}$ denote the region of real matrix and complex matrix, respectively, which have the size of $M \times N$. Finally, letters $\mathbb{S}^{N}_{+}$ and $\mathbb{S}^{N}_{++}$ denote the region of positive semidefinite matrices and positive definite matrices, respectively, with size $N \times N$.



\section{System Model}\label{sec:system_model}

We consider the localization task in an IRS-assisted 6G cellular system as illustrated in Fig. \ref{fig:system model}, which consists of one single-antenna user as the transmit anchor, one BS equipped with $M_{\text{B}} \ge 1$ antennas as the receive anchor, and $K$ passive targets to be localized. %\textcolor[rgb]{0.00,0.07,1.00}{The set containing all targets in the system is defined as $\mathcal{K} = \{1,\dots,K\}$.}
The 2D coordinates of the user, the BS, and the $k$-th target are denoted as $(x_{\text{U}},y_{\text{U}})$, $(x_{\text{B}},y_{\text{B}})$, and $(x_k,y_k)$, $\forall k$, respectively. We focus on a challenging scenario where the LOS paths between the targets and the BS are blocked by the obstacles such that the targets cannot be localized via the user-target-BS path as in the conventional bi-static radar system. To tackle this challenge, one IRS equipped with $M_\text{I}$ reflecting elements is deployed at a proper location, denoted by $(x_{\text{I}},y_{\text{I}})$, with LOS paths to the targets and the BS, as shown in Fig. \ref{fig:system model}. Under the above setup, the user emits OFDM signals in the uplink \cite{3gpp_2024_nr}, which will be reflected to the BS via the user-target-IRS-BS path. The BS can perform ISAC to its received signals via decoding user messages and localizing the targets simultaneously. Since IRS-assisted communication has been widely studied in the literature, this paper mainly focuses on localizing the targets based on the signals received at the BS. 

\subsection{Signal Propagation Model}

In this paper, we consider the transmission of OFDM symbols over $V$ consecutive coherence blocks, where each coherence block consists of $Q$ OFDM symbols. Moreover, we assume that there are $N$ OFDM sub-carriers and the sub-carrier spacing is $\Delta f$ Hz such that the bandwidth is $B = N\Delta f$ Hz. In each coherence block $t$, let $s_{n,t}^{(q)}$ denote the pilot signal of the user at the $n$-th sub-carrier in each OFDM symbol $q$, $\forall t, n, q$. Then, the frequency-domain pilot signal generated by the user over all the $N$ sub-carriers of the $q$-th symbol in coherence block $t$ is denoted by $\boldsymbol{s}_t^{(q)}=[s_{1,t}^{(q)},\cdots,s_{N,t}^{(q)}]^T$. Assume that the user transmits with identical power at all sub-carriers, which is denoted by $p$. Then, in each coherence block $t$, the time-domain OFDM signal generated by the user in the $q$-th OFDM symbol duration is denoted by
\begin{align}\label{eq:tx signal}
    \boldsymbol{\chi}_t^{(q)}= [\chi_{1,t}^{(q)},\cdots,\chi_{N,t}^{(q)}]^T = \boldsymbol{W}^H\sqrt{p}\boldsymbol{s}_t^{(q)},\quad \forall q,t, 
\end{align}
where $\chi_{n,t}^{(q)}$ denotes the $n$-th temporal-domain sample from the user at the $q$-th OFDM symbol duration in coherence block $t$, and $\boldsymbol{W}$ is the $N\times N$ discrete Fourier transform (DFT) matrix. At the beginning of each OFDM symbol $q$, a cyclic prefix (CP) consisting of $J$ OFDM samples is inserted to eliminate the inter-symbol interference. Therefore, in coherence block $t$, the overall time-domain pilot signal transmitted by the user in the $q$-th OFDM symbol duration is denoted by
\begin{align}\label{eq:tx signal with cp}
    \boldsymbol{\bar{\chi}}_t^{(q)}=[\chi_{N-J+1,t}^{(q)}, \ldots, \chi_{N,t}^{(q)}, \chi_{1,t}^{(q)}, \ldots, \chi_{N,t}^{(q)}]^T, \quad\forall q,t.
\end{align}

Define $\boldsymbol{\bar{G}}\in\mathbb{C}^{M_{\text{B}}\times M_{\text{I}}}$ as the channel from the IRS to the BS, and $\boldsymbol{r}_{k,t} \in \mathbb{C}^{M_{\text{I}} \times 1}$ as the channel of the user-target $k$-IRS path in coherence block $t$, $\forall k,t$. Moreover, define $\phi_{m_\text{I},t}^{(q)}$ as the reflecting coefficient of the $m_\text{I}$-th IRS element with $|\phi_{m_\text{I},t}^{(q)}| = 1$ during the $q$-th OFDM symbol duration in coherence block $t$, and $\boldsymbol{\phi}_t^{(q)}=[\phi_{1,t}^{(q)},\cdots,\phi_{M_\text{I},t}^{(q)}]^T \in \mathbb{C}^{M_\text{I}\times 1}$. Therefore, the effective channel of the user-target $k$-IRS-BS path at the $q$-th OFDM symbol transmission in coherence block $t$ can be defined as
%\vspace{-5pt}
\begin{align}\label{eq:cascaded channel}
    \boldsymbol{h}_{k,t}^{(q)} = \boldsymbol{\bar{G}}\text{diag}(\boldsymbol{\phi}_t^{(q)})\boldsymbol{r}_{k,t}\in\mathbb{C}^{M_{\text{B}} \times 1}, \quad \forall q,t,k.
\end{align}
Then, at the $n$-th sample period of the $q$-th OFDM symbol duration in coherence block $t$, the signal received by the BS is given as\footnote{In the case that the LOS path between the user and the IRS is not blocked, the signals can also be propagated over the user-IRS-BS path. Since the user, the IRS, and the BS are deployed at fixed and known locations, the LOS user-IRS channel and IRS-BS channel are known \cite{assup_known_channel1,assup_known_channel2}. As a result, the signals via the user-IRS-BS path can be canceled from the received signal such that (\ref{eq:time domain received signal}) can be interpreted as the processed received signal that is useful for localizing the targets.}
\vspace{-10pt}
\begin{align}\label{eq:time domain received signal}
    \boldsymbol{\upsilon}_{n,t}^{(q)} = \sum_{k=1}^{K}\boldsymbol{h}_{k,t}^{(q)}\bar{\chi}_{n-l_k,t}^{(q)} + \boldsymbol{z}_{n,t}^{(q)},\quad\forall n,q,t,
\end{align}where $l_k$ denotes the propagation delay (in terms of OFDM samples) of the user-target $k$-IRS-BS path, and $\boldsymbol{z}_{n,t}^{(q)}\sim \mathcal{CN}(0,\sigma^2\boldsymbol{I})\in\mathbb{C}^{M_{\text{B}} \times 1}$ is the Gaussian noise of the BS at the $n$-th sample period of the $q$-th OFDM symbol duration in coherence block $t$. In an $L$-tap multi-path environment where $L$ is the maximum number of resolvable paths, define 
\begin{align}\label{eq:range cluster l}
    \Omega_l = \{k| l_k = l\},~l=1,\ldots,L,
\end{align}
as the set of targets in range cluster $l$, i.e., for all the targets in $\Omega_l$, their echo signals will be propagated to the BS simultaneously with a delay of $l$ OFDM samples. Moreover, define $K_l=|\Omega_l|$ as the number of targets in range cluster $l$. $\forall l$. At the $q$-th OFDM symbol duration in coherence block $t$, define the effective channel from the user to all the targets in range cluster $l$ to the IRS to the BS as
\begin{align}\label{eq:multipath channel l tap}
\boldsymbol{\bar{h}}_{l,t}^{(q)} = \left\{\begin{matrix}
\sum_{k\in\Omega_l}\boldsymbol{h}_{k,t}^{(q)}, & \text{if}~\Omega_l \neq \emptyset, \\
\boldsymbol{0}, & \text{otherwise},
\end{matrix}\right. \quad\forall q,l,t.
\end{align}Moreover, define
\vspace{-5pt}
\begin{align}\label{eq:multipath channel}
    \boldsymbol{\bar{H}}_t^{(q)} = [\boldsymbol{\bar{h}}_{1,t}^{(q)},\cdots,\boldsymbol{\bar{h}}_{L,t}^{(q)}]^T\in \mathbb{C}^{L\times M_{\text{B}}},\quad \forall q,t,
\end{align}
as the $L$-tap multi-path channel from the user to the BS at the $q$-th OFDM symbol duration in coherence block $t$. Then, the time-domain received signal given in \eqref{eq:time domain received signal} can be expressed as
\vspace{-5pt}
\begin{align}
   \boldsymbol{\upsilon}_{n,t}^{(q)} = \sum_{l=1}^{L}\boldsymbol{\bar{h}}_{l,t}^{(q)}\bar{\chi}_{n-l,t}^{(q)} + \boldsymbol{z}_{n,t}^{(q)},\quad\forall n,q,t.
\end{align}After removing CP and performing the DFT operation to the above time-domain signal, the frequency-domain received signal over all the $N$ sub-carriers of the $q$-th OFDM symbol in coherence block $t$ is given as
\vspace{-5pt}
\begin{align}\label{eq:received signal}
    \boldsymbol{\bar{\Upsilon}}_t^{(q)} & = [\boldsymbol{\bar{\upsilon}}_{1,t}^{(q)},\cdots,\boldsymbol{\bar{\upsilon}}_{N,t}^{(q)}]^T \notag \\ &=  \sqrt{p}~ \text{diag}({\boldsymbol{s}}_t^{(q)})\boldsymbol{E}\boldsymbol{\bar{H}}_t^{(q)} + \boldsymbol{\bar{Z}}_t^{(q)},\quad \forall q, 
\end{align}where $\boldsymbol{E}\in\mathbb{C}^{N\times L}$ with the element on the $n$-th row and $l$-th column denoted by $E_{n,l} = e^{-j\frac{2\pi(n-1)(l-1)}{N}}$, and $\boldsymbol{\bar{Z}}_t^{(q)}=[\boldsymbol{z}_{1,t}^{(q)},\cdots,\boldsymbol{z}_{N,t}^{(q)}]^T\in\mathbb{C}^{N\times M_{\text{B}}}$.



\subsection{Channel Model}\label{subsec:channel model}

It is worth noting that the range and AOA information of all the targets are embedded in the multi-path channels $\boldsymbol{\bar{h}}_{l,t}^{(q)}$'s. In the following, we introduce the channel model that reveals the relation between the target locations and the channels. 

Specifically, because the IRS is physically large, its near-field region is in general large such that some targets are in the near-field region of the IRS, and some targets are in the far-field region of the IRS. Therefore, in this paper, we adopt the general near-field steering vector model at the IRS side. In particular, towards a point of distance $d$ and angle $\eta$, let $\boldsymbol{a}_{\text{I}}(d,\eta) = [a_{\text{I},{1}}(d,\eta),\cdots,a_{\text{I},M_{\text{I}}}(d,\eta)]^T$ denote the near-field steering vector of the IRS. For example, if the IRS elements are deployed under the uniform linear array (ULA) model with element spacing $d^{\text{I}}$ meters, the $m_{\text{I}}$-th response in the steering vector can be characterized as \cite{near_field_survey}
\vspace{-5pt}
\begin{align}\label{eq:a_m}
a_{\text{I},m_{\text{I}}}(d,\eta) &= e^{-j\frac{2\pi}{\lambda}(d-\sqrt{d^2+(m_{\text{I}}d^{\text{I}})^2-2m_{\text{I}}d d^{\text{I}}\cos\eta})}, \notag \\
&\approx e^{-j\frac{2\pi}{\lambda} (m_{\text{I}} d^{\text{I}}\cos\eta + \frac{(m_{\text{I}}d^{\text{I}}\sin\eta)^2}{2d})}, ~\forall m_I,
\end{align}where $\lambda$ in meter is the wavelength. Note that mathematically, the far-field steering vector is a special case of the near-field steering vector when $d$ goes to infinity. For example, under the ULA model, we have
\begin{align}\label{eq:a_m far field}
a_{\text{I},m_{\text{I}}}(d\rightarrow \infty,\eta) \!= \! e^{-j\frac{2\pi}{\lambda} m_{\text{I}} d^{\text{I}}\cos\eta}, \forall m_I,
\end{align}which is the response in the classic far-field steering vector. Define 
\begin{align}
& d_k=\sqrt{(x_{\text{I}}-x_k)^2+(y_{\text{I}}-y_k)^2}, \forall k, \label{eqn:distance}\\
& \theta_k=\arctan \frac{y_{\text{I}}-y_k}{x_{\text{I}}-x_k}, ~ \forall k, \label{eqn:AOA}
\end{align}
as the distance and AOA from target $k$ to the IRS, respectively. Then, the channel of the user-target $k$-IRS path at coherence block $t$ is given by \cite{near_field_survey}
\begin{align}\label{eq:nlos channel k}
\boldsymbol{r}_{k,t} &= \beta_k \gamma_{k,t}\boldsymbol{a}_\text{I}(d_{k},\theta_k),\quad \forall k,t, 
\end{align}where $\beta_k$, $\forall k$, denotes the path loss factor of this path, and $\gamma_{k,t}$ is the radar cross section (RCS) of target $k$ in coherence block $t$. For example, under the Swerling target model \cite{swerling_model1}, $\gamma_{k,t}$'s are Gaussian distributed (their norms are Rayleigh distributed) independent over $k$ and $t$.
 
Next, the LOS MIMO channel between the IRS and the BS is given by
\vspace{-5pt}
\begin{align}\label{eq:los irs bs channel}
    \boldsymbol{\bar{G}} = \delta \boldsymbol{G},
\vspace{-5pt}
\end{align}where $\delta$ is the path-loss factor, and the element on the $m_\text{B}$-th row and $m_\text{I}$-th column of $\boldsymbol{G} \in \mathbb{C}^{M_{\text{B}} \times M_{\text{I}}}$ is denoted by \cite{near_field_survey}
\vspace{-5pt}
\begin{align}\label{eq:near irs bs channel}
    G_{m_{\text{B}},m_\text{I}} = e^{-j\frac{2\pi}{\lambda}d_{m_{\text{B}},m_\text{I}}^{\text{IB}}},\quad \forall m_{\text{B}},m_\text{I},
\end{align}with $d_{m_{\text{B}},m_{\text{I}}}^{\text{IB}}$ in meter being the distance between the $m_{\text{B}}$-th BS antenna and the $m_\text{I}$-th IRS element, $\forall m_{\text{B}},m_{\text{I}}$. In particular, when the BS is in the far-field region of the IRS, the LOS channel between the IRS and the BS can be simplified as
\begin{align}\label{equ:far_irs_bs_channel}
  \boldsymbol{\bar{G}} = \delta \boldsymbol{a}_{\rm B}(d \to \infty,\kappa) \boldsymbol{a}_{\rm I}^{\rm T}(d \to \infty,\xi),
\end{align}
where $\kappa$ and $\xi$ denote the AOA and angle of departure (AOD) of the LOS channel from the IRS to the BS, respectively. Since the locations of all BS antennas and IRS elements are known, we assume that $\boldsymbol{\bar{G}}$ is known in this paper.

\subsection{Problem Statement}

In this paper, our goal is to estimate the location $(x_k,y_k)$ of each target $k$ based on the BS received signals $\boldsymbol{\bar{\Upsilon}}_t^{(q)}$, $\forall q,t$, given in (\ref{eq:received signal}). Because there are no LOS links between targets and the BS, it is impossible to estimate the propagation delay and the AOA from each target to the BS based on $\boldsymbol{\bar{\Upsilon}}_t^{(q)}$, $\forall q,t$. Instead, we treat the IRS as a passive anchor and localize the targets by estimating their range and AOA information with regarding to the IRS, since LOS paths exist between targets and the IRS. 



\section{Three-Phase Localization Protocol}\label{sec:protocol}

This paper adopts a three-phase protocol to exploit the IRS as a passive anchor for localizing the targets. In the first phase, we aim to estimate the CIRs $\boldsymbol{\bar{H}}_t^{(q)}$'s as given in (\ref{eq:multipath channel}) based on the signals received by the BS $\boldsymbol{\bar{\Upsilon}}_t^{(q)}$'s as given in (\ref{eq:received signal}). In the second phase, we aim to extract the range and AOA information of the targets with regarding to the IRS and/or the user from the estimated channels, since $\boldsymbol{\bar{H}}_t^{(q)}$'s are functions of delays and AOAs of the targets as shown in \eqref{eq:multipath channel l tap} and \eqref{eq:nlos channel k}. Specifically, we first detect whether there exist targets in each range cluster $l$. If a target is detected to be in range cluster $l$, then the propagation delay over the user-target-IRS-BS path is of $l$ OFDM samples. Next, given each range cluster $l$ with $\Omega_l \neq \emptyset$, we will design a MUSIC algorithm based approach on $\boldsymbol{\bar{h}}_{l,t}^{(q)}$, $\forall q, t$, to 1. detect the near-field and far-field targets; 2. estimate the AOA and the range from each near-field target to the IRS; and 3. estimate the AOA from each far-field target to the IRS. This is because $\boldsymbol{\bar{h}}_{l,t}^{(q)}$ can be viewed as a narrowband channel contributed by all the targets in range cluster $l$. In the third phase, by treating the IRS as a passive anchor, we aim to localize the targets based on the range and AOA information estimated in the second phase. In the following, we introduce each phase in details.

\subsection{Phase I: CIR Estimation}

Since this paper focuses on localizing static targets, the range cluster sets $\Omega_l$'s given in \eqref{eq:range cluster l} do not change with time. Based on \eqref{eq:multipath channel l tap}, given any $l$ with $\Omega_l = \emptyset$, it follows that the $l$-th row of $\boldsymbol{\bar{H}}_t^{(q)}$ given in \eqref{eq:multipath channel} is a zero vector, i.e., $\boldsymbol{\hat{h}}_{l,t}^{(q)} = \boldsymbol{0}, \forall q, t$. 
Therefore, we can exploit the above sparsity in all $\boldsymbol{\bar{H}}_t^{(q)}$'s to perform channel estimation over the $V$ coherence blocks jointly by solving the following group least absolute shrinkage and selection operator (LASSO) problem \cite{Yuan_2005_JRSSS}
%\begin{subequations}\label{equ:H_est_prob}
%\begin{align}
%   \min_{\{\boldsymbol{\bar{H}}_t^{(q)}\}_{q=1,t=1}^{Q,T}}~ & \sum_{l=1}^{L} \left\| \left\| \boldsymbol{\bar{h}}_{l,t}^{(q)} \right\|_2^2 \right\|_0 \label{equ:obj_H_est_prob} \\
%   \text{s.t.} \quad\quad~ &  \sum_{t=1}^{T}\sum_{q=1}^{Q} \left\| \boldsymbol{\bar{\Upsilon}}_t^{(q)} - \text{diag}(\boldsymbol{s}_t^{(q)})\boldsymbol{E}\boldsymbol{\bar{H}}_t^{(q)} \right\|_\text{F}^2 \leq \epsilon,
%\end{align}
%\end{subequations}
%where $\epsilon$ is the threshold of the fitting error. However, this problem cannot be directly solved due to the non-convex objective function (\ref{equ:obj_H_est_prob}). To tackle this challenge, we can relax the $l_0$-norm by the $l_{2}$-norm and reformulate a regularized least squared (RLS) problem 
%\begin{small}
\begin{align}\label{eq:group_lasso}
    \min_{\{\boldsymbol{\bar{H}}_t^{(q)}\}_{q=1,t=1}^{Q,V}}~ & \sum_{t=1}^{V}\sum_{q=1}^{Q} \left\| \boldsymbol{\bar{\Upsilon}}_t^{(q)} - \text{diag}(\boldsymbol{s}_t^{(q)})\boldsymbol{E}\boldsymbol{\bar{H}}_t^{(q)} \right\|_F^2 \notag \\
    & + \omega  \sum_{l=1}^{L}  \left(\sum_{q=1}^{Q}\sum_{t=1}^{V} \left\|\boldsymbol{\bar{h}}_{l,t}^{(q)}\right\|_2^2 \right)^{\frac{1}{2}}, %\left\|\boldsymbol{\bar{h}}_{l} \right\|_2,
\end{align}
%\end{small}
where $\omega \geq 0$ is some given parameter to control the sparsity of the solution. The above problem (\ref{eq:group_lasso}) is a convex problem, which can be globally solved. Let $\boldsymbol{\tilde{H}}_t^{(q)}=[\boldsymbol{\tilde{h}}_{1,t}^{(q)},\cdots,\boldsymbol{\tilde{h}}_{L,t}^{(q)}]^T$ denote the optimal solution, $\forall q,t$.%, which will be used for the AOA and/or range information derivation in Phase II.
%\begin{align}\label{eq:ori_sparse_prob}
%    \min_{\{\boldsymbol{\bar{H}}_t^{(q)}\}_{q=1,t=1}^{Q,T}}~ &\sum_{t=1}^{T}\sum_{q=1}^{Q} \left\| \boldsymbol{\bar{\Upsilon}}_t^{(q)} - \text{diag}(\boldsymbol{s}_t^{(q)})\boldsymbol{E}\boldsymbol{\bar{H}}_t^{(q)} \right\|_\text{F}^2 \notag \\
%    %& + \omega  \sum_{l=1}^{L}  \left(\sum_{q=1}^{Q}\sum_{t=1}^{T} \left\|\boldsymbol{\bar{h}}_{l,t}^{(q)}\right\|_2^2 \right)^{\frac{1}{2}}. %\left\|\boldsymbol{\bar{h}}_{l} \right\|_2,
%    &+ \alpha \sum_{l=1}^{L} \varpi\left(\left\{\boldsymbol{\bar{h}}_{l,t}^{(q)}\right\}_{q=1,t=1}^{Q,T}\right),
%\end{align}
%%where $\bar{\boldsymbol{h}}_l = \left[(\bar{\boldsymbol{h}}_{l,1}^{(1)})^{\rm T},\dots,(\bar{\boldsymbol{h}}_{l,1}^{(1)})^{\rm T},\dots,(\bar{\boldsymbol{h}}_{l,T}^{(1)})^{\rm T},\dots,(\bar{\boldsymbol{h}}_{l,T}^{(Q)})^{\rm T}\right]^{\rm T}$ denotes the stacked channel vector of the $l$-th path, $\forall l$.
%In the above problem, $\alpha \geq 0$ is a given parameter to balance the fitting error $\sum_{t=1}^{T}\sum_{q=1}^{Q} \left\| \boldsymbol{\bar{\Upsilon}}_t^{(q)} - \text{diag}(\boldsymbol{s}_t^{(q)})\boldsymbol{E}\boldsymbol{\bar{H}}_t^{(q)} \right\|_\text{F}^2$ and the sparsity-inducing penalty function $\varpi\left(\left\{\boldsymbol{\bar{h}}_{l,t}^{(q)}\right\}_{q=1,t=1}^{Q,T}\right)$.



\subsection{Phase II: Range and AOA Estimation for Near-Field and Far-Field Targets}\label{sec:phase_II}

In the second phase, we aim to estimate the target range and AOA information that is embedded in the estimated channels $\boldsymbol{\tilde{H}}_t^{(q)}$, $\forall q,t$. First of all, according to (\ref{eq:multipath channel l tap}), we declare that there are targets in range cluster $l$ if the estimated channels $\boldsymbol{\tilde{h}}_{l,t}^{(q)}$'s are strong, $\forall q,t$, and there are no targets in range cluster $l$ otherwise. Mathematically, the above detection is modeled as
%\vspace{-5pt}
\begin{align}
\hat{\Omega}_l\left\{\begin{array}{ll} \neq \emptyset, & {\rm if} ~ \sum\limits_{t=1}^V\sum\limits_{q=1}^Q \left\|\boldsymbol{\tilde{h}}_{l,t}^{(q)}\right\|^2\geq \rho_l, \\ =\emptyset, & \text{otherwsie}, \end{array}\right. ~ l=1,\dots,L, 
\end{align}where $\hat{\Omega}_l$ is an estimation of the set consisting of all the targets in range cluster $l$, i.e., $\Omega_l$ defined in (\ref{eq:range cluster l}), and $\rho_l>0$ is some pre-designed threshold. 
For convenience, define 
\begin{align}
    \Phi=\{l:\hat{\Omega}_l \neq \emptyset, \forall l\}
\end{align}
as the set consisting of the indices of all the range clusters with targets detected. For each $l \in \Phi$, the propagation delay from the user to all the targets in $\hat{\Omega}_l$ to the IRS to the BS is estimated to be of $l$ OFDM sample durations. Therefore, for each target in range cluster $l \in \Phi$, the range of the corresponding user-target-IRS-BS path is estimated as \cite{dvc}
\begin{align}\label{equ:d_UTIB}
    \hat{d}^{\rm UTIB}_l = \frac{(l-1)c_0}{N \Delta f} + \frac{1}{2N \Delta f} = \frac{(2l-1)c_0}{2N \Delta f}, \forall l,
\end{align}
where $c_0$ denotes the speed of the light. Moreover, based on the locations of the target $k$, the IRS, and the user, the range of the path from the user to target $k$ to the IRS and that from the IRS to the BS are
%\vspace{-5pt}
\begin{align}\label{eq:range user t irs}
    d^{\text{UTI}}(x_k,y_k) =&~ d^{\text{UT}}(x_k,y_k) + d^{\text{TI}}(x_k,y_k) \notag \\
    =&~ \sqrt{(x_{\text{I}}-x_k)^2 + (y_{\text{I}}-y_k)^2} \notag \\
    &~+\sqrt{(x_{\text{U}}-x_k)^2+(y_{\text{U}}-y_k)^2}, \forall k, \\
    d^{\rm IB} =&~ \!\sqrt{(x_{\text{I}}-x_{\text{B}})^2\!+\!(y_{\text{I}}-y_{\text{B}})^2}\!.
\end{align}
Therefore, for a target $k$ that is estimated in range cluster $l \in \Phi$, we have
\begin{align}\label{eqn:sum distance}
 d^{\text{UTI}}(x_k,y_k) &= \hat{d}^{\rm UTIB}_l - d^{\rm IB}+\!\mu_k, ~\forall k\in \hat{\Omega}_l,
 %d^{\rm IB} &= \!\sqrt{(a_{\text{I}}-a_{\text{B}})^2\!+\!(b_{\text{I}}-b_{\text{B}})^2}\!,
\end{align}
where $\mu_k$ denotes the range estimation error for target $k$. In the case of perfect range estimation with $\mu_k=0$, (\ref{eqn:sum distance}) indicates that each target $k\in \hat{\Omega}_l$ should be on an ellipse because the sum of its distance to the IRS and that to the user is fixed. 

\begin{figure}[t]
   \centering
    \includegraphics[width=.4\textwidth]{diagram_NF.eps}
    \vspace{-0.3cm}
    \caption{A toy example for coexistence of near-field and far-field targets, where target $1$ and target $2$ coexist within the range cluster $l$. They are in the different regions for the IRS. Specifically, target $1$ is in the near-field region of the IRS, while target $2$ is in the far-field region of the IRS.}\label{fig:demo_coexistence}
    \vspace{-0.5cm}
\end{figure}

After detecting the range clusters with targets in them, we have three goals listed as below:
\begin{itemize}
    \item {\bf{Goal $1$}}: Estimate the number of targets in each range cluster $l \in \Phi$, which is denoted by $\hat{K}_l = \left|\hat{\Omega}_l\right|$;
    \item {\bf{Goal $2$}}: Given each range cluster $l \in \Phi$, detect the set of targets that are in the far-field region of the IRS, denoted by the set $\hat{\Omega}^{\rm F}_{l}$, and the set of targets that are in the near-field region of the IRS, denoted by the set of $\hat{\Omega}^{\rm N}_{l}$;\footnote{One example is given in Fig. \ref{fig:demo_coexistence} to show that even in the same range cluster, it is likely that some users are in the near-field region of the IRS, while others are in the far-field region.}
    \item {\bf{Goal $3$}}: If a target $k$ is detected to be a far-field target, then we estimate its AOA to the IRS, denoted by $\hat{\theta}_{k}$, based on the relation between the AOA and the far-field steering vector $\boldsymbol{a}_\text{I}(d\to\infty,\theta)$; otherwise, we estimate both its AOA and distance to the IRS, denoted by $\hat{d}_k$ and $\hat{\theta}_{k}$, respectively, based on the relation among the AOA, the distance, and the near-field steering vector $\boldsymbol{a}_\text{I}(d,\theta)$.
\end{itemize}
%Without loss of generality, we propose to achieve the above three goals just based on the estimated channels during the transmission of the 1st OFDM symbol in each coherence block $t$, i.e., $\hat{\boldsymbol{h}}_{l,t}^{(1)}$'s.


Because it takes quite large space to show how to achieve \textbf{Goals} 1-3, in the rest of this section, we will assume that the target number $\hat{K}_l$, far-field target set $\hat{\Omega}^{\rm F}_{l}$, near-field target set $\hat{\Omega}^{\rm N}_{l}$, AOAs $\hat{\theta}_{k}$'s for far-field targets, AOA and range pairs $(\hat{d}_k,\hat{\theta}_k)$ for near-field targets have all been successfully estimated and merely introduce how to localize the targets based on these estimations in Phase III. This may help readers quickly understand the overall process of our proposed three-phase protocol. We will show how to achieve the above three goals of Phase II in Section \ref{sec:rank_two_case}.
%The specific schemes to solve these three problems will be introduced in the next section

%In the literature, there exist plenty of classic signal processing methods to deal with the system described in \eqref{eq:estimated_channel cluster l} \cite{aoa_survey}. However, to make the signal processing methods in \cite{aoa_survey} work, we need to make sure that the effective steering vectors do not change over different coherence blocks. This work adopts the popular MUSIC algorithm \cite{music} to realize the three Goals listed in the above.
%Therefore, the parameters including the AOA and/or range information for the targets in the range cluster $l$ can be estimated by finding which effective steering vectors $\boldsymbol{\psi}^{(q)}_t(\bar{d}_k,\theta_k)$ are contained in the channel $\boldsymbol{\tilde{h}}_{l,t}^{(q)}$.
%However, there is an interesting observation that if the rank of the LOS MIMO channel between the BS and the IRS is smaller than or equal to the number of the targets in the range cluster $l$.
%In this section, we focus on the case where $\text{rank}(\mathbf{G}) > K_l$, where $K_l = |\Omega_{l}|$ is the number of targets in range cluster $l$ to introduce how to apply the MUSIC algorithm for the AOA and/or range information estimation. In this case, we propose to 

%However, to make the signal processing methods in \cite{aoa_survey} work, we need to make sure that the effective steering vectors at the $q$th OFDM symbol duration in different coherence blocks are identical. %do not change over different coherence blocks at least. 
%To satisfy the above condition, we set a common IRS reflecting pattern at the $q$th OFDM symbol duration over all the coherence blocks, i.e.,
%\begin{align}\label{equ:phi_equ_t}
%    \boldsymbol{\phi}_{t}^{(q)}=\boldsymbol{\bar{\phi}}^{(q)},\quad \forall t, q.
%\end{align}
%In this case, the effective steering vectors in (\ref{eq:effect steering}) reduce to
%\begin{align}\label{eq:new array}
%    \boldsymbol{\psi}^{(q)}_t(\bar{d}_k,\theta_k) &= \boldsymbol{\psi}^{(q)}(\bar{d}_k,\theta_k) \notag \\
%    &= \boldsymbol{G}\text{diag}(\boldsymbol{\bar{\phi}}^{(q)})\boldsymbol{a}_\text{I}(\bar{d}_k,\theta_k), \forall k,q,t.
%\end{align}
%and the estimated channels given in (\ref{eq:estimated_channel cluster l}) reduce to
%\begin{align}\label{equ:est_channel_reduce}
%  \tilde{\boldsymbol{h}}_{l,t}^{(q)}=\sum_{k \in \Omega_{l}} v_{k,t} \boldsymbol{\psi}^{(q)}(\bar{d}_k,\theta_k) + \boldsymbol{\tilde{z}}_{l,t}^{(q)}, ~\forall t, l\in \Phi.
%\end{align}
%Then we apply the MUSIC algorithm \cite{music} on \eqref{equ:est_channel_reduce} to achieve the three goals listed in the above. 
%However, how to apply the MUSIC algorithm for the target detection is closely related to the rank of the LOS MIMO channel between the BS and the IRS. 
%Here, the estimation algorithm for the AOA and/or range information of targets in the range cluster $l$ are investigated for two cases


%\subsection{Phase II: Range and AOA Estimation}
%In the second phase, we aim to estimate the target range and AOA information that is embedded in the estimated channels $\boldsymbol{\hat{H}}_t^{(q)}$, $\forall q,t$. First of all, according to (\ref{eq:multipath channel l tap}), we declare that there are targets in range cluster $l$ if the estimated channels $\boldsymbol{\hat{h}}_{l,t}^{(q)}$'s are strong, $\forall q,t$, and there are no targets in range cluster $l$ otherwise. Mathematically, the above detection is modeled as
%\begin{align}
%\hat{\Omega}_l\left\{\begin{array}{ll} \neq \emptyset, & {\rm if} ~ \sum\limits_{t=1}^T\sum\limits_{q=1}^Q \|\boldsymbol{\hat{h}}_{l,t}^{(q)}\|^2\geq \rho_l, \\ =\emptyset, & {\rm otherwsie}, \end{array}\right. ~ l=1,\cdots,L, 
%\end{align}where $\hat{\Omega}_l$ is an estimation of $\Omega_l$ defined in (\ref{eq:range cluster l}), and $\rho_l>0$ is some pre-designed threshold. If $\hat{\Omega}_l$ is detected not to be an empty set for some $l$, it indicates that the propagation delay from the user to all the targets in $\hat{\Omega}_l$ to the IRS to the BS is estimated to be of $l$ OFDM sample durations. Therefore, we have
%\begin{align}\label{eqn:sum distance}
%& \sqrt{(a_{\text{I}}-x_k)^2+(b_{\text{I}}-y_k)^2}+\sqrt{(a_{\text{U}}-x_k)^2+(b_{\text{U}}-y_k)^2} \nonumber \\ =&\frac{lc_0}{N\Delta f}-\sqrt{(a_{\text{I}}-a_{\text{B}})^2+(b_{\text{I}}-b_{\text{B}})^2}+\mu_k, ~ \forall k\in \hat{\Omega}_l,
%\end{align}where $c_0$ denotes the speed of the light, and $\mu_k$ denote the range estimation error. In the case of perfect range estimation with $\mu_k=0$, (\ref{eqn:sum distance}) indicates that each target $k\in \hat{\Omega}_l$ should be on an ellipsoid because the sum of its distance to the IRS and that to the BS is fixed. 
%
%Next, we introduce for each target located in range cluster $l$, how to detect whether this target is in the far-field region of the IRS or the near-field region of the IRS, and how to estimate its AOA to the IRS based on the relation between the AOA and the far-field steering vector $\boldsymbol{a}_\text{I}(d\rightarrow \infty,\theta)$ in the former case, and both of its AOA and distance to the IRS based on the relation among the AOA, the distance, and the near-field steering vector $\boldsymbol{a}_\text{I}(d,\theta)$ in the latter case. (show how to achieve the above gaol in the case when rank of G is sufficiently large and number of time slots is sufficient)

\subsection{Phase III: Localization}

In the third phase, we aim to estimate the locations of the targets, i.e., $(x_k,y_k)$'s, $\forall k$, assuming that \textbf{Goals} 1-3 in Phase II listed in the above have been achieved. 

First, we show how to localize the users that are in the far-field region of the IRS. Given each far-field target $k \in \hat{\Omega}_l^{\rm F}$ with some $l \in \Phi$, the range of the path from the user to it to the IRS to the BS is estimated as $\hat{d}^{\rm UTI}$ according to \eqref{equ:d_UTIB}, and its AOA to the IRS is estimated as $\hat{\theta}_k$ in Phase II. Then, the weighted residual minimization problem to localize the far-field target $k \in \hat{\Omega}_{l}^{\rm F}$ can be formulated as
%\begin{subequations}\label{equ:loc_far_target_k}
\begin{small}
\begin{align}\label{equ:loc_far_target_k}
    \min_{(x_k,y_k)} &~ (1-\varpi_{\rm F}) \left( d^{\rm UTI}(x_k,y_k) + d^{\rm IB} - \hat{d}^{\rm UTIB}_l \right)^2 \notag  \\
    &~+ \varpi_{\rm F} \left( \arctan\frac{y_{\rm I} - y_k}{x_{\rm I} - x_k} - \hat{\theta}_{k} \right)^2, 
\end{align}
%\end{subequations}
%\begin{align}\label{equ:loc_far_target_k}
%    \min_{(x_k,y_k)} &~ (1-\varpi_{\rm F}) \left( d^{\rm UTI}(x_k,y_k) + d^{\rm IB} - \frac{(2l-1) c_0}{2N\Delta f} \right)^2 \label{equ:far_distance_obj} \notag \\
%    &~+ \varpi_{\rm F} \left( \arctan\frac{b_{\rm I} - y_k}{a_{\rm I} - x_k} - \hat{\theta}_{k} \right)^2, \label{equ:far_angle_obj}
%\end{align}
\end{small}where $\varpi_{\rm F} \in [0,1]$ is the tuning parameter to balance the angle estimation residual and the range estimation residual of a far-field target. Then, the unique solution to problem \eqref{equ:loc_far_target_k} can be considered as the estimation of the location of this far-field target $k$. 
%\textcolor[rgb]{0.00,0.07,1.00}{In fact, it is easy to observe that the optimal value of the objective function of the problem \eqref{equ:loc_far_target_k} is equal to zero. As such, the unique solution to lead zero-value to both of the functions \eqref{equ:far_distance_obj} and \eqref{equ:far_angle_obj} is defined as the estimation of the location for this far-field target.}
It can be show that the following solution (consider targets are only located in one side of the IRS) leads to zero-residual to the above problem
\begin{small}
\begin{subequations}
\begin{align}
    \hat{x}_k =& \frac{ \left(\frac{(2l-1) c_0}{2N\Delta f} - d^{\rm IB} \right)^2 - (x_{\rm I} - x_{\rm U})^2 - (y_{\rm I} - y_{\rm U})^2}{2\left(x_{\rm I} - x_{\rm U} + (y_{\rm I} - y_{\rm U})\tan\hat{\theta}_k - \frac{\frac{(2l-1) c_0}{2N\Delta f} - d^{\rm IB}}{\cos\hat{\theta}_k}\right)} + a_{\rm I},  \\
    \hat{y}_k =&  \frac{ \left(\frac{(2l-1) c_0}{2N\Delta f} - d^{\rm IB} \right)^2 - (x_{\rm I} - x_{\rm U})^2 - (y_{\rm I} - y_{\rm U})^2}{2\left(x_{\rm I} - x_{\rm U} + (y_{\rm I} - y_{\rm U})\tan\hat{\theta}_k - \frac{\frac{(2l-1) c_0}{2N\Delta f} - d^{\rm IB}}{\cos\hat{\theta}_k}\right)} \tan\hat{\theta}_k + b_{\rm I}.
\end{align}
\end{subequations}
\end{small}Therefore, the above solution is optimal. 


%Moreover, given each range cluster $l \in \Phi$, some near-field targets may be detected, and their distances and AOAs to the IRS are in the set $\hat{\Xi}_{l}^{\rm N}$. Therefore, for each near-field target $k$, two kinds of distances are available: its distance to the IRS, and the sum of its distance to the IRS and that to the user. 
%Here, we adopt both of the two kind of distances for the near-field target localization and then we can formulate the localization problem as\
Second, we show how to localize the targets that are in the near-field region of the IRS. Given each near-field target $k \in \hat{\Omega}_{l}^{\rm N}$ with some $l \in \Phi$, the range of the path from the target to the IRS is estimated as $\hat{d}_{k}$, and its AOA to the IRS is estimated as $\hat{\theta}_k$ in Phase II. We can then formulate the weighted residue minimization problem to localize the near-field target $k \in \hat{\Omega}_{l}^{\rm N}$ as %(follow far-field case to introduce 25)
\begin{small}
\begin{align}\label{equ:loc_near_target_k}
    \min_{(x_k,y_k)} &~ \varpi_{{\rm N},1}  \left( \arctan\frac{y_{\rm I} - y_k}{x_{\rm I} -  x_k} - \hat{\theta}_{k} \right)^2  \notag \\
    &~+ \varpi_{{\rm N},2} \left( d^{\rm UTI}(x_k,y_k) + d^{\rm IB} - \frac{(2l-1) c_0}{2N\Delta f} \right)^2  \notag \\
    &~+ (1 - \varpi_{{\rm N},1} - \varpi_{{\rm N},2})\left( d^{\rm TI}(x_k,y_k) - \hat{d}_k \right)^2, 
\end{align}
\end{small}where $\varpi_{{\rm N},1} \in [0,1]$ and $\varpi_{{\rm N},2} \in [0,1]$ with $\varpi_{{\rm N},1} + \varpi_{{\rm N},2} \in [0,1]$ are the tuning parameters to balance the estimation error by Algorithm \ref{alg:pri_music} and the range estimation error introduced by the sum distance estimation in Phase I. 
In contrast to the far-field target localization, it is hard to get the closed-form solution for the problem \eqref{equ:loc_near_target_k}, and we can apply Gauss-Newton algorithm \cite{Torrieri_1984_TAES} to solve it.
%Similar to the far-field target localization, the optimal value of the objective function should be zero and then we can get the estimated location of the near-field target as
%\begin{align}\label{equ:est_N_xy}
%  (\hat{x}_k,\hat{y}_k) &= \left(a_{\rm I} + \hat{d}_k \cos \hat{\theta}_k , b_{\rm I} + \hat{d}_k \sin \hat{\theta}_k \right).
%\end{align}

%However, the estimation error of sum distance relies on the channel bandwidth, which can be large when the bandwidth is not sufficiently large. Therefore, we only utilize the distance to the IRS to localize the near-field targets. For the near-field target $k$, suppose its distance and AOA to the IRS are estimated as $(\hat{d}_k, \hat{\theta}_k) \in \hat{\Theta}_{l}^{\rm N}$. Then, via utilizing the IRS as the anchor, its location is estimated as
%\begin{align}\label{equ:est_N_xy}
%  (\hat{x}_k,\hat{y}_k) &= \left(a_{\rm I} + \hat{d}_k \cos \hat{\theta}_k , b_{\rm I} + \hat{d}_k \sin \hat{\theta}_k \right).
%\end{align}

%i.e.,
%\begin{align}\label{equ:theta_F_k}
%  \frac{b_{\rm I} - \hat{y}_k}{a_{\rm I} - \hat{x}_k} = \tan{\hat{\theta}_k}.
%\end{align}
%Moreover, because this target is detected in range cluster $l$, the sum of its distance to the IRS and to the user is estimated as 
%\begin{align}\label{equ:dis_F_k}
%  &\sqrt{(a_{\rm I} - \hat{x}_k)^2 + (b_{\rm I} - \hat{y}_k)^2} + \sqrt{(a_{\rm U} - \hat{x}_k)^2 + (b_{\rm U} - \hat{y}_k)^2} \notag \\
%  &= \frac{l c_0}{N \Delta f} - \sqrt{(a_{\rm B} - a_{\rm I})^2 + (b_{\rm B} - b_{\rm I})^2},
%\end{align}
%where the estimation error is omitted in (\ref{eqn:sum distance}), i.e., $\mu_k = 0$.
%Then, the unique solution to (\ref{equ:theta_F_k}) and (\ref{equ:dis_F_k}) can be defined as the estimation of the location of this far-field target.

%Moreover, given each range cluster $l \in \Phi$, some near-field targets may be detected, and their distances and AOAs to the IRS are in the set $\hat{\Theta}_{l}^{\rm N}$. Therefore, for each near-field target, two kinds of distances are available: its distance to the IRS, and the sum of its distance to the IRS and that to the user. However, the estimation error of sum distance relies on the channel bandwidth, which can be large when the bandwidth is not sufficiently large. Therefore, we only utilize the distance to the IRS to localize the near-field targets. For the near-field target $k$, suppose its distance and AOA to the IRS are estimated as $(\hat{d}_k, \hat{\theta}_k) \in \hat{\Theta}_{l}^{\rm N}$. Then, via utilizing the IRS as the anchor, its location is estimated as
%\begin{align}\label{equ:est_N_xy}
%  (\hat{x}_k,\hat{y}_k) &= \left(a_{\rm I} + \hat{d}_k \cos \hat{\theta}_k , b_{\rm I} + \hat{d}_k \sin \hat{\theta}_k \right).
%\end{align}


%After Phase II, we already have the range information of the user-target-IRS paths as shown in (\ref{eqn:sum distance}), the AOA information of the targets towards the IRS as shown in (), and the range information of the target-IRS paths for the near-field targets as shown in (). In the third phase, we aim to estimate the locations of the targets, i.e., $(x_k,y_k)$, $\forall k$, based on the above range and AOA information obtained in Phase II. 
%
%Specifically, for the far-field targets, we utilize both the IRS and the user as the anchors, and localize these targets based on the range information (\ref{eqn:sum distance}) and AOA information (). (introduce how to get target location based on range and AOA)
%
%On the other hand, for the near-field targets, we merely utilize the IRS as the passive anchor, and localize these targets based on their AOAs to the IRS shown in () and distances to the IRS shown in (). Note that in contrast to the estimation of distance of the user-target-IRS paths shown in (\ref{eqn:sum distance}), the estimation of $d_k$'s is based on their relation to the near-field steering vectors $\boldsymbol{a}_\text{I}(d_{k},\theta_k)$'s such that its accuracy is not limited by the channel bandwidth. That is the reason why we utilize the range information of the target-IRS paths, rather than that of the user-target-IRS paths, to localize the near-field targets.  (show how to get target location based on range and AOA).

 

%\subsection{Step I: Channel Estimation and Model-free Range Estimation}
%
%The goal of Step I of our proposed protocol is to first estimate $\boldsymbol{\bar{H}}_{t}^{(q)}$'s based on the received signal given in \eqref{eq:received signal} and then extract the model-free range information as well as obtain the effective cascaded channel information. It is observed from \eqref{eq:multipath channel l tap} that $\boldsymbol{\bar{H}}_{t}^{(q)}$ is a row-sparse matrix, i.e., $\boldsymbol{\bar{h}}_{l,t}^{(q)} = \boldsymbol{0}$ for many $l$. This is because if there is no path causing a delay of $l$ OFDM samples for the $m_{\text{B}}$-th receive antenna, then such a path does not exist for the other receive antennas. Based on the row-sparse property, $\boldsymbol{\bar{H}}_t^{(q)}$'s can be estimated based on the group-LASSO technique by solving the following problem
%Based on $\boldsymbol{\hat{H}}_{t}^{(q)}$'s, we need to estimate the support of $\boldsymbol{\bar{H}}_{t}^{(q)}$'s so as to obtain the model-free range information of the user - some target - IRS path. In this paper, we adopt a threshold-based strategy to achieve the above goal. Specifically, with some given threshold $\tau$, define $\Phi=\{l|\sum_{t=1}^{T}\sum_{q=1}^Q\|\boldsymbol{\hat{h}}_{l,t}^{(q)}\|^2 \geq \tau\}$. Then, for any $l\in\Phi$, we declare that $\boldsymbol{\hat{h}}_{l,t}^{(q)}\neq\boldsymbol{0}$, $\forall q,t$ and there exist some targets in range cluster $l$, i.e., $\Omega_l \neq \emptyset$. Moreover, ${d}_{k_l}^{\text{TxTI}}$ can be estimated by deducing the known distance of the IRS - BS path from that of the user - target $k_l$ - IRS -BS path, i.e.,
%\begin{align}\label{eq:est range step 1}
%    \hat{d}_{k_l}^{\text{TxTI}} = \hat{d}_{l}^{\text{TxTI}} = \frac{(2l-1)c_0}{2B}-d^{\text{IR}},\quad  \forall k_l\in \Omega_l,l \in\Phi,
%\end{align}where $d^{\text{IR}}$ in meter is the known distance between the IRS and the BS. 
%
%To summarize, after Phase I, the BS obtains the model-free range estimation of the user - target $k_l$ in range cluster $l\in\Phi$ - IRS path, i.e., $\hat{d}_{k_l}^{\text{TxTI}}$'s, and the corresponding cascaded channel estimation, i.e., $\boldsymbol{\hat{h}}_{l,t}^{(q)}$'s, $\forall l\in \Phi$. After that, the next job is to estimate the model-based range and AOA information of the target $k_l$ - IRS path, as will be introduced in the next subsection.
%
%\subsection{Step II: Model-based AOA and Range Estimation} \label{subsec:step II}
%
%
%According to \eqref{eq:multipath channel l tap}, if $l\in\Phi$, it then indicates some targets may coexist within range cluster $k$. These targets could be a mix of far-field targets and near-field targets relative to the IRS. This is due to the fact that while some targets within the same range cluster $l$ might share the same $d_k^{\text{TxTI}}$, their $d_k^{\text{IT}}$ can differ. For instance, as illustrated in Fig. \ref{fig:demo_coexistence}, target $1$ and target $2$ are located within the same range cluster, with target $1$ being a near-field target and target $2$ being a far-field target. Define $\Omega_l^\text{F}$ and $\Omega_l^\text{N}$ as the set of far-field targets and near-field targets in range cluster $l$, such that $\Omega_l^\text{F}\cap\Omega_l^\text{N}=\emptyset$ and $\Omega_l^\text{F}\cup\Omega_l^\text{N}=\Omega_l$, $\forall l$. Then, the BS aims to estimate $\theta_{k_l}$'s and $d_{k_l}^\text{IT}$'s based on the following relation
%\begin{align}\label{eq:received signal cluster l}
%    \boldsymbol{\hat{h}}_{l,t}^{(q)} &= \boldsymbol{\bar{h}}_{l,t}^{(q)}+\boldsymbol{\hat{z}}_{l,t}^{(q)} \notag\\
%    &=\delta\boldsymbol{G}\text{diag}(\boldsymbol{\phi}_t^{(q)}) \big(\sum_{k_l\in\Omega_l^\text{F}} \beta_{k_l} \gamma_{k_l,t} \boldsymbol{a}_\text{I}(\infty,\theta_{k_l}) \notag \\
%    &\qquad+\sum_{k_l\in\Omega_l^\text{N}} \beta_{k_l} \gamma_{k_l,t}  \boldsymbol{a}_\text{I}(d_{k_l}^{\text{IT}},\theta_{k_l})\big) + \boldsymbol{\hat{z}}_{l,t}^{(q)} \notag \\
%    &=\boldsymbol{\Psi}_t^{(q)}(\boldsymbol{\bar{d}}_l^\text{IT},\boldsymbol{\theta}_l)\boldsymbol{x}_{l,t} + \boldsymbol{\hat{z}}_{l,t}^{(q)},\quad\forall q,t,l\in\Phi,
%\end{align}where 
%\begin{align}
%&\boldsymbol{\bar{d}}_l^\text{IT} = [\bar{d}_1^\text{IT},\cdots,\bar{d}_{K_l}^\text{IT}],\\
%\hspace*{-0.1cm}&\bar{d}_{k_l}^{\text{IT}}=\left\{\begin{matrix}
% d_{k_l}^\text{IT}, & \text{if}~k_l\in\Omega_l^\text{N},  \\ 
% \infty, & \text{if}~k_l\in\Omega_l^\text{F},
%\end{matrix}\right.\label{eq:model range}\\
%&\boldsymbol{\theta}_l = [\theta_1,\cdots,\theta_{K_l}],\\
%&\boldsymbol{\Psi}_t^{(q)}(\boldsymbol{\bar{d}}_l^\text{IT},\boldsymbol{\theta}_l)= \boldsymbol{G}\text{diag}(\boldsymbol{\phi}_t^{(q)})\boldsymbol{A}(\boldsymbol{\bar{d}}_l^\text{IT},\boldsymbol{\theta}_l), \label{eq:array matrix}\\
%&\boldsymbol{A}(\boldsymbol{\bar{d}}_l^\text{IT},\boldsymbol{\theta}_l) = [\boldsymbol{a}_\text{I}(\bar{d}_1^{\text{IT}},\theta_1),\cdots,\boldsymbol{a}_\text{I}(\bar{d}_{K_l}^{\text{IT}},\theta_{K_l})],\\
%&\boldsymbol{x}_{l,t} = [\delta\beta_{1}\gamma_{1,t},\cdots,\delta\beta_{K_l}\gamma_{K_l,t}]^\text{T},\label{eq:x}
%\end{align}and $\boldsymbol{\hat{z}}_{l,t}^{(q)}$ is the estimation error introduced by Step I. 
%
%According to \eqref{eq:near irs bs channel} and \eqref{eq:far irs bs channel}, the LOS channel from the IRS to the BS, i.e., $\boldsymbol{\bar{G}}$, is not always of full row rank (or equivalently, $\boldsymbol{G}$ is not always of full row rank). This indicates that some rows in $\boldsymbol{\bar{G}}$ can be expressed as a linear combination of the others. Consequently, the information of these rows (or equivalently, receive antennas) is redundant and does not contribute to the estimation problem we are interested in. In other words, although the BS is equipped with multiple antennas, not all the receive antennas provide additional information on AOA and range of the LOS path from each target to the IRS. This is actually not surprising. For example, if the channel between the IRS and the BS is a far-field channel such that \eqref{eq:far irs bs channel} holds, then in the ideal case when there is no estimation error, i.e., $\boldsymbol{\hat{z}}_{l,t}^{(q)}=\boldsymbol{0}$, $\forall q,t,l\in\Phi$, at the $q$-th OFDM symbol duration in coherence block $t$, the cascaded channel between the user and the $m_1$-th receive antenna as well as that between the user and the $m_2$-th receive antenna have the following relation
%\begin{align}\label{eq:received signal two antennas far}
%    \hat{h}_{m_2,l,t}^{(q)} = \frac{a_{\text{B},m_2}(\infty,\varphi)}{a_{\text{B},m_1}(\infty,\varphi)}\hat{h}_{m_1,l,t}^{(q)},\quad \forall m_2\neq m_1,q,
%\end{align}where $a_{\text{B},m_{\text{B}}}(\infty,\varphi)$ is obtained by substituting $d$ with $\infty$ in \eqref{eq:steer vectors}. This implies that given the channel of the path from the user to the $m_1$-th receive antenna, that from the user to any other antenna $m_2 \neq m_1$ provides no information about $\boldsymbol{\theta}_l$'s and $\boldsymbol{\bar{d}}_l^{\text{IT}}$'s. Instead, it only contains the information of the AOA of the path from the IRS to the BS, i.e., $\varphi$. In other words, if the LOS channel from the IRS to the BS is a far-field channel, then the phase differences among the different antennas at the BS only contain the information about the AOA from the IRS. However, we are interested in estimating the AOAs from the targets to the IRS for localization. 
%
%Therefore, we propose first selecting a subset of useful receive antennas based on $\boldsymbol{G}$. Then, we can estimate the AOA and range information just based on the selected receive antennas. Specifically, we aim to identify an index set of receive antennas $\mathcal{M}_\text{S}$ with a cardinality of $\text{rank}(\boldsymbol{G})$ based on the known matrix $\boldsymbol{G}$, such that the channels between the IRS and the selected receive antennas are linearly independent. Let $\mathcal{M}_{\text{S}}(m_{\text{S}})$ denote the $m_{\text{S}}$-th element of $\mathcal{M}_{\text{S}}$ and $M_{\text{S}}$ denote the cardinality of $\mathcal{M}_{\text{S}}$. Then, we achieve the above goal by solving the following problem 
%\begin{align}\label{prb:find antenna}
%    &\text{find}~\mathcal{M}_{\text{S}}\notag\\
%    &\text{s.t.}~\text{rank}([(\boldsymbol{g}_{\mathcal{M}_\text{S}(1)})^\text{T},\cdots,(\boldsymbol{g}_{\mathcal{M}_\text{S}(M_\text{S})})^\text{T}]^\text{T}) = \text{rank}(\boldsymbol{G}),
%\end{align}where $\boldsymbol{g}_{\mathcal{M}_\text{S}(m_\text{S})}$ is the $\mathcal{M}_\text{S}(m_\text{S})$-th row of $\boldsymbol{G}$. The above problem can be efficiently solved by the pivoted QR decomposition approach \cite{pivoted_QR}. Let $\hat{\mathcal{M}}_\text{S}$ denote the solution to problem \eqref{prb:find antenna} and $\hat{M}_\text{S}$ as the cardinality of $\hat{\mathcal{M}}_\text{S}$. Additionally, define 
%\begin{align}
%    &\boldsymbol{\tilde{G}} = [\boldsymbol{g}_{\hat{\mathcal{M}}_\text{S}(1)},\cdots,\boldsymbol{g}_{\hat{\mathcal{M}}_\text{S}(M_\text{S})}]^T\in\mathbb{C}^{\hat{M}_\text{S}\times M_\text{I}}, \\
%    &\boldsymbol{\tilde{\Psi}}_t^{(q)}(\boldsymbol{\bar{d}}_l^\text{IT},\boldsymbol{\theta}_l)= [\tilde{\psi}_t^{(q)}(\bar{d}_1^\text{IT},\theta_1),\cdots,\tilde{\psi}_t^{(q)}(\bar{d}_{K_l}^\text{IT},\theta_{K_l})] \notag\\& ~\quad\qquad\qquad= \boldsymbol{\tilde{G}}\text{diag}(\boldsymbol{\phi}_t^{(q)})\boldsymbol{A}(\boldsymbol{\bar{d}}_l^\text{IT},\boldsymbol{\theta}_l),\\
%    &\boldsymbol{\tilde{z}}_{l,t}^{(q)} = [\hat{z}_{\hat{\mathcal{M}}_\text{S}(1),l,t},\cdots,\hat{z}_{\hat{\mathcal{M}}_\text{S}(|\hat{\mathcal{M}}_{\text{S}}|),l,t}]^\text{T}.
%\end{align}Thus, we only need to focus on the receive antennas in $\hat{\mathcal{M}}_{\text{S}}$ for estimating $\boldsymbol{\bar{d}}_l^\text{IT}$ and $\boldsymbol{\theta}_l$, i.e., 
%\begin{align}\label{eq:selected received signal cluster l}
%    \boldsymbol{\tilde{h}}_{l,t}^{(q)} &= [\hat{h}_{\hat{\mathcal{M}}_\text{S}(1),l,t}^{(q)},\cdots,\hat{h}_{\hat{\mathcal{M}}_\text{S}(\hat{M}_{\text{S}}),l,t}^{(q)}]^\text{T} \in \mathbb{C}^{\hat{M}_{\text{S}} \times 1} \notag \\
%    &= \boldsymbol{\tilde{\Psi}}_t^{(q)}(\boldsymbol{\bar{d}}_l^\text{IT},\boldsymbol{\theta}_l)\boldsymbol{x}_{l,t} + \boldsymbol{\tilde{z}}_{l,t}^{(q)},\quad\forall q,t,l\in\Phi.
%\end{align}Without loss of generality, we assume that the IRS repeats the same reflecting pattern over different coherence blocks, i.e.,
%\begin{align}
%    \boldsymbol{\phi}_{t_1}^{(q)} = \boldsymbol{\phi}_{t_2}^{(q)},\quad \forall t_1 \neq t_2, q.
%\end{align}Thus, $\boldsymbol{\tilde{\Psi}}_t^{(q)}(\boldsymbol{\bar{d}}_l^\text{IT},\boldsymbol{\theta}_l)$ does not change over different coherence blocks, such that \eqref{eq:selected received signal cluster l} reduces to
%\begin{align}\label{eq:final received signal cluster l}
%    \boldsymbol{\tilde{h}}_{l,t}^{(q)} = \boldsymbol{\tilde{\Psi}}^{(q)}(\boldsymbol{\bar{d}}_l^\text{IT},\boldsymbol{\theta}_l)\boldsymbol{x}_{l,t} + \boldsymbol{\tilde{z}}_{l,t}^{(q)},\quad\forall q,t,l\in\Phi.
%\end{align}Based on the number of useful receive antennas in $\hat{\mathcal{M}}_\text{S}$ and the number of coherence blocks, we discuss different strategies for estimating $\boldsymbol{\bar{d}}_l^\text{IT}$ and $\boldsymbol{\theta}_l$.
%
%%One direct approach for estimating $\boldsymbol{\bar{d}}_l^\text{IT}$ and $\boldsymbol{\theta}_l$ is to apply traditional MUSIC algorithm or IAA algorithm on \eqref{eq:selected received signal cluster l}. However, 
%
%
%%Note that $\hat{M}_\text{S}$ (or equivalently, $\text{rank}(\boldsymbol{\tilde{G}})$) may be larger than $K_l$ or smaller than $K_l$. Thus, $\text{rank}(\boldsymbol{\tilde{\Psi}}^{(q)}(\boldsymbol{\bar{d}}_l^\text{IT},\boldsymbol{\theta}_l))$ may be equal to $K_l$ or smaller than $K_l$ because
%%\begin{align}
%%    \text{rank}(\boldsymbol{\tilde{\Psi}}^{(q)}(\boldsymbol{\bar{d}}_l^\text{IT},\boldsymbol{\theta}_l))\leq \min (\text{rank}(\boldsymbol{\tilde{G}}),K_l).
%%\end{align}In the following, we discuss how to estimate $\boldsymbol{\bar{d}}_l^\text{IT}$'s and $\boldsymbol{\theta}_l$'s based on \eqref{eq:selected received signal cluster l} under different cases.
%
%\subsubsection{Case I} 
%
%We first discuss the case when $\hat{M}_\text{S} > K_l$ and $T \geq K_l$. Under such a setup, one straightforward approach to estimate $\boldsymbol{\bar{d}}_l^\text{IT}$'s and $\boldsymbol{\theta}_l$'s based on \eqref{eq:final received signal cluster l} is to apply the conventional MUSIC algorithm \cite{music,2d_music}. Specifically, let us assume that the first OFDM symbol transmission in each coherence block $t$ is utilized for estimation. The first step is to calculate the sample covariance matrix of the received signals as the estimation of the true covariance matrix. In particular, the sample covariance matrix over $T$ coherence blocks is given as
%\begin{align}\label{eq:est cov}
%    \boldsymbol{\hat{R}}_l^\text{MUSIC} = \frac{1}{T} \sum_{t=1}^{T} \boldsymbol{\tilde{h}}_{l,t}^{(1)}(\boldsymbol{\tilde{h}}_{l,t}^{(1)})^H\in \mathbb{C}^{\hat{M}_\text{S}\times \hat{M}_\text{S}},~ \forall l.
%\end{align}Then, define the eigenvalue decomposition (EVD) of $\boldsymbol{\hat{R}}_l^{\text{MUSIC}}$ as $\boldsymbol{\hat{R}}_l^{\text{MUSIC}}=\boldsymbol{V}_l\boldsymbol{\Lambda}_l\boldsymbol{V}_l^H$, where $\boldsymbol{\Lambda}_l={\rm diag}([\bar{\lambda}_{l,1},\cdots,\bar{\lambda}_{l,\hat{M}_\text{S}}]^T)$ whose diagonal elements are the eigenvalues of $\boldsymbol{\hat{R}}_l^{\text{MUSIC}}$, and $\boldsymbol{V}_l=[\boldsymbol{v}_{l,1},\cdots, \boldsymbol{v}_{l,\hat{M}_\text{S}}]$ consists of the corresponding eigenvectors. Without loss of generality, we assume that $\bar{\lambda}_{l,1}\geq \bar{\lambda}_{l,2}\geq \cdots \geq \bar{\lambda}_{l,\hat{M}_\text{S}}$. Then, we estimate the number of targets $K_l$ in range cluster $l\in\Phi$. If the signal-to-noise ratio (SNR) is high, we can set a threshold on $\bar{\lambda}_{l,\hat{m}_\text{S}}$ to estimate the number of targets \cite{2d_music}. Otherwise, we can estimate the number of targets based on the Akaike information criterion (AIC) approach \cite{aic_music,aic}, by solving
%\begin{align}\label{eq:est k}
%    \hat{K}_l = &\arg\max_{K_l} \log\left(\frac{\prod_{i=K_l+1}^{\hat{M}_\text{S}}\bar{\lambda}_{l,i}^{1/(\hat{M}_\text{S}-K_l)}}{1/(\hat{M}_\text{S}-K_l)\sum_{j=K_l+1}^{\hat{M}_\text{S}}\bar{\lambda}_{l,j}}\right)^{\hat{M}_\text{S}-K_l} \notag \\& \qquad\qquad - K_l(2\hat{M}_\text{S} - K_l),\quad \forall l.
%\end{align}Next, define $\boldsymbol{\bar{V}}_l = [\boldsymbol{v}_{l,\hat{K}_l+1},\cdots,\boldsymbol{v}_{l,\hat{M}_\text{S}}]\in\mathbb{C}^{\hat{M}_\text{S} \times (\hat{M}_\text{S}-\hat{K}_l)}$. Since the far-field channel model of a target is a special case of its near-field channel model with $d_{k_l}^\text{IT}\to\infty$, the spectrum for estimating the AOAs of far-field targets in range cluster $l$ is given as  
%\begin{align}\label{eq:spectrum 1D}
%     P_l^{\text{1D}}(\infty,\theta)=\frac{\boldsymbol{\tilde{\psi}}^{(1)}(\infty,\theta)^H\boldsymbol{\tilde{\psi}}^{(1)}(\infty,\theta)}{\boldsymbol{\tilde{\psi}}^{(1)}(\infty,\theta)^H\bar{\boldsymbol{V}}_l\boldsymbol{\bar{V}}_l^H\boldsymbol{\tilde{\psi}}^{(1)}(\infty,\theta)},\forall \theta\in [0,\pi).
%\end{align}Moreover, the spectrum used for estimating the AOAs and ranges of near-field targets in range cluster $l$ is defined as
%\begin{align}\label{eq:spectrum 2D}
%     P_l^{\text{2D}}(d^\text{IT},\theta)=\frac{\boldsymbol{\tilde{\psi}}^{(1)}(d^\text{IT},\theta)^H\boldsymbol{\tilde{\psi}}^{(1)}(d^\text{IT},\theta)}{\boldsymbol{\tilde{\psi}}^{(1)}(d^\text{IT},\theta)^H\bar{\boldsymbol{V}}_l\boldsymbol{\bar{V}}_l^H\boldsymbol{\tilde{\psi}}^{(1)}(d^\text{IT},\theta)},\forall (d^\text{IT},\theta)\in \mathcal{R},
%\end{align}where $\mathcal{R}$ is the collection of discrete search grids. In particular, let $\Delta d$ and $\Delta \theta$ denote the searching step size of the angle and that of the range, respectively. Then, $\mathcal{R}$ is given by
%\begin{align}\label{eq:2d search}
%    \mathcal{R} = \{(\theta_\xi,d_\varsigma^\text{IT})|&\theta_\xi = \xi\Delta\theta,\xi=1,\cdots,\lceil{\frac{\pi}{\Delta \theta}\rceil}, \notag \\ &d_\varsigma^\text{IT}=\varsigma \Delta d,\varsigma=1,\cdots,\lceil\frac{d_\text{max}}{\Delta d}\rceil,\},
%\end{align}where $d_\text{max}$ is the maximum possible range of the paths from targets to the IRS. Finally, we can perform a one-dimension search over $P_l^{\text{1D}}(\infty,\theta)$ and a two-dimension search over $P_l^{\text{2D}}(d^\text{IT},\theta)$ to find peaks of the above spectrums, and selected $\hat{K}_l$ largest peaks. The corresponding angles and ranges will be the estimations of the AOAs and ranges of the LOS paths from the targets in range cluster $l$ to the IRS. 
%
%However, the number of search grids in $\mathcal{R}$ is large, and it is of prohibitive complexity to calculate the spectrum for all these search grids, as required by the MUSIC algorithm \eqref{eq:spectrum 2D}. To tackle the above challenge, in this paper, we propose a prior information-assisted MUSIC algorithm, where the model-free range information obtained in Phase I is leveraged as prior information to reduce the number of search grids. Specifically, when localizing targets in range cluster $l$, given any search grid $(d_\varsigma^\text{IT},\theta_\xi)\in \mathcal{R}$ and the location of the IRS, we can obtain the corresponding target location as $(x_l(d_\varsigma^\text{IT},\theta_\xi),y_l(d_\varsigma^\text{IT},\theta_\xi))$. Moreover, the location of this target should satisfy the range constraints, i.e.,
%\begin{align}
%    &\sqrt{(x_l(d_\varsigma^\text{IT},\theta_\xi)-x_\text{I})^2+(y_l(d_\varsigma^\text{IT},\theta_\xi)-y_\text{I})^2} + \notag \\ &\sqrt{(x_l(d_\varsigma^\text{IT},\theta_\xi) - x_\text{T})^2+(y_l(d_\varsigma^\text{IT},\theta_\xi) - y_\text{T})^2} \geq \hat{d}_{l}^{\text{TxTI}} - \frac{c_0}{2B},\label{eq:ellipse1}\\
%    &\sqrt{(x_l(d_\varsigma^\text{IT},\theta_\xi)-x_\text{I})^2+(y_l(d_\varsigma^\text{IT},\theta_\xi)-y_\text{I})^2} + \notag \\ &\sqrt{(x_l(d_\varsigma^\text{IT},\theta_\xi) - x_\text{T})^2+(y_l(d_\varsigma^\text{IT},\theta_\xi) - y_\text{T})^2} \leq \hat{d}_{l}^{\text{TxTI}} + \frac{c_0}{2B},\label{eq:ellipse2}
%\end{align}where $\hat{d}_{l}^{\text{TxTI}}$ is the model-free range information obtained by \eqref{eq:est range step 1}, and the term $\frac{c_0}{2B}$ accounts for the model-free range estimation error in Step I. As a result, when localizing targets in range cluster $l$, we actually only need to search over the following region
%\begin{align}\label{eq:proposed search}
%    \hat{\mathcal{R}}_l = \{(d_\varsigma^\text{IT},\theta_\xi)|\forall &(d_\varsigma^\text{IT},\theta_\xi)\in \mathcal{R}~\text{and}~\notag \\&(d_\varsigma^\text{IT},\theta_\xi)~\text{satisfies}~\eqref{eq:ellipse1}~\text{and}~ \eqref{eq:ellipse2}\}.
%\end{align}As compared to the conventional searching region $\mathcal{R}$ given in \eqref{eq:2d search}, $\hat{\mathcal{R}}_l$'s are much smaller. Moreover, $\hat{\mathcal{R}}_l$'s can be pre-computed offline such that it does not add to runtime complexity. Thus, the new prior information assisted spectrum for estimating the AOAs and ranges of near-field targets in range cluster $l$ is given by
%\begin{align}\label{eq:spectrum prior}
%     P_l^{\text{Prior}}(d^\text{IT},\theta)=&\frac{\boldsymbol{\tilde{\psi}}^{(1)}(d^\text{IT},\theta)^H\boldsymbol{\tilde{\psi}}^{(1)}(d^\text{IT},\theta)}{\boldsymbol{\tilde{\psi}}^{(1)}(d^\text{IT},\theta)^H\bar{\boldsymbol{V}}_l\boldsymbol{\bar{V}}_l^H\boldsymbol{\tilde{\psi}}^{(1)}(d^\text{IT},\theta)},\notag \\
%     &\qquad\qquad\qquad\qquad\qquad\forall (d^\text{IT},\theta)\in \hat{\mathcal{R}}_l.
%\end{align}The above prior information-assisted MUSIC algorithm is summarized in Algorithm \ref{alg:music}.
%
%\begin{algorithm}[t]
%	\caption{Prior information-assisted MUSIC Algorithm for AOA and Range Estimation in Range Cluster $l$}\label{alg:music}
%	    {\bf Input}: $ \boldsymbol{\tilde{h}}_{l,t}^{(q)}$ given in \eqref{eq:final received signal cluster l};\\
%	    {\bf Initialization (Offline)}: Obtain $\hat{\mathcal{R}}_l$'s given in \eqref{eq:proposed search};
%        \begin{enumerate}
%         \item [1.] Estimate the covariance matrix $\boldsymbol{\hat{R}}_l^\text{MUSIC}$ by \eqref{eq:est cov}, and perform EVD on $\boldsymbol{\hat{R}}_l^\text{MUSIC}$ to obtain eigenvalues $\bar{\lambda}_{l,1}\geq \bar{\lambda}_{l,2}\geq \cdots \geq \bar{\lambda}_{l,\hat{M}_\text{S}}$ and the corresponding eigenvectors $\boldsymbol{v}_{l,1},\boldsymbol{v}_{l,2},\cdots, \boldsymbol{v}_{l,\hat{M}_\text{S}}$; \Comment{Step 1} 
%         \item [2.] Estimate the total number of targets in range cluster $l$ by \eqref{eq:est k} as $\hat{K}_l$; \Comment{Step 2}
%        \item [3.] Find the peaks of the $1$-D spectrum \eqref{eq:spectrum 1D} and find the peaks of the $2$-D spectrum \eqref{eq:spectrum prior}; \Comment{Step 3}
%        \item [4.] Jointly select $\hat{K}_l$ largest peaks of the $1$-D and $2$-D spectrums obtained in Step $3$; \Comment{Step 4}
%     \end{enumerate}
%        {\bf{Output}}: AOAs of far-field targets corresponding to the selected peaks from the $1$-D spectrum, AOAs and ranges of near-field targets corresponding to the selected peaks from the $2$-D spectrum.
%\end{algorithm}
%
%To gain some insight, we provide a numerical example to verify the effectiveness of the prior information to reduce the search complexity. In this example, we assume that there is one user located at $(0,0)$ in meter, one IRS located at $(20,20)$ in meter, and one BS located at $(25,15)$ in meter. Then, we randomly generate the location of one target such that it is located within a quarter circle with the IRS as the center and $d^\text{max}=80$ in meter as the radius. Moreover, we assume the angle search step size is $0.1$ in degree, i.e., $\Delta \theta = 0.1$, and the range search step size is $0.1$ in meter, i.e., $\Delta d = 0.1$. As a result, under this setup, the cardinality of $\mathcal{R}$ is $|\mathcal{R}|=900\times 800=7.2\times 10 ^5$. Last, we adopt the average number of search grids as the performance metric. Fig. \ref{fig:comparison_search_grids} shows the average number of search grids required by the conventional $2$-D search and that required by the proposed approach with prior information, i.e., $\hat{\mathcal{R}}_l$'s defined in \eqref{eq:proposed search}, for different bandwidth ranging from $100$ MHz to $400$ MHz. It is observed that the proposed approach significantly reduces the number of search grids, and thus significantly reduces the search complexity. For example, when the bandwidth is $400$ MHz, the conventional $2$-D search requires $7.2 \times 10^5$ grids on average, while our proposed approach only requires $8.5\times 10^3$ grids on average. This shows that the proposed approach is able to reduce over $98$\% search grids as compared to the conventional approach. 
%
%\begin{figure}[t]
%   \centering
%    \includegraphics[width=8cm]{Fig3.eps}
%    \caption{Comparison between the number of search grids required by $2$-D search and our proposed approach.}\label{fig:comparison_search_grids}
%\end{figure}
%
%Note that the key to applying the MUSIC algorithm is the covariance matrix. In particular, the rank of the true signal covariance matrix should be equal to the number of targets $K_l$ in range cluster $l$, i.e.,
%\begin{align}\label{eq:music necessary 1}
%    \text{rank}(\boldsymbol{\tilde{\Psi}}^{(q)}(\boldsymbol{\bar{d}}_l^\text{IT},\boldsymbol{\theta}_l)\boldsymbol{x}_{l,t}\boldsymbol{x}_{l,t}^H(\boldsymbol{\tilde{\Psi}}^{(q)}(\boldsymbol{\bar{d}}_l^\text{IT},\boldsymbol{\theta}_l))^H)=K_l.
%\end{align}Furthermore, the rank of the estimated covariance matrix should also be larger than the number of targets $K_l$, i.e.,
%\begin{align}\label{eq:music necessary 2}
%    \text{rank}(\boldsymbol{\hat{R}}_l^\text{MUSIC}) \geq K_l.
%\end{align}
%%However, in our considered system, we encounter two challenges related to \emph{rank deficiency} issue in both the true and estimated covariance matrix. In these cases, either \eqref{eq:music necessary 1} or \eqref{eq:music necessary 2} is violated. These cases are discussed as follows.
%
%\subsubsection{Case II}Then, we discuss the case when $\hat{M}_\text{S}\leq K_l$ and $T \geq K_l$. In this case, the necessary condition \eqref{eq:music necessary 1} does not hold. This is because of the rank-deficient channel between the BS and the IRS. Specifically, according to problem \eqref{prb:find antenna}, if $\text{rank}(\boldsymbol{G})$ is less than $K_l$, then the cardinality of its solution will also be less than $K_l$. Consequently, $\text{rank}(\boldsymbol{\tilde{\Psi}}^{(q)}(\boldsymbol{\bar{d}}_l^\text{IT},\boldsymbol{\theta}_l))$ is less than $K_l$. This implies that it is impossible to apply the classic methods for estimating $\boldsymbol{\theta}_l$ and $\boldsymbol{\bar{d}}_l^\text{IT}$ based on the covariance matrix of the cascaded channel \eqref{eq:selected received signal cluster l}. This is actually not surprising, as it is analogous to attempting to solve for unknown variables with a set of equations where the number of unknown variables exceeds the number of equations, rendering it unsolvable.
%
%This rank deficiency situation is similar to the channel rank-deficient phenomenon in LOS MIMO communication systems, where some antennas do not contribute to increasing the channel capacity \cite{los_mimo}. However, the rank deficiency issue discussed in this paper is much more challenging than the communication case. This is because the role of a communication system is to transmit/decode messages. Even though the channel rank is $1$, the communication system still functions. In contrast, in the sensing scenario, the rank deficiency problem prohibits the classic estimation methods from working. Therefore, it is crucial to investigate how to tackle the rank-deficiency challenge in the sensing setup. Although the spatial-domain channels do not work when the LOS channel between the IRS and the BS is rank deficient, we show a novel and interesting result in Section \ref{sec:step II MUSIC}: via properly creating virtual multi-dimension channels between the user and the BS in the \emph{temporal domain}, we are able to estimate the model-based AOAs and ranges of the LOS paths from targets to the IRS based on the virtual channels, even if only one receive antenna is utilized. 
%
%\subsubsection{Case III}Next, we discuss the case when $T < K_l$. In practice, the number of available coherence blocks is small. In this case, according to \eqref{eq:covariance estimation}, $\text{rank}(\boldsymbol{\hat{R}}_l^\text{MUSIC}) < K_l$, violating the necessary condition \eqref{eq:music necessary 2}. This also indicates that it is impossible to apply the classic MUSIC algorithm for estimating $\boldsymbol{\bar{d}}_l^\text{IT}$ and $\boldsymbol{\theta}_l$. We address this challenge in Section \ref{sec:step II IAA}.

%\section{Step II: Model-based AOA and Range Estimation with A Large Number of Coherence Blocks}\label{sec:step II MUSIC}
\section{Achieving Goals 1-3 in Phase II}\label{sec:rank_two_case}

In this section, we give the specific schemes to realize \textbf{Goals} 1-3 in Phase II listed in Section \ref{sec:phase_II} to complete our three-phase protocol. 

%\textcolor[rgb]{0.00,0.07,1.00}{For illustration, the target set for range cluster $l$ in \eqref{eq:range cluster l} is also represented as $\Omega_l = \{m_{l,1},\dots,m_{l,K_l}\}$ with $1 \le m_{l,1} < \dots < m_{l,i} < \dots < m_{l,K_l} \le K, ~\forall l \in \Phi$.}

For illustration, we define $\Omega_l^\text{F}$ and $\Omega_l^\text{N}$ as the ture sets of far-field targets and near-field targets in range cluster $l \in \Phi$, respectively, such that $\hat{\Omega}_{l}^{\rm F}$ and $\hat{\Omega}_{l}^{\rm N}$ in \textbf{Goal} 2 are their estimations.
According to \eqref{eq:cascaded channel}, \eqref{eq:nlos channel k}, and \eqref{eq:los irs bs channel}, the effective channels of all targets in range cluster $l \in \Phi$ that are estimated via problem \eqref{eq:group_lasso} satisfy
\begin{subequations}\label{eq:estimated_channel cluster l}
\begin{align}%
    %\boldsymbol{\tilde{h}}_{l,t}^{(q)} &= \boldsymbol{\bar{h}}_{l,t}^{(q)}+\boldsymbol{\tilde{z}}_{l,t}^{(q)} \notag\\
%    &=\sum_{k\in\Omega_l}\delta\beta_{k}\gamma_{k,t}\boldsymbol{G}\text{diag}(\boldsymbol{\phi}_t^{(q)})\boldsymbol{a}_\text{I}(\bar{d}_{k},\theta_k) + \boldsymbol{\tilde{z}}_{l,t}^{(q)}\notag\\
%    &\overset{(a)}{=}\delta\boldsymbol{G}\text{diag}(\boldsymbol{\phi}_t^{(q)}) \Bigg(\sum_{k\in\Omega_l^\text{F}} \beta_{k} \gamma_{k,t} \boldsymbol{a}_\text{I}(d_{k} \to \infty,\theta_{k}) \notag \\
%    &~\quad + \sum_{k\in\Omega_l^\text{N}} \beta_{k} \gamma_{k,t} \boldsymbol{a}_\text{I}(d_{k},\theta_{k})\Bigg) + \boldsymbol{\tilde{z}}_{l,t}^{(q)} \notag \\
%    %&= \sum_{k\in\Omega_l} v_{k,t} \boldsymbol{\psi}^{(q)}_t(\bar{d}_k,\theta_k) + \boldsymbol{\tilde{z}}_{l,t}^{(q)} \notag \\
%    &= \boldsymbol{\Psi}_{l,t}^{(q)}(\boldsymbol{\bar{d}}_{l},\boldsymbol{\theta}_{l}) \boldsymbol{\upsilon}_{l,t} + \boldsymbol{\tilde{z}}_{l,t}^{(q)},  \quad\forall t, l \in \Phi,%l~\text{with}~\Omega_l \neq \emptyset, 
%    %\notag \\ 
%    %&= \boldsymbol{\Psi}^{(q)}_{l,t} \boldsymbol{v}_{l,t} + \boldsymbol{\tilde{z}}_{l,t}^{(1)}, 
    \boldsymbol{\tilde{h}}_{l,t}^{(q)} &= \boldsymbol{\bar{h}}_{l,t}^{(q)}+\boldsymbol{\tilde{z}}_{l,t}^{(q)} \label{equ:noiseless+noise} \\
    &=\sum_{k\in\Omega_l}\delta\boldsymbol{G}\text{diag}(\boldsymbol{\phi}_t^{(q)}) \boldsymbol{r}_{k,t} + \boldsymbol{\tilde{z}}_{l,t}^{(q)} \label{equ:expand_estimated_channel} \\
    &= \delta\boldsymbol{G}\text{diag}(\boldsymbol{\phi}_t^{(q)}) \Big(\sum_{k\in\Omega_l^\text{F}} \beta_{k} \gamma_{k,t} \boldsymbol{a}_\text{I}(d_{k} \to \infty,\theta_{k}) \notag  \\
    &~\quad + \sum_{k\in\Omega_l^\text{N}} \beta_{k} \gamma_{k,t} \boldsymbol{a}_\text{I}(d_{k},\theta_{k})\Big) + \boldsymbol{\tilde{z}}_{l,t}^{(q)} \label{equ:expand_far_near}  \\
    &= \sum_{k\in\Omega_l} v_{k,t} \boldsymbol{\psi}^{(q)}_t(\bar{d}_k,\theta_k) + \boldsymbol{\tilde{z}}_{l,t}^{(q)} \label{equ:weighted_steering} \\
    &= \boldsymbol{\Psi}_{t}^{(q)}(\Theta_{l}) \boldsymbol{\upsilon}_{l,t} + \boldsymbol{\tilde{z}}_{l,t}^{(q)},  \quad\forall t, l \in \Phi. \label{equ:estimate_h_matrix_form} %l~\text{with}~\Omega_l \neq \emptyset, 
    %\notag \\ 
    %&= \boldsymbol{\Psi}^{(q)}_{l,t} \boldsymbol{v}_{l,t} + \boldsymbol{\tilde{z}}_{l,t}^{(1)}, 
\end{align}
\end{subequations}
%where
%\begin{align}
%    \boldsymbol{\Psi}_{l,t}^{(q)} =  \boldsymbol{G}\text{diag}(\phi_t^{(q)})\boldsymbol{a}_{\rm I}
%\end{align}
In the above, $\boldsymbol{\tilde{z}}_{l,t}^{(q)}$ in \eqref{equ:noiseless+noise} is the error of estimating $\bar{\boldsymbol{h}}_{l,t}^{(q)}$; 
$\boldsymbol{\psi}_t^{(q)}(\bar{d}_{k},\theta_{k})$ in \eqref{equ:weighted_steering} is defined as
\begin{align}\label{equ:effective_steering}
    \boldsymbol{\psi}_t^{(q)}(\bar{d}_{k},\theta_{k}) = \boldsymbol{G}\text{diag} (\boldsymbol{\phi}_t^{(q)}) \boldsymbol{a}_\text{I}(\bar{d}_{k},\theta_k), \forall k \in \Omega_l,
\end{align}
where $\bar{d}_k$ is the effective distance of target $k$ to the IRS, which is defined as
\begin{align}
\bar{d}_{k}=\left\{\begin{matrix}
 d_{k}, & \text{if}~k\in\Omega_l^{\rm N},  \\ 
 \infty, & \text{if}~k\in\Omega_l^{\rm F};
\end{matrix}\right.\label{eq:model range}
\end{align}
$\boldsymbol{\Psi}_{t}^{(q)}(\Theta_{l}) = \boldsymbol{G}\text{diag} (\boldsymbol{\phi}_t^{(q)})\boldsymbol{A}_{\rm I}(\Theta_l) \in \mathbb{C}^{M_{\rm B} \times K_l}$ in \eqref{equ:estimate_h_matrix_form} is defined as the effective steering matrix and $\boldsymbol{A}_{\rm I}(\Theta_l) \in \mathbb{C}^{M_{\rm I} \times K_l}$ is the steering matrix for the target-IRS channels with each column given by $\boldsymbol{a}_{\rm I}(\bar{d}_k,\theta_k), ~\forall k \in \Omega_l$, where %$\mathcal{D}_l = \{\bar{d}_k | k \in \Omega_l\}, \forall l$ is the set consisting of the effective distances of all targets located in the range cluster $l$, and $\Theta_l = \{\theta_{k} | k \in \Omega_l \}$ is the set consisting of the AOAs of all targets located in the range cluster $l$;
$\Theta_l = \{ (\bar{d}_k,\theta_{k}) | k \in \Omega_l \}$ is the set consisting of the pairs of the effective distances and AOAs of all targets located in the range cluster $l \in \Phi$;
%$\boldsymbol{\Psi}^{(q)}_{l,t} = [\boldsymbol{\psi}^{(q)}_t(\bar{d}_k,\theta_k),\boldsymbol{\psi}^{(q)}_t(\bar{d}_k,\theta_k)]$
% \begin{align}
%     &\boldsymbol{\Psi}_{t}^{(q)}(\boldsymbol{\bar{d}}_{l},\boldsymbol{\theta}_{l}) \notag \\
%     &= \boldsymbol{G}\text{diag}(\boldsymbol{\phi}_t^{(q)}) \boldsymbol{A}_{\rm I}(\boldsymbol{\bar{d}}_{l},\boldsymbol{\theta}_{l}) \notag \\
%     &= \boldsymbol{G}\text{diag}(\boldsymbol{\phi}_t^{(q)})[\boldsymbol{a}_\text{I}(\bar{d}_{l,m_{1}},\theta_{l,m_1}),\dots,\boldsymbol{a}_\text{I}(\bar{d}_{l,m_{K_l}},\theta_{l,m_{K_l}})] \notag \\
%     &= [\boldsymbol{\psi}_{t}^{(q)}(\bar{d}_{l,1},\theta_{l,1}),\dots,\boldsymbol{\psi}_{t}^{(q)}(\bar{d}_{l,m_{K_l}},\theta_{l,m_{K_l}})] \in \mathbb{C}^{M_{\rm B} \times K_l}
% \end{align}
%$\boldsymbol{\Psi}_{l,t}^{(q)}(\boldsymbol{\bar{d}}_{l},\boldsymbol{\theta}_{l}) = [\boldsymbol{\psi}_{l,t}^{(q)}(\bar{d}_{l,1},\theta_{l,1}),\dots,\boldsymbol{\psi}_{l,t}^{(q)}(\bar{d}_{l,m_{K_l}},\theta_{l,m_{K_l}})] \in \mathbb{C}^{M_{\rm B} \times K_l}$ 
% is defined as the effective steering matrix where each column $\boldsymbol{\psi}^{(q)}_{t}(\bar{d}_{l,m_i},\theta_{l,m_i})$ is the effective steering vector of the IRS towards a point with distance $\bar{d}_{l,m_i}$ and AOA $\theta_{l,m_i}$ given by
% \begin{align}\label{eq:effect steering}
%     \boldsymbol{\psi}^{(q)}_{t}(\bar{d}_{l,m_i},\theta_{l,m_i}) = \boldsymbol{G}\text{diag}(\boldsymbol{\phi}_t^{(q)})\boldsymbol{a}_\text{I}(\bar{d}_{l,m_{i}},\theta_{l,m_i}),
% \end{align}
% with each element of $\boldsymbol{a}_\text{I}(\bar{d}_k,\theta_k)$ given in \eqref{eq:a_m}; 
and $\boldsymbol{v}_{l,t} \in \mathbb{C}^{K_l \times 1}$ is the vector consisting of $v_{k,t}$'s, $\forall k \in \Omega_l$, in range cluster $l$. Note that \eqref{equ:expand_far_near} holds because the steering vector of a far-field target can be expressed as $\boldsymbol{a}_\text{I}(d\to\infty,\theta)$ as shown in Section \ref{subsec:channel model}. 
%Note that for a far-field target $k$, we define its effective distance to the IRS as infinity, because its steering vector reduces to $\boldsymbol{a}_\text{I}(d\to\infty,\theta)$.

%Note that based on the signal model (\ref{eq:estimated_channel cluster l}).


%From (\ref{eq:estimated_channel cluster l}), this is an interesting observation that 

%However

%We may treat $\boldsymbol{\psi}^{(q)}_t(\bar{d}_k,\theta_k)$ given in \eqref{eq:effect steering} as the effective steering vector of the IRS towards a point with distance $\bar{d}_k$ and AOA $\theta_k$. 
As mentioned before, one key challenge to leverage the IRS as a passive anchor is the passive nature of the IRS - we need to estimate the AOAs and/or ranges from the targets to the IRS based on the signals received by the BS. Amazingly, by treating $\boldsymbol{\psi}_{t}^{(q)}(\bar{d}_k,\theta_k)$ in \eqref{equ:effective_steering} as the steering vector of the IRS towards the target $k$, whose AOA and effective distance to the IRS are denoted by $\theta_k$ in \eqref{eqn:AOA} and $\bar{d}_{k}$ in \eqref{eq:model range}, equation \eqref{eq:estimated_channel cluster l} mathematically describes a virtual system where the effective observations, $\boldsymbol{\tilde{h}}_{l,t}^{(q)}$'s, are the weighted sum of the IRS's effective steering vectors towards the targets in the range cluster $l$. 
Such a virtual system is similar to the conventional multi-antenna system for estimating AOA and/or range information of the near-field and far-field targets, where there are LOS paths between targets and the multi-antenna receive anchor, and the signals received by the anchor is the weighted sum of steering vectors towards the targets \cite{Zheng_2019_TAP}. Therefore, our localization task is to extract the ranges and AOAs from the estimated channels $\{\boldsymbol{\tilde{h}}_{l,t}^{(q)}\}_{q=1,t=1}^{Q,V}$. In the literature, there exist plenty of classic signal processing methods to deal with the system described in \eqref{eq:estimated_channel cluster l}, such as the MUSIC algorithm and the ESPRIT algorithm \cite{aoa_survey}. In this paper, we will apply the MUSIC algorithm to extract the AOA information of the far-field targets and the AOA together with the range information of the near-field targets from $\{\boldsymbol{\tilde{h}}_{l,t}^{(q)}\}_{q=1,t=1}^{Q,V}$. 
%To make the MUSIC algorithms work, we need to make sure that the effective steering vectors at the $q$th OFDM symbol duration in different coherence blocks are identical. Therefore, we set a common IRS reflecting pattern at the $q$th OFDM symbol duration over all the coherence blocks, i.e., 

However, there are two differences between our AOA and/or range estimation system \eqref{eq:estimated_channel cluster l} and the conventional multi-antenna localization system \cite{Zheng_2019_TAP} that make it impossible to directly apply the MUSIC algorithm on \eqref{eq:estimated_channel cluster l}.

\begin{itemize}
\item \textbf{Difference 1}: In the conventional localization system, the steering vectors are static over time. However, in our system \eqref{eq:estimated_channel cluster l}, because the IRS reflecting coefficients $\boldsymbol{\phi}_t^{(q)}$'s can change over different OFDM symbol durations and different coherence blocks, the steering vectors can be dynamic. 

\item \textbf{Difference 2}: In the conventional localization system, it is required that the number of receive antennas is larger than the number of targets. Mathematically, this implies that the rank of the matrix that contains the steering vectors towards all the targets is equal to the number of targets such that the null space of this matrix exists to implement the MUSIC algorithm. However, the rank of $\boldsymbol{\Psi}_{t}^{(q)}(\Theta_{l})$ in \eqref{equ:estimate_h_matrix_form} can be smaller than the number of targets, even when the number of BS antennas and the number of IRS reflecting elements are both larger than the number of targets in each range cluster. This is because the rank of $\boldsymbol{\Psi}_{t}^{(q)}(\Theta_{l}) = \boldsymbol{G} \text{diag} (\boldsymbol{\phi}^{(q)}) \boldsymbol{A}_{\rm I}(\Theta_l)$ is limited by the rank of $\boldsymbol{G}$. For example, when the BS and the IRS are in the far-field region of each, it can be shown that the rank of the LOS channel between the BS and the IRS, which is denoted by $r_{\boldsymbol{G}}$, is one. In this case, the rank of $\boldsymbol{\Psi}_{t}^{(q)}(\Theta_{l})$ is also one, and it is impossible to estimate the AOA and/or range information of the targets in each range cluster $l \in \Phi$ based on the MUSIC algorithm.
\end{itemize}

In the following, we propose a novel method to tackle the above two challenges for enabling the MUSIC algorithm on \eqref{eq:estimated_channel cluster l}. There are two key ideas under our proposed method. First, given any $q$th OFDM symbol duration, we set a common IRS reflecting pattern, denoted by $\boldsymbol{\bar{\phi}}^{(q)}$, over all the coherence blocks, i.e., 
\begin{align}\label{equ:phi_equ_t} 
    \boldsymbol{\phi}_{t}^{(q)}=\boldsymbol{\bar{\phi}}^{(q)},\quad \forall t, q.
\end{align}
In this case, the effective steering vectors in (\ref{equ:effective_steering}) reduce to
\begin{align}\label{eq:new array}
    \boldsymbol{\bar{\psi}}^{(q)}(\bar{d}_{k},\theta_{k}) = \boldsymbol{G}\text{diag}(\boldsymbol{\bar{\phi}}^{(q)}) \boldsymbol{a}_\text{I}(\bar{d}_{k},\theta_{k}), \forall q,t, k \in \Omega_l,
\end{align}
and the estimated channels given in (\ref{eq:estimated_channel cluster l}) reduce to
\begin{align}\label{equ:est_channel_reduce}
  \boldsymbol{\tilde{h}}_{l,t}^{(q)} &= \sum_{k\in\Omega_l} v_{k,t} \boldsymbol{\bar{\psi}}^{(q)}(\bar{d}_k,\theta_k) + \boldsymbol{\tilde{z}}_{l,t}^{(q)} \notag \\
  &= \bar{\boldsymbol{\Psi}}^{(q)}(\Theta_{l}) \boldsymbol{\upsilon}_{l,t} + \boldsymbol{\tilde{z}}_{l,t}^{(q)}, ~\forall t, l\in \Phi,
\end{align}
where $\bar{\boldsymbol{\Psi}}^{(q)}(\Theta_{l})  \in \mathbb{C}^{M_{\rm B} \times K_l}$ has similar definition to $\boldsymbol{\Psi}_{t}^{(q)}(\Theta_{l})$.
%where $\bar{\boldsymbol{\Psi}}^{(q)}(\mathcal{D}_{l},\Theta_{l}) = [\boldsymbol{\bar{\psi}}^{(q)}(\bar{d}_{k_1},\theta_{k_1}),\dots,\boldsymbol{\bar{\psi}}^{(q)}(\bar{d}_{k_{|\Omega_l|}},\theta_{k_{|\Omega_l|}})]$ with $k_i \in \Omega_l, \forall i \in \{1,\dots,|\Omega_l|\}$.

%where $\boldsymbol{\Psi}^{(q)}(\boldsymbol{\bar{d}}_{l},\boldsymbol{\theta}_{l}) = [\boldsymbol{\psi}^{(q)}(\bar{d}_{l,m_1},\theta_{l,m_1}),\dots, \boldsymbol{\psi}^{(q)}(\bar{d}_{l,m_{K_l}},\theta_{l,m_{K_l}})]$.
%Then we can adopt the MUSIC algorithm \cite{music} on \eqref{equ:est_channel_reduce} to achieve the three goals listed in the above.
%In this work, we adopt the MUSIC algorithm to extract the range and AOA information from the estimated channels $\{\boldsymbol{\tilde{h}}_{l,t}^{(q)}\}_{q=1,t=1}^{Q,T}$ in Phase II. 
%However, certain conditions on the effective steering matrix $\boldsymbol{\Psi}_{l,t}^{(q)}(\boldsymbol{\bar{d}}_{l},\boldsymbol{\theta}_{l})$ should be satisfied to guarantee the accurate identification of $K_l$ sources even if there is no estimation error in \eqref{eq:estimated_channel cluster l}, i.e., $\boldsymbol{\tilde{h}}_{l,t}^{(q)} = \boldsymbol{\bar{h}}_{l,t}^{(q )}$. 
% However, there still exists one challenge to implement the MUSIC algorithm, where the effective steering matrix $\boldsymbol{\Psi}_{l,t}^{(q)}(\boldsymbol{\bar{d}}_{l},\boldsymbol{\theta}_{l})$ should be satisfy the condition $\text{rank}(\boldsymbol{\Psi}_{l,t}^{(q)}(\boldsymbol{\bar{d}}_{l},\boldsymbol{\theta}_{l})) \ge K_l$ to guarantee the identification of $K_l$ sources even if there is no estimation error in \eqref{eq:estimated_channel cluster l}, i.e., $\boldsymbol{\tilde{h}}_{l,t}^{(q)} = \boldsymbol{\bar{h}}_{l,t}^{(q )}$.
% %One condition is that  is that $\text{rank}(\boldsymbol{\Psi}_{l,t}^{(q)}(\boldsymbol{\bar{d}}_{l},\boldsymbol{\theta}_{l})) \ge K_l$. 
% Based on the relationship $\text{rank}(AB) \le \min\{\text{rank}(A),\text{rank}(B)\}$, we have $r_{\boldsymbol{G}} \ge K_l$ and $\text{rank}(\boldsymbol{A}(\boldsymbol{\bar{d}}_{l},\boldsymbol{\theta}_{l})) \ge K_l$ by defining $r_{\boldsymbol{G}} = \text{rank}(\boldsymbol{G})$. Assuming the IRS is built by the ULA architecture, $\text{rank}(\boldsymbol{A}(\boldsymbol{\bar{d}}_{l},\boldsymbol{\theta}_{l})) = K_l$ is always satisfied \cite{linear_indpend} and thus the condition $\text{rank}(\boldsymbol{G}) \ge K_l$ should be satisfied. Based on the channel model introduced in Section \ref{subsec:channel model}, this condition cannot be always satisfied, e.g., $r_{\boldsymbol{G}} = 1$ in the scenario where the BS is located at the far-field region of the IRS. In this work, both of the rank-sufficient case with $r_{\boldsymbol{G}} \ge K_l$ and rank-deficient case with $r_{\boldsymbol{G}} < K_l$ shall be considered and the corresponding schemes based on MUSIC for range and AOA information extraction in these two cases will be proposed. This section concentrates on the introduction of the three-phase localization framework and the design of the specific scheme to extract the AOA and range information will be specified in the next section.
%As such, depending whether $\text{rank}(\boldsymbol{G}) \ge K_l$ or $\text{rank}(\boldsymbol{G}) < K_l$, we shall design two specific schemes to extract the range and AOA information , whose details will be illustrated in the next section.

\begin{figure}[t]
	\centering
    %\includegraphics[width=.9\textwidth]{SignalModel.eps}
    \includegraphics[width=.47\textwidth]{orignal_new_signal.eps}
    \vspace{-0.3cm}
    \caption{Illustration of original signal model and newly created signal model. Specifically, we combine all the estimated channels of the first $Q_0$ OFDM symbols within one coherence block together to create a new virtual high-dimension signal, which contains information in both spatial (BS antennas) and temporal (OFDM symbols) domain.}\label{fig:multisignal}
    \vspace{-0.6cm}
    %\vspace{-0.6cm}
\end{figure}

After adopting the first idea to employ a common IRS reflecting pattern at the same OFDM symbol durations of different coherence blocks in \eqref{equ:phi_equ_t}, the two challenges listed in the above still exist in the new system described in \eqref{eq:estimated_channel cluster l}. In the following, we introduce the second idea that can tackle these two challenges together based on \eqref{equ:est_channel_reduce} - leveraging temporal domain signals. Specifically, at each coherence block $t$, given any range cluster $l \in \Phi$, we combine the temporal domain channels over the first $Q_0 \le Q$ OFDM symbol durations together to create a new virtual channel vector as follows
\begin{align}\label{equ:virtual_channel_l}
    \boldsymbol{\tilde{h}}_{l,t} = [(\boldsymbol{\tilde{h}}_{l,t}^{(1)})^T,\dots,(\boldsymbol{\tilde{h}}_{l,t}^{(Q_0)})^T]^T, ~\forall t, l \in \Phi.
\end{align}
The selection of $Q_0$ will be discussed later.
According to \eqref{equ:est_channel_reduce}, it can be shown that
\begin{align}%\label{equ:estimated_channel_virtual}
    \boldsymbol{\tilde{h}}_{l,t} &= \sum_{k\in\Omega_l} v_{k,t} \boldsymbol{\breve{\psi}}(\bar{d}_k,\theta_k) + \boldsymbol{\tilde{z}}_{l,t} \notag \\ 
    &= \boldsymbol{\breve{\Psi}}(\Theta_{l}) \boldsymbol{\upsilon}_{l,t}  + \boldsymbol{\tilde{z}}_{l,t}, ~\forall t, l \in \Phi,
    %&~~~~~= \left[\breve{\boldsymbol{\psi}}(\bar{d}_{l,m_1},\theta_{l,m_1}), \dots, \breve{\boldsymbol{\psi}}(\bar{d}_{l,m_{K_l}},\theta_{l,m_{K_l}})\right]\boldsymbol{\upsilon}_{l,t}, 
    \label{equ:estimated_channel_deficient_noisy}
\end{align}
where
\begin{align}
    %\breve{\boldsymbol{\Psi}}_{l,t} =&~ [\breve{\boldsymbol{\psi}}_{l,t,1}, \dots, \breve{\boldsymbol{\psi}}_{l,t,K_l}], \\  %[\text{diag}(\boldsymbol{\phi}^{(1)}),\dots,\text{diag}(\boldsymbol{\phi}^{(Q)})]^T \sum_{k \in \Omega_{l}} \beta_k\gamma_{k,t}\boldsymbol{a}(\bar{d}_k,\theta_{k})
    \boldsymbol{\breve{\Psi}}(\Theta_{l}) &= \boldsymbol{P}\boldsymbol{A}_{\rm I}(\Theta_{l}), ~\forall l \in \Phi, \label{equ:vir_steering} \\ %= \begin{bmatrix}
%                              %\boldsymbol{G}\text{diag}(\boldsymbol{\bar{\phi}}^{(1)}) 
%                              \boldsymbol{P}^{(1)} \boldsymbol{a}_{\rm I}(\bar{d}_{k},\theta_{k}) \\
%                              \vdots \\
%                              %\boldsymbol{G}\text{diag}(\boldsymbol{\bar{\phi}}^{(Q)})
%                              \boldsymbol{P}^{(Q_0)} \boldsymbol{a}_{\rm I}(\bar{d}_{k},\theta_{k}) 
%                            \end{bmatrix}, ~\forall k \in \Omega_l, \label{equ:vir_steering} \\
    \boldsymbol{P} &= [(\boldsymbol{P}^{(1)})^T,\dots,(\boldsymbol{P}^{(Q_0)})^T]^T, \\
    \boldsymbol{P}^{(q)} &= \boldsymbol{G}\text{diag}(\boldsymbol{\bar{\phi}}^{(q)}), ~\forall q \in \{1,\dots,Q_0\},\\
    \boldsymbol{\tilde{z}}_{l,t} &= [(\boldsymbol{\tilde{z}}_{l,t}^{(1)})^T,\dots,(\boldsymbol{\tilde{z}}_{l,t}^{(Q_0)})^T]^T, ~\forall t, l \in \Phi. \label{equ:vir_noise}
    %\boldsymbol{A}_{{\rm I},l} =&~ [\boldsymbol{a}_{\rm I}(\bar{d}_{l,1},\theta_{l,1}),\dots,\boldsymbol{a}_{\rm I}(\bar{d}_{l,K_l},\theta_{l,K_l})],                        
\end{align}

The above creation of the new virtual signals by combining the signals in both the spatial and temporal domains is illustrated in Fig. \ref{fig:multisignal}.
Next, we show that the two challenges listed in the above disappear in the new system described in \eqref{equ:estimated_channel_deficient_noisy} such that we can apply the MUSIC algorithm to estimate the AOA and/or range information of all the targets in each range cluster $l \in \Phi$. First, it is observed that the virtual steering matrices $\boldsymbol{\breve{\Psi}}(\Theta_{l})$'s in \eqref{equ:vir_steering} are static over time. Second, we tackle the challenge about matrix rank. 
For convenience, we define $K^{\rm max} = \max\{K_l, \forall l \in \Phi\}$
% and $\boldsymbol{W}^{\rm DFT} = [\boldsymbol{w}^{\rm DFT}_1,\dots,\boldsymbol{w}^{\rm DFT}_{M_{\rm I}}] \in \mathbb{C}^{M_{\rm I} \times M_{\rm I}}$ as the discrete fourier transform (DFT) matrix. We also define two matrices as $\boldsymbol{W} \in \mathbb{C}^{M_{\rm I} \times Q_0}$ with each column given by $\boldsymbol{w}_{q} = \boldsymbol{w}^{\rm DFT}_{\kappa(q)}$ and $\boldsymbol{W}^c\in \mathbb{C}^{M_{\rm I} \times (M_{\rm I} - Q_0)}$ with each column given by $\boldsymbol{w}^c_{m} = \boldsymbol{w}^{\rm DFT}_{\kappa(Q_0+m)}$, where $\{\kappa(1),\dots,\kappa(M_{\rm I})\}$ is a reordered sequence of $\{1,\dots,M_{\rm I}\}$, 
\begin{theorem}\label{theorem1}
We have $\text{rank}(\breve{\boldsymbol{\Psi}}(\Theta_{l})) = K_l$, $\forall l \in \Phi$ almost surely when the following conditions are satisfied:
\begin{enumerate}
        \item The number of utilized OFDM symbols satisfies $Q_0 \ge K^{\rm max}$;
        %\item Given the array manifold of the IRS $\mathcal{A}_{\rm I}$, any $\tilde{M}_{\rm I} \ge K^{\rm max}$ different steering vectors $\boldsymbol{a}_{\rm I}(\bar{d},\theta)$'s from the array manifold are linearly independent;
        \item The IRS reflecting elements follow ULA with element spacing $\frac{\lambda}{2}$ and the number of IRS elements satisfies $M_{\rm I} \ge K^{\rm max}$;
        \item The coefficient of each reflecting element $m_{\rm I}$ during OFDM symbol $q$ is set as
        \vspace{-6pt}
        \begin{align}\label{equ:IRS_phi_design}
            &\bar{\phi}^{(q)}_{m_{\rm I}} = \frac{w_{m_{\rm I},q}^{\rm I}}{G_{1,m_{\rm I}}}e^{j (m_{\rm I}-1) \vartheta}, \notag \\
            & \forall m_{\rm I} \in \{1,\dots,M_{\rm I}\}, q \in \{1,\dots,Q_0\},
        \end{align}
        where $w_{m_{\rm I},q}^{\rm I}$ is the element on the $m_{\rm I}$-th row and the $q$-th column element of the DFT matrix $\boldsymbol{W}^{\rm I} \in \mathbb{C}^{M_{\rm I} \times M_{\rm I}}$ and $\vartheta$ is an arbitrary angle in the region of $[0, 2\pi)$.
\end{enumerate}
\end{theorem}


\begin{IEEEproof}
Please refer to Appendix \ref{appendix1}.
\end{IEEEproof}



%\textcolor[rgb]{0.00,0.07,1.00}{
%\begin{theorem}\label{theorem1}
%%The rank of $\breve{\boldsymbol{\Psi}}(\mathcal{D}_{l},\Theta_{l}),~\forall l$ is equal to $K_l, ~\forall l \in \Phi$, almost surely suppose the following conditions are satisfied
%    %For the matrix $\breve{\boldsymbol{\Psi}}(\mathcal{D}_{l},\Theta_{l}),~\forall l$ given in \eqref{equ:estimated_channel_deficient_noisy}, suppose that 
%In the case where the BS-IRS channel follows the far-field channel model \eqref{equ:far_irs_bs_channel}, suppose the following conditions are all satisfied:
%    \begin{itemize}
%        %\item[1.] The product of the rank of the channel $\boldsymbol{G}$ and the number of OFDM symbols in each coherence block is larger than the number of targets in each range cluster $l$, i.e., $S = Q \times r_{\boldsymbol{G}}>K_l,~\forall l$, while the number of IRS elements is larger than the number of targets in each range cluster $l$, i.e., $M_{\rm I} > K_l, \forall l \in \Phi$.
%        \item [1)] For the matrix $\boldsymbol{P} = [(\boldsymbol{P}^{(1)})^T,\dots,(\boldsymbol{P}^{(Q_0)})^T]^T \in \mathbb{C}^{Q_0M_\text{B} \times M_\text{I}}$, its row size is larger than the maximal number of targets in each range cluster $l \in \Phi$, i.e., $Q_0 M_\text{B} > \max\{K_l, l \in \Phi\}$, and its rank is also no smaller than $\max\{K_l, l \in \Phi\}$, i.e., $\text{rank}(\boldsymbol{P}) \ge \max\{K_l, l \in \Phi\}$.
%        \item[2)] Any $M_{\rm I}$ different steering vectors $\boldsymbol{a}(\bar{d},\theta)$'s for the IRS are linearly independent; %(for instance, ULA with element spacing smaller than $\lambda/2$ is adopted);
%        %The IRS is modeled as a ULA, where the number of IRS elements is larger than the number of targets in each range cluster $l$, i.e., $M_\text{I}>K_l$, $\forall l$, and the IRS element spacing is smaller than the half of the wavelength $\frac{\lambda}{2}$;
%        \item[3)] The column space of matrix $\boldsymbol{P}^T$ can be represented as $\mathcal{N}(\boldsymbol{P}^{T}) = \text{span}\left(\boldsymbol{a}_{\rm I}(\tilde{d}_{1},\tilde{\theta}_{1}),\dots,\boldsymbol{a}_{\rm I}(\tilde{d}_{M_{\rm I}-r},\tilde{\theta}_{M_{\rm I}-r})\right)$ and $\{(\tilde{d}_{1},\theta_{1}),\dots,(\tilde{d}_{M_{\rm I}-r},\tilde{\theta}_{M_{\rm I}-r})\} \cap \Lambda_{(\bar{d},\theta)} = \emptyset$ for $\max\{K_l, l \in \Phi\} \le r$. Here, $\Lambda_{(\bar{d},\theta)}$ denote the interested regions of all the pairs of effective distance and AOA.
%        %There exists $\{(\bar{d}^{0}_{1},\theta^{0}_{1}),\dots,(\bar{d}^{0}_{M_{\rm I}-K_l},\theta^{0}_{M_{\rm I}-K_l})\}$ with $\{(\bar{d}^{0}_{1},\theta^{0}_{1}),\dots,(\bar{d}^{0}_{M_{\rm I}-K_l},\theta^{0}_{M_{\rm I}-K_l})\} \cap \Lambda_{(\bar{d},\theta)} = \emptyset$ and $\Lambda_{(\bar{d},\theta)}$ denoting the interested regions of all the pairs of effective distance and AOA. It also holds that $\mathcal{N}(\boldsymbol{P}) \in \text{span}(\boldsymbol{a}_{\rm I}(\bar{d}^{0}_{1},\theta^{0}_{1}),\dots,\boldsymbol{a}_{\rm I}(\bar{d}^{0}_{M_{\rm I}-K_l},\theta^{0}_{M_{\rm I}-K_l}))$. %and $\text{rank}(\boldsymbol{P}) > K_l, ~\forall l$.
%        %\item[3.] For any $(\bar{d},\theta) \in (\Lambda_{\bar{d},l},\Lambda_{\theta,l})$, we have $\breve{\psi}_i(\bar{d},\theta) \ne 0, ~\forall i,l \in \Phi$ with $\breve{\psi}_i(\bar{d},\theta)$ denoting the $i$th element of  $\breve{\boldsymbol{\psi}}_i(\bar{d},\theta)$. Here, $\Lambda_{\bar{d},l}$ and $\Lambda_{\theta,l}$ denote the interested regions of the effective distance and AOA in range cluster $l$;
%       %\item[3.] For any $\{(\bar{d}_k,\theta_k) \in (\Lambda_{\bar{d}},\Lambda_{\theta})\}_{k=1}^{\tilde{M} + 1}$, we have $\text{spark}\left( [\breve{\boldsymbol{\psi}}(\bar{d}_1,\theta_1), \dots, \breve{\boldsymbol{\psi}}(\bar{d}_{S + 1},\theta_{S + 1}) ] \right) = S + 1$,
%        %\item[3.] The IRS adopts the random reflection approach, where the reflection coefficient of each reflection element is given by $\bar{\phi}_{m}^{(q)} = e^{-j\varrho_{m}^{(q)}}$ and $\varrho_{m}^{(q)},~\forall m$ is independently and identically distributed in the region $[0,\pi)$.
%    \end{itemize} %it holds almost surely that $\text{rank}\left(\breve{\boldsymbol{\Psi}}(\mathcal{D}_{l},\Theta_{l})\right) = K_l$, $\forall l$.
%we have $\text{rank}(\breve{\boldsymbol{\Psi}}(\Theta_{l})) = K_l$, $\forall l \in \Phi$.
%\end{theorem}}
%
%\begin{IEEEproof}
%Please refer to Appendix \ref{appendix1}.
%\end{IEEEproof}

%\textcolor[rgb]{0.00,0.07,1.00}{
%Here, we give a scheme for the reflection pattern design of the IRS to satisfy the above three conditions of Theorem \ref{theorem1} in the case where the BS-IRS channel follows far-field channel model. In specific, we propose to utilize $Q_0 \ge \max\{K_l, l \in \Phi\}$ OFDM symbols in each coherence block and the ULA with $\lambda/2$ element spacing is adopted for the IRS. Consider one set $\{ \tilde{\theta}_{1}, \dots, \tilde{\theta}_{Q_0} \}$ satisfying $\{ (\infty,\tilde{\theta}_{1}), \dots, (\infty,\tilde{\theta}_{Q_0}) \} \cap \Lambda_{(\bar{d},\theta)} = \emptyset$, we determine the reflection coefficient of each element $m_{\rm I}$ at the IRS during the OFDM symbol $q$ as
%\begin{align}\label{equ:phi_design_far_field}
%    \boldsymbol{\phi}^{(q)}_{m_{\rm I}} = \frac{a_{{\rm I},m_{\rm I}}(\infty,\tilde{\theta}_{q})}{a_{{\rm I},m_{\rm I}}(\infty,\xi)}, ~\forall m_{\rm I} \in \{1,\dots,M_{\rm I}\}, q \in \{1,\dots,Q_0\},
%\end{align}
%where $a_{{\rm I},m_{\rm I}}(\infty,\theta_{q})$ is the $m_{\rm I}$th element of the steering vector $\boldsymbol{a}_{{\rm I}}(\infty,\theta_{q})$.}

%\textcolor[rgb]{0.00,0.07,1.00}{
%Based on the above design, it is obvious that the condition 2) holds under the ULA setup with $\lambda/2$ element spacing \cite{linear_indpend}. Denote the $n$th row of the matrix $\boldsymbol{P}^{(q)}$ as $\boldsymbol{p}^{(q)}_{n}$, we can easily obtain the $\text{rank}(\boldsymbol{P}) = \text{rank}([(\boldsymbol{p}^{(1)}_{1})^T,\dots,(\boldsymbol{p}^{(Q_0)}_{1})^T]^T) = Q_0$ based on the far-field channel model and the condition 2), which indicates that condition 1) is satisfied. Finally, we can have $\boldsymbol{p}^{(q)}_{n} = \varpi_n^{(q)} (\boldsymbol{a}_{\rm I}(\infty,\tilde{\theta}_q))^T$ based on the reflection design in \eqref{equ:phi_design_far_field}, which indicates that condition 3) holds as well. Therefore, the proposed reflection pattern design guarantees that $\text{rank}(\breve{\boldsymbol{\Psi}}(\Theta_{l})) = K_l$, $\forall l$.}


The first condition can be easily satisfied since the number of OFDM symbols in each coherence block is much larger than $K^{\rm max}$ in practice. Moreover, in practice, the IRS consists of a huge number of reflecting elements, and a common spacing of $\lambda/2$ between adjacent elements is a standard IRS pattern.\footnote{Although we cannot prove this rigorously, we find in all the numerical examples that Theorem 1 is still true if the IRS element spacing is less than $\lambda/2$.} Thus, the second condition also holds. 
%As usual, the targets are only distributed in a part of the whole area, thus we can carefully select $\boldsymbol{W}$ and $\vartheta$ to ensure that $\text{rank}(\breve{\boldsymbol{\Psi}}(\Theta_{l})) = K_l$, $\forall l \in \Phi$, always satisfies with condition 3).
Theorem 1 thus implies that as long as the IRS reflecting coefficients are set as \eqref{equ:IRS_phi_design}, then the MUSIC algorithm can work on the signals \eqref{equ:virtual_channel_l} almost surely. Note that if $Q_0=1$, i.e., $\boldsymbol{\tilde{h}}_{l,t} = \boldsymbol{\tilde{h}}^{(1)}_{l,t}, ~\forall t, l \in \Phi$, we do not combine channels over different symbols as shown in \eqref{equ:virtual_channel_l}, then the rank of $\breve{\boldsymbol{\Psi}}(\Theta_{l})$ is limited by the rank of the BS-IRS channel, i.e., $r_{\boldsymbol{G}}$. In this case, the MUSIC algorithm cannot work on the signals given in \eqref{equ:estimated_channel_deficient_noisy} if $r_{\boldsymbol{G}}< K^{\rm max}$. Theorem \ref{theorem1} shows that by creating a virtual signal as in \eqref{equ:virtual_channel_l} that consists of channel vectors over a sufficient number of OFDM symbols, the MUSIC algorithm can work no matter what is the value of $r_{\boldsymbol{G}}$. 

One more comment is about the first condition in Theorem \ref{theorem1}. Although we cannot prove it rigorously, we find via a vast number of numerical examples that $\text{rank}(\breve{\boldsymbol{\Psi}}(\Theta_{l})) = K_l, \forall l \in \Phi$ even when the first condition in Theorem \ref{theorem1} is relaxed to $Q_0 r_{\boldsymbol{G}} \ge K^{\rm max}$. Intuitively, this is because $\text{rank}(\boldsymbol{P})=Q_0 r_{\boldsymbol{G}}$ and according to \eqref{equ:vir_steering}, the rank of $\breve{\boldsymbol{\Psi}}(\Theta_{l}) = \boldsymbol{P} \boldsymbol{A}(\mathcal{D}_l)$ should be upper bounded by $Q_0 r_{\boldsymbol{G}}$. Note that a small $Q_0$ is preferred in practice because the complexity to implement the MUSIC algorithm to \eqref{equ:estimated_channel_deficient_noisy} depends on the dimension of the virtual signals. Therefore, in practice, we can set $Q_0=\ceil*{\frac{K^{\rm max}}{r_{\boldsymbol{G}}}}$ to make the MUSIC algorithm applicable to \eqref{equ:estimated_channel_deficient_noisy} with the minimum complexity.  
Based on Theorem \ref{theorem1}, we can have the following corollary.
\begin{corollary}
In the asymptotic regime with $V \to \infty$, both of the far-field and near-field targets can be perfectly detected by the MUSIC algorithm when the three conditions in Theorem \ref{theorem1} are satisfied. In particular, the AOA information of the far-field targets as well as the AOA and range information of the near-field targets can be perfectly estimated. 
\end{corollary}
\begin{IEEEproof}
In the asymptotic regime with $V \to \infty$, the sample covariance matrix of the virtual channel $\tilde{\boldsymbol{h}}_{l,t}$ is equal to the true covariance matrix. Then, the virtual steering vectors $\boldsymbol{\breve{\psi}}(\bar{d}_k,\theta_k)$'s of all targets can be perfectly detected by the MUSIC algorithm \cite{Wax_1989_TASSP} when all conditions in Theorem \ref{theorem1} hold. Since there is a one-to-one mapping between the pair $(\bar{d}_k,\theta_k)$ and $\boldsymbol{\breve{\psi}}(\bar{d}_k,\theta_k)$, the effective distances and AOAs of all targets can also be perfectly derived. From \eqref{eq:model range}, we can perfectly detect which target is in the far-field region and which target is in the near-field region. Due the fact that $\bar{d}_k \to \infty$ for far-field targets by \eqref{eq:model range}, only the AOA information can be perfectly estimated for each far-field target, while both of the AOA and range information for each near-field target can be perfectly estimated.
\end{IEEEproof}

%\textcolor[rgb]{0.00,0.07,1.00}{
%By empirical results, we find that $\text{rank}(\breve{\boldsymbol{\Psi}}(\Theta_{l})) = K_l$, $\forall l \in \Phi$ can usually be satisfied when $Q_0 r_{\boldsymbol{G}} \ge K^{\rm max}$ holds and the random reflection pattern with $\angle\phi^{(q)}_{m_{\rm I}} \sim \mathcal{U}(0,2\pi), ~\forall m_{\rm I},q$ is adopted. Intuitively, this is because $\text{rank}(\boldsymbol{P}) \ge K^{\rm max}$ might be enough to support $\text{rank}(\breve{\boldsymbol{\Psi}}(\Theta_{l})) = K_l$, $\forall l \in \Phi$ and $\text{rank}(\boldsymbol{P}) = Q_0 r_{\boldsymbol{G}}$ can usually be realized by the mentioned random reflection pattern design. Therefore, the condition 2) is possible to be relaxed by $Q_0 r_{\boldsymbol{G}} \ge K^{\rm max}$ in the practical implementation. When $r_{\boldsymbol{G}} \le M_{\rm B}$ is large, the complexity to extract AOA and/or range information from a high-dimension signal $\{\tilde{\boldsymbol{h}}_{l,t}\}_{t=1}^{T}$ is huge if $Q_0$ is also large. Therefore, it is beneficial to set $Q_0$ as the minimum integer that satisfies these conditions, e.g., $Q_0^{\rm min} = \ceil*{\frac{K^{\rm max}}{r_{\boldsymbol{G}}}}$, when $\boldsymbol{G}$ can be known when the BS and the IRS are deployed at known sites. Moreover, in practice, we can use historical data to see at most, how many targets can be in a range cluster. Then, we can set a proper value of $Q_0$. 
%}

%\textcolor[rgb]{0.00,0.07,1.00}{
%The above Theorem \ref{theorem1} provides the reflection pattern design in the case with the far-field BS-IRS channel.
%Intuitively, in the case where the BS-IRS channel follows the near-field channel model \eqref{eq:near irs bs channel}, we can adopt the similar reflection pattern design in Theorem \ref{theorem1} to guarantee that $\text{rank}(\breve{\boldsymbol{\Psi}}(\Theta_{l})) = K_l$, $\forall l \in \Phi$ if we use $Q_0 \ge \max\{K_l, \forall l \in \Phi\}$ OFDM symbols in each coherence block. However, when $Q_0 < \max\{K_l, \forall l \in \Phi\}$, it is hard to design a reflection pattern design that can strictly guarantee $\text{rank}(\breve{\boldsymbol{\Psi}}(\Theta_{l})) = K_l$, $\forall l \in \Phi$. By empirical results, we find that $\text{rank}(\breve{\boldsymbol{\Psi}}(\Theta_{l})) = K_l$, $\forall l \in \Phi$ usually holds by adopting the random reflection pattern design, where the phase of the coefficient of each reflection element $\angle\phi^{(q)}_{m_{\rm I}}$ of the IRS is independently and identically distributed with $\angle\phi^{(q)}_{m_{\rm I}} \sim \mathcal{U}(0,2\pi), ~\forall m_{\rm I},q$. In the following simulations, such reflection pattern design is utilized when $Q_0 < \max\{K_l, \forall l \in \Phi\}$.}

%\textcolor[rgb]{0.00,0.07,1.00}{
%In practice, there is usually a lot of OFDM symbols%\footnote{In 5G NR system, there are 14 OFDM symbols in each slot and the number of slots in each frame is usually larger than 10 \cite{3gpp_5G_nr}, which ensures us to accurately detect the targets in each range cluster}
%in each coherence block that is larger than the passive targets in each range cluster, i.e, $Q > K_l, ~\forall l \in \Phi$. Therefore, the condition $1)$ in Theorem \ref{theorem1} can be always satisfied. We also propose to adopt random reflection pattern for the IRS reflection coefficient design. It is hard to verify that such setting strictly satisfies condition 3), but empirical results show that such setting can realize accurate target localization.}

%\begin{remark}
%In the special case where the LOS channel between the IRS and the BS follows the far-field channel model in \eqref{equ:far_irs_bs_channel}, we can obtain $r_{\boldsymbol{G}} = 1$. As such, the number of OFDM symbols for sensing in each coherence block should be no less than $K_l$ to satisfy the first condition in Theorem \ref{theorem1}. 
%%This also indicates that by properly designing the reflection coefficients, we are able to estimate the AOA and range information of the targets even if the received signals at only one receive antenna is utilized. 
%\end{remark}


%\begin{remark}
%According to Theorem \ref{theorem1} and the above discussion, $Q_0$ should be large enough to satisfy the conditions 1). However, if $Q_0$ is too large, the complexity to extract AOA and/or range information from a high-dimension signal $\{\tilde{\boldsymbol{h}}_{l,t}\}_{t=1}^{T}$ is huge. Therefore, it is beneficial to set $Q_0$ as the minimum integer that satisfies these conditions when $\boldsymbol{G}$ can be known when the BS and the IRS are deployed at known sites. Moreover, in practice, we can use historical data to see at most, how many targets can be in a range cluster. Then, we can set a proper value of $Q_0$. 
%%If all the $Q$ OFDM symbols in each coherence block are utilized, we can find that the computational complexity of the MUSIC algorithm is $\mathcal{O}(Q^3M_{\rm B}^3 + SQM_{\rm B}(QM_{\rm B}-\hat{K}_l))$ where $S$ is number of search grids in MUSIC. However, we have no need to utilize the signals at all receive antennas at the BS if $r_{\boldsymbol{G}} < M_{\rm B}$ based on our previous work \cite{music_irs}. By utilizing the received signal of carefully selected $r_{\boldsymbol{G}}$ antennas and utilizing $Q_0 \le Q$ OFDM symbols in each coherence block with $Q_0 r_{\boldsymbol{G}} > \hat{K}_l$ for localization, the above complexities can be reduced to $\mathcal{O}(Q_0^3 r_{\boldsymbol{G}}^3 + SQ_0r_{\boldsymbol{G}}(Q_0 r_{\boldsymbol{G}}-\hat{K}_l)))$. 
%%\textcolor[rgb]{0.00,0.07,1.00}{Thus, in the following section \ref{sec:simulation}, we utilize $Q_0$ OFDM symbols in each coherence block to localize the far-field and near-field targets in simulations.}
%\end{remark}

%\begin{remark}
%Theorem \ref{theorem1} also implies that there is no need to utilize all the $Q$ OFDM symbols in each coherence block for localization if $Q_0 \times r_{\boldsymbol{G}} > K_l$ for some $Q_0 \le Q$. For instance, if LOS channel between the IRS and the BS follows the near-field channel model in \eqref{eq:near irs bs channel} and $r_{\boldsymbol{G}} \ge K_l, ~\forall l \in \Phi$, we can only utilize the $1$st OFDM symbol in each coherence block for localization. However, in practical case, we have no prior knowledge of the target number in the system and thus we usually should utilize all the $Q$ OFDM symbols in each coherence block to estimate $K_l$.
%\end{remark}


%To summarize, via a proper design of IRS reflecting patterns over time as shown in \eqref{equ:phi_equ_t}, and combing the temporal domain signals as shown in \eqref{equ:est_channel_reduce}, we create a virtual system \eqref{equ:estimated_channel_deficient_noisy} which is the same as the conventional multi-antenna localization system. 
In the following, we show how to achieve \textbf{Goals} 1-3 listed in Section \ref{sec:phase_II} via applying the MUSIC algorithm on the system described in \eqref{equ:estimated_channel_deficient_noisy}.

%(you may think whether we should put figure 4 here to emphasize our idea)

%Before the usage of the specific algorithm to estimate the target number and extract the range and AOA information for the range cluster $l$, there still remains one problem that how the BS can know the relationship between $r_{\boldsymbol{G}}$ and $K_l$ for $l \in \Phi$ since $K_l$ is unknown in prior and usually different for different $l \in \Phi$. In the below, we can show that this problem can be easily solved along with the target number estimation process for each range cluster $l \in \Phi$.
%Without loss of generality, we only use the $1$st OFDM symbol in each coherence block to calculate the sampled covariance matrix as $\boldsymbol{\tilde{R}}_{l}^{(1)} = \frac{1}{T}\sum_{t=1}^{T} \boldsymbol{\tilde{h}}^{(1)}_{l,t} \left(\boldsymbol{\tilde{h}}^{(1)}_{l,t}\right)^H$.
%\begin{subequations}
%\begin{align}\label{equ:est_cov_mat_h}
%    \boldsymbol{\tilde{R}}_{l}^{(1)} &= \frac{1}{T}\sum_{t=1}^{T} \boldsymbol{\tilde{h}}^{(1)}_{l,t} \left(\boldsymbol{\tilde{h}}^{(1)}_{l,t}\right)^H.
%    %&= \boldsymbol{\Psi}^{(1)}(\boldsymbol{\bar{d}}_{l},\boldsymbol{\theta}_{l}) \boldsymbol{V}_{l} \left(\boldsymbol{\Psi}^{(1)}(\boldsymbol{\bar{d}}_{l},\boldsymbol{\theta}_{l})\right)^{H} + \sigma_{\boldsymbol{\tilde{z}}}^2 \boldsymbol{I}, \label{equ:decom_est_cov_h} \\
%%    &\overset{(a)}{=} \boldsymbol{U}_l^{(1)} \boldsymbol{\Lambda}_{l}^{(1)} \left( \boldsymbol{U}_l^{(1)} \right)^{H}, \forall l \in \Phi,
%\end{align}
%\end{subequations}
%where $\boldsymbol{V}_{l} = \mathbb{E}\left\{\boldsymbol{v}_{l,t}\boldsymbol{v}_{l,t}^H\right\}$, $(a)$ is obtained by the  eigenvalue decomposition (EVD) on $\mathbb{E}\left\{ \boldsymbol{\tilde{h}}^{(1)}_{l,t} \left(\boldsymbol{\tilde{h}}^{(1)}_{l,t}\right)^H \right\}$ where $\boldsymbol{U}_l^{(1)} = [\boldsymbol{u}_{l,1}^{(1)},\dots,\boldsymbol{u}_{l,M_{\rm B}}^{(1)}]$ is the eigenvector matrix and $\boldsymbol{\Lambda}_{l}^{(1)} = \text{diag}\left\{\lambda_{l,1}^{(1)},\dots,\lambda_{l,M_{\rm B}}^{(1)} \right\}$ is the eigenvalue matrix with $\lambda_{l,1}^{(1)} \ge \lambda_{l,2}^{(1)} \ge \dots \ge \lambda_{l,M_{\rm B}}^{(1)}$.
First, we consider the target number estimation problem of \textbf{Goal} $1$, which is referred as the standard model selection problem \cite{Akaike_1974_TAC}.
Then, we propose to utilize the commonly adopted Akaike information criterion (AIC) criterion \cite{Akaike_1974_TAC} to solve this problem. In detail, we first calculate the sample covariance matrix of the virtual channel in \eqref{equ:estimated_channel_deficient_noisy} as
\begin{align}\label{equ:sampled_cov_vs_h}
    \boldsymbol{\tilde{R}}_{l} &= \frac{1}{V} \sum_{t=1}^{V} \boldsymbol{\tilde{h}}_{l,t} \boldsymbol{\tilde{h}}_{l,t}^H, ~\forall l \in \Phi.
\end{align}
By performing EVD, we can derive $\boldsymbol{\tilde{R}}_{l} = \boldsymbol{U}_l \boldsymbol{\it \Lambda}_{l} \left( \boldsymbol{U}_l \right)^{H}$, where $\boldsymbol{U}_l = [\boldsymbol{u}_{l,1},\dots,\boldsymbol{u}_{l,Q_0 M_{\rm B}}]$ is the eigenvector matrix and $\boldsymbol{\it \Lambda}_{l} = \text{diag}([\lambda_{l,1},\dots,\lambda_{l,Q_0 M_{\rm B}}] )$ is the eigenvalue matrix with $\lambda_{l,1} \ge \lambda_{l,2} \ge \dots \ge \lambda_{l,Q_0 M_{\rm B}}$. Then the estimated target number $\hat{K}_l$ based on the AIC criterion is given by
\begin{small}
\begin{align}\label{eq:est k}
    \hat{K}_l = &\arg\max_{K_l} ~\log\left(\frac{\prod_{i=K_l+1}^{Q_0 M_\text{B}}\lambda_{l,i}^{\frac{1}{M_\text{B}-K_l}}}{\frac{1}{M_\text{B}-K_l}\sum_{j=K_l+1}^{Q_0 M_\text{B}}\lambda_{l,j}}\right)^{Q_0 M_\text{B}-K_l} \notag \\
    & \qquad\qquad - 2K_l(Q_0 M_\text{B} - K_l),\quad \forall l \in \Phi.
\end{align}
\end{small}Note that $\hat{K}_l$ can be easily obtained via exhaustive search.



%Assuming the number of coherence blocks is large enough, we can have
%\begin{subequations}
%\begin{align}
%    \boldsymbol{\tilde{R}}_{l} &= \frac{1}{T} \sum_{t=1}^{T} \boldsymbol{\tilde{h}}_{l,t} \left(\boldsymbol{\tilde{h}}_{l,t}\right)^H \\
%    &\approx \mathbb{E}\left\{ \boldsymbol{\tilde{h}}_{l,t} \left(\boldsymbol{\tilde{h}}_{l,t}\right)^H \right\} \\
%    %&= \boldsymbol{\Psi}^{(1)}(\boldsymbol{\bar{d}}_{l},\boldsymbol{\theta}_{l}) \mathbb{E}\left\{\boldsymbol{v}_{l,t}\boldsymbol{v}_{l,t}^H\right\} \left(\boldsymbol{\Psi}^{(1)}(\boldsymbol{\bar{d}}_{l},\boldsymbol{\theta}_{l})\right)^{H} + \mathbb{E}\left\{ \boldsymbol{\tilde{z}}_{l,t}^{(1)} \left( \boldsymbol{\tilde{z}}_{l,t}^{(1)} \right)^H \right\} \notag \\
%    &= \mathbb{E}\left\{ \boldsymbol{\bar{h}}^{(1)}_{l,t} \left(\boldsymbol{\bar{h}}^{(1)}_{l,t}\right)^H\right\} + \sigma_{\boldsymbol{\tilde{z}}}^2 \boldsymbol{I} \\
%    &= \boldsymbol{\Psi}^{(1)}(\boldsymbol{\bar{d}}_{l},\boldsymbol{\theta}_{l}) \boldsymbol{V}_{l} \left(\boldsymbol{\Psi}^{(1)}(\boldsymbol{\bar{d}}_{l},\boldsymbol{\theta}_{l})\right)^{H} + \sigma_{\boldsymbol{\tilde{z}}}^2 \boldsymbol{I}, \label{equ:decom_est_cov_h} \\
%    &\overset{(a)}{=} \boldsymbol{U}_l^{(1)} \boldsymbol{\Lambda}_{l}^{(1)} \left( \boldsymbol{U}_l^{(1)} \right)^{H}, \forall l \in \Phi,
%\end{align}
%\end{subequations}
%where $\boldsymbol{V}_{l} = \mathbb{E}\left\{\boldsymbol{v}_{l,t}\boldsymbol{v}_{l,t}^H\right\}$, $(a)$ is obtained by the  eigenvalue decomposition (EVD) on $\mathbb{E}\left\{ \boldsymbol{\tilde{h}}^{(1)}_{l,t} \left(\boldsymbol{\tilde{h}}^{(1)}_{l,t}\right)^H \right\}$ where $\boldsymbol{U}_l^{(1)} = [\boldsymbol{u}_{l,1}^{(1)},\dots,\boldsymbol{u}_{l,M_{\rm B}}^{(1)}]$ is the eigenvector matrix and $\boldsymbol{\Lambda}_{l}^{(1)} = \text{diag}\left\{\lambda_{l,1}^{(1)},\dots,\lambda_{l,M_{\rm B}}^{(1)} \right\}$ is the eigenvalue matrix with $\lambda_{l,1}^{(1)} \ge \lambda_{l,2}^{(1)} \ge \dots \ge \lambda_{l,M_{\rm B}}^{(1)}$. Assuming $r_{\boldsymbol{G}} \ge K_l$, the target number $K_l$ can be easily obtained by searching the index of smallest diagonal element of $\boldsymbol{\Lambda}^{(1)}_{l} - \lambda_{l,M_{\rm B}}^{(1)}\boldsymbol{I}$ being larger than $0$.
%In practical case where $\boldsymbol{\tilde{R}}_{l}^{(1)} \ne \mathbb{E}\left\{ \boldsymbol{\tilde{h}}^{(1)}_{l,t} \left(\boldsymbol{\tilde{h}}^{(1)}_{l,t}\right)^H \right\}$, we can perform EVD on $\boldsymbol{\tilde{R}}_{l}^{(1)}$ as $\boldsymbol{\tilde{R}}_{l}^{(1)} = \boldsymbol{\tilde{U}}_l^{(1)} \boldsymbol{\tilde{\Lambda}}_{l}^{(1)} \left( \boldsymbol{\tilde{U}}_l^{(1)} \right)^{H}$ with $\boldsymbol{\tilde{\Lambda}}_{l}^{(1)} = \text{diag}\left\{\tilde{\lambda}_{l,1}^{(1)},\dots,\tilde{\lambda}_{l,M_{\rm B}}^{(1)} \right\}$ and $\tilde{\lambda}_{l,1}^{(1)} \ge \tilde{\lambda}_{l,2}^{(1)} \ge \dots \ge \tilde{\lambda}_{l,M_{\rm B}}^{(1)}$. Then, we can estimate the number of targets $K_l$ in the range $l \in \Phi$ by utilizing the Akaike information criterion (AIC) approach \cite{aic_music}:
%\begin{align}\label{eq:est k}
%    \hat{K}_l = &\arg\max_{K_l} ~\log\left(\frac{\prod_{i=K_l+1}^{M_\text{B}}\tilde{\lambda}_{l,i}^{1/(M_\text{B}-K_l)}}{1/(M_\text{B}-K_l)\sum_{j=K_l+1}^{M_\text{B}}\tilde{\lambda}_{l,j}}\right)^{M_\text{B}-K_l} \notag \\
%    & \qquad\qquad - K_l(2M_\text{B} - K_l),\quad \forall l \in \Phi.
%\end{align}
%Note that $\hat{K}_l$ can be obtained via the exhaustive search method.
%On the other hand, it is noted that the rank of the first term in the right hand side of \eqref{equ:decom_est_cov_h} is equal to $r_{\boldsymbol{G}}$ if $r_{\boldsymbol{G}} < K_l$ and is equal to $K_l$ otherwise. Thus, by comparing $\hat{K}_l$ and $r_{\boldsymbol{G}}$, we can easily obtain that $r_{\boldsymbol{G}} > K_l$ if $\hat{K}_l < r_{\boldsymbol{G}}$ and $r_{\boldsymbol{G}} \le K_l$ if $\hat{K}_l = r_{\boldsymbol{G}}$. The above indicates that the target number estimation results enable us to identify rank-sufficient case and rank-deficient case for each range cluster $l \in \Phi$.
%However, the target estimation result given by \eqref{eq:est k} in the rank-deficient case is not accurate and we will also illustrate how to estimate $K_l$ in the rank-deficient case along with the specific AOA and range information extraction scheme design.

%Therefore, if the signals in range cluster $l$ belong to the rank-sufficient case, we will utilize the developed MUSIC-based method in Section \ref{sec:rank_sufficient} to extract the AOA and range information with the estimated target number $\hat{K}_l$. While if the signals in range cluster $l$ belong to the rank-deficient case, an innovative method based on the MUSIC technique will be proposed in Section \ref{sec:rank_deficient} to realize all the listed three goals.


%\subsection{Rank-Sufficient Case}\label{sec:rank_sufficient}


%In the subsection, we consider the AOA and range information estimation scheme design for the rank-sufficient case where $r_{\boldsymbol{G}} \ge K_l$. Here, we also only utilize the $1$st OFDM symbol in each coherence block to estimate the AOA and range information of the targets in the range cluster $l \in \Phi$, which can save the computational cost on the AOA and range information extraction. In the rank-sufficient case, {\bf{Goal}} $1$ can be easily realized based on \eqref{eq:est k}, and then we consider to realize {\bf{Goal $2$}} and {\bf{Goal $3$}} jointly by leveraging the MUSIC technique.

%To guarantee the successful usage of the MUSIC algorithm, the number of coherence blocks where the received signals are used for target detection should be larger than the number $K_l = |\Omega_l|$ of targets in each range cluster $l \in \Phi$, i.e., $T > K_l$.
%In this section, we just consider the rank-sufficient case to introduce the design of the MUSIC algorithm for target detection, where the rank of $\boldsymbol{G}$ is assumed to be no smaller than the number $K_l = |\Omega_l|$ of targets in each range cluster $l \in \Phi$, i.e., $\text{rank}(\boldsymbol{G}) \ge K_l$. In contrast, the rank-deficient case where the rank of $\boldsymbol{G}$ is smaller than or equal to the number of targets in each range cluster $l \in \Phi$, i.e., $\text{rank}(\boldsymbol{G}) < K_l$ will be discussed in the next section.
%%Moreover, we assume that both the rank of $\boldsymbol{G}$ and the number of coherence blocks are larger than the number of targets in each range cluster $l \in \Phi$. 
%Without loss of generality, the IRS reflecting pattern at all the OFDM symbol durations in each coherence block are also set to be identical and thus the IRS reflecting pattern during the $T$ coherence blocks keeps unchanged, i.e,
%\begin{align}\label{equ:IRS_fp_id}
%    \boldsymbol{\phi}_t^{(q)} = \boldsymbol{\phi}^{(q)} = \boldsymbol{\bar{\phi}},\quad \forall t, q,
%\end{align}
%and thus we can simplify the effective steering vector \eqref{eq:new array} as
%\begin{align}\label{equ:array_sufficient}
%    \boldsymbol{\psi}^{(q)}(\bar{d}_k,\theta_k) &= \boldsymbol{\psi}(\bar{d}_k,\theta_k) \notag \\
%    &= \boldsymbol{G}\text{diag}(\boldsymbol{\bar{\phi}})\boldsymbol{a}_\text{I}(\bar{d}_k,\theta_k), \quad\forall q,t.
%\end{align}
%%Then we apply the MUSIC algorithm \cite{music} on \eqref{equ:est_channel_reduce} to achieve the three goals listed in the above. 
%Then, under the MUSIC framework, the first step is to calculate the sample covariance matrix of $\tilde{\boldsymbol{h}}_{l,t}^{(q)}$, $\forall l \in \Phi$, as the estimation of the true covariance matrix. In particular, the sample covariance matrix of $\tilde{\boldsymbol{h}}_{l,t}$ over $T$ coherence blocks is given as
%\begin{align}\label{eq:est cov}
%    \boldsymbol{\tilde{R}}_l = \frac{1}{TQ} \sum_{t=1}^{T}\sum_{q=1}^{Q} \boldsymbol{\tilde{h}}_{l,t}^{(q)}(\boldsymbol{\tilde{h}}_{l,t}^{(q)})^\text{H}\in \mathbb{C}^{M_\text{B}\times M_\text{B}}, ~\forall l \in \Phi.
%\end{align}
%Then, define the eigenvalue decomposition (EVD) of $\boldsymbol{\tilde{R}}_l$ as $\boldsymbol{\tilde{R}}_l=\boldsymbol{V}_l\boldsymbol{\Lambda}_l\boldsymbol{V}_l^\text{H}$, where $\boldsymbol{\Lambda}_l={\rm diag}([\lambda_{l,1},\cdots,\lambda_{l,M_\text{B}}]^T)$ whose diagonal elements are the eigenvalues of $\boldsymbol{\tilde{R}}_l$, and $\boldsymbol{V}_l=[\boldsymbol{v}_{l,1},\cdots, \boldsymbol{v}_{l,M_\text{B}}]$ consists of the corresponding eigenvectors. Without loss of generality, we assume that $\lambda_{l,1}\geq \lambda_{l,2}\geq \cdots \geq \lambda_{l,M_\text{B}}$. 
%Then, we estimate the number of targets $K_l$ in range cluster $l \in \Phi$, adopting the Akaike information criterion (AIC) approach \cite{aic_music}:
%\begin{align}\label{eq:est k}
%    \hat{K}_l = &\arg\max_{K_l} ~\log\left(\frac{\prod_{i=K_l+1}^{M_\text{B}}\lambda_{l,i}^{1/(M_\text{B}-K_l)}}{1/(M_\text{B}-K_l)\sum_{j=K_l+1}^{M_\text{B}}\lambda_{l,j}}\right)^{M_\text{B}-K_l} \notag \\
%    & \qquad\qquad - K_l(2M_\text{B} - K_l),\quad \forall l \in \Phi.
%\end{align}Note that $\hat{K}_l$ can be obtained via the exhaustive search method.
%Then, we jointly consider {\bf{Goal $2$}} and {\bf{Goal $3$}}. 

Then we consider to realize {\bf{Goal $2$}} and {\bf{Goal $3$}} jointly.
Define $\boldsymbol{\bar{U}}_l =[\boldsymbol{u}_{l,\hat{K}_l+1},\cdots,\boldsymbol{u}_{l,Q_0 M_\text{B}}]\in\mathbb{C}^{Q_0 M_\text{B} \times (Q_0 M_\text{B}-\hat{K}_l)}, \forall l \in \Phi$. Note that the key difference between a near-field target and a far-field target lies in the steering vector - the effective distance of target $k$, i.e., $\bar{d}_k$, goes to infinity if $k\in\Omega_l^\text{F}$, and is finite if $k\in\Omega_l^\text{N}$. As a result, we may use this difference to detect whether a target is a far-field target or a near-field target, and estimate its AOA and range information to the IRS. 
Specifically, for near-field targets in range cluster $l \in \Phi$, the MUSIC algorithm require us to calculate the following 2D spectrum
\begin{small}
\begin{align}\label{eq:spectrum 2D}
     P_l^{\text{N}}(d,\theta) =&~\frac{\boldsymbol{\breve{\psi}}(d,\theta)^H\boldsymbol{\breve{\psi}}(d,\theta)}{\boldsymbol{\breve{\psi}}(d,\theta)^H\boldsymbol{\bar{U}}_l\boldsymbol{\bar{U}}_l^H\boldsymbol{\breve{\psi}}(d,\theta)}, \theta \in [0,\pi),d \in (0,\infty), %\notag \\
     %&\quad\quad\quad\quad \forall \theta \in [0,\pi),d \in (0,\infty),
\end{align}
\end{small}where $\boldsymbol{\breve{\psi}}(d,\theta)$ is similarly defined as each column vector in \eqref{equ:vir_steering}.
Define $d_{\rm max}=L c_0/N\Delta f - d^{\rm IR}$ as the maximum distance that can be detected in the $L$-tap environment. Then, given the step size $\Delta d$ for searching $d$ and step size $\Delta {\theta}$ for searching $\theta$, define 
\begin{small}
\begin{align}\label{equ:R_dtheta}
  \mathcal{R}^{\rm N}=\bigg\{(d,\theta)| &d = \zeta\Delta{d}, \zeta = 1,\dots,\ceil*{\frac{d_{\rm max}}{\Delta d}}, \notag \\
                           &\theta = \mu\Delta{\theta}, \mu = 1,\dots,\ceil*{\frac{\pi}{\Delta {\theta}}} \bigg\} \cap \mathcal{L}^{\rm N},
\end{align}
\end{small}as the set consisting of all discrete grids in the search region of $d$ and $\theta$, where $\mathcal{L}^{\rm N}$ is the near-field region for the IRS and can be derived in advance. Over $\mathcal{R}^{\rm N}$, we can perform a 2D exhaustive search to find the peaks of $P_l^{\rm N}(d,\theta), \forall l\in \Phi$. Define $\Xi_l^{\rm N}$ as the set of all pairs of $(d,\theta)$ that lead to peaks of $P_l^{\rm N}(d,\theta)$. Then, the pairs of distance and AOA of all the near-field targets in range cluster $l \in \Phi$ are contained in $\Xi_l^{\rm N}$.

%Specifically, for far-field targets in range cluster $l \in \Phi$, we may use the following spectrum
Next, we consider the detection of far-field targets in the range cluster $l \in \Phi$. Define the 1D spectrum as
\begin{small}
\begin{align}\label{eq:spectrum 1D}
     P_l^{\text{F}}(\theta) %=&~ P_l^{\text{F}}(d\to\infty,\theta) \notag \\
     =&~ \frac{\boldsymbol{\breve{\psi}}(d\to\infty,\theta)^H\boldsymbol{\breve{\psi}}(d\to\infty,\theta)}{\boldsymbol{\breve{\psi}}(d\to\infty,\theta)^H \bar{\boldsymbol{U}}_l\boldsymbol{\bar{U}}_l^H\boldsymbol{\breve{\psi}}(d\to\infty,\theta)}, \theta\in [0,\pi). %\notag \\
     %&\qquad\qquad\qquad\qquad\qquad\forall \theta\in [0,\pi).
\end{align}
\end{small}Since the effective distance of far-field targets are considered to be infinity, we only perform a 1D angle search over $P_l^{\text{F}}(\theta)$ to find its peaks. 
The set of search grids for far-field targets can be given as
\begin{align}\label{equ:R_theta}
    \mathcal{R}^{\rm F} = \left\{ \theta | \theta = \xi\Delta{\theta}, \xi = 1,\dots,\ceil*{\frac{\pi}{\Delta {\theta}}} \right\}.
\end{align}
Define $\Xi_l^\text{F}$ as the AOA set that leads to peaks of $P_l^{\text{F}}(\theta), \forall l \in \Phi$.

%Next, we consider near-field targets in range cluster $l \in \Phi$. Define the spectrum of the near-field targets in range cluster $l \in \Phi$ as 
%\begin{align}\label{eq:spectrum 2D}
%     P_l^{\text{N}}(d,\theta)=&\frac{\boldsymbol{\bar{\psi}}(d,\theta)^\text{H}\boldsymbol{\bar{\psi}}(d,\theta)}{\boldsymbol{\bar{\psi}}(d,\theta)^\text{H}\bar{\boldsymbol{V}}_l\boldsymbol{\bar{V}}_l^\text{H}\boldsymbol{\bar{\psi}}(d,\theta)},~\forall \theta \in [0,\pi),d \in (0,\infty).
%\end{align}
%Define $d_{\rm max}=L c_0/N\Delta f$ as the maximum distance that can be detected in the $L$-tap environment. Then, given the step size $\Delta d$ for searching $d$ and step size $\Delta {\theta}$ for searching $\theta$, define 
%\begin{align}\label{equ:R_dtheta}
%  \mathcal{R}^{\rm N}=\bigg\{(d,\theta)| &d = \zeta\Delta_d, \zeta = 1,\dots,\ceil*{\frac{d_{\rm max}}{\Delta d}}, \notag \\
%                           &\theta = \xi\Delta_{\theta}, \xi = 1,\dots,\ceil*{\frac{\pi}{\Delta {\theta}}} \bigg\},
%\end{align}
%as the set consisting of all the discrete grids in the search regime of $d$ and $\theta$. Over $\mathcal{R}$, we can perform a 2D exhaustive search to find the peaks of $P_l^{\rm N}(d,\theta), \forall l\in \Phi$. 

However, to realize precise estimation on the AOAs and/or ranges of targets, the selected step sizes $\Delta d$ and $\Delta {\theta}$ is usually much smaller than $d_{\rm max}$ and $\pi$, respectively. So that the number of seach grids in $\mathcal{R}^{\rm N}$ is quite large and it is of prohibitive complexity to calculate the 2D spectrum, as required by the MUSIC algorithm. In fact, we can leverage the prior information of the range cluster to help reduce the search region, leading to much lower searching complexity. Such method is named as the prior information-assisted MUSIC algorithm. Specifically, if targets considered to be in the range cluster $l$, their positions $(x_k, y_k)$'s must satisfy the following condition
\begin{align} \label{equ:pri_inf_cluster_l}
    \frac{(l-1)c_0}{N \Delta f} \le d^{\rm UTI}(x_k,y_k) + d^{\rm IB}  < & \frac{lc_0}{N \Delta f},  ~k \in \Omega_{l}.
\end{align} 
%The graphical diagram of a toy example of the target distribution in the range cluster $l$ is shown in Fig. \ref{fig:demo_coexistence}.

Define $\mathcal{R}^{\rm N}_{l}$ as the set of search grids for near-field targets in the range cluster $l$ via the 2D-MUSIC algorithm. Each search grid $(d_{\rm pri},\theta_{\rm pri}) \in \mathcal{R}^{\rm N}_{l}$ should also satisfy the above condition (\ref{equ:pri_inf_cluster_l}) by replacing $(x_{k}, y_{k})$ with $(x_{\rm pri}(d_{\rm pri},\theta_{\rm pri}), y_{\rm pri}(d_{\rm pri},\theta_{\rm pri}))$ defined as
\begin{align}
    x_{\rm pri}(d_{\rm pri},\theta_{\rm pri}) &= x_{\rm I} + d_{\rm pri} \cos\theta_{\rm pri}, \\
    y_{\rm pri}(d_{\rm pri},\theta_{\rm pri}) &= y_{\rm I} + d_{\rm pri} \sin\theta_{\rm pri}.
\end{align}
%On the other hand, the near-field region of the IRS also provides prior information for the target positions that the distances between the targets and the IRS should be less than the Rayleigh distance $d_{\rm R}$. Thus, the search grids should also satisfy the condition
%\begin{align}\label{equ:pri_NF}
%    d_{\rm pri} \le d_{\rm R}.
%\end{align}
Therefore, the set of search grids for near-field targets in the cluster range $l$ can be presented as
\begin{align}\label{equ:R_N_l}
    \mathcal{R}^{\rm N}_{l} = &~\{ (d,\theta) | (d,\theta) \text{ satisfies conditions (\ref{equ:pri_inf_cluster_l})}, \notag \\
    &~\forall (d,\theta) \in \mathcal{R}^{\rm N} \}, \quad \forall l \in \Phi.
\end{align}

Similarly, for the detection of far-field targets, the number of search grids for the 1D spectrum can also be reduced to save the searching cost with the prior information. The possible positions $(x_{\rm F}, y_{\rm F})$ of far-field targets should satisfy the condition (\ref{equ:pri_inf_cluster_l}) and
\begin{align}
    (d(x_{\rm F}, y_{\rm F}),\theta(x_{\rm F}, y_{\rm F})) \in \mathcal{L}^{\rm F},
\end{align}
where $\mathcal{L}^{\rm F} = \mathcal{L} \backslash \mathcal{L}^{\rm N}$ is the far-field region with $\mathcal{L}$ being the whole position region for targets in the considered system.
Then, from $\mathcal{L}^{\rm F}$, we can obtain the AOA region of the far-field targets in the range cluster $l$ as $\mathcal{A}_l$. Finally, the set of search grids for far-field targets via the MUSIC algorithm can be given by
\begin{align}\label{equ:R_F_l}
    \mathcal{R}^{\rm F}_{l} = \mathcal{A}_l \cap \mathcal{R}^{F}, \quad \forall l \in \Phi.
\end{align}

\begin{figure}[t]
   \centering
    \includegraphics[width=.45\textwidth]{Ave_num_grids.eps}
    \vspace{-0.5cm}
    \caption{Comparison of the average number of search grids for the near-field and far-field targets between the proposed method and the conventional method.}\label{fig:comparison_search_grids}
    \vspace{-0.5cm}
\end{figure}

\begin{remark}
To validate the efficiency by utilizing the prior information, an example is provided in this section. In this example, we assume that there is one user located at $(0,0)$ in meter, one IRS located at $(20,20)$ in meter, and one BS located at $(20,15)$ in meter. Then, we randomly generate the location of one target such that it is located within a quarter circle with the IRS as the center and $d^\text{max}=80$ (meter) as the radius. 
Here, we consider that the near-field region is defined as $\mathcal{L}^{\rm N} = \{(d,\theta)| d \le d_{\rm R} \}$ with $d_{\rm R} = 30$ (meter).
%Here, we consider to use the Rayleigh distance \cite{Kris_2017_APM} to distinguish the near-field and far-field regions and thus $\mathcal{L}^{\rm N} = \{(d,\theta)| d \le d_{\rm R} \}$ with $d_{\rm R} = 30$ (meter) denoting the Rayleigh distance. 
Moreover, we assume the step size of angular search is $0.1$ in degree, i.e., $\Delta \theta = \frac{\pi}{1800}$, and the step size of range search is $0.1$ in meter, i.e., $\Delta d = 0.1$ (meter). As a result, under this setup, the total number of search grids in $\mathcal{R}^{\rm N}$, known as the cardinality of $\mathcal{R}^{\rm N}$, is $|\mathcal{R}^{\rm N}|=900\times 300=2.7 \times 10 ^5$. Last, we adopt the average number of search grids as the performance metric. Fig. \ref{fig:comparison_search_grids} plots the average number of search grids for near-field targets and far-field targets under different bandwidth ranging from $50$ MHz to $400$ MHz. In particular, the utilization of the prior information can significantly reduce the number of search grids for the near-field targets. For example, when the bandwidth is $400$ MHz, the number of the search grids in $\mathcal{R}^{\rm N}$ is $2.7 \times 10^5$ grids on average, while our proposed approach only requires $3858$ grids on average. This result indicates that more than $98$\% search grids can be saved in our proposed method, compared to the conventional approach without considering the prior information. However, for far-field targets, the average number of search grids is only slightly decreased by about $8.8$\% when the bandwidth is $400$ MHz, since the far-field region is much larger than the near-field region. Therefore, utilizing the prior information can mainly save the searching cost for the near-field target detection.
\end{remark}

\begin{algorithm}[t]
	\caption{Prior Information-Assisted MUSIC Algorithm for AOA and Range Estimation in the Range Cluster $l$}\label{alg:pri_music}
	    {\bf Input}: $ \{\boldsymbol{\tilde{h}}_{l,t}^{(q)}\}_{q=1,t=1}^{Q_0,V}$ given in \eqref{eq:estimated_channel cluster l};\\
	    {\bf Initialization (Offline)}: Obtain $\mathcal{R}_l^{\rm N}$'s and $\mathcal{R}_l^{\rm F}$ given in \eqref{equ:R_N_l} and \eqref{equ:R_F_l};
        \begin{enumerate}
        \item [1.] Calculate the sampled covariance matrix $\boldsymbol{\tilde{R}}_l$ via \eqref{equ:sampled_cov_vs_h}, and perform EVD on $\boldsymbol{\tilde{R}}_l$ to obtain eigenvalues $\{\lambda_{l,m}\}_{m=1}^{Q_0M_{\rm B}}$ and the corresponding eigenvectors $\{\boldsymbol{v}_{l,m}\}_{m=1}^{Q_0M_{\rm B}}$; \Comment{Step 1} 
        \item [2.] Estimate the number $\hat{K}_l$ of targets in range cluster $l$ by \eqref{eq:est k}; \Comment{Step 2}
        \item [3.] Search the peaks of the $2$D spectrum \eqref{eq:spectrum 2D} over $\mathcal{R}_{l}^{\rm N}$ and the peaks in $1$D spectrum \eqref{eq:spectrum 1D} over $\mathcal{R}_l^{\rm F}$; \Comment{Step 3}
        \item [4.] Obtain the two sets $\tilde{\mathcal{P}}_{l}^{\rm F}$ and $\tilde{\mathcal{P}}_{l}^{\rm N}$ that contain the largest $\hat{K}_l$ peaks from the $1$D and $2$D spectrums obtained in Step $3$, respectively. Then reserve the peaks larger than the threshold $\varsigma^{\rm F}$ for the set $\tilde{\mathcal{P}}_{l}^{\rm F}$ and reserve the peaks larger than the threshold $\varsigma^{\rm N}$ for the set $\tilde{\mathcal{P}}_{l}^{\rm N}$; \Comment{Step 4}
     \end{enumerate}
        {\bf{Output}}: The AOA information $\hat{\Xi}_l^{\rm F}$ of far-field targets; the AOA and range information $\hat{\Xi}_l^{\rm N}$ of near-field targets.
\end{algorithm}

Until now, we have detected some peaks in $P_l^{\rm F}(\theta)$ and $P_l^{\rm N}(d,\theta)$ for each $l \in \Phi$. However, the number of peaks in $\left\{P_l^{\rm F}(\theta), \forall \theta \in \Xi_l^{\rm F}\right\}$ and $\left\{P_l^{\rm N}(d,\theta), \forall (d,\theta) \in \Xi_l^{\rm N} \right\}$ is usually much larger than $\hat{K}_l$ because some ghost targets may be detected due to noise. To tackle this problem, we propose to determine the final peak sets $\mathcal{P}_l^{\rm F}$ and $\mathcal{P}_l^{\rm N}$ of each range cluster $l \in \Phi$ by thresholding method, which selects all the peaks being larger than the pre-determined thresholds. Note that totally $\hat{K}_l$ targets are considered in each range cluster $l \in \Phi$, meaning that either the number of far-field targets or near-field targets should be no larger than $\hat{K}_l$. Therefore, we first select the largest $\hat{K}_l$ peaks in $P_l^{\rm F}(\theta)$ and the largest $\hat{K}_l$ peaks in $P_l^{\rm N}(d,\theta)$ for each range cluster $l \in \Phi$ to contribute the set $\tilde{\mathcal{P}}^{\rm F}_l$ and $\tilde{\mathcal{P}}^{\rm N}_l$, respectively. Then we utilize two pre-selected thresholds $\varsigma^{\rm F}$ and $\varsigma^{\rm N}$ to determine far-field targets and near-field targets in the range cluster $l \in \Phi$, respectively. Finally, the range and/or AOA information sets for far-field and near-field targets are, respectively, given by
\begin{align}
  \hat{\Xi}_l^{\rm F} &= \{ \theta | P_l^{\rm F}(\theta) > \varsigma^{\rm F}  \text{ and } P_l^{\rm F}(\theta) \in \tilde{\mathcal{P}}_l^{\rm F} \}, \forall l \in \Phi, \label{equ:F_AOA} \\
  \hat{\Xi}_l^{\rm N} &=\{(d,\theta)| P_l^{\rm N}(d,\theta) > \varsigma^{\rm N}  \text{ and } P_l^{\rm N}(d,\theta) \in \tilde{\mathcal{P}}_l^{\rm N} \}, \forall l \in \Phi. \label{equ:N_dAOA}
\end{align}


With the above, the prior information-assisted MUSIC algorithm can be outlined in Algorithm \ref{alg:pri_music}.
After finishing the task in Phase II, target locations will be estimated in Phase III with the estimated AOA and range information of targets.

To summarize, in Phase II, given each range cluster $l \in \Phi$ where exists targets, we 1. estimate the target number in this range cluster as $\hat{K}_l$ given in (\ref{eq:est k}); 2. detect which targets are far-field targets and which targets are near-field targets; 3. estimate the AOAs from the far-field targets to the IRS in the set $\hat{\Xi}_l^{\rm F}$ defined in (\ref{equ:F_AOA}) and the distances and AOAs from the near-field targets to the IRS in the set $\hat{\Xi}_l^{\rm N}$ defined in (\ref{equ:N_dAOA}).

%Until now, we have detected some peaks in $P_l^{\rm F}(\theta)$ and $P_l^{\rm N}(d,\theta)$ for each $l \in \Phi$. For convenience, define the set that consists of all the peaks in $P_l^{\rm F}(\theta)$ and $P_l^{\rm N}(d,\theta)$ for each $l \in \Phi$ as $\mathcal{P}_l=\left\{P_l^{\rm F}(\theta), \forall \theta \in \Xi_l^{\rm F}\right\} \cup \left\{P_l^{\rm N}(d,\theta), \forall (d,\theta) \in \Xi_l^{\rm N} \right\}$.
%Note that $\hat{K}_l$ targets are detected in range cluster $l \in \Phi$, where $\hat{K}_l$ is given in (\ref{eq:est k}). However, the number of peaks in the set $\mathcal{P}_l$ is usually larger than $\hat{K}_l$ because some ghost targets that do not exist may be detected due to noise. To tackle this problem, we select the $\hat{K}_l$ largest elements in $\mathcal{P}_l$ to construct a new set $\mathcal{P}_l^{\rm max}$ and claim that the peaks in this set are contributed by the $\hat{K}_l$ detected targets, which corresponds to Step 4 in Algorithm \ref{alg:pri_music}. Therefore, given $\Xi_l^{\rm F}$ and $\Xi_l^{\rm N}$ that contain the AOA and/or distance information of all targets (including the true targets and ghost targets), we are only interested in the elements that contribute to the peaks in $\mathcal{P}_l^{\rm max}$. In particular, define the sets that consist of the AOAs from the far-field targets to the IRS and the distances and AOAs from the near-field targets to the IRS respectively as
%\begin{align}
%  \hat{\Xi}_l^{\rm F} &=\{\theta| \theta \in \Xi_l^{\rm F} \text{ and } P_l^{\rm F}(\theta) \in \mathcal{P}_l^{\rm max}\}, \forall l \in \Phi, \label{equ:F_AOA} \\
%  \hat{\Xi}_l^{\rm N} &=\{(d,\theta)| (d,\theta) \in \Xi_l^{\rm N} \text{ and } P_l^{\rm N}(d,\theta) \in \mathcal{P}_l^{\rm max} \}, \forall l \in \Phi. \label{equ:N_dAOA}
%\end{align}
%With the estimated AOA and range information of targets, the location of targets can be estimated in Phase III.
%
%To summarize, in Phase II, given each range cluster $l \in \Phi$ where targets are detected, we 1. estimate the number of targets in this range cluster as $\hat{K}_l$ given in (\ref{eq:est k}); 2. detect which targets are far-field targets and which targets are near-field targets; 3. estimate the AOAs from the far-field targets to the IRS in the set $\hat{\Xi}_l^{\rm F}$ defined in (\ref{equ:F_AOA}) and the distances and AOAs from the near-field targets to the IRS in the set $\hat{\Xi}_l^{\rm N}$ defined in (\ref{equ:N_dAOA}).

%\subsection{Rank-Deficient Case}\label{sec:rank_deficient}


%In the subsection, we shall introduce how to realize all the three goals in Phase II for the rank-deficient case where $r_{\boldsymbol{G}} < K_l$.
%Compared with the rank-sufficient case, it is much harder to realize the three goals listed in Section \ref{sec:phase_II}.
%Specifically, as illustrated in Section \ref{sec:phase_II}, even the \textbf{Goal} $1$ cannot be realized in the rank-deficient case by checking the covariance matrix $\boldsymbol{\tilde{R}}_{l}^{(1)}$ following \eqref{eq:est k} when the number of coherence blocks is large enough. Then, even if we can get the accurate estimation of the target number $K_l$, \textbf{Goal} $2$ and \textbf{Goal} $3$ also cannot be realized, since we cannot identify $K_l$ sources via MUSIC if $\text{rank}\left(\mathbb{E}\left\{ \boldsymbol{\bar{h}}^{(1)}_{l,t} \left(\boldsymbol{\bar{h}}^{(1)}_{l,t}\right)^H \right\}\right) = r_{\boldsymbol{G}} < K_l$.
%To cope with these challenges, we thus provide an innovative method to create a virtual-spatial signal model and then leverage the MUSIC algorithm for achieving the three goals. 
%
%
%%In this case, the proposed method and the reflection coefficient pattern design in the previous section cannot directly applied to estimate the AOA and/or range information of the targets, whose reason can be explained as follows. Here, we assume $\text{rank}(\boldsymbol{G}) = r_{\boldsymbol{G}}$ with $r_{\boldsymbol{G}} \le \min\{ M_{\rm B}, M_{\rm I}, K_{l}-1 \}$. 
%%%In fact, the rank-deficient case is common in the considered system. For example, if the BS is located in the far-field region of the IRS, the rank of the LOS MIMO channel will be equal one.
%%Without loss of generality, we also assume that the first $\text{rank}(\boldsymbol{G})$ rows of $\boldsymbol{G}$ are linearly independent. Therefore, each of remaining ($M_{\rm B} - r_{\boldsymbol{G}}$) rows can be represented as a linear combination of the first $\text{rank}(\boldsymbol{G})$ rows as
%%\begin{align}\label{equ:row_linear}
%%    \boldsymbol{g}_{m_{\rm B}} =&~ \sum_{r=1}^{r_{\boldsymbol{G}}} \mu_{m_{\rm B},r}(a_{\rm U},b_{\rm U},a_{\rm I},b_{\rm I}) \boldsymbol{g}_{r} \notag \\
%%     &\quad\quad\quad\quad m_{\rm B} \in \{r_{\boldsymbol{G}}+1,\dots,M_{\rm B}\}, 
%%\end{align}
%%where $\boldsymbol{g}_{m_{\rm B}}$ is the $m_{\rm B}$th row of the channel $\boldsymbol{G}$.
%%Consequently, the information of these remaining ($M_{\rm B} - r_{\boldsymbol{G}}$) rows (or equivalently, receive antennas) is redundant and does not contribute to the estimation problem we are interested in. In other words, although the BS is equipped with multiple antennas, not all the receive antennas provide additional information on AOA and range of the LOS path from each target to the IRS. This is actually not surprising. Based on the above assumption, in the ideal case when there is no estimation error, i.e., $\boldsymbol{\tilde{z}}_{l,t}^{(q)}=\boldsymbol{0}$, $\forall q,t,l\in\Phi$, at the $q$-th OFDM symbol duration in coherence block $t$, the cascaded channel between the user and the $m_{\rm B}$-th receive antenna as well as that between the user and the first $r_{\rm \boldsymbol{G}}$ receive antennas have the following relation
%%\begin{align}\label{eq:received signal relation_deficient}
%%    \tilde{h}_{l,t,m_{\rm B}}^{(q)} =&~ \sum_{r=1}^{r_{\boldsymbol{G}}} \mu_{m_{\rm B},r}(a_{\rm U},b_{\rm U},a_{\rm I},b_{\rm I}) \tilde{h}_{l,t,r}^{(q)}, \notag \\ 
%%    &\quad \forall q, t, m_{\rm B} \in \{r_{\boldsymbol{G}}+1,\dots,M_{\rm B}\}.
%%\end{align}
%%This implies that given the channel of the path from the user to the first $r_{\boldsymbol{G}}$ receive antennas, that from the user to any other antenna $m_{\rm B} \in \{r_{\boldsymbol{G}}+1,\dots,M_{\rm B}\}$ provides no information about $\theta_{k}$'s and $d_k$'s for $k \in \Omega_{l}$. Instead, it only contains the location information of the IRS to the BS, i.e., $(a_{\rm U},b_{\rm U})$ and $(a_{\rm I},b_{\rm I})$. However, we are interested in estimating the AOAs of the channel from the targets to the IRS for localization.
%
%%In other words, if the LOS channel from the IRS to the BS satisfies $r_{\boldsymbol{G}} \le \min\{ M_{\rm B}, M_{\rm I}, K_{l} \}$, then the differences among the different antennas at the BS only contain the information about the AOA from the IRS. However, we are interested in estimating the AOAs from the targets to the IRS for localization. Then for the general case with $1 \le r_{\boldsymbol{G}} \le \min\{ M_{\rm B}, M_{\rm I}, K_{l} \}$, we can easily obtain the similar conclusions that only the first $r_{\boldsymbol{G}}$ is useful to extract the AOA and/or range information from the targets the IRS.
%
%%For a toy example, if the channel between the IRS and the BS is a far-field channel such that \eqref{equ:far_irs_bs_channel} and $r_{\boldsymbol{G}} = 1$ hold, then in the ideal case when there is no estimation error, i.e., $\boldsymbol{\tilde{z}}_{l,t}^{(q)}=\boldsymbol{0}$, $\forall q,t,l\in\Phi$, at the $q$-th OFDM symbol duration in coherence block $t$, the cascaded channel between the user and the $m_{\rm B}$-th receive antenna as well as that between the user and the first receive antenna have the following relation
%%\begin{align}\label{eq:received signal two antennas far}
%%    \tilde{h}_{l,t,m_{\rm B}}^{(q)} =& \frac{a_{\text{B},m_{\rm B}}(d\to\infty,\phi)}{a_{\text{B},1}(d\to\infty,\phi)}\tilde{h}_{l,t,1}^{(q)}, \notag \\
%%    &\quad \forall q, t, m_{\rm B} \in \{r_{\boldsymbol{G}}+1,\dots,M_{\rm B}\},
%%\end{align}
%%where $a_{\text{B},m_{\text{B}}}(d\to\infty,\varphi)$ is the $m_{\rm B}$-th element of $\boldsymbol{a}_{\text{B}}$. This implies that given the channel of the path from the user to the first receive antenna, that from the user to any other antenna $m_{\rm B} \neq 1$ provides no information about $\theta_{k}$'s and $d_k$'s for $k \in \Omega_{l}$. Instead, it only contains the information of the AOA of the path from the IRS to the BS, i.e., $\phi$. In other words, if the LOS channel from the IRS to the BS is a far-field channel, then the phase differences among the different antennas at the BS only contain the information about the AOA from the IRS. However, we are interested in estimating the AOAs from the targets to the IRS for localization. Then for the general case with $1 \le r_{\boldsymbol{G}} \le \min\{ M_{\rm B}, M_{\rm I}, K_{l} \}$, we can easily obtain the similar conclusions that only the first $r_{\boldsymbol{G}}$ is useful to extract the AOA and/or range information from the targets the IRS.
%
%Considering that only using the $1$st OFDM symbol in each coherence block is not enough to extract the AOA and range information of targets, we thus propose to utilize all the $Q$ OFDM symbols in all the $T$ coherence blocks for the AOA and range information extraction in the rank-deficient case without loss of generality. 
%%If following the previous section by setting the reflection coefficient pattern to be identical in the whole $T$ coherence blocks as \eqref{equ:IRS_fp_id}, we will find that $\text{rank}(\tilde{\boldsymbol{R}}_{l}) \le r_{\boldsymbol{G}} \le K_{l}$. However, the necessary condition of the MUSIC algorithm to identify $K_{l}$ source is that $\text{rank}(\tilde{\boldsymbol{R}}_{l}) \ge K_l + 1$ and thus the AOA and/or range information of targets cannot extracted by leveraging the noise subspace of $\tilde{\boldsymbol{R}}_{l}$ via the MUSIC algorithm by \eqref{eq:spectrum 2D} and \eqref{eq:spectrum 1D}. To cope with this challenge problem in the rank-deficient case, we provide an innovative method to create a virtual multi-dimension channel in the temporal domain for achieving the three goals listed in the previous section. Under the proposed method, we determine the reflection coefficient patterns during different OFDM symbols in each coherence block to be different while to still satisfy the condition \eqref{equ:phi_equ_t}, i.e., 
%%\begin{align}\label{equ:reflection_deficient}
%%    \phi_{t_1}^{(q_1)} \ne \phi_{t_1}^{(q_2)}, \phi_{t_1}^{(q_1)} = \phi_{t_2}^{(q_1)} = \phi^{(q_1)}, ~\forall q_1 \ne q_2, t_1 \ne t_2.
%%\end{align}
%By defining $\boldsymbol{\tilde{h}}_{l,t} = \left[\left(\boldsymbol{\tilde{h}}_{l,t}^{(1)}\right)^T,\dots,\left(\boldsymbol{\tilde{h}}_{l,t}^{(Q)}\right)^T\right]^T$, $\boldsymbol{\bar{h}}_{l,t} = \left[\left(\boldsymbol{\bar{h}}_{l,t}^{(1)}\right)^T,\dots,\left(\boldsymbol{\bar{h}}_{l,t}^{(Q)}\right)^T\right]^T$, and $\boldsymbol{\tilde{z}}_{l,t} = \left[\left(\boldsymbol{\tilde{z}}_{l,t}^{(1)}\right)^T,\dots,\left(\boldsymbol{\tilde{z}}_{l,t}^{(Q)}\right)^T\right]^T$, the estimated channel vector $\boldsymbol{\tilde{h}}_{l,t}$ in each coherence block $t$ can be represented as
%\begin{align}\label{equ:estimated_channel_deficient}
%    \tilde{\boldsymbol{h}}_{l,t} &= \boldsymbol{\bar{h}}_{l,t}  + \tilde{\boldsymbol{z}}_{l,t},
%\end{align} 
%where
%\begin{align}
%    &\boldsymbol{\bar{h}}_{l,t} =\breve{\boldsymbol{\Psi}}(\boldsymbol{\bar{d}}_{l},\boldsymbol{\theta}_{l}) \boldsymbol{\upsilon}_{l,t} \notag \\
%    &~~~~~= \left[\breve{\boldsymbol{\psi}}(\bar{d}_{l,m_1},\theta_{l,m_1}), \dots, \breve{\boldsymbol{\psi}}(\bar{d}_{l,m_{K_l}},\theta_{l,m_{K_l}})\right]\boldsymbol{\upsilon}_{l,t}, \label{equ:estimated_channel_deficient_noiseless} \\
%    %\breve{\boldsymbol{\Psi}}_{l,t} =&~ [\breve{\boldsymbol{\psi}}_{l,t,1}, \dots, \breve{\boldsymbol{\psi}}_{l,t,K_l}], \\  %[\text{diag}(\boldsymbol{\phi}^{(1)}),\dots,\text{diag}(\boldsymbol{\phi}^{(Q)})]^T \sum_{k \in \Omega_{l}} \beta_k\gamma_{k,t}\boldsymbol{a}(\bar{d}_k,\theta_{k})
%    &\breve{\boldsymbol{\psi}}(\bar{d}_{l,m_i},\theta_{l,m_i}) = \begin{bmatrix}
%                              \boldsymbol{G}\text{diag}(\boldsymbol{\phi}^{(1)})\boldsymbol{a}_{\rm I}(\bar{d}_{l,m_i},\theta_{l,m_i}) \\
%                              \vdots \\
%                              \boldsymbol{G}\text{diag}(\boldsymbol{\phi}^{(Q)})\boldsymbol{a}_{\rm I}(\bar{d}_{l,m_i},\theta_{l,m_i}) 
%                            \end{bmatrix}. \label{equ:vir_steering}
%    %\boldsymbol{A}_{{\rm I},l} =&~ [\boldsymbol{a}_{\rm I}(\bar{d}_{l,1},\theta_{l,1}),\dots,\boldsymbol{a}_{\rm I}(\bar{d}_{l,K_l},\theta_{l,K_l})],                        
%\end{align}
%%with $\boldsymbol{\upsilon}_{l,t} = [\upsilon_{t,k_{1}},\dots,\upsilon_{t,k_{K_l}}]^T$. 
%It is seen that \eqref{equ:estimated_channel_deficient} -- \eqref{equ:vir_steering} can be regarded to mathematically describe a virtual-spatial signal model similar to \eqref{eq:estimated_channel cluster l}, where each BS can be considered to be equipped with $QM_{\rm B}$ virtual antennas and $\breve{\boldsymbol{\psi}}(\bar{d}_{l,m_i},\theta_{l,m_i})$ in \eqref{equ:vir_steering} is the effective steering vector in the virtual-spatial signal model for target $m_i \in \Omega_{l}$.
%In particular, the graphic descriptions of the signal model for the rank-sufficient case and the rank-deficient case are given in Fig. \ref{fig:multisignal}.
%Note that via a proper design of the IRS reflection coefficient patterns $\left\{\boldsymbol{\phi}^{(q)}\right\}_{q=1}^{Q}$, the rank of $\breve{\boldsymbol{\Psi}}(\boldsymbol{\bar{d}}_{l},\boldsymbol{\theta}_{l})$ can be equal to $K_l$ and then we can utilize the MUSIC algorithm to extract the AOA and range information of targets in the range cluster $l$ from the sampled covariance matrix $\breve{\boldsymbol{R}}_l = \frac{1}{T} \sum_{t=1}^{T} \tilde{\boldsymbol{h}}_{l,t}\tilde{\boldsymbol{h}}_{l,t}^H, ~\forall l$. The following theorem is given to validate that the rank of $\breve{\boldsymbol{\Psi}}(\boldsymbol{\bar{d}}_{l},\boldsymbol{\theta}_{l})$ can be almost surely equal to $K_l$ under the random reflection coefficient design.
%
%
%%As discussed in Section \ref{subsec:step II}, the key research question is how to create a new virtual signal with dimension $M_\text{V}=Q\hat{M}_{\text{S}} > K_l$, where the signals in the other $M_\text{V}-1$ dimensions all provide new information about $\boldsymbol{\theta}_{l}$'s and $\boldsymbol{\bar{d}}_{l}^{\text{IT}}$'s. In other words, the necessary condition \eqref{eq:music necessary 1} can be satisfied. In the following, we provide an innovative method to create a virtual multi-dimension channel in the temporal domain for achieving the above goal. Under our proposed scheme, the channel from the user to receive antennas set $\hat{\mathcal{M}}_{\text{S}}$ over $Q$ OFDM symbols in each coherence block $t$ forms a new block. Let
%%\begin{align}
%%    \boldsymbol{\breve{h}}_{l,t} =[(\boldsymbol{\tilde{h}}_{l,t}^{(1)})^\text{T},\cdots,(\boldsymbol{\tilde{h}}_{l,t}^{(Q)})^\text{T}]^\text{T}\in\mathbb{C}^{M_\text{V} \times 1},~\forall l,t,
%%\end{align}collect the channels of the paths from the user to targets in range cluster $l$ to the IRS to the receive antennas $\hat{\mathcal{M}}_{\text{S}}$ over $Q$ OFDM symbol transmissions. It then follows that
%%\begin{align}\label{eq:temporal signal}
%%    \boldsymbol{\breve{h}}_{l,t} = \boldsymbol{\breve{\Psi}}(\boldsymbol{\bar{d}}_l^\text{IT},\boldsymbol{\theta}_l)\boldsymbol{x}_{l,t} + \boldsymbol{\breve{z}}_{l,t},\quad \forall l,t,
%%\end{align}where $\boldsymbol{\breve{z}}_{l,t} = [(\boldsymbol{\tilde{z}}_{l,t}^{(q)})^\text{T},\cdots,(\boldsymbol{\tilde{z}}_{l,t}^{(q)})^\text{T}]^\text{T}\in \mathbb{C}^{M_\text{V} \times 1}$ and
%%\begin{align}\label{eq:steering matrix}
%%    \boldsymbol{\breve{\Psi}}(\boldsymbol{\bar{d}}_l^\text{IT},\boldsymbol{\theta}_l) &= \begin{bmatrix}
%%    \boldsymbol{\tilde{G}}\text{diag}(\boldsymbol{\phi}^{(1)})\boldsymbol{A}_\text{I}(\boldsymbol{\bar{d}}_l^\text{IT},\boldsymbol{\theta}_l)\\
%%    \vdots  \\
%%    \boldsymbol{\tilde{G}}\text{diag}(\boldsymbol{\phi}^{(Q)})\boldsymbol{A}_\text{I}(\boldsymbol{\bar{d}}_l^\text{IT},\boldsymbol{\theta}_l)
%%    \end{bmatrix}\in \mathbb{C}^{M_\text{V} \times K_l},\notag \\
%%    & = [\boldsymbol{\breve{\psi}}(\bar{d}_1^{\text{IT}},\theta_1),\cdots,\boldsymbol{\breve{\psi}}(\bar{d}_{K_l}^{\text{IT}},\theta_{K_l})],
%%\end{align}with
%%\begin{align}\label{eq:new steering vector}
%%    \boldsymbol{\breve{\psi}}(\bar{d}_{k_l}^{\text{IT}},\theta_{k_l}) = \begin{bmatrix}
%%    \boldsymbol{\tilde{G}}\text{diag}(\boldsymbol{\phi}^{(1)})\boldsymbol{a}_\text{I}(\bar{d}_{k_l}^\text{IT},\theta_{k_l})\\
%%    \vdots  \\
%%    \boldsymbol{\tilde{G}}\text{diag}(\boldsymbol{\phi}^{(Q)})\boldsymbol{a}_\text{I}(\bar{d}_{k_l}^\text{IT},\theta_{k_l})
%%    \end{bmatrix}\in \mathbb{C}^{M_\text{V} \times 1}.
%%\end{align}Note that via a proper design of the IRS reflection coefficients, $\boldsymbol{\breve{\Psi}}(\boldsymbol{\bar{d}}_l^\text{IT},\boldsymbol{\theta}_l)$ can be of full column rank, i.e., $\text{rank}(\boldsymbol{\breve{\Psi}}(\boldsymbol{\bar{d}}_l^\text{IT},\boldsymbol{\theta}_l))=K_l$, as will be examined in the following theorem.
%
%
%Based on the virtual-spatial signal model \eqref{equ:estimated_channel_deficient} and Theorem \ref{theorem1}, we can follow Section \ref{sec:phase_II} to apply the AIC approach to estimate the target number $K_l$. 
%Then, the proposed prior information-aided MUSIC algorithm in Algorithm \ref{alg:pri_music} can be used to estimate $\theta_{l,m_i}$'s and $\bar{d}_{l,m_i}$'s from $\breve{\boldsymbol{R}}_{l}$ by replacing $\boldsymbol{\psi}(\bar{d},\theta)$ with $\breve{\boldsymbol{\psi}}(\bar{d},\theta)$. Finally, the locations of these targets can be estimated based on the same operations in Phase III.

%\begin{figure*}[ht]
%	\centering
%    \includegraphics[width=15cm]{SignalModel.eps}
%    \caption{Illustration of signals in Case I and Case II: In case I, the spatial-domain signals provide information on AOA and range information. However, in case II, the spatial-domain signals provide no information on AOA and range information, while our designed temporal-domain signals provide information on AOA and range information.}\label{fig:multisignal}
%\end{figure*}

%\begin{figure}[t]
%	\centering
%	\subfigure[Traditional multi-dimensional signal for estimation AOA and range information \cite{music}\cite{2d_music}.]{\includegraphics[height=6.6cm]{signal_virtualization1.pdf}\label{fig:signal_traditional}}
%	\subfigure[Our proposed multi-dimensional signal for estimation AOA and range information.]{\includegraphics[height=6.6cm]{signal_virtualization2.pdf}\label{fig:signal_proposed}}
%    \caption{Comparison between traditional multi-dimensional signal and our proposed multi-dimensional signal.} \label{fig:all_multisignal}
%\end{figure}





%%% IAA section
%\section{Model-based AOA and Range Estimation with A Small Number of Coherence Blocks}\label{sec:step II IAA}
%
%The above considers the regime when a large number of coherence blocks is available, such that the covariance can be estimated by \eqref{eq:est cov}. In the following, we introduce how to estimate the model-based AOA information and range information of the paths from targets to the IRS when a small number of coherence blocks is available, even with only one coherence block.
%
%To achieve the above goal, we adopt the iterative adaptive approach (IAA) framework \cite{iaa} to design the prior information-assisted IAA algorithm. Specifically, the IAA approach aims to iteratively estimate the signals from all the possible angles $\theta$'s and ranges $\bar{d}^\text{IT}$'s, i.e., $x_{l,t}(\bar{d}^\text{IT},\theta)$, while treating the signals from other angles and ranges as interference. In other words, the IAA estimates $x_{l,t}(\bar{d}^\text{IT},\theta)$ based on the following model
%\begin{align}
%    \boldsymbol{\breve{h}}_{l,t} &= \underbrace{\boldsymbol{\breve{\psi}}(\bar{d}^\text{IT},\theta)x_{l,t}(\bar{d}^\text{IT},\theta)}_{\text{Interested signal}} \notag \\&+ \underbrace{\sum_{\breve{\theta}\neq\theta}\sum_{\breve{d}^\text{IT}\neq \bar{d}^\text{IT}}\boldsymbol{\breve{\psi}}(\breve{d}^\text{IT},\breve{\theta})x_{l,t}(\breve{d}^\text{IT},\breve{\theta}) + \boldsymbol{\tilde{z}}_{l,t}}_{\text{Interference and noise}}.
%\end{align}If the corresponding signal power from an angle $\theta$ and range $\bar{d}^\text{IT}$ is large, we declare that a target exists with an angle $\theta$ and range $\bar{d}^\text{IT}$. 
%
%Note that with prior information, we only need to estimate the signals from the search grids $\hat{\mathcal{R}}_l$ for near-field targets and the search grids for far-field targets. Specifically, define
%\begin{align}
%    \tilde{\mathcal{R}}_l = \hat{\mathcal{R}}_l \cup \{(d_\varsigma^\text{IT},\theta_\xi)|\theta_\xi = \xi\Delta\theta,\xi=1,\cdots,\lceil{\frac{\pi}{\Delta \theta}\rceil}, d_{\varsigma_l}^\text{IT}=\infty\}.
%\end{align}Then, we only need to estimate the signals from $(\bar{d}^\text{IT},\theta)\in \tilde{\mathcal{R}}_l$. Starting from 
%\begin{align}\label{eq:power initialize}
%    P_l^{(0),\text{IAA}}(\bar{d}^\text{IT},\theta) =& \frac{\sum_{t=1}^{T}|\boldsymbol{\breve{\psi}}^H(\bar{d}^\text{IT},\theta)\boldsymbol{\breve{h}}_{l,t}|^2}{(\boldsymbol{\breve{\psi}}^H(\bar{d}^\text{IT},\theta)\boldsymbol{\breve{\psi}}(\bar{d}^\text{IT},\theta))^2 T},(\bar{d}^\text{IT},\theta)\in \tilde{\mathcal{R}}_l,
%\end{align}the IAA algorithm proceeds at each iteration $z$, $z=1,\cdots,Z$, as
%\begin{align}
%    &\boldsymbol{R}_l^{(z),\text{IAA}}(\bar{d}^\text{IT},\theta) \notag \\
%    &=\sum_{\substack{\breve{\theta}\neq \theta \\ \breve{d}^\text{IT}\neq \bar{d}^\text{IT}}} P_l^{(z-1),\text{IAA}}(\breve{d}^\text{IT},\breve{\theta})\boldsymbol{\breve{\psi}}(\breve{d}^\text{IT},\breve{\theta})\boldsymbol{\breve{\psi}}^H(\breve{d}^\text{IT},\breve{\theta}),(\bar{d}^\text{IT},\theta)\in \tilde{\mathcal{R}}_l, \label{eq:covariance estimation} \\
%    &x_{l,t}^{(z)}(\bar{d}^\text{IT},\theta) = \frac{\boldsymbol{\breve{\psi}}^H(\bar{d}^\text{IT},\theta)(\boldsymbol{R}_{l}^{(z),\text{IAA}}(\bar{d}^\text{IT},\theta))^{-1}\boldsymbol{\breve{h}}_{l,t}}{\boldsymbol{\breve{\psi}}^H(\bar{d}^\text{IT},\theta)(\boldsymbol{R}_{l}^{(z),\text{IAA}}(\bar{d}^\text{IT},\theta))^{-1}\boldsymbol{\breve{\psi}}(\bar{d}^\text{IT},\theta)}, \notag \\&\quad\quad\qquad\qquad\qquad\qquad\qquad\qquad\qquad(\bar{d}^\text{IT},\theta)\in \tilde{\mathcal{R}}_l,\label{eq:signal estimation} \\
%    &P_l^{(z),\text{IAA}}(\bar{d}^\text{IT},\theta) = \frac{1}{T} \sum_{t=1}^{T}|x_{l,t}^{(z)}(\bar{d}^\text{IT},\theta)|^2, (\bar{d}^\text{IT},\theta)\in \tilde{\mathcal{R}}_l.\label{eq:power iteration}
%\end{align}
%In the above iterations, \eqref{eq:power initialize} initializes the signal power from angle $\theta$ and range $\bar{d}^\text{IT}$. \eqref{eq:covariance estimation} calculates the covariance matrix of the interference and the noise in each iteration. \eqref{eq:signal estimation} estimates the signal from angle $\theta$ and range $\bar{d}^\text{IT}$ based on the weighted least square approach. \eqref{eq:power iteration} calculates the signal power from angle $\theta$ and range $\bar{d}^\text{IT}$ in each iteration $z$.
%After the convergence of the IAA algorithm, we can first perform a one-dimension search to find peaks of $P^{(Z),\text{IAA}}(\infty,\theta)$, and then a two-dimensional search over $P^{(Z),\text{IAA}}(d^\text{IT},\theta)$, $\forall (d^\text{IT},\theta)\in\hat{\mathcal{R}}_l$. The corresponding ranges and angles will be the estimations of the ranges and AOAs of the LOS paths from targets in range cluster $l$ to the IRS. The above procedures are summarized in Algorithm \ref{alg:iaa}.
%\begin{algorithm}[t]
%	\caption{Prior information assisted IAA Algorithm for AOA and Range Estimation in Range Cluster $l$}\label{alg:iaa}
%	    {\bf Input}: $\boldsymbol{\breve{h}}_{l,t}$;\\
%	    {\bf Initialization (Offline)}: Obtain $\hat{\mathcal{R}}_l$'s given in \eqref{eq:proposed search};
%     \begin{enumerate}
%         \item [1.] Initialize the signal power as \eqref{eq:power initialize};\Comment{Step 1}
%         \item [2.] Iteratively perform \eqref{eq:covariance estimation}-\eqref{eq:power iteration} until convergence to obtain $P_l^{(Z),\text{IAA}}(\bar{d}^\text{IT},\theta)$;\Comment{Step 2}
%         \item [3.] Find the peaks of the $1$-D spectrum $P_l^{(Z),\text{IAA}}(\infty,\theta)$ and the $2$-D spectrum $P_l^{(Z),\text{IAA}}(d^\text{IT},\theta)$;\Comment{Step 3}
%     \end{enumerate}
%        {\bf{Output}}: AOAs of far-field targets corresponding to the selected peaks from the $1$-D spectrum, AOAs and ranges of near-field targets corresponding to the selected peaks from the $2$-D spectrum.
%\end{algorithm}

\section{Numerical Results}\label{sec:simulation}

\begin{figure*}[t]
  \centering
  \subfigure[AOA spectrum for far-field targets.]{
    \label{fig:1D_spec}
    \includegraphics[width=.32\textwidth]{Far_1D_MUSIC.eps}}
  \subfigure[AOA spectrum for near-field targets.]{
    \label{fig:2D_spec_AOA}
    \includegraphics[width=.32\textwidth]{Near_2D_MUSIC_AOA.eps}}
  \subfigure[Range spectrum for near-field targets.]{
    \label{fig:2D_spec}
    \includegraphics[width=.32\textwidth]{Near_2D_MUSIC_Range.eps}}
    \vspace{-0.3cm}
  \caption{The MUSIC spectrums in Phase II for far-field and near-field targets with $M_{\rm B} = 4$ and $Q_0=4$.}\label{fig:spec}
  \vspace{-0.7cm}
\end{figure*}

In this section, we provide numerical examples to verify the effectiveness of the proposed three-phase localization protocol. The transmit power of the user is $39$ dBm and the power spectrum density of the noise at the receive BS is $-169$ dBm/Hz. We consider $N = 834$ sub-carriers in the OFDM system with bandwidth equaling $100$ MHz and $L=88$ solvable paths. We utilize the received signals in $V=32$ coherence blocks for target localization. Then, the IRS and the BS are modeled as ULA with $\frac{\lambda}{2}$ as the element and antenna spacing. In particular, we assume that the number of the IRS elements is $256$. Next, we assume that the BS, the IRS, and the user are located in $(50,43)$, $(50,50)$, and $(0,0)$ in meter. The near-field region is defined as the circle region centered by the IRS and radius $90$ in meter.
In particular, the challenging case is considered to validate the performance superiority of the proposed method, where there are multiple targets located in the same range cluster.
We generate $\iota_{\rm max}=10^4$ realizations of target locations, while under each realization $\iota$, we consider that targets are located in $3$ range clusters and each of them contains $K=8$ targets randomly either in the near-field or far-field region of the IRS. 

\subsection{Performance Evaluation of Prior Information-Assisted MUSIC in Phase II}

First, we validate the feasibility for extracting the AOA and/or range information by using the prior-information assisted MUSIC algorithm in Phase II in Fig. \ref{fig:spec}. We consider $M_{\rm B}=4$ and $Q_0=4$. Due to the usage of the prior information from the range cluster estimation, the spectrum for the range of the near-field targets only focuses on a narrow region and thus we only need to search the peak in a much small region, which thus reduces the searching cost. It is observed that AOAs of far-field targets can be accurately estimated based on the 1D spectrum, while both of the range and AOA information of near-field targets can be obtained by the 2D spectrum. In addition, it shows that the AOAs of near-field targets can be estimated more accurately than their range information, which reveals that the detection error of near-field targets mainly comes from the range estimation. Since the AOA and/or range information can be extracted via MUSIC, both far-field and near-field targets are able to be localized in Phase III, which also verifies the feasibility to uniquely identify different targets and localize them in the IRS-assisted system.

%In this section, we provide numerical results to verify the effectiveness of the proposed sequential localization protocol for IRS-assisted bistatic NLOS sensing. In these numerical examples, we assume that the channel bandwidth is $100$ MHz. The identical transmit power of the user is $39$ dBm. The power spectrum density of the noise at the BS is $-174$ dBm/Hz. Next, all of the user, the IRS, and the BS are modeled as ULA with $\frac{\lambda}{2}$ as the element spacing. In particular, we assume that the number of the receive antennas is $8$ and the number of the IRS elements is $256$. Last, we assume that the BS, the IRS, and the user are located in $(25,25)$, $(20,20)$, and $(0,0)$ in meter. The targets are randomly generated within range clsuter $17$. 

%To evaluate the performance of the proposed scheme, we consider the following benchmark schemes for performance comparison.
%\begin{itemize}
%    \item {\bf{Benchmark Scheme 1}}: The prior information of the range cluster obtained by the channel estimation results in Phase I is ignored under the MUSIC algorithm. Under this scheme, we perform exhaustive search over $\mathcal{R}^{\rm N}$ and $\mathcal{R}^{\rm F}$ defined in \eqref{equ:R_dtheta} and \eqref{equ:R_theta} to estimate the AOAs and/or ranges of the targets. 
%    \item {\bf{Benchmark Scheme 2}}: The range clusters which contain target and the numbers of the near-field and far-field targets are perfectly known. In such case, the channel estimation is assumed to be perfect and thus this method can be considered as the performance lower bound.
%    %\item {\bf{Benchmark Scheme 2: Alternative Optimization (AO) based Maximum Likelihood (ML) framework for model-based AOA and range estimation.}} In this scheme, we extend \cite{ao_ml} to the near-field case. Assume that the number of targets $K_l$ is known, the conventional ML approach aims to solve the following problem
%%    \begin{align}\label{prb:ml}
%%        \min_{(\boldsymbol{\bar{d}}_l^\text{IT},\boldsymbol{\theta}_l)} \sum_{t=1}^{T}f_{t}(\boldsymbol{\bar{d}}_l^\text{IT},\boldsymbol{\theta}_l),
%%    \end{align}where
%%    \begin{align}
%%      &f_{t}(\boldsymbol{\bar{d}}_l^\text{IT},\boldsymbol{\theta}_l)=  \notag \|\boldsymbol{\breve{h}}_{l,t}- \notag \\
%%      &\boldsymbol{\breve{\Psi}}(\boldsymbol{\bar{d}}_l^\text{IT},\boldsymbol{\theta}_l)
%%    (\boldsymbol{\breve{\Psi}}(\boldsymbol{\bar{d}}_l^\text{IT},\boldsymbol{\theta}_l)^H\boldsymbol{\breve{\Psi}}(\boldsymbol{\bar{d}}_l^\text{IT},\boldsymbol{\theta}_l))^{-1}\boldsymbol{\breve{\Psi}}(\boldsymbol{\bar{d}}_l^\text{IT},\boldsymbol{\theta}_l)\boldsymbol{\breve{h}}_{l,t}\|^2.
%%    \end{align}Note that problem \eqref{prb:ml} is a multi-dimensional non-convex problem, which typically requires a multi-dimensional search over $\boldsymbol{\bar{d}}_l^\text{IT}$ and $\boldsymbol{\theta}_l$. \cite{ao_ml} proposes to use the AO approach to decouple the multi-dimensional search into several two-dimensional searches. Specifically, this approach first assumes that only one target exists, whose AOA and range are estimated by performing a two-dimensional search of the following problem
%%    \begin{align}
%%        (\hat{\bar{d}}_1^{(0),\text{IT}},\hat{\theta}_1^{(0)}) = \arg\min_{(\bar{d}_1^\text{IT},\theta_1)}&\sum_{t=1}^{T}f_{t}(\bar{d}_1^{0,\text{IT}},\theta_1^{(0)}), \notag\\
%%        &\quad\forall (\bar{d}_1^{0,\text{IT}},\theta_1^{(0)})\in\hat{\mathcal{R}}_l.
%%    \end{align}Then, this approach assumes two targets exist in the system, and the first target's AOA and range are $\hat{\bar{d}}_1^{(0),\text{IT}}$ and $\hat{\theta}_1^{(0)}$. The second target's AOA and range are estimated by
%%    \begin{align}(\hat{\bar{d}}_2^{(0),\text{IT}},\hat{\theta}_2^{(0)})=&\arg\min_{(\bar{d}_2^{(0),\text{IT}},\theta_2^{(0)})}\sum_{t=1}^{T}f_{t}([\hat{\bar{d}}_1^{(0),\text{IT}},\bar{d}_2^{(0),\text{IT}}],\notag \\&[\hat{\theta}_1^{(0)},\theta_2^{(0)}]), \quad\forall (\bar{d}_2^{0,\text{IT}},\theta_2^{(0)})\in\hat{\mathcal{R}}_l.
%%    \end{align}At the $z$-th iteration, this approach iteratively estimates the AOA and range of the $k_l$-th target, i.e., $(\hat{\bar{d}}_{k_l}^{(z),\text{IT}},\hat{\theta}_{k_l}^{(z)})$, while fixing other targets' estimation. We term this scheme as Benchmark $2$.
%\end{itemize}

%\begin{figure}[t]
%	\centering
%	\subfigure[Localization performance of Far-field Targets]{\includegraphics[height=6.6cm]{far_MUSIC.eps}\label{fig:perf_music_far}}
%	\subfigure[Localization performance of Near-field Targets]{\includegraphics[height=6.6cm]{near_MUSIC.eps}\label{fig:perf_music_near}}
%    \caption{Evaluation of Algorithm \ref{alg:music} under an IRS-assisted ISAC system.} \label{fig:all_music}
%\end{figure}

%\begin{figure}[t]
%	\centering
%	\subfigure[Localization performance of Far-field Targets]{\includegraphics[height=6.6cm]{far_IAA.eps}\label{fig:perf_iaa_far}}
%	\subfigure[Localization performance of Near-field Targets]{\includegraphics[height=6.6cm]{near_IAA.eps}\label{fig:perf_iaa_near}}
 %   \caption{Evaluation of Algorithm \ref{alg:music} under an IRS-assisted ISAC system.} \label{fig:all_iaa}
%\end{figure}

% \begin{figure*}
%     \begin{minipage}[t]{0.33\textwidth}
%         \centering
%         \includegraphics[width=1\textwidth]{J_new_pro_Re_Benchmark.eps}
%         \caption{Localization performance versus the radius of the effective detection region with $M_{\rm B} = 4$ and $Q_0=4$.}
%     \end{minipage} \hspace{0.2cm}
%     \begin{minipage}[t]{0.33\textwidth}
%         \centering
%         \includegraphics[width=1\textwidth]{J_new_pro_Mr_Benchmark.eps}
%         \caption{Localization performance versus the number of BS antennas $M_{\rm B}$ with $Q_0=1$ when the BS-IRS channel follows near-field channel model.}
%     \end{minipage} \hspace{0.2cm}
%     \begin{minipage}[t]{0.33\textwidth}
%         \centering
%         \includegraphics[width=1\textwidth]{J_new_pro_Mr_far-BS-IRS.eps}
%         \caption{Localization performance versus the number of BS antennas $M_{\rm B}$ with $Q_0=10$ when the BS-IRS channel follows far-field channel model.}
%     \end{minipage}
% \end{figure*}

\begin{figure}
  \centering
  \includegraphics[width=.42\textwidth]{J_new_pro_Re_Benchmark.eps}
  \vspace{-0.3cm}
  \caption{Localization performance versus the radius of the effective detection region with $M_{\rm B} = 4$ and $Q_0=4$.}\label{fig:prob_Re}
  \vspace{-0.3cm}
\end{figure}

\subsection{Performance Evaluation of the Three-Phase Protocol}


Two performance metrics will be considered - missed detection and false alarm probabilities for near-field and far-field target detection. 
First, we can define the effective detection region for each true target $k$ as a circle region centered by itself with radius $R_{e}$. Specifically, under the each realization, we claim that a missed detection event occurs for a near-field (far-field) target $k$ if the locations of all the detected near-field (far-field) targets estimated by our proposed method lie outside the effective detection region of this target $k$. Moreover, we claim that a false alarm event occurs for a detected near-field (far-field) target $k$ if its estimated location lies outside the effective detection region of all the near-field (far-field) targets. 
Let $K_{\iota}^{\rm N}$ and $K_{\iota}^{\rm F}$ denote the numbers of near-field and far-field targets generated in realization $\iota$, $K_{\iota}^{\rm N,MD}$, $K_{\iota}^{\rm F,MD}$, $K_{\iota}^{\rm N,FA}$, and $K_{\iota}^{\rm F,FA}$ denote the number of missed detection events for near-field targets and those for far-field targets, and false alarm events for near-field targets and those for far-field targets at realization $\iota$. Then, after $\iota_{\rm max} = 10^4$ realizations, the missed detection probabilities for detecting the near-field and far-field targets are calculated as $P_{\text{MD}}^{\rm N} = \frac{\sum_{\iota=1}^{\iota_\text{max}}K^{{\rm N,MD}}_{\iota}}{\sum_{\iota=1}^{\iota_{\text{max}}} K_{\iota}^{\rm N}}$ and $P_{\text{MD}}^{\rm F} = \frac{\sum_{\iota=1}^{\iota_\text{max}}K^{{\rm F,MD}}_{\iota}}{\sum_{\iota=1}^{\iota_{\text{max}}} K_{\iota}^{\rm F}}$, and the false alarm probabilities for detecting the near-field and far-field targets are defined as $P^{\rm N}_{\text{FA}} = \frac{\sum_{\iota=1}^{\iota_\text{max}}K^{{\rm N,FA}}_{\iota}}{\sum_{\iota=1}^{\iota_{\text{max}}} K_{\iota}^{\rm N}}$ and $P^{\rm F}_{\text{FA}} = \frac{\sum_{\iota=1}^{\iota_\text{max}}K^{{\rm F,FA}}_{\iota}}{\sum_{\iota=1}^{\iota_{\text{max}}} K_{\iota}^{\rm F}}$. The representative sparsity-based algorithm of simultaneous orthogonal matching pursuit (S-OMP) \cite{Tropp_2006_SP} is also served as the benchmark scheme. 

\begin{figure}[t]
  \centering
  \includegraphics[width=.42\textwidth]{J_new_pro_Mr_Benchmark.eps}%\vspace{-0.3cm}
  \vspace{-0.3cm}
  \caption{Localization performance versus the number of BS antennas $M_{\rm B}$ with $Q_0=1$ when the BS-IRS channel follows near-field channel model.}\label{fig:prob_MB}%\vspace{-0.6cm}
  \vspace{-0.3cm}
\end{figure}

\begin{figure}[t]
  \centering
  \includegraphics[width=.42\textwidth]{J_new_pro_Mr_far-BS-IRS.eps}%\vspace{-0.3cm}
  \vspace{-0.3cm}
  \caption{Localization performance versus the number of BS antennas $M_{\rm B}$ with $Q_0=10$ when the BS-IRS channel follows far-field channel model.}\label{fig:prob_MB_far-BS-IRS}\vspace{-0.5cm}
\end{figure}

The impact of the detection radius $R_{e}$ on the performance\footnote{The selection of thresholds $\varsigma^{\rm F}$ and $\varsigma^{\rm N}$ realizes a trade-off between the performance of false alarm probability and missed detection probability. In simulations, thresholds $\varsigma^{\rm F}$ and $\varsigma^{\rm N}$ are chosen to realize close performance of false alarm probability and missed detection probability for fair comparison.} of the proposed localization protocol is first evaluated in Fig. \ref{fig:prob_Re} with $M_{\rm B} = 4$ and $Q_0=4$. We can observe that enlarging the detection region $R_e$ can enhance the localization performance of both near-field and far-field targets for both the MUSIC and S-OMP algorithms. This improvement, however, comes with an increases in the tolerable positioning error. Additionally, the MUSIC algorithm can significantly outperform the S-OMP algorithm, since inadequate number of measurements usually leads to limited performance for CS-based algorithms. It is also shown that the missed detection/false alarm probability of far-field and near-field targets approaches their minimal value for $R_e \ge 1.1$ (meter) and $R_e \ge 1.5 $ (meter), respectively. If we set $R_e = 1$ (meter), missed detection and false alarm probabilities of far-field and near-field targets are all less than 0.04, indicating that precise meter-level localization performance can be realized via the proposed three-phase protocol. In the following simulations, we set $R_e = 1$ (meter).


\begin{figure}[t]
  \centering
  \includegraphics[width=.42\textwidth]{J_new_pro_Q_fixed-QMr_Benchmark.eps}
  \vspace{-0.3cm}
  \caption{Localization performance versus the number of utilized OFDM symbols $Q_0$ used in each coherence block for localization under $Q_0 M_{\rm B} = 12$.}\label{fig:prob_Q}
  \vspace{-0.3cm}
\end{figure}

\begin{figure}[t]
  \centering
  \includegraphics[width=.42\textwidth]{J_new_pro_K_Benchmark.eps}
  \vspace{-0.3cm}
  \caption{Localization performance versus the number of targets $K$ with $M_{\rm B} = 4$ and $Q_0=4$.}\label{fig:prob_K}\vspace{-0.5cm}
\end{figure}

We then evaluate the localization performance under different numbers of BS antennas $M_{\rm B}$ in Fig. \ref{fig:prob_MB} with $Q_0=1$ when the BS-IRS channel follows the near-field channel model. It is observed that both of false alarm and missed detection probabilities for near-field and far-field targets are reduced with the $M_{\rm B}$ increases, because more measurements for target localization are obtained. 
In particular, the missed detection and false alarm probabilities of far-field targets has much more reduction with $M_{\rm B}$ increasing, compared with near-field targets. This is because increasing $M_{\rm B}$ can provide significant performance improvement on the AOA estimation while the range information estimation performance is only slightly improved, which is similar to results in Fig. \ref{fig:spec}. We also consider the case where the BS-IRS channel follows the far-field channel model in Fig. \ref{fig:prob_MB_far-BS-IRS}. It is observed that solely increasing the number of BS antennas cannot provide any enhancement in localization accuracy, which validates our previous analysis that increasing the number of BS antennas in such case only provides redundant information for localization.
%both of the false alarm probabilities and missed detection probabilities to detect the near-field targets and the far-field targets are below 0.02 when $M_{\rm B} \ge 18$, implying that targets can be accurately detected by the proposed localization protocol, with a maximum-$1.5$-meter localization error. 
%With more BS antennas, the detection performance for both near-field and far-field targets is improved, because the LASSO problem with larger $M_{\rm B}$ can provide more accurate estimation of user-target-IRS-BS channels and the MUSIC algorithm with a larger-size sampled covariance matrix can also give more precise AOA and/or range information in such case.
%\subsection{Localization and Complexity Performance of Algorithm \ref{alg:pri_music} in the Rank-Sufficient Case}


The impact of the number of utilized OFDM symbols $Q_0$ in each coherence block on the localization performance is shown in Fig. \ref{fig:prob_Q}. Here, we fix the dimension of measurements with $Q_0 M_{\rm B} = 12$. It is observed that increasing the number $Q_0$ of utilized OFDM symbols used in each coherence block can significantly boost the localization performance. This is because the target detection performance is limited to the ill-conditioned matrix $\breve{\boldsymbol{\Psi}}(\Theta_l)$ resulting from the channel matrix $\boldsymbol{G}$ when $Q_0$ is small. By increasing $Q_0$, the matrix $\breve{\boldsymbol{\Psi}}(\Theta_l)$ can become a better-conditioned matrix and thus the target detection performance is enhanced. These results also indicate that we can use more OFDM symbols in each coherence block to reduce the localization error if the number of BS antennas is small due to hardware limits.



We also evaluate the localization performance of the proposed protocol under different target numbers in Fig. \ref{fig:prob_K} with $M_{\rm B} = 4$ and $Q_0=4$. It is observed that missed detection and false alarm probabilities are both enlarged with the number $K$ of targets increases. With more targets existing in the system, more reflection signals by targets are interfered with each other, leading to degraded performance. Therefore, we should utilize more OFDM symbols in each coherence block to maintain the performance when there are target number is very large in the system, as revealed in Fig. \ref{fig:prob_Q}.

%\begin{figure}[t]
%  \centering
%  \includegraphics[width=.48\textwidth]{J_Pro_BW.eps}
%  \caption{Target detection performance of the proposed protocol versus the bandwidth $B$ with $M_{\rm B} = 16$ and $Q=1$.}\label{fig:prob_BW}
%\end{figure}
%
%\begin{figure}[t]
%  \centering
%  \includegraphics[width=.48\textwidth]{J_MSE_BW.eps}
%  \caption{Localization error performance of the proposed protocol versus the bandwidth $B$ with $M_{\rm B} = 16$ and $Q=1$.}\label{fig:mse_BW}
%\end{figure}
%
%At last, we give the performance of target detection and localization error under different bandwidths in Fig. \ref{fig:prob_BW} and Fig. \ref{fig:mse_BW}, respectively, with $M_{\rm B} = 16$ and $Q=1$. Here, the performance metric for localization error adopts the mean squared error (MSE) defined as $\text{MSE} = \frac{1}{K}\sum_{k=1}^{K} [(\hat{x}_k-x_k)^2 + (\hat{y}_k-y_k)^2]$. The results demonstrate that target detection performance has limited improvement and will saturate when $B$ is large enough, because increasing the bandwidth can only provide more precise range information but provide no information for the AOA estimation via the MUSIC algorithm. However, it is observed that increasing the bandwidth can greatly reduce the localization error for the far-field targets, since the range resolution becomes smaller and then the range estimation error for far-field targets is also diminished. However, for the near-field targets, the MSE keeps almost unchanged under different bandwidth, because the range information for them is estimated by 2D-MUSIC algorithm and independent to the bandwidth. From the results, it is also revealed that exploiting IRS as a passive anchor can be potential to provide centimeter-level localization performance when the bandwidth is large.

%To start with, we show the numerical results of Algorithm \ref{alg:music} under the case when the number of coherence blocks is larger than the number of targets, i.e., $T>K$. In this example, we assume that the number of coherence blocks is $16$, i.e., $T=16$. Fig. \ref{fig:perf_music_far} and Fig. \ref{fig:perf_music_near} show the performance comparison between our proposed Algorithm \ref{alg:music} and the two benchmark schemes, in terms of localization accuracy. First, Fig. \ref{fig:perf_music_far} and Fig.\ref{fig:perf_music_near} show the false alarm and miss detection probabilities for localizing far-field targets and near-field targets achieved by Algorithm $1$.
%The false alarm probability and the miss detection probability are calculated as follows. First, in the $\iota$-th realization, we obtain the number of targets that are correctly localized, denoted by $K_\iota ^{\text{correct}}$. In particular, if the estimated target location lies within a radius of $1.2$ meter from the true target location, then this target is correctly localized. Then, let $K_\iota^{\text{detect}}$ denote the number of estimated targets. Then, the false alarm probability and the miss detection probability are defined as 
%\begin{align}
%    &P_{\text{FA}} = \frac{\sum_{\iota=1}^{\iota_\text{max}}(K_\iota^{\text{detect}}-K_\iota ^{\text{correct}})}{K\times \iota_\text{max}}, \\
%    &P_{\text{MD}} = \frac{\sum_{\iota=1}^{\iota_\text{max}}(K-K_\iota ^{\text{correct}})}{K\times \iota_\text{max}},
%\end{align}where $\iota_\text{max}$ is number of Monte Carlo realizations. It is observed from Fig. \ref{fig:perf_music_far} that both the false alarm and miss detection probability of our proposed algorithm are much lower than that achieved by the AO-ML approach. This is because the objective function \eqref{prb:ml} is highly non-convex, and the AO approach is very likely to converge to a local optimum point. However, our proposed approach does not rely on the optimization approach. Second, it is observed that our proposed approach archives a similar performance as that of the benchmark scheme $1$. This indicates that our proposed prior information-assisted approach is very effective. Last, it is observed that the localization error probability of near-field targets is smaller than that of the far-field targets. This shows that the model-based range estimation is more accurate than the model-free range estimation.
%
%Second, we evaluate the complexity of the above schemes in terms of CPU time in seconds. When $K=6$, the average CPU time of the proposed scheme is $0.4394$ in second, while that of Benchmark scheme $1$ is $5.4107$ in second and that of Benchmark scheme $2$ is $87.3706$ in second. This result verifies that our proposed approach is of much lower complexity thanks to the leverage of prior information \eqref{eq:proposed search}.

%\subsection{Localization and Complexity Performance of Algorithm \ref{alg:pri_music} in the Rank-Deficient Case}

%Next, we show the numerical results of Algorithm \ref{alg:iaa} under the case when the number of coherence blocks is small. In this example, we assume that the number of coherence blocks is $1$. Fig. \ref{fig:perf_iaa_far} and Fig. \ref{fig:perf_iaa_near} show the localization accuracy of Algorithm \ref{alg:iaa} when the number of OFDM systems varies from $10$ to $14$. It is observed from Fig. \ref{fig:perf_iaa_far} and Fig. \ref{fig:perf_iaa_near} that similar to the case with a large number of coherence blocks, the performance of Algorithm $2$ is much better than that of Benchmark scheme $2$, and very close to that of Benchmark scheme $1$. Moreover, when there are $6$ targets, the average CPU time of our proposed scheme is $36.3722$ in second, which is lower than $45.4474$ in second achieved by Benchmark scheme $1$ and $37.3922$ in second achieved by Benchmark scheme $2$. This thus verifies that the effectiveness of our proposed algorithm even with a limited number of transmit antennas.

%\begin{figure}[t]
%	\centering
%	\subfigure[Localization performance of Far-field Targets]{\includegraphics[height=6.6cm]{far_compare.eps}\label{fig:compare_far}}
%	\subfigure[Localization performance of Near-field Targets]{\includegraphics[height=6.6cm]{near_compare.eps}\label{fig:compare_near}}
 %   \caption{Comparison between Algorithm \ref{alg:music} and Algorithm \ref{alg:iaa} with different number of transmit antennas.} \label{fig:all_compare}
%\end{figure}

%\subsection{Effect of the Number of Transmit Antennas on Localization Performance}

%Last, we evaluate the effect of the number of transmit antennas on localization performance. Fig. \ref{fig:compare_far} and \ref{fig:compare_near} show the performance of Algorithm \ref{alg:music} and Algorithm \ref{alg:iaa} when the number of transmit antennas ranges from $1$ to $9$ and the number of targets is $6$. It is observed that in the regime where the number of transmit antennas is small, the performance of Algorithm \ref{alg:iaa} is better than Algorithm \ref{alg:music}. However, in the regime where the number of transmit antennas is large, Algorithm \ref{alg:music} achieves a better performance as compared to Algorithm \ref{alg:iaa}. This is because Algorithm \ref{alg:music} requires the covariance matrix to be accurately estimated by \eqref{eq:est cov}, without which Algorithm \ref{alg:music} will fail. Therefore, we can implement Algorithm $1$ with a large number of transmit antennas and Algorithm $2$ with a small number of antennas.  

\section{Conclusions}

In this paper, we considered the localization problem for passive targets in an IRS-assisted bi-static 6G ISAC network, where LOS paths between the targets and the BS do not exist, and the IRS serves as a passive anchor to localize the targets. The main challenges lied in how to apply the subspace-based method to jointly localize these mixed far-field and near-field targets from the received signals at the BS. To tackle these challenges, we proposed a three-phase localization protocol that is able to exploit the BS received signals to detect which targets are in the near-field region and which are in the far-field region of the IRS, and localize these targets. Numerical results were provided to verify the effectiveness of our proposed three-phase localization protocol, which demonstrate the feasibility of exploiting the IRSs as passive anchors for the localization of passive targets in the future 6G network to realize ISAC.

\begin{appendices}
\section{Proof of Theorem \ref{theorem1}}\label{appendix1}

First, it is easy to know that the matrix $\boldsymbol{P}$ should satisfy $\text{rank}(\boldsymbol{P}) \ge K^{\rm max}$ due to the inequality $\boldsymbol{\breve{\Psi}}(\Theta_l) \le \min\{\boldsymbol{P}, \boldsymbol{A}_{\rm I}(\Theta_l)\}$. Since $\text{rank}(\boldsymbol{P}) \le Q_0 r_{\boldsymbol{G}}$ and $r_{\boldsymbol{G}} \ge 1$, condition 1) holds to ensure that $\text{rank}(\boldsymbol{P}) \ge K^{\rm max}$ can be realized in both the cases where the BS-IRS channel follows the far-field and near-field channel models. Then, the condition 2) is to ensure that $K^{\rm max}$ distinct targets can be identified without ambiguity since any $M_{\rm I}$ steering vectors in ULA array manifold are linearly independent \cite{linear_indpend}. The condition 2) also supports that the condition 3) can be realized. Finally, assuming that conditions 1) and 2) both hold, based on the reflecting pattern design in \eqref{equ:IRS_phi_design}, the first row of each matrix $\boldsymbol{P}^{(q)}$ can be expressed as $\boldsymbol{p}^{(q)}_{1,:} = (\boldsymbol{w}_{q}^{\rm I})^T\boldsymbol{D}(\vartheta)$ with $\boldsymbol{D}(\vartheta) = \text{diag}([1,\dots,e^{j(M_{\rm I}-1)\vartheta}])$. Define the matrix $\boldsymbol{P}_1 = [(\boldsymbol{p}^{(1)}_{1,:})^T,\dots,(\boldsymbol{p}^{(Q_0)}_{1,:})^T]^T$. We can get the relationship
\begin{align}\label{equ:col_space_P1}
    \mathcal{N}(\boldsymbol{P}_1) = \text{span}(\boldsymbol{D}(\vartheta)\boldsymbol{w}_{Q_0 + 1}^{\rm I},\dots,\boldsymbol{D}(\vartheta)\boldsymbol{w}_{M_{\rm I}}^{\rm I}).
\end{align}
We can then follow \cite[Proposition 1]{Amini_2005_SPL} to prove that the matrix $\boldsymbol{\breve{\Psi}}_1(\Theta_l) = \boldsymbol{P}_1 \boldsymbol{A}_{\rm I}(\Theta_l)$ is full-column rank with $\text{rank}(\boldsymbol{\breve{\Psi}}_1(\Theta_l)) = K_l$ by contraction if $\boldsymbol{D}(\vartheta) \boldsymbol{w}_{m}^{\rm I} \ne \boldsymbol{a}_{\rm I}(\bar{d}_k,\theta_k))$, $\forall m \in \{Q_0+1,\dots,M_{\rm I}\}, k \in \{1,\dots,K\}$.
However, it is easy to know that the probability 
\begin{align}
    &\text{Pr}(\boldsymbol{D}(\vartheta) \boldsymbol{w}_{m}^{\rm I} = \boldsymbol{a}_{\rm I}(\bar{d}_k,\theta_k)) = 0, \notag \\
    &\qquad\qquad \forall m \in \{Q_0+1,\dots,M_{\rm I}\}, k \in \{1,\dots,K\}.
\end{align}
Due to the fact $\mathcal{N}(\boldsymbol{P}) \subseteq \mathcal{N}(\boldsymbol{P}_1)$, we can obtain that $\text{rank}(\boldsymbol{\breve{\Psi}}(\Theta_l)) = K_l$ also holds almost surely. 
Therefore, the probability of $\text{rank}(\boldsymbol{\breve{\Psi}}(\Theta_l)) = K_l$ is equal to one.
The proof is completed.

%First, the condition 1) is to ensure that the steering vector of from different targets to the IRS can be identified without ambiguity. Then, we know that $\text{rank}(\boldsymbol{P}^{(q)}) = \text{rank}(\boldsymbol{G}) =  1$ based on the far-field model for the BS-IRS channel. Define $\boldsymbol{P} = [(\boldsymbol{P}^{(1)})^T,\dots,(\boldsymbol{P}^{(Q_0)})^T]^T$, we can easily obtain that $\boldsymbol{P} \le Q_0$. Therefore, we cannot identify the $K_l$ targets if the condition 2) does not hold by the inequality $\text{rank}(\boldsymbol{P}\boldsymbol{A}_{\rm I}(\Theta_{l})) \le \min\{\boldsymbol{P},\boldsymbol{A}_{\rm I}(\Theta_{l})\}, ~\forall l \in \Phi$. 
%%With the conditions 1) and 2), we can finally prove that $\text{rank}(\breve{\Psi}(\Theta_l)) = K_l, ~\forall l \in \Phi$ if the condition 3) holds. 
%Finally, based on the reflection pattern design in \eqref{equ:IRS_phi_design}, each row of the matrix $\boldsymbol{P}^{(q)}$ can be expressed as $\varpi_n^{(q)} \boldsymbol{a}_{\rm I}^T(\tilde{d}_q,\tilde{\theta}_q)$ with $\varpi_n^{(q)} \ne 0 \in \mathbb{C}, ~\forall q$. As such, we can obtain $\text{rank}(\boldsymbol{P}) = Q_0$ and 
%\begin{align}\label{equ:col_space_P}
%    \mathcal{N}(\boldsymbol{P}) = \text{span}\{\boldsymbol{a}_{\rm I}(\tilde{d}_1,\tilde{\theta}_1),\dots,\boldsymbol{a}_{\rm I}(\tilde{d}_{Q_0},\tilde{\theta}_{Q_0})\}.
%\end{align}
%Based on the above, it is easy to find that \eqref{equ:col_space_P} can be regarded as the 2D case of the condition in \cite[Proposition 1]{Amini_2005_SPL} which only considers the far-field channel model.
%Then we can similarly prove the condition 3) by contraction, which is omitted here. Thus, the proof is completed.

%First, it is easy to prove that conditions 1) and 2) hold. Since $\text{rank}(\boldsymbol{P}\boldsymbol{A}_{\rm I}(\mathcal{D}_{l},\Theta_{l})) \le \min(\text{rank}(\boldsymbol{P}), \text{rank}(\boldsymbol{A}_{\rm I}(\mathcal{D}_{l},\Theta_{l})))$, we should have $\text{rank}(\boldsymbol{P}) \ge \max\{K_l, l \in \Phi\}$ and $\text{rank}(\boldsymbol{A}(\mathcal{D}_{l},\Theta_{l})) = K_l$, which ensures that the $K_l$ virtual steering vector can be identified without no ambiguity, respectively. As well, we should have $Q_0 M_r > \max\{K_l, l \in \Phi\}$ to guarantee the existence of the noise subspace in the covariance matrix for the MUSIC algorithm. Then, with conditions 1) and 2) hold, we prove that $\text{rank}\left(\breve{\boldsymbol{\Psi}}(\mathcal{D}_{l},\Theta_{l})\right) = K_l$, $\forall l$ when condition 3) holds. We can regard the condition 3) as a generalized two-dimensional case of that in \cite[Proposition 1]{Amini_2005_SPL}, which can be similarly proved by contraction \cite{Amini_2005_SPL}. 

%If $\text{rank}\left(\breve{\boldsymbol{\Psi}}(\mathcal{D}_{l},\Theta_{l})\right) < K_l$, we can obtain that there exists at least one vector $\boldsymbol{\omega} \ne \boldsymbol{0} \in \mathbb{C}^{K_l \times 1}$ with $\breve{\boldsymbol{\Psi}}(\mathcal{D}_{l},\Theta_{l})\boldsymbol{\omega} = \boldsymbol{0}$. As such, we have
%\begin{align}
%    \sum_{i=1}^{M_{\rm I}-K_l} \kappa_{i} \boldsymbol{a}_{\rm I}(\bar{d}_{i}^{0},\theta_i^0) = \sum_{k=1}^{K_l} \omega_k \boldsymbol{a}_{\rm I}(\bar{d}_{k},\theta_k),
%\end{align}
%which contracts condition 2). Therefore, with the above three conditions being satisfied, we obtain $\text{rank}\left(\breve{\boldsymbol{\Psi}}(\mathcal{D}_{l},\Theta_{l})\right) = K_l$, $\forall l$.

%
%For simplicity, we use $\boldsymbol{a}_k$ to denote $\boldsymbol{a}_\text{I}(\bar{d}_k,\theta_k)$ for $k \in \{1,\dots,K\}$ and denote the $i$th row of $\boldsymbol{G}$ as $\boldsymbol{g}_i^T \in \mathbb{C}^{1 \times M_{\rm I}}$ here. Without loss of generality, by assuming the first $r_{\boldsymbol{G}}$ rows of $\boldsymbol{G}$ are mutually linearly independent, we can define $\breve{\boldsymbol{G}} = [\boldsymbol{g}_1,\dots,\boldsymbol{g}_{r_{\boldsymbol{G}}}]^T$ as the matrix containing the first $r_{\boldsymbol{G}}$ rows of $\boldsymbol{G}$ and
%%\begin{align}
%%   \boldsymbol{A} = [(\boldsymbol{a}_1)^\text{T},\cdots,(\boldsymbol{a}_T)^\text{T}]^\text{T}= \begin{bmatrix}
%%    \boldsymbol{g}_{\mathcal{M}_{\text{S}}(1)}\text{diag}(\boldsymbol{\phi}^{(1)})\\
%%    \vdots  \\
%%    \boldsymbol{g}_{\mathcal{M}_{\text{S}}(M_\text{S})}\text{diag}(\boldsymbol{\phi}^{(1)}) \\
%%    \vdots \\
%%    \boldsymbol{g}_{\mathcal{M}_{\text{S}}(1)}\text{diag}(\boldsymbol{\phi}^{(Q)})\\
%%    \vdots \\
%%    \boldsymbol{g}_{\mathcal{M}_{\text{S}}(M_\text{S})}\text{diag}(\boldsymbol{\phi}^{(Q)})
%%    \end{bmatrix}\in \mathbb{C}^{T \times I},
%%\end{align}
%\begin{align}
%   \boldsymbol{B} &= [\boldsymbol{b}_1,\cdots,\boldsymbol{b}_S]^T= \begin{bmatrix}
%    %\boldsymbol{g}_{1}\text{diag}(\boldsymbol{\phi}^{(1)})\\
%%    \vdots  \\
%%    \boldsymbol{g}_{r_{\boldsymbol{G}}}\text{diag}(\boldsymbol{\phi}^{(1)}) \\
%%    \vdots \\
%%    \boldsymbol{g}_{1}\text{diag}(\boldsymbol{\phi}^{(Q)})\\
%%    \vdots \\
%%    \boldsymbol{g}_{r_{\boldsymbol{G}}}\text{diag}(\boldsymbol{\phi}^{(Q)})
%    \breve{\boldsymbol{G}}\text{diag}(\boldsymbol{\phi}^{(1)}) \\
%    \vdots \\
%    \breve{\boldsymbol{G}}\text{diag}(\boldsymbol{\phi}^{(Q)})
%    \end{bmatrix} \notag \\
%    &= \boldsymbol{E} \odot \breve{\boldsymbol{G}} \in \mathbb{C}^{S \times Q},
%\end{align}
%with $\boldsymbol{E} = [\bar{\boldsymbol{\phi}}^{(1)},\dots,\bar{\boldsymbol{\phi}}^{(Q)}]^T$. hen the virtual steering vector can be represent as $\breve{\boldsymbol{\psi}}_k = \boldsymbol{B}\boldsymbol{a}_k, ~\forall k$. Thus, the condition 1. ensures that the necessary condition to identify $K_l$ virtual steering vectors satisfies.
%Since $M_{\rm I}$ is usually much larger than $S$, i.e., $S \ll M_{\rm I}$, it can ensure that the matrix $\boldsymbol{B}$ is in full row rank if the channel $\boldsymbol{G}$ satisfies a) $\text{spark}(\breve{\boldsymbol{G}}) = r_{\boldsymbol{G}}+1$, and b) $\text{spark}(\boldsymbol{E}) = Q+1$, whose proof is simply by referring to \cite[Lemma 1]{Guo_2017_TWC}. 
%%We can adjust the location and orientation of the IRS and the antenna array at the BS to satisfy the condition a). While for condition b), it can usually be satisfied by designing the DFT-like random reflection pattern or the independent and identical distributed random reflection pattern.
%Based on the channel modelling in \eqref{eq:near irs bs channel}, we can obtain that each vector $\boldsymbol{g}_i$ can also represented in the form of $\boldsymbol{a}_{\rm I}(\bar{d}_{i}^{\rm B},\theta_{i}^{\rm B})$, $\forall i \le r_{\boldsymbol{G}}$. Therefore, the matrix $\breve{\boldsymbol{G}}$ is sure to satisfy the condition a) based on the condition 2.




%where $t = (q-1)|\mathcal{M}_\text{S}|+m_\text{S}$ and 
%\begin{align}
%    \boldsymbol{a}_t = \boldsymbol{g}_{M_\text{S}(m_\text{S})}\text{diag}(\boldsymbol{\phi}^{(q)}),\quad \forall t.
%\end{align}



%In the following, we aim to show that $\boldsymbol{B}\boldsymbol{a}_k$'s, $\forall k$, are linearly independent with probability $1$ by contradiction.
%
%Suppose that $\{\boldsymbol{B}\boldsymbol{a}_k\}_{k=1}^{K}$ are linearly dependent. According to the definition of the linear dependence, there exists $K$ scalars $\vartheta_k \in \mathbb{C}$ (not all $\vartheta_k$'s are equal to zero) such that
%\begin{align}\label{eq:def linear independ}  
%    \vartheta_1\boldsymbol{B}\boldsymbol{a}_1+\cdots+\vartheta_K\boldsymbol{B}\boldsymbol{a}_K =\boldsymbol{0}.
%\end{align}
%Note that \eqref{eq:def linear independ} is equivalent to
%\begin{align}\label{eq:def linear independ 2}
%    \boldsymbol{b}_s^T(\vartheta_1\boldsymbol{a}_1+\cdots+\vartheta_K\boldsymbol{a}_K)=0,\quad \forall s.
%\end{align}
%In the following, we show that \eqref{eq:def linear independ 2} does not hold almost surely when $S>K$. If this is true, then $\boldsymbol{B}\boldsymbol{a}_k$'s are linearly independent with probability $1$.
%
%Let $\mathcal{E}$ denote the event that \eqref{eq:def linear independ 2} holds, and $\mathcal{E}_s$ denote the event that the $s$-th equation in \eqref{eq:def linear independ 2} holds. Note that 
%\begin{align}\label{eq:prob inequality}
%    \text{P}(\mathcal{E}) = \text{P}(\mathcal{E}_1\cap \cdots \cap \mathcal{E}_S) \leq \text{max}\{\text{P}(\mathcal{E}_1),\cdots,\text{P}(\mathcal{E}_S)\}.
%\end{align}
%In the following, we show \eqref{eq:prob inequality} holds almost surely by showing that 
%\begin{align}\label{eq:prob subevent}
%    \text{P}(\mathcal{E}_s)=0, ~\forall s \in \{1, \dots, S\},
%\end{align}
%holds almost surely. First, because the IRS satisfies condition 2, $\boldsymbol{a}_k$'s are linearly independent \cite{linear_indpend} and $(\vartheta_1\boldsymbol{a}_1+\cdots+\vartheta_K\boldsymbol{a}_K) \neq \boldsymbol{0}$ holds. Second, because the IRS adopts the random reflection approach, $\boldsymbol{b}_s\neq\boldsymbol{0}$ holds. As a result, \eqref{eq:prob subevent} holds if and only if the random variable $\vartheta_1\boldsymbol{b}_s^T\boldsymbol{a}_1+\cdots+\vartheta_K\boldsymbol{b}_s^T\boldsymbol{a}_K$ equals to zero. Note that $\vartheta_1\boldsymbol{b}_s^T\boldsymbol{a}_1+\cdots+\vartheta_K\boldsymbol{b}_s^T\boldsymbol{a}_K$ is the sum of $K$ log-normal distributed variables, which is a continuous variable \cite{sum_lognormal}. The probability that a continuous random variable equals to a certain value is $0$, which means that equation \eqref{eq:prob subevent} holds. Theorem \ref{theorem1} is thus proved.

\end{appendices}

\bibliographystyle{IEEEtran}
\bibliography{ref}

\end{document}



