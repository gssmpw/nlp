
\section{Preliminary and Background} \label{sec:prelim}
\subsection{Workflow}

A \emph{workflow} defines a structured process designed to accomplish a specific task or goal within a particular scenario. For instance, in a hospital appointment booking scenario, a typical workflow involves steps such as querying the user for their preferred hospital, department, and time, retrieving available appointment slots using relevant tools, confirming the details with the user, and completing the booking. 
Formally, we can represent a workflow as a directed acyclic graph (DAG) denoted by $\mathcal{G}(\mathcal{V}, \mathcal{E})$ \citep{WorFBench,AFlow}, where $\mathcal{V}$ represents the set of nodes, each corresponding to an atomic operation (e.g., querying the user, invoking an API, retrieving from a knowledge base), and $\mathcal{E}$ represents the directed edges that define the temporal and dependency relationships between these operations, effectively specifying the workflow's progression.


\subsection{Workflow Agent} \label{subsec:workflow-agent}
A \emph{workflow agent} is designed to assist users in completing tasks by interacting with them and utilizing available tools. It can be conceptualized as an agent making sequential decisions within an environment composed of the user and the available tools. This interaction can be modeled as a Markov Decision Process (MDP), which provides a valuable framework for understanding the agent's decision-making process over time. In this framework, the agent transitions through a sequence of states ($s$), takes actions ($a$) based on the current state, and receives feedback ($r$) from the environment (user responses or tool-generated outputs). This process can be represented as $\{(s_{0}, a_{0}, r_{0}), (s_{1}, a_{1}, r_{1}), \dots, (s_{t-1}, a_{t-1}, r_{t-1})\}$. Consequently, the decision-making process of the workflow agent can be expressed as:
\begin{equation}
    a_t \leftarrow \mathcal{A}(\mathcal{H}_{t-1}, \mathcal{G}),
\end{equation}
where $\mathcal{H}_{t-1}$ encompasses all actions and observations up to time $t-1$, and $\mathcal{G}$ serves as the guide for the agent's actions.

Based on how the workflow is represented and integrated into the agent's decision-making process, workflow agents can be broadly classified into two categories: \emph{rule-based agents} and \emph{prompt-based agents}. Rule-based agents rely on explicitly programmed procedures to guide the workflow, while prompt-based agents utilize a single language model to autonomously manage the entire decision-making and dialogue process.

The first category, \textbf{rule-based agents}, implements the workflow procedure through explicit programming. Examples include Coze~\citep{coze}, Dify~\citep{dify}, and Flowise~\citep{flowise}. In these systems, the transitions between nodes are rigidly controlled by the program, with the LLM acting as a component within specific nodes to generate user responses, predict parameters for tool calls, or facilitate node transitions (e.g., classifying user intent).  In this paradigm, the information accessible to the agent and its action space are limited, which can be expressed as:
\begin{equation}
    a_t \leftarrow \mathcal{M}^{v}(\phi^{v}(\mathcal{H}_{t-1}), \psi^{v}(\mathcal{G})),
\end{equation}
where $v$ is the current node, $\phi^{v}(\mathcal{H}_{t-1})$ is the selected information visible to $v$, $\psi^{v}(\mathcal{G})$ is a subgraph of $\mathcal{G}$ expanded from $v$, and $\mathcal{M}^{v}$ denotes the language model associated with node $v$. 

The second category, \textbf{prompt-based agents} \citep{FlowBench,KnowAgent}, represents the workflow as text $\mathcal{W}^{(f)}$ using a specific format $f$, and a single language model $\mathcal{M}$ autonomously manages the entire decision-making and dialogue process. This process can be represented as:
\begin{equation}
    a_t \leftarrow \mathcal{M}(\mathcal{H}_{t-1}, \mathcal{G}^{(f)}),
\end{equation}
where $\mathcal{G}^{(f)}$ represents the graph structure implicitly encoded within $\mathcal{W}^{(f)}$. 
