\section{Related Work} 
\subsection{LLM-Driven Conversational Systems} \label{subsec:llm-tod} 
The evolution of task-oriented dialogue (TOD) systems has transitioned from modular pipelines \citep{LLM-dialogue-survey} to end-to-end LLM paradigms. While traditional systems suffered from error propagation across NLU, DST, and NLG modules \citep{SPACE-3,PPTOD}, modern approaches leverage LLMs for holistic dialogue management via workflow-guided interactions \citep{FlowBench,SystemPrompt}. 
This shift necessitates new evaluation metrics focusing on task success rates over modular accuracy \citep{DAG-test-gen}, motivating our framework's dual focus on procedural compliance and adaptive flexibility.


\subsection{Agentic Workflow Architectures} 
The progression of LLMs has led to the development of LLM-based agents across various domains \citep{GenerativeAgents,MedAgents,ChatDev}.
LLM-based agents enhance task execution through tool usage and dynamic planning \citep{ReAct,Toolformer,KnowledGPT,KnowAgent}. We distinguish two paradigms: 1) \emph{Workflow generation} creates procedures via LLM reasoning \citep{AutoFlow,CoRE,LLM+P,AutoAgents,PlanBench}, and 2) \emph{Workflow execution} operates within predefined structures \citep{FlowBench,WorFBench}. 

Our research primarily focuses on the latter paradigm, treating workflows as predefined knowledge to build robust, user-centric agents. Within this context, two main approaches are adopted to integrate structured workflows with linear-text-processing language models: 
1) \emph{Rule-based Approach}: This method involves hard-coding workflow transition rules as fixed logic, defining the current node and state transitions explicitly in the program.
2) \emph{Prompt-based Approach}: Here, workflows are represented in flexible formats such as natural language, code (or pseudocode), or flowchart syntax \citep{FlowBench,KnowAgent}.
Each method presents unique challenges: rule-based systems often lack flexibility, while prompt-based methods might deviate from intended procedures. Our solution aims to strike a balance between process control and adaptability, ensuring workflows are both structured and responsive to dynamic interactions.
