\tableDatasetStat
\tableDatasetCompare


\section{Methodology} \label{sec:method}
In this work, we introduce a novel procedural description language (PDL) designed to represent workflows, alongside \model, an execution framework that enhances the agent's behavioral control. 

\subsection{PDL Syntax} \label{subsec:pdl-syntax}
\begin{figure}[!t]
\begin{lstlisting}[language={}, basicstyle=\small\ttfamily] 
APIs:
- name: check_hospital
    pre: []
- name: check_department
    pre: ['check_hospital']
- name: query_appointment
    pre: ['check_department']
- name: register_appointment
    pre: ['query_appointment']
- name: recommend_other_hospitals
    pre: ['register_appointment']

    ANSWERs:
- name: inform_appointment_result
    pre: ['register_appointment']
...
- name: answer_out_of_workflow_questions
- name: request_information
\end{lstlisting}
\vspace{-0.2cm}
\caption{Example of Node Definations in PDL} \label{fig:node_example}
\vspace{-0.2cm}
\end{figure}

\begin{figure*}[!t]
\begin{lstlisting}[language={}, basicstyle=\small\ttfamily, numbers=left] 
while not API.check_hospital(hospital) or not API.check_department(hospital, department):
    hospital, department = ANSWER.request_information('hospital', 'department')
time = ANSWER.request_information('time')
available_list = query_appointment(hospital, department, time)
try:
    # ... collect necessory information for registration
    result = API.register_appointment(hospital, ...)
    ANSWER.inform_appointment_result(result)
except:
    # if registration fails, recommend other hospitals
    available_list = API.recommend_other_hospitals(department, time)
    # ... try to register again
\end{lstlisting}
\vspace{-0.2cm}
\caption{Example of Procedure Description in PDL} \label{fig:procedure_example}
\vspace{-0.2cm}
\end{figure*}

PDL consists of three primary components:
1) \textit{Meta Information}: Basic workflow details such as name and description.
2) \textit{Node Definitions}: Resources accessible to the agent, which include\emph{API} nodes (for external tool calls) and \emph{ANSWER} nodes (for user interaction).
2) \textit{Procedure Description}: The procedural logic of the task, expressed in a mix of natural language and pseudocode.

For illustration, in the \emph{Hospital Appointment} workflow, Fig.\ref{fig:node_example} presents a segment of the \emph{node definitions}
\footnote{For brevity, certain details have been omitted; see App.\ref{appendix:pdl_example} for the complete PDL specification.}.
Fig.\ref{fig:procedure_example} illustrates a portion of the \emph{procedure description}.  Key features of PDL include:
1) \textit{Precondition Specification}: Nodes include a \emph{preconditions} attribute, defining dependencies between nodes. For example, \codef{check\_department} requires \codef{check\_hospital} as a prerequisite, ensuring hospital selection before department inquiry.
2) \textit{Hybrid Representation}: The integration of natural language and code in the procedure description ensures a concise and yet flexible workflow representation, maintaining the clarity of NL with the accuracy of code.

\algo

\subsection{\model Architecture} \label{subsec:arch}

To enhance the compliance of workflow agents, we introduce \model, an execution framework tightly integrated with PDL. \model enforces a set of controllers that govern the agent's decision-making process, thereby promoting reliable action execution without sacrificing the LLM's autonomy.

Algorithm~\ref{algo} outlines \model's overall execution. Each round begins with a user query (line~\ref{alg:user}), which the agent interprets to produce a response or a tool call (line~\ref{alg:bot-action}), ultimately generating a user-facing response (line~\ref{alg:bot-response}).

To ensure decision-making stability, \model incorporates two categories of controllers: \emph{pre-decision} controllers ($\mathcal{C}_{\text{pre}} = \{c^{\text{pre}}_i\}_{i=1}^{C_{\text{pre}}}$) and \emph{post-decision} controllers ($\mathcal{C}_{\text{post}} = \{c^{\text{post}}_j\}_{j=1}^{C_{\text{post}}}$).
\textbf{Pre-decision controllers} proactively guide the agent's actions by evaluating the current state and providing feedback to the LLM (e.g., identifying unreachable nodes based on the dependency graph $\mathcal{G}^{(pdl)}$). This feedback, denoted by $\mathcal{R}_{\text{pre}}$, serves as a form of \emph{soft control}.
However, LLMs can still generate unstable outputs even with pre-decision guidance. Therefore, \textbf{post-decision controllers} provide \emph{hard constraints} by assessing the validity of proposed agent actions.  

We designed modular controllers to adjust the behavior of the workflow agent across multiple dimensions, such as \emph{enforcing node dependencies}, \emph{constraining API call repetition}, and \emph{limiting conversation length}. 
Below, using the workflow shown in Fig.~\ref{fig:node_example} as an example, we briefly introduce the \textbf{node dependency controller}.
It can operate in both pre- and post-decision modes. As a pre-decision controller ($c^{\text{pre}}_{\text{dep}}$), the controller analyzes the agent's current node and identifies inaccessible nodes by examining the dependency graph. For example, if the agent is at \codef{check\_hospital}, $c^{\text{pre}}_{\text{dep}}$ prevents the LLM from prematurely transitioning to \codef{query\_appointment} (soft control). As a post-decision controller ($c^{\text{post}}_{\text{dep}}$), the controller validates proposed node transitions.  For instance, if the agent attempts to transition to \codef{query\_appointment} without completing \codef{check\_department}, the controller denies the request, providing feedback to the agent.
