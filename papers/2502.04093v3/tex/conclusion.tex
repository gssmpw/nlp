\section{Conclusions and Future Work}
\label{sec:conclusion}

In this work, we present IPComp, an interpolation-based progressive lossy compression solution designed to address the growing need for efficient scientific data storage and retrieval. Our approach accomplishes progressiveness effectively with an interpolation prediction model, multi-level bitplanes, and predictive coding techniques. It is equipped with an optimizer to minimize the data volume during retrieval under given error bound or bitrate targets. 
% Unlike existing residual-based progressive compression techniques, our solution significantly reduces computational overhead by enabling direct retrieval at various fidelity levels without requiring multiple decompression passes.

Experimental evaluations conducted on six real-world scientific datasets demonstrate the effectiveness of our solution. IPComp consistently achieves the fastest speed, the highest compression ratios, the lowest data retrieval volume, and the highest data fidelity compared to state-of-the-art alternatives. Additionally, compared with residual-based solutions that only support limited retrieval options, our approach is very flexible on fidelity control as it takes arbitrary error bounds and bitrates as retrieval options. 

Our findings suggest that IPComp represents a significant advancement in progressive lossy compression and is a practical choice for scientific applications. The future work will focus on optimizing hardware acceleration (e.g., GPU and tensor cores), integrating with scientific workflows like HDF5, and expanding large-scale HPC evaluations. These improvements will further refine IPComp for broader application scenarios.