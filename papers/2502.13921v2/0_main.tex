%%
%% This is file `sample-sigconf.tex',
%% generated with the docstrip utility.
%%
%% The original source files were:
%%
%% samples.dtx  (with options: `sigconf')
%% 
%% IMPORTANT NOTICE:
%% 
%% For the copyright see the source file.
%% 
%% Any modified versions of this file must be renamed
%% with new filenames distinct from sample-sigconf.tex.
%% 
%% For distribution of the original source see the terms
%% for copying and modification in the file samples.dtx.
%% 
%% This generated file may be distributed as long as the
%% original source files, as listed above, are part of the
%% same distribution. (The sources need not necessarily be
%% in the same archive or directory.)
%%
%% Commands for TeXCount
%TC:macro \cite [option:text,text]
%TC:macro \citep [option:text,text]
%TC:macro \citet [option:text,text]
%TC:envir table 0 1
%TC:envir table* 0 1
%TC:envir tabular [ignore] word
%TC:envir displaymath 0 word
%TC:envir math 0 word
%TC:envir comment 0 0
%%
%%
%% The first command in your LaTeX source must be the \documentclass command.
\documentclass[sigconf,9pt]{acmart}
%% ASP-DAC: 9pt or 10pt font.
%% NOTE that a single column version may be required for 
%% submission and peer review. This can be done by changing
%% the \doucmentclass[...]{acmart} in this template to 
%% \documentclass[manuscript,screen]{acmart}
%% 
%% To ensure 100% compatibility, please check the white list of
%% approved LaTeX packages to be used with the Master Article Template at
%% https://www.acm.org/publications/taps/whitelist-of-latex-packages 
%% before creating your document. The white list page provides 
%% information on how to submit additional LaTeX packages for 
%% review and adoption.
%% Fonts used in the template cannot be substituted; margin 
%% adjustments are not allowed.
%%
%%
%% \BibTeX command to typeset BibTeX logo in the docs
\AtBeginDocument{%
  \providecommand\BibTeX{{%
    \normalfont B\kern-0.5em{\scshape i\kern-0.25em b}\kern-0.8em\TeX}}}

%% Rights management information.  This information is sent to you
%% when you complete the rights form.  These commands have SAMPLE
%% values in them; it is your responsibility as an author to replace
%% the commands and values with those provided to you when you
%% complete the rights form.
\setcopyright{acmlicensed}
% \setcopyright{rightsretained}
\copyrightyear{2025}
\acmYear{2025}
\acmDOI{10.1145/3658617.3697616}

%% These commands are for a PROCEEDINGS abstract or paper.
\acmConference[ASP-DAC'25]{Asia and South Pacific Design Automation Conference
}{January 20--23, 2025}{Tokyo Odaiba Miraikan, Japan}
%
%  Uncomment \acmBooktitle if th title of the proceedings is different
%  from ``Proceedings of ...''!
%
%\acmBooktitle{Woodstock '18: ACM Symposium on Neural Gaze Detection,
%  June 03--05, 2018, Woodstock, NY} 
\acmISBN{979-8-4007-0635-6/25/01}


%%%%%%%%%%%%Newly-added Package Begin%%%%%%%%%%%%%

\usepackage{amsmath,amsfonts}
\usepackage{graphicx}
\usepackage{textcomp}
\usepackage{bbding}
\usepackage{pifont}
\usepackage{multicol}

\setlength{\tabcolsep}{3pt}
\usepackage{booktabs}
\usepackage[binary-units=true]{siunitx}
\usepackage{array,multirow}
\usepackage{hyperref}
\usepackage{amsmath}
\usepackage{enumitem}
\usepackage{graphicx}
\usepackage{float}
\usepackage[symbol, hang,flushmargin]{footmisc}
\usepackage{bm}
\usepackage{subfigure}

\renewcommand{\thefootnote}{\fnsymbol{footnote}}
\usepackage{tablefootnote}

\usepackage[noend]{algpseudocode}

\usepackage[ruled,linesnumbered]{algorithm2e}
\SetKwComment{Comment}{$\triangleright$\ }{}
% for multiple inputs in algorithms
\newlength\mylenin
\newcommand\myinput[1]{%
\settowidth\mylenin{\KwIn{}}%
\setlength\hangindent{\mylenin}%
\hspace*{\mylenin}#1\\}

\let\oldnl\nl % Store \nl in \oldnl
\newcommand{\nonl}{\renewcommand{\nl}{\let\nl\oldnl}} % Remove line number for one line

\newlength\mylenout
\newcommand\myoutput[1]{%
\settowidth\mylenout{\KwOut{}}%
\setlength\hangindent{\mylenout}%
\hspace*{\mylenout}#1\\}

\fboxsep=1mm%padding thickness
\fboxrule=1pt%border thickness

\newcommand{\figref}[1]{Figure~\ref{#1}}
\newcommand{\secref}[1]{Section~\ref{#1}}
\newcommand{\subsecref}[1]{Subsection~\ref{#1}}
\newcommand{\tabref}[1]{Table~\ref{#1}}
\newcommand{\appendref}[1]{Appendix~\ref{#1}}
% \newcommand{\algref}[1]{Algorithm~\ref{#1}}
% \newcommand{\eqref}[1]{Equation~\ref{#1}}

\newcommand{\Hsection}[1]{\vspace{0.5\baselineskip}\par\noindent\textit{#1}~\textbf{---}~}
\DeclareMathOperator*{\argmax}{argmax}
\DeclareMathOperator*{\argmin}{argmin}


\newcolumntype{L}[1]{>{\raggedright\let\newline\\\arraybackslash\hspace{0pt}}m{#1}}
\newcolumntype{C}[1]{>{\centering\let\newline\\\arraybackslash\hspace{0pt}}m{#1}}
\newcolumntype{R}[1]{>{\raggedleft\let\newline\\\arraybackslash\hspace{0pt}}m{#1}}

% correct bad hyphenation here
\hyphenation{op-tical net-works semi-conduc-tor}

\usepackage{soul}
% \usepackage[dvipsnames]{xcolor}
\usepackage{xurl}
\usepackage{xspace}
\usepackage{tablefootnote}
%%%%%%%%%%%%Newly-added Package End%%%%%%%%%%%%%


%%
%% Submission ID.
%% Use this when submitting an article to a sponsored event. You'll
%% receive a unique submission ID from the organizers
%% of the event, and this ID should be used as the parameter to this command.
%%\acmSubmissionID{123-A56-BU3}

%%
%% For managing citations, it is recommended to use bibliography
%% files in BibTeX format.
%%
%% You can then either use BibTeX with the ACM-Reference-Format style,
%% or BibLaTeX with the acmnumeric or acmauthoryear sytles, that include
%% support for advanced citation of software artefact from the
%% biblatex-software package, also separately available on CTAN.
%%
%% Look at the sample-*-biblatex.tex files for templates showcasing
%% the biblatex styles.
%%

%%
%% The majority of ACM publications use numbered citations and
%% references.  The command \citestyle{authoryear} switches to the
%% "author year" style.
%%
%% If you are preparing content for an event
%% sponsored by ACM SIGGRAPH, you must use the "author year" style of
%% citations and references.
%% Uncommenting
%% the next command will enable that style.
%%\citestyle{acmauthoryear}

%%
%% end of the preamble, start of the body of the document source.

\begin{document}
\settopmatter{printacmref=false} % Removes citation information below abstract
%%
%% The "title" command has an optional parameter,
%% allowing the author to define a "short title" to be used in page headers.
\title{Exploring Code Language Models for Automated HLS-based Hardware Generation: Benchmark, Infrastructure and Analysis}

%%
%% The "author" command and its associated commands are used to define
%% the authors and their affiliations.
%% Of note is the shared affiliation of the first two authors, and the
%% "authornote" and "authornotemark" commands
%% used to denote shared contribution to the research.

\author{
Jiahao Gai$^{2}$, Hao (Mark) Chen$^{1}$, Zhican Wang$^3$, Hongyu Zhou$^4$, \\ Wanru Zhao$^2$, Nicholas Lane $^2$, Hongxiang Fan$^{1\&2}$\\
{\normalsize $^1$Imperial College London, $^2$University of Cambridge, $^3$Shanghai Jiao Tong University, $^4$University of Sydney}\\ 
{\normalsize Email: jg2123@cam.ac.uk, hongxiangfan@ieee.org}\\
}
%%
%% By default, the full list of authors will be used in the page
%% headers. Often, this list is too long, and will overlap
%% other information printed in the page headers. This command allows
%% the author to define a more concise list
%% of authors' names for this purpose.
\renewcommand{\shortauthors}{Gai et al.}
\renewcommand{\shorttitle}{Exploring Code Language Models for Automated HLS-based
Hardware Generation}
%%
%% The abstract is a short summary of the work to be presented in the
%% article.
\begin{abstract}
Recent advances in code generation have illuminated the potential of employing large language models (LLMs) for general-purpose programming languages such as \textit{Python} and \textit{C++}, opening new opportunities for automating software development and enhancing programmer productivity. 
The potential of LLMs in software programming has sparked significant interest in exploring automated hardware generation and automation.
Although preliminary endeavors have been made to adopt LLMs in generating hardware description languages (HDLs) such as \textit{Verilog} and \textit{SystemVerilog}, several challenges persist in this direction.
First, the volume of available HDL training data is substantially smaller compared to that for software programming languages. 
Second, the pre-trained LLMs, mainly tailored for software code, tend to produce HDL designs that are more error-prone. 
Third, the generation of HDL requires a significantly higher number of tokens compared to software programming, leading to inefficiencies in cost and energy consumption.
To tackle these challenges,
this paper explores leveraging LLMs to generate High-Level Synthesis (\textit{HLS})-based hardware design.
Although code generation for domain-specific programming languages is not new in the literature,
we aim to provide experimental results, insights, benchmarks, and evaluation infrastructure to investigate the suitability of \textit{HLS} over low-level HDLs for LLM-assisted hardware design generation.
To achieve this, we first finetune pre-trained models for \textit{HLS}-based hardware generation, using a collected dataset with text prompts and corresponding reference \textit{HLS} designs.
An LLM-assisted framework is then proposed to automate end-to-end hardware code generation, which also investigates the impact of chain-of-thought and feedback loops promoting techniques on \textit{HLS}- design generation.
Comprehensive experiments demonstrate the effectiveness of our methods.
Limited by the timeframe of this research, we plan to evaluate more advanced reasoning models in the future.
\end{abstract}

\begin{CCSXML}
<ccs2012>
 <concept>
  <concept_id>00000000.0000000.0000000</concept_id>
  <concept_desc>Do Not Use This Code, Generate the Correct Terms for Your Paper</concept_desc>
  <concept_significance>500</concept_significance>
 </concept>
 <concept>
  <concept_id>00000000.00000000.00000000</concept_id>
  <concept_desc>Do Not Use This Code, Generate the Correct Terms for Your Paper</concept_desc>
  <concept_significance>300</concept_significance>
 </concept>
 <concept>
  <concept_id>00000000.00000000.00000000</concept_id>
  <concept_desc>Do Not Use This Code, Generate the Correct Terms for Your Paper</concept_desc>
  <concept_significance>100</concept_significance>
 </concept>
 <concept>
  <concept_id>00000000.00000000.00000000</concept_id>
  <concept_desc>Do Not Use This Code, Generate the Correct Terms for Your Paper</concept_desc>
  <concept_significance>100</concept_significance>
 </concept>
</ccs2012>
\end{CCSXML}

%%%%%%%%%%%%%%% For submission: begin %%%%%%%%%%%%%%%%
% \ccsdesc[500]{Do Not Use This Code~Generate the Correct Terms for Your Paper}
% \ccsdesc[300]{Do Not Use This Code~Generate the Correct Terms for Your Paper}
% \ccsdesc{Do Not Use This Code~Generate the Correct Terms for Your Paper}
% \ccsdesc[100]{Do Not Use This Code~Generate the Correct Terms for Your Paper}
%%%%%%%%%%%%%%% For submission: end %%%%%%%%%%%%%%%%


%%
%% Keywords. The author(s) should pick words that accurately describe
%% the work being presented. Separate the keywords with commas.

%%%%%%%%%%%%%%% For submission: begin %%%%%%%%%%%%%%%%
% \keywords{Automated Hardware Code Generation, Generative Artificial Intelligence, High-Level Synthesis}
%%%%%%%%%%%%%%% For submission: end %%%%%%%%%%%%%%%%

%% A "teaser" image appears between the author and affiliation
%% information and the body of the document, and typically spans the
%% page.
% \received{20 February 2007}
% \received[revised]{12 March 2009}
% \received[accepted]{5 June 2009}

%%
%% This command processes the author and affiliation and title
%% information and builds the first part of the formatted document.

\maketitle
\textbf{\small ACM Reference Format:} \newline
{\small Jiahao Gai, Hao (Mark) Chen, Zhican Wang, Hongyu Zhou, Wanru Zhao, Nicholas Lane, Hongxiang Fan. 2025. \shorttitle. In \textit{Proceedings of Asia and South Pacific Design Automation Conference (ASP-DAC’25)}. ACM, New York,
NY, USA, 8 pages. \url{https://doi.org/10.1145/3658617.3697616}}
\section{Introduction}

Video generation has garnered significant attention owing to its transformative potential across a wide range of applications, such media content creation~\citep{polyak2024movie}, advertising~\citep{zhang2024virbo,bacher2021advert}, video games~\citep{yang2024playable,valevski2024diffusion, oasis2024}, and world model simulators~\citep{ha2018world, videoworldsimulators2024, agarwal2025cosmos}. Benefiting from advanced generative algorithms~\citep{goodfellow2014generative, ho2020denoising, liu2023flow, lipman2023flow}, scalable model architectures~\citep{vaswani2017attention, peebles2023scalable}, vast amounts of internet-sourced data~\citep{chen2024panda, nan2024openvid, ju2024miradata}, and ongoing expansion of computing capabilities~\citep{nvidia2022h100, nvidia2023dgxgh200, nvidia2024h200nvl}, remarkable advancements have been achieved in the field of video generation~\citep{ho2022video, ho2022imagen, singer2023makeavideo, blattmann2023align, videoworldsimulators2024, kuaishou2024klingai, yang2024cogvideox, jin2024pyramidal, polyak2024movie, kong2024hunyuanvideo, ji2024prompt}.


In this work, we present \textbf{\ours}, a family of rectified flow~\citep{lipman2023flow, liu2023flow} transformer models designed for joint image and video generation, establishing a pathway toward industry-grade performance. This report centers on four key components: data curation, model architecture design, flow formulation, and training infrastructure optimization—each rigorously refined to meet the demands of high-quality, large-scale video generation.


\begin{figure}[ht]
    \centering
    \begin{subfigure}[b]{0.82\linewidth}
        \centering
        \includegraphics[width=\linewidth]{figures/t2i_1024.pdf}
        \caption{Text-to-Image Samples}\label{fig:main-demo-t2i}
    \end{subfigure}
    \vfill
    \begin{subfigure}[b]{0.82\linewidth}
        \centering
        \includegraphics[width=\linewidth]{figures/t2v_samples.pdf}
        \caption{Text-to-Video Samples}\label{fig:main-demo-t2v}
    \end{subfigure}
\caption{\textbf{Generated samples from \ours.} Key components are highlighted in \textcolor{red}{\textbf{RED}}.}\label{fig:main-demo}
\end{figure}


First, we present a comprehensive data processing pipeline designed to construct large-scale, high-quality image and video-text datasets. The pipeline integrates multiple advanced techniques, including video and image filtering based on aesthetic scores, OCR-driven content analysis, and subjective evaluations, to ensure exceptional visual and contextual quality. Furthermore, we employ multimodal large language models~(MLLMs)~\citep{yuan2025tarsier2} to generate dense and contextually aligned captions, which are subsequently refined using an additional large language model~(LLM)~\citep{yang2024qwen2} to enhance their accuracy, fluency, and descriptive richness. As a result, we have curated a robust training dataset comprising approximately 36M video-text pairs and 160M image-text pairs, which are proven sufficient for training industry-level generative models.

Secondly, we take a pioneering step by applying rectified flow formulation~\citep{lipman2023flow} for joint image and video generation, implemented through the \ours model family, which comprises Transformer architectures with 2B and 8B parameters. At its core, the \ours framework employs a 3D joint image-video variational autoencoder (VAE) to compress image and video inputs into a shared latent space, facilitating unified representation. This shared latent space is coupled with a full-attention~\citep{vaswani2017attention} mechanism, enabling seamless joint training of image and video. This architecture delivers high-quality, coherent outputs across both images and videos, establishing a unified framework for visual generation tasks.


Furthermore, to support the training of \ours at scale, we have developed a robust infrastructure tailored for large-scale model training. Our approach incorporates advanced parallelism strategies~\citep{jacobs2023deepspeed, pytorch_fsdp} to manage memory efficiently during long-context training. Additionally, we employ ByteCheckpoint~\citep{wan2024bytecheckpoint} for high-performance checkpointing and integrate fault-tolerant mechanisms from MegaScale~\citep{jiang2024megascale} to ensure stability and scalability across large GPU clusters. These optimizations enable \ours to handle the computational and data challenges of generative modeling with exceptional efficiency and reliability.


We evaluate \ours on both text-to-image and text-to-video benchmarks to highlight its competitive advantages. For text-to-image generation, \ours-T2I demonstrates strong performance across multiple benchmarks, including T2I-CompBench~\citep{huang2023t2i-compbench}, GenEval~\citep{ghosh2024geneval}, and DPG-Bench~\citep{hu2024ella_dbgbench}, excelling in both visual quality and text-image alignment. In text-to-video benchmarks, \ours-T2V achieves state-of-the-art performance on the UCF-101~\citep{ucf101} zero-shot generation task. Additionally, \ours-T2V attains an impressive score of \textbf{84.85} on VBench~\citep{huang2024vbench}, securing the top position on the leaderboard (as of 2025-01-25) and surpassing several leading commercial text-to-video models. Qualitative results, illustrated in \Cref{fig:main-demo}, further demonstrate the superior quality of the generated media samples. These findings underscore \ours's effectiveness in multi-modal generation and its potential as a high-performing solution for both research and commercial applications.
\section{Background} \label{section:LLM}

% \subsection{Large Language Model (LLM)}   

Figure~\ref{fig:LLaMA_model}(a) shows that a decoder-only LLM initially processes a user prompt in the “prefill” stage and subsequently generates tokens sequentially during the “decoding” stage.
Both stages contain an input embedding layer, multiple decoder transformer blocks, an output embedding layer, and a sampling layer.
Figure~\ref{fig:LLaMA_model}(b) demonstrates that the decoder transformer blocks consist of a self attention and a feed-forward network (FFN) layer, each paired with residual connection and normalization layers. 

% Differentiate between encoder/decoder, explain why operation intensity is low, explain the different parts of a transformer block. Discuss Table II here. 

% Explain the architecture with Llama2-70B.

% \begin{table}[thb]
% \renewcommand\arraystretch{1.05}
% \centering
% % \vspace{-5mm}
%     \caption{ML Model Parameter Size and Operational Intensity}
%     \vspace{-2mm}
%     \small
%     \label{tab:ML Model Parameter Size and Operational Intensity}    
%     \scalebox{0.95}{
%         \begin{tabular}{|c|c|c|c|c|}
%             \hline
%             & Llama2 & BLOOM & BERT & ResNet \\
%             Model & (70B) & (176B) & & 152 \\
%             \hline
%             Parameter Size (GB) & 140 & 352 & 0.17 & 0.16 \\
%             \hline
%             Op Intensity (Ops/Byte) & 1 & 1 & 282 & 346 \\
%             \hline
%           \end{tabular}
%     }
% \vspace{-3mm}
% \end{table}

% {\fontsize{8pt}{11pt}\selectfont 8pt font size test Memory Requirement}

\begin{figure}[t]
    \centering
    \includegraphics[width=8cm]{Figure/LLaMA_model_new_new.pdf}
    \caption{(a) Prefill stage encodes prompt tokens in parallel. Decoding stage generates output tokens sequentially.
    (b) LLM contains N$\times$ decoder transformer blocks. 
    (c) Llama2 model architecture.}
    \label{fig:LLaMA_model}
\end{figure}

Figure~\ref{fig:LLaMA_model}(c) demonstrates the Llama2~\cite{touvron2023llama} model architecture as a representative LLM.
% The self attention layer requires three GEMVs\footnote{GEMVs in multi-head attention~\cite{attention}, narrow GEMMs in grouped-query attention~\cite{gqa}.} to generate query, key and value vectors.
In the self-attention layer, query, key and value vectors are generated by multiplying input vector to corresponding weight matrices.
These matrices are segmented into multiple heads, representing different semantic dimensions.
The query and key vectors go though Rotary Positional Embedding (RoPE) to encode the relative positional information~\cite{rope-paper}.
Within each head, the generated key and value vectors are appended to their caches.
The query vector is multiplied by the key cache to produce a score vector.
After the Softmax operation, the score vector is multiplied by the value cache to yield the output vector.
The output vectors from all heads are concatenated and multiplied by output weight matrix, resulting in a vector that undergoes residual connection and Root Mean Square layer Normalization (RMSNorm)~\cite{rmsnorm-paper}.
The residual connection adds up the input and output vectors of a layer to avoid vanishing gradient~\cite{he2016deep}.
The FFN layer begins with two parallel fully connections, followed by a Sigmoid Linear Unit (SiLU), and ends with another fully connection.
\section{Dataset Details}
\label{appendixdata}
The full list of songs we selected are detailed in Tables \ref{tab:en_list}, \ref{tab:zh_list}, \ref{tab:fr_list}, \ref{tab:ko_list}. The lyrics of these
songs will not be publicly released but will be available upon request for research purposes only, ensuring their appropriate use. The songs are selected from leaderboards on music platforms such as Apple Music, with release years ranging from 1930 to 2012. We manually ensure that all of the songs are copyright protected. U.S., China, France, and Korea are all signatories to the Berne Convention \cite{berne_convention}, which is an international treaty that ensures that works created in one member country are automatically protected by copyright in all other member countries without the need for formal registration \cite{ricketson2022international}. This means that a Chinese song is automatically protected by U.S. copyright law as soon as it is created and fixed in a tangible medium (e.g., recorded or written down).

\section{\tool System}\label{sec:system_description}

\begin{figure*}[t]
    \centering
    \includegraphics[width=0.85\textwidth]{src/img/inference.pdf}
    \vspace{-2mm}
    \caption{Our sketch-aware prompt recommendation first builds a semantic space through data-driven analysis of key semantic elements covering single object and cross object properties. Then the semantic space is integrated with retrieval of attributes and relationships reference from semantic dataset. Finally, these semantic guidance is combined with users' initial sketch to form a sketch-aware multi-modal prompt to the MLLM to support spatial-aware inference.}
    \Description{Our sketch-aware prompt recommendation first builds a semantic space through data-driven analysis of key semantic elements covering single object and cross-region properties. Then the semantic space is integrated with retrieval of attributes and relationships reference from semantic dataset. Finally, these semantic guidance is combined with users' initial sketch to form a sketch-aware multi-modal prompt to the MLLM to support spatial-aware inference.}
    \label{fig:inference}
    % \vspace{-3mm}
\end{figure*}

Based on the design goals, we design \tool, an interactive system that allows users to generate images controlled by inputs of rough sketches and simple prompts, while generating semantically cohesive and region-controlled images.
The overall framework of \tool is shown in Figure~\ref{fig:workflow}. 
The framework consists of three stages: 1) the user can draw a sketch, assign a corresponding regional prompt, and use automatic prompt recommendation to refine their initial prompt \textbf{(G1, G2)}; 2) the sketch with object decomposition and single-object generation \textbf{(G3)}; and 3) spatial adjustment and anchoring of object shapes   \textbf{(G1, G3)}.


\subsection{Sketch-Aware Prompt Recommendation}
\label{ssec:prompt_rec}
Crafting effective prompts for rough sketch-based image generation is a challenging task, as users must not only create prompts for each individual region but also ensure coherence across the entire image. 
We introduce a prompt recommendation method that automatically enhances the user’s initial input, to produce a spatially cohesive prompt that aligns with the overall composition.
Figure~\ref{fig:inference} shows the overall workflow of this process.


\subsubsection{Semantic Space Reasoning}
\begin{table}[thb] \small
    \centering
    \caption{Common examples in the semantic space across various T2I description datasets.}
    \vspace{-2mm}
    \begin{tabular}{|l|l|c|}
    \hline 
    \textbf{Space Item} & \textbf{Property} & \textbf{Instance} \\
    \hline 
    \multirow{3}{*}{\textbf{Single Object}} 
    & Type &   Man, Car, Dog, Tree, Window \\
    && Table, Ocean, Park, Wall\\ \cline{2-3}
    & Attribute & Wooden, Tall, Red, Large\\
    &&Fluffy, Round, Slim, Silver   \\ \cline{2-3}
    & State &   Standing, Moving, Swaying, \\
    &&Broken, Sleeping, Lying\\\hline
    \multirow{2}{*}{\textbf{Cross Object}} 
    &  Direction  &  Facing to, Aligned in, \\
    &&Diagonally placed\\\cline{2-3}
    & Relationship & Next to, Under, Parked on, \\
    &&Sitting by, Supporting\\ \hline
    \multirow{3}{*}{\textbf{Overall}}  
    & Lightning  &   Natrual daylight, indoor lighting, \\
    && Soft light, Moon light\\\cline{2-3}
    & Camera  &   Close-up shot, Wide-angle shot, \\
    && Overhead shot, Extreme long shot \\\cline{2-3}
    & Style  &   Realistic, Minimalist, Cinematic, \\
    && Abstractm, Anime, Oil painting\\
    \hline
    \end{tabular}
    \vspace{-1mm}
    \label{tab:example_space}
\end{table}



Previous prompt-tuning methods have primarily focused on text-to-image models, which offer limited support for refining individual region prompts and often struggle to ensure cohesiveness across the entire image.
To address this, the first step is to identify the types of prompts needed to generate a coherent image under rough sketch-based control.
Specifically, we define an "\emph{semantic space}" that provides intuitive guidance that helps users to easily input and adjust their prompts in the appropriate regions.
The process of constructing this semantic space involves analyzing online sources~\cite{sd,wang2022diffusiondb,civitai,xie2023prompt} and prompt-guideline literature~\cite{oppenlaender2023prompting,oppenlaender2023taxonomy,liu2022design} to identify critical elements that contribute to high-quality image generation. Additionally, examining image description datasets in computer vision—including traditional task datasets~\cite{caesar2018coco,pham2021learning,krishna2017visual} and recent datasets tailored for image generation~\cite{onoe2024docci}—helps uncover relevant dimensions for describing images.

\begin{figure*}[t]
    \centering
    \includegraphics[width=0.995\textwidth]{src/img/refinement.pdf}
    \vspace{-2mm}
    \caption{Spatial-condition sketch refinement can help novice users refine their sketch by generating more realistic and accurate sketch for each object through single object decomposition and generation, and subsequently allowing users to interactively refine the sketch by object selection and spatial adjustment.}
    \Description{Spatial-condition sketch refinement can help novice users refine their sketch by generating more realistic and accurate sketch for each object through single object decomposition and generation, and subsequently allowing users to interactively refine the sketch by object selection and spatial adjustment.}
    \label{fig:decompose}
    \vspace{-3mm}
\end{figure*}

\begin{table*}[thb] 
    \centering
    \caption{Example Statistics of objects, attributes and relationships in Visual Genome~\cite{krishna2017visual} and VAW~\cite{pham2021learning}.}
    \vspace{-2mm}
    \begin{tabular}{lccccc}
    \hline 
    \textbf{Space Item} & \textbf{1st} & \textbf{2nd} & \textbf{3rd} & \textbf{20th} & \textbf{50th}  \\
    \hline 
    Object & window (52k)  & man (52k) & shirt (39k) & trees (17k) & sidewalk (8k)  \\
    Attribute & white (311k) & black (195k) &  blue (118k) & clear (15k) & colorful (5.8k)   \\
    Relationship  & on (645k) & has (245k) &  in (219k) & sitting on (13k) & laying on (3.5k)    \\
    \hline
    \end{tabular}
    \vspace{-1mm}
    \label{tab:statsitic}
\end{table*}

We summarize these prompt aspects and conduct experiments to identify the key components essential for high-quality output. 
% Finally, we propose a semantic space that integrates these necessary elements to ensure coherence when generating images from rough sketches.
Table~\ref{tab:example_space} presents the common dimensions and example instances of the identified semantic space within the T2I datasets.
The specific elements within the space are listed below.
\begin{itemize}
    \item \textbf{Single Object Prompt} contains local prompts for each single object. 
    It contains \textit{type} of the object; \textit{attribute} of object including main attributes such as color, texture, shape; and \textit{state} indicates how the object acts, including still, standing, running, etc. The background is a special object that only has type and attribute.
    \item \textbf{Cross Object Prompt} explicitly specifies how objects in different regions interact, which is crucial for the coherence of the generated image.
    It includes \textit{direction} of objects and \textit{relationship} that multiple objects interact with each other.
    \item \textbf{Overall Prompt} does not directly affect multi-object cohesiveness but allows users to optionally specify the overall visual effect, including \textit{lighting}, \textit{style} and \textit{camera}.
    % \item \textbf{Overall Prompt} is not indispensable for a semantically cohesive image generation, but has significant impact in visual effect, include \textit{lighting}, \textit{style} and \textit{camera}. 
    % Previous works consider quality modifiers (e.g., best quality, high resolution). In our experiment, the state-of-the-art model can already generate high quality images without these modifiers.
\end{itemize}




Figure~\ref{fig:inference} (right) illustrates a completed semantic space for a sketch featuring a girl, a cat, and a background (\textit{i.e.}, areas without objects).
Once the individual prompt components are defined, the separate prompts within a single region are concatenated into one unified prompt using commas (\textit{e.g.}, "type: girl, attribute: long hair" becomes "girl, long hair"). 
% To ensure that the concatenated prompt remains within the 77-token limit imposed by CLIP, we employ the bag-of-conditions approach~\cite{omost}.}



\subsubsection{Sketch-Augmented Prompting}
While the semantic space simplifies the prompt input and adjustment process, manually entering all prompts and identifying the appropriate ones can still be labor intensive and cognitively demanding, particularly when dealing with many objects. 
To alleviate this burden, we utilize a MLLM GPT-4o, to automatically complete the semantic space based on the user's sketch and initial prompt.

Specifically, the user’s initial prompt and rough sketch are input into the MLLM, with the semantic space acting as a contextual guide within the prompt template. The model generates prompts to populate the semantic space, incorporating both the initial input prompt and the sketch. It utilizes spatial reasoning, guided by the Chain-of-Thought~\cite{wei2022chain} strategy, to account for the \textit{shape}, \textit{location}, and \textit{interaction} of the objects within the user's sketch.
However, relying solely on the MLLM to fill the semantic space can be risky, as it may overfit to certain content or produce results with reduced coherence~\cite{cao2023beautifulprompt}. 
To address this, we further enhance the process by retrieving reference attributes and relationships between objects from crowd-sourced text-image datasets based on real-world images~\cite{pham2021learning,krishna2017visual}. 
The datasets are organized into a dictionary, where object names serve as keys and their corresponding attributes or relationships are stored as values. 
During retrieval, \tool randomly samples $k=10$ examples from the values based on the given object names as keys, providing the MLLM with reference data. 
The raw datasets are sourced and publicly available from ~\cite{pham2021learning,krishna2017visual}.
Example statistics are shown in Table~\ref{tab:statsitic}. 
The rich semantics in these datasets enhance both diversity and coherence in the completed semantic space.

% Table~\ref{tab} presents example statistics of objects, attributes, and relationships within these datasets. 
The overall prompt generation process, as shown in Figure~\ref{fig:inference}, incorporates user input, semantic space, and retrieved attributes and relationships to guide generation. 
We employ few-shot learning~\cite{wang2023large} to enhance the quality and robustness of the generated prompts.


\begin{figure*}[t]
    \centering
    \includegraphics[width=0.995\textwidth]{src/img/ablation_zoom.pdf}
    \vspace{-2mm}
    \caption{Ablation study shows that prompt recommendation avoids common issues like missing objects and unrealistic relationships while sketch refinement further enhances fine-grained control.}
    \Description{Ablation study shows that prompt recommendation avoids common issues like missing objects and unrealistic relationships while sketch refinement further enhances fine-grained control.}
    \label{fig:ablation}
    \vspace{-4mm}
\end{figure*}

\subsection{Spatial-Condition Sketch Refinement}
\label{ssec:sketch_refine}
To help users refine their rough sketches into fine-grained shapes that align with their intentions and allow for iterative refinement, we propose a decompose-and-recompose approach. 
As Figure~\ref{fig:decompose} shows, the sketch is first decomposed into individual objects.
The users can then generate, select, and adjust the desired single-object images. 
Finally, these selected objects, along with their spatial conditions, are combined to generate the final result.

\subsubsection{Single Object Decomposition}
Instead of directly using a single object sketch for generation, we first classify objects into two categories: \textbf{thing} as foreground object with specific shapes, such as humans, animals, or chairs, and \textbf{stuff} as background object without a defined shape, like oceans, grass, or sky, based on ~\cite{caesar2018coco}. 
% \new{"Things" are objects with specific shapes, such as humans, animals, or chairs, while "stuff" refers to objects without a defined shape, like oceans, grass, or the sky.}
To classify an object, we compute the word embedding of its type and find the nearest match in a category list containing "things" and "stuff" in the object list in the COCO-Stuff dataset~\cite{caesar2018coco}.
During single object decomposition stage, each thing object is extracted using FAST SAM (Segment Anything)~\cite{zhao2023fast}, to make sure the single object generation maximally preserves consistency to the original sketch.
Once the single object sketch is decomposed, the generation of each individual object is carried out. 
Since the goal is to generate fine-grained object shapes but not the final image, the process can be accelerated by using low-step inference (6 steps) with the Lightning Diffusion model~\cite{luo2023lcm} and a lower resolution (512x512). 
In our experiment, generating 12 images took approximately 4 seconds on a GTX 4090.


To ensure that the generated result aligns more closely with the user's sketch, we filter the generated images based on the Intersection over Union (IoU) and CLIP score~\cite{hessel2021clipscore}, which measure spatial correspondence and semantic alignment, respectively, between the user sketch and the generated single object. 
We compute a weighted sum of the IoU and CLIP scores, then sort the images and select the top four for the user to choose from.
Once the user selects an image containing the desired object shape, the target object is automatically extracted using FAST SAM~\cite{zhao2023fast}.
Each object with refined shape will automatically replace the original rough sketch.
If users have no desired image in one generation, they can perform multi-round generation until finding the desired one.
Figure~\ref{fig:decompose} shows an example in which the rough sketch of a girl and a dog is refined to specific shapes.


\begin{figure*}[t]
    \centering
    \includegraphics[width=0.8\textwidth]{src/img/interface.pdf}
    \vspace{-3mm}
    \caption{\tool interface consists of (a) Canvas view, (b) Prompt Recommend view, (c) Sketch Refine view and (d) Result view.}
    \Description{SketchFlex interface consists of (a) Canvas view, (b) Prompt Recommend view, (c) Sketch Refine view and (d) Result view.}
    \label{fig:interface}
    \vspace{-3mm}
\end{figure*}

\subsubsection{Single Object Adjustment}
Our system provides flexible control not only over the shapes of objects but also over their size and position. As shown in Figure~\ref{fig:decompose}
, users can easily adjust the size and spatial placement of each object. 
Once adjustments are made, the corresponding single object shape mask is moved accordingly. 
All individual object shapes are then combined into an "anchor" image.
% The term "anchor" represents the idea that the generated result, like a boat floating within the rough sketch regions, may vary with each generation, just as a boat drifts on water. 
% However, with the anchor in place, the boat remains fixed to a specific location—similar to how the selected object shapes remain fixed in the generated result. 
Shape anchoring refers to the process of fixing the object shapes in the generated result.
To apply this shape anchoring, we extract the edges of the selected object using Canny edge detection and feed them into ControlNet to ensure that the final output adheres to the user's shape preferences. 
Once users have made their desired adjustments, they can generate an image with the exact fine-grained shapes they prefer.


A related issue with the original rough sketch-based generation is that a single prompt is applied to each region separately. 
To generate images that capture relationships between objects, we create a joint mask, \textit{i.e.}, the union of object masks, for two related objects, allowing the relationship prompt to influence the interaction between them. 
Equation 1 illustrates the creation of a joint mask for the relationship between region $i$ and region $j$, 
where $\oplus$ denotes the concatenation operation.

\begin{equation}
M_{ij} = [M_i \oplus M_j].
\end{equation}

However, using a joint mask alone can sometimes result in multiple objects being generated within a single object area. 
To address this, we apply negative prompts to exclude objects outside the intended region, preventing unwanted elements from appearing in the wrong areas, as shown in Equations 2-3.
In Equation 2, the cross-attention map that correlates text and image patches is updated such that the text embedding $C_i$ is amplified by a scalar $\lambda_{m_i}$ within the masked region $M_i$.
In Equation 3, the relationship embedding condition is reinforced within $M_{ij}$, while the influence of $C_i / C_j$ is reduced in the complementary areas.

\begin{equation}
A_i \leftarrow \lambda_{m_i} \cdot C_i \odot M_i,
\end{equation}

\begin{equation}
A_{(i,j)} \leftarrow \lambda_{m_{ij}} \cdot C_{ij} \odot M_{ij} - \lambda_{m_{ij}} \cdot C_i \odot (M_{ij}-M_i) - \lambda_{m_{ij}} \cdot C_j \odot (M_{ij} - M_j).
\end{equation}



Figure~\ref{fig:ablation} shows the ablation results of our system’s functions.
Without prompt recommendation and sketch refinement, the rough sketch-based generation often produces undesirable outcomes, such as missing objects, unrealistic relationships, and incorrect perspectives.
By incorporating our prompt recommendation, which refines prompts to be spatially aligned with the sketch, the results become more stable and coherent, with the generated objects closely matching the original sketch.
Finally, with prompt recommendation and sketch refinement, users can achieve fine-grained control, enabling them to generate detailed results that align with their sketches and creative intentions. For example, in the first column, the posture of the man lying on the chair with both arms spread out is more accurately captured in our result compared to the other two. 
In the third column, the girl's head is positioned slightly to the right and below the car within the mask, which aligns with our result. In contrast, in the other two results, the girl's head overlaps with the car, resulting in an incorrect spatial relationship.










\section{Evaluation}

\subsection{Experimental Setup}\label{subsec:exp_setup}

In our evaluation, we adopt \textit{Code-Llama-7B} as the pre-trained model for fine-tuning, employing the low-rank-adaption (QLoRA)~\cite{hu2021lora, dettmers2024qlora} technique for faster training and lower memory consumption. 
Key configurations include loading the model in 8-bit, a sequence length of 4096, sample packing, and padding to sequence length. We set the warmup steps to 100, with a gradient accumulation of 4 steps, a micro-batch size of 4, and an inference batch size of 2.
For both syntax and functionality checks, we measure pass@3 accuracy as metrics. In the ablation study from~\secref{subsec:exp_finetune} to~\secref{subsec:complexity}, we adopt \textit{MachineGen} for evaluation.

Experiments are conducted on a server with four NVIDIA L20 GPUs (48 GB each), an 80 vCPU Intel® Xeon® Platinum 8457C, and 100GB of RAM. This setup ensures sufficient computational power and memory to handle the intensive demands of fine-tuning and inference efficiently, especially for long data sequences in the feedback loop experiment. 

\subsection{Effect of Supervised Finetuning}\label{subsec:exp_finetune}
Our first ablation study investigates the effect of the model fine-tuning.
We evaluated the performance based on both syntax and functionality checks. 
As shown in~\figref{fig:finetune_cot}(a), the results demonstrate that the finetuning dramatically increases syntax correctness from $54.85$\% to $88.44$\%. 
More importantly, the impact of finetuning is even more pronounced in the functionality evaluation, where the non-finetuned model failed to achieve any correct functionality test, but the accuracy is improved to $53.20$\% in the finetuned model. These enhancements highlight the critical role of finetuning in producing not only syntactically correct but also functionally viable codes, which demonstrates the benefits of finetuning LLMs for hardware design in the HLS code generation task.




\subsection{Effect of Chain-of-Thought Prompting}\label{subsec:exp_cot}
To assess the effect of the chain-of-thought (CoT) technique,
we perform both syntax and functionality evaluation on the fine-tuned model with and without the use of CoT.
As indicated in~\figref{fig:finetune_cot}(b), incorporating CoT leads to a noticeable improvement in both metrics. 
Specifically, syntax correctness increases from $88.44$\% to $94.33$\%, and functionality score rises from $53.20$\% to $61.45$\%. 
The result demonstrates the effectiveness of CoT in enhancing the reasoning capability, thereby improving its overall performance.


\subsection{Effect of Feedback Loops}\label{subsec:exp_feedback}
Our two-step feedback loop provides both syntax and functionality feedback. We evaluate the impact of these feedback loops with different numbers of iterations, ranging from 0 to 2.The results, shown in Figure ~\figref{fig:syntax_feedback} and ~\figref{fig:func_feedback}, indicate that both syntax and functionality feedback loops significantly improve model performance, especially when combined with COT prompting. The initial feedback loop yields substantial accuracy improvements in both syntax correctness and functionality evaluation, though the second loop shows diminishing returns.Syntax feedback loops enhance both syntax correctness and functionality performance, suggesting that iterative refinement is particularly effective for complex tasks. Similarly, functionality feedback loops not only improve functionality checks but also boost syntax accuracy, indicating that enhancements in functional understanding contribute to better syntactic performance.

 \begin{figure}[t]
    \centering
    \includegraphics[width=1\linewidth]{./figures/merged_finetune_cot.pdf}
    \vspace{-5mm}
    \caption{Effect of fine-tuning and chain-of-thought.}
    \label{fig:finetune_cot}
\end{figure}

\begin{figure}[t]
    \centering
    \includegraphics[width=0.95\linewidth]{./figures/Effect_of_Syntax_Feedback_Loop.pdf}
    \vspace{-2mm}
    \caption{Effect of syntax feedback loop.}
    \vspace{-2mm}
    \label{fig:syntax_feedback}
\end{figure}
\begin{figure}[t]
    \centering
    \includegraphics[width=0.95\linewidth]{./figures/Effect_of_Functionality_Feedback_Loop.pdf}
    \caption{Effect of functionality feedback loop.}
    \vspace{-2mm}
    \label{fig:func_feedback}
\end{figure}


\subsection{Time Cost and Hardware Performance}\label{subsec:exp_timecost}

\figref{fig:time_cost} shows the time cost for generating 120 data entries under different conditions, measuring the impact of CoT and feedback loops. Without a feedback loop, CoT significantly reduces the time. Adding a syntax feedback loop increases the time, but CoT continues to notably decrease the duration. The functionality feedback loop is the most time-consuming, though CoT still provides a notable reduction, albeit less dramatic. This demonstrates CoT's effectiveness in reducing operational times across varying complexities.

For the test set,
we evaluate the latency and resource consumption of the generated \textit{HLS} designs using a Xilinx VCU118 as our target FPGA, with a clock frequency of $200$MHz and Xilinx Vivado 2020.1 for synthesis.
As shown in~\tabref{tb:perf_resource}, all \textit{HLS} designs demonstrate reasonable performance, with BRAM usage consistently remained at zero due to the design scale.

\begin{figure}
    \centering
    \includegraphics[width=0.95\linewidth]{./figures/Time_Cost_Analysis.pdf}
    \vspace{-2mm}
    \caption{Time cost of code generation.}
    \label{fig:time_cost}
\end{figure}


\begin{table}[htb]
\centering
\caption{Latency and resource usage of LLM-generated designs synthesized on a VCU118 FPGA.}
\label{tb:perf_resource}
\setlength\tabcolsep{1pt} 
\scalebox{0.8}{
% \begin{tabular}{L{2cm}ccC{1.5cm}C{2.5cm}}
\begin{tabular}{C{2.5cm}|C{1.9cm}|C{1.5cm}|C{1.5cm}|C{1.3cm}|C{1.3cm}}
\toprule
{}& \textbf{Latency} (ms)& \textbf{LUTs} & \textbf{Registers} & \textbf{DSP48s} & \textbf{BRAMs} \\ \midrule
{\textbf{Available}} & - & 1182240 & 2364480 & 6840 & 4320 \\ \midrule
{\textit{ellpack}} & 0.304 & 1011 & 1079 & 11 & 0 \\
{\textit{syrk}} & 21.537 & 1371 & 1621 & 19 & 0 \\
{\textit{syr2k}} & 40.626 & 1572 & 1771 & 19 & 0 \\
{\textit{stencil2d}} & 1.368 & 287 & 123 & 3 & 0 \\
{\textit{trmm-opt}} & 15.889 & 1262 & 1239 & 11 & 0 \\
{\textit{stencil3d}} & 21.537 & 1173 & 1271 & 20 & 0 \\
{\textit{symm}} & 24.601 & 1495 & 1777 & 19 & 0 \\
{\textit{symm-opt}} & 16.153 & 1361 & 1608 & 19 & 0 \\
{\textit{symm-opt-medium}} & 579.0 & 2223 & 2245 & 22 & 0 \\
\bottomrule
\end{tabular}}
\end{table}

\subsection{Effect of Task Complexity}\label{subsec:complexity}
We analyze the effects of code complexity on the performance of fine-tuning our language model with CoT prompting and tested without the use of any feedback loops during inference. 
We categorize \textit{MachineGen} into three classes according to their code complexity: easy, medium, and difficult.
The results shown in the \tabref{tab:model_performance} indicates a clear trend: as the complexity of the generated code increases, both syntax and functionality correctness rates decline. This outcome could be attributed to several factors. First, more complex code inherently presents more challenges in maintaining syntactic integrity and functional accuracy. Second, the absence of feedback loops in the inference phase may have limited the model's ability to self-correct emerging errors in more complicated code generations.

\begin{table}[h]
\centering
\caption{Performance across different complexity levels.}
\scalebox{0.8}{
\begin{tabular}{c|c|c}
\hline
\textbf{Test Set} & \textbf{Syntax Check} & \textbf{Functionality} \\
\hline
Easy & 96.67\% & 63.33\% \\
Medium & 96.67\% & 53.33\% \\
Difficult & 90\% & 53.33\% \\
\hline
\end{tabular}}
\label{tab:model_performance}
\end{table}

\subsection{Analysis of \textit{MachineGen} and \textit{HumanRefine}}
\begin{table}[h]
\centering
\caption{Performance on \textit{MachineGen} and \textit{HumanRefine}.}
\scalebox{0.9}{
\begin{tabular}{c|c|c}
\hline
\textbf{Test Set} & \textbf{Syntax Check} & \textbf{Functionality Check} \\
\hline
\textit{MachineGen} & 93.83\% & 62.24\% \\
\hline
\textit{HumanRefine} & 47.29\% & 21.36\% \\
\hline
\end{tabular}}
\vspace{-3mm}
\label{table:eval_comparison}
\end{table}

As shown in~\tabref{table:eval_comparison}, this section compares the performance of our model on \textit{MachineGen} and \textit{HumanRefine} test sets.
Our findings reveal that the performance on the \textit{HumanRefine} is significantly lower than on the \textit{MachineGen}. This disparity suggests that the model is more adept at handling machine-generated prompts. The primary reasons for this are: the model's training data bias towards machine-generated prompts, the increased complexity and nuanced nature of human-generated prompts, and the conciseness and clarity of human-generated prompts that often omit repetitive or explicit details found in machine-generated prompts, making it harder for the model to generate syntactically and functionally correct code.

\subsection{Thoughts, Insights, and Limitations}

\noindent \textbf{1. \textit{HLS} versus \textit{HDL} for AI-assisted code generation:} The selection of programming language for hardware code generation should mainly depend on two  factors:
\begin{itemize}[leftmargin=*]
    \item \textit{Quality of Generated Hardware Design}: The evaluation of hardware design's quality includes syntax correctness, functionality, and hardware performance.
    Since \textit{HLS} shares similar semantics and syntax with programming languages commonly used during LLM pre-training, this work demonstrates that the LLM-assisted code generation for \textit{HLS} has the potential to achieve high syntax and functional correctness in hardware designs. While this work does not leverage hardware performance as feedback for design generation, it identifies this aspect as a key direction for future research and enhancements.
    \item \textit{Runtime Cost of Hardware Generation}: Although \textit{HLS}-based designs typically require fewer tokens compared to \textit{HDL} during the code generation phase—suggesting potentially lower costs—the overall runtime costs associated with HLS synthesis must also be considered. A more comprehensive quantitative comparison of these runtime costs is planned for our future work. 
\end{itemize}

\noindent \textbf{2. Input instructions and datasets are crucial}: The fine-tuning of pre-trained LLMs on \textit{HLS} dataset can bring a significant improvement in the design quality, echoing findings from previous studies on \textit{Verilog} code generation~\cite{thakur2023verigen}. 
Additionally, during our evaluation, we found that employing simple CoT prompting largely improves hardware design quality. 
This result contrasts with the application of CoT in general-purpose programming languages, where a specialized form of CoT is necessary~\cite{li2023structured}.
Therefore, future efforts for further enhancement can focus on collecting high-quality datasets and exploring better refinement of input prompts.

\noindent \textbf{3. Limitations}: At the time of this research, more advanced reasoning models, such as DeepSeek-R1~\cite{guo2025deepseek}, were not available for evaluation. Additionally, test-time scaling approaches~\cite{welleck2024decoding} could be incorporated to further enhance performance in the future.
Moreover, we observe that the diversity of hardware designs in the benchmark is limited, which may impact the generalizability of our findings.
We intend to address these limitations in our future work.

\paragraph{Summary}
Our findings provide significant insights into the influence of correctness, explanations, and refinement on evaluation accuracy and user trust in AI-based planners. 
In particular, the findings are three-fold: 
(1) The \textbf{correctness} of the generated plans is the most significant factor that impacts the evaluation accuracy and user trust in the planners. As the PDDL solver is more capable of generating correct plans, it achieves the highest evaluation accuracy and trust. 
(2) The \textbf{explanation} component of the LLM planner improves evaluation accuracy, as LLM+Expl achieves higher accuracy than LLM alone. Despite this improvement, LLM+Expl minimally impacts user trust. However, alternative explanation methods may influence user trust differently from the manually generated explanations used in our approach.
% On the other hand, explanations may help refine the trust of the planner to a more appropriate level by indicating planner shortcomings.
(3) The \textbf{refinement} procedure in the LLM planner does not lead to a significant improvement in evaluation accuracy; however, it exhibits a positive influence on user trust that may indicate an overtrust in some situations.
% This finding is aligned with prior works showing that iterative refinements based on user feedback would increase user trust~\cite{kunkel2019let, sebo2019don}.
Finally, the propensity-to-trust analysis identifies correctness as the primary determinant of user trust, whereas explanations provided limited improvement in scenarios where the planner's accuracy is diminished.

% In conclusion, our results indicate that the planner's correctness is the dominant factor for both evaluation accuracy and user trust. Therefore, selecting high-quality training data and optimizing the training procedure of AI-based planners to improve planning correctness is the top priority. Once the AI planner achieves a similar correctness level to traditional graph-search planners, strengthening its capability to explain and refine plans will further improve user trust compared to traditional planners.

\paragraph{Future Research} Future steps in this research include expanding user studies with larger sample sizes to improve generalizability and including additional planning problems per session for a more comprehensive evaluation. Next, we will explore alternative methods for generating plan explanations beyond manual creation to identify approaches that more effectively enhance user trust. 
Additionally, we will examine user trust by employing multiple LLM-based planners with varying levels of planning accuracy to better understand the interplay between planning correctness and user trust. 
Furthermore, we aim to enable real-time user-planner interaction, allowing users to provide feedback and refine plans collaboratively, thereby fostering a more dynamic and user-centric planning process.


\bibliographystyle{ACM-Reference-Format}
\bibliography{./text/aspdac_llmhls}


\end{document}
\endinput
%%
%% End of file `sample-sigconf.tex'.
