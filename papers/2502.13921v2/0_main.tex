%%
%% This is file `sample-sigconf.tex',
%% generated with the docstrip utility.
%%
%% The original source files were:
%%
%% samples.dtx  (with options: `sigconf')
%% 
%% IMPORTANT NOTICE:
%% 
%% For the copyright see the source file.
%% 
%% Any modified versions of this file must be renamed
%% with new filenames distinct from sample-sigconf.tex.
%% 
%% For distribution of the original source see the terms
%% for copying and modification in the file samples.dtx.
%% 
%% This generated file may be distributed as long as the
%% original source files, as listed above, are part of the
%% same distribution. (The sources need not necessarily be
%% in the same archive or directory.)
%%
%% Commands for TeXCount
%TC:macro \cite [option:text,text]
%TC:macro \citep [option:text,text]
%TC:macro \citet [option:text,text]
%TC:envir table 0 1
%TC:envir table* 0 1
%TC:envir tabular [ignore] word
%TC:envir displaymath 0 word
%TC:envir math 0 word
%TC:envir comment 0 0
%%
%%
%% The first command in your LaTeX source must be the \documentclass command.
\documentclass[sigconf,9pt]{acmart}
%% ASP-DAC: 9pt or 10pt font.
%% NOTE that a single column version may be required for 
%% submission and peer review. This can be done by changing
%% the \doucmentclass[...]{acmart} in this template to 
%% \documentclass[manuscript,screen]{acmart}
%% 
%% To ensure 100% compatibility, please check the white list of
%% approved LaTeX packages to be used with the Master Article Template at
%% https://www.acm.org/publications/taps/whitelist-of-latex-packages 
%% before creating your document. The white list page provides 
%% information on how to submit additional LaTeX packages for 
%% review and adoption.
%% Fonts used in the template cannot be substituted; margin 
%% adjustments are not allowed.
%%
%%
%% \BibTeX command to typeset BibTeX logo in the docs
\AtBeginDocument{%
  \providecommand\BibTeX{{%
    \normalfont B\kern-0.5em{\scshape i\kern-0.25em b}\kern-0.8em\TeX}}}

%% Rights management information.  This information is sent to you
%% when you complete the rights form.  These commands have SAMPLE
%% values in them; it is your responsibility as an author to replace
%% the commands and values with those provided to you when you
%% complete the rights form.
\setcopyright{acmlicensed}
% \setcopyright{rightsretained}
\copyrightyear{2025}
\acmYear{2025}
\acmDOI{10.1145/3658617.3697616}

%% These commands are for a PROCEEDINGS abstract or paper.
\acmConference[ASP-DAC'25]{Asia and South Pacific Design Automation Conference
}{January 20--23, 2025}{Tokyo Odaiba Miraikan, Japan}
%
%  Uncomment \acmBooktitle if th title of the proceedings is different
%  from ``Proceedings of ...''!
%
%\acmBooktitle{Woodstock '18: ACM Symposium on Neural Gaze Detection,
%  June 03--05, 2018, Woodstock, NY} 
\acmISBN{979-8-4007-0635-6/25/01}


%%%%%%%%%%%%Newly-added Package Begin%%%%%%%%%%%%%

\usepackage{amsmath,amsfonts}
\usepackage{graphicx}
\usepackage{textcomp}
\usepackage{bbding}
\usepackage{pifont}
\usepackage{multicol}

\setlength{\tabcolsep}{3pt}
\usepackage{booktabs}
\usepackage[binary-units=true]{siunitx}
\usepackage{array,multirow}
\usepackage{hyperref}
\usepackage{amsmath}
\usepackage{enumitem}
\usepackage{graphicx}
\usepackage{float}
\usepackage[symbol, hang,flushmargin]{footmisc}
\usepackage{bm}
\usepackage{subfigure}

\renewcommand{\thefootnote}{\fnsymbol{footnote}}
\usepackage{tablefootnote}

\usepackage[noend]{algpseudocode}

\usepackage[ruled,linesnumbered]{algorithm2e}
\SetKwComment{Comment}{$\triangleright$\ }{}
% for multiple inputs in algorithms
\newlength\mylenin
\newcommand\myinput[1]{%
\settowidth\mylenin{\KwIn{}}%
\setlength\hangindent{\mylenin}%
\hspace*{\mylenin}#1\\}

\let\oldnl\nl % Store \nl in \oldnl
\newcommand{\nonl}{\renewcommand{\nl}{\let\nl\oldnl}} % Remove line number for one line

\newlength\mylenout
\newcommand\myoutput[1]{%
\settowidth\mylenout{\KwOut{}}%
\setlength\hangindent{\mylenout}%
\hspace*{\mylenout}#1\\}

\fboxsep=1mm%padding thickness
\fboxrule=1pt%border thickness

\newcommand{\figref}[1]{Figure~\ref{#1}}
\newcommand{\secref}[1]{Section~\ref{#1}}
\newcommand{\subsecref}[1]{Subsection~\ref{#1}}
\newcommand{\tabref}[1]{Table~\ref{#1}}
\newcommand{\appendref}[1]{Appendix~\ref{#1}}
% \newcommand{\algref}[1]{Algorithm~\ref{#1}}
% \newcommand{\eqref}[1]{Equation~\ref{#1}}

\newcommand{\Hsection}[1]{\vspace{0.5\baselineskip}\par\noindent\textit{#1}~\textbf{---}~}
\DeclareMathOperator*{\argmax}{argmax}
\DeclareMathOperator*{\argmin}{argmin}


\newcolumntype{L}[1]{>{\raggedright\let\newline\\\arraybackslash\hspace{0pt}}m{#1}}
\newcolumntype{C}[1]{>{\centering\let\newline\\\arraybackslash\hspace{0pt}}m{#1}}
\newcolumntype{R}[1]{>{\raggedleft\let\newline\\\arraybackslash\hspace{0pt}}m{#1}}

% correct bad hyphenation here
\hyphenation{op-tical net-works semi-conduc-tor}

\usepackage{soul}
% \usepackage[dvipsnames]{xcolor}
\usepackage{xurl}
\usepackage{xspace}
\usepackage{tablefootnote}
%%%%%%%%%%%%Newly-added Package End%%%%%%%%%%%%%


%%
%% Submission ID.
%% Use this when submitting an article to a sponsored event. You'll
%% receive a unique submission ID from the organizers
%% of the event, and this ID should be used as the parameter to this command.
%%\acmSubmissionID{123-A56-BU3}

%%
%% For managing citations, it is recommended to use bibliography
%% files in BibTeX format.
%%
%% You can then either use BibTeX with the ACM-Reference-Format style,
%% or BibLaTeX with the acmnumeric or acmauthoryear sytles, that include
%% support for advanced citation of software artefact from the
%% biblatex-software package, also separately available on CTAN.
%%
%% Look at the sample-*-biblatex.tex files for templates showcasing
%% the biblatex styles.
%%

%%
%% The majority of ACM publications use numbered citations and
%% references.  The command \citestyle{authoryear} switches to the
%% "author year" style.
%%
%% If you are preparing content for an event
%% sponsored by ACM SIGGRAPH, you must use the "author year" style of
%% citations and references.
%% Uncommenting
%% the next command will enable that style.
%%\citestyle{acmauthoryear}

%%
%% end of the preamble, start of the body of the document source.

\begin{document}
\settopmatter{printacmref=false} % Removes citation information below abstract
%%
%% The "title" command has an optional parameter,
%% allowing the author to define a "short title" to be used in page headers.
\title{Exploring Code Language Models for Automated HLS-based Hardware Generation: Benchmark, Infrastructure and Analysis}

%%
%% The "author" command and its associated commands are used to define
%% the authors and their affiliations.
%% Of note is the shared affiliation of the first two authors, and the
%% "authornote" and "authornotemark" commands
%% used to denote shared contribution to the research.

\author{
Jiahao Gai$^{2}$, Hao (Mark) Chen$^{1}$, Zhican Wang$^3$, Hongyu Zhou$^4$, \\ Wanru Zhao$^2$, Nicholas Lane $^2$, Hongxiang Fan$^{1\&2}$\\
{\normalsize $^1$Imperial College London, $^2$University of Cambridge, $^3$Shanghai Jiao Tong University, $^4$University of Sydney}\\ 
{\normalsize Email: jg2123@cam.ac.uk, hongxiangfan@ieee.org}\\
}
%%
%% By default, the full list of authors will be used in the page
%% headers. Often, this list is too long, and will overlap
%% other information printed in the page headers. This command allows
%% the author to define a more concise list
%% of authors' names for this purpose.
\renewcommand{\shortauthors}{Gai et al.}
\renewcommand{\shorttitle}{Exploring Code Language Models for Automated HLS-based
Hardware Generation}
%%
%% The abstract is a short summary of the work to be presented in the
%% article.
\begin{abstract}
Recent advances in code generation have illuminated the potential of employing large language models (LLMs) for general-purpose programming languages such as \textit{Python} and \textit{C++}, opening new opportunities for automating software development and enhancing programmer productivity. 
The potential of LLMs in software programming has sparked significant interest in exploring automated hardware generation and automation.
Although preliminary endeavors have been made to adopt LLMs in generating hardware description languages (HDLs) such as \textit{Verilog} and \textit{SystemVerilog}, several challenges persist in this direction.
First, the volume of available HDL training data is substantially smaller compared to that for software programming languages. 
Second, the pre-trained LLMs, mainly tailored for software code, tend to produce HDL designs that are more error-prone. 
Third, the generation of HDL requires a significantly higher number of tokens compared to software programming, leading to inefficiencies in cost and energy consumption.
To tackle these challenges,
this paper explores leveraging LLMs to generate High-Level Synthesis (\textit{HLS})-based hardware design.
Although code generation for domain-specific programming languages is not new in the literature,
we aim to provide experimental results, insights, benchmarks, and evaluation infrastructure to investigate the suitability of \textit{HLS} over low-level HDLs for LLM-assisted hardware design generation.
To achieve this, we first finetune pre-trained models for \textit{HLS}-based hardware generation, using a collected dataset with text prompts and corresponding reference \textit{HLS} designs.
An LLM-assisted framework is then proposed to automate end-to-end hardware code generation, which also investigates the impact of chain-of-thought and feedback loops promoting techniques on \textit{HLS}- design generation.
Comprehensive experiments demonstrate the effectiveness of our methods.
Limited by the timeframe of this research, we plan to evaluate more advanced reasoning models in the future.
\end{abstract}

\begin{CCSXML}
<ccs2012>
 <concept>
  <concept_id>00000000.0000000.0000000</concept_id>
  <concept_desc>Do Not Use This Code, Generate the Correct Terms for Your Paper</concept_desc>
  <concept_significance>500</concept_significance>
 </concept>
 <concept>
  <concept_id>00000000.00000000.00000000</concept_id>
  <concept_desc>Do Not Use This Code, Generate the Correct Terms for Your Paper</concept_desc>
  <concept_significance>300</concept_significance>
 </concept>
 <concept>
  <concept_id>00000000.00000000.00000000</concept_id>
  <concept_desc>Do Not Use This Code, Generate the Correct Terms for Your Paper</concept_desc>
  <concept_significance>100</concept_significance>
 </concept>
 <concept>
  <concept_id>00000000.00000000.00000000</concept_id>
  <concept_desc>Do Not Use This Code, Generate the Correct Terms for Your Paper</concept_desc>
  <concept_significance>100</concept_significance>
 </concept>
</ccs2012>
\end{CCSXML}

%%%%%%%%%%%%%%% For submission: begin %%%%%%%%%%%%%%%%
% \ccsdesc[500]{Do Not Use This Code~Generate the Correct Terms for Your Paper}
% \ccsdesc[300]{Do Not Use This Code~Generate the Correct Terms for Your Paper}
% \ccsdesc{Do Not Use This Code~Generate the Correct Terms for Your Paper}
% \ccsdesc[100]{Do Not Use This Code~Generate the Correct Terms for Your Paper}
%%%%%%%%%%%%%%% For submission: end %%%%%%%%%%%%%%%%


%%
%% Keywords. The author(s) should pick words that accurately describe
%% the work being presented. Separate the keywords with commas.

%%%%%%%%%%%%%%% For submission: begin %%%%%%%%%%%%%%%%
% \keywords{Automated Hardware Code Generation, Generative Artificial Intelligence, High-Level Synthesis}
%%%%%%%%%%%%%%% For submission: end %%%%%%%%%%%%%%%%

%% A "teaser" image appears between the author and affiliation
%% information and the body of the document, and typically spans the
%% page.
% \received{20 February 2007}
% \received[revised]{12 March 2009}
% \received[accepted]{5 June 2009}

%%
%% This command processes the author and affiliation and title
%% information and builds the first part of the formatted document.

\maketitle
\textbf{\small ACM Reference Format:} \newline
{\small Jiahao Gai, Hao (Mark) Chen, Zhican Wang, Hongyu Zhou, Wanru Zhao, Nicholas Lane, Hongxiang Fan. 2025. \shorttitle. In \textit{Proceedings of Asia and South Pacific Design Automation Conference (ASP-DAC’25)}. ACM, New York,
NY, USA, 8 pages. \url{https://doi.org/10.1145/3658617.3697616}}
\section{Introduction}\label{sec:intro}

In computational finance, Monte Carlo simulations are used extensively to estimate the expected value of financial payoffs based on the solution of stochastic differential equations (SDEs) which model the evolution of stock prices, interest rates, exchange rates and other quantities \cite{glasserman04}.  Monte Carlo methods are very general and flexible, but for high accuracy it requires generating a large number of costly SDE path approximations, which has motivated research into a number of variance reduction or, equivalently, cost reduction techniques. One such method is
Multilevel Monte Carlo (MLMC), which was proposed in \cite{GILES2008} and was adapted for various applications that are summarised in \cite{Giles_overview17} and successfully combined with other methods such as quasi-Monte Carlo methods. The main idea of MLMC is to approximate the payoff using different time stepping resolutions when numerically solving the underlying SDE and to generate an optimal number of samples on each level, such that the overall computational cost is minimised subject to the desired bound on the variance. %, such that the total computational cost is minimised. 
The computational savings come from the fact that most samples are computed on the coarser levels and hence are less expensive while only a few samples from the finest levels are required \cite{GILES2008}.


Among the directions in which the computational cost 
of MLMC methods could further be reduced, an important avenue is the use of lower precision calculations, especially for the first Monte Carlo levels where the targeted accuracy is relatively low. 
 An overview of the research on mixed precision for the standard Monte Carlo (MC) framework is provided in \cite{ChowMixedPrecisionStandardMC} but only a few references study the potential of low precision computation in the MLMC framework \cite{Rounding_error_oliver}. To the best of our knowledge, the only MLMC framework with customised precision in the literature is \cite{brugger2014mixed}, but they use a uniform precision for all operations on each Monte Carlo level instead of optimising 
 the precision of each intermediary variable to reduce as much as possible the cost of path generation.
 
An important motivation for an MLMC framework with variable precision would be performing the low precision computations on reconfigurable hardware devices such as Field Programmable Gate Arrays (FPGAs). FPGAs contain customizable logic blocks and connectors that make it easy to adapt the digital circuit architecture for a specific application, leading to a highly parallel and optimised implementation. Therefore they are successfully exploited in applications that require high speed and have high computational workload, such as signal processing \cite{woods2008fpga}, and real time applications like high frequency trading \cite{HFT1,HFT2}. That is why a number of previous works in hardware architecture design implemented the MLMC algorithm to price financial options using FPGAs as accelerators, which resulted in improved speed and power efficiency compared to full CPU architectures \cite{Schryver2013AMM}. The paper \cite{lindsey2016domain} also proposed 
a Domain Specific Language to automate the configuration of FPGAs for this specific application. However, only \cite{brugger2014mixed} proposed a heuristic to reduce the precision in calculations.

In addition, all aforementioned works considered that the random number generation (RNG) is performed in single or double precision. Yet in most cases an important portion of the workload in the overall MLMC simulation comes from the RNG and in \cite{brugger2014mixed} this limited the total computational savings.
To reduce the cost of MLMC simulations in particular those based on the Geometric Brownian Motion (GBM), \cite{approximateICDF_Oliver, NestedOliver} have proposed to use approximate random numbers that are generated by applying an approximation of the inverse CDF to uniform random numbers. In \cite{NestedOliver}, the authors proposed a way to integrate these lower precision random variables into a \textit{nested} MLMC framework and completed a numerical analysis to bound the resulting error at each MC level by a product of the time step and the error in the random number approximation. The same authors show in \cite{approximateICDF_Oliver} that using approximate random variables reduces the cost of path generation by a factor 7.


In this paper we propose a nested MLMC framework that combines the use of approximate random normal variables and lower precision calculations to reduce the computational cost of MLMC even further than \cite{brugger2014mixed,NestedOliver}. We illustrate the efficiency of our framework in Matlab, after making several assumptions on the cost of operations and size of the errors that we carefully justify. We focus on the case of GBM and use the approximate RNG methods presented in \cite{approximateICDF_Oliver} as well as a new slightly modified method that combines CDF inversion and the central limit theorem. To choose the precision of the variables in the low precision path generation, we introduce a novel method to optimise the bit-widths. This optimisation is performed before the main path generation loop is executed and is based on a linear model of the payoff error  
due to rounding when computing in low precision. The error model relies on algorithmic differentiation in a similar manner to \cite{unifying-bwoptim,bitwidth-AD,ADAPT}. The bit-width optimisation procedure can be performed off-line, so this stage can be excluded from the on-line time complexity of our framework. The user specified desired accuracy is then enforced by calculating on-line the number of samples that need to be generated.

In terms of hardware design, we suggest implementing the low precision path generation on FPGAs and the full-precision ones on a CPU or GPU. 
The FPGA offers enough flexibility to define a separate bit-width for every variable in the low precision path generation, and can be reconfigured periodically to update the bit-widths when the market parameters have changed considerably. 


The paper is organized as follows : \Cref{sec:MLMC} introduces MLMC and nested MLMC to make clear the estimator that is implemented in our framework. Then in \Cref{sec:RNG} we detail the methods that could be used to obtain approximate random normally distributed numbers very cheaply for the low precision path generation. In \Cref{sec:error_model} and \Cref{sec:costModel} we propose an error model and a cost model (resp.) that we then use to formulate the optimisation problem that is solved to obtain the optimal bit-widths of fixed point variables in \Cref{sec:optimisation}. Finally we summarise our results and future directions in \Cref{sec:conclusion}.



\section{Background}
\label{sec:background}


\subsection{Preliminaries}

{\color{red}[TODO: LLMs? in-context learning?]}

\subsection{Problem Definition}

{\color{red}[TODO: define the problem of citation intent]}

\begin{figure*}[t]
  \centering
  \includegraphics[width=\textwidth, height=10cm]{images/mindmap2.pdf} 
  \caption{Mindmap showing Data Collection and Rewrite Desiderata}
  \label{fig:mindmap}
\end{figure*}
% \begin{figure*}[t]
%   \centering
%   \includegraphics[width=\textwidth]{images/process.pdf} 
%   \caption{Dataset Creation Pipeline}
%   \label{fig:process}
% \end{figure*}
\section{Constructing a Dataset for Visual Instruction Rewriting}
\label{sec:datasets}

Task-oriented conversational AI systems rely on a semantic parser to interpret user intent and extract structured arguments \cite{louvan2020recent,aghajanyan2020conversational}. For example, when a user says,\textit{ "Add the team meeting to my calendar for Friday at 3 PM"}, the system must parse the intent (\textit{CreateCalendarEvent}) and extract arguments such as the \textit{EventTitle} (``team meeting''), \textit{EventDate} (``Friday''), and \textit{EventTime} (``3 PM'') to schedule the event correctly. Unlike purely text-based interactions, multimodal instructions, particularly those directed at conversational AI assistants on AR/VR devices (\textit{e.g.,} Apple's Siri for Apple Vision Pro), introduce additional challenges such as ellipsis and coreference resolution. For instance, a user may look at a book cover and ask, \textit{“Who wrote this?”} or point at a product in an AR interface and say, \textit{“How much does this cost?”} Traditional text-based semantic parsers struggle with such instructions since critical visual context is missing. Thus, to bridge the gap between multimodal input and existing conversational AI stacks, we introduce a dataset specifically designed for \textit{rewriting multimodal instructions} into structured text that can be processed by standard text-based semantic parsers. Figure \ref{fig:mindmap} illustrates a representation of the dataset collection requirement, highlighting the transformation of multimodal inputs into text-based rewrites.

To construct our dataset, we first define an ontology of intents and arguments, as existing ontologies in conversational AI and semantic parsing are often proprietary and unavailable for research use. We take inspiration from \newcite{goel2023presto} for ontology and extend it to accommodate multimodal task-oriented interactions. Figure \ref{fig:intent_argument_box} (ref. Appendix) presents an overview of the intents and arguments in our ontology. Next, we curate a diverse set of images covering various real-world multimodal interaction scenarios, including book covers, product packaging, paintings, mobile screenshots, flyers, signboards, and landmarks. These images are sourced from publicly available academic datasets, such as OCR-VQA\footnote{\url{https://ocr-vqa.github.io/}}, CD and book cover datasets, Stanford mobile image datasets\footnote{\url{http://web.cs.wpi.edu/~claypool/mmsys-dataset/2011/stanford/}}, flyer OCR datasets\footnote{\url{https://github.com/Skeletonboi/ocr-nlp-flyer.git}}, signboard classification datasets\footnote{\url{https://github.com/madrugado/signboard-classification-dataset}}, Google Landmarks\footnote{\url{https://github.com/cvdfoundation/google-landmark}}, and Products-10K\footnote{\url{https://products-10k.github.io/}}.

\begin{table}[t]
    \centering
    \scriptsize
    \label{tab:dataset_statistics}
    \begin{tabular}{llccc}
        \toprule
        \textbf{Category} & \textbf{Total} & \textbf{Train} & \textbf{Test} \\
        \midrule
        Book              & 485 / 500                               & 386 / 399                               & 101 / 101                               \\
        Business Card     & 26 / 960                                & 26 / 772                                & 26 / 188                                \\
        CD               & 27 / 1,020                              & 27 / 835                                & 27 / 185                                \\
        Flyer & 159 / 5,940                             & 159 / 4,742                             & 159 / 1,198                             \\
        Landmark         & 511 / 19,274                            & 511 / 15,420                            & 511 / 3,854                             \\
        Painting & 27 / 980                                & 27 / 774                                & 27 / 206                                \\
        Product          & 499 / 10,349                            & 499 / 8,276                             & 492 / 2,073                             \\
        \midrule
        \textbf{Total}   & \textbf{1,734 / 39,023}                 & \textbf{1,635 / 31,218}                 & \textbf{1,343 / 7,805}                  \\
        \bottomrule
    \end{tabular}
    \caption{Number of Images/Instructions per Category}
    \label{tab:sources}
\end{table}
\begin{table}[t]
    \centering
    \footnotesize
    \begin{tabular}{l  c}
        \toprule
         \textbf{Annotator}& \textbf{Percentage of Correct Captions}\\ 
         \midrule
         Annotator 1	& 90.62\%\\ 
         Annotator 2	& 87.23\%\\
         Annotator 3	& 86.35\%\\
         \midrule
         \textbf{At least two }& \textbf{92.18}\%\\
         \midrule
         \textit{All three }& \textit{74.63}\% \\
         \bottomrule
    \end{tabular}
    \caption{GPT-4 Instruction Rewriting Validation Results from Amazon Mechanical Turk }
    \label{tab:annotator_data}
\end{table}
\begin{figure}[t]
\includegraphics[width=\columnwidth]{images/intent.png}
  \caption{Dataset Distributions By Intent}
  \label{fig:intent}
\end{figure}
Upon identifying and verifying the images, we employ the GPT-4 model from OpenAI \cite{achiam2023gpt} to systematically generate and refine multimodal instructions into rewritten text-based instructions. The process begins with a bootstrap phase, where GPT-4 is prompted to generate 20 direct questions per image by explicitly referencing visible objects or textual elements while adhering to the intent list defined in Figure \ref{fig:intent_argument_box}. A second prompting phase then validates the generated questions against the corresponding image, filtering out ambiguous or irrelevant instructions to ensure alignment with the visual context. 

In the rewriting phase, GPT-4 is tasked with paraphrasing the validated instructions, ensuring that the transformed questions are fully self-contained and interpretable without requiring the image. This transformation is crucial for enabling multimodal conversational AI systems to process instructions using purely text-based stacks. Finally, a verification phase prompts the model to assess the rewritten questions in relation to both the original instruction and the image, ensuring semantic fidelity and eliminating inconsistencies. This multi-stage prompting strategy resulted in a dataset of 39,023 original-rewritten instruction pairs, derived from 1,734 images, with an 80\%-20\% train-test split. Table \ref{tab:sources} provides a breakdown of image sources.

While automated validation ensures consistency across different stages, human evaluation remains critical for verifying the dataset’s reliability. To this end, we conducted an annotation task via Amazon Mechanical Turk (AMT) to validate rewritten instructions within the test set for indirect image-based instructions. Each annotation task followed a structured validation guideline, where annotators reviewed an image, its original multimodal instruction, and the rewritten text-only instruction, determining whether the reformulation preserved the intent and meaning of the original instruction. Annotators were instructed to select "Accept" if the rewritten instruction was correct or "Reject" if it failed to capture the original meaning. Annotators are incentivized appropriately for this binary grading task. Agreement analysis, as shown in Table \ref{tab:annotator_data}, indicates that in 92.2\% of cases, at least two annotators agreed on "Accept," while 74.6\% of instructions achieved full consensus across all three annotators. Despite a Fleiss' Kappa score of 0.278—suggesting fair inter-annotator agreement—the high rate of majority consensus supports the dataset’s reliability for real-world use. Given these results, we publicly release the full dataset along with raw AMT responses, enabling further analysis, filtering, and refinements by the research community.

Figure \ref{fig:intent} presents the distribution of intents in our dataset, categorized into training and test splits. The distribution reflects practical usage patterns in real-world multimodal conversational AI systems, with a higher occurrence of general QA and web search, alongside diverse task-oriented intents such as reminders, messaging, and navigation, ensuring coverage of frequent user interactions.



% In this study, we utilize a comprehensive multimodal dataset curated from various sources to facilitate research in multimodal instruction rewriting using compact models. Table~\ref{tab:dataset_statistics} provides an overview of the dataset's composition, detailing the number of images and corresponding instructions sourced from different domains. This diverse dataset is designed to challenge models in interpreting and rewriting instructions based on both visual and textual information embedded within images.

% The dataset is organized into a single TSV file, \texttt{all\_data.tsv}, which consolidates all the data for streamlined processing and analysis.

% The dataset is publicly accessible and can be downloaded from our Hugging Face repository:
% \url{https://huggingface.co/datasets/utischoolnlp/multimodal_instruction_rewrites}.

% \begin{table}[h]
%     \centering
%     \caption{Dataset Statistics}
%     \label{tab:dataset_statistics}
%     \resizebox{0.5\textwidth}{!}{%
%         \begin{tabular}{|l|l|c|c|}
%             \hline
%             \textbf{Data Source} & \textbf{Type} & \textbf{Number of Images} & \textbf{Number of instructions} \\ \hline
%             \href{https://github.com/gulvarol/grocerydataset}{Grocery Store Dataset} & Grocery Dataset & 287 & 5,945 \\ \hline
%             \href{https://amazon-berkeley-objects.s3.amazonaws.com/index.html}{Amazon Berkeley Objects} & Amazon Dataset & 187 & 3,890 \\ \hline
%             \href{https://products-10k.github.io/}{Products-10K} & E-commerce Dataset & 23 & 472 \\ \hline
%             \href{https://www.kaggle.com/datasets/vikashrajluhaniwal/fashion-images}{Fashion Images} & Fashion Clothing Dataset & 2 & 42 \\ \hline
%             \textbf{Total} & & \textbf{499} & \textbf{10,349} \\ \hline
%         \end{tabular}
%     }
% \end{table}


% \subsection*{Additional Dataset Statistics}

% To provide a deeper understanding of the dataset's characteristics, we present the following statistics derived from \texttt{all\_data.tsv}:

% \begin{itemize}
%     \item \textbf{Prompt Length}:
%     \begin{itemize}
%         \item \textbf{Average Prompt Length}: 80.99 tokens
%         \item \textbf{Maximum Prompt Length}: 160 tokens
%         \item \textbf{Minimum Prompt Length}: 28 tokens
%     \end{itemize}
    
%     \item \textbf{Rewritten Question Length}:
%     \begin{itemize}
%         \item \textbf{Average Rewritten Question Length}: 56.94 tokens
%         \item \textbf{Maximum Rewritten Question Length}: 160 tokens
%         \item \textbf{Minimum Rewritten Question Length}: 28 tokens
%     \end{itemize}
% \end{itemize}

% These statistics highlight the complexity and variability of the prompts and their corresponding rewritten questions, providing a robust foundation for training and evaluating multimodal instruction rewriting models.

% \subsection*{Dataset Composition}

% The dataset is consolidated into a single TSV file, \texttt{all\_data.tsv}, which includes all image-instruction pairs. This unified format simplifies data handling and ensures consistency across training and evaluation phases. The structure of \texttt{all\_data.tsv} is as follows:


% \begin{itemize}
%     \item \textbf{Columns}:
%     \begin{itemize}
%         \item \texttt{Image\_ID}: Unique identifier for each image.
%         \item \texttt{Image\_URL}: Direct link to the image file.
%         \item \texttt{Prompt}: Original instruction associated with the image.
%         \item \texttt{Rewritten\_Question}: Reformulated version of the original instruction.
%     \end{itemize}
% \end{itemize}

% \subsection*{Dataset Accessibility}

% Researchers and practitioners can access the dataset and its associated resources through our Hugging Face repository:
% \url{https://huggingface.co/datasets/utischoolnlp/multimodal_instruction_rewrites}.

% The dataset is organized in a structured format, including:
% \begin{itemize}
%     \item \texttt{all\_data.tsv}: Consolidated dataset containing all image-instruction pairs.
%     \item \texttt{images.zip}: Compressed archive of all dataset images.
%     \item \texttt{README.md}: Detailed instructions and metadata descriptions for dataset usage.
% \end{itemize}

% \subsection*{Discussion}

% The diversity of data sources, ranging from grocery items to fashion clothing, ensures that the dataset covers a wide array of visual and textual contexts. This variety is crucial for training models that are robust and generalizable across different domains. The substantial number of instructions relative to images indicates that each image is associated with multiple instructions, providing ample data for effective model training and evaluation.

% By consolidating all data into a single TSV file, we streamline the data processing pipeline, facilitating easier integration with various modeling frameworks and tools. The comprehensive statistics on prompt and rewritten question lengths further underscore the dataset's complexity, challenging models to handle a wide range of instruction formulations.

% \section*{Conclusion}

% Our multimodal instruction rewriting dataset offers a comprehensive resource for researchers aiming to develop and evaluate models in this domain. By providing a diverse and sizeable dataset, we aim to facilitate advancements in multimodal understanding and contribute to the broader field of artificial intelligence.

% \section*{References}

% \begin{itemize}
%     \item \href{https://github.com/gulvarol/grocerydataset}{Grocery Store Dataset}
%     \item \href{https://amazon-berkeley-objects.s3.amazonaws.com/index.html}{Amazon Berkeley Objects}
%     \item \href{https://products-10k.github.io/}{Products-10K}
%     \item \href{https://www.kaggle.com/datasets/vikashrajluhaniwal/fashion-images}{Fashion Images Dataset}
% \end{itemize}

% \label{sec:dataset}
\section{\tool System}\label{sec:system_description}

\begin{figure*}[t]
    \centering
    \includegraphics[width=0.85\textwidth]{src/img/inference.pdf}
    \vspace{-2mm}
    \caption{Our sketch-aware prompt recommendation first builds a semantic space through data-driven analysis of key semantic elements covering single object and cross object properties. Then the semantic space is integrated with retrieval of attributes and relationships reference from semantic dataset. Finally, these semantic guidance is combined with users' initial sketch to form a sketch-aware multi-modal prompt to the MLLM to support spatial-aware inference.}
    \Description{Our sketch-aware prompt recommendation first builds a semantic space through data-driven analysis of key semantic elements covering single object and cross-region properties. Then the semantic space is integrated with retrieval of attributes and relationships reference from semantic dataset. Finally, these semantic guidance is combined with users' initial sketch to form a sketch-aware multi-modal prompt to the MLLM to support spatial-aware inference.}
    \label{fig:inference}
    % \vspace{-3mm}
\end{figure*}

Based on the design goals, we design \tool, an interactive system that allows users to generate images controlled by inputs of rough sketches and simple prompts, while generating semantically cohesive and region-controlled images.
The overall framework of \tool is shown in Figure~\ref{fig:workflow}. 
The framework consists of three stages: 1) the user can draw a sketch, assign a corresponding regional prompt, and use automatic prompt recommendation to refine their initial prompt \textbf{(G1, G2)}; 2) the sketch with object decomposition and single-object generation \textbf{(G3)}; and 3) spatial adjustment and anchoring of object shapes   \textbf{(G1, G3)}.


\subsection{Sketch-Aware Prompt Recommendation}
\label{ssec:prompt_rec}
Crafting effective prompts for rough sketch-based image generation is a challenging task, as users must not only create prompts for each individual region but also ensure coherence across the entire image. 
We introduce a prompt recommendation method that automatically enhances the user’s initial input, to produce a spatially cohesive prompt that aligns with the overall composition.
Figure~\ref{fig:inference} shows the overall workflow of this process.


\subsubsection{Semantic Space Reasoning}
\begin{table}[thb] \small
    \centering
    \caption{Common examples in the semantic space across various T2I description datasets.}
    \vspace{-2mm}
    \begin{tabular}{|l|l|c|}
    \hline 
    \textbf{Space Item} & \textbf{Property} & \textbf{Instance} \\
    \hline 
    \multirow{3}{*}{\textbf{Single Object}} 
    & Type &   Man, Car, Dog, Tree, Window \\
    && Table, Ocean, Park, Wall\\ \cline{2-3}
    & Attribute & Wooden, Tall, Red, Large\\
    &&Fluffy, Round, Slim, Silver   \\ \cline{2-3}
    & State &   Standing, Moving, Swaying, \\
    &&Broken, Sleeping, Lying\\\hline
    \multirow{2}{*}{\textbf{Cross Object}} 
    &  Direction  &  Facing to, Aligned in, \\
    &&Diagonally placed\\\cline{2-3}
    & Relationship & Next to, Under, Parked on, \\
    &&Sitting by, Supporting\\ \hline
    \multirow{3}{*}{\textbf{Overall}}  
    & Lightning  &   Natrual daylight, indoor lighting, \\
    && Soft light, Moon light\\\cline{2-3}
    & Camera  &   Close-up shot, Wide-angle shot, \\
    && Overhead shot, Extreme long shot \\\cline{2-3}
    & Style  &   Realistic, Minimalist, Cinematic, \\
    && Abstractm, Anime, Oil painting\\
    \hline
    \end{tabular}
    \vspace{-1mm}
    \label{tab:example_space}
\end{table}



Previous prompt-tuning methods have primarily focused on text-to-image models, which offer limited support for refining individual region prompts and often struggle to ensure cohesiveness across the entire image.
To address this, the first step is to identify the types of prompts needed to generate a coherent image under rough sketch-based control.
Specifically, we define an "\emph{semantic space}" that provides intuitive guidance that helps users to easily input and adjust their prompts in the appropriate regions.
The process of constructing this semantic space involves analyzing online sources~\cite{sd,wang2022diffusiondb,civitai,xie2023prompt} and prompt-guideline literature~\cite{oppenlaender2023prompting,oppenlaender2023taxonomy,liu2022design} to identify critical elements that contribute to high-quality image generation. Additionally, examining image description datasets in computer vision—including traditional task datasets~\cite{caesar2018coco,pham2021learning,krishna2017visual} and recent datasets tailored for image generation~\cite{onoe2024docci}—helps uncover relevant dimensions for describing images.

\begin{figure*}[t]
    \centering
    \includegraphics[width=0.995\textwidth]{src/img/refinement.pdf}
    \vspace{-2mm}
    \caption{Spatial-condition sketch refinement can help novice users refine their sketch by generating more realistic and accurate sketch for each object through single object decomposition and generation, and subsequently allowing users to interactively refine the sketch by object selection and spatial adjustment.}
    \Description{Spatial-condition sketch refinement can help novice users refine their sketch by generating more realistic and accurate sketch for each object through single object decomposition and generation, and subsequently allowing users to interactively refine the sketch by object selection and spatial adjustment.}
    \label{fig:decompose}
    \vspace{-3mm}
\end{figure*}

\begin{table*}[thb] 
    \centering
    \caption{Example Statistics of objects, attributes and relationships in Visual Genome~\cite{krishna2017visual} and VAW~\cite{pham2021learning}.}
    \vspace{-2mm}
    \begin{tabular}{lccccc}
    \hline 
    \textbf{Space Item} & \textbf{1st} & \textbf{2nd} & \textbf{3rd} & \textbf{20th} & \textbf{50th}  \\
    \hline 
    Object & window (52k)  & man (52k) & shirt (39k) & trees (17k) & sidewalk (8k)  \\
    Attribute & white (311k) & black (195k) &  blue (118k) & clear (15k) & colorful (5.8k)   \\
    Relationship  & on (645k) & has (245k) &  in (219k) & sitting on (13k) & laying on (3.5k)    \\
    \hline
    \end{tabular}
    \vspace{-1mm}
    \label{tab:statsitic}
\end{table*}

We summarize these prompt aspects and conduct experiments to identify the key components essential for high-quality output. 
% Finally, we propose a semantic space that integrates these necessary elements to ensure coherence when generating images from rough sketches.
Table~\ref{tab:example_space} presents the common dimensions and example instances of the identified semantic space within the T2I datasets.
The specific elements within the space are listed below.
\begin{itemize}
    \item \textbf{Single Object Prompt} contains local prompts for each single object. 
    It contains \textit{type} of the object; \textit{attribute} of object including main attributes such as color, texture, shape; and \textit{state} indicates how the object acts, including still, standing, running, etc. The background is a special object that only has type and attribute.
    \item \textbf{Cross Object Prompt} explicitly specifies how objects in different regions interact, which is crucial for the coherence of the generated image.
    It includes \textit{direction} of objects and \textit{relationship} that multiple objects interact with each other.
    \item \textbf{Overall Prompt} does not directly affect multi-object cohesiveness but allows users to optionally specify the overall visual effect, including \textit{lighting}, \textit{style} and \textit{camera}.
    % \item \textbf{Overall Prompt} is not indispensable for a semantically cohesive image generation, but has significant impact in visual effect, include \textit{lighting}, \textit{style} and \textit{camera}. 
    % Previous works consider quality modifiers (e.g., best quality, high resolution). In our experiment, the state-of-the-art model can already generate high quality images without these modifiers.
\end{itemize}




Figure~\ref{fig:inference} (right) illustrates a completed semantic space for a sketch featuring a girl, a cat, and a background (\textit{i.e.}, areas without objects).
Once the individual prompt components are defined, the separate prompts within a single region are concatenated into one unified prompt using commas (\textit{e.g.}, "type: girl, attribute: long hair" becomes "girl, long hair"). 
% To ensure that the concatenated prompt remains within the 77-token limit imposed by CLIP, we employ the bag-of-conditions approach~\cite{omost}.}



\subsubsection{Sketch-Augmented Prompting}
While the semantic space simplifies the prompt input and adjustment process, manually entering all prompts and identifying the appropriate ones can still be labor intensive and cognitively demanding, particularly when dealing with many objects. 
To alleviate this burden, we utilize a MLLM GPT-4o, to automatically complete the semantic space based on the user's sketch and initial prompt.

Specifically, the user’s initial prompt and rough sketch are input into the MLLM, with the semantic space acting as a contextual guide within the prompt template. The model generates prompts to populate the semantic space, incorporating both the initial input prompt and the sketch. It utilizes spatial reasoning, guided by the Chain-of-Thought~\cite{wei2022chain} strategy, to account for the \textit{shape}, \textit{location}, and \textit{interaction} of the objects within the user's sketch.
However, relying solely on the MLLM to fill the semantic space can be risky, as it may overfit to certain content or produce results with reduced coherence~\cite{cao2023beautifulprompt}. 
To address this, we further enhance the process by retrieving reference attributes and relationships between objects from crowd-sourced text-image datasets based on real-world images~\cite{pham2021learning,krishna2017visual}. 
The datasets are organized into a dictionary, where object names serve as keys and their corresponding attributes or relationships are stored as values. 
During retrieval, \tool randomly samples $k=10$ examples from the values based on the given object names as keys, providing the MLLM with reference data. 
The raw datasets are sourced and publicly available from ~\cite{pham2021learning,krishna2017visual}.
Example statistics are shown in Table~\ref{tab:statsitic}. 
The rich semantics in these datasets enhance both diversity and coherence in the completed semantic space.

% Table~\ref{tab} presents example statistics of objects, attributes, and relationships within these datasets. 
The overall prompt generation process, as shown in Figure~\ref{fig:inference}, incorporates user input, semantic space, and retrieved attributes and relationships to guide generation. 
We employ few-shot learning~\cite{wang2023large} to enhance the quality and robustness of the generated prompts.


\begin{figure*}[t]
    \centering
    \includegraphics[width=0.995\textwidth]{src/img/ablation_zoom.pdf}
    \vspace{-2mm}
    \caption{Ablation study shows that prompt recommendation avoids common issues like missing objects and unrealistic relationships while sketch refinement further enhances fine-grained control.}
    \Description{Ablation study shows that prompt recommendation avoids common issues like missing objects and unrealistic relationships while sketch refinement further enhances fine-grained control.}
    \label{fig:ablation}
    \vspace{-4mm}
\end{figure*}

\subsection{Spatial-Condition Sketch Refinement}
\label{ssec:sketch_refine}
To help users refine their rough sketches into fine-grained shapes that align with their intentions and allow for iterative refinement, we propose a decompose-and-recompose approach. 
As Figure~\ref{fig:decompose} shows, the sketch is first decomposed into individual objects.
The users can then generate, select, and adjust the desired single-object images. 
Finally, these selected objects, along with their spatial conditions, are combined to generate the final result.

\subsubsection{Single Object Decomposition}
Instead of directly using a single object sketch for generation, we first classify objects into two categories: \textbf{thing} as foreground object with specific shapes, such as humans, animals, or chairs, and \textbf{stuff} as background object without a defined shape, like oceans, grass, or sky, based on ~\cite{caesar2018coco}. 
% \new{"Things" are objects with specific shapes, such as humans, animals, or chairs, while "stuff" refers to objects without a defined shape, like oceans, grass, or the sky.}
To classify an object, we compute the word embedding of its type and find the nearest match in a category list containing "things" and "stuff" in the object list in the COCO-Stuff dataset~\cite{caesar2018coco}.
During single object decomposition stage, each thing object is extracted using FAST SAM (Segment Anything)~\cite{zhao2023fast}, to make sure the single object generation maximally preserves consistency to the original sketch.
Once the single object sketch is decomposed, the generation of each individual object is carried out. 
Since the goal is to generate fine-grained object shapes but not the final image, the process can be accelerated by using low-step inference (6 steps) with the Lightning Diffusion model~\cite{luo2023lcm} and a lower resolution (512x512). 
In our experiment, generating 12 images took approximately 4 seconds on a GTX 4090.


To ensure that the generated result aligns more closely with the user's sketch, we filter the generated images based on the Intersection over Union (IoU) and CLIP score~\cite{hessel2021clipscore}, which measure spatial correspondence and semantic alignment, respectively, between the user sketch and the generated single object. 
We compute a weighted sum of the IoU and CLIP scores, then sort the images and select the top four for the user to choose from.
Once the user selects an image containing the desired object shape, the target object is automatically extracted using FAST SAM~\cite{zhao2023fast}.
Each object with refined shape will automatically replace the original rough sketch.
If users have no desired image in one generation, they can perform multi-round generation until finding the desired one.
Figure~\ref{fig:decompose} shows an example in which the rough sketch of a girl and a dog is refined to specific shapes.


\begin{figure*}[t]
    \centering
    \includegraphics[width=0.8\textwidth]{src/img/interface.pdf}
    \vspace{-3mm}
    \caption{\tool interface consists of (a) Canvas view, (b) Prompt Recommend view, (c) Sketch Refine view and (d) Result view.}
    \Description{SketchFlex interface consists of (a) Canvas view, (b) Prompt Recommend view, (c) Sketch Refine view and (d) Result view.}
    \label{fig:interface}
    \vspace{-3mm}
\end{figure*}

\subsubsection{Single Object Adjustment}
Our system provides flexible control not only over the shapes of objects but also over their size and position. As shown in Figure~\ref{fig:decompose}
, users can easily adjust the size and spatial placement of each object. 
Once adjustments are made, the corresponding single object shape mask is moved accordingly. 
All individual object shapes are then combined into an "anchor" image.
% The term "anchor" represents the idea that the generated result, like a boat floating within the rough sketch regions, may vary with each generation, just as a boat drifts on water. 
% However, with the anchor in place, the boat remains fixed to a specific location—similar to how the selected object shapes remain fixed in the generated result. 
Shape anchoring refers to the process of fixing the object shapes in the generated result.
To apply this shape anchoring, we extract the edges of the selected object using Canny edge detection and feed them into ControlNet to ensure that the final output adheres to the user's shape preferences. 
Once users have made their desired adjustments, they can generate an image with the exact fine-grained shapes they prefer.


A related issue with the original rough sketch-based generation is that a single prompt is applied to each region separately. 
To generate images that capture relationships between objects, we create a joint mask, \textit{i.e.}, the union of object masks, for two related objects, allowing the relationship prompt to influence the interaction between them. 
Equation 1 illustrates the creation of a joint mask for the relationship between region $i$ and region $j$, 
where $\oplus$ denotes the concatenation operation.

\begin{equation}
M_{ij} = [M_i \oplus M_j].
\end{equation}

However, using a joint mask alone can sometimes result in multiple objects being generated within a single object area. 
To address this, we apply negative prompts to exclude objects outside the intended region, preventing unwanted elements from appearing in the wrong areas, as shown in Equations 2-3.
In Equation 2, the cross-attention map that correlates text and image patches is updated such that the text embedding $C_i$ is amplified by a scalar $\lambda_{m_i}$ within the masked region $M_i$.
In Equation 3, the relationship embedding condition is reinforced within $M_{ij}$, while the influence of $C_i / C_j$ is reduced in the complementary areas.

\begin{equation}
A_i \leftarrow \lambda_{m_i} \cdot C_i \odot M_i,
\end{equation}

\begin{equation}
A_{(i,j)} \leftarrow \lambda_{m_{ij}} \cdot C_{ij} \odot M_{ij} - \lambda_{m_{ij}} \cdot C_i \odot (M_{ij}-M_i) - \lambda_{m_{ij}} \cdot C_j \odot (M_{ij} - M_j).
\end{equation}



Figure~\ref{fig:ablation} shows the ablation results of our system’s functions.
Without prompt recommendation and sketch refinement, the rough sketch-based generation often produces undesirable outcomes, such as missing objects, unrealistic relationships, and incorrect perspectives.
By incorporating our prompt recommendation, which refines prompts to be spatially aligned with the sketch, the results become more stable and coherent, with the generated objects closely matching the original sketch.
Finally, with prompt recommendation and sketch refinement, users can achieve fine-grained control, enabling them to generate detailed results that align with their sketches and creative intentions. For example, in the first column, the posture of the man lying on the chair with both arms spread out is more accurately captured in our result compared to the other two. 
In the third column, the girl's head is positioned slightly to the right and below the car within the mask, which aligns with our result. In contrast, in the other two results, the girl's head overlaps with the car, resulting in an incorrect spatial relationship.










\section{Experimental Protocol}
\label{sec:evaluation}
\subsection{Model and Dataset}
\begin{figure}[t]
    \centering
\includegraphics[width=\linewidth]{fig/instruction.png}
    \caption{Instructions given to the LLM for the bias detecrtion.}
    \label{fig:instruction}
\end{figure}
We utilized Stable Diffusion 3.5-large~\cite{sd3} as our text-to-image (T2I) model and employed GPT-4o~\cite{gpt4} for bias detection as a blackbox model, and DeepSeek-V3~\cite{liu2024deepseek} as an open-sourced model. The LLM receives prompts as illustrated in Figure \ref{fig:instruction}. Through in-context learning techniques, we enhance model performance by exposing it to an exemplar task~\cite{brown2020language}. To evaluate the debiasing performance for occupations, we used the occupation dataset from Stable Bias~\cite{Luccioni_2023} (hereafter referred to as the stable bias profession dataset), which contains 131 occupations sourced from the U.S. Bureau of Labor Statistics (BLS). The dataset composition is detailed in the Appendix A of~\cite{Luccioni_2023}. All input prompts were formatted as ``A portrait photo of [profession]'' to ensure that the T2I model interprets them specifically as occupations rather than other potential meanings. To assess the performance in removing implicit social biases present in prompts beyond occupations, we used the Parti Prompt dataset~\cite{yu2022scaling}, which consists of over 1,600 diverse English prompts designed to comprehensively evaluate text-to-image generation models and test their limitations. For attribute rebalancing, we employed the uniform distribution, as our primary goal was to verify the debiasing capability of our latent variable guidance.

% For experiments involving bias adjustment using employment statistics log-probabilities, we conducted experiments to mitigate gender bias using BLS2022 statistical data for five occupation prompts mentioned in \cite{naik2023social}: ``CEO'', ``doctor'', ``computer programmer'', ``house keeper'', and ``nurse''.

\subsection{Human Evaluation}
For each prompt, nine images are generated using three methods: a baseline method without debiasing, and two LLM-assisted debiasing methods employing GPT-4o and DeepSeek-V3. These images are arranged in a 3 $\times$ 3 grid, and evaluators assess pairs of images based on image quality, prompt reflection, and diversity of generations. Image quality refers to the aesthetic appeal, high resolution, natural appearance, and detailed refinement of the images. Prompt adherence measures the degree to which the generated images reflect the input text. Diversity of generations evaluates the variety of generated results, particularly whether the images avoid stereotypes and fixed patterns. For each criterion, evaluators rate the results on a 5-point scale, ranging from 1 (very poor) to 5 (very good). To facilitate relative comparisons, images generated by different models for the same input prompt are presented in consecutive questions. This comparative evaluation across the three criteria enables a detailed assessment of the proposed methods' relative strengths and limitations. We randomly selected 50 prompts from Stable Bias profession dataset and Parti Prompt dataset. The subset used for the human evaluation is detailed in Table\ref{tab:sd_subset} and Table\ref{tab:pp_subset} in the supplementary materials. Responses were collected from 20 evaluators, ensuring a diverse range of perspectives. 
\subsection{Non-parametric Evaluation}
Quantitative evaluation of generation diversity presents significant challenges. To address this, we adopt the clustering-based evaluation methodology proposed in Stable Bias~\cite{Luccioni_2023}, implementing a nonparametric diversity assessment using k-Nearest Neighbors (kNN)~\cite{fix1985discriminatory}. Specifically, we generate anchor images based on prompts structured as ``a portrait of a [ethnicity] [gender] at work,'' creating nine images for each combination of ethnicity and gender. This analysis employs 18 ethnic labels from Stable Bias and three gender categories: ``male'', ``female'', and ``non-binary'' (detailed ethnic labels are provided in the Appendix A of~\cite{Luccioni_2023}).

For image embeddings, we utilize Google's VertexAI multimodal embedding model\footnote{https://cloud.google.com/vertex-ai/docs/generative-ai/embeddings/get-multimodal-embeddings}, which converts 512 $\times$ 512 images into 1048-dimensional vector representations. For each prompt in the identity dataset, 30 unique images are generated, yielding a total of 54 $\times$ 30 $=$ 1620 images that serve as anchor points for classification. To examine local trends linked to specific professions, we follow the methodology outlined in \cite{naik2023social}, generating 210 images per method for five professions: ``CEO'', ``computer programmer'', ``doctor'', ``nurse'', and ``housekeeper''. The classification results are visualized to uncover potential biases or distinct patterns specific to each profession.

% In addition, to capture global trends across the entire profession dataset, we generate nine images per profession prompt for each method. These classification results provide an overarching perspective on diversity and potential biases in the generated outputs.

\section{Concluding Remarks}
In this paper, we proposed a novel approach utilizing multimodal LLMs to generate gesture-aware speech recognition transcripts for patients with language disorders. Our framework integrates verbal speech and iconic gestures, enabling the generation of enriched transcripts that capture the latent meaning conveyed through both modalities. Through extensive experimentation, we demonstrated that the proposed method effectively contextualizes incomplete or disfluent speech by incorporating gesture information, leading to more accurate and meaningful representations of the speaker's intent. These findings highlight the potential of our approach to significantly contribute to the field of speech and language therapy, offering innovative tools that can enhance the quality of life for individuals with language disorders by facilitating better communication and assessment methods.

\subsection{Ethical Statement} 
Our dataset was obtained from AphasiaBank with the approval of the Institutional Review Board (IRB) and adheres to the data sharing guidelines set by TalkBank\footnote{https://talkbank.org/share/ethics.html}. This includes complying with the Ground Rules for all TalkBank databases, which are based on the American Psychological Association Code of Ethics~\cite{american2002ethical}.

\subsection{Limitation \& Future Work} 
%This study represents a preliminary investigation into using multimodal LLMs to generate gesture-aware speech recognition transcripts. 
While the results are promising, we recognize several limitations and outline our plans to extend this work further.

One primary limitation is the absence of a definitive ground truth for quantitative evaluation. Since our model generates transcripts by synthesizing speech and gesture data from scratch, traditional benchmarks, such as comparisons with standard speech recognition outputs, are insufficient. Moreover, existing original transcripts lack gesture annotations, making direct comparisons challenging. In future work, we aim to address this gap by collaborating with certified pathologists to conduct qualitative assessments, such as A-B preference tests, to evaluate the effectiveness of gesture-enriched transcripts in accurately conveying the speaker's intentions.

To support quantitative evaluations, we plan to develop novel metrics that assess transcript quality, including grammar accuracy, semantic consistency, and the integration of multimodal information. Such metrics will provide a more objective basis for assessing our model's performance and facilitate comparisons with other multimodal and unimodal approaches.

Another limitation of this study is its focus on structured gestures from a specific task, the Peanut Butter Sandwich Task. While this task offers a controlled context for testing our approach, it does not encompass the diversity of gestures and communication patterns seen in everyday scenarios. As part of our future work, we plan to expand the scope of our model to include tasks such as the Cinderella Story Recall Task~\cite{bird1996cinderella}, which involves unstructured and complex narrative gestures. This expansion will allow us to evaluate the adaptability and robustness of our model in handling varied linguistic and gestural contexts.

In summary, while this study establishes a strong foundation for gesture-aware speech recognition, we aim to refine and extend our methods through collaborative qualitative evaluations, the development of robust quantitative metrics, and broader task applications. These efforts will ensure that our approach continues to evolve, ultimately contributing to more effective communication tools and interventions for individuals with language disorders.





\bibliographystyle{ACM-Reference-Format}
\bibliography{./text/aspdac_llmhls}


\end{document}
\endinput
%%
%% End of file `sample-sigconf.tex'.
