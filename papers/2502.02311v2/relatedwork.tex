\section{Related Works}
Multi-agent task allocation (MATA) has been extensively studied in the fields of robotics, autonomous systems, and artificial intelligence. Existing approaches can be broadly categorized into optimization-based methods, market-based mechanisms, and learning-based approaches. Furthermore, distinctions between centralized and decentralized task allocation, as well as homogeneous and heterogeneous multi-agent settings, are critical in determining the efficiency and adaptability of such systems.

\subsection{Optimization-Based Approaches}
Optimization techniques, e.g., the linear sum assignment problem (LSAP) and the Hungarian algorithm, have been widely applied in multi-agent task allocation \cite{goarin2024graph}. While these methods provide optimal solutions, they often rely on centralized computation, rendering them unsuitable for large-scale real-time applications. Ismail et al. \cite{ismail2017decentralized} proposed a novel decentralized-based Hungarian method to solve this problem. Another decentralized version of the Hungarian method is proposed by Xia et al. \cite{xia} to solve task allocation for underwater vehicles. These methods extend the Hungarian approach to decentralized settings by ensuring task allocation remains optimal as long as agent networks remain connected. Kong et al. \cite{greedy} proposed a new optimization strategy that combines the improved particle swarm optimization and the greedy IPSO-G algorithm. Moreover, two different stochastic approaches, the Genetic Algorithm (GA) and the Ant-Colony Optimization (ACO) algorithm, were introduced by \cite{genetic} to solve the multi-robot task allocation problem. Peter et al. \cite{neurofleets} introduced a swarm intelligence-based task allocation method utilizing decentralized decision-making and subgoal-based path formation.

\subsection{Market-Based Mechanisms for Task Allocation}
Market-based strategies, particularly auction-based methods, have been widely explored for decentralized task allocation. Zhong et al. \cite{zhong2018stable} proposed an extended auction-based driver-passenger matching system for ride-sharing applications. The system optimizes both stability and fairness in assignments by allowing drivers to bid on passengers based on individual profitability. Liu et al. \cite{liu2024multi} extended auction-based methods to multi-UAV systems, incorporating Dubins path-based flight cost estimation to refine task allocations. While auction-based strategies provide high adaptability, their efficiency heavily depends on accurate bid valuation models. An alternative hybrid approach is introduced in the Harmony Drone Task Allocation (DTA) method \cite{harmonyDTA}, which integrates a consensus-based auction mechanism with a gossip-based consensus strategy. The Harmony DTA optimizes multi-drone task allocation under complex time constraints by balancing task urgency with resource availability while minimizing communication load.

\subsection{Multi-Agent Deep Reinforcement Learning (DRL) in Task Allocation}

The integration of DRL into MATA has led to promising advancements, particularly through MARL frameworks. Agrawal et al. \cite{agrawal2023rtaw} introduced an attention-inspired DRL method for warehouse-based task allocation that optimizes decision-making efficiency and scalability. Moreover, recent studies explore Multi-Agent Proximal Policy Optimization (MAPPO) to enhance collaborative decision-making and mitigate non-stationary challenges in multi-agent environments \cite{mappo2023arxiv}. Additionally, it incorporates message pooling and weight scheduling mechanisms to enhance agent communication, improving efficiency in cooperative decision-making. Lowe et al. \cite{lowe2017multi} use Q-learning to do cooperative and competitive tasks. QMIX is a CTDE approach that shows \cite{rashid2018qmix} promising results in cooperative tasks.

\subsection{Graph Neural Networks (GNNs) for Decentralized Task Allocation}
Recent advances in graph-based learning methods have demonstrated substantial potential for improving decentralized task allocation. GNNs enable robots to model and process inter-agent relationships, allowing task allocations to be computed based on local graph structures rather than centralized control. Goarin et al. \cite{goarin2024graph} proposed DGNN-GA, a decentralized GNN-based goal assignment method that optimizes communication efficiency in multi-robot planning. Heterogeneous multi-agent systems pose additional complexities due to differences in robot capabilities, sensor modalities, and mobility constraints. The GATAR framework \cite{peng2024graph} introduced a graph-based task allocation framework for multi-robot target localization, specifically designed for heterogeneous robot systems. Furthermore, Blumenkamp et al. \cite{blumenkamp2022framework} develop a real-world framework for deploying decentralized GNN-based policies in multi-robot systems, facilitating seamless sim-to-real transfers. 

Despite advancements in MATA, several challenges remain unaddressed or partially solved. Many DRL-based solutions struggle to generalize to larger teams due to the inherent non-stationarity of multi-agent environments. While GNN-based models enhance decentralized coordination, ensuring robust communication under limited bandwidth conditions remains a challenge. The transition from simulation to real-world applications is hindered by the lack of standardized multi-agent benchmarks and sim-to-real transfer limitations. 
 Future research should focus on hybrid approaches that combine optimization, DRL, and GNNs to balance efficiency, adaptability, and real-time feasibility in multi-agent decentralized task allocation.