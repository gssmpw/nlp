
\subsection{\textbf{Preliminaries: Material Model}}
Building upon previous work~\cite{rodriguezpardo2023UMat}, we use a physically-based material model based on microfacets reflectance~\cite{burley2012physically}, into which we incorporate additional parameters to enable transmittance effects. Our material model aggregates a diffuse component (i.e. the material albedo) $\mathbf{A} \in \mathbb{R}^{3 \times x  \times y}$, with a grayscale, isotropic specular GGX~\cite{walter2007microfacet} lobe $s_{l,v} \in \mathbb{R}^{x  \times y}$, which depends on the surface normal $\mathbf{N}$, its specularity $\mathbf{S}$ and roughness $\mathbf{R}$. The shading model $f_{l,v}^\textrm{BSDF} \in \mathbb{R}^{4 \times x  \times y}$ for a particular light $l$ and camera $v$ has an additional transparency term which depends on the material binary opacity $\mathbf{O} \in \mathbb{Z}_2^{x  \times y}$ and its transmittance  $\mathbf{T} \in \mathbb{R}^{x  \times y}$ , as follows: 

\begin{align}
	\label{eq:mat_model}
		f_{l,v}^\textrm{BSDF} (\mathbf{A},\mathbf{N},\mathbf{S},\mathbf{R},\mathbf{O},\mathbf{T})   = \mathbf{O}\cdot( \underbrace{\frac{\mathbf{A}}{\pi} + s_{l,v}(\mathbf{N},\mathbf{S},\mathbf{R})}_\text{reflectance $f_{l,v}^\textrm{BRDF}$}+ \underbrace{(\mathbf{T}\cdot\mathbf{A}))}_\text{transmittance}%
\end{align}


The transmittance is modeled as the base albedo $\mathbf{A}$ modulated by a  gray scale value $\mathbf{T}$. This assumes that the light scattered through the material is a linear attenuation of the reflectance wavelength (albedo). Finally, both reflectance and transmission are weighted by the binary operator $\mathbf{O}$, which differentiates areas with partial and total transmission.
Finally, both reflectance and transmission are weighted by the binary operator $\mathbf{O}$, which differentiates areas with partial transmission from fully transparent pixels. The distinction between both $\mathbf{O}$ and $\mathbf{T}$, being the former just a particular threshold on the continuous transmittance $\mathbf{T}$, is due to its traditional use as a binary mask in several rendering methods, to reduce shader execution time by discarding pixels.



Although there are richer and more complex models for transmittance and sub-surface scattering phenomena (E.g.:~\cite{burley2015extending}), we find that this thin-layer diffuse transmission model suffices to represent a large proportion of materials that can be captured with a scanner, while having low requirements for real time visualization and less memory consumption that a multi-channel transmission map. 

\section{Method}
\label{sec:method}

Our method takes as input a single image of the material and estimates its spatially-varying SVBSDF material parameters, including reflection and transmission per-pixel coefficients. 
The input image can be obtained with any capture device that provides mostly \textit{uniform} lighting, such as the one provided by flatbed scanners. %
Our algorithm has two steps. 
In the first step, described in Section~\ref{sec:delighting}, we use a  cycle-consistent residual generative network to \textit{delight} the material and obtain an albedo-like reflectance map. After our processing, the resulting map lacks micro-reflections and shadows that might be originally present due to directional lighting hitting the material.
In the second step, described in Section~\ref{sec:svbsdf_estimation}, we use this image as input of an attention-guided U-Net that estimate the remaining material maps, to convey reflection and transmission. 
\begin{figure*}[t!]
	\centering
	\includegraphics[width=1.0\textwidth]{figs/pdf/method_del_v0.pdf}
	\caption{From an image $I_{l}$ captured with any flatbed scanner, we first estimate its albedo $I_d$ using a residual generative model $\mathcal{M}_{D}$, which removes specular highlights and shading. Taking $I_d$ as input, a second model $\mathcal{M}_{BSDF}$ estimates the rest of the SVBSDF, namely the surface normals, roughness, specular, transmittance, and opacity maps. These can be then rendered to generate photo-realistic images.    }
	\label{fig:method}
\end{figure*}

