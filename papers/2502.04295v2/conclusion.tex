\section{Conclusion}

% In this research, we have introduced \fullsysname{}, a novel approach that innovatively addresses the challenges of optimizing both the content and format of prompts in a unified manner. \sysname{} enhances LLM performance on various tasks by understanding the complex interplay between content and format. This approach represents a significant step forward in prompt engineering, suggesting the potential for more advanced optimization strategies.

% As LLMs evolve and tackle more complex tasks, the demand for sophisticated prompt engineering methods like \sysname{} increases. Our research underscores the limitations of traditional optimization and proposes a nuanced, model-specific strategy. \sysname{}'s dual exploration mechanism is a notable advancement, offering a flexible solution that improves LLM utility across different scenarios.

This paper introduces \fullsysname{}, a innovative methodology that concurrently optimizes both prompt content and format. \sysname{} incorporates the \textit{Prompt Renderer} and the \textit{Query Format} within a structured prompt template. By leveraging distinct optimization strategies, \sysname{} discovers high-performing prompts that outperform content-only methods, addressing a critical gap in existing research. Our results demonstrate the substantial significant influence of format on LLM performance, underscoring the necessity of a joint optimization approach. These findings emphasize the importance of integrating content and format considerations in prompt engineering. CFPO represents a significant advancement, empowering developers to design effective and robust prompts and unlocking the full potential of LLMs across diverse applications.
