\begin{figure}[H]
    \centering
    \includegraphics[width=\textwidth,keepaspectratio]{fig/GDRNPP.eps}
    \caption{Architecture of GDRNPP / GDR-Net \cite{wang2021gdr}}
    \label{fig:Architecture of GDRNPP}
\end{figure}

\section{Preliminaries}
\label{sec:preliminaries}
GDRNPP \cite{liuShanicelGdrnpp_bop20222024} is a 6D object pose estimation architecture, that constitutes an enhanced version of GDR-Net \cite{wang2021gdr}.
It estimates the pose of an object given an RGB image, by firstly detecting relevant image regions, containing the object region, then predicting relevant features in these image regions using Convolutional Neural Networks (CNN) in the form of a backbone and a subsequent Geo Net, based on which a PnP-Module directly regresses the rotation and translation from the learned features. Subsequently a depth-based pose refinement can be performed as an optional step.\\
%
We choose to adopt the GDRNPP architecture as a base for the proposed scalable 6D object pose estmation models, due to its high performance in the BOP Challenge \cite{BOPBenchmark6Da} and for its highly adoptable architecture, as demonstrated in a range of diverse extensions \cite{epro, stereo}.
To identify relevant parts of the architecture for inference time optimization, we subdivide the process of GDRNPP into six conceptual stages, i.e., \textit{Data Load}, \textit{Backbone}, \textit{Geo Head}, \textit{Data Process}, \textit{Patch PnP} and \textit{Data Process Afterwards}, which are illustrated with the architecture of GDR-Net (respectively GDRNPP) in Fig. \ref{fig:Architecture of GDRNPP}.
Preliminary experiments show that the major part of inference time are caused by the Data Load, Backbone and Geo Head stages (Fig. \ref{fig:runtimeportions}). Therefore, in the following we propose multiple changes to the architecture in these conceptual stages to reduce inference time, while aiming to preserve accuracy.

\begin{figure}[htbp]
    \centering
    \includegraphics[width=0.6\textwidth,keepaspectratio]{fig/processing_times.eps}
    \caption{Relative required runtime of the six conceptual stages during inference of GDR-Net / GDRNPP.}
    \label{fig:runtimeportions}
\end{figure}