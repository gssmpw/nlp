\section{Introduction}
In the field of computer vision, object pose estimation can be considered one of the most important tasks.
It aims to determine the rotation and translation of an object in three dimensions respectively.
Due to its wide range of applications such as robotics, autonomous driving, and augmented reality, researchers have attempted to tackle this problem.
%\cite{zhu2014single}.
In robotics, for example, high-quality 6D pose estimation plays a crucial role in autonomous robotic object manipulation in real-world scenarios such as picking up industrial bins \cite{deng2020self,}.\\
%强调速度在姿态估计中的重要意义
In applications requiring real-time feedback, low inference times play an important role in pose estimation. During robot navigation or autonomous driving, pose estimation systems must be capable of analyzing and responding rapidly and accurately to changes in the environment in order to prevent collisions and ensure the safety of the robot or passenger.
Fast pose estimation can be used for sports analysis and video surveillance to capture critical motions and behaviors, which can be utilized for real-time decision support, as well.
As a result, enhancing the processing speed of a pose estimation algorithm is essential to broaden its practical applications.\\
Often the inference time of a model is closely linked to its accuracy.
Based on this observation, there are a lot of applications that have strong constraints regarding their inference time budget.
For these applications, finding the most accurate model, that fulfils its inference time constraints is key.
Tan et al. \cite{tan2020efficientdet} introduce a family of algorithms for Object Detection, that perform well under different time budgets, giving the user the possibility to chose the variation, that achieves the highest accuracy given their time budget.
Pöllabauer et al. \cite{pollabauer2024fast} propose such a family of architectures for 6D pose estimation.\\
In this work, we propose multiple candidate architectures based on GDRNPP \cite{liuShanicelGdrnpp_bop20222024}, which constitutes an enhanced version of GDR-Net \cite{wang2021gdr}, that optimize inference time while maintaining or surpassing its accuracy on multiple BOP challenge \cite{BOPBenchmark6Da} datasets.
Furthermore, we introduce the Adaptive Margin-Dependent Iterative Selection (AMIS) algorithm that selects a subset out of candidate architectures, that constitute a beneficial trade-off between inference time and accuracy over multiple datasets.
The proposed AMIS algorithm can be applied to a diverse range of task-specific datasets, allowing the choice of a model that reflects the domain-specific requirements with regard to a trade-off between inference time and accuracy.

In summary,
\begin{itemize}
    \item we propose 40 candidate architectures based on modifications of GDRNPP \cite{liuShanicelGdrnpp_bop20222024} by adapting backbone and Geo Head architecture with the primary goal to enhance the resulting inference time while maintaining high accuracy,
    \item we present the AMIS algorithm, which identifies a suitable set of candidate models that constitute an optimal trade-off between inference time and their 6D pose estimation quality over multiple datasets, and
    \item we present quantitative results of the candidate models identified by the proposed AMIS algorithm for the LM-O, YCB-V, T-LESS, and ITODD datasets \cite{xiang2017posecnn, drost2017introducing, hodan2017t,}.
\end{itemize}