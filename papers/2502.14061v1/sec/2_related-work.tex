\section{Related Work}
\subsection{6D Object pose estimation}

The estimation of 6D positions from monocular RGB-D pictures has a multitude of applications in the industry, e.g., robot grasping, and therefore constitutes an important computer vision task overall.
Especially in the industrial area the need for real-time 6D object pose estimation is prevalent \cite{gorschluter2022survey}, therefore this work focuses on high accuracy applications under constraint inference time.\\
%
Hodan et al. \cite{hodan2024bop} give a comprehensive overview over the recent developments in 6D object pose estimation as benchmarked by the BOP challenge, that uses a wide variety of datasets and relevant metrics.
Li et al. \cite{li2019cdpn} increase accuracy and robustness by introducing Coordinates-based Disentangled Pose Network (CDPN) that uniquely separates the prediction of rotation from the prediction of translation, demonstrating a high level of flexibility and efficiency even with texture-less and occluded objects.
Labbe et al. \cite{labbe2020cosypose} propose Cosypose, which is able to estimate the 6D pose of multiple objects in scenes captured based on unknown camera viewpoints. From individual images, hypotheses of object poses are generated, which are matched  across different views to estimate both the camera viewpoints and the poses in a unified scene framework.
By employing an object-level bundle adjustment, the method innovatively manages object symmetries without requiring depth information, improving the accuracy and robustness  by minimizing reprojection errors in complex multi-object, multi-view scenarios.
Wang et al. \cite{wang2021gdr} propose GDR-Net, introducing a dense correspondence-based intermediate geometric representations, which in contrast to earlier strategies that set up 2D-3D correspondences and subsequently apply PnP/RANSAC strategies, allows for end-to-end training. As a result of this approach, the 6D pose can be directly regressed, combining the advantages of both direct and indirect methods.
GDRNPP \cite{liuShanicelGdrnpp_bop20222024} upgrades GDR-Net by introducing stronger domain randomization operations, such as background replacement and color enhancements, and most prominently replacing the ResNet-34 backbone with ConvNeXt.
We discuss the architecture of GDR-Net and GDRNPP in more detail in section \ref{sec:preliminaries}.
Haugaard and Buch \cite{haugaard2022surfemb} propose an unsupervised approach, that is employed to learn dense, continuous 2D-3D correspondence distributions on object surfaces without prior knowledge of visual ambiguities such as symmetry. They utilize a compact, fully connected key model and an encoder-decoder query model, both operating within object-specific latent spaces.
Zebrapose \cite{su2022zebrapose} introduces a discrete descriptor that provides a more detailed and accurate mapping of the object surface as compared to previous methods, which allows encoding the surface of an object more efficiently by incorporating a hierarchical binary grouping. Furthermore, they propose a novel coarse-to-fine training strategy that enhances the accuracy by enabling fine-grained correspondence prediction.


\subsection{Speed and Efficiency in Deep Learning}
Modern deep learning architectures have been specifically tailored to address the computational bottlenecks in 6D pose estimation. For instance, lightweight neural networks that incorporate depthwise separable convolutions, such as MobileNet and EfficientNet, have demonstrated significant reductions in computational complexity and latency without compromising accuracy by reducing the number of parameters and operations, thereby speeding up the inference time and allowing for scaling and using different configurations \cite{liang2021efficient, wang2024lightweight,}.\\
% \textbf{Algorithmic Enhancements}: 
Beyond architectural changes, algorithmic adjustments also play a crucial role. For example, employing more sophisticated loss functions that focus on critical parts of the pose estimation task or the application of better regularization techniques can lead to more efficient learning dynamics and prevent overfitting. \cite{liu2023linear, gonzalez2020effective}.\\
Pöllabauer et al. \cite{pollabauer2024fast} propose a set of scalable 6d pose estimation architectures over a wide scale of inference time budgets based on a fixed set of datasets from the BOP challenge \cite{BOPBenchmark6Da}. In contrast to that, we propose a strategy that is able to automatically identify architectures with a beneficial trade-off between inference time and accuracy for arbitrary 6D pose estimation datasets.