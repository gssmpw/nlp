\section{Results}

In this section, we provide a quantitative analysis of the proposed Geo Head architecture improvements (see Sec. \ref{sec:results:preliminary}), proposing 40 candidate architectures based on the best performing ones.
Furthermore, we show the results of the models selected by the proposed AMIS for multiple datasets (see Sec. \ref{sec:results:amis}), showing that the selected models outperform GDRNPP and provide a reasonable model selection regarding a desired trade-off between inference time and accuracy on multiple datasets.


\subsection{Preliminary Experiments}
\label{sec:results:preliminary}

In sections \ref{sec:methodology:geohead} and \ref{sec:methodology:misc} we proposed alterations to the Geo Head architecture of GDRNPP with the goal of reducing inference time while maintaining a high level of accuracy.
We evaluate the performance of GDRNPP using these Geo Head architectures on the LM-O dataset.
The results of this evaluation are illustrated in Tab. \ref{tab:Comparison of Different Modifications on LM-O}.
These results indicate, that the proposed general optimizations \textit{B0} and \textit{C0} positively influence the inference time, while maintaining the desired level of precision, measured by MSPD, MSSD, VSD and AR, which give insight into a wide variety of aspects of 6D object pose estimation provided by the BOP Toolkit\cite{hodan2024bop}.
Furthermore, the results indicate, that the introduced variations of the Geo Head architecture reduce the inference time even further.
We ablate the influence of architectural changes in more detail in section \ref{sec:results:geohead}.\\
%
Based on the results, we chose the Geo Head architectures \textit{C0}, \textit{E0}, \textit{F0}, and \textit{F2} due to their high performance in comparison to the required inference time.
We combine those architectures with the previously identified candidate backbones to obtain 20 combinations.
Those candidates are assigned with a number for their identification according to Tab. \ref{tab:Encode Experiment Candidates}.
We evaluate each of these architecture combinations with and without the optional refinement step of GDR-NET in the subsequent experiments, essentially resulting in 40 candidate architectures.

\begin{table}[H]
\centering
\caption[Comparison of Different Modifications on LM-O]{\raggedright
Quantitative results of the proposed Geo Head candidates ond the LM-O dataset regarding MSPD, MSSD, VSD, AR and Inference time. The Geo Head architectures that we chose to construct candidate architecture are indicated in \textbf{bold}.}
\label{tab:Comparison of Different Modifications on LM-O}
\begin{tabular}{@{}l|l|c|c|c|c|c@{}}
\toprule
\multirow{2}{*}{\textbf{Row}} & \multirow{2}{*}{\textbf{Method}} & \textbf{MSPD} & \textbf{MSSD} & \textbf{VSD} & \textbf{AR} & \textbf{Time} \\
 &  &  \textbf{\%} & \textbf{\%} & \textbf{\%} & \textbf{\%} & \textbf{ms}\\
\midrule
A0 & GDRNPP & 87.14 & 67.03 & 52.38 & 68.85 & 28.8 \\
\midrule
B0 & A0: Optimizations & 87.14 & 67.07 & 52.38 & 68.86 & 26.72 \\
\midrule
\textbf{C0} & \textbf{B0: Data Process Optimization} & 87.14 & 67.03 & 52.38 & 68.86 & \textbf{25.29} \\
\midrule
D0 & C0: Add connection in location 1 & \textbf{87.66} & \textbf{67.29} & \textbf{52.53} & \textbf{69.16} & 28.79 \\
D1 & C0: Add connection in location 2 & 86.89 & 66.75 & 51.91 & 68.82 & 29.17 \\
D2 & C0: Add connection in location 3 & 87.39 & 66.64 & 52.01 & 68.68 & 25.68 \\
\midrule
\textbf{E0} & \textbf{C0: Vanilla Geo Head$\rightarrow$variation 1} & \textbf{87.11} & \textbf{67.78} & \textbf{52.58} & \textbf{69.15} & \textbf{24.05} \\
E1 & E0: Add connection in location 1 & 86.71 & 67.2 & 52.11 & 68.7 & 24.47 \\
E2 & E0: Add connection in location 2 & 87.16& 67.13& 52.42& 68.9 & 24.32 \\
E3 & E0: Add connection in location 3 & 87.30& 66.83& 51.90 & 68.68 & 24.13 \\
\midrule
\textbf{F0} & \textbf{C0: Vanilla Geo Head$\rightarrow$variation 2} & 85.23 & 65.40 & 50.61 & 67.08 & 22.66 \\
F1 & F0: Add connection in location 1 & 85.29& 65.72& 50.88 & 67.3 & 22.22 \\
\textbf{F2} & \textbf{F0: Add connection in location 2} & \textbf{84.96}& \textbf{66.26}& \textbf{51.27}& \textbf{67.49} & 23.20 \\
F3 & F0: Add connection in location 3 & 81.46& 58.30& 44.82 & 61.53 & 23.11 \\
\midrule
G0 & F2: convnext\_base$\rightarrow$convnextv2\_nano & 83.47 & 63.00 & 48.15 & 64.87 & \textbf{17.12} \\
\midrule
H0 & E0: convnext\_base$\rightarrow$convnextv2\_base & 87.31 & 67.62 & 52.63 & \textbf{69.18} & 24.05 \\
\bottomrule
\end{tabular}
\end{table}

\begin{table}[H]
\centering
\caption[Coding Experiment Candidates]{\raggedright Candidate Models consisting of one of the previously identified backbones and adapted Geo Head candidates (F0, F1, E0, E1).} 
\label{tab:Encode Experiment Candidates}
\begin{tabular}{@{}l|cccc@{}}
\toprule
\multirow{2}{*}{\textbf{Backbone}} & \multicolumn{4}{c}{\textbf{configuration}} \\
 & F0 & F2 & E0 & C0 \\
\midrule
fastvit\_s12 & 1 & 2 & 3 & 4 \\
convnextv2\_nano & 5 & 6 & 7 & 8 \\
convnext\_base & 9 & 10 & 11 & 12 \\
convnextv2\_base & 13 & 14 & 15 & 16 \\
maxxvit\_small & 17 & 18 & 19 & 20 \\
\bottomrule
\end{tabular}
\end{table}

\subsection{Geo Head}
\label{sec:results:geohead}
In this section, we summarize the findings from the proposed alterations to the Geo Head part of GDR-Net / GDRNPP as evaluated on the LM-O dataset (see Tab. \ref{tab:Comparison of Different Modifications on LM-O}).\\
Our results indicate that the proposed optimizations (see Sec. \ref{sec:methodology:misc}) improve the inference time without a relevant impact on the measured accuracy.\\
The proposed Geo Head variation 1 leads to a speed-up in inference time and an improvement in accuracy, constituting it a successful optimization to the Geo Head architecture.
In contrast to that, the proposed Geo Head variation 2, further increased inference time, while reducing accuracy significantly. The availability of such an adaption, however, is highly beneficial when looking for highly accurate 6D pose estimation models under varying inference time budgets.\\
Adding connections within the Geo Head was generally found to negatively impact inference speed, particularly in the vanilla Geo Head setup, where it was deemed not worthwhile due to significant slowdowns without notable accuracy benefits. However, in the second variation of Geo Head, adding a connection at the first location improved both speed and accuracy, making it a promising adjustment for balancing performance metrics.

\begin{figure}[b]   
    \includegraphics[width=\textwidth,keepaspectratio]{fig/final_seperate_dataset.eps}
    \caption[Average of MSPD, MSSD, VSD]{Average over MSPD, MSSD, VSD of the 5 candidate architectures with desireable trade-off between inference time and accuracy using the AMIS algorithm on LM-O (left), YCB-V(left), T-LESS (center) and ITODD (right) dataset.}
    \label{fig:Average of MSPD, MSSD, VSD_reference}
\end{figure}

\subsection{AMIS}
\label{sec:results:amis}


We employ the aforementioned AMIS algorithm (see Sec. \ref{sec:methodology:amis}) to identify a suitable subset of models within the previously identified candidate architectures on the IMO, LM-O, YCB-V, T-LESS, and ITODD datasets \cite{xiang2017posecnn, drost2017introducing, hodan2017t,}.
The results of the five identified candidates are illustrated in Fig. \ref{fig:Average of MSPD, MSSD, VSD_reference} on scatterplot showing inference time and 6D object pose estimation accuracy.
Furthermore, the average results of the models are illustrated in Tab. \ref{tab:Result of selected Candidates vs GDRNPP}.\\
The experimental results show, that even with minimal time budget, i.e., when we expect our model to perform the fastest inference, the required inference time is reduced by 35\% in comparison to GDRNPP, while the achieved performance only drops by 3\% measured by the average of MSPD, MSSD, and VSD.
As the time budget increases, the performance of our candidates also gradually improves. This variation can adapt to the complex scenarios of different time and accuracy requirements in industrial environments. It is noteworthy that compared to GDRNPP, using about 31\% additional time can lead to approximately a 25\% improvement in performance.
Fig. \ref{fig:Average of MSPD, MSSD, VSD_reference} furthermore shows that the selected models show an increase in accuracy with increased inference time for all 4 datasets.

\begin{table}[ht]
\centering
\caption{Quantitative results of the candidate architectures identified by the AMIS algorithm, showing that they constitute a desirable trade-off between inference time and accuracy.}
\label{tab:Result of selected Candidates vs GDRNPP}
\begin{tabular}{@{}l|cccc@{}}
\toprule
\multicolumn{5}{c}{\textbf{5 selected Candidates vs GDRNPP}}\\
\midrule
\multirow{2}{*}{\textbf{Candidates}}& \textbf{Average of} & \multirow{3}{*}{\textbf{ADD}}& \textbf{Average of }&  \multirow{3}{*}{\textbf{Time}}\\
& \textbf{MSPD, MSSD} &  & \textbf{MSPD, MSSD,} &   \\
\multirow{2}{*}{\textbf{Number}}& \textbf{VSD} &  & \textbf{VSD, ADD} &   \\
& \% & \% & \% & \%  \\
\midrule
6 without ref. & -2.82 & -16.62 & -3.42 & -35.71\\
\midrule
9 without ref. & +2.06 & -9.82 & +1.66 & -24.86\\
\midrule
11 without ref.{ } & +10.12 & +3.0 & +10.44 & -17.61\\
\midrule
7 with ref. & +16.24 & +41.28 & +20.56 & +15.81\\
\midrule
11 with ref. & +25.14 & +50.83 & +30.35 & +30.74\\
\bottomrule
\end{tabular}
\end{table}