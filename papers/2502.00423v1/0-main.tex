\documentclass[12pt,letterpaper]{article}

%----------------------------------------------------
%
% Packages
%
\usepackage{amsmath,amssymb,amsfonts,amsthm,bbm}
%\usepackage{amsmath, amsthm, amsfonts, bm}  % bm for \boldsymbol with amsmath % bbm for \mathbbm{1}
\usepackage{mathtools}   % for \norm

\usepackage{enumitem}   % for customized enumerate

\usepackage{authblk} % for multiple author
\renewcommand{\Affilfont}{\normalsize}
\renewcommand{\Authfont}{\normalsize}

\usepackage[titletoc,title]{appendix}  % Appendix A

%%
%%  Style
%%

%%   Font
%%   https://tug.org/FontCatalogue/
\usepackage{kpfonts}
%\usepackage[sfmath]{kpfonts}
% \renewcommand*\familydefault{\sfdefault}
\usepackage[T1]{fontenc}

%%   Color
%% set ref link color, a triad of colors
\usepackage[table,xcdraw,dvipsnames]{xcolor}   % for \textcolor, \colorlet, ...
\usepackage{hyperref}
\newcommand\myshade{85}
\colorlet{mylinkcolor}{YellowOrange}
\colorlet{mycitecolor}{Aquamarine}
\colorlet{myurlcolor}{violet}


\hypersetup{
  linkcolor  = mylinkcolor!\myshade!black,
  citecolor  = mycitecolor!\myshade!black,
  urlcolor   = myurlcolor!\myshade!black,
  colorlinks = true,
}

%%
%% Figures
%%
\usepackage{caption}
\usepackage{subcaption}

\usepackage{comment}
%%
%% algorithms
%%
%\usepackage{algorithm}
%\usepackage{algpseudocode}
\section{The general case: Proof of \texorpdfstring{\Cref{thm:main-decomp}}{Theorem 1.6}}\label{sec:algo}

First, we show that data structure of \Cref{l:max_min_query} can be used to compute distances witnessed by shortest paths that pass through a constant-size separator.

\begin{lemma}\label{l:single_adhesion}
Fix a constant $k \in \mathbb{N}$. There exists an algorithm which as the input receives an edge-weighted graph $G$ on $n$ vertices and $m$ edges together with a partition of its vertices into three sets $A, B, C$ such that $|B| \leq k$ and there are no edges between $A$ and $C$, and as the output computes $\max_{c \in C} \dist(a, c)$ for every $a \in A$. The running time is $\Oh(m \log n + n \log^{k - 1} n)$.
\end{lemma}

\begin{proof}
Let $B = \{b_1, \ldots, b_k\}$. For any $a \in A, c \in C$, we have $\dist(a, c) = \min_{i \in [k]} \dist(a, b_i) + \dist(c, b_i)$. First, we run Dijkstra's algorithm from every vertex in $B$ to find $\dist(v, b_i)$ for every $v \in V(G)$ and $i \in [k]$. Next, we use \Cref{l:max_min_query} to construct a data structure $\mathbb{D}$ for the point set $\{(\dist(c, b_1), \dots, \dist(c, b_k))\colon c\in C\}\subseteq \mathbb{R}^k$. Now, the value $\max_{c \in C} \dist(a, c)$ for any given $a$ is equal to the answer of $\mathbb{D}$ to the query with argument $(\dist(a, b_1), \dots, \dist(a, b_k))$.
\end{proof}

After computing the distances over a constant-size separator, we will use the following observation to simplify one of the sides of the separation.

\begin{lemma}\label{l:inserting_paths}
Let $G$ be a edge-weighted connected graph and let $A, B, C$ be a partition of its vertices such that there are no edges between $A$ and $C$. For every pair of vertices $u, v \in B$, let $P_{u, v}$ be any shortest path from $u$ to $v$ with all internal vertices in $C$ (assuming such a path exists).

Let $G'$ denote a graph obtained from $G[A \cup B]$ by adding an edge from $u$ to $v$ of weight equal to the length of $P_{u, v}$, for all $u, v \in B$ for which $P_{u, v}$ exists. Then,  $$\dist_G(s, t) = \dist_{G'}(s, t)\qquad\textrm{for all }s,t\in A\cup B.$$
\end{lemma}
\begin{proof}
Let $G''$ be the graph obtained by adding new edges of $G'$ to $G$.
Fix any $s, t \in A \cup B$ and let $P$ denote the shortest path from $s$ to $t$ in $G''$ which minimizes the number of vertices from $C$ visited. Naturally, the weight of $P$ is equal $\dist_G(s, t)$. Assume that such path visits at least one vertex of $C$. Then, the path $P$ is of the form $s \xrightarrow{P_1} x \xrightarrow{P_2} y \xrightarrow{P_3} t$, where $x, y \in B$ and all the internal vertices of $P_2$ are in $C$. By the construction of $G'$, $P_2$ can be replaced with a direct edge from $x$ to $y$ of the same weight. We obtain a same weight path with a smaller number of vertices of $C$ visited, which is a contradiction. Therefore, $P$ is entirely contained in $A \cup B$, hence it exists in $G'$. This shows that $\dist_G(s, t) = \dist_{G'}(s, t)$.
\end{proof}


The next lemma encapsulates the main algorithmic content of the proof of \Cref{thm:main-decomp}. The algorithm will split the tree decomposition provided on input into smaller parts for which the eccentricities are easier to calculate. We use the following lemma to handle a single such part.
\begin{lemma}\label{l:star}
Fix constants $k, g \in \mathbb{N}, 0 < \delta < \frac{1}{54}$. Assume we are given $n \in \mathbb{N}$, an edge-weighted graph $G$ on at most $n$ vertices with a weight function $w \colon E(G) \to \mathbb{N}$, a vertex subset $A$ and a collection of non-empty vertex subsets $V_0, V_1, \dots, V_\ell$ satisfying the following conditions:
\begin{itemize}[nosep]
	\item The sum of weights of all the edges in $G$ is bounded by $\Oh(n)$.
	\item $V(G) \setminus A = V_0 \cup V_1 \cup \dots \cup V_\ell$.
	\item $|A| \leq k$.
	\item For every $i \in [\ell]$, $G[V_i \setminus V_0]$ is connected, $N_G(V_i \setminus V_0) = V_i \cap V_0$, $|V_i| = \Oh(n^\delta)$, and $|V_0 \cap V_i| \leq 4$.
	\item For all $i, j \in [\ell], i \neq j$, $V_i \setminus V_0$ and $V_j \setminus V_0$ are disjoint and non-adjacent in $G$.
	\item Every edge $uv \in E(G)$ with $u, v \not\in A$ is contained in $G[V_i]$ for some $i\in \{0,1,\ldots,\ell\}$.
	\item The graph obtained by taking $G[V_0]$ and adding a clique on $V_0 \cap V_i$ for every $i \in [\ell]$ has Euler genus bounded by $g$.
\end{itemize}
Then, we can compute the eccentricity of every vertex of $G$ in time $\Oh \left( n^{1 + \frac{150 + 54 \delta}{151}} \log^k n \right)$.
\end{lemma}

\begin{proof}
Fix $\delta' = \frac{1 + 97 \delta}{151}$; we have $\delta' - \delta = \frac{1 - 54\delta}{151} > 0$.
Let $E_i$ denote the set of edges with one endpoint in $V_i$ and the other endpoint in $V_i \setminus V_0$. For $i \in [\ell]$, we shall say that $V_i$ is {\em{heavy}} if the sum of weights of $E_i$ is larger than $n^{\delta'}$. Since the sets $E_i$ are pairwise disjoint and the total sum of weights of all the edges is bounded by $\Oh(n)$, the number of heavy subsets is bounded by $\Oh(n^{1 - \delta'})$. Without loss of generality, we may assume that $V_{\ell' + 1}, \dots, V_\ell$ are heavy and $V_1, \dots, V_{\ell'}$ are not, for some $\ell'\in \{0,\ldots,\ell\}$.


For any source vertex $s$, we can calculate distances from $s$ to every vertex of $G$  using breadth first search in time $\Oh(\sum_{e \in E(G)} w(e)) = \Oh(n)$.
In particular, for every $\ell' < i \leq \ell$, we can compute the distances from every vertex of $V_i$ to every vertex of $G$ in total time $\Oh(n^{2 - \delta' + \delta})$, because $$|V_{\ell'+1}\cup \ldots\cup V_{\ell}|\leq n^{1-\delta'}\cdot \Oh(n^\delta)=\Oh(n^{1-\delta'+
\delta}).$$
Additionally, we calculate distances $\dist_G(a, v)$ for every $a \in A, v \in V(G)$ in time $O(n)$.

For every $i \in [\ell]$ and $u,v \in V_0 \cap V_i$, there exists a shortest path $P_{i,u,v}$ from $u$ to $v$ with all internal vertices belonging to $V_i - V_0$ due to the assumption that $G[V_i - V_0]$ is connected and $N_G(V_i - V_0) = V_i \cap V_0$. Therefore, the distance from $u$ to $v$ is bounded by the sum of weights of edges in $E_i$. In particular, for $i \in [\ell']$, $\dist_G(u, v) \leq n^{\delta'}$.

We define $\widetilde{G}$ to be the graph obtained by taking $G[A \cup V_0 \cup \dots \cup V_{\ell'}]$ and applying the following operation for every $i \in \{\ell' + 1, \dots, \ell\}$:
for each pair of vertices $u, v \in A \cup (V_0 \cap V_i)$, add an edge in $\widetilde{G}$ between $u$ and $v$ with weight equal to the total weight of $P_{i,u,v}$. For a fixed $i, u$, we can find $P_{i, u, v}$ for all $v$ using breadth first search in time $\Oh(n)$. Taking a sum over all $i, u$, we get that $\tilde{G}$ can be computed in total time $\Oh(n^{2 - \delta'})$.


\begin{claim}\label{cl:wG}
The sum of the edge weights in $\widetilde{G}$ is $\Oh(n)$. Moreover, for all $u, v \in V(\widetilde{G})$, we have $\dist_{\widetilde{G}}(u, v) = \dist_{G}(u, v)$.
\end{claim}

\begin{proof}
Consider $i \in \{\ell' + 1, \dots, \ell\}$ and any $u, v \in A \cup (V_0 \cap V_i)$ for which we added an edge. Its weight is bounded by the sum of weights of edges in $E_i$. Therefore, the total weight of all edges added is at most
$$
\sum_{i \in \{\ell' + 1, \dots, \ell\}} \left( |A \cup (V_0 \cap V_i)|^2 \sum_{e \in E_i} w(e) \right) \leq (4 + k)^2 \sum_{e \in E(G)} w(e) = \Oh(n).
$$
This proves the first part of the claim.

For the second part of the claim, consider any $i \in \{\ell' + 1, \dots, \ell \}$ and observe that by our assumptions, $A \cup (V_0 \cap V_i)$ separates $(V_0 \cup \dots \cup V_{\ell'} \cup V_{i + 1} \cup \dots \cup V_\ell) \setminus V_i$ from $V_i \setminus V_0$. Hence it suffices to repeatedly apply \Cref{l:inserting_paths}.
\end{proof}

For every $u \in V(\widetilde{G})$, we have $\ecc_G(u) = \max(\ecc_{\widetilde{G}}(v), \max_{v \in V(G) \setminus V(\widetilde{G})} \dist_G(u, v))$. Note, that we already know all the distances $\dist_G(u, v)$ for $v \in V(G) \setminus V(\widetilde{G})$. Similarly, we can already compute $\ecc_G(u)$ for every $u \in V(G) \setminus V(\widetilde{G})$. Therefore, it remains to compute $\ecc_{\widetilde{G}}(v)$ for each $v \in V(\widetilde{G})$. Our goal is to show that this can be done efficiently using \Cref{l:main_ecc}.

Now, let $G'$ be the graph obtained from $\tilde{G}$ by replacing every edge $e$ non-indicent to $A$ with $w(e)\geq 2$ with a path of length $w(e)$ consisting of unit-weight edges. This operation again preserves the distances. Since the sum of edge weights in $\tilde{G}$ is of $\Oh(n)$, the total number of vertices in $G'$ is of $\Oh(n)$. For $0 \leq i \leq \ell'$, we write $V'_i$ to denote the set $V_i$ together with all the vertices added as a part of a path between two endpoints in $V_i$.
As $V_i$ is not heavy for each $i\in [\ell']$, we have
$$
|V'_i \setminus V'_0| \leq |V_i| + \sum_{e \in E_i} w(e) = \Oh(n^{\delta'})\qquad \textrm{for all }i\in [\ell'].
$$

Let $G_0$ denote the graph $G'[V'_0]$ and let $G_0^*$ denote the graph $G'- A$ with $V'_i - V'_0$ contracted to a single vertex $v_i^*$, for each $i \in [\ell']$; note that, all edges of $G_0$ and $G_0^*$ have unit weight.

\begin{claim}
	The graph $G_0^*$ is does not contain $K_{t}$ as a minor, where $t = \Oh(\sqrt{g})$.
\end{claim}

\begin{proof}
Let $\bar{G}_0$ denote the graph obtained by taking $G_0$ and adding a clique on $V_0 \cap V_i$ for every $i \in [\ell']$.
By lemma assumptions and the fact that subdividing edges does not increase the Euler genus, $\bar{G}_0$ has Euler genus at most $g$. In particular, $\bar{G}_0$ is $K_{t'}$-minor-free for some $t' = \Oh(\sqrt{g})$, because the Euler genus of $K_{t'}$ is $\Omega({t'}^2)$.

Similarly, let $\bar{G}_0^*$ be the graph obtained by taking $G_0^*$ and adding a clique on each $V_0 \cap V_i$.
Note, that $\bar{G}_0^* - \{v_1^*, \dots, v_{\ell'}^*\}$ is precisely $\bar{G}_0$. Let $t = \max(t', 6)$.
Recall that a minor model of a clique $K_t$ consists of $t$ pairwise vertex-disjoint connected subgraphs, called
branch sets, such that there is at least one edge between each pair of the branch sets.
Consider a minor model $\varphi$ of $K_{t}$ in $\bar{G}^*_0$.
Note that $\varphi$ cannot contain any singleton branch set of the form $\{v^*_i\}$, for the degree of $v^*_i$ in $\bar{G}^*_0$ is at most $4 < t - 1$. Furthermore, since $N_{\bar{G}^*_0}(v^*_i) = V_0 \cap V_i$, any branch set containing $v^*_i$ and at least one other vertex contains some $u \in V_0 \cap V_i$, and $N_{\bar{G}^*_0}(v^*_i)\subseteq N_{\bar{G}^*_0}(u)$, hence removing $v^*_i$ from this branch set preserves the model. Therefore, we can assume without loss of generality that all branch sets of $\varphi$ are disjoint from $\{v^*_1, \dots, v^*_{\ell'}\}$, hence $\varphi$ is a minor model of $K_{t}$ in $\bar{G}_0$. This is a contradiction, as $t \geq t'$ and $\bar{G}_0$ is $K_{t'}$-minor-free. Therefore, $\bar{G}_0^*$ is $K_t$-minor-free, hence $G_0^*$ also.
\end{proof}

Let $\rho' = \frac{2 - 108 \delta}{151} > 0$. The graph $G^*_0$ is a unit-weight graph and is $K_{t}$-minor-free.
Hence, by applying \Cref{t:r_division} to $G^*_0$ (with $\varepsilon = \rho'/2$)
we obtain an $n^{\rho'}$-division $\mathcal{R}_0$ in time $\Oh(n^{1 + \rho'})$.
We extend it to $G' - A$ by mapping every contracted vertex $v^*_i$ to $N_{G' - A}[V'_i - V'_0] = (V'_i - V'_0) \cup (V_0 \cap V_i)$. Formally, we put $V''_i \coloneqq N_{G' - A}[V'_i - V'_0]$ and 
$$
\mathcal{R} \coloneqq \left\{ (R_0 \cap V'_0) \cup \bigcup_{i \colon v^*_i \in R_0} V''_i \colon R_0 \in \mathcal{R}_0 \right\}.
$$

Now, we argue that $\mathcal{R}$ is a reasonable division of $G' - A$. Clearly, all sets $R \in \mathcal{R}$ are connected in $G' - A$. Pick any $R \in \mathcal{R}$ and let $R_0$ be its corresponding set in $\mathcal{R}_0$.
Every vertex $v^*_i$ is mapped to a set of size $\Oh(n^{\delta'})$, therefore
$$|R| \leq |R_0| \cdot \Oh(n^{\delta'}) = \Oh(n^{\rho' + \delta'}).$$

By our construction, for every $i \in [\ell']$, $R$ is either disjoint from $V'_i - V'_0$ or contains whole $N_{G' - A}[V'_i - V'_0]$. This means that no vertex belonging to any $V'_i - V'_0$ can be in $\partial R$, hence $\partial R \subseteq V'_0$.

Pick any $u \in \partial R \cap R_0$. Assume that $u \not\in \partial R_0$. Then every vertex of $N_{G_0^*}(u)$ must be in $R_0$, hence $N_{G - A'}(u) \subseteq R$, which is a contradiction. This means that $\partial R \cap R_0 \subseteq \partial R_0$.

Pick any $u \in \partial R - R_0$. Then, $u \in V_0 \cap V_i$ for some $i \in [\ell']$ such that $v_i^* \in R_0$. Moreover, $v_i^* \in \partial R_0$ and is adjacent to $u$ in $G_0^*$. The number of such $u$ is bounded by $4 |\partial R_0 \cap \{ v_1^*, \dots, v_{\ell'}^* \}|$.

Putting two cases together, we obtain:
$$
\sum_{R \in \mathcal{R}} |\partial R| = \sum_{R \in \mathcal{R}} \left(|\partial R \cap R_0| + |\partial R - R_0|\right) \leq \sum_{R_0 \in \mathcal{R}_0} \left(|\partial R_0| + 4 |\partial R_0 \cap \{ v_1^*, \dots, v_{\ell'}^* \}|\right) = \Oh(n^{1 - \frac{1}{2}\rho'}).
$$

It remains to show the following claim.

\begin{claim}
Pick any $R \in \mathcal{R}, s_R \in R$. The number of different distance profiles on $R$ relative to $s_R$ in $G' - A$ is of $\Oh(n^{48\rho' + 54\delta'})$.
\end{claim}
\begin{proof}
We look at every vertex $v \in V(G') \setminus A$ and consider three cases: $v \in R$, $v \in V'_0$, and $v \in V'_i \setminus (V'_0 \cup R)$ for some $i \in [\ell']$. By our construction, $R \cap V'_0$ is non-empty, hence w.l.o.g. we can assume that $s_R \in V'_0$ as whether two vertices have the same profile on $R$ is independent of the choice of the pivot vertex.

In the first case, there are at most $|R| = \Oh(n^{\rho' + \delta'})$ such vertices, hence they realise at most that many profiles.

In the second case, we want to observe that profile of any vertex $v \in V'_0$ on $R$ depends only on its profile on $R \cap V'_0$ (relative to $s_R$). Pick any $t \in R - V'_0$. Then $t \in V'_i - V'_0$ for some $i \in [\ell']$, $V_i \cap V_0 \subseteq R \cap V'_0$, and every path from $v$ to $t$ intersects $V_i \cap V_0$. In particular, distances from $v$ to vertices of $V_i \cap V_0$ determine its distance to $t$, which proves the observation.

Let $\tilde{G}_0$ denote the graph obtained by taking $G'[V'_0]$ and for every $i \in [\ell'], u, v \in V_0 \cap V_i$ adding a disjoint path from $u$ to $v$ of length $\dist(u, v)$. Let $P_i$ denote the vertex set of paths added between $V_0 \cap V_i$. For every $t \in V'_0$ we have $\dist_{G' - A}(v, t) = \dist_{\tilde{G}_0}(v, t)$, so it suffices to bound the number of profiles on $R \cap V'_0$ in $\tilde{G}_0$. By our assumptions, $\tilde{G}_0$ has Euler genus bounded by $g$ and all $P_i$ are of size $\Oh(n^{\delta'})$.

Let $R_0$ be the set of $\mathcal{R}_0$ corresponding to $R$. Let $\tilde{R}_0$ denote the set $(R \cap V'_0) \cup \bigcup_{i : v^*_i \in R_0} P_i$. Such set is connected in $\tilde{G}_0$. Moreover, similarly to $R$, its size is $\Oh(n^{\rho' + \delta'})$. Applying \Cref{thm:distprofiles}, we get that the number of distance profiles on $\tilde{R}_0$ in $\tilde{G}_0$ is $\Oh(n^{12(\rho' + \delta')})$, which also bounds the number of profiles on $R$ in $G' - A$ realised by $V'_0$.

For the third case, assume $v \in V'_i \setminus (V'_0 \cup R)$ for some $i\in [\ell']$. Every path from $v$ to any vertex of $R$ in $G' - A$ intersects $V_i \cap V_0$. Let $v_1, \dots v_p$ be the vertices of $V_i \cap V_0$, where $p \leq 4$. The profile of $v$ on $R$ is then determined by the following:
\begin{itemize}[nosep]
 \item[(a)] the profile of each $v_j$ on $R$,
 \item[(b)] $\dist_{G' - A}(v, v_j) - \dist_{G' - A}(v, v_1)$ for each $2 \leq j \leq p$, and
 \item[(c)] $\dist_{G' - A}(s_R, v_j) - \dist_{G' - A}(s_R, v_1)$ for each $2 \leq j \leq p$ where $s_R$ is some pivot vertex of $R$.
\end{itemize}
By the previous case, the number of distance profiles of each $v_j$ is $\Oh(n^{12(\rho' + \delta')})$. The distances between $v$ and $v_j$ are bounded by $|V'_i|$, hence each quantity described in (b) can take $\Oh(n^{\delta'})$ different possible values. Similarly, since $v_1$ and $v_j$ are connected via $V'_i$, $|\dist_{G' - A}(s_R, v_j) - \dist_{G' - A}(s_R, v_1)| \leq \Oh(n^{\delta'})$. The number of different possible profiles of such $v$ is therefore bounded by $\Oh(n^{48(\rho' + \delta') + 6\delta'}) = \Oh(n^{48\rho' + 54\delta'})$. This finishes the proof of the claim.
\end{proof}

Now we can apply \Cref{l:main_ecc} to graph $G'$ with apex set $A$, $X = V(\widetilde{G})$, and the following constants: $$\rho = \rho' + \delta',\qquad \gamma = 1 - \frac{1}{2}\rho',\quad \textrm{and}\quad \alpha = 48\rho' + 54 \delta'.$$ This allows us to calculate all $V(\widetilde{G})$-eccentricities in $G'$ in time
$$
\Oh \left( \left(
	n^{ 2 - \frac{1}{2} \rho' } +
	n^{ 1 + 48\rho' + 54 \delta' }
\right) \log^k n \right) =
\Oh \left( n^{1 + \frac{150 + 54 \delta}{151}} \log^k n \right).
$$
Since for each $v\in V(\widetilde{G})$ we have $\ecc_{\widetilde{G}}(v) = \max_{u \in V(\widetilde{G})} \dist_{\widetilde{G}}(v, u) = \max_{u \in V(\widetilde{G})} \dist_{G'}(v, u)$, this means that we have successfully computed all the eccentricities in $\widetilde{G}$; and as we argued, this is enough to compute all the eccentricities in $G$ as well.

Finally, the total running time of the algorithm is
$$
\Oh \left( n^{1 + \frac{150 + 54 \delta}{151}} \log^k n + n^{2 - \delta' + \delta} \right) =
\Oh \left( n^{1 + \frac{150 + 54 \delta}{151}} \log^k n \right).
$$\qedhere
\end{proof}


\begin{lemma}\label{l:star2}
Fix constants $k, g \in \mathbb{N}, 0 < \delta < \frac{1}{54}$. Assume we are given $n \in \mathbb{N}$, an edge-weighted graph $G$ on at most $n$ vertices with a weight function $w \colon E(G) \to \mathbb{N}$, a vertex subset $A$ and a collection of non-empty vertex subsets $V_0, V_1, \dots, V_\ell$ satisfying the same conditions as in \Cref{l:star} with the following differences:
\begin{itemize}
	\item we don't require $G[V_i - V_0]$ to be connected and $V_i - V_0$ to be adjacent to whole $V_i \cap V_0$;
	\item instead of $|V_0 \cap V_i| \leq 4$, we require $|V_0 \cap V_i| \leq k$.
\end{itemize}
Then, we can compute the eccentricity of every vertex of $G$ in time $\Oh \left( n^{1 + \frac{150 + 54 \delta}{151}} \log^{k + 5g} n \right)$.
\end{lemma}

\begin{proof}
We will reduce our input to one which will satisfy the conditions of \Cref{l:star}. We start by addressing the adhesions $V_0 \cap V_i$ containing too many vertices.

Let $G_0$ denote the graph $G[V_0]$ with cliques placed at $V_0 \cap V_i$ for every $i \in [\ell]$.
For every $i \in [\ell]$ we repeat the following procedure: while $|V_0 \cap V_i| > 4$,
remove arbitrary $5$ vertices from $V_0 \cap V_i$. Since $|V_0 \cap V_i| \leq k$ for each $i\in [\ell]$,
this procedure can be implemented in total time $\Oh(n)$. As a result, at the end we have $|V_0 \cap V_i| \leq 4$ for all $i \in [\ell]$. Let $M$ be the set of all the removed vertices. By our assumptions, $G_0$ has Euler genus bounded by $g$, hence it cannot contain $g + 1$ pairwise disjoint copies of $K_5$
(as the Euler genus of a graph is the sum of the Euler genera of its 2-connected components~\cite{StahlB77} and $K_5$ is not planar). Each removed quintiple of vertices induces a $K_5$ in $G_0$, hence we have $|M| \leq 5g$. We set $A' = A \cup M$ and may thus assume that $V_i$ is disjoint from $A'$ for all $0 \leq i \leq \ell$.

Now, fix $i \in [\ell]$. Let $C^i_1, \dots, C^i_{r_i}$ denote the connected components of $V_i - V_0$ in $G - A'$. We define $W^i_j := N_{G - A'}[C^i_j]$ for every $j \in [r_i]$. Clearly, all $W^i_j$ induce a connected subgraph of $G$ and satisfy $N_{G - A'}(W^i_j - V_0) = W^i_j \cap V_0$. We put $V'_0 := V_0$ and enumerate
$$
\{V'_1, V'_2, \dots V'_{\ell'}\} := \{ W^i_j \colon i \in [\ell], j \in [r_i] \}.
$$
It is easy to verify that the sets $A'$ and $V'_0, V'_1, \dots, V'_{\ell'}$ satisfy the conditions of \Cref{l:star}. We apply said lemma to calculate the eccentricity of every vertex of $G$ in the desired time.
\end{proof}



The next statement is a reformulation of \Cref{thm:main-decomp}.

\begin{theorem}
Fix constants $k, g \in \mathbb{N}$. Assume we are given a graph $G$ on $n$ vertices together with its tree decomposition $(T, \beta)$ and a set of private apices $A_t \subseteq \beta(t)$ for each node $t\in V(T)$ such that the following conditions hold:
\begin{itemize}[nosep]
 \item For every node $t \in V(T)$, we have $|A_t| \leq k$.
 \item For every edge $st \in E(T)$,  we have $|\beta(v) \cap \beta(u)|\leq k$.
 \item For every node $t \in V(T)$, graph obtained by taking $G[\beta(t)] - A_t$ and turning  $(\beta(t) \cap \beta(s))\setminus A_t$ into a clique for every edge $st \in E(T)$ has Euler genus bounded by $g$.
\end{itemize}
Then, we can compute the eccentricity of every vertex of $G$ in time $\Oh \left( n^{1 + \frac{355}{356}} \log^{k + 5g} n \right)$.
\end{theorem}

\begin{proof}
We may assume that $|V(T)|\leq n$, for every tree decomposition with no two bags comparable by inclusion has this property; and adjacent comparable bags can be merged by contracting the edge between them.

For a node $t\in V(T)$, by the {\em{weight}} of $t$ we mean the size of the corresponding bag, that is, $|\beta(t)|$. For any subset of nodes $S \subseteq V(T)$, we define $\beta(S) \coloneqq \bigcup_{t \in S} \beta(t)$ By the {\em{weight}} of $S$, we mean the total weight of the elements of $S$, that is, $\sum_{t\in S} |\beta(t)|$. 

\begin{claim}\label{cl:weight-T}
The weight of $V(T)$ is of $\Oh(n)$.
\end{claim}

\begin{proof}
The sets $\beta'(t) := \beta(t) - \bigcup_{s \in N_T(t)} \beta(s)$ are pairwise disjoint. We have
$$
\sum_{t \in V(T)} |\beta(t)| =
\sum_{t \in V(T)} |\beta'(t)| + 2 \cdot \sum_{st \in E(T)} |\beta(s) \cap \beta(t)| \leq
|V(T)| + 2k|E(T)| = \Oh(n).
$$
\end{proof}

Since every bag induces a graph of bounded Euler genus, the number of edges contained in a bag is linear in its size. In particular, this implies that the total number of edges of $G$ is also bounded by $\Oh(n)$.

We set $$\delta \coloneqq \frac{1}{356}\qquad\textrm{and}\qquad \Delta \coloneqq \frac{355}{356}.$$ Root the tree $T$ in an arbitrarily chosen node; this naturally imposes an ancestor-descendant relation in $T$ (for convenience, every node is considered its own ancestor and descendant).

We start by partitioning $T$ into connected subtrees using the following procedure.
We proceed bottom-up over $T$, processing nodes in any order so that a node is processed after all its strict descendants have been processed. Along the way, we mark some nodes and split the edges of $T$ into heavy and light. Let $t \in V(T)$ be the currently processed non-root node of $T$ and let $e \in E(T)$ be the edge connecting $t$ with its parent. If the total weight of all the unmarked nodes that are descendants of $t$ is at least $n^\delta$ (recall that this includes $t$ itself as well), then we declare $e$ heavy and mark all the descendants of $t$ that were unmarked so far. Otherwise, the edge $e$ is declared light and the procedure proceeds to further nodes of $T$.

Observe that
removing all heavy edges splits $T$ into connected subtrees, say $T'_1, \cdots T'_m$. All of the subtrees, except for possibly the subtree containing the root node, are of weight at least $n^\delta$. In particular, the number of subtrees $m$, and therefore the number of heavy edges, is  bounded by $\Oh(n^{1 - \delta})$. Moreover, in every subtree $T'_i$, removing the node closest to the root splits $T'_i$ into smaller components, each of weight less than $n^\delta$.

Fix a heavy edge $e$ and let $T^e_1$ and $T^e_2$ be the two subtrees into which $T$ splits after removing~$e$. Let $X^e_i = \beta(T^e_i)$ for $i \in \{1, 2\}$. Put $A_e = X^e_1 \setminus X^e_2$, $C_e = X^e_2 \setminus X^e_1$, and $B_e = X^e_1 \cap X^e_2$. By the properties of tree decompositions, such choice of $A_e, B_e, C_e$ satisfies the conditions of \Cref{l:single_adhesion}, hence in time $\Oh(n \log^{k - 1} n)$ we can compute $\max_{v \in X^e_2} \dist_G(u,v)$ for every $u \in X^e_1$, and $\max_{u \in X^e_1} \dist_G(u,v)$ for every $v \in X^e_2$. Computing this for every heavy edge $e$ takes total time $\Oh(n^{2 - \delta} \log^{k - 1} n)$.

Fix any subtree $T'=T'_j$. Let $e_1 = t^{e_1}_1t^{e_1}_2, e_2 = t^{e_2}_1 t^{e_2}_2, \dots, e_\ell = t^{e_\ell}_1 t^{e_\ell}_2$ denote the heavy edges incident to $T'$, where $t^{e_i}_1 \in V(T')$ and $V(T') \subseteq V(T_1^{e_i})$ for every $i \in [\ell]$.
For a vertex $v \in \beta(T')$, let
$$d_0(v) = \max_{u \in \beta(T')} \dist_G(v, u)\qquad\textrm{and}\qquad d_i(v) = \max_{u \in X_2^{e_i}}\dist_G(v,u),\quad\textrm{for } i \in [\ell].$$ We have $\ecc(v) = \max \{ d_i(v)\colon i\in \{0,1,\ldots,\ell\}\}$.The values of $d_i(v)$ are already calculated for all $i\in [\ell]$, hence it remains to compute $d_0(v)$.

For every $i \in [\ell]$ and every pair of vertices $u, v \in \beta(t^{e_i}_1) \cap \beta(t^{e_i}_2)$ we find a shortest path between $u$ and $v$ with all internal vertices inside $X^{e_i}_2$ (or determine that it doesn't exist). For a fixed $u, v$ this can be done in time $\Oh(n)$. Since in total we perform this step at most $2k^2$ times per heavy edge, it takes $\Oh(n^{2 - \delta})$ time in total. Let $P_{i, u, v}$ denote such path, assuming it exists.

Let $G'$ denote the graph obtained from $G[\beta(T')]$ by taking every $i, u, v$ for which $P_{i, u, v}$ exists and adding an edge between $u$ and $v$ of weight equal to the total weight of $P_{i, u, v}$.
The weight of every edge inserted in $\beta(t^{e_i}_1) \cap \beta(t^{e_i}_2)$ is bounded by $|X^{e_i}_2|+1$. The total weight of all edges inserted is therefore at most
$$
\sum_{i \in [\ell]} |\beta(t^{e_i}_1) \cap \beta(t^{e_i}_2)|^2 \cdot (|X^{e_i}_2|+1) \leq
k^2 \sum_{i \in [\ell]} (|X^{e_i}_2|+1) = \Oh(n),
$$
where the last equality follows from the fact that all the trees $T^{e_i}_2$ are pairwise disjoint.
By \Cref{l:inserting_paths}, we have $\dist_{G'}(u, v) = \dist_G(u, v)$ for each $u, v \in \beta(T')$. Hence, computing $d_0(v)$ for every $v \in \beta(T')$ is equivalent to computing the eccentricity of every vertex in $G'$.

If the size of $\beta(T')$ is smaller than $n^\Delta$, we compute the eccentricities naively in time $\Oh(|\beta(T')|^2)$, 
noting that $G'$ has $\Oh(|\beta(T')|)$ edges (thanks to Claim~\ref{cl:weight-T} and bounded genus assumption 
of the last bullet of the theorem statement). Otherwise, we argue that we can use the algorithm in \Cref{l:star} as follows.

Let $t$ be the node of $T'$ closest to the root. Let $s_1, \dots, s_p$ be the children of $t$ in $T$ and let $T''_i$ denote the connected component of $T' - \{t\}$ containing $s_i$. Set $V_0 = \beta(t)$ and $V_i = \beta(T''_i)$ for $i \in [p]$.

It is now easy to verify that $G'$ and sets $A, \{V_i\colon 0\leq i\leq p\}$ selected this way satisfy the assumptions of \Cref{l:star2}. This allows us to use it to compute the eccentricities in $G'$ in time
$$
\Oh \left( n^{1 + \frac{150 + 54\delta}{151}} \log^{k + 5g} n \right) =
\Oh \left( n^{1 + \frac{354}{356}} \log^{k + 5g} n \right).
$$
As we argued, from these eccentricities, we may easily compute all the eccentricities in $G$.

Now, let us analyse the total running time of the whole algorithm. We invoke \Cref{l:star} $\Oh(n^{1 - \Delta})$ times, since we apply it only to subtrees $T'_i$ of size at least $n^\Delta$. The total running time of those applications is hence
$$
\Oh \left( n^{2 + \frac{354}{356} - \Delta} \log^{k + 5g} n \right) =
\Oh \left( n^{1 + \frac{355}{356}} \log^{k + 5g} n \right).
$$
We compute the eccentricities naively for subtrees smaller than $n^\Delta$, hence the total running time of this computation is
$$
\sum_{i \in [m] \colon |\beta(T'_i)| \leq n^\Delta} |\beta(T'_i)|^2 \leq
n^\Delta \cdot \sum_{i \in m} |\beta(T'_i)| = \Oh(n^{1 + \Delta})=\Oh\left(n^{1+\frac{355}{356}}\right).
$$
The rest of computation can be done in $\Oh(n^{2 - \delta} \log^k n)$. Therefore, the whole algorithm runs in time $\Oh \left( n^{1 + \frac{355}{356}} \log^{k + 5g} n \right)$.
\end{proof}

% \usepackage[most]{tcolorbox}
\usepackage{listings}
\usepackage{float}

\lstset{basicstyle=\ttfamily, columns=flexible, breaklines=true, mathescape=true}


\tcbset{
  aibox/.style={
    width=474.18663pt,
    top=10pt,
    colback=white,
    colframe=black,
    colbacktitle=black,
    enhanced,
    center,
    attach boxed title to top left={yshift=-0.1in,xshift=0.15in},
    boxed title style={boxrule=0pt,colframe=white,},
  }
}
\newtcolorbox{AIbox}[2][]{aibox,title=#2,#1}

% \definecolor{codegreen}{rgb}{0,0.6,0}
% \definecolor{codegray}{rgb}{0.5,0.5,0.5}

% \definecolor{backcolour}{RGB}{245,248,250}
% \definecolor{emph}{RGB}{166,88,53}
% \definecolor{nightblue}{RGB}{9,49,105}
% \definecolor{keywords}{RGB}{207,33,46}
% \definecolor{lightpurple}{RGB}{130,81,223}

% \lstdefinestyle{mystyle}{
%     backgroundcolor=\color{backcolour},   
%     commentstyle=\color{codegreen},
%     keywordstyle=\color{keywords},
%     stringstyle=\color{nightblue},
%     basicstyle=\fontsize{7}{8}\ttfamily,
%     breakatwhitespace=true,         
%     breaklines=true,                 
%     captionpos=b,                    
%     keepspaces=true,                 
%     numberstyle=\tiny\color{codegray},
%     numbersep=2pt,                  
%     showspaces=false,                
%     showstringspaces=false,
%     showtabs=false,                  
%     tabsize=2,
%     emph={dsp,Example,sample,annotate,knn,crossval,generate,retrieve,retrieve\_ensemble,majority,fused_retrieval,Template, Transformation,rank,branch},
%     emphstyle={\color{lightpurple}},
%     linewidth=0.98\columnwidth,
%     frame=tb,    
%     xrightmargin=0pt,
%     xleftmargin=0.23cm,
%     numbers=left,
%     aboveskip=0.4cm,
%     belowskip=0.4cm,
% }

% \lstset{style=mystyle}


%---------------------------------------------------
%
% Self-defined macros
%
% Copy right @ 2019
% All right reserved by Elynn Y. Chen.
% This file defines short-hands for math formulas.
% It started with  Prof. Jianqing Fan's macro definition.
% Later it incorporated some of my own macro definition.
%
%
% Other references:
% -- math guide: http://mirrors.rit.edu/CTAN/info/short-math-guide/short-math-guide.pdf
% -- bold math symbol: https://tex.stackexchange.com/questions/595/how-can-i-get-bold-math-symbols
%
%

\renewcommand{\hat}{\widehat}
\renewcommand{\tilde}{\widetilde}
\renewcommand{\bar}{\overline}

%--------------------------------
%
%  Boldfaces in Letters
%
% \newcommand{\bfm}[1]{\ensuremath{\mathbf{#1}}}
\newcommand{\bfm}[1]{\ensuremath{\boldsymbol{#1}}} % bm

\def\bzero{\bfm 0}
\def\bone{\bfm 1}
\def\bbone{\mathbbm{1}} % package bm

\def\ba{\bfm a}   \def\bA{\bfm A}  \def\AA{\mathbb{A}}
\def\bb{\bfm b}   \def\bB{\bfm B}  \def\BB{\mathbb{B}}
\def\bc{\bfm c}   \def\bC{\bfm C}  \def\CC{\mathbb{C}}
\def\bd{\bfm d}   \def\bD{\bfm D}  \def\DD{\mathbb{D}}
\def\be{\bfm e}   \def\bE{\bfm E}  \def\EE{\mathbb{E}}
\def\bff{\bfm f}  \def\bF{\bfm F}  \def\FF{\mathbb{F}}
\def\bg{\bfm g}   \def\bG{\bfm G}  \def\GG{\mathbb{G}}
\def\bh{\bfm h}   \def\bH{\bfm H}  \def\HH{\mathbb{H}}
\def\bi{\bfm i}   \def\bI{\bfm I}  \def\II{\mathbb{I}}
\def\bj{\bfm j}   \def\bJ{\bfm J}  \def\JJ{\mathbb{J}}
\def\bk{\bfm k}   \def\bK{\bfm K}  \def\KK{\mathbb{K}}
\def\bl{\bfm l}   \def\bL{\bfm L}  \def\LL{\mathbb{L}}
\def\bm{\bfm m}   \def\bM{\bfm M}  \def\MM{\mathbb{M}}
\def\bn{\bfm n}   \def\bN{\bfm N}  \def\NN{\mathbb{N}}
\def\bo{\bfm o}   \def\bO{\bfm O}  \def\OO{\mathbb{O}}
\def\bp{\bfm p}   \def\bP{\bfm P}  \def\PP{\mathbb{P}}
\def\bq{\bfm q}   \def\bQ{\bfm Q}  \def\QQ{\mathbb{Q}}
\def\br{\bfm r}   \def\bR{\bfm R}  \def\RR{\mathbb{R}}
\def\bs{\bfm s}   \def\bS{\bfm S}  \def\SS{\mathbb{S}}
\def\bt{\bfm t}   \def\bT{\bfm T}  \def\TT{\mathbb{T}}
\def\bu{\bfm u}   \def\bU{\bfm U}  \def\UU{\mathbb{U}}
\def\bv{\bfm v}   \def\bV{\bfm V}  \def\VV{\mathbb{V}}
\def\bw{\bfm w}   \def\bW{\bfm W}  \def\WW{\mathbb{W}}
\def\bx{\bfm x}   \def\bX{\bfm X}  \def\XX{\mathbb{X}}
\def\by{\bfm y}   \def\bY{\bfm Y}  \def\YY{\mathbb{Y}}
\def\bz{\bfm z}   \def\bZ{\bfm Z}  \def\ZZ{\mathbb{Z}}

\def\calA{{\cal  A}} \def\cA{{\cal  A}}
\def\calB{{\cal  B}} \def\cB{{\cal  B}}
\def\calC{{\cal  C}} \def\cC{{\cal  C}}
\def\calD{{\cal  D}} \def\cD{{\cal  D}}
\def\calE{{\cal  E}} \def\cE{{\cal  E}}
\def\calF{{\cal  F}} \def\cF{{\cal  F}}
\def\calG{{\cal  G}} \def\cG{{\cal  G}}
\def\calH{{\cal  H}} \def\cH{{\cal  H}}
\def\calI{{\cal  I}} \def\cI{{\cal  I}}
\def\calJ{{\cal  J}} \def\cJ{{\cal  J}}
\def\calK{{\cal  K}} \def\cK{{\cal  K}}
\def\calL{{\cal  L}} \def\cL{{\cal  L}}
\def\calM{{\cal  M}} \def\cM{{\cal  M}}
\def\calN{{\cal  N}} \def\cN{{\cal  N}}
\def\calO{{\cal  O}} \def\cO{{\cal  O}}
\def\calP{{\cal  P}} \def\cP{{\cal  P}}
\def\calQ{{\cal  Q}} \def\cQ{{\cal  Q}}
\def\calR{{\cal  R}} \def\cR{{\cal  R}}
\def\calS{{\cal  S}} \def\cS{{\cal  S}}
\def\calT{{\cal  T}} \def\cT{{\cal  T}}
\def\calU{{\cal  U}} \def\cU{{\cal  U}}
\def\calV{{\cal  V}} \def\cV{{\cal  V}}
\def\calW{{\cal  W}} \def\cW{{\cal  W}}
\def\calX{{\cal  X}} \def\cX{{\cal  X}}
\def\calY{{\cal  Y}} \def\cY{{\cal  Y}}
\def\calZ{{\cal  Z}} \def\cZ{{\cal  Z}}

\def\calh{{\cal h}}

%--------------------------------
%
%  Boldfaces in Greek
%
\newcommand{\bfsym}[1]{\ensuremath{\boldsymbol{#1}}}

 \def\balpha{\bfsym \alpha}
 \def\bbeta{\bfsym \beta}
 \def\bgamma{\bfsym \gamma}             \def\bGamma{\bfsym \Gamma}
 \def\bdelta{\bfsym {\delta}}           \def\bDelta {\bfsym {\Delta}}
 \def\bfeta{\bfsym {\eta}}              \def\bfEta {\bfsym {\Eta}}
 \def\bmu{\bfsym {\mu}}                 \def\bMu {\bfsym {\Mu}}
 \def\bnu{\bfsym {\nu}}
 \def\btheta{\bfsym {\theta}}           \def\bTheta {\bfsym {\Theta}}
 \def\beps{\bfsym \varepsilon}          \def\bEps{\bfsym{\mathcal E}}
 \def\bepsilon{\bfsym \varepsilon}
 \def\bsigma{\bfsym \sigma}             \def\bSigma{\bfsym \Sigma}
  \def\bsig{\bfsym \sigma}             \def\bSig{\bfsym \Sigma}
 \def\blambda {\bfsym {\lambda}}        \def\bLambda {\bfsym {\Lambda}}
 \def\bomega {\bfsym {\omega}}          \def\bOmega {\bfsym {\Omega}}
 \def\brho   {\bfsym {\rho}}
 \def\btau{\bfsym {\tau}}
 \def\bxi{\bfsym {\xi}}          \def\bXi{\bfsym {\Xi}}
 \def\bzeta{\bfsym {\zeta}}
 \def\bpi{\bfsym {\pi}}          \def\bPi{\bfsym {\Pi}}
\def\bphi{\bfsym {\phi}}          \def\bPhi{\bfsym {\Phi}}
\def\bpsi{\bfsym {\psi}}          \def\bPsi{\bfsym {\Psi}}
\def\bchi{\bfsym {\chi}}          \def\bChi{\bfsym {\Chi}}
\def\bupsi{\bfsym {\upsilon}}          \def\bUpsi{\bfsym {\Upsilon}}

 %--------------------------------
 %
 %  Hat (boldface) in Greek
 %
  \def\halpha{\hat{\alpha}}              \def\hbalpha{\hat{\bfsym \alpha}}
  \def\hbeta{\hat{\beta}}                \def\hbbeta{\hat{\bfsym \beta}}
  \def\hgamma{\hat{\gamma}}              \def\hgamma{\hat{\bfsym \gamma}}
  \def\hGamma{\hat{ \Gamma}}             \def\hbGamma{\hat{\bfsym \Gamma}}
  \def\hdelta{\hat{\delta}}              \def\hbdelta{\hat{\bfsym {\delta}}}
  \def\hDelta {\hat{\Delta}}             \def\hbDelta{\hat{\bfsym {\Delta}}}
  \def\heta{\hat {\eta}}                 \def\hbfeta {\hat{\bfsym {\eta}}}
  \def\hmu{\hat{\mu}}                    \def\hbmu {\hat{\bfsym {\mu}}}
  \def\hnu{\hat{\nu}}                    \def\hbnu {\hat{\bfsym {\nu}}}
  \def\htheta{\hat {\theta}}             \def\hbtheta {\hat{\bfsym {\theta}}}
  \def\hTheta{\hat {\Theta}}             \def\hbTheta {\hat{\bfsym {\Theta}}}
  \def\hbeps{\hat{\bfsym \varepsilon}}   \def\hbepsilon{\hat{\bfsym \varepsilon}}
  \def\hsigma{\hat{\sigma}}              \def\hbsigma{\hat{\bfsym \sigma}}
  \def\hSigma{\hat{\Sigma}}              \def\hbSigma{\hat{\bfsym \Sigma}}
  \def\hlambda{\hat{\lambda}}            \def\hblambda{\hat{\bfsym \lambda}}
  \def\hLambda{\hat{\Lambda}}            \def\hbLambda{\hat{\bfsym \Lambda}}
  \def\homega {\hat {\omega}}            \def\hbomega {\hat{\bfsym {\omega}}}
  \def\hOmega {\hat {\omega}}            \def\hbOmega {\hat{\bfsym {\Omega}}}
  \def\hrho   {\hat {\rho}}              \def\hbrho {\hat{\bfsym {\rho}}}
  \def\htau   {\hat {\tau}}              \def\hbtau {\hat{\bfsym {\tau}}}
  \def\hxi{\hat{\xi}}                    \def\hbxi{\hat{\bfsym {\xi}}}
  \def\hzeta{\hat{\zeta}}                \def\hbzeta{\hat{\bfsym {\bzeta}}}


 %--------------------------------
 %
 %  Operator
 %
\providecommand{\abs}[1]{\left\lvert#1\right\rvert}
\providecommand{\norm}[1]{\left\lVert#1\right\rVert}

\providecommand{\angles}[1]{\left\langle #1 \right\rangle}
\providecommand{\paran}[1]{\left( #1 \right)}
\providecommand{\brackets}[1]{\left[ #1 \right]}
\providecommand{\braces}[1]{\left\{ #1 \right\}}

\DeclarePairedDelimiter\ceil{\lceil}{\rceil}
\DeclarePairedDelimiter\floor{\lfloor}{\rfloor}

\providecommand{\convprob}{\stackrel{\calP}{\longrightarrow}}
\providecommand{\convdist}{\stackrel{\calD}{\longrightarrow}}
\providecommand{\defeq}{\triangleq}
% \providecommand{\defeq}{\dot{=}}

\usepackage{mathtools}
% K-L divergence
\DeclarePairedDelimiterX{\infdivx}[2]{(}{)}{%
  #1 \; \delimsize\| \; #2%
}
\newcommand{\infdiv}{\calD \infdivx}
\newcommand{\kldiv}{\calD_{KL} \infdivx}

%--------------------------------
%
% Regular font in math equation
%

\DeclareMathOperator{\argmin}{argmin}
\DeclareMathOperator{\corr}{corr}
\DeclareMathOperator{\cov}{cov}
\DeclareMathOperator{\Cov}{Cov}
\DeclareMathOperator{\Dev}{Dev}
\DeclareMathOperator{\diag}{diag}
%\DeclareMathOperator{\E}{E}
\newcommand{\E}[1]{{\mathbb{E}} \left[ #1 \right]}
\DeclareMathOperator{\logit}{logit}
\DeclareMathOperator{\rank}{rank}
\DeclareMathOperator{\RSS}{RSS}
\DeclareMathOperator{\sgn}{sgn}
\DeclareMathOperator{\supp}{supp}
\DeclareMathOperator{\Var}{Var}
\DeclareMathOperator{\var}{var}
\DeclareMathOperator{\SE}{SE}
\DeclareMathOperator{\Prob}{Pr}
%\DeclareMathOperator{\tr}{tr}
\DeclareMathOperator{\Tr}{Tr}
%\DeclareMathOperator{\vect}{vec}
\newcommand{\vect}[1]{{\textsc{vec}} \left( #1 \right)}
\newcommand{\mat}[1]{{\textsc{mat}} \left( #1 \right)}


%--------------------------------
%
%  Thms, Props, Assumps, Remarks, et al
%

%\newtheorem{definition}{Definition}[section]
%\newtheorem{assumption}[definition]{Assumption}
%\newtheorem{lemma}[definition]{Lemma}
%\newtheorem{proposition}[definition]{Proposition}
%\newtheorem{theorem}[definition]{Theorem}
%\newtheorem{corollary}[definition]{Corollary}
%\newtheorem{exercise}[definition]{Exercise}
%\newtheorem{example}[definition]{Example}
%
%\newenvironment{solution}{\paragraph{Solution:}}{\hfill$\square$}
%
%\renewcommand*{\theassumption}{\Alph{assumption}}

% \newtheorem{definition}{Definition}
% \newtheorem{assumption}{Assumption}
% \newtheorem{lemma}{Lemma}
% \newtheorem{proposition}{Proposition}
% \newtheorem{theorem}{Theorem}
% \newtheorem{corollary}{Corollary}

%\def\remark{{\noindent\underbar{\bf Remark}\quad}}
%\newtheorem{remark}{Remark}

%--------------------------------
%
%  Bachmann-Landau or asymptotic notations
%
% determiniastic convergence
\newcommand{\bigO}[1]{ \mathcal{O} \left( #1 \right) }
\newcommand{\bigOmega}[1]{ \Omega \left( #1 \right) }
\newcommand{\bigTheta}[1]{ \Theta \left( #1 \right) }
\newcommand{\smlo}[1]{{\rm o} \left( #1 \right) }
\newcommand{\smlomega}[1]{{\omega} \left( #1 \right) }

% converge in probability
\newcommand{\Op}[1]{{\mathcal{O}_p} \left( #1 \right) }
\newcommand{\op}[1]{{\rm o_p} \left( #1 \right) }

%--------------------------------
%
%  Markers
%
% need \usepackage[table,xcdraw,dvipsnames]{xcolor}   % for \textcolor

% colorful text emphasizer
\providecommand{\textem}[1]{\textit{\textcolor{Cerulean}{#1}}}

% comments
\definecolor{royalpurple}{rgb}{0.47, 0.32, 0.66}
\providecommand{\elynn}[1]{\textcolor{RubineRed}{\bf [Elynn: #1]}}
\providecommand{\attn}[1]{\textcolor{red}{\bf (#1)}}
\providecommand{\xuening}[1]{\textcolor{blue}{\bf [Xuening: #1]}}

 %--------------------------------
%
%  Semi-parametric specials
%

\newcommand{\jbracket}[1]{ J_{[]}\braces{#1} }
\newcommand{\nbracket}[1]{ N_{[]}\braces{#1} }

 %--------------------------------
 %
 %  Shorthands
 %

\def\eps{\varepsilon}
\def\lam {\lambda}
\def\blam {\blambda}
\def\bLam {\bLambda}
\def\bdiag{{\rm bdiag}}

\def\vec{\mbox{vec}}
\def\wh{\widehat}
\def\blue{\color{blue}}
\def\red{\color{red}}

% May add more in future.

\newenvironment{question}{ \color{teal} \textbf{To settle:}  }{\hfill$\clubsuit$}

% more info here
% https://texblog.org/2015/09/30/fancy-boxes-for-theorem-lemma-and-proof-with-mdframed/

\usepackage[framemethod=TikZ]{mdframed} % box environment & flexible box designs

% or
\usepackage{fancybox}

\newcommand{\oC}{\overline{C}}
\DeclareMathOperator*{\argmax}{\arg\!\max}

%---------------------------------------------------
%
% Format control
%
\usepackage[top=1in, bottom=1in, left=1in, right=1in]{geometry}

%\setcounter{tocdepth}{1} % Show sections
%\setcounter{tocdepth}{2} % + subsections
%\setcounter{tocdepth}{3} % + subsubsections
%\setcounter{tocdepth}{4} % + paragraphs
%\setcounter{tocdepth}{5} % + subparagraphs

% spacing
%\setlength{\parindent}{4em}  % paragraph indent
%\setlength{\parskip}{1em}  % between paragraphs
%\renewcommand{\baselinestretch}{1.5}
\usepackage{setspace}
%\setstretch{1.9}
%\onehalfspacing

%---------------------------------------------------
%
% Bib file, image folder
%
\usepackage[authoryear]{natbib}  %[numbers]
\newcommand{\mybibsty}{chicago}%{hsiamplain}%{agsm}
\newcommand{\mybib}{main}

\newcommand{\cyan}{\color{cyan}}

\definecolor{DSgray}{cmyk}{0,1,0,0}
\newcommand{\Authornote}[2]{{\small\textcolor{DSgray}{\sf$<${  #1: #2
			}$>$}}}
\newcommand{\Authormarginnote}[2]{\marginpar{\parbox{2cm}{\raggedright\tiny
			\textcolor{DSgray}{#1: #2}}}}
\newcommand{\wenbo}[1]{{\Authornote{Wenbo}{#1}}}

\usepackage{graphicx}
\graphicspath{{figs/}}

%---------------------------------------------------
%
% More self-defined macros
% add at the end of "macros/self-defined.tex"
%
% Specific for this paper
%

%\pdfminorversion=4
% NOTE: To produce blinded version, replace "0" with "1" below.
\newcommand{\blind}{1}
%---------------------------------------------------
%
%  document body
%
\begin{document}
\pagenumbering{arabic}

\def\spacingset#1{\renewcommand{\baselinestretch}%
{#1}\small\normalsize} \spacingset{1}

%---------------------------------------------------
%
% Title page
%
% https://www.math.uh.edu/~torok/math_6298/latex/MANUALS/NASA_Hypertext-Help-with-LaTeX/latex/ltx-407.html
%

\def\TITLE{Stochastic Linear Bandits with \\ Latent Heterogeneity}


\if1\blind
{
\title{\bf \TITLE}
\author{
Elynn Chen$^\diamondsuit$ \hspace{2ex}
Xi Chen$^\natural$%\thanks{Corresponding author. The authors gratefully acknowledge \textit{please remember to list all relevant funding sources in the unblinded version}} 
\hspace{2ex}
Wenbo Jing$^\sharp$ \hspace{2ex}
Xiao Liu$^\flat$ \hspace{2ex} \\ \normalsize
\medskip
$^{\diamondsuit,\natural,\sharp,\flat}$ New York University 
}
\maketitle
} \fi

\if0\blind
{
  \bigskip
  \bigskip
  \bigskip
  \begin{center}
    {\LARGE\bf Title}
\end{center}
  \medskip
} \fi

\bigskip
\begin{abstract}
\spacingset{1}
This paper addresses the critical challenge of latent heterogeneity in online decision-making, where individual responses to business actions vary due to unobserved characteristics. While existing approaches in data-driven decision-making have focused on observable heterogeneity through contextual features, they fall short when heterogeneity stems from unobservable factors such as lifestyle preferences and personal experiences. We propose a novel latent heterogeneous bandit framework that explicitly models this unobserved heterogeneity in customer responses, with promotion targeting as our primary example. Our methodology introduces an innovative algorithm that simultaneously learns latent group memberships and group-specific reward functions. Through theoretical analysis and empirical validation using data from a mobile commerce platform, we establish high-probability bounds for parameter estimation, convergence rates for group classification, and comprehensive regret bounds. Notably, our theoretical analysis reveals two distinct types of regret measures: a ``strong regret'' against an oracle with perfect knowledge of customer memberships, which remains non-sub-linear due to inherent classification uncertainty, and a ``regular regret'' against an oracle aware only of deterministic components, for which our algorithm achieves a sub-linear rate that is minimax optimal in horizon length and dimension. We further demonstrate that existing bandit algorithms ignoring latent heterogeneity incur constant average regret that accumulates linearly over time. Our framework provides practitioners with new tools for decision-making under latent heterogeneity and extends to various business applications, including personalized pricing, resource allocation, and inventory management.
\end{abstract}

\noindent%
{\it Keywords: Stochastic bandits; Latent heterogeneity; High-dimensional Statistics; Mixed linear regression; Logistic models; }  
\vfill

%-------
%
%  main text
%
%\begin{singlespace}
%\tableofcontents
%\end{singlespace}

\spacingset{1.8}


%!TEX root = 0-main.tex

\section{Introduction}  \label{sec:intro}

Recent studies have highlighted the critical importance of latent heterogeneity across various domains, including economics \citep{blundell2007labor,bonhomme2022discretizing}, business \citep{cherry1999unobserved,lewis2024latent}, and healthcare \citep{zhou2018using,chen2024reinforcement}. The prevalence of latent heterogeneity stems from the inherent complexity of human behavior in societal applications, where individual responses to the same policy or intervention can vary significantly based on unobserved factors. For instance, consumer preferences and purchasing decisions often extend beyond observable demographic characteristics, influenced by psychological factors, personal experiences, and social interactions that are difficult to quantify. While existing approaches in data-driven decision-making have focused on leveraging observable contextual features \citep{keskin2017chasing,keskin2021dynamic,chen2019welfare,chen2021nonparametric,simchi2024simple,chen2022transfer,chen2024reinforcement,chai2025deep}, these methods fall short in addressing latent heterogeneity where the sources of variation are unobserved or unknown a priori. This limitation poses significant challenges for businesses, potentially leading to suboptimal resource allocation, reduced marketing effectiveness, and missed opportunities for personalization in customer engagement.

To address this critical challenge in business decision-making, we propose a novel decision framework that explicitly accounts for \emph{latent heterogeneity} by building upon the classical stochastic bandits model --- a prototype for modeling single-stage decision-making with learning in numerous business applications \citep{keskin2022data,ji2022risk,aramayo2023multiarmed,dong2024value,li2024dynamic,simchi2024simple,chen2025express}. We use promotion targeting as our running example throughout the paper, though the proposed model with latent heterogeneity extends to other settings of sequential personalized targeting, advertising, and recommendations.

Consider an e-commerce platform that observes one customer and needs to decide on a coupon to issue to that customer at a sequence of time points $[T]:= \{ 1, \dots, T \}$. The platform observes contextual information stored in a vector $\bz_{i}$, which includes consumer characteristics (such as demographics and historical purchase amount), product characteristics (such as product category and brand tier) from the live stream video watched by the consumer, and environmental characteristics (such as video characteristics of each livestream channel). At each decision point $i \in [T]$, there are $K$ available coupons to be offered to consumers, with each coupon specified by a discount amount and a minimum purchase requirement (e.g., ``\$15 off \$99''). We use $\bx_{i,k}\in\RR^d$, $i\in[T]$, $k\in[K]$ to represent the contextual vector concatenating the customer and coupon features for the $i$-th customer. The stochastic reward $y_{i,k}$ associated with customer $i$ receiving coupon $k$ typically denotes the customer's payment to the platform after receiving the coupon.

Extensive research has documented substantial heterogeneity in customer preferences and purchasing behaviors \citep{cherry1999unobserved,bonhomme2015grouped,lewis2024latent,hess2024latent}. In particular, customer responses to recommendations or advertisements often depend on latent factors such as lifestyle preferences or brand perceptions that are not directly measurable. To model this \emph{latent heterogeneity}, we consider two possible latent statuses for customer $i$, denoted by $g_i = 1$ or $2$, without loss of generality. The stochastic reward $y_{i,k}$ exhibits {\em different associations} for customers with different latent status $g_i$:
\begin{align} \label{eqn:hetero-f}
(y_{i,k} \cond g_i = 1) = f_1(\bx_{i,k}) + \epsilon_i , \quad\text{and}\quad
(y_{i,k} \cond g_i = 2) = f_2(\bx_{i,k}) + \epsilon_i, 
\end{align}
where $f_1(\cdot)\ne f_2(\cdot)$ and $\epsilon_i$ is a mean-zero random noise. The \emph{latent} status of customer $i$ is modeled as a random draw from a binomial distribution:
\begin{align} \label{eqn:hetero-status}
\Pr(g_i = 1 \cond \bz_i) = p(\bz_i^\top \btheta^*), \quad\text{and}\quad
\Pr(g_i = 2 \cond \bz_i) = 1-p(\bz_i^\top \btheta^*),
\end{align}
where $p(x) = 1/\paren{1 + \exp(-x)}$. 
%In our theoretical analysis, we focus on the linear reward case where the coefficients $\bbeta_{g_i}^*$ of the linear functions $f_{g_i}$ differ, specifically, $f_1(\bx_{i,k}) = \angles{\bx_{i,k}, \bbeta_1^*}$, $f_2(\bx_{i,k}) = \angles{\bx_{i,k}, \bbeta_2^*}$, and $\bbeta_1^* \ne \bbeta_2^*$.

Our proposed latent heterogeneous bandit model, characterized by \eqref{eqn:hetero-f} and \eqref{eqn:hetero-status}, differs fundamentally from the classic stochastic bandit model in two interrelated aspects: the reward functions $f_1$ and $f_2$ vary across latent groups, and these group memberships $g_i$ are unobservable. This joint presence of unobservable group membership and group-specific reward functions poses unique challenges that existing methods cannot readily address. Naive approaches such as classification followed by group-specific bandit algorithms are infeasible since the latent groups are not directly observable. Moreover, standard clustering techniques combined with group-specific bandit algorithms are inadequate because cluster membership may not align with the underlying reward associations --- the key source of heterogeneity in decision outcomes.

To address these challenges, we make three main contributions. First, we propose a novel modeling framework that explicitly captures latent heterogeneity in sequential decision-making scenarios. This framework provides a rigorous foundation for analyzing the complexities inherent in business decisions where customer segments respond differently to the same actions. Second, focusing on the linear reward setting, we develop an innovative algorithm that simultaneously learns latent group memberships and group-specific reward functions. Third, we establish comprehensive theoretical guarantees for our approach, including high-probability bounds for parameter estimation, convergence rates for group classification, regret upper bound for the optimal policy, and the minimax lower bound.

Our theoretical analysis reveals several important insights about decision-making under latent heterogeneity. We show that existing algorithms, which ignore latent heterogeneity, inevitably incur constant average regret due to their inability to account for group-specific associations. We further distinguish between two types of regret based on different oracle comparisons. The ``strong regret,'' measured against an oracle with perfect knowledge of customer memberships under the binomial distribution, remains non-sub-linear due to inherent uncertainty in group classification. This is unavoidable due to the minimax lower bound on misclassification rates, which stems from the inherent randomness in group classification. In contrast, the ``regular regret,'' compared against an oracle aware only of deterministic components, achieves a sub-linear rate that is minimax optimal in terms of both time horizon $T$ and feature dimension $d$.

\subsection{Related Works}\label{sec:related-works}

Our work intersects online stochastic contextual bandits, statistical learning, and decision-making under latent heterogeneity. We review the most relevant literature and distinguish our contributions.

\smallskip
\noindent
\textbf{Online Stochastic Contextual Bandits.}
The bandit problem has been extensively studied across computer science, operations, and statistics \citep{li2019dimension, russo2020simple,chen2021multi, russo2022satisficing, si2023distributionally,simchi2023regret,ren2024dynamic,tang2024stochastic,chen2025express}; see \cite{lattimore2020bandit} for a comprehensive review. Contextual bandits incorporate additional information to predict action quality \citep{auer2002using,dani2008stochastic,li2010contextual,chu2011contextual,ji2022risk}. While adversarial settings achieve $\bigO{d\sqrt{T}}$ regret \citep{abbasi2011improved}, stochastic settings --- suitable for applications like news recommendation and clinical trials --- can improve bounds dependent on $T$ from $\sqrt{T}$ to $\log(T)$ for homogeneous bandits.

In low dimensions, \cite{goldenshluger2013linear} achieved $\bigO{d^3\log(T)}$ regret, though this becomes unfavorable as the dimension increases. For high-dimensional settings with sparsity $s$ ($s \ll d$), several approaches have emerged: \cite{bastani2020online}'s LASSO bandit achieved $\bigO{K s^2 \log^2(dT)}$; %with a $\bigO{s^2\log(d)\log(T)}$ exploration stage; 
\cite{wang2024online}'s MCP-based method improved this to $\bigO{Ks^2(s+\log d)\log T}$; and \cite{kim2019doubly}'s doubly-robust approach reached $\bigO{s\log(dT)\sqrt{T}}$; %with $\bigO{\sqrt{T\log(dT)\log(T)}}$ exploration. 
%These methods require known sparsity $s$. 
In contrast, \cite{oh2020sparsity}'s sparse-agnostic approach achieves %$\bigO{s^2\log(d) + s\sqrt{T\log(dT)}}$, improvable to 
$\bigO{\sqrt{sT\log(dT)}}$ under restricted eigenvalue conditions. 
%\cite{ariu2020thresholded}'s Threshold LASSO achieves $\bigO{\log(d) + \sqrt{T\log(T)}}$, reducible to $\bigO{\log(d) + \log(T)}$ under margin conditions.

All of the aforementioned papers work within the classic model of stochastic bandits, which does not address the widely encountered setting of latent heterogeneity in business and economics. Under latent heterogeneity, these algorithms will obtain biased estimation due to the misspecified single linear expectation, incurring non-negligible regret at each decision point and thus linear expected regret over time.
In contrast, our proposed latent heterogeneous bandit model explicitly incorporates an unobservable subgroup structure. Our specially designed algorithm learns both the latent groups and the true parameters for each group. 
%During the decision-making stage, group membership is predicted and the optimal policy for the respective group is applied. Our analysis differentiates between two types of regrets based on different oracle comparisons. The "strong regret," measured against an oracle with perfect knowledge of customer memberships under the binomial distribution, remains non-sub-linear due to inherent uncertainty in group classification. In contrast, the "regular regret," compared against an oracle aware only of deterministic components, achieves $\bigO{\sqrt{Ts\log(d)}}$. We establish the minimax lower bounds for both regret types, demonstrating the optimality of our proposed heterogeneous algorithm.

Recent work has applied the bandits to various management problems, including newsvendor problems \citep{keskin2023nonstationary}, dynamic pricing \citep{keskin2017chasing,chen2019dynamic,keskin2021dynamic,chen2019welfare,chen2021nonparametric,den2022dynamic,bastani2022meta,keskin2022data,chen2024dynamic,li2024dynamic}, joint advertising and assortment \citep{gao2021assortment,gurkan2023dynamic}, resource allocation \citep{dong2024value}, inventory control \citep{che2024stochastic,qin2023sailing}, sub-exponential rewards \citep{yuan2021marrying}, sourcing \citep{tang2024online}, network revenue management \citep{chen2023network}, and smart order routing \citep{ji2022risk}. These works operate within the classic bandit framework and do not consider latent heterogeneity. Our proposed latent heterogeneous bandits framework can be incorporated into these decision scenarios. It is of great interest to explore the effect of latent heterogeneity of the customer population on various decision scenarios and to develop efficient methods or strategies to overcome the challenge of latent heterogeneity. We leave those for future work.

\smallskip
\noindent
\textbf{Statistical learning with Latent heterogeneity.}
In linear regression settings, researchers have explored two major approaches to latent heterogeneity. The non-parametric approach employs grouping penalization on pairwise differences \citep{shen2010grouping,ke2013homogeneity,ma2017concave}, avoiding distributional assumptions but limiting predictive capabilities for new samples. For prediction-oriented tasks, mixture-model-based approaches have shown greater promise. Key developments include: rigorous EM algorithm guarantees for symmetric mixed linear regressions (MLR) where the mixing proportion is known to be $1/2$ \citep{balakrishnan2017statistical}, efficient fixed-parameter algorithms \citep{li2018learning}, computational-statistical tradeoffs \citep{fan2018curse}, improved EM convergence analysis \citep{mclachlan2019finite,klusowski2019estimating}, robust estimation under corruptions \citep{shen2019iterative}, high-dimensional MLR with unknown but constant mixing proportions \citep{zhang2020estimation}, and convergence analysis for federated learning \citep{su2024global, niu2024collaborative}. Our model \eqref{eqn:hetero-f} and \eqref{eqn:hetero-status} extends this literature by addressing variable mixing proportions in high-dimensional MLR, a previously unexplored challenge.

\smallskip
\noindent
\textbf{Decision with Heterogeneity.}
Research addressing learning and decision-making with heterogeneity remains relatively sparse. Notable work spans several domains: personalized dynamic pricing with high-dimensional features and heterogeneous elasticity \citep{ban2021personalized}; regime-switching bandits with temporal heterogeneity \citep{cao2019nearly,zhou2021regime}; and convergence analysis of Langevin diffusion under mixture distributions \citep{dong2022spectral}, where multiple density components can significantly impact sampling efficiency. In sequential decision settings, recent work has explored policy evaluation and optimization with latent heterogeneity in reinforcement learning \citep{chen2024reinforcement,bian2024off}. However, these approaches focus primarily on multi-stage aspects of reinforcement learning, leaving unexplored the fundamental challenge that latent heterogeneity poses to the exploration-exploitation trade-off. Our work addresses this gap through the lens of the bandit problem—essentially a one-step reinforcement learning.


\subsection{Notations and Organization}\label{sec:notations}

For a positive integer $n$, let $[n] := {1,\dots,n}$. For any vector $\bv$, $\norm{\bv}_0$, $\norm{\bv}_1$, and $\norm{\bv}_2$ denote the $\ell_0$ (number of non-zero elements), $\ell_1$, and $\ell_2$ norms respectively. For a matrix $\bA$, $\lambda_{\min}(\bA)$ and $\lambda_{\max}(A)$ denote its minimum and maximum eigenvalues. For positive sequences ${a_n}$ and ${b_n}$, we write $a_n \lesssim b_n$, $a_n=\bigO{b_n}$, or $b_n=\Omega(a_n)$ if there exists $C > 0$ such that $a_n \leq Cb_n$ for all $n$. We write $a_n \asymp b_n$ if $a_n \lesssim b_n$ and $b_n \lesssim a_n$.

The remainder of this paper is organized as follows. Section \ref{sec:model} formulates our latent heterogeneous bandit model and defines two types of regret measures. Section \ref{sec:method} presents our proposed methodology. Section \ref{sec:theory} establishes theoretical guarantees, including estimation error bounds, misclassification rates, and minimax optimal regret bounds. Sections \ref{sec:numerical} validate our approach through simulation studies and an empirical application using cash bonus data from a mobile commerce platform. Section \ref{sec:conclusion} concludes with discussions.

\section{Problem Formulation} \label{sec:model}

In this section, we formulate the linear bandits problem under latent heterogeneity. Section \ref{sec:problem} introduces the latent heterogeneous linear bandit model, which extends the classical stochastic linear bandit setting by incorporating unobserved group structures. Section \ref{sec:two-type-regret} defines two types of regret—strong regret and regular regret—that evaluate the performance of a policy in the presence of latent heterogeneity.

\subsection{Latent Heterogeneous Linear Bandits}\label{sec:problem}

The latent heterogeneous bandit model \eqref{eqn:hetero-f} and \eqref{eqn:hetero-status} introduced at the beginning of this paper is a general framework that allows for arbitrary functional forms of the mean rewards $f_1(\bx)$ and $f_2(\bx)$. We focus on linear functional form in this work and leave non-linear and non-parametric function approximation for future research.
Without loss of generality, we consider the setting for two latent subgroups, as the extension to any known finite number of latent subgroups follows naturally. 
Each customer $i\in[T]$ is characterized by a customer feature $\bz_i\in\RR^{d'}$. 
For any customer $i$, there are $K$ possible arms (coupons) to offer. 
The combined features of customer $i$ and an arm $k$ are denoted as $\bx_{i,k}\in\RR^d$. 
The latent heterogeneous linear bandits are characterized by
\begin{equation} \label{eqn:lhlb}
\begin{aligned}
&\text{(Subgroup model):} & \Pr(g_i = 1 \cond \bz_i) = p(\bz_i^\top \btheta^*), \quad
\Pr(g_i = 2 \cond \bz_i) = 1-p(\bz_i^\top \btheta^*), \\
&\text{(Reward model):} & (y_{i,k} \cond g_i = 1) = \angles{\bx_{i,k}, \bbeta_1^*} + \epsilon_i , \quad
(y_{i,k} \cond g_i = 2) = \angles{\bx_{i,k}, \bbeta_2^*} + \epsilon_i, 
\end{aligned}
\end{equation}
where $\bbeta_1^*\ne\bbeta_2^*$, $p(x) = 1/\paren{1 + \exp(-x)}$, and $\epsilon_i \sim \calN(0, \sigma^2)$. We refer to the two equations as the ``subgroup model'' and the ``reward model.''

For each customer $i$, while the contextual features $\big\{ \{\bx_{i,k}\}_{k\in[K]}, \; \bz_i \big\}$ are observable, the true group membership $g_i$ remains unknown. 
The objective of the platform is to select one coupon $k\in[K]$ for each customer $i$ to maximize the aggregated rewards across all $T$ customers. 
Our goal is to design a sequential decision rule (policy) $\pi$ that maximizes the expected cumulative reward over the time horizon while simultaneously estimating the model parameters and predicting the latent group $g_i$. 
%Specifically, given $\big\{ \{\bx_{i,k}\}_{k\in[K]}, \; \bz_i \big\}$, the platform uses {\red algorithm} to obtain estimators ($\hat{\btheta}, \hat{\bbeta}_1, \hat{\bbeta}_2$), predict the subgroup membership (or environmental status) $\hat{g}_i$, and prescribe an action $\hat{a}_i\in[K]$. %according to 
%\begin{equation}
%\hat{a}_i = \underset{k \in [K]}{\arg \max} \; \angles{\bx_{i,k}, \hat\bbeta_{\hat g_i}}. 
%\end{equation}

\subsection{Two Types of Regrets under Latent Heterogeneity}\label{sec:two-type-regret}

To evaluate the performance of any policy $\pi$, we must account for an important source of randomness: the subgroup model in \eqref{eqn:lhlb}  only specifies probabilities of group assignments rather than deterministic membership. 
As a result, when a customer $i$ arrives, there exist two different types of oracle: (1) the ``ex-post'' oracle who is able to precisely predict the true realized group membership $g_i$, and (2) the ``ex-ante'' oracle who knows the true parameter $\btheta^*$ of the subgroup model and thus knows the group probability $p(\bz_i^{\top}\btheta^*)$. 

This distinction gives rise to two types of regret measures when comparing against optimal policies derived from the two types of oracles. 
Let $\pi^{*}$ denote the {\it strong oracle rule}, which ``knows'' not only the true parameters $\bbeta_1^*$, $\bbeta_2^*$, and $\btheta^*$, but also the realized group $g_i$ beyond the probabilistic structure of the subgroup model. 
For each customer $i$, the strong oracle rule prescribes
\begin{equation}
a_i^{*} = \underset{k \in [K]}{\arg \max} \; \angles{\bx_{i,k}, \bbeta^*_{g_i}}.
\end{equation}
Alternatively, we let $\tilde{\pi} $ denote the  {\it regular oracle rule}, which ``knows'' the true parameters $\bbeta_1^*$, $\bbeta_2^*$, and $\btheta^*$, but {\em not} the realized group $g_i$. 
For each customer $i$, the regular oracle rule prescribes
\begin{equation}
\tilde{a}_i = \underset{k \in [K]}{\arg \max} \; \angles{\bx_{i,k}, \bbeta^*_{\tilde g_i}},
\end{equation}
where $\tilde g_i$ is estimated group using the oracle parameter $\btheta^*$  according to the decision rule in \eqref{eqn:lhlb}, i.e., $\tilde g_i = 1$ if $p(\bz_i^{\top}\btheta^*) \geq 1/2$ and $\tilde g_i=2$ otherwise.

To evaluate any allocation policy $\hat{\pi}$, we measure its performance relative to the two oracle rules.
Let $\hat{a}_i$ denote the action chosen by policy $\hat{\pi}$ for customer $i$. We define the {\em instant strong regret} comparing against the strong oracle,
\begin{equation}
\mathrm{reg}_i^{*} = %\EE\brackets{ 
	\underset{k\in [K]}{\max}\; \angles{\bx_{i,k}, \bbeta^*_{g_i}} - \angles{\bx_{i,\hat a_i}, \bbeta^*_{g_i}}, 
\end{equation}
 and the  {\em instant regular regret} comparing against the regular oracle,
\begin{equation}  
\tilde{\mathrm{reg}}_i = %\EE\brackets{
	 \angles{\bx_{i,\tilde a_i}, \bbeta^*_{g_i}} - \angles{\bx_{i,\hat a_i}, \bbeta^*_{g_i}}.
\end{equation}
The expected cumulative strong regret and regular regret at time $T$ are respectively defined as
\begin{equation}
{\mathrm{Reg}}^*(T) = \EE\brackets{\sum_{i=1}^{T} \mathrm{reg}^*_i},  \quad\text{and}\quad
\tilde{\mathrm{Reg}} (T) = \EE\brackets{\sum_{i=1}^{T} \tilde{\mathrm{reg}}_i},
\end{equation}
where the expectation is taken over the randomness in the feature vectors $\{\bx_{i,k}\}_{1\le k \le K}$ and $\bz_i$, group membership $g_i$, and the stochastic noise $\epsilon_i$. 
Our objective is to develop a policy that minimizes both types of expected cumulative regrets. 

\section{Learning and Decision under Latent Heterogeneity} \label{sec:method}

In this section, we present our methodological framework for addressing latent heterogeneity in the linear bandits problem. We first propose our phased learning algorithm that accounts for the latent group structures in Section \ref{sec:phased}, followed by a tailored expectation-maximization (EM) algorithm for parameter estimation in Section \ref{sec:EM}.

\subsection{Phased Learning and Greedy Decisions}\label{sec:phased}

\begin{algorithm}[ht!]
	\caption{Phased Learning and Greedy Decision under Latent Heterogeneity}
	\label{alg:em-bandit}
	\SetKwInOut{Input}{Input}
	\SetKwInOut{Output}{Output}
	\Input{Features $\{\bx_{i, k}, k \in [K]\}$ and $\bz_i$ for sequentially arriving customers $i$, and the minimal episode length $n_0$.}
	
%	\Output{Estimated optimal $a_i$ for $i\ge 1$.}
	
	%Initialize $\tau \leftarrow 1$, $\ba_1 \leftarrow \cU([K])$, $\hat{\btheta}$, $\hat{\bbeta}_1 \leftarrow 0$, $\hat{\bbeta}_1 \leftarrow 0$.
	
	\For{each episode $\tau=0, 1, 2, \dots$}{
		Set the length of the $\tau$-th episode as $N_{\tau} = 2^{\tau}n_0$ and define an index set $\calI_{\tau}$ with cardinality $N_{\tau}$;
		
		\If{$\tau=0$}{
			\For{$i\in\cI_{\tau}$}{
				Receive features $\{\bx_{i,k}, \bz_{i}\}_{k \in [K]}$ ;\\
				Select $a_i \sim {\rm Uniform}([K])$;\\
				Receive the reward $y_i$;
			}
		}
		\Else{
			
			%\tcc{Estimate parameter estimators using data from the previous episode.}
			
			Call Algorithm \ref{alg:em} ``Learning under Latent Heterogeneity'' to obtain $\hat{\btheta}^{(\tau)}$, $\hat{\bbeta}_1^{(\tau)}$, and $\hat{\bbeta}_2^{(\tau)}$ using data collected in the $(\tau - 1)$ -th episode, i.e.,  $\calD_{\tau-1}$;
			
			%\tcc{At each time point of the $\tau$-th episode, act greedily.}
			\For{$i\in\cI_{\tau}$}{
				
				Receive features $\{\bx_{i,k}, \bz_{i}\}_{k \in [K]}$;\\
				Predict the group membership $\hat{g}_i = 1$ if $\bz_i^{\top}\hat{\btheta}^{(\tau)}\ge 0$ and $\hat{g}_i = 2$ otherwise;
				
				Prescribe the optimal action based on estimation and prediction, that is, 
				\[
				a_i=\argmax_{k \in [K]}\; \angles{\bx_{i,k}, \hat{\bbeta}_{\hat{g}_i}^{(\tau)}};
				\]
				Receive the reward $y_i$;
			}
		}
		Collect the dataset $\calD_{\tau}=\{y_i, \bx_{i, a_i}, \bz_{i}\}_{i \in \cI_{\tau}}$;
	}
\end{algorithm}

Our proposed method exploits a key structural property of model \eqref{eqn:lhlb}: customers within the same latent group share common reward parameter $\bbeta^*_1$ or $\bbeta^*_2$ for different actions. The reward observed from taking an action provides information about the rewards of other potential actions due to the shared parametric structure. We leverage this property to develop an exploration-free algorithm that achieves minimax optimal regret.

The proposed Algorithm \ref{alg:em-bandit} implements a phased learning approach that divides the time horizon into non-overlapping episodes, indexed by $\tau =0,1,2,\dots$, and let $i \ge 1$ index sequentially arriving customers.
%The length of the $\tau$-th episode is denoted by $N_{\tau}$. 
In the initial episode 0, the actions are uniformly selected from the $K$ arms, which generates the necessary samples for the learning procedure in episode 1.
%When $\tau\ge 1$, the actions are chosen greedily according to model \eqref{eqn:lhlb}, the predicted subgroup membership $\hat{g}_i$, and the estimated model parameters $\hat{\btheta}$, $\hat{\bbeta}_1$, and $\hat{\bbeta}_2$.

For subsequent episodes ($\tau \geq 1$), model parameters $\hat{\btheta}$, $\hat{\bbeta}_1$, and $\hat{\bbeta}_2$ are updated at the start of each episode using Algorithm \ref{alg:em}, which employs expectation-maximization (EM) iterations and is detailed in Section \ref{sec:EM}. The updates utilize only the samples $\calD_{\tau-1}$ collected from the previous episode.%, which ensures that the actions chosen in each episode are independent of the model noises in the same episode. %which enables us to decouple the action selection from the noise terms, yielding sharper concentration bounds in our theoretical analysis.

With the updated parameter estimates, actions are chosen greedily by first predicting the customer's group membership $\hat{g}_i$ using current $\btheta^*$ estimates, then selecting the action that maximizes expected reward under the predicted group and current $\hbbeta_{\hat{g}_i}$ estimates.
The length of each episode, denoted by $N_{\tau}$,  increases geometrically as $N_{\tau} = n_0 2^{\tau}$, allowing for a more accurate estimate as the episodes progress.
While the algorithm terminates at the end of the horizon (time $T$), it does not require prior knowledge of $T$.



\subsection{Learning under Latent Heterogeneity and High-dimensionality} \label{sec:EM}

We now present the details of the learning procedure under latent heterogeneity in Algorithm \ref{alg:em}. 
Given the samples $\calD_{\tau-1}=\braces{ y_i, \bx_i := \bx_{i, a_i}, \bz_i}_{i \in \calI_{\tau-1}}$ collected in episode $\tau-1$, our goal is to estimate the unknown parameters in model \eqref{eqn:lhlb} via maximum likelihood estimator (MLE).
For notational clarity, we denote the index set of the samples input to Algorithm \ref{alg:em} as $\cI_{\tau-1}$ with size $N_{\tau-1}=\abs{\cI_{\tau-1}}$. 
The MLE aims to maximize the log-likelihood of the observed data $\calD_{\tau-1}$: 
\begin{equation}\label{eqn:llklh}
\ell_{N_{\tau-1}}(y_i,\bx_i,\bz_i; \bgamma) = \frac{1}{N_{\tau-1}}\sum_{i \in \cI_{\tau-1}}\log\brackets{p\paren{\bz_i^\top\btheta} \cdot
\phi\paren{ \frac{y_i - \bx_i^\top\bbeta_1}{\sigma} }
+
\paren{1-p\paren{\bz_i^\top\btheta}} \cdot
\phi\paren{ \frac{y_i - \bx_i^\top\bbeta_2}{\sigma}}},
\end{equation}
where we denote the unknown parameter by $\bgamma=(\btheta,\bbeta_1,\bbeta_2)$ and the standard normal density function by $\phi(\cdot)$. 
Directly searching for the maximizer of $\ell_{N_{\tau-1}}(y_i,\bx_i,\bz_i; \bgamma)$ is computationally intractable due to its non-convexity. Moreover, in the early episodes ($\tau$ is small), the dimension $d$ can substantially exceed the available sample size, making the estimation problem statistically infeasible.

\begin{algorithm}[t!]
\caption{Learning under Latent Heterogeneity in Episode $\tau$ ($\tau \geq 1$)}
\label{alg:em}
\SetKwInOut{Input}{Input}
\SetKwInOut{Output}{Output}
\Input{Batch data $\calD_{\tau-1}=\braces{y_i, \bx_i := \bx_{i, a_i}, \bz_i}_{i \in \cI_{\tau-1}}$, batch size $N_{\tau-1} = \abs{\cI_{\tau-1}}$, initial estimators $\bgamma^{(\tau, 0)}=\big(\btheta^{(\tau, 0)}, \bbeta_1^{(\tau, 0)},\bbeta_2^{(\tau, 0)}\big)$, maximum number of iterations $t_{\tau, \max}$, regularization parameters $\big\{\lambda_{n_{\tau}}^{(t)}\big\}_{t \in [t_{\tau, \max}]}$. \\
%constants $\kappa\in(0,1)$ and $C_{\lambda}>0$. 
}

\Output{estimates $\hat{\bgamma}^{(\tau)}=\big(\hat{\btheta}^{(\tau)},\hat{\bbeta}_1^{(\tau)},\hat{\bbeta}_2^{(\tau)}\big)$.}

Split $\calI_{\tau-1}$ into $t_{\tau, \max}$ subsets $\big\{\cI_{\tau-1}^{(t)}\big\}_{t \in [t_{\tau, \max}]}$, each of size $n_{\tau}=N_{\tau-1}/t_{\tau, \max}$; %\tcp{for theoretical consideration}

\For{$t=1, \dots, t_{\tau, \max}$}{
	
%\tcc{Update pseudo-soft-labels:}
	
For each $i\in\calI_{\tau-1}^{(t)}$, calculate $\omega_{i}^{(\tau, t)} = \omega\paren{y_i, \bx_i,  \bz_i; \bgamma^{(\tau, t-1)}}$, where $\omega$ is defined by
\begin{equation} \label{eqn:omega-main}
	\omega(y, \bx, \bz ; \bgamma) 
	= \frac{p(\bz^\top\btheta) \cdot
		\phi\paren{\frac{y - \bx^\top\bbeta_1}{\sigma}}}{p(\bz^\top\btheta) \cdot \phi\paren{\frac{y - \bx^\top\bbeta_1}{\sigma}} 
		+ \paren{1-p(\bz^\top\btheta)} \cdot \phi\paren{\frac{y - \bx^\top\bbeta_2}{\sigma}}},
\end{equation}
with $\bgamma=(\btheta,\bbeta_1,\bbeta_2)$,  $p(x) = 1/\paren{1 + \exp(-x)}$, and $\phi(x)=\frac{1}{\sqrt{2\pi}}e^{-x^2/2}$. 
	
Update each elements of $\bgamma^{(\tau, t)}$ by
\begin{equation} \label{eqn:coeff-updates}
\begin{aligned}
\bbeta_{1}^{(\tau, t)} & := \underset{\bbeta_1}{\arg\!\min} \;  Q_{n_{\tau}1}\paren{\bbeta_1 \cond \bgamma^{(\tau, t-1)}}  
+ \lambda_{n_{\tau}}^{(t)} \norm{\bbeta_1}_1,  \\
\bbeta_2^{(\tau, t)} & := \underset{\bbeta_2}{\arg\!\min} \; Q_{n_{\tau}2}\paren{\bbeta_2 \cond \bgamma^{(\tau, t-1)}}  
+ \lambda_{n_{\tau}}^{(t)} \norm{\bbeta_2}_1,   \\
\btheta^{(\tau, t)} & := \underset{\btheta}{\arg\!\min} \; Q_{n_{\tau}3}\paren{\btheta \cond \bgamma^{(\tau, t-1)}} 
+ \lambda_{n_{\tau}}^{(t)} \norm{\btheta}_1,
\end{aligned}
\end{equation}
 where $Q_{1n_{\tau}}$, $Q_{2n_{\tau}}$, and $Q_{3n_{\tau}}$ are defined in \eqref{eqn:Qn123} with the $t$-th subset $\cI_{\tau-1}^{(t)}$.

%\tcc{Update pseudo-soft-labels:}
%Update  $\omega_{i}^{(t+1)} = \omega\paren{\bx_i, y_i, \bz_i; \bgamma^{(t+1)}}$. 
}
Assign $\hat{\bgamma}^{(\tau)} = \bgamma^{(\tau, t_{\tau, \max})}$. 
\end{algorithm}

Algorithm \ref{alg:em} addresses these challenges through an EM algorithm, handling both the non-convexity of $\ell_{N_{\tau-1}}(\bgamma; y_i,\bx_i,\bz_i)$ and the high-dimensionality of the parameter space.
The EM algorithm is essentially an alternating maximization method, iterating between identifying the latent group membership $\{g_i\}$ and estimating the unknown parameter $\bgamma=(\btheta,\bbeta_1,\bbeta_2)$.
To ensure independence between samples across iterations, we first partition the entire sample set into disjoint subsets. 
In the E-step of the $t$-th iteration during episode $\tau$, given the parameters $\bgamma^{(\tau, t-1)}=(\btheta^{(\tau, t-1)},\bbeta_1^{(\tau, t-1)},\bbeta_2^{(\tau, t-1)})$ estimated from the previous iteration, the conditional probability of the $i$-th sample belonging to group 1 given the observed data is 
\begin{equation}
\omega_i^{(\tau, t)} = \omega_i(\bgamma^{(\tau, t-1)})
= \PP(g_i=1\cond y_i, \bx_i, \bz_i;\bgamma^{(\tau, t-1)})  =  \omega(y_i, \bx_i,  \bz_i; \bgamma^{(\tau, t-1)}),    
\end{equation}
where $\omega(y, \bx, \bz; \bgamma)$ is defined in \eqref{eqn:omega-main}. 
Let $\ell\paren{y_i, \bx_i,  \bz_i, g_i; \bgamma}$ be the log-likelihood of complete data where $g_i$ is also observable. 
Thus, conditional on the current estimate $\bgamma^{(\tau, t-1)}$, the conditional log-likelihood function can be calculated as
\begin{equation}
\begin{aligned}
Q_{n_{\tau}}(\bgamma \cond \bgamma^{(\tau, t-1)}) & := \; 
\sum_{i \in \cI_{\tau}^{(t)}}\EE\brackets{ \ell\paren{y_i, \bx_i, \bz_i, g_i; \bgamma} \cond y_i, \bx_i, \bz_i; \bgamma^{(\tau, t-1)}} \\
& = \; Q_{n_{\tau}1}(\bbeta_1 \cond \bgamma^{(\tau, t-1)}) 
+ Q_{n_{\tau}2}(\bbeta_2 \cond \bgamma^{(\tau, t-1)})
+ Q_{n_{\tau}3}(\btheta \cond \bgamma^{(\tau, t-1)}),
\end{aligned}
\end{equation}
where
\begin{equation} \label{eqn:Qn123}
\begin{aligned}
    Q_{n_{\tau}1}(\bbeta_1 \cond \bgamma^{(\tau, t-1)}) & :=  \frac{1}{2n_{\tau}} \sum_{i \in \cI_{\tau}^{(t)}} \omega_i^{(\tau, t)}\cdot  \frac{(y_i - \bx_i^\top\bbeta_1)^2 }{\sigma^2}, \\
    Q_{n_{\tau}2}(\bbeta_2 \cond \bgamma^{(\tau, t-1)}) & :=  \frac{1}{2n_{\tau}} \sum_{i \in \cI_{\tau}^{(t)}} (1- \omega_i^{(\tau, t)}) \cdot \frac{(y_i - \bx_i^\top\bbeta_2)^2 }{\sigma^2}, \quad\text{and}\quad \\
    Q_{n_{\tau}3}(\btheta \cond \bgamma^{(\tau, t-1)}) & := - \frac{1}{n_{\tau}} \sum_{i \in \cI_{\tau}^{(t)}} \paren{ \omega_i^{(\tau, t)} \cdot \log p(\bz_i^\top\btheta) + (1- \omega_i^{(\tau, t)}) \cdot \log(1-p(\bz_i^\top\btheta)) }.
\end{aligned}
\end{equation}
The M-step proceeds by maximizing $Q_{n_{\tau}}(\bgamma \cond \bgamma^{(\tau, t-1)})$, which is equivalent to maximizing $Q_{n_{\tau}1}(\bbeta_1 \cond \bgamma^{(\tau,t-1)})$, $Q_{n_{\tau}2}(\bbeta_2 \cond \bgamma^{(\tau,t-1)})$, and $Q_{n_{\tau}3}(\btheta \cond \bgamma^{(\tau,t-1)})$ simultaneously. 
However, in the high-dimensional setting, direct maximization of these objectives tends to overfit the data. To address this challenge, our algorithm incorporates regularization terms $\norm{\bbeta_1}_1$, $\norm{\bbeta_2}_1$, and $\norm{\btheta}_1$ to induce sparsity in the parameter estimates.
The specific updates of the model parameters are presented in \eqref{eqn:coeff-updates}. 
%We also update $\omega_{i}^{(t+1)} = \omega\paren{\bx_i, y_i, \bz_i; \bgamma^{(t+1)}}$. 
The algorithm then proceeds iteratively, alternating between the E-step and M-step until convergence.


\section{Theoretical Analysis}\label{sec:theory}

%The classical low-dimensional setting is considered in \cite{zhu2004hypothesis}.
%We consider a high-dimensional setting. 
 
In this section, we establish theoretical guarantees for the estimation error of  $\big(\hbtheta^{(\tau)}, \hbbeta_1^{(\tau)}, \hbbeta_2^{(\tau)}\big)$ learned in the $\tau$-th episode using $N_{\tau-1}=n_02^{\tau-1}$ samples collected from the previous episode.
%Let $\calI_{\tau-1}$ be the set of time indices collected from the $(\tau-1)$-th episode. 
%We first establish the properties of estimators $\big(\hat\btheta^{(\tau)}, \hat\bbeta_1^{(\tau)}, \hat\bbeta_2^{(\tau)}\big)$ obtained  at the $\tau$-th episode. 
We begin by introducing the parameter space that characterizes the sparsity of the true parameters $( \btheta^*, \bbeta_1^*, \bbeta_2^*)$:
\begin{equation}
\btheta^*, \bbeta_1^*, \bbeta_2^* \in \Theta(d, s) = \braces{ \btheta, \bbeta_1, \bbeta_2 \in \RR^d:\norm{\btheta}_0 \le s,
\norm{\bbeta_1}_0 \le s, 
\norm{\bbeta_2}_0 \le s
},
\end{equation}
where $\btheta^*$ and $\bbeta^*$ are assumed to share the same dimension $d$ and sparsity level $s$ without loss of generality. 
Our analysis relies on the following regularity conditions:
\begin{enumerate}[label=(A\arabic*)]
\item Assume that there exists $0< \xi < 1/2$ such that $\xi < p(\bz_i^\top\btheta^*) < 1 - \xi$ or equivalently, there exists $C_{\xi} > 0$ such that $\abs{\bz_i^{\top}\btheta^*} < C_{\xi}$ for all $i$. \label{A1}
\item Assume that the initial estimators in the first episode satisfy
$\big\|\bbeta_1^{(1, 0)}-\bbeta_1^*\big\|_2 + \big\|\bbeta_2^{(1, 0)}-\bbeta_2^*\big\|_2+ \norm{\btheta^{(1, 0)}-\btheta^*}_2 \le \delta_{1, 0}$ %\min\braces{\xi, 1-\xi, \norm{\bbeta_1^*-\bbeta_2^*}_2}$ 
for a sufficiently small constant $\delta_{1, 0}$. \label{A2}
\item\label{A3} Define the signal strength $\Delta^* \defeq \norm{\bbeta_1^*-\bbeta_2^*}_2$. Assume that the signal-to-noise ratio (SNR), defined as $\Delta^*/ \sigma$, satisfies $\Delta^*/ \sigma \ge C_{\mathrm{SNR}}$ for a sufficiently large constant $C_{\mathrm{SNR}}$, where $\sigma$ is the standard deviation of the noise $\epsilon_i$. 
\item For each $k\in [K]$, assume that the i.i.d. covariates $(\bx_{i, k}, \bz_i)$ are sub-Gaussian. Moreover, let $\bSigma_{x, k}\defeq\EE[\bx_{i,k}\bx_{i, k}^{\top}]$ and $\bSigma_z\defeq\EE[\bz_{i}\bz_i^{\top}]$, and assume that there exists a costant $M>1$ such that  $1/ M < \lambda_{\min}(\bSigma_{x, k}) \le \lambda_{\max}(\bSigma_{x, k}) < M$ for all $k \in [K]$ and $1/ M < \lambda_{\min}(\bSigma_z) \le \lambda_{\max}(\bSigma_z) < M$. \label{A4}
\end{enumerate}

Assumption \ref{A1} requires non-degenerate group assignment probabilities, preventing the variance of the group indicator $g_i$ from vanishing, as $\Var(g_i) = p_i(1-p_i)$ is bounded away from zero. This condition is standard in high-dimensional logistic regression literature \citep{van2014asymptotically,abramovich2018high,athey2018approximate}.
Assumption \ref{A2} assumes the initial estimators to be within a neighborhood of the true parameters, which can be achieved through any consistent initialization procedure with sufficient samples in episode 0. 
%Due to the non-concavity of the log-likelihood, such a condition is common in the literature of mixed linear regression, {\red \citep{???}}.  
We provide a detailed initialization algorithm in Remark \ref{rmk: Initialization}.
Assumption \ref{A3} characterizes the minimal signal strength required for effective group separation, which is necessary for distinguishing different groups in mixture models \citep{loffler2021optimality, ndaoud2022sharp}. For clear presentation, we defer the explicit specification of constants $\delta_{1, 0}$ and $C_{\mathrm{SNR}}$  to Section  \ref{sec:proof-coeff-bound}  of the supplementary material.
Assumption \ref{A4}  imposes standard regularity conditions that ensure the non-singularity and upper-boundedness of the population covariance matrices.

\begin{remark}[Initialization]\label{rmk: Initialization}
	The initial estimators $\left(\btheta^{(1, 0)}, \bbeta^{(1, 0)}_{1}, \bbeta^{(1, 0)}_{2}\right)$ in the first episode can be obtained through any algorithm that generates consistent estimators for $(\btheta^*, \bbeta^*_{1}, \bbeta^*_{2})$ that satisfy Assumption \ref{A2}.  We recommend the following approach, which proceeds in three stages. First, using all the samples $\{\by_i, \bx_{i}\}_{i \in \calI^{(0)}}$ collected in episode 0, we fit a single LASSO model to select the features in $\bx$ with non-zero coefficients.  Second, we apply Gaussian Mixture clustering to $\{y_i, \bx_{i}'\}_{i\in \calI^{(0)}}$, where $\bx_{i}'$ denotes the selected features observation $i$, to cluster the initial data into two groups. Third, we use these group labels to fit a logistic regression model, obtaining the estimator $\btheta^{(1, 0)}$, and separately fit two LASSO models within the two groups to obtain  $\bbeta^{(1, 0)}_{1}$ and $\bbeta^{(1, 0)}_{2}$. Prior research has demonstrated that Gaussian Mixture clustering can achieve high accuracy of the two groups when the signal-to-noise ratio is sufficiently large  \citep{loffler2021optimality, ndaoud2022sharp}, which, combined with \ref{A3} and sufficiently large $n_0$, ensures that our initial estimators satisfy \ref{A2}. 
\end{remark}

 \subsection{Learning Performance} \label{sec:theory-learning}
 
 
Building upon the above assumptions, we present our theoretical results, which characterize the estimation error of the learned parameters.


\begin{theorem}%[Coefficients bound under the high-dimensional setting.]
	 \label{thm:coeff-bound-hd}
Suppose Assumptions \ref{A1}--\ref{A4} hold and $s^2\log d\log n_0 \lesssim n_0$. Let the initial estimators $\bgamma^{(\tau, 0)}=\hat\bgamma^{(\tau-1)}$ for $\tau \geq 2$. Furthermore, select $t_{\tau, \max}\asymp 
\log n_0$  for $\tau=1$ and $t_{\tau, \max}\asymp 
1$ for $\tau\geq 2$. By choosing appropriate regularization parameters $\big\{\lambda_{n_{\tau}}^{(t)}\big\}$, we have
\begin{equation}\label{eq:est-bound-l2}
\norm{\hat{\bbeta}_{1}^{(\tau)} - \bbeta_{1}^*}_2 + \norm{\hat{\bbeta}_{2}^{(\tau)} - \bbeta_{2}^*}_2 + \norm{\hat{\btheta}^{(\tau)} - \btheta^*}_2
\lesssim 
\sqrt{\frac{s\log d\log n_0}{N_{\tau-1}}},
\end{equation}
and
\begin{equation}\label{eq:est-bound-l1}
	\norm{\hat{\bbeta}_{1}^{(\tau)} - \bbeta_{1}^*}_1 + \norm{\hat{\bbeta}_{2}^{(\tau)} - \bbeta_{2}^*}_1 + \norm{\hat{\btheta}^{(\tau)} - \btheta^*}_1
	\lesssim 
	\sqrt{\frac{s^2\log d\log n_0}{N_{\tau-1}}},
\end{equation}
with probability at least $1-d^{-1}$, where $\big(\hat{\btheta}^{(\tau)},\hat{\bbeta}_{1}^{(\tau)},\hat{\bbeta}_{2}^{(\tau)}\big)$ are obtained from Algorithm \ref{alg:em} in the $\tau$-th episode.
\end{theorem}

Theorem \ref{thm:coeff-bound-hd} establishes the statistical convergence rates for both $\ell_2$ and $\ell_1$ estimation errors. The $\ell_2$ error bound scales as  $\bigO{\sqrt{s\log d \log n_0 / N_{\tau-1}}}$, while the $\ell_1$ estimation error scales as $\bigO{\sqrt{s^2\log d \log n_0 / N_{\tau-1}}}$, matching the minimax optimal rates for high-dimensional sparse estimation up to a logarithm factor in $n_0$. We note that Theorem \ref{thm:coeff-bound-hd} serves as a simplified version of our theoretical results, a complete version of which is provided as Theorem \ref{thm:1-detailed} in Section \ref{sec:proof-coeff-bound} of the supplemental materials, where we explicitly specify the choice of the regularization parameters $\big\{\lambda_{n_{\tau}}^{(t)}\big\}$, the concrete requirements for the constants $\delta_{1, 0}$ and $C_{\mathrm{SNR}}$, and more accurate probability bounds.

To better understand the convergence behavior, we provide a more detailed analysis in Remark \ref{rmk:logn0}, which elucidates the role of $t_{\tau,\max}$ and explains the origin of the additional $\sqrt{\log n_0}$ factor in the error bounds. This analysis reveals how the algorithm's phased learning structure and the initial estimation error influence the final convergence rates.

\begin{remark}\label{rmk:logn0}
	Our theoretical analysis in Section \ref{sec:proof-coeff-bound} of the supplemental material establishes a finer error bound for the estimators $\big(\hbtheta^{(\tau)}, \hbbeta_1^{(\tau)},  \hbbeta_2^{(\tau)}\big)$:
	\begin{equation}\label{eq:est-bound-t}
		\norm{\hbbeta_1^{(\tau)} - \bbeta_1^*}_2 + \norm{\hbbeta_2^{(\tau)} - \bbeta_2^*}_2 + \norm{\hbtheta^{(\tau)} - \btheta^*}_2 
		\lesssim
		\rho^{t_{\tau, \max}}\delta_{\tau, 0}
		+ \sqrt{\frac{st_{\tau, \max}\log d}{N_{\tau-1}}},
	\end{equation}
	where $\rho<1$ is a contraction factor, and $\delta_{\tau, 0}:=
	\big\|\bbeta_1^{(\tau, 0)} - \bbeta_1^*\big\|_2 + \big\|\bbeta_2^{(\tau, 0)} - \bbeta_2^*\big\|_2 + \norm{\btheta^{(\tau, 0)} - \btheta^*}_2$. The first term in \eqref{eq:est-bound-t} quantifies the contraction of initial estimation error, while the second term represents the statistical error rate. For $\tau=1$, since Assumption \ref{A2} assumes the initial estimation error $\delta_{1, 0}$ to be a constant, we require $t_{1, \max} \asymp \log n_0$ iterations to ensure the first term $\rho^{t_{\tau, \max}}\delta_{1, 0}$ is dominated by the second term, yielding a rate of  $\sqrt{\frac{s\log n_0 \log d}{N_{\tau-1}}}$. For $\tau \geq 2$, since the initial estimation error $\delta_{\tau, 0} \asymp  \sqrt{\frac{s\log n_0 \log d}{N_{\tau-2}}}$, a constant number of iterations suffices. If we strengthen \ref{A2} to require that $\delta_{1, 0}=\bigO{a_{n_0}}$ for some $a_{n_0}=o(1)$, then the extra $\sqrt{\log n_0}$ factor in \eqref{eq:est-bound-l2} and \eqref{eq:est-bound-l1} can be reduced to $\sqrt{\log (n_0a_{n_0}^2)}$.
	\end{remark}

%\section{Theoretical Analysis of Decision Performance}


\subsection{Classification Accuracy}\label{sec:theory-classification}

In this section, we provide a theoretical guarantee for the statistical accuracy of the latent group identification procedure Algorithm \ref{alg:em-bandit}. 
%For ease of presentation, we denote the estimated coefficients for decision in the $\tau$-th episode as $\hat\bgamma = \hat\bgamma^{(T)}:= (\hat\bbeta_1^{(T)}, \hat\bbeta_2^{(T)}, \hat\btheta^{(T)})$. 
In the $\tau$-th episode, after obtaining $\big(\hat\btheta^{(\tau)}, \hat\bbeta_1^{(\tau)}, \hat\bbeta_2^{(\tau)}\big)$, Algorithm \ref{alg:em-bandit} employs a Bayes classifier $G\big(\bz_i; \hat\btheta^{(\tau)}\big)$, defined as  
\begin{align*} \label{eqn:bayes-classifier}
G(\bz_i; \btheta) = 
\begin{cases}
1, & p(\bz_i^\top\btheta)\ge 1/2, \\
2, & p(\bz_i^\top\btheta) < 1/2.
\end{cases}
\end{align*}

To characterize the classification performance, we define the optimal misclassification error achievable with the true parameter,
$
R(\btheta^*) := \EE\brackets{\bbone(G(\bz_i;\btheta^*) \ne g_i)},
$
and the misclassification error of the estimated classifier,
$
R\big(\hat\btheta\big) := \EE\brackets{\bbone(G(\bz_i;\hat\btheta) \ne g_i) \mid \hbtheta}.
$

\begin{theorem} \label{thm:miss-class-rate}
Let $\hat{\btheta}^{(\tau)}$ be the estimator obtained in the $\tau$-th episode for $\tau \geq 1$. Under the assumptions of Theorem \ref{thm:coeff-bound-hd}, we have that the excess misclassification error  satisfies
\begin{align*}
R\big(\hat\btheta^{(\tau)}\big)-R(\btheta^*) \lesssim \sqrt{\frac{s\log d\log n_0}{N_{\tau-1}}},
\end{align*}
with probability at least $1-d^{-1}$.
\end{theorem}

Theorem \ref{thm:miss-class-rate} reveals that the excess misclassification error converges at the same rate as the parameter estimation error bound established in Theorem \ref{thm:coeff-bound-hd}, which is useful for our subsequent regret analysis in Section \ref{sec:theory-regret}. %The proof of Theorem \ref{thm:miss-class-rate} is provided in Section \ref{sec:proof-misclustering} of the supplemental material.

\subsection{Regret Analysis}
\label{sec:theory-regret}

Recall that, for any policy $\hat{\pi}$, we define two types of regrets in Section \ref{sec:two-type-regret}: (1) the strong regret, $\reg^{*}_i = \underset{a\in [K]}{\max}\; \angles{\bx_{i,a}, \bbeta^*_{g_i}} - \angles{\bx_{i,\hat a_i}, \bbeta^*_{g_i}}$, where $\hat{a}_i$ is the action chosen by policy $\hat{\pi}$ for customer $i$, and (2) the regular regret, $\tilde{\reg}_i= \angles{\bx_{i,\tilde{a}_i}, \bbeta^*_{g_i}} - \angles{\bx_{i,\hat a_i}, \bbeta^*_{g_i}}$, where $\tilde a_i = \underset{a\in [K]}{\arg \max} \angles{\bx_{i, a}, \bbeta_{\tilde g_i}}$ and $\tilde g_i = G(\bz_i; \btheta^*)$. These two regret formulations arise from comparing $\hat{\pi}$ against two types of oracles: the strong oracle, which has access to the value of $(\bbeta_1^*, \bbeta_2^*)$ and the latent group labels $g_i$, and the regular oracle, which knows $(\btheta^*, \bbeta_1^*, \bbeta_2^*)$.  Notably, the regular oracle is weaker than the strong oracle since even with the known $\btheta^*$, there remains an inherent positive misclassification error $R(\btheta^*)$ due to probabilistic nature of the subgroup model in \eqref{eqn:lhlb}. Consequently, the strong regret necessarily exceeds the regular regret.

The cumulative strong and  regular regrets over time horizon$T$ are defined, respectively, as 
\begin{equation*}
\Reg^{*} (T)= \sum_{\tau=0}^{\tau_{\max}}\sum_{i\in\calN_{\tau}}\EE[\reg^{*}_i], \quad\text{and}\quad 
\tilde{\Reg} (T)= \sum_{\tau=0}^{\tau_{\max}}\sum_{i\in\calN_{\tau}}\EE\left[\tilde{\reg}_i\right],
\end{equation*}
where $\tau_{\max}:=[\log_2(T/n_0+1)]-1$ is the maximum number of episodes within horizon length $T$. To establish theoretical bounds for $\Reg^{*} (T)$ and $\tilde{\Reg} (T)$, we require the following additional conditions:
\begin{enumerate}[label=(B\arabic*)]
%\item Sparsity assumption: There exist positive constants $c$ and $s$ such that $\norm{\bbeta_k^*}_1 \le c$ and $\norm{\bbeta_k^*}_0 \le s$ for all $k=1,2$. 
%\item Boundedness: There exists a positive constant $B$ such that $\norm{\bx_i}_{\infty} \le B$ for all $i\in[T]$. 
\item  Assume that $\norm{\bx_{i, k}}_{\infty} \leq \overline{x}$ for some constant $\overline{x}>0$ and the coefficient $\big\|\bbeta_g^*\big\|_1 \leq \overline{L}$ for some constant $\overline{L}>0$ for $g=1,2$. Consequently, the reward function is bounded: $\abs{\angles{\bx_{i, k}, \bbeta_g^*}} \leq \overline{R}=\overline{x}\overline{L}$ for all $i$ and $g=1, 2$. \label{B1}
\item  Assume that there exist positive constants $C$ and $\overline{h}$ such that, for all $h\in\big[0, \overline{h}\big]$, we have $\PP\paren{\angles{\bx_{i, \tilde{a}_{i, g}}, \bbeta_g^*} \le \max\limits_{k\ne \tilde{a}_{i,g}}\angles{\bx_{i, k}, \bbeta^*_{g}} + h} \le Ch $ for $g=1,2$, where $\tilde{a}_{i, g}:=\underset{k\in [K]}{\arg \max} \angles{\bx_{i, k}, \bbeta_g^*}$. \label{B2}
\end{enumerate}

Assumptions \ref{B1} and \ref{B2} both are standard conditions that have been widely assumed in the linear bandit literature \citep{goldenshluger2013linear, bastani2020online, bastani2020mostly, li2021regret, wang2024online}. Specifically, Assumption \ref{B1} ensures that the maximum regret at each time is upper bounded.  Assumption \ref{B2} plays an important role in controlling the distribution of the covariates in the neighborhood of decision boundaries, where small perturbations in parameter estimates can lead to changes in the selected actions. This assumption is satisfied under relatively mild conditions, for example, the density of $\angles{x_{i, k}, \bbeta_g^*}$ is uniformly bounded for all $k \in [K]$. Remark \ref{rmk:margin} further explains the necessity of Assumption \ref{B2} in the context of latent heterogeneity.

We then establish theoretical upper bounds for both the strong and regular regrets of Algorithm \ref{alg:em-bandit}.

\begin{theorem} \label{thm:regret}
Assume the assumptions in Theorem \ref{thm:coeff-bound-hd} and Assumptions \ref{B1} and \ref{B2} hold. We have the cumulative strong regret 
\begin{equation}\label{eq:upper-strong}
\Reg^*(T) \lesssim  \overline{R}n_0 +\overline{x}^2s^2\log d \log n_0\log T+ \overline{x} \norm{\bbeta_2^* - \bbeta_1^*}_1R(\btheta^*) T.
\end{equation}
and the cumulative regular regret 
\begin{equation}\label{eq:upper-regular}
\tilde{\Reg}(T) \lesssim \overline{R}n_0+ \overline{x}^2 \norm{\bbeta_2^*-\bbeta_1^*}_1\sqrt{s^2\log d \log n_0 } \sqrt{T}.
\end{equation}
\end{theorem}

%The proof of Theorem \ref{thm:regret} is provided in Section \ref{sec:proof-upper} in the supplemental materials.
The upper bound for strong regret comprises three terms. The first term captures the cost of initial exploration in episode 0, and the second term stems from the small failure probabilities established in Theorems \ref{thm:coeff-bound-hd} and \ref{thm:miss-class-rate}. The third term, which dominates the other two terms for large $T$, reflects the impact of misclassification. As defined in Section \ref{sec:theory-classification}, the classifier $G\big(\bz_i; \btheta^{(\tau)}\big)$ misclassifies a customer $i$ in episode $\tau$ with probability $R\big(\btheta^{(\tau)}\big)$, which converges to a constant $R(\btheta^*)$  as characterized by Theorem \ref{thm:miss-class-rate}. Under such misclassification, the instant strong regret $\reg^{*}_i$ attains a constant level $\overline{x} \norm{\bbeta_2^* - \bbeta_1^*}_1$, leading to the linear term $\overline{x} \norm{\bbeta_2^* - \bbeta_1^*}_1R(\btheta^*) T$.

However, this linear term vanishes in the regular regret bound. This occurs because the regular regret compares our algorithm against oracle actions selected using the true value of $\btheta^*$, which itself incurs a constant $R(\btheta^*)$ misclassification rate. The two constants cancel each other, leaving only an $O(\sqrt{T\log d})$ term that emerges from the failure of the event defined in Assumption \ref{B2}.


\begin{remark}\label{rmk:margin}
	Assumption \ref{B2} is essential for establishing bound \eqref{eq:upper-regular} for the regular regret under latent heterogeneity. Violation of this assumption makes it fundamentally difficult to distinguish the optimal arm from the sub-optimal arms, resulting in sub-optimal selection, i.e.,  $\hat{a}_{i}=\argmax_k \angles{\bx_{i, k}, \hbbeta_{g}} \neq \tilde{a}_{i, g}$. This issue remains manageable for regret analysis within a single group since $\angles{\bx_{i, \tilde{a}_{i,g}}, \bbeta^*_{g}}-\angles{\bx_{i, \hat{a}_{i}}, \bbeta^*_{g}}=\max_k \angles{\bx_{i, k}, \bbeta^*_{g}}-\max_k \angles{\bx_{i, k}, \hbbeta_{g}}+\angles{\bx_{i, \hat{a}_i}, \hbbeta_{g}}-\angles{\bx_{i, \hat{a}_{i}}, \bbeta^*_{g}} \leq 2\sup \abs{\angles{\bx_{i, k},\hbbeta_{g}- \bbeta^*_{g}}}$ is still bounded by the $\ell_1$ estimation error $\big\|\hbbeta_{g}-\bbeta^*_{g}\big\|_1$. However, with latent heterogeneity, the consequence of sub-optimal selection becomes more severe. Consider a customer $i$ with $p(\bz_i^{\top}\btheta^*)<1/2$, who can be assigned to group $g_i=1$ with a positive probability. In such case, if the estimated action $\hat{a}_{i}=\argmax_k \angles{\bx_{i, k}, \hbbeta_{2}} \neq \tilde{a}_{i, 2}$, the regular regret $\angles{\bx_{i, \tilde{a}_{i,2}}, \bbeta^*_{1}}-\angles{\bx_{i, \hat{a}_{i}}, \bbeta^*_{1}}$ can attain a constant level, regardless of estimation accuracy. 
\end{remark}

We further develop minimax lower bounds for the strong and regular regrets.  For any policy $\hat\pi$, define \[\EE_{\hat\pi}[\reg^*_{i}]:=\EE\left[\underset{k\in [K]}{\max}\; \angles{\bx_{i,k}, \bbeta^*_{g_i}}\right] -\EE_{\hat\pi}\left[\angles{\bx_{i,\hat a_i}, \bbeta^*_{g_i}}\right], \quad \EE_{\hat\pi}[\widetilde{\reg}_{i}]:=\EE\left[\angles{\bx_{i,\widetilde{a}_i}, \bbeta^*_{g_i}}\right] -\EE_{\hat\pi}\left[\angles{\bx_{i,\hat a_i}, \bbeta^*_{g_i}}\right], \]
where $\EE_{\hat\pi}$ represents that the actions $\hat{a}_i$ are chosen according to the policy $\hat\pi$. We then establish the following lower-bound result.
\begin{theorem}\label{thm:lower-bound} Let $\mu(y, \bx, \bz; \bgamma^*)$ denote a distribution of $(y_i, \bx_i, \bz_i)$ that satisfies model \eqref{eqn:lhlb}  with $\bgamma^*=(\btheta^*, \bbeta_1^*, \bbeta_2^*)$, and define $\calP_{d, s,\overline{x},\overline{L}}$ as the collection of all the distributions $\mu(y, \bx, \bz; \bgamma^*)$ such that $\bgamma^* \in \Theta(d, s)$ and Assumptions \ref{A1}, \ref{A4}, \ref{B1}, and \ref{B2} hold with constants $\overline{x}$ and $\overline{L}$. Then we have
	\begin{equation}\label{eq:lower-strong}
			\inf_{\hat\pi}\sup_{\mu \in \calP_{d, s, \overline{x}, \overline{L}}}\sum_{i=1}^{T}\EE_{\hat\pi}[\reg^*_{i}]\gtrsim \overline{x}\overline{L}R(\btheta^*)T,  \quad	\inf_{\hat\pi}\sup_{\mu \in \calP_{d,s,\overline{x},\overline{L}}} \sum_{i=1}^{T}\EE_{\hat\pi}[\widetilde{\reg}_{i}]\ \gtrsim \overline{x}\overline{L}\sqrt{s\log d}\sqrt{T},
	\end{equation}
	where the infimum is taken over all the possible policies $\hat{\pi}$.
\end{theorem}

Compared to the upper bounds established in Theorem \ref{thm:regret}, the lower bounds replace $\norm{\bbeta_1^{*}-\bbeta_2^{*}}_1$ by $\overline{L}$, which are equivalent since $\sup\norm{\bbeta_1^{*}-\bbeta_2^{*}}_1 = 2\overline{L}$ under Assumption \ref{B1}. The lower bound $\Omega(R(\btheta^*)T)$ for the strong regret precisely matches the dominant linear term in the upper bound \eqref{eq:upper-strong}, while the lower bound $\Omega(\sqrt{Ts\log d})$ aligns with the upper bound \eqref{eq:upper-regular} in terms of $T$ and $d$, demonstrating the minimax optimality of our proposed method under latent heterogeneity. The gap between the upper and lower bounds for the regular regret amounts to a factor of $\sqrt{s\log n_0}$, which, as discussed in Remark \ref{rmk:logn0}, stems from the sample splitting procedure and can be reduced by assuming stronger conditions on the initialization.
%The proof of Theorem \ref{thm:lower-bound} is provided in Section \ref{sec:proof-lower} of the supplementary material.


\section{Numerical Study} \label{sec:numerical}

In this section, we evaluate the performance of our proposed heterogeneous algorithm through numerical studies. Section \ref{sec:simul} presents simulation studies to show the effectiveness of our algorithm and validate our theoretical findings under various settings. Section \ref{sec:real} further illustrates the practical utility of our method through an application to a cash bonus dataset from a mobile commerce platform.

\subsection{Simulations} \label{sec:simul}


In this section, we conduct numerical simulations to demonstrate the performance of our proposed algorithm under varying conditions. Our simulation is implemented based on the model: $y_{i, k}= \angles{\bx_{i, k}, \bbeta_{g_i}^*}+\epsilon_i$ for $\bx_{i, k} \in \RR^d$, $k \in [K]$ and $g_i \in \{1, 2\}$, with $\epsilon_i \sim \calN(0, \sigma^2)$.  We let the number of available actions $K=2$ and generate the covariates $\bx_{i, k} \sim \calN(\bmu_{k}, \bSigma_k)$ ($k = 1,2$). Each entry of $\bmu_{1}$ and $\bmu_{2}$ is independently generated from $\calN(1, (0.5)^2)$ and $\calN(-1, (0.5)^2)$, respectively, and the covariance matrix $\bSigma_1=\bSigma_2$ is set to be an AR(1) matrix with correlation $0.5$, that is, $(\bSigma_1)_{j_1, j_2}=(\bSigma_2)_{j_1, j_2}=0.5^{\abs{j_1-j_2}}$. For the parameters $\bbeta_g^*$, we set $\bbeta_{1, j}^* = \overline{L}\bbone(1\leq j \leq s) / s$ and $\bbeta_{2, j}^* = -\overline{L}\bbone(d/2\leq j \leq d/2+s) / s $ for $j=1,\dots,d$, which ensures that $\norm{\bbeta_1^*}_1=\norm{\bbeta_2^*}_1=\overline{L}$. Specifically, we fix the noise level $\sigma=1$ and the sparsity $s=20$ and vary $\overline{L} \in \{2.5, 5\}$ and $d \in \{500, 1000\}$. The group assignment probability is determined by $\PP(g_i = 1 | \bz_i) = 1 / (1 + \exp(-z^{\top}_i \btheta^*))$, where $z_i \in \RR^{50}$ is generated from $N(0, I_{50})$, and $\btheta^*$ has nonzero entries in its first 10 dimensions, drawn uniformly from $[-1, 1]$. 

To assess the performance of our proposed algorithm (Algorithm \ref{alg:em-bandit}) and verify our theoretical results, we compare the average strong and regular regrets, i.e., $\frac{1}{T}\Reg^*(T)$ and $\frac{1}{T}\widetilde{\Reg}(T)$, against a benchmark---a single LASSO method applied without considering the latent group structure. In both algorithms, the regularization parameters $\lambda$ are chosen through cross-validation, and the maximum number of iterations $t_{\max}$ in Algorithm \ref{alg:em} is set as 1 for all episodes. Figure \ref{fig:regret} presents the average strong and regular regrets of our proposed algorithm (``hetero'') and the single LASSO (``single'') for $s=20$, $d\in\{500, 1000\}$, and $\overline{L} \in \{2.5, 5\}$. All results are averaged over 100 independent runs. %Additional simulation results, including those for $s=5$, are relegated to the supplementary material. 

\begin{figure}[!t]
	\centering
	\begin{subfigure}[b]{0.45\textwidth}
		\centering
		\includegraphics[width=\textwidth]{figs/regret_p=500_rho=2.5_s=20.png}
		\caption{$d=500, \overline{L}=2.5$}
		\label{fig:regret_500_2.5}
	\end{subfigure}
	\hfill
	\begin{subfigure}[b]{0.45\textwidth}
		\centering
		\includegraphics[width=\textwidth]{ figs/regret_p=500_rho=5_s=20.png}
		\caption{$d=500, \overline{L}=5$}
		\label{fig:regret_500_5}
	\end{subfigure}
	
	\begin{subfigure}[b]{0.45\textwidth}
		\includegraphics[width=\textwidth]{ figs/regret_p=1000_rho=2.5_s=20.png}
		\caption{$d=1000, \overline{L}=2.5$}
		\label{fig:regret_1000_2.5}
	\end{subfigure}
	\hfill
	\begin{subfigure}[b]{0.45\textwidth}
		\centering
		\includegraphics[width=\textwidth]{ figs/regret_p=1000_rho=5_s=20.png}
		\caption{$d=1000, \overline{L}=5$}
		\label{fig:regret_1000_5}
	\end{subfigure}
	\caption{Average strong and regular regrets with $s=20$, $d=\{500, 1000\}$ and $ \overline{L}\in \{2.5, 5\}$. The horizontal axis ``time'' represents the sample size $T$. }
	\label{fig:regret}
\end{figure}



As shown in Figure \ref{fig:regret}, our proposed heterogeneous algorithm significantly outperforms the ``single'' algorithm, demonstrating the critical importance of identifying latent group heterogeneity. Under all scenarios, the average regular regret of our algorithm (``regular regret - hetero'') approaches zero as $T$ increases, while the average strong regret (``strong regret - hetero'') stabilizes at a constant level. This observation aligns closely with Theorem \ref{thm:regret}, where we theoretically establish that the cumulative strong regret grows linearly in $T$, and the cumulative regular regret exhibits sublinear growth. 


Furthermore, when the parameter magnitude $\overline{L}$ increases from $2.5$ to $5$, the constant level of average strong regret approximately doubles. This behavior is consistent with the theoretical bound derived in equations \eqref{eq:upper-strong} since $\norm{\bbeta^*_1-\bbeta_2^*}_1 \propto \overline{L}$ in our experimental setup. Moreover, the dimensionality of the problem $d$ demonstrates minimal impact on the regret performance when varying from $500$ to $1000$, illustrating our algorithm's robustness and effectiveness in handling high-dimensional parameter spaces.

\begin{figure}[!t]
	\centering
	\begin{subfigure}[b]{0.45\textwidth}
		\centering
		\includegraphics[width=\textwidth]{ figs/error_p=500_rho=2.5_s=20.png}
		\caption{$d=500$}
		\label{fig:error_500_2.5}
	\end{subfigure}
	\hfill
	\begin{subfigure}[b]{0.45\textwidth}
		\centering
		\includegraphics[width=\textwidth]{ figs/error_p=1000_rho=2.5_s=20.png}
		\caption{$d=1000$}
		\label{fig:error_1000_2.5}
	\end{subfigure}
	\caption{Estimation errors of the parameters $( \btheta^*,\bbeta_1^*, \bbeta^*_2)$ with $s=20$, $\overline{L}=2.5$, and $d\in\{500, 1000\}$. The horizontal axis ``time'' represents the sample size $T$.}
	\label{fig:error}
\end{figure}

Moreover, we present the $\ell_2$ estimation error for the parameters $(\btheta^*, \bbeta_1^*, \bbeta_2^*)$  in Figure \ref{fig:error}. Due to the potential for the algorithm to interchange the two groups, we compute the $\ell_2$ estimation errors of $\bbeta_g^*$ and $\btheta^*$ as the minimum error between the estimated group and the alternative group, that is, $\min\left\{\norm{\hat\bbeta_g-\bbeta_g^*}_2, \norm{\hat\bbeta_{\{1, 2\} \setminus g}-\bbeta_g^*}_2\right\}$ and $ \min\left\{\norm{\hat\btheta-\btheta^*}_2, \norm{\hat\btheta+\btheta^*}_2\right\}$,
respectively. As illustrated in Figure \ref{fig:error}, the estimation errors consistently decrease with increasing sample size $T$, indicating that our algorithm effectively estimates the model parameters in high-dimensional settings, which leads to the reduction in regret. The estimation errors for $d = 1000$ are only slightly larger than those for $d = 500$, which aligns with our theoretical result that the estimation error scales with $\sqrt{\log d}$. 


\subsection{Real Data Analysis} \label{sec:real}

%\begin{figure}[!t]
%	\centering
%	\includegraphics[width=0.5\textwidth]{ figs/regret_real.png}
%	\caption{The average regret of different methods on the cash bonus dataset}
%	\label{fig:regret_real}
%\end{figure}


In this section, we illustrate the usefulness of our proposed method with application to the cash bonus dataset, originally presented by \cite{chen2022bcrlsp}. The dataset was collected from a mobile app, Taobao Special Offer Edition, which provided its users with a daily cash bonus that could be subtracted from the final payment at the time of purchase within 24 hours. The aim is to determine the optimal amount of cash bonus allocated to each user that leads to the highest payment.

Each observation in the dataset consists of customer features (user demographics and behavior information), a consecutive action variable $a_i$ (the amount of cash bonus) ranging from 0.25 to 1.95 with increment 0.01, and a continuous reward $y_i$ (the actual payment when the user redeems the cash bonus). To improve the computational efficiency, we compute the top 100 principal components of the customer features as the feature variable $\bz_{i}$. The contextual features $\bx_{i, a_{i}}$ are defined as $\bx_{i, a_{i}}=[a_i, a_i^2, \bz_i, a_i \bz_i, a_i^2\bz_i]$, incorporating quadratic terms of $a_i$ and their interaction terms with $\bz_i$. Furthermore, we divide the dataset into two groups based on the user's consumption level and remove this variable from the dataset to assume that it is unknown in practice. %where the high-consumption and low-consumption groups contain and 290,494 observations, respectively. 
Following model \eqref{eqn:lhlb}, we use the entire dataset to fit an $\ell_1$-regularized logistic regression model for group classification and two LASSO models for the two groups separately. The fitted parameters are viewed as the ground truth $(\btheta^*, \bbeta_1^*, \bbeta_2^*)$, which are further used to generate the unobserved rewards $y_{i, k}$ in model \eqref{eqn:lhlb} for all of the actions $k$. 

Similar to the simulation studies, we evaluate the performance of our proposed heterogeneous method by comparing its average strong and regular regrets with a ``single'' method that applies a single LASSO in each episode without considering the heterogeneous grouping. Additionally, we also implement a ``separate'' method that applies LASSO algorithms separately for the two groups, assuming the group assignment is known. The ``separate'' method indicates the optimal performance one can expect in the heterogeneous setting with unknown groups. 

\begin{figure}[!t]
	\centering
	\begin{subfigure}[b]{0.45\textwidth}
		\centering
		\includegraphics[width=\textwidth]{ figs/regret_real_g1.png}
		\caption{high-consumption group}
		\label{fig:real_g1}
	\end{subfigure}
	\hfill
	\begin{subfigure}[b]{0.45\textwidth}
		\centering
		\includegraphics[width=\textwidth]{ figs/regret_real_g2.png}
		\caption{low-consumption group}
		\label{fig:real_g2}
	\end{subfigure}
	\caption{Average strong and regular regrets of different methods on the cash bonus dataset}
	\label{fig:real}
\end{figure}


Figure \ref{fig:real} presents a comparative analysis of the average regrets across different methods for both groups, where we set the initial sample size $n_0$ to be 256. Our proposed heterogeneous method demonstrates superior performance compared to the single LASSO approach in both groups, though with differences in the magnitude and pattern of improvement. For the high-consumption group,  the ``single'' method shows consistently higher regret levels, while our heterogeneous method achieves significantly lower regret values. For the low-consumption group, while the improvement is less clear in early episodes, it becomes particularly evident as the sample size increases. 

The performance difference of the ``single'' method between the two groups reveals the heterogeneity in purchasing behaviors and transaction values. The higher regret level in the high-consumption group reflects larger deviations from optimal bonus allocation, likely due to their higher baseline spending and larger transaction amounts. %Conversely, the lower regret levels observed in the low-consumption group reflect smaller deviations from optimal bonus allocation, consistent with their lower transaction values. 
Since the ``single'' method fits only one LASSO model per episode, it effectively tracks only the behavior patterns of low-consumption users, resulting in more accurate predictions for this group but substantial errors for high-consumption users.

In contrast, our proposed heterogeneous method successfully captures the behavior patterns of both groups, verified by its comparable regret to the ``separate'' method that represents the performance of LASSO under known group assignments. This superior performance indicates that platforms can achieve more efficient bonus allocation strategies by incorporating latent group structures, potentially leading to improved user engagement and platform profitability across different user segments.

\section{Conclusion} \label{sec:conclusion}
This paper advances the field of online decision-making by addressing the critical challenge of latent heterogeneity in stochastic linear bandits. We introduce a novel framework that explicitly models unobserved customer characteristics affecting their responses to business actions, filling a significant gap in existing approaches primarily focusing on observable heterogeneity. Our methodology introduces an innovative algorithm that simultaneously learns both latent group memberships and group-specific reward functions, effectively handling the challenge of not having direct observations of group labels.

Our theoretical analysis reveals important insights about decision-making under latent heterogeneity. We establish that while the ``strong regret'' against an oracle with perfect group knowledge remains non-sub-linear due to inherent classification uncertainty, the ``regular regret'' against an oracle aware of only deterministic components achieves a minimax optimal rate in terms of $T$ and $d$. Importantly, we demonstrate that traditional bandit algorithms that ignore latent heterogeneity incur linear regret, highlighting the necessity of our approach.

Through empirical validation using data from a mobile commerce platform, we demonstrate the practical value of our framework. The results show that our approach effectively handles real-world scenarios where customer heterogeneity plays a crucial role, such as in personalized pricing, resource allocation, and inventory management. This work provides practitioners with theoretically grounded tools for making sequential decisions under latent heterogeneity, bridging the gap between theoretical understanding and practical implementation in business applications.

Several important directions remain for future research. First, extending our framework to nonlinear structured bandit models would enable applications in more complex decision scenarios. Developing online tests for latent heterogeneity and studying heterogeneous treatment effects in bandits would further enhance the practical utility of our approach. Moreover, our finding that the ``strong regret'' remains non-sub-linear due to inherent classification uncertainty suggests a fundamental limitation that cannot be addressed through algorithmic improvements alone. This points to the necessity of mechanism design approaches that could incentivize customers to reveal their latent group memberships, potentially opening a new avenue of research at the intersection of bandit algorithms and mechanism design.


%-------
%
%  bibliography
%

\clearpage
%\nocite{*} % include all bib's
\bibliographystyle{\mybibsty}
\bibliography{\mybib}

%-------
%
%  appendix
%
\clearpage
\appendix
\section{Proofs} \label{sec:proof}

%!TEX root = 0-main.tex

\section{Proof of Theorem \ref{thm:coeff-bound-hd}} \label{sec:proof-coeff-bound}

The EM algorithm is essentially an alternating maximization method, which alternatively optimizes between the identification of hidden labels $\{g_i\}$ and the estimation of parameter $\bgamma=(\btheta, \bbeta_1, \bbeta_2)$.
In the $\tau$-th episode, we utilize $N_{\tau-1} = n_02^{\tau - 1}$ samples from the previous episode. 
%We let $N=N_{\tau-1}$ and drop the index $\tau$ in this section since we are considering only one episode $\tau$. 
In the $(t+1)$-th iteration of the EM-algorithm, we have i.i.d samples in the set $\calI_{\tau-1}^{(t+1)}$ of size $n_{\tau}=N_{\tau-1}/t_{\tau, \max}$.

By the sub-Gaussianity of $\bx_i$, $\bz_i$, and $\epsilon_i$, there exist a constant $C$ such that \[\EE\left[(\bx_i^{\top}\bv)^{m}\right] \leq C^m \sigma_{x}^mm^{m/2}, \quad \EE[\epsilon_{i}^m]\leq C^m \sigma^mm^{m/2},
\text{ and } 
\PP\paren{\abs{\bz_i^{\top}\bv} \leq \mu} < 2e^{-\frac{\mu^2}{2\sigma^2_{z}}},\] 
for all non-negative integers $m$ and all $\bv$ such that $\norm{\bv}_2=1$, where $\sigma_x$, $\sigma_z$, and $\sigma$ are the sub-Gaussian parameters of $\bx_i$, $\bz_i$, and $\epsilon_i$, respectively. Let $\eta_x =\sigma_x/\sigma$. Moreover, we further define that $\delta_{\bgamma}^{(\tau, 0)}=\begin{cases}
	\delta_{0}, & \text{if } \tau=1;\\
	 \sqrt{\frac{s\log d \log n_0}{n_0}}, & \text{if } \tau \geq 2,\end{cases}$
where $\delta_{0}=c_1\min\big\{\xi, 1-\xi, \norm{\bbeta_1^*-\bbeta_2^*}_2\big\}$. We will show that, there exists a constant $\oC$ such that, by letting $\kappa<(2\oC^2)^{-1}$ and $\widetilde\kappa=\oC^2\kappa<1/2$, we have the following theorem, which is a more detailed version of Theorem $\ref{thm:coeff-bound-hd}$:

\begin{theorem}\label{thm:1-detailed}
	Suppose Assumptions \ref{A1}--\ref{A4} hold with the constants  $c_1$, $c_2$ that satisfy $c_2  \geq \max\left\{1/(2^{1/4}\sigma_x), 256MC^2\sigma_x\eta_x/\kappa, 384C^2\sigma_x^2M/\sqrt{\kappa}\right\}$ and $c_1 \leq \min\{1/4M, 2/\mu_0\}$, where $\mu_0$ is defined by $\mu_0= \sqrt{2}\sigma_z\sqrt{\log\paren{\frac{2144C^4\sigma_x^4c_2^2}{\kappa}}}$. Furthermore,  assume that $\sqrt{\frac{s^2\log d\log n_0}{n_0}} =o(1)$. Let $\left(\hat{\btheta}^{(\tau)},\hat{\bbeta}_{1}^{(\tau)},\hat{\bbeta}_{2}^{(\tau)}\right)=\left(\btheta^{(\tau,t_{\tau, \max})},\bbeta^{(\tau, t_{\tau, \max})}_{1},\bbeta^{(\tau, t_{\tau, \max})}_{2}\right)$ be the output of Algorithm \ref{alg:em} in the $\tau$-th episode. By letting the number of iterations $t_{\tau, \max} \asymp \log n_0$ for $\tau=1$ and $t_{\tau, \max} \asymp 1$  for $\tau \geq 2$ and choosing \[\lambda_{n_{\tau}}^{(t+1)}=\frac{2\oC(1-(2\widetilde\kappa)^{t+1})}{1-2\widetilde\kappa}\sqrt{\frac{\log d}{n_{\tau}}}+\frac{\oC\kappa(2\widetilde\kappa)^{t}}{\sqrt{s}}\delta_{\bgamma}^{(\tau, 0)}, \]
	we have
	\begin{equation}
		\norm{\hat{\bbeta}_{1} - \bbeta_{1}^*}_2 + \norm{\hat{\bbeta}_{2} - \bbeta_{2}^*}_2 + \norm{\hat{\btheta} - \btheta^*}_2
		\lesssim \frac{1}{\kappa(1-2\widetilde\kappa)}\sqrt{\frac{s\log d\log n_0}{N_{\tau-1}}}, 
	\end{equation}
	and
	\begin{equation}
		\norm{\hat{\bbeta}_{1} - \bbeta_{1}^*}_1 + \norm{\hat{\bbeta}_{2} - \bbeta_{2}^*}_1 + \norm{\hat{\btheta} - \btheta^*}_1
		\lesssim   \frac{1}{\kappa(1-2\widetilde\kappa)}\sqrt{\frac{s^2\log d\log n_0}{N_{\tau-1}}},
	\end{equation}
	with probability at least $1-\frac{c\log^3 n_0}{\max^2\{N_{\tau-1}, d\}}$ for some constant $c$.
\end{theorem}

We temporarily drop the index $\tau$ in $N_{\tau-1}$, $n_{\tau}$, $\cI_{\tau-1}^{(t)}$, $t_{\tau, \max}$, $\omega_i^{(\tau, t)}$, $\bgamma^{(\tau, t)}$, and $\lambda_{n_{\tau}}^{(t)}$ when we are considering a single episode $\tau$. 
 The overall objective function $Q_n$ in the M-Step is the sum of the single-observation objective functions, that is, 
\begin{equation}
    Q_n(\bgamma \cond \bgamma^{(t)})
    \; := \; 
    \sum_{i \in \calI^{(t+1)}}\EE\brackets{ \ell\paren{\bx_i, y_i, \bz_i, g_i; \bgamma} \cond \bx_i, y_i, \bz_i; \bgamma^{(t)}}.
\end{equation}
For clearness, we replace $i \in \calI^{(t+1)}$ with $\sum_{i=1}^{n}$ when there is no ambiguity.
Simple calculation yields that 
\begin{align}\label{eqn:Qn}
    Q_n(\bgamma \cond \bgamma^{(t)})
    & = 
   \frac{1}{n} \sum_{i=1}^{n} \left[\omega_i^{(t)}\cdot \paren{ \log p(\bz_i^\top\btheta)-
    \frac{(y_i - \bx_i^\top\bbeta_1)^2 }{2\sigma^2 }} \right.   \\
    &\quad +
    \left. (1- \omega_i^{(t)}) \cdot \paren{\log(1-p(\bz_i^\top\btheta)) -
    \frac{(y_i - \bx_i^\top\bbeta_2)^2}{2\sigma^2}} \right]  \nonumber \\
   & = -  \frac{1}{2n} \sum_{i=1}^{n} \omega_i^{(t)} \cdot  \frac{(y_i - \bx_i^\top\bbeta_1)^2 }{\sigma^2} - \frac{1}{2n} \sum_{i=1}^{n} (1- \omega_i^{(t)}) \cdot \frac{(y_i - \bx_i^\top\bbeta_2)^2 }{\sigma^2}  \nonumber   \\
   & \quad + \frac{1}{n} \sum_{i=1}^{n}\left[ \omega_i^{(t)} \cdot \log p(\bz_i^\top\btheta) + (1- \omega_i^{(t)}) \cdot \log(1-p(\bz_i^\top\btheta))\right]  \nonumber 
\end{align}
where $\omega_i^{(t)} = \omega_i(\bgamma^{(t)}) = \omega(\bx_i, y_i, \bz_i ; \bgamma^{(t)})$ is defined by
\[
\omega(\bx, y, \bz ; \bgamma) 
= \frac{p(\bz^\top\btheta) \cdot
	    \phi\paren{\frac{y - \bx^\top\bbeta_1}{\sigma}}}{p(\bz^\top\btheta) \cdot \phi\paren{\frac{y - \bx^\top\bbeta_1}{\sigma}} 
	    + \paren{1-p(\bz^\top\btheta)} \cdot \phi\paren{\frac{y - \bx^\top\bbeta_2}{\sigma}}}.
\]
Let 
\begin{equation} %\label{eqn:Qn123}
\begin{aligned}
    Q_{n1}(\bbeta_1 \cond \bgamma^{(t)}) & =  \frac{1}{2n} \sum_{i=1}^{n} \omega_i^{(t)}\cdot  \frac{(y_i - \bx_i^\top\bbeta_1)^2 }{\sigma^2}, \\
    Q_{n2}(\bbeta_2 \cond \bgamma^{(t)}) & =  \frac{1}{2n} \sum_{i=1}^{n} (1- \omega_i^{(t)}) \cdot \frac{(y_i - \bx_i^\top\bbeta_2)^2 }{\sigma^2}, \quad\text{and}\quad \\
    Q_{n3}(\btheta \cond \bgamma^{(t)}) & = - \frac{1}{n} \sum_{i=1}^{n} \paren{ \omega_i^{(t)} \cdot \log p(\bz_i^\top\btheta) + (1- \omega_i^{(t)}) \cdot \log(1-p(\bz_i^\top\btheta)) }.
\end{aligned}
\end{equation}
Then, in the $(t+1)$-th iteration, the M-step is implemented as 
\begin{align}
    \bbeta_1^{(t+1)} & := \underset{\bbeta_1}{\arg\min} \;  Q_{n1}\paren{\bbeta_1 \cond \bgamma^{(t)}}  
    + \lambda_n^{(t)} \norm{\bbeta_1}_1,  \label{eqn:beta-1-opt}\\
    \bbeta_2^{(t+1)} & := \underset{\bbeta_2}{\arg\min} \; Q_{n2}\paren{\bbeta_2 \cond \bgamma^{(t)}}  
    + \lambda_n^{(t)} \norm{\bbeta_2}_1,  \label{eqn:beta-2-opt} \\
    \btheta^{(t+1)} & := \underset{\btheta}{\arg\min} \; Q_{n3}\paren{\btheta \cond \bgamma^{(t)}} 
    + \lambda_n^{(t)} \norm{\btheta}_1.  \label{eqn:theta-opt}
\end{align}

%Given a suitable initialization, the proposed EM algorithm then proceeds by iterating between the E-step and the M-step, which is summarized in the Algorithm \ref{alg:em-bandit}.

To prove Theorem \ref{thm:1-detailed}, we first present Lemmas \ref{thm:lemma-expectation} and \ref{thm:lemma-sample} on iterative estimation bounds. Let $\omega_i^* = \omega_i(\bgamma^*) = \omega(\bx_i, y_i, \bz_i ; \bgamma^*)$.

\begin{lemma}[Population EM iterates]  
\label{thm:lemma-expectation}
Assume Assumptions (A1) and (A4) hold.  For any constant $\kappa>0$ and any constants $c_1, c_2$ satisfying the requirements in Theorem \ref{thm:1-detailed}, 
if $\norm{\bbeta_1^{(t)} - \bbeta_1^*}_2 + \norm{\bbeta_2^{(t)} - \bbeta_2^*}_2 + \norm{\btheta^{(t)} - \btheta^*}_2 \le c_1 \min\braces{\xi, (1-\xi), \norm{\bbeta_1^*-\bbeta_2^*}_2}$ and $\norm{\bbeta_1^*-\bbeta_2^*}_2 > c_2$, then we have
\begin{align*}
    \abs{ \EE[\omega_i^{(t)} ] 
        - \EE[\omega_i^*] }  
    & \le \kappa \cdot \paren{\norm{\bbeta_1^{(t)} - \bbeta_1^*}_2 + \norm{\bbeta_2^{(t)} - \bbeta_2^*}_2 + \norm{\btheta^{(t)} - \btheta^*}_2}, 
    \\
    \norm{ \EE[ \omega_i^{(t)} \bx_i ( \bx_i^\top \bbeta_1^* - y_i) ] 
        - \EE[ \omega_i^* \bx_i ( \bx_i^\top \bbeta_1^* - y_i) ] }_2  
    & \le \kappa \cdot \paren{\norm{\bbeta_1^{(t)} - \bbeta_1^*}_2 + \norm{\bbeta_2^{(t)} - \bbeta_2^*}_2 + \norm{\btheta^{(t)} - \btheta^*}_2}, 
    \\
    \norm{ \EE[ ( \omega_i^{(t)} - p(\bz_i^\top\btheta^{*}) ) \bz_i ] 
       - \EE[ (\omega_i^* - p(\bz_i^\top\btheta^{*}) ) \bz_i ] }_2
    & \le  \kappa \cdot \paren{\norm{\bbeta_1^{(t)} - \bbeta_1^*}_2 + \norm{\bbeta_2^{(t)} - \bbeta_2^*}_2 + \norm{\btheta^{(t)} - \btheta^*}_2}. 
\end{align*}
\end{lemma}
\noindent Proof of Lemma \ref{thm:lemma-expectation} is provided in Section \ref{sec:proof-lemma-expectation}. 

\begin{lemma}[Sample EM iterates]  \label{thm:lemma-sample}
Under the assumptions of Theorem \ref{thm:coeff-bound-hd}, suppose that $\bgamma^{(t)}$ is independent with $\{\bx_i, y_i, \bz_i\}$'s, then there exists some constant $C > 0$, such that, with probability at least $1 - \frac{4}{\max\{n, d\}^2}$,  
\begin{align*}
\norm{ \frac{1}{n}\sum_{i=1}^n \left[\omega_i^{(t)} \bx_i ( \bx_i^\top \bbeta_1^* - y_i) \right] 
    - \EE\left[ \omega_i^{(t)} \bx_i ( \bx_i^\top \bbeta_1^* - y_i) \right] }_\infty 
& \le  C \sqrt{ \frac{\log \max\{n, d\}}{n} },  \\
\norm{  \frac{1}{n}\sum_{i=1}^n \left[( \omega_i^{(t)} - p(\bz_i^\top\btheta^{*}) ) \bz_i \right]
    - \EE\left[ (\omega_i^{(t)} - p(\bz_i^\top\btheta^{*}) ) \bz_i \right] }_\infty
& \le C \sqrt{ \frac{\log \max\{n, d\}}{n} }.
\end{align*}
\end{lemma}
\noindent Proof of Lemma \ref{thm:lemma-sample} is provided in Section \ref{sec:proof-lemma-sample}. 

\medskip

\noindent In Steps 1 and 2 in the following proof, we use $\hat{\bbeta_1}$, $\hat{\bbeta_2}$, $\hat{\btheta}$, $\lambda$ to denote $\bbeta_1^{(t+1)}$, $\bbeta_2^{(t+1)}$,  $\btheta^{(t+1)}$, and $\lambda_n^{(t+1)}$ for simplicity. Also, we suppose that $\norm{\bbeta_1^{(t)} - \bbeta_1^*}_2 + \norm{\bbeta_2^{(t)} - \bbeta_2^*}_2 + \norm{\btheta^{(t)} - \btheta^*}_2 \le c_1 \min\braces{\xi, (1-\xi), \norm{\bbeta_1^*-\bbeta_2^*}_2}$, which is satisfied with $t=0$ and will be shown by induction in Step 3.

\noindent
\textbf{STEP 1. Sample EM iterative bounds for $\bbeta_1$ and $\bbeta_2$.} We make use of the definition of \eqref{eqn:beta-1-opt} and \eqref{eqn:beta-2-opt} and a decomposition of the main objective function $Q_{n1}$ as follows.
\begin{align}
Q_{n1}\paren{{\bbeta_1^*} \cond \bgamma^{(t)}} - Q_{n1}\paren{\hat\bbeta_1 \cond \bgamma^{(t)}}
& = \frac{1}{n}\angles{\sum_{i=1}^{n}\omega_i^{(t)}\paren{y_i-\bx_i^\top\bbeta_1^*} \bx_i,\; \hat\bbeta_1-\bbeta_1^*} \nonumber\\
& - \paren{\hat\bbeta_1-\bbeta_1^*}^\top 
\paren{\frac{1}{2n}\sum_{i=1}^n\omega_i^{(t)}\bx_i\bx_i^\top}
\paren{\hat\bbeta_1-\bbeta_1^*}. \label{eqn:Qn-decomp}
\end{align}
By \eqref{eqn:beta-1-opt}, we have
\begin{equation} \label{eqn:Qn-diff-lb-1}
\lambda\paren{\norm{\hat\bbeta_1}_1-\norm{\bbeta_1^*}_1} 
\le Q_{n1}\paren{{\bbeta_1^*} \cond \bgamma^{(t)}} - Q_{n1}\paren{\hat\bbeta_1 \cond \bgamma^{(t)}}.
\end{equation}

Combine \eqref{eqn:Qn-decomp} and \eqref{eqn:Qn-diff-lb-1}, we have
\begin{equation}
\begin{aligned}
&\quad  \paren{\hat\bbeta_1-\bbeta_1^*}^\top 
\paren{\frac{1}{2n}\sum_{i=1}^n\omega_i^{(t)}\bx_i^\top\bx_i}
\paren{\hat\bbeta_1-\bbeta_1^*} \\
&\le 
\frac{1}{n}\angles{\sum_{i=1}^{n}\omega_i^{(t)}\paren{y_i-\bx_i^\top\bbeta_1^*} \bx_i,\; \hat\bbeta_1-\bbeta_1^*} - \lambda\paren{\norm{\hat\bbeta_1}_1-\norm{\bbeta_1^*}_1}.
\end{aligned}
\end{equation}
Applying the decomposition (where we use  $\EE\brackets{\omega_i^*\paren{y_i-\bx_i^\top\bbeta_1^*} \bx_i}=0$, which is implied by the fact that $(\btheta, \bbeta_1^*, \bbeta_2^*)$ is the minimizer of the population likelihood)
\begin{align*}
&\quad \angles{\frac{1}{n}\sum_{i=1}^{n}\omega_i^{(t)}\paren{y_i-\bx_i^\top\bbeta_1^*} \bx_i,\; \hat\bbeta_1-\bbeta_1^*} \\
& \le \abs{\angles{\frac{1}{n}\sum_{i=1}^{n}\omega_i^{(t)}\paren{y_i-\bx_i^\top\bbeta_1^*} \bx_i - \EE\brackets{\omega_i^{(t)}\paren{y_i-\bx_i^\top\bbeta_1^*} \bx_i},\; \hat\bbeta_1-\bbeta_1^*}} \\
& ~~~ + \abs{\angles{\EE\brackets{\omega_i^{(t)}\paren{y_i-\bx_i^\top\bbeta_1^*} \bx_i} - \EE\brackets{\omega_i^*\paren{y_i-\bx_i^\top\bbeta_1^*} \bx_i},\; \hat\bbeta_1-\bbeta_1^*}} \nonumber
\end{align*} 
and Lemmas \ref{thm:lemma-expectation} and \ref{thm:lemma-sample}, we obtain, with probability at least $1-2/\max\{n, d\}^2$, 
\begin{equation} \label{eqn:Qn-diff-ub}
\begin{aligned}
& \angles{\frac{1}{n}\sum_{i=1}^{n}\omega_i^{(t)}\paren{y_i-\bx_i^\top\bbeta_1^*} \bx_i,\; \hat\bbeta_1-\bbeta_1^*} \\
& \le  
C \sqrt{\frac{\log \max\{n, d\} }{ n}}\cdot\norm{\hat\bbeta_1-\bbeta_1^*}_1 
+ \kappa\cdot\paren{\norm{\bbeta_1^{(t)} - \bbeta^*}_2 + \norm{\bbeta_2^{(t)} - \bbeta^*}_2 + \norm{\btheta^{(t)} - \btheta^*}_2}\cdot\norm{\hat\bbeta_1-\bbeta_1^*}_2. 
\end{aligned}
\end{equation}

\begin{lemma} \label{thm:diff-L1-Bound}
Let $S = {\rm supp}(\bbeta_1^*)$ and $s=|S|$. When \[\lambda\ge 3C\sqrt{\frac{\log \max\{d, n\}}{ n}} + \frac{\kappa}{\sqrt{s}}\cdot\paren{\norm{\bbeta_1^{(t)} - \bbeta^*}_2 + \norm{\bbeta_2^{(t)} - \bbeta^*}_2 + \norm{\btheta^{(t)} - \btheta^*}_2},\] we have
\begin{equation}
\norm{ \hat\bbeta_1 - \bbeta_1^*}_1 \leq 5\sqrt{s}\norm{ \hat\bbeta_1 - \bbeta_1^*}_2. 
\end{equation}
\end{lemma}
\begin{proof}{Proof of Lemma \ref{thm:diff-L1-Bound}}
Let $\bu = \hat\bbeta_1 - \bbeta_1^*$. Combining the inequality from the definition of each iterates
\begin{equation} 
\lambda\paren{\norm{\hat\bbeta_1}_1-\norm{\bbeta_1^*}_1} 
\le Q_{n1}\paren{{\bbeta_1^*} \cond \bgamma^{(t)}} - Q_{n1}\paren{\hat\bbeta_1 \cond \bgamma^{(t)}} 
\end{equation}
and the inequality that
\begin{equation*}
\norm{\hat\bbeta_1}_1-\norm{\bbeta_1^*}_1 
\ge \norm{\bbeta_1^* + \bu_{S^C}}_1 - \norm{\bu_{S}}_1-\norm{\bbeta_1^*}_1
= \norm{\bu_{S^C}}_1 - \norm{\bu_{S}}_1,
\end{equation*}
we obtain
\begin{equation} \label{eqn:Qn-diff-lb}
\lambda\paren{\norm{\bu_{S^C}}_1 - \norm{\bu_{S}}_1}
\le 
Q_{n1}\paren{{\bbeta_1^*} \cond \bgamma^{(t)}} - Q_{n1}\paren{\hat\bbeta_1 \cond \bgamma^{(t)}}.
\end{equation}
Combining \eqref{eqn:Qn-decomp}, \eqref{eqn:Qn-diff-ub} and \eqref{eqn:Qn-diff-lb}, we have 
\begin{align*}
\lambda\paren{\norm{\bu_{S^C}}_1 - \norm{\bu_{S}}_1}
& \le Q_{n1}\paren{{\bbeta_1^*} \cond \bgamma^{(t)}} - Q_{n1}\paren{\hat\bbeta_1 \cond \bgamma^{(t)}} \\
& \le \frac{1}{n}\angles{\sum_{i=1}^{n}\omega_i^{(t)}\paren{y_i-\bx_i^\top\bbeta_1^*} \bx_i,\; \hat\bbeta_1-\bbeta_1^*} \\
& \le 
C \sqrt{\frac{\log \max\{d, n\}} {n}}\cdot\norm{\bu}_1 \\
&\quad + \frac{\kappa}{\sqrt{s}}\cdot\paren{\norm{\bbeta_1^{(t)} - \bbeta^*}_2 + \norm{\bbeta_2^{(t)} - \bbeta^*}_2 + \norm{\btheta^{(t)} - \btheta^*}_2}\cdot\sqrt{s}\norm{\bu}_2. 
\end{align*} 
Let \[\lambda\ge 3C\sqrt{\frac{\log \max\{d, n\}}{ n}} + \frac{\kappa}{\sqrt{s}}\cdot\paren{\norm{\bbeta_1^{(t)} - \bbeta^*}_2 + \norm{\bbeta_2^{(t)} - \bbeta^*}_2 + \norm{\btheta^{(t)} - \btheta^*}_2},\] we have
% \begin{equation*}
% \norm{\bu_{S^C}}_1 - \norm{\bu_{S}}_1 \le 1/3 (\norm{\bu_{S^C}}_1 + \norm{\bu_{S}}_1) + \sqrt{s}\norm{\bu}_2. 
% \end{equation*}
\begin{equation*}
\norm{\bu_{S^C}}_1 \le 2\norm{\bu_{S}}_1 + 3/2\sqrt{s}\norm{\bu}_2 \le 4\sqrt{s}\norm{\bu}_2,
\end{equation*}
and hence $\norm{\bu} \leq \norm{\bu_{S}}_1+\norm{\bu_{S^c}}_1 \leq 5\sqrt{s}\norm{\bu}_2$. \hfill\qed
\end{proof}

Then, applying Lemma \ref{thm:diff-L1-Bound} to \eqref{eqn:Qn-diff-ub}, we have
\begin{equation}
\begin{aligned}
&\quad  \paren{\hat\bbeta_1-\bbeta_1^*}^\top 
\paren{\frac{1}{2n}\sum_{i=1}^n\omega_i^{(t)}\bx_i^\top\bx_i}
\paren{\hat\bbeta_1-\bbeta_1^*} \label{eqn:Q1n-2nd-mm-ub}\\
& \lesssim \sqrt{\frac{s\log \max\{d, n\}} {n}}\cdot\norm{\hat\bbeta_1-\bbeta_1^*}_2
+ \kappa\cdot\paren{\norm{\bbeta_1^{(t)} - \bbeta_1^*}_2 + \norm{\bbeta_2^{(t)} - \bbeta_2^*}_2 + \norm{\btheta^{(t)} - \btheta^*}_2}\cdot\norm{\hat\bbeta_1-\bbeta_1^*}_2  \\
&\quad + \lambda\sqrt{s}\norm{\hat\bbeta_1-\bbeta_1^*}_2. \nonumber
\end{aligned}
\end{equation}
Second, we establish a lower bound of the second-order term. 
Assumption (A1) ensures that $p(\bz_i^{\top}\btheta^*) \in (\xi, 1-\xi)$. By Hoeffding's inequality and Lemma \ref{thm:lemma-expectation}, 
%when $\norm{\bbeta_1^{(t)} - \bbeta^*}_2 + \norm{\bbeta_2^{(t)} - \bbeta^*}_2 + \norm{\btheta^{(t)} - \btheta^*}_2 \le ...$, 
we have, with probability at least $1-\frac{2}{\max\{d, n\}^2}$,  \[\abs{\frac{1}{n}\sum_{i=1}^{n}\omega_i^{(t)}-\EE[\omega_i^*]} \leq \sqrt{\frac{\log \max \{d, n\}}{n}} + \kappa \paren{\norm{\bbeta_1^{(t)} - \bbeta_1^*}_2 + \norm{\bbeta_2^{(t)} - \bbeta_2^*}_2 + \norm{\btheta^{(t)} - \btheta^*}_2}.\]
Since $\paren{\norm{\bbeta_1^{(t)} - \bbeta_1^*}_2 + \norm{\bbeta_2^{(t)} - \bbeta_2^*}_2 + \norm{\btheta^{(t)} - \btheta^*}_2} \leq c_1 \min\{\xi, 1-\xi\}$, by taking $\kappa$ sufficiently small, we have $\abs{\frac{1}{n}\sum_{i=1}^{n}\omega_i^{(t)}-\EE[\omega_i^*]} \leq \frac{1}{2} \min\{\xi, 1-\xi\}$ with probability at least $1-\frac{2}{\max\{d, n\}^2}$. Since $\EE[\omega_i^*] = \EE[p(\bz_i^{\top}\btheta^*)] \geq \xi$,  we have that $\frac{1}{n}\sum_{i=1}^{n}\omega_i^{(t)} \geq \frac{\xi}{2}$, and thus $n_{a}:=\sum_{i=1}^n\bbone\braces{\omega_i^{(t)} \ge \frac{\xi}{4}} \ge \frac{\xi}{4}n$. (Otherwise, $\frac{1}{n}\sum_{i=1}^{n}\omega_i^{(t)} \leq \frac{\xi}{4}+ \frac{\xi}{4}(1- \frac{\xi}{4}) < \xi/2$.) %, with probability approaching one. 
As a result, 
\begin{align}
& \paren{\hat\bbeta_1-\bbeta_1^*}^\top 
\paren{\frac{1}{2n}\sum_{i=1}^n\omega_i^{(t)}\bx_i\bx_i^\top}
\paren{\hat\bbeta_1-\bbeta_1^*} 
%= \frac{1}{2n}\sum_{i=1}^n\omega_i^{(t)}\paren{\bx_i^\top\paren{\hat\bbeta_1-\bbeta_1^*}}^2
\nonumber\\
& \ge \frac{\xi}{8}(\hat\bbeta_1-\bbeta_1^*)^{\top}\cdot \frac{1}{n}\sum_{\omega_i^{(t)} \geq \xi/4} \paren{\bx_i\bx_i^\top} (\hat\bbeta_1-\bbeta_1^*)\nonumber\\
& \ge \frac{\xi^2}{32M} \norm{\hat\bbeta_1-\bbeta_1^*}_2^2 +  \frac{\xi}{8}(\hat\bbeta_1-\bbeta_1^*)^{\top}\cdot \frac{1}{n}\left[\sum_{\omega_i^{(t)} \geq \xi/4} \big(\bx_i\bx_i^\top-\EE[\bx_i\bx_i^{\top}]\big)\right] (\hat\bbeta_1-\bbeta_1^*) \nonumber\\
& \ge \frac{\xi^2}{64M} \norm{\hat\bbeta_1-\bbeta_1^*}_2^2, \label{eqn:Q1n-2nd-mm-lb}
\end{align}
where we use the standard result that $\frac{1}{n_a}\left\|\sum_{\omega_i^{(t)} \geq \xi/4} \big(\bx_i\bx_i^\top-\EE[\bx_i\bx_i^{\top}]\big)\right\|_{\max} \lesssim \sqrt{\frac{\log \max\{n_a, d\}}{n_a}}$ with probability at least $1-2\max\{n_a, d\}^{-2}$, the result in Lemma  \ref{thm:diff-L1-Bound}, and the assumption that $s\sqrt{\frac{\log \max \{n, d\}}{n}}=o(1)$.
Combining \eqref{eqn:Q1n-2nd-mm-ub} and \eqref{eqn:Q1n-2nd-mm-lb}, we have with probability at least $1-c_0\max\{n, d\}^{-2}$, 
\begin{equation} \label{eqn:beta1-one-iter}
\norm{\hat\bbeta_1-\bbeta_1^*}_2 
\lesssim    
\sqrt{s \log  \max\{d, n\} / n}
+ \kappa\cdot\paren{\norm{\bbeta_1^{(t)} - \bbeta_1^*}_2 + \norm{\bbeta_2^{(t)} - \bbeta_2^*}_2 + \norm{\btheta^{(t)} - \btheta^*}_2}
+ \lambda\sqrt{s},
\end{equation}
for some constant $c_0>0$.
Similarly, we obtain that with with probability at least $1-c_0\max\{n, d\}^{-2}$,
\begin{equation}\label{eqn:beta2-one-iter}
\norm{\hat\bbeta_2-\bbeta_2^*}_2
\lesssim \sqrt{s \log  \max\{d, n\} / n}
+ \kappa\cdot\paren{\norm{\bbeta_1^{(t)} - \bbeta_1^*}_2 + \norm{\bbeta_2^{(t)} - \bbeta_2^*}_2 + \norm{\btheta^{(t)} - \btheta^*}_2}
+ \lambda\sqrt{s}.
\end{equation}

\medskip
\noindent
\textbf{STEP 2. Sample EM iterative bounds for $\btheta$.} 
We make use of the definition of \eqref{eqn:theta-opt}. 
Let $\bu = \hat\btheta$ and $S = {\rm supp}(\btheta^*)$.
Firstly, by the definition of estimator, we have that 
\begin{equation} \label{eqn:Qn3-diff-lb}
\lambda\paren{\norm{\hat\btheta}_1 - \norm{\btheta^*}_1} 
\le 
Q_{n3}\paren{\btheta^* \cond \bgamma^{(t)}} - Q_{n3}\paren{\hat\btheta \cond \bgamma^{(t)}} 
\end{equation}
In addition, we have
\begin{align}
Q_{n3}\paren{\hat\btheta \cond \bgamma^{(t)}}
- Q_{n3}\paren{{\btheta^*} \cond \bgamma^{(t)}}
& = 
- \frac{1}{n} \sum_{i=1}^{n} \paren{ \omega_i^{(t)} \cdot \log p(\bz_i^\top\hat\btheta) + (1- \omega_i^{(t)}) \cdot \log(1-p(\bz_i^\top\hat\btheta)) } \nonumber\\
& + \frac{1}{n} \sum_{i=1}^{n} \paren{ \omega_i^{(t)} \cdot \log p(\bz_i^\top\btheta^*) + (1- \omega_i^{(t)}) \cdot \log(1-p(\bz_i^\top\btheta^*)) } \nonumber\\
& = \angles{-\frac{1}{n} \sum_{i=1}^{n} \paren{\omega_i^{(t)}- p(\bz_i^\top\btheta^*)}\bz_i, \; (\hat\btheta-\btheta^*) } \nonumber\\
& + (\hat\btheta-\btheta^*)^\top \paren{\frac{1}{n} \sum_{i=1}^{n}p(\bz_i^\top\btheta')(1-p(\bz_i^\top\btheta'))\bz_i\bz_i^\top} (\hat\btheta-\btheta^*), \label{eqn:Qn3-decomp}
\end{align}
for some $\btheta'$ between $\hat\btheta$ and $\btheta^*$. 
Thus, we have
\begin{align*}
&\quad (\hat\btheta-\btheta^*)^\top \paren{\frac{1}{n} \sum_{i=1}^{n}p(\bz_i^\top\btheta')(1-p(\bz_i^\top\btheta'))\bz_i\bz_i^\top} (\hat\btheta-\btheta^*)\\
& \le \abs{\angles{-\frac{1}{n} \sum_{i=1}^{n} \paren{\omega_i^{(t)}- p(\bz_i^\top\btheta^*)}\bz_i, \; (\hat\btheta-\btheta^*) }} + \lambda \paren{\norm{\btheta^*}_1 - \norm{\hat\btheta}_1}  \\
& \le \abs{\angles{\frac{1}{n} \sum_{i=1}^{n} (\omega_i^{(t)}- p(\bz_i^\top\btheta^*))\bz_i - \EE\brackets{(\omega_i^{(t)}- p(\bz_i^\top\btheta^*))\bz_i}, \; (\hat\btheta-\btheta^*) }} \\
& + \abs{\angles{\EE\brackets{(\omega_i^{(t)}- p(\bz_i^\top\btheta^*))\bz_i} - \EE\brackets{(\omega_i^*- p(\bz_i^\top\btheta^*))\bz_i}, \; (\hat\btheta-\btheta^*) }}  + \lambda \paren{\norm{\btheta^*}_1 - \norm{\hat\btheta}_1} . 
\end{align*}
Applying Lemmas \ref{thm:lemma-expectation} and \ref{thm:lemma-sample}, we have
\begin{align*}
&  \abs{\angles{-\frac{1}{n} \sum_{i=1}^{n} \paren{\omega_i^{(t)}- p(\bz_i^\top\btheta^*)}\bz_i, \; (\hat\btheta-\btheta^*) }}  \\
& \le 
C \sqrt{\log \max\{d,n\} / n}\cdot\norm{\hat\btheta-\btheta^*}_1 
+ \kappa\cdot\paren{\norm{\bbeta_1^{(t)} - \bbeta^*}_2 + \norm{\bbeta_2^{(t)} - \bbeta^*}_2 + \norm{\btheta^{(t)} - \btheta^*}_2}\cdot\norm{\hat\btheta-\btheta^*}_2.
\end{align*}
Similar to Lemma \ref{thm:diff-L1-Bound}, by taking $\lambda > 3C\sqrt{\frac{\log \max\{d,n\}}{ n}} + \frac{\kappa}{\sqrt{s}}\cdot\paren{\norm{\bbeta_1^{(t)} - \bbeta^*}_2 + \norm{\bbeta_2^{(t)} - \bbeta^*}_2 + \norm{\btheta^{(t)} - \btheta^*}_2}$, it holds that
\begin{equation}\label{eqn:Q3n-2nd-mm-upper}
	\begin{aligned}
		& (\hat\btheta-\btheta^*)^\top \paren{\frac{1}{n} \sum_{i=1}^{n}p(\bz_i^\top\btheta')(1-p(\bz_i^\top\btheta'))\bz_i\bz_i^\top} (\hat\btheta-\btheta^*)\\
		& \le 
		\angles{-\frac{1}{n} \sum_{i=1}^{n} \paren{\omega_i^{(t)}-\log p(\bz_i^\top\btheta^*)}\bz_i, \; (\hat\btheta-\btheta^*) } + \lambda \paren{\norm{\hat\btheta}_1 - \norm{\btheta^*}_1} \\
		& \le 
		C \sqrt{\frac{s \log \max\{d,n\} }{ n}}\cdot\norm{\hat\btheta-\btheta^*}_2 
		+ \kappa\cdot\paren{\norm{\bbeta_1^{(t)} - \bbeta^*}_2 + \norm{\bbeta_2^{(t)} - \bbeta^*}_2 + \norm{\btheta^{(t)} - \btheta^*}_2}\cdot\norm{\hat\btheta-\btheta^*}_2 \\
		&\quad + \sqrt{s}\lambda \norm{\hat\btheta - \btheta^*}_2.
	\end{aligned}
\end{equation}

At the same time, similar to Step 1, we have $\frac{1}{n}\sum_{i=1}^{n}p(\bz_i^\top\btheta')(1-p(\bz_i^\top\btheta'))\ge\xi^2/8$ for any $\btheta'$ between $\hat\btheta$ and $\btheta^*$, with probability at least $1-c_0\max\{n, d\}^{-2}$, and hence
\[\sum_{i=1}^{n}\bbone\braces{(\bz_i^\top\btheta')(1-p(\bz_i^\top\btheta')) \geq \frac{\xi^2}{16}} \geq \frac{\xi^2}{4}n.\]
Then, we have 
\begin{equation} \label{eqn:Q3n-2nd-mm-lower}
(\hat\btheta-\btheta^*)^\top \paren{\frac{1}{n} \sum_{i=1}^{n}p(\bz_i^\top\btheta')(1-p(\bz_i^\top\btheta'))\bz_i\bz_i^\top} (\hat\btheta-\btheta^*)
\ge (\xi^4/64M) \norm{\hat\btheta-\btheta^*}_2^2.
\end{equation}
Combining \eqref{eqn:Q3n-2nd-mm-upper} and \eqref{eqn:Q3n-2nd-mm-lower}, we have with probability at least $1-c_0\max\{n, d\}^{-2}$, 
\begin{equation}\label{eqn:theta-one-iter}
\norm{\hat\btheta-\btheta^*}_2^2 
\lesssim    
 \sqrt{s \log \max\{d,n\} / n}
+ \kappa\cdot\paren{\norm{\bbeta_1^{(t)} - \bbeta_1^*}_2 + \norm{\bbeta_2^{(t)} - \bbeta_2^*}_2 + \norm{\btheta^{(t)} - \btheta^*}_2}
+ 
\lambda\sqrt{s}.
\end{equation}

\medskip
\noindent
\textbf{STEP 3. Proof by induction.}
Combining \eqref{eqn:beta1-one-iter},  \eqref{eqn:beta2-one-iter}, and \eqref{eqn:theta-one-iter}, we have that, with probability at least $1-c\max\{n, d\}^{-2}$
\begin{equation}\label{eq:cgamma}
	\begin{aligned}
		&\norm{\bbeta^{(t+1)}_1 - \bbeta_1^*}_2 + \norm{\bbeta^{(t+1)}_2 - \bbeta_2^*}_2 + \norm{\btheta^{(t+1)} - \btheta^*}_2 \\
		&\leq C_{\gamma} \brackets{
			\sqrt{s \log \max\{d,n\} / n} 
			+ \kappa\cdot\paren{\norm{\bbeta_1^{(t)} - \bbeta_1^*}_2 + \norm{\bbeta_2^{(t)} - \bbeta_2^*}_2 + \norm{\btheta^{(t)} - \btheta^*}_2}
			+ 
			\lambda_n^{(t+1)}\sqrt{s}}, 
	\end{aligned}
\end{equation}

when
\begin{equation*}
\lambda^{(t+1)}_n \ge C_\lambda \sqrt{\log \max\{d,n\} / n} + \kappa/\sqrt{s}\cdot\paren{\norm{\bbeta_1^{(t)} - \bbeta_1^*}_2 + \norm{\bbeta_2^{(t)} - \bbeta_2^*}_2 + \norm{\btheta^{(t)} - \btheta^*}_2},
\end{equation*}
for some absolute constants $c$, $C_{\gamma}$ and $C_{\lambda}$. Let $\oC: = \max\braces{C_{\lambda}, C_{\gamma}, 1}$, choose $\kappa<(2\oC^2)^{-1}$,  let $\widetilde\kappa:=\oC^2\kappa<1/2$, and define
\[\delta_{\bgamma}^{(t)}:=  \norm{\bbeta_1^{(t)} - \bbeta_1^*}_2 + \norm{\bbeta_2^{(t)} - \bbeta_2^*}_2 + \norm{\btheta^{(t)} - \btheta^*}_2.\]

We will show by induction that, by choosing 
\[\lambda_n^{(t+1)}=\frac{2\oC(1-(2\widetilde\kappa)^{t+1})}{1-2\widetilde\kappa}\sqrt{\frac{\log \max\{d,n\}}{n}}+\frac{\oC\kappa(2\widetilde\kappa)^{t}}{\sqrt{s}}\delta_{\bgamma}^{(0)}, \]
it holds that
\[\delta_{\bgamma}^{(t)} \leq \frac{2(1-(2\widetilde\kappa)^{t+1})}{\kappa(1-2\widetilde\kappa)}\sqrt{\frac{s\log \max\{d,n\}}{n}}+(2\widetilde\kappa)^t\delta_{\bgamma}^{(0)}, \]
and 
\[\lambda_n^{(t+1)} \geq C_\lambda\sqrt{\frac{\log \max\{d,n\}}{n}} + \frac{\kappa}{\sqrt{s}}\delta_{\bgamma}^{(t)}.\]

The case $t=0$ is trivial. Assume that the above two inequalities are true for $t$. Consider $t+1$. By \eqref{eq:cgamma}, we have
\begin{align*}
	\delta_{\bgamma}^{(t+1)} 
	&\leq \oC \brackets{1+\frac{2(1-(2\widetilde\kappa)^{t+1})}{1-2\widetilde\kappa}+\frac{2\oC(1-(2\widetilde\kappa)^{t+1})}{1-2\widetilde\kappa}}\sqrt{\frac{s\log \max\{d,n\}}{n}} +\paren{\oC\kappa+\oC^2\kappa}(2\widetilde{\kappa})^t\delta_{\bgamma}^{(0)}\\
	& \leq \frac{1}{\kappa} \brackets{\oC\kappa+\frac{2\oC\kappa(1-(2\widetilde\kappa)^{t+1})}{1-2\widetilde\kappa}+\frac{2\oC^2\kappa(1-(2\widetilde\kappa)^{t+1})}{1-2\widetilde\kappa}}\sqrt{\frac{s\log \max\{d,n\}}{n}}+(2\widetilde{\kappa})^{t+1}\delta_{\bgamma}^{(0)}\\
	& \leq \frac{1}{\kappa} \brackets{2+\frac{2\widetilde\kappa(1-(2\widetilde\kappa)^{t+1})}{1-2\widetilde\kappa}+\frac{2\widetilde\kappa(1-(2\widetilde\kappa)^{t+1})}{1-2\widetilde\kappa}}\sqrt{\frac{s\log \max\{d,n\}}{n}}+(2\widetilde{\kappa})^{t+1}\delta_{\bgamma}^{(0)}\\
	&\leq\frac{2(1-(2\widetilde\kappa)^{t+2})}{\kappa(1-2\widetilde\kappa)}\sqrt{\frac{s\log \max\{d,n\}}{n}}+(2\widetilde{\kappa})^{t+1}\delta_{\bgamma}^{(0)}.
\end{align*}
Furthermore,
\begin{align*}
	&\quad C_\lambda\sqrt{\frac{\log \max\{d,n\}}{n}} + \frac{\kappa}{\sqrt{s}}\delta_{\bgamma}^{(t+1)}\\
	& \leq  \brackets{\oC+\oC\kappa+\frac{2\oC\kappa(1-(2\widetilde\kappa)^{t+1})}{1-2\widetilde\kappa}+\frac{2\oC^2\kappa(1-(2\widetilde\kappa)^{t+1})}{1-2\widetilde\kappa}}\sqrt{\frac{\log \max\{d,n\}}{n}}+\frac{\kappa}{\sqrt{s}}(2\widetilde{\kappa})^{t+1}\delta_{\bgamma}^{(0)}\\
	& \leq \brackets{2\oC+\frac{2\oC^3\kappa(1-(2\widetilde\kappa)^{t+1})}{1-2\widetilde\kappa}+\frac{2\oC^3\kappa(1-(2\widetilde\kappa)^{t+1})}{1-2\widetilde\kappa}}\sqrt{\frac{\log \max\{d,n\}}{n}}+\frac{\kappa}{\sqrt{s}}(2\widetilde{\kappa})^{t+1}\delta_{\bgamma}^{(0)}	\leq\lambda_{n}^{(t+2)}.
\end{align*}
 
Therefore, we have shown that
\begin{equation*}
	\begin{aligned}
&\quad \norm{\bbeta_1^{(t)} - \bbeta_1^*}_2 + \norm{\bbeta_2^{(t)} - \bbeta_2^*}_2 + \norm{\btheta^{(t)} - \btheta^*}_2 \\
& \leq
(2\widetilde\kappa)^t\cdot\paren{\norm{\bbeta_1^{(0)} - \bbeta_1^*}_2 + \norm{\bbeta_2^{(0)} - \bbeta_2^*}_2 + \norm{\btheta^{(0)} - \btheta^*}_2}
+ \frac{2}{\kappa(1-2\widetilde\kappa)}\sqrt{\frac{s\log \max\{d,n\}}{n}}.
	\end{aligned}
\end{equation*}

 Since we focus on the high-dimensional setting where $\log N_{\tau-1}  \lesssim \log d$ for all $\tau$, we replace $\log\max\{d, n\}$ with  $\log d$ from now on. Recall that we use sample splitting $n_{\tau}=N_{\tau-1}/t_{\tau, \max}$ to make sure the solutions obtained in each iteration are independent with each other (which satisfies the assumption of Lemma \ref{thm:lemma-sample}). When $\tau=1$, we have $N_0=n_0$ and  $\delta_{\gamma}^{(0)}\leq \delta_{0}$. Hence,  by taking $t_{1, \max} \asymp \frac{1}{2\log(1/2\tilde{\kappa})}\log (n_0)$, we obtain 
\[
\norm{\hat{\bbeta}_1^{(1)} - \bbeta_1^*}_2 + \norm{\hbbeta_2^{(1)} - \bbeta_2^*}_2 + \norm{\hbtheta^{(1)} - \btheta^*}_2  \lesssim \sqrt{\frac{s\log d \log n_0}{n_0}},
\]
with probability at least $1-ct_{1,\max}^3 / \max\{n_0, d\}^2$.
For $\tau=2$, the initials $\bgamma^{(2, 0)}=\hat{\bgamma}^{(1)}$, and thus $\delta_0^{(2, 0)}\lesssim \sqrt{\frac{s \log d \log n_0}{n_0}}$. By taking $t_{2, \max} \asymp \log(\sqrt{2})/\log(1/(2\tilde{\kappa}))$, which is a constant, we obtain
 \[
 \norm{\hat{\bbeta}_1^{(2)} - \bbeta_1^*}_2 + \norm{\hbbeta_2^{(2)} - \bbeta_2^*}_2 + \norm{\hbtheta^{(2)} - \btheta^*}_2  \lesssim \sqrt{\frac{s\log d \log n_0}{2n_0}} = \sqrt{\frac{s\log d \log n_0}{N_1}}.
 \]
 The same argument shows that, for all $\tau \geq 2$, by taking $t_{\tau, \max} \asymp \log(\sqrt{2})/\log(1/(2\tilde{\kappa}))$,
  \[
 \norm{\hat{\bbeta}_1^{(\tau)} - \bbeta_1^*}_2 + \norm{\hbbeta_2^{(\tau)} - \bbeta_2^*}_2 + \norm{\hbtheta^{(\tau)} - \btheta^*}_2  \lesssim \sqrt{\frac{s\log d \log n_0}{N_{\tau-1}}},
 \]
 with probability at least $1-c t_{\tau, \max}^3 / \max\{N_{\tau-1}, d\}^2$.
%\begin{equation*}
%\begin{aligned}
%&\quad 	\norm{\bbeta_1^{(t_{\max})} - \bbeta_1^*}_2 + \norm{\bbeta_2^{(t_{\max})} - \bbeta_2^*}_2 + \norm{\btheta^{(t_{\max})} - \btheta^*}_2 \\
%&	\leq
%	(2\widetilde\kappa)^{t_{\max}}\cdot\paren{\norm{\bbeta_1^{(0)} - \bbeta_1^*}_2 + \norm{\bbeta_2^{(0)} - \bbeta_2^*}_2 + \norm{\btheta^{(0)} - \btheta^*}_2}
%	+ \frac{2}{\kappa(1-2\widetilde\kappa)}\sqrt{\frac{s\log \max\{d,N_{\tau-1}\}}{N_{\tau-1}/t_{\max}}},
%	\end{aligned}
%\end{equation*}
%where $N_{\tau-1}$ is the sample size of the $\tau$-th episode. 
%By taking the number of iteration $t_{\max} \asymp \log N_{\tau-1}$ , it holds that 
%\[	(2\widetilde\kappa)^{t_{\max}}\cdot\paren{\norm{\bbeta_1^{(0)} - \bbeta_1^*}_2+\norm{\bbeta_2^{(0)} - \bbeta_2^*}_2+\norm{\btheta^{(0)} - \btheta^*}_2} \lesssim \sqrt{\frac{s\log d \log N_{\tau-1}}{N_{\tau-1}}}, \]
%and thus we obtain that
%\[
%\norm{\bbeta_1^{(t_{\max})} - \bbeta_1^*}_2 + \norm{\bbeta_2^{(t_{\max})} - \bbeta_2^*}_2 + \norm{\btheta^{(t_{\max})} - \btheta^*}_2
%\lesssim
%\sqrt{\frac{s\log \max\{d,N_{\tau-1}\} \log N_{\tau-1}}{N_{\tau-1}}},
%\]
% with probability at least $1-c\log^3 N_{\tau-1} \max\{N_{\tau-1}, d\}^{-2}$ for some constant $c$.
  By Lemma \ref{thm:diff-L1-Bound}, we also have that
$$
 	\norm{\hbbeta_1^{(\tau)} - \bbeta_1^*}_1 + \norm{\hbbeta_2^{(\tau)} - \bbeta_2^*}_1 + \norm{\hbtheta^{(\tau)} - \btheta^*}_1
 	\lesssim
 	\sqrt{\frac{s^2\log d \log n_0}{N_{\tau-1}}},
$$ which concludes the proof of Theorem \ref{thm:1-detailed}.
% Finally, since we focus on the high-dimensional setting where $\log N_{\tau-1}  \lesssim \log d$ for all $\tau$, we have $\log\max\{d, N_{\tau-1}\} \lesssim \log d$, 





%\input{9-proof-misclustering.tex}

%!TEX root = 0-main.tex

\section{Proof for the Regret Results}
\label{sec:proof-regret}

\subsection{Proof for the Excess Misclassification Rate}\label{sec:proof-misclustering}

\begin{proof}{Proof of Theorem \ref{thm:miss-class-rate}}
%For ease of presentation, we denote $n = N_{\tau-1}$. 
The misclassification error for $\btheta^*$ can be expressed by
\begin{align*}
	R(\btheta^*) & = \EE\brackets{\EE\big[\bbone(g_i^* \neq G_{\btheta^*}(\bz_i)) \cond \bz_i\big]} \\
	&=\EE\brackets{\EE\big[\bbone(g_i^* =1, \bz_i^{\top}\btheta^*\leq0) \cond \bz_i\big]+\EE\big[\bbone(g_i^* =2, \bz_i^{\top}\btheta^*>0) \cond \bz_i\big]} \\
	&=\EE\brackets{\bbone(\bz_i^{\top}\btheta^* \leq 0)p(\bz_i^{\top}\btheta^*)+\bbone(\bz_i^{\top}\btheta^* > 0)(1-p(\bz_i^{\top}\btheta^*))} \\
	& = \EE \Big[\min\braces{(1 - p(\bz_i^\top\btheta^*), p(\bz_i^\top\btheta^*)}\Big],
\end{align*}
and 
\begin{align*}
	R\big(\hat\btheta\big) & = \EE\brackets{\EE\Big[\bbone(g_i^* \neq G_{\hat\btheta}(\bz_i)) \cond \bz_i\Big]}\\
	&=\EE\Big[\EE\big[\bbone(g_i^* =1, \bz_i^{\top}\btheta^*\leq0, \bz_i^{\top}\hat{\btheta} \leq 0) \cond \bz_i\big]+\EE\big[\bbone(g_i^* =1, \bz_i^{\top}\btheta^*>0, \bz_i^{\top}\hat{\btheta} \leq 0) \cond \bz_i\big]\\
	&\quad\quad +\EE\big[\bbone(g_i^* =2, \bz_i^{\top}\btheta^*\leq0, \bz_i^{\top}\hat{\btheta} >0) \cond \bz_i\big]+\EE\big[\bbone(g_i^* =2, \bz_i^{\top}\btheta^*>0, \bz_i^{\top}\hat{\btheta} > 0) \cond \bz_i\big]\Big] . %=\int_{\calZ}\sum_{k=1,2}\bbone(G_{\hat\btheta}(\bz_i) \ne k,  g_i^*=k)\PP(g^*_i=k\cond\bz_i) \mu(d\bz_i),
\end{align*}
Note that
\[\EE\big[\bbone(g_i^* =1, \bz_i^{\top}\btheta^*\leq0, \bz_i^{\top}\hat{\btheta} \leq 0) \cond \bz_i\big] \leq \EE\big[\bbone(g_i^* =1, \bz_i^{\top}\btheta^*\leq0) \cond \bz_i\big] \leq R(\btheta^*),\]
\begin{align*}
	\EE\big[\bbone(g_i^* =1, \bz_i^{\top}\btheta^*>0, \bz_i^{\top}\hat{\btheta} \leq 0) \cond \bz_i\big] & \leq \EE\big[\bbone\left(\abs{\bz_i^{\top}(\hat\btheta-\btheta^*)}\geq \bz_i^{\top}\btheta^*>0\right)p(\bz_i^{\top}\btheta^*) \cond \bz_i\big] \\
	&\leq R(\btheta^*) + \EE\brackets{\bbone\left(\abs{\bz_i^{\top}(\hat\btheta-\btheta^*)}\geq \bz_i^{\top}\btheta^*>0\right)(2p(\bz_i^{\top}\btheta^*)-1)}\\
	&\leq R(\btheta^*) + \EE\brackets{2p\paren{\abs{\bz_i^{\top}(\hat\btheta-\btheta^*)}}-1},
\end{align*}
\begin{align*}
	\EE\big[\bbone(g_i^* =2, \bz_i^{\top}\btheta^*\leq0, \bz_i^{\top}\hat{\btheta} > 0) \cond \bz_i\big] &\leq \EE\big[\bbone\left(\abs{\bz_i^{\top}(\hat\btheta-\btheta^*)} \leq\bz_i^{\top}\btheta^*\leq0\right)(1-p(\bz_i^{\top}\btheta^*)) \cond \bz_i\big]\\
	&\leq R(\btheta^*) + \EE\brackets{\bbone\left(\abs{\bz_i^{\top}(\hat\btheta-\btheta^*)}\leq \bz_i^{\top}\btheta^*<0\right)(1-2p(\bz_i^{\top}\btheta^*))}\\
	&\leq R(\btheta^*) + \EE\brackets{1-2p\paren{\abs{\bz_i^{\top}(\hat\btheta-\btheta^*)}}},
\end{align*}
\[\EE\big[\bbone(g_i^* =2, \bz_i^{\top}\btheta^*>0, \bz_i^{\top}\hat{\btheta} > 0) \cond \bz_i\big] \leq \EE\big[\bbone(g_i^* =2, \bz_i^{\top}\btheta^*>0) \cond \bz_i\big] \leq R(\btheta^*),\]
and only one of the above four indicator functions equals one. Therefore, we obtain that
\[R\big(\hat\btheta\big)-R(\btheta^*) \leq \EE\brackets{\abs{2p\left(\abs{\bz_i^{\top}(\hat\btheta-\btheta^*)}\right)-1}} \leq 2\EE\brackets{p\left(\abs{\bz_i^{\top}(\hat\btheta-\btheta^*)}\right)-p(0)} \leq \frac{1}{2}\EE\brackets{\abs{\bz_i^{\top}(\hat\btheta-\btheta^*)}}.\]
Furthermore, since $\lambda_{\max}(\EE[\bz_i\bz_i^{\top}]) \leq M$, we finally obtain
\[R\big(\hat\btheta\big)-R(\btheta^*) \lesssim \EE\brackets{\abs{\bz_i^{\top}(\hat\btheta-\btheta^*)}} \leq \sqrt{\EE\brackets{(\hat\btheta-\btheta^*)^{\top}\bz_i\bz_i^{\top}(\hat\btheta-\btheta^*)}}.\]
Define a ``good'' event $\cE_i$ as
\begin{equation*}
	\cE_i := \left\{ \norm{\hat{\bbeta}_{1}^{(\tau)} - \bbeta_{1}^*}_{2} + \norm{\hat{\bbeta}_{2}^{(\tau)} - \bbeta_{2}^*}_{2}  + \norm{\hat{\btheta}^{(\tau)} - \btheta^*}_{2}
	\leq C	\sqrt{\frac{s\log d \log n_0}{N_{\tau-1}}}\right\},
\end{equation*}
where $\hat{\bbeta}_{1}^{(\tau)}$, $\hat{\bbeta}_{2}^{(\tau)}$, $\hat{\btheta}^{(\tau)}$ are the estimators obtained using the samples in the $(\tau-1)$-th phase.  By Theorem \ref{thm:1-detailed}, it holds that $\Pr(\cE_i^c) \leq c \frac{\log^3 n_0}{\max\{N_{\tau-1}, d\}^2}  \lesssim  \frac{1}{N_{\tau-1}}$ for some constants $c$ and $C$, which yields
\begin{equation*}
	R\left(\hat\btheta^{(\tau)}\right)-R(\btheta^*) \lesssim 1/N_{\tau-1} + \sqrt{\EE\brackets{\big(\hat\btheta^{(\tau)}-\btheta^*\big)^{\top}\bz_i\bz_i^{\top}\big(\hat\btheta^{(\tau)}-\btheta^*\big) \cond \cE_i}} \lesssim \sqrt{\frac{s\log d \log n_0}{N_{\tau-1}}}.
\end{equation*} \hfill\qed
%Note that for any $p,\hat p\in(0,1)$, we have $|\min\{p,1-p\}-\min\{\hat p,1-\hat p\}|\le |p-\hat p|$. In fact, to prove this claim, it suffices to show this inequality holds if $\hat p<1/2<p$. When $\hat p<1/2<p$, $|\min\{p,1-p\}-\min\{\hat p,1-\hat p\}|=|1-p-\hat p|\le |p-\hat p|$. Therefore,
%\begin{align*}
%	R(\btheta^*)-  R\big(\hat\btheta\big)\le& \int_{\calZ}
%	\abs{p(\bz_i^\top\hat\btheta)-p(\bz_i^\top\btheta^*)} \mu(d\bz_i)\\
%	\le& \frac{1}{4} \int_{\calZ}
%	\abs{\bz_i^\top\hat\btheta-\bz_i^\top\btheta^*}   \mu(d\bz_i)\\
%	\le& \sqrt{\norm{\Cov(\bz_i)}_2}\cdot \norm{\hat\btheta-\btheta^*}_2 \\
%	\le& \sqrt{M \frac{s\log d \log N_{\tau-1}}{N_{\tau-1}}}.
%\end{align*}
\end{proof}

\subsection{Proof for the Regret Upper Bound}
\label{sec:proof-upper}
\begin{proof}{Proof of Theorem \ref{thm:regret}}
Let $\calI_{\tau}$ be the set of indices in the $\tau$-th episode and $N_{\tau}=n_02^{\tau}$ be the cardinality of $\calI_{\tau}$. 
The expected cumulative regret over a length of horizon $T$ can be expressed as
\begin{equation*}
\Reg(T) = \sum_{\tau = 0}^{\tau_{\max}} \sum_{i\in\calN_{\tau}} \EE\brackets{\reg_i},
\end{equation*}
where $\tau_{\max}=[\log_2(T/n_0+1)]-1$.

We will show the following results for the instant regret:
Suppose the conditions in Theorem \ref{thm:regret} hold. For an observation $i$ in the $\tau$-th episode ($\tau \geq 2$), we have
\begin{equation} \label{eqn:E[reg_i]-strong}
\EE[\reg_i^{*}] \lesssim   \frac{\overline{x}^2s^2\log d \log n_0}{N_{\tau-1}} + \overline{x} \norm{\bbeta_2^* - \bbeta_1^*}_1\cdot R(\btheta^*), 
\end{equation}
and
\begin{equation} \label{eqn:E[reg_i]-weak}
\EE[\tilde{\reg}_i] \lesssim \overline{x} \norm{\bbeta_2^*-\bbeta_1^*}_1\sqrt{\frac{s\log d \log n_0}{N_{\tau-1}}}. 
\end{equation}
	
We first deal with the instant strong regret. Let $(\hat{\btheta}, \hat{\bbeta}_{1}, \hat{\bbeta}_{2})$ be the estimator in the $\tau$-th episode, which is obtained using the data collected in the $(\tau-1)$-th episode. By the proof of Theorem \ref{thm:coeff-bound-hd}, we have that 
	\[
	\norm{\hat{\bbeta}_{1} - \bbeta_{1}^*}_1 + \norm{\hat{\bbeta}_{2} - \bbeta_{2}^*}_1 + \norm{\hat{\btheta} - \btheta^*}_1
	\leq C
	\sqrt{\frac{s^2\log d \log n_0}{N_{\tau-1}}},
	\]
	with probability at least $1-c \frac{\log^3 n_0}{\max\{N_{\tau-1}, d\}^2}$ for some constant $c, C$.
Define ``good'' events $\cE_i$ and $\cG_i$ as
	\begin{align*}
		\cE_i& := \left\{ \norm{\hat{\bbeta}_{1} - \bbeta_{1}^*}_{1} + \norm{\hat{\bbeta}_{2} - \bbeta_{2}^*}_{1} + \norm{\hat{\btheta} - \btheta^*}_{1}
		\leq C
		\sqrt{\frac{s^2\log d \log n_0}{N_{\tau-1}}}\right\}\\
		%&	\underset{a\in [K]}{\max}\; \abs{\angles{\bx_{i,a}, \bbeta_g^* - \hat \bbeta_g}} \leq  C\sqrt{\log K+\log(d \vee N_{\tau-1})}\cdot \norm{\hat \bbeta_g - \bbeta_g^*}_2 \text{ for } g= 1, 2,\\
			\cG_i&:=\left\{\angles{\bx_{i, \tilde{a}_{i, g}}, \bbeta_g^*} > \max_{a\ne \tilde{a}_{i, g}}\angles{\bx_{i, a}, \bbeta^*_{g}} + 2C\overline{x}\sqrt{\frac{s^2\log d  \log n_0}{N_{\tau-1}}}, \text{ for } g= 1, 2 \right\},
	\end{align*}
	where $\hat{\bbeta}_{1}$, $\hat{\bbeta}_{2}$, $\hat{\btheta}$ are the estimators obtained using the samples in the $(\tau-1)$-th phase, and $\tilde{a}_{i, g}:=\underset{a\in [K]}{\arg \max}\; \angles{\bx_{i,a}, \bbeta^*_g}$.  Then it holds that $\Pr(\cE_i^c) \leq c \frac{\log^3 n_0}{\max\{N_{\tau-1}, d\}^2}  \lesssim  \frac{1}{N_{\tau-1}}$, which yields
	\begin{equation}\label{eq:Eec-strong}
		\EE[\reg^*_i \cond \cE_i^c] \lesssim \overline{R}/N_{\tau-1}.
	\end{equation}%By the sub-Gaussianity of $\bx_{i,a}$ and the maximal sub-Gaussian inequality, 
	%\[
%	\underset{a\in [K]}{\max}\; \abs{\angles{\bx_{i,a}, \bbeta_g^* - \hat \bbeta_g} }\lesssim \sqrt{\log K+\log(d \vee N_{\tau-1})}\cdot \norm{\hat \bbeta_g - \bbeta_g^*}_2,
%	\]
%	with probability  at least $1-c_0 \frac{1}{\max\{N_{\tau-1}, d\}}$.
 By Assumption \ref{B2}, we have that
	\[\angles{\bx_{\tilde{a}_{i, g}, i}, \bbeta_g^*} \ge \max_{a\ne \tilde{a}_{i, g}}\angles{\bx_{a, i}, \bbeta^*_{g}} + 2C\overline{x}\sqrt{\frac{s^2\log d  \log n_0}{N_{\tau-1}}}, \text{ for } g= 1, 2,\]
	hold with probability at least $1-2C_1C\overline{x}\sqrt{\frac{s^2\log d  \log n_0}{N_{\tau-1}}}$. Therefore,
	\[
	\Pr(\cG_i^c) \lesssim \overline{x}\sqrt{\frac{s^2\log d  \log n_0}{N_{\tau-1}}}.
	\]

	
	Since of $\norm{\bx_{i,a}}_{\infty} \leq \overline{x}$, under $\cE_i$, we have for $g=1,2$,
	\[\max_{a \in [K]}\angles{\bx_{i,a}, (\bbeta_g^* - \hat \bbeta_g)} \leq \overline{x}\norm{\bbeta_g^* - \hat \bbeta_g}_1 \leq C\overline{x}\sqrt{\frac{s^2\log d  \log n_0}{N_{\tau-1}}}.\]
	If $\cE_i\cap\cG_i$ holds, then
	\[\max\limits_{a\in [K]}\abs{\angles{\bx_{i,a}, \hat \bbeta_{g}} - \angles{\bx_{i,a}, \bbeta^*_{g}}} \leq \frac{1}{2}\left( \angles{\bx_{i, \tilde{a}_{i, g}}, \bbeta_g^*} - \max_{a\ne \tilde{a}_{i, g}}\angles{\bx_{i, a}, \bbeta^*_{g}}\right),\]
	which implies that, for any $a \neq \tilde{a}_{i, g}= \underset{a\in [K]}{\arg \max}\; \angles{\bx_{i,a}, \bbeta^*_{g}}$,
	\begin{equation}
		\begin{aligned}
			\angles{\bx_{i, a}, \hat\bbeta_{g}} & \leq \abs{\angles{\bx_{i,a}, \hat \bbeta_{g}} - \angles{\bx_{i,a}, \bbeta^*_{g}}}  + \angles{\bx_{i, a}, \bbeta^*_{g}}	\\
			&\leq \max\limits_{a\in [K]}\abs{\angles{\bx_{i,a}, \hat \bbeta_{g}} - \angles{\bx_{i,a}, \bbeta^*_{g}}}  +\max_{a\ne \tilde{a}_{i, g}}\angles{\bx_{i, a}, \bbeta^*_{g}}\\
			& \leq \angles{\bx_{i, \tilde{a}_{i, g}}, \bbeta_g^*}-\frac{1}{2}\left( \angles{\bx_{i, \tilde{a}_{i, g}}, \bbeta_g^*} - \max_{a\ne \tilde{a}_{i, g}}\angles{\bx_{i, a}, \bbeta^*_{g}}\right)\\
			& \leq  \angles{\bx_{i, \tilde{a}_{i, g}}, \hat{\bbeta}_g}.
		\end{aligned}
	\end{equation}
	Therefore, we have $\underset{a\in [K]}{\arg \max}\; \angles{\bx_{i,a}, \hat\bbeta_{g}}=\underset{a\in [K]}{\arg \max}\; \angles{\bx_{i,a}, \bbeta^*_{g}}$ for $g=1,2$.

Now we consider two different cases of $g_i$ and $\hat g_i$. When $\hat g_i = g_i$, $\reg^{*}_i = \underset{a\in [K]}{\max}\; \angles{\bx_{i,a}, \bbeta_{g_i}^*} - \angles{\bx_{i,\hat a_i}, \bbeta_{g_i}^*}$. Under $\cE_i\cap\cG_i$, since $\underset{a\in [K]}{\arg \max}\; \angles{\bx_{i,a}, \hat\bbeta_{g}}=\underset{a\in [K]}{\arg \max}\; \angles{\bx_{i,a}, \bbeta^*_{g}}$ for $g=g_i$, we have $\reg^{*}_i=0$. Otherwise
\begin{align*} 
	\EE(\reg^{*}_i \cond \hat g_i = g_i, \cE_i \cap \cG_i^c)  &= \EE\brackets{\underset{a\in [K]}{\max}\; \angles{\bx_{i,a}, \bbeta_1^*} - \angles{\bx_{i,\hat a_i}, \bbeta_1^*} \cond \cE_i \cap \cG_i^c}  \Pr( \cE_i \cap \cG_i^c) \\
& = \EE\brackets{ \underset{a\in [K]}{\max}\; \angles{\bx_{i,a}, \bbeta_1^*} - \underset{a\in [K]}{\max}\;  \angles{\bx_{i,a}, \hat \bbeta_1} 
+ \angles{\bx_{i,\hat a_i}, \hat\bbeta_1} - \angles{\bx_{i, \hat a_i}, \bbeta_1^*} }\Pr( \cE_i \cap \cG_i^c)
\\
& \le 2\; \EE\abs{\underset{a\in [K]}{\max}\; \angles{\bx_{i,a}, (\bbeta_1^* - \hat \bbeta_1)}} \Pr( \cE_i \cap \cG_i^c)\\
& \lesssim \frac{\overline{x}^2s^2\log d  \log n_0}{N_{\tau-1}}.
%\le 2 \underset{a\in [K]}{\max}\;\norm{\bx_{i,a}}_2 \cdot \norm{\hat \bbeta_1 - \bbeta_1^*}_2. 
\end{align*}
% and sub-Gaussian maximal inequality, we have that
%\begin{equation}\label{eqn:x-beta-diff}
%\EE \abs{\underset{a\in [K]}{\max}\; \angles{\bx_{i,a}, (\bbeta_1^* - \hat \bbeta_1)}} \lesssim  \sqrt{\log K}\cdot \norm{\hat \bbeta_1 - \bbeta_1^*}_2. 
%\end{equation}

As a result, we obtain that 
\begin{equation*}
 \EE(\reg^{*}_i \cond \hat g_i =  g_i, \cE_i) \lesssim \frac{\overline{x}^2s^2\log d  \log n_0}{N_{\tau-1}}.
\end{equation*}
%By a similar argument, it can be shown that 
%\begin{equation*}
%\EE(\reg^{*}_i \cond \hat g_i = 2, g_i = 2, \cE_i) \lesssim \sqrt{\frac{s\log(d \vee N_{\tau-1}) \log N_{\tau-1} \log K}{N_{\tau-1}}}.
%\end{equation*}

If $g_i = 1$, but the algorithm mistakenly clusters it to $\hat g_i = 2$, the greedy policy prescribes
$\hat a_i = \underset{a\in [K]}{\arg \max}\; \angles{\bx_{i,a}, \hat \bbeta_2}$. 
The instant strong regret is
\begin{equation}\label{eq:51}
	\begin{aligned}
		&\quad \EE(\reg^{*}_i \cond \hat g_i = 2, g_i = 1, \cE_i)  =  \EE\brackets{\underset{a\in [K]}{\max}\; \angles{\bx_{i,a}, \bbeta_1^*} - \angles{\bx_{i,\hat a_i}, \bbeta_1^*}}  \\
		& = \EE\Big[ \underset{a\in [K]}{\max}\; \angles{\bx_{i,a}, \bbeta_1^*} - \underset{a\in [K]}{\max}\; \angles{\bx_{i,a}, \bbeta_2^*}
		+ \underset{a\in [K]}{\max}\; \angles{\bx_{i,a}, \bbeta_2^*} \\
		%- \underset{a\in [K]}{\max}\;  \angles{\bx_{i,a}, \hat \bbeta_2}\\
		%& \quad + \angles{\bx_{i,\hat a_i}, \hat\bbeta_2}
		 &\quad \quad -\angles{\bx_{i,\hat a_i}, \bbeta^*_2} +\angles{\bx_{i,\hat a_i},\bbeta^*_2}-\angles{\bx_{i,\hat a_i}, \bbeta^*_1}\Big]
		\\
		& \le \EE\brackets{2\abs{\underset{a\in [K]}{\max}\; \angles{\bx_{i,a}, \bbeta_2^* - \bbeta_1^*}} 
			+  \underset{a\in [K]}{\max}\; \angles{\bx_{i,a}, \bbeta_2^*}-\angles{\bx_{i,\hat a_i}, \bbeta^*_2}} \\
		& \lesssim \overline{x}\norm{\bbeta_2^*-\bbeta_1^*}_1
		+  \frac{\overline{x}^2s^2\log d \log n_0}{N_{\tau-1}}.
	\end{aligned}
\end{equation}

By a similar argument, we have 
\begin{equation}\label{eq:52}
\EE(\reg^{*}_i \cond \hat g_i = 1, g_i = 2,\cE_i) \lesssim \overline{x}\norm{\bbeta_2^*-\bbeta_1^*}_1
+  \frac{\overline{x}^2s^2\log d \log n_0}{N_{\tau-1}}.
\end{equation}
In summary, we have
\begin{align*}
&\quad	\EE[\reg_i^{*}\cond \cE_{i}] \\
& = \EE[\reg^{*}_i \cond \hat g_i = g_i, \cE_i] \cdot \left(1-R\big(\hat\btheta\big)\right) + \EE[\reg^{*}_i \cond \hat g_i \neq g_i, \cE_i] R\left(\hat\btheta\right)\\
& \lesssim \frac{\overline{x}^2s^2\log d  \log n_0}{N_{\tau-1}} + \paren{\overline{x}\norm{\bbeta_2^* - \bbeta_1^*}_1 +    \frac{\overline{x}^2s^2\log d \log n_0}{N_{\tau-1}} } \cdot \paren{R(\btheta^*) + \sqrt{\frac{s\log d \log n_0}{N_{\tau-1}}}} \\
& \lesssim   \frac{\overline{x}^2s^2\log d  \log n_0}{N_{\tau-1}} + \overline{x} \norm{\bbeta_2^* - \bbeta_1^*}_1\cdot R(\btheta^*),
\end{align*}  
where we apply Theorem \ref{thm:miss-class-rate}. Combing \eqref{eq:Eec-strong} with the above inequality leads to \eqref{eqn:E[reg_i]-strong}.

Now we deal with the instant regular regret. Similar to \eqref{eq:Eec-strong}, we have 
	\begin{equation}\label{eq:Eec-regular}
	\EE[\tilde{\reg}_i \cond \cE_i^c] \lesssim \overline{R}/N_{\tau-1}, \quad \EE[\tilde{\reg}_i \cond \cG_i^c] \lesssim \overline{x}\sqrt{\frac{s^2\log d  \log n_0}{N_{\tau-1}}}.
\end{equation}
%By the proof of Theorem \ref{thm:coeff-bound-hd}, we have that 
%\[
%\norm{\hat{\bbeta}_{1} - \bbeta_{1}^*}_1 + \norm{\hat{\bbeta}_{2} - \bbeta_{2}^*}_2 + \norm{\hat{\btheta} - \btheta^*}_2
%\lesssim 
%\sqrt{\frac{s\cdot\log(d \vee N_{\tau-1}) \log N_{\tau-1}}{N_{\tau-1}}},
%\]
%with probability at least $1-c_0 \frac{\log^2 N_{\tau-1}}{\max\{N_{\tau-1}, d\}}$. By the sub-Gaussianity of $\bx_{i,a}$ and the maximal sub-Gaussian inequality, 
%\[
%\underset{a\in [K]}{\max}\; \abs{\angles{\bx_{i,a}, \bbeta_g^* - \hat \bbeta_g} }\lesssim \sqrt{\log K+\log(d \vee N_{\tau-1})}\cdot \norm{\hat \bbeta_g - \bbeta_g^*}_2,
%\]
%with probability  at least $1-c_0 \frac{1}{\max\{N_{\tau-1}, d\}}$. By Assumption \ref{assump:reward}, we have that the third inequality in $\cE_i$,
%\[\angles{\bx_{\tilde{a}_{i, g}, i}, \bbeta_g^*} \le \max_{a\ne \tilde{a}_{i, g}}\angles{\bx_{a, i}, \bbeta^*_{g}} + 2C^2\sqrt{\frac{s\cdot\log(d \vee N_{\tau-1}) \log N_{\tau-1}(\log K+\log(d \vee N_{\tau-1}))}{N_{\tau-1}}}, \text{ for } g= 1, 2,\]
%hold with probability $1-2C^2\sqrt{\frac{s\cdot\log(d \vee N_{\tau-1}) \log N_{\tau-1}(\log K+\log(d \vee N_{\tau-1}))}{N_{\tau-1}}}$. Therefore,
%\[
%\Pr(\cE_i^c) \lesssim  \sqrt{\frac{s\cdot\log(d \vee N_{\tau-1}) \log N_{\tau-1}(\log K+\log(d \vee N_{\tau-1}))}{N_{\tau-1}}},
%\]
%and thus
%\begin{equation}\label{eq:Eec}
%	\EE[\tilde{\reg}_i \cond \cE_i^c] \lesssim 2\overline{R} \sqrt{\frac{s\cdot\log(d \vee N_{\tau-1}) \log N_{\tau-1}(\log K+\log(d \vee N_{\tau-1}))}{N_{\tau-1}}}.
%\end{equation}
%
%In the sequel, we consider the conditional  expectation on $\cE_i$. 

Note that
\begin{equation}\label{eq:Ee}
	\begin{aligned}
		&\quad \EE[\tilde{\reg}_i\cond \cE_{i} \cap \cG_{i}]\\
		&=\EE\brackets{\angles{\bx_{i,\tilde{a}_i}, \bbeta^*_{g_i}} - \angles{\bx_{i,\hat a_i}, \bbeta^*_{g_i}} \cond \cE_{i} \cap \cG_{i} }\\
		& =\EE\brackets{\underset{a\in [K]}{\max}\; \angles{\bx_{i,a}, \bbeta^*_{g_i}} - \angles{\bx_{i,\hat a_i}, \bbeta^*_{g_i}} \cond \cE_{i} \cap \cG_{i} } - \EE\brackets{\underset{a\in [K]}{\max}\; \angles{\bx_{i,a}, \bbeta^*_{g_i}} - \angles{\bx_{i, \tilde{a}_i}, \bbeta^*_{g_i}}\cond \cE_{i} \cap \cG_{i} }\\
		& = \EE\brackets{\underset{a\in [K]}{\max}\; \angles{\bx_{i,a}, \bbeta^*_{g_i}} - \angles{\bx_{i,\hat a_i}, \bbeta^*_{g_i}} \cond g_i=\widehat{g}_i, \cE_{i} \cap \cG_{i}} \left(1-R\big(\hat{\btheta}\big)\right) \\
		&\quad + \EE\brackets{\underset{a\in [K]}{\max}\; \angles{\bx_{i,a}, \bbeta_{g_i}^*} - \angles{\bx_{i,\hat a_i}, \bbeta_{g_i}^*} \cond g_i\neq\widehat{g}_i,  \cE_{i} \cap \cG_{i}} R\big(\hat{\btheta}\big)\\
		&  \quad - \EE\brackets{\underset{a\in [K]}{\max}\; \angles{\bx_{i,a}, \bbeta^*_{g_i}} - \angles{\bx_{i,\tilde a_i}, \bbeta^*_{g_i}} \cond g_i=\tilde{g}_i,  \cE_{i} \cap \cG_{i}} \left(1-R(\btheta^*)\right)\\
		&\quad - \EE\brackets{\underset{a\in [K]}{\max}\; \angles{\bx_{i,a}, \bbeta_{g_i}^*} - \angles{\bx_{i,\tilde a_i}, \bbeta_{g_i}^*} \cond g_i\neq\tilde{g}_i,  \cE_{i} \cap \cG_{i}} R(\btheta^*)
	\end{aligned}
\end{equation}


% Under $\cE_i$, 
%it holds that 
%\[\max\limits_{a\in [K]}\abs{\angles{\bx_{i,a}, \hat \bbeta_{g}} - \angles{\bx_{i,a}, \bbeta^*_{g}}} \leq \frac{1}{2}\left( \angles{\bx_{i, \tilde{a}_{i, g}}, \bbeta_g^*} - \max_{a\ne \tilde{a}_{i, g}}\angles{\bx_{i, a}, \bbeta^*_{g}}\right),\]
%which implies that, for any $a \neq \tilde{a}_{i, g}= \underset{a\in [K]}{\arg \max}\; \angles{\bx_{i,a}, \bbeta^*_{g}}$,
%\begin{equation}
%	\begin{aligned}
%		\angles{\bx_{i, a}, \hat\bbeta_{g}} & \leq \abs{\angles{\bx_{i,a}, \hat \bbeta_{g}} - \angles{\bx_{i,a}, \bbeta^*_{g}}}  + \angles{\bx_{i, a}, \bbeta^*_{g}}	\\
%		&\leq \max\limits_{a\in [K]}\abs{\angles{\bx_{i,a}, \hat \bbeta_{g}} - \angles{\bx_{i,a}, \bbeta^*_{g}}}  +\max_{a\ne \tilde{a}_{i, g}}\angles{\bx_{i, a}, \bbeta^*_{g}}\\
%		& \leq \angles{\bx_{i, \tilde{a}_{i, g}}, \bbeta_g^*}-\frac{1}{2}\left( \angles{\bx_{i, \tilde{a}_{i, g}}, \bbeta_g^*} - \max_{a\ne \tilde{a}_{i, g}}\angles{\bx_{i, a}, \bbeta^*_{g}}\right)\\
%		& \leq  \angles{\bx_{i, \tilde{a}_{i, g}}, \hat{\bbeta}_g}.
%	\end{aligned}
%\end{equation}
Since $\underset{a\in [K]}{\arg \max}\; \angles{\bx_{i,a}, \hat\bbeta_{g}}=\underset{a\in [K]}{\arg \max}\; \angles{\bx_{i,a}, \bbeta^*_{g}}$ for $g=1,2$ under $\cE_i \cap \cG_i$, we have $\hat{a}_i = \underset{a\in [K]}{\arg \max}\; \angles{\bx_{i,a}, \hat \bbeta_{g_i}}= \underset{a\in [K]}{\arg \max}\; \angles{\bx_{i,a}, \bbeta^*_{g_i}} =\tilde{a}_i$ when $g_i=\hat{g}_i=\tilde{g}_i$. 
Therefore, 
\begin{align*}
	&\quad \EE\brackets{\underset{a\in [K]}{\max}\; \angles{\bx_{i,a}, \bbeta^*_{g_i}} - \angles{\bx_{i,\hat a_i}, \bbeta^*_{g_i}} \cond g_i=\widehat{g}_i,  \cE_{i} \cap \calG_{i}} \left(1-R\big(\hat{\btheta}\big)\right) \\
	&\quad - \EE\brackets{\underset{a\in [K]}{\max}\; \angles{\bx_{i,a}, \bbeta^*_{g_i}} - \angles{\bx_{i,\tilde a_i}, \bbeta^*_{g_i}} \cond g_i=\tilde{g}_i, \cE_{i} \cap \calG_{i}} \left(1-R(\btheta^*)\right)\\
	%& = \EE\brackets{\underset{a\in [K]}{\max}\; \angles{\bx_{i,a}, \bbeta^*_{g_i}} - \angles{\bx_{i,\hat a_i}, \bbeta^*_{g_i}} \cond g_i=\widehat{g}_i,  \cE_{i}\cap \calG_{i}} \left(R(\btheta^*)-R\big(\hat{\btheta}\big)\right)\\
	& = 0.
\end{align*}
And 
\begin{align*}
	&\quad \EE\brackets{\underset{a\in [K]}{\max}\; \angles{\bx_{i,a}, \bbeta^*_{g_i}} - \angles{\bx_{i,\hat a_i}, \bbeta^*_{g_i}} \cond g_i \neq \widehat{g}_i,  \cE_{i} \cap \calG_{i}} R\big(\hat{\btheta}\big) \\
	&\quad - \EE\brackets{\underset{a\in [K]}{\max}\; \angles{\bx_{i,a}, \bbeta^*_{g_i}} - \angles{\bx_{i,\tilde a_i}, \bbeta^*_{g_i}} \cond g_i\neq \tilde{g}_i, \cE_{i} \cap \cG_i} R(\btheta^*)\\
	& = \EE\brackets{\underset{a\in [K]}{\max}\; \angles{\bx_{i,a}, \bbeta^*_{g_i}} - \angles{\bx_{i,\hat a_i}, \bbeta^*_{g_i}} \cond g_i \neq\widehat{g}_i,  \cE_{i} \cap \calG_{i}} \left(R\big(\hat{\btheta}\big)-R(\btheta^*)\right)\\
	& \lesssim \left(\overline{x}\norm{\bbeta_2^*-\bbeta_1^*}_1
	+  \overline{x}\sqrt{\frac{s^2\log d \log n_0 }{N_{\tau-1}}}\right) \cdot \sqrt{\frac{s\log d \log n_0}{N_{\tau-1}}},
\end{align*}
where the last inequality follows from \eqref{eq:51}, \eqref{eq:52}, and Theorem \ref{thm:miss-class-rate}. Combining \eqref{eq:Eec-regular} and the above two inequalities, we obtain that 
\[ \EE[\tilde{\reg}_i] \lesssim \overline{x}\norm{\bbeta_2^*-\bbeta_1^*}_1\sqrt{\frac{s\log d \log n_0}{N_{\tau-1}}}.\]

Now we return to the cumulative regrets. The regret accumulated in the $\tau$-th phase can be bounded in two different cases. 
\begin{enumerate}[label=(\roman*)]
	\item When $\tau\leq1$, we have $N_{\tau} \leq 2n_0$, then the boundedness of rewards in Assumption \ref{B1} implies that
	\begin{align*}
		& \sum_{i\in\calN_{\tau}} \EE\brackets{\reg^{*}_i} \leq 2\overline{R} N_{\tau} \lesssim \overline{R}n_0, \quad  \sum_{i\in\calN_{\tau}} \EE\brackets{\tilde{\reg}_i} \leq 2\overline{R} N_{\tau} \lesssim \overline{R}n_0. 
	\end{align*}
	%Compared with the simple linear setting, the additional $\log N_{\tau-1}$ is incurred by the iterative algorithm and the sample splitting in theoretical analysis. 
	\item When $\tau \geq 2$, by \eqref{eqn:E[reg_i]-strong}, the expected strong regret in the $\tau$-th phase satisfies
	\begin{align*}
		\sum_{i\in \calN_{\tau}}\EE[\reg^{*}_i] \lesssim
		\overline{x}^2s^2\log d \log n_0 + \overline{x} \norm{\bbeta_2^* - \bbeta_1^*}_1\cdot R(\btheta^*)N_{\tau-1}.
	\end{align*} 
	By \eqref{eqn:E[reg_i]-weak}, the expected regular regret in the $\tau$-th phase satisfies
	\begin{align*}
		\sum_{i\in \calN_{\tau}}\EE[\tilde{\reg}_i] \lesssim \overline{x} \norm{\bbeta_2^*-\bbeta_1^*}_1\sqrt{s\log d \log n_0 \cdot N_{\tau-1}}.
	\end{align*} 
\end{enumerate}

Hence, the total expected {\em strong} regret is 
\begin{align*}
	\Reg^{*}(T) & = \sum_{\tau = 0}^{\tau_{\max}} \sum_{i\in\calN_{\tau}} \EE\brackets{\reg^{*}_i} \\
	& \lesssim  \overline{R}n_0 +\overline{x}^2s^2\log d \log n_0 \cdot \log T+ \overline{x} \norm{\bbeta_2^* - \bbeta_1^*}_1\cdot R(\btheta^*) \cdot T.
\end{align*}
The total expected regular regret is
\begin{align*}
	\tilde{\Reg}(T) & = \sum_{\tau = 1}^{\tau_{\max}} \sum_{i\in\calN_{\tau}} \EE\brackets{\tilde{\reg}_i}  \\
	& \lesssim \overline{R}n_0 + \sum_{\tau = 2}^{\tau_{\max}}  \overline{x} \norm{\bbeta_2^*-\bbeta_1^*}_1\sqrt{sn_02^{\tau-1}\log d \log n_0 }\\
	& \lesssim \overline{R}n_0+ \overline{x} \norm{\bbeta_2^*-\bbeta_1^*}_1\sqrt{s\log d \log n_0 } \sqrt{T}.
\end{align*} \hfill\qed
\end{proof}


\subsection{Proof for the Regret Lower Bound}\label{sec:proof-lower}
\begin{proof}{Proof of Theorem \ref{thm:lower-bound}}
	We first show the lower bound for the instant strong regret. Given constants $\overline{L}>0$ and $\overline{x}>0$, let $\bbeta_{1}^{*}=(\overline{L}, 0, \dots, 0)$, $\bbeta_2^*=(-\overline{L}, 0, \dots, 0)$, and the $j$-th entry of $\bx_{i, a_i}$ be $x_{ij} + \frac{\overline{x}}{2}(3- 2a_i)$, where $x_{ij} \stackrel{i.i.d.}{\sim} \calU[-\overline{x}/2, \overline{x}/2]$ for $j=1,2,\dots, d$, and $a_i \in \left\{1, 2\right\}$. For simplicity, we denote $(\bx_{i, 1}, \bx_{i, 2})$ by $\bx_i$ for any $i$. The parameter $\btheta^* \in \RR^d$ and the distribution of $\bz_i$ can be chosen arbitrarily as long as they satisfy $\norm{\btheta^*}_0 \leq s$, \ref{A1}, \ref{A4}, and $\bz_i$ is independent of $x_{i1}$. Then it is straightforward to verify that this choice of $\mu(\bx, y, \bz;\bgamma^*)$ belongs to $ \calP_{d,s,\overline{x},\overline{L}}$. 
	
	We have that 
	\[\angles{\bx_{i,a_i}, \bbeta^*_{g_i}} =\overline{L} \left(x_{i1}+\frac{\overline{x}}{2}(3- 2a_i)\right)(3-2g_i)=\overline{L}x_{i1}(3-2g_i)+\overline{L}\overline{x}(3 - 2a_i)(3-2g_i)/2,\] and hence 
	\begin{equation}\label{eq:max-regret-lower}
		\max_{a_i \in [2]}\angles{\bx_{i,a_i}, \bbeta^*_{g_i}} =\overline{L}x_{i1}(3-2g_i)+\overline{L}\overline{x}/2.
	\end{equation}
	
	On the other hand, let $\hat\pi(a_i \mid \bx_i, \bz_i, \calH_{i-1})$ denote a policy for choosing $a_i$, i.e., a conditional distribution of $a_i$ given the present features $(\bx_i, \bz_i)$ and the past history $\calH_{i-1}:=(\bx_{i-1}, \bz_{i-1}, y_{i-1}, \dots, \bx_{1}, \bz_{1}, y_{1})$. Let $\hat\pi_1:=\hat\pi(a_i=1 \mid \bx_i, \bz_i, \calH_{i-1})$ and $p_1:=\PP(g_i=1 \mid \bx_i, \bz_i, \calH_{i-1})=p(\bz_i^{\top}\btheta^*)$. Note that given $\bz_i$, the group $g_i$ is independent of the action $\widehat{a}_i$ under $\pi$. Then the conditional expected reward with $\widehat{a}_i \sim \hat\pi(a_i \mid \bx_i, \bz_i, \calH_{i-1})$ can be written as
	\begin{equation}\label{eq:pi-hat-regret}
	\begin{aligned}
			&\quad \EE_{\hat\pi}\left[\angles{\bx_{i,\widehat{a}_i}, \bbeta^*_{g_i}} \mid  \bx_i, \bz_i, \calH_{i-1} \right]\\
			&=\overline{L}\Big\{p_1\left[(x_{i1}+\overline{x}/2)\hat\pi_1+(x_{i1}-\overline{x}/2)(1-\hat\pi_1)\right]+(1-p_1)\left[(-x_{i1}-\overline{x}/2)\hat\pi_1+(\overline{x}/2-x_{i1})(1-\hat\pi_1)\right]\Big\},
	\end{aligned}
	\end{equation}
	and hence, for all $\hat\pi(a_i \mid \bx_i, \bz_i, \calH_{i-1})$,
	\begin{align*}
	 \EE_{\hat\pi}\left[\angles{\bx_{i,\widehat{a}_i}, \bbeta^*_{g_i}} \mid  \bx_i, \bz_i, \calH_{i-1} \right] &\leq \overline{L}(2p_1-1) \left(x_{i1}+\overline{x}\sgn[2p_1-1]/2\right)\\
		&=\overline{L}(2p(\bz_i^{\top}\btheta^*)-1)x_{i1}+\frac{\overline{x}\overline{L}}{2}\abs{2p(\bz_i^{\top}\btheta^*)-1}.
	\end{align*}
	Note that \eqref{eq:max-regret-lower} implies
	\begin{equation}\label{eq:expected-max-regret-lower}
		\begin{aligned}
				\EE\left[\max_{a_i \in [2]}\angles{\bx_{i,a_i}, \bbeta^*_{g_i}} \mid \bx_i, \bz_i, \calH_{i-1}\right] 
				&= p_1(\overline{L}x_{i1}+\overline{L}\overline{x}/2)+(1-p_1)(-\overline{L}x_{i1}+\overline{L}\overline{x}/2)\\
				&=\overline{L}(2p(\bz_i^{\top}\btheta^*)-1)x_{i1}+\overline{L}\overline{x}/2.
		\end{aligned}
	\end{equation}
	As  $R(\btheta^*)=\EE \Big[\min\braces{(1 - p(\bz_i^\top\btheta^*), p(\bz_i^\top\btheta^*)}\Big]=\frac{1}{2}\EE\left[1-\abs{2p(\bz_i^{\top}\btheta^*)-1}\right]$, we obtain, for any policy $\hat\pi$, 
	\[\sup_{\mu \in \calP_{s,d,\overline{x},\overline{L}}}\EE_{\hat\pi}[\reg^*_{i}] \geq \frac{\overline{x}\overline{L}}{2}\EE\left[1-\abs{2p(\bz_i^{\top}\btheta^*)-1}\right] \gtrsim \overline{x}\overline{L}R(\btheta^*).\]
	Hence, the cumulative regret 
		\[\inf_{\hat\pi}\sup_{\mu \in \calP_{s,d,\overline{x},\overline{L}}}\sum_{i=1}^{T}\EE_{\hat\pi}[\reg^*_{i}]\gtrsim \overline{x}\overline{L}R(\btheta^*)T.\]
		
	We then show the lower bound for the instant regular regret, where we first introduce the following lemma on the lower bound of the excess misclassification rate for the sparse logistic model:
	\begin{lemma}[\cite{abramovich2018high}, Section \uppercase\expandafter{\romannumeral 6\relax}]\label{lem:risk-lower-bound}
		Define a sparse logistic model $(y, \bz) \sim \calL_{\btheta^*}$ as $y  \sim \mathrm{Bernoulli}(p)$ with $p=\frac{\exp(\bz^{\top}\btheta^*)}{1+\exp(\bz^{\top}\btheta^*)}$, where $\btheta^*\in \RR^d$ and $\norm{\btheta^*}_0 \leq s$. Then we have
		\[\inf_{\hat\eta}\sup_{\norm{\btheta^*}_0 \leq s} \big[\EE_{\{(y_i, \bz_i)\}_{i=1}^n \sim \calL_{\btheta^*}}[R_{\btheta^*}(\hat\eta)]-R_{\btheta^*}(\eta^*)\big] \gtrsim \sqrt{\frac{s\log(d/s)}{n}},\]
		where $R_{\btheta^*}(\eta):=\EE_{(y, \bz) \in \calL_{\btheta^*}}[\one(\eta(\bz) \neq y)]$, $\eta^*(\bz)=\one(\bz^{\top}\btheta^*>0)$, and the infimum is taken over all classifiers $\hat\eta: \RR^d \to \{0, 1\}$ learned from random samples $\{(y_i, \bz_i)\}_{i=1}^n$.
	\end{lemma}

Let $\calH_{i-1, \bz}:=\{\bz_{1}, \bz_{2}, \dots, \bz_{i-1}\}$. 
By \eqref{eq:pi-hat-regret} and \eqref{eq:expected-max-regret-lower}, we have that 
\begin{align*}
	&\quad \EE\left[\max_{a_i \in [2]}\angles{\bx_{i,a_i}, \bbeta^*_{g_i}} \mid \bz_i,  \calH_{i-1, \bz}\right] - \EE_{\hat\pi}\left[\angles{\bx_{i,\widehat{a}_i}, \bbeta^*_{g_i}} \mid  \bz_i,  \calH_{i-1, \bz}\right] \\
		&=\EE_{\bx}\left[\EE\left[\max_{a_i \in [2]}\angles{\bx_{i,a_i}, \bbeta^*_{g_i}} \mid \bx_i, \bz_i, \calH_{i-1}\right]\right] - \EE_{\bx}\left[\EE_{\hat\pi}\left[\angles{\bx_{i,\widehat{a}_i}, \bbeta^*_{g_i}} \mid  \bx_i, \bz_i, \calH_{i-1} \right]\right] \\
		&= \overline{x}\overline{L}\Big[p_1(1-\EE_{\bx}[\hat\pi_1])+(1-p_1)\EE_{\bx}[\hat\pi_1]\Big],
\end{align*}
%\[
%	\EE\left[\max_{a_i \in [2]}\angles{\bx_{i,a_i}, \bbeta^*_{g_i}} \mid \bx_i, \bz_i, \calH_{i-1}\right]- \EE_{\hat\pi}\left[\angles{\bx_{i,\widehat{a}_i}, \bbeta^*_{g_i}} \mid  \bx_i, \bz_i, \calH_{i-1} \right]
%	= \overline{x}\overline{L}\Big[p_1(1-\hat\pi_1)+(1-p_1)\hat\pi_1\Big].
%\]
where $\EE_{\bx}$ is taken over $\bx_1,\dots, \bx_{i}$. In particular,
\[	\EE\left[\max_{a_i \in [2]}\angles{\bx_{i,a_i}, \bbeta^*_{g_i}} \mid   \bz_i, \calH_{i-1,\bz}\right] - \EE\left[\angles{\bx_{i,\widetilde{a}_i}, \bbeta^*_{g_i}} \mid \bz_i, \calH_{i-1,\bz} \right] = \overline{x}\overline{L}[p_1(1-\widetilde\pi_1)+(1-p_1)\widetilde\pi_1],\]
where $\widetilde\pi_1=\one(\bz_i^{\top}\btheta^* \geq 0)$. Note that $\EE_{\bx}[\hat\pi_1]=\EE_{\bx}[\hat\pi(a_i=1 \mid \bx_i, \bz_i, \calH_{i-1})]$ can be viewed a function of $\bz_i$ that is learned based on $\calH_{i-1, \bz}$, and thus we can correspondingly define an estimated classifier $\hat\eta_{\hat\pi}$ such that $\hat\eta_{\hat\pi}=1$ with probability $\EE_{\bx}[ \hat\pi_1]$ and $\hat\eta_{\hat\pi}=2$ with probability $1-\EE_{\bx}[\hat\pi_1]$. Then, using $\EE_{\bz}$ to denote the expectation taken over $\bz_1,\dots, \bz_{i}$, we have $\EE_{\bz}\big[p_1(1-\EE_{\bx}[\hat\pi_1])+(1-p_1)\EE_{\bx}[\hat\pi_1]\big]=\EE_{\bz}[R_{\btheta^*}(\hat\eta_{\hat\pi})]$ and $\EE_{\bz}[p_1(1-\widetilde\pi_1)+(1-p_1)\widetilde\pi_1]=R_{\btheta^*}(\eta^*)$, where $\eta^*(\bz)$ is the classifier such that $g_i=1$ if $\bz^{\top}\btheta^*\geq 0$ and $g_i=2$ otherwise. Therefore, for any policy $\hat\pi$, 
\begin{align*}
	&\quad \EE \left[\angles{\bx_{i, \widetilde{a}_i}, \bbeta^*_{g_i}}\right] - \EE_{\hat\pi} \left[\angles{\bx_{i, \widehat{a}_i}, \bbeta^*_{g_i}}\right]\\
	&=\EE_{\bz}\bigg[\EE\left[\max_{a_i \in [2]}\angles{\bx_{i,a_i}, \bbeta^*_{g_i}} \mid \bz_i,  \calH_{i-1, \bz}\right] - \EE_{\hat\pi}\left[\angles{\bx_{i,\widehat{a}_i}, \bbeta^*_{g_i}} \mid  \bz_i,  \calH_{i-1, \bz}\right]\bigg]\\
	&\quad  - \EE_{\bz}\bigg[\EE\left[\max_{a_i \in [2]}\angles{\bx_{i,a_i}, \bbeta^*_{g_i}} \mid   \bz_i, \calH_{i-1,\bz}\right] - \EE\left[\angles{\bx_{i,\widetilde{a}_i}, \bbeta^*_{g_i}} \mid \bz_i, \calH_{i-1,\bz} \right] \bigg]\\
	&=\overline{x}\overline{L}[\EE_{\bz}[R_{\btheta^*}(\hat\eta_{\hat\pi})]-R_{\btheta^*}(\eta^*)].
\end{align*}
By Lemma \ref{lem:risk-lower-bound}, we obtain that
\[ \inf_{\hat\pi}\sup_{\mu \in \calP_{d,s,\overline{x},\overline{L}}}\EE \left[\angles{\bx_{i, \widetilde{a}_i}, \bbeta^*_{g_i}}\right] - \EE_{\hat\pi} \left[\angles{\bx_{i, \widehat{a}_i}, \bbeta^*_{g_i}}\right] \gtrsim \overline{x}\overline{L}\sqrt{\frac{s\log d}{i-1}}.\]
Hence, for $n_0\gtrsim s\log d$, the cumulative regular regret 
\[ \inf_{\hat\pi}\sup_{\mu \in \calP_{d,s,\overline{x},\overline{L}}}\sum_{i=n_0}^{T}\bigg[\EE \left[\angles{\bx_{i, \widetilde{a}_i}, \bbeta^*_{g_i}}\right] - \EE_{\hat\pi} \left[\angles{\bx_{i, \widehat{a}_i}, \bbeta^*_{g_i}}\right]\bigg] \gtrsim \overline{x}\norm{\bbeta_1^*-\bbeta_2^*}_1\sqrt{\frac{s\log d}{i-1}}\gtrsim \overline{x}\overline{L}\sqrt{sT\log d},\]
where we use the fact that $\sqrt{n}\leq \sum_{i=1}^{n}\frac{1}{\sqrt{i}} \leq 2\sqrt{n}$. \hfill\qed
\end{proof}


\section{Proof for the Technical Lemmas}
%!TEX root = 0-main.tex

\subsection{Proof of Lemma \ref{thm:lemma-expectation}} \label{sec:proof-lemma-expectation}

\begin{comment}
This lemma is related to Lemma 2 and Definition 4 (5.5) in \cite{balakrishnan2017statistical}. 
See also (A.5a) on page 37 of the paper. 
Lemma A.1 in \cite{zhang2020estimation}.
\end{comment}
Recall that \begin{equation}
	\omega(\bx, y, \bz ; \bgamma) 
	= \frac{p(\bz^\top\btheta) \cdot
		\phi\paren{\frac{y - \bx^\top\bbeta_1}{\sigma}}}{p(\bz^\top\btheta) \cdot \phi\paren{\frac{y - \bx^\top\bbeta_1}{\sigma}} 
		+ \paren{1-p(\bz^\top\btheta)} \cdot \phi\paren{\frac{y - \bx^\top\bbeta_2}{\sigma}}}.
\end{equation}
%Define a function $f_1(\btheta, \bbeta_1, \bbeta_2) 
%\defeq \frac{1-p(\bz^\top\btheta)}{p(\bz^\top\btheta)} \times \frac{\phi\paren{\frac{y - \bx^\top\bbeta_2}{\sigma}}}{\phi\paren{\frac{y - \bx^\top\bbeta_1}{\sigma}}} 
%= \frac{1-p(\bz^\top\btheta)}{p(\bz^\top\btheta)} \times \exp\left(\frac{(y - \bx^\top \bbeta_1)^2 - (y - \bx^\top \bbeta_2)^2}{2\sigma^2}\right)$.
%We rewrite $\omega(\bx, y, \bz ; \bgamma) = \frac{1}{1 + f_1(\btheta, \bbeta_1, \bbeta_2)}$.
We calculate the partial derivatives of $\omega(\bx, y, \bz ; \bgamma)$ with respect to each component of $\bgamma\defeq [\bbeta_1^\top, \bbeta_2^\top, \btheta^\top]^\top$ as%, given by:
%\begin{equation*}
%\frac{\partial\omega(\bx, y, \bz ; \bgamma)}{\partial \bgamma} = - \frac{1}{(1 + f_1(\bgamma))^2}\cdot\frac{\partial f_1(\bgamma)}{\partial \bgamma},   
%\end{equation*}
%where
\begin{equation}\label{eq:partial}
	\begin{aligned}
		&\quad \frac{\partial \omega}{\partial \btheta} \\
		& =\frac{p(\bz^\top\btheta)(1-p(\bz^\top\btheta))\bz}{\left[p(\bz^\top\btheta)+(1-p(\bz^\top\btheta))\exp\left(\frac{(y-\bx^{\top}\bbeta_1)^2-(y-\bx^{\top}\bbeta_2)^2}{2\sigma^2}\right)\right]\left[p(\bz^\top\btheta)\exp\left(\frac{(y-\bx^{\top}\bbeta_2)^2-(y-\bx^{\top}\bbeta_1)^2}{2\sigma^2}\right)+1-p(\bz^\top\btheta)\right]}, \\
		%%------------------------
		&\quad \frac{\partial \omega}{\partial \bbeta_1} \\
		& = \frac{p(\bz^\top\btheta)(1-p(\bz^\top\btheta))(y-\bx^\top\bbeta_1)\bx/\sigma^2}{\left[p(\bz^\top\btheta)+(1-p(\bz^\top\btheta))\exp\left(\frac{(y-\bx^{\top}\bbeta_1)^2-(y-\bx^{\top}\bbeta_2)^2}{2\sigma^2}\right)\right]\left[p(\bz^\top\btheta)\exp\left(\frac{(y-\bx^{\top}\bbeta_2)^2-(y-\bx^{\top}\bbeta_1)^2}{2\sigma^2}\right)+1-p(\bz^\top\btheta)\right]},\\
		%%------------------------
		&\quad \frac{\partial \omega}{\partial \bbeta_2} \\
		& = \frac{-p(\bz^\top\btheta)(1-p(\bz^\top\btheta))(y-\bx^\top\bbeta_2)\bx/\sigma^2}{\left[p(\bz^\top\btheta)+(1-p(\bz^\top\btheta))\exp\left(\frac{(y-\bx^{\top}\bbeta_1)^2-(y-\bx^{\top}\bbeta_2)^2}{2\sigma^2}\right)\right]\left[p(\bz^\top\btheta)\exp\left(\frac{(y-\bx^{\top}\bbeta_2)^2-(y-\bx^{\top}\bbeta_1)^2}{2\sigma^2}\right)+1-p(\bz^\top\btheta)\right]}.
	\end{aligned}
\end{equation}

Let $\bgamma^{(t)}=\left[\big(\bbeta_1^{(t)}\big)^{\top}, \big(\bbeta_2^{(t)}\big)^{\top}, \big(\btheta^{(t)}\big)^{\top}\right]^\top$, $\bdelta_\gamma=\bgamma^{(t)}-\bgamma^*$ and $\bgamma_u=\bgamma^*+u\bdelta_\gamma$ for $u\in[0,1]$, then
\begin{align*}
& \quad \EE[\omega(\bx, y, \bz ; \bgamma^{(t)})  ] - \EE[\omega(\bx, y, \bz ; \bgamma^*) ] = \EE\brackets{\int_0^1\angles{\frac{\partial\omega}{\partial \bgamma} \big|_{\bgamma_u}, \bdelta_\gamma} \mathrm{d} u} \\
& = \brackets{\int_0^1\angles{\EE\frac{\partial\omega}{\partial \btheta}\big|_{\bgamma_u}, \bdelta_\theta} \mathrm{d} u} + \sum_{g=1,2}\EE\brackets{\int_0^1\angles{\EE\frac{\partial\omega}{\partial \bbeta_{g}}\big|_{\bgamma_{u}}, \bdelta_{\beta_g}} \mathrm{d} u}.
\end{align*}
It suffices to show that, for any constant $\kappa>0$, when $c_1$ is sufficiently small and $c_2$ is sufficiently large, we have
\begin{equation}\label{eq:partial-bound}
	\sup_{u \in [0, 1]}\norm{\EE\left[\frac{\partial\omega}{\partial \bbeta_g}\big|_{\bgamma_{u}}\right]}_2 \leq \kappa \text{ for } g=1, 2, \text{ and } \sup_{u \in [0, 1]}\norm{\EE\left[\frac{\partial\omega}{\partial \btheta}\big|_{\bgamma_{u}}\right]}_2 \leq \kappa.
\end{equation}

We first show that $\sup\limits_{u \in [0, 1]}\norm{\EE\left[\frac{\partial\omega}{\partial \bbeta_1}\big|_{\bgamma_{u}}\right]}_2 \leq \kappa$. In the sequel, we omit the subscript $u$, i.e., we use $(\bbeta_1, \bbeta_2, \btheta)$ to denote an arbitrary parameter between $(\bbeta_1^*, \bbeta_2^*, \btheta^*)$ and $\big(\bbeta_1^{(t)}, \bbeta_2^{(t)}, \btheta^{(t)}\big)$. Define $\cE_1= \braces{\abs{\bz^{\top}(\btheta-\btheta^*)}<\mu\norm{\btheta-\btheta^*}_2}$, where $\mu$ is a constant to be determined. By the sub-Gaussianity of $\bz$, we have that $\PP\paren{\cE_1^c} < 2e^{-\mu^2/(2\sigma_z^2)}$. 

Under $\cE_1$, note that $p(\bz^{\top}\btheta)(1-p(\bz^{\top}\btheta)) \leq \frac{1}{4}$ and $\abs{p(\bz^{\top}\btheta)-p(\bz^{\top}\btheta^*)} \leq \frac{1}{4}\abs{\bz^{\top}(\btheta-\btheta^*)} \leq  \frac{1}{4}\mu\norm{\btheta-\btheta^*}_2<\xi/2$ if $\norm{\btheta-\btheta^*}_2 \leq  c_1\xi$ for $c_1 \leq 2/\mu$. Then we obtain that $\xi/2 <p(\bz^{\top}\btheta) < 1-\xi/2$, and thus \[\left[p(\bz^\top\btheta)+(1-p(\bz^\top\btheta))\exp\left(\frac{(y-\bx^{\top}\bbeta_1)^2-(y-\bx^{\top}\bbeta_2)^2}{2\sigma^2}\right)\right] \geq \frac{\xi}{2}\brackets{1+ \exp\left(\frac{(y-\bx^{\top}\bbeta_1)^2-(y-\bx^{\top}\bbeta_2)^2}{2\sigma^2}\right)},\]
	\[\left[p(\bz^\top\btheta)\exp\left(\frac{(y-\bx^{\top}\bbeta_2)^2-(y-\bx^{\top}\bbeta_1)^2}{2\sigma^2}\right)+1-p(\bz^\top\btheta)\right]\geq \frac{\xi}{2}\brackets{1+ \exp\left(\frac{(y-\bx^{\top}\bbeta_2)^2-(y-\bx^{\top}\bbeta_1)^2}{2\sigma^2}\right)},\]
which implies that
\begin{equation}
	\label{eq:E-partial-oemga-1}
	\norm{\EE\left[\frac{\partial\omega(\bgamma)}{\partial \bbeta_1} \big| \cE_1 \right]}_2 \leq \sup\limits_{\norm{\bv}_2=1}\frac{1}{\xi^2}\EE\brackets{\frac{\abs{y-\bx^{\top}\bbeta_1}(\bx^{\top}\bv)/\sigma^2}{\exp\paren{\abs{\frac{(y-\bx^{\top}\bbeta_1)^2-(y-\bx^{\top}\bbeta_2)^2}{2\sigma^2}}}}} .
	\end{equation}

By Cauchy-Schwarz, we further have
\begin{equation}
	\label{eq:E-partial-oemga-2}
	\EE\brackets{\frac{\abs{y-\bx^{\top}\bbeta_1}(\bx^{\top}\bv)/\sigma^2}{\exp\paren{\abs{\frac{(y-\bx^{\top}\bbeta_1)^2-(y-\bx^{\top}\bbeta_2)^2}{2\sigma^2}}}}} \leq \sqrt{\EE\brackets{\frac{(y-\bx^{\top}\bbeta_1)^2(\bx^{\top}\bv)^2}{\sigma^4}}\EE{\exp\paren{-\abs{\frac{(y-\bx^{\top}\bbeta_1)^2-(y-\bx^{\top}\bbeta_2)^2}{\sigma^2}}}}}.
\end{equation}
By Jensen's inequality, 
\begin{align*}
	\EE{\exp\paren{-\abs{\frac{(y-\bx^{\top}\bbeta_1)^2-(y-\bx^{\top}\bbeta_2)^2}{\sigma^2}}}} &\leq \exp\paren{-\EE\frac{\abs{(y-\bx^{\top}\bbeta_1)^2-(y-\bx^{\top}\bbeta_2)^2}}{\sigma^2}}\\ &\leq \exp\paren{-\frac{\abs{\EE(y-\bx^{\top}\bbeta_1)^2-\EE(y-\bx^{\top}\bbeta_2)^2}}{\sigma^2}}\\
	&=\begin{cases}
		\exp\paren{-\frac{\abs{\norm{\bbeta_2^*-\bbeta_1}^2_{\Sigma}-\norm{\bbeta_2^*-\bbeta_2}^2_{\Sigma}}}{\sigma^2}} & \text{ if } y=\bx^{\top}\bbeta_2^*+\epsilon, \\
		\exp\paren{-\frac{\abs{\norm{\bbeta_1^*-\bbeta_2}^2_{\Sigma}-\norm{\bbeta_1^*-\bbeta_1}^2_{\Sigma}}}{\sigma^2}} & \text{ if } y=\bx^{\top}\bbeta_1^*+\epsilon,
	\end{cases}\\
	&\leq \exp\paren{-\frac{\frac{1}{2}\norm{\bbeta_2^*-\bbeta^*_1}^2_{\Sigma}-\norm{\bbeta_1^*-\bbeta_1}^2_{\Sigma}-\norm{\bbeta_2^*-\bbeta_2}^2_{\Sigma}}{\sigma^2}}\\
	&\leq \exp\paren{-c_3\frac{\norm{\bbeta_2^*-\bbeta^*_1}^2_{2}}{\sigma^2}},
\end{align*}
where $\norm{\bv}_{\Sigma} := \sqrt{\bv^{\top}\Sigma\bv}$ for any vector $\bv$, $\Sigma = \EE [\bx\bx^{\top}]$, and $c_3=\frac{1}{2M}-Mc_1^2 \geq \frac{1}{4M}$, using the assumption that $\norm{\bbeta_1^*-\bbeta_1}_{2}+\norm{\bbeta_2^*-\bbeta_2}_{2} \leq c_1 \norm{\bbeta_1^*-\bbeta_2^*}_2$,  $M^{-1}<\lambda_{\min}(\Sigma)<\lambda_{\max}(\Sigma)<M$, and $c_1<1/(2M)$.

Now we bound $\EE\brackets{\frac{(y-\bx^{\top}\bbeta_1)^2(\bx^{\top}\bv)^2}{\sigma^4}}$. The sub-Gaussianity of $\bx$ ensures that, for any $\bv$ such that $\norm{\bv}_2=1$,  \[\EE(\bx^{\top}\bv)^m \leq C^{m}\sigma_{x}^mm^{m/2},\] for some constant $C >0$.

If $y=\bx^{\top}\bbeta_2^*+\epsilon$, then 
\begin{align*}
\EE\brackets{\frac{(y-\bx^{\top}\bbeta_1)^2(\bx^{\top}\bv)^2}{\sigma^4}} &\leq \EE\brackets{\frac{(\bx^{\top}\bbeta_2^*+\epsilon-\bx^{\top}\bbeta_1)^2(\bx^{\top}\bv)^2}{\sigma^4}} \\
&\leq 3\EE\brackets{\frac{\brackets{(\bx^{\top}(\bbeta_2^*-\bbeta_1^*))^2+(\bx^{\top}(\bbeta_1^*-\bbeta_1))^2+\epsilon^2}(\bx^{\top}\bv)^2}{\sigma^4}}\\
& \leq \frac{48C^4\sigma_x^4\norm{\bbeta^*_2-\bbeta^*_1}_2^2+48C^4\sigma_x^4\norm{\bbeta_1-\bbeta^*_1}_2^2+48C^4\sigma^2\sigma_x^2}{\sigma^4} \\
&\leq 48C^4\sigma_x^2 \frac{(1+c_1^2)\sigma_x^2\norm{\bbeta^*_2-\bbeta^*_1}_2^2+\sigma^2}{\sigma^4}.
\end{align*}
Otherwise, if $y=\bx^{\top}\bbeta_1^*+\epsilon$, then 
\begin{align*}
	\EE\brackets{\frac{(y-\bx^{\top}\bbeta_1)^2(\bx^{\top}\bv)^2}{\sigma^4}} &\leq \EE\brackets{\frac{(\bx^{\top}\bbeta_1^*+\epsilon-\bx^{\top}\bbeta_1)^2(\bx^{\top}\bv)^2}{\sigma^4}} \\
	& \leq 32C^4\sigma_x^2 \frac{\sigma_x^2\norm{\bbeta_1-\bbeta^*_1}_2^2+\sigma^2}{\sigma^4}\\
	& \leq 32C^4\sigma_x^2 \frac{c_1^2\sigma_x^2\norm{\bbeta^*_2-\bbeta^*_1}_2^2+\sigma^2}{\sigma^4}.
\end{align*}
Therefore, by combining the above inequalities with \eqref{eq:E-partial-oemga-1} and \eqref{eq:E-partial-oemga-2} , we obtain that 
\[	\norm{\EE\left[\frac{\partial\omega(\bgamma)}{\partial \bbeta_1} \big| \cE_1 \right]}_2 \leq 4\sqrt{3}C^2\sigma_x\frac{\sqrt{2\sigma_x^2\norm{\bbeta^*_2-\bbeta^*_1}_2^2+\sigma^2}}{\sigma^2}\exp\paren{-c_3\frac{\norm{\bbeta_2^*-\bbeta^*_1}^2_{2}}{2\sigma^2}}.\]
By some algebra, it holds that, when $\frac{\norm{\bbeta_2^*-\bbeta^*_1}^2_{2}}{\sigma^2} > c_2^2$ with $c_2 \geq \max\big\{\frac{256MC^2\sigma_x\eta_x}{\kappa}, \frac{1}{\sqrt{2}\sigma_x}\big\}$, the above quantity $\norm{\EE\left[\frac{\partial\omega(\bgamma)}{\partial \bbeta_1} \big| \cE_1 \right]}_2 \leq \kappa/2$. 

Under $\cE_1^c$, we simply apply the fact that 
\[\frac{p(\bz^\top\btheta)(1-p(\bz^\top\btheta))}{\left[p(\bz^\top\btheta)+(1-p(\bz^\top\btheta))\exp\left(\frac{(y-\bx^{\top}\bbeta_1)^2-(y-\bx^{\top}\bbeta_2)^2}{2\sigma^2}\right)\right]\left[p(\bz^\top\btheta)\exp\left(\frac{(y-\bx^{\top}\bbeta_2)^2-(y-\bx^{\top}\bbeta_1)^2}{2\sigma^2}\right)+1-p(\bz^\top\btheta)\right]} \leq \frac{1}{4},\]
and the result we derived above
\[\EE (y-\bx^\top\bbeta_1)(\bx^{\top}\bv)/\sigma^2 \leq 4\sqrt{3}C^2\sigma_x\frac{\sqrt{2\sigma_x^2\norm{\bbeta^*_2-\bbeta^*_1}_2^2+\sigma^2}}{\sigma^2},\]
then we obtain
\[\norm{\EE\left[\frac{\partial\omega(\bgamma)}{\partial \bbeta_1} \big| \cE_1^c \right]}_2\PP(\cE_1^c) \leq 8\sqrt{3}C^2\sigma_x\frac{\sqrt{2\sigma_x^2\norm{\bbeta^*_2-\bbeta^*_1}_2^2+\sigma^2}}{\sigma^2} e^{-\mu^2/(2\sigma_z^2)}.\]
By taking $\mu \geq  \sqrt{2}\sigma_z\sqrt{\log\paren{\frac{32\sqrt{3}C^2\sigma_x^2c_2}{\kappa}}}$ , the above quantity $\norm{\EE\left[\frac{\partial\omega(\bgamma)}{\partial \bbeta_1} \big| \cE_1^c \right]}_2\PP(\cE_1^c)$ is less than $\kappa/2$.
Then
\[\norm{\EE\left[\frac{\partial\omega(\bgamma)}{\partial \bbeta_1} \right]}_2 \leq  \norm{\EE\left[\frac{\partial\omega(\bgamma)}{\partial \bbeta_1} \big| \cE_1\right]}_2\PP(\cE_1) + \norm{\EE\left[\frac{\partial\omega(\bgamma)}{\partial \bbeta_1} \big| \cE_1^c \right]}_2\PP(\cE_1^c) \leq \kappa.\]
The other two inequalities $\norm{\EE\left[\frac{\partial\omega(\bgamma)}{\partial \bbeta_2}\right]}_2 \leq \kappa$ and $\norm{\EE\left[\frac{\partial\omega(\bgamma)}{\partial \btheta}\right]}_2 \leq \kappa$ can be shown in a similar way. 

We then use the same technique to  establish the other two inequalities in Lemma \ref{thm:lemma-expectation}. 
%By the fact that
%\begin{align*}
%&\EE[ \omega(\bx, y, \bz; \bgamma) \bx ( \bx^\top \bbeta_1^* - y) ] - \EE[ \omega(\bx, y, \bz; \bgamma^*)  \bx ( \bx^\top \bbeta_1^* - y) ]\\ 
%& = \EE\brackets{\int_0^1\angles{\frac{\partial\omega(\bgamma_u)}{\partial \btheta}\bx^{\top} ( \bx^\top \bbeta_1^* - y),\; \bdelta_\theta} \mathrm{d} u}
%+ \sum_{g=1,2}\EE\brackets{\int_0^1\angles{\frac{\partial\omega(\bgamma_u)}{\partial \bbeta_g}\bx ( \bx^\top \bbeta_1^* - y),\; \bdelta_{\beta_g}} \mathrm{d} u},
%\end{align*}
It suffices to show that, for some $\kappa \in (0, 1)$,
\[\sup_{u \in [0, 1]}\norm{\EE\left[\frac{\partial\omega}{\partial \bbeta_g}\big|_{\bgamma_{u}}\bx^{\top} ( \bx^\top \bbeta_1^* - y)\right]}_2 \leq \kappa \text{ for } g=1, 2, \text{ and } \sup_{u \in [0, 1]}\norm{\EE\left[\frac{\partial\omega}{\partial \btheta}\big|_{\bgamma_{u}}\bx^\top ( \bx^\top \bbeta_1^* - y)\right]}_2 \leq \kappa.\]
Compared to the proof of \eqref{eq:partial-bound}, the only difference is we replace $\EE\brackets{\frac{(y-\bx^{\top}\bbeta_1)^2(\bx^{\top}\bv)^2}{\sigma^4}}$ with
$\EE\brackets{\frac{(y-\bx^{\top}\bbeta_1)^2(y-\bx^{\top}\bbeta_1^*)^2(\bx^{\top}\bv)^4}{\sigma^4}}$ and need to bound it. Note that $(y-\bx^{\top}\bbeta_1)^2(y-\bx^{\top}\bbeta_1^*)^2(\bx^{\top}\bv)^4 \leq \frac{1}{2}\brackets{(y-\bx^{\top}\bbeta_1)^4(\bx^{\top}\bv)^4 + (y-\bx^{\top}\bbeta_1^*)^4 (\bx^{\top}\bv)^4}$. Either when $y=\bx^{\top}\bbeta^*_1+\epsilon$ or $y=\bx^{\top}\bbeta^*_2+\epsilon$, we have $(y-\bx^{\top}\bbeta_1)^4(\bx^{\top}\bv)^4 + (y-\bx^{\top}\bbeta_1^*)^4 (\bx^{\top}\bv)^4 \leq 35(\bx^{\top}\bbeta_2^*-\bx^{\top}\bbeta_1^*)^4(\bx^{\top}\bv)^4+35 (\bx^{\top}\bbeta_1^*-\bx^{\top}\bbeta_1)^4(\bx^{\top}\bv)^4 + 35\epsilon^4(\bx^{\top}\bv)^4$. By the sub-Gaussianity of $\bx$, it holds that \[\EE\brackets{\frac{(y-\bx^{\top}\bbeta_1)^2(y-\bx^{\top}\bbeta_1^*)^2(\bx^{\top}\bv)^4}{\sigma^4}} \leq \frac{35\cdot 8^4 \cdot C^8\sigma_x^4}{2\sigma^4} \left(\sigma^4+(1+c_1^4)\sigma_x^4\norm{\bbeta_1^*-\bbeta_2^*}_2^4\right),\] 
which yields that 
\[\norm{\EE\left[\frac{\partial\omega}{\partial \bbeta_g}\big|_{\bgamma_{u}}\bx^{\top} ( \bx^\top \bbeta_1^* - y) \big| \cE_1 \right]}_2  \leq 268C^4\sigma_x^2\sqrt{ 1+2\sigma_x^4\frac{\norm{\bbeta_1^*-\bbeta_2^*}_2^4}{\sigma^4}}\exp\paren{-c_3\frac{\norm{\bbeta_2^*-\bbeta^*_1}^2_{2}}{2\sigma^2}},\]
and 
\[\norm{\EE\left[\frac{\partial\omega}{\partial \bbeta_g}\big|_{\bgamma_{u}}\bx^{\top} ( \bx^\top \bbeta_1^* - y) \big| \cE_1^c \right]}_2 \PP(\cE_1^c) \leq 536C^4\sigma_x^2\sqrt{ 1+2\sigma_x^4\frac{\norm{\bbeta_1^*-\bbeta_2^*}_2^4}{\sigma^4}}e^{-\mu^2/(2\sigma_z^2)}.\]
By some algebra, we can show that when $\frac{\norm{\bbeta_1^*-\bbeta_2^*}_2}{\sigma} \geq c_2 \geq \max\left\{384C^2\sigma_x^2M/\sqrt{\kappa}, 1/(2^{1/4}\sigma_x)\right\}$,  we have $268C^4\sigma_x^2\sqrt{ 1+2\sigma_x^4\frac{\norm{\bbeta_1^*-\bbeta_2^*}_2^4}{\sigma^4}}\exp\paren{-c_3\frac{\norm{\bbeta_2^*-\bbeta^*_1}^2_{2}}{2\sigma^2}} \leq \frac{\kappa}{2}$. Moreover, if $\mu$ satisfies $\mu \geq  \sqrt{2}\sigma_z\sqrt{\log\paren{\frac{2144C^4\sigma_x^4c_2^2}{\kappa}}}$, we have $536C^4\sigma_x^2\sqrt{ 1+2\sigma_x^4\frac{\norm{\bbeta_1^*-\bbeta_2^*}_2^4}{\sigma^4}}e^{-\mu^2/(2\sigma_z^2)} \leq \kappa/2$. As conclusion, when $c_1 \leq \min\{1/4M, 2/\mu_0\}$  and $c_2 \geq \max\left\{1/(2^{1/4}\sigma_x), 256MC^2\sigma_x\eta_x/\kappa, 384C^2\sigma_x^2M/\sqrt{\kappa}\right\}$, where $\mu_0=\sqrt{2}\sigma_z\sqrt{\log\paren{\frac{2144C^4\sigma_x^4c_2^2}{\kappa}}}$ is larger than  $\sqrt{2}\sigma_z\sqrt{\log\paren{\frac{32\sqrt{3}C^2\sigma_x^2c_2}{\kappa}}}$, the first two inequalities in Lemma \ref{thm:lemma-expectation} hold.

The proof for the upper bound of $\EE[ ( \omega(\bx, y, \bz;\bgamma) - p(\bz_i^\top\btheta^{*}) ) \bz_i ] - \EE[ (\omega(\bx, y, \bz;\bgamma^*) - p(\bz_i^\top\btheta^{*}) ) \bz_i ]= \EE[ \omega(\bx, y, \bz;\bgamma) \bz_i ] - \EE[ \omega(\bx, y, \bz;\bgamma^*) \bz_i ]$ can be established in the same way.
%\begin{align*}
%&\EE[ ( \omega_i^{(t)} - p(\bz_i^\top\btheta^{*}) ) \bz_i ] - \EE[ (\omega_i^* - p(\bz_i^\top\btheta^{*}) ) \bz_i ] \\
%& = \EE\brackets{\int_0^1\angles{\frac{\partial\omega(\bx_i, y_i, \bz_i ; \bgamma)}{\partial \btheta}\bz_i^\top,\; \bdelta_\theta} du} + \sum_{k=1,2}\EE\brackets{\int_0^1\angles{\frac{\partial\omega(\bx_i, y_i, \bz_i ; \bgamma)}{\partial \bbeta_k}\bz_i^\top,\; \bdelta_{\beta_k}} du}
%\end{align*}
%By Lemma \ref{thm:exp-deriv-l2-3}, we establish that
%\begin{equation*}
%\norm{ \EE[ ( \omega_i^{(t)} - p(\bz_i^\top\btheta^{*}) ) \bz_i ] - \EE[ (\omega_i^* - p(\bz_i^\top\btheta^{*}) ) \bz_i ] }_2 \le  \kappa \cdot \paren{\norm{\bbeta_1^{(t)} - \bbeta_1^*}_2 + \norm{\bbeta_2^{(t)} - \bbeta_2^*}_2 + \norm{\btheta^{(t)} - \btheta^*}_2}
%\end{equation*}

%For $\bgamma^{(t)}\defeq [(\bbeta_1^{(t)})^\top, (\bbeta_2^{(t)})^\top, (\btheta^{(t)})^\top]^\top$, 


%\subsection{Technical Lemmas used in Lemma \ref{thm:lemma-expectation}}
%
%\begin{lemma}[Technical lemma used in Lemma \ref{thm:lemma-expectation}]\label{thm:exp-deriv-l2-1}
%Suppose that $\norm{\bbeta_1 - \bbeta_1^*}_2 + \norm{\bbeta_2 - \bbeta_2^*}_2 + \norm{\btheta - \btheta^*}_2 \le c \Delta^*$, we have that 
%\begin{align}
%\norm{\EE\brackets{\frac{f_1(\btheta, \bbeta_1, \bbeta_2)\cdot\bz}{(1 + f_1(\bgamma))^2}}}_2 & \le \kappa/2 \\
%\norm{\EE\brackets{\frac{f_1(\btheta, \bbeta_1, \bbeta_2)\cdot(y-\bx^\top\bbeta_1)\bx}{\sigma^2(1 + f_1(\bgamma))^2}}}_2 & \le \kappa/2 \\
%\norm{\EE\brackets{\frac{f_1(\btheta, \bbeta_1, \bbeta_2)\cdot(y-\bx^\top\bbeta_2)\bx}{\sigma^2(1 + f_1(\bgamma))^2}}}_2 & \le \kappa/2.
%\end{align}
%\end{lemma}
%\begin{proof}[Proof of Lemma \ref{thm:exp-deriv-l2-1}]
%\end{proof}
%
%\begin{lemma}[Technical lemma used in Lemma \ref{thm:lemma-expectation}]\label{thm:exp-deriv-l2-2}
%Suppose that $\norm{\bbeta_1 - \bbeta_1^*}_2 + \norm{\bbeta_2 - \bbeta_2^*}_2 + \norm{\btheta - \btheta^*}_2 \le c \Delta^*$, we have that 
%\begin{align}
%\norm{\EE\brackets{\frac{f_1(\btheta, \bbeta_1, \bbeta_2)\cdot\bz\bx^\top(\bx^\top \bbeta_1^* - y)}{(1 + f_1(\bgamma))^2}}}_2 & \le \kappa/2 \\
%\norm{\EE\brackets{\frac{f_1(\btheta, \bbeta_1, \bbeta_2)\cdot(y-\bx^\top\bbeta_1)\bx\bx^\top(\bx^\top \bbeta_1^* - y)}{\sigma^2(1 + f_1(\bgamma))^2}}}_2 & \le \kappa/2 \\
%\norm{\EE\brackets{\frac{f_1(\btheta, \bbeta_1, \bbeta_2)\cdot(y-\bx^\top\bbeta_2)\bx\bx^\top(\bx^\top \bbeta_1^* - y)}{\sigma^2(1 + f_1(\bgamma))^2}}}_2 & \le \kappa/2.
%\end{align}
%\end{lemma}
%\begin{proof}[Proof of Lemma \ref{thm:exp-deriv-l2-2}]
%\end{proof}
%
%\begin{lemma}[Technical lemma used in Lemma \ref{thm:lemma-expectation}]\label{thm:exp-deriv-l2-3}
%Suppose that $\norm{\bbeta_1 - \bbeta_1^*}_2 + \norm{\bbeta_2 - \bbeta_2^*}_2 + \norm{\btheta - \btheta^*}_2 \le c \Delta^*$, we have that 
%\begin{align}
%\norm{\EE\brackets{\frac{f_1(\btheta, \bbeta_1, \bbeta_2)\cdot\bz\bz^\top}{(1 + f_1(\bgamma))^2}}}_2 & \le \kappa/2 \\
%\norm{\EE\brackets{\frac{f_1(\btheta, \bbeta_1, \bbeta_2)\cdot(y-\bx^\top\bbeta_1)\bx\bz^\top}{\sigma^2(1 + f_1(\bgamma))^2}}}_2 & \le \kappa/2 \\
%\norm{\EE\brackets{\frac{f_1(\btheta, \bbeta_1, \bbeta_2)\cdot(y-\bx^\top\bbeta_2)\bx\bz^\top}{\sigma^2(1 + f_1(\bgamma))^2}}}_2 & \le \kappa/2.
%\end{align}
%\end{lemma}
%\begin{proof}[Proof of Lemma \ref{thm:exp-deriv-l2-3}]
%\end{proof}


\subsection{Proof of Lemma \ref{thm:lemma-sample}} \label{sec:proof-lemma-sample}

%We first introduce two standard lemmas  in \cite{boucheronconcentration} which are useful in establish our concentration inequalities.
%
%\begin{lemma}[Symmetrization Lemma]\label{thm:symmetrization}
%	Let $\ba_1, \dots, \ba_n \in \calA$ be independent random vectors and $f\in\calF$ for some function class $\calF$ defined on $\calA$.  We have
%	\begin{equation*}
%		\EE\brackets{\underset{f\in\calF}{\sup}\abs{\sum_{i=1}^n \big(f(\ba_i) - \EE\brackets{f(\ba_i)}\big)}} \le 2\EE\brackets{\underset{f\in\calF}{\sup}\abs{\sum_{i=1}^n \sigma_i\cdot f(\ba_i)}},
%	\end{equation*}
%	where $\sigma_1, \dots, \sigma_n$ are i.i.d. Rademacher random variables that are independent of $\ba_1, \dots, \ba_n$. 
%\end{lemma}
%
%\begin{lemma}[Talagrand's Lemma]\label{thm:Talagrand}
%	Let $\ba_1, \dots, \ba_n \in \calA$ be random samples and $g\in\calG$ for some function class $\calG$ defined on $\calA$. 
%	Let $l_i(\cdot)$ be a $L$-Lipschitz function that satisfy $l_i(0)=0$ and 
%	\begin{equation}
%		\abs{l_i(x) - l_i(x')} \le L\cdot\abs{x-x'},\quad\text{for all}\quad x, x'\in\RR. 
%	\end{equation}
%We have
%	\begin{equation*}
%		\EE_{\bsigma}\brackets{\underset{f\in\calF}{\sup}\abs{\sum_{i=1}^n \sigma_i\cdot [ l_i\circ g(\ba_i)]}} \le 2L\EE_{\bsigma}\brackets{\underset{f\in\calF}{\sup}\abs{\sum_{i=1}^n \sigma_i\cdot g(\ba_i)}} 
%	\end{equation*}
%	where $\sigma_1, \dots, \sigma_n$ are i.i.d. Rademacher random variables that are independent of $\ba_1, \dots, \ba_n$, and $\EE_{\bsigma}$ denotes the expectation taken over $(\sigma_1,\dots, \sigma_n)$. 
%\end{lemma}
	We first prove the first concentration inequality: For some constants $c$ and $C$, with probability at least $1-\frac{c}{\max\{n, d\}^2}$,
	\begin{equation*}
		 \norm{ \frac{1}{n}\sum_{i=1}^n \omega_i(\bgamma^{(t)}) \bx_i ( \bx_i^\top \bbeta_1^* - y_i) 
			- \EE[ \omega_i(\bgamma^{(t)}) \bx_i ( \bx_i^\top \bbeta_1^* - y_i) ] }_\infty \le  C \sqrt{ \frac{\log \max\{n, d\}}{n}},
	\end{equation*}
	which is equivalent to 
	\begin{equation*}
		\max_{j \in [d]} \abs{ \frac{1}{n}\sum_{i=1}^n \omega_i(\bgamma^{(t)}) x_{ij} ( \bx_i^\top \bbeta_1^* - y_i) 
			- \EE[ \omega_i(\bgamma^{(t)}) x_{ij} ( \bx_i^\top \bbeta_1^* - y_i) ] } \le  C \sqrt{ \frac{\log \max\{n, d\}}{n} } .
	\end{equation*}
	Let $\ba_{i} = (\bx_i, y_i, \bz_i)$ and $f(\ba_{i})=\omega(\bgamma^{(t)};\bx_i, y_i, \bz_i) x_{ij} ( \bx_i^\top \bbeta_1^* - y_i)$. 
	Note that $\bgamma^{(t)}$ is independent of $(\bx_i, y_I, \bz_i)$, and thus, given $\bgamma^{(t)}$, the $\{f(\ba_{i})\}$ are i.i.d. sequence $\omega_i(\bgamma)=\omega(\bgamma;\bx_i, y_i, \bz_i)$. Furthermore, since $\abs{f(\ba_i)} \leq \abs{x_{ij}(\bx_i^\top \bbeta_1^* - y_i)}$, we have
	\[\norm{f(\ba_i)-\EE[f(\ba_i)]}_{\psi_{1}} \lesssim \norm{x_{ij}}_{\psi_2}\norm{\bx_i^\top \bbeta_1^* - y_i}_{\psi_{2}} \lesssim \sigma_x^2\norm{\bbeta_1^*-\bbeta_2^*}_2+\sigma_x\sigma < \infty.\]
	By Proposition 5.16 of \cite{vershynin2010introduction}, we have
	\[\Pr\paren{\abs{\sum_{i=1}^n \left(f(\ba_i)-\EE[f(\ba_i)]\right)} > t} \leq 2\exp\brackets{-c'\min\left\{\frac{t^2}{n(C')^2}, \frac{t}{C'}\right\}},\]
	for some constant $c'$ and $C'$. Letting $t = \sqrt{3(C')^2n \log \max\{n, d\}/c'}$ yields that 
	\[\Pr\paren{\abs{\sum_{i=1}^n \left(f(\ba_i)-\EE[f(\ba_i)]\right)} > \sqrt{3(C')^2n \log \max\{n, d\}/c'}} \leq \frac{2}{\max\{n, d\}^3}.\]
	
%By applying Lemmas \ref{thm:symmetrization} and \ref{thm:Talagrand} with $\ba_{i} = (\bx_i, y_i, \bz_i)$, $f(\ba_{i})=\omega(\bgamma^{(t)};\bx_i, y_i, \bz_i) x_{ij} ( \bx_i^\top \bbeta_1^* - y_i)$, $\calF=\{f\}$, $l_i(u)=\omega(\gamma^{(t)}; \bx_i, y_i, \bz_i)u$, $g(\ba_i)=x_{ij}(\bx_i^{\top}\bbeta_1^*-y_i)$, and $\calG=\{g\}$, we obtain
%	\begin{equation*}
%		\begin{aligned}
% \EE\brackets{\abs{\sum_{i=1}^n \left(f(\ba_i)-\EE[f(\ba_i)]\right)} } & \lesssim \EE \brackets{\abs{\sum_{i=1}^n \sigma_if(\ba_i)}}\\
%			& = \EE_{\bx, y, \bz}\EE_{\bsigma}\brackets{\abs{\sum_{i=1}^n \sigma_il_i \circ g(\ba_i)}} \\
%			& \lesssim \EE_{\bx, y, \bz}\EE_{\bsigma} \brackets{\abs{\sum_{i=1}^n \sigma_i g(\ba_i)}}\\
%			&=\EE \brackets{\abs{\sum_{i=1}^n \sigma_i g(\ba_i)}}.
%		\end{aligned}
%	\end{equation*}

%Then we introduce the concentration inequality for suprema of unbounded functions established in \cite{adamczak2008tail}.
%
%\begin{lemma}\label{thm:lemma-tail}
%		Let $\ba_1, \dots, \ba_n \in \calA$ be independent random vectors and let $\mathcal{F}$ be a set of countable real-value functions such that all functions $f \in \mathcal{F}$ are measurable, square-integrable and satisfy $\EE f\left(\ba_i\right)=0$. Moreover, for some $\alpha \in (0, 1]$ and all $i$, $\norm{\sup_{f \in \calF}\abs{f(\ba_i)}}_{\psi_{\alpha}}  < \infty$. Define 
%		\[\Upsilon:=\sup_{f \in \mathcal{F}}\abs{\sum_{i=1}^n f\left(\ba_i\right)},\]
%		and $\sigma_{s}^2=\sum_{i=1}^n \sup_{f \in \mathcal{F}} \EE f^2\left(\ba_i\right)$. Then for all $\eta \in (0, 1)$ and $\delta>0$, there exists a constant $C'=C(\alpha, \eta, \sigma_s)$, such that for all $t>0$, we have
%		\[\mathbb{P}\paren{\Upsilon>(1+\eta)\E \Upsilon +t}< \exp\paren{-\frac{t^2}{2(1+\delta)\sigma_{s}^2}}+3\exp\paren{-\paren{\frac{t}{C'\norm{\max_i\sup_{f\in \calF}\abs{f(\ba_i)}}_{\psi_{\alpha}}}}^{\alpha}}.\]
%\end{lemma}

%\begin{lemma}[\cite{bousquet2003concentration}]	\label{lem:Bousquet}
%	Let $\ba_1, \dots, \ba_n \in \calA$ be independent random vectors. Let $\mathcal{F}$ be a set of countable real-value functions such that all functions $f \in \mathcal{F}$ are measurable, square-integrable and satisfy $\E f\left(\bz_i\right)=0$. Assume $\sup_{f,\bz} f\left(\bz\right) \leq 1$. Define 
%	\[\Upsilon:=\sup_{f \in \mathcal{F}}\sum_{i=1}^n f\left(\bz_i\right).\]
%	If $\sum_{i=1}^n \sup_{f \in \mathcal{F}} \E f^2\left(\bz_i\right) \leq n\sigma^2$, then for all $x>0$, we have
%	\[\mathbb{P}\paren{\Upsilon>\E \Upsilon + \sqrt{2x\paren{n\sigma^2+2\E \Upsilon}}+\frac{x}{3}}< e^{-x}.\]
%\end{lemma}

%\begin{lemma}[Symmetrization Lemma]\label{thm:symmetrization}
%Let $z_1, \cdots, z_n$ be the $n$ independent realization of the random vector $Z\in\calZ$ for some Borel space $\calZ$, and $f\in\calF$ for some function class $\calF$ defined on $\calZ$. 
%For any increasing convex function $h(\cdot)$, we have
%\begin{equation*}
%\EE\brackets{h\paren{\underset{f\in\calF}{\sup}\abs{\sum_{i=1}^n \big(f(z_i) - \EE\brackets{f(Z)}\big)}}} \le \EE\brackets{h\paren{2\underset{f\in\calF}{\sup}\abs{\sum_{i=1}^n b_i\cdot f(z_i)}}},
%\end{equation*}
%where $b_1, \cdots, b_n$ are i.i.d.~Rademacher random variables that are independent of $z_1, \cdots, z_n$. 
%\end{lemma}
%
%\begin{lemma}[Talagrand's Lemma]\label{thm:Talagrand}
%Let $z_1, \cdots, z_n$ be the $n$ independent realization of the random vector $Z\in\calZ$ for some Borel space $\calZ$, and $f\in\calF$ for some function class $\calF$ defined on $\calZ$. 
%Let $l_i(\cdot)$ be a $L$-Lipschitz function that satisfy $l_i(0)=0$ and 
%\begin{equation}
%\abs{l_i(x) - l_i(x')} \le L\cdot\abs{x-x'},\quad\text{for all}\quad x, x'\in\RR. 
%\end{equation}
%For any increasing convex function $h(\cdot)$, we have
%\begin{equation*}
%\EE\brackets{h\paren{\underset{f\in\calF}{\sup}\abs{\sum_{i=1}^n b_i\cdot l_i\circ f(z_i)}}} \le \EE\brackets{h\paren{2L\cdot\underset{f\in\calF}{\sup}\abs{\sum_{i=1}^n b_i\cdot f(z_i)}}} 
%\end{equation*}
%where $b_1, \cdots, b_n$ are i.i.d.~Rademacher random variables that are independent of $z_1, \cdots, z_n$. 
%\end{lemma}
%
%\begin{lemma}[Panchenko's Lemma]\label{thm:Panchenko}
%Suppose that $Z_1$ and $Z_2$ are two random variables that satisfy $\EE\brackets{(Z_1-a)_{+}}\le\EE\brackets{(Z_2-a)_{+}}$ for any $a \in \RR$. 
%Assuming that $ \PP(Z_2 \ge t) \le \Gamma e^{-\gamma t}$ for all $t\ge0$, we have
%\begin{equation*}
%\PP\paren{Z_1\ge t} \le \Gamma e^{1-\gamma t}. 
%\end{equation*}
%\end{lemma}

%\begin{proof}[Proof of Lemma \ref{thm:lemma-sample}]
%We first prove the first concentration inequality: For some constants $c$ and $C$, with probability at least $1-\frac{c^2\log \max\{n, d\}}{\max\{n, d\}^2}$,
%\begin{equation*}
%\sup_{\bgamma} \norm{ \frac{1}{n}\sum_{i=1}^n \omega_i(\bgamma) \bx_i ( \bx_i^\top \bbeta_1^* - y_i) 
%- \EE[ \omega_i(\bgamma) \bx_i ( \bx_i^\top \bbeta_1^* - y_i) ] }_\infty \le  C \sqrt{ \frac{\log \max\{n, d\}}{n}},
%\end{equation*}
%which is equivalent to 
%\begin{equation*}
%\max_{j \in [d]} \sup_{\bgamma}  \abs{ \frac{1}{n}\sum_{i=1}^n \omega_i(\bgamma) x_{ij} ( \bx_i^\top \bbeta_1^* - y_i) 
%- \EE[ \omega_i(\bgamma) x_{ij} ( \bx_i^\top \bbeta_1^* - y_i) ] } \le  C \sqrt{ \frac{\log \max\{n, d\}}{n} } .
%\end{equation*}
%Note that $\omega_i(\bgamma)=\omega(\bgamma;\bx_i, y_i, \bz_i)$. Let $\ba_{i} = (\bx_i, y_i, \bz_i)$,  \[f_{\bgamma}(\ba_{i})=\omega(\bgamma;\bx_i, y_i, \bz_i) x_{ij} ( \bx_i^\top \bbeta_1^* - y_i)-\EE[\omega(\bgamma;\bx_i, y_i, \bz_i) x_{ij} ( \bx_i^\top \bbeta_1^* - y_i)],\] and $\calF=\big\{f_{\bgamma}; \bgamma \in \QQ^{3d}\big\}$. Here we remark that $\bgamma=(\btheta^{\top}, \bbeta_1^{\top}, \bbeta_2^{\top})^{\top} \in \RR^{3d}$, and we first consider $\bgamma \in \QQ^{3d}$, i.e., those $\bgamma$ with rational entries, due to the requirement in Lemma \ref{thm:lemma-tail} that $\calF$ is countable. Note that 
%\[\norm{\sup_{f \in \calF}\abs{f(\ba_i)}}_{\psi_{1}} \lesssim \norm{x_{ij}}_{\psi_2}\norm{\bx_i^\top \bbeta_1^* - y_i}_{\psi_{2}} \lesssim \sigma_x^2\norm{\bbeta_1^*-\bbeta_2^*}_2+\sigma_x\sigma < \infty,\]
%and by the proof of Lemma \ref{thm:lemma-expectation},
%\[\sigma_s^2 = \sum_{i=1}^n \sup_{f \in \mathcal{F}} \EE f^2\left(\ba_i\right) \lesssim n\sigma_x^2\left(\sigma_x^2\norm{\bbeta^*_2-\bbeta^*_1}_2^2+\sigma^2\right).\]
%Furthermore, by Lemma 8.2 in \cite{kosorok2008introduction}, 
%\[\norm{\max_i\sup_{f\in \calF}\abs{f(\ba_i)}}_{\psi_{1}} \lesssim \sqrt{\log (n+1)}\norm{\sup_{f \in \calF}\abs{f(\ba_i)}}_{\psi_{1}}. \]
%Therefore, applying Lemma \ref{thm:lemma-tail} with $t=C\sqrt{n\log \max\{n, d\}}$ for sufficiently large $C$ leads to
%		\[\mathbb{P}\paren{\Upsilon < (1+\eta)\E \Upsilon + C\sqrt{n\log \max\{n, d\}}}>1-\frac{c}{\max\{n, d\}^3},\]
%for some constant $c$.
%
%Then we focus on bounding $\EE[\Upsilon]$. Let $\sigma_1, \cdots, \sigma_n$ be i.i.d.  Rademacher random variables that are independent of $\ba_{i}$, and define $g_i(\bgamma)=\omega(\bgamma;\bx_i, y_i, \bz_i) x_{ij} ( \bx_i^\top \bbeta_1^* - y_i)$. By Lemma \ref{thm:symmetrization},
%\begin{equation*}
%	\begin{aligned}
%		\EE\brackets{\Upsilon}
%		&\lesssim \EE\brackets{\underset{\bgamma}{\sup}\abs{\sum_{i=1}^n \sigma_i\omega(\bgamma;\bx_i, y_i, \bz_i) x_{ij} ( \bx_i^\top \bbeta_1^* - y_i)}}\\
%		& = \EE_{\bx, y, \bz}\EE_{\sigma_1,\dots,\sigma_{n-1}}\EE_{\sigma_n}\brackets{\underset{\bgamma}{\sup}\abs{\sum_{i=1}^n \sigma_i\omega(\bgamma;\bx_i, y_i, \bz_i) x_{ij} ( \bx_i^\top \bbeta_1^* - y_i)}}\\
%		& = \EE_{\bx, y, \bz}\EE_{\sigma_1,\dots,\sigma_{n-1}} \brackets{\frac{1}{2}\underset{\bgamma}{\sup}\abs{u_{n-1}(\bgamma)+g_i(\bgamma)}+\frac{1}{2}\underset{\bgamma}{\sup}\abs{u_{n-1}(\bgamma)- g_i(\bgamma)}}
%	\end{aligned}
%\end{equation*}
%For any $\eps>0$, there exists $\bgamma_1$, $\bgamma_2$ such that $(1-\eps)\underset{\bgamma}{\sup}\abs{u_{n-1}(\bgamma)+g_i(\bgamma)} \leq \abs{u_{n-1}(\bgamma_1)+g_i(\bgamma_1)}$
%where the last inequality follows from applying Lemma \ref{thm:Talagrand} with $l_i(u)=\omega(\bgamma_0;\bx_i, y_i, \bz_i) \cdot u$ where $\bgamma_0=\argmax_{\bgamma} \abs{\sum_{i=1}^n \sigma_i\omega(\bgamma;\bx_i, y_i, \bz_i) x_{ij} ( \bx_i^\top \bbeta_1^* - y_i)}$.

which yields that 
\begin{equation*}
\PP\paren{\max_j\abs{ \frac{1}{n}\sum_{i=1}^n \omega_i^{(t)} x_{ij} ( \bx_i^\top \bbeta_1^* - y_i) 
- \EE[ \omega_i^{(t)} x_{ij} ( \bx_i^\top \bbeta_1^* - y_i) ] } > C\sqrt{\frac{ \log \max\{n, d\}}{n}} } \le \frac{2}{\max\{n, d\}^2}. 
\end{equation*}
where $C=C'\sqrt{3/c'}$.
Similary, the second concentration inequality
\begin{equation*}
\norm{  \frac{1}{n}\sum_{i=1}^n ( \omega_i^{(t)} - p(\bz_i^\top\btheta^{*}) ) \bz_i 
- \EE[ (\omega_i^{(t)} - p(\bz_i^\top\btheta^{*}) ) \bz_i ] }_\infty \le C \sqrt{ \frac{\log \max\{n, d\}}{n} }
\end{equation*}
is equivalent to 
\begin{equation*}
\max_j \abs{  \frac{1}{n}\sum_{i=1}^n ( \omega_i^{(t)} - p(\bz_i^\top\btheta^{*}) ) \bz_i 
- \EE[ (\omega_i^{(t)} - p(\bz_i^\top\btheta^{*}) ) \bz_i ] } \le C \sqrt{ \frac{\log \max\{n, d\}}{n} }.
\end{equation*}
Since 
\begin{equation*}
\norm{( \omega_i^{(t)} - p(\bz_i^\top\btheta^{*}) ) \bz_i }_{\psi_1} \lesssim \norm{ \omega_i^{(t)} - p(\bz_i^\top\btheta^{*})}_{\psi_2} \norm{\bz_i }_{\psi_2}  < \infty,
\end{equation*}
we can similarly establish the desired result. 




%\listoffigures
%\listoftables

\end{document}