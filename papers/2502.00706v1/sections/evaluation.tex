\section{Evaluation}
\label{sec:eval}


We evaluate our proposed provenance testing approach experimentally. Our evaluation aims to assess both the effectiveness of the approach and examine the validity of its core two assumptions. Specifically, we seek to answer the following research questions:\begin{enumerate}[label=(RQ\arabic*)]
\item How accurate is our provenance tester in practice and how does the number of prompts affect its performance?
\item To what extent do derived models maintain similarity to their parents?
\item How does the size and selection of control models impact the tester?\item How effective are the query reduction approaches?
\end{enumerate}

%
%
\paragraph{Experimental Setup.} 
We run our model provenance testers on a Linux machine with 64-bit Ubuntu 22.04.3 LTS, 128GB RAM and
2x 24 CPU AMD EPYC 7443P @1.50GHz and 4x NVIDIA A40 GPUs with 48GB RAM. All experiments are implemented using PyTorch framework~\cite{paszke2019pytorch} and the Hugging Face Transformers library~\cite{transformers}.

%
%
\paragraph{Models and Provenance Pairs.}  
We collect model candidates for all provenance pairs from the Hugging Face (HF) platform\cite{huggingface}.
To avoid selection bias, we used download counts as our selection criterion, taking the most popular models subject only to hardware constraints on model size. 
To increase variety of candidates, we create two distinct benchmarks \bencho and \bencht, that differ in aspects such as model sizes, choice of pre-trained models, and ground-truth verification procedure. 
The full procedure of collection of models and constructions of benchmarks is described in Appendix~\ref{sec:appendix:benchmarks} and their brief comparison is given in Table~\ref{tab:eval:bench-a-b}.
%
\begin{table}[t]
\setlength{\tabcolsep}{5pt}
%
\begin{center}
\caption{Comparison of \bencho to \bencht on different features.}
\label{tab:eval:bench-a-b}
\begin{tabular}{l||l|l}
\hline
Feature &  \bencho & \bencht\\ \hline 
pre-trained models & 10 & 57 \\
derived models & 100 & 383 \\ 
total models   & 100 & 531 \\
model parameters  &  1B-4B  & $<$ 1B \\
compilation method & manual (partially) & automatic \\
ground-truth verification & higher & lower 
\end{tabular}
\end{center}
\end{table}

%
%
\paragraph{Framework.}

Our default evaluation framework assumes extended model provenance testing (refer to Algorithm~\ref{alg:unknown_parent_tester}) as it allows to test for all parents at once.
For sanity check, we also include an alternative framework, that considers the case where one parent is suspected (based on Algorithm~\ref{alg:basic_tester}), and this is discussed at the end of Section~\ref{sec:eval:basic} and evaluated in Appendix~\ref{sec:appendix:known_parent}. 
We use the common value for significance parameter $\alpha=0.05$. Sampling of prompts is described in Appendix~\ref{sec:appendix:sampling_prompts}.

\paragraph{Selection of control set.}
%
In all of our provenance tests, we use the complete set of available pre-trained models from the benchmark as control models - 10 models for \bencho and 57 for \bencht. This selection was done to demonstrates that effective control sets can be constructed without careful manual curation or domain-specific analysis.
%
Specifically, we make no effort to align control models with particular parent models' domains or capabilities. We neither analyze the outputs of parent models $f$ nor attempt to match control models to specific use cases. Instead, we simply include all pre-trained models that rank among the most popular on the Hugging Face platform. This sampling approach, while simple, helps avoid introducing obvious selection bias while ensuring broad coverage of model types and capabilities.
%
This straightforward selection strategy allows us to evaluate whether provenance testing can be effective even without carefully chosen control sets. 
%

\teo{How does the precision and recall change with the significance level?}
\ivica{did not test this}

\begin{comment}
\begin{figure}[h]
  \includegraphics[width=8cm]{plots/precision_recall_large.png}
  \caption{Precision and recall for \bencho.}
  \label{fig:eval:accuracy-a}
\end{figure}

\begin{figure}[h]
\includegraphics[width=8cm]{plots/precision_recall_small.png}
\caption{Precision and recall for \bencht.}
\label{fig:eval:accuracy-b}
\end{figure}
\end{comment}


\begin{figure}[h]
    \centering
    \subfigure%
    {
        \includegraphics[width=8cm]{plots/precision_recall_large.png}
        \label{fig:eval:accuracy-a}
    }
    \subfigure%
    {
        \includegraphics[width=8cm]{plots/precision_recall_small.png}
        \label{fig:eval:accuracy-b}
    }
    \caption{Precision and recall of the model provenance tester with different number of prompts on \bencho (top) and \bencht (bottom). }
    \label{fig:eval:accuracy}
\end{figure}


\subsection{Accuracy of Model Provenance Tester}
\label{sec:eval:basic}

We evaluate the accuracy of the provenance tester by examining its performance on both \bencho and \bencht under different numbers of prompts. 
Figure~\ref{fig:eval:accuracy} shows
the precision and recall results from these experiments. The tester demonstrates similar performance patterns on both benchmarks, with slightly better results on \bencho.

The precision is notably high (approximately $0.95$) when the tester uses up to $1,000$ prompts.
Interestingly, however, the precision reduces as the number of prompts (test samples) increases. This is in direct contrast to common hypothesis testing, where larger sample size leads to smaller standard errors, thus higher precision. 
We get different results because our model provenance tester relies on detecting similarities of models. When using a smaller number of prompts, it can detect only the stronger similarities which are usually due to model provenance. However, as we increase the prompts, it starts detecting similar models that not necessarily have provenance relation. This leads to misclassification and reduced precision.

The recall behavior shows an opposite trend - it improves with a larger number of prompts, eventually reaching $80\%-90\%$ depending on the benchmark. This follows expected behavior: more prompts increase the statistical power of our hypothesis tests, enabling detection of small but significant differences in similarities. This increased sensitivity leads to higher recall rates, as the tester can detect more subtle provenance relationships that might be missed with fewer prompts.

We also examine the impact the randomness of  prompt sampling on the tester's accuracy. We conduct experiments on both benchmarks using five different randomly sampled sets of $1,000$ prompts\footnote{We chose smaller number of prompts due to larger computation effort required to complete five full runs of both benchmarks. The running time is completely dominated by producing outputs from the models, which in theory is parallelizable, but in our case it was not due to limited GPU resources.}, with the same set of prompts used in all testers, and record the precision and recall for each run -- see Table~\ref{tab:eval:random_sampling} of Appendix~\ref{sec:appendix:tables}. The results show that these values vary by $1-4\%$ between runs, indicating consistent performance across different prompt samples.


\custombox{1}{\newtext{
%
Our model provenance tester demonstrates high accuracy across different benchmarks, achieving precision of $90\%-95\%$ and recall of $80\%-90\%$ with $3,000$ prompts per model. 
Simply increasing the number of prompts does not guarantee uniformly better results, reflecting a fundamental trade-off: gains in recall might be accompanied by losses in precision.
}}


The evaluations above are in the default framework, which assumes no candidate parent is given in each provenance test. 
We run similar experiments when the candidate parent is given  in Appendix~\ref{sec:appendix:known_parent}. This is an easier problem (to make a wrong prediction, one needs not only to have conclusive hypothesis test that output a wrong parent, but also it should match the candidate parent), and the results confirm this: the recall of the tester in this framework is similar to the recall on the default framework, whereas the precision is very close to $100\%$. 



\subsection{Correctness of Assumptions}
\label{sec:eval:assumptions}

As discussed in Section~\ref{sec:analysis}, our approach relies on two key assumptions. While the high accuracy demonstrated in the previous section indirectly validates these assumptions, we provide here a detailed experimental analysis of both.

Our first assumption posits that derived models maintain significant similarity to their parent models. To evaluate this, we analyzed the similarity rankings across all provenance tests using $3,000$ prompts. For each derived model, we examined where its true parent ranked among all models in terms of similarity ratio $\mu$. The results strongly support this: in \bencho, the true parent had the highest similarity ratio in $93\%$ of cases, while in \bencht this occurred in $89\%$ of cases. When considering whether parents ranked in the top $50$th percentile by similarity, these percentages increased to $98\%$ and $96\%$ respectively. Thus we can conclude that our experiments indicate that derived models do indeed maintain strong similarity patterns with their parent models. Inadvertently, we have shown as well that with $3,000$ prompts the tester almost approaches the statistical limit (only the model with highest similarity ratio can be identified as a parent), as the recalls are very close to the percentages of highest similarity ($89\%$ recall vs. $93\%$ highest parent similarity, and $82\%$ recall vs. $89\%$ similarity, for the two benchmarks, respectively).

\custombox{2}{\newtext{The assumption that derived models show significant similarity to their parent models is valid for most provenance pairs.}}

%

Our second assumption concerns whether control models can effectively establish a baseline for similarity between unrelated models. We stress that in our experiments we have chosen the control models to be simply the set of all pre-trained models in an unbiased way, without any special selection or optimization for particular parent models they are tested against. 
We empirically observe that such unbiased selection of control model establishes a good baseline similarity as evident from the accuracy results presented thus far.

We further examine how the size and quality of the set of control models might affect tester accuracy. We conducted experiments varying the size of the control set while keeping other parameters constant ($3,000$ prompts per test). We randomly sampled different-sized subsets from our full control sets ($10$ models for \bencho and $57$ for \bencht) and ran $100$ complete benchmark tests for each size, and averaged the outcomes. The results, shown in Figure~\ref{fig:eval:assumption_control}, reveal that both size and quality of the control set significantly impact tester performance. 
%
For \bencho, even with just $4$ control models, the tester achieved $55\%$ precision. This relatively good performance with few controls can be attributed to \bencho consisting entirely of general-purpose, well-trained LLMs - thus any subset of these models provides a reasonable baseline for measuring similarity between unrelated models. However, for \bencht, the randomly sampled $4$-model control set yielded less than $10\%$ precision. This poor performance stems from \bencht containing a much more diverse set of models, including domain-specific models (e.g., for medical or coding tasks) and smaller models with varying capabilities. With such diversity, a small random subset of control models is unlikely to establish good baselines for all test cases - for instance, when testing a coding-focused model, we need coding-related models in the control set to establish proper baselines\footnote{Note that in practice, unlike our random sampling experiments, one can deliberately select control models matching the domain and capabilities of the suspected parent model, thus reducing significantly the impact of size of control sets, and leaving quality of the control set as the main factor on efficiency of the tester.}. Performance improves steadily as control set size increases in both benchmarks, since larger control sets are more likely to include appropriate baseline models for each test case. 

\begin{figure}[t]
    \centering
    \subfigure
    {
        \includegraphics[width=8cm]{plots/assumption_control_large.png}
        %
    }
    \subfigure
    {
        \includegraphics[width=8cm]{plots/assumption_control_small.png}
        %
    }
    \caption{Precision and recall of \bencht (top) and \bencho (bottom) with respect to smaller control set size. }
    \label{fig:eval:assumption_control}
\end{figure}



\custombox{3}{\newtext{The tester's performance significantly degrades when the control set is too small or poorly selected.
}}




\subsection{Reducing Query Complexity}
\label{sec:eval:online}

%
Certain pre-trained models from \bencho and \bencht exhibit a high degree of similarity when comparing their output tokens generated from random prompts.
Table~\ref{tab:eval:similarity_pre_train} of Appendix~\ref{sec:appendix:tables} presents the top $5$ most similar model pairs from \bencht, measured by the percentage of matching output tokens when tested on 1,000 random prompts (column $k=1$).

To reduce the online complexity, we implement an advanced rejection prompt sampling strategy as detailed in Section~\ref{sec:query}. 
%
%
We evaluate this strategy using different parameter values $k=4,16,$ and $64$ (recall, $k$ defines how many random samples are used to produce one selected sample), comparing it to the standard provenance testing without rejection ($k=1$).

Table~\ref{tab:eval:similarity_pre_train} demonstrates how the percentage of matching tokens changes with rejection sampling (columns $k=4,16,$ and $64$). For example, the most similar pair of models shows a reduction in matching output tokens from $64\%$ ($k=1$) to merely $16\%$ ($k=64$), indicating that rejection sampling significantly reduces token overlap between models. This improvement directly enhances the efficiency of provenance testing by reducing the tester's online complexity.



\begin{figure}[t]
  \includegraphics[width=8cm]{plots/recall_online_small.png}
  \caption{Recall for \bencht with different values of advanced prompt sampling defined with $k$.}
  \label{fig:eval:online_recall}
\end{figure}
Figure~\ref{fig:eval:online_recall} compares the tester's recall across different values of $k$. Notable improvements are visible even at $k=4$, with higher values of $k$ showing better results (though with diminishing returns). Specifically, the recall achieved with $1,000$ prompts at $k=1$ can be matched using only about $250$ prompts at $k=64$, representing a four-fold reduction in online complexity. %
Figure~\ref{fig:eval:online_full} provides
a comprehensive comparison between $k=1$ and $k=64$ for both precision and recall across both benchmarks, using $4-5$ times fewer queries for $k=64$ 
(note, in Figure~\ref{fig:eval:online_full} the number of prompts for $k=64$ are given at the top of the plots). 
The results demonstrate that the tester maintains its effectiveness despite the significant reduction in queries to the tested models. 
For example, advanced prompt sampling achieves high levels of $90-95$\% precision and $80-90$\% recall while reducing the required number of prompts from $3,000$ to just $500$ per model.
\begin{figure}[t]
    \centering
    \subfigure
    {
        \includegraphics[width=8cm]{plots/online_full_large.png}
    }
    \subfigure
    {
        \includegraphics[width=8cm]{plots/online_full_small.png}
    }
    \caption{Comparison of precision/recall for \bencht (top) and \bencht (bottom) when advanced online prompt sampling with $k=64$  uses four times less prompts than no advanced sampling ($k=1$). }
    \label{fig:eval:online_full}
\end{figure}



%
%


We next evaluate strategies for reducing offline complexity, which refers to the number of queries made to pre-trained models during testing. We implement this reduction using BAI, as described in Section~\ref{sec:query} and given in Algorithm~\ref{alg:offline_mab}.
%
We test this approach on both benchmarks by setting a target budget of $T$ queries (prompts) per pre-trained model. For example, with $T=1000$ on \bencho, which contains $10$ pre-trained models, the BAI-based provenance tester has a maximum budget of $10 \cdot 1,000 = 10,000$ total queries to make its decision.

Table~\ref{tab:eval:bai} compares the performance of the base tester and the BAI-enhanced version across different query budgets $T\in \{500,1000,2000\}$. The results show that the BAI tester successfully reduces offline complexity by $10\%-30\%$ (as shown in the ``avg queries'' column). However, this reduction comes at a significant cost to recall, while precision remains largely unchanged.
%
For instance, with $T=1,000$ on \bencho, BAI reduces the average number of queries from $1,000$ to $605$, but recall drops from $0.86$ to $0.63$. Similarly, on \bencht, the average queries decrease from $1,000$ to $809$, but recall falls from $0.68$ to $0.42$. This pattern persists across different values of $T$ and both benchmarks, suggesting that the trade-off between query reduction and recall preservation is not favorable in most cases.



\custombox{4}{\newtext{The online query optimization strategy leads to a 4-6 fold query reduction without accuracy drop, whereas the offline approach performs only marginally better and has a negative impact on recall.}}

\begin{table}[h]
\setlength{\tabcolsep}{5pt}
%
\begin{center}
\caption{Precision and recall of the base vs BAI tester on \bencho and \bencht.}
\label{tab:eval:bai}
\small
\begin{tabular}{r|c|c|r|r|r}
\hline
allowed & benchmark & tester & avg  & precision & recall \\
queries $T$ &  &  & queries &  &  \\
\hline
500         & \bencho   & base   &  500    &  1.00  &  0.81 \\
500         & \bencho   & BAI    &  450    &  0.98  &  0.29 \\
\hline
500         & \bencht   & base   &  500    &  0.95  &  0.56 \\
500         & \bencht   & BAI    &  452    &  0.98  &  0.29 \\
\hline
1,000       & \bencho   & base   & 1,000   &  0.99  &  0.86 \\
1,000       & \bencho   & BAI    &   605   &  1.00  &  0.63 \\
\hline
1,000       & \bencht   & base   & 1,000   &  0.94  &  0.68 \\
1,000       & \bencht   & BAI    &   809   &  0.98  &  0.42 \\
\hline 
2,000       & \bencho   & base   & 2,000   &  0.98  &  0.89 \\
2,000       & \bencho   & BAI    & 1,482   &  0.97  &  0.54 \\
\hline 
2,000       & \bencht   & base   & 2,000   &  0.92  &  0.77 \\
2,000       & \bencht   & BAI    & 1,482   &  0.97  &  0.54 \\
\hline
 \end{tabular}
\end{center}
\end{table}

%

