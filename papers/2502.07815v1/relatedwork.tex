\section{Related work}
\label{sec:Literature review 
}
Detecting sensitive information such as Personally Identifiable Information (PII) and Protected Health Information (PHI) has been widely studied in the domains of data security, natural language processing (NLP), and pattern-matching techniques. Existing approaches primarily fall into three categories: regex-based detection, AI-driven Named Entity Recognition (NER), and hybrid methods combining both.

\subsection{Regex-Based Detection}
Regular expressions (regex) have long been used for pattern-based text matching, forming the backbone of many data loss prevention (DLP) systems \cite{garfinkel2022differential}. Popular regex engines such as PCRE, Google RE2, and Hyperscan have been benchmarked for efficiency in large-scale text scanning \cite{venkateshliterary}. While regex-based approaches offer deterministic accuracy and speed, they struggle with pattern generalization and require frequent updates to accommodate evolving data structures. Furthermore, regex engines like PCRE suffer from backtracking issues, leading to unpredictable execution times \cite{van2022investigation}.

\subsection{AI-Driven Named Entity Recognition}
Recent advancements in NLP have enabled deep learning-based NER models to identify sensitive entities beyond strict pattern matching. Models such as BERT \cite{kenton2019bert} and spaCy's NER \cite{honnibal2020spacy} have demonstrated strong recall in detecting complex entities across diverse linguistic contexts. However, AI-based approaches introduce challenges such as higher computational costs, false positives, and the need for extensive labeled datasets \cite{li2020survey}.

\subsection{Hybrid AI + Regex Approaches}
Several studies have explored hybrid methods that combine regex with machine learning for enhanced detection accuracy. Souza et al. \cite{souza2024combining} proposed an approach where regex serves as a pre-filtering mechanism, followed by an AI model to refine entity classification. Similarly, Friebely et al. \cite{friebely2022analyzing} demonstrated that integrating regex with deep learning improves precision while maintaining high recall, making such approaches more suitable for real-time applications.

\subsection{Comparative Benchmarking and Efficiency}
Prior research has also focused on benchmarking various detection techniques for performance and scalability. Hyperscan has been identified as the fastest regex engine but comes with hardware constraints \cite{wang2019hyperscan}. Meanwhile, RE2 has been praised for its balance between speed and memory efficiency, making it a practical choice for large-scale deployments \cite{pike2010re2}. AI-based solutions, while powerful, tend to be resource-intensive and less predictable in execution time \cite{strubell2020energy}.

\subsection{Contributions of This Work}
While previous studies have explored regex, AI-based NER, and hybrid detection models separately, our work systematically benchmarks these approaches under real-world conditions. We evaluate regex engines such as RE2, PCRE, and Hyperscan alongside AI-driven detection methods to identify an optimal balance between accuracy, speed, and scalability. By integrating regex with AI in a hybrid model, we achieve improved detection accuracy while maintaining computational efficiency, making our approach well-suited for large-scale data security applications.