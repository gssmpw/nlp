\section{Related Work}
Our work belongs to the stream of research on \emph{perpetual}, or {\em temporal voting} \cite{lackner2020perpetual,lackner2023proportionalPV,elkind2024temporalelections}.
This line of work started by considering temporal extensions of popular multiwinner voting rules and simple axiomatic properties.
Bulteau et al.~\shortcite{bulteau2021jrperpetual} were the first to adapt notions of proportional representation to the temporal setting. They 
consider several temporal variants of JR and PJR, and 
show that a JR outcome can be computed in polynomial time even for the most
demanding notion of JR among the ones they consider; for PJR, 
they prove the existence of an outcome satisfying the axiom, but their
proof, while constructive, is based on an exponential-time algorithm. Page et al.~\shortcite{page2021electing} proposed very similar concepts in the context of electing the executive branch. This work is extended by Chandak et al.~\shortcite{chandak23}, who also propose a temporal variant of EJR, 
and show that several multiwinner voting rules
that satisfy PJR/EJR in the standard model can be extended
to the temporal setting so as to satisfy temporal variants of PJR/EJR.
Bredereck et al.~\shortcite{bredereck2020successivecommittee,bredereck2022committeechange} look at sequential committee elections whereby 
an entire committee is elected in each round, and impose
constraints on the extent a committee can change, whilst ensuring that the candidates retain sufficient support from the electorate.
Zech et al.~\shortcite{zech2024multiwinnerchange} study a similar model, but focus on the class of Thiele rules.
Elkind et al.~\shortcite{elkind2024temporal} offer a systematic review of recent work 
on temporal voting.

The temporal voting model captures apportionment with approval preferences \cite{brill2024partyapportionment,delemazure2022spelection}, by endowing the voters with preferences that do not change over time.
Other models in the social choice literature that include temporal elements include
public decision-making \cite{alouf2022better,conitzer2017fairpublic,fain2018publicgoods,skowron2022proppublic,lackner2023freeriding,MPS23a,neoh2025strategicchores}, scheduling \cite{elkind2022temporalslot,patro2022virtualconf}, resource allocation over time \cite{bampis2018fairtime,allouah2023fairallocationtime,elkind2024temporalfairdivision}, online committee selection \cite{do2022onlinecommittee}, and dynamic social choice \cite{parkes2013dynamicsocialchoice,freeman2017dynamicsocialchoice}.

Lackner et al.~\shortcite{lackner2021longtermpb} propose a framework for studying long-term participatory budgeting, and study fairness considerations in that setting.