\subsection{How do these factors collectively influence the preference on valuation of contrastive and selective explanations?}
Considering all factors including individual-, context-, and explanation-dependent factors, we investigate the interplay of factors in affecting the valuation of explanations.
As described in Section~\ref{sec:stat}, structural equation models (SEM) are used to test the effects of various factors in explaining whether a participant prefers a particular explanation or not for five explanatory values. Fig.~\ref{fig:explanation-analysis} provides a summary of the significant factors identified by the path analysis for each explanatory strategy. For predicting five explanatory values in each of the explanation styles (e.g., \cbhe), the estimated standardized effects of the significant factors are shown with a 95\% confidence interval.

\begin{figure*}[t]%
\centering
\includegraphics[width=\textwidth]{figures/SEM-all-updated.pdf}
\caption{Path analysis results for each explanation strategy. Factors significantly influencing the valuation process of each explanation strategy identified in per-strategy path analysis are highlighted.}\label{fig:explanation-analysis}
\end{figure*}

\textbf{Overview.} Our path analysis reveals a combination of direct and indirect effects of individual-, context-, and explanation-dependent factors on the valuation process. Across the results of path analysis over six strategies, the most impactful factor was \textit{germane} load, directly impacting all explanatory values. Conversely, for \cf and \cbho, \textit{intrinsic} load was significantly higher in high-stakes and professional contexts (\loanN, \mediN, \mediP). 

Sociotechnical contexts also significantly impacted the valuation of explanations either directly or indirectly through cognitive engagement. In those cases, contexts were primarily high-stakes or professional, affecting the explanatory values indirectly by invoking people to perceive higher motivation, lower opportunity, or lower ability (in the high-stakes or highly professional contexts), while also directly influencing valuation due to given contexts. 

For individual characteristics, some demographic traits were significantly associated with decision-making styles (Male $\rightarrow$ Rational (***0.19\footnote{Standardized path coefficient with statistical significance (*p $\leq$ 0.05, **p $\leq$ 0.01, ***p $\leq$ 0.001)}); Younger $\rightarrow$ Dependent(**0.12) and Avoidant (***0.14); HigherEdu $\rightarrow$ Dependent (**0.10) and Intuitive (**-0.12)). The decision-making styles, in turn, tended to influence cognitive engagement. For instance, rational thinkers tended to demonstrate higher motivation (*0.07), opportunity (***0.12), and ability (*0.07), different than avoidant thinkers who tended to have lower ability (**-0.11). These associations together formulate indirect effects of demographic traits towards cognitive engagement via demographic styles. For example, younger individuals, often avoidant, tended to have lower perceived ability. On the other hand, extraneous and germane loads were not significantly associated with individual characteristics.  

\textbf{Strategy-specific characteristics of valuation process.} The SEMs fitted to each explanation strategy provides insights into how the valuation process differs across strategies.

First, \comp explanations' values were mostly shaped by positive direct effects of individual and context-dependent factors. For example, \comp was found more understandable for users with higher education or within medical decision contexts (\mediP, \mediN) requiring advanced professional knowledge. Conversely, \ctt was the counterpart of \comp, as indicated by negative direct associations in a professional context (\mediP $\rightarrow$ Sufficiently-detailed; **-0.11) and positive associations in a low-stakes context (\recomN $\rightarrow$ Preference; **0.12). 

For \cf explanations, both indirect and direct effects of context-dependent factors were notable. \cf was highly valued in \loanN across four explanatory values and in high-stakes contexts due to their high intrinsic load (\mediN, \mediP, \drivN $\rightarrow$ Intrinsic; *0.11, ***0.18, ***0.17, Intrinsic $\rightarrow$ Preference; *0.06), or when individuals have higher perceived ability (Ability $\rightarrow$  Understandable, Sufficiently-detailed; *-0.09, *-0.07). The valuation process of \cbho and \cbhe were the opposite of \cf's. \cbho was less valued directly in \loanN context or indirectly in contexts associated with higher \textit{intrinsic} load (Intrinsic $\rightarrow$ Preference; ***-0.11). \cbhe was positively valued as useful when individuals have lower perceived ability (Ability $\rightarrow$ Useful; ***-0.1).