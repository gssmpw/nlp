%%
%% This is file `sample-sigconf-authordraft.tex',
%% generated with the docstrip utility.
%%
%% The original source files were:
%%
%% samples.dtx  (with options: `all,proceedings,bibtex,authordraft')
%% 
%% IMPORTANT NOTICE:
%% 
%% For the copyright see the source file.
%% 
%% Any modified versions of this file must be renamed
%% with new filenames distinct from sample-sigconf-authordraft.tex.
%% 
%% For distribution of the original source see the terms
%% for copying and modification in the file samples.dtx.
%% 
%% This generated file may be distributed as long as the
%% original source files, as listed above, are part of the
%% same distribution. (The sources need not necessarily be
%% in the same archive or directory.)
%%
%%
%% Commands for TeXCount
%TC:macro \cite [option:text,text]
%TC:macro \citep [option:text,text]
%TC:macro \citet [option:text,text]
%TC:envir table 0 1
%TC:envir table* 0 1
%TC:envir tabular [ignore] word
%TC:envir displaymath 0 word
%TC:envir math 0 word
%TC:envir comment 0 0
%%
%%
%% The first command in your LaTeX source must be the \documentclass
%% command.
%%
%% For submission and review of your manuscript please change the
%% command to \documentclass[manuscript, screen, review]{acmart}.
%%
%% When submitting camera ready or to TAPS, please change the command
%% to \documentclass[sigconf]{acmart} or whichever template is required
%% for your publication.
%%
%%
\documentclass[sigconf]{acmart}
\DeclareUnicodeCharacter{E904}{\textbf{?}}
%%
%% \BibTeX command to typeset BibTeX logo in the docs
\AtBeginDocument{%
  \providecommand\BibTeX{{%
    Bib\TeX}}}

%% Rights management information.  This information is sent to you
%% when you complete the rights form.  These commands have SAMPLE
%% values in them; it is your responsibility as an author to replace
%% the commands and values with those provided to you when you
%% complete the rights form.
\setcopyright{acmlicensed}
\copyrightyear{2018}
\acmYear{2018}
\acmDOI{XXXXXXX.XXXXXXX}

%% These commands are for a PROCEEDINGS abstract or paper.
\acmConference[Conference acronym 'XX]{Make sure to enter the correct
  conference title from your rights confirmation emai}{June 03--05,
  2018}{Woodstock, NY}
%%
%%  Uncomment \acmBooktitle if the title of the proceedings is different
%%  from ``Proceedings of ...''!
%%
%%\acmBooktitle{Woodstock '18: ACM Symposium on Neural Gaze Detection,
%%  June 03--05, 2018, Woodstock, NY}
\acmISBN{978-1-4503-XXXX-X/18/06}


%%
%% Submission ID.
%% Use this when submitting an article to a sponsored event. You'll
%% receive a unique submission ID from the organizers
%% of the event, and this ID should be used as the parameter to this command.
%%\acmSubmissionID{123-A56-BU3}

%%
%% For managing citations, it is recommended to use bibliography
%% files in BibTeX format.
%%
%% You can then either use BibTeX with the ACM-Reference-Format style,
%% or BibLaTeX with the acmnumeric or acmauthoryear sytles, that include
%% support for advanced citation of software artefact from the
%% biblatex-software package, also separately available on CTAN.
%%
%% Look at the sample-*-biblatex.tex files for templates showcasing
%% the biblatex styles.
%%

%%
%% The majority of ACM publications use numbered citations and
%% references.  The command \citestyle{authoryear} switches to the
%% "author year" style.
%%
%% If you are preparing content for an event
%% sponsored by ACM SIGGRAPH, you must use the "author year" style of
%% citations and references.
%% Uncommenting
%% the next command will enable that style.
%%\citestyle{acmauthoryear}

% \usepackage[dvipsnames,table,xcdraw,svgname]{xcolor}
% \usepackage[colorlinks=true, allcolors=blue]{hyperref}
% \usepackage{setspace}
\usepackage[backgroundcolor=white,textsize=tiny]{todonotes}
\newcommand{\smalltodo}[2][] 
    {\todo[size=\scriptsize,caption={#2}, #1]
    {\ttfamily  
    \begin{spacing}{0.5}#2\end{spacing}}} 
\usepackage{aliascnt}
\usepackage{tabularx, booktabs}
\usepackage{array}

\usepackage{natbib}
\usepackage{multirow}
\usepackage{xspace}
% \usepackage{array}% http://ctan.org/pkg/array
\usepackage{soul}
\usepackage{caption}
\usepackage{placeins}
\usepackage{url}
% \usepackage{lmodern}
% \renewcommand*{\ttdefault}{qcr}

\newcolumntype{L}[1]{>{\raggedright\let\newline\\\arraybackslash\hspace{0pt}}p{#1}}
\newcolumntype{C}[1]{>{\centering\let\newline\\\arraybackslash\hspace{0pt}}p{#1}}
\newcolumntype{R}[1]{>{\raggedleft\let\newline\\\arraybackslash\hspace{0pt}}p{#1}}


\newcommand{\ys}[1]{{\small\color{blue}{[YS: #1]}}}
\newcommand{\ysv}[1]{{\textcolor{blue}{#1}}}
\newcommand{\ma}[1]{{\small\textcolor{red}{[MA: #1]}}}
\newcommand{\ec}[1]{{\small\textcolor{orange}{[EC: #1]}}}
\newcommand{\todoo}[1]{{\small\textcolor{blue}{[ToDo: #1]}}}
\newcommand{\yrl}[1]{{\small\textcolor{orange}{[YRL: #1]}}}
\newcommand{\yrv}[1]{{\textcolor{brown}{#1}}}
\newcommand{\yrq}[1]{{\textcolor{red}{#1}}}
\newcommand{\yrnote}[1]{\smalltodo[linecolor=olive,backgroundcolor=white,bordercolor=purple]{YRL:#1}}
\newcommand{\tofill}[1]{{\textcolor{red}{#1}}}



\newcommand{\loanN}{{\sf \small \color{black} Loan-}\xspace}
\newcommand{\drivN}{{\sf \small \color{black} Driv-}\xspace}
\newcommand{\mediP}{{\sf \small \color{black} Medi+}\xspace}
\newcommand{\mediN}{{\sf \small \color{black} Medi-}\xspace}
\newcommand{\recomP}{{\sf \small \color{black} Recom+}\xspace}
\newcommand{\recomN}{{\sf \small \color{black} Recom-}\xspace}

\newcommand{\comp}{{\sf \small \color{black} Complete}\xspace}
\newcommand{\cbhe}{{\sf \small \color{black} Case-based (hetero)}\xspace}
\newcommand{\cbho}{{\sf \small \color{black} Case-based (homo)}\xspace}
\newcommand{\cto}{{\sf \small \color{black} Contrastive (o)}\xspace}
\newcommand{\ctt}{{\sf \small \color{black} Contrastive (t)}\xspace}
\newcommand{\ct}{{\sf \small \color{black} Contrastive}\xspace}
\newcommand{\cf}{{\sf \small \color{black} Counterfactual}\xspace}

\newcommand{\cbhelb}{{\tt \color{black} Case-based} \\ {\tt \color{black}(hetero)}\xspace}
\newcommand{\cbholb}{{\tt \color{black} Case-based} \\ {\tt \color{black}(homo)}\xspace}

\newcommand{\darkgray}[1]{{\textcolor[HTML]{8d8d8d}{#1}}}
\newcommand{\orange}[1]{{\textcolor{orange}{#1}}}
\newcommand{\darkorange}[1]{{\textcolor[HTML]{cd5f00}{#1}}}
\newcommand{\green}[1]{{\textcolor[HTML]{679092}{#1}}}
\newcommand{\darkgreen}[1]{{\textcolor[HTML]{006160}{#1}}}
\newcommand{\darkpurple}[1]{{\textcolor[HTML]{4b39ce}{#1}}}

\newcommand{\app}{{\textcolor{blue}{\it Appendix} (Section~\ref{sec:appendix})}\xspace}
% \newcommand{\app}[1]{{\textcolor{blue}{{\it Appendix} (Section~\ref{sec:appendix})}}\xspace}
\newcommand{\appsec}[1]{{\textcolor{blue}{{\it Appendix}~}(Section~\ref{#1})}\xspace}


\usepackage[flushleft]{threeparttable}
\usepackage{booktabs}
\usepackage{multirow}
\usepackage[normalem]{ulem} %% for strickethrough
\usepackage{nicematrix}
\usepackage{colortbl}
\usepackage{graphicx}
% \usepackage{ulem}

%%
%% end of the preamble, start of the body of the document source.
\begin{document}

%%
%% The "title" command has an optional parameter,
%% allowing the author to define a "short title" to be used in page headers.
\title[Human-centered explanation does not fit all]{Human-centered explanation does not fit all: The Interplay of sociotechnical, cognitive, and individual factors in the effect of AI explanations in algorithmic decision-making}

%%
%% The "author" command and its associated commands are used to define
%% the authors and their affiliations.
%% Of note is the shared affiliation of the first two authors, and the
%% "authornote" and "authornotemark" commands
%% used to denote shared contribution to the research.
\author{Yongsu Ahn}
\email{yongsu.ahn@pitt.edu}
\affiliation{%
  \institution{University of Pittsburgh}
  \city{Pittsburgh}
  \state{Pennsylvania}
  \country{USA}
}

\author{Yu-Ru Lin}
\email{yurulin@pitt.edu}
\affiliation{%
  \institution{University of Pittsburgh}
  \city{Pittsburgh}
  \state{Pennsylvania}
  \country{USA}
}

\author{Malihe Alikhani}
\email{m.alikhani@northeastern.edu}
\affiliation{%
  \institution{Northeastern University}
  \city{Boston}
  \state{Massachusetts}
  \country{USA}
}

\author{Eunjeong Cheon}
\email{echeon@syr.edu}
\affiliation{%
  \institution{Syracuse University}
  \city{Syracuse}
  \state{New York}
  \country{USA}
}

%%
%% By default, the full list of authors will be used in the page
%% headers. Often, this list is too long, and will overlap
%% other information printed in the page headers. This command allows
%% the author to define a more concise list
%% of authors' names for this purpose.
\renewcommand{\shortauthors}{ Ahn et al.}

%%
%% The abstract is a short summary of the work to be presented in the
%% article.
\begin{abstract}
  Recent XAI studies have investigated what constitutes a \textit{good} explanation in AI-assisted decision-making. Despite the widely accepted human-friendly properties of explanations, such as contrastive and selective, existing studies have yielded inconsistent findings. To address these gaps, our study focuses on the cognitive dimensions of explanation evaluation, by evaluating six explanations with different contrastive strategies and information selectivity and scrutinizing factors behind their valuation process. Our analysis results find that contrastive explanations are not the most preferable or understandable in general; Rather, different contrastive and selective explanations were appreciated to a different extent based on who they are, when, how, and what to explain -- with different level of cognitive load and engagement and sociotechnical contexts. Given these findings, we call for a nuanced view of explanation strategies, with implications for designing AI interfaces to accommodate individual and contextual differences in AI-assisted decision-making.
\end{abstract}

%%
%% The code below is generated by the tool at http://dl.acm.org/ccs.cfm.
%% Please copy and paste the code instead of the example below.
%%
\begin{CCSXML}
<ccs2012>
 <concept>
  <concept_id>00000000.0000000.0000000</concept_id>
  <concept_desc>Do Not Use This Code, Generate the Correct Terms for Your Paper</concept_desc>
  <concept_significance>500</concept_significance>
 </concept>
 <concept>
  <concept_id>00000000.00000000.00000000</concept_id>
  <concept_desc>Do Not Use This Code, Generate the Correct Terms for Your Paper</concept_desc>
  <concept_significance>300</concept_significance>
 </concept>
 <concept>
  <concept_id>00000000.00000000.00000000</concept_id>
  <concept_desc>Do Not Use This Code, Generate the Correct Terms for Your Paper</concept_desc>
  <concept_significance>100</concept_significance>
 </concept>
 <concept>
  <concept_id>00000000.00000000.00000000</concept_id>
  <concept_desc>Do Not Use This Code, Generate the Correct Terms for Your Paper</concept_desc>
  <concept_significance>100</concept_significance>
 </concept>
</ccs2012>
\end{CCSXML}

\ccsdesc[500]{Do Not Use This Code~Generate the Correct Terms for Your Paper}
\ccsdesc[300]{Do Not Use This Code~Generate the Correct Terms for Your Paper}
\ccsdesc{Do Not Use This Code~Generate the Correct Terms for Your Paper}
\ccsdesc[100]{Do Not Use This Code~Generate the Correct Terms for Your Paper}

%%
%% Keywords. The author(s) should pick words that accurately describe
%% the work being presented. Separate the keywords with commas.
\keywords{AI explanations, Explainable AI, Algorithmic decision-making, AI-assisted decision-making, Counterfactual explanations, Contrastive explanations, Decision-making style, Cognitive load, Demographic traits}
%% A "teaser" image appears between the author and affiliation
%% information and the body of the document, and typically spans the
%% page.
% \begin{teaserfigure}
%   \includegraphics[width=\textwidth]{sampleteaser}
%   \caption{Seattle Mariners at Spring Training, 2010.}
%   \Description{Enjoying the baseball game from the third-base
%   seats. Ichiro Suzuki preparing to bat.}
%   \label{fig:teaser}
% \end{teaserfigure}

\received{20 February 2007}
\received[revised]{12 March 2009}
\received[accepted]{5 June 2009}

%%
%% This command processes the author and affiliation and title
%% information and builds the first part of the formatted document.
\maketitle

\section{Introduction}
\label{sec:intro}

% Artificial intelligence (AI) has made a significant impact on our daily lives, from personalized purchase discounts to media recommendations. As there is a growing concern that AI systems may generate unfair or harmful decisions, it has become increasingly imperative to ensure transparency and explainability in the decision-making process \cite{lipton2018mythos}. While the field of eXplainable AI (XAI) has introduced various explanation strategies to enhance transparency, recent studies have shifted the focus towards more user-centered explainability and raised a crucial question: what makes an explanation \textit{good} in AI-driven decision-making?

% Introduced by \cite{miller2019explanation}, the properties of human-friendly explanations -- selective, contrastive, and conversational -- brought from findings in social and cognitive science have gained much attention. Among them, contrastive and selective explanations, which provide a minimal set of causes against the opposite case or focus on relevant information, have been widely accepted to a number of studies ranging from algorithmic methods to empirical studies.

% \begin{figure}[t]%
% \centering
% \includegraphics[width=0.7\columnwidth]{figures/teaser.pdf}
% \vspace{-1em}
% \caption{The process of humans' valuation of explanations. This study explores the interplay of \darkgray{individual-}, \orange{context-}, and \darkgreen{explanation-dependent} factors influencing the \darkpurple{valuation} of explanation strategies.}
% \label{fig:teaser}
% \end{figure}


% Despite the theoretical foundations that advocate for intuitive properties of such explanations, recent studies have yielded inconsistent findings. For instance, contrastive explanation was helpful in improving decision quality in emotion recognition task \cite{RelatableExplainableAIPerceptualProcess}, while they failed to calibrate the trust of AI in human perception in \cite{AreExplanationsHelpfulComparative}. Similarly, while comprehensive and detailed explanations have been found to facilitate an understanding of AI processes \cite{TooMuchTooLittlea, ExplainableAutonomyStudyExplanation}, they can also lead to over-reliance on AI-generated decisions \cite{RoleExplanationsTrustReliance} and pose a threat to user trust \cite{HowMuchInformationEffects}. These findings raise questions about whether the effect of contrastive and selective explanations, widely accepted as intuitive and human-friendly, really hold true or in what contexts they may be more beneficial than others.

% We argue that this misalignment can be attributed to a lack of comprehensive scrutiny of the underlying factors that shape laypeople’s valuation process over explanations. Studies in social and cognitive science have found that cognitive and individual differences, such as cognitive load and thinking styles, are recognized as significant determinants of the efficacy of explanations \cite{PartneringPeopleDeepLearning, SpanishvalidationGeneralDecisionMakingStyle, Doeshighereducationhonecognitive, Domainspecificpreferencesintuitiondeliberationdecision}. While recent XAI literature \cite{CapturingTrendsApplicationsIssues, ScienceHumanAIDecisionMakingSurvey, ehsan2023charting} have conceptually examined that AI transparency is entangled with sociotechnical contexts or individual differences, existing studies remain constrained by evaluating explanations within a single context and limited range of study subjects, failing to demonstrate how various factors can influence the impact of explanations.
  
% In response to these research gaps, our study shifts focus to the cognitive aspects of explanation evaluation in two aspects: 1) We evaluate explanation strategies that are widely accepted as intuitive -- contrastive and selective explanations, by conducting an experiment with laypeople to assess their preferences. Specifically, we explore six explanation styles from existing literature, differentiated by their use of contrastive strategies and information selectivity; 2) We further scrutinize how various factors -- individual, contextual, and cognitive factors, as identified through a review of cognitive and social science literature -- shape the valuation process. These factors are synthesized into a framework to explain when, why, and for whom specific explanation strategies are found to be preferable.

% In light of our analysis results, we challenge the assumption that selective and contrastive explanations are always intuitive or advantageous. Instead, we demonstrate that their effectiveness is highly context-dependent and varying depending on the context and the specific comparison or information being explained. Overall, our study provides evidence for a more context-sensitive and cognitively grounded approach to designing AI systems, highlighting the importance of considering contextual and cognitive factors in evaluating explanations. The contributions of this work are as follows:

% \begin{itemize}
%     \item The findings from our experiment contribute to a nuanced understanding of the efficacy of contrastive and selective explanations. We reveal that the perceived value of explanations is highly context-specific and varies among individuals, challenging the theory-driven properties of human-intuitive explanations.
%     \item We introduce a novel framework (Fig. \ref{fig:teaser}) designed to probe the process of human valuation of explanations. This framework enables us to scrutinize this intricate dynamics as a social and explanatory process synthesizing various factors—\darkgray{individual differences}, \orange{sociotechnical contexts}, and \darkgreen{cognitive aspects}— into the framework. While our study aimed to examine specific types of explanations, the framework is generally applicable to any other explanations in AI-assisted decision contexts.
%     \item Expanding upon our analysis, we provide pragmatic design implications for XAI interfaces to better accommodate individual and contextual differences. We suggest mechanisms to guide the generative process for explanations towards specific explanatory strategies or values, enhancing the usability and effectiveness of XAI systems.
% \end{itemize}
% \vspace{-0.5em}

Artificial intelligence (AI) has increasingly become integrated into everyday life, influencing decisions from personalized purchase recommendations to media content suggestions. However, concerns about AI's potential to produce unfair or harmful outcomes have highlighted the need for greater transparency and explainability in AI-driven decision-making \cite{lipton2018mythos}. In response, the field of eXplainable AI (XAI) has developed various explanation strategies aimed at making AI systems more transparent; However, a critical question still remains: what constitutes a {\it good} explanation in the context of AI decision-making?

Drawing from cognitive and social sciences, Miller \cite{miller2019explanation} introduced human-centered explanation properties---such as selective, contrastive, and conversational explanations---that have since gained considerable attention. Of particular interest are {\it contrastive} explanations, which highlight the differences between a decision and its alternatives, and {\it selective explanations}, which focus on the most relevant information. These strategies have been widely studied, from algorithmic methods to user-facing empirical studies.

\begin{figure}[t]%
\centering
\includegraphics[width=\columnwidth]{figures/teaser.pdf}
\caption{The process of humans' valuation of explanations. This study explores the interplay of \darkgray{individual-}, \orange{context-}, and \darkgreen{explanation-dependent} factors influencing the \darkpurple{valuation} of explanation strategies.}
\label{fig:teaser}
\end{figure}

However, despite their theoretical appeal, findings on these explanation types have been inconsistent. For instance, while contrastive explanations have been shown to improve decision quality in certain tasks like emotion recognition \cite{RelatableExplainableAIPerceptualProcess}, they have also failed to align AI trust with human perception in other contexts \cite{AreExplanationsHelpfulComparative}. Similarly, while detailed explanations can enhance users' understanding of AI processes \cite{TooMuchTooLittlea, ExplainableAutonomyStudyExplanation}, they also risk fostering over-reliance on AI decisions \cite{RoleExplanationsTrustReliance} and reducing trust in AI systems \cite{HowMuchInformationEffects}. These mixed findings raise the question: are contrastive and selective explanations universally beneficial, or do their advantages depend on specific contexts?

We argue that the inconsistencies in previous studies stem from insufficient attention to the cognitive and contextual factors that influence how users evaluate explanations. Research in cognitive science suggests that individual traits such as cognitive load and decision-making style play a critical role in how explanations are perceived and valued \cite{PartneringPeopleDeepLearning, SpanishvalidationGeneralDecisionMakingStyle}. Recent XAI literature has touched on the influence of sociotechnical contexts and user diversity on AI transparency \cite{CapturingTrendsApplicationsIssues, ScienceHumanAIDecisionMakingSurvey, ehsan2023charting}, but most studies are limited to single contexts or small sample sizes, preventing a comprehensive understanding of how these factors shape the impact of explanations.

To address these gaps, our study focuses on the cognitive dimensions of explanation evaluation. Specifically, we examine the following:
(1) We evaluate the effectiveness of widely accepted explanation strategies---contrastive and selective---through an experiment involving laypeople. By comparing six explanation styles drawn from existing literature, we assess how different approaches to contrast and selectivity influence user preferences.
(2) We explore how individual, contextual, and cognitive factors shape the evaluation process. Drawing on insights from cognitive and social science, we synthesize these factors into a comprehensive framework to explain when, why, and for whom specific explanation strategies are most effective.

\textbf{Study contexts.} While a variety of contexts in AI-assisted decision-making has been studied in existing literature, we particularly focus on the context of algorithmic decision-making that determines various aspects of individuals’ everyday lives, such as loan approval, medical diagnosis, autonomous driving, and movie recommendations. These contexts typically entail a passive nature of decisions made by algorithms embedded within AI systems or infrastructures in various socio-technical contexts, where decisions are notified or presented to data subjects, triggering their cognitive responses or emotions on the given decisions. This contrasts with the general or more objective nature of AI-assisted decision-making contexts, such as image/sentiment classification, where explanations can assist with better understanding of feature attribution. Due to this nature, the design of our study, which best approximates the context with written scenarios then asks subjective ratings of users, centers on exploring their perceived and cognitive responses, along with the various factors that shape them, particularly when explanations describe them, contain personal information, and contrast with others’.

Overall, our findings challenge the assumption that selective and contrastive explanations are universally intuitive or advantageous. Instead, we demonstrate that their effectiveness is highly dependent on context and user characteristics. This evidence supports the need for a more context-sensitive, user-centered approach to designing XAI systems. The key contributions of our study are as follows:
\begin{itemize}
\item {\bf Context-specific values of explanations:} Our study reveals that the perceived value of contrastive and selective explanations varies significantly by context and individual traits, challenging the assumption that these strategies are inherently human-friendly.
\item {\bf Novel framework for explanation evaluation:} We introduce a comprehensive framework (Fig. \ref{fig:teaser}) that captures the complex interplay of \darkgray{individual differences}, \orange{sociotechnical contexts}, and \darkgreen{cognitive aspects} in explanation valuation. While our study focuses on specific types of explanations, this framework can be applied broadly across various AI decision-making contexts.
\item {\bf Design implications for XAI:} We offer actionable insights for designing more interactive and personalized XAI systems. Our recommendations emphasize the need to adapt explanation strategies based on user characteristics and contextual factors, improving both the usability and trustworthiness of AI systems.
\end{itemize}

\section{Related Work}
\label{sec:related-work}

\subsection{Valuation of Explanation Strategies in AI-assisted Decision-making}
\label{sec:related-work-exp}

Evaluating post-hoc explanations in the form of text has gained much attention as it provides rationales on AI-based decisions and impacts users' perceived trust and understandability  \cite{vilone2020explainable, lipton2018mythos, ExplainableSoftwareAnalytics}. Such explanations incorporating a variety of strategies and logics vary by techniques \cite{AreExplanationsHelpfulComparative, adhikari2019leafage} including feature importance or nearest neighbors or logics/strategies such as contrastive \cite{miller2019explanation, ConversationalProcessesCausalExplanation, DeductiveApproachCausalInference}, counterfactual \cite{Contrastscounterfactualscauses, CounterfactualStoryReasoningGeneration, CounterfactualExplanationsOpeningBlack, PsychologicalStudiesCausalCounterfactualReasoning} explanations in social and cognitive science, and case-based explanations from expert system studies \cite{Similaritymeasuresattributeselectioncasebased, GainingInsightCasebasedExplanation, EvaluationUsefulnessCaseBasedExplanation, SurveyCBRApplicationAreas}. 

However, recent XAI research has produced contradictory findings regarding the effect of explanation strategies. For example, providing such explanations in AI-assisted decision were found to help justify the decision and increase users' trust when compared with no explanation provided in some contexts such as medical chatbot \cite{ExploringPromotingDiagnosticTransparency} or self-driving context \cite{TrustingXAIEffectsdifferenttypes}. On the other hand, they tend to be distracting or lead to either overreliance \cite{ExplanationsCanReduceOverrelianceAI, PartneringPeopleDeepLearning} or cognitive overload \cite{WaitWhyAssessingBehaviorExplanation, EvolutionCognitiveLoadTheoryMeasurement} for lay users without a careful design of how explanations are presented \cite{CheXplainEnablingPhysiciansExploreUnderstand}. 
The impact of different explanation strategies in AI-assisted decision-making is also diverging. For example, complete explanations with a greater information complexity, for example, were proved to gain trust better in medical diagnosis \cite{EffectExplanationStylesUser, TooMuchTooLittlea, HowMuchInformationEffects} while they led to over-reliance on AI in medical diagnosis in \cite{RoleExplanationsTrustReliance}. Contrastive explanations known as intuitive and human-friendly were helpful in improving decision quality with semantic evidence in emotion recognition task \cite{RelatableExplainableAIPerceptualProcess}, while it was found to fail to calibrate the trust of AI in human perception in \cite{AreExplanationsHelpfulComparative}.

A recent XAI study \cite{SelectiveExplanationsLeveragingHumanInput} has advanced the discussion on making explanations more selective. The proposed framework draws on literature exploring how humans produce explanations selectively, focusing on aspects such as abnormality \cite{hilton1986knowledge}, relevance \cite{woodward2006sensitive}, and changeability \cite{hilton2007course}. Empirical results from this study suggest that allowing users to contribute to the generation of explanations reduces over-reliance on automated systems.

In this study, we pursue a deeper and comprehensive understanding of users' valuation on explanations strategies. By conducting a survey-based experiment, we find that the valuation process is a complex interplay of individual and contextual factors, highlighting the importance of personalizing the degree of explainability carefully based on individual traits and cognitive abilities.


\subsection{Individual and Cognitive Dimension of Explanatory Process}
\label{sec:related-work-cognition}

When making sense of a social situation such as interpersonal communication or decision-making processes, individuals go through certain cognitive processes to analyze, interpret, and remember information \cite{SocialCognition}. Previous studies found that these process are influenced by individuals’ own cognitive tendency or given contexts. For example, people tend to have their own decision-making style \cite{DecisionmakingstylesreallifedecisionChoosing,Individualdifferencesadultdecisionmakingcompetence, DecisionMakingStyleDevelopmentAssessmentNew} -- whether they rely on hunches or a thorough search for information. According to the Motivation-Opportunity-Ability (MOA) model \cite{EnhancingMeasuringConsumersMotivationOpportunity}, individuals are either empowered or hindered in exhibiting behavioral changes or attributing decisions when making sense of their success and failure in their careers or education \cite{weiner1972attribution}. These cognitive processes can further influence the degree to which individuals seek more information to reason about situations, manifest in either spontaneous or deliberate modes within the dual processing theory \cite{SocialCognition, CogitoergoquidEffectCognitive, InfluenceCognitiveStylesUsersUnderstandinga}.

A number of studies \cite{PartneringPeopleDeepLearning, Doeshighereducationhonecognitive, Domainspecificpreferencesintuitiondeliberationdecision} have found that all these cognitive traits are highly dependent on their demographics. For example, older adults rely more on emotions and experience rather than being rational due to aging cognitive ability \cite{lockenhoff2018aging}. The level of education also influences the cognitive load, intelligence, thinking, and working memory \cite{Doeshighereducationhonecognitive}.  Rational thinkers tend to actively seek explanations when finding the best recommendations than intuitive thinkers. 

Despite these findings, there is limited understanding of the complex cognitive dimensions that affect individuals' willingness to seek explanations and their valuation of specific explanation strategies in AI-assisted decision-making. Our study explores these connections, highlighting the need for careful designed explanations that consider these cognitive processes. 


\subsection{Evaluating Explanations in Various AI-assisted Decision Contexts}
\label{sec:related-work-context}

The effectiveness of various explanation types in enhancing trust and understanding has been studied across diverse AI-assisted decision contexts, including human-robot interaction \cite{ExplainableAgentsRobotsResults, GuidelinesDevelopingExplainableCognitive, SelfExplainingSocialRobotsVerbal, TheoryExplanationsHumanRobotCollaboration, DifferentXAIDifferentHRI}, self-driving technologies \cite{ExplainableAutonomyStudyExplanation, TextualExplanationsSelfDrivingVehicles, TrustingXAIEffectsdifferenttypes}, and medical diagnosis \cite{humanbodyblackboxsupporting, EffectExplanationStylesUser, jimenez2020drug}.

For instance, in medical context, providing explanations helped improve health awareness, facilitate learning, and aid decision-making by offering patients new information about their symptoms \cite{ExploringPromotingDiagnosticTransparency} or help laypeople understand complex medical concepts during cancer diagnosis \cite{EffectExplanationStylesUser}. In self-driving contexts, explanations provided in a timely manner during sequential driving scenes have been found to improve understanding \cite{TrustingXAIEffectsdifferenttypes, ExplainableAutonomyStudyExplanation, TextualExplanationsSelfDrivingVehicles}.

Despite all these studies highlighting the impact of various explanation styles within a certain context, it remains uncertain how different explanation types affect the levels of trust, understandability, and other aspects of user perceived values across multiple contexts. While previous studies \cite{CapturingTrendsApplicationsIssues, ScienceHumanAIDecisionMakingSurvey, ehsan2023charting} have conceptually examined that AI transparency is entangled with sociotechnical contexts, our research empirically demonstrates that the effectiveness of explanations varies based on the application context, highlighting the need for context-aware design in explainable AI systems.

\section{Study Purpose and Overview}
%Our study aims to examine two facets of purposes as follows: 1) how contrastive and selective explanations -- as widely accepted properties of good explanations -- also hold preferable when presented to lay people in AI-assisted decision-making contexts, and 2) what factors come into play in evaluating such explanation strategies on AI-assisted decisions. In our study, we consider a variety AI-assisted decision-making scenarios where data subjects are given a decision related to their individual circumstances in diverse sociotechnical contexts, followed by a textual explanation presented to them as the rationale on why they were given the decision. To illustrate them with concrete examples, we began by developing a handful of written scenarios spanning multiple decision-making contexts, such as a medical diagnosis scenario as illustrated below:
Our study aims to examine two key aspects of explanation strategies in AI-assisted decision-making: (1) how contrastive and selective explanations—commonly accepted as properties of good explanations—are perceived by laypeople, and (2) which factors influence the evaluation of these explanation strategies in AI-assisted decisions. We explore various AI-assisted decision-making scenarios, where human subjects receive decisions related to their personal circumstances in different sociotechnical contexts. These decisions are accompanied by textual explanations that provide the rationale behind the outcomes.

To ground our study in concrete examples, we developed a range of written scenarios across various decision-making contexts, such as medical diagnoses and loan approvals. These scenarios demonstrate how explanations are provided and how participants might evaluate them in different sociotechnical contexts. Below, we present one of the six selected scenarios.

% \textbf{Motivating scenario.} On Sunday morning, Jane, in her age of sixty-seven, was preparing to go for a walk as usual. When headed outside home, however, she suddenly felt a bit dizzy and had some chest pain. She wanted to check the health status with a diagnosis app tracking the sensor data and medical records. She entered the symptom and checked if this is related to some disease. Then the AI-based diagnosis told her that “you are at a risk for Acute Coronary Syndrome and are recommended to visit your primary care physician.” It was really an unexpected result for her because when looking back at her past experiences, her dizziness used to be gone in a minute, and she assumed it to be anemia.

% Jane found it crucial to her long-term health to know but there seemed to be many underlying factors that are hard to understand without medical knowledge. She was also unsure what medical or personal factors led to such a diagnosis, and also wanted to know what she can do to get better. She found that, in the application, there was an AI-based chatbot available for seeking explanations about why she got a diagnosis from the app, so she asked for it. In response, the chat service responded back with an explanation.

\textbf{Loan approval case as a motivating scenario:} On a Tuesday morning, Jane, a sixty-seven-year-old retiree, applied for a loan to renovate her home. A few hours later, she received a notification from the bank stating that her loan request had been denied. The AI-based system explained the decision was based on her credit score and financial history. This result surprised Jane, as she had always been responsible with her finances. She wanted to understand why the loan was denied---whether it was her age, retirement status, or something else in her profile.

\begin{table*}[h]
\caption{Types of explanation strategies.}
\label{tab:strategies}
\setlength{\tabcolsep}{3pt}
\centering
\begin{tabular}{@{}lllll@{}}
\toprule
\multirow{2}{*}{Explanations} & \multicolumn{2}{l}{Contrastive strategies}                                                                                                                         & \multicolumn{2}{l}{Information selectivity}                                                                                                            \\ \cmidrule(l){2-5} 
                              & \begin{tabular}[c]{@{}l@{}}How to compare\end{tabular} & What/Who to compare                                                                                     & \begin{tabular}[c]{@{}l@{}}Information complexity\end{tabular} & \begin{tabular}[c]{@{}l@{}}Information alignment with \\  prior beliefs\end{tabular} \\ \midrule
\comp          & No strategy                                              & -                                                                                                       & Thorough (All the causes)                                        & Low bias, High variance                                                             \\
\cf            & Contrastive                                              & vs. me (hypothetical status of myself)                                                                  & Simple (Only minimal causes)                                     & High bias, Low variance                                                             \\
\cto           & Contrastive                                              & vs. others (with the opposite outcome)                                                                  & Simple (Only minimal causes)                                     & High bias, Low variance                                                             \\
\begin{tabular}[c]{@{}l@{}}\ctt \\ \newline  \end{tabular}        & \begin{tabular}[c]{@{}l@{}}Contrastive   \\ \newline   \end{tabular}                                        & \begin{tabular}[c]{@{}l@{}}vs. me (previous status of myself \\ with the opposite outcome)\end{tabular} & \begin{tabular}[c]{@{}l@{}}Simple (Only minimal causes)  \\ \newline \end{tabular}                                   & \begin{tabular}[c]{@{}l@{}} High bias, Low variance  \\ \newline  \end{tabular}                                                         \\
\cbhe          & Analogous                                                & vs. others (with the same outcome)                                                                      & Simple (Only minimal causes)                                     & High bias, Low variance                                                             \\
\cbho          & Analogous                                                & vs. others (with the same outcome)                                                                      & Thorough (All the causes)                                        & Low bias, High variance                                                             \\ \bottomrule
\end{tabular}
\begin{tablenotes}
  \small
  \item *Note that explanation strategies that entail contrasting the information (\cf, \cto, and \ctt) inherently deal with simple information by their definition.
\end{tablenotes}
\end{table*}

Jane found an AI-based chatbot on the bank's website offering explanations for loan decisions. She asked the chatbot why her loan was denied, and it provided a detailed response outlining the factors behind the decision.

\subsection{Evaluation of Contrastive and Selective Explanations}
% In such scenarios, AI systems may concern how to not only provide explanations to satisfy the explainability but also make it better and tailored to fit into users' different characteristics, level of understanding and expectations in a given AI-assisted decision-making contexts.

% Our choice of explanations based on literature review allows us a thorough examination of potential variants of contrastive and selective explanations. As summarized in Table \ref{tab:strategies}, these explanations can be also viewed in terms of two explanatory strategies, contrastive explanations and information selectivity. First, an explanation depends on contrastive strategies (i.e., \textit{whether} or \textit{how} to compare), such as contrastive (to compare with the opposite case), analogous (to compare with a similar case), or no strategy. Within each contrastive strategy, we include variants of explanations discussed in existing literature depending on \textit{what/who} to compare, such as making contrast against others (\cto), their own status in the past (\ctt), or hypothetical status of myself (\cf) with the opposite outcome. 

% Also, the variants of explanations in our study are also differentiated based on information selectivity, with the matter of information complexity and alignment as its sub-components. terms of information complexity, contrastive and counterfactual explanations (\cto, \ctt, and \cf) as well as \cbhe provide simple explanations with only minimal set of features that differentiate cases described in the texts, while \comp and \cbho describe all the features in a thorough manner. This variation may further influence information selectivity, i.e., how information presented in the explanations, depending on whether it is partial vs. all the causes, are aligned with people's prior beliefs. To provide concrete examples of explanation, Fig. \ref{fig:exp-strategies} illustrates six explanations with their definitions in a scenario of loan approval scenario where a data subject (denoted as `Me') was given a decision of loan denial. As shown in the examples, explanations with simple information include partial set of features as the minimal causes, as compared with thorough explanations with all the features. We hypothesize that the value of explanations are likely affected by how these features are aligned with data subjects' prior beliefs. From our perspective, this can be better understood through an analogy to a trade-off between bias and variance, a well-known concept in the context of machine learning. When a partial set of information is presented (i.e., providing only minimal causes), it leads to high bias (less likely to fit users' prior beliefs), but once it fits in, its low variance will lead to a greater precision, leading to more satisfaction, while presenting a broader scope of information (i.e., providing all the causes) is the opposite case. 

% Overall, our study is designed to take a deeper dive into variants of such explanations, on whether, what, and how to compare (contrastive strategies) and what information and how much to explain (information selectivity). 

In such scenarios, AI systems must not only provide explanations to meet explainability requirements but also tailor them to users' varying characteristics, levels of understanding, and expectations within AI-assisted decision-making contexts.

Our selection of explanation types, informed by a thorough literature review, enables us to examine different forms of contrastive and selective explanations. As summarized in Table \ref{tab:strategies}, these explanations fall into two main categories: contrastive strategies and information selectivity.

{\bf Contrastive strategies.} Contrastive explanations compare a given decision to alternative outcomes using different strategies ({\it how} to compare). These strategies include comparing to an opposite outcome (contrastive), comparing to a similar case (analogous), or providing no comparison at all. The focus of the comparison ({\it what/who} to compare) can also vary, such as comparing the user's outcome to someone else's (\cto) or comparing their current status to their past (\ctt) \cite{miller2019explanation, RemoteCausesBadExplanations}, or considering a hypothetical alternative outcome they could face (\cf). Each of these variants is rooted in the existing literature (see detailed review in Section \ref{sec:related-work-exp}) and represents a specific approach to helping users understand the AI's decision-making process.

{\bf Information selectivity.} The explanations in our study are also differentiated by information selectivity, which involves two sub-components: information complexity and alignment with users' prior beliefs. 

In terms of information complexity, some explanations, such as contrastive and counterfactual explanations (\cto, \ctt, and \cf) and \cbhe \cite{ EvaluationUsefulnessCaseBasedExplanationa, stolpmann1999optimierung}, provide simple explanations that highlight only the minimal features differentiating the cases. In contrast, other explanations, like \comp and \cbho \cite{el2015case, ExplanationsCaseBasedReasoningFoundational}, offer a comprehensive overview, describing all relevant features in detail. This variation further influences how much and what information is going to be described in each explanation. Figure \ref{fig:exp-strategies} illustrates six different explanation strategies in the context of a loan denial decision. As shown, explanations with minimal information focus on a few key features, while thorough explanations include all relevant factors.

We hypothesize that the perceived value of these explanations depends on how well the selected features align with users' prior beliefs. This relationship can be compared to the trade-off between bias and variance in prediction. Presenting minimal information may result in higher bias (less alignment with users' expectations), but when the explanation aligns with their expectations, it can lead to greater satisfaction due to its greater precision. In contrast, presenting more comprehensive explanations tends to reduce bias but may introduce greater variance, potentially overwhelming users with information.


\subsection{Understanding of Human Valuation Process of Explanations} 
Given an array of contrastive and selective explanations, which explanation strategies will people find preferable and intuitive? While theoretical frameworks advocate for these types of explanations as generally intuitive and human-friendly, we posit that their value is highly contextual — depending on factors such as \textit{when}, \textit{how}, and \textit{for whom} the explanations are provided.

\begin{figure*}[h]%
\centering
\includegraphics[width=\textwidth]{figures/exp-strategies}
\vspace{-2em}
\caption{The detailed definitions and examples of six variants of explanations.}\label{fig:exp-strategies}
% \vspace{-0.75em}
\end{figure*}

Prior research in cognitive and social science has presented numerous findings on how individuals engage with explanatory processes or decision-making contexts differently based on their personal traits and situational contexts. Drawing from the literature review (see details in Section \ref{sec:variable-selection}), we present the framework for conceptualizing human valuation of explanation. As presented in Fig. \ref{fig:study-design}A, we synthesize them as an interplay of various factors, which can fall into one of the following dimensions: 1) \darkgray{individual-dependent} factors: Individuals are known to have inherent \darkgray{demographic traits} and \darkgray{decision-making style} (Fig. \ref{fig:study-design}A-a) that shape their preferences and ability based on their inherent traits such as demographics and decision-making style. 2) \orange{context-dependent} factors: When given a decision in a \orange{sociotechnical context} (e.g., denied a loan) (Fig. \ref{fig:study-design}A-b), individuals may perceive the context (e.g., does the decision have a significant consequences?), in turn exhibiting a different level of \darkorange{cognitive engagement} (Fig. \ref{fig:study-design}A-c), depending on whether they are motivated, perceive a certain level of opportunities or ability to seek explanations. When presented with an \green{explanation} from an AI system (Fig. \ref{fig:study-design}A-d), individuals may experience different levels of \darkgreen{cognitive load} (Fig. \ref{fig:study-design}A-e), impacting their interpretation of the provided explanation. Depending on these factors, users will evaluate explanations regarding different \darkpurple{explanatory values} (Fig. \ref{fig:study-design}A-f). In our study, we scrutinize this framework by investigating RQ3 and RQ4 as outlined in Section \ref{sec:rqs}, aiming to identify key aspects of the human valuation process of explanations.


\subsection{Research Questions}\label{sec:rqs}
Based on two facets of our study purposes as described above, we distill them into four key research questions listed below: \\

% \textbf{Evaluating lay people's preferences over contrastive and selective explanations.} The first part of our study is dedicated to evaluate contrastive and selective explanations in different AI-assisted decision scenarios. We take both quantitative and qualitative approach to validate the relative preferences of different types of explanations via a survey experiment and further scrutinize the rationales behind their preferences through an analysis of responses on open-ended questions to see why they preferred certain explanations (i.e., what properties of explanations led to their preferences) (see detailed methods on Section \ref{sec:study-design} and \ref{sec:method-open}). With this purpose, we challenge the widely accepted assumption of the value of contrastive and selective explanations.
{\bf Evaluating laypeople's preferences for contrastive and selective explanations:} The first part of our study focuses on evaluating preferences for contrastive and selective explanations across various AI-assisted decision-making scenarios. We use both quantitative and qualitative approaches: a survey experiment to measure preferences and an analysis of responses to open-ended questions to explore the reasons behind these preferences (i.e., what features of the explanations influenced their choices) (see detailed methods in Sections \ref{sec:study-design} and \ref{sec:method-open})). Through this, we aim to critically examine the commonly held assumption that contrastive and selective explanations are inherently valuable in all contexts.
\begin{itemize}
    \item \textbf{RQ1.} How are contrastive and selective explanations preferred in general and within specific sociotechnical contexts?
    \item \textbf{RQ2.} Which aspects and properties of explanations are linked to the distinct valuation of those explanation strategies?
\end{itemize}

% \textbf{Understanding factors influencing the preference of contrastive and selective explanations.} In addition to evaluating the preferences, we further explore the study framework to contribute to a nuanced understanding of human valuation of explanations. In our experiment, we translate the factors included in the framework into the survey items and conduct the analysis for the between-variable and collective relationship among all factors.
{\bf Exploring factors influencing preferences for contrastive and selective explanations:} In addition to evaluating preferences, we aim to gain deeper insights into the factors that shape how people value different explanations. In our experiment, we translate these factors into specific survey items and analyze their interactions, examining both individual variables and their combined effects. This design seeks to explain the factors influencing preferences for contrastive and selective explanations.

\begin{itemize}
    \item \textbf{RQ3.} How do individual, contextual, and cognitive factors interact in the valuation process of explanations?
    \item \textbf{RQ4.} How do these factors collectively influence the preference on valuation of contrastive and selective explanations?
\end{itemize}

\begin{figure*}[h]%
\centering
\includegraphics[width=\textwidth]{figures/study-design.pdf}
\caption{A) \textbf{Study framework}: Our study aims to systematically explore the inner workings of how individuals attribute AI-generated decisions when confronted with AI-driven decisions (e.g., why was I denied a loan?) and evaluate explanations presented by AI systems. The study framework for human valuation on explanations presents a cognitive journey (in the order from a to f) explainees may go through while processing explanations about the decision. B) \textbf{Survey design}: To examine the valuation process, we distilled the framework into a survey for scenario-based experiment. C) \textbf{AI-assisted decision scenarios}: These decision scenarios are carefully selected to be contrasted in three aspects: high-stakes, professional, and timely.}
\label{fig:study-design}
\end{figure*}

\subsection{Theoretical Foundation and Rationale behind Variable Selection} \label{sec:variable-selection}
To design the framework, we have reviewed literature throughout a wide range of disciplines. Studies in social and cognitive science have been to examine theories and findings about how people attribute decisions and process information based on a given context, leading to shaping individual, cognitive, and contextual factors. An extensive review from and outside of XAI literature have led us to examine explanatory values (i.e., criteria for evaluating explanations in AI contexts).

\textbf{From various disciplines outside of XAI literature.} Literature from social and cognitive science have established theories and findings on individuals' cognitive capability and status in the decision-making and explanatory process. Our objective is to integrate these insights into AI decision contexts, to not only theoretically support our framework but also make connections between human and AI studies, from human-human and human-AI communication.

\begin{itemize}
    \item \textbf{\darkgray{Decision-making styles}} (Fig \ref{fig:study-design}A-a): Research in Psychology and Business has studied how individuals have different levels of cognitive tendency based on inner traits such as gender, age, or education. Especially when processing decisions and relevant information, individuals tend to exhibit different decision-making styles (See detailed literature review in Section \ref{sec:related-work-cognition}). We introduce five different styles in a scale called General Decision Making Style (GDMS) \cite{examinationgeneraldecisionmakingstyle}, including rational, avoidant, intuitive, dependent, and swift (details in Table \ref{tab:cog_vars}).
    \item \textbf{\darkorange{Cognitive engagement}} (Fig \ref{fig:study-design}A-c): In the literature from social science such as Education and Consumer Behavior, scholars have attempted to establish theories, often referred to as social attribution, theorizing individuals' internal cognitive states that drive behavioral changes (See detailed literature review in Section \ref{sec:related-work-cognition}) triggered by events such as a failure or success of their career or educational performances. In this study, we employ the MOA (Motivation, Opportunity, Ability) model \cite{EnhancingMeasuringConsumersMotivationOpportunity, jepson2018applying, fazio2014attitude} integrating three internal states, indicating whether they are more willing to act or change when they feel more motivated, perceive more opportunities or ability depending on surrounding environments or situations (details in Table \ref{tab:cog_vars}).
    \item \textbf{\darkgreen{Cognitive load}} (Fig \ref{fig:study-design}A-e): Cognitive load theory is a widely used construct to measure cognitive burden in humans’ information processing  (See detailed literature review in Section \ref{sec:related-work-cognition}). We examine three facets of cognitive load, including intrinsic, extraneous, and germane load. Each facet of cognitive load serves a distinct purpose: Intrinsic load is contingent on the difficulty of information covered in explanations, heavily influenced by specific contexts. On the other hand, extraneous and germane load pertain more to the presentation of information, and this variability is tied to the explanation strategies (details in Table \ref{tab:cog_vars}).
\end{itemize}

\textbf{From XAI literature.} Recent XAI studies have examined AI explainability specific to a context or multiple contexts to evaluate its effect in enhancing various explanatory values.

\begin{itemize}
    \item \textbf{\darkpurple{Explanatory values}} (Fig \ref{fig:study-design}A-f): We examined XAI surveys \cite{MetricsExplainableAIChallenges, TaxonomyHumanSubjectEvaluation, ScienceHumanAIDecisionMakingSurvey} investigating measures, criteria, and taxonomy for evaluating explainable AI. In our study, we employ multiple values for users to evaluate the effect of explanations, whether explanations have sufficiency (detailed and complete), understandability (helping understand why the decision was made), usefulness (facilitating comprehension of actions to take), trust (increasing the willingness to act on the basis of, the recommendations, actions, and decisions of an artificially intelligent decision aid)\footnote{A recent study \cite{HowEvaluateTrustAIAssistedDecision} provides the definition of trust in evaluating XAI systems and how it should be integrated in the experimental protocols. We note that our experimental setting and measure satisfy the three elements of trust discussed in the paper, including attitude (i.e., measures in the experiment capture how participants perceive the AI system), vulnerability (i.e., the given scenario involves uncertainty of the outcomes of a decision), positive expectation (i.e., the instruction ensures participants that the AI system is reliable and accurate).}, and overall preference.
    \item \textbf{\orange{Sociotechnical contexts}} (Fig \ref{fig:study-design}A-b): The six decision scenarios (Fig. \ref{fig:study-design}C) were selected to examine the impact of sociotechnical contexts. These decision contexts were carefully selected to differentiate between them in three contextual aspects: (a) {\it high-stakes}: a decision in the scenario involves significant consequences, (b) {\it professional}: a decision in the scenario requires one to have the professional knowledge to reason the given information, and (c) {\it timely} (or {\it time-critical}): a decision in the scenario must be made promptly.

    \indent We selected four application contexts widely studied in AI explanability studies including loan approval, medical diagnosis, driving app, and movie recommendation. We further differentiate them with the outcome of a decision -- whether the decision is favorable (positive) or unfavorable (negative) -- may alter people's expectations regarding the explanations. As illustrated in Fig.~\ref{fig:study-design}C, we create six scenarios featuring the various aspects: loan decision (\loanN) and driving app (\drivN) with undesirable decision, medical diagnosis with desirable (\mediP) and undesirable decision (\mediN), desirable (\recomP) and undesirable (\recomN) movie recommendation. The positive and negative signs indicate whether the provided information is generally desirable or undesirable.
\end{itemize}
\section{Method}\label{sec:method}

\subsection{Study Design}\label{sec:study-design}

We designed a survey experiment to investigate the process of evaluating explanation and answer the research questions. The survey was designed as a between-subject experiment, where participants were randomly assigned to one of the six decision scenarios.

Overall, the survey was designed to emulate the process of humans' valuation of explanation strategies (Fig. \ref{fig:study-design}A), consisting of multiple stages (a-f) as illustrated in Fig. \ref{fig:study-design}B (see the details for the survey material in the \appsec{sec:survey}): a) Participants were asked questions about their individual characteristics including demographics and decision-making style; b) A randomly assigned scenario was presented to engage participants in the AI-assisted decision-making context. To devise each scenario, we consulted with domain experts to determine the details of the outcomes and features, and various user cases that serve as counterparts of the focal data subject regarding contrastive and analogous comparisons; c) Questions regarding perceived contextual properties (i.e., high-stakes, professional, timely) and cognitive engagement (i.e., motivation, opportunity, ability) were prompted to examine context-dependent factors; d) Six explanation strategies were presented; e,f) users were asked to rank explanations to rate their cognitive load as well as explanatory values. Finally, we collected participants' opinions and reasoning for the explanation variants in free-form text responses.

% \begin{figure}[t]%
% \centering
% \includegraphics[width=\columnwidth]{figs/contexts.pdf}
% \caption{The six scenarios of AI-assisted decision-making in our study. These decision scenarios are carefully selected to be contrasted in three aspects: high-stakes, professional, and timely.}
% \label{fig:contexts}
% \end{figure}

\subsection{Study Implementation}
\label{sec:study-procedure}

\paragraph{Participants}\label{sec:participants}
We recruited participants from Prolific crowdsourcing platform. Workers living in the US (age 18+) are fluent in English were eligible to participate in this study to make sure their ability to reason about the given scenario in English. One of the three rounds aimed to recruit senior people (age 55 and older), as they are typically underrepresented on the crowdsourcing platform. A total of 839 participants took part in the study. We excluded respondents who left their responses as default or who did not respond to all survey questions. This yielded a final sample of 698 participants for our data analysis (390 females, age: 222 participants were between the ages of 18 and 24 and 62 were 65 or older; 1 participant preferred not to state age, and 3 participants preferred not to say their gender). Fig. \ref{fig:demographics} in the \appsec{sec:demo-breaks} provides the demographic breakdown of the participants. The experiment was approved by the Institutional Review Board of the University\footnote{The university name was omitted for blind review.}. We pre-registered the experiment on the Open Science Framework (OSF)\footnote{\url{https://osf.io/hp86t/?view_only=f49fb230b8e8478288d4844869a88863}}. 

\paragraph{Sample size determination}
We used the Mann-Whitney test to determine the sample size. Based on the power analysis with a significance level of 0.05 and a medium effect size, we determined that at least 106 samples per group were needed, resulting in a suggested minimum sample size of 636 for six survey variants, in order to ensure that the effect of the explanations could be tested.

\paragraph{Procedure}
The survey experiment was conducted on Prolific for three rounds in 2022, on January 29, March 29, and April 25. Participants receive a small amount of compensation with the base rate of \$7.74 per hour in exchange for their effort. The survey consisted of 19 questions, seven of which asked for demographic and survey identification information. The median time for a respondent to complete the survey was 11.6 minutes. 

In the context of AI-based decision-making, participants were tasked with evaluating various types of explanations as decision-making rationales. The survey began with questions regarding demographics. Participants were then asked to read a paragraph describing one of six AI-assisted decision-making scenarios provided by the researchers. Given the context, participants first rated their perceptions of contextual properties and cognitive engagement when presented with the scenario. These questions use a five-point  Likert scale with values ranging from 1 to 5. Participants were then instructed to read each explanation style presented in random order and rank them according to their preferences. At the end of the study, participants provided written comments on the rationale behind their choice of the preferred explanations and thoughts about the AI system.

\subsection{Statistical Analyses}
\label{sec:stat}

We conducted multiple types of statistical analyses to examine the research questions listed in Section \ref{sec:rqs}. As a pre-processing step, we converted all ranking responses to a {\it relative rating} variable, such that higher scores indicate positive cognitive loads and explanatory values. 

For RQ3, we employed the Mann-Whitney U test to determine if significant differences existed between two distinct sets of scenarios (e.g., high-stakes vs. low-stakes contexts). The Wilcoxon signed-rank test was additionally utilized for identifying significant differences between pairs of styles, acknowledging that the relative ratings across these styles are interdependent. In our pairwise testing, we applied the Bonferroni correction to adjust for multiple comparisons. Given that our analysis involved both 6 scenarios and 6 styles, the significance level was accordingly adjusted to 0.05/6, which is approximately 0.0083.

To examine RQ4 in scrutinizing the interplay of individual-, cognitive-, and explanation-dependent factors, we used Structural Equation Modeling (SEM) to take into account all direct and indirect relationships between variables included in our study framework (Fig. \ref{fig:study-design}A). For these models per explanation type, we converted participants' ratings of explanation styles into a binary variable indicating whether a particular style was absolutely favorable (i.e., ranked as top or second one) or not.


\subsection{Analysis for Open-ended Questions}
\label{sec:method-open}

In the survey, we further collected open-ended responses from participants to ask for their detailed rationales for their overall preferences with two questions: (1) Detailed rationales on their valuation process of ranking the explanations with the question, “please briefly describe the rationale on ranking the explanations,” and (2) general perceptions on values of explanations and AI systems, “what is the most important aspect of explanations for you? Do you have any additional comments about the AI system?”

\begin{figure}[h]%
\centering
\includegraphics[width=\columnwidth]{figures/overallPref.pdf}
\caption{The relative ratings (along the $x$-axis) for each of the six explanation variants for the {\it extraneous} and {\it germane} cognitive capacities influenced by how information is presented, as well as the preference rankings in five distinct value dimensions. To facilitate the summary, participants' {\it overall} preference was highlighted in gray.}\label{fig:overallPref}
\end{figure}

To extract and summarize the characteristics of their evaluation process from texts, we took a two-step approach. First, we used qualitative coding to identify recurring themes in participants' valuations. Two coders read a random sample of 30\% of written responses and engaged in multiple rounds of discussion to group them into high-level categories, resulting in four distinct types of valuations: explanatory properties, features, comparison, and XAI system. With these categories, we iterated over all written responses to quantitatively and qualitatively identify whether each written response contains any of four valuation categories (e.g., explanatory values) and what specific values (e.g., simple, easy) are encoded. The definitions and methods of analysis for four categories are summarized below:

\begin{itemize}
    \item \textbf{Property-oriented valuation}: 524 responses mentioned the desired properties of explanations regarding information quantity, logic, or value (e.g., brief, simple, informative). We extracted adjectives that express positive explanatory values by conducting the part-of-text analysis and filtering out negative and irrelevant words.
    
    \item \textbf{Feature-oriented valuation}: 135 responses referred to certain feature(s) given in the decision-making scenario (e.g., credit score, BMI, weather, genre, or age) as a rationale of why they chose certain explanations as the least/most preferred. From these responses, we extract some findings on what aspects of information selectivity mainly influence the valuation of explanations.
    
    \item \textbf{Comparison-related valuation}: 77 responses preferred certain type(s) of contrastive strategies. As this type of feedback was expressed in an unstructured manner, we manually coded the types of comparisons mentioned in their responses.
    \item \textbf{XAI-system-related valuation}
    25 responses expressed their expectations on how explanations in AI systems were received or what they are supposed to do. Similarly, we manually coded whether each response contained system-related valuation.
\end{itemize}




\section{Results}

\subsection{How do participants' perceived value of explanations differ, overall and within sociotechnical contexts?}\label{sec:values}

First, we examine how explanation strategies were perceived differently in terms of perceived values of explanations overall and within each decision context, which are summarized in Fig.~\ref{fig:overallPref} and Fig.~\ref{fig:overallPref-context} respectively.

Our analysis found \comp explanations were evaluated as higher in all explanatory values than any other explanation types. Conversely, \cbhe was considered the least favorable in all valuation aspects. \cf and \cbho were ranked second-to-worst in terms of {\it sufficient} and {\it understandable}, and {\it trustworthiness} and {\it usefulness} respectively.

\begin{figure}[H]%
\centering
\includegraphics[width=\columnwidth]{figures/overallPref-by-context}
\vspace{-2em}
\caption{In each scenario, the relative ratings (along the $x$-axis) for each of the six explanation variants for the {\it extraneous} and {\it germane} cognitive capacities, as well as the preference rankings in five distinct value dimensions. To facilitate the summary, participants' {\it overall} preference was highlighted in gray.}\label{fig:overallPref-context}
% \vspace{-0.75em}
\end{figure}

However, when analyzing the ratings of explanatory values for each decision scenario, we found that participants’ perceptions of these explanations varied depending on the scenarios they encountered. In high-stakes and time-sensitive scenarios such as \drivN, \loanN, and \mediN, three explanation styles stood out. The \ctt received a rating comparable to the \comp in \drivN. The \ctt explanation, ``\textit{The estimated arrival time was predicted 20 minutes later than usual because there was heavy snow last night ..., while your usual travel with on-time arrival happened without snow with good traffic flow},'' was considered to be a {\it sufficient}, {\it understandable}, {\it trustworthy}, and {\it useful} explanation for the AI-decision ``{\it your arrival time as being 20 minutes later than usual ...}'' This suggests that an explanation contrasting outcomes at different times is particularly applicable to highly time-sensitive situations. In \mediN, the \cto (``...\textit{You are at a high risk for Acute Coronary Syndrome because you have high cholesterol and blood pressure between 90-145. In contrast, a user who wasn't....}'') received ratings comparable to the \comp. This indicates that justifications for the AI's decision based on contrasting results due to different causes work well in medical contexts.

The \loanN scenario exhibited a distinct pattern of valuation compared to all other scenarios. Participants in this scenario favored the \comp less than those in other scenarios. On the other hand, the \cf (``{\it You could have been granted a loan if you had been employed with an annual income more than...}'') as well as \cto explanations were the most popular choices. This suggests that both explanation approaches were particularly useful in a situation where there was either a win or a loss, such as with \loanN.
\subsection{Which aspects and properties of explanations are linked to the distinct valuation of those explanation strategies?}
Based on the analysis of open-ended responses, we examine participants' detailed rationale on what aspects and properties of explanations made them prefer certain explanations over the others. We find evidence on all facets of explanation strategies listed in Table \ref{tab:comp_strategies} including contrastive strategies (how and what/who to compare) and information selectivity (information complexity and alignment) as well as general expectations over AI systems' explainability as follows. 

\subsubsection{Participants appreciate different types of explanatory properties, mostly pertaining to information complexity.}
\label{sec:exp-properties}
First, most of the participants (524/698, 75.1\%) tended to evaluate explanation strategies based on a variety of explanatory properties to further elaborate on their preferences in the quantitative results. The five most frequent properties across all explanation styles were easy (126 times), clear (66), relevant (38), specific (36), and detailed (34). 

\begin{figure}[H]%
% \vspace{-1em}
\centering
\includegraphics[width=.8\columnwidth]{figures/exp-properties.pdf}
% \vspace{-2.5em}
\caption{The most frequent explanatory properties in association with the explanation strategies.}
\label{fig:exp-properties}
\end{figure}

It was noticeable that the top explanatory properties in each explanation type were mostly pertaining to information complexity such as detailed, simple, or brief. The lists of top-10 most associated properties in favor of each explanation strategy (Fig. \ref{fig:exp-properties}) were extracted based on FREX score (FRequency and EXclusivity) \cite{bischof2012summarizingfrex}, which identifies words that are both frequent in and exclusive to a topic of interest, regarding each explanation style. Specifically, \comp was perceived as comprehensive and detailed as it described ``\textit{every aspect of why the decision was made}'' and the comprehensive information allowed them to``\textit{form a big picture in [their] mind.}.'' \cbho was appreciated for not only being thorough but also its comparative characteristic, ``\textit{being detailed and comparative would help to highlight the urgency of the situation.}'' On the other hand, contrastive explanations (including \cto and \ctt, and \cf) were commonly appreciated by virtue of their clear, brief, and short information representation as it gives them ``\textit{the most exact and particular reason what the key problem is to improve}''.

On the other hand, contrastive/counterfactual explanations were preferred due to both comparative strategies. For example, \cto and \cf were received more actionable because it helps ``\textit{take away something from and apply it to my life.}'' due to being descriptive of their status in contrast to  the ones with the opposite outcome. On the other hand, \ctt explanations contrasting with previous state of the user were regarded as more relevant as it ``\textit{relates to me more and compares less with others.}''


\subsubsection{Participants favor explanations they can readily accept based on their prior knowledge or beliefs.}
\label{sec:selective}
In the 135 responses (19.3\%), participants expressed preferences for explanation styles when the explanation mentioned one or more feature(s) they perceived as critical or unnecessary in the decision-making. 

These feature-oriented valuation, spanning 34 types of features for 205 times, varied across decision contexts but appeared mainly in less knowledge-intensive decision-making contexts. For instance, in the loan approval context, participants often confidently referred to employment (\loanN: 17) or annual income (\loanN: 17) as critical factors of loan approval based on their beliefs or prior experiences as commented, ``\textit{income matters.}'' or ``\textit{not being employed is the most critical problem. There is no other reason that matters.}''. In movie recommendations, the browsing history was often considered as a critical factor (\recomN: 30, \recomP: 23) because this feature ``\textit{made the most sense}'' and they feel ``\textit{this will give me the best choice for movies.}'' The movie genre (\recomN: 9, \recomP: 7) was also often perceived as an understandable and important feature to the given recommendations, ``\textit{the relation to the genre and my previous recommendations make it the best and easiest to understand.}'' On the other hand, the least frequent contexts were medical diagnosis (\mediN: 10, \mediP: 12) with mentions of some features such as blood pressure (\mediN: 2, \mediP: 3) or cholesterol (\mediN: 2, \mediP: 1). This suggests that, while some participants prefer explanations that seem more plausible to them, possibly due to their prior beliefs and knowledge, this preference may vary depending on the level of professional knowledge required for the decision context.


On the other hand, some features, especially those related to demographics, also provoked negative preferences over explanations. A number of participants (\recomN: 15, \recomP: 12, \loanN: 5) considered all demographic-related features (such as gender, marriage, or age) as not useful, necessary, or relevant to the AI's decision. Some of them raised their concerns about privacy or profiling issues of collecting and processing personal data. 

\addtolength{\tabcolsep}{-4pt}
\begin{table*}[h]
\caption{Types of comparative strategies disliked (red) or liked (blue) by users.}
\label{tab:comp_strategies}
\begin{tabular}{@{}lllllllll@{}}
\toprule
 &
   &
  \multicolumn{6}{l}{\textbf{\begin{tabular}[c]{@{}l@{}}Explanation strategies ranked by\\ preference scores\end{tabular}}} &
   \\ \cmidrule(lr){3-8}
\multirow{-3}{*}{\textbf{\begin{tabular}[c]{@{}l@{}}Types of comparative strategies\\ disliked/liked by users\\ in written responses\end{tabular}}} &
  \multirow{-2}{*}{\textbf{\begin{tabular}[c]{@{}l@{}}\#\\ users\end{tabular}}} &
  \begin{tabular}[c]{@{}l@{}}\textbf{1st}\end{tabular} &
  \textbf{2nd} &
  \textbf{3rd} &
  \textbf{4th} &
  \textbf{5th} &
  \textbf{6th} \\ \midrule
\cellcolor[HTML]{FFCCC9}\textit{"I \uline{don't like} any comparisons."} &
  13 &
  {\color[HTML]{000000} CP} &
  \cellcolor[HTML]{FFCCC9}{\color[HTML]{000000} CT-T} &
  \cellcolor[HTML]{FFCCC9}{\color[HTML]{000000} CT-O} &
  \cellcolor[HTML]{FFCCC9}{\color[HTML]{000000} CF} &
  \cellcolor[HTML]{FFCCC9}{\color[HTML]{000000} CB-HO} &
  \cellcolor[HTML]{FFCCC9}{\color[HTML]{000000} CB-HT} \\ \midrule
\cellcolor[HTML]{FFCCC9}\textit{\begin{tabular}[c]{@{}l@{}}"I \uline{don't like} being compared to \\ others."\end{tabular}} &
  14 &
  {\color[HTML]{000000} CF} &
  {\color[HTML]{000000} CT-T} &
  \cellcolor[HTML]{FFCCC9}{\color[HTML]{000000} CT-O} &
  \cellcolor[HTML]{FFCCC9}{\color[HTML]{000000} CP} &
  \cellcolor[HTML]{FFCCC9}{\color[HTML]{000000} CB-HO} &
  \cellcolor[HTML]{FFCCC9}{\color[HTML]{000000} CB-HT} \\ \midrule
\cellcolor[HTML]{FFCCC9}\textit{\begin{tabular}[c]{@{}l@{}}"I \uline{don't like} being compared to\\ others who are similar to me."\end{tabular}} &
  14 &
  {\color[HTML]{000000} CP} &
  {\color[HTML]{000000} CT-O} &
  {\color[HTML]{000000} CF} &
  \cellcolor[HTML]{FFCCC9}{\color[HTML]{000000} CT-T} &
  \cellcolor[HTML]{FFCCC9}{\color[HTML]{000000} CB-HO} &
  \cellcolor[HTML]{FFCCC9}{\color[HTML]{000000} CB-HT} \\ \midrule
\cellcolor[HTML]{CBCEFB}\textit{\begin{tabular}[c]{@{}l@{}}"I \uline{like} being compared to others \\ who are typical or similar to me."\end{tabular}} &
  17 &
  \cellcolor[HTML]{CBCEFB}{\color[HTML]{000000} CB-HO} &
  \cellcolor[HTML]{CBCEFB}{\color[HTML]{000000} CB-HT} &
  {\color[HTML]{000000} CP} &
  {\color[HTML]{000000} CF} &
  {\color[HTML]{000000} CT-T} &
  {\color[HTML]{000000} CT-O} \\ \midrule
\cellcolor[HTML]{CBCEFB}\textit{\begin{tabular}[c]{@{}l@{}}"I \uline{like} being compared to others \\ who have the opposite outcome."\end{tabular}} &
  6 &
  \cellcolor[HTML]{CBCEFB}{\color[HTML]{000000} CT-O} &
  {\color[HTML]{000000} CF} &
  {\color[HTML]{000000} CP} &
  {\color[HTML]{000000} CT-T} &
  {\color[HTML]{000000} CB-HO} &
  {\color[HTML]{000000} CB-HT} \\ \midrule
\cellcolor[HTML]{CBCEFB}\textit{\begin{tabular}[c]{@{}l@{}}"I \uline{like} being compared to \\ my previous condition."\end{tabular}} &
  4 &
  \cellcolor[HTML]{CBCEFB}{\color[HTML]{000000} CT-T} &
  \cellcolor[HTML]{FFFFFF}{\color[HTML]{000000} CP} &
  {\color[HTML]{000000} CT-O} &
  {\color[HTML]{000000} CB-HO} &
  {\color[HTML]{000000} CF} &
  {\color[HTML]{000000} CB-HT} \\ \bottomrule
\end{tabular}
\end{table*}
\addtolength{\tabcolsep}{4pt}

% \begin{table}[]
% \begin{threeparttable}
% \caption{\label{tab:comp_strategies} Users' negative and positive preferences over different comparison modes.} \ys{to see if it should go to appendix.}
% \yrl{Suggest removing this table and the corresponding results unless you can make the presentation clear. Issues: This table is very confusing. First, the abbreviations (CT, CF, etc.) have not been clearly defined. Second, it's not clear how the preference scores are calculated. Are they some transformation of the rankings, or derived elsewhere? How could you get positive and negative scores? How did the scores being matched with the "type" (in general, to others, etc.)? Third, how do we know whether the "Dominant contexts" were mentioned with positive or negative preferences? }

% \begin{tabular}{@{}lllllllll@{}}
% \toprule
% \textbf{Type} &
%   \textbf{Cnt} &
%   \textbf{CP} &
%   \textbf{CF} &
%   \textbf{CT-O} &
%   \textbf{CT-T} &
%   \textbf{CB-HO} &
%   \textbf{CB-HT} &
%   \textbf{Dominant contexts} \\ \midrule
% \multicolumn{9}{l}{\textbf{Dislike being compared}}
%    \\\midrule
% in general &
%   13 &
%   {\color[HTML]{3531FF} 5.15} &
%   {\color[HTML]{FE0000} \textbf{3.23}} &
%   {\color[HTML]{FE0000} \textbf{3.54}} &
%   {\color[HTML]{FE0000} \textbf{3.56}} &
%   {\color[HTML]{FE0000} \textbf{3.08}} &
%   {\color[HTML]{FE0000} \textbf{2.46}} &
%   \loanN(4), \mediN(4), \recomP(3) \\
% to others &
%   14 &
%   {\color[HTML]{FE0000} 3.86} &
%   {\color[HTML]{3531FF} 5.14} &
%   {\color[HTML]{FE0000} 3.93} &
%   {\color[HTML]{3531FF} 4.36} &
%   {\color[HTML]{FE0000} \textbf{1.86}} &
%   {\color[HTML]{FE0000} \textbf{1.86}} &
%   \loanN(5), \mediN(5), \mediP(4) \\
% \begin{tabular}[c]{@{}l@{}}to others who are\\ typical or similar to me\end{tabular} &
%   \begin{tabular}[c]{@{}l@{}} 14 \\ \newline  \end{tabular}&
%   \begin{tabular}[c]{@{}l@{}}{\color[HTML]{3531FF} 4.93} \\ \newline \end{tabular}&
%   \begin{tabular}[c]{@{}l@{}}{\color[HTML]{3531FF} 3.71} \\ \newline \end{tabular}&
%   \begin{tabular}[c]{@{}l@{}}{\color[HTML]{3531FF} 4.07 } \\ \newline \end{tabular}&
%   \begin{tabular}[c]{@{}l@{}}{\color[HTML]{3531FF} 3.57 } \\ \newline \end{tabular}&
%   \begin{tabular}[c]{@{}l@{}}{\color[HTML]{FE0000} \textbf{3.14}} \\ \newline \end{tabular}&
%   \begin{tabular}[c]{@{}l@{}}{\color[HTML]{FE0000} \textbf{1.57}}  \\ \newline \end{tabular} &
%   \begin{tabular}[c]{@{}l@{}} \mediP(6), \loanN(3), \recomP(2) \\ \newline \end{tabular} \\ \midrule
% \multicolumn{9}{l}{\textbf{Like being compared}}
%    \\ \midrule
% \begin{tabular}[c]{@{}l@{}}to others who are\\ typical or similar to me\end{tabular} &
%   \begin{tabular}[c]{@{}l@{}} 17 \\ \newline  \end{tabular}&
%   \begin{tabular}[c]{@{}l@{}}{\color[HTML]{FE0000} 3.29} \\ \newline \end{tabular}&
%   \begin{tabular}[c]{@{}l@{}}{\color[HTML]{FE0000} 3.29} \\ \newline \end{tabular}&
%   \begin{tabular}[c]{@{}l@{}}{\color[HTML]{FE0000} 2.76 } \\ \newline \end{tabular}&
%   \begin{tabular}[c]{@{}l@{}}{\color[HTML]{FE0000} 3.12 } \\ \newline \end{tabular}&
%   \begin{tabular}[c]{@{}l@{}}{\color[HTML]{3531FF} \textbf{4.65}} \\ \newline \end{tabular}&
%   \begin{tabular}[c]{@{}l@{}}{\color[HTML]{3531FF} \textbf{3.88}}  \\ \newline \end{tabular} &
%   \begin{tabular}[c]{@{}l@{}} \mediP(6), \recomP(5), \recomN(3) \\ \newline\end{tabular} \\
%   \begin{tabular}[c]{@{}l@{}}to others who have\\ the opposite outcome\end{tabular} &
% \begin{tabular}[c]{@{}l@{}} 6 \\ \newline  \end{tabular}&
%   \begin{tabular}[c]{@{}l@{}}{\color[HTML]{FE0000} 3.67} \\ \newline \end{tabular}&
%   \begin{tabular}[c]{@{}l@{}}{\color[HTML]{FE0000} 3.83} \\ \newline \end{tabular}&
%   \begin{tabular}[c]{@{}l@{}}{\color[HTML]{3531FF} \textbf{5.33} } \\ \newline \end{tabular}&
%   \begin{tabular}[c]{@{}l@{}}{\color[HTML]{FE0000} 3.33 } \\ \newline \end{tabular}&
%   \begin{tabular}[c]{@{}l@{}}{\color[HTML]{FE0000} 2.5} \\ \newline \end{tabular}&
%   \begin{tabular}[c]{@{}l@{}}{\color[HTML]{FE0000} 2.33}  \\ \newline \end{tabular} &
%   \begin{tabular}[c]{@{}l@{}} \mediP(3) \\ \newline \end{tabular} \\
% \begin{tabular}[c]{@{}l@{}}to my previous \\ condition\end{tabular} &
%   \begin{tabular}[c]{@{}l@{}}4  \\ \newline \end{tabular}&
%   \begin{tabular}[c]{@{}l@{}}{\color[HTML]{FE0000} 4.00} \\ \newline\end{tabular} &
%   \begin{tabular}[c]{@{}l@{}}{\color[HTML]{FE0000} 2.25} \\ \newline\end{tabular} &
%   \begin{tabular}[c]{@{}l@{}}{\color[HTML]{FE0000} 4.5} \\ \newline\end{tabular} &
%   \begin{tabular}[c]{@{}l@{}}{\color[HTML]{3531FF} \textbf{5.75}} \\ \newline\end{tabular} &
%   \begin{tabular}[c]{@{}l@{}}{\color[HTML]{FE0000} 2.75} \\ \newline\end{tabular} &
%   \begin{tabular}[c]{@{}l@{}}{\color[HTML]{FE0000} 1.75} \\ \newline\end{tabular} &
%   \begin{tabular}[c]{@{}l@{}}\mediN(2) \\ \newline\end{tabular} \\ \bottomrule
% \end{tabular}
% \begin{tablenotes}
%   \small
%   \item *Positive/negative preferences are colored as blue/red.
%   \item *Bold texts are preference scores conforming to each comparison mode of whether to dislike/like being compared.
% \end{tablenotes}
% \end{threeparttable}
% \end{table}

\subsubsection{Rather than simply contrastive, human-friendly explanations depend on whether, what, and how to compare.}
\label{sec:contrastive}
77 participants (11.0\%) showed diverse user preferences over different types of comparisons, based on factors like what or whom is being compared against their own status. In Table \ref{tab:comp_strategies}, explanations are presented based on their average ranking over a group of participants and highlighting liked (red) or disliked (blue) strategies. 

We found that some participants did not want any types of comparisons and preferred \comp over all others (+: \comp; -: all others). They preferred explanations to ``\textit{focus on themselves}'' rather than others because ``\textit{everyone has different personal context}'' but AI may not consider all contexts when making comparisons. Another group of participants did not like being compared to others but rather to their own previous or hypothetical state (+: \cf, \ctt; -: all others). On the other hand, others preferred certain comparative explanations. For example, a group of participants favored comparison with typical or similar others (+: \cbho, \cbhe; -: all others) or those with opposite outcome (+: \cto, \ctt; -: all others).

\subsubsection{Participants have different expectations over AI explainability.}
\label{sec:ai-system}

Participants' expectations about the role of explanations in AI systems varied significantly across the 25 responses (3.5\%). First, some expressed their views on the objectives of XAI systems. Ten participants, for instance, anticipated that AI systems would provide professional information or be used for complex tasks. Notably, eight participants preferred explanations that included more statistics and numbers, citing that such information ``{\it helps [in] visualizing the status.}'' Two participants mentioned that AI systems should present professional information that could provide more learning opportunities, as opposed to merely presenting well-known terms and factors such as BMI or blood pressure.

Second, four participants argued that AI systems should do less, suggesting that the use of XAI should be limited or restricted to human experts. They voiced concerns such as, ``{\it I [would] rather hear this from a human [than AI]; it is useful when used [alongside] doctors}'' and ``{\it AI is just for diagnosing; to tell me whether I should go to a doctor.}''
\subsection{How do individual, contextual, and cognitive factors interact in the valuation process of explanations?} 

In the second part of our study, we conduct an in-depth analysis of how the distinct values of various explanation strategies, as discussed above, are shaped by a range of factors and their complex interactions. We begin by examining the relationships between sociotechnical and cognitive factors and how they interact with the evaluation of explanations.

\textbf{Sociotechnical contexts are perceived as having distinct contextual properties.}\label{sec:scenario-properties} First, we confirmed that the six scenarios in our study exhibited different characteristics along the three dimensions, including {\it high-stakes}, {\it professional}, and {\it timely} based on participants' ratings (Fig.\ref{fig:situ}). For instance, \loanN, \mediN, and \drivN were rated as more {\it high-stakes} and {\it timely} than \recomP and \recomN (Mann-Whitney U test with $p<0.001$ on each pair of the former and latter groups). Similarly, \mediN and \mediP were rated as more {\it professional} contexts than \recomP and \recomN (Mann-Whitney U test with $p<0.001$ on each pair of the former and latter groups). Fig.\ref{fig:situ} in the \appsec{sec:contextual-prop-and-cog-engagement} illustrates the characteristics of the six different decision scenarios and their impact on participants' comprehension processes.

\textbf{Participants' cognitive engagement tend to differ across sociotechnical contexts.}\label{sec:scenario-properties} We also find that, depending on decision contexts with different contextual properties, participants tend to exhibit different level of having motivations, opportunities, and abilities. For instance, participants in more high-stakes scenarios, such as \loanN and \mediN, showed higher {\it motivation} to investigate the information due to perceived benefits or threats from the consequences of the decision, compared to those in less high-stakes scenarios, such as \recomP and \recomN (Mann-Whitney U test with $p<0.001$ on every pair of the former and latter groups).
In the scenarios with more professional knowledge required (\mediP and \mediN), participants generally felt less capable of making sense of the information on their own. This was supported by the Mann-Whitney U test results, showing a significant difference ($p < 0.001$) between \mediP (\mediN) and almost all other scenarios. In a time-sensitive scenario such as \drivN, participants were more likely to perceive a lower {\it opportunity} to evaluate options thoroughly. The average rating for {\it intrinsic} cognitive capacity was greater than 3.5 across all scenarios (higher values indicate more manageable complexity), suggesting that the inherent level of difficulty of the given information was considered manageable in all cases.

\textbf{Participants' cognitive load tend to change with different explanation strategies.}\label{sec:exp-overall} Fig.~\ref{fig:overallPref} shows the relative ratings of each of the six explanation variants for the {\it extraneous} and {\it germane} cognitive load, as well as for the preference rankings in terms of different value aspects. To facilitate comparison, all ranking results were inversely coded such that a higher value indicates positive rating (see Section~\ref{sec:stat} for details).

Overall, explanation styles with higher information complexity (\comp and \cbho) received significantly lower \textit{extraneous} ratings (Wilcoxon signed rank test with $p<0.001$), whereas others with lower complexity received higher ratings on extraneous load. The \cf explanation with the lowest amount of information received a relatively high {\it extraneous} rating than all others. This suggests that presenting a small amount of information made participants feel easier to distinguish between important and unimportant information when presented with explanation styles other than the \comp and \cbho. However, this did not necessarily lead to an enhanced understanding of why you were given the decision (i.e., higher \textit{germane} load). Participants tended to give a lower {\it extraneous} rating but the highest {\it germane} rating to the \comp explanation (the Wilcoxon signed rank test showed a significantly lower {\it extraneous} rating for \comp than for four of the others, as well as a significantly higher {\it germane} rating for \comp than for others, with all $p<0.001$), suggesting that while the \comp explanation is not perceived as effective in how the information is presented, it helps enhance their understanding of the given decision.
\subsection{How do these factors collectively influence the preference on valuation of contrastive and selective explanations?}
Considering all factors including individual-, context-, and explanation-dependent factors, we investigate the interplay of factors in affecting the valuation of explanations.
As described in Section~\ref{sec:stat}, structural equation models (SEM) are used to test the effects of various factors in explaining whether a participant prefers a particular explanation or not for five explanatory values. Fig.~\ref{fig:explanation-analysis} provides a summary of the significant factors identified by the path analysis for each explanatory strategy. For predicting five explanatory values in each of the explanation styles (e.g., \cbhe), the estimated standardized effects of the significant factors are shown with a 95\% confidence interval.

\begin{figure*}[t]%
\centering
\includegraphics[width=\textwidth]{figures/SEM-all-updated.pdf}
\caption{Path analysis results for each explanation strategy. Factors significantly influencing the valuation process of each explanation strategy identified in per-strategy path analysis are highlighted.}\label{fig:explanation-analysis}
\end{figure*}

\textbf{Overview.} Our path analysis reveals a combination of direct and indirect effects of individual-, context-, and explanation-dependent factors on the valuation process. Across the results of path analysis over six strategies, the most impactful factor was \textit{germane} load, directly impacting all explanatory values. Conversely, for \cf and \cbho, \textit{intrinsic} load was significantly higher in high-stakes and professional contexts (\loanN, \mediN, \mediP). 

Sociotechnical contexts also significantly impacted the valuation of explanations either directly or indirectly through cognitive engagement. In those cases, contexts were primarily high-stakes or professional, affecting the explanatory values indirectly by invoking people to perceive higher motivation, lower opportunity, or lower ability (in the high-stakes or highly professional contexts), while also directly influencing valuation due to given contexts. 

For individual characteristics, some demographic traits were significantly associated with decision-making styles (Male $\rightarrow$ Rational (***0.19\footnote{Standardized path coefficient with statistical significance (*p $\leq$ 0.05, **p $\leq$ 0.01, ***p $\leq$ 0.001)}); Younger $\rightarrow$ Dependent(**0.12) and Avoidant (***0.14); HigherEdu $\rightarrow$ Dependent (**0.10) and Intuitive (**-0.12)). The decision-making styles, in turn, tended to influence cognitive engagement. For instance, rational thinkers tended to demonstrate higher motivation (*0.07), opportunity (***0.12), and ability (*0.07), different than avoidant thinkers who tended to have lower ability (**-0.11). These associations together formulate indirect effects of demographic traits towards cognitive engagement via demographic styles. For example, younger individuals, often avoidant, tended to have lower perceived ability. On the other hand, extraneous and germane loads were not significantly associated with individual characteristics.  

\textbf{Strategy-specific characteristics of valuation process.} The SEMs fitted to each explanation strategy provides insights into how the valuation process differs across strategies.

First, \comp explanations' values were mostly shaped by positive direct effects of individual and context-dependent factors. For example, \comp was found more understandable for users with higher education or within medical decision contexts (\mediP, \mediN) requiring advanced professional knowledge. Conversely, \ctt was the counterpart of \comp, as indicated by negative direct associations in a professional context (\mediP $\rightarrow$ Sufficiently-detailed; **-0.11) and positive associations in a low-stakes context (\recomN $\rightarrow$ Preference; **0.12). 

For \cf explanations, both indirect and direct effects of context-dependent factors were notable. \cf was highly valued in \loanN across four explanatory values and in high-stakes contexts due to their high intrinsic load (\mediN, \mediP, \drivN $\rightarrow$ Intrinsic; *0.11, ***0.18, ***0.17, Intrinsic $\rightarrow$ Preference; *0.06), or when individuals have higher perceived ability (Ability $\rightarrow$  Understandable, Sufficiently-detailed; *-0.09, *-0.07). The valuation process of \cbho and \cbhe were the opposite of \cf's. \cbho was less valued directly in \loanN context or indirectly in contexts associated with higher \textit{intrinsic} load (Intrinsic $\rightarrow$ Preference; ***-0.11). \cbhe was positively valued as useful when individuals have lower perceived ability (Ability $\rightarrow$ Useful; ***-0.1).
\section{Discussion}\label{sec:discussion}
% Recent research in explainable AI (XAI) has sought to theorize about the characteristics of user-friendly explanations and cognitive frameworks. In the following sections, we discuss how our findings both support and challenge these existing frameworks, and offer  implications for the design of XAI interfaces (e.g., AI chatbots), with an emphasis on promoting interactive and personalized explanation processes.
Recent research in explainable AI (XAI) has focused on identifying the key features of user-friendly explanations and exploring cognitive frameworks that enhance users' understanding of AI-assisted decisions. In this section, we discuss how our findings both support and challenge existing theories of explanation design. We also provide practical recommendations for designing XAI interfaces, such as AI chatbots, highlighting the need to prioritize interactive and personalized explanation processes tailored to diverse user profiles and decision-making contexts.

\subsection{Effectiveness of Contrastive and Selective explanations}
% Alignment between analysis results and the properties of human-friendly explanations
% \label{sec:discussion-properties} 
% As discussed in Section \ref{sec:intro}, Miller \cite{miller2019explanation}, the properties of human-friendly explanation, particularly contrastive and selective explanations, have gained attention in a number of recent XAI literature and been used as a theoretical backbone in some studies \cite{SelectiveMutableDialogicXAIReview, SelectiveExplanationsLeveragingHumanInput}. Overall, our study finds empirical evidence that challenges two properties in AI-assisted decision contexts -- (1) \textbf{Contrastive}: In our study, contrastive explanation was not perceived as the most preferred or understandable in some contexts. Contrary to findings advocating for its intuitive and human-friendly nature, their values were highly dependent to given contexts (e.g., \ctt in time-sensitive and \cto in a win or loss situation). Participants' preferences also differentiated based on contrastive strategies (i.e., how and who/what to compare), for example, \cto or \ctt with contrastive received as actionable and relevant respectively depending on whether to contrast with other with the opposite outcome or previous state of myself. As described in Section \ref{sec:contrastive}, some participants perceived \cto explanation as less relevant than \ctt or \cf as it was not solely pertaining to themselves; (2) \textbf{Selective}: We found evidence from the analysis of open-ended responses that this property does not hold in general but should be understood in a nuanced manner. As summarized in Section \ref{sec:selective}, some participants found certain explanations preferable when they mention particular types of features (e.g., annual income, browsing history) that conform to prior beliefs. However, this type of rationale on their preferences were particularly salient in the loan approval or movie recommendations which require less professional knowledge compared with medical contexts with only a few of responses regarding information selectivity.

% Overall, our study challenges the assumption that selective and contrastive explanations are not always intuitive or advantageous. Instead, these strategies should be carefully introduced into AI systems with an understanding of individual and contextual differences.

As discussed in Section \ref{sec:intro}, the properties of human-friendly explanations---especially contrastive and selective explanations---have garnered increasing attention in recent XAI research and are often used as theoretical foundations \cite{SelectiveMutableDialogicXAIReview, SelectiveExplanationsLeveragingHumanInput}. However, our study presents empirical evidence that challenges two key properties in AI-assisted decision-making contexts:

\begin{enumerate}
\item {\bf Contrastive} explanations: In some contexts, contrastive explanations were not perceived as the most understandable or preferred. While earlier studies highlight their intuitive and human-friendly nature, our findings suggest their effectiveness is highly context-dependent (e.g., contrastive explanations were less effective in time-sensitive situations or when comparing outcomes like winning vs. losing). Participant preferences also varied based on the type of contrast (e.g., comparing with others' outcomes or with their own previous state). For example, when contrasting with another individual's outcome (\cto), participants found explanations actionable, whereas comparing their current state to a previous one (\ctt) was perceived as more relevant to personal decision-making. However, as discussed in Section \ref{sec:contrastive}, some participants found \cto explanations less relevant, as they often focused on their own circumstances.
\item {\bf Selective} explanations: Our analysis of open-ended responses revealed that selective explanations should not be assumed universally effective. Participants preferred explanations that highlighted specific features (e.g., annual income, browsing history), especially when those features aligned with their expectations. However, this preference was more pronounced in simpler decision contexts, such as loan approvals or movie recommendations, where professional knowledge is not required. In more complex contexts, like medical decision-making, participants were less likely to express preferences related to selective information, as noted in Section \ref{sec:selective}.
\end{enumerate}

Overall, our findings challenge the assumption that contrastive and selective explanations are universally intuitive or advantageous. These strategies should be carefully tailored to individual needs and contextual factors when incorporated into AI systems.


\subsection{Need to Balance Individual Engagement and Cognitive Load}
%Need of personalized cognitive device or framework.

% Recent XAI literature has introduced cognitive devices or frameworks such as cognitive forcing \cite{TrustThinkCognitiveForcing} or evaluative AI \cite{miller2023explainable} to enhance user engagement and mitigate the risk of over-reliance problems. These studies propose cognitive interfaces to let them engage in slow and deliberate thinking mode \cite{kahneman2011thinking} or a framework to let users contrast between multiple explanation options with different outcomes and generate hypotheses rather than recommendation-driven explanations. While these ideas offer ways to enhance user engagement, such mechanisms can impose a higher cognitive load especially for individuals predisposed to intuitive thinking. It can be hardly expected that participants who do not attend to explanations presented by systems at hand will actively engage in more complex explainability tasks. Our findings suggest that such explainability mechanisms need to be tailored to individual level of engagement rather than adopted as one-size-fits-all solutions. 

Recent XAI literature has introduced cognitive frameworks, such as cognitive forcing \cite{TrustThinkCognitiveForcing} and evaluative AI \cite{miller2023explainable}, aimed at enhancing user engagement and mitigating the risk of over-reliance. These studies suggest using cognitive interfaces that encourage slow, deliberate thinking \cite{kahneman2011thinking} or offer multiple explanations with contrasting outcomes, allowing users to generate their own hypotheses rather than relying solely on system-driven recommendations. While these approaches can improve engagement, they also risk increasing cognitive load, particularly for users inclined toward intuitive thinking. It is unrealistic to expect participants who typically overlook system-generated explanations to actively engage with more complex explainability tasks. Our findings suggest that these approaches should be tailored to individual levels of engagement rather than implemented as one-size-fits-all solutions.

% \textbf{Design Implications for AI Chatbot Interfaces to Facilitate Interactive and Personalized Explanations} Given the recent widespread use of chat-based AI services or large language models, more interactive forms of explanatory processes may take place in chatbot interfaces to provide rationales for AI-based decisions. Our study suggests several design implications for such interactive systems: 

% \begin{itemize}
%     \item \textbf{Personalized explanations}: Given various factors within individuals' cognitive traits and sociotechnical contexts, it is recommended to tailor explanations based on the cognitive states of users. This can be achieved by prompting questions to gather information about the user's level of engagement, knowledge, and contextual properties, and then tailoring the explanations to address the unique needs of each user. With sufficient user data available, a data-driven approach can be employed to accurately infer a user's cognitive states and other crucial attributes such as their demographic traits and decision-making styles. To alleviate potential errors in prediction, it is recommended to offer clear explanations about the methods used to make the prediction and also provide the users with the opportunity to rectify any incorrect predictions.
%     \item \textbf{Impact of prior knowledge}: To prevent the risk of biased preferences towards explanations that conform to incorrect prior knowledge (Section \ref{sec:selective}), the system can support participants to select a part of texts on explanations or press interfaces such as a question mark or ask icon to know more about features in the explanation (e.g., what does the feature indicate? How is it attributed as important?)
%     \item \textbf{Interactive generation of explanation strategies}: The systems can offer an interface with explicit options to let participants explore and choose various explanation strategies or modify explanations towards desirable explanatory properties such as simple, detailed, easy, or professional (Section \ref{sec:exp-properties}) or explanatory values such as understandable, trustworthy, learnable, actionable, selective, caring, or privacy-preserving.
% \end{itemize}

\subsection{The Role of Content and Tone in Shaping User Perceptions of AI Explanations}
%Negative Perception on Explanations Influenced by Content and Tone

% Our research discovered that explanation design has the potential to influence participants' negative perceptions. While most explanations were well received, 15 participants expressed dissatisfaction, particularly with the use of demographic information.
% Five people in the study reported feeling uneasy; one of them said, ``{\it [Anything] involving demographics or comparisons raises questions of profiling.}'' ``{\it [The] data they collect on me makes me distrust the system},'' said another participant. These responses suggest that the use of personal and comparative data can make some people feel uneasy or even distrustful of AI systems. 
% Moreover, our study revealed that the emotional tone of explanations is crucial. Six participants emphasized the importance of the explanation's tone, with one participant suggesting that "{\it explanations shouldn't sound cold or sarcastic.}" They favored AI systems that mimic human-like interactions by communicating politely and conversationally. This implies that creating a good explanation requires not only clarity and utility, but also emotional satisfaction. Although the emotional aspect of explanations was outside the scope of our study, it has been identified as a crucial factor in explanation design, especially in chat support scenarios, as noted in previous research \cite{UnderstandingBenefitsChallengesDeployingConversationala}.

Our research revealed that the design of explanations significantly affects participants' perceptions. While most explanations were well received, 15 participants expressed dissatisfaction, particularly with the use of demographic data. One participant noted, ``{\it [Anything] involving demographics or comparisons raises questions of profiling,}'' while another expressed distrust in the system, stating, ``{\it [The] data they collect on me makes me distrust the system.}'' These responses indicate that incorporating personal or comparative data can cause discomfort and raise concerns about privacy and fairness in AI systems.

Additionally, the tone of explanations emerged as an important factor in shaping user perceptions. Six participants stressed that the explanations should not sound ``{\it cold or sarcastic.}'' They preferred AI systems that communicated in a polite, conversational manner, suggesting that the emotional tone is as crucial as the content itself. While our study primarily focused on clarity and utility, these findings highlight the importance of incorporating human-like warmth into AI explanations, particularly in customer support scenarios, as noted in prior research \cite{UnderstandingBenefitsChallengesDeployingConversationala}.


\subsection{Design implications for AI chatbot interfaces}

With the growing adoption of chat-based AI services and large language models, chatbot interfaces have the potential to provide more interactive explanations for AI-driven decisions. Our study highlights several design recommendations for these systems:

\begin{itemize}
    \item {\bf Personalized explanations:} Explanations should be tailored to individual users based on their cognitive traits and contextual factors. This can be achieved by gathering information about the user's engagement, knowledge, and circumstances through interactive prompts. Where sufficient data is available, a data-driven approach can infer user preferences and cognitive states, though it is crucial to explain how these predictions are made and allow users to correct any inaccuracies.
\item {\bf Managing prior knowledge:} To prevent users from favoring explanations that reinforce incorrect prior knowledge (Section \ref{sec:selective}), the system should enable users to explore specific parts of the explanation. Features like question icons or clickable sections can provide further clarification (e.g., ``What does this feature mean?'' or ``Why is this feature important?'').
\item {\bf Interactive explanation customization:} Systems should offer users options to customize the style of explanations they receive. Users could select between simple or detailed explanations, or those that emphasize values such as trustworthiness, clarity, or privacy (Section \ref{sec:exp-properties}). This interactivity allows users to better control their experience and improve the relevance of the information provided.
\end{itemize}




% \subsubsection{Negative Perception on Explanations Influenced by Content and Tone.} 
% We discovered that explanation design has the potential to influence participants' negative perceptions. For example, 15 participants expressed their uneasy or distrustful feeling with the use of demographic information, for example, ``{\it [The] data they collect on me makes me distrust the system},''. Six participants emphasized the importance of the explanation's tone, such as "{\it explanations shouldn't sound cold or sarcastic.}" They favored AI systems that mimic human-like interactions by communicating politely and conversationally. This implies that creating a good explanation requires not only clarity and utility, but also emotional satisfaction.

% \subsection{Limitations}
% In this section, we summarize the limitation and the possible future work in three aspects: study scope, explanation design, and survey design.

% \paragraph{Application domains} The goal of our research was to simulate AI-assisted decision-making for data subjects, with explanations serving as an algorithmic rationale for AI's actions. However, the field of AI-assisted decision-making extends beyond this, involving a broad spectrum of users, tasks, and various aspects of human-AI interaction. For instance, previous research has considered problem-solving tasks, such as sentiment or content classification \cite{TrustThinkCognitiveForcing,HumanAIInteractionHealthcareThree}, or proxy tasks, such as maze-solving \cite{ExplanationsCanReduceOverrelianceAI}. These scenarios typically involve laypeople who, despite a lack of specialist knowledge, are required to make final decisions with the assistance of AI. Other studies have looked at professional contexts such as medical diagnosis \cite{DesigningAITrustCollaborationTimeConstrained, ExplainableAIDeadLongLive}, where clinicians use AI to help them diagnose diseases. Although our findings, such as individual differences in how explanations are valued based on cognitive characteristics and contexts, may largely apply to these tasks, other tasks may involve different human roles or AI-human collaborations. This could alter the value placed on different explanation strategies. Nonetheless, the nuances of each task, as well as its unique human-AI dynamics, should guide the explanation strategy approach.

% \paragraph{Algorithm or explanation aversion} Our survey was not designed to investigate the phenomenon known as explanation or algorithm aversion \cite{logg:2019algorithm, dietvorst:2015algorithm}, which refers to a dislike or distrust of AI explanations or systems. The reason for this is that our survey was not meant to compare the preference for explanations to the absence of explanations, but rather to evaluate preferences for various explanation strategies. For example, out of all our participants, only three expressed distrust with comments like, "{\it I don’t trust or respect its explanations}" and "{\it AI should not be diagnosing.}" Unfortunately, the design of our study did not permit us to delve further into these negative sentiments expressing algorithm aversion (or, conversely, algorithm appreciation). Understanding the nuance of these attitudes would benefit greatly from a more in-depth study. By contrasting a preference for explanations versus no explanation, we could gain a deeper understanding of the cognitive and contextual factors that may contribute to explanation or algorithm aversion.

% \paragraph{Measuring the perceived understandability.} Our study was intended to evaluate the perceived impact of explanation strategies. Given the subjective nature of our survey, the results may not provide an objective indication of participants' enhanced understanding of decisions regarding some factors such as understandability or germane load. In light of this limitation, future research could aim to evaluate how participants accurately make sense of decisions with the help of given explanations, under tasks designed to measure accuracy, task completion, or response time.

% \paragraph{Explanation quality.} Our study assumes that the explanations used in our experiment are accurate and faithful to the prediction. However, there is a potential for these explanations to be designed inaccurately or with malicious intent to manipulate users into accepting a decision. To cope with this issue, XAI systems may necessitate enhanced transparency, such as providing information on uncertainty or the input used to generate explanations. Exploring user vulnerability to ill-designed explanations could be a potential avenue for future research, involving comparative analysis and an investigation into user behavior and perception.


\subsection{Limitations and Future Directions}

This section outlines the study's limitations and potential future research across four key areas: application domains, explanation aversion, understandability measures, and explanation quality.

\paragraph{Application domains}
Our research focused on simulating AI-assisted decision-making for human data subjects, where explanations provide a rationale for the AI's actions. However, the broader field of AI-assisted decision-making encompasses a wide variety of tasks, users, and interactions. Previous studies have examined tasks such as sentiment analysis, content classification \cite{TrustThinkCognitiveForcing,HumanAIInteractionHealthcareThree}, and maze-solving \cite{ExplanationsCanReduceOverrelianceAI}, typically involving laypeople making decisions with AI assistance. In contrast, other studies have explored professional settings like medical diagnosis \cite{DesigningAITrustCollaborationTimeConstrained, ExplainableAIDeadLongLive}, where AI supports clinicians in diagnosing diseases. While our findings on individual differences in valuing explanations may apply to these diverse tasks, other domains may involve different human roles or AI-human collaboration dynamics, potentially altering the effectiveness of various explanation strategies. Future research should consider the unique requirements and challenges of specific application domains to refine explanation strategies accordingly.

\paragraph{Algorithm and explanation aversion}
Our survey did not specifically examine the phenomenon of algorithm or explanation aversion \cite{logg:2019algorithm, dietvorst:2015algorithm}, where users exhibit a general distrust of AI systems or their explanations. Instead, we focused on assessing preferences for different explanation strategies. Only three participants expressed explicit distrust, with comments such as, "{\it I don't trust or respect its explanations}" and "{\it AI should not be diagnosing.}" However, due to the design of our study, we were unable to explore these negative perceptions in detail. Future research should delve deeper into these attitudes, possibly by comparing user preferences for receiving explanations versus none. Gaining a clearer understanding of the cognitive and contextual factors that contribute to explanation or algorithm aversion could help in designing more effective and trustworthy AI systems.

\paragraph{Measuring perceived and actual understandability}
While our study focused on evaluating the perceived effectiveness of various explanation strategies, subjective perceptions alone may not fully capture the depth of participants' understanding. Perceived understandability is important, as it influences user satisfaction and trust, but it doesn't necessarily indicate how well users comprehend AI-driven decisions. Future research may incorporate objective metrics---such as task accuracy, decision-making speed, and users' ability to make informed decisions based on explanations---to better gauge actual understanding. By combining subjective and objective measures, researchers can gain a holistic view of how explanation strategies impact both user confidence and cognitive comprehension. This would provide more robust insights into the design of AI systems that not only explain decisions clearly but also empower users to act on them effectively. Improving both perceived and actual understanding will be crucial in building trustworthy, reliable AI systems that meet user needs.

\paragraph{Explanation quality}
In our study, we assumed that the explanations provided were accurate and trustworthy. However, there is always a risk that explanations could be designed to mislead or manipulate users. To address this concern, future XAI systems should prioritize transparency, offering users information about uncertainties or the data used to generate the explanations. Investigating user vulnerability to misleading explanations could be an important area of future research. This would involve exploring how users react to inaccurate or biased explanations and developing safeguards to protect against manipulation.

\section{Conclusion}\label{sec:conclusion}

% Prior studies have shown inconsistent findings on the effect of explanation strategies. Based on a review of literature in cognitive science, our study seeks to investigate a deeper understanding of what factors shape individuals’ valuation of different explanation strategies. We designed a scenario-based survey and conducted between-subject experiment that distilled the complex relationship among human valuations, explanations, and various contextual properties, built on social and cognitive science literature. Our findings showed that the properties of human-friendly explanations -- selective and contrastive -- are not necessarily advantageous: complete explanations are the most preferable for most values; selective explanations are preferable much more in less professional context; and participants prefer different comparative strategies for different reasons rather than contrastive explanations. Throughout the analysis, we found that various factors ranging from demographic traits, cognitive load to contexts impact individual differences in favoring different types of explanation and comparative strategies. We discussed that these findings imply that cognitive devices or framework that promote deliberate thinking and tasks with a one-size-fits-all  approach may fail to successfully engage in individuals with different cognitive abilities.

% In the upcoming era of AI systems affecting nearly everyone's lives, it is urgently needed to tailor the explainability to individuals’ contexts. In this context, this study provides a nuanced view of explanation strategies, and conveys implications for designing more interactive and personalized AI services to accommodate individual and context differences. We believe this will make ways to improve user interaction and trust, and make the technology more accessible. 

Previous research on explanation strategies has yielded mixed results, leaving gaps in understanding how individuals value different types of explanations. Drawing from cognitive and social science literature, our study aimed to explore the specific factors influencing user preferences for various explanation strategies. Using a scenario-based survey and a between-subject experiment, we investigated the relationships between explanation types, human valuations, and contextual factors.

Our findings challenge the assumption that human-friendly explanations, such as selective and contrastive strategies, are always beneficial. We showed that complete explanations were generally preferred across most values, while selective explanations were favored primarily in less professional contexts. Additionally, participants showed diverse preferences for comparative strategies, instead of the contrastive approach typically emphasized in prior studies.

We identified several key factors---demographics, cognitive load, and contextual conditions---that significantly shape individual preferences for explanation and comparative strategies. These results suggest that one-size-fits-all cognitive frameworks, including those promoting deliberate thinking, may fail to effectively engage users with varying cognitive abilities and needs.

As AI systems increasingly influence daily decision-making, there is an urgent need to personalize explainability according to users' individual contexts and cognitive traits. Our study offers actionable insights for designing AI systems that deliver more interactive and tailored explanations. By accommodating user diversity, these systems can enhance user engagement, foster trust, and make AI technologies more accessible to a broader audience.


%%
%% The acknowledgments section is defined using the "acks" environment
%% (and NOT an unnumbered section). This ensures the proper
%% identification of the section in the article metadata, and the
%% consistent spelling of the heading.
% \begin{acks}
% To Robert, for the bagels and explaining CMYK and color spaces.
% \end{acks}

%%
%% The next two lines define the bibliography style to be used, and
%% the bibliography file.
\bibliographystyle{ACM-Reference-Format}
\bibliography{references, references-zotero}

%%
%% If your work has an appendix, this is the place to put it.
\appendix
% \begin{appendices}
% \end{appendices}

% \setcounter{table}{0}
% \renewcommand{\thetable}{S\arabic{table}}%
% \setcounter{figure}{0}
% \renewcommand{\thefigure}{S\arabic{figure}}%

\clearpage
\section{Demographic breakdown}\label{sec:demo-breaks}

\begin{figure}[H]%
\centering
\includegraphics[width=0.5\columnwidth]{figures/demographics}
\vspace{-1em}
\caption{Demographics of participants.}\label{fig:demographics}
\vspace{-1em}
\end{figure}

We collected 839 responses to the survey using the Prolific crowdsourcing platform. Participants had to be 18 or older, fluent in English, and reside in the United States at the time of the survey to ensure the study's validity. To control the quality of these responses, we filtered out those that met one of the exclusion criteria: (1) all five ranking answers (including overall preference and four explanatory values) are identical; (2) the ranking answer for overall preferences was only modified once from its default ranking. This procedure yielded 698 responses as our analysis pool. Based on four demographic questions in the survey, we found that participants were fairly distributed by gender (female: 55.1\%, male: 42.4\%) and age (younger (18-54): 76.7\%, older ($\ge$ 55): 23.4\%).  The majority of the population had a bachelor's degree or higher (55.1\%), and they were the majority across four racial and ethnic groups (White: 77.0\%, Asian: 7.1\%, Black: 6.8\%, Hispanic: 6.6\%, Others: 2.4\%).

% \subsection{Context-wise valuation of explanation strategies}\label{sec:exp-values-in-contexts}

% \begin{figure}[H]%
% \centering
% \includegraphics[width=.7\columnwidth]{figures/overallPref_by_context}
% \vspace{-2em}
% \caption{In each scenario, the relative ratings (along the $x$-axis) for each of the six explanation variants for the {\it extraneous} and {\it germane} cognitive capacities, as well as the preference rankings in five distinct value dimensions. To facilitate the summary, participants' {\it overall} preference was highlighted in gray.}\label{fig:overallPref-context}
% % \vspace{-0.75em}
% \end{figure}


\section{Distinct characteristics of six decision scenarios and participants' different level of cognitive engagement}\label{sec:contextual-prop-and-cog-engagement}
\begin{figure}[h]%
\centering
\includegraphics[width=0.5\columnwidth]{figures/situ_properties_intrinsic_by_context}
\vspace{-2em}
\caption{Decision scenarios have distinct characteristics ({\it high-stakes}, {\it professional}, and {\it timely}) that influence participants' comprehending processes, as measured by three MOA variables (individuals' motivation, opportunity, and ability) and the {\it intrinsic} cognitive load (the level of information difficulty covered in the explanations).}\label{fig:situ}
\vspace{-1.5em}
\end{figure}

% \subsection{Properties of explanation strategies in the analysis of open responses}\label{sec:exp-properties-in-open-responses}
% \begin{figure}[H]%
% % \vspace{-1em}
% \centering
% \includegraphics[width=1\columnwidth]{figures/exp-properties.pdf}
% % \vspace{-2.5em}
% \caption{The most frequent explanatory properties in association with the explanation strategies.}
% \label{fig:exp-properties-in-open-responses}
% \end{figure}

% \section{Decision-making styles by demographic information}
% \input{figs/figA2}

\section{Survey material}\label{sec:survey}

\begin{figure}[h]%
\centering
\includegraphics[width=\columnwidth]{figures/survey1.pdf}
\caption{Introduction and basic demographics page in the survey.}\label{fig:survey1}
\end{figure}

\begin{figure}[h]%
\includegraphics[width=\columnwidth]{figures/survey2.pdf}
\caption{Scenario, explanations, and cognition page in the survey.}\label{fig:survey2}
\end{figure}

\begin{figure}[h]%
\centering
\includegraphics[width=\columnwidth]{figures/survey3.pdf}
\caption{Questions on explanatory value in the survey.}\label{fig:survey3}
\end{figure}

\clearpage
\onecolumn
\section{Definitions of Individual/Cognitive Variables}
\addtolength{\tabcolsep}{2pt}
\begin{table*}[h]
\caption{Variables in cognitive load, decision styles, and Motivation-Opportunities-Ability (MOA) model}
\label{tab:cog_vars}
\centering
\begin{NiceTabular}[t]{@{}p{2cm}ll@{}}
\CodeBefore 
   % \rowcolors{2}{gray!20}{white}
\Body
\toprule
\textbf{Variable}                                          & \textbf{Definition}                                                                                                                                                              & \textbf{Scale}                                                                                                                                                                                    \\ \midrule
\multicolumn{3}{l}{\textbf{Cognitive Load Variables} \cite{CognitiveLoadTheoryFormatInstruction,Workingmemorylimitedimprovingknowledge, CognitiveArchitectureInstructionalDesign}}                                                                                                                                                                                                                                                                                                                                                                                             \\ \midrule
\begin{tabular}[t]{@{}l@{}}Intrinsic load\end{tabular}  & \begin{tabular}[t]{@{}l@{}}The inherent level of difficulty associated with \\a specific topic or context\end{tabular}                                                    & \begin{tabular}[t]{@{}l@{}}How difficult is the information covered\\ in the explanations overall for you to understand?\end{tabular}                                                          \\[1.5em]
\begin{tabular}[t]
{@{}l@{}}Extraneous load\end{tabular} & \begin{tabular}[t]{@{}l@{}}Ineffective load by the manner in which information \\ is presented to individuals and is under the control of \\ material\end{tabular} & \begin{tabular}[t]{@{}l@{}}How difficult is each explanation for you \\ to distinguish important and unimportant \\ information for your decision-making from these \\ explanations?\end{tabular} \\[4em]
\begin{tabular}[t]{@{}l@{}}Germane load\end{tabular}    & \begin{tabular}[t]{@{}l@{}}Effective cognitive load devoted to integrating new \\ information, the creation and modification of schema\end{tabular}                           & \begin{tabular}[t]{@{}l@{}}How did each explanation enhance your \\ understanding of why you were given the decision?\end{tabular}                                                             \\ \midrule
\multicolumn{3}{l}{\textbf{Decision Style Variables } \cite{DecisionMakingStyleDevelopmentAssessmentNew}}                                                                                                                                                                                                                                                                                                                                                                                             \\ \midrule
Rational                                                   & \begin{tabular}[t]{@{}l@{}}A thorough search for and logical evaluation of alternatives\end{tabular}                                                                          & \begin{tabular}[t]{@{}l@{}}I make decisions in a logical and systematic way\end{tabular}                                                                                                       \\[0.5em]
Intuitive & A reliance on hunches and feelings & \begin{tabular}[t]{@{}l@{}}When I make decisions, I tend to rely on my intuition\end{tabular}     \\[0.5em]
Avoidant                                                   & An attempt to avoid decision making                                                                                                                                              & \begin{tabular}[t]{@{}l@{}}I avoid making important decisions until the pressure\\ is on\end{tabular}                                                                                            \\[1.5em]
Dependent                                                  & \begin{tabular}[t]{@{}l@{}}A search for advice and direction from others\end{tabular}                                                                                         & \begin{tabular}[t]{@{}l@{}}I rarely make important decisions without consulting \\ other people\end{tabular}                                                                                      \\[1.5em]
Spontaneous                                                & \begin{tabular}[t]{@{}l@{}}A sense of immediacy and a desire to get through the \\ decision-making process as soon as possible\end{tabular}                                   & I generally make snap decisions                                                                                                                                                                   \\ \midrule
\multicolumn{3}{l}{\textbf{Motivation-Opportunities-Ability (MOA) model} \cite{EnhancingMeasuringConsumersMotivationOpportunity}}                                                                                                                   \\ \midrule
Motivation                                                 & \begin{tabular}[t]{@{}l@{}}Conscious and unconscious cognitive processes that \\ directand inspire behavior\end{tabular}                                                      & \begin{tabular}[t]{@{}l@{}}I feel it is necessary for me to deliberately think and \\ investigate the rationale behind the decision.\end{tabular}                                              \\[1.5em]
Opportunity                                                & \begin{tabular}[t]{@{}l@{}}External factors that make a behavior possible\end{tabular}                                                                                        & \begin{tabular}[t]{@{}l@{}}I feel I have enough time to deliberately think and \\ investigate the rationale behind the decision.\end{tabular}                                                  \\[1.5em]
Ability                                                    & \begin{tabular}[t]{@{}l@{}}Individual’s psychological and physical ability to participate\\ in an activity\end{tabular}                                                       & \begin{tabular}[t]{@{}l@{}}I feel I have the required domain knowledge to \\ understand the information regarding my status \\ used in this decision.\end{tabular}                                \\ \bottomrule
\end{NiceTabular}
\begin{tablenotes}
  \small
  \item *All variables were measured using 5-point Likert scale.
\end{tablenotes}
\vspace{-1em}
\end{table*}

% \section{Definitions of Explanation Strategies and Individual/Cognitive Variables}
% \begin{table*}[h]
\caption{Types of explanation strategies.}
\label{tab:strategies}
\setlength{\tabcolsep}{3pt}
\centering
\begin{tabular}{@{}lllll@{}}
\toprule
\multirow{2}{*}{Explanations} & \multicolumn{2}{l}{Contrastive strategies}                                                                                                                         & \multicolumn{2}{l}{Information selectivity}                                                                                                            \\ \cmidrule(l){2-5} 
                              & \begin{tabular}[c]{@{}l@{}}How to compare\end{tabular} & What/Who to compare                                                                                     & \begin{tabular}[c]{@{}l@{}}Information complexity\end{tabular} & \begin{tabular}[c]{@{}l@{}}Information alignment with \\  prior beliefs\end{tabular} \\ \midrule
\comp          & No strategy                                              & -                                                                                                       & Thorough (All the causes)                                        & Low bias, High variance                                                             \\
\cf            & Contrastive                                              & vs. me (hypothetical status of myself)                                                                  & Simple (Only minimal causes)                                     & High bias, Low variance                                                             \\
\cto           & Contrastive                                              & vs. others (with the opposite outcome)                                                                  & Simple (Only minimal causes)                                     & High bias, Low variance                                                             \\
\begin{tabular}[c]{@{}l@{}}\ctt \\ \newline  \end{tabular}        & \begin{tabular}[c]{@{}l@{}}Contrastive   \\ \newline   \end{tabular}                                        & \begin{tabular}[c]{@{}l@{}}vs. me (previous status of myself \\ with the opposite outcome)\end{tabular} & \begin{tabular}[c]{@{}l@{}}Simple (Only minimal causes)  \\ \newline \end{tabular}                                   & \begin{tabular}[c]{@{}l@{}} High bias, Low variance  \\ \newline  \end{tabular}                                                         \\
\cbhe          & Analogous                                                & vs. others (with the same outcome)                                                                      & Simple (Only minimal causes)                                     & High bias, Low variance                                                             \\
\cbho          & Analogous                                                & vs. others (with the same outcome)                                                                      & Thorough (All the causes)                                        & Low bias, High variance                                                             \\ \bottomrule
\end{tabular}
\begin{tablenotes}
  \small
  \item *Note that explanation strategies that entail contrasting the information (\cf, \cto, and \ctt) inherently deal with simple information by their definition.
\end{tablenotes}
\end{table*}
% \addtolength{\tabcolsep}{2pt}
\begin{table*}
\caption{Variables in cognitive load, decision styles, and Motivation-Opportunities-Ability (MOA) model}
\label{tab:cog_vars}
\begin{NiceTabular}[t]{@{}p{2cm}ll@{}}
\CodeBefore 
   % \rowcolors{2}{gray!20}{white}
\Body
\toprule
\textbf{Variable}                                          & \textbf{Definition}                                                                                                                                                              & \textbf{Scale}                                                                                                                                                                                    \\ \midrule
\multicolumn{3}{l}{\textbf{Cognitive Load Variables} \cite{CognitiveLoadTheoryFormatInstruction,Workingmemorylimitedimprovingknowledge, CognitiveArchitectureInstructionalDesign}}                                                                                                                                                                                                                                                                                                                                                                                             \\ \midrule
\begin{tabular}[t]{@{}l@{}}Intrinsic \\ load\end{tabular}  & \begin{tabular}[t]{@{}l@{}}The inherent level of difficulty \\ associated with a specific \\ topic or context\end{tabular}                                                    & \begin{tabular}[t]{@{}l@{}}How difficult is the information covered\\ in the explanations overall for you \\ to understand?\end{tabular}                                                          \\[3em]
\begin{tabular}[t]
{@{}l@{}}Extraneous \\ load\end{tabular} & \begin{tabular}[t]{@{}l@{}}Ineffective load by the manner in \\ which information is presented \\ to individuals and is under the control \\ of material\end{tabular} & \begin{tabular}[t]{@{}l@{}}How difficult is each explanation for you \\ to distinguish important and unimportant \\ information for your decision-making from \\ these explanations?\end{tabular} \\[4em]
\begin{tabular}[t]{@{}l@{}}Germane \\ load\end{tabular}    & \begin{tabular}[t]{@{}l@{}}Effective cognitive load devoted to \\ integrating new information, the \\ creation and modification of schema\end{tabular}                           & \begin{tabular}[t]{@{}l@{}}How did each explanation enhance your \\ understanding of why you were given the decision?\end{tabular}                                                             \\ \midrule
\multicolumn{3}{l}{\textbf{Decision Style Variables } \cite{DecisionMakingStyleDevelopmentAssessmentNew}}                                                                                                                                                                                                                                                                                                                                                                                             \\ \midrule
Rational                                                   & \begin{tabular}[t]{@{}l@{}}A thorough search for and \\ logical evaluation of alternatives\end{tabular}                                                                          & \begin{tabular}[t]{@{}l@{}}I make decisions in a logical and systematic way\end{tabular}                                                                                                       \\[1.5em]
Intuitive                                                  & A reliance on hunches and feelings                                                                                                                                               & \begin{tabular}[t]{@{}l@{}}When I make decisions, I tend to rely on my intuition\end{tabular}                                                                                                  \\[0.75em]
Avoidant                                                   & An attempt to avoid decision making                                                                                                                                              & \begin{tabular}[t]{@{}l@{}}I avoid making important decisions until the pressure\\ is on\end{tabular}                                                                                            \\[1.5em]
Dependent                                                  & \begin{tabular}[t]{@{}l@{}}A search for advice and \\ direction from others\end{tabular}                                                                                         & \begin{tabular}[t]{@{}l@{}}I rarely make important decisions without consulting \\ other people\end{tabular}                                                                                      \\[1.5em]
Spontaneous                                                & \begin{tabular}[t]{@{}l@{}}A sense of immediacy and a desire \\ to get through the decision-making \\ process as soon as possible\end{tabular}                                   & I generally make snap decisions                                                                                                                                                                   \\ \midrule
\multicolumn{3}{l}{\textbf{Motivation-Opportunities-Ability (MOA) model} \cite{EnhancingMeasuringConsumersMotivationOpportunity}}                                                                                                                   \\ \midrule
Motivation                                                 & \begin{tabular}[t]{@{}l@{}}Conscious and unconscious \\ cognitive processes that \\ directand inspire behavior\end{tabular}                                                      & \begin{tabular}[t]{@{}l@{}}I feel it is necessary for me to deliberately think and \\ investigate the rationale behind the decision.\end{tabular}                                              \\[2.75em]
Opportunity                                                & \begin{tabular}[t]{@{}l@{}}External factors that \\ make a behavior possible\end{tabular}                                                                                        & \begin{tabular}[t]{@{}l@{}}I feel I have enough time to deliberately think and \\ investigate the rationale behind the decision.\end{tabular}                                                  \\[1.5em]
Ability                                                    & \begin{tabular}[t]{@{}l@{}}Individual’s psychological and \\ physical ability to participate\\ in an activity\end{tabular}                                                       & \begin{tabular}[t]{@{}l@{}}I feel I have the required domain knowledge to \\ understand the information regarding my status \\ used in this decision.\end{tabular}                                \\ \bottomrule
\end{NiceTabular}
\begin{tablenotes}
  \small
  \item *All variables were measured using 5-point Likert scale.
\end{tablenotes}
\vspace{-1em}
\end{table*}

% \clearpage
% \subsection{Results of Bayesian regression models}\label{sec:model-results}
% \input{tabs/bayes-predict-extraneous}
% \input{tabs/bayes-predict-germane}
% \input{tabs/bayes-predict-overall}
% \input{tabs/bayes-predict-sufficient}
% \input{tabs/bayes-predict-understandable}
% \input{tabs/bayes-predict-trust}
% \input{tabs/bayes-predict-useful}


\end{document}
\endinput
%%
%% End of file `sample-sigconf-authordraft.tex'.
