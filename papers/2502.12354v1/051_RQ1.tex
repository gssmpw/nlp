\subsection{How do participants' perceived value of explanations differ, overall and within sociotechnical contexts?}\label{sec:values}

First, we examine how explanation strategies were perceived differently in terms of perceived values of explanations overall and within each decision context, which are summarized in Fig.~\ref{fig:overallPref} and Fig.~\ref{fig:overallPref-context} respectively.

Our analysis found \comp explanations were evaluated as higher in all explanatory values than any other explanation types. Conversely, \cbhe was considered the least favorable in all valuation aspects. \cf and \cbho were ranked second-to-worst in terms of {\it sufficient} and {\it understandable}, and {\it trustworthiness} and {\it usefulness} respectively.

\begin{figure}[H]%
\centering
\includegraphics[width=\columnwidth]{figures/overallPref-by-context}
\vspace{-2em}
\caption{In each scenario, the relative ratings (along the $x$-axis) for each of the six explanation variants for the {\it extraneous} and {\it germane} cognitive capacities, as well as the preference rankings in five distinct value dimensions. To facilitate the summary, participants' {\it overall} preference was highlighted in gray.}\label{fig:overallPref-context}
% \vspace{-0.75em}
\end{figure}

However, when analyzing the ratings of explanatory values for each decision scenario, we found that participants’ perceptions of these explanations varied depending on the scenarios they encountered. In high-stakes and time-sensitive scenarios such as \drivN, \loanN, and \mediN, three explanation styles stood out. The \ctt received a rating comparable to the \comp in \drivN. The \ctt explanation, ``\textit{The estimated arrival time was predicted 20 minutes later than usual because there was heavy snow last night ..., while your usual travel with on-time arrival happened without snow with good traffic flow},'' was considered to be a {\it sufficient}, {\it understandable}, {\it trustworthy}, and {\it useful} explanation for the AI-decision ``{\it your arrival time as being 20 minutes later than usual ...}'' This suggests that an explanation contrasting outcomes at different times is particularly applicable to highly time-sensitive situations. In \mediN, the \cto (``...\textit{You are at a high risk for Acute Coronary Syndrome because you have high cholesterol and blood pressure between 90-145. In contrast, a user who wasn't....}'') received ratings comparable to the \comp. This indicates that justifications for the AI's decision based on contrasting results due to different causes work well in medical contexts.

The \loanN scenario exhibited a distinct pattern of valuation compared to all other scenarios. Participants in this scenario favored the \comp less than those in other scenarios. On the other hand, the \cf (``{\it You could have been granted a loan if you had been employed with an annual income more than...}'') as well as \cto explanations were the most popular choices. This suggests that both explanation approaches were particularly useful in a situation where there was either a win or a loss, such as with \loanN.