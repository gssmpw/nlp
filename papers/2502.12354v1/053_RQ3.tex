\subsection{How do individual, contextual, and cognitive factors interact in the valuation process of explanations?} 

In the second part of our study, we conduct an in-depth analysis of how the distinct values of various explanation strategies, as discussed above, are shaped by a range of factors and their complex interactions. We begin by examining the relationships between sociotechnical and cognitive factors and how they interact with the evaluation of explanations.

\textbf{Sociotechnical contexts are perceived as having distinct contextual properties.}\label{sec:scenario-properties} First, we confirmed that the six scenarios in our study exhibited different characteristics along the three dimensions, including {\it high-stakes}, {\it professional}, and {\it timely} based on participants' ratings (Fig.\ref{fig:situ}). For instance, \loanN, \mediN, and \drivN were rated as more {\it high-stakes} and {\it timely} than \recomP and \recomN (Mann-Whitney U test with $p<0.001$ on each pair of the former and latter groups). Similarly, \mediN and \mediP were rated as more {\it professional} contexts than \recomP and \recomN (Mann-Whitney U test with $p<0.001$ on each pair of the former and latter groups). Fig.\ref{fig:situ} in the \appsec{sec:contextual-prop-and-cog-engagement} illustrates the characteristics of the six different decision scenarios and their impact on participants' comprehension processes.

\textbf{Participants' cognitive engagement tend to differ across sociotechnical contexts.}\label{sec:scenario-properties} We also find that, depending on decision contexts with different contextual properties, participants tend to exhibit different level of having motivations, opportunities, and abilities. For instance, participants in more high-stakes scenarios, such as \loanN and \mediN, showed higher {\it motivation} to investigate the information due to perceived benefits or threats from the consequences of the decision, compared to those in less high-stakes scenarios, such as \recomP and \recomN (Mann-Whitney U test with $p<0.001$ on every pair of the former and latter groups).
In the scenarios with more professional knowledge required (\mediP and \mediN), participants generally felt less capable of making sense of the information on their own. This was supported by the Mann-Whitney U test results, showing a significant difference ($p < 0.001$) between \mediP (\mediN) and almost all other scenarios. In a time-sensitive scenario such as \drivN, participants were more likely to perceive a lower {\it opportunity} to evaluate options thoroughly. The average rating for {\it intrinsic} cognitive capacity was greater than 3.5 across all scenarios (higher values indicate more manageable complexity), suggesting that the inherent level of difficulty of the given information was considered manageable in all cases.

\textbf{Participants' cognitive load tend to change with different explanation strategies.}\label{sec:exp-overall} Fig.~\ref{fig:overallPref} shows the relative ratings of each of the six explanation variants for the {\it extraneous} and {\it germane} cognitive load, as well as for the preference rankings in terms of different value aspects. To facilitate comparison, all ranking results were inversely coded such that a higher value indicates positive rating (see Section~\ref{sec:stat} for details).

Overall, explanation styles with higher information complexity (\comp and \cbho) received significantly lower \textit{extraneous} ratings (Wilcoxon signed rank test with $p<0.001$), whereas others with lower complexity received higher ratings on extraneous load. The \cf explanation with the lowest amount of information received a relatively high {\it extraneous} rating than all others. This suggests that presenting a small amount of information made participants feel easier to distinguish between important and unimportant information when presented with explanation styles other than the \comp and \cbho. However, this did not necessarily lead to an enhanced understanding of why you were given the decision (i.e., higher \textit{germane} load). Participants tended to give a lower {\it extraneous} rating but the highest {\it germane} rating to the \comp explanation (the Wilcoxon signed rank test showed a significantly lower {\it extraneous} rating for \comp than for four of the others, as well as a significantly higher {\it germane} rating for \comp than for others, with all $p<0.001$), suggesting that while the \comp explanation is not perceived as effective in how the information is presented, it helps enhance their understanding of the given decision.