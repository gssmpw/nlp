% \begin{appendices}
% \end{appendices}

% \setcounter{table}{0}
% \renewcommand{\thetable}{S\arabic{table}}%
% \setcounter{figure}{0}
% \renewcommand{\thefigure}{S\arabic{figure}}%

\clearpage
\section{Demographic breakdown}\label{sec:demo-breaks}

\begin{figure}[H]%
\centering
\includegraphics[width=0.5\columnwidth]{figures/demographics}
\vspace{-1em}
\caption{Demographics of participants.}\label{fig:demographics}
\vspace{-1em}
\end{figure}

We collected 839 responses to the survey using the Prolific crowdsourcing platform. Participants had to be 18 or older, fluent in English, and reside in the United States at the time of the survey to ensure the study's validity. To control the quality of these responses, we filtered out those that met one of the exclusion criteria: (1) all five ranking answers (including overall preference and four explanatory values) are identical; (2) the ranking answer for overall preferences was only modified once from its default ranking. This procedure yielded 698 responses as our analysis pool. Based on four demographic questions in the survey, we found that participants were fairly distributed by gender (female: 55.1\%, male: 42.4\%) and age (younger (18-54): 76.7\%, older ($\ge$ 55): 23.4\%).  The majority of the population had a bachelor's degree or higher (55.1\%), and they were the majority across four racial and ethnic groups (White: 77.0\%, Asian: 7.1\%, Black: 6.8\%, Hispanic: 6.6\%, Others: 2.4\%).

% \subsection{Context-wise valuation of explanation strategies}\label{sec:exp-values-in-contexts}

% \begin{figure}[H]%
% \centering
% \includegraphics[width=.7\columnwidth]{figures/overallPref_by_context}
% \vspace{-2em}
% \caption{In each scenario, the relative ratings (along the $x$-axis) for each of the six explanation variants for the {\it extraneous} and {\it germane} cognitive capacities, as well as the preference rankings in five distinct value dimensions. To facilitate the summary, participants' {\it overall} preference was highlighted in gray.}\label{fig:overallPref-context}
% % \vspace{-0.75em}
% \end{figure}


\section{Distinct characteristics of six decision scenarios and participants' different level of cognitive engagement}\label{sec:contextual-prop-and-cog-engagement}
\begin{figure}[h]%
\centering
\includegraphics[width=0.5\columnwidth]{figures/situ_properties_intrinsic_by_context}
\vspace{-2em}
\caption{Decision scenarios have distinct characteristics ({\it high-stakes}, {\it professional}, and {\it timely}) that influence participants' comprehending processes, as measured by three MOA variables (individuals' motivation, opportunity, and ability) and the {\it intrinsic} cognitive load (the level of information difficulty covered in the explanations).}\label{fig:situ}
\vspace{-1.5em}
\end{figure}

% \subsection{Properties of explanation strategies in the analysis of open responses}\label{sec:exp-properties-in-open-responses}
% \begin{figure}[H]%
% % \vspace{-1em}
% \centering
% \includegraphics[width=1\columnwidth]{figures/exp-properties.pdf}
% % \vspace{-2.5em}
% \caption{The most frequent explanatory properties in association with the explanation strategies.}
% \label{fig:exp-properties-in-open-responses}
% \end{figure}

% \section{Decision-making styles by demographic information}
% \input{figs/figA2}

\section{Survey material}\label{sec:survey}

\begin{figure}[h]%
\centering
\includegraphics[width=\columnwidth]{figures/survey1.pdf}
\caption{Introduction and basic demographics page in the survey.}\label{fig:survey1}
\end{figure}

\begin{figure}[h]%
\includegraphics[width=\columnwidth]{figures/survey2.pdf}
\caption{Scenario, explanations, and cognition page in the survey.}\label{fig:survey2}
\end{figure}

\begin{figure}[h]%
\centering
\includegraphics[width=\columnwidth]{figures/survey3.pdf}
\caption{Questions on explanatory value in the survey.}\label{fig:survey3}
\end{figure}

\clearpage
\onecolumn
\section{Definitions of Individual/Cognitive Variables}
\addtolength{\tabcolsep}{2pt}
\begin{table*}[h]
\caption{Variables in cognitive load, decision styles, and Motivation-Opportunities-Ability (MOA) model}
\label{tab:cog_vars}
\centering
\begin{NiceTabular}[t]{@{}p{2cm}ll@{}}
\CodeBefore 
   % \rowcolors{2}{gray!20}{white}
\Body
\toprule
\textbf{Variable}                                          & \textbf{Definition}                                                                                                                                                              & \textbf{Scale}                                                                                                                                                                                    \\ \midrule
\multicolumn{3}{l}{\textbf{Cognitive Load Variables} \cite{CognitiveLoadTheoryFormatInstruction,Workingmemorylimitedimprovingknowledge, CognitiveArchitectureInstructionalDesign}}                                                                                                                                                                                                                                                                                                                                                                                             \\ \midrule
\begin{tabular}[t]{@{}l@{}}Intrinsic load\end{tabular}  & \begin{tabular}[t]{@{}l@{}}The inherent level of difficulty associated with \\a specific topic or context\end{tabular}                                                    & \begin{tabular}[t]{@{}l@{}}How difficult is the information covered\\ in the explanations overall for you to understand?\end{tabular}                                                          \\[1.5em]
\begin{tabular}[t]
{@{}l@{}}Extraneous load\end{tabular} & \begin{tabular}[t]{@{}l@{}}Ineffective load by the manner in which information \\ is presented to individuals and is under the control of \\ material\end{tabular} & \begin{tabular}[t]{@{}l@{}}How difficult is each explanation for you \\ to distinguish important and unimportant \\ information for your decision-making from these \\ explanations?\end{tabular} \\[4em]
\begin{tabular}[t]{@{}l@{}}Germane load\end{tabular}    & \begin{tabular}[t]{@{}l@{}}Effective cognitive load devoted to integrating new \\ information, the creation and modification of schema\end{tabular}                           & \begin{tabular}[t]{@{}l@{}}How did each explanation enhance your \\ understanding of why you were given the decision?\end{tabular}                                                             \\ \midrule
\multicolumn{3}{l}{\textbf{Decision Style Variables } \cite{DecisionMakingStyleDevelopmentAssessmentNew}}                                                                                                                                                                                                                                                                                                                                                                                             \\ \midrule
Rational                                                   & \begin{tabular}[t]{@{}l@{}}A thorough search for and logical evaluation of alternatives\end{tabular}                                                                          & \begin{tabular}[t]{@{}l@{}}I make decisions in a logical and systematic way\end{tabular}                                                                                                       \\[0.5em]
Intuitive & A reliance on hunches and feelings & \begin{tabular}[t]{@{}l@{}}When I make decisions, I tend to rely on my intuition\end{tabular}     \\[0.5em]
Avoidant                                                   & An attempt to avoid decision making                                                                                                                                              & \begin{tabular}[t]{@{}l@{}}I avoid making important decisions until the pressure\\ is on\end{tabular}                                                                                            \\[1.5em]
Dependent                                                  & \begin{tabular}[t]{@{}l@{}}A search for advice and direction from others\end{tabular}                                                                                         & \begin{tabular}[t]{@{}l@{}}I rarely make important decisions without consulting \\ other people\end{tabular}                                                                                      \\[1.5em]
Spontaneous                                                & \begin{tabular}[t]{@{}l@{}}A sense of immediacy and a desire to get through the \\ decision-making process as soon as possible\end{tabular}                                   & I generally make snap decisions                                                                                                                                                                   \\ \midrule
\multicolumn{3}{l}{\textbf{Motivation-Opportunities-Ability (MOA) model} \cite{EnhancingMeasuringConsumersMotivationOpportunity}}                                                                                                                   \\ \midrule
Motivation                                                 & \begin{tabular}[t]{@{}l@{}}Conscious and unconscious cognitive processes that \\ directand inspire behavior\end{tabular}                                                      & \begin{tabular}[t]{@{}l@{}}I feel it is necessary for me to deliberately think and \\ investigate the rationale behind the decision.\end{tabular}                                              \\[1.5em]
Opportunity                                                & \begin{tabular}[t]{@{}l@{}}External factors that make a behavior possible\end{tabular}                                                                                        & \begin{tabular}[t]{@{}l@{}}I feel I have enough time to deliberately think and \\ investigate the rationale behind the decision.\end{tabular}                                                  \\[1.5em]
Ability                                                    & \begin{tabular}[t]{@{}l@{}}Individual’s psychological and physical ability to participate\\ in an activity\end{tabular}                                                       & \begin{tabular}[t]{@{}l@{}}I feel I have the required domain knowledge to \\ understand the information regarding my status \\ used in this decision.\end{tabular}                                \\ \bottomrule
\end{NiceTabular}
\begin{tablenotes}
  \small
  \item *All variables were measured using 5-point Likert scale.
\end{tablenotes}
\vspace{-1em}
\end{table*}

% \section{Definitions of Explanation Strategies and Individual/Cognitive Variables}
% \begin{table*}[h]
\caption{Types of explanation strategies.}
\label{tab:strategies}
\setlength{\tabcolsep}{3pt}
\centering
\begin{tabular}{@{}lllll@{}}
\toprule
\multirow{2}{*}{Explanations} & \multicolumn{2}{l}{Contrastive strategies}                                                                                                                         & \multicolumn{2}{l}{Information selectivity}                                                                                                            \\ \cmidrule(l){2-5} 
                              & \begin{tabular}[c]{@{}l@{}}How to compare\end{tabular} & What/Who to compare                                                                                     & \begin{tabular}[c]{@{}l@{}}Information complexity\end{tabular} & \begin{tabular}[c]{@{}l@{}}Information alignment with \\  prior beliefs\end{tabular} \\ \midrule
\comp          & No strategy                                              & -                                                                                                       & Thorough (All the causes)                                        & Low bias, High variance                                                             \\
\cf            & Contrastive                                              & vs. me (hypothetical status of myself)                                                                  & Simple (Only minimal causes)                                     & High bias, Low variance                                                             \\
\cto           & Contrastive                                              & vs. others (with the opposite outcome)                                                                  & Simple (Only minimal causes)                                     & High bias, Low variance                                                             \\
\begin{tabular}[c]{@{}l@{}}\ctt \\ \newline  \end{tabular}        & \begin{tabular}[c]{@{}l@{}}Contrastive   \\ \newline   \end{tabular}                                        & \begin{tabular}[c]{@{}l@{}}vs. me (previous status of myself \\ with the opposite outcome)\end{tabular} & \begin{tabular}[c]{@{}l@{}}Simple (Only minimal causes)  \\ \newline \end{tabular}                                   & \begin{tabular}[c]{@{}l@{}} High bias, Low variance  \\ \newline  \end{tabular}                                                         \\
\cbhe          & Analogous                                                & vs. others (with the same outcome)                                                                      & Simple (Only minimal causes)                                     & High bias, Low variance                                                             \\
\cbho          & Analogous                                                & vs. others (with the same outcome)                                                                      & Thorough (All the causes)                                        & Low bias, High variance                                                             \\ \bottomrule
\end{tabular}
\begin{tablenotes}
  \small
  \item *Note that explanation strategies that entail contrasting the information (\cf, \cto, and \ctt) inherently deal with simple information by their definition.
\end{tablenotes}
\end{table*}
% \addtolength{\tabcolsep}{2pt}
\begin{table*}
\caption{Variables in cognitive load, decision styles, and Motivation-Opportunities-Ability (MOA) model}
\label{tab:cog_vars}
\begin{NiceTabular}[t]{@{}p{2cm}ll@{}}
\CodeBefore 
   % \rowcolors{2}{gray!20}{white}
\Body
\toprule
\textbf{Variable}                                          & \textbf{Definition}                                                                                                                                                              & \textbf{Scale}                                                                                                                                                                                    \\ \midrule
\multicolumn{3}{l}{\textbf{Cognitive Load Variables} \cite{CognitiveLoadTheoryFormatInstruction,Workingmemorylimitedimprovingknowledge, CognitiveArchitectureInstructionalDesign}}                                                                                                                                                                                                                                                                                                                                                                                             \\ \midrule
\begin{tabular}[t]{@{}l@{}}Intrinsic \\ load\end{tabular}  & \begin{tabular}[t]{@{}l@{}}The inherent level of difficulty \\ associated with a specific \\ topic or context\end{tabular}                                                    & \begin{tabular}[t]{@{}l@{}}How difficult is the information covered\\ in the explanations overall for you \\ to understand?\end{tabular}                                                          \\[3em]
\begin{tabular}[t]
{@{}l@{}}Extraneous \\ load\end{tabular} & \begin{tabular}[t]{@{}l@{}}Ineffective load by the manner in \\ which information is presented \\ to individuals and is under the control \\ of material\end{tabular} & \begin{tabular}[t]{@{}l@{}}How difficult is each explanation for you \\ to distinguish important and unimportant \\ information for your decision-making from \\ these explanations?\end{tabular} \\[4em]
\begin{tabular}[t]{@{}l@{}}Germane \\ load\end{tabular}    & \begin{tabular}[t]{@{}l@{}}Effective cognitive load devoted to \\ integrating new information, the \\ creation and modification of schema\end{tabular}                           & \begin{tabular}[t]{@{}l@{}}How did each explanation enhance your \\ understanding of why you were given the decision?\end{tabular}                                                             \\ \midrule
\multicolumn{3}{l}{\textbf{Decision Style Variables } \cite{DecisionMakingStyleDevelopmentAssessmentNew}}                                                                                                                                                                                                                                                                                                                                                                                             \\ \midrule
Rational                                                   & \begin{tabular}[t]{@{}l@{}}A thorough search for and \\ logical evaluation of alternatives\end{tabular}                                                                          & \begin{tabular}[t]{@{}l@{}}I make decisions in a logical and systematic way\end{tabular}                                                                                                       \\[1.5em]
Intuitive                                                  & A reliance on hunches and feelings                                                                                                                                               & \begin{tabular}[t]{@{}l@{}}When I make decisions, I tend to rely on my intuition\end{tabular}                                                                                                  \\[0.75em]
Avoidant                                                   & An attempt to avoid decision making                                                                                                                                              & \begin{tabular}[t]{@{}l@{}}I avoid making important decisions until the pressure\\ is on\end{tabular}                                                                                            \\[1.5em]
Dependent                                                  & \begin{tabular}[t]{@{}l@{}}A search for advice and \\ direction from others\end{tabular}                                                                                         & \begin{tabular}[t]{@{}l@{}}I rarely make important decisions without consulting \\ other people\end{tabular}                                                                                      \\[1.5em]
Spontaneous                                                & \begin{tabular}[t]{@{}l@{}}A sense of immediacy and a desire \\ to get through the decision-making \\ process as soon as possible\end{tabular}                                   & I generally make snap decisions                                                                                                                                                                   \\ \midrule
\multicolumn{3}{l}{\textbf{Motivation-Opportunities-Ability (MOA) model} \cite{EnhancingMeasuringConsumersMotivationOpportunity}}                                                                                                                   \\ \midrule
Motivation                                                 & \begin{tabular}[t]{@{}l@{}}Conscious and unconscious \\ cognitive processes that \\ directand inspire behavior\end{tabular}                                                      & \begin{tabular}[t]{@{}l@{}}I feel it is necessary for me to deliberately think and \\ investigate the rationale behind the decision.\end{tabular}                                              \\[2.75em]
Opportunity                                                & \begin{tabular}[t]{@{}l@{}}External factors that \\ make a behavior possible\end{tabular}                                                                                        & \begin{tabular}[t]{@{}l@{}}I feel I have enough time to deliberately think and \\ investigate the rationale behind the decision.\end{tabular}                                                  \\[1.5em]
Ability                                                    & \begin{tabular}[t]{@{}l@{}}Individual’s psychological and \\ physical ability to participate\\ in an activity\end{tabular}                                                       & \begin{tabular}[t]{@{}l@{}}I feel I have the required domain knowledge to \\ understand the information regarding my status \\ used in this decision.\end{tabular}                                \\ \bottomrule
\end{NiceTabular}
\begin{tablenotes}
  \small
  \item *All variables were measured using 5-point Likert scale.
\end{tablenotes}
\vspace{-1em}
\end{table*}

% \clearpage
% \subsection{Results of Bayesian regression models}\label{sec:model-results}
% \input{tabs/bayes-predict-extraneous}
% \input{tabs/bayes-predict-germane}
% \input{tabs/bayes-predict-overall}
% \input{tabs/bayes-predict-sufficient}
% \input{tabs/bayes-predict-understandable}
% \input{tabs/bayes-predict-trust}
% \input{tabs/bayes-predict-useful}
