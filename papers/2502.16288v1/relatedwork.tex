\section{Related Works}
\label{sec2}

Due to the ability of HINs to characterize complex information, much research has been dedicated to developing specialized graph mining techniques. Node similarity discovery plays a central role in these techniques.

The traditional algorithms for computing node similarity are path-based methods. PathSim \cite{sun2011pathsim} was the first method to calculate similarity between nodes in HINs, which introduces the innovative concept of meta-paths to assess relatedness between objects of the same type through symmetric paths. HeteSim \cite{shi2014hetesim} was a more general solution capable of quantifying the similarity between heterogeneous nodes regardless of whether the node types are the same or different. It adopted a path-constrained design to capture the semantic information present in the graph. Despite the awareness of different meta-paths in heterogeneous graphs, both PathSim and HeteSim treated each meta-path equally. These approaches overlooked the varied impact across different paths, leading to a loss of valuable semantic information. Howsim \cite{wang2020howsim} introduced a decay graph to encode the aggregation of similarities across various relations, which can capture semantics automatically from HINs. Despite Howsim recognizing the varying importance of different meta-paths, it solely utilized edge type information, neglecting other factors like node centrality. This limitation renders it incapable of handling complex situations, such as distinguishing between nodes of the same type connected by edges of the same type. Additionally, link-based methods do not leverage the content information from the HINs, resulting in less accurate results for similarity search.

With the advent of machine learning, the dominant approach to processing graphs has shifted towards using graph embedding techniques, which map graph elements to vectorized representations. DeepWalk \cite{perozzi2014deepwalk} introduced the method of word embedding to graph embedding by treating nodes as words and mapping the adjacent relationships of nodes into sentences, opening the door to graph representation learning. Node2vec \cite{grover2016node2vec} combined depth-first search strategy and breadth-first search strategy to sample nodes and generate biased second-order random walk sequences. Considering the heterogeneity of the graph, MetaPath2Vec \cite{dong2017metapath2vec} introduces a meta-path-based random walk, applying skip-gram to heterogeneous graphs. These methods either utilized only structural information or combined structural and semantic information without incorporating node content information. In terms of similarity accuracy, they all fell short of expectations.

Graph Neural Networks (GNNs) are widely recognized for their effective graph modeling and have consequently found application in HINs. They leverage deep neural networks to not only aggregate link information but also aggregate content information from neighboring nodes, thereby enhancing the power of the aggregated embeddings. Inspired by Graph Attention Network (GAT) \cite{velivckovic2017graph}, which initially introduced attention mechanisms for aggregating node-level information in homogeneous networks, HAN \cite{wang2019heterogeneous} proposed a two-level attention Heterogeneous Graph Neural Network (HGNN), incorporating both node-level and semantic-level attention. To leverage additional information from HINs, MAGNN \cite{fu2020magnn} employed three major components to encapsulate information from node content, intermediate semantic nodes, and multiple meta-paths. HetGNN \cite{zhang2019heterogeneous} jointly considers both structural and content information of each node without predefining meta-paths. However, when dealing with queries with ad-hoc meta-paths, these methods fall short. They either rely on meta-path-free approaches for automatic weighting across all data, making it challenging to respond to user preferences, or they calculate similarity based on predefined paths. The former struggles to adapt to user preferences, and the latter only produces good results when pre-defined meta-paths match the user-given ones. Aligning meta-paths is crucial for optimal results, but pre-defining all possible meta-paths is nearly impossible for the considerable scale and high complexity of HINs. Although theoretically feasible, retraining node embeddings is highly time-consuming and practically challenging to apply.

% Several endeavors have arisen during the last decade to evaluate the similarity of nodes in HINs. 

% It is conceivable disregarding the different patterns across links leads to a loss of valuable semantic information. The other is to specify meta-paths in advance, which raise several issues. Firstly, each specific meta-path can only capture a particular semantics and these methods suffer from requiring experts to define meta-paths in advance. However, real-world HINs often exhibit considerable scale and high complexity, rendering the enumeration of all possible meta-paths infeasible and unnecessary. Additionally, the question of how to aggregate different meta-paths also poses a challenge. Many simplistic methods \cite{sun2011pathsim, shi2014hetesim} treat each meta-path equally, disregarding the diverse patterns across links, and resulting in a loss of valuable semantic information. Consequently, the traditional methods become incompetent for large and intricate HINs.

% . HAN relies on meta-paths, while HetGNN does not

% However, training graph embeddings can be computationally expensive. More critically, real-world datasets are often dynamically updated, necessitating retraining of embeddings whenever the dataset changes. This drawback renders them unsuitable for scenarios requiring rapid responses. 

% Furthermore, these approaches tend to overlook node content attributes and fail to distinguish between different node types. While some prior works \cite{b22, b23} have addressed node heterogeneity, they either focus solely on intrinsic node attributes or consider only global properties. In contrast, our investigation reveals the significance of inherent attributes, global properties, and local properties, each of which plays a crucial role in achieving comprehensive graph analysis. Even for the same node in different contexts (with identical global attributes), their local properties can still result in varying impacts. For example, in the case of different papers authored by the same individual, their roles as the ``first author" or ``last author" may lead to different impacts on the papers.

As a result, there is a lack of an existing similarity method to handle queries with ad-hoc meta-paths for HINs to integrate both content and path information holistically and efficiently.