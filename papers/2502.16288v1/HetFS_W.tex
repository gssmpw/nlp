%Version 2.1 April 2023
% See section 11 of the User Manual for version history
%
%%%%%%%%%%%%%%%%%%%%%%%%%%%%%%%%%%%%%%%%%%%%%%%%%%%%%%%%%%%%%%%%%%%%%%
%%                                                                 %%
%% Please do not use \input{...} to include other tex files.       %%
%% Submit your LaTeX manuscript as one .tex document.              %%
%%                                                                 %%
%% All additional figures and files should be attached             %%
%% separately and not embedded in the \TeX\ document itself.       %%
%%                                                                 %%
%%%%%%%%%%%%%%%%%%%%%%%%%%%%%%%%%%%%%%%%%%%%%%%%%%%%%%%%%%%%%%%%%%%%%

%%\documentclass[referee,sn-basic]{sn-jnl}% referee option is meant for double line spacing

%%=======================================================%%
%% to print line numbers in the margin use lineno option %%
%%=======================================================%%

%%\documentclass[lineno,sn-basic]{sn-jnl}% Basic Springer Nature Reference Style/Chemistry Reference Style

%%======================================================%%
%% to compile with pdflatex/xelatex use pdflatex option %%
%%======================================================%%

%%\documentclass[pdflatex,sn-basic]{sn-jnl}% Basic Springer Nature Reference Style/Chemistry Reference Style


%%Note: the following reference styles support Namedate and Numbered referencing. By default the style follows the most common style. To switch between the options you can add or remove “Numbered” in the optional parenthesis. 
%%The option is available for: sn-basic.bst, sn-vancouver.bst, sn-chicago.bst, sn-mathphys.bst. %  
 
%%\documentclass[sn-nature]{sn-jnl}% Style for submissions to Nature Portfolio journals
%%\documentclass[sn-basic]{sn-jnl}% Basic Springer Nature Reference Style/Chemistry Reference Style
\documentclass[sn-mathphys,Numbered]{sn-jnl}% Math and Physical Sciences Reference Style
%%\documentclass[sn-aps]{sn-jnl}% American Physical Society (APS) Reference Style
%%\documentclass[sn-vancouver,Numbered]{sn-jnl}% Vancouver Reference Style
%%\documentclass[sn-apa]{sn-jnl}% APA Reference Style 
%%\documentclass[sn-chicago]{sn-jnl}% Chicago-based Humanities Reference Style
%%\documentclass[default]{sn-jnl}% Default
%%\documentclass[default,iicol]{sn-jnl}% Default with double column layout

%%%% Standard Packages
%%<additional latex packages if required can be included here>

\usepackage{amssymb}
\usepackage{utfsym}
\usepackage{graphicx}%
\usepackage{multirow}%
\usepackage{amsmath,amssymb,amsfonts}%
\usepackage{amsthm}%
% \usepackage{mathtools}
\usepackage{mathrsfs}%
\usepackage[title]{appendix}%
\usepackage{xcolor}%
\usepackage{textcomp}%
\usepackage{manyfoot}%
\usepackage{booktabs}%
\usepackage{algorithm}%
\usepackage{algorithmic}%
% \usepackage{algorithmicx}%
% \usepackage{algpseudocode}%
\usepackage{listings}%
\usepackage[utf8]{inputenc}
\usepackage{dutchcal}
\usepackage{tabularx} 
% \usepackage[table]{xcolor}
\usepackage{array}
\usepackage{makecell}
\usepackage{enumitem}
\usepackage{units}
%\nicefrac
\setenumerate[1]{itemsep=1pt,partopsep=0pt,parsep=\parskip,topsep=5pt}
\setitemize[1]{itemsep=1pt,partopsep=0pt,parsep=\parskip,topsep=5pt}
\setdescription{itemsep=1pt,partopsep=0pt,parsep=\parskip,topsep=5pt}
% \graphicspath{ {image/} }
\renewcommand{\algorithmicrequire}{ \textbf{Input:}}
\renewcommand{\algorithmicensure}{ \textbf{Output:}}
%%%%

%%%%%=============================================================================%%%%
%%%%  Remarks: This template is provided to aid authors with the preparation
%%%%  of original research articles intended for submission to journals published 
%%%%  by Springer Nature. The guidance has been prepared in partnership with 
%%%%  production teams to conform to Springer Nature technical requirements. 
%%%%  Editorial and presentation requirements differ among journal portfolios and 
%%%%  research disciplines. You may find sections in this template are irrelevant 
%%%%  to your work and are empowered to omit any such section if allowed by the 
%%%%  journal you intend to submit to. The submission guidelines and policies 
%%%%  of the journal take precedence. A detailed User Manual is available in the 
%%%%  template package for technical guidance.
%%%%%=============================================================================%%%%

%\jyear{2021}%

%% as per the requirement new theorem styles can be included as shown below
\theoremstyle{thmstyleone}%
\newtheorem{theorem}{Theorem}%  meant for continuous numbers
%%\newtheorem{theorem}{Theorem}[section]% meant for sectionwise numbers
%% optional argument [theorem] produces theorem numbering sequence instead of independent numbers for Proposition
\newtheorem{proposition}[theorem]{Proposition}% 
%%\newtheorem{proposition}{Proposition}% to get separate numbers for theorem and proposition etc.

\theoremstyle{thmstyletwo}%
\newtheorem{example}{Example}%
\newtheorem{remark}{Remark}
%\usepackage{xcolor}
% \usepackage{soul}
% % 定义2种颜色
% \definecolor{Seashell}{RGB}{255, 245, 238} %背景色浅一点的
% \definecolor{Firebrick4}{RGB}{255, 0, 0}%文字颜色红一点的
% \newcommand{\code}[1]{
%   \begingroup
%   \sethlcolor{Seashell}%背景色
%   \textcolor{Firebrick4}{\hl{#1}}%textcolor里面对应文字颜色
%   \endgroup
% }

\theoremstyle{thmstylethree}%
\newtheorem{definition}{Definition}%

\raggedbottom
%%\unnumbered% uncomment this for unnumbered level heads

\begin{document}

\title[Article Title]{HetFS: A Method for Fast Similarity Search with Ad-hoc Meta-paths on Heterogeneous Information Networks}

%%=============================================================%%
%% Prefix	-> \pfx{Dr}
%% GivenName	-> \fnm{Joergen W.}
%% Particle	-> \spfx{van der} -> surname prefix
%% FamilyName	-> \sur{Ploeg}
%% Suffix	-> \sfx{IV}
%% NatureName	-> \tanm{Poet Laureate} -> Title after name
%% Degrees	-> \dgr{MSc, PhD}
%% \author*[1,2]{\pfx{Dr} \fnm{Joergen W.} \spfx{van der} \sur{Ploeg} \sfx{IV} \tanm{Poet Laureate} 
%%                 \dgr{MSc, PhD}}\email{iauthor@gmail.com}
%%=============================================================%%

\author[1,3]{\fnm{Xuqi} \sur{Mao}}\email{xqmao17@fudan.edu.cn}

\author[2,3]{\fnm{Zhenyi} \sur{Chen}}\email{zhenyichen20@fudan.edu.cn}
% % \equalcont{These authors contributed equally to this work.}

\author[1, 3]{\fnm{Zhenying} \sur{He}}\email{zhenying@fudan.edu.cn}

\author[1, 3]{\fnm{Yinan} \sur{Jing}}\email{jingyn@fudan.edu.cn}

\author[1, 3]{\fnm{Kai} \sur{Zhang}}\email{zhangk@fudan.edu.cn}

\author*[1,2,3]{\fnm{X. Sean} \sur{Wang}}\email{xywangCS@fudan.edu.cn}
% % \equalcont{These authors contributed equally to this work.}

\affil*[1]{\orgdiv{School of Computer Science}, \orgname{Fudan University}, \orgaddress{\state{Shanghai}, \country{China}}}

\affil[2]{\orgdiv{Software School}, \orgname{Fudan University},  \orgaddress{\state{Shanghai}, \country{China}}}

\affil[3]{\orgname{Shanghai Key Laboratory of Data Science},  \orgaddress{\state{Shanghai}, \country{China}}}

%%==================================%% 
%% sample for unstructured abstract %%
%%==================================%%

\abstract{Numerous real-world information networks form \textbf{H}eterogeneous \textbf{I}nformation \textbf{N}etworks (HINs) with diverse objects and relations represented as nodes and edges in heterogeneous graphs. Similarity between nodes quantifies how closely two nodes resemble each other, mainly depending on the similarity of the nodes they are connected to, recursively. Users may be interested in only specific types of connections in the similarity definition, represented as meta-paths, i.e., a sequence of node and edge types. Existing \textbf{H}eterogeneous \textbf{G}raph \textbf{N}eural \textbf{N}etwork (HGNN)-based similarity search methods may accommodate meta-paths, but require retraining for different meta-paths. Conversely, existing path-based similarity search methods may switch flexibly between meta-paths but often suffer from lower accuracy, as they rely solely on path information. This paper proposes HetFS, a \textbf{F}ast \textbf{S}imilarity method for ad-hoc queries with user-given meta-paths on \textbf{Het}erogeneous information networks. HetFS provides similarity results based on path information that satisfies the meta-path restriction, as well as node content. Extensive experiments demonstrate the effectiveness and efficiency of HetFS in addressing ad-hoc queries, outperforming state-of-the-art HGNNs and path-based approaches, and showing strong performance in downstream applications, including link prediction, node classification, and clustering.
}

% Extensive experiments on various datasets demonstrate that HetFS outperforms state-of-the-art baselines with high accuracy and immediate response speed. 
%%================================%%
%% Sample for structured abstract %%
%%================================%%

% \abstract{\textbf{Purpose:} The abstract serves both as a general introduction to the topic and as a brief, non-technical summary of the main results and their implications. The abstract must not include subheadings (unless expressly permitted in the journal's Instructions to Authors), equations or citations. As a guide the abstract should not exceed 200 words. Most journals do not set a hard limit however authors are advised to check the author instructions for the journal they are submitting to.
% 
% \textbf{Methods:} The abstract serves both as a general introduction to the topic and as a brief, non-technical summary of the main results and their implications. The abstract must not include subheadings (unless expressly permitted in the journal's Instructions to Authors), equations or citations. As a guide the abstract should not exceed 200 words. Most journals do not set a hard limit however authors are advised to check the author instructions for the journal they are submitting to.
% 
% \textbf{Results:} The abstract serves both as a general introduction to the topic and as a brief, non-technical summary of the main results and their implications. The abstract must not include subheadings (unless expressly permitted in the journal's Instructions to Authors), equations or citations. As a guide the abstract should not exceed 200 words. Most journals do not set a hard limit however authors are advised to check the author instructions for the journal they are submitting to.
% 
% \textbf{Conclusion:} The abstract serves both as a general introduction to the topic and as a brief, non-technical summary of the main results and their implications. The abstract must not include subheadings (unless expressly permitted in the journal's Instructions to Authors), equations or citations. As a guide the abstract should not exceed 200 words. Most journals do not set a hard limit however authors are advised to check the author instructions for the journal they are submitting to.}

\keywords{heterogeneous information network, similarity search method, user-given meta-path, ad-hoc query}

%%\pacs[JEL Classification]{D8, H51}

%%\pacs[MSC Classification]{35A01, 65L10, 65L12, 65L20, 65L70}

\maketitle

\section{Introduction}\label{sec1}

Similarity search of nodes on information networks serves as the foundation for numerous data analytics techniques  \cite{zhang2020continuously, echihabi2022hercules, mccauley2018set, wang2021deep, willkomm2019efficient} and has wide-ranging applications, including online advertising \cite{dey2020p}, recommendation systems \cite{zhang2020graph}, biomedical analysis \cite{selvitopi2023extreme, pavlopoulos2018bipartite, yi2022graph}, spatial-temporal systems \cite{patroumpas2020similarity, athanasiou2019big}. Most real-world information networks are heterogeneous information networks (HINs) \cite{zheng2022semantic, liao2022contrastive}, characterized by the coexistence of edges connecting nodes of various types (structural information) and properties associated with each node (often in terms of unstructured information). These connections illustrate the intricate semantics inherent in networks. We use a movie information network, illustrated in Fig.~\ref{fig:illustrationExampleOfHIN}(a), as a running example. The network consists of actors, movies, and directors, each interconnected by different types of relations.
% As a result, the evolution of information networks with complex content, semantics, and structure makes measuring object similarity over HINs a necessary task.

\begin{figure}[h]
\centering\includegraphics[width=0.7\textwidth]{illustrationExampleOfHIN.eps}
\caption{Examples of HINs. (a) depicts a movie HIN comprising objects of different types such as actors (A), movies (M), and directors (D), each carrying varied properties including actor ID, actor name, author order, movie ID, movie name, director ID, and director name. These objects are interconnected through different types of relations, such as acting, acted-by, directed-by, and directing. ``MAM" (Movie-Actor-Movie) and ``DMD" (Director-Movie-Director) are two meta-path examples of it. (b) showcases objects and relations within an academic network, with ``PAP" (Paper-Author-Paper) and ``PVP" (Paper-Venue-Paper) as two example meta-paths.} \label{fig:illustrationExampleOfHIN}
\end{figure} 

% The similarity between two nodes is usually defined recursively by the similarity of the nodes themselves and that of the nodes they are respectively connected to. When the types of connection are restricted to specific ones at each step of the recursion, the similarity is said to be under a specific meta-path. Calculation of node similarity in HINs is challenging due to the varied structural and attribute-based relationships these entities share. In essence, meta-paths intuitively represent the connection pattern between the two nodes. For example, meta-paths like ``PAP” and ``PVP” in Fig. 1(a) indicate two papers with shared authorship or publication venues. Users often focus on the similarity of nodes associated with a particular meta-path, necessitating the handling of queries with ad-hoc meta-paths. User queries with ad-hoc meta-paths refer to the queries whose meta-paths are not predefined but tailored to specific querying needs, allowing systems to capture user search intentions and meet user expectations more accurately. To precisely align with user intent not only enhances user satisfaction but also improves the practical utility of the network. For instance, Table 1 displays movies similar to ``Terminator 2: Judgment Day" (abbreviated as ``Terminator 2" hereinafter)  under different meta-paths, 可以看出不同meta-path-guided下与同一步电影最相似的5部电影区别非常大。第一列电影(Jingle all the way等)都与Terminator 2 有共同的主演施瓦辛格,第二列电影(Titanic)都与Terminator 2都由卡梅隆执导。不同的meta-paths会导致搜索出的结果天差地别,highlighting the necessity of discerning user intent. 

The similarity between two nodes is generally defined recursively, considering both the nodes themselves and those to which they are connected. When the types of connection are restricted to specific ones at each step of the recursion, the similarity is said to be under a specific meta-path. Meta-paths effectively represent the patterns of connection between two nodes, encapsulating the structural and attribute-based relationships within the network. For example, meta-paths such as ``MAM" (Movie-Actor-Movie) and ``MDM" (Movie-Director-Movie) in Fig.~\ref{fig:illustrationExampleOfHIN}(a) illustrate movies linked either by shared actors or directors. Given their specificity, users often focus on node similarities that align with particular meta-paths, necessitating the ability to handle queries with ad-hoc meta-paths. These ad-hoc meta-paths, tailored to specific querying needs, are not predefined but are instead designed to capture user search intentions more precisely, thereby meeting user expectations more effectively. Accurately aligning query handling with user intent not only enhances user satisfaction but also increases the practical utility of the network.

For instance, when looking into the movie information network in our running example, we found that the movies most similar to ``Terminator 2: Judgment Day" (abbreviated as ``Terminator 2" hereinafter) exhibit considerable variation under different meta-paths. The first column in Table~\ref{tab:top5movies_terminator2} lists movies like ``Terminator," sharing similar lead actors with ``Terminator 2," whereas the second column includes films like ``True Lies," directed by similar directors with ``Terminator 2". A special case in similarity search is where all meta-paths are considered without restriction to provide a comprehensive view of similarity, named meta-path-free similarity. The last column of Table~\ref{tab:result_hetfs} gives the movies that are most similar to ``Terminator 2" under meta-path-free semantics in the movie information network of our running example. These examples underscore the importance of discerning user intent, as different meta-paths can lead to vastly different search results, highlighting the critical role of precise meta-path selection in achieving relevant outcomes. 

% Table generated by Excel2LaTeX from sheet 'merge'
\begin{table}[htbp]
\centering
\caption{\centering Top 5 similar movies of ``Terminator 2" under two meta-paths and a meta-path-free scenarios using HetFS.} \label{tab:top5movies_terminator2}
\begin{tabular}{clll}
\toprule
Rank  & meta-path: MAM   & meta-path: MDM   & meta-path-free \\
\midrule
1     & Terminator & True Lies & Terminator \\
2     & True Lies & Aliens & Terminator 3 \\
3     & Total Recall & The Abyss & True Lies \\
4     & Jingle All the Way & Titanic & Titanic \\
5     & End of Days & Ghosts of the Abyss & The Abyss \\
\bottomrule
\end{tabular}%
\label{tab:result_hetfs}%
\end{table}%

% While existing methods broaden the scope of similarity search, they often fall short when handling user queries with ad-hoc meta-paths in highly complex heterogeneous graphs. 

The above examples highlight the need for approaches that can effectively adapt to ad-hoc meta-paths, catering specifically to distinct user preferences. Despite advancements in visual preference learning \cite{wen2024unified}, contrastive learning \cite{zheng2023exploiting}, and attention networks \cite{pan2023iui, chen2023disentangling, wang2023intention} that have improved item recommendations and search accuracy by better aligning with user intentions \cite{rasch2023going, xu2023int}, there is a notable gap in research that focuses on leveraging ad-hoc meta-path configurations to enhance recommendation systems further.
% Users often focus on the similarity of nodes associated with a particular meta-path, necessitating the handling of queries with ad-hoc meta-paths. User queries with ad-hoc meta-paths refer to the queries whose meta-paths are not predefined but tailored to specific querying needs, allowing systems to capture user search intentions and meet user expectations more accurately.

% For example, Fig. 1 illustrates the distinct paths through various interactions between nodes. These paths are abstracted as meta-paths, i.e., sequences of node types and edge types defining composite relations between objects, as detailed in Section 3, to represent the relationships between different types of nodes. Consequently, each specific meta-path encapsulates a unique type of relationship between nodes, conveying distinct semantics. For instance, ``PAP" is a meta-path, which indicates that two nodes are considered similar if the node they are connected to, with the connection type ``P-A'' (i.e., paper-authored-by) to the same node (in this case, an author node). In other words, it signifies two papers are written by a single author. As another example, ``PVP'' denotes two papers published in the identical venue. It's essential to note that the similarity conveyed under different meta-paths varies, and it becomes crucial to discern and evaluate the similarity with distinct meta-paths. Yet another example of a meta-path is ``PAPVPAP''. This means that two papers are similar if they are authored by similar authors, which in turn is defined as having authored similar papers, which is determined if the papers are published in an identical venue. It's notable that two nodes may be connected via various paths that satisfy the same meta-path, and the similarity needs to be aggregated. And by its nature, meta-paths should always be symmetric in order to be meaningful.

% In practice, users often explicitly focus on the similarity of nodes associated with a particular meta-path, conducting searches for nodes based on the specific semantics of a meta-path. Hence, user queries with ad-hoc meta-paths need to be addressed. As an example, Table 1 displays the top 5 movies that are most similar to ``Terminator 2: Judgment Day" (abbreviated as ``Terminator 2" hereinafter) under different meta-path semantics: (a) common actors shared by two movies, i.e., with meta-path ``MAM," and (b) common directors shared by two movies with meta-path ``MDM." The distinct results obtained under each meta-path assumption highlight the substantial differences in similarity, emphasizing the necessity of discerning user intent when handling ad-hoc queries. 

% A special case is what is called meta-path-free similarity. In this case, all meta-paths are taken into account. The last column of Table 1 gives the movies that are most similar to ``Terminator 2'' under meta-path-free semantics. 

% Existing methods encounter limitations when dealing with user-specified meta-paths, as they either compute similarity based on pre-defined meta-paths or operate in a meta-path-free manner across all data. There is a need to calculate similarity based on user-specified ad-hoc meta-paths.

% The task of similarity search with ad-hoc meta-paths in HINs encounters several challenges. Most GNN-based methods designed for similarity search on HINs \cite{wang2019heterogeneous, hu2020heterogeneous, zhang2019heterogeneous} are meta-path-free. These methods face limitations in handling queries with ad-hoc meta-paths, leading to search results that do not align with the user's intention. On the other hand, while some GNN-based methods \cite{fu2020magnn} calculate similarity based on pre-defined meta-paths, they can only deliver accurate results if the pre-defined paths match the user-given meta-path. If the pre-defined meta-paths fail to cover the ad-hoc meta-paths during querying, the resulting similarity will be suboptimal. Enumerating all possible meta-paths in the training phase is a laborious and impractical task, rendering these methods less feasible. While theoretically feasible to train node embeddings online for each user query, such a process is prohibitively expensive, especially in terms of time overhead, significantly compromising the user experience. 

% Another class of similarity search methods is the path-based similarity search methods \cite{sun2011pathsim, wang2020howsim}. They primarily calculate similarity based on the topology of the graph, utilizing limited information and yielding lower accuracy compared to GNN-based methods.

% Given the high cost associated with retraining with the GNN-based methods, we turn to path-based similarity search methods to maintain flexibility in handling user queries with ad-hoc meta-paths. To enhance the accuracy of path-based methods, we comprehensively calculate node similarity by integrating complex content information, intrinsic node characteristics, unique edge properties, and the overall graph structure. Each of these elements plays a role in conducting a thorough graph analysis.
% to capture more structural information of the graph,

% We emphasize that complex content offers significant additional information for path-based similarity computation on HINs. For instance, rich textual content, unique ranking information, diverse image data, and so forth. Supplementing these details into similarity calculations is certain to capture more insights from HINs, thereby enhancing the accuracy of similarity assessments. For example, in the case of different papers authored by the same author, their roles as the ``first author" or ``last author" may lead to different impacts on the similarity of papers. Similarly, the same actor playing the lead and supporting roles in the two films brings different similarities to the two films. Besides, many existing path-based works often overlook the significance of node centrality \cite{brin1998pagerank}, treating all nodes equally and assigning them equal weights. Centrality primarily assigns significance to nodes in a network, such importance also significantly influences the determination of node similarity. For instance, a newcomer to a field would convey a different level of similarity compared to a seasoned expert with extensive experience in the domain. Another essential feature to consider is to combine node properties with edge properties. Two paths with nodes of the same properties may differ significantly if the edge properties are different. While some prior works \cite{fu2020magnn, zhang2019heterogeneous} have noticed the heterogeneity of HINs, they either focus solely on intrinsic node properties or consider only properties of edges. 

% However, numerous studies have found that not only the topology structure but also the content can contribute to node similarity. In addition, we augment path information with content information to address the inadequacy of information utilized by traditional path-based methods.

Existing methods face limitations in dealing with user-specified meta-paths, either relying on pre-defined meta-paths or operating in a meta-path-free manner. GNN-based methods \cite{wang2019heterogeneous, hu2020heterogeneous, zhang2019heterogeneous} designed for HIN similarity search are often meta-path-free, leading to search results that may not align with user intentions. On the other hand, pre-defined meta-path methods \cite{fu2020magnn} fail to deliver accurate results if they don't match the user-given meta-paths. Training node embeddings online for each query is impractical due to time constraints.

Path-based similarity \cite{sun2011pathsim, wang2020howsim} search methods primarily use graph topology, resulting in lower accuracy compared to GNN-based methods. To enhance accuracy, we propose to calculate node similarity by integrating complex content information, intrinsic node characteristics, unique edge properties, and overall graph structure. For instance, rich textual content and diverse image data can capture more insights from HINs, enhancing similarity assessments. Different roles of an author (first author vs. last author) or an actor (lead vs. supporting roles) can influence similarity. Many existing path-based works overlook node centrality \cite{brin1998pagerank}, treating all nodes equally and assigning them equal weights. Centrality (a newcomer vs. a seasoned expert), which signifies node importance, significantly influences node similarity. Additionally, combining node properties with edge properties is crucial, as paths with similar nodes can differ if edge properties vary.

Considering the above limitations in the current approaches, this paper introduces HetFS, a Fast Similarity method with ad-hoc meta-paths on Heterogeneous information networks. HetFS integrates content information into path information by extracting and aggregating multi-level information, including content, node, edge, and structure. It utilizes type-specific transformation functions to project heterogeneous content, potentially of unequal dimensions, into a unified latent space. For node information, HetFS distinguishes between different types of nodes and assigns weights to node properties based on their types and degrees. To dig deeper into semantic information, HetFS extracts diverse semantics from edges to measure the varying impacts of heterogeneous edges. It's worth noting that, despite alignment between nodes and semantics, differences in node structure can result in varying influences on other nodes. Therefore, structure information still plays a crucial role in integrating graph topology. Finally, all the aforementioned information is aggregated to form the ultimate similarity calculation. 

% Table 1 summarizes the key advantages of HetFCF, compared to a number of state-of-the-art models.

% \begin{table}[h]
% \setlength{\tabcolsep}{2pt}
% \fontsize{12}{7}\selectfont
% \caption{Model Comparison: (1) HG - heterogeneous graph? (2) CA - content aware? (3) LNA - local node aware? (4) GNA - global node aware? (5) SeA - semantic aware? (6) StA - structure aware? (7) HStA - heterogeneous Structure aware? (8) FUM - flexible user-given meta-path?}\label{ModelComparison} 
% % (8) MPF - meta-path free? 
% \scriptsize
% % \resizebox{0.5\textwidth}{!}{
% \begin{tabular}{cccccccccccm{2cm}}
% \toprule
% Property & PS & w-PPR & DW & MP2V & GAT & HAN & HetGNN & HGT & MAGNN & HetFCF \\
% \midrule
% HIN & \checkmark & \checkmark & \scalebox{0.75}{\usym{2613}} & \checkmark & \scalebox{0.75}{\usym{2613}} & \checkmark & \checkmark & \checkmark & \checkmark & \checkmark \\
% CA & \scalebox{0.75}{\usym{2613}} & \scalebox{0.75}{\usym{2613}} & \scalebox{0.75}{\usym{2613}} & \scalebox{0.75}{\usym{2613}} & \scalebox{0.75}{\usym{2613}} & \scalebox{0.75}{\usym{2613}} & \checkmark & \scalebox{0.75}{\usym{2613}} & \checkmark & \checkmark \\
% LNA & \scalebox{0.75}{\usym{2613}} & \scalebox{0.75}{\usym{2613}} & \scalebox{0.75}{\usym{2613}} & \scalebox{0.75}{\usym{2613}} & \scalebox{0.75}{\usym{2613}} & \scalebox{0.75}{\usym{2613}} & \scalebox{0.75}{\usym{2613}} & \checkmark & \scalebox{0.75}{\usym{2613}} & \checkmark \\
% GNA & \scalebox{0.75}{\usym{2613}} & \scalebox{0.75}{\usym{2613}} & \scalebox{0.75}{\usym{2613}} & \scalebox{0.75}{\usym{2613}} & \checkmark & \checkmark & \checkmark & \scalebox{0.75}{\usym{2613}} & \checkmark & \checkmark \\
% SeA & \scalebox{0.75}{\usym{2613}} & \checkmark & \scalebox{0.75}{\usym{2613}} & \scalebox{0.75}{\usym{2613}} & \scalebox{0.75}{\usym{2613}} & \checkmark & \checkmark & \checkmark & \checkmark & \checkmark \\
% StA & \checkmark & \checkmark & \checkmark & \checkmark & \checkmark & \checkmark & \checkmark & \checkmark & \scalebox{0.75}{\usym{2613}} & \checkmark \\
% HStA & \scalebox{0.75}{\usym{2613}} & \checkmark & \scalebox{0.75}{\usym{2613}} & \scalebox{0.75}{\usym{2613}} & \scalebox{0.75}{\usym{2613}} & \scalebox{0.75}{\usym{2613}} & \scalebox{0.75}{\usym{2613}} & \scalebox{0.75}{\usym{2613}} & \scalebox{0.75}{\usym{2613}} & \checkmark \\
% % MPF & \scalebox{0.75}{\usym{2613}} & \checkmark & \scalebox{0.75}{\usym{2613}} & \scalebox{0.75}{\usym{2613}} & \scalebox{0.75}{\usym{2613}} & \scalebox{0.75}{\usym{2613}} & \checkmark & \checkmark & \scalebox{0.75}{\usym{2613}} & \checkmark \\
% FUM & \checkmark & \checkmark & \scalebox{0.75}{\usym{2613}} & \scalebox{0.75}{\usym{2613}} & \scalebox{0.75}{\usym{2613}} & \scalebox{0.75}{\usym{2613}} & \scalebox{0.75}{\usym{2613}} & \scalebox{0.75}{\usym{2613}} & \scalebox{0.75}{\usym{2613}} & \checkmark \\
% \bottomrule
% \end{tabular}
% % }
% \end{table}

In summary, this work makes the following main contributions:
\begin{itemize}
\item It introduces HetFS to answer flexible queries with ad-hoc meta-paths, integrating the heterogeneous information of content, node, semantics, and structure.
\item HetFS projects heterogeneous content into a unified latent space to integrate content information with path information, thereby enhancing the final similarity accuracy.
\item HetFS incorporates a refined weighting approach, assigning weights to nodes based on both node type and centrality and to edges based on their contribution to adjacent nodes.
% \item HetFS scales the content similarity to enable aggregation with path similarity, yielding the final similarity.
\item We perform comprehensive experiments to showcase the accuracy and responsiveness of HetFS for queries with ad-hoc meta-paths in comparison to HGNNs and path-based approaches. Additionally, we show that HetFS has robust performance in various downstream applications, including link prediction, node classification, and clustering.

% HetFCF projects heterogeneous content into a unified latent space and aggregates them through a multi-layer transformation function to form the node representation.
% \item We optimize the model based on a random walk, which makes it capable of handling tasks on dynamic graphs.
\end{itemize}

% \begin{figure}[h]
% \centering\includegraphics[width=\textwidth]{excerptOfDBLPNew.png}
% \caption{An excerpt of DBLP. DBLP is an example of a HIN that consists of various types of objects, including authors (A), papers (P), and venues (V), as well as multiple types of relations, such as writing, citing, and publishing.} \label{fig:excerptOfDBLP}
% \end{figure}

% \begin{figure}[h]
% \centering\includegraphics[width=\textwidth]{workFlowNew.png}
% \caption{The Workflow of HetFS. Firstly, we clean the source dataset to obtain the contribution graph and index graph. The contribution graph can be automatically generated from the data graph or specified based on the input. Using the contribution graph as a basis, the query graph can be generated taking into account the input query from users. With the index graph and the query graph, our algorithm can calculate the similarity scores and obtain the result list based on these scores.} \label{fig:workflow}
% \end{figure}

The structure of this paper is as follows. In Section 2, we conduct a survey of the related work. Then we provide the necessary definitions in Section 3 and present our HetFS in Section 4. We report the results of our experimental studies in Section 5 and conclude the paper in Section 6.

\section{Related Works}\label{sec2}

Due to the ability of HINs to characterize complex information, much research has been dedicated to developing specialized graph mining techniques. Node similarity discovery plays a central role in these techniques.

The traditional algorithms for computing node similarity are path-based methods. PathSim \cite{sun2011pathsim} was the first method to calculate similarity between nodes in HINs, which introduces the innovative concept of meta-paths to assess relatedness between objects of the same type through symmetric paths. HeteSim \cite{shi2014hetesim} was a more general solution capable of quantifying the similarity between heterogeneous nodes regardless of whether the node types are the same or different. It adopted a path-constrained design to capture the semantic information present in the graph. Despite the awareness of different meta-paths in heterogeneous graphs, both PathSim and HeteSim treated each meta-path equally. These approaches overlooked the varied impact across different paths, leading to a loss of valuable semantic information. Howsim \cite{wang2020howsim} introduced a decay graph to encode the aggregation of similarities across various relations, which can capture semantics automatically from HINs. Despite Howsim recognizing the varying importance of different meta-paths, it solely utilized edge type information, neglecting other factors like node centrality. This limitation renders it incapable of handling complex situations, such as distinguishing between nodes of the same type connected by edges of the same type. Additionally, link-based methods do not leverage the content information from the HINs, resulting in less accurate results for similarity search.

With the advent of machine learning, the dominant approach to processing graphs has shifted towards using graph embedding techniques, which map graph elements to vectorized representations. DeepWalk \cite{perozzi2014deepwalk} introduced the method of word embedding to graph embedding by treating nodes as words and mapping the adjacent relationships of nodes into sentences, opening the door to graph representation learning. Node2vec \cite{grover2016node2vec} combined depth-first search strategy and breadth-first search strategy to sample nodes and generate biased second-order random walk sequences. Considering the heterogeneity of the graph, MetaPath2Vec \cite{dong2017metapath2vec} introduces a meta-path-based random walk, applying skip-gram to heterogeneous graphs. These methods either utilized only structural information or combined structural and semantic information without incorporating node content information. In terms of similarity accuracy, they all fell short of expectations.

Graph Neural Networks (GNNs) are widely recognized for their effective graph modeling and have consequently found application in HINs. They leverage deep neural networks to not only aggregate link information but also aggregate content information from neighboring nodes, thereby enhancing the power of the aggregated embeddings. Inspired by Graph Attention Network (GAT) \cite{velivckovic2017graph}, which initially introduced attention mechanisms for aggregating node-level information in homogeneous networks, HAN \cite{wang2019heterogeneous} proposed a two-level attention Heterogeneous Graph Neural Network (HGNN), incorporating both node-level and semantic-level attention. To leverage additional information from HINs, MAGNN \cite{fu2020magnn} employed three major components to encapsulate information from node content, intermediate semantic nodes, and multiple meta-paths. HetGNN \cite{zhang2019heterogeneous} jointly considers both structural and content information of each node without predefining meta-paths. However, when dealing with queries with ad-hoc meta-paths, these methods fall short. They either rely on meta-path-free approaches for automatic weighting across all data, making it challenging to respond to user preferences, or they calculate similarity based on predefined paths. The former struggles to adapt to user preferences, and the latter only produces good results when pre-defined meta-paths match the user-given ones. Aligning meta-paths is crucial for optimal results, but pre-defining all possible meta-paths is nearly impossible for the considerable scale and high complexity of HINs. Although theoretically feasible, retraining node embeddings is highly time-consuming and practically challenging to apply.

% Several endeavors have arisen during the last decade to evaluate the similarity of nodes in HINs. 

% It is conceivable disregarding the different patterns across links leads to a loss of valuable semantic information. The other is to specify meta-paths in advance, which raise several issues. Firstly, each specific meta-path can only capture a particular semantics and these methods suffer from requiring experts to define meta-paths in advance. However, real-world HINs often exhibit considerable scale and high complexity, rendering the enumeration of all possible meta-paths infeasible and unnecessary. Additionally, the question of how to aggregate different meta-paths also poses a challenge. Many simplistic methods \cite{sun2011pathsim, shi2014hetesim} treat each meta-path equally, disregarding the diverse patterns across links, and resulting in a loss of valuable semantic information. Consequently, the traditional methods become incompetent for large and intricate HINs.

% . HAN relies on meta-paths, while HetGNN does not

% However, training graph embeddings can be computationally expensive. More critically, real-world datasets are often dynamically updated, necessitating retraining of embeddings whenever the dataset changes. This drawback renders them unsuitable for scenarios requiring rapid responses. 

% Furthermore, these approaches tend to overlook node content attributes and fail to distinguish between different node types. While some prior works \cite{b22, b23} have addressed node heterogeneity, they either focus solely on intrinsic node attributes or consider only global properties. In contrast, our investigation reveals the significance of inherent attributes, global properties, and local properties, each of which plays a crucial role in achieving comprehensive graph analysis. Even for the same node in different contexts (with identical global attributes), their local properties can still result in varying impacts. For example, in the case of different papers authored by the same individual, their roles as the ``first author" or ``last author" may lead to different impacts on the papers.

As a result, there is a lack of an existing similarity method to handle queries with ad-hoc meta-paths for HINs to integrate both content and path information holistically and efficiently.

\section{Preliminaries}
This section gives an overview of HIN and SimRank. Table~\ref{tab:notationTable} presents the notations used in this paper.

\begin{table}[h]
\caption{\centering Notation Table.}\label{tab:notationTable}
% \scriptsize
\begin{tabular}{cm{9cm}}
\toprule
% \makebox[2cm][c]{Notation} & {Description} \\
Notation & \makecell[c]{Description}\\
\hline
% $G$, $\mathcal{T}_G$ & the data graph, the network schema of $G$\\
$G$ & the data graph\\
$u$, $v$ & a node in the data graph\\
$e$ & an edge in the data graph\\
$V$, $E$ & the node set and the edge set of the data graph\\
$A$, $R$ & the node type and the edge type of the data graph\\
$\mathcal{A}$, $\mathcal{R}$ & the node type set and the edge type set of the data graph\\
$\phi$, $\psi$ & the type mapping function of the node and edge\\
$n$, $m$ &  the numbers of nodes and edges in $G$\\
% $N(u)$, $N_A(u)$, $N_R(u)$ & the neighbor set of $u$, the type-A neighbor set of $u$, and the neighbor set of $u$ along relation $R$ \\
$N(u)$, $N_R(u)$ & the neighbor set of $u$ and the neighbor set of $u$ along relation $R$ \\
$u^{\prime}$, $u_R^{\prime}$ & one of neighbors of $u$ and one of neighbors of node $u$ along relation $R$\\
$s(u, v)$ & the similarity score of $u$ and $v$\\
$c$ & the decay factor of SimRank\\
$c_n$ & the node decay factor\\
$t$ & a tour traversed during the random walk\\
% $s_p(u, v)$ & the similarity score of $u$ and $v$ under instance path $p$\\
$\mathcal{P}, \lvert\overline{\mathcal{P}}\rvert$ & the meta-path set and the average number of meta-path\\
$P, \lvert\overline{P}\rvert$ & the path set of a meta-path and the average number of paths in each meta-path\\
\bottomrule
\end{tabular}
\end{table}

\subsection{Heterogeneous Information Network}

\begin{definition}[Information Network]
An information network is a directed graph $G(V, E)$, where each node $v \in V$ is mapped to a specific node type $\phi(v) \in \mathcal{A}$ by $\phi : V \rightarrow \mathcal{A}$ and each edge $e \in E$ is mapped to a specific relation type $\psi(e) \in \mathcal{R}$ by $\psi : E \rightarrow \mathcal{R}$.
\end{definition}

An information network is said to be a Heterogeneous Information Network (HIN) when $|\mathcal{A}|$ \textgreater 1 or $|\mathcal{R}|$ \textgreater 1. If $|\mathcal{A}| = |\mathcal{R}| = 1$, then the information network is considered a homogeneous one.

% \noindent\textbf{Definition 2} Network Schema. The network schema of a HIN is a directed graph serving as a meta-template for HIN, defined on the node type set $\mathcal{A}$ and edge type set $\mathcal{R}$, represented as $\mathcal{T}_G(\mathcal{A}, \mathcal{R})$.

% % The network schema is akin to the ER (Entity-Relationship) model in database systems.

% The network schema provides valuable insights into the types of objects existing in the network graph as well as the potential relations between them. 



\begin{definition}[Meta-Path]
A meta-path $P$ is a sequence of node types in $\mathcal{A}$ and relation types in $\mathcal{R}$. It is typically represented with the notation of $A_1 \stackrel {R_1} {\longrightarrow} A_2 \stackrel {R_2} { \longrightarrow} \cdots \stackrel {R_l} {\longrightarrow} A_{l+1}$, where $A_1$ denotes the starting node type, $A_{l+1}$ denotes the ending node type, and each intermediate relation type $R_i$ denotes a connection type from node type $A_i$ and to node type $A_{i+1}$. To simplify notation, we can use node types along the meta-path to specify a meta-path if no more than one edge type exists connecting the identical pair of node types. 
\end{definition}

As an illustration, the co-author relation in DBLP, denoted by a length-2 meta path $A\stackrel{writing}{\longrightarrow}P\stackrel{written-by}{\longrightarrow}A$, can be represented in a shorter form $APA$ with no ambiguity.

% A reverse meta-path $P^{-1}$ serves as the reverse of $P$, whose relation is absolutely inverse to that of $P$. $P_1 = (A_1A_2...A_k)$ and $P_2 = ({A_1^{'}}{A_2^{'}}...{A_k^{'}})$, are considered connectable if and only if $A_k = A_1^{'}$. The connected meta-path $(A_1A_2...A_k{A_2^{'}}...{A_k^{'}})$ can be denoted as $P = (P_1P_2)$.

% Similarity score of a node pair $x \in A_1$ and $y \in A_l$ can be computed if there exists a meta-path $P = (A_1A_2...A_l)$ between them. The computation involves aggregating the similarity scores of each instance of the meta-path $P$. Besides, we denote path count as $PC$, which corresponds to the number of path instances of meta-path $P$ between $x$ and $y$, given by $PC = |p:p \in P|$.

\subsection{SimRank}

SimRank is a similarity measure defined on homogeneous information networks with a long history and broad applications across different domains. It is based on two fundamental principles: \textit{a}) two nodes are considered similar if connected to similar nodes; \textit{b}) a node has a similarity score of 1 with itself. Given an unweighted directed graph, ${G\,(\,V, E\,)}$, the SimRank score of two nodes $u \in V$ and $v \in V$ is defined recursively:
% \begin{small}
\begin{equation} 
	\label{(simrankEquation)}
	s(u, v)=
	\left\{
	\begin{array}{llr}
		1, & \text{if } u = v; & \\
		\frac{c}{|N(u)||N(v)|}\sum\limits_{u^{\prime} \in N(u)}\sum\limits_{v^{\prime} \in N(v)}s(u^{\prime}, v^{\prime}), & \text{otherwise.}
	\end{array}
	\right.
\end{equation}
% \end{small}
where $N(u)$ and $N(v)$ denote the neighbor set of node $u$ and $v$, respectively, with $c$ $\in$ (0,1) commonly set to 0.6 \cite{wang2021exactsim} or 0.8 \cite{jeh2002simrank}, is a decay factor to ensure that the similarity between different nodes does not equal 1. Due to the recursiveness of Eq.~\ref{(simrankEquation)}, SimRank can integrate similarities not only from their direct neighbors of $u$ and $v$ but also from their indirect neighbors (multi-hop neighbors), resulting in accurate, reliable, and meaningful outcomes.

To gain a more intuitive understanding of the calculated values, SimRank introduced a random surfer-pairs model \cite{jeh2002simrank}. This model interprets $s(u,v)$ as the expected first-meeting distance for two random surfers, each originating from nodes $u$ and $v$ and destinating at the same node $w_l$. A tour $t$ may be written as $t= \langle w_1, \cdots w_l(w_{\text{-}l}),\cdots, w_{\text{-}1}\rangle $, and $l$ is said to be the length of the tour. Intuitively, the first surfer walks from $w_1$ to $w_2, \cdots, w_l$ and the second surfer walks from $w_{\text{-}1}$ to $w_{\text{-}2}, \cdots, w_{\text{-}l}$, and we assume $w_l = w_{\text{-}l}$. When $l = 0$, we have $w_1 = w_{\text{-}1}$. For a tour $t$ traversed during the random surf from node $u$ and $v$, here $u = w_1$ and $v = w_{\text{-}i}$, $s(u,v)$ can be reformulated based on this random surfer-pairs model as:
\begin{equation} 
    \label{(random_walk)}
    s(u, v)=\sum\limits_{t}Pr(t)c^{l}
\end{equation}
where
\begin{equation} 
    \label{(Pr)}
    Pr(t)=
    \left\{
    \begin{array}{llr}
    1, & \text{if } l = 0; & \\
    \prod\limits_{w_i \in  t}\frac{1}{|N(w_i)|}, & \text{otherwise.}
    \end{array}
    \right.
\end{equation}
% \begin{equation} 
%     \label{(random_walk)}
%     s(u, v)=\sum\limits_{t:(a,b)\leadsto(w_l,w_l)}Pr(t)c^{l}
% \end{equation}
% \begin{equation} 
%     \label{(first_meeting_probability)}
% \end{equation}
and $t= \langle w_1, \cdots w_l(w_{\text{-}l}),\cdots, w_{\text{-}1} \rangle$ ranges over all the tours that the random surfers start from node $u (w_1)$ and $v (w_{\text{-}1})$ and reach a node $w_l (w_{\text{-}l})$, and $Pr(t)$ denotes the first meeting probability of tour $t$, where $N(w_i)$ denotes the set of neighbors of node $w_i$.


\section{Methodology}\label{sec3}

% In various domains, similar nodes are more likely to connect than dissimilar nodes. For example, customers loyal to similar brands tend to purchase similar products, and researchers publishing papers at similar conferences are likely to have connections. Consequently, the similarity between two nodes can be deduced from their shared references. Inspired by this intuition, 
% Inspired by the definition of SimRank and its random surfer-pairs model, we extend SimRank \cite{jeh2002simrank} to HINs by aggregating multi-level information along the random tours into the original definition. The abundant content within HINs offers intricate information. Neglecting this rich content information within heterogeneous graphs will inevitably result in computed similarity values that fail to capture the similarity between nodes fully. This is the reason why we combine content information and path information to propose our HetFS method.

Due to the effectiveness of SimRank \cite{jeh2002simrank} in capturing path information, we propose HetFS to extend SimRank to HINs by aggregating multi-level information and combining content information with path information to capture node similarity fully.

This section introduces HetFS, a fast similarity search method with ad-hoc meta-paths for HINs. HetFS integrates multi-level information from content, nodes, edges, and structure to enhance search capabilities, as depicted in Fig.~\ref{fig:architecture}, which illustrates its architecture. Note that while HetFS is capable of processing both symmetric and asymmetric meta-paths, it primarily focuses on symmetric meta-paths, aligning with the principles established by PathSim~\cite{sun2011pathsim} to ensure balanced and comparable visibility between nodes. Further details will be discussed in the following sections.

% HetFS computes similarity with a deep understanding of heterogeneous nodes. 
% The architecture of HetFS is illustrated in Fig. 2, depicting the process of aggregating content similarity to path similarity to generate the final similarity.

\begin{figure}[h]
\centering\includegraphics[width=\textwidth]{architecture.eps}
\caption{The overall architecture of HetFS. HetFS first projects heterogeneous content information, e.g., textual data, from each node into a unified domain via a transformation function. Subsequently, the content information is integrated with node centrality, edge contribution, and structural topology to form the ultimate similarity computation.} \label{fig:architecture}
\end{figure}


% \subsection{Content Similarity}

\subsection{Content Information}

Nodes can contain diverse types of information as their content, such as textual data, image data, and more. Most existing methods overlook the content information carried by the nodes, which makes the information in the HIN not fully utilized, and the final result may not be satisfactory. HetFS extracts content information, integrating it into the path-based similarity, thereby enhancing the overall accuracy of the final similarity score.

Considering the unique content characteristic associated with each dataset, we project heterogeneous content $C_u$ of each node $u \in V$ into a given domain through a transformation function $f$. The $i$-th content in $C_u$ is denoted as $x_i(u)$. Note that $x_i(u)$ can undergo various processing depending on the type of content. For instance, we employ tf-idf to process textual content and CNN for image content. Specifically, for textual content, we begin by tokenizing each document, and then perform lemmatization and remove common stop words. Subsequently, we utilize Word2Vec \cite{mikolov2013efficient} to capture corpora characteristics, creating corpora-specific stop-word and synonym lists, which were then used to enhance the tokenization results of the content. Note that we generate individual stop-word lists and synonym lists for each textual dataset, with each dataset comprising a collection of text based on how it is associated with the nodes. For example, texts may be from a title or an abstract, forming two specific datasets. Finally, we calculate content score based on tf-idf. Formally, the content score of node $u$, represented as $\chi(u)$ can be formulated:
\begin{equation} 
    \label{(content score)}
    \chi(u)=\sum\limits_{i = 1, \cdots, |C_u|} f_i(x_i(u)) 
\end{equation}
where $C_u$ is the heterogeneous content of node $u$, and $f$ is the specific transformation function for each content in $C_u$.

% \subsection{Path Similarity}

\subsection{Node Information}
% \noindent\textbf{Node Information}

Node centrality $\alpha$ is a measure that gives the importance of a node. For instance, academic giants have higher academic influence compared to novice entrants. Inspired by PageRank, we say that a node is of high centrality if it is referenced by many nodes that themselves possess high centrality. Intuitively, a node that is well linked with other nodes is usually of high centrality.

To distinguish centrality with respect to nodes of different types, we calculate centrality separately for each node type and then aggregate the results. Based on the discussion above, we define the node centrality recursively as follows:
\begin{equation} 
    \label{nodecentrality}
    \alpha(u)=c_n \cdot \sum\limits_{R \in \mathcal{R}}\sum\limits_{u_R^{\prime}\in N_R(u)}\alpha(u_R^{\prime}) \cdot \mathbf{P}(u, u^{\prime})
\end{equation}
where $c_n$ is the node decay factor to deal with dangling links, $u_R^{\prime}$ is one of the neighbors of $u$ along relation $R$, and $\mathbf{P(u, u^{\prime})}$ denotes the transition factor as follows:
\begin{equation} 
    \label{(transition_factor)}
    \mathbf{P}(u, u^{\prime})=
    \left\{
    \begin{array}{llr}
    \frac{1}{|N_R(u)|}, & \text{if } u^{\prime} \in N_R(u); & \\
    0, & \text{otherwise.} 
    \end{array}
    \right.
\end{equation}
in which $N_R(u)$ denotes the neighbors of $u$ along relation $R$. Eq.~\ref{nodecentrality} is recursive, aligning with the intuition that a node's centrality is distributed evenly among its neighbors, thereby contributing to the centrality of the connected nodes. It can be computed by initializing with some random values and iterating the computation until convergence or reaching a specific iteration.

% $arctan(x)$

\subsection{Edge Information}

% In heterogeneous graphs, we observe variations not only in edge categories but also in the additional information carried by edges of the same category. For instance, a particular edge may carry a rank within its category of edges. The former is described in this paper as the global edge contribution $\mu$, while the latter as the local edge contribution $\lambda$. It is not difficult to understand that edges of different types convey different information. Regarding the latter, for example, the information conveyed to an article by the first author and the second author is also different. Customizing the weight of relations, referred to as the edge contribution, is crucial for HetFS, as weights embody the importance of different relations. 

% The inspiration for the global edge contribution comes from the well-known numerical statistic called TF-IDF, commonly used in information retrieval. 
% Given a network schema $T_G{(\mathcal{A, R})}$,
In HINs, the diverse types of edges convey distinct semantics, thereby contributing differently to the overall similarity score. By analyzing the statistics and structure of the data graph, we can derive the edge contribution for each edge type, which can also be adjusted based on user input and preferences. Given an HIN $G(V, E)$ with relation types $R$, a global edge contribution $\mu_R$ maps each relation type $R \in \mathcal{R}$ to a fraction:
\begin{equation} 
    \label{(contributionFunction)}
    \mu_R = RF(R) \cdot IRF(R)
\end{equation}
where $RF(R)$ and $IRF(R)$ are calculated based on the following heuristics:
\begin{enumerate}
    \item [1)] $RF(R)$ is the normalized relation frequency of relation $R$, which can be calculated as $\frac{m_R}{m}$, i.e., the fraction of the count of edges with relation $R$ in $G$ within the total count of edges in $G$, preventing the contribution coefficient from biasing toward frequently-appear edges.
    \item [2)] $IRF(R)$ is the inverse frequency of nodes associated with relation $R$, $IRF(R) = \ln \frac{n}{|n_R|}$, which is the Napierian logarithmic of the total count of nodes over the count of nodes involved with relation $R$. This count illustrates the differentiation capability among the data graph. In other words, the fewer neighbors of a relation $R$, the better a relation can differentiate a node. 
\end{enumerate}

By combining the edge contributions, we can construct a contribution graph $M(\mathcal{A}, \mathcal{R}, \mu_R)$, which is a directed graph with weighted edges, where $\mathcal{A}$ and $\mathcal{R}$ denotes the node type set and edge type set, respectively. This contribution graph illustrates how a node accumulates similarities from its neighbors across different relations, as well as how its neighbors contribute similarities to the node across different relations. Fig.~\ref{fig:contribution_graph} is an example of the contribution graph of the movie network. We can see that for the ``movie-to-movie'' similarity calculation, considering only edge contribution, 47\% similarity originates from shared actors, 41\% similarity from being part of the same film series, and 30\% from common directors.

\begin{figure}[h]
\centering\includegraphics[width=0.7\textwidth]{contribution_graph.eps}
\caption{The contribution graph of movie network.} \label{fig:contribution_graph}
\end{figure}

% which is a directed graph that shares the same structure as $T_G{(\mathcal{A, R})}$ but with weighted edges

% The process of extracting the contribution graph is outlined in Algorithm 1, and further details can be found therein.

% \begin{algorithm}
% \caption{Extraction of Contribution Graph}
% \label{alg:extractionOfContributionGraph}
% \begin{algorithmic}[1]
% \REQUIRE heterogeneous graph $G$, number of nodes $n$, number of edges $m$
% \ENSURE contribution graph in the form of map consisting of tuples $M(R, \mu_R)$ for each $R \in \mathcal{R}$ 
% \STATE load the heterogeneous graph $G$, initialize the map $M(R, \mu_R)$ 
% \STATE $iRF, m_R \gets 0$ /* $m_R$ is to record all the in-degree of a node */
% \FOR{each $R \in \mathcal{R}$}
%     \FOR{each $R_i \in R$}
%         \STATE $O_R \gets O_R.add(v_{out-R_i})$
%     \ENDFOR
%     \FOR{each $v \in O_R$}
%         \FOR{each in-node of $v$}
%             \STATE $m_R \gets m_R + 1$negatively
%         \ENDFOR
%     \ENDFOR
%     \STATE $RF \gets \frac{m_R}{m}$
%     \STATE $INF \gets \ln{\frac{n}{|O_R|}}$
%     \STATE $AVG(IND_R) \gets \frac{m_R}{|O_R|}$
%     \STATE $\mu_R \gets \frac{RF * IRF}{AVG(IND_R)}$
% \ENDFOR
% \RETURN $M(R, \mu_R)$
% \end{algorithmic}
% \end{algorithm}

% Concerning local edge contribution $\lambda$, we adopt a straightforward approach to distinguish edges of different orders. Based on the additional rank information of each edge, we initially sorted these edges in reverse order, and the element's position served as the sorted order of edges. The local edge contribution $\lambda_R(a)$ of node $a$ along relation $R$ can be formulated as follows:
% \begin{equation} 
%     \label{(localEdgeContribution)}
%     \lambda_R(a) = h_R \left( \Tilde{o} \left( a \right) \right)
% \end{equation}
% where $h_R$ denotes the differentiation function of relation $R$ that distinguishes edges by different orders and $\Tilde{o}(a)$ is the specific order of node $a$ in local context. Fig. 3 (b) illustrates how edge order affects the local edge contribution.

% $l1$

\subsection{Structure Information}
% \noindent\textbf{Structure information}

Similar to SimRank, we consider that information is distributed among the neighbors of a node. 
Here, we calculate the structure weight for each type of neighbor of a node separately. Given the heterogeneity of HINs, we define structure weight $\beta$ for each node $u$ based on the count of neighbors connected via type $R$ as follows:
\begin{equation} 
    \label{(strctureWeight)}
    \beta_R(u) = |N_R(u)|
\end{equation}
% \begin{equation} 
%     \label{(strctureWeight)}
%     \beta(u) = \sum\limits_{A\in \mathcal{A}}\frac{1}{|N_A(u)|}
% \end{equation}
where $|N_R(u)|$ is the count of the neighbors of $u$ along relation $R$. The structure weight ensures that the impact of a node on its neighbors is related to the of its neighbors. 
% inversely

% The preprocessing is now almost complete. Algorithm 1 presents the pseudocode for the preprocessing of HetFS, where Line 3, Line 4, and Line 5 extract the content information, node information, and structure information for each node, respectively. Line 8 calculates the contribution for each edge $R \in E$.

% \begin{algorithm}
% \caption{HetFS Preprocess}
% \label{alg:HetFSPreprocess}
% \begin{algorithmic}[1]
% \REQUIRE heterogeneous graph $G(V, E)$
% \ENSURE $\chi(u)$, $\alpha(u)$, $\beta_R(u)$, $\mu_R$
% \STATE load the heterogeneous graph $G$
% \FOR{each $u \in V$}
% \STATE $\chi(u) \gets \sum\limits_{i \in |C_u|} f(x_i(u))$
% \STATE $\alpha(u) \gets c_n \cdot \sum\limits_{R \in \mathcal{R}}\sum\limits_{u_R^{'}\in N_R(u)}\alpha(u_R^{'}) \cdot \mathbf{P}$
% \STATE $\beta_R(u) \gets |N_R(u)|$
% \ENDFOR
% \FOR{each $R \in E$}
% \STATE $\mu_R \gets RF(R) \cdot IRF(R)$
% \ENDFOR
% \end{algorithmic}
% \end{algorithm}

% % \subsection{An alternative interpretation of HetFS}

% Drawing inspiration from the Monte Carlo method, we define a random walk in HIN initiating from a node, surfing at each step following the following rules:
% \begin{enumerate}
%     \item [1)] it moves to one of their in-nodes uniformly with a probability of $\sqrt{\mu}$;
%     \item [2)] It stops with a probability of $1 - \sqrt{\mu}$.
% \end{enumerate}
% We refer to this type of random walk as a $\sqrt{\mu}$-walk in the context of HetFS. Subsequently, we introduce a lemma that establishes a connection between $\sqrt{\mu}$-walks and our HetFS.

% \noindent\textbf{LEMMA 1} Let $W_i$ and $W_j$ represent two $\sqrt{\mu}$-walks starting from nodes $v_i$ and $v_j$ respectively. their meeting probability equals to $HetFS(u_i, u_j)$.

% To approximate the HetFS score, the solution maintains a set of raw random walks for each node and efficiently computes meeting vertices online.

% The algorithm conducts $R$ random samples to estimate $s(u, v)$. In each sample, two random walkers $W_{ui}$ and $W_{vi}$ are sampled starting from vertices u and v in $G_r$, then the delta HetFS score $\Delta{u, v}$ is computed by finding the first-meeting node between $W_{ui}$ and $W_{vi}$ (line 5), finally, it finishes this sample by increasing s(u, v) with $\Delta{u, v}$. Theorem 1 guarantees the accuracy of the estimated HetFS score with respect to the number of samples.
 
% The basic idea for the estimation based on Monte Carlo is to sample two collections of random walks and identify their meeting nodes. Then we can calculate $s(u, v)$ consistent with Eq.(11): 
% \begin{equation} 
%     \label{equ:HetFS}
%     s(u, v)=
%     \sum\limits_{R\in \mathcal{R}(A)}\prod\limits_{u'\in I_R(u)}\prod\limits_{v'\in I_R(v)}
%     \frac{\sqrt{\mu_R}}{|I_R(u)||I_R(v)|} 
% \end{equation}
% where $\sqrt{\mu_R}$ denotes the moving probability from the present node to one of its in-nodes along the relation $R$. The core idea of the random-walk-based approach involves maintaining a set of one-step random walks for every node, called a one-step graph. This one-step graph approximates the complete random-walk tree for each node. During query time, HetFS estimates similarity via the one-step graph, thereby avoiding redundant computations. Additionally, the one-step graph can be efficiently updated when modifications are made to the original graph. The following sections provide detailed explanations of the two stages involved in this process.

% Meta-paths are collapsible and iterable in our algorithm, which is to say, a meta-path can have collapsed versions. A collapsed meta-path is represented as $P^{-k}$, where $k \leq |\frac{l}{2}| + 1$ with $l$ representing the uncollapsed meta-path length. Specifically, a meta-path $P_1^{-1} = (A_2A_3...A_l)$ is a collapsed version of the meta-path $P_1 = (A_1A_2...A_{l+1})$. Similarly, a meta-path $P_2^{-1} = (A_3A_4...A_{l-1})$ is a collapsed version of the meta-path $P_2 = (A_2A_3...A_l)$. This collapsing process can be iterated, for example, a meta-path $P_1^{-2} = (A_3A_4...A_{l-1})$ is a collapsed version of the meta-path $P_1 = (A_1A_2...A_{l+1})$, and so on.

% Analogously, the semantics path is collapsible and iterable, too. A semantics path $p_1^{-1} = (a_2a_3...a_l)$ is a collapsed version of a meta-path $p_1 = (a_1a_2...a_{l+1})$ and a meta-path $p_1^{-2} = (a_3a_4...a_{l-1})$ is a collapsed version of a meta-path $p_1 = (a_1a_2...a_{l+1})$ iterably.

\subsection{Integration}
% \noindent\textbf{Integration}

Integrating content information, node information, edge information, and structure information gives the overall similarity. For a HIN $G(V, E)$, the similarity $s(u, v)$ of two nodes $u \in V$ and $v \in V$ is defined as:
% \begin{equation} 
%     \label{(pathSimilarity)}
%     s_p(u, v)=
%     \left\{
%     \begin{array}{lr}
%     0,  \hspace{20.8em} \rm{if  u = v} & \\
%     c\sum\limits_{R\in \mathcal{R}}\sum\limits_{u^{'} \in N(u)}\sum\limits_{v^{'} \in N(v)}\lvert\lvert \tanh{(\alpha \lambda)}\mu\rvert\rvert_{1} \beta s(u^{'}, v^{'}),  \hspace{0.8em} \rm{otherwise}
%     \end{array}
%     \right.
% \end{equation}
\begin{equation} 
    \label{power_method}
    s(u, v)=
    \left\{
    \begin{array}{llr}
    1, & \text{if } u = v; & \\
    \sum\limits_{R\in \mathcal{R}_{\mathcal{P}}}\sum\limits_{u^{\prime} \in N_R(u)}\sum\limits_{v^{\prime} \in N_R(v)} \frac{c \cdot \chi(u^{\prime}) \cdot \chi(v^{\prime}) \cdot \alpha(u^{\prime}) \cdot \alpha(v^{\prime}) \cdot \mu_R}{\beta_R(u^{\prime}) \cdot \beta_R(v^{\prime})} \cdot s(u^{\prime}, v^{\prime}), & \text{otherwise.}
    \end{array}
    \right.
\end{equation}
where $\mathcal{R}_{\mathcal{P}}$ denotes the edge types set corresponding to specific meta-paths, $c$ ($0 < c < 1$) is the decay factor inherited from SimRank, which ensures that no other node, except the node itself, has a similarity value of 1 with that node, $\chi(u^{\prime})$ and $\chi(v^{\prime})$ are the content information of $u^{\prime}$ and $v^{\prime}$, respectively, $\alpha(u^{\prime})$ and $\alpha(v^{\prime})$ are the node centrality of $u^{\prime}$ and $v^{\prime}$, respectively, $\mu(R)$ is the edge contribution of edge type $R$, $\beta(u^{\prime})$ and $\beta(v^{\prime})$ is the structure weight of $u^{\prime}$ and $v^{\prime}$, respectively. The similarity score is 1 if $u$ and $v$ are the same node by definition. Otherwise, Eq.~\ref{power_method} combines the similarities from $N(u)$ and $N(v)$ through each $R \in \mathcal{R}_{\mathcal{P}}$. 

% Next, we give the matrix representation for Eq.(6) using the linearization method. Given a HIN $G(V, E)$, the transition matrix $Q$ of transpose graph $G^T$ is defined:
% \begin{equation}
%     \label{(transitionMatrix)}
%     Q_{i, j} = 
%     \left\{
%     \begin{array}{llr}
%     \frac{\chi(u^{'}) \cdot \chi(v^{'}) \cdot \alpha(u^{'}) \cdot \alpha(v^{'}) \cdot \mu_R}{\beta_R(u^{'}) \cdot \beta_R(v^{'})}, & \text{if } \exists (i \rightarrow j) \in E; & \\
%     0, & \rm{otherwise.}
%     \end{array}
%     \right.
% \end{equation}
% where $I_R(j)$ and $I_R(i))$ represent the in-neighbor set of $j \in V$ and that of $i \in V$, respectively. $(i \rightarrow j) \in E$ represents $i$ and $j$ is connected over $G$ For a node type $A \in \mathcal{A}$, we define $S_A$ as the matrix equivalent to Eq.(6):
% \begin{equation}
%     \label{(HetFSMatrixRepresentation)}
%     S_A = max\{\sum_{R \in R(A)} c \cdot (Q \cdot S \cdot Q^T), I_n\} = (\sum_{R \in R(A)} c \cdot Q \cdot S \cdot Q^T) \vee I_n
% \end{equation}
% where $Q^T$ is the matrix transpose of $Q$, $I_n$ is the $n \times n$ identity matrix, $max\{,\}$ and $\vee$ are both element-wise maximum operator. In other words, the entry $s_A(i, j)$ of the matrix $C \vee D$ is given by $max\{C_{ij}, D_{ij}\}$.
\begin{algorithm}
\caption{Brute-force HetFS Computation}
\label{alg:bruteforce_computation}
\begin{algorithmic}[1]
\REQUIRE heterogeneous graph $G(V, E)$, number of iterations $l$, $\chi(u)$, $\alpha(u)$, $\beta_R(u)$, $\mu_R$, node $u$, $v$
\ENSURE similarity $s$
\FOR{$i \gets 0$ to $l-1$}
\IF{$u = v$}
\STATE $s(u,v) \gets 1$
\ELSIF{$i < l-1$}
\FOR{each $u^{\prime} \in N(u)$}
\FOR{each $v^{\prime} \in N(v)$}
\STATE $s(u,v) \gets s(u,v) + \frac{c \cdot \chi(u^{\prime}) \cdot \chi(v^{\prime}) \cdot \alpha(u^{\prime}) \cdot \alpha(v^{\prime}) \cdot \mu_R}{\beta_R(u^{\prime}) \cdot \beta_R(v^{\prime})} \cdot s(u^{\prime}, v^{\prime})$
\STATE $i \gets i+1$
\ENDFOR
\ENDFOR
\ELSE
\STATE $s(u,v) \gets 0$
\ENDIF
\ENDFOR
\RETURN $s(u,v)$
\end{algorithmic}
\end{algorithm}

Algorithm~\ref{alg:bruteforce_computation} provides the pseudocode for a brute-force HetFS computation. If $u = v$, their similarity is set to 1, as shown in Line 4. From Line 5 to Line 13,  brute-force HetFS iteratively calculates the similarity between the two nodes. When the iteration count exceeds the predefined threshold $l$, indicating that the nodes are no longer potentially similar, Line 13 sets their similarity to 0.

SimRank has substantial computational costs in both time and space. Similarly, the brute-force approach of HetFS also encounters significant challenges in terms of computational overhead. In the next section, we propose an efficient optimized strategy of HetFS based on the random surfer method.

% % For each node $u \in V$, if the accessed node during every iteration is equal to node $u$, the HetFS score is assigned as 1. Otherwise, the similarity score is computed using vector multiplication as described in Eq.10. 
% % The time complexity of this computation is $O(ln^2)$, and storing the similarity matrix requires $O(n^2)$ space.

\subsection{A Path Enumeration Strategy}

The random surfer-pairs model of SimRank provides an intuitive understanding of its computation process, and we extended it from homogeneous information networks to HINs to make HetFS more intuitive. 
 % but also makes the calculation of similarity more efficient.
% To address these challenges, an efficient similarity search approach based on the Monte Carlo method is introduced. It assumes two surfers starting from nodes $u$ and $v$, randomly choosing and moving to adjacent nodes. The probability of the two nodes meeting for the first time is then considered their similarity. We define the trajectory of their encounter as an instance path.

\noindent\textbf{Definition 3} A tour of a meta-path. A tour of a meta-path $A_1\cdots A_{l-1}A_lA_{l-1}\cdots A_1$, is defined as $t = \langle w_1\cdots w_{l}w_{l}\cdots w_{\text{-}1} \rangle$ in the HIN through the meta-path $P$, if it connects node $u$ and $v$ with $u = w_1 and v = w_{\text{-}1}$ $\phi(u) = \phi(v) = A_i$, satisfying the conditions: for each node $i = 1, \cdots, l, \phi(w_i) = \phi(w_{\text{-}i}) = A_i$, and $(w_1, w_2), \cdots, (w_{l\text{-}1}, w_l), (w_{\text{-}l}, w_{\text{-}l+1}), \cdots, (w_{\text{-}2}, w_{\text{-}1})$ are all edges in $G$. 

% Then we give the definition of structure similarity along a specific instance path $p$ between $u$ and $v$ based on the above discussion:
% \begin{equation} 
%     \label{(similarityOverSemanticsPath)}
%     s_p(u,v)=\prod\limits_{i=1}^{l}\beta(u) \beta(v)
% \end{equation}
% where $l$ is the length of $p$. 
% , and we only consider symmetric meta-paths. Extensions are possible but are discussed elsewhere.
Here, we assume the edge type between node types $(A_i, A_{i+1})$ is unique. Given an tour $t$ connecting node $u$ and node $v$, in this case $w_1 = u$ and $w_{\text{-}1} = v$, we get the tour similarity $Pr(t)$ of two nodes $u \in V$ and $v \in V$ as follows:
\begin{equation} 
    \label{first_meeting_probability}
    Pr(t)=
    \left\{
    \begin{array}{llr}
    1, & \text{if } l = 0; & \\
    \prod\limits_{i \in \{1, \cdots, l-1, \text{-}l+1, \cdots, \text{-}1\}}\frac{c \cdot \chi(w_i) \cdot \alpha(w_i) \cdot \mu_{R_i}}{\beta_R(w_i)}, & \text{otherwise.}
    \end{array}
    \right.
\end{equation}
where $c$ is the decay factor adopted from SimRank, $l$ is the length of the instance path, $t = (w_1\cdots w_{l}\cdots w_{\text{-}1})$ is the tour, and $R_i$ is the relation between $(w_i, w_{i+1})$. Then the calculation of the node similarity $s(u, v)$ in Eq.~\ref{power_method} can be simplified as the aggregation of all tour similarity, each of which can be calculated by Eq.~\ref{first_meeting_probability}:

% \begin{equation}
%     \label{(pathSimilarity)}
%     s_{p}(u,v)=c\prod\limits_{i=1}^{l}\alpha_i \cdot \lambda_i \cdot \mu_i \cdot \beta_i
% \end{equation}

\begin{equation}
    \label{pathSimilarity}
    s(u,v)=\sum\limits_{P\in \mathcal{P}}\sum\limits_{t\in P}Pr(t)
\end{equation}
where $\mathcal{P}$ is the set of meta-path user defined, $P$ is one meta-path in $\mathcal{P}$, and $t$ is a tour of $u$ and $v$ along the meta-path $P$.

Algorithm~\ref{alg:path_enumeration_computation} outlines the pseudocode for random surfer-pairs model of HetFS. Line 1 generates random walks for node $u$ and $v$ within $l$ steps, respectively. Their meeting probability is then added to the final similarity if the random surfers encounter each other, as depicted in Line 9.
% For each node $u \in V$, if the accessed node during every iteration is equal to node $u$, the HetFS score is assigned as 1. Otherwise, the similarity score is computed using vector multiplication as described in Eq.10. 
% The time complexity of this computation is $O(ln^2)$, and storing the similarity matrix requires $O(n^2)$ space.

% Algorithm 2 provides the pseudocode for HetFS computation. If $u = v$, their similarity is set to 1, as shown in Line 4. From Line 5 to Line 13, HetFS iteratively calculates the similarity between the two nodes. When the iteration count exceeds the predefined threshold $l$, indicating that the nodes are no longer potentially similar, Line 13 sets their similarity to 0.
% For each node $u \in V$, if the accessed node during every iteration is equal to node $u$, the HetFS score is assigned as 1. Otherwise, the similarity score is computed using vector multiplication as described in Eq.10. 
% The time complexity of this computation is $O(ln^2)$, and storing the similarity matrix requires $O(n^2)$ space.

\begin{algorithm}
\caption{HetFS Computation with random surfer-pairs model}\label{alg2}
\label{alg:path_enumeration_computation}
\begin{algorithmic}[1]
\REQUIRE heterogeneous graph $G(V, E)$, number of iterations $l$, $\chi(u)$, $\alpha(u)$, $\beta_R(u)$, $\mu_R$, node $u$, $v$
\ENSURE similarity $s$
\STATE generate $k$ tours $t_u = \langle w_1w_2\cdots w_l \rangle $, $t_v = \langle w_{\text{-}1} w_{\text{-}2}\cdots w_{\text{-}l} \rangle $ from $u$ and $v$, respectively, within $l$ steps
\STATE $s(u,v) \gets 0$
\IF{$u = v$}
\STATE $s(u,v) = 1$
\ELSE
\FOR{each tour $t$}
\IF{$w_l = w_{-l}$}
\STATE let $t = \langle w_1w_2\cdots w_l(w_{\text{-}l})\cdots w_{\text{-}2}w_{\text{-}1} \rangle$
\STATE $s(u,v) \gets s(u,v) + Pr(t)$
\ENDIF
\ENDFOR
\ENDIF
\RETURN $s(u,v)$
\end{algorithmic}
\end{algorithm}

% {\noindent\textbf{A running example}

% To identify movies most similar to a given movie ``Terminator 2" in the ``IMDB'' dataset, due to the complexity of semantics in HINs, it is necessary to specify meta-paths for conducting the search. Potential meta-paths might include MAM, MMM, MDM, or their various combinations. Initially, we pre-compute and store certain data locally, such as the tf-idf scores between movie titles, descriptions, node centrality, edge contribution, and structure information.

\begin{figure}[h]
\centering\includegraphics[width=0.95\textwidth]{running_example.eps}
\caption{A running example of movie network.} \label{fig:running_example}
\end{figure}

% We use a sub-network of the movie information network in our running example, shown in Fig.~\ref{fig:running_example}, to illustrate our algorithm. Suppose a user wants to find movies most similar to a given movie ``Terminator 2" from the ``IMDB" dataset. This user query corresponds to a combination of the MAM, MMM, and MDM meta-paths. To retrieve top-$k$ most similar movies to ``Terminator 2", similarities between the specific movie and other movies should be calculated first. There are two approaches for computing these similarities in HetFS. One utilizes the power method, a straightforward brute-force computation detailed in Eq.~\ref{power_method}. The other approach employs a random-surfer-pairs model, leveraging a path enumeration strategy by Eq.~\ref{first_meeting_probability} and Eq.~\ref{pathSimilarity}.

We illustrate our algorithm using a sub-network from the ``IMDB" dataset, aimed at identifying movies similar to ``Terminator 2" (T2) via the MAM, MMM, and MDM meta-paths, as shown in Fig.~\ref{fig:running_example}. There are two methods for computing similarities: the power method (Eq.~\ref{power_method}) and a random-surfer-pairs model (Eq.~\ref{pathSimilarity}), both necessitating the preprocessing of content relatedness involving movie data and initial online computations.

% The power method calculates similarities between each pair of movie nodes, while the path enumeration method focuses specifically on similarities involving ``Terminator 2" and other movies. Both methods require the pre-processing of content scores from disk, which includes data on movie titles, descriptions, node centrality, edge contributions, and structural information, which are then loaded online for initial computations.
% Each iteration calculates the similarity scores between the specified node and all other nodes, storing these in a matrix according to Eq.~\ref{power_method}. This process continues through iterations until the last meta-path is processed, which incurs significant time and space costs.

% This method factors in the top three actors and director of each movie, calculating text-related similarities and node centrality. For example, the similarity scores between "Terminator 2" and other movies in the series like "Terminator" and "Terminator 3" are computed using actor centrality, director contributions, and movie series connections.

The power method computes similarities across all movie pairs, while the path enumeration method specifically assesses similarities involving T2. Both methods require the pre-processing and loading data such as content relatedness, node centrality, edge contributions, and structural weight. Here we focus on explaining the path enumerating approach. The text relatedness between T and T2 is 2.71, and between T and T3 is 1.58. Schwarzenegger starred in 17 movies, resulting in a centrality score of 5.67, Hamilton in 8 movies with a score of 2.67, and Furlong in 9 movies with a score of 3.0. Directors James Cameron and Jonathan Mostow have centrality scores of 2.45 and 0.67, respectively. Edges contributions of starred-in, directed-by, and movie series are 0.47, 0.30, and 0.41, respectively. Then, the similarity is calculated as follows:
\begin{itemize}
    \item $s(\text{T2, T}) = 0.47\cdot s_{\text{MAM}}(\text{T2, T})+0.3\cdot s_{\text{MDM}}(\text{T2, T})+0.41\cdot s_{\text{MMM}}(\text{T2, T}) = 0.47*\frac{5.67+2.67}{3*3})+0.3*(\frac{6}{17*17})+0.41*(\frac{2.71}{2*2}) = 0.72$.
    \item $s(\text{T2, T3}) = 0.47\cdot s_{\text{MAM}}(\text{T2, T3})+0.3\cdot s_{\text{MDM}}(\text{T2, T3})+0.41\cdot s_{\text{MMM}}(\text{T2, T3}) = (0.47*\frac{5.67}{3*3})+0.3*0+0.41*(\frac{1.58}{2*2}) = 0.46$.
    \item  Analogously, we have $s(\text{T2, True Lies}) = 0.30$, $s(\text{T2, Titanic}) = 0.01$.
\end{itemize}

% In detailing the path enumeration method, we consider textual and topological aspects of each meta-path to provide deep textual insights and precise similarity computations. HetFS enhances responsiveness to user queries by effectively managing complex semantics and ad-hoc meta-paths, significantly outperforming traditional methods in practical applications.

% 对于每部电影,我们考虑它的前三名演员、导演和其系列片。电影Terminator 2的三名演员按照先后顺序分别是施瓦辛格、哈密尔顿和福隆。首先我们计算电影title之间和description之间的相关性,Terminator、Terminator 3与Terminator 2的text 相关性分别是2.71,1.58。
% 然后计算每个节点的node centrality。以actor节点为例,数据库里记录了17部施瓦辛格参演的电影,8部哈密尔顿参演的电影,9部福隆参演的电影。于是我们根据公式~\ref{nodeCentrality}计算得到这三个actor node 的centrality分别是5.67,2.67,3.0。同样地,James Cameron执导了18部电影,Jonathan Mostow执导了2部电影,这两个director node 的centrality分别是2.45和0.67.根据公式计算得到edge contribution of acted-by, directed-by and being part of the same series are 0.47, 0.30 and 0.41,respectively。我们有$s (terminator2,terminator) = 0.47*s_MAM(terminator2,terminator)+0.3*s_MDM(terminator2,terminator)+0.41*s_MMM(terminator2,terminator) = 0.47*frac{5.67+2.67}{3*3})+0.3*(\frac{6}{17*17})+0.41*(\frac{2.71}{2*2}) = 0.72$, 而$s (terminator2,terminator3) = 0.47*s_MAM(terminator2,terminator3)+0.3*s_MDM(terminator2,terminator3)+0.41*s_MMM(terminator2,terminator3) = 0.47*frac{5.67}{3*3})+0.3*0+0.41*(\frac{1.58}{2*2}) = 0.0.46$,同样地,我们可以得到$s (terminator2,True lies) = 0.30$, $s (terminator2, Titanic) = 0.01$.}

% the random-surfer model calculates similarities only between the specified node and nodes involved in the ad-hoc meta-path tours, which are generated online. Utilizing the structural information of these tours, pre-computed content scores are integrated to compute the tour similarity for each meta-path. The aggregated similarities from all generated tours provide the final results, significantly reducing time and space requirements.}

HetFS captures deeper content knowledge and integrates them into similarity calculations, enhancing accuracy over traditional methods that may overlook text-related elements. By utilizing advanced techniques, HetFS swiftly processes ad-hoc meta-paths, improving responsiveness and scalability in real-world applications, thus outperforming conventional path-based and graph neural network methods.

% Regarding the processing of the MMM meta-path, which reflects the similarity based on movie series from titles and descriptions, traditional path-based methods either ignore the similarity brought by text or uniformly compute topological similarities by treating simple text segmentation as nodes. Such methods are only suitable for handling straightforward texts like abstracts or keywords. In contrast, HetFS captures deeper textual knowledge, thereby calculating more precise tour similarities for each meta-path.

% By employing sophisticated, multi-faceted multi-facet techniques, HetFS can swiftly respond to user queries with ad-hoc meta-paths online. This responsiveness, along with its powerful handling of complex semantics, greatly boosts the practical utility and scalability of HetFS in real-world applications, outperforming traditional path-based or graph neural network methods.

% Furthermore, our method not only computes targeted similarity for ad-hoc meta-paths but also extracts deep insights from existing information in the graph. 

\subsection{Analysis}

% In this section, we analyze the theoretical property of the HetFS algorithm. 
% , focusing on its efficiency and accuracy in handling complex queries
% \noindent\textbf{Time Complexity for worst-case graphs}
% \noindent\textbf{Time Complexity}

% TSF+panther

The similarity search in HetFS, as detailed in Algorithm~\ref{alg2}, is highly efficient. Theorem~\ref{thm1} demonstrates that HetFS can estimate a similarity score between a specific node and other involving nodes in $\mathcal{O}(\overline{d}^lm + \log k)$ time. This performance is independent of the graph size, showcasing the scalability of HetFS.
% independent of the size of the graph.

\begin{theorem}\label{thm1}
In HetFS, the average time required to compute a similarity score between a specific node and other involving nodes in a user query with ad-hoc meta-paths is bounded by $\mathcal{O}(\overline{d}^lm)$. The total time for processing the query is bounded by $\mathcal{O}(\overline{d}^lm + \log k)$.

% Each iteration in tour runs $\mathcal{O}$ time, so a tour $t_u$ of length $l$ runs in $\mathcal{O}(l\cdot n)$ time. Suming up for $k$ tours follows that the expected running time of $\mathcal{O}(\frac{m}{\varepsilon^2}\log{\frac{n}{\delta}})$.    
\end{theorem}

\begin{proof}
The estimation of top-$k$ similarity score involves two main parts.

Calculation Phase. To calculate $Pr(t)$ for a single tour $t$, HetFS begins by traversing over each tour within $l$ steps. We notice that in each iteration of the random walk, each edge in the graph is traversed at most once. Thus, the time complexity of each path is $\mathcal{O}(m)$. Following this, it aggregates tours across various meta-paths. Considering the average degree in the graph as $\overline{d}$, the average number of meta-path as $\lvert\overline{\mathcal{P}}\rvert$, and the average number of paths in each meta-path as $\lvert\overline{P}\rvert$, the overall average number of paths $\lvert\overline{P}\rvert\lvert\overline{\mathcal{P}}\rvert$ is proportional to $\overline{d}^l$. Consequently, the calculating cost is $\mathcal{O}(\overline{d}^lm)$.

Search Phase. The process of retrieving top-$k$ similar nodes based on the HetFS similarity scores introduces an additional time cost of $\mathcal{O}(\log k)$.

In summary, the overall time complexity for retrieving top-$k$ similar nodes is $\mathcal{O}(\overline{d}^lm + \log k)$, confirming the efficiency of the algorithm.

% \noindent\textbf{Space Complexity}

% The space complexity for storing paths

 % $\sum_{i=1}^(i=l)\overline{d}^i$.

\end{proof}

% \noindent\textbf{Time Complexity for power-law graphs}

% \noindent\textbf{Accuracy Guarantees}

% % TSF

% We now show that Algorithm~\ref{alg2} indeed gives an accurate and unbiased estimation of the HetFS similarity score $s(u,v)$ for each $v \in V, u \neq v$.

% \noindent\textbf{Lemma 1}\label{lm1}. For any node $v \in V$ and $v \neq u$, Algorithm~\ref{alg2} provides an estimator $\tilde{s}(u, v)$ such that $\text{E}[\tilde{s}(u, v)]=s(u,v)$.

% % Following this, Lemma 2 posits that if a random surf tour $t = \langle w_1\cdots w_{l}w_{l}\cdots w_{\text{-}1} \rangle$ is conducted, then after $l$ iterations, the similarity score $s(u,v)$ is equivalent to $Pr(t)$, the first meeting probability with respect to tour $t$.

% We need the following Lemma, which states that if we start a random surf tour $t = \langle w_1\cdots w_{l}w_{l}\cdots w_{\text{-}1} \rangle$, then the similarity score computed by the HetFS algorithm is exactly the probability that node $u$ and node $v$ first meet at $w_l$.

% \noindent\textbf{Lemma 2}. For any node $v \in V, v \neq u$, after the $l-$iteration, $s(u,v)$ is equal to $Pr(t)$, the first meeting probability of $v$ with respect to tour $t$.

% % \begin{proof}
% % We proof the following claim: let $Pr(t)$ denotes the similarity score of node $u$ and $v$ with tour $t$ after $l$ iterations.
% % \end{proof}

% Let $\tilde{s}(u, v)$ be the most accurate similarity score which can be obtained by HetFS, we have the following Theorem that states the error of the estimator $\tilde{s}(u, v)$ provided by Algorithm~\ref{alg2} can be bounded with high probability.

% \begin{theorem}\label{thm2}
% For each node $v \in V, v \neq u$, and a given error bound $\varepsilon$, Algorithm~\ref{alg2} returns an estimation $\tilde{s}(u, v)$ for $s(u,v)$ with a high probability that the error does not exceed $\varepsilon$, formalized as $\text{Prob} \left[ \forall v \in V, \lvert \tilde{s}(u, v)-s(u,v)\rvert > \varepsilon \right] \leq 2\varepsilon^{-\varepsilon^2\frac{\lvert P \rvert \lvert \mathcal{P}\rvert}{(B-c)^2}}$.
% \end{theorem}

% \begin{proof}[Proof]
% Assuming $\tilde{s}(u, v)$ as the estimated similarity score from random surfing, the most accurate $\tilde{s}(u, v)$ can be calculated by identifying all meeting nodes surfing from node $u$ and $v$. With $\varepsilon$ as the error bound and $B$ defined as $min\{1, \frac{c}{1-c}\}$, the specific inequality follows from Hoeffding's inequality with the defined bounds and assumptions regarding the error threshold and node interactions.
% \end{proof}

% This structured approach underscores the capability of HetFS to handle complex queries involving ad-hoc meta-paths efficiently but also highlights its superiority in computational speed and accuracy over traditional methods.

% By Lemma 1 and Chernoff bound, we have the following Theorem that states by performing $n_r = \frac{3c}{\varepsilon}\log{\frac{n}{\delta}}$ independent trials, the error of the estimator $\tilde{s}(u, v)$ provided by Algorithm~\ref{alg2} can be bounded with high probability.

% \begin{theorem}\label{thm2}
% For every node $v \in V, v \neq u$, and a given error bound $\varepsilon$, Algorithm~\ref{alg2} returns an estimation $\tilde{s}(u, v)$ for $s(u,v)$ such that $\text{Probe} \left[ \forall v \in V, \lvert \tilde{s}(u, v)-s(u,v)\rvert \leq \varepsilon \right] \geq 1-\delta$.
% \end{theorem}

% We need the following form of Chernoff bound:

% \noindent\textbf{Lemma 3 (Chernoff Bound \cite{fogaras2005scaling}).} For any set ${x_i} (i \in [1, n_x])$ of i.i.d random variables with mean $\mu$ and $x_i \in [0,1]$, $\text{Probe} \left\{ \left\lvert \sum_{i=1}^{n_x}x_i-n_x\mu \right\rvert \geq n_x\varepsilon\right\} \leq \text{exp}\left( -\frac{n_x \cdot \varepsilon^2}{\frac{2}{3}\varepsilon+2\mu} \right)$.

% \begin{proof}[Proof of Theorem~{\upshape\ref{thm2}}]
% We first note that in each trial $k$, the estimator $\tilde{s_k}(u, v)$ is a value in $[0,1]$. It is obvious that $\tilde{s_k}(u, v) \geq 1$. To see that $\tilde{s}(u, v) \leq 1$, notice that $\tilde{s_k}(u, v) = \sum_{i=2}^lPr(t)$ is a probability. More precisely, it is the probability that a random surf from node $u$ meets a random surf from node $v$ at the same steps.

% Thus, the final estimator $\tilde{s}(u, v) = \frac{1}{n_r}\sum_{k=1}^{n_r}\tilde{s}_k(u, v)$ is the average of $n_r$ i.i.d random variables whose values lie in the range $[0,1]$. Thus, we can apply Chernoff bound to get: $\text{Probe}\left\{ \lvert \tilde{s}(u, v) - s(u, v) \rvert \geq \varepsilon \right\} \leq \text{exp} \left( -\frac{n_r \cdot \varepsilon^2}{3 \cdot s(u, v)} \right)$. Recall that $n_r = \frac{3c}{\varepsilon^2}\log{\frac{n}{\delta}}$, and notice that $s(u,v) \leq c$, it follows that: $\text{Probe}\left\{ \lvert \tilde{s}(u, v) - s(u, v) \rvert \geq \varepsilon \right\} \leq \text{exp} \left(-\log{\frac{n}{\delta}}\right) = \frac{\delta}{n}$. Taking union bound over all nodes $v \in V$, it follows that: $\text{Probe}\left\{\forall v \in V, \lvert \tilde{s}(u, v) - s(u, v) \rvert \geq \varepsilon \right\} \leq \delta$, and the Theorem follows.
% \end{proof}

% While HetFS serves as a unified framework for homogeneous and heterogeneous information networks, the exact computation of HetFS can be time-consuming. To address this, we explore an alternative interpretation of HetFS using random walks to reduce the computation cost.

% Given a HIN $G(V, E)$, path similarity has the following properties for node $a_1$, $a_2 \in V$ similar to SimRank:
% \begin{enumerate}
%     \item [1)] Self-maximum: if $a_1 = a_2$, $s(a_1, a_2) = 1$.
%     \item [2)] Symmetry: $s(a_1, a_2) = s(a_2, a_1)$.
%     \item [3)] Monotonicity: $0 \leq s_l(a_1, a_2) \leq s_{l+1}(a_1, a_2) \leq 1$.
%     \item [4)] Existence: The iterative equation of HetFS always has a solution converging to a fixed value.
%     \item [5)] Uniqueness: The solution to the iterative equation of HetFS is unique.
% \end{enumerate}

% Next, we discuss the calculation of content similarity.

% \noindent\textbf{Proof.}
% \begin{enumerate}
%     \item [1)] Self-maximum: Following the intuition of SimRank, we can easily get that a node has a similarity score of 1 with itself.
%     \item [2)] Symmetry: Since $\phi(a_1) = \phi(a_2)$, given $I_R(a_1)$ and $I_R(a_2)$, $\mu_R(a_1)$ and $\mu_R(a_2)$ is the same for both $s_R(a_1, a_2)$ and $s_R(a_2, a_1)$. If $(I_R(a_1))_i = (I_R(a_2))_j$, the similarity score $s((I_R(a_1))_i, (I_R(a_2))_j)$ is symmetric and equal to 1. Analogously, $s((I_R(a_1))_j, (I_R(a_2))_i)$ is also symmetric and equal to 1. Otherwise, if $(I_R(a_1))_i \neq (I_R(a_2))_j$, Eq.(8) will continue the counting until a node pair $(u_k, v_k)$ is found in the $k$-th iteration where $(I_R(u_k))_i = (I_R(v_k))_j$. At this point, the similarity score $s(a_k, b_k)$ becomes symmetric in the $k$-th iteration.
%     \item [3)] Monotonicity: In each iteration k, $\hat{s}_{k+1}(a_1,a_2)$ is non-negative, thus $\hat{s}_{k+1}(a_1,a_2) = s_{k+1}(a_1,a_2) - s_k(a_1,a_2) = \mu \sum\limits_{i=1}^n \sum\limits_{j=1}^n Q_{a_1,i}(S_{k}(i,j) - S_{k-1}(j,i) Q_{j,a_2} \geq 0$. So $0 \leq S_{k+1}(a_1, a_2) = \mu \sum\limits_{i=1}^n Q_{a_1,i} \sum\limits_{j=1}^n S_{k}(i,j) Q_{j,a_2} \leq \mu \sum\limits_{i=1}^n Q_{a_1,i} \sum\limits_{j=1}^n Q_{j,a_2} = c \leq 1$. By induction, we have $0 \leq \hat{s}_{k}(a_1, a_2) \leq \hat{s}_{k+1}(a_1, a_2) \leq 1$ for any $k$.
%     \item [4)] Existence: The property of monotonicity ensures that $S_l(a_1, a_2)$ is bounded and non-decreasing as $l$ increases. $\mathop{lim}_{l \rightarrow \infty} S_{l}(a_1, a_2) = lim_{l \rightarrow \infty} S_{l+1}(a_1, a_2) = S_{a_1,a_2}$. If $a_1 = a_2$, $s(a_1,a_2) = 1$, otherwise $S_{a_1,a_2} = \mathop{lim}\limits_{l \rightarrow \infty} S_{l}(a_1, a_2) = \mu \mathop{lim}\limits_{l \rightarrow \infty} \sum\limits_{i=1}^n \sum\limits_{j=1}^n Q_{a_1,i} S_{l}(a_1, a_2) Q_{j,a_2} = \mu \sum\limits_{i=1}^n \sum\limits_{j=1}^n Q_{a_1,i} \mathop{lim}\limits_{l \rightarrow \infty} S_{l}(a_1, a_2) Q_{j,a_2} =  \mu \sum\limits_{i=1}^n \sum\limits_{j=1}^n Q_{a_1,i} S(a_1, a_2) Q_{j,a_2}$. Thus the existing property always holds.
%     \item [5)] Uniqueness: Suppose $s(*,*)$ and $\tilde{s}(*,*)$ are two solutions to the HetFS equations. For any entries, $a_1, a_2 \in V$, let $\Delta = s(a_1,a_2) - \tilde{s}(a_1,a_2)$ be the difference of the aforementioned two solutions. If the maximum absolute value of the difference, denoted as $|\Delta|$, is 0, then the two solutions can be proved as identically the same one. Obviously $|\Delta| = 0$ if $a_1 = a_2$. Otherwise $\Delta = s(a_1,a_2) - \tilde{s}(a_1,a_2) = \mu \sum\limits_{i=1}^n \sum\limits_{j=1}^n Q_{a_1,i} (S(a_1, a_2) - \tilde{S}(a_1,a_2)) Q_{j,a_2}$. Therefore, $|\Delta| = |s(a_1,a_2) - \tilde{s}(a_1,a_2)| = |\mu \sum\limits_{i=1}^n \sum\limits_{j=1}^n Q_{a_1,i} (S(a_1, a_2) - \tilde{S}(a_1,a_2)) Q_{j,a_2}| \leq \mu \sum\limits_{i=1}^n \sum\limits_{j=1}^n Q_{a_1,i} \Delta Q_{j,a_2} = \mu \Delta$. Thus if $a_1 \neq a_2$, $|\Delta| = 0$. In summary, we have $|\Delta| = 0$ whether or not $a_1$ equals $a_2$.
% \end{enumerate}

% \subsection{Power Method}

% As mentioned earlier, our HetFS heuristics builds upon the concept of SimRank \cite{jeh2002simrank}, which suggests ``two nodes are similar if there are similar nodes referencing them." We introduce several properties of HetFS to adapt the above idea on HINs:
% \begin{enumerate}
%     \item [1)] Only same-type nodes can yield valid similarity, while any similarity between nodes of different types is disregarded.
%     \item [2)] If two nodes share similar neighbors, they are considered similar. The similarity between nodes is influenced by the similarity of their neighboring nodes.
%     \item [3)] A node is assigned a similarity score of 1 with itself, indicating its inherent similarity to its own identity.
% \end{enumerate}

% To adhere to Property (1), the similarities between nodes in a HIN are stored in $\mathcal{|A|}$ matrices, each of which to a specific node type \textit{A} $\in$ $\mathcal{|A|}$. Thus, there are $\mathcal{|A|}$ independent matrices for HetFS rather than a single $V \times V$ matrix. This separation allows us to capture the distinct similarities within each type. Regarding Property (2), we consider the out-neighbor set $N_R(a)$ of node $a$ through all relations \textit{R} $\in$ $\mathcal{R(A)}$, recognizing the various roles that node \textit{a} can play across various relations. This consideration acknowledges the nuanced relationships and their impact on the similarity calculation. Property (3) ensures the normalization of similarity scores, aligning with the original SimRank definition. By normalizing the scores, we maintain consistency and facilitate meaningful comparisons.

% \subsection{Aggregation of Similarity}

% Now that we have obtained content similarity and path similarity, combining them yields the final similarity. However, during the merging process, a new challenge emerges. 

% As illustrated in Figure 3(a), the value range and variation magnitude of content similarity differ from those of path similarity. Notably, the content similarity values for the top 500 results are consistently greater than 0.8, which is understandable given that even after removing stop words, text content tends to exhibit similarity due to frequent meaningful comparisons. If directly added together, there's a risk that content similarity might overshadow path similarity, masking the impact of the latter. Additionally, the change trend of content similarity is relatively gentle, with the top 500 results decreasing only from 0.4 to 0.08, reflecting a 20-fold change. We consider such a gradual trend insufficient for effectively distinguishing the similarity between different documents. In contrast, path similarity exhibits a more pronounced variation, decreasing from 0.09 to 8.4e-7, indicating a change of over 10,000 times. Moreover, the decline in path similarity is gradual and progressive, suggesting that most of the decrease in these 500 steps is meaningful. These aspects underscore the distinguishing value of path similarity.

% Consequently, we decide to stretch the values of content similarity. This adjustment aims to enhance its discriminative capability and align its value range with that of path similarity, ensuring a more balanced contribution to the final similarity measure.

% \begin{figure}[h]
%     \centering
%         \includegraphics[width=0.8\textwidth]{similarityCompare.png}
%     \caption{The comparison of path similarity and content similarity.} \label{fig:similarityCompare}
% \end{figure}

% We propose a novel method to integrate both similarity measures using SimRank, technically. We treat content in the corpus as one set of nodes and the decomposed vocabulary as another set of nodes, establishing a bipartite graph. Subsequently, we employ the intuition of SimRank to compute the similarity between content. 

% Take the texture content for example, the term frequency (TF) values encompass term centrality, while the inverse document frequency (IDF) values encapsulate the contribution of edges in the bipartite graph. Treating TF-IDF values as weights for both nodes and edges, we integrate structural information from the content with SimRank, thereby enhancing the differentiation of TF-IDF values and improving the overall similarity calculation. The final similarity is defined as:
% \begin{equation} 
%     \label{(pathSimilarity)}
%     s(u, v)=c\sum\limits_{p^{'}\in P(u,v)}\prod\limits_{i=1}^{l}\alpha_i \lambda_i \mu_i \beta_i + \sum\limits_{t\in T}\frac{f(t_u)f(t_v)}{\lvert C_t\rvert\lvert C_t\rvert}
% \end{equation}
% % \begin{equation} 
% %     \label{(pathSimilarity)}
% %     s(u, v)=c\sum\limits_{p\in P(u,v)}\prod\limits_{i=1}^{l}\lvert\lvert \tanh{(\alpha \lambda)}\mu\rvert\rvert_{1} \beta + \sum\limits_{t\in T}\frac{\rm{\scriptstyle TF}\text{-}\rm{\scriptstyle IDF}\scriptstyle (t_u\scriptstyle )\rm{\scriptstyle TF}\text{-}\rm{\scriptstyle IDF}\scriptstyle (t_v\scriptstyle )}{\lvert C_t\rvert\lvert C_t\rvert}
% % \end{equation}
% where $t$ is the term tokenized from content of nodes $u$ and $v$, the set of which is $T$. $f(t_u)$ and $f(t_v)$ represent the content similarity of term $t$ in the content of nodes $u$ and $v$ respectively, while $\lvert C_t \rvert$ denotes the number of documents in which the term $t$ appears.

% After scaling, content similarity exhibits distinguishable variations, and its value range highly overlaps with path similarity. In this way, the combined similarity can effectively leverage both path information and content information, as depicted in Fig.3 (b).




% \subsection{This is an example for second level head---subsection head}\label{subsec2}

% \subsubsection{This is an example for third level head---subsubsection head}\label{subsubsec2}

% Sample body text. Sample body text. Sample body text. Sample body text. Sample body text. Sample body text. Sample body text. Sample body text. 


% \subsection{On Dynamic Graphs}

% HetFS distinguishes itself from other random-walk-based solutions by enabling efficient updates to the one-step graph when modifications are made to the graph $G(V, E)$. This efficiency arises from two crucial factors. Firstly, the one-step graph directly indexes the original random walks, ensuring the explicit preservation of the relationship between the one-step graph and $G(V, E)$. Secondly, the one-step graph is constructed through independent node random walks. Consequently, when modifications occur in $G(V, E)$, Updation only takes place among the relevant vertexes and their in-neighbors in the one-step graph. We provide a detailed explanation of the process involved in updating the one-step graph when modifying a node or an edge in a HIN $G(V, E)$ as follows.

% \textbf{Edge modification in $G(V, E)$}. When a directed edge $e = <a_ia_{i+1}> \in E$ is added to or removed from the graph, the modifications solely affect the incoming edges of node $v$ within the one-step graphs. This is due to the fact that the number of incoming edges affects how the random walk surfs. Hence, to update the one-step graphs for node $v$, it is sufficient to re-index the edge using the updated set of incoming edges for node $v$. The time complexity of updating the one-step graph to modify edge is $O(1)$.

% \textbf{node modification in $G(V, E)$}. When a node $v$ is modified, it can be decomposed into a sequence of edge modifications. The overall effect of this node modification can be explained as follows: each source node of not only incoming edges but also outgoing edges to node $v$ undergoes an edge modification, experiencing $|I(v)|+|O(v)|$ edge modifications. The incoming edges, as well as outgoing edges of the node, need to be re-indexed, and node $v$ and its involving edges are discarded. The time complexity of updating the one-step graph to modify node is $O(\frac{m}{n})$.

\section{Experiments}\label{sec5}

Extensive experiments are conducted to address the following research questions. This section reports our results.
\begin{itemize}
    \item \textbf{RQ1}. How does HetFS perform compared to state-of-the-art methods for user queries with ad-hoc meta-paths in terms of both effectiveness and efficiency?
    \item \textbf{RQ2}. How does HetFS perform compared to state-of-the-art methods across various downstream graph mining tasks, including link prediction, node clustering, and node classification?
    \item \textbf{RQ3}. What is the impact of different components, such as node information, edge information, and content information, on the performance of HetFS?
    % \item \textbf{RQ4}. How do various hyper-parameters affect the performance of HetFS?
\end{itemize}

\subsection{Experimental Design}

We describe the details of datasets, methods, how to set the ground truth and parameters, and how to assess the effectiveness of each method in our experiments.

\subsubsection{Data Description} 

We use 3 real-world datasets described as follows to evaluate the performance of HetFS as compared to state-of-the-art baselines.
\begin{itemize}
\item DBLP $\footnote{https://aminer.org/data}$: is an academic dataset \cite{tang2008arnetminer} extracted from website sources. We adopt a subset of Academic that includes information about paper writing, citations, publications, titles, and abstracts, covering three types of nodes and other related content. Authors are categorized into four research areas: DB, DM, NLP, and CV. Specifically, we extracted the titles and abstracts of all papers and aggregated them into a title corpus and an abstract corpus. Using these two corpora as references, HetFS calculated the textual similarity between each pair of papers.
% \item DBLP $\footnote{https://web.cs.ucla.edu/~yzsun/data/}$: a bibliographic website for computer science. We adopt a subset of DBLP that includes information about paper writing, citations, publications, and titles, encompassing four types of nodes. Authors are categorized into four research areas: Database, Data Mining, Artificial Intelligence, and Information Retrieval.
% \item ACM $\footnote{https://github.com/Jhy1993/HAN/tree/master/data}$: an academic citation network that extracts papers published in conferences such as KDD, SIGMOD, SIGCOMM, MobiCOMM, and VLDB. It categorizes the papers into three classes according to their published conference: Database, Wireless Communication, and Data Mining.
\item IMDB $\footnote{https://grouplens.org/datasets/hetrec-2011/}$: is a dataset that encompasses information about both movies and TV shows. It includes details such as cast, production crew, plot summaries, and related information, covering four types of nodes and other related content. The movies are categorized into five classes based on their genre information: Action, Comedy, Drama, Romance, and Thriller.
\item LastFM $\footnote{https://grouplens.org/datasets/hetrec-2011/}$: is an online music website containing data on music listening behavior. It comprises information about users, artists, and tags, along with their interactions. We use the dataset released by MAGNN \cite{fu2020magnn} for the link prediction task.
\end{itemize}
Table 3 summarizes the dataset statistics.

\begin{table}[h]
\fontsize{12}{12}\selectfont
\caption{\centering Statistics of datasets.}\label{statisticOfDatasets}
% \scriptsize
\footnotesize
% \resizebox{0.49\textwidth}{!}{
\begin{tabular}{cccccccccm{5cm}}
\toprule
Dataset & \multicolumn{2}{c}{Entity} & \multicolumn{2}{c}{Edge} \\
\midrule
\multirow{2}*{Academic} & \textbf{A}uthor: 28645 & \textbf{P}aper: 21044 & A-P: 69311 & P-P: 34238 \\
 & \textbf{T}erm: 22551 & \textbf{V}enue: 18 & P-T: 171774 & V-P: 21044 \\
\midrule
% \multirow{2}*{DBLP} & Author: 4057 & Paper: 14328 & writing: 39290 & mentioning: 171620 \\
%  & Term: 7723 & Venue: 20 & publishing: 28656 \\
% \midrule
% \multirow{2}*{ACM} & Author: 5959 & Paper: 3025 & writing: 100797 & citing: 165904 \\
%  & Subject: 56 & Term: 1902 & publishing: 42872 & mentioning: 284156 \\
% \midrule
\multirow{3}*{IMDB} & \textbf{M}ovie: 8742 & \textbf{D}irector: 3279 & D-M: 6582 & A-M: 19612 \\
 & \textbf{A}ctor: 9418 & \textbf{T}erm: 7295 & M-T: 16247 & M-G: 4914 \\
 & \textbf{G}enre: 17 & & & \\
\midrule
\multirow{2}*{LastFM} & \textbf{U}ser: 1892 & \textbf{A}rtist: 17632 & U-U: 12717 & U-A: 92834 \\
 & \textbf{T}ag: 1088 & & A-T: 23253 & \\
\bottomrule
\end{tabular}
% }
\end{table}

\subsubsection{Baseline} 

We compare HetFS against various methods, including traditional graph similarity search methods, graph embedding models, and GNNs. The baseline methods are outlined below.

\begin{itemize}
\item HGT \cite{hu2020heterogeneous} is a heterogeneous GNN that extends the transformer architecture to the graph-structured data. It utilizes the self-attention mechanism of the transformer to incorporate the unique characteristics of HINs. 
\item HetGNN \cite{zhang2019heterogeneous} is a heterogeneous GNN with the ability to capture both structural and content-related information from heterogeneous graphs. It utilizes attention mechanisms to aggregate information from neighboring nodes.
\item MHGNN \cite{li2023metapath} is a meta-path-based heterogeneous GNN with the ability to capture both structural and content-related information from heterogeneous graphs. 
\item MAGNN \cite{fu2020magnn} is a heterogeneous GNN designed to capture information from various levels by incorporating node attributes, intermediate semantic nodes, and multiple meta-paths. This enables MAGNN to leverage both network topology and content-related information for effective representation learning in HINs.
\item HAN \cite{wang2019heterogeneous} is a heterogeneous model that proposes an attention mechanism to address the heterogeneity inherent in nodes and edges. Specifically, it employs two layers of attention structures to capture weights at both the node and semantic levels intricately.
\item RGCN \cite{schlichtkrull2018modeling} is an extension of GCN for relational (multiple edge types) graphs, which can be conceptualized as a weighted combination of standard graph convolutions, each tailored to a specific edge type within the graph.
\item DeepWalk \cite{perozzi2014deepwalk} is a homogeneous graph embedding method that pioneered the use of random walks to transform the graph structure into sequences, subsequently utilizing these sequences to embed nodes of the graph.
% \item Node2Vec \cite{grover2016node2vec} is an enhanced homogeneous model derived from DeepWalk. While DeepWalk essentially utilizes a depth-first traversal algorithm with the ability to revisit visited nodes to generate node sequences, Node2Vec adopts a novel random walk strategy. It combines breadth-first and depth-first traversals by introducing parameters to balance local and global similarity.
% \item Metapath2Vec \cite{dong2017metapath2vec} is a heterogeneous model that generates random walk sequences based on pre-defined meta-paths to construct heterogeneous neighbors for each node.
\item PathSim \cite{sun2011pathsim} is a traditional heterogeneous similarity measure calculated by the path instance count along a specific meta-path. It addresses the multi-edge similarity search problem between two nodes.
\end{itemize}

\subsubsection{Experimental Setup}

When dealing with meta-path-free scenarios, PathSim aggregates the results from all possible paths and then merges them and sorts them to obtain the final ranking. Specifically, we set $\epsilon$ to 0.000005 and $k$ to 1000, that is, we consider results with similarity values above this threshold  $\epsilon$ and rank $k$ within the top 1000 nodes as similar nodes. Since DeepWalk handles meta-path-free scenarios, it does not distinguish between node types and conducts random walks on the entire graph to generate preprocessed paths. For the HetGNN algorithm, we adopt the experimental parameters from the original paper.

All experiments are conducted on a Linux machine featuring an Intel(R) Xeon(R) Silver 4208 CPU running at 2.1GHz and 128GB of memory.

% However, for meta-path-free experiments, since there is no ready-made ground truth for the meta-path-free 
 
% \noindent\textbf{Ground Truth} Due to the absence of an established ground truth for similarity calculation based on ad-hoc meta-paths, and considering that HetGNN leverages a richer set of graph-related information than many other HGNNs by incorporating not only node and network information but also content information, we choose to employ the results of HetGNN on specific meta-paths as the benchmark for Task 1.
 
% For meta-path-based experiments, we calculate $s(u,v)$ by PathSim on the whole dataset as the ground truth.situation, we conduct a blinded user study using pooling to evaluate outcomes \cite{schutze2008introduction}, a commonly used approach in Information Retrieval systems for assessing the ranking quality of top-k documents when obtaining ground truth ranking scores is challenging.

% \noindent\textbf{Metrics} We divide our effectiveness experiments into two main parts: the meta-path-based part and the meta-path-free part. Precision and recall are two critical metrics used to evaluate effectiveness. They describe the accuracy and completeness of the system in returning results. A method with high precision often returns high-quality results but may miss some genuinely relevant results. On the other hand, a method with a high recall rate can retrieve more relevant results but may also produce some irrelevant ones simultaneously. Therefore, in practical applications, it is necessary to consider both precision and recall comprehensively to find a balance point that allows the system to return accurate and complete results. Here, we measure the effectiveness of different algorithms by mAP (mean Average Precision), and recall is described as follows.

% The mean Average Precision at $k (mAP@k)$ for the given query set $Q$ is defined as the mean average precision of the top-$k$ similar nodes returned by a specific method, denoted as $\mathcal{mAP}_{A, k} = \frac{1}{Q} \sum_{i=1}^{|Q|}\mathcal{AP}_{A, k}(q_i)$, where $\mathcal{AP}_{A, k}(q_i)$ represents the average precision of the top-$k$ similar nodes to $q_i$ (the $i$-th node in $Q$), defined as $\mathcal{AP}_{A, k}(q_i) = \frac{1}{|k|}\sum_{j=1}^{|k|}\mathcal{P}(q_i, v_j)$, where $j$ is the rank of $v_j$ in the returned list, and $\mathcal{P}(q_i, v_j)$ is the precision function for evaluating the effectiveness of similarity of similarity between $q_i$ and $v_j$, which is defined as $\mathcal{P}(q_i, v_j) = \frac{1}{|k|}\mathcal{H}(q_i, v_j)$. $\mathcal{H}(q_i, v_j)$ is the hit count that records the number of results in $v_j$ that match the ground truth, which is defined as $\mathcal{H}(q_i, v_j) = \sum_{j=1}^{|k|}rel(j)$, where $rel(j)$ is just an indicator that says whether that $j^{th}$ item is relevant ($rel(j) = 1$) or not ($rel(j) = 0)$. Notice that mAP values are larger when there have been more successes in front of it - that is because the precision of the k$_th$ subset is higher the more correct guesses a method has had up to point k. Thus, AP rewards it for front-loading the most likely accurate recommendations.

% The recall of the top $k$ for a given query set Q, which represents the proportion of relevant objects among those returned by a specific method, is defined as $\mathcal{R}_{A, k} = \frac{1}{Q} \sum_{i=1}^{|Q|}\mathcal R_{A, k}(q_i)$, where $\mathcal R_{A, k}(q_i)$ is the recall of the returned top k objects similar to $q_i$ (the $i$-th object in $Q$), which is defined as $\mathcal R_{A, k}(q_i) = \frac{1}{|k|}\mathcal{H}(q_i, v_j)$, where $\mathcal{H}(q_i, v_j)$ is the hit counter that records the number of results matching the ground truth in the returned list.

% \subsection{Experimental analysis}

\subsection{Ad-hoc Queries (RQ1)}

As mentioned earlier, various meta-paths convey varied semantic meanings, leading to diverse query results. We focus on the differences in effectiveness and efficiency of the algorithm under ad-hoc query scenarios.

First, a case study is conducted on the ``IMDB" dataset. We use the movie ``Terminator 2: Judgment Day", a sci-fi film directed by James Cameron and starring Arnold Schwarzenegger, among others, as an example and query the top 8 movies that are most similar to it. The experiment compares outcomes for three scenarios: two meta-paths (``MDM", i.e., movie-director-movie, ``MAM," i.e., movie-actor-movie) and meta-path-free. These scenarios represent co-actors, movies directed by the same director, and a comprehensive search for similar movies using all available information. Graph embedding methods and HGNNs rank nodes based on their preference score, computed as the inner product between their embeddings.

As previously highlighted, diverse meta-paths encapsulate unique semantic meanings, reflecting varied user interests in information. This underscores our emphasis on the disparate outcomes yielded by searches conducted through different paths. Table~\ref{tab:result_imdb} displays the similarity results obtained under different specific meta-paths, revealing notable variations. The experimental results are consistent with the expectations. Movies retrieved based on the ``MAM" meta-path have similar leading actors to ``Terminator 2", while movies retrieved based on the ``MDM" meta-path have similar directors to ``Terminator 2".

% Table generated by Excel2LaTeX from sheet 'merge'
\begin{table*}[htbp]
\centering
\caption{\centering Case study on query ``Terminator 2" under different meta-paths on ``IMDB" dataset.}
\footnotesize
\renewcommand\arraystretch{1.2}
\resizebox{\textwidth}{!}{
\begin{tabular}{|c|ll|ll|}
\multicolumn{1}{c}{\multirow{1}[0]{*}{}} & \multicolumn{2}{c}{(a) PathSim} & \multicolumn{2}{c}{(b) DeepWalk} \\
\hline
Rank  & meta-path: MAM   & meta-path: MDM   & meta-path: MAM   & meta-path: MDM \\
\hline
1     & True Lies & True Lies & Hercules in New York & True Lies \\
2     & Eraser & The Abyss & Pumping Iron & The Abyss \\
3     & Jingle All the Way & Aliens & Detroit Rock City & Aliens \\
4     & Conan the Barbarian & Titanic & King Kong Lives & Titanic \\
5     & Total Recall & Ghosts of the Abyss & The Kid \& I & Aliens of the Deep \\
6     & End of Days & Aliens of the Deep & Kindergarten Cop & Ghosts of the Abyss \\
7     & The Running Man & -     & Brainscan & My Name Is Bruce \\
8     & Red Heat & -     & Children of the Corn & On\_Line \\
% 9     & Twins & -     & Jingle All the Way & Passion of Mind \\
% 10    & Collateral Damage & -     & American Heart & Ma vie en rose \\
\hline
\multicolumn{1}{c}{\multirow{1}[0]{*}{}} & \multicolumn{2}{c}{(c) HetGNN} & \multicolumn{2}{c}{(d) HetFS} \\
\hline
Rank  & meta-path: MAM   & meta-path: MDM   & meta-path: MAM   & meta-path: MDM \\
\hline
1     & Jingle All the Way & Titanic & Terminator & True Lies \\
2     & End of Days & Ghosts of the Abyss & True Lies & Aliens \\
3     & Total Recall & The Abyss & Total Recall & The Abyss \\
4     & American History X & Aliens of the Deep & Jingle All the Way & Titanic \\
5     & Terminator 3 & True Lies & End of Days & Ghosts of the Abyss \\
6     & Pumping Iron & Aliens & The Running Man & Aliens of the Deep \\
7     & Kindergarten Cop & El Callejon de los Milagros & Red Heat & - \\
8     & Hercules in New York & Smokey and the Bandit Part 3 & Pumping Iron & - \\
% 9     & Twins & Care Bears Movie II: A New Generation & Eraser & - \\
% 10    & True Lies & Over the Hedge & King Kong Lives & - \\
\hline
\end{tabular}%
}
\label{tab:result_imdb}%
\end{table*}%

Next, we enumerate all paths with a length of 2, treating them as the complete semantics, to conduct meta-path-free experiments. The outcomes are listed in Table~\ref{tab:result_imdb_meta-path-free}. Our method, HetFS, demonstrates the ability to retrieve movies associated with the provided movie effectively. Specifically, HetFS not only identifies movies with the same main actor or director but also discovers series such as ``Terminator 3" and ``Terminator." The former appears only in the top ten search results of PathSim among the other three methods, while the latter does not appear in the top ten search results of any other methods. Series are generally considered to have high relevance. 
% Specifically, ``Terminator 3" shares the same main actor, Arnold Schwarzenegger, with ``Terminator 2", while ``Terminator" and ``Terminator 2" share no common actors or directors besides being part of the same series. 

% For convenience, we have counted the number of results each method found in the user study and displayed them in the last row of the table.

\begin{table*}[h]
\centering
\caption{\centering Case study on query ``Terminator 2" on ``IMDB" dataset under meta-path-free scenarios.}
\footnotesize
\renewcommand\arraystretch{1.2}
\resizebox{\textwidth}{!}{
\begin{tabular}{|c|l|l|l|l|}
% \multicolumn{1}{c}{\multirow{1}[0]{*}{}} &  \multicolumn{1}{c}{(a) PathSim} &  \multicolumn{1}{c}{(b) DeepWalk} &  \multicolumn{1}{c}{(c) HetGNN} &  \multicolumn{1}{c}{(d) HetFS} \\
\hline
Rank & PathSim & DeepWalk & HetGNN & HetFS \\
\hline
1 & The Abyss & True Lies & True Lies & Terminator \\
2 & Aliens & Ghosts of the Abyss & Aliens of the Deep & Terminator 3 \\
3 & Titanic & Aliens of the Deep & Pumping Iron & True Lies \\
4 & Ghosts of the Abyss & Detroit Rock City & Ghosts of the Abyss & Titanic \\
5 & Aliens of the Deep & Brainscan & Hercules in New York & The Abyss \\
% 6 & True Lies & King Kong Lives & Aliens & Aliens \\
% 7 & Terminator 3 & Children of the Corn & Red Heat & Hercules in New York \\
% 8 & Terminator Salvation & Hercules in New York & The Abyss & Pumping Iron \\
% 9 & Eraser & Pumping Iron & Kindergarten Cop & Collateral Damage \\
% 10 & Jingle All the Way & The Kid \& I & Collateral Damage & Jingle All the Way \\
\hline
\end{tabular}%
}
\label{tab:result_imdb_meta-path-free}%
\end{table*}%

Next, We utilize the ``DBLP" dataset and retrieve the top 8 authors most similar to Jiawei Han, who has the highest number of paper records in the dataset (154 papers). The experiment compares results for three scenarios: two meta-paths and meta-path-free, representing co-authors, co-conference participants, and a comprehensive search for similar authors using all available information. 
 % (``APA", i.e., author-paper-author, ``APVPA", i.e., author-paper-venue-paper-author)

The outcomes are displayed in Table~\ref{tab:result_dblp}. Notably, Yizhou Sun was a student of Jiawei Han, and now being an influential mentor in the field of graph data mining. They are close collaborators with strong connections. HetFS and DeepWalk rank Yizhou Sun at the top, while HetGNN is in the third position, and PathSim is in the sixth position. However, the results of HetGNN are not satisfactory. For instance, author Philip Yu did not appear in the top ten results of other algorithms but was considered by HetGNN as the most similar author, ranked first.

\begin{table*}[h]
% \fontsize{12}{12}\selectfont \cellcolor{gray!40}
\caption{\centering Case study on query Jiawei Han under different meta-paths on ``DBLP" dataset.}\label{tab:result_dblp}
\scriptsize
% \footnotesize
\renewcommand\arraystretch{1.3}
\resizebox{\textwidth}{!}{
\begin{tabular}{|c|ccc|ccc|m{2cm}}
\multicolumn{1}{c}{\multirow{1}[0]{*}{}} & \multicolumn{3}{c}{\textbf{(a) PathSim}} & \multicolumn{3}{c}{\textbf{(b) DeepWalk}} \\
\hline
\textbf{Rank} & meta-path: APA & meta-path: AVA & meta-path-free & meta-path: APA & meta-path: AVA & meta-path-free \\
\hline
1 & \textbf{Yizhou Sun} & Philip S. Yu & Philip S. Yu & Xiaolei Li & Haixun Wang & \textbf{Yizhou Sun} \\
2 & Xifeng Yan & Christos Faloutsos & Christos Faloutsos & Tim Weninger & Michalis Vazirgiannis & Xiao Yu \\
3 & Chi Wang & Hui Xiong & Divesh Srivastava & Brandon Norick & Kevin Chen-Chuan & Marina Danilevsky \\
4 & Xiao Yu & Wei Wang & Xifeng Yan & \textbf{Yizhou Sun} & Min Wang & Bo Zhao \\
5 & Jing Gao & Tao Li & Gerhard Weikum & George Brova & C. Lee Giles & Yintao Yu \\
6 & Bo Zhao & Jian Pei & \textbf{Yizhou Sun} & Marina Danilevsky & Francesco Bonchi & Tim Weninger \\
7 & Dong Xin & Jie Tang & Aristides Gionis & Yintao Yu & Jianyong Wang & Zhijun Yin \\
8 & Bolin Ding & Haixun Wang & Wei Wang & Bo Zhao & Enhong Chen & Fangbo Tao \\
9 & Deng Cai & Lei Chen & Francesco Bonchi & Tianyi Wu & HweeHwa Pang & Jialu Liu \\
10 & Xiaofei He & Wei Fan & Tao Li & Roland Kays & Aixin Sun & Xiaolei Li \\
\hline
\multicolumn{1}{c}{\multirow{1}[0]{*}{}} & \multicolumn{3}{c}{\textbf{(c) HetGNN}} & \multicolumn{3}{c}{\textbf{(d) HetFS}} \\
\hline
\textbf{Rank} & meta-path: APA & meta-path: AVA & meta-path-free & meta-path: APA & meta-path: AVA & meta-path-free \\
\hline
1 & Jimeng Sun & Rajeev Rastogi & Philip Yu & Quanquan Gu & Philip S. Yu & Quanquan Gu\\
2 & Brandon Norick & Tamraparni Dasu & Chen Chen & \textbf{Yizhou Sun} & Christos Faloutsos &  \textbf{Yizhou Sun} \\
3 & Guojie Song & Guiquan Liu & \textbf{Yizhou Sun} & Dong Xin & Divesh Srivastava & Dong Xin \\
4 & Zeng Lian & Dashun Wang & Xifeng Yan & Deng Cai & Wei Wang & Deng Cai \\
5 & Marina Danilevsky & Xiao Yu & Bolin Ding & Chi Wang & Yongxin Tong & Chi Wang \\
6 & \textbf{Yizhou Sun} & Ming Li & Xiao Yu & Tim Weninger & Lei Chen & Tim Weninger \\
7 & Christos Boutsidis & Neal E. Young & Peixiang Zhao & Philip S. Yu & Haixun Wang & Philip S. Yu \\
8 & Jinyan Li & Jian Xu & Spiros Papadimitriou & Xiaofei He & Tao Li & Xiaofei He \\
9 & Yang Cao & Robson Leonardo & Sangkyum Kim & Xiao Yu & Jian Pei & Xiao Yu \\
10 & Yu-Ru Lin & Chengnian Sun & Hong Cheng & Jing Gao & Hui Xiong & Jing Gao \\
\hline
\end{tabular}
}
\end{table*}

In order to meet the demands of ad-hoc queries from users, the algorithm's responsiveness in switching meta-paths is crucial. We then recorded the CPU time required by different methods to search for similar nodes based on different paths. Fig.~\ref{fig:time_cost_compare} illustrates the efficiency performance of the compared methods. A logarithmic scale is used in the graph to visualize the differences in results better. The graph embedding method and HGNNs require a significant amount of time to train node embeddings. In the meta-path-free scenario, the most time-consuming algorithm (HetGNN) takes 63,038,114 ms (approximately 105 min) and 5,281,670 ms (approximately 88 min) to process queries on the ``DBLP'' dataset and ``IMDB'' dataset, respectively. Such long waiting times are almost intolerable for ad-hoc searches. On the other hand, PathSim and HetFS do not require time-consuming training, and since they support single-source queries, their average time costs for processing a query of the ``DBLP'' dataset are only 6 ms and 8 ms, respectively. 

\begin{figure}[h]
    \centering
        \includegraphics[width=0.9\textwidth]{time_cost_compare.eps}
    \caption{The time cost under different meta-paths on ``DBLP'' and ``IMDB'' datasets.} \label{fig:time_cost_compare}
\end{figure}


\subsection{Applications (RQ2)}

To validate the practicality of HetFS, we compare it with measures across various tasks.

\subsubsection{Link Prediction}

% Given a time $Ts$, prior to the split year ($Ts$), data is assigned as training data, while data post-Ts is utilized as test data. We set $Ts$ to 2014 in the ``DBLP" dataset. We conduct two types of tasks, collaboration link prediction and citing link prediction. In the former, we remove author-paper edges post-2014 and retrieve the top-k similar authors of author $u$ based on existing links. These $k$ candidate authors are then regarded as having a collaboration link with author $u$. Similarly, in the latter task, we eliminate paper-paper citing edges after 2014 and retrieve the top-k similar papers for each paper based on existing links. Subsequently, we consider citing links to exist between these $k$ candidate papers and the authors of paper $u$. $AUC$ and $F1$ are utilized for quality measurement, where $F1$ = $\frac{2*recall*precision}{recall+precision}$, with $TP$ = $|candidates\cap deleted \text{ } edges|$, $TP+FP$ = $|deleted \text{ } edges|$, $TP+FN = |candidates|$, $precision = \frac{TP}{TP+FP}$, $recall = \frac{TP}{TP+FN}$. Moreover, only new links among nodes in the test data are considered, and duplicated links are removed during evaluation. 

% We conduct two types of tasks: collaboration link prediction and citing link prediction. In the former. For the latter task, we eliminate paper-paper citing edges after 2014 and retrieve the top-k similar papers for each paper based on existing links. Subsequently, we consider citing links to exist between these $k$ candidate papers and the authors of paper $u$.

Two split settings are employed for the link prediction task. For the ``DBLP'' dataset, given a time $Ts$, prior to the split year ($Ts$), data is assigned as training data, while data post-Ts is utilized as test data. We set $Ts$ to 2014 in the ``DBLP" dataset. We remove author-paper edges post-2014 and retrieve the top-k similar authors of author $u$ based on existing links. These $k$ candidate authors are then regarded as having a collaboration link with author $u$. For the ``LastFM'' dataset, we randomly sample a subset of links for training and reserve the remaining links for evaluation, maintaining a train/test ratio of 7:3. $AUC$ and $F1$ are utilized for quality measurement. Moreover, during evaluation, only new links between nodes in the test data are considered, and any duplicated links are removed.

% In the ``IMDB" dataset, we randomly sample a portion of links for training and utilize the remaining links for evaluation, with a train/test ratio (in terms of movie numbers) set to 7:3. 

Table~\ref{linkprediction} presents the performance metrics for all models, highlighting superior results in bold. It is observed that HetFS exhibits relatively favorable outcomes compared to the graph embedding method and HGNNs, indicating that the systematic integration of path similarity and content similarity effectively captures graph information for link prediction. We noticed that the results of PathSim are not very satisfactory. This is because PathSim tends to treat authors with comparable connections to a common intermediate node as similar. This is because PathSim fails to incorporate content information in the graph and its computational principles. PathSim tends to equate similarity with the consistency of path quantities between two nodes and intermediate nodes. While theoretically sound, this often leads to errors in practical applications.

% For example, authors who share a similar number of paper publications at the same conferences are identified as similar. While this conceptualization is theoretically sound, PathSim tends to consider authors who share the same number of papers presented at the same conference as identical, assigning them a maximum similarity value of 1. Similarly, if two authors collaborate across all their published works, PathSim considers them completely similar. However, instances where authors collaborate on all their works are infrequent, causing the similarity of collaboratively engaged authors to be overshadowed by those who share conference participation. Consequently, authors with collaborative ties may be ranked lower than those who share conference participation in PathSim's link prediction outcomes, resulting in less than optimal results.

% HetFS outperforms all baselines, particularly excelling in citation link prediction.

\begin{table*}[h]
\centering
% \fontsize{12}{12}\selectfont
\caption{\centering Experimental results (\%) on ``DBLP'' and ``LastFM'' datasets for the link prediction tasks. Vacant positions (``-") are due to a lack of similarity based on that method.}\label{linkprediction}
% \scriptsize
\footnotesize
\resizebox{\textwidth}{!}{
\begin{tabular}{ccccccccccc}
\toprule
          &       & \multicolumn{1}{l}{PathSim} & \multicolumn{1}{l}{DeepWalk} & \multicolumn{1}{l}{RGCN} & \multicolumn{1}{l}{MAGNN} & \multicolumn{1}{l}{HetGNN} & \multicolumn{1}{l}{MHGNN} & \multicolumn{1}{l}{HGT} & \multicolumn{1}{l}{HetFS} \\
    \midrule
    \multirow{2}[0]{*}{DBLP} & AUC   & 54.15  & 71.67  & 75.02 & -  & 76.60  & 75.64 & 75.02 & \textbf{76.91} \\
          & MRR   & 58.9  & 73.84  & \textbf{79.76} & -  & 78.48  & 78.13 & 77.94 & 78.63 \\
    \midrule
    \multirow{2}[0]{*}{LastFM} & AUC   & 43.56  & 50.49  & 55.87 & 54.66 & \textbf{59.04} & 58.43 & 54.44 & 58.29 \\
          & MRR   & 58.06  & 65.53  & 72.08 & 68.12 & 77.56 & 75.5  & 73.18 & \textbf{79.37} \\
% AUC (type-1)   & 0.541  & 0.736  & \textbf{0.760}  & 0.742  \\
% F1 (type-1)    & 0.177  & 0.650  & \textbf{0.714}  & 0.681  \\
% AUC (type-2)   & 0.600  & 0.778  & 0.785  & \textbf{0.812}  \\
% F1 (type-2)    & 0.339  & 0.741  & 0.768  & \textbf{0.810}  \\
\bottomrule

\end{tabular}
}
\end{table*}


% \subsubsection{Personal Recommendation}

% In the recommendation task, we evaluate the performance of venue recommendation (author-venue link) in the DBLP data. We follow the same data splitting setting as in the link prediction task. Specifically, we use the training data to extract information from the graph. The ground truth for recommendation is based on the author's presence (having papers) in venues during testing. We believe that authors with similar interests are likely to attend similar venues. Therefore, for each target author, we identify the top-k most similar authors in a meta-path-free manner. The similarity of each author serves as a coefficient multiplied by their respective attendance at the most frequented venue, generating a venue preference score. We then accumulate the venue preference scores for each venue, and the top-ranked venue is recommended to the target author. We employ Recall (Rec), Precision (Pre), and F1 scores in the top-k recommendation list as evaluation metrics. Additionally, only new links among author-venue in the test data are considered and duplicated author-venue pairs are eliminated from the evaluation. 

% The reported scores represent the average values across all evaluated authors. PathSim lacks node recommendation results, as it only supports similarity computation for nodes of the same type. Consequently, it cannot handle the preference of authors for venues.

% The results of various models are presented in Table 8, with the optimal outcomes highlighted in bold. Notably, among the baselines, graph neural network methods exhibit superior performance. Additionally, HetFS performs impressively well, affirming its effectiveness in leveraging graph information for node recommendation.

% \begin{table*}[h]
% \centering
% % \fontsize{12}{12}\selectfont
% \caption{\centering Node recommendation results. Vacant positions (``-") are due to a lack of similarity based on that method.}\label{statisticOfDatasets}
% % \scriptsize
% \footnotesize
% \resizebox{0.7\textwidth}{!}{
% \begin{tabular}{cccccm{5cm}}
% \toprule
% DBLP & \textbf{PathSim} & \textbf{DeepWalk} & \textbf{HetGNN} & \textbf{HetFS} \\
% \midrule
% Recall & -     & 0.637  & \textbf{0.688}  & \textbf{0.688}  \\
% Precision & -     & 0.278  & \textbf{0.302}  & 0.249  \\
% F1    & -     & 0.387  & \textbf{0.419}  & 0.366  \\
% \bottomrule
% \end{tabular}
% }
% \end{table*}

\subsubsection{Node Clustering and Classification}

For node clustering and classification tasks, we follow the methodology outlined in HetGNN \cite{zhang2019heterogeneous}. We categorize authors in the ``DBLP" dataset into four distinct research domains: DM, CV, NLP, and DB. The top three venues for each domain are selected, and authors are labeled based on the predominant publication venues within these domains. Authors lacking publications in these venues are omitted from the evaluation. Node information is derived from the complete dataset.

To cluster node, we assumed that authors sharing similar research interests were likely to be active in similar domains. Consequently, we classified authors based on the conference categories where they exhibited the highest participation. More precisely, we record the most frequently attended conference for each author and multiply this information by the similarity value between the target author and others, which is calculated as the HetFS score in a meta-path-free manner. Then, the label preference scores were accumulated for each label, with the top-ranked label used for clustering. This method was applied to derive clustering results for HetFS and PathSim. For graph embedding methods and HGNNs, we utilized the learned node embeddings as input for a k-means clustering algorithm. Clustering performance was assessed using metrics such as normalized mutual information (NMI) and adjusted Rand index (ARI).

A methodology akin to node clustering was applied for the multi-label classification task to derive node classification outcomes for PathSim and HetFS. For graph embedding methods and HGNNs, the acquired node embeddings served as input for a logistic regression classifier. Notably, the train/test ratio was set at 1:9. Evaluation metrics encompass both Micro-F1 and Macro-F1.

Table~\ref{clustering&classification} presents the results of all methods, with the best outcomes highlighted in bold. The observations are as follows: (1) Most models exhibit excellent performance in multi-label classification, achieving high Macro-F1 and Micro-F1 scores (over 0.88 in the ``DBLP'' dataset and 0.55 in the ``IMDB'' dataset). This is reasonable given the distinct nature of authors across the four selected domains. (2) Graph embedding methods and HGNNs yield the best classification results, while HetFS also performs well. This indicates that HetFS effectively learns graph information for the node clustering task. (3) HetFS outperforms PathSim in both node clustering and node classification tasks, demonstrating that leveraging content information enhances embedding performance.

% The superior classification results of PathSim can be attributed to its consideration of authors who co-participate in conferences as highly similar. Since we also classify authors based on their most frequently attended conferences in the node clustering task, PathSim aligns well with this clustering objective.

\begin{table*}[h]
\centering
% \fontsize{12}{12}\selectfont
\caption{\centering Experimental results (\%) on ``DBLP'' and ``IMDB'' datasets for the node clustering and classification task.}\label{clustering&classification}
% \scriptsize
\footnotesize
\resizebox{\textwidth}{!}{
\begin{tabular}{ccccccccccc}
\toprule
          &       & \multicolumn{1}{l}{PathSim} & \multicolumn{1}{l}{DeepWalk} & \multicolumn{1}{l}{RGCN} & \multicolumn{1}{l}{HAN} & \multicolumn{1}{l}{MAGNN} & \multicolumn{1}{l}{HetGNN} & \multicolumn{1}{l}{MHGNN} & \multicolumn{1}{l}{HGT} & \multicolumn{1}{l}{HetFS} \\
    \midrule
    \multirow{4}[0]{*}{DBLP} & NMI   & 83.21  & 88.6  & 87.55 & 90.33 & 90.21 & 89.6  & 88.07 & \textbf{91.22} & 90.77 \\
          & ARI   & 85.29  & 91.34  & 89.49 & 90.89 & 91.3  & \textbf{93.1}  & 91.26 & 92.13 & 91.46 \\
          & macro-f1  & 86.18  & 90.45  & 89.88 & 91.67 & \textbf{92.16} & 91.74 & 91.84 & 92.1  & 92.3 \\
          & micro-f1  & 86.64  & 91.58  & 90.49 & 92.05 & 92.24 & 92.53 & \textbf{92.97} & 92.78 & 92.54 \\
    \midrule
    \multirow{4}[0]{*}{IMDB} & NMI    & 53.23  & 54.09  & 57.31 & 57.04 & 56.32 & 56.19 & 56.44 & \textbf{57.89} & 57.12 \\
          & ARI    & 55.06  & 55.34  & 59.2  & 62.77 & \textbf{63.07} & 62.96 & 61.86 & 60.43 & 61.88 \\
          & macro-f1  & 55.21  & 56.77  & 58.85 & 57.74 & 56.49 & 58.28 & 58.01 & 59.12 & \textbf{59.78} \\
          & micro-f1  & 57.76  & 58.4  & 62.05 & 64.63 & 64.67 & 64.3  & 63.94 & \textbf{65.2}  & 63.36 \\
% DBLP & \textbf{PathSim} & \textbf{DeepWalk} & \textbf{HetGNN} & \textbf{HetFS} \\
% \midrule
% NMI   & 0.832  & \textbf{0.886}  & \textbf{0.886}  & 0.820  \\
% ARI   & 0.882  & 0.919  &\textbf{ 0.921}  & 0.868  \\
% \midrule
% MacroF1 & 0.947  & \textbf{0.976}  & 0.974  & 0.945  \\
% MicroF1 & 0.951  & \textbf{0.977}  & 0.975  & 0.947  \\
\bottomrule
\end{tabular}
}
\end{table*}


\subsection{Ablation Study}

HetFS is an integrated similarity search method designed to aggregate information at different levels. How does node information impact the total performance? Does semantic information optimize information processing? Is content information effective for enhancing the performance of the path information-based method? We conducted ablation studies to evaluate the performance of various components to answer the above questions (RQ3), including (1) - node, removing node centrality from the method and treating all nodes as identical; (2) - semantics, excluding edge contributions from the method, and treating all edge weights as identical; (3) - content, removing content information from the method and relying solely on path information in the HIN for similarity search. The results of node clustering and node classification on the IMDB dataset are shown in Fig.~\ref{fig:ablation_study}.

\begin{figure}[h]
    \centering
        \includegraphics[width=\textwidth]{ablation_study.eps}
    \caption{Performances without various components on ``IMDB" datasets.} \label{fig:ablation_study}
\end{figure}

According to Fig.~\ref{fig:ablation_study}, we can observe that: (1) HetFS performs better than all partial methods with different components removed, suggesting that node information, semantics information, and content information all contribute to the aggregation of heterogeneous information to varying extents. (2) Among all the different components, the removal of semantics information has the most pronounced impact, indicating that content provides more effective information to the method compared to other components.

% \subsection{Hyper-parameter Sensitivity}

% In this section, we employ a pooling approach to assess the effectiveness of meta-path-free cases. The pooling algorithm, commonly employed in the domain of information retrieval, serves as an evaluation method to gauge the quality of document retrieval results. Typically, this method is applied to evaluate the top-K document ranking, where the system returns the K most pertinent documents relying on a user query. In the pooling algorithm, evaluators annotate each document in the retrieval results, with the annotation result being either binary (relevant/irrelevant) or a rating (e.g., on a scale of 1-5). The annotations from multiple evaluators are then aggregated to calculate the average score for each document. This allows for the ranking of the documents, with those ranked higher being considered more relevant by more evaluators.

% The underlying concept of pooling is as follows. Let's consider evaluating $l$ similarity measures $A_1, A_2, \cdot, A_l$, where each measure retrieves k results most similar to the given query. To begin, the top-k results are merged into a pool, eliminating any duplicates. Next, we present the pool of results to domain experts for evaluation. Taking into account the experts’ feedback, we select out top-k results, which will be used as the reference to access the top-k outcomes returned by $A_1, A_2, \cdot, A_l$. Specifically, we handpicked 14 widely recognized movies and presented 126 related movies as a reference. After crafting a survey, we invited 400 experts to participate and obtained 389 responses. Following a rigorous screening process, we identified 230 high-caliber responses to serve as the benchmark for statistical analysis. Next, we calculate the frequency of each movie and rank them. The frequency is considered as the true similarity score, and the order is ranking. We eventually reserve 13 movies to form the input query set, and the selected frequency and the ranking of 87 movies related to these 13 movies are regarded as the ground truth.

% With the ground truth in hand, we carry out a series of experiments on IMDB to assess the performance of all the methods using the two metrics.

% A similar property that HetFS performs outstandingly is shown in Fig.12(b), Fig. 12(c), and Fig.12(d), among which the last two graphs analyze the results on global datasets. Note that wPPR even briefly outperforms HetFS in top-1 and top-2 results on IMDB-C and IMDB-D, but the mAP after convergence is still lower than HetFS. The difference between local datasets and global datasets suggests that increasing the dataset size simply has little impact on mAP.

 % There is a detail that even though the initial mAP and the converged mAP of SimRank-2 and SimRank-3 are the same, the top-2 and top-3 mAPs of SimRank-2 are higher than those of SimRank-3. This suggests that more iteration does not necessarily lead to more similar results. In other words, the similarity between two items may exist in a relatively local relationship without considering too much complexity.

% \begin{figure}[h]
%     \centering
%         \includegraphics[width=\textwidth]{images/mapKMethodDataSetOnUserStudy14.png}
%     \caption{The mAP with varying K on IMDB} \label{fig:mapKMethodDataSetOnUserStudy14}
% \end{figure}

All the experiments above demonstrated that a combination of path information and content information was effective and efficient in mining graph data. Ad-hoc query experiments revealed that the results obtained by searching with different meta-paths focused on different content aspects. HetFS not only delved deeply into graph data to provide meaningful results but also swiftly adapted to user-given meta-paths for answering ad-hoc queries. Downstream applications further demonstrated that, while ensuring rapid responses in on-the-fly queries, HetFS delivered results comparable to state-of-the-art methods.

% \section{Cross referencing}\label{sec8}

% Environments such as figure, table, equation and align can have a label
% declared via the \verb+\label{#label}+ command. For figures and table
% environments use the \verb+\label{}+ command inside or just
% below the \verb+\caption{}+ command. You can then use the
% \verb+\ref{#label}+ command to cross-reference them. As an example, consider
% the label declared for Figure~\ref{fig1} which is
% \verb+\label{fig1}+. To cross-reference it, use the command 
% \verb+Figure \ref{fig1}+, for which it comes up as
% ``Figure~\ref{fig1}''. 

% To reference line numbers in an algorithm, consider the label declared for the line number 2 of Algorithm~\ref{algo1} is \verb+\label{algln2}+. To cross-reference it, use the command \verb+\ref{algln2}+ for which it comes up as line~\ref{algln2} of Algorithm~\ref{algo1}.

% \subsection{Details on reference citations}\label{subsec7}

% Standard \LaTeX\ permits only numerical citations. To support both numerical and author-year citations this template uses \verb+natbib+ \LaTeX\ package. For style guidance please refer to the template user manual.

% Here is an example for \verb+\cite{...}+: \cite{bib1}. Another example for \verb+\citep{...}+: \citep{bib2}. For author-year citation mode, \verb+\cite{...}+ prints Jones et al. (1990) and \verb+\citep{...}+ prints (Jones et al., 1990).

% All cited bib entries are printed at the end of this article: \cite{bib3}, \cite{bib4}, \cite{bib5}, \cite{bib6}, \cite{bib7}, \cite{bib8}, \cite{bib9}, \cite{bib10}, \cite{bib11}, \cite{bib12} and \cite{bib13}.

% \section{Examples for theorem like environments}\label{sec10}

% For theorem like environments, we require \verb+amsthm+ package. There are three types of predefined theorem styles exists---\verb+thmstyleone+, \verb+thmstyletwo+ and \verb+thmstylethree+ 

% %%=============================================%%
% %% For presentation purpose, we have included  %%
% %% \bigskip command. please ignore this.       %%
% %%=============================================%%
% \bigskip
% \begin{tabular}{|l|p{19pc}|}
% \hline
% \verb+thmstyleone+ & Numbered, theorem head in bold font and theorem text in italic style \\\hline
% \verb+thmstyletwo+ & Numbered, theorem head in roman font and theorem text in italic style \\\hline
% \verb+thmstylethree+ & Numbered, theorem head in bold font and theorem text in roman style \\\hline
% \end{tabular}
% \bigskip
% %%=============================================%%
% %% For presentation purpose, we have included  %%
% %% \bigskip command. please ignore this.       %%
% %%=============================================%%

% For mathematics journals, theorem styles can be included as shown in the following examples:

% \begin{theorem}[Theorem subhead]\label{thm1}
% Example theorem text. Example theorem text. Example theorem text. Example theorem text. Example theorem text. 
% Example theorem text. Example theorem text. Example theorem text. Example theorem text. Example theorem text. 
% Example theorem text. 
% \end{theorem}

% Sample body text. Sample body text. Sample body text. Sample body text. Sample body text. Sample body text. Sample body text. Sample body text.

% \begin{proposition}
% Example proposition text. Example proposition text. Example proposition text. Example proposition text. Example proposition text. 
% Example proposition text. Example proposition text. Example proposition text. Example proposition text. Example proposition text. 
% \end{proposition}

% Sample body text. Sample body text. Sample body text. Sample body text. Sample body text. Sample body text. Sample body text. Sample body text.

% \begin{example}
% Phasellus adipiscing semper elit. Proin fermentum massa
% ac quam. Sed diam turpis, molestie vitae, placerat a, molestie nec, leo. Maecenas lacinia. Nam ipsum ligula, eleifend
% at, accumsan nec, suscipit a, ipsum. Morbi blandit ligula feugiat magna. Nunc eleifend consequat lorem. 
% \end{example}

% Sample body text. Sample body text. Sample body text. Sample body text. Sample body text. Sample body text. Sample body text. Sample body text.

% \begin{remark}
% Phasellus adipiscing semper elit. Proin fermentum massa
% ac quam. Sed diam turpis, molestie vitae, placerat a, molestie nec, leo. Maecenas lacinia. Nam ipsum ligula, eleifend
% at, accumsan nec, suscipit a, ipsum. Morbi blandit ligula feugiat magna. Nunc eleifend consequat lorem. 
% \end{remark}

% Sample body text. Sample body text. Sample body text. Sample body text. Sample body text. Sample body text. Sample body text. Sample body text.

% \begin{definition}[Definition sub head]
% Example definition text. Example definition text. Example definition text. Example definition text. Example definition text. Example definition text. Example definition text. Example definition text. 
% \end{definition}

% Additionally a predefined ``proof'' environment is available: \verb+\begin{proof}+ \verb+...+ \verb+\end{proof}+. This prints a ``Proof'' head in italic font style and the ``body text'' in roman font style with an open square at the end of each proof environment. 

% \begin{proof}
% Example for proof text. Example for proof text. Example for proof text. Example for proof text. Example for proof text. Example for proof text. Example for proof text. Example for proof text. Example for proof text. Example for proof text. 
% \end{proof}

% Sample body text. Sample body text. Sample body text. Sample body text. Sample body text. Sample body text. Sample body text. Sample body text.

% \begin{proof}[Proof of Theorem~{\upshape\ref{thm1}}]
% Example for proof text. Example for proof text. Example for proof text. Example for proof text. Example for proof text. Example for proof text. Example for proof text. Example for proof text. Example for proof text. Example for proof text. 
% \end{proof}

% \noindent
% For a quote environment, use \verb+\begin{quote}...\end{quote}+
% \begin{quote}
% Quoted text example. Aliquam porttitor quam a lacus. Praesent vel arcu ut tortor cursus volutpat. In vitae pede quis diam bibendum placerat. Fusce elementum
% convallis neque. Sed dolor orci, scelerisque ac, dapibus nec, ultricies ut, mi. Duis nec dui quis leo sagittis commodo.
% \end{quote}

% Sample body text. Sample body text. Sample body text. Sample body text. Sample body text (refer Figure~\ref{fig1}). Sample body text. Sample body text. Sample body text (refer Table~\ref{tab3}). 

% \section{Methods}\label{sec11}

% Topical subheadings are allowed. Authors must ensure that their Methods section includes adequate experimental and characterization data necessary for others in the field to reproduce their work. Authors are encouraged to include RIIDs where appropriate. 

% \textbf{Ethical approval declarations} (only required where applicable) Any article reporting experiment/s carried out on (i)~live vertebrate (or higher invertebrates), (ii)~humans or (iii)~human samples must include an unambiguous statement within the methods section that meets the following requirements: 

% \begin{enumerate}[1.]
% \item Approval: a statement which confirms that all experimental protocols were approved by a named institutional and/or licensing committee. Please identify the approving body in the methods section

% \item Accordance: a statement explicitly saying that the methods were carried out in accordance with the relevant guidelines and regulations

% \item Informed consent (for experiments involving humans or human tissue samples): include a statement confirming that informed consent was obtained from all participants and/or their legal guardian/s
% \end{enumerate}

% If your manuscript includes potentially identifying patient/participant information, or if it describes human transplantation research, or if it reports results of a clinical trial then  additional information will be required. Please visit (\url{https://www.nature.com/nature-research/editorial-policies}) for Nature Portfolio journals, (\url{https://www.springer.com/gp/authors-editors/journal-author/journal-author-helpdesk/publishing-ethics/14214}) for Springer Nature journals, or (\url{https://www.biomedcentral.com/getpublished/editorial-policies\#ethics+and+consent}) for BMC.

% \section{Discussion}\label{sec12}

% Discussions should be brief and focused. In some disciplines use of Discussion or `Conclusion' is interchangeable. It is not mandatory to use both. Some journals prefer a section `Results and Discussion' followed by a section `Conclusion'. Please refer to Journal-level guidance for any specific requirements. 

\section{Conclusion}\label{sec13}

This paper presented HetFS, an efficient and effective similarity measure designed to address four limitations in existing solutions for HINs: (1) high computational overhead on ad-hoc queries with user-given meta-paths, (2) disregarding node centrality, (3) omission of edge contribution, and (4) neglecting node content features. Specifically, HetFS leveraged content, node, edge, and structural information to aggregate data from HINs. Content information, integrated into path information later, underwent transformation via type-specific functions to unify it into a latent space. Path information incorporated node centrality and edge contribution with structural information, utilizing an iterative algorithm on heterogeneous graphs. The culmination of all these elements formed the ultimate similarity method. Experiments indicated that HetFS effectively handled ad-hoc queries with swift response times and comprehensive search outcomes, delivering commendable results compared to state-of-the-art methods in tasks such as link prediction, node clustering, and node classification. Future work involves adapting this heterogeneous graph mining framework to mine complex relation similarity from HINs.

\bmhead{Acknowledgments}

This work was mainly supported by the National Natural Science Foundation of China (NSFC No. 61732004).

\bmhead{Funding}

This work was mainly supported by the National Natural Science Foundation of China (NSFC No. 61732004).


% \backmatter

% \bmhead{Supplementary information}

% If your article has accompanying supplementary file/s please state so here. 

% Authors reporting data from electrophoretic gels and blots should supply the full unprocessed scans for key as part of their Supplementary information. This may be requested by the editorial team/s if it is missing.

% Please refer to Journal-level guidance for any specific requirements.

% \bmhead{Acknowledgments}

% Acknowledgments are not compulsory. Where included they should be brief. Grant or contribution numbers may be acknowledged.

% Please refer to Journal-level guidance for any specific requirements.

\section*{Statements and Declarations}

\begin{itemize}
\item This work was mainly supported by the National Natural Science Foundation of China (NSFC No. 61732004).
\item The authors declared no potential conflicts of interest with respect to the research, authorship, and/or publication of this article.
\item Ethics approval not applicable.
\item Some data used in this study were obtained from the publicly accessible websites described in Section 5. Other data are available from the corresponding author upon reasonable request.
\item Code is available from the corresponding author upon reasonable request.
\item Xuqi Mao: conceptualization of this study, methodology, software. Zhenyi Chen: conceptualization, methodology, software. Zhenying He: supervision. X. Sean Wang:
conceptualization, methodology.
\end{itemize}

% \noindent
% If any of the sections are not relevant to your manuscript, please include the heading and write `Not applicable' for that section. 

% %%===================================================%%
% %% For presentation purpose, we have included        %%
% %% \bigskip command. please ignore this.             %%
% %%===================================================%%
% \bigskip
% \begin{flushleft}%
% Editorial Policies for:

% \bigskip\noindent
% Springer journals and proceedings: \url{https://www.springer.com/gp/editorial-policies}

% \bigskip\noindent
% Nature Portfolio journals: \url{https://www.nature.com/nature-research/editorial-policies}

% \bigskip\noindent
% \textit{Scientific Reports}: \url{https://www.nature.com/srep/journal-policies/editorial-policies}

% \bigskip\noindent
% BMC journals: \url{https://www.biomedcentral.com/getpublished/editorial-policies}
% \end{flushleft}

% \begin{appendices}

% \section{Section title of first appendix}\label{secA1}

% An appendix contains supplementary information that is not an essential part of the text itself but which may be helpful in providing a more comprehensive understanding of the research problem or it is information that is too cumbersome to be included in the body of the paper.

% %%=============================================%%
% %% For submissions to Nature Portfolio Journals %%
% %% please use the heading ``Extended Data''.   %%
% %%=============================================%%

% %%=============================================================%%
% %% Sample for another appendix section			       %%
% %%=============================================================%%

% %% \section{Example of another appendix section}\label{secA2}%
% %% Appendices may be used for helpful, supporting or essential material that would otherwise 
% %% clutter, break up or be distracting to the text. Appendices can consist of sections, figures, 
% %% tables and equations etc.

% \end{appendices}

%%===========================================================================================%%
%% If you are submitting to one of the Nature Portfolio journals, using the eJP submission   %%
%% system, please include the references within the manuscript file itself. You may do this  %%
%% by copying the reference list from your .bbl file, paste it into the main manuscript .tex %%
%% file, and delete the associated \verb+\bibliography+ commands.                            %%
%%===========================================================================================%%

\bibliography{ref}% common bib file
%% if required, the content of .bbl file can be included here once bbl is generated
%%\input sn-article.bbl
% \bibliography{sn-bibliography}


\end{document}


% \begin{table*}[h]
% % \fontsize{12}{12}\selectfont \cellcolor{gray!40}
% \caption{\centering Case study on query ``Terminator 2" under different meta-paths on ``IMDB" dataset.}\label{statisticOfDatasets}
% % \scriptsize
% \footnotesize
% \renewcommand\arraystretch{1.2}
% \resizebox{\textwidth}{!}{
% \begin{tabular}{|c|lll|lll|m{2cm}}
% \multicolumn{1}{c}{\multirow{1}[0]{*}{}} & \multicolumn{3}{c}{\textbf{(a) PathSim}} & \multicolumn{3}{c}{\textbf{(b) DeepWalk}} \\
% \hline
% \textbf{Rank} & MAM & MDM & meta-path-free & MAM & MDM & meta-path-free \\
% \hline
% 1 & True Lies & True Lies & The Abyss & Hercules in New York & True Lies & True Lies \\
% 2 & Eraser & The Abyss & Aliens & Pumping Iron & The Abyss & Ghosts of the Abyss \\
% 3 & Jingle All the Way & Aliens & Titanic & Detroit Rock City & Aliens & Aliens of the Deep \\
% 4 & Conan the Barbarian & Titanic & Ghosts of the Abyss & King Kong Lives & Titanic & Detroit Rock City \\
% 5 & Total Recall & Ghosts of the Abyss & Aliens of the Deep & The Kid \& I & Aliens of the Deep & Brainscan \\
% 6 & End of Days & Aliens of the Deep & True Lies & Kindergarten Cop & Ghosts of the Abyss & King Kong Lives \\
% 7 & The Running Man & - & Terminator 3 & Brainscan & My Name Is Bruce & Children of the Corn \\
% 8 & Red Heat & - & Terminator Salvation & Children of the Corn & On\_Line & Hercules in New York \\
% % 9 & Twins & - & Eraser & Jingle All the Way & Passion of Mind & Pumping Iron \\
% % 10 & Collateral Damage & - & Jingle All the Way & American Heart & Ma vie en rose & The Kid \& I \\
% \hline
% \multicolumn{1}{c}{\multirow{1}[0]{*}{}} & \multicolumn{3}{c}{\textbf{(c) HetGNN}} & \multicolumn{3}{c}{\textbf{(d) HetFS}} \\
% \hline
% \textbf{Rank} & MAM & MDM & meta-path-free & MAM & MDM & meta-path-free \\
% \hline
% 1 & Jingle All the Way & Ghosts of the Abyss & True Lies & Hercules in New York & True Lies & Terminator 3 \\
% 2 & End of Days & True Lies & Aliens of the Deep & Pumping Iron & Titanic & Terminator Salvation \\
% 3 & Total Recall & The Abyss & Pumping Iron & Collateral Damage & The Abyss & True Lies \\
% 4 & American History X & Aliens of the Deep & Ghosts of the Abyss & Jingle All the Way & Aliens & Titanic \\
% 5 & Terminator 3 & Titanic & Hercules in New York & Kindergarten Cop & Ghosts of the Abyss & The Abyss \\
% 6 & Pumping Iron & Terminator 3 & Aliens & The Running Man & Aliens of the Deep & Aliens \\
% 7 & Kindergarten Cop & American History X & Red Heat & Red Heat & - & Hercules in New York \\
% 8 & Hercules in New York & Pumping Iron & The Abyss & Terminator 3 & - & Pumping Iron \\
% % 9 & Twins & Kindergarten Cop & Kindergarten Cop & Eraser & - & Collateral Damage \\
% % 10 & True Lies & Hercules in New York & Collateral Damage & King Kong Lives & - & Jingle All the Way \\
% \hline
% \end{tabular}
% }
% \end{table*}


% Table generated by Excel2LaTeX from sheet 'merge'
% \begin{table}[htbp]
% \centering
% \caption{Add caption}
% \begin{tabular}{rllll}
% \multicolumn{1}{c}{\multirow{2}[0]{*}{Rank}} & \multicolumn{2}{c}{HetGNN} & \multicolumn{2}{c}{HetFS} \\
%       & MAM   & MDM   & MAM   & MDM \\
% 1     & Jingle All the Way & Ghosts of the Abyss & Hercules in New York & True Lies \\
% 2     & End of Days & True Lies & Pumping Iron & Titanic \\
% 3     & Total Recall & The Abyss & Collateral Damage & The Abyss \\
% 4     & American History X & Aliens of the Deep & Jingle All the Way & Aliens \\
% 5     & Terminator 3: Rise of the Machines & Titanic & Kindergarten Cop & Ghosts of the Abyss \\
% 6     & Pumping Iron & Terminator 3: Rise of the Machines & The Running Man & Aliens of the Deep \\
% 7     & Kindergarten Cop & American History X & Red Heat & - \\
% 8     & Hercules in New York & Pumping Iron & Terminator 3: Rise of the Machines & - \\
% 9     & Twins & Kindergarten Cop & Eraser & - \\
% 10    & True Lies & Hercules in New York & King Kong Lives & - \\
% \end{tabular}%
% \label{tab:addlabel}%
% \end{table}%


% \begin{table*}[htbp]
% \centering
% \caption{\centering Case study on query ``Terminator 2" under different meta-paths on ``IMDB" dataset using HetFS method. Vacant positions (``-") indicate no additional similar results found in the database for that meta-path.}
% \resizebox{\textwidth}{!}{
% \begin{tabular}{lllllr}
% \toprule
% Rank & MAM   & MDM   & MGM   & meta-path-free &  \\
% \midrule
% 1 & Hercules in New York & True Lies & Space Cowboys & Terminator 3 & \\
% 2 & Pumping Iron & Titanic & Superman & Terminator Salvation & \\
% 3 & Collateral Damage & The Abyss & Breakfast of Champions & True Lies & \\
% 4 & Jingle All the Way & Aliens & Ghost Busters & Titanic & \\
% 5 & Kindergarten Cop & Ghosts of the Abyss & Ghostbusters II & The Abyss & \\
% 6 & The Running Man & Aliens of the Deep & Sleeper & Aliens & \\
% 7 & Red Heat & - & Meteor & Hercules in New York & \\
% 8 & Terminator 3 & - & Mars Attacks! & Pumping Iron & \\
% 9 & Eraser & - & The Handmaid's Tale & Collateral Damage & \\
% 10 & King Kong Lives & - & Sneakers & Jingle All the Way & \\
% \bottomrule
% \end{tabular}%
% }
% \label{tab:addlabel}%
% \end{table*}%


% \begin{table*}[htbp]
% \centering
% \caption{\centering Case study on query ``Jiawei Han" on ``DBLP" dataset using different methods (meta-path-free).}
% \resizebox{0.9\textwidth}{!}{
% \begin{tabular}{rllll}
% \toprule
% \multicolumn{1}{l}{Rank} & PathSim & DeepWalk & HetGNN & HetFS \\
% \midrule
% 1 & Philip S. Yu & Yizhou Sun & Philip Yu & Yizhou Sun \\
% 2 & Christos Faloutsos & Xiao Yu & Chen Chen & Quanquan Gu \\
% 3 & Divesh Srivastava & Marina Danilevsky & Yizhou Sun & Scott W. McQuiggan \\
% 4 & Xifeng Yan & Bo Zhao & Xifeng Yan & Xiao Yu \\
% 5 & Gerhard Weikum & Yintao Yu & Bolin Ding & Elke A. Rundensteiner \\
% 6 & Yizhou Sun & Tim Weninger & Xiao Yu & Philip S. Yu \\
% 7 & Aristides Gionis & Zhijun Yin & Peixiang Zhao & Karen Works \\
% 8 & Wei Wang & Fangbo Tao & Spiros Papadimitriou & Jian Pei \\
% 9 & Francesco Bonchi & Jialu Liu & Sangkyum Kim & Xiaofei He \\
% 10 & Tao Li & Xiaolei Li & Hong Cheng & Mingli Wu \\
% \bottomrule
% \end{tabular}%
% }
% \label{tab:addlabel}%
% \end{table*}%



% Table generated by Excel2LaTeX from sheet 'merge'
% \begin{table*}[htbp]
% \centering
% \caption{Add caption}
% \resizebox{\textwidth}{!}{
% \begin{tabular}{|c|ll|ll|}
% \multicolumn{1}{c}{\multirow{1}[0]{*}{}} & \multicolumn{2}{c}{(a) PathSim} & \multicolumn{2}{c}{(b) DeepWalk} \\
% \hline
% Rank  & APA   & AVA   & APA   & AVA \\
% \hline
% 1     &  Yizhou Sun & Philip S. Yu & Xiaolei Li & Haixun Wang \\
% 2     &  Xifeng Yan & Christos Faloutsos & Tim Weninger & Michalis Vazirgiannis \\
% 3     &  Chi Wang & Hui Xiong & Brandon Norick & Kevin Chen-Chuan \\
% 4     &  Xiao Yu & Wei Wang & Yizhou Sun & Min Wang \\
% 5     &  Jing Gao & Tao Li & George Brova & C. Lee Giles \\
% 6     &  Bo Zhao & Jian Pei & Marina Danilevsky & Francesco Bonchi \\
% 7     &  Dong Xin & Jie Tang & Yintao Yu & Jianyong Wang \\
% 8     &  Bolin Ding & Haixun Wang & Bo Zhao & Enhong Chen \\
% 9     &  Deng Cai & Lei Chen & Tianyi Wu & HweeHwa Pang \\
% 10    &  Xiaofei He & Wei Fan & Roland Kays & Aixin Sun \\
% \hline
% \multicolumn{1}{c}{\multirow{1}[0]{*}{}} & \multicolumn{2}{c}{(c) HetGNN} & \multicolumn{2}{c}{(d) HetFS} \\
% \hline
% Rank  & APA   & AVA   & APA   & AVA \\
% \hline
% 1     &  Jimeng Sun & Rajeev Rastogi & Quanquan Gu & Philip S. Yu \\
% 2     &  Brandon Norick & Tamraparni Dasu & Yizhou Sun & Christos Faloutsos \\
% 3     &  Guojie Song & Guiquan Liu & Dong Xin & Divesh Srivastava \\
% 4     &  Zeng Lian & Dashun Wang & Deng Cai & Wei Wang \\
% 5     &  Marina Danilevsky & Xiao Yu & Chi Wang & Yongxin Tong \\
% 6     &  Yizhou Sun & Ming Li & Tim Weninger & Lei Chen \\
% 7     &  Christos Boutsidis & Neal E. Young & Philip S. Yu & Haixun Wang \\
% 8     &  Jinyan Li & Jian Xu & Xiaofei He & Tao Li \\
% 9     &  Yang Cao & Robson Leonardo & Xiao Yu & Jian Pei \\
% 10    &  Yu-Ru Lin & Chengnian Sun & Jing Gao & Hui Xiong \\
% \hline
% \end{tabular}%
% }
% \label{tab:addlabel}%
% \end{table*}%