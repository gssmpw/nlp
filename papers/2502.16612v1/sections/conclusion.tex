\section{Conclusions and Future Work}
\label{sec:conclusions}
In this study, we introduce a \memex{} dataset for propagandistic and hateful meme detection and natural explanation generation, making it the \textit{first} resource of its kind. To address both detection and explanation generation tasks and ensure efficient VLMs model training on this dataset, we also propose a multi-stage optimization procedure.  
To evaluate the multilingual capability of the model, we developed Arabic and English explanations for Arabic memes. The inclusion of English explanations benefits non-Arabic speakers, whereas providing explanations in the native language ensures that cultural nuances are accurately conveyed. With our multi-stage training procedure, we demonstrate improved detection performance for both \textit{ArMeme} and hateful memes. The higher performance of explanation generation further demonstrates the efficacy of our multi-stage training approach.  
We foresee several future directions to extend this research and explore the following: \textit{(a)} training the model with additional data through data augmentation, which could help it become an instruction-generalized model and potentially enhance its performance further; \textit{(b)} incorporating pseudo and self-labeled data using an active learning procedure to incrementally improve the model's capabilities; and \textit{(c)} developing a task-generalized model that addresses multiple tasks.
% within the domain of visual sentiment analysis.

