Showing in Table~\ref{tab:command_list}, we design the following actions for Repo2Run to facilitate its invocation.

\begin{table}[htbp]
    \centering
    \caption{Command list and their functions}
    \begin{tabular}{|l|p{9cm}|}
        \hline
        \textbf{Command} & \textbf{Function} \\
        \hline
        \texttt{waitinglist add -p package\_name} & Add item into waiting list. If no ``version\_constraints'' are\\
        \texttt{[-v version\_constraints] -t tool} & specified, the latest version will be downloaded by default. \\
        \hline
        \texttt{waitinglist addfile file\_path} & Add all entries from a file similar to requirements.txt format to\\
        & the waiting list. Format should be package\_name [version\_constraints]. \\
        \hline
        \texttt{waitinglist clear} & Clear all items in the waiting list. \\
        \hline
        \texttt{conflictlist solve -v} & Resolve the conflict for the first element in the conflict list, and \\
        ``\texttt{[version\_constraints]}'' & update the version constraints for the corresponding package\_name and tool to version\_constraints. If no ``version\_constraints'' are specified, the latest version will be downloaded by default. The package\_name and tool in the original waiting list must match one of the elements in the conflictlist. Here, the version\_constraints are specified. \\
        \hline
        \texttt{conflictlist solve -u} & Keep the original version constraint that exists in the waiting list, and discard the other version constraints with the same name and tool in the conflict list. \\
        \hline
        \texttt{conflictlist clear} & Clear all items in the conflict list. \\
        \hline
        \texttt{conflictlist show} & Show all items in the conflict list. \\
        \hline
        \texttt{waitinglist show} & Show all items in the waiting list. \\
        \hline
        \texttt{download} & Download all pending items in the waiting list at once, and the conflict list must be empty before executing. \\
        \hline
        \texttt{runtest} & Check if the configured environment is correct using``\texttt{pytest}''. \\
        \hline
        \texttt{poetryruntest} & Check if the configured environment is correct in the poetry environment. If you want to run tests in the poetry environment, run it. \\
        \hline
        \texttt{runpipreqs} & Generate \texttt{requirements\_pipreqs.txt} and \texttt{pipreqs\_output.txt} and \texttt{pipreqs\_error.txt}.\\
        \hline
        \texttt{change\_python\_version} & Switching the Python version in the Docker container will forgo\\
        \texttt{python\_version} & any installations made prior to the switch. The Python version number should be represented directly with numbers and dots, without any quotation marks. \\
        \hline
        \texttt{clear\_configuration} & Reset all the configuration to the initial setting of \texttt{python:3.10}. \\
        \hline
    \end{tabular}
    \label{tab:command_list}
\end{table}