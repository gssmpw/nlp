

% The training data collected for the development of the Salamandra models aims to cover 35 European languages and a wide variety of domains. 
The training data collected for the development of the Salamandra models prioritizes the official Spanish languages, including Spanish, Catalan, Basque, Galician and Occitan, while also covering 30 additional European languages and a wide variety of domains. 
During the processing stage, we distinguish between curated and web data, which are processed by different pipelines but still undergo the same steps, including language identification, deduplication and heuristic filtering. 
The distinction between curated and web data is mainly due to the differences in scale and nature, requiring the use of different deduplication and filtering techniques. 
In this section, we outline the process of data selection, conversion, normalization, deduplication, quality filtering, and language sampling, mainly for the data used during pre-training.

\subsubsection{Data selection}
\label{subsubsec:data-selection}

%\todo{Why and how we select the sources?}

The training data is the pivotal point which will influence the performance in downstream applications and real-world use cases that will be built on top of the Salamandra models. When compiling the training data, the selection of data sources plays a crucial role in determining the distribution of words that will be learned by the model. Therefore, in order to select the sources that we want to include in the training corpus, we define the following set of requirements:

\begin{itemize}
     %\item \textbf{Linguistic relevance}: Datasets should include languages relevant to the research goals and provide a sufficient amount of data to support effective training for a given language. The content of these datasets should also be aligned with the topics relevant to the intended applications of the model.
     \item \textbf{Linguistic relevance}: Datasets should be relevant to the Spanish and European languages, and provide a sufficient amount of data to support effective training for any given language in the set. The content of these datasets should also be aligned with the topics relevant to the intended applications of the model.
     \item \textbf{Quality and integrity}: Datasets should be error-free, updated, and relevant to the desired time period, and consist of human-produced content to avoid redundancy and increase the diversity of the training data.
     \item \textbf{Availability of resources}: We consider the need for having sufficient hardware resources, such as disk space and computing power, to handle the preprocessing of datasets. Latest versions of the datasets are used to keep up-to-date knowledge and ensure the relevance of the training data.
\end{itemize}

%Based on these requirements, we compile a list of monolingual and multilingual sources from both heterogeneous sources such as Common Crawl and more specific repositories covering different domains and languages. The data sources are described in detail in the appendix \ref{app:datasheet}.

Based on these requirements, we compile a list of monolingual and multilingual sources from both heterogeneous sources such as Common Crawl and more specific repositories covering 35 languages and different domains, which ensures that the models are able to generalize across linguistic structures and domains.

Optimising the number of languages in multilingual model training to improve cross-lingual transfer is still an open area of research. Studies show that scaling the number of languages leads to better cross-lingual performance up to a point, after which increasing model and vocabulary capacity can help, but overall performance on monolingual and cross-lingual benchmarks tends to deteriorate, especially for high-resource languages, in what \citet{conneau_unsupervised_2020} calls the "curse of multilinguality". \citet{wang_negative_2020} reports "negative interference", i.e. performance degradation, for both high-resource and low-resource languages. More recently, \citet{chang_when_2023} reports that moderate amounts of multilingual data improve performance for low-resource languages, while it consistently degrades performance for high-resource languages.

%\footnote{\citet{conneau_unsupervised_2020} reports significant performance degradation on XNLI when scaling XLM-R 270M above 7 languages, and on XLM-R 550M when scaling above 15 languages. \citet{scao_what_2022} shows that a 1.3B parameter-sized model pre-trained on 13 languages also significantly underperforms an English model trained from the same data source in terms of zero-shot generalization. \citet{lin_few-shot_2022} conducts a similar experiment where XGLM 7.5B, shared among 30 languages, underperforms on few-shot tasks for high-resource languages against English-centric models.}

%While recent research has focused on minimising the negative interference of languages by changing the architecture or training technique with specialised models \cite{zhou_moe-lpr_2024, blevins_breaking_2024}, training a single "dense" embedding space remains the standard technique for multilingual model training \cite{avramidis_occiglot_2024, martins_eurollm_2024, ali_teuken-7b-base_2024}, and usually involves grouping together related languages, based on the hypothesis that languages with similar syntactic structure\footnote{By syntactic structure, we refer mainly to the word order between elements such as subject-object-verb, subject-verb and object-verb order, which is the most studied phenomenon in cross-lingual transfer learning studies.} are able to improve performance in low resource languages by using high resource languages \cite{chai_crosslingual_2022, philippy_towards_2023}\todo{I understand this, but it took me a couple of rereads. Maybe reword or add more detail for clarity}.

Recent research has focused on minimising the negative interference of languages by changing the architecture or training technique using specialised models \cite{zhou_moe-lpr_2024, blevins_breaking_2024}. However, training a single 'dense' embedding space remains the standard technique for multilingual model training \cite{avramidis_occiglot_2024, martins_eurollm_2024, ali_teuken-7b-base_2024}, where related languages are usually grouped together in the training data, based on the hypothesis that languages with similar syntactic structures\footnote{By syntactic structure, we refer mainly to the word order between elements such as subject-object-verb, subject-verb and object-verb order, which is the most studied phenomenon in cross-lingual transfer learning studies.} can be used to improve performance in low-resource languages \cite{chai_crosslingual_2022, philippy_towards_2023}. In particular, European languages provide a useful range of typological diversity for the Spanish languages, and are well represented in widely available datasets in the NLP community \cite{zhou_moe-lpr_2024, joshi_state_2021, blevins_breaking_2024}.

The inclusion of varied domains, such as legal, medical, technical, and conversational data, is crucial for training models that can perform effectively across tasks and applications \cite{hashimoto_model_2021, miranda_beyond_2023, xie_doremi_2023, fan_doge_2023}. 
We also include the Starcoder training data \cite{li_starcoder_2023}%\todo{This citation needs to go between parethesis I think? -> Yes, it's \cite, not \citet}
, since it has been proven that a portion of code in the training data is able to improve performance in downstream tasks \cite{muennighoff_scaling_2023, liang_holistic_2023, ma_at_2023}. 
The data sources are described in detail in Appendix \ref{app:datasheet}.

% \begin{figure}[!htbp]
%     \centering
%     \includegraphics[width=\textwidth]{figures/data/license_dist_barplot.pdf}
%     \caption{License distribution of the sources used in the Salamandra pre-training dataset. The graph highlights the counts of different license types across the datasets, categorized as "Permissive" or "Restrictive".}
%     \label{fig:license_dist}
% \end{figure}

%When collecting data from different sources, languages and domains, data provenance becomes a concern, especially within the NLP field, where inconsistencies in dataset documentation and licensing are very common \cite{datasheets, fries_dataset_2022}. 
%The licence distribution graph in Figure \ref{fig:license_dist} provides insight into the different licences that apply to the datasets used in the training corpus of the Salamandra models. A significant proportion of the datasets fall under permissive licences, such as CC BY and CC0, which allow broad reuse with minimal restrictions. Restrictive licences, including CC BY-SA and CC BY-NC, make up a smaller but significant proportion, reflecting restrictions on reuse.
%Notably, many datasets are categorised as ``unspecified'', indicating a lack of explicit licensing information about their sources. We inferred potential licensing terms and ensured the use of such data only within the processing pipelines described below in Section \ref{subsubsec:data-processing-pipelines}.

%\footnote{Datasets labeled as "Unspecified" are categorized under the "Restrictive" group because the absence of explicit licensing information creates legal ambiguity and limits their use beyond the processing pipeline.}
%{Without a clear license, there is no explicit permission to use, modify, or distribute the data, which defaults to "all rights reserved" under most copyright laws.}

%In such cases, additional research has been done to infer potential licensing terms.
%, although for some datasets the exact licences still remain unclear.

\subsubsection{Processing Pipelines}
\label{subsubsec:data-processing-pipelines}

%\todo{Distinction between web and "curated" data, requirements and characteristics for each type}

%\todo{How the requirements for each type of data fit within each pipeline and why is it necessary to appply different approaches}

One of the most prominent sources of data for the development of LLMs is web data, where Common Crawl (CC) stands as the largest repository of internet data, which is updated periodically with copies from internet webpages that are distributed as monthly data dumps. 
CC is often considered to be a representative snapshot of the web due to the size of the dumps, but in fact it is often incomplete in terms of the amount of content and URLs it collects, and in terms of the diversity of languages and domains, since the page ranking method it uses to select the crawled web pages prioritises content that is linked from other sites, usually sites hosted in the United States, and in most cases it favours the default version of multilingual sites, which is usually in English \cite{back_critical_2024}.
While CC is the largest resource in the multilingual environment, curated data can fill the gap for the limitations of web data, as it is data that comes from thematically related repositories that are selected by third parties based on their content value, providing a wider range of content that may not be readily available on web-crawled data.

Common Crawl data has been found to contain non-linguistic content (code, poorly encoded documents), unnatural language (short text, boilerplate content), and undesired data for LLM pretraining like adult or offensive content. Even in filtered subsets of CommonCrawl, like C4 \cite{raffel_exploring_2023}, The Pile-CC \cite{gao_pile_2020} or OSCAR \cite{abadji_cleaner_2020}, documents are classified by language identifiers and due to mislabeling, these problems are exacerbated for low-resource languages \cite{caswell_quality_2021}. 

% In order to deal with the heterogeneous and noisy nature of web data, the \textbf{Ungoliant} pipeline \cite{abadji_ungoliant_2021} was used to produce the Colossal OSCAR 1.0 corpus for the OSCAR project, and uses the following modules:

In order to deal with the heterogeneous and noisy nature of web data, the \textbf{Ungoliant} pipeline \cite{abadji_ungoliant_2021} was used to produce the Colossal OSCAR corpus for the OSCAR project, from which we include 20 CommonCrawl snapshots\footnote{Filtered data from the OSCAR project has been included for the 35 languages listed in Table \ref{tab:langs-corpus} for the following CommonCrawl snapshots: 2015-14, 2016-40, 2017-43, 2018-47, 2019-22, 2020-24, 2020-45, 2021-49, 2022-05, 2022-21, 2022-27, 2022-33, 2022-40, 2022-49, 2023-06, 2023-14, 2023-23, 2023-40, 2023-50, 2024-10.}, originally in WET format, containing the extracted plain text from the web pages, converted to UTF-8, and headers containing the metadata of each crawled document. Ungoliant uses the following modules:

\begin{itemize}
    \item Normalization: Ensures consistency in text encoding, removing noise, normalizing text formatting, and encoding all content into UTF-8.
    \item Language detection: Sentence-based language identification is performed using embedded pretrained FastText models \cite{joulin_bagtrick_2016, joulin_fasttext_2016}.
    \item Prefiltering: Documents are filtered out based on heuristic criteria, such as removing documents with a low number of characters or low language detection scores.
    \item Computation of quality warnings: Ungoliant generates quality warnings for each document which are then used for subsequent filtering stages.
    \item Computation of harmful-perplexity: Harmful content is identified using perplexity scores based on a pretrained KenLM model \cite{jansen_perplexed_2020}. This model evaluates documents to determine whether they contain harmful content.
\end{itemize}

On the other hand, for curated data, which are the rest of the sources which are not Colossal OSCAR 1.0, we use the \textbf{CURATE} pipeline \cite{palomar-giner_curated_2024}, which works as follows:

\begin{itemize}
    \item Normalization: CURATE normalizes multiple sources into TSV files that are equally treated by the pipeline modules. Similar to Ungoliant, text data is uniformly formatted and encoded. All text data is encoded in UTF-8 to maintain a standard character encoding format, excessive whitespace is trimmed, and inconsistent spacing is corrected.
    \item Language detection: CURATE uses FastText's language identification models to detect the primary language of each sentence in the documents. The character percentage of each language is calculated, and the document's main language is determined if it exceeds a threshold. For the Salamandra training corpus, the main language of a given document has a character percentage above 0.5.
    \item Deduplication: CURATE employs a three-step exact deduplication process involving hash computation and parallel processing for scalability. %\todo{Cite DistributedDocumentDedup from Adri}.
    \item Scoring: CURATE combines multiple quality scoring heuristics to assign a continuous score between 0 and 1, making it intuitive for data sampling. This provides good control over quality versus quantity, which is crucial for mid- and low-resourced languages. For the Salamandra training corpus, only documents with scores above 0.8 are retained.
    %\item Classification modules: CURATE includes modules for anonymization and adult content filtering, among others. These classifiers leverage fine-tuned Transformer models for topic labeling and are triggered if the document's vocabulary matches with word lists corresponding to each classifier module. Since CURATE was only used for selected data sources for which we know its origin, we did not use classification modules for the Salamandra training corpus 
\end{itemize}

\begin{figure}[htbp!]
    \includegraphics[width=\textwidth]{figures/data/source_dist_pointplot.pdf}
    \caption{Distribution of sources in the Salamandra pre-training dataset. Each data point represents a source, with colours indicating the type and circle size indicating the relative number of words. The logarithmic scale is used to capture variability in dataset size, which spans several orders of magnitude, so that smaller significant sources remain visible alongside larger datasets. Sources with less than 1\% of the words are listed in the lower right text box for completeness.}
    \label{fig:source_dist}
\end{figure}

The resulting source distribution from the aforementioned efforts is illustrated in Figure \ref{fig:source_dist}, where each dot is a single dataset, categorised as either data from curated sources, which includes a variety of third-party domain-specific sources, mostly under the 1B word limit; or from internally generated sources, which reflects our dedicated efforts towards domain-specific data in Spanish languages; or from web crawled data from Common Crawl, which is the dominant class in terms of size, although it only spans 4 sources.

\subsubsection{Language Distribution}
\label{subsubsec:language-distribution}

% Explain our efforts on Spanish (LEGAL/SCIENTIFIC/BIO) and Catalan, and emphasize dialectal variety. Mention inclusion and relevance of low-res and minoritized langs like Basque and Galician, Occitan.

%Efforts in Spanish and Catalan have focused on collecting data from sociolectually and dialectally diverse backgrounds. Spanish was enriched by compiling three different domain-specific corpora, each of which was applied exact document deduplication, language identification in Spanish, and heuristic filtering with a score above 0.2, following the CURATE pipeline described in section \ref{subsubsec:data-processing-pipelines}, which are described below:
%\begin{itemize}
    %item Biomedical corpus: Collection of documents regarding life sciences, articles, and clinical notes, including the CoWeSe (Corpus Web Salud Español) dataset \cite{carrino_largest_2022}, clinical case reports, electronic health record (EHR) documents (discharge reports, clinical course notes and X-ray reports), SciELO\footnote{https://scielo.isciii.es/scielo.php} publications until 2017, sections from biomedical abbreviation recognition datasets (BARR2) \cite{intxaurrondo_biomedical_2017}, Wikipedia articles related to life sciences, Google Patents\footnote{https://patents.google.com/} in Medical Domain for Spain (Spanish), European Medicines Agency (EMEA)\footnote{https://www.ema.europa.eu/en/homepage} documents, Spanish biomedical scientific literature from Medline\footnote{https://medlineplus.gov/spanish/}, and open-access articles from PubMed\footnote{https://pubmed.ncbi.nlm.nih.gov/}.
    %\item Legal corpus: Collection of state-related documents crawled from repositories from public institutions until 2023, including bulletins, decrees and orders from the \textit{Boletín Oficial del Estado} (BOE)\footnote{https://www.boe.es/buscar/legislacion.php}, the \textit{Boletín Oficial del Registro Mercantil} (BORME)\footnote{https://www.boe.es/diario\_borme/}, the \textit{Boletín Oficial de las Cortes Generales} (BOCG)\footnote{https://www.senado.es/web/actividadparlamentaria/publicacionesoficiales/senado/boletinesoficiales/index.html}, the \textit{Diari Oficial de la Generalitat de Catalunya} (DOGC)\footnote{https://dev.socrata.com/foundry/analisi.transparenciacatalunya.cat/n6hn-rmy7}; plenary sittings from the \textit{Diarios de Sesiones del Senado de España} (DSSE); and court orders from the \textit{Consejo General del Poder Judicial}\footnote{https://www.poderjudicial.es/search/indexAN.jsp}.
    %\item Scientific: Collection of articles from scholar repositories until 2023, including Dialnet\footnote{https://dialnet.unirioja.es/}, Docta Complutense\footnote{https://docta.ucm.es/home}, SciELO\footnote{https://scielo.isciii.es/scielo.php}, and Tesis Doctorals en Xarxa (TDX)\footnote{https://www.tesisenred.net/}.
%\end{itemize}

Efforts in Spanish and Catalan have focused on collecting data from sociolectually and dialectally diverse backgrounds. Spanish was enriched by compiling three different domain-specific corpora, each of which was applied exact document deduplication, language identification in Spanish, and heuristic filtering with a score above 0.2, following the CURATE pipeline described in section \ref{subsubsec:data-processing-pipelines}, which include corpora from the biomedical, scientific, and legal domain in Spanish. For Catalan, as described in \citet{palomar-giner_curated_2024}, data has been drawn from dialects such as Central, Valencian and Balearic in order to capture the full range of linguistic expression.

To complement these efforts, significant resources have been devoted to minority languages, including Basque, Galician and Occitan. These languages often have a significant lack of digital textual data, requiring collaboration with local organisations and open access repositories for their inclusion in the training data.

The pre-training corpus shows a large variation in token volume across languages. In order to deal with this, factor sampling was used to balance the representation, mainly for English and code, which were considered dominant and were therefore undersampled by half. On the other hand, oversampling was necessary to ensure that the languages of interest, including Spanish, Catalan, Galician and Basque, had sufficient token presence. This approach prevents the model from being biased towards a single language and maintains multilingual coverage. The adjusted token distribution is detailed in Table \ref{tab:langs-corpus} and illustrated in Figure \ref{fig:lang_distribution}.

\begin{figure}[htbp!]
    \centering
    \includegraphics[width=\textwidth]{figures/data/lang_dist_treemap.pdf}
    \caption{Distribution of tokens in the pre-training and continued training phase corpus after applying epoch sampling. The languages are grouped under families, represented with the ISO 639-1 codes.}
    \label{fig:lang_distribution}
\end{figure}

\sisetup{
  group-digits            = integer,          % Group only integer part
  group-separator         = {.},              % European style: . as the group separator
  group-minimum-digits    = 1,                % Group digits if at least 4
  round-mode              = places,           % Round to the specified decimal places
  output-decimal-marker   = {,},              % European style: , as the decimal marker
  scientific-notation     = false,            % Do not use scientific notation
  per-mode                = symbol,           % Formatting for per symbols
  fixed-exponent          = 6,                % Use M for million (10^6)
  detect-weight           = true,             % Detect font weight
  detect-family           = true              % Detect font family
}

%\begin{table}[!ht]
%\centering
%\resizebox{\textwidth}{!}{
%\begin{tabular}{lccccr}
%\toprule

%\textbf{Language} & \textbf{ISO 639-1} & \textbf{Family} & \textbf{Epochs} & %\textbf{Documents (M)} & \textbf{Words (M)} \\ \hline
%English & en & Germanic & 1 & \num{1062,578577} & \num{938843,669484} \\
%Spanish & es & Romance & 2 & \num{338,562896} & \num{373005,219686} \\
%French & fr & Romance & 1 & \num{67,464468} & \num{148362,286203} \\
%Code & - & N/A & 0,5 & \num{118,807026} & \num{130181,487841} \\
%Russian & ru & Balto-Slavic & 1 & \num{64,949614} & \num{107686,062137} \\
%German & de & Germanic & 1 & \num{82,657329} & \num{99010,351135} \\
%Hungarian & hu & Uralic & 1 & \num{11,765566} & \num{88830,963897} \\
%Portuguese & pt & Romance & 1 & \num{53,637987} & \num{49719,277463} \\
%Italian & it & Romance & 1 & \num{35,866230} & \num{45475,819759} \\
%Dutch & nl & Germanic & 1 & \num{30,747041} & \num{26412,654104} \\
%Polish & pl & Balto-Slavic & 1 & \num{21,302804} & \num{23027,826967} \\
%Ukrainian & uk & Balto-Slavic & 1 & \num{27,570613} & \num{20947,545120} \\
%Greek & el & Hellenic & 1 & \num{48,181852} & \num{19781,259240} \\
%Czech & cs & Balto-Slavic & 1 & \num{82,943680} & \num{19570,962510} \\
%Catalan & ca & Romance & 2 & \num{53,336438} & \num{39126,730182} \\
%Romanian & ro & Romance & 1 & \num{6,135787} & \num{15583,828315} \\
%Slovak & sk & Balto-Slavic & 1 & \num{8,396537} & \num{14031,934780} \\
%Bulgarian & bg & Balto-Slavic & 1 & \num{14,769832} & \num{13493,873623} \\
%Swedish & sv & Germanic & 1 & \num{11,968205} & \num{11078,405725} \\
%Norwegian & no & Germanic & 1 & \num{16,985857} & \num{7852,902036} \\
%Danish & da & Germanic & 1 & \num{6,852978} & \num{7795,384359} \\
%Finnish & fi & Uralic & 1 & \num{7,043679} & \num{7670,615327} \\
%Slovenian & sl & Balto-Slavic & 1 & \num{8,624371} & \num{7416,077292} \\
%Serbian & sr & Balto-Slavic & 1 & \num{13,891607} & \num{6006,728341} \\
%Croatian & hr & Balto-Slavic & 1 & \num{16,886466} & \num{5502,557803} \\
%Estonian & et & Uralic & 1 & \num{6,661360} & \num{4306,472218} \\
%Galician & gl & Romance & 2 & \num{19,804642} & \num{7215,781778} \\
%Lithuanian & lt & Balto-Slavic & 1 & \num{2,735243} & \num{3480,914767} \\
%Basque & eu & Euskera & 2 & \num{11,827052} & \num{4959,598918} \\
%Latvian & lv & Balto-Slavic & 1 & \num{1,542944} & \num{2107,951131} \\
%Maltese & mt & Semitic & 1 & \num{,774813} & \num{2044,999536} \\
%Welsh & cy & Celtic & 1 & \num{,389080} & \num{158,309390} \\
%Irish & ga & Celtic & 1 & \num{,056595} & \num{157,267952} \\
%Serbo-Croatian & sh & Balto-Slavic & 1 & \num{,414883} & \num{110,836051} \\
%Occitan & oc & Romance & 1 & \num{,082799} & \num{68,565718} \\
%Norwegian Nynorsk & nn & Germanic & 1 & \num{,150206} & \num{50,531133} \\
%\bottomrule \\
%\end{tabular}
%}
%\caption{List of languages present in the pre-training corpus, with the corresponding number of documents and words after applying the epoch sampling.}
%\label{tab:langs-corpus-old}
%\end{table}

%%% https://docs.google.com/spreadsheets/d/1rAlDsSPCduiIX97wN6Qb5_wTxcxZmHjTQYilYOqHvxc/edit?gid=1284881236#gid=1284881236

\begin{table}[!ht]
\centering
\resizebox{\textwidth}{!}{
\begin{tabular}{lccccr}
\toprule

\textbf{Language} & \textbf{ISO 639-1} & \textbf{Family} & \textbf{Epochs} & \textbf{Documents (M)} & \textbf{Words (M)} \\ \hline
English & en & Germanic & 1 & \num{773,938944} M (39,332\%) & \num{938843,669484} M (41,706\%) \\
Spanish & es & Romance & 2 &  \num{338,562896} M (17,206\%) & \num{373005,219686} M (16,570\%) \\
French & fr & Romance & 1 &  \num{67,464468} M (3,429\%) & \num{148362,286203} M (6,591\%) \\
Code & - & N/A & 0,5 &  \num{118,807026} M (6,038\%) & \num{130181,487841} M (5,783\%) \\
Russian & ru & Balto-Slavic & 1 &  \num{64,949614} M (3,301\%) & \num{107686,062137} M (4,784\%) \\
German & de & Germanic & 1 &  \num{82,657329} M (4,201\%) & \num{99010,351135} M (4,398\%) \\
Hungarian & hu & Uralic & 1 &  \num{11,765566} M (0,598\%) & \num{88830,963897} M (3,946\%) \\
Portuguese & pt & Romance & 1 &  \num{53,637987} M (2,726\%) & \num{49719,277463} M (2,209\%) \\
Italian & it & Romance & 1 &  \num{35,866230} M (1,823\%) & \num{45475,819759} M (2,020\%) \\
Catalan & ca & Romance & 2 &  \num{53,336438} M (2,711\%) & \num{39126,730182} M (1,738\%) \\
Dutch & nl & Germanic & 1 &  \num{30,747041} M (1,563\%) & \num{26412,654104} M (1,173\%) \\
Polish & pl & Balto-Slavic & 1 &  \num{21,302804} M (1,083\%) & \num{23027,826967} M (1,023\%) \\
Ukranian & uk & Balto-Slavic & 1 & \num{27,570613} M (1,401\%) & \num{20947,545120} M (0,931\%) \\
Greek & el & Hellenic & 1 & \num{48,181852} M (2,449\%) & \num{19781,259240} M (0,879\%) \\
Czech & cs & Balto-Slavic & 1 & \num{82,943680} M (4,215\%) & \num{19570,962510} M (0,869\%) \\
Romanian & ro & Romance & 1 & \num{6,135787} M (0,312\%) & \num{15583,828315} M (0,692\%) \\
Slovak & sk & Balto-Slavic & 1 & \num{8,396537} M (0,427\%) & \num{14031,934780} M (0,623\%) \\
Bulgarian & bg & Balto-Slavic & 1 & \num{14,769832} M (0,751\%) & \num{13493,873623} M (0,599\%) \\
Swedish & sv & Germanic & 1 & \num{11,968205} M (0,608\%) & \num{11078,405725} M (0,492\%) \\
Norwegian & no & Germanic & 1 & \num{16,985857} M (0,863\%) & \num{7852,902036} M (0,349\%) \\
Danish & da & Germanic & 1 & \num{6,852978} M (0,348\%) & \num{7795,384359} M (0,346\%) \\
Finnish & fi & Uralic & 1 & \num{7,043679} M (0,358\%) & \num{7670,615327} M (0,341\%) \\
Slovenian & sl & Balto-Slavic & 1 & \num{8,624371} M (0,438\%) & \num{7416,077292} M (0,329\%) \\
Galician & gl & Romance & 2 & \num{19,804642} M (1,006\%) & \num{7215,781778} M (0,321\%) \\
Serbian & sr & Balto-Slavic & 1 & \num{13,891607} M (0,706\%) & \num{6006,728341} M (0,267\%) \\
Hungarian & hr & Balto-Slavic & 1 & \num{16,886466} M (0,858\%) & \num{5502,557803} M (0,244\%) \\
Basque & eu & Euskera & 2 & \num{11,827052} M (0,601\%) & \num{4959,598918} M (0,220\%) \\
Estonian & et & Uralic & 1 & \num{6,661360} M (0,339\%) & \num{4306,472218} M (0,191\%) \\
Lithuanian & lt & Balto-Slavic & 1 & \num{2,735243} M (0,139\%) & \num{3480,914767} M (0,155\%) \\
Latvian & lv & Balto-Slavic & 1 & \num{1,542944} M (0,078\%) & \num{2107,951131} M (0,094\%) \\
Maltese & mt & Semitic & 1 & \num{,774813} M (0,039\%) & \num{2044,999536} M (0,091\%) \\
Welsh & cy & Celtic & 1 & \num{,389080} M (0,020\%) & \num{158,309390} M (0,007\%) \\
Irish & ga & Celtic & 1 & \num{,056595} M (0,003\%) & \num{157,267952} M (0,007\%) \\
Serbo-Croatian & sh & Balto-Slavic & 1 & \num{,414883} M (0,021\%) & \num{110,836051} M (0,005\%) \\
Occitan & oc & Romance & 1 & \num{,082799} M (0,004\%) & \num{68,565718} M (0,003\%) \\
Norwegian Nynorsk & nn & Germanic & 1 & \num{,150206} M (0,008\%) & \num{50,531133} M (0,002\%) \\ \hline
Total & - & - & - & \num{1967,727425} M (100\%) & \num{2251075,651921} M (100\%) \\
\bottomrule \\
\end{tabular}
}
\caption{List of languages present in the pre-training corpus, with the corresponding number of documents and words after applying the epoch sampling. Percentages for each language are given in brackets for documents and words.}
\label{tab:langs-corpus}
\end{table}


%\subsubsection{Data Analysis}

%\todo{Statistics of number of bytes/docs/words/tokens after and before each step, including deduplication and filtering.}

%\subsubsection{Data Quality Assessment}

%\todo{What is high quality? Filtering approach in CURATE and human assessment of CATalog}


\section{Model Architectures}
\label{sect:architecture}

The principal architecture introduced in this work is the \nomdef{Long Short-Term Memory AE with Soft Actor-Critic}{\lstmaesac}. This model leverages a RAE, specifically an LSTM-based AE as previously defined in \autoref{sect:rae}, to handle path history feature extraction. The LSTMAE is coupled with the \nomdef{Soft Actor-Critic}{SAC} algorithm, which has demonstrated robust performance in continuous action spaces with large observation spaces. The \lstmaesac architecture is designed to efficiently process temporal information from the agent's path history, potentially reducing the need for an extremely large network as observed in previous work \citep{ewers_deep_2025}.

\begin{table*}[htb!]
    \centering
    \caption{Definitions of architectures}
    \label{tbl:method:definitions_of_architectures}
    \begin{tabular}{@{}l|lL{3cm}L{2.5cm}|L{2.5cm}@{}}
        \toprule
        Architecture title & RL Algorithm & Path Observation Augmentation & Inner-model Feature Extraction & Observation Space                                \\ \midrule
        \lstmaesac         & SAC          & LSTM-AE                       & None                           & $\left( z_\mathrm{path}, s_\mathrm{PDM} \right)$ \\ \midrule
        \lstmaeppo         & PPO          & LSTM-AE                       & None                           & $\left( z_\mathrm{path}, s_\mathrm{PDM} \right)$ \\
        \lstmppo           & RPPO         & None                          & None                           & $\left( s_\mathrm{pos}, s_\mathrm{PDM}\right)$   \\
        \fssaclstm         & SAC          & Frame Stacking                & LSTM                           & $\left( s_\mathrm{path},s_\mathrm{PDM}\right)$   \\
        \fssacfcn          & SAC          & Frame Stacking                & Fully Connected Network        & $\left( s_\mathrm{path},s_\mathrm{PDM}\right)$   \\
        \fssacconvtwod     & SAC          & Frame Stacking                & 2D Convolution                 & $\left( s_\mathrm{path},s_\mathrm{PDM}\right)$   \\
        \fsppolstm         & PPO          & Frame Stacking                & LSTM                           & $\left( s_\mathrm{path},s_\mathrm{PDM}\right)$   \\
        \fsppofcn          & PPO          & Frame Stacking                & Fully Connected Network        & $\left( s_\mathrm{path},s_\mathrm{PDM}\right)$   \\
        \fsppoconvtwod     & PPO          & Frame Stacking                & 2D Convolution                 & $\left( s_\mathrm{path},s_\mathrm{PDM}\right)$   \\ \bottomrule
    \end{tabular}
\end{table*}

To comprehensively evaluate the efficacy of \lstmaesac, a suite of comparative architectures were developed. These vary in their DRL algorithms, path observation augmentation techniques, and inner-model feature extraction methods. The key variants are defined in \autoref{tbl:method:definitions_of_architectures} and were designed to systematically explore the impact of different components:
\begin{itemize}
    \item Efficacy of path observation augmentation,
    \item Differences in DRL algorithm,
    \item Impact on inner-model feature extraction in lue of path observation augmentation.
\end{itemize}

\begin{figure*}[htbp]
    \centering

    \tikzset{
        block/.style = {draw, fill=white, rectangle, minimum height=3em},
        input/.style = {fill=none, rectangle},
        output/.style= {fill=none, rectangle},
    }
    \subfloat[Policy for the LSTMAE variation. Note that the LSTM encoder module is frozen and its parameters do not get updated during training. \label{fig:method:lstmae_policy}]{
        \begin{tikzpicture}[]
            {[start chain]
                \node [input, on chain] (spath) {$s_\mathrm{pos}$};
                \node [block, ml/encoder, on chain, dashed] (lstm_e) {LSTM AE};
                \node [sum, on chain] (concatenate)  {};
                \node [block, on chain] (policy) {FCN};
                \node [output, on chain] (action) {$a$};
            }
            \node [input, above of=spath, densely dotted] (h)  {$h_{t-1}$};
            \draw [->, ml/weights/arrow] (h) -| (lstm_e);
            \node [input, below of=spath] (spdm)  {$s_\mathrm{PDM}$};
            \draw [->, ml/weights/arrow] (spdm) -| (concatenate);
            \path (concatenate) -- node[above] {$s$}(policy);
            \path (lstm_e) -- node[above] {$z$}(concatenate);
        \end{tikzpicture}
    }
    \hfill
    \subfloat[RPPO policy where $h_{t-1}$ is the LSTM hidden state from the previous time-step.\label{fig:method:rppo_policy}]{
        \begin{tikzpicture}[]
            {[start chain]
                \node [input, on chain] (spath) {$s_\mathrm{pos}$};
                \node [sum, on chain] (concatenate)  {};
                \node [block, on chain] (policy_lstm) {LSTM};
                \node [block, on chain] (policy_fcn) {FCN};
                \node [output, on chain] (action) {$a$};
            }
            \node [input, above of=spath] (h)  {$h_{t-1}$};
            \node [input, below of=spath] (spdm)  {$s_\mathrm{PDM}$};
            \draw [->, ml/weights/arrow] (spdm) -| (concatenate);
            \draw [->, ml/weights/arrow, densely dotted] (h) -| (policy_lstm);
            \path (concatenate) -- node[above] {$s$}(policy_lstm);
        \end{tikzpicture}
    }
    \hfill
    \subfloat[FCN inner-model feature extraction.\label{fig:method:fcn_policy}]{
        \begin{tikzpicture}[]
            {[start chain]
                \node [input, on chain] (spath) {$s_\mathrm{path}$};
                \node [ml/encoder, on chain] (path_fe) {FCN};
                \node [sum, on chain] (concatenate)  {};
                \node [block, on chain] (policy) {FCN};
                \node [output, on chain] (action) {$a$};
            }
            \node [input, below of=spath] (spdm)  {$s_\mathrm{PDM}$};
            \draw [->, ml/weights/arrow] (spdm) -| (concatenate);
            \path (concatenate) -- node[above] {$s$}(policy);
        \end{tikzpicture}
    }
    \vfill
    \subfloat[LSTM inner-model feature extraction.\label{fig:method:lstm_policy}]{
        \begin{tikzpicture}[]
            {[start chain]
                \node [input, on chain] (spath) {$s_\mathrm{path}$};
                \node [ml/encoder, on chain] (path_fe_lstm) {LSTM};
                \node [ml/encoder, on chain] (path_fe_fcn) {FCN};
                \node [sum, on chain] (concatenate)  {};
                \node [block, on chain] (policy) {FCN};
                \node [output, on chain] (action) {$a$};
            }
            \node [input, below of=spath] (spdm)  {$s_\mathrm{PDM}$};
            \node[rectangle, inner sep=0.2cm, draw=none, fit=(path_fe_lstm)] (path_fe_lstm_border) {};
            \draw [->, ml/weights/arrow, densely dotted]
            (path_fe_lstm.east)++(0,0.15)
            -- ($(path_fe_lstm_border.east)+(0,0.15)$)
            -- (path_fe_lstm_border.north east)
            -- node[above] {$h$} (path_fe_lstm_border.north west)
            -- ($(path_fe_lstm_border.west)+(0,0.15)$)
            -- ($(path_fe_lstm.west)+(0,0.15)$);

            \path (concatenate) --node[above] {$s$} (policy);
            \draw [->, ml/weights/arrow] (spdm) -| (concatenate);
        \end{tikzpicture}
    }
    \hfill
    \subfloat[2D CNN inner-model feature extraction.\label{fig:method:cnn_policy}]{
        \begin{tikzpicture}[]
            {[start chain]
                \node [input, on chain] (spath) {$s_\mathrm{path}$};
                \node [ml/encoder, on chain] (path_fe_cnn) {2D CNN};
                \node [ml/encoder, on chain] (path_fe_fcn) {FCN};
                \node [sum, on chain] (concatenate)  {};
                \node [block, on chain] (policy) {FCN};
                \node [output, on chain] (action) {$a$};
            }
            \node [input, below of=spath] (spdm)  {$s_\mathrm{PDM}$};

            \path (concatenate) --node[above] {$s$} (policy);
            \draw [->, ml/weights/arrow] (spdm) -| (concatenate);
        \end{tikzpicture}
    }
    \caption{The five proposed policy architectures for use with either PPO, RPPO, or SAC. Figure \protect\subref{fig:method:rppo_policy} is only used with RPPO.}
    \label{<label>}
\end{figure*}

% \begin{figure}[htbp]
%     \centering
%     \subfloat[
%         2D convolution
%         \label{fig:buffered_linestring_with_overlap}
%     ]{
%         \includegraphics[height=6cm]{path_feature_extractor_conv2d.pdf}
%     }
%     \hfill
%     \subfloat[
%         LSTM
%         \label{fig:buffered_linestring_triangulated}
%     ]{
%         \includegraphics[height=6cm]{path_feature_extractor_lstm.pdf}
%     }
%     \hfill
%     \subfloat[
%         Fully cconnected network
%         \label{fig:buffered_linestring_triangulated}
%     ]{
%         \includegraphics[height=6cm]{path_feature_extractor_fcn.pdf}
%     }
%     \caption{Visualizations of concepts related to the buffered polygon representation of the seen area.}
%     \label{fig:buffered_linestring}
% \end{figure}


\section{Methodology}
\label{methodology}
To address the three challenges outlined in Section \ref{introduction} and develop a method to predict LLM capabilities prior to pre-training, we meticulously construct 1.5 trillion tokens of high-quality pre-training data and train base models at three different sizes. We then utilize a multi-template complementation mechanism to accurately evaluate the model’s knowledge retention capbility on CBQA tasks. Furthermore, we retrieve these knowledge triples from the pre-training data, introduce the SMI metric and establish a predictive equation that maps SMI to the ACC on CBQA tasks using linear regression. Finally, we compute $\text{R}^2$ and MSE to assess the predictive performance.


\subsection{LLM Pre-training}
\label{3-1}

\paragraph{Pre-training data.} 
Our pre-training data is composed of four main categories: English, Chinese, multilingual, and code. We utilize four high-quality and widely used open-source datasets: Falcon RefinedWeb~\cite{refinedweb}, SlimPajama~\cite{cerebras2023slimpajama}, Wikipedia~\cite{wikidump}, and Starcoderdata~\cite{DBLP:journals/tmlr/LiAZMKMMALCLZZW23}. For Chinese data, we collect diverse content from the internet, including blogs, books, chats, encyclopedias, and other categories.

To ensure data quality, we implement a rigorous cleaning process. First, during data filtering, we remove short texts, special characters, and texts lacking punctuation, and we use Toxigen~\cite{hartvigsen2022toxigen} to filter out toxic content. Second, for data deduplication, we apply Locality-Sensitive Hashing to eliminate duplicate documents and employ strict matching techniques to remove duplicate sentences. Finally, for data inspection, we sample 1\% of the data and use the Qwen2-1.5B~\cite{DBLP:journals/corr/abs-2407-10671} to calculate the loss, followed by manual review of the high-loss data to ensure it met our quality standards. The detailed composition of our pre-training data is provided in Figure~\ref{pre-training_data}.

\paragraph{Model architectures.} 
Our models are built on Transformer architectures similar to Llama2 (Table~\ref{model_architecture}). The 1.6B model consists of 24 layers with a maximum sequence length of 2048 tokens. The 7B model features 32 layers with a maximum sequence length of 4096 tokens, while the 13B model includes 40 layers, also supporting a maximum sequence length of 4096 tokens. We train the tokenizer using ten thousand tokens from minority languages, leveraging the Wikipedia dataset and the Byte Pair Encoding method. These tokens are then integrated with the tokenizer from InternLM~\cite{DBLP:journals/corr/abs-2403-17297}.


\subsection{Evaluating LLM capabilities}
\label{3-2}

\paragraph{Knowledge triples.}
We center our research on the ability of LLMs to memorize knowledge triples. A knowledge triple is represented as a tuple $t = (s, r, o)$, where $s$ is the subject, $r$ is the relation, and $o$ is the object \cite{DBLP:conf/acl/JuCY0DZL24}. For a given LLM, denoted as $F$, we define $F$ as mastering the knowledge triple $t$ if the following condition holds:

\begin{equation}
    F(q_{s,r}) = o, \quad (q \in \mathbf{Q}, o \in \mathbf{O}).
\end{equation}

Here, \(q_{s,r}\) represents the combined representation of the subject \(s\) and the relation \(r\), while \(o\) denotes the representation of the object. For instance, if \(s = \text{Apple}\) and \(r = \text{headquarter}\), then \(q_{s,r}\) can be expressed as statements like ``The headquarters of Apple is in...'', ``Apple is headquartered in...'', or ``Apple's head office is based in...''. Similarly, \(o\) corresponds to the object representation, such as ``Cupertino'', ``Cupertino, California'', or ``Cupertino city''. The sum of all \(q_{s,r}\) representations is denoted as \(\mathbf{Q}\), and the sum of all \(o\) representations is denoted as \(\mathbf{O}\).

\paragraph{Multi-template complementation mechanism.}
Accurately evaluating LLMs’ memory of knowledge is challenging due to the diverse range of queries formed by the subject and relation of a knowledge triple. To address this, we implement a multi-template complementation mechanism, which has demonstrated strong performance in CBQA tasks~\cite{DBLP:journals/corr/abs-2409-15825}. Experimental results indicate that memory levels assessed using this mechanism closely align with the model’s actual CBQA task performance. Moreover, the distribution of memory levels is strongly correlated with the model’s performance after fine-tuning, further validating the effectiveness of the multi-template complementation mechanism in evaluating LLMs’ knowledge retention.

Specifically, we generate a large set of query templates that are semantically similar but vary in form. From this set, we selecte 20 templates with diverse lengths and structures for each type of knowledge triple. Each template $q$ represents a specific instance within the query set \(\mathbf{Q}\), and the 20 selected $q$’s are used to approximate the entire \(\mathbf{Q}\).


% 
\begin{figure}[tb]
% \vskip 0.2in
\begin{center}
\centerline{\includegraphics[width=\columnwidth]{pictures/eval_data.pdf}}
\caption{Statistics of the evaluation set.}
\label{eval_data}
\end{center}
\vskip -0.4in
\end{figure}
% 
\begin{figure}[tb]
\vskip 0.2in
\begin{center}
\centerline{\includegraphics[width=\columnwidth]{pictures/cooccur_frequency.pdf}}
\caption{Histogram of co-occurrence frequencies of knowledge triples in the evaluation set within the pre-training data.}
\label{cooccur_frequency}
\end{center}
\vskip -0.3in
\end{figure}

\subsection{Predicting LLM capabilities}
\label{3-3}

\paragraph{MI metric.}
We are committed to investigating the relationship between the capabilities of LLMs and their pre-training data. Existing research has explored how the co-occurrence of questions and answers within pre-training data influences model ACC, identifying a general trend: higher co-occurrence frequencies are typically associated with improved response ACC in LLMs~\cite{DBLP:journals/corr/abs-2404-05405, DBLP:journals/jmlr/ChowdheryNDBMRBCSGSSTMRBTSPRDHPBAI23}. However, co-occurrence metrics alone fail to account for the specificity of knowledge. When the subject and object of a knowledge triple are common words that frequently appear in pre-training data, the specificity is low, making it more challenging for the model to recall the information accurately. In contrast, when the subject and object are less frequent, the specificity increases, thereby improving the model’s ability to retrieve the knowledge.

For instance, the headquarters address of NVIDIA is easier to recall than that of Apple because ``Apple'' often appears in diverse contexts, reducing its specificity. According to information theory, Mutual Information (MI) accounts for both the co-occurrence of two variables and their individual specificity \cite{DBLP:journals/bstj/Shannon48}. Thus, we propose leveraging MI to address the challenge of predicting LLM capabilities.

We introduce the MI metric in the pre-training data. Given a pre-training data \(\mathbf{P}\) consisting of \(\mathbf{N}\) paragraphs and a knowledge triple $t = (s, r, o)$, we first calculate three indicators: P(s), the proportion of paragraphs containing the subject $s$, P(o), the proportion of paragraphs containing the object $o$, and P(s, o), the proportion of paragraphs containing both $s$ and $o$. The MI between $s$ and $o$ is then defined as:

\begin{equation}
    I(s, o) = P(s, o) \log \left( \frac{P(s, o)}{P(s)P(o)} \right).
\end{equation}

In this formula, $P(s, o)$ measures the co-occurrence frequency of the subject and object, serving as our baseline. The denominator in the logarithmic term, $P(s)$ and $P(o)$, penalizes the individual occurrences of the subject and object, respectively, highlighting their mutual connection. The formula, in its entirety, quantifies the amount of information shared between $s$ and $o$, which can also be interpreted as the reduction in uncertainty about $o$ given $s$. In the context of LLMs, this shared information corresponds to the likelihood of the model generating the object given the subject. Given the highly skewed distribution of MI values, we apply a logarithm to this metric and normalize it to the range between 0 and 1.

\begin{equation}
    MI(s, o) = Norm(\log(I(s, o))).
\end{equation}

\paragraph{Size-dependent MI metric.}
The MI metric focuses solely on the distribution of pre-training data, without considering the memory capacity of LLMs. However, knowledge retention in LLMs is influenced by both data distribution and model size~\cite{DBLP:journals/corr/abs-2403-00510}. OpenAI’s research highlights that the capabilities of LLMs improve as model size increases \cite{DBLP:journals/corr/abs-2001-08361}. To address this limitation, we propose an enhancement to the MI metric by incorporating model size \(\Phi\) (measured in billions of parameters).

We introduce the SMI metric, an exponential function where the MI metric serves as the base and $1 + \frac{1}{\Phi}$ is the exponent. The MI metric quantifies the informational content of a knowledge triple in the pre-training data, while the model size \(\Phi\) reflects the memory capacity of LLMs. Together, these two factors govern the model’s knowledge retention:

\begin{equation}
    SMI(s, o, \Phi) = Norm(\log(I(s, o)))^{1 + \frac{1}{\Phi}}.
\end{equation}

Since the MI metric is less than 1, as the model size increases, $1 + \frac{1}{\Phi}$ decreases, leading to an increase in the SMI metric value, which indicates stronger memory capabilities of the model.

\paragraph{Predicting LLM capabilities.}
We are committed to linking SMI metric with the ACC of the model. For each knowledge triple, we retrieve the entire pre-training data and calculate three key metrics: the co-occurrence metric, the MI metric, and the SMI metric. Similar to the MI metric, the co-occurrence metric is logarithmically transformed and normalized. Next, we use the knowledge triples to construct questions and test them on LLMs to determine their ACC. Finally, we fit a predictive equation that captures the relationship between the metrics for all knowledge triples in the evaluation set and the observed ACC.

