


\begin{figure}[tb]
\vskip 0.1in
\begin{center}
\centerline{\includegraphics[width=\columnwidth]{pictures/pretraining_data.pdf}}
\caption{The composition statistics of pre-training data, including Chinese, English, multilingual, and code.}
\label{pre-training_data}
\end{center}
\vskip -0.3in
\end{figure}

\section{Model Architecture}
\label{model_architecture}

\begin{figure}[t!]
    \centering
    \begin{subfigure}[b]{0.4\linewidth}
        \centering
        \includegraphics[width=0.7\linewidth]{assets/discrete_mamba.png}
        \caption{
        The Discrete Mamba-2 block \cite{mohawk} modifies the original Mamba-2 architecture by removing both post-convolution activation and pre-output projection normalization. Additionally, the Discrete Mamba-2 sequence mixer eliminates the $\Delta$ discretization parameter and directly projects the $\mathbf{A}$ matrix from the input.
        }
        \label{fig:discrete_mamba}
    \end{subfigure}
    \hspace{0.1\linewidth} % Adjust the spacing
    \begin{subfigure}[b]{0.4\linewidth}
        \centering
        \includegraphics[width=0.62\linewidth]{assets/llamba_architecture.png}
        \caption{Llamba models—Llamba-1B, Llamba-3B, and Llamba-8B—are based on the architecture of their Llama teacher models. Each block comprises two sub-blocks with residual connections:
        (1) RMS Normalization followed by a Discrete Mamba-2 layer.
        (2) RMS Normalization followed by a feed-forward layer.
        }
        \label{fig:llamba_architecture}
    \end{subfigure}
    \caption{Comparison of the Discrete Mamba-2 block and the Llamba architecture.}
    \label{fig:comparison}
\end{figure}

Unlike the Mamba and Mamba-2 architectures, which were designed for training from scratch, \textit{Llamba is directly motivated by architectural distillation}.
In particular, the Mohawk distillation framework involves aligning sub-networks of the model at various levels of granularity (\Cref{sec:distillation}). 
This constraints Llamba to retain the overall architecture of the teacher model, ideally modifying only the attention matrix mixer by replacing it with a subquadratic alternative.

The Llamba models—Llamba-1B, Llamba-3B, and Llamba-8B—comprise 16, 28, and 32 residual Mamba-2 blocks, respectively, followed by feed-forward layers. These models share the tokenizer and vocabulary of Llama-3.1, with hidden dimensions of 2048 for Llamba-1B, 3072 for Llamba-3B, and 4096 for Llamba-8B. 
In addition, Llamba differs from the original Mamba-2 architecture \citep{mamba2} in the following ways (see \Cref{fig:llamba_architecture}):
\begin{itemize}[leftmargin=*]

\item \textbf{Alternating MLP blocks}:
Llamba interleaves Llama’s Gated MLP components between each Mamba-2 mixing layer, unlike Mamba-1 and Mamba-2, which consist solely of SSM blocks.

\item \textbf{Multi-head structure}: Llama-3.x models use grouped-query attention (GQA) \citep{gqa, mqa}, which employs 32 query heads and 8 key-value heads to boost inference speed and reduce the size of the decoding cache. However, Mamba’s recurrent layers don’t rely on a cache, so these optimizations aren’t needed. Instead, \textit{Llamba blocks feature a Multi-Head variant} of Mamba-2 with 32 heads and dimensions of 64, 96, or 128, along with a state size of 64. While this design differs from Mamba-2’s ``multi-value attention'' (MVA) architecture, it still keeps inference costs low.

\item \textbf{Non-linearities}: We remove the normalization before the output projection and the activation after convolution, as these are non-linear operations that do not exist in the attention block and hurts alignment (See \Cref{subsec:mohawk}). 

\item \textbf{Discretization}: Llamba uses \textit{Discrete-Mamba-2}, a variant that projects the matrix $\mathbf{A}$ directly from the input, eliminating the discretization parameters $\Delta$ to better match the inherently discrete attention mechanisms. 
\end{itemize}

Notably, these changes not only facilitate the distillation process but also improve training efficiency. Alternating with MLPs \textbf{reduces the number of temporal mixing layers}, enabling Llamba to achieve faster computation than other models of comparable size (see \Cref{subsec:throughput}). Furthermore, training becomes simpler and more efficient by eliminating normalization-related all-reduce operations.



\section{Methodology}
\label{methodology}
To address the three challenges outlined in Section \ref{introduction} and develop a method to predict LLM capabilities prior to pre-training, we meticulously construct 1.5 trillion tokens of high-quality pre-training data and train base models at three different sizes. We then utilize a multi-template complementation mechanism to accurately evaluate the model’s knowledge retention capbility on CBQA tasks. Furthermore, we retrieve these knowledge triples from the pre-training data, introduce the SMI metric and establish a predictive equation that maps SMI to the ACC on CBQA tasks using linear regression. Finally, we compute $\text{R}^2$ and MSE to assess the predictive performance.


\subsection{LLM Pre-training}
\label{3-1}

\paragraph{Pre-training data.} 
Our pre-training data is composed of four main categories: English, Chinese, multilingual, and code. We utilize four high-quality and widely used open-source datasets: Falcon RefinedWeb~\cite{refinedweb}, SlimPajama~\cite{cerebras2023slimpajama}, Wikipedia~\cite{wikidump}, and Starcoderdata~\cite{DBLP:journals/tmlr/LiAZMKMMALCLZZW23}. For Chinese data, we collect diverse content from the internet, including blogs, books, chats, encyclopedias, and other categories.

To ensure data quality, we implement a rigorous cleaning process. First, during data filtering, we remove short texts, special characters, and texts lacking punctuation, and we use Toxigen~\cite{hartvigsen2022toxigen} to filter out toxic content. Second, for data deduplication, we apply Locality-Sensitive Hashing to eliminate duplicate documents and employ strict matching techniques to remove duplicate sentences. Finally, for data inspection, we sample 1\% of the data and use the Qwen2-1.5B~\cite{DBLP:journals/corr/abs-2407-10671} to calculate the loss, followed by manual review of the high-loss data to ensure it met our quality standards. The detailed composition of our pre-training data is provided in Figure~\ref{pre-training_data}.

\paragraph{Model architectures.} 
Our models are built on Transformer architectures similar to Llama2 (Table~\ref{model_architecture}). The 1.6B model consists of 24 layers with a maximum sequence length of 2048 tokens. The 7B model features 32 layers with a maximum sequence length of 4096 tokens, while the 13B model includes 40 layers, also supporting a maximum sequence length of 4096 tokens. We train the tokenizer using ten thousand tokens from minority languages, leveraging the Wikipedia dataset and the Byte Pair Encoding method. These tokens are then integrated with the tokenizer from InternLM~\cite{DBLP:journals/corr/abs-2403-17297}.


\subsection{Evaluating LLM capabilities}
\label{3-2}

\paragraph{Knowledge triples.}
We center our research on the ability of LLMs to memorize knowledge triples. A knowledge triple is represented as a tuple $t = (s, r, o)$, where $s$ is the subject, $r$ is the relation, and $o$ is the object \cite{DBLP:conf/acl/JuCY0DZL24}. For a given LLM, denoted as $F$, we define $F$ as mastering the knowledge triple $t$ if the following condition holds:

\begin{equation}
    F(q_{s,r}) = o, \quad (q \in \mathbf{Q}, o \in \mathbf{O}).
\end{equation}

Here, \(q_{s,r}\) represents the combined representation of the subject \(s\) and the relation \(r\), while \(o\) denotes the representation of the object. For instance, if \(s = \text{Apple}\) and \(r = \text{headquarter}\), then \(q_{s,r}\) can be expressed as statements like ``The headquarters of Apple is in...'', ``Apple is headquartered in...'', or ``Apple's head office is based in...''. Similarly, \(o\) corresponds to the object representation, such as ``Cupertino'', ``Cupertino, California'', or ``Cupertino city''. The sum of all \(q_{s,r}\) representations is denoted as \(\mathbf{Q}\), and the sum of all \(o\) representations is denoted as \(\mathbf{O}\).

\paragraph{Multi-template complementation mechanism.}
Accurately evaluating LLMs’ memory of knowledge is challenging due to the diverse range of queries formed by the subject and relation of a knowledge triple. To address this, we implement a multi-template complementation mechanism, which has demonstrated strong performance in CBQA tasks~\cite{DBLP:journals/corr/abs-2409-15825}. Experimental results indicate that memory levels assessed using this mechanism closely align with the model’s actual CBQA task performance. Moreover, the distribution of memory levels is strongly correlated with the model’s performance after fine-tuning, further validating the effectiveness of the multi-template complementation mechanism in evaluating LLMs’ knowledge retention.

Specifically, we generate a large set of query templates that are semantically similar but vary in form. From this set, we selecte 20 templates with diverse lengths and structures for each type of knowledge triple. Each template $q$ represents a specific instance within the query set \(\mathbf{Q}\), and the 20 selected $q$’s are used to approximate the entire \(\mathbf{Q}\).


% 
\begin{figure}[tb]
% \vskip 0.2in
\begin{center}
\centerline{\includegraphics[width=\columnwidth]{pictures/eval_data.pdf}}
\caption{Statistics of the evaluation set.}
\label{eval_data}
\end{center}
\vskip -0.4in
\end{figure}
% 
\begin{figure}[tb]
\vskip 0.2in
\begin{center}
\centerline{\includegraphics[width=\columnwidth]{pictures/cooccur_frequency.pdf}}
\caption{Histogram of co-occurrence frequencies of knowledge triples in the evaluation set within the pre-training data.}
\label{cooccur_frequency}
\end{center}
\vskip -0.3in
\end{figure}

\subsection{Predicting LLM capabilities}
\label{3-3}

\paragraph{MI metric.}
We are committed to investigating the relationship between the capabilities of LLMs and their pre-training data. Existing research has explored how the co-occurrence of questions and answers within pre-training data influences model ACC, identifying a general trend: higher co-occurrence frequencies are typically associated with improved response ACC in LLMs~\cite{DBLP:journals/corr/abs-2404-05405, DBLP:journals/jmlr/ChowdheryNDBMRBCSGSSTMRBTSPRDHPBAI23}. However, co-occurrence metrics alone fail to account for the specificity of knowledge. When the subject and object of a knowledge triple are common words that frequently appear in pre-training data, the specificity is low, making it more challenging for the model to recall the information accurately. In contrast, when the subject and object are less frequent, the specificity increases, thereby improving the model’s ability to retrieve the knowledge.

For instance, the headquarters address of NVIDIA is easier to recall than that of Apple because ``Apple'' often appears in diverse contexts, reducing its specificity. According to information theory, Mutual Information (MI) accounts for both the co-occurrence of two variables and their individual specificity \cite{DBLP:journals/bstj/Shannon48}. Thus, we propose leveraging MI to address the challenge of predicting LLM capabilities.

We introduce the MI metric in the pre-training data. Given a pre-training data \(\mathbf{P}\) consisting of \(\mathbf{N}\) paragraphs and a knowledge triple $t = (s, r, o)$, we first calculate three indicators: P(s), the proportion of paragraphs containing the subject $s$, P(o), the proportion of paragraphs containing the object $o$, and P(s, o), the proportion of paragraphs containing both $s$ and $o$. The MI between $s$ and $o$ is then defined as:

\begin{equation}
    I(s, o) = P(s, o) \log \left( \frac{P(s, o)}{P(s)P(o)} \right).
\end{equation}

In this formula, $P(s, o)$ measures the co-occurrence frequency of the subject and object, serving as our baseline. The denominator in the logarithmic term, $P(s)$ and $P(o)$, penalizes the individual occurrences of the subject and object, respectively, highlighting their mutual connection. The formula, in its entirety, quantifies the amount of information shared between $s$ and $o$, which can also be interpreted as the reduction in uncertainty about $o$ given $s$. In the context of LLMs, this shared information corresponds to the likelihood of the model generating the object given the subject. Given the highly skewed distribution of MI values, we apply a logarithm to this metric and normalize it to the range between 0 and 1.

\begin{equation}
    MI(s, o) = Norm(\log(I(s, o))).
\end{equation}

\paragraph{Size-dependent MI metric.}
The MI metric focuses solely on the distribution of pre-training data, without considering the memory capacity of LLMs. However, knowledge retention in LLMs is influenced by both data distribution and model size~\cite{DBLP:journals/corr/abs-2403-00510}. OpenAI’s research highlights that the capabilities of LLMs improve as model size increases \cite{DBLP:journals/corr/abs-2001-08361}. To address this limitation, we propose an enhancement to the MI metric by incorporating model size \(\Phi\) (measured in billions of parameters).

We introduce the SMI metric, an exponential function where the MI metric serves as the base and $1 + \frac{1}{\Phi}$ is the exponent. The MI metric quantifies the informational content of a knowledge triple in the pre-training data, while the model size \(\Phi\) reflects the memory capacity of LLMs. Together, these two factors govern the model’s knowledge retention:

\begin{equation}
    SMI(s, o, \Phi) = Norm(\log(I(s, o)))^{1 + \frac{1}{\Phi}}.
\end{equation}

Since the MI metric is less than 1, as the model size increases, $1 + \frac{1}{\Phi}$ decreases, leading to an increase in the SMI metric value, which indicates stronger memory capabilities of the model.

\paragraph{Predicting LLM capabilities.}
We are committed to linking SMI metric with the ACC of the model. For each knowledge triple, we retrieve the entire pre-training data and calculate three key metrics: the co-occurrence metric, the MI metric, and the SMI metric. Similar to the MI metric, the co-occurrence metric is logarithmically transformed and normalized. Next, we use the knowledge triples to construct questions and test them on LLMs to determine their ACC. Finally, we fit a predictive equation that captures the relationship between the metrics for all knowledge triples in the evaluation set and the observed ACC.

