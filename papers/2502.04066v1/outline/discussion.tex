\section{Discussion}
\label{discussion}

In this section, based on our SMI mertric and experimental results, we offer the following two recommendations:

\paragraph{Optimizing knowledge distribution in pre-training data.}
The occurrence frequencies and specificity of knowledge in the pre-training data significantly affect the model's ability to retain knowledge. Therefore, when constructing pre-training data, it is essential to optimize the distribution of knowledge. Specifically, the frequencies of critical knowledge, particularly those with low specificity and low occurrence frequencies, can be appropriately increased to improve the model's retention of such knowledge.

At the same time, to avoid redundancy caused by excessively high frequencies of certain types of knowledge, which can lead to wasted computational resources, strategies can be employed to reduce the frequencies of overrepresented knowledge, ensuring balanced occurrence frequencies. This approach helps maintain diversity and equilibrium within the pre-training data.

\paragraph{Balancing pre-training data and model size.}
The memory capacity of a model influences its ability to retain knowledge. Our findings show that smaller models can retain knowledge effectively, but they place greater demands on the pre-training data distribution. Research institutions with limited computational resources may need to focus more on optimizing the data distribution. If the existing data cannot achieve a high SMI for critical knowledge, increasing the model size could be a viable solution to retain more knowledge. In contrast, larger models are more robust to variations in pre-training data, allowing for greater flexibility in adjusting the data distribution.

In practical applications, the knowledge coverage strategy for pre-training data should be tailored to the model’s size. For smaller models, the focus should be on increasing the frequencies of critical knowledge to make the most of the limited memory capacity. In contrast, for larger models, broadening the coverage of knowledge with medium to high SMI can enhance performance across a wider range of tasks.

