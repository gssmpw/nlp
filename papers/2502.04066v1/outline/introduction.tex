

\usepackage[T1]{fontenc}
\usepackage{amsfonts,amsmath,amsthm,amssymb,mathtools}  %

\mathtoolsset{centercolon}
\usepackage{xfrac,nicefrac}
\usepackage{xcolor}
\usepackage{mathdots}
\usepackage{mleftright}  %
\let\left\mleft
\let\right\mright

\usepackage{xspace}
\xspaceaddexceptions{]\}}  %
\usepackage{regexpatch}

\usepackage{bm,bbm,dsfont}  %
\usepackage{caption}
\usepackage[normalem]{ulem}
\usepackage{enumitem}

\usepackage{graphicx}
\usepackage{float}
\usepackage{subcaption}  %
\usepackage{tcolorbox}
\usepackage{tikz}
\usetikzlibrary{decorations.pathreplacing}
\usetikzlibrary{calc}
\usetikzlibrary{positioning}
\usetikzlibrary{arrows.meta}

\usepackage[linesnumbered,boxed,ruled,vlined]{algorithm2e}
\usepackage{algpseudocode}

\usepackage{thmtools,thm-restate}
\theoremstyle{plain}
\newtheorem{prob}{Problem}  %
\newtheorem{open}[prob]{Open Question}
\newtheorem{theorem}{Theorem}[section]  %
\newtheorem{lemma}[theorem]{Lemma}
\newtheorem{fact}[theorem]{Fact}
\newtheorem{observation}[theorem]{Observation}
\newtheorem{prop}[theorem]{Proposition}
\newtheorem{cor}[theorem]{Corollary}
\newtheorem{claim}[theorem]{Claim}
\theoremstyle{definition}  %
\newtheorem{definition}[theorem]{Definition}
\newtheorem{remark}[theorem]{Remark}
\newtheorem{property}[theorem]{Property}
\newenvironment{proofof}[1]{\begin{proof}[Proof of #1]}{\end{proof}}
\newenvironment{proofsketch}{\begin{proof}[Proof Sketch]}{\end{proof}}
\newenvironment{proofsketchof}[1]{\begin{proof}[Proof Sketch of #1]}{\end{proof}}

\usepackage[colorlinks,citecolor=blue,linkcolor=blue,urlcolor=red]{hyperref}
\usepackage[capitalise]{cleveref}
\crefname{algocf}{Algorithm}{Algorithms}
\Crefname{algocf}{Algorithm}{Algorithms}
\crefname{prob}{Problem}{Problems}
\crefname{claim}{Claim}{Claims}
\crefname{cor}{Corollary}{Corollaries}
\crefname{fact}{Fact}{Facts}

\DeclarePairedDelimiter{\ceil}{\lceil}{\rceil}
\DeclarePairedDelimiter{\floor}{\lfloor}{\rfloor}
\DeclarePairedDelimiter{\norm}{\lVert}{\rVert}
\DeclarePairedDelimiter{\bk}{(}{)}
\DeclarePairedDelimiter{\Bk}{[}{]}
\DeclarePairedDelimiter{\BK}{\{}{\}}
\DeclarePairedDelimiter{\rb}{(}{)}
\DeclarePairedDelimiter{\sqb}{[}{]}
\DeclarePairedDelimiter{\cb}{\{}{\}}
\DeclarePairedDelimiter{\angbk}{\langle}{\rangle}
\DeclarePairedDelimiter{\abs}{\lvert}{\rvert}

\DeclareMathOperator*{\E}{\mathbb{E}}
\DeclareMathOperator*{\PrAux}{Pr}
\let\Pr\PrAux
\DeclareMathOperator{\poly}{poly}
\DeclareMathOperator{\polylog}{polylog}
\DeclareMathOperator*{\argmax}{arg\,max}
\DeclareMathOperator*{\argmin}{arg\,min}

\newcommand{\F}{\mathbb{F}}
\renewcommand{\d}{\mathrm{d}}
\renewcommand{\tilde}{\widetilde}
\newcommand{\eqdef}{\eqqcolon}
\newcommand{\defeq}{\coloneqq}
\newcommand{\eps}{\varepsilon}
\newcommand{\T}{\mathcal{T}}
\newcommand{\N}{\mathbb{N}}
\newcommand{\R}{\mathbb{R}}
\newcommand{\Z}{\mathbb{Z}}
\renewcommand{\l}{\ell}
\renewcommand{\emptyset}{\varnothing}
\renewcommand{\epsilon}{\eps}
\newcommand{\Patrascu}{\textup{P{\v{a}}tra{\c{s}}cu}\xspace}

\newcommand{\numberthis}{\addtocounter{equation}{1}\tag{\theequation}}
\newcommand{\smallsub}{\scriptscriptstyle}
\newcommand{\tallsub}{{\textstyle\mathstrut}}
\newcommand{\tall}{\vphantom\sum}
\newcommand{\matwrap}[1]{{\begin{matrix}#1\end{matrix}}}
\newcommand{\matwrapdisplay}[1]{{\begin{matrix}\displaystyle #1\end{matrix}}}


\usepackage{regexpatch}
\makeatletter
\xpatchcmd\thmt@restatable{%
\csname #2\@xa\endcsname\ifx\@nx#1\@nx\else[{#1}]\fi
}{%
\ifthmt@thisistheone
\csname #2\@xa\endcsname\ifx\@nx#1\@nx\else[{#1}]\fi
\else
\csname #2\@xa\endcsname[{Restated}]
\fi}{}{}
\makeatother

\newcommand{\creflastconjunction}{, and\nobreakspace}


\newcommand{\jingxunworking}{\begin{center}\color{blue} ------------------------------------WORKING------------------------------------ \end{center}}
\newcommand{\jingxun}[1]{{\color{blue}[Jingxun: #1]}}
\newcommand{\josh}[1]{{\color{red}[Josh: #1]}}
\newcommand{\defn}[1]{\textbf{\emph{#1}}}

\newcommand{\nnz}{\textup{nnz}}
\newcommand{\rank}[1][\F_p]{\textup{rank}^{#1}}
\newcommand{\sparsity}{\textup{sparsity}}
\renewcommand{\R}{\mathcal{R}}
\newcommand{\signR}{\text{sign-}\mathcal{R}}
\newcommand{\ip}{\text{IP}}
\newcommand{\Pcc}{\text{P}^{\text{cc}}}
\newcommand{\BPP}{\text{BPP}^{\text{cc}}}
\newcommand{\NP}{\text{NP}^{\text{cc}}}
\newcommand{\PH}{\text{PH}^{\text{cc}}}
\newcommand{\AM}{\text{AM}^{\text{cc}}}
\newcommand{\LTFoLTF}{\text{LTF} \circ \text{LTF}}
\newcommand{\bool}{\textup{bool}}
\newcommand{\sign}{\textup{sign}}

\renewcommand{\vec}[1]{\bm{\mathrm{#1}}}

\newcommand{\circledM}{\tikz[baseline=(char.base)]{\node[shape=circle,draw,inner sep=0.4pt,minimum size=0.4em] (char) {\fontsize{3}{4}\selectfont M};}}
\newcommand{\maj}[2][n]{#2^{\circledM #1}}
\newcommand{\kro}[2][n]{#2^{\otimes #1}}
\newcommand{\Maj}{\textup{Maj}}

\newcommand{\indicator}[1]{\mathbbm{1}{\Bk*{#1}}}
\newcommand{\boolRigidity}[2]{\R_{#1}^{\F_p^{\bool}}\bk*{#2}}

\newcommand{\pre}{\text{pre}}

\newcommand{\C}{\mathbb{C}}


\section{Introduction}
\label{sec:intro}

\begin{figure*}[tb]
    \centering
    \includegraphics[width=0.848\linewidth]{figs/circuitnn.pdf} 
    \caption{Illustration of differentiable CircuitNN. CircuitNN is designed based on differentiable NAND gates. After DAS is guided by PI and PO pairs of the truth table, CircuitNN can get the precise circuit architecture logic equivalent to the truth table.}
    \label{fig:circuitnn}
\end{figure*}

% 1. Describe the importance of logic synthesis
% 2. Existing Problems
% (a) Neural Architecture Search: Unstable, Predefined Setting, etc.
% (b) Circuit Generation: Probabilistic Model, Logic Equivalence

With the rapid advancement of technology, the scale of integrated circuits (ICs) has expanded exponentially. 
This expansion has introduced significant challenges in chip manufacturing, particularly concerning power and area metrics.
A primary objective in IC design is achieving the same circuit function with fewer transistors, thereby reducing power usage and area occupancy.

Logic synthesis~\cite{hachtel2005logicsynth}, a critical step in electronic design automation (EDA), transforms behavioral-level circuit designs into optimized gate-level circuits, ultimately yielding the final IC layout. 
The primary goal of logic synthesis is to identify the physical implementation with the fewest gates for a given circuit function. 
This task constitutes a challenging NP-hard combinatorial optimization problem. 
Current logic synthesis tools~\cite{brayton2010abc, wolf2013yosys} rely on human-designed heuristics, often leading to sub-optimal outcomes.

Differentiable architecture search (DAS) techniques~\cite{liu2018darts, chu2020darts} offer novel perspectives on addressing challenges in this problem.
Circuit functions can be represented through truth tables, which map binary inputs to their corresponding outputs. 
Truth tables provide a precise representation of input-output relationships, ensuring the design of functionally equivalent circuits.
Inspired by this, researchers~\cite{deepmind2024ai4sys, wang2024tnet} have begun exploring the application of DAS to synthesize circuits directly from truth tables.
Specifically, \citet{deepmind2024ai4sys} proposed CircuitNN, a framework that learns differentiable connection structures with logic gates, enabling the automatic generation of logic circuits from truth tables.
This approach significantly reduces the complexity of traditional circuit generation. 
Building on this, \citet{wang2024tnet} introduced T-Net, a triangle-shaped variant of CircuitNN, incorporating regularization techniques to enhance the efficiency of DAS.

Despite these advancements, several challenges remain. 
The computational complexity of DAS grows quadratically with the number of gates, posing scalability issues.
Although triangle-shaped architecture~\cite{wang2024tnet} partially mitigates this problem, redundancy persists. 
%Additionally, DAS is susceptible to converging to local optima, limiting the ability to search architectures that satisfy the given truth tables~\cite{liu2018darts}. 
%Furthermore, hyperparameters (network depth and layer width) require extensive searches, introducing complexity and prolonging the synthesis process. 
Additionally, DAS is susceptible to converging to local optima~\cite{liu2018darts} and hyperparameters (network depth and layer width) require extensive searches. 
The challenges arise from the vast search space in DAS. 
% Even with predefined settings for CircuitNN, finding a configuration that meets the truth table requires extensive trial and error during the DAS process. 
Intuitively, limiting the search space through predefined parameters (network depth, gates per layer, and connection probabilities) can significantly reduce the complexity.

Recent advances~\cite{openai2023gpt4, abramson2024alphafold3, esser2024sd3, li2024mar} in conditional generative models have demonstrated remarkable performance across language, vision, and graph generation tasks. 
Motivated by these developments, we propose a novel approach to circuit generation that generates preliminary circuit structures to guide DAS in generating refined circuits matching specified truth tables. 
Firstly, we introduce CircuitVQ, a tokenizer with a discrete codebook for circuit tokenization. 
Built upon our Circuit AutoEncoder framework~\cite{hou2022graphmae,li2023maskgae,wu2025mgvga}, CircuitVQ is trained through a circuit reconstruction task. 
Specifically, the CircuitVQ encoder encodes input circuits into discrete tokens using a learnable codebook, while the decoder reconstructs the circuit adjacency matrix based on these tokens.
Subsequently, the CircuitVQ encoder serves as a circuit tokenizer for CircuitAR pretraining, which employs a masked autoregressive modeling paradigm~\cite{chang2022maskgit, li2023mage}. 
In this process, the discrete codes function as supervision signals. 
After training, CircuitAR can generate discrete tokens progressively, which can be decoded into initial circuit structures by the decoder of the CircuitVQ. 
These prior insights can guide DAS in producing refined circuits that match the target truth tables precisely.

Our key contributions can be summarized as follows:
\begin{itemize}
\item We introduce CircuitVQ, a circuit tokenizer that facilitates graph autoregressive modeling for circuit generation, based on our Circuit AutoEncoder framework;
\item Develop CircuitAR, a model trained using masked autoregressive modeling, which generates initial circuit structures conditioned on given truth tables;
\item Propose a refinement framework that integrates differentiable architecture search to produce functionally equivalent circuits guided by target truth tables;
\item Comprehensive experiments demonstrating the scalability and capability emergence of our CircuitAR and the superior performance of the proposed circuit generation approach.
\end{itemize}

% Motivation
% (a) Diffusion (Vision, Graph), Autoregressive (Language, Vision)
% (b) Circuit Generation for Predefined Setting
% (c) Neural Architecture Search for Strict Logic Equivalence

% Contribution
% (a) Circuit Tokenizer (new transformer arch, training strategy)
% (b) CircuitAR (train and gen strategies, post-ar strategy)
% (c) Extensive Evaluation including BitD (Bit Distance) for Scalability


\section{Introduction}
\label{introduction}

The GPT-4 technical report of OpenAI states, ``We registered predictions for GPT-4’s performance on HumanEval before training completed, using only information available prior to training''~\cite{DBLP:journals/corr/abs-2107-03374, DBLP:journals/corr/abs-2303-08774}. However, the report lacks detailed explanations of the methodologies underlying this prediction technique.

This technique is critical for the pre-training of Large Language Models (LLMs). On the one hand, the cost of a single pre-training run can reach hundreds of millions of dollars, making re-pre-training a highly challenging endeavor~\cite{DBLP:journals/corr/abs-2401-04088, DBLP:journals/corr/abs-2407-10671, DBLP:journals/corr/abs-2407-21783}. The prediction technique, which allows for accurate performance forecasting on certain tasks based only on the pre-training data, helps mitigate the resource waste associated with unnecessary re-pre-training aimed at improving a model’s knowledge retention. On the other hand, it can guide the construction of pre-training data that are more aligned with target tasks, optimizing resource usage. This is particularly valuable when pre-training LLMs for specialized domains, as it reduces the creation of redundant data and minimizes unnecessary expenditures~\cite{Taoli-LLama, xiong2023doctorglm, li2024deeplearningllmbasedmethods}.

To achieve this vision, we focus on make predictions for the CBQA task, as it is closely linked to both the pre-training data and the model's knowledge retention~\cite{DBLP:conf/acl/WangL020, DBLP:conf/iclr/Sun0TYZ23}.
However, we encounter three major challenges (Figure~\ref{head}):

\textbf{Challenge1: Mastering the entire pre-training process, especially the construction of pre-training data.} Currently, most open-source base LLMs do not fully disclose their pre-training data, making it challenging to gain a comprehensive understanding of the datasets’ contents. Starting pre-training from scratch is prohibitively expensive, requiring a vast amount of data collection and substantial computational resources. For instance, Meta’s technical report reveals that pre-training the 405B-parameter Llama 3 model on 15.6 trillion tokens involved the use of 16k H100 GPUs, with the total cost approaching several hundred million dollars~\cite{DBLP:journals/corr/abs-2407-21783}.

\textbf{Challenge2: Evaluating whether the pre-trained model retains specific knowledge.} Based on the characteristics of the CBQA tasks, we can assess the model’s accuracy (ACC) on these tasks to evaluate its retention of knowledge. However, most evaluation methods face challenges, such as being overly sensitive to specific in-context examples and having coarse granularity in test data segmentation~\cite{DBLP:conf/emnlp/MinLHALHZ22, DBLP:conf/acl/SrivastavaGD024}. These issues make it difficult to accurately assess whether the pre-trained model has effectively retained specific knowledge.

\textbf{Challenge3: Predicting task-specific knowledge retention using only information available prior to training.} Solving target tasks relies on the model’s ability to learn world knowledge during pre-training, with this retention being strongly influenced by the data~\cite{DBLP:conf/icml/Allen-ZhuL24, DBLP:conf/emnlp/WangYXQD00GJX0C24}. During pre-training, the model generalizes knowledge by learning from terabytes of unlabeled data. However, there is currently no effective method to predict the retention of specific knowledge prior to training. Research from Meta and Palm suggests that analyzing the co-occurrence of questions and answers within the pre-training data might aid in making such predictions~\cite{DBLP:journals/corr/abs-2404-05405, DBLP:journals/jmlr/ChowdheryNDBMRBCSGSSTMRBTSPRDHPBAI23}. Nevertheless, this approach fails to account for the specificity of the knowledge and the impact of the model’s memory capacity on knowledge retention. For example, despite having the same co-occurrence frequency, a model might struggle to recall the location of Apple’s headquarters while easily recalling the location of NVIDIA’s headquarters.

Our work is primarily focused on effectively addressing these three challenges. An overview of our approach is provided in Figure~\ref{intro}. Specifically:

First, we allocate substantial resources to pre-training three base models with parameter sizes of 1.6B, 7B, and 13B, utilizing 1.5 trillion tokens of data. This process incurs a cost of 560k dollars and consumes 520k GPU hours. A detailed overview of the models’ overall performance is provided in Section~\ref{experimental_results}. With full access to the pre-training data, we are able to conduct an in-depth analysis and thoroughly evaluate the models’ performance on specific tasks.

Second, we utilize knowledge triples to perform a comprehensive retrieval and analysis of the pre-training data, focusing on specific CBQA tasks. This enables a detailed examination of the information embedded within the data. Additionally, inspired by the work of~\citet{DBLP:journals/corr/abs-2409-15825}, we implement a robust multi-template complementation mechanism to precisely assess the model’s knowledge retention.

Finally, we introduce the Size-dependent Mutual Information (SMI) metric, an information-theoretic approach to predict a model’s retention of specific knowledge using only information available prior to training. This method considers both the occurrence frequency and specificity of knowledge, as well as the model’s memory capacity. We conduct experiments on our three pre-trained models and the TinyLlama-1.1B model~\cite{DBLP:journals/corr/abs-2401-02385}. Our experiments show that the SMI metric effectively predicts knowledge retention, with a strong linear correlation between the SMI and ACC on CBQA tasks across various LLM sizes (i.e., 1.1B, 1.6B, 7B, and 13B), with Coefficient of Determination ($\text{R}^2$) values over 0.84.

% In addition, based on the characteristics of the SMI mertric and experimental results, we provide several recommendations for the pre-training phase. Regarding data composition, it is essential to consider not only the occurrence frequency of the data but also its specificity. Furthermore, in selecting the parameter size of the pre-trained model, attention should be given to the quality of the data. For details, please refer to Section~\ref{discussion}.

Overall, our contributions are threefold:
% Overall, our contributions are fourfold:
\begin{enumerate}
    \item We allocate substantial resources to pre-train three base models of varying sizes and release the weights and most of the pre-training data for the 1.6B model to support further research in this field.
    \item We propose an information-theoretic method and introduce the SMI metric, which accurately reflects a model's ability to retain task-specific knowledge solely based on the information available prior to training.
    \item Across LLMs of various sizes, the SMI metric demonstrates a strong linear correlation with ACC on CBQA tasks, achieving $\text{R}^2$ values greater than 0.84.
    % \item Based on our experimental results, we provide practical recommendations regarding the composition and quality of pre-training data, as well as the selection of model sizes.
\end{enumerate}