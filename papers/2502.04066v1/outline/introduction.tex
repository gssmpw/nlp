

\usepackage[american]{babel}
\usepackage{natbib}
    \bibliographystyle{plainnat}
    \renewcommand{\bibsection}{\subsubsection*{References}}
\usepackage{mathtools}
\usepackage{booktabs}
\usepackage{tikz}

% \usepackage[utf8]{inputenc} % allow utf-8 input
\usepackage[T1]{fontenc}    % use 8-bit T1 fonts
\usepackage{hyperref}       % hyperlinks
\usepackage{url}            % simple URL typesetting
\usepackage{amsfonts}       % blackboard math symbols
\usepackage{nicefrac}       % compact symbols for 1/2, etc.
\usepackage{microtype}      % microtypography
\usepackage{svg}
\usepackage{amsmath}
\usepackage{multirow}
\usepackage{centernot}
\usepackage{amsthm}
\usepackage{thmtools}
\usepackage{thm-restate}
\usepackage{cleveref}

\newtheorem{theorem}{Theorem}[section]
\newtheorem{corollary}{Corollary}[section]
\newtheorem{lemma}{Lemma}[section]
\newtheorem{proposition}{Proposition}[section]
\newtheorem{definition}{Definition}[section]
\newtheorem{example}{Example}[section]
\newenvironment{proofsketch}{%
  \renewcommand{\proofname}{Proof Sketch}\proof}{\endproof}

% Algorithms
\usepackage{algorithmicx}
\usepackage[noend]{algpseudocode}
\usepackage{algorithm}
% \usepackage[linesnumbered]{algorithm2e}

\usepackage{comment}

\usepackage{cases}

\usepackage{xcolor}
\usepackage{hyperref}
% \usepackage{graphicx}
\usepackage{subcaption}
\usepackage{paralist}
\usepackage{comment}

% CI
\newcommand{\indep}{\perp\kern-6pt\perp}
\newcommand{\dep}{\centernot{\perp\kern-6pt\perp}}
% Orientations
\newcommand{\starleft}{*\kern-5pt}
\newcommand{\starright}{\kern-5pt*}

\usepackage{acronym}
\acrodef{AID}{adjustment identification distance}
\acrodef{ATE}{average treatment effect}
\acrodef{CI}{conditional independence}
\acrodef{CPDAG}{complete partially directed acyclic graph}
\acrodef{DAG}{directed acyclic graph}
\acrodef{KCI}{Kernel-based conditional independence}
\acrodef{LCD}{local causal discovery}
\acrodef{MB}{Markov blanket}
\acrodef{MEC}{Markov equivalence class}
\acrodef{SHD}{structural Hamming distance}
\acrodef{SNAP}{Sequential Non-Ancestor Pruning}

\section{Introduction}


\begin{figure}[t]
\centering
\includegraphics[width=0.6\columnwidth]{figures/evaluation_desiderata_V5.pdf}
\vspace{-0.5cm}
\caption{\systemName is a platform for conducting realistic evaluations of code LLMs, collecting human preferences of coding models with real users, real tasks, and in realistic environments, aimed at addressing the limitations of existing evaluations.
}
\label{fig:motivation}
\end{figure}

\begin{figure*}[t]
\centering
\includegraphics[width=\textwidth]{figures/system_design_v2.png}
\caption{We introduce \systemName, a VSCode extension to collect human preferences of code directly in a developer's IDE. \systemName enables developers to use code completions from various models. The system comprises a) the interface in the user's IDE which presents paired completions to users (left), b) a sampling strategy that picks model pairs to reduce latency (right, top), and c) a prompting scheme that allows diverse LLMs to perform code completions with high fidelity.
Users can select between the top completion (green box) using \texttt{tab} or the bottom completion (blue box) using \texttt{shift+tab}.}
\label{fig:overview}
\end{figure*}

As model capabilities improve, large language models (LLMs) are increasingly integrated into user environments and workflows.
For example, software developers code with AI in integrated developer environments (IDEs)~\citep{peng2023impact}, doctors rely on notes generated through ambient listening~\citep{oberst2024science}, and lawyers consider case evidence identified by electronic discovery systems~\citep{yang2024beyond}.
Increasing deployment of models in productivity tools demands evaluation that more closely reflects real-world circumstances~\citep{hutchinson2022evaluation, saxon2024benchmarks, kapoor2024ai}.
While newer benchmarks and live platforms incorporate human feedback to capture real-world usage, they almost exclusively focus on evaluating LLMs in chat conversations~\citep{zheng2023judging,dubois2023alpacafarm,chiang2024chatbot, kirk2024the}.
Model evaluation must move beyond chat-based interactions and into specialized user environments.



 

In this work, we focus on evaluating LLM-based coding assistants. 
Despite the popularity of these tools---millions of developers use Github Copilot~\citep{Copilot}---existing
evaluations of the coding capabilities of new models exhibit multiple limitations (Figure~\ref{fig:motivation}, bottom).
Traditional ML benchmarks evaluate LLM capabilities by measuring how well a model can complete static, interview-style coding tasks~\citep{chen2021evaluating,austin2021program,jain2024livecodebench, white2024livebench} and lack \emph{real users}. 
User studies recruit real users to evaluate the effectiveness of LLMs as coding assistants, but are often limited to simple programming tasks as opposed to \emph{real tasks}~\citep{vaithilingam2022expectation,ross2023programmer, mozannar2024realhumaneval}.
Recent efforts to collect human feedback such as Chatbot Arena~\citep{chiang2024chatbot} are still removed from a \emph{realistic environment}, resulting in users and data that deviate from typical software development processes.
We introduce \systemName to address these limitations (Figure~\ref{fig:motivation}, top), and we describe our three main contributions below.


\textbf{We deploy \systemName in-the-wild to collect human preferences on code.} 
\systemName is a Visual Studio Code extension, collecting preferences directly in a developer's IDE within their actual workflow (Figure~\ref{fig:overview}).
\systemName provides developers with code completions, akin to the type of support provided by Github Copilot~\citep{Copilot}. 
Over the past 3 months, \systemName has served over~\completions suggestions from 10 state-of-the-art LLMs, 
gathering \sampleCount~votes from \userCount~users.
To collect user preferences,
\systemName presents a novel interface that shows users paired code completions from two different LLMs, which are determined based on a sampling strategy that aims to 
mitigate latency while preserving coverage across model comparisons.
Additionally, we devise a prompting scheme that allows a diverse set of models to perform code completions with high fidelity.
See Section~\ref{sec:system} and Section~\ref{sec:deployment} for details about system design and deployment respectively.



\textbf{We construct a leaderboard of user preferences and find notable differences from existing static benchmarks and human preference leaderboards.}
In general, we observe that smaller models seem to overperform in static benchmarks compared to our leaderboard, while performance among larger models is mixed (Section~\ref{sec:leaderboard_calculation}).
We attribute these differences to the fact that \systemName is exposed to users and tasks that differ drastically from code evaluations in the past. 
Our data spans 103 programming languages and 24 natural languages as well as a variety of real-world applications and code structures, while static benchmarks tend to focus on a specific programming and natural language and task (e.g. coding competition problems).
Additionally, while all of \systemName interactions contain code contexts and the majority involve infilling tasks, a much smaller fraction of Chatbot Arena's coding tasks contain code context, with infilling tasks appearing even more rarely. 
We analyze our data in depth in Section~\ref{subsec:comparison}.



\textbf{We derive new insights into user preferences of code by analyzing \systemName's diverse and distinct data distribution.}
We compare user preferences across different stratifications of input data (e.g., common versus rare languages) and observe which affect observed preferences most (Section~\ref{sec:analysis}).
For example, while user preferences stay relatively consistent across various programming languages, they differ drastically between different task categories (e.g. frontend/backend versus algorithm design).
We also observe variations in user preference due to different features related to code structure 
(e.g., context length and completion patterns).
We open-source \systemName and release a curated subset of code contexts.
Altogether, our results highlight the necessity of model evaluation in realistic and domain-specific settings.






\section{Introduction}
\label{introduction}

The GPT-4 technical report of OpenAI states, ``We registered predictions for GPT-4’s performance on HumanEval before training completed, using only information available prior to training''~\cite{DBLP:journals/corr/abs-2107-03374, DBLP:journals/corr/abs-2303-08774}. However, the report lacks detailed explanations of the methodologies underlying this prediction technique.

This technique is critical for the pre-training of Large Language Models (LLMs). On the one hand, the cost of a single pre-training run can reach hundreds of millions of dollars, making re-pre-training a highly challenging endeavor~\cite{DBLP:journals/corr/abs-2401-04088, DBLP:journals/corr/abs-2407-10671, DBLP:journals/corr/abs-2407-21783}. The prediction technique, which allows for accurate performance forecasting on certain tasks based only on the pre-training data, helps mitigate the resource waste associated with unnecessary re-pre-training aimed at improving a model’s knowledge retention. On the other hand, it can guide the construction of pre-training data that are more aligned with target tasks, optimizing resource usage. This is particularly valuable when pre-training LLMs for specialized domains, as it reduces the creation of redundant data and minimizes unnecessary expenditures~\cite{Taoli-LLama, xiong2023doctorglm, li2024deeplearningllmbasedmethods}.

To achieve this vision, we focus on make predictions for the CBQA task, as it is closely linked to both the pre-training data and the model's knowledge retention~\cite{DBLP:conf/acl/WangL020, DBLP:conf/iclr/Sun0TYZ23}.
However, we encounter three major challenges (Figure~\ref{head}):

\textbf{Challenge1: Mastering the entire pre-training process, especially the construction of pre-training data.} Currently, most open-source base LLMs do not fully disclose their pre-training data, making it challenging to gain a comprehensive understanding of the datasets’ contents. Starting pre-training from scratch is prohibitively expensive, requiring a vast amount of data collection and substantial computational resources. For instance, Meta’s technical report reveals that pre-training the 405B-parameter Llama 3 model on 15.6 trillion tokens involved the use of 16k H100 GPUs, with the total cost approaching several hundred million dollars~\cite{DBLP:journals/corr/abs-2407-21783}.

\textbf{Challenge2: Evaluating whether the pre-trained model retains specific knowledge.} Based on the characteristics of the CBQA tasks, we can assess the model’s accuracy (ACC) on these tasks to evaluate its retention of knowledge. However, most evaluation methods face challenges, such as being overly sensitive to specific in-context examples and having coarse granularity in test data segmentation~\cite{DBLP:conf/emnlp/MinLHALHZ22, DBLP:conf/acl/SrivastavaGD024}. These issues make it difficult to accurately assess whether the pre-trained model has effectively retained specific knowledge.

\textbf{Challenge3: Predicting task-specific knowledge retention using only information available prior to training.} Solving target tasks relies on the model’s ability to learn world knowledge during pre-training, with this retention being strongly influenced by the data~\cite{DBLP:conf/icml/Allen-ZhuL24, DBLP:conf/emnlp/WangYXQD00GJX0C24}. During pre-training, the model generalizes knowledge by learning from terabytes of unlabeled data. However, there is currently no effective method to predict the retention of specific knowledge prior to training. Research from Meta and Palm suggests that analyzing the co-occurrence of questions and answers within the pre-training data might aid in making such predictions~\cite{DBLP:journals/corr/abs-2404-05405, DBLP:journals/jmlr/ChowdheryNDBMRBCSGSSTMRBTSPRDHPBAI23}. Nevertheless, this approach fails to account for the specificity of the knowledge and the impact of the model’s memory capacity on knowledge retention. For example, despite having the same co-occurrence frequency, a model might struggle to recall the location of Apple’s headquarters while easily recalling the location of NVIDIA’s headquarters.

Our work is primarily focused on effectively addressing these three challenges. An overview of our approach is provided in Figure~\ref{intro}. Specifically:

First, we allocate substantial resources to pre-training three base models with parameter sizes of 1.6B, 7B, and 13B, utilizing 1.5 trillion tokens of data. This process incurs a cost of 560k dollars and consumes 520k GPU hours. A detailed overview of the models’ overall performance is provided in Section~\ref{experimental_results}. With full access to the pre-training data, we are able to conduct an in-depth analysis and thoroughly evaluate the models’ performance on specific tasks.

Second, we utilize knowledge triples to perform a comprehensive retrieval and analysis of the pre-training data, focusing on specific CBQA tasks. This enables a detailed examination of the information embedded within the data. Additionally, inspired by the work of~\citet{DBLP:journals/corr/abs-2409-15825}, we implement a robust multi-template complementation mechanism to precisely assess the model’s knowledge retention.

Finally, we introduce the Size-dependent Mutual Information (SMI) metric, an information-theoretic approach to predict a model’s retention of specific knowledge using only information available prior to training. This method considers both the occurrence frequency and specificity of knowledge, as well as the model’s memory capacity. We conduct experiments on our three pre-trained models and the TinyLlama-1.1B model~\cite{DBLP:journals/corr/abs-2401-02385}. Our experiments show that the SMI metric effectively predicts knowledge retention, with a strong linear correlation between the SMI and ACC on CBQA tasks across various LLM sizes (i.e., 1.1B, 1.6B, 7B, and 13B), with Coefficient of Determination ($\text{R}^2$) values over 0.84.

% In addition, based on the characteristics of the SMI mertric and experimental results, we provide several recommendations for the pre-training phase. Regarding data composition, it is essential to consider not only the occurrence frequency of the data but also its specificity. Furthermore, in selecting the parameter size of the pre-trained model, attention should be given to the quality of the data. For details, please refer to Section~\ref{discussion}.

Overall, our contributions are threefold:
% Overall, our contributions are fourfold:
\begin{enumerate}
    \item We allocate substantial resources to pre-train three base models of varying sizes and release the weights and most of the pre-training data for the 1.6B model to support further research in this field.
    \item We propose an information-theoretic method and introduce the SMI metric, which accurately reflects a model's ability to retain task-specific knowledge solely based on the information available prior to training.
    \item Across LLMs of various sizes, the SMI metric demonstrates a strong linear correlation with ACC on CBQA tasks, achieving $\text{R}^2$ values greater than 0.84.
    % \item Based on our experimental results, we provide practical recommendations regarding the composition and quality of pre-training data, as well as the selection of model sizes.
\end{enumerate}