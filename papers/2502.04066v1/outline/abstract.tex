\begin{abstract}  
The GPT-4 technical report from OpenAI suggests that model performance on specific tasks can be predicted prior to training, though methodologies remain unspecified. This approach is crucial for optimizing resource allocation and ensuring data alignment with target tasks.
To achieve this vision, we focus on predicting performance on Closed-book Question Answering (CBQA) tasks, which are closely tied to pre-training data and knowledge retention. We address three major challenges: 1) mastering the entire pre-training process, especially data construction; 2) evaluating a model’s knowledge retention; and 3) predicting task-specific knowledge retention using only information available prior to training.
To tackle these challenges, we pre-train three large language models (i.e., 1.6B, 7B, and 13B) using 560k dollars and 520k GPU hours. We analyze the pre-training data with knowledge triples and assess knowledge retention using established methods. Additionally, we introduce the \textbf{SMI} metric, an information-theoretic measure that quantifies the relationship between pre-training data, model size, and task-specific knowledge retention.
Our experiments reveal a strong linear correlation ($\text{R}^2 > 0.84$) between the SMI metric and the model’s accuracy on CBQA tasks across models of varying sizes (i.e., 1.1B, 1.6B, 7B, and 13B). The dataset, model, and code are available at \href{https://github.com/yuhui1038/SMI}{https://github.com/yuhui1038/SMI}.
\end{abstract}