
\section{Templates for Knowledge Triples}
\begin{table}[!ht]
    \centering
    \small
    \caption{The prompt template for classifying the topic and format of a web page. The first two row shows the templates for system and user prompts, in which {\tt$\{$domain$\}$} becomes either ``topic'' and ``format'' and {\tt$\{$instructions$\}$} are substituted with the content of the bottom two rows.}
    \icmlskip{0.1in}
\begin{tabular}{lp{0.8\textwidth}}
\toprule
& \multicolumn{1}{c}{Prompt templates} \\
\midrule
System & {\tt Your task is to classify the $\{$domain$\}$ of web pages into one of the following 24 categories:} \\
    & {\tt $\{$choices$\}$ } \\
    & \\
    & {\tt $\{$instructions$\}$} \\
\midrule
User & {\tt Consider the following web page:} \\
    & \\
    & {\tt URL: `$\{$url$\}$`} \\
    & {\tt Content: ```} \\
    & {\tt $\{$text$\}$} \\
    & {\tt ```} \\
    & \\
    & {\tt Your task is to classify the $\{$domain$\}$ of web pages into one of the following 24 categories:} \\
    & {\tt $\{$choices$\}$ } \\
    & \\
    & {\tt $\{$instructions$\}$} \\
\midrule
& \multicolumn{1}{c}{Instructions} \\
\midrule
{\topics Topic} & \tt Choose which topic from the above list is the best match for describing what the web page content is about. If the content is about multiple topics, choose the one that is most prominent.
Remember to focus on the topic, and not the format, e.g., a book excerpt about a first date is related to `Social Life' and not `Literature'.
The URL might help you understand the content. Avoid shortcuts such as word overlap between the page and the topic descriptions or simple patterns in the URL.
Start your response with the single-letter ID of the correct topic followed by an explanation. \\
\midrule
{\formats Format} & \tt
Choose which format from the above list is the best match for describing the style, purpose and origin of the web page content. If the content has multiple formats, choose the one that is most prominent.
Remember to focus on the format, and not the topic, e.g., a research paper about legal issues does not count as `Legal Notices'.
The URL might help you understand the content. Avoid shortcuts such as word overlap between the page and the format descriptions or simple patterns in the URL, for example `.../blog/...' may also occur for organizational announcements, comment sections, and other formats.
Start your response with the single-letter ID of the correct format followed by an explanation. \\
\bottomrule
\end{tabular}
    \icmlskip{-0.1in}
    \label{tab:templates}
\end{table}

We create twenty templates for each type of knowledge triple. Table~\ref{templates} illustrates the templates for the \(\mathbf{BORN}\) relation, along with their corresponding evaluation ACC on the 13B model. As shown, the choice of template has a significant impact on the model’s ACC in answering questions. To address the instability of evaluations based on a single template, we calculate the average accuracy across all 20 templates.



\section{Case Study}

\begin{table*}[htbp]
    \centering
    \small
    \begin{tabular}{p{14cm}}
     \toprule
\#\#\#  Objective: \\
Generate a 5-day family travel itinerantry that satisfies all specified requirements while adhering to highly fine-grained constraints. The generated itinerary should balance real-time adaptability, strict hard attributes, and semantic soft attributes. \\

\#\#\# User Profile: \\
 - Travelers: 2 adults + 1 child (age 8) \\
 - Budget: $<=$ \$300/day (total \$1,500 for the trip) \\
 - Activity Balance: 70\% educational/cultural experiences, 20\% relaxation, 10\% family-friendly shopping. \\

\#\#\# Hard Attributes: \\
- Activity Scheduling: \\
\quad- Each activity must have a defined start and end time, ensuring there is no overlap between activities. \\
\quad- A break period from 13:00-14:30 is mandatory daily. \\
\quad- Each activity must fit within a 2-hour window unless otherwise specified. \\

- Budget Requirements: \\
\quad- Each day’s total cost (including transportation, food, and activities) must not exceed \$300. \\
\quad- Transportation is limited to metro and walking only, with a maximum of 3 metro rides per day. \\

- Location Constraints: \\
\quad- Must-visit locations: City Zoo (Day 1) and Science Museum (Day 3). \\
\quad- Activities must occur in geographically adjacent areas to minimize walking distance. \\

- Keyword Requirements: \\
\quad- Each day’s description must include specific keywords. For example: \\
\quad- Day 1: “wildlife,” “exploration,” and “interactive learning.” \\
\quad- Day 3: “science,” “innovation,” and “hands-on exhibits.” \\

- Structure Constraints: \\
\quad- Each day’s itinerary must consist of 4 sections: \\
\quad\quad- Morning activity \\
\quad\quad- Break/lunch period \\ 
\quad\quad- Afternoon activity \\
\quad\quad- Evening summary (limited to 50 words) \\

\#\#\# Soft Attributes \\
- Tone and Emotion: \\ 
\quad- Day 1: Use a tone that conveys “excitement and discovery.” \\ 
\quad- Day 3: Use a tone that conveys “curiosity and wonder.” \\
- Language Style: \\ 
\quad- Use descriptive, vivid, and family-friendly language throughout. \\
\quad- Include at least one metaphor or simile per day (e.g., "The Science Museum felt like stepping into the future!"). \\
- Visual Details: \\
\quad- Each activity must include specific sensory details (e.g., "the bright colors of the parrots at the zoo" or "the tinkling sound of water fountains at the park").

- Adaptive Adjustments (Real-time Constraints): \\
\quad- Weather Sensitivity: \\
\quad\quad- If the rain forecast exceeds 60\%, replace outdoor activities with indoor alternatives while keeping the overall tone and keywords intact. \\ 
\quad- Physical Endurance: \\
\quad\quad- If a day’s total walking distance exceeds 10 kilometers, the next day’s activities must reduce walking by 30\%. \\
\quad- Health Responsiveness: \\
\quad\quad- If a health-related issue arises (e.g., fatigue or illness), adjust the itinerary dynamically to: \\
\quad\quad- Reduce activity duration to half. \\ 
\quad\quad- Substitute the activity with a more relaxing or passive option. \\
\bottomrule
    \end{tabular}
    \caption{The complete travel planner case study.}
    \label{tab:travel_planner_case}
\end{table*}
Table~\ref{case_study} highlights four counterintuitive cases. From these, we observe the following: (1) In the first two cases, the subject and object rarely co-occur, suggesting a seemingly weak relationship. However, whenever the subject does appear, the object almost always appears alongside it, resulting in high accuracy. (2) In the last two cases, despite frequent co-occurrence between the subject and object, their individual occurrence frequencies are significantly higher, leading to low accuracy. For example, when prompted with ``Paris is located in'', the model is more likely to respond with ``France'' rather than ``Europe''.

These examples highlight the pivotal role of knowledge specificity in the memory of LLMs. It is not simply the frequencies of knowledge occurrence that enhance retention, but rather the specificity of the knowledge that proves more influential. This insight suggests that in downstream task training, even when domain-specific data is significantly fewer than the pre-training data, minimizing repetition with the pre-training data can increase the SMI. This makes it easier for LLMs to retain the relevant knowledge.


\section{Experimental Results on Each Relation}
\begin{figure*}[h]
\vskip 0.2in
\begin{center}
\begin{minipage}[b]{0.49\textwidth} % 左侧
    \centering
    \includegraphics[width=\textwidth]{pictures/relations_R2.pdf}\\
    (a) $\text{R}^2$ ($\uparrow$)
\end{minipage}
\hfill
\begin{minipage}[b]{0.49\textwidth} % 右侧
    \centering
    \includegraphics[width=\textwidth]{pictures/relations_MSE.pdf}\\
    (b) MSE ($\downarrow$)
\end{minipage}
\caption{Results of the 13B model across 15 relations: (a) $\text{R}^2$ ($\uparrow$ higher is better) and (b) MSE ($\downarrow$ lower is better).}
\label{15relations}
\end{center}
% \vskip -0.2in
\end{figure*}

Figure~\ref{15relations} illustrates the evaluation results for each knowledge triple relation in the 13B model. We observe that: (1) The $\text{R}^2$ for individual relations is lower than that of the entire evaluation set, probably due to insufficient data for specific relations. (2) The results vary significantly across different relations, indicating that when applying our method, it is essential to account for biases introduced by the relation types of the knowledge triples.

