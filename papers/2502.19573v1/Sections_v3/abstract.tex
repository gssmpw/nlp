\begin{abstract}
Large Language Models (LLMs) have emerged as highly capable systems and are increasingly being integrated into various uses. However, the rapid pace of their deployment has outpaced a comprehensive understanding of their internal mechanisms and a delineation of their capabilities and limitations. A desired attribute of an intelligent system is its ability to recognize the scope of its own knowledge. To investigate whether LLMs embody this characteristic, we develop a benchmark designed to challenge these models to enumerate all information they possess on specific topics. This benchmark evaluates whether the models recall excessive, insufficient, or the precise amount of information, thereby indicating their awareness of their own knowledge. Our findings reveal that all tested LLMs, given sufficient scale, demonstrate an understanding of how much they know about specific topics. While different architectures exhibit varying rates of this capability’s emergence, the results suggest that awareness of knowledge may be a generalizable attribute of LLMs. Further research is needed to confirm this potential and fully elucidate the underlying mechanisms.
\end{abstract}

