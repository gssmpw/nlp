\section{Traffic Generation}
In order to have a wide variety of benign traffic and attack traffic, we developed a traffic generator based on Python. We decided to develop our own generator that allows us to orchestrate multiple attacks and also according benign traffic. Considering a company network, users will have benign interaction with the services provided on a network, these benign traffic usually mixes up with the traffic of attackers trying to attack such services. This overlap creates a realistic challenge for intrusion detection systems to detect the attacks inside the normal, benign noise. By implementing our own traffic generator, we can utilize the generator to execute benign actions that are closer to the attack patterns, e.g., utilizing the same web server endpoints. The generator supports multiple protocols, including HTTP(S), FTP, SMTP, SSH, ICMP, DNS, TCP, and UDP, ensuring broad coverage of typical network traffic scenarios. Additionally, the generator applies randomization techniques to introduce variability in request timings and traffic patterns, ensuring that generated traffic does not rely on static characteristics. This helps mitigate model overfitting to features like precise timing patterns that follow the specific generation process. Currently, the traffic generator is available on request to other researchers, it is planned to release it open source.

In the following, we will describe the two primary modes of the traffic generator: benign mode, where different actions are randomly triggered to simulate normal user behavior, and attack mode, where 13 different attack types can be executed.

\subsection{Benign Mode}
In benign mode, the traffic generator simulates normal user behavior across various protocols, including web browsing (HTTP(S)), file transfers via FTP, email exchanges through SMTP, and remote server interactions over SSH.

The traffic generator simulates benign HTTP(S) interactions using automated bots and a web crawler to replicate realistic web browsing behavior. Bots perform randomized actions on an OWASP Juice Shop instance, such as browsing pages, submitting feedback, and logging in or out. The web crawler complements this by navigating internal links from a list of URLs, scraping page content, and updating a graph-based crawl frontier to emulate natural user navigation. Both components introduce variability through randomized actions and delays. 
Similarly, benign FTP interactions involve randomized directory navigation, file uploads, and downloads, with variability in file names, sizes, permissions, and login attempts.
For SMTP, the traffic generator simulates randomized email exchanges by varying the subject, body length, and recipient count.
Finally, SSH interactions involve establishing remote sessions and executing random commands on servers, including delays and optionally simulated failures, creating realistic server access behaviors.

\subsection{Attack Mode}
In attack mode, the traffic generator simulates 13 different malicious activities across supported protocols. Table~\ref{tab:attacks} shows the different attacks grouped by the service.

For HTTP(S), various web-based attacks are executed on an OWASP Juice Shop instance, including SQL injection (targeting login and search functionalities, with optional payload obfuscation), cross-site scripting (XSS) through feedback forms, denial of service (DoS) via sqlmap, brute force login attempts using Hydra, server-side request forgery (SSRF) targeting predefined URLs, and reverse shell exploits leveraging server-side template injection (SSTI) to execute commands on the victim server. The implemented FTP attacks include fingerprinting via version detection and brute force attempts with randomized or injected credentials to simulate successful and unsuccessful login scenarios. For the attacks on the SMTP protocol, we focus on enumeration of users and fingerprinting of the mail server configuration. Furthermore, SSH attacks use brute force to compromise login credentials, incorporating successful breaches. Finally, miscellaneous attacks include host sweeps and port scans, with parameters such as target ranges randomized for variability. Each attack dynamically adjusts parameters, timing, and target specifics to ensure realistic simulation.

\begin{table}[]
\caption{Implemented attacks in the traffic generator}
\label{tab:attacks}
\resizebox{\columnwidth}{!}{%
\begin{tabular}{|l|l|l|l|}
\hline
\multicolumn{1}{|c|}{\textbf{Service}} & \multicolumn{1}{c|}{\textbf{Attack}} & \multicolumn{1}{c|}{\textbf{Alias}} & \multicolumn{1}{c|}{\textbf{Tool(s)}}                    \\ \hline
\textbf{FTP}                           &                                      &                                     &                                                          \\ \hline
                                       & Fingerprinting                       & ftp\_version                        & Metasploit (auxiliary/scanner/ftp/ftp\_version)   \\ \hline
                                       & Bruteforce                           & ftp\_login                          & Metasploit (auxiliary/scanner/ftp/ftp\_login)     \\ \hline
\textbf{HTTP/S}          &                                      &                                     &                                                          \\ \hline
                                       & Cross-site scripting         & xss                                 & Selenium WebDriver                                       \\ \hline
                                       & SQL-Injection                        & sqli                                & Selenium WebDriver, python-requests, sqlmap              \\ \hline
                                       & Denial of Service              & dos                                 & sqlmap                                                   \\ \hline
                                       & Bruteforce                           & bruteforce                          & Hydra                                                    \\ \hline
                                       & Server-side request forgery   & ssrf                                & Selenium WebDriver                                       \\ \hline
                                       & Reverse Shell                        & revshell                            & Selenium WebDriver, netcat                               \\ \hline
\textbf{SMTP}                          &                                      &                                     &                                                          \\ \hline
                                       & Fingerprinting                       & smtp\_version                       & Metasploit (auxiliary/scanner/smtp/smtp\_version) \\ \hline
                                       & User Enumeration                     & smtp\_enum                          & Metasploit (auxiliary/scanner/smtp/smtp\_enum)    \\ \hline
\textbf{SSH}                           &                                      &                                     &                                                          \\ \hline
                                       & Bruteforce                           & ssh\_login                          & Metasploit (auxiliary/scanner/ssh/ssh\_login)     \\ \hline
\textbf{Misc}  &                                      &                                     &                                                          \\ \hline
                                       & Portscan                             & portscan                            & Nmap (-sS flag)                                 \\ \hline
                                       & Hostsweep                            & hostsweep                           & Nmap (-sn and -Pn flags)                                          \\ \hline
\end{tabular}%
}
\end{table}