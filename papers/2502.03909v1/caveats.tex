\section{Notes on interpreting classifier results}
\label{sec:notes}
Some of the attacks presented in this dataset are not detectable by inspecting a single flow. Meaning, when a flow-based classifier detects such an attack correctly, it is likely that an overfitting issue occurred. The attacks \texttt{revshell} and \texttt{Server-side request forgery} are usually only successful when the victim server creates a new connection to a host specified in one of the attacks. This results in that proper detection is only guaranteed when at least two flows are analyzed. By only inspecting a single flow, the encrypted payload sent to a server should not be differentiable from a benign payload. Identifying these attacks requires either the inspection of unencrypted payloads or the inspection of resulting flows. However, during capturing the dataset we did not include a mechanism, that would be able to indicate, whether a stream resulted from a previous stream. With some uncertainty, it probably could be inferred, by inspecting the flows and matching on the flow that goes from the victim's IP to the attacker's IP, after a \texttt{revshell} samples is observed. For \texttt{ssrf} the target was randomly chosen out of a list of 5 hosts on the public internet, typically our victim server is not performing web requests. Therefore, this could be used as an indicator.