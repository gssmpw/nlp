\documentclass[10pt,conference,letterpaper]{IEEEtran}
\IEEEoverridecommandlockouts
\usepackage{cite}
\usepackage{amsmath,amssymb,amsfonts}
%\usepackage{algorithm}
\usepackage{graphicx}
\usepackage{algorithmicx}
\usepackage{booktabs}
%\usepackage{tabular}
\usepackage[noend]{algpseudocode}
\usepackage{textcomp}
\usepackage{xcolor}
\usepackage{enumitem}
\usepackage{array}
\usepackage{footnote}
\usepackage{url}
%\usepackage{tabularx}
\usepackage{balance}
\usepackage{multirow}
\usepackage{hyperref}
\usepackage{gensymb}

%Layouting
\clubpenalty = 10000
\widowpenalty = 10000 
\displaywidowpenalty = 10000

\emergencystretch=20pt

%% Anteil an Seite ab der Figure eine einzelne Seite erhaelt
\renewcommand{\floatpagefraction}{.8}
%% Mindest-Anteil Text pro Seite
%\renewcommand{\textfraction}{.01}
\renewcommand{\topmargin}{-.7in}

%Maximaler Anteil der Seite, die fuer Abbildungen, die `t'op bzw. `b'ottom
%plaziert werden, verwendet werden darf. Empfehlenswert sind Werte
%zwischen 50% und 85% fuer \topfraction und 20% bis 50% fuer
%\bottomfraction. 
%Einer dieser beiden Werte sollte stets groesser als \floatpagefraction
%sein! 
  \renewcommand{\topfraction}{0.8}     % vorher: .7
  \renewcommand{\bottomfraction}{.8}  % vorher: .3

%Maximale Anzahl der Abbildungen und Tafeln, die auf einer Seite `t'op
%bzw. `b'ottom bzw. insgesamt auf der Seite plaziert werden.
  \setcounter{topnumber}{3} % vorher: 2
  \setcounter{bottomnumber}{3} % vorher: 1
  \setcounter{totalnumber}{5} % vorher: 3


\makeatletter
\newcommand\fs@spaceruled{\def\@fs@cfont{\bfseries}\let\@fs@capt\floatc@ruled
  \def\@fs@pre{\vspace*{0.2cm}\hrule height.8pt depth0pt \kern2pt}%
  \def\@fs@post{\kern2pt\hrule\relax}%
  \def\@fs@mid{\kern2pt\hrule\kern2pt}%
  \let\@fs@iftopcapt\iftrue}
\makeatother

\begin{document}

\title{Technical Report:\\ Generating the WEB-IDS23 Dataset}

\author{ 
Eric Lanfer, Dominik Brockmann,  Nils Aschenbruck\\

Osnabrück University - Institute of Computer Science, Osnabrück, Germany\\
\{lanfer, dobrockmann, aschenbruck\}@uos.de\\
}
\IEEEoverridecommandlockouts



%\author{Anonymous authors}

\maketitle

\begin{abstract}
Anomaly-based Network Intrusion Detection Systems (NIDS) require correctly labelled, representative and diverse datasets for an accurate evaluation and development. However, several widely used datasets do not include labels which are fine-grained enough and, together with small sample sizes, can lead to overfitting issues that also remain undetected when using test data. Additionally, the cybersecurity sector is evolving fast, and new attack mechanisms require the continuous creation of up-to-date datasets. To address these limitations, we developed a modular traffic generator that can simulate a wide variety of benign and malicious traffic. It incorporates multiple protocols, variability through randomization techniques and can produce attacks along corresponding benign traffic, as it occurs in real-world scenarios. Using the traffic generator, we create a dataset capturing over 12 million samples with 82 flow-level features and 21 fine-grained labels. Additionally, we include several web attack types which are often underrepresented in other datasets.
\end{abstract}

\begin{IEEEkeywords}
Dataset, Network Intrusion Detection, Machine Learning, Web Attacks
\end{IEEEkeywords}
\section{Introduction}
\label{sec:introduction}
The business processes of organizations are experiencing ever-increasing complexity due to the large amount of data, high number of users, and high-tech devices involved \cite{martin2021pmopportunitieschallenges, beerepoot2023biggestbpmproblems}. This complexity may cause business processes to deviate from normal control flow due to unforeseen and disruptive anomalies \cite{adams2023proceddsriftdetection}. These control-flow anomalies manifest as unknown, skipped, and wrongly-ordered activities in the traces of event logs monitored from the execution of business processes \cite{ko2023adsystematicreview}. For the sake of clarity, let us consider an illustrative example of such anomalies. Figure \ref{FP_ANOMALIES} shows a so-called event log footprint, which captures the control flow relations of four activities of a hypothetical event log. In particular, this footprint captures the control-flow relations between activities \texttt{a}, \texttt{b}, \texttt{c} and \texttt{d}. These are the causal ($\rightarrow$) relation, concurrent ($\parallel$) relation, and other ($\#$) relations such as exclusivity or non-local dependency \cite{aalst2022pmhandbook}. In addition, on the right are six traces, of which five exhibit skipped, wrongly-ordered and unknown control-flow anomalies. For example, $\langle$\texttt{a b d}$\rangle$ has a skipped activity, which is \texttt{c}. Because of this skipped activity, the control-flow relation \texttt{b}$\,\#\,$\texttt{d} is violated, since \texttt{d} directly follows \texttt{b} in the anomalous trace.
\begin{figure}[!t]
\centering
\includegraphics[width=0.9\columnwidth]{images/FP_ANOMALIES.png}
\caption{An example event log footprint with six traces, of which five exhibit control-flow anomalies.}
\label{FP_ANOMALIES}
\end{figure}

\subsection{Control-flow anomaly detection}
Control-flow anomaly detection techniques aim to characterize the normal control flow from event logs and verify whether these deviations occur in new event logs \cite{ko2023adsystematicreview}. To develop control-flow anomaly detection techniques, \revision{process mining} has seen widespread adoption owing to process discovery and \revision{conformance checking}. On the one hand, process discovery is a set of algorithms that encode control-flow relations as a set of model elements and constraints according to a given modeling formalism \cite{aalst2022pmhandbook}; hereafter, we refer to the Petri net, a widespread modeling formalism. On the other hand, \revision{conformance checking} is an explainable set of algorithms that allows linking any deviations with the reference Petri net and providing the fitness measure, namely a measure of how much the Petri net fits the new event log \cite{aalst2022pmhandbook}. Many control-flow anomaly detection techniques based on \revision{conformance checking} (hereafter, \revision{conformance checking}-based techniques) use the fitness measure to determine whether an event log is anomalous \cite{bezerra2009pmad, bezerra2013adlogspais, myers2018icsadpm, pecchia2020applicationfailuresanalysispm}. 

The scientific literature also includes many \revision{conformance checking}-independent techniques for control-flow anomaly detection that combine specific types of trace encodings with machine/deep learning \cite{ko2023adsystematicreview, tavares2023pmtraceencoding}. Whereas these techniques are very effective, their explainability is challenging due to both the type of trace encoding employed and the machine/deep learning model used \cite{rawal2022trustworthyaiadvances,li2023explainablead}. Hence, in the following, we focus on the shortcomings of \revision{conformance checking}-based techniques to investigate whether it is possible to support the development of competitive control-flow anomaly detection techniques while maintaining the explainable nature of \revision{conformance checking}.
\begin{figure}[!t]
\centering
\includegraphics[width=\columnwidth]{images/HIGH_LEVEL_VIEW.png}
\caption{A high-level view of the proposed framework for combining \revision{process mining}-based feature extraction with dimensionality reduction for control-flow anomaly detection.}
\label{HIGH_LEVEL_VIEW}
\end{figure}

\subsection{Shortcomings of \revision{conformance checking}-based techniques}
Unfortunately, the detection effectiveness of \revision{conformance checking}-based techniques is affected by noisy data and low-quality Petri nets, which may be due to human errors in the modeling process or representational bias of process discovery algorithms \cite{bezerra2013adlogspais, pecchia2020applicationfailuresanalysispm, aalst2016pm}. Specifically, on the one hand, noisy data may introduce infrequent and deceptive control-flow relations that may result in inconsistent fitness measures, whereas, on the other hand, checking event logs against a low-quality Petri net could lead to an unreliable distribution of fitness measures. Nonetheless, such Petri nets can still be used as references to obtain insightful information for \revision{process mining}-based feature extraction, supporting the development of competitive and explainable \revision{conformance checking}-based techniques for control-flow anomaly detection despite the problems above. For example, a few works outline that token-based \revision{conformance checking} can be used for \revision{process mining}-based feature extraction to build tabular data and develop effective \revision{conformance checking}-based techniques for control-flow anomaly detection \cite{singh2022lapmsh, debenedictis2023dtadiiot}. However, to the best of our knowledge, the scientific literature lacks a structured proposal for \revision{process mining}-based feature extraction using the state-of-the-art \revision{conformance checking} variant, namely alignment-based \revision{conformance checking}.

\subsection{Contributions}
We propose a novel \revision{process mining}-based feature extraction approach with alignment-based \revision{conformance checking}. This variant aligns the deviating control flow with a reference Petri net; the resulting alignment can be inspected to extract additional statistics such as the number of times a given activity caused mismatches \cite{aalst2022pmhandbook}. We integrate this approach into a flexible and explainable framework for developing techniques for control-flow anomaly detection. The framework combines \revision{process mining}-based feature extraction and dimensionality reduction to handle high-dimensional feature sets, achieve detection effectiveness, and support explainability. Notably, in addition to our proposed \revision{process mining}-based feature extraction approach, the framework allows employing other approaches, enabling a fair comparison of multiple \revision{conformance checking}-based and \revision{conformance checking}-independent techniques for control-flow anomaly detection. Figure \ref{HIGH_LEVEL_VIEW} shows a high-level view of the framework. Business processes are monitored, and event logs obtained from the database of information systems. Subsequently, \revision{process mining}-based feature extraction is applied to these event logs and tabular data input to dimensionality reduction to identify control-flow anomalies. We apply several \revision{conformance checking}-based and \revision{conformance checking}-independent framework techniques to publicly available datasets, simulated data of a case study from railways, and real-world data of a case study from healthcare. We show that the framework techniques implementing our approach outperform the baseline \revision{conformance checking}-based techniques while maintaining the explainable nature of \revision{conformance checking}.

In summary, the contributions of this paper are as follows.
\begin{itemize}
    \item{
        A novel \revision{process mining}-based feature extraction approach to support the development of competitive and explainable \revision{conformance checking}-based techniques for control-flow anomaly detection.
    }
    \item{
        A flexible and explainable framework for developing techniques for control-flow anomaly detection using \revision{process mining}-based feature extraction and dimensionality reduction.
    }
    \item{
        Application to synthetic and real-world datasets of several \revision{conformance checking}-based and \revision{conformance checking}-independent framework techniques, evaluating their detection effectiveness and explainability.
    }
\end{itemize}

The rest of the paper is organized as follows.
\begin{itemize}
    \item Section \ref{sec:related_work} reviews the existing techniques for control-flow anomaly detection, categorizing them into \revision{conformance checking}-based and \revision{conformance checking}-independent techniques.
    \item Section \ref{sec:abccfe} provides the preliminaries of \revision{process mining} to establish the notation used throughout the paper, and delves into the details of the proposed \revision{process mining}-based feature extraction approach with alignment-based \revision{conformance checking}.
    \item Section \ref{sec:framework} describes the framework for developing \revision{conformance checking}-based and \revision{conformance checking}-independent techniques for control-flow anomaly detection that combine \revision{process mining}-based feature extraction and dimensionality reduction.
    \item Section \ref{sec:evaluation} presents the experiments conducted with multiple framework and baseline techniques using data from publicly available datasets and case studies.
    \item Section \ref{sec:conclusions} draws the conclusions and presents future work.
\end{itemize}
\section{Traffic Generation}
In order to have a wide variety of benign traffic and attack traffic, we developed a traffic generator based on Python. We decided to develop our own generator that allows us to orchestrate multiple attacks and also according benign traffic. Considering a company network, users will have benign interaction with the services provided on a network, these benign traffic usually mixes up with the traffic of attackers trying to attack such services. This overlap creates a realistic challenge for intrusion detection systems to detect the attacks inside the normal, benign noise. By implementing our own traffic generator, we can utilize the generator to execute benign actions that are closer to the attack patterns, e.g., utilizing the same web server endpoints. The generator supports multiple protocols, including HTTP(S), FTP, SMTP, SSH, ICMP, DNS, TCP, and UDP, ensuring broad coverage of typical network traffic scenarios. Additionally, the generator applies randomization techniques to introduce variability in request timings and traffic patterns, ensuring that generated traffic does not rely on static characteristics. This helps mitigate model overfitting to features like precise timing patterns that follow the specific generation process. Currently, the traffic generator is available on request to other researchers, it is planned to release it open source.

In the following, we will describe the two primary modes of the traffic generator: benign mode, where different actions are randomly triggered to simulate normal user behavior, and attack mode, where 13 different attack types can be executed.

\subsection{Benign Mode}
In benign mode, the traffic generator simulates normal user behavior across various protocols, including web browsing (HTTP(S)), file transfers via FTP, email exchanges through SMTP, and remote server interactions over SSH.

The traffic generator simulates benign HTTP(S) interactions using automated bots and a web crawler to replicate realistic web browsing behavior. Bots perform randomized actions on an OWASP Juice Shop instance, such as browsing pages, submitting feedback, and logging in or out. The web crawler complements this by navigating internal links from a list of URLs, scraping page content, and updating a graph-based crawl frontier to emulate natural user navigation. Both components introduce variability through randomized actions and delays. 
Similarly, benign FTP interactions involve randomized directory navigation, file uploads, and downloads, with variability in file names, sizes, permissions, and login attempts.
For SMTP, the traffic generator simulates randomized email exchanges by varying the subject, body length, and recipient count.
Finally, SSH interactions involve establishing remote sessions and executing random commands on servers, including delays and optionally simulated failures, creating realistic server access behaviors.

\subsection{Attack Mode}
In attack mode, the traffic generator simulates 13 different malicious activities across supported protocols. Table~\ref{tab:attacks} shows the different attacks grouped by the service.

For HTTP(S), various web-based attacks are executed on an OWASP Juice Shop instance, including SQL injection (targeting login and search functionalities, with optional payload obfuscation), cross-site scripting (XSS) through feedback forms, denial of service (DoS) via sqlmap, brute force login attempts using Hydra, server-side request forgery (SSRF) targeting predefined URLs, and reverse shell exploits leveraging server-side template injection (SSTI) to execute commands on the victim server. The implemented FTP attacks include fingerprinting via version detection and brute force attempts with randomized or injected credentials to simulate successful and unsuccessful login scenarios. For the attacks on the SMTP protocol, we focus on enumeration of users and fingerprinting of the mail server configuration. Furthermore, SSH attacks use brute force to compromise login credentials, incorporating successful breaches. Finally, miscellaneous attacks include host sweeps and port scans, with parameters such as target ranges randomized for variability. Each attack dynamically adjusts parameters, timing, and target specifics to ensure realistic simulation.

\begin{table}[]
\caption{Implemented attacks in the traffic generator}
\label{tab:attacks}
\resizebox{\columnwidth}{!}{%
\begin{tabular}{|l|l|l|l|}
\hline
\multicolumn{1}{|c|}{\textbf{Service}} & \multicolumn{1}{c|}{\textbf{Attack}} & \multicolumn{1}{c|}{\textbf{Alias}} & \multicolumn{1}{c|}{\textbf{Tool(s)}}                    \\ \hline
\textbf{FTP}                           &                                      &                                     &                                                          \\ \hline
                                       & Fingerprinting                       & ftp\_version                        & Metasploit (auxiliary/scanner/ftp/ftp\_version)   \\ \hline
                                       & Bruteforce                           & ftp\_login                          & Metasploit (auxiliary/scanner/ftp/ftp\_login)     \\ \hline
\textbf{HTTP/S}          &                                      &                                     &                                                          \\ \hline
                                       & Cross-site scripting         & xss                                 & Selenium WebDriver                                       \\ \hline
                                       & SQL-Injection                        & sqli                                & Selenium WebDriver, python-requests, sqlmap              \\ \hline
                                       & Denial of Service              & dos                                 & sqlmap                                                   \\ \hline
                                       & Bruteforce                           & bruteforce                          & Hydra                                                    \\ \hline
                                       & Server-side request forgery   & ssrf                                & Selenium WebDriver                                       \\ \hline
                                       & Reverse Shell                        & revshell                            & Selenium WebDriver, netcat                               \\ \hline
\textbf{SMTP}                          &                                      &                                     &                                                          \\ \hline
                                       & Fingerprinting                       & smtp\_version                       & Metasploit (auxiliary/scanner/smtp/smtp\_version) \\ \hline
                                       & User Enumeration                     & smtp\_enum                          & Metasploit (auxiliary/scanner/smtp/smtp\_enum)    \\ \hline
\textbf{SSH}                           &                                      &                                     &                                                          \\ \hline
                                       & Bruteforce                           & ssh\_login                          & Metasploit (auxiliary/scanner/ssh/ssh\_login)     \\ \hline
\textbf{Misc}  &                                      &                                     &                                                          \\ \hline
                                       & Portscan                             & portscan                            & Nmap (-sS flag)                                 \\ \hline
                                       & Hostsweep                            & hostsweep                           & Nmap (-sn and -Pn flags)                                          \\ \hline
\end{tabular}%
}
\end{table}
%\vspace{-3mm}
\subsection{Multi-destination active message format}
\label{section:message_format}
\vspace{-0.7cm}
\begin{figure}[h!]
	\scriptsize
        \centering
    % \hspace{-1cm}
    \includegraphics[width=1\columnwidth]{diagrams/message_format.pdf}
    \vspace{-0.4cm}
	\caption{Message format} 
	\label{fig:message_format}
	\vspace{-.3cm}
\end{figure}
\textit{Nexus Machine} extends the fundamental Active Message primitives to accommodate a multi-destination based routing mechanism. 
Fig.~\ref{fig:message_format} illustrates the message format: the first 12 bits specify intermediate destinations (\textit{R1}, \textit{R2}, \textit{R3}), based on our workload analysis. 
The next 4 bits contain the Program Counter (PC) for the next instruction (\textit{N\_PC}), followed by 4 bits for the \textit{Opcode}. 
A single bit (\textit{Res\_c}) indicates if the message carries a result. 
The subsequent 2 bits (\textit{Op1\_c} and \textit{Op2\_c}) identify whether \textit{Op1} and \textit{Op2} are addresses or values. 
Depending on \textit{Res\_c}, the \textit{Result} field contains the final result or its address, while the next 16 bits hold data for Operand1 (\textit{Op1}) and Operand2 (\textit{Op2}).

When a message arrives at a router, the first destination (\textit{R1}) is processed by the \textit{Route Computation} logic and then allocated to the appropriate output port. After reaching \textit{R1}, the message is handled by the \textit{Input Network Interface}, and the remaining destinations are cyclically rotated, making \textit{R2} the first and \textit{R3} the second. 

In the \textit{Nexus Machine}, a message is equivalent to a packet or flit (all messages are a single-flit packet).
\begin{comment}
\begin{figure*}[h!]
	\scriptsize
	\centering
	\includegraphics[width=\textwidth]{diagrams/architecture.pdf}
	\caption{\textit{Nexus Machine} microarchitecture. \textit{Nexus Machine} is a fabric of homogenous PEs interconnected by a mesh network for communicating Active messages, enhancing fabric utilization by executing messages en-route.} 
	\label{fig:detail_arch}
	%\vspace{-.5cm}
\end{figure*}
\end{comment}
%\vspace{-3mm}
\subsection{Nexus Machine Micro-architecture}
\begin{figure*}[h!]
	\scriptsize
	\centering
	\includegraphics[width=0.9\textwidth]{diagrams/architecture.pdf}
    \vspace{-.15cm}
	\caption{\textit{Nexus Machine} microarchitecture. A fabric of homogenous PEs interconnected by a mesh network for communicating Active Messages which carry instructions that can be launched en-route at any PE, enhancing fabric utilization and runtime.} 
    \vspace{-0.3cm}
 %\color{red}{\bf Peh: I suggest replacing (d) with one of the Compute Unit, cos it's a major component of Nexus and yet do not feature in any figure. We need to highlight to reviewers that our compute unit consists of ALU :)}} 
	\label{fig:detail_arch}
	%\vspace{-.5cm}
\end{figure*}
%\subsubsection{Top Level}
As presented in Fig.~\ref{fig:detail_arch}(a), the \textit{Nexus Machine}'s fabric comprises homogeneous processing elements (PEs) interconnected with a mesh network, with a global termination detector. Each PE is linked to four neighboring PEs in North, East, South, and West directions.
The off-chip memory is connected to the four PEs located along the left edge.


\subsubsection{Processing Elements (PEs).}
%As presented in figure~\ref{fig:detail_arch}(b), each PE combines a compute unit, a dynamic router for network connectivity with congestion control, a decode unit with local data memory, an Input Network Interface which contains an instruction memory for handling incoming AMs and an AM Network Interface unit for spawning new AMs. \\
As presented in Fig.~\ref{fig:detail_arch}(b), each PE combines a compute unit, a dynamic router for network connectivity with congestion control, a decode unit, and two Network Interface logic.
Specifically, \textit{Input Network Interface} unit is responsible for efficiently handling incoming AMs from the NoC, while the AM Network Interface unit initiates the injection of new messages into the NoC.

\textbf{Input Network Interface.}
%The Input Network Interface logic triggers the loading of subsequent instruction on AM arrival.
%The arrival of a Decode AM triggers loading of the data element 
The \textit{Input Network Interface} unit manages \textit{incoming AMs} to a PE.
Depending on the message, \\%it performs either of these two operations.\\
%(a) It either updates the instruction contained in the message based on the next Program Counter (N\_PC) value provided within the message body.\\
(a) If it pertains to an ALU operation, it is directed to the \textit{Compute Unit} for execution.\\
(b) Alternatively, in case of a memory operation, the message is forwarded to the \textit{Decode} unit. 
This unit initiates a load or store operation, utilizing the operand address information (\textit{Op1} or \textit{Op2}) contained in the message.\\
Once these operations are completed, the resulting \textit{output dynamic AM} is dispatched to the \textit{AM Network Interface} for injecting into the network.
%The message enters the network via the local input port, which feeds the \textit{Compute} unit.

\textbf{Compute Unit.}
The \textit{compute unit} within a PE can perform 16-bit arithmetic operations, logic operations, multiplication, and division on its ALU.

An incoming AM at the \textit{Input Network Interface} dispatches two operands, \textit{Op1} and \textit{Op2} along with the \textit{Opcode} field in the message to the compute unit.
After computation, it generates an output that is combined with the original AM, replacing the \textit{Op1} field in the message.
Finally, this modified AM is forwarded to the \textit{AM Network Interface} for injecting into the network.

%{\bf Peh: There needs to be detailed information on how an AM launches computation! This is the thesis of AM! For instance, what's the format of the AM, when it's received, which field is used to configure the ALU? How is the PC set? What happens in the beginning of execution? read config memory? are there registers? what happens if data operand is not present -- can that happen? stall? Lots of details needed here.}

\textbf{Decode Unit.}
The \textit{Decode Unit}, as shown in Fig.~\ref{fig:detail_arch}(e), can be flexibly configured to operate in dereference and streaming modes.
In \textbf{dereference mode}, the operand address field (\textit{Op1} or \textit{Op2}) in the message triggers the loading of a single element. This gets embedded into the output \textit{dynamic AM}.
Conversely, in \textbf{streaming mode}, the message initiates the loading of multiple elements from memory, generating multiple output AMs.
In this mode, the operand address is considered the base address, along with a count to access and load the elements from memory sequentially.
These two modes suffice for our benchmarks; however, our architecture allows for integration of additional modes if needed.

\textbf{Active Message (AM) Network Interface.}
The \textit{AM Network Interface logic} is responsible for injecting AMs into the network.
%The AM Network Interface logic consists of an AM Queue and a configuration memory.
%The AM Queue is a 1KB FIFO, initialized with 44-bit precompiled entries.
%The configuration memory is 16-bit wide, containing 8 configurations.
This module comprises two primary components: an \textit{AM Queue} and a \textit{configuration memory}. 
The \textit{AM Queue} is a 16KB FIFO initialized with 70-bit precompiled entries. 
The \textit{configuration memory}, 10-bit wide, accommodates 8 distinct configurations.

%Depending on the availability of the output dynamic AM from the \textit{Input Network Interface}, it either
It either performs these two operations, as shown in Fig.~\ref{fig:detail_arch}(b):
(1) If the output \textit{dynamic AM} is available from \textit{Input Network Interface}, the subsequent configuration is loaded from memory based on the \textit{N\_PC} field of the AM (see Fig.~\ref{fig:message_format}). 
This configuration is combined with the output \textit{dynamic AM} and forwarded into the injection port of the router.\\
(2) Alternatively, a \textit{static AM} is injected into the network to keep it occupied. 
This \textit{static AM} is the concatenation of the next precompiled entry from the \textit{AM Queue} with the first configuration loaded from memory.
The generation rate of \textit{static AMs} is determined by the backpressure signal at the router's injection port.

The highlighted blue fields in the message format (see Fig.~\ref{fig:message_format}) depict data from the configuration memory used to construct the subsequent dynamic AM, with fields \textit{Res\_c}, \textit{Op1\_c}, and \textit{Op2\_c} stored to prevent redundancy.
%The AM Network Interface logic consists of a 1KB AM Queue, a FIFO containing 44-bit pre-compiled entries.
%, alongside a 13-bit wide configuration register. 

%As shown in Figure~\ref{fig:detail_arch}(e), the output dynamic AMs from \textit{Input Network Interface} trigger loading the next subsequent configuration from the memory with the N\_PC field of the AM.
%These are further concatenated with the output dynamic AMs and pushed into the injection port of the router.
%The injection rate is managed by the backpressure signal at the injection port of the router.

%To keep the network occupied, static AMs are containing the first precompiled entry from AM queue
%As shown in Figure~\ref{fig:detail_arch}(e), it concatenates an AM Queue entry with the first configuration loaded from the memory to generate a static AM, which is subsequently pushed intothe injection port of the router. 

\subsubsection{Dynamic and Congestion Aware Routing.}
\textit{Nexus Machine} supports turn model routing~\cite{noc_peh}, with each router containing five input and five output ports.
%Specifically, these input ports correspond to AM, local, north, east, south and west, whereas output ports correspond to local and four directions.
Specifically, these input ports are designated for messages coming from \textit{AM Network Interface} unit, as well as north, east, south, and west directions, whereas output ports are designated for messages going to \textit{Input Network Interface} unit and four directions.
%The AM input port receives recently generated messages from the \textit{AM Network Interface unit}, while the local port handles messages coming from the \textit{Input Network Interface}. 
Each input port has a buffer comprising three registers to manage in-flight messages, accompanied by congestion control logic. \textit{Nexus Machine}'s design choice of employing only three registers is motivated by the goal of minimizing overall power consumption.

As presented in Fig.~\ref{fig:detail_arch}(c), each router contains a Route Computation Unit, Separable Allocator, and a Crossbar.

\textbf{Route Computation} logic considers the destination of messages from all the input ports. It compares it with the positional ID of the PE, and calculates the output port to be requested. This is sent as an input to the allocator.

%{\bf Peh: Separable alllocation is a well-known previously proposed technique... so there's no need to elaborate... just cite a NoCs textbook}
A toy example of \textbf{Separable Allocation} process is presented in Fig.~\ref{fig:detail_arch}(d)~\cite{noc_peh}.
\iffalse
The request matrix's rows correspond to input ports, and columns correspond to output ports.
The process consists of two stages of 6:1 and 5:1 fixed priority arbiters. The first stage prunes the matrix to ensure that each output port (or resource) receives requests from at most one input port (or requestor). Subsequently, the backpressure signal is applied to each output port, enabling congestion control, as explained below. The second stage further prunes the matrix to guarantee one grant per input port.
The allocator executes within a single cycle, marking granted requests as issued immediately to prevent them from bidding again.
\fi

\textbf{On/Off Congestion control} involves the transmission of a signal to the upstream router when the count of available buffers falls below a threshold, ensuring all in-flight messages will have buffers on arrival. Each of the five ports transmits an OFF signal when their corresponding available buffer space is reduced to 1, i.e., $T_{OFF} = 1$, and conversely, an ON signal when their buffer space reaches 2, i.e., $T_{ON} = 2$.

The output of the allocator is sent to a 6x5 \textbf{Crossbar}.
%, which forms many-to-many connections among internal and external datapaths.\\ Peh: A crossbar by definition forms many-to-many connections between its input and output ports, so no need to explain. 

\subsubsection{Off-chip Memory Datapath.}
Each off-chip memory port connects to a row of the PE array via an AXI bus, delivering a combined bandwidth of 1.28GBps. During data loading, data transfers from off-chip memory to the \textit{AM queues} and \textit{data memory} in each PE. 
The \textit{AM queues} are actively consumed during execution, effectively hiding data loading latency by performing it concurrently with the execution. 
However, data loading into \textit{data memories} occurs after tile execution is complete.

\subsubsection{Bit-vector Scanners.}
The first sparse operand is encoded in \textit{static AMs}. For subsequent sparse operands, bit-vector scanner hardware assists in efficient iteration, providing coordinates within compressed vectors as described in \cite{capstan}. \textit{Nexus Machine} integrates a modified version of this with its AXI bus controller to obtain these coordinates. It can vectorize 16 non-zeros within 128 elements, allowing it to handle matrices with densities exceeding 12\%.
\vspace{-0.2cm}
\subsection{Deadlock avoidance}
%\textcolor{blue}{\bf Peh: Write up a short blurb on how Nexus machine addresses various deadlock scenarios and explain design choice: (1) Within network: Flow control deadlocks addressed by bubble buffer [cite bubble flow control] instead of VCs so as to minimize buffering; Routing deadlocks addressed by turn model, so as to provide high throughput without complex adaptive routing hardware; (2) AM also introduces potential network-PE deadlocks -- addressed by compiler preventing such cyclic dependencies + runtime timeouts} 

Given the dynamic nature, \textit{Nexus Machine} can potentially encounter deadlock without careful design. We address various deadlock scenarios with these specific design choices: 
(1) To mitigate flow control deadlocks within the network, we adopt the bubble NoC~\cite{bubble_flow} approach over Virtual Channels (VCs), with the aim to minimize buffering. 
(2) Routing deadlocks are mitigated by using the turn model~\cite{noc_peh}, that ensures high throughput without the need for complex adaptive routing hardware.
(3) AMs can potentially create deadlocks between the network and PEs. 
These are effectively mitigated by the compiler through strategic data placement and runtime timeouts.
%\textit{Nexus Machine} currently uses a simple heuristic for data placement strategy within the compiler.
%\textcolor{red}{\bf Peh: Is the simple heuristic related to deadlocks? Cos the above sentence seems to contradict the earlier sentence. Elaborate on the heuristic and timeouts??}
Future research will explore more optimized data placement strategies.
\begin{table}[t]
  \centering
    \caption{Dataset statistics}
    \resizebox{0.49\textwidth}{!}{
        % \begin{threeparttable}

\begin{tabular}{lrrrrr}
    \toprule
    \textbf{ DATASET } & \textbf{ \#Users } & \textbf{ \#Items } & \textbf{\#Interactions}  & \textbf{Avg.Inter.} & \textbf{Sparsity}\\
    \midrule
     Gowalla  & 29,858 & 40,981& 1,027,370 & 34.4 & 99.92\% \\
     Yelp2018  & 31,668 & 38,048 & 1,561,406 & 49.3 & 99.88\% \\ 
     MIND  & 38,441 & 38,000 & 1,210,953 & 31.5 & 99.92\% \\ 
     MIND-Large  & 111,664 & 54,367 & 3,294,424 & 29.5 & 99.95\% \\
    \bottomrule
    \end{tabular}
        % \end{threeparttable}
    }
  \label{tab:datasets}%
\end{table}%
\section{Notes on interpreting classifier results}
\label{sec:notes}
Some of the attacks presented in this dataset are not detectable by inspecting a single flow. Meaning, when a flow-based classifier detects such an attack correctly, it is likely that an overfitting issue occurred. The attacks \texttt{revshell} and \texttt{Server-side request forgery} are usually only successful when the victim server creates a new connection to a host specified in one of the attacks. This results in that proper detection is only guaranteed when at least two flows are analyzed. By only inspecting a single flow, the encrypted payload sent to a server should not be differentiable from a benign payload. Identifying these attacks requires either the inspection of unencrypted payloads or the inspection of resulting flows. However, during capturing the dataset we did not include a mechanism, that would be able to indicate, whether a stream resulted from a previous stream. With some uncertainty, it probably could be inferred, by inspecting the flows and matching on the flow that goes from the victim's IP to the attacker's IP, after a \texttt{revshell} samples is observed. For \texttt{ssrf} the target was randomly chosen out of a list of 5 hosts on the public internet, typically our victim server is not performing web requests. Therefore, this could be used as an indicator.
\section{Conclusion}
In this technical report, we provide details on the creation of the WEB-IDS23 dataset. The dataset is publicly available and generated by a traffic generator developed by ourselves in a virtual environment. The aim of creating this new dataset is to encounter the lack of datasets with low numbers of attack samples to enable the community a balanced learning. Moreover, we include a set of web attacks that are used in real attack scenarios. We mixed the attack traffic with realistic benign traffic utilizing the same services as the attackers to form realistic scenarios.

The dataset is also pruned to some limitations, one big issue that it is synthetically generated and does not contain real-world traffic. Furthermore, only certain protocols are included, due to the high effort needed in implementing such actions into a traffic generator. Additionally, as described before in Sec.~\ref{sec:notes}, some attacks are only detectable by inspecting two flows, we did not implement a mechanism to trace back which flow was the result of a previous flow. 

In future work, when recording new datasets, it would be advisable to record the unencrypted payloads. They might be the only way to make some attacks differentiable from real traffic. This would also help to detect attacks that result in a second stream. By inspecting  the payloads, e.g., a revshell could be detected.

\section*{Acknowledgements}
We would like to thank Marty Schüller for his work on the traffic generator and for helping in recording and curating the dataset.
\balance
\bibliographystyle{IEEEtranS}
\bibliography{IEEEabrv, bibliography}

\end{document}
