\documentclass[10pt,conference,letterpaper]{IEEEtran}
\IEEEoverridecommandlockouts
\usepackage{cite}
\usepackage{amsmath,amssymb,amsfonts}
%\usepackage{algorithm}
\usepackage{graphicx}
\usepackage{algorithmicx}
\usepackage{booktabs}
%\usepackage{tabular}
\usepackage[noend]{algpseudocode}
\usepackage{textcomp}
\usepackage{xcolor}
\usepackage{enumitem}
\usepackage{array}
\usepackage{footnote}
\usepackage{url}
%\usepackage{tabularx}
\usepackage{balance}
\usepackage{multirow}
\usepackage{hyperref}
\usepackage{gensymb}

%Layouting
\clubpenalty = 10000
\widowpenalty = 10000 
\displaywidowpenalty = 10000

\emergencystretch=20pt

%% Anteil an Seite ab der Figure eine einzelne Seite erhaelt
\renewcommand{\floatpagefraction}{.8}
%% Mindest-Anteil Text pro Seite
%\renewcommand{\textfraction}{.01}
\renewcommand{\topmargin}{-.7in}

%Maximaler Anteil der Seite, die fuer Abbildungen, die `t'op bzw. `b'ottom
%plaziert werden, verwendet werden darf. Empfehlenswert sind Werte
%zwischen 50% und 85% fuer \topfraction und 20% bis 50% fuer
%\bottomfraction. 
%Einer dieser beiden Werte sollte stets groesser als \floatpagefraction
%sein! 
  \renewcommand{\topfraction}{0.8}     % vorher: .7
  \renewcommand{\bottomfraction}{.8}  % vorher: .3

%Maximale Anzahl der Abbildungen und Tafeln, die auf einer Seite `t'op
%bzw. `b'ottom bzw. insgesamt auf der Seite plaziert werden.
  \setcounter{topnumber}{3} % vorher: 2
  \setcounter{bottomnumber}{3} % vorher: 1
  \setcounter{totalnumber}{5} % vorher: 3


\makeatletter
\newcommand\fs@spaceruled{\def\@fs@cfont{\bfseries}\let\@fs@capt\floatc@ruled
  \def\@fs@pre{\vspace*{0.2cm}\hrule height.8pt depth0pt \kern2pt}%
  \def\@fs@post{\kern2pt\hrule\relax}%
  \def\@fs@mid{\kern2pt\hrule\kern2pt}%
  \let\@fs@iftopcapt\iftrue}
\makeatother

\begin{document}

\title{Technical Report:\\ Generating the WEB-IDS23 Dataset}

\author{ 
Eric Lanfer, Dominik Brockmann,  Nils Aschenbruck\\

Osnabrück University - Institute of Computer Science, Osnabrück, Germany\\
\{lanfer, dobrockmann, aschenbruck\}@uos.de\\
}
\IEEEoverridecommandlockouts



%\author{Anonymous authors}

\maketitle

\begin{abstract}
Anomaly-based Network Intrusion Detection Systems (NIDS) require correctly labelled, representative and diverse datasets for an accurate evaluation and development. However, several widely used datasets do not include labels which are fine-grained enough and, together with small sample sizes, can lead to overfitting issues that also remain undetected when using test data. Additionally, the cybersecurity sector is evolving fast, and new attack mechanisms require the continuous creation of up-to-date datasets. To address these limitations, we developed a modular traffic generator that can simulate a wide variety of benign and malicious traffic. It incorporates multiple protocols, variability through randomization techniques and can produce attacks along corresponding benign traffic, as it occurs in real-world scenarios. Using the traffic generator, we create a dataset capturing over 12 million samples with 82 flow-level features and 21 fine-grained labels. Additionally, we include several web attack types which are often underrepresented in other datasets.
\end{abstract}

\begin{IEEEkeywords}
Dataset, Network Intrusion Detection, Machine Learning, Web Attacks
\end{IEEEkeywords}
\documentclass[../main.tex]{subfiles}
\graphicspath{{../images/}}
\makeatletter
\def\input@path{{../images/}}
\makeatother
\begin{document}
\section{Introduction}
\begin{figure}
\centering
\begin{tikzpicture}
\node[inner sep=0pt] (ws) at (0, 0) {
\includegraphics[height=.4\textwidth, trim={10cm 0 10cm 0},clip]{world_space.png}};
\node[inner sep=0pt] (cs) at (6,0) {\includegraphics[height=.4\textwidth, trim={10cm 1cm 10cm 4cm},clip]{conf_space.png}};
\end{tikzpicture}
\vspace{-5pt}
\label{fig:pbrm_intro}
\caption{\textbf{Left}: Shows world space obstacles as grey spheres. Robots start and goal configuration is colored red and green, respectively. Configurations along the computed path are colored transparent blue. \textbf{Right:} Mapped world space scenario to configuration space. Obstacle region is the grey mesh. Red spheres are collision-free regions computed by the neural SCDF. The optimized shortest path in the convex corridor is the blue curve.}
\vspace{-25pt}
\end{figure}
Motion planning is the problem of finding a collision-free trajectory that connects a given start and goal configuration. The planning takes place in the configuration space of the robot. For single body robots, like mobile robots or drones, the configuration space and the world space are usually the same. This simplifies the planning, since explicit obstacle representations are available which enables geometrical tools like separating hyperplanes, smallest distance to obstacles etc., to be used when designing motion planning algorithms. For multi-body robots like manipulators, the situation is completely different. The world space obstacles are usually mapped to non-convex regions, and to make the problem even harder, the mapping is usually not known. Forming explicit representations of the obstacle region in the configuration space is usually too expensive or intractable. Despite all of this, sampling based planners are used with great success, which mainly is due to their use of implicit representations of the obstacle region. The basic idea is to construct a graph in the configuration space that covers and connects the collision-free region. From this graph, a path can be extracted that connects a given start and goal configuration. The approach is computationally expensive, since the graph is constructed with the smallest geometrical building block available, points, which represents a collision-check. Furthermore, the extracted paths from the graph are non-smooth and jagged due to the stochastic nature of the approach. This adds an additional post-processing step to the process, where the paths are shortcutted and smoothened, before the path can be used for tracking. Clearly a lot of time is invested to form this graph and produce smooth paths. Thus, if the obstacles start to move, then all of this work is done in no use, since all points that make up this graph need to be re-verified, which is simply too time consuming to be done in real time.
\\\\
In this work, we want to address the existing drawbacks of the sampling based planners. Our main contribution is an improved motion planner where each vertex in the graph covers a collision-free region in the form of a sphere instead of a point and where the edges are formed with neighboring intersecting spheres. This representation has the advantage of instead of returning piecewise linear paths, returning a sequence of overlapping spheres, i.e. a convex corridor, that connects a given start and goal configuration, illustrated in Figure \ref{fig:pbrm_intro}. This convex corridor allows us to use convex optimization to produce smooth trajectories, instead of computationally expensive post-processing methods. The representation further allows us to estimate the coverage of the collision-free space, which gives us awareness and feedback in the offline roadmap construction phase. Finally, our representation is simple to adapt to moving obstacles, simply requery for the new radii and recheck for intersections. 
\\\\
The spherical collision-free regions are formed using a signed distance function (SDF), which is a function that returns the smallest distance from an arbitrary point to the boundary of an obstacle. As the name implies, the distance is signed, thus if the point is inside the obstacle it is negative otherwise positive. If the distance is positive, a sphere with radius equal to the distance is guaranteed to cover a collision-free region. Using an SDF in motion planning is not new, but what is novel about our approach is that we express the distance in the configuration space instead of the world space and by doing so allows us to form these convex collision-free regions. We refer to the resulting SDF as a signed configuration distance function (SCDF). Computing an SCDF analytically is non-trivial, our approach is therefore to parameterize the SCDF with a deep neural network and learn the mapping by supervised learning. Our resulting neural SCDF can compute distances for different parameter values of obstacle shapes and we also show how multiple distances can be combined, thus making our approach flexible.
\section{Related work}
Motion planning algorithms can roughly be divided into three families, grid-based, sampling based and optimization based methods. Grid-based methods (GBM) discretize the planning space from which a graph is then compiled. A standard search method is A$^\star$ \citep{a_star}, which is classified as an \textit{informed} search method, since it employs a heuristic function to speed up the search. A$^\star$ guarantees to return an optimal path at the level of discretization used. GBMs usually discretize the planning space by a regular lattice and this limits the GBMs to problems with low dimensionality due to the curse of dimensionality. Thus, GBMs are usually limited to single-body robots where the degrees of freedom (DOF) are low. To overcome the inherent scaling problem with the GBMs, stochastic methods are usually used for multi-body robots. These methods are termed as sampling-based methods (SBM) and core members within this family are the rapidly-exploring random trees (RRT) \citep{rrt} and the probabilistic roadmap (PRM) \citep{prm}. RRT grows a tree from the start configuration and explores the collision-free region in a rapid way until it is able to connect to the goal region. RRT is usually improved by bi-directional planning \citep{rrt_connect}, i.e. an additional tree is grown from the goal configuration and the trees are tested for connection after any tree has been expanded. RRT is a single-query method, thus it searches for a path from scratch each time it is queried. Contrary to this, PRM is a multi-query method, which solves for multiple queries without starting from scratch. PRM does this by creating a roadmap (graph) that covers the collision-free space as an offline step. The graph is then used to solve for multiple queries. PRMs are used in cases where the environment does not change since the extra offline step is too computationally costly and needs to be re-done if the environment is changed. In our work, we address this inherent issue by using a different roadmap representation. Our vertices in the graph cover a collision-free region in the form of spheres and we form the edges by checking for intersecting spheres. If something in the environment changes, we recompute the spheres radii and recheck the intersections, without relying on collision detection. We use a trained neural network to compute the sphere radius, therefore querying for the radius can be done fast, hence our representation enables the PRM for dynamic environments.
\\\\
In the recent decades, optimization based methods (OBM) \citep{chomp, schulman, itomp, stomp} have been introduced as an alternative to SBM for multi-body robots. Like the SBM, the OBMs scale well to higher dimensional problems and produce smoother motion. It is common to use a SDF in the optimization since it is a smooth function, thus enabling gradient-based methods. However, the standard way of expressing the SDF is in world space. The distance therefore needs to be mapped to the configuration space by the forward kinematics. This mapping makes the optimization problem a non-linear program (NLP), which is computationally expensive to solve. Recently, a different approach has been proposed. In \cite{mp_gcs} motion planning is formulated as a convex optimization problem by using the graph of convex sets framework \citep{gcs}. The underlying idea is to decompose the collision-free space into intersecting convex sets from which a convex optimization problem is formulated. In cases where an explicit representation of the obstacles in the configuration space exists, like for single-body robots, creating collision-free convex regions can be done fast \citep{iris}. For multi-body robots, this is non-trivial. Existing work does this successfully \citep{iris_nlp, iris_c} by an optimization based approach, but the methods are still too time consuming to be used in the presence of moving obstacles. Our approach is instead to use deep learning to learn an SDF expressed in the configuration space. With this, we can query for shortest distances to the collision boundary, which allows us to expand spherical regions which are collision-free. Our approach is fast and therefore enables our suggested roadmap planner to be used in dynamic environments.
\\\\
Recent research has focused on learning collision detection \citep{fk_kernel_distance, diffco, graphdistnet} by predicting the signed distance between the robot links and the surrounding obstacles in the world space. The learned SDF is used in trajectory optimization but since the distance is expressed in the world space, the problem becomes an NLP and therefore takes a long time to solve. We take a novel approach and suggest to instead express the signed distance in the configuration space. This allows us to improve the PRM at the same time as it enables convex optimization for trajectory optimization, which runs faster and is more reliable than NLP solvers. In \cite{cspf} a learned signed distance function in the configuration space is proposed similar to our approach. However, their approach is restricted to point cloud representations, while we propose to represent the obstacles as parameterized geometric shapes, e.g. spheres. Furthermore, we also show how to use our learned SCDF to improve an existing roadmap planner.
\section{Problem formulation}
A robot is located in the world space, $\W \subset \R^3 $. The unique location of the robot is given by its configuration $\q \in \C$, where $\C$ is the configuration space. The set of points covered by the robots bodies at a certain configuration is expressed as $\B(\q) \subset \W$. The robot is surrounded by $\NrObst$ obstacles $\O = \bigcup_{i=1}^{\NrObst} \O_i$, where  $\O_i \subset \W$. The representation of the obstacle in the configuration space is the set $\C\O_i = \{\q \in \C \: |\: \B(\q) \cap \O_i \neq \emptyset \}$. The obstacle space is formed as $\Co = \bigcup_{i=1}^{\NrObst} \C \O_i$. The complement is referred to as the free space, $\Cf = \C \setminus \Co$. The path planning problem is a tuple, ($\Cf$, $\qStart$, $\qGoal$), where we want to connect a query pair, consisting of a start, $\qStart$, and goal configuration, $\qGoal$, with a geometric path, $\q(s): [0, 1] \mapsto \Cf$, such that $\q(0)=\qStart$ and $\q(1)=\qGoal$, or report correctly when such a path does not exist.
\end{document}

\section{Traffic Generation}
In order to have a wide variety of benign traffic and attack traffic, we developed a traffic generator based on Python. We decided to develop our own generator that allows us to orchestrate multiple attacks and also according benign traffic. Considering a company network, users will have benign interaction with the services provided on a network, these benign traffic usually mixes up with the traffic of attackers trying to attack such services. This overlap creates a realistic challenge for intrusion detection systems to detect the attacks inside the normal, benign noise. By implementing our own traffic generator, we can utilize the generator to execute benign actions that are closer to the attack patterns, e.g., utilizing the same web server endpoints. The generator supports multiple protocols, including HTTP(S), FTP, SMTP, SSH, ICMP, DNS, TCP, and UDP, ensuring broad coverage of typical network traffic scenarios. Additionally, the generator applies randomization techniques to introduce variability in request timings and traffic patterns, ensuring that generated traffic does not rely on static characteristics. This helps mitigate model overfitting to features like precise timing patterns that follow the specific generation process. Currently, the traffic generator is available on request to other researchers, it is planned to release it open source.

In the following, we will describe the two primary modes of the traffic generator: benign mode, where different actions are randomly triggered to simulate normal user behavior, and attack mode, where 13 different attack types can be executed.

\subsection{Benign Mode}
In benign mode, the traffic generator simulates normal user behavior across various protocols, including web browsing (HTTP(S)), file transfers via FTP, email exchanges through SMTP, and remote server interactions over SSH.

The traffic generator simulates benign HTTP(S) interactions using automated bots and a web crawler to replicate realistic web browsing behavior. Bots perform randomized actions on an OWASP Juice Shop instance, such as browsing pages, submitting feedback, and logging in or out. The web crawler complements this by navigating internal links from a list of URLs, scraping page content, and updating a graph-based crawl frontier to emulate natural user navigation. Both components introduce variability through randomized actions and delays. 
Similarly, benign FTP interactions involve randomized directory navigation, file uploads, and downloads, with variability in file names, sizes, permissions, and login attempts.
For SMTP, the traffic generator simulates randomized email exchanges by varying the subject, body length, and recipient count.
Finally, SSH interactions involve establishing remote sessions and executing random commands on servers, including delays and optionally simulated failures, creating realistic server access behaviors.

\subsection{Attack Mode}
In attack mode, the traffic generator simulates 13 different malicious activities across supported protocols. Table~\ref{tab:attacks} shows the different attacks grouped by the service.

For HTTP(S), various web-based attacks are executed on an OWASP Juice Shop instance, including SQL injection (targeting login and search functionalities, with optional payload obfuscation), cross-site scripting (XSS) through feedback forms, denial of service (DoS) via sqlmap, brute force login attempts using Hydra, server-side request forgery (SSRF) targeting predefined URLs, and reverse shell exploits leveraging server-side template injection (SSTI) to execute commands on the victim server. The implemented FTP attacks include fingerprinting via version detection and brute force attempts with randomized or injected credentials to simulate successful and unsuccessful login scenarios. For the attacks on the SMTP protocol, we focus on enumeration of users and fingerprinting of the mail server configuration. Furthermore, SSH attacks use brute force to compromise login credentials, incorporating successful breaches. Finally, miscellaneous attacks include host sweeps and port scans, with parameters such as target ranges randomized for variability. Each attack dynamically adjusts parameters, timing, and target specifics to ensure realistic simulation.

\begin{table}[]
\caption{Implemented attacks in the traffic generator}
\label{tab:attacks}
\resizebox{\columnwidth}{!}{%
\begin{tabular}{|l|l|l|l|}
\hline
\multicolumn{1}{|c|}{\textbf{Service}} & \multicolumn{1}{c|}{\textbf{Attack}} & \multicolumn{1}{c|}{\textbf{Alias}} & \multicolumn{1}{c|}{\textbf{Tool(s)}}                    \\ \hline
\textbf{FTP}                           &                                      &                                     &                                                          \\ \hline
                                       & Fingerprinting                       & ftp\_version                        & Metasploit (auxiliary/scanner/ftp/ftp\_version)   \\ \hline
                                       & Bruteforce                           & ftp\_login                          & Metasploit (auxiliary/scanner/ftp/ftp\_login)     \\ \hline
\textbf{HTTP/S}          &                                      &                                     &                                                          \\ \hline
                                       & Cross-site scripting         & xss                                 & Selenium WebDriver                                       \\ \hline
                                       & SQL-Injection                        & sqli                                & Selenium WebDriver, python-requests, sqlmap              \\ \hline
                                       & Denial of Service              & dos                                 & sqlmap                                                   \\ \hline
                                       & Bruteforce                           & bruteforce                          & Hydra                                                    \\ \hline
                                       & Server-side request forgery   & ssrf                                & Selenium WebDriver                                       \\ \hline
                                       & Reverse Shell                        & revshell                            & Selenium WebDriver, netcat                               \\ \hline
\textbf{SMTP}                          &                                      &                                     &                                                          \\ \hline
                                       & Fingerprinting                       & smtp\_version                       & Metasploit (auxiliary/scanner/smtp/smtp\_version) \\ \hline
                                       & User Enumeration                     & smtp\_enum                          & Metasploit (auxiliary/scanner/smtp/smtp\_enum)    \\ \hline
\textbf{SSH}                           &                                      &                                     &                                                          \\ \hline
                                       & Bruteforce                           & ssh\_login                          & Metasploit (auxiliary/scanner/ssh/ssh\_login)     \\ \hline
\textbf{Misc}  &                                      &                                     &                                                          \\ \hline
                                       & Portscan                             & portscan                            & Nmap (-sS flag)                                 \\ \hline
                                       & Hostsweep                            & hostsweep                           & Nmap (-sn and -Pn flags)                                          \\ \hline
\end{tabular}%
}
\end{table}
\begin{figure*}[t]
\vskip 0.2in
\begin{center}
\centerline{
\includegraphics[width=\textwidth, height=9cm]{figures/architecture_img.pdf}}   
\vspace{-3mm}
\caption{\textbf{Overview of our method at the blending stage. }
% condition
Two input images or concepts are encoded into embeddings, mapped to a shared text space via the Linear Prior Converter from unCLIP~\citep{ramesh2022hierarchical}. These embeddings condition the U-Net: one for downsampling, the other for upsampling.
% module
During the blending stage, a blending latent $L_b$ initialized with Gaussian noise is processed in the Feedback Interpolation Module, conditioned on image embeddings. Noise $\epsilon$ is added to the embeddings to generate initial auxiliary latents, which are interpolated into $L^{(t)}_{b}$ with an increasing weight $p$. The  $L^{(t)}_{a}$ is combined with interpolated latent $L'^{(t)}_{b}$ by proportion $p$. All updated $L'^{(t)}_{a}$ are refined in the auxiliary inference to retain original features using the text prompt for corresponding categories, and $L'^{(t)}_{b}$ is denoised via the blending inference.
% refinement
Finally, the refined $ L_b $ is passed into the VAE decoder to generate the final blending image. 
}
\label{architecture}
\end{center}
% \vskip -0.4in
\vspace{-8mm}
\end{figure*}

\begin{table*}
\label{datasetdescription}
\centering
\caption{Overview of the evaluation dataset, where M denotes malware and B denotes benign applications.}
\begin{tabular}{c|c|c|c|c|c|c|c|c} 
\hline
           & \textbf{Time Interval} & \textbf{Sample Size} & \begin{tabular}[c]{@{}c@{}}\textbf{The number of}\\\textbf{Existing family}\end{tabular} & \begin{tabular}[c]{@{}c@{}}\textbf{The number of}\\\textbf{New family}\end{tabular} & \textbf{Packed} & \textbf{Malicious} & \textbf{Benign} & \textbf{M/(M+B)\%}  \\ 
\hline
Test set 1 & 2020.05 - 2021.01 & 3015  & 21                                                               & 24                                                          & 18     & 284       & 2731   & 9.42                       \\ 
\hline
Test set 2 & 2021.01 - 2021.12 & 3015      & 28                                                               & 32                                                          & 30     & 298       & 2717   & 9.88                      \\ 
\hline
Test set 3 & 2021.12 - 2023.12 & 3016      & 34                                                               & 36                                                          & 40     & 302       & 2714   & 10.01                       \\
\hline                  
\end{tabular}
\end{table*}
\section{Notes on interpreting classifier results}
\label{sec:notes}
Some of the attacks presented in this dataset are not detectable by inspecting a single flow. Meaning, when a flow-based classifier detects such an attack correctly, it is likely that an overfitting issue occurred. The attacks \texttt{revshell} and \texttt{Server-side request forgery} are usually only successful when the victim server creates a new connection to a host specified in one of the attacks. This results in that proper detection is only guaranteed when at least two flows are analyzed. By only inspecting a single flow, the encrypted payload sent to a server should not be differentiable from a benign payload. Identifying these attacks requires either the inspection of unencrypted payloads or the inspection of resulting flows. However, during capturing the dataset we did not include a mechanism, that would be able to indicate, whether a stream resulted from a previous stream. With some uncertainty, it probably could be inferred, by inspecting the flows and matching on the flow that goes from the victim's IP to the attacker's IP, after a \texttt{revshell} samples is observed. For \texttt{ssrf} the target was randomly chosen out of a list of 5 hosts on the public internet, typically our victim server is not performing web requests. Therefore, this could be used as an indicator.
\section{Conclusion}
In this technical report, we provide details on the creation of the WEB-IDS23 dataset. The dataset is publicly available and generated by a traffic generator developed by ourselves in a virtual environment. The aim of creating this new dataset is to encounter the lack of datasets with low numbers of attack samples to enable the community a balanced learning. Moreover, we include a set of web attacks that are used in real attack scenarios. We mixed the attack traffic with realistic benign traffic utilizing the same services as the attackers to form realistic scenarios.

The dataset is also pruned to some limitations, one big issue that it is synthetically generated and does not contain real-world traffic. Furthermore, only certain protocols are included, due to the high effort needed in implementing such actions into a traffic generator. Additionally, as described before in Sec.~\ref{sec:notes}, some attacks are only detectable by inspecting two flows, we did not implement a mechanism to trace back which flow was the result of a previous flow. 

In future work, when recording new datasets, it would be advisable to record the unencrypted payloads. They might be the only way to make some attacks differentiable from real traffic. This would also help to detect attacks that result in a second stream. By inspecting  the payloads, e.g., a revshell could be detected.

\section*{Acknowledgements}
We would like to thank Marty Schüller for his work on the traffic generator and for helping in recording and curating the dataset.
\balance
\bibliographystyle{IEEEtranS}
\bibliography{IEEEabrv, bibliography}

\end{document}
