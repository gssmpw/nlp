\section{Experimental Results}
\label{experimental_results}

This section presents the evaluation of the adversarial universal stickers attack.


\subsection{Baseline Performance of Street Sign Classification}

The evaluation is performed on 3 randomly selected images: Stop, Yield, and Merge. The input images are from Street View, not from the LISA data set. Thus the training set (LISA images) is different from the testing set (Street View images). LISA-CNN performs well, with average over $80$\% confidence scores in the correctly predicted images. Specifically, Table~\ref{table_adversarial} shows the confidence scores for the street signs.

\begin{table}[t]
\centering
\caption{Baseline images and their confidence scores.}
%\begin{adjustbox}{width=0.48\textwidth}
\label{table_adversarial}
\small
\begin{tabular}{|p{1.8cm}|p{1.8cm}|p{1.8cm}|p{1.2cm}|}
\hline 
\textbf{Image} & \textbf{Predicted Label}  & \textbf{Confidence Score (\%)}
\\ \hline \hline
Stop &  Stop &  85.42 \\ \hline 
Yield  & Yield & 88.87 \\ \hline 
Merge & Merge & 76.51 \\ \hline 

\end{tabular}
%\end{adjustbox}
\end{table}




\subsection{One Sticker Attacks}

The first evaluation was done by generating black and white stickers, one at a time. Stickers of rectangular shape were generated. The width and height were selected to be in ranges from 5\% to 50\% of the street sign size. All combinations of widths and heights form 5 to 50, in step of 5, were tested. Figures~\ref{fig:one_sticker_black_Images} and~\ref{fig:one_sticker_white_Images} show the best location for the black and white stickers respectively. Table~\ref{table_adversarial_timing} shows the confidence scores, and Table~\ref{table_adversarial_labels} shows the (incorrectly classified) labels for these street signs. It can be seen that with one white or black sticker, placed at same location or any of the signs, the machine learning classification results are highly incorrect and attack can succeed.




\begin{figure*}[t]
    \begin{subfigure}[b]{0.25\textwidth}
        \centering
\includegraphics[width=2.2cm]{plots/attack_images/two_sticker/first_black_second_black/Best_stop_45_25_best_sticker_size.png}
        \caption{\small \centering Stop Black Stickers Image}
        \label{fig:Best_stop_black_black_two_sticker}
    \end{subfigure}
    \hfill
    \begin{subfigure}[b]{0.25\textwidth}
        \centering
         \includegraphics[width=2.2cm]{plots/attack_images/two_sticker/first_black_second_black/Best_yield_45_25_best_sticker_size.png}
        \caption{\small \centering Yield Black Stickers Image}
        \label{fig:Best_yield_black_black_two_sticker}
    \end{subfigure}
    \hfill
    \begin{subfigure}[b]{0.25\textwidth}
        \centering
  \includegraphics[width=2.2cm]{plots/attack_images/two_sticker/first_black_second_black/Best_merge_45_25_best_sticker_size.png}
        \caption{\small \centering Merge Black Stickers Image}
        \label{fig:Best_merge_black_black_two_sticker}
    \end{subfigure}
    \vspace{-1em}
    \caption{Two Black sticker attack images.}
    \label{fig:two_sticker_black_black_Images}
\end{figure*}



\begin{figure*}[t]
    \begin{subfigure}[b]{0.25\textwidth}
        \centering
\includegraphics[width=2.2cm]{plots/attack_images/two_sticker/first_black_second_white/Best_stop_50_10_best_sticker_size.png}
        \caption{\small \centering Stop Black and White Stickers Image}
        \label{fig:Best_stop_black_white_two_sticker}
    \end{subfigure}
    \hfill
    \begin{subfigure}[b]{0.25\textwidth}
        \centering
         \includegraphics[width=2.2cm]{plots/attack_images/two_sticker/first_black_second_white/Best_yield_50_10_best_sticker_size.png}
        \caption{\small \centering Yield Black and White Stickers Image}
        \label{fig:Best_yield_black_white_two_sticker}
    \end{subfigure}
    \hfill
    \begin{subfigure}[b]{0.25\textwidth}
        \centering
  \includegraphics[width=2.2cm]{plots/attack_images/two_sticker/first_black_second_white/Best_merge_50_10_best_sticker_size.png}
        \caption{\small \centering Merge Black and White Stickers Image}
        \label{fig:Best_merge_black_white_two_sticker}
    \end{subfigure}
    \vspace{-1em}
    \caption{Two Black and White sticker attack images.}
    \label{fig:two_sticker_black_white_Images}
\end{figure*}


\begin{figure*}[t]
    \begin{subfigure}[b]{0.25\textwidth}
        \centering
\includegraphics[width=2.2cm]{plots/attack_images/two_sticker/first_white_second_black/Best_stop_45_25_best_sticker_size.png}
        \caption{\small \centering Stop White and Black Stickers Image}
        \label{fig:Best_stop_white_black_two_sticker}
    \end{subfigure}
    \hfill
    \begin{subfigure}[b]{0.25\textwidth}
        \centering
         \includegraphics[width=2.2cm]{plots/attack_images/two_sticker/first_white_second_black/Best_yield_45_25_best_sticker_size.png}
        \caption{\small \centering Yield White and Black Stickers Image}
        \label{fig:Best_yield_white_black_two_sticker}
    \end{subfigure}
    \hfill
    \begin{subfigure}[b]{0.25\textwidth}
        \centering
  \includegraphics[width=2.2cm]{plots/attack_images/two_sticker/first_white_second_black/Best_merge_45_25_best_sticker_size.png}
        \caption{\small \centering Merge White and Black Stickers Image}
        \label{fig:Best_merge_white_black_two_sticker}
    \end{subfigure}
    \vspace{-1em}
    \caption{Two White and Black sticker attack images.}
    \label{fig:two_sticker_white_black_Images}
\end{figure*}


\begin{figure*}[h!]
    \begin{subfigure}[b]{0.25\textwidth}
        \centering
\includegraphics[width=2.2cm]{plots/attack_images/two_sticker/first_white_second_white/Best_stop_50_25_best_sticker_size.png}
        \caption{\small \centering Stop White and White Stickers Image}
        \label{fig:Best_stop_white_white_two_sticker}
    \end{subfigure}
    \hfill
    \begin{subfigure}[b]{0.25\textwidth}
        \centering
         \includegraphics[width=2.2cm]{plots/attack_images/two_sticker/first_white_second_white/Best_yield_50_25_best_sticker_size.png}
        \caption{\small \centering Yield White and White Stickers Image}
        \label{fig:Best_yield_white_white_two_sticker}
    \end{subfigure}
    \hfill
    \begin{subfigure}[b]{0.25\textwidth}
        \centering
 \includegraphics[width=2.2cm]{plots/attack_images/two_sticker/first_white_second_white/Best_merge_50_25_best_sticker_size.png}
        \caption{\small \centering Merge White and White Stickers Image}
        \label{fig:Best_merge_white_white_two_sticker}
    \end{subfigure}
    \vspace{-1em}
    \caption{Two White and White sticker attack images.}
    \label{fig:two_sticker_white_white_Images}
\end{figure*}

\subsection{Two Stickers}

The second evaluation was done by generating two black and white stickers, two at a time. Stickers of rectangular shape were again generated. The width and height were selected to be in ranges from 5\% to 50\% of the street sign size. All combinations of widths and heights form 5 to 50, in step of 5, were tested. Figures~\ref{fig:two_sticker_black_black_Images} to~\ref{fig:two_sticker_white_white_Images} show the best location for the different combinations of black and white stickers. Table~\ref{table_adversarial_timing} shows the confidence scores, and Table~\ref{table_adversarial_labels} shows the (incorrectly classified) labels for these street signs. It can be seen that with two white or black stickers in any combination, placed at same location or any of the signs, the machine learning classification results are highly incorrect and attack can succeed.

\subsection{Confidence Scores and Misclassified Labels Patterns}

From Table~\ref{table_adversarial_timing} and Table~\ref{table_adversarial_labels} we observed two patterns.
First, almost always it is possible to find a universal adversarial sticker for one or two stickers. The range of the confidence values (for the incorrectly classified signs) ranges from about 23\% to over 95\%. Only in two cases one of the signs was not misclassified.
Second, when misclassification occurs, the image is most often misclassified as the pedestrian crossing sign. As we treat LISA-CNN as a black-box, we do not have insights into why the pedestrian crossing sign shows up most often. However, this could be abused in future attacks where the attacker knows that the misclassification is likely to give a certain street sign class. E.g., instead of merging on a highway, the vehicle will (incorrectly) detect pedestrian crossing sign and stop, resulting in a crash on a highway.

\subsection{Sticker Size Evaluation}

We performed further extensive study to test what sticker sizes work best. Again, all combinations of widths and heights form 5 to 50, in step of 5, were tested. Tables~\ref{table_one_sticker_black} to~\ref{table_two_sticker_first_white_second_white} show the results.
For single black sticker, highest average confidence in correctly classified sign was over 80\% for 40x50 sticker.
For single white sticker, highest average confidence in correctly classified sign was over 42\% for 25x35 sticker.
For two stickers, black and black, highest average confidence in correctly classified sign was over 63\% for 25x45 sticker.
For two stickers, black and white, highest average confidence in correctly classified sign was over 56\% for 10x50 sticker.
For two stickers, white and black, highest average confidence in correctly classified sign was over 70\% for 25x45 sticker.
And for two stickers, white and white, highest average confidence in correctly classified sign was over 39\% for 25x50 sticker.

\subsection{Sticker Size Patterns}

We observed some expected and some unexpected pattern. For single black sticker, as the sticker size increases (form top-left to bottom-right in the table), the confidence in misclassified images increases. This follows the intuition that as the sticker gets bigger, it obscures larger portion of the street sign, making it less likely to be recognized correctly. On the other hand, for single white sticker, this pattern does not hold. Going form top-left to bottom-right in the table, as sticker dimensions increase, the confidence increases, but then half-way through the table it starts to decrease.

For two stickers, we observed a yet different, but consistent pattern. Going form top-left to bottom-right in the table, as sticker dimensions increase, the confidence in misclassified signs increases, then decreases a bit, until finally the size of the combined stickers is too large and does not fit in the mask area (X in the table entries). This pattern also makes sense. The intuition is that as stickers get bigger, they cause the signs to be incorrectly classified, similar to one black sticker. However, due to size of the stickers, no universal sticker can be found since for large sizes two stickers do not fit in the combined mask.

\begin{table*}[t]
\centering
\caption{Confidence scores of the best universal adversarial images.}
% \begin{adjustbox}{width=0.98\textwidth}
\label{table_adversarial_timing}
\small
\begin{tabular}{|p{1.5cm}|p{1.5cm}|p{1.5cm}|p{1.5cm}|p{1.5cm}|p{1.5cm}|p{1.5cm}|}
\hline 
\textbf{Adversarial Image} & \textbf{One Sticker Black}  & \textbf{One Sticker  White} & \textbf{Two Sticker Black, Black} & \textbf{Two Sticker Black, White} & \textbf{Two Sticker White, Black} & \textbf{Two Sticker White, White} 
\\ \hline \hline
Stop & 86.85 & 37.79 & 60.67 & 49.17& 67.25 & 25.98   \\ \hline 
Yield  & 65.44 & $\times$ & 55.24  & 23.70& 58.87& $\times$ \\ \hline 
Merge & 88.42 & 90.49 & 75.27 &95.41  &87.05 &91.50  \\ \hline 
\end{tabular}
% \end{adjustbox}
\end{table*}




\begin{table*}[t]
\centering
\caption{Predicted labels of the adversarial images. In all but two cases the attack worked. The labels correspond to the confidences shown in entries in Table~\ref{table_adversarial_confidence}.}
\begin{adjustbox}{width=0.98\textwidth}
\label{table_adversarial_labels}
\small
\begin{tabular}{|p{1.5cm}|p{1.5cm}|p{1.5cm}|p{1.5cm}|p{1.5cm}|p{1.5cm}|p{1.5cm}|p{1.5cm}|p{1.5cm}|p{1.5cm}|}
\hline 
\textbf{Adversarial Image} & \textbf{Snowball 1}  & \textbf{Snowball 2} & \textbf{Snowball 3} & \textbf{Snowball 4} & \textbf{Snowball 5} & \textbf{Snowball 6} & \textbf{Snowball 7} & \textbf{Snowball 8} & \textbf{Snowball 9}
\\ \hline \hline
Stop & Speed Limit 25 & Yield& Speed Limit 25& Yield& Speed Limit 45& Signal Ahead& Speed Limit 25& Speed Limit 25& Speed Limit 25\\ \hline 
Yield  & Speed Limit 35 & Speed Limit 35& Speed Limit 35& Speed Limit 35& Ped. Crossing& Ped. Crossing& -& Speed Limit 35& - \\ \hline
Ped. Crossing  & Stop Ahead & Stop Ahead& Stop Ahead& Stop Ahead& Stop Ahead& Stop Ahead& Stop Ahead& Stop Ahead& Stop Ahead \\ \hline  
Merge & Ped. Crossing & Ped. Crossing & Ped. Crossing & Ped. Crossing & Ped. Crossing & Ped. Crossing & Ped. Crossing & Ped. Crossing & Ped. Crossing   \\ \hline 
Turn Right & Stop & Stop & Added Lane & Stop & Added Lane & Ped. Crossing & Stop & Stop & Stop \\ \hline  
\end{tabular}
\end{adjustbox}
\end{table*}




\begin{table*}[t]
    \centering
    \caption{Average Confidence scores for universal adversarial images using a single black sticker.}
    \label{table_one_sticker_black}
    \begin{adjustbox}{width=\textwidth}
    \small
        \begin{tabular}{|p{1.5cm}|p{1.5cm}|p{1.5cm}|p{1.5cm}|p{1.5cm}|p{1.5cm}|p{1.5cm}|p{1.5cm}|p{1.5cm}|p{1.5cm}|p{1.5cm}|p{1.5cm}|}
            \hline
            \textbf{Height, Width} & 
            \textbf{5} &
            \textbf{10} & \textbf{15} & \textbf{20} & \textbf{25} & \textbf{30} & \textbf{35} & \textbf{40} & \textbf{45} & \textbf{50} \\ \hline
             \textbf{5} & - & - & - & - & - & - & - & - & - & -  \\ \hline
             \textbf{10} & - & - & - & - & 12.83 & 14.63 & 16.31 & 19.19 & 21.05  & 31.10 \\ \hline
             \textbf{15} & - & - & - & 15.74 & 19.39 & 22.46 & 26.04 & 36.27 & 41.86 & 47.98 \\ \hline
             \textbf{20} & - & - & 15.62 & 20.42 & 23.79 & 26.17 & 32.09 & 43.09 & 49.95 & 54.09 \\ \hline
            \textbf{25} & - & 14.79 & 19.28 & 24.75 & 26.85 & 30.43 & 42.95 & 55.94 & 62.54 & 72.31 \\ \hline
            \textbf{30} & - & 16.57 & 23.03 & 26.98 & 31.69 & 27.27 & 40.36 & 53.55 & 60.12 & 66.01 \\ \hline
           \textbf{35} & - & 18.07 & 24.91 & 28.01 & 45.13 & 40.38 & 47.83 & 59.38 & 67.27 & 73.54\\ \hline
            \textbf{40} & - & 19.63 & 26.07 & 35.19 & 53.13 & 49.37 & 56.81 & 65.90 & 73.55 & \textbf{80.24}  \\ \hline
           \textbf{45} & - & 20.69 & 33.72 & 41.32 & 55.24 & 54.27 & 61.89 & 69.23 & 70.08 & 77.52 \\ \hline
            \textbf{50} & - & 21.11 & 38.51 & 43.98 & 57.20 & 57.76 & 64.63 & 72.11 & 73.72 & 80.23 \\ \hline
        \end{tabular}
    \end{adjustbox}
\end{table*}

\begin{table*}[t]
    \centering
    \caption{Average Confidence scores for universal adversarial images using a single white sticker.}
    \label{table_one_sticker_white}
    \begin{adjustbox}{width=\textwidth}
    \small
        \begin{tabular}{|p{1.5cm}|p{1.5cm}|p{1.5cm}|p{1.5cm}|p{1.5cm}|p{1.5cm}|p{1.5cm}|p{1.5cm}|p{1.5cm}|p{1.5cm}|p{1.5cm}|p{1.5cm}|}
            \hline
            \textbf{Height, Width} & 
            \textbf{5} &
            \textbf{10} & \textbf{15} & \textbf{20} & \textbf{25} & \textbf{30} & \textbf{35} & \textbf{40} & \textbf{45} & \textbf{50} \\ \hline
             \textbf{5}  & -  & -  & 20.25  &  29.53  & 30.60  & 31.41  & 31.79  & 31.81  & 31.80  & 31.36  \\ \hline
            \textbf{10}  & -  & -  & 23.51  & 31.68  & 31.98  & 32.01  & 31.98  & 31.83  & 31.40  & 31.35  \\ \hline
            \textbf{15}  & 20.28  & 30.07  & 31.86  & 31.98  &  31.98 & 31.79  & 31.56  & 31.43  & 31.32  & 29.99  \\ \hline
            \textbf{20}  & 29.49 & 31.74  & 32.03  & 31.98  & 31.85  & 31.56  & 31.15  &  31.03 & 30.57  & 30.53  \\ \hline
            \textbf{25}  & 30.53  & 32.01  & 32.04 & 31.67  & 31.67  & 41.72  & \textbf{42.76}  & 31.52  & 31.26  & 30.47  \\ \hline
            \textbf{30}  & 31.35  & 31.84  & 31.80  & 37.95  & 40.34  & 31.41  & 31.56  & 31.52  & 31.18 & 27.62  \\ \hline
            \textbf{35}  & 31.50  & 31.78  & 31.68 &  32.13 & 32.14 & 31.65 & 31.77  & 31.31 & 28.21  & 25.42  \\ \hline
            \textbf{40}  & 31.54  & 31.70  & 31.73  & 39.55  & 40.35 & 31.83  &  30.79  & 30.03  & 24.59  & 22.88  \\ \hline
            \textbf{45}  & 31.35 & 31.77  & 31.79  &  32.41 & 32.38  & 28.75  & 25.55 & 23.86  & 21.17  & 18.85  \\ \hline
            \textbf{50}  & 31.33  & 31.91  & 31.96  & 31.73  & 27.75  & 24.33  & 23.85  & 19.96  & 19.51  & 18.59  \\ \hline
        \end{tabular}
    \end{adjustbox}
\end{table*}


\begin{table*}[t]
    \centering
    \caption{Average Confidence scores for universal adversarial images using two stickers (first black, second black).\vspace{-0.5em}}
\label{table_two_sticker_first_black_second_black}
    \begin{adjustbox}{width=\textwidth}
    \small
        \begin{tabular}{|p{1.5cm}|p{1.5cm}|p{1.5cm}|p{1.5cm}|p{1.5cm}|p{1.5cm}|p{1.5cm}|p{1.5cm}|p{1.5cm}|p{1.5cm}|p{1.5cm}|}
            \hline
            \textbf{Height, Width} & 
            \textbf{5} &
            \textbf{10} & \textbf{15} & \textbf{20} & \textbf{25} & \textbf{30} & \textbf{35} & \textbf{40} & \textbf{45} & \textbf{50} \\ \hline
             \textbf{5}  & -  & -  & -  & -  & -  & -  & -  & -  & -  & -  \\ \hline
            \textbf{10}  & -  & -  & -  & 11.40  & 14.41  & 16.22  & 18.67  & 27.16  & 31.69  & 34.01  \\ \hline
            \textbf{15}  & -  & -  & 12.74  & 17.13  & 25.00  & 31.09  & 33.73  & 36.89  & 40.49  & 43.55  \\ \hline
            \textbf{20}  & -  & 12.66  & 17.20  & 29.86  & 35.86  & 42.68  & 48.24  & 49.70  & 52.02  & 53.02  \\ \hline
            \textbf{25}  & -  & 16.06  & 27.41  & 34.65  & 42.22  & 50.39  & 53.44  & 57.99  & \textbf{63.73}  & 55.19  \\ \hline
            \textbf{30}  & -  & 17.14  & 28.85  & 40.52  & 44.21  & 34.50  & 38.14  & 42.35  & 39.08  & 20.87  \\ \hline
            \textbf{35}  & -  & 18.45  & 32.36  & 38.02  & 44.16  & 39.94  & 36.15  & 31.07  & 11.03  & $\times$  \\ \hline
            \textbf{40}  & -  & 17.11  & 41.25  & 41.75  & 39.77  & 31.83  & 29.96  & $\times$  & $\times$  & $\times$  \\ \hline
            \textbf{45}  & -  & 16.75  & 33.42  & 40.16  & 33.63  & 7.80  & $\times$  & $\times$ & $\times$  & $\times$ \\ \hline
            \textbf{50}  & -  & 17.15  & 27.93  & 38.82  & $\times$  & $\times$ & $\times$  & $\times$ & $\times$  & $\times$ \\ \hline
        \end{tabular}
    \end{adjustbox}
\end{table*}


\begin{table*}[t]
    \centering
     \caption{Average Confidence scores for universal adversarial images using two stickers (first black, second white).}
    \label{table_two_sticker_first_black_second_white}
    \begin{adjustbox}{width=\textwidth}
    \small
        \begin{tabular}{|p{1.5cm}|p{1.5cm}|p{1.5cm}|p{1.5cm}|p{1.5cm}|p{1.5cm}|p{1.5cm}|p{1.5cm}|p{1.5cm}|p{1.5cm}|p{1.5cm}|}
            \hline
            \textbf{Width, Height} & 
            \textbf{5} &
            \textbf{10} & \textbf{15} & \textbf{20} & \textbf{25} & \textbf{30} & \textbf{35} & \textbf{40} & \textbf{45} & \textbf{50} \\ \hline
             \textbf{5}  & -  & -          & 19.11  & 29.39  & 30.83  & 31.63  & 32.02  & 32.04  & 31.97  & 31.65  \\ \hline
            \textbf{10}  & -  & -  & 23.73          & 31.86  & 32.18  & 32.24  & 37.62  & 47.71  & 55.40  & \textbf{56.09}  \\ \hline
            \textbf{15}  & 16.59  & 29.42  & 31.94  & 32.23  & 43.31  & 46.49  & 44.18  & 53.84  & 48.54  & 46.16  \\ \hline
            \textbf{20}  & 29.40  & 31.73  & 32.06  & 32.15  & 44.74  & 44.30  & 43.09  & 42.43  & 43.47  & 36.32  \\ \hline
            \textbf{25}  & 31.19  & 31.88  & 31.72  & 39.44  & 41.62  & 39.19  & 40.92  & 36.12  & 31.94  & 36.66  \\ \hline
            \textbf{30}  & 31.76  & 31.97   & 31.55  & 34.09  & 31.23  & 30.22  & 30.29  & 29.22  & 28.30  & 32.04  \\ \hline
            \textbf{35}  & 31.65  & 31.55 & 29.34  & 28.05  & 32.18  & 28.12  & 25.71  & 24.54  & 17.05  & $\times$  \\ \hline
            \textbf{40}  & 24.96  & 25.32  & 20.39  & 26.00  & 28.17  & 22.81  & 20.06  & $\times$  & $\times$  & $\times$  \\ \hline
            \textbf{45}  & 20.61  & 20.85  & 20.25  & 22.68  & 22.46  & 15.51  & $\times$  & $\times$ & $\times$  & $\times$  \\ \hline
            \textbf{50}  & 19.09  & 19.43  & 16.56  & 18.23  & $\times$  & $\times$  & $\times$ & $\times$  & $\times$  & $\times$  \\ \hline
        \end{tabular}
    \end{adjustbox}
\end{table*}


\begin{table*}[t]
    \centering
    \caption{Average Confidence scores for universal adversarial images using two stickers (first white, second black).}
    \label{table_two_sticker_first_white_second_black}
    \begin{adjustbox}{width=\textwidth}
    \small
        \begin{tabular}{|p{1.5cm}|p{1.5cm}|p{1.5cm}|p{1.5cm}|p{1.5cm}|p{1.5cm}|p{1.5cm}|p{1.5cm}|p{1.5cm}|p{1.5cm}|p{1.5cm}|p{1.5cm}|}
            \hline
            \textbf{Height, Width} & 
            \textbf{5} &
            \textbf{10} & \textbf{15} & \textbf{20} & \textbf{25} & \textbf{30} & \textbf{35} & \textbf{40} & \textbf{45} & \textbf{50} \\ \hline
             \textbf{5}     &      - &      - &  20.25 &  27.59 &  27.79 &  24.40 &  24.83 &  25.08 &  25.90 &  24.97 \\ \hline
\textbf{10}    &      - &      - &  19.35 &  28.09 &  23.58 &  23.84 &  25.01 &  24.88 &  22.58 &  32.95 \\ \hline
\textbf{15}    &  18.65 &  19.23 &  22.63 &  23.44 &  23.00 &  22.35 &  25.45 &  35.11 &  45.80 &  52.51 \\ \hline
\textbf{20}    &  14.99 &  21.98 &  22.46 &  21.66 &  34.01 &  47.95 &  58.98 &  64.28 &  65.50 &  69.16 \\ \hline
\textbf{25}    &  17.56 &  22.78 &  20.23 &  34.17 &  49.53 &  61.45 &  66.84 &  70.02 &  \textbf{70.39} &  67.56 \\ \hline
\textbf{30}    &  18.89 &  19.09 &  28.26 &  47.14 &  60.37 &  54.23 &  56.72 &  60.39 &  59.07 &  41.45 \\ \hline
\textbf{35}    &  13.76 &  22.06 &  36.16 &  56.99 &  65.09 &  57.93 &  57.57 &  62.80 &  50.05 &      $\times$ \\ \hline
\textbf{40}    &  14.07 &  24.37 &  39.47 &  59.52 &  64.66 &  55.28 &  60.09 &      $\times$ &      $\times$ &      $\times$ \\ \hline
\textbf{45}    &  15.85 &  27.32 &  42.01 &  57.20 &  58.19 &  55.44 &     $\times$ &      $\times$ &     $\times$ &      $\times$ \\ \hline
\textbf{50}    &  18.06 &  27.92 &  41.33 &  41.64 &      $\times$ &      $\times$ &      $\times$ &      $\times$ &    $\times$  & $\times$ \\ \hline
        \end{tabular}
    \end{adjustbox}
\end{table*}




\begin{table*}[t]
    \centering
    \caption{Average Confidence scores for universal adversarial images using two stickers (first white, second white).}
    \label{table_two_sticker_first_white_second_white}
    \begin{adjustbox}{width=\textwidth}
    \small
        \begin{tabular}{|p{1.5cm}|p{1.5cm}|p{1.5cm}|p{1.5cm}|p{1.5cm}|p{1.5cm}|p{1.5cm}|p{1.5cm}|p{1.5cm}|p{1.5cm}|p{1.5cm}|p{1.5cm}|}
            \hline
            \textbf{Height, Width} & 
            \textbf{5} &
            \textbf{10} & \textbf{15} & \textbf{20} & \textbf{25} & \textbf{30} & \textbf{35} & \textbf{40} & \textbf{45} & \textbf{50} \\ \hline
            \textbf{5}  &      - &      - &  20.25 &  29.68 &  29.86 &  31.11 &  31.63 &  31.71 &  31.39 &  31.36 \\ \hline
\textbf{10} &      - &  21.04 &  31.39 &  30.85 &  29.81 &  28.94 &  28.87 &  28.94 &  28.50 &  27.01 \\ \hline
\textbf{15} &  24.50 &  31.33 &  31.27 &  31.32 &  30.50 &  28.84 &  28.30 &  27.67 &  28.01 &  29.08 \\ \hline
\textbf{20} &  29.39 &  31.31 &  31.21 &  30.29 &  28.29 &  26.81 &  26.21 &  27.30 &  28.70 &  29.88 \\ \hline
\textbf{25} &  29.81 &  31.24 &  31.37 &  30.77 &  30.04 &  28.93 &  29.07 &  29.51 &  30.19 &  \textbf{39.16} \\ \hline
\textbf{30} &  31.04 &  31.43 &  30.92 &  28.73 &  27.31 &  29.05 &  29.23 &  28.88 &  29.73 &  30.94 \\ \hline
\textbf{35} &  30.51 &  25.26 &  26.34 &  21.57 &  22.40 &  23.39 &  24.67 &  21.71 &  22.82 &      $\times$ \\ \hline
\textbf{40} &  21.59 &  21.11 &  23.39 &  22.56 &  23.91 &  24.00 &  23.81 &      $\times$ &      $\times$ &      $\times$ \\ \hline
\textbf{45} &  20.00 &  20.23 &  23.39 &  21.52 &  22.37 &  22.53 &      $\times$&      $\times$ &      $\times$ &      $\times$ \\ \hline
\textbf{50} &  17.90 &  20.94 &  21.81 &  18.03 &      $\times$ &      $\times$ &      $\times$ &      $\times$ &      $\times$&      $\times$ \\ \hline
            
        \end{tabular}
    \end{adjustbox}
\end{table*}




