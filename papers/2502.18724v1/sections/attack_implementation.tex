\section{Universal Perturbation Attack on Street Signs}

This section introduces a universal perturbation attack targeting traffic sign classification systems. Unlike traditional per-instance adversarial attacks, our approach applies a common perturbation across multiple traffic signs to induce consistent misclassification. By strategically placing a universal adversarial mask in a fixed location, we demonstrate the vulnerability of deep learning models to input-agnostic physical perturbations. Our attack is evaluated on real-world traffic sign images, highlighting its robustness under varying environmental conditions. This study underscores the risks posed by universal perturbations and their potential impact on machine learning-driven traffic sign recognition systems.

\subsection{Attack Workflow}

Figure~\ref{fig_universal_perturbation_workflow} shows the attack workflow. The goal of this work is to design simple, yet effective framework for universal perturbations on street signs. In our context {\em universal perturbation means a perturbation that is applied to the same location on the face of any street sign.} As a first step to achieving this, the street signs need to be analyzed to generate the universal perturbation mask region. This is the region common to all street signs such that a perturbation, i.e. sticker, can be placed on all street signs in same location.

In order to find the common region, the street signs are analyzed to generate a mask, i.e. identify the region occupied by the street sign, as shown in middle-left of Figure~\ref{fig_universal_perturbation_workflow}. Next, the masks are merged, to generate the universal perturbation mask region, as shown in middle-right of Figure~\ref{fig_universal_perturbation_workflow}. Any sticker placed within the merged mask region is quarantined to fit on any of the street signs.

Next, the best location for the stickers is searched within the merged mask. An exhaustive search is performed where a sticker is placed at the same location on each street sign, and the confidence scores are collected. The location that gives the highest average confidence score (for a wrongly predicted street sign) is selected as the target location. Finally, the selected location is used to apply the sticker, as shown in bottom-left of the Figure~\ref{fig_universal_perturbation_workflow}. Note that the white cross-marks are only shown to illustrate and highlight that the stickers are placed in same location on all the street signs, they are not part of the attack.


The adversarial stickers used in this work are simple black or white stickers. Unlike existing work on (non-universal) adversarial stickers~\cite{9779913} which used various colored stickers, our work focuses on simple stickers with just two colors. Examples of stickers, overlaid on top of street signs, are shown later, e.g., in Figures~\ref{fig:one_sticker_black_Images} to~\ref{fig:two_sticker_white_white_Images}.

\begin{figure*}[t]
    \begin{subfigure}[b]{0.25\textwidth}
        \centering
\includegraphics[width=2.2cm]{plots/attack_images/one_sticker/black/Best_stop_50_40_best_sticker_size.png}
        \caption{\small \centering Stop Black Sticker Image}
        \label{fig:Best_stop_black_one_sticker}
    \end{subfigure}
    \hfill
    \begin{subfigure}[b]{0.25\textwidth}
        \centering
         \includegraphics[width=2.2cm]{plots/attack_images/one_sticker/black/Best_yield_50_40_best_sticker_size.png}
        \caption{\small \centering Yield Black Sticker Image}
        \label{fig:Best_yield_black_one_sticker}
    \end{subfigure}
    \hfill
    \begin{subfigure}[b]{0.25\textwidth}
        \centering
   \includegraphics[width=2.2cm]{plots/attack_images/one_sticker/black/Best_merge_50_40_best_sticker_size.png}
        \caption{\small \centering Merge Black Sticker Image}
        \label{fig:Best_merge_black_one_sticker}
    \end{subfigure}
    \vspace{-1em}
    \caption{Single Black sticker attack images.}
    \label{fig:one_sticker_black_Images}
\end{figure*}

\begin{figure*}[t]
    \begin{subfigure}[b]{0.25\textwidth}
        \centering
\includegraphics[width=2.2cm]{plots/attack_images/one_sticker/white/Best_stop_35_25_best_sticker_size.png}
        \caption{\small \centering Stop White Sticker Image}
        \label{fig:Best_stop_white_one_sticker}
    \end{subfigure}
    \hfill
    \begin{subfigure}[b]{0.25\textwidth}
        \centering
         \includegraphics[width=2.2cm]{plots/attack_images/one_sticker/white/Best_yield_35_25_best_sticker_size.png}
        \caption{\small \centering Yield White Sticker Image}
        \label{fig:Best_yield_white_one_sticker}
    \end{subfigure}
    \hfill
    \begin{subfigure}[b]{0.25\textwidth}
        \centering
   \includegraphics[width=2.2cm]{plots/attack_images/one_sticker/white/Best_merge_35_25_best_sticker_size.png}
        \caption{\small \centering Merge White Sticker Image}
        \label{fig:Best_merge_white_one_sticker}
    \end{subfigure}
    \vspace{-1em}
    \caption{Single White sticker attack images.}
    \label{fig:one_sticker_white_Images}
\end{figure*}
