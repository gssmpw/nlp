\documentclass[conference]{IEEEtran}
\IEEEoverridecommandlockouts
\usepackage{cite}
\usepackage{amsmath,amssymb,amsfonts}
\usepackage{algorithm}
\usepackage{algpseudocode}
\usepackage{graphicx}
\usepackage{textcomp}
\usepackage{xcolor}
\usepackage{balance}
\usepackage{epsfig}
\usepackage[left=0.6in,right=0.6in,top=0.72in, bottom=0.72in]{geometry}
\usepackage{float}
\usepackage{verbatim}
\usepackage{tabularx}
\usepackage{svg}

% BibTeX command fix for IEEEtran
\def\BibTeX{{\rm B\kern-.05em{\sc i\kern-.025em b}\kern-.08em
    T\kern-.1667em\lower.7ex\hbox{E}\kern-.125emX}}

\begin{document}

% Title
\title{Text2Net: Transforming Plain-text To A Dynamic Interactive Network Simulation Environment}

\author{
\IEEEauthorblockN{Alireza Marefat, Abbaas Alif Mohamed Nishar, Ashwin Ashok}\\
\IEEEauthorblockA{\textit{[amarefatvayghani1, amohamednishar1]}@student.gsu.edu,} \textit{aashok@gsu.edu}\\
\\ \IEEEauthorblockA{\textit{Georgia State University, Atlanta, USA}}
}

\maketitle
\noindent{\footnotesize \textcopyright\ 2025 IEEE. Personal use of this material is permitted. Permission from IEEE must be obtained for all other uses, \\
in any current or future media, including reprinting/republishing this material for advertising or promotional purposes, \\
creating new collective works, for resale or redistribution to servers or lists, or reuse of any copyrighted component of this work in other works.}

% Abstract
\begin{abstract}
This paper introduces Text2Net, an innovative text-based network simulation engine that leverages natural language processing (NLP) and large language models (LLMs) to transform plain-text descriptions of network topologies into dynamic, interactive simulations. Text2Net simplifies the process of configuring network simulations, eliminating the need for users to master vendor-specific syntaxes or navigate complex graphical interfaces. Through qualitative and quantitative evaluations, we demonstrate Text2Net's ability to significantly reduce the time and effort required to deploy network scenarios compared to traditional simulators like EVE-NG. By automating repetitive tasks and enabling intuitive interaction, Text2Net enhances accessibility for students, educators, and professionals. The system facilitates hands-on learning experiences for students that bridge the gap between theoretical knowledge and practical application. The results showcase its scalability across various network complexities, marking a significant step toward revolutionizing network education and professional use cases, such as proof-of-concept testing.
\end{abstract}

% Keywords
\begin{IEEEkeywords}
Network Simulation and Emulation, Educational Technology, AI in Education, Interactive Learning Environments, Network Configuration Automation, AI-driven Network
\end{IEEEkeywords}

% Main Sections
% 
% 
The widespread integration of communication networks and smart devices in modern control systems has increased the vulnerability of industrial systems to online cyber-attacks, e.g., Industroyer, Blackenergy, etc \citep{osti_1505628}.
% Modern control systems have seen a large push to include communication networks and smart devices to increase performance, made possible by improvements in communication device cost and energy consumption. This trend has been coupled with the usage of open-standard communication protocols among industrial control systems, making them vulnerable to online cyber-attacks such as Industroyer, Blackenergy, etc \citep{osti_1505628}. 
To counter this, methods have been developed to improve security by achieving attack detection, mitigation, and monitoring, among others \citep{sandberg2022secure}. This paper focuses on active attack diagnosis to mitigate stealthy attacks. 
%
%\subsection{Literature review}

Active diagnosis techniques rely on the inclusion of additional moduli to control systems
% inclusion within the control system of additional moduli 
to alter the behavior of the system compared to information known by the attacker. 
For instance, the concept of additive watermarking was introduced in \cite{mo2015physical}, where noise signals of known mean and variance are added at the plant and compensated for it at the controller. 
This compensation, however, is not exact, causing some performance degradation. Thus, trade-offs between performance and detectability  are necessary \citep{zhu2023detection}.
% A later work \citep{zhu2023detection} designs the watermark signal by trading performance for detection. Thus, although additive watermarking serves as a good detection scheme, they endure performance losses even in the nominal case. 

In encrypted control \citep{darup2021encrypted}, the sensor data is encrypted, sent to the controller, and then operated on directly. Encrypted input signals are sent back to the plant for decryption. Although encryption is widespread in IT security, in control systems it presents some concerns, such as the introduction of time delays \citep{stabile2024verifiable}, while it may present inherent weaknesses \citep{alisic2023model}.
% they are not preferred as they introduce time delays \citep{stabile2024verifiable} which can cause instability, and some encryption schemes can be very weak  \citep{alisic2023model}. 

In moving target defense \citep{griffioen2020moving}, the plant is augmented with fictitious dynamics, known to the controller. The plant output is transmitted to the controller along with the fictitious states over a network under attack. 
The additional measurements then aide in the detection of attacks. 
This comes at the cost of higher communication bandwidth needs, which increases rapidly with the dimension of the augmented systems.
% Since the dynamics of the fictitious dynamics are exactly known to the controller, the attack is detected easily. However, when the scale of the system increases, the communication bandwidth used by moving the target defense approach increases rapidly. 

Other recently proposed works include two-way coding \citep{fang2019two}, a weak encryuption technique, and dynamic masking \citep{abdalmoaty2023privacy}, which enhances privacy as well as security, have been shown to be effective against zero-dynamics attacks.
% Two-way coding \citep{fang2019two} and dynamic masking \citep{abdalmoaty2023privacy} are other recently proposed approaches. Two-way coding is another form of weak encryption technique whilst dynamic masking proposes an architecture that enhances both privacy and security. These schemes are shown to be effective against zero dynamics attacks but remain to be studied for other classes of attacks. 
% Recent extensions include \citep{mukherjee2021secure,ramos2024privacy}.
% Some other works which are related are \citep{mukherjee2021secure}, an extension of \cite{fang2019two}. The work \citep{ramos2024privacy} is an extension of moving target defense for multi-agent systems. 
Furthermore, filtering techniques for attack detection are proposed by \cite{murguia2020security,hashemi2022codesign,escudero2023safety}, while not focusing on stealthy attacks.
% The works \citep{murguia2020security,hashemi2022codesign,escudero2023safety} develop filtering techniques to guarantee safety, without being focused on stealthy covert attacks.

Multiplicative watermarking (mWM) has been proposed by the authors as a diagnosis technique \citep{ferrari2020switching}. mWM consists of a pair of filters on each communication channel between the plant and its controller; the scheme is affine to weak encryption, whereby ``encoding'' and ``decoding'' are done by changing signals' dynamic characteristics through inverse pairs of filters. This enables original signals to be recovered exactly, and thus does not lead to performance degradation.
% A multiplicative watermark is an affine to a weak encryption technique, through which the signal is ``encoded'' by a filter, changing its dynamic behavior. The use of inverse pairs means that the original signal can be recovered, through ``decoding'' via an inverse filter. As such, differently to techniques based on additive watermarking, no performance is lost due to the injection of noise, and there are no bandwidth limitations.

%\subsection{Contributions}
One of the critical features of multiplicative watermarking is that to detect stealthy attacks, the mWM filter parameters must be switched over time. In this paper, an algorithm to optimally design the mWM parameters after a switching event is presented, enhancing detection performance, without changing the switching time.
% This is done without changing the switching time, which is taken as given.

\textcolor{black}{
To formalize the filter design problem, we suppose the defender is interested in optimal performance against adversaries injecting covert attacks with matched system parameters \citep{smith2015covert}, including the mWM parameters prior to the switch. This scenario represents a worst case where malicious agents can take full control of the system while remaining undetected.
Thus, the attack strategy is explicitly included within the formulation of the closed-loop system, and the mWM filters are chosen by solving an optimization problem minimizing the attack-energy-constrained output-to-output gain (AEC-OOG) \citep{anand2023risk}, a variation of the output-to-output gain proposed in  \cite{teixeira2015strategic}.
}
The main contributions of this paper are:
% We consider an adversary injecting a covert attack with matched system parameters \citep{smith2015covert}, i.e., an attacker with full knowledge of the control system parameters, including those of the mWM filters before the switch. This scenario is taken as a worst case, as it has been shown that this class of attacks can be made stealthy. To quantitatively define a cost, the output-to-output gain (OOG) \citep{teixeira2015strategic} is leveraged,
% a metric introduced to evaluate the impact of an additive attack in a control system. %Specifically, OOG evaluates the worst-case performance loss that an attacker injecting an undetectable attack can obtain. 
% Here, the maximum performance loss caused by a stealthy adversary with limited energy is taken, the attack-energy-constrained OOG (AEC-OOG) \citep{anand2023risk}. The main contributions of this paper are:
\begin{enumerate}
%[label=\alph*.]
\item The problem of optimally designing the switching mWM filters is formulated as an optimization problem, with the AEC-OOG is taken as the objective;%where the AEC-OOG is taken as the impact metric; 
\item The worst-case scenario of a covert attack with exact knowledge of plant and mWM filter parameters is embedded within the design problem;
% The optimization problem is defined to incorporate the worst-case scenario of a covert attack with exact knowledge of plant and mWM filter parameters;
\item The feasibility of the optimization problem is shown to be dependent only on stability conditions; 
\item A solution scheme is proposed to promote randomization of the mWM filter parameters such that an eavesdropping adversary cannot remain stealthy.
\end{enumerate} 

This builds on the results of \cite{ferrari2020switching}, where the focus was on the design of the switching protocols, rather than the parameters themselves.
Compared to previous work \citep{gallo2021design}, this paper introduces an optimization problem which is always feasible (thanks to the use of AEC-OOG in the objective), while also considering a more sophisticated class of covert attacks, where the presence of watermark is known to the adversary. 
Moreover, this paper poses a different objective than \citep{zhang2023hybrid}; indeed, while \citep{zhang2023hybrid} provided a design strategy to ensure certain privacy properties, in this paper we address the problem of optimal parameter design following a switching event.


%\subsection{Organization}
The rest of the paper is organized as follows. 
After formulating the problem in Section~\ref{sec:PF}, we propose our design algorithm in Section~\ref{sec:main}, and analyze its properties. It is then evaluated through a numerical example in Section~\ref{sec:NE}, and concluding remarks are given Section~\ref{sec:Con}.
% We provide the problem background in Section~\ref{sec:PF}. We formulate the design problem in Section~\ref{sec:main}, together with an analysis of its properties. The proposed algorithm is evaluated through a numerical example in Section \ref{sec:NE}. Concluding remarks are offered in Section \ref{sec:Con}.
\section{Related Work}
% Goal-oriented dialogue requires agents to complete a specific task through multi-round dialogue~\cite{bordes2016learning,rajendran2018learning,williams2007partially}. 

% Although goal-oriented spoken and text-based dialogues have been studied for many years in the field of Natural Language Processing\cite{bordes2016learning,rajendran2018learning,williams2007partially}, goal-oriented visual dialogue moves the scene into a more realistic visual environment, making it a relatively more practical and challenging field. 

% The goal of GuessWhat?!~\cite{de2017guesswhat} is to distinguish a defined object in an image through dialogue, while the goal of GuessWhich~\cite{das2017learning} is to identify the correct image from a series of images. 

% There are usually two dialogue agents, Questioner and Oracle. The Questioner keeps asking questions to find the defined but undisclosed target, and the Oracle defines the target object in advance and answers questions accordingly.
% In a dialogue, there are typically two agent types, {\it i.e.}, the Questioner and the Oracle. The Questioner consists of two sub-models, QGen and Guesser. 
% They all involve QGen, Guesser and Oracle. 
% Our main focus is on the QGen. Please refer to the supplementary materials for more details about Oracle and Guesser.



% \subsection{Oracle}

% In the initial work of GuessWhat?!, a baseline Oracle was proposed, which concatenates the question encoding and the spatial and category information of the target object together and inputs them into the MLP layer to predict the final answer. However, without the introduction of visual information, the baseline Oracle may have difficulty understanding questions that involve color, shape, and object relations. Tu et al.\cite{tu2021learning} introduced visual features predicted by object detection models such as Faster-RCNN\cite{ren2015faster} into Oracle's decision-making process, but the way did not effectively help Oracle understand questions that involve information such as object relations or color.

% \subsection{Guesser}

% Guesser not only needs to perform referring expression comprehension for dialogue describing visual objects but also needs to perform reasoning. The initial work proposed a model that combines the encoding of the entire dialogue history with each object category and spatial information to predict the target object\cite{de2017guesswhat,strub2017end}. Later work\cite{shukla2019should,lu202012,deng2018visual} treated the entire dialogue history as a whole. However, the Guesser model does not encode any visual information. Considering that the lack of turn-level visual grounding can cause the Guesser to confuse the object referred to in each question, some methods\cite{simonyan2014very,pang2020guessing} introduced features such as VGG and Faster-RCNN into the Guesser model. Considering the dynamic characteristic of multi-turn dialogue reasoning, Pang et al.\cite{pang2020guessing} proposed to decompose the dialogue into turn-level and use state tracking to dynamically update the guessing confidence, demonstrating a significant performance improvement. Recent work\cite{tu2021learning} introduced a Visual-Linguistic pre-trained model, giving the agent more visual language shared representations and prior knowledge, which has achieved good results.


% \subsection{Question Generator}

%CHANGED-0614
\subsection{Question Generator (QGen)}
% \textbf{QGen.} 
The QGen plays a core role in the goal-oriented visual dialogue, as it not only needs to ask questions that can acquire certain information gain but also guides the dialogue towards the direction of the target.  
De Vries et al.~\shortcite{de2017guesswhat} propose the first QGen model with an encoder-decoder structure, in which the dialogue history is encoded by a Hierarchical Recurrent Encoder-Decoder (HRED)~\cite{serban2015hierarchical}, and the image is conditionally encoded as VGG features~\cite{simonyan2014very}.
Strub et al.~\shortcite{strub2017end} introduce the approach of RL and provide a 0-1 reward, where 1 indicates successful finding of the target in the dialogue. Built upon this approach, Zhang et al.~\shortcite{zhang2018goal} propose intermediate rewards from three dimensions to improve the model performance. 
Shekhar et al.~\shortcite{shekhar2018beyond} introduce a shared dialogue state encoder for Guesser and QGen, in which the visual encoder is based on ResNet~\cite{he2016deep}, and the language encoder is based on LSTM~\cite{hochreiter1997long}. Pang et al.~\shortcite{pang2020visual} introduce a turn-level object state tracking mechanism to QGen. Tu et al.~\shortcite{tu2021learning} introduce a Visual-Linguistic pre-trained model to QGen, which makes the object's semantic coverage more comprehensive and better.
Our main focus is on how to train QGen. 
The fundamental difference between TSADE and prior work lies in its clever use of a non-goal-oriented questioning strategy~(NGOQS) to find target, whereas prior works~\cite{zhang2018goal,shukla2019should,testoni2021looking} utilize a goal-oriented questioning strategy~(GOQS). 
We experimentally prove that flexibly using NGOQS is more useful than simply using GOQS, and GOQS can benefit from NGOQS.



%Please refer to the supplementary materials for the difference between our method and prior work, as well as for more details about Oracle and Guesser.
% Please refer to the supplementary materials for more details about Oracle and Guesser.



\subsection{Answer Distribution Estimator (ADE)}
% \textbf{Answer Distribution Estimator (ADE).}
Given a question, ADE actually employs an internal Oracle to answer all objects in the image to obtain an answer distribution. Lee et al.~\shortcite{lee2018answerer} first introduce the ADE module to propose an Answerer in Questioner’s Mind (AQM) algorithm to obtain question in each round.
In this work, ADE refers to an approximated model of the original Oracle explicitly trained by AQM's Questioner. 
It abandons the paradigm of deep learning, and uses mathematics and the approximated model to directly calculate information gain to select question from training data in each round. 
% However, this paradigm of selecting question from training data has great limitations. The fixed training data usually can't cover the huge actual scenes in life. 
% And the information gain of all training data must be calculated in each round, making the calculation cost very high.
% Different from AQM, TSADE is a paradigm based on question generation, which has stronger generalization and lower computational cost. TSADE employs the answer distribution to dynamically update the real-time candidate objects and calculate reward score for the quality of each question. 
% Then the reward score is put into RL to optimize question generation.
Zhang et al.~\shortcite{zhang2018goal} propose three intermediate rewards to optimize the model in RL. 
% It explicitly obtains higher rewards with fewer rounds. 
Based on the goal-oriented way, it hope that the probability of ground truth (target) will progressively increase during the whole process. It uses ADE to avoid useless questions based on answer distribution. However, it does not consider what kind of questions are most useful. The difference is that TSADE takes the issue into account and uses ADE to achieve the same final goal in a non-goal-oriented way, without paying attention to which target is during the whole process.
Testoni and Bernardi \shortcite{testoni2021looking} propose the ``confirm-it'' strategy to select question that can gradually increase the probability of the target from the candidate questions. It uses an internal Oracle to provide answers specific to the target for a set of candidate questions. These answers are then used by the Guesser to compute a probability distribution over candidate objects. 
% In contrast, TSADE uses the internal Oracle to obtain an answer distribution over the candidate objects. The former's internal Oracle responds to target based on a set of questions, while the latter's internal Oracle responds to candidate objects based on a single question.
%We can see that existing methods do not have an efficient and intuitive strategy to guide question generation. Previous research\cite{strub2017end,shukla2019should,zhang2018goal,zhao2018improving} has used Reinforcement Learning methods to learn the Questioner/Guesser model by designing different rewards, such as end-game success or information gain from question generation. However, the question-generation strategy under these methods is fuzzy, uninterpretable, and inefficient. This paper proposes an Answer Distribution Estimator (ADE) that explicitly uses a binary search strategy to guide question generation, further integrates the state distributions of different agents, and enhances the fusion of visual and textual information.

% \begin{figure}[h]
%   \centering
%   \includegraphics[width=0.8\linewidth]{images/fig1_emnlp.pdf}
%   \caption{It shows an example of the GuessWhat?! game that describes the process of attention transfer in dialogue based on the Tree-structured strategy. The excluded objects are in the lower-right candidate box. The target object is highlighted in green box.}
%   \label{fig:example of strategy}
% \end{figure}
\section{Text2Net Methodology}
As depicted in Figure~\ref{fig: system model}, Text2Net comprises five modules: User, Software-Adaptor, instructed LLM, Text Extractor, and Simulator. The user inputs a network topology scenario, which the Software-Adaptor forwards to the instructed LLM, OpenAI's ChatGPT-4T, via API calls. The LLM processes this input and returns Structured Command Strings (SCS) to the Software-Adaptor. Utilizing NLP tools (SpaCy), along with RegEx and pattern matching, the Software-Adaptor extracts the desired key-value pairs, formatting them into a JSON dictionary. This JSON file is input to the EVE-NG emulator to provision the live network topology and configurations. 

\begin{figure}[ht!]
    \centering
    \includegraphics[width=\columnwidth]{Figures/flowchart_svg-raw.pdf}
    \caption{System flow perspectives.}
    \vspace{-3mm}
    \label{fig: flowchart}
    \vspace{-3mm}
\end{figure}

\noindent{\bf Overall System Flow.}
Figure ~\ref{fig: flowchart} depicts the system model, showcasing both, the system and user perspectives.
As illustrated on the right side of Figure~\ref{fig: flowchart}, the system flow from the user perspective begins with the initiation of Text2Net followed by the display of the welcome page. Text2Net greets the user and prompts for a network topology scenario. If the input scenario is valid and includes all required technical details, the system proceeds to provision the network topology in the emulator. If the scenario lacks necessary information, the system requests further details from the user. Ultimately, the user can interact with the fully configured live network topology.
The left side of Figure~\ref{fig: flowchart} illustrates the background processes of the system, which operate transparently to the user and concurrently with the functions depicted on the right side, highlighted in the same color, which will be discussed in detail in this section. 
%The initial step involves launching the emulator environment on a cloud virtual machine and starting the Software Adaptor code. Following this, the system processes the user input through the instructed LLM, retrieving the relevant SCS after performing necessary sanity checks and adhering to provided arguments and instructions. Subsequently, using NLP and Regex pattern matching, the system constructs a key-value pair dictionary in JSON format. The next step involves connecting to the EVE-NG emulator to create nodes, link interfaces, establish Telnet sessions for each node, and push configurations to test the scenario. Once these steps are completed, the fully provisioned network topology is ready for further user interaction.

\vspace{-2mm}
\subsection{System preparation}

%\subsubsection{Preparing the Simulation Environment}
The initial setup of Text2Net involves using EVE-NG, a network emulator (it is beyond simulator that replicates real-world environments and supports actual device images from manufacturers like Cisco, Juniper, and HPE). For Text2Net, Cisco devices are primarily used due to their commonality in networking. EVE-NG can be deployed either via an ISO file on virtual machines like VMware or directly on physical hardware to avoid performance issues associated with nested virtualization. To broaden Text2Net’s accessibility, it is hosted on the Google Cloud Platform (GCP) using an n2-standard-8 machine with 8 vCPUs and 32 GB of memory, running Linux Ubuntu. After the EVE-NG installation, the system is configured with a static IP, and HTTPS and SSH ports are opened to ensure it is accessible from any location, verified by navigating to the public IP address in a web browser to reach the EVE-NG login screen.

To leverage OpenAI's ChatGPT-4T for Text2Net, we trained the model to interpret and generate SCS from plain text descriptions of network topologies, commonly presented in computer network lectures. The model was trained to precisely extract and structure key information into command strings with key-value pairs essential for network topology provisioning.
The model recognizes detailed textual descriptions of network setups, outputting accurate command strings without superfluous content. For valid, complete inputs, the model confirms with returning the phrase "Understood", moving to the next phase. For inputs that are empty, incomplete, or incorrect, it prompts the user to refine their input.
The initial user interaction with Text2Net involves a user-friendly interface where users are prompted to input network topologies in plain English. This input becomes the basis for generating network configurations. A significant challenge was standardizing how users describe network topologies; this was addressed through a qualitative survey to establish a standard input format.
Text2Net is equipped to assess the validity and completeness of user inputs, ensuring no essential details, like IP formatting or technical configurations, are missing. This capability ensures the system efficiently transitions from user input to network configuration. 
When an input scenario includes an invalid IP address, such as "192.168.0.300," Text2Net automatically detects the error. For configurations involving specific protocols like static routing that lack necessary details, the system does not simply accept the input. Instead, it prompts the user \textit{``Please provide additional details about the static route"} before proceeding with generating the SCS.

\vspace{-2mm}
\subsection{Extracting Structured Command Strings}
To efficiently extract key-value pairs from the plain text, the text is decomposed into segments known as Structured Command Strings a.k.a. SCSs. These SCSs are derived from the plain text by employing the instructed GPT-4T model. Each SCS consists of short strings that encapsulate one or a few specific key-value pairs, ensuring clarity and specificity in data extraction.
Thanks to prompt engineering, the system is able to extract the same SCSs as an output from the following different scenarios that explain the same network topology with different styles of explanation. Figure~\ref{fig: network topology} shows the network diagram for which we have considered input/output configuration testing across three different user-input scenarios that may be possible:

\begin{figure}[t!]
    \centering
    \includegraphics[scale=0.3]{Figures/network_topology.png}
    \caption{Static-route Scenario Network Topology}
    \vspace{-6mm}
    \label{fig: network topology}
\end{figure}

\vspace{1mm}\noindent{\bf Scenario 1 -}
\textit{“R-1” is a router that is connected to “R-2”. “R-1” interface gi 0/0 has IP address 192.168.0.1/24 and is connected to  “R-2” interface Gi 0/0 with IP address 192.168.0.2/24. 
R-2 is connected to “R-3” via interface Gi 0/1 using IP address 192.168.100.1/24. “R-3” is connected back to “R-2” using interface Gi 0/0 with IP address 192.168.100.2/24. 
A static route is configured on “R-1” to reach “R-3” as well as a static router on “R-3” to reach to “R-1” through “R-2”.}


\vspace{1mm}\noindent{\bf Scenario 2 -}
\textit{The network has three routers: R-1, R-2, and R-3, interconnected in a specific manner.
R-1 connects to R-2 through its interface Gi 0/0, with the IP address 192.168.0.1/24, while R-2's corresponding interface, Gi 0/0, has the IP address 192.168.0.2/24.
R-2 establishes a connection with R-3 via interface Gi 0/1, with R-2 assigned the IP address 192.168.100.1/24 for this link.
The reverse connection from R-3 to R-2 is achieved through R-3's interface Gi 0/0, configured with the IP address 192.168.100.2/24.
For seamless communication between R-1 and R-3, static routes are set up on both routers through R-2, ensuring efficient routing between them.}


\begin{figure}[ht!]
    \centering
    \includegraphics[width=1\columnwidth]{Figures/verbatim_svg-raw.pdf}
    % \includesvg[inkscapelatex=false, scale=0.6]{Figures/venn diagram.svg}
    \caption{Structured Command Strings (SCSs)}
    \vspace{-4mm}
    \label{fig: SCS}
\end{figure}

% \begin{small}
% \begin{verbatim} 
% - device-1 type is router
% - device-1 name is R-1
% - device-1 interface Gi0/0 has IP address 192.168.0.1 with subnet- 
% mask 255.255.255.0
% - device-1 interface Gi0/0 is connected to device-2 interface Gi0/0
% - device-2 type is router
% - device-2 name is R-2
% - device-2 interface Gi0/0 has IP address 192.168.0.1 with subnet-
% mask 255.255.255.0
% - device-2 interface Gi0/0 is connected to device-1 interface Gi0/0
% - device-2 interface Gi0/1 has IP address 192.168.100.1 with subnet- 
% mask 255.255.255.0
% - device-2 interface Gi0/1 is connected to device-3 interface Gi0/0
% - device-3 type is router
% - device-3 name is R-3
% - device-3 interface Gi0/0 has IP address 192.168.100.2 with subnet- 
% mask 255.255.255.0
% - device-3 interface Gi0/0 is connected to device-2 interface Gi0/1
% - static route configured from R-1 to R-3 through R-2
% - static route configured from R-3 to R-1 through R-2
% \end{verbatim}
% \end{small}

Figure~\ref{fig: SCS} shows the same output as SCS from the two above scenarios in our system. Such examples demonstrate Text2Net's capability to deliver consistent output across different storytelling (variable network configuration explanations from users) approaches, provided that the underlying network topology remains the same. This highlights the robustness of this system in recognizing and interpreting the essential elements of network configurations, even when the narrative descriptions vary. This consistency ensures that Text2Net can be reliably used, and is easy to program for changes in input styles, in educational settings where diverse narrative styles are employed to describe similar network setups.

The scenario below illustrates how the same network topology can be described differently, akin to the variations tested in the previous two scenarios. However, there is a key difference, this description lacks detailed information about ``static routing". This incomplete scenario provides an opportunity to observe how Text2Net manages scenarios where critical information is missing.

\vspace{-2mm}
\SetAlgoLined
\DontPrintSemicolon
\begin{algorithm}[]
\caption{\method: API Generation}
\label{algo:api}
\SetKwData{Data}{Data}
\SetKw{Continue}{continue}
\SetKwInOut{KwInit}{Initialize}
\SetKwProg{Fn}{Function}{}{end}

\KwData{Questions $\mathcal{Q}$}
$\mathcal{S} \gets \{\} $\tcp*[r]{Signatures}
$\mathcal{A} \gets \{\text{Vision Models}\}$\tcp*[r]{API Methods}
\For{batch $B \subset \mathcal{Q}$} {
    $\mathcal{S} \gets \mathcal{S} \cup  \texttt{SignatureAgent}(B)$
}
\For{$S \in \mathcal{S}$} {
    $e_S \gets 0$ \tcp*[r]{Error count}
    $A \gets \texttt{ImplementationAgent}(S)$ \;
    $E \gets \texttt{TestAgent}(A)$\;
    \eIf{Python Exception $E$}{
        \lIf{$e_S = 5$}{\Continue}
        \uElseIf{$E$ is ``undefined method $U$''}{
            $e_S \gets e_S + 1$\;
            Recursively implement $U$
        }
        \Else{
            $e_S \gets e_S + 1$\;
            Re-implement $S$ using $E$
        }
    } {
    $\mathcal{A} \gets \mathcal{A} \cup A$
    }
}
\Return{$\mathcal{A}$}
\end{algorithm}
\vspace{-3mm}
\noindent{\bf Scenario 3-}
\textit{This network architecture is designed to facilitate efficient communication between multiple network segments, each identified by distinct IP subnets.
Routers R-1, R-2, and R-3 serve as intermediaries for routing data packets between these segments.
R-1 functions as the gateway router, connecting a potential local network segment to the wider network. It is directly linked to R-2 through its interface Gi 0/0, with R-1's IP address on this interface being 192.168.0.1/24 and R-2's IP address set to 192.168.0.2/24.
R-2 operates as a central hub, facilitating connections between multiple network segments. It interfaces with R-1 through Gi 0/0 and with R-3 through Gi 0/1. On interface gi 0/1, R-2 is assigned the IP address 192.168.100.1/24.
R-3 serves as a bridge between different network segments. It connects back to R-2 through its interface Gi 0/0, configured with the IP address 192.168.100.2/24.
The network's functionality relies on the careful configuration of IP addresses and static routes. This ensures that data packets are routed efficiently between devices connected to R-1, R-2, and R-3, facilitating seamless communication across the entire network infrastructure.}

In this case, the SCSs is generated but not for the static route section. Text2Net detected the missing information and specifically asked about it when returned ``However, I need additional information about the static routing configuration to provide complete command strings. Could you specify the source, destination, and through devices for each static route?''


\vspace{-2mm}
\subsection{Extracting Key-value pairs}

\iffalse
\begin{figure}[ht!]
    \centering
    \includegraphics[width=0.65\columnwidth]{Figures/venn_diagram.svg}
    % \includesvg[inkscapelatex=false, scale=0.6]{Figures/venn diagram.svg}
    \caption{Entities relationship}
    \vspace{-6mm}
    \label{fig: entity-relationship}
\end{figure}
\fi

To develop a comprehensive key-value pair dictionary, the first step is to establish a detailed entity relationship. Understanding the scope of the system is crucial for designing the relationships between entities to structure the corresponding JSON dictionary effectively. In the current phase of Text2Net, we leveraged RegEX and pattern matching to implement the system's functionality for routing in networking. However, scaling this approach to cover all networking concepts would be labor-intensive and inefficient. As future work, we aim to explore more on NLP techniques, also including Retrieval Augmented Generators (RAGs) in our model, to enhance scalability and extend the system's capabilities.

%Figure~\ref{fig: entity-relationship} illustrates the entities and their relationships that are used in Text2Net. 
% Main entities are "Devices" and "Node Configurations" each comprising sub entities that have relationship to each other. Devices includes sub-entities like Name, Network Attributes, and Type. Network Attributes contains its child entities such as "Int" for interface, "IP" for IP address, "SM" for Subnet Mask, "NN" for Network Number, "NM" for Network Mask, "WC" for Wildcard, and "NID" for Network Identifier. On the other hand, there are three different Types defined for Text2Net such as Router, Switch, and PC (Personal Computer). 
% %This will be further expanded later in future works to support more devices like firewalls. 
% Node Configuration has two sub-entities called "Basic Configurations" and "L3 Configurations". Basic Configuration includes "HN" for Host Name that is connected to Names, and "Links" which itself is connected to Int, IP, SM, and NN. L3 Configurations contains three of the most popular Layer 3  routing protocols such as Static routing, OSPF (Open-Shortest Path First), and EIGRP (Enhanced Interior-Gateway Routing Protocol).


\begin{figure}[ht!]
    \centering
    \includegraphics[width=0.55\columnwidth]{Figures/verbatim_svg-raw.pdf}
    % \includesvg[inkscapelatex=false, scale=0.6]{Figures/venn diagram.svg}
    \caption{key-value pairs output}
    \vspace{-3mm}
    \label{fig: key-value pairs output}
\end{figure}

Algorithm~\ref{algorithm} facilitates the structured extraction and processing of network topology data. $\mathcal{D} = \{d_1, d_2, \ldots, d_n\}$ represents the set of all devices, where in  $\mathcal \{d_1, d_2, \ldots, d_n\}$ each element is a tuple containing key-value pairs $\mathcal(k, v)$ that describe network devices parameters and their information. The output, $\mathcal{J}$, is a JSON object structured to include detailed device and connection configurations necessary to be used for the network simulation. During processing, $\mathcal{V}$ serves as a temporary dictionary to accumulate the detailed attributes of each device, while $\mathcal{C}$ and $\mathcal{L}$ store basic and Layer 3 configurations respectively. \textit{l} represents each line of SCS that the algorithm iterates through for extraction. Functions such as \verb|ExtractNode()|, \verb|ExtractName()|, and \verb|ExtractInterfaceDetails()| parse specific details from textual descriptions. \verb|AssignUniqueID()| assigns unique network identifiers, ensuring each component is distinctly recognized in the simulation environment. Together, these elements systematically transform plain text input into a structured format that is both accurate and suitable for any layer3 topology generation.
Figure~\ref{fig: key-value pairs output} shows Key-Value pair dictionary based JSON output is a generated template for one device to show case the format of the algorithm output.


% \begin{small}
% \begin{verbatim}
%     "devices": [
%         {
%             "name": "device_1",
%             "details": [
%                 {"Node_Type": "router"},
%                 {"Node_Name": "R-1"},
%                 {
%                     "interface": "Gi 0/0",
%                     "ip_address": "192.168.0.1",
%                     "subnet_mask": "255.255.255.0",
%                     "network_number": "192.168.0.0",
%                     "network_mask": "255.255.255.0",
%                     "wildcard_mask": "0.0.0.255",
%                     "network_id": "Network_1"
%                 }
%             ],
%             "node_configs": {
%                 "basic_configs": {
%                     "hostname": "R-1",
%                     "interfaces": ["Gi 0/0"]
%                 },
%                 "L3_configs": {
%                     "static_route": {
%                         "interface": "Gi 0/0",
%                         "network": "192.168.0.0",
%                         "mask": "255.255.255.0"
%                     }
%                 }
%             }
%         }
% \end{verbatim}
% \end{small}
% \vspace{-2mm}

\vspace{-2mm}
\subsection{Network topology provisioning}
The integration with the simulation environment, EVE-NG, for network topology provisioning leverages the previously structured JSON containing key-value pairs of all devices, including their technical configurations. This JSON serves as the blueprint for the entire network topology within EVE-NG.\\ 
\noindent{\bf Preparation and Initialization:}
The process begins by parsing the structured JSON, which details each network device's required configuration such as device type, interfaces, and routing protocols. This data guides the creation and configuration of each virtual device within EVE-NG.\\
\noindent{\bf Node Creation:}
The \verb|create_node| function dynamically creates nodes in EVE-NG based on the specifications extracted from the JSON. It configures various attributes like device type, associated images, and hardware specifications (CPU, RAM). Depending on whether the node is a router, switch or PC, specific templates and additional parameters such as QEMU options are set. \\
%The function also determines the placement of these nodes within the virtual lab space, either by generating random coordinates or using predefined positions to ensure an organized layout.
\noindent{\bf Network Linking:}
Following node creation, the \verb|create_network| function establishes links between the nodes. This function reads interface details directly from the JSON and uses them to configure correct connections, ensuring that all network interfaces are linked as per the topology requirements.\\ 
%It manages network visibility and bridge settings through EVE-NG’s API, effectively translating logical designs into operational network setups.
\noindent{\bf Operational Execution and API Interaction:}
With nodes and networks in place, the script executes operational commands to activate and initially configure the devices via the EVE-NG API. This includes setting up interfaces and applying any predefined network routes or policies as specified in the JSON. Each action is carefully monitored through the API responses to handle exceptions and ensure successful deployment.\\
\noindent{\bf Session Management and Debugging:}
The entire process is supported by robust session management, where authentication and session cookies are handled to maintain a persistent connection with the EVE-NG API. Debugging information, such as device creation and network linkage statuses, is logged to assist in troubleshooting and validating the network setup.








\begin{figure}[t]
    \centering
    \includegraphics[width=\columnwidth]{Figures/scenario1.png}
    \caption{Steps and time comparison for Text2Net and EVE-NG Network Simulator - Scenario1}
    \vspace{-4mm}
    \label{fig: scenario1}
\end{figure}

\section{Evaluation}
The evaluation of Text2Net was conducted using both qualitative and quantitative methods. For the quantitative analysis, we compared Text2Net’s performance with manual configuration in the EVE-NG simulation environment, focusing on two parameters: time and steps. We measured only the time to input commands, excluding thinking or troubleshooting time, ensuring the results represent the best-case scenario for manual configuration. This assumption of an error-free manual process further highlights Text2Net’s competitiveness.

To assess scalability and efficiency, we analyzed three network scenarios of increasing complexity. The results demonstrate that Text2Net significantly reduces time and steps, with its advantages growing as complexity increases. This improvement stems from eliminating repetitive commands and tasks inherent in traditional workflows.

Scenario 1 involves configuring a router with basic settings, including date/time, hostname, disabling DNS lookups, configuring an interface, and verifying the configuration. As shown in Figure~\ref{fig: scenario1} (Gantt chart), manual configuration in EVE-NG requires 12 steps, such as launching the simulator, logging in, creating a lab environment, starting a node, and configuring the interface. These steps, common across network platforms, take 200 seconds.
In contrast, Text2Net completes the same task in just two steps and 110 seconds with the prompt:
\textit{“Configure a router as R1 with basic setup. Configure the interface Fast Ethernet 0/1 with IP address 192.168.0.1 and subnet mask 255.255.255.0, and finally check the configurations.”}



Scenario 2 introduces greater complexity with two routers, each having internal networks configured as loopback interfaces and interconnected with static routes. The steps from Scenario 1 are repeated for each node, including configuring loopback interfaces, setting static routes, verifying configurations, and running ping tests. Manual configuration in EVE-NG requires 510 seconds, while Text2Net completes the task in 250 seconds using the prompt:
\textit{“Configure Router 1 as R1 and Router 2 as R2 with basic configurations. On R1, configure the interface Fast Ethernet 0/1 with IP address 192.168.0.1 and subnet mask 255.255.255.0. Configure loopback 1 interface to act as Network 1 with IP address 192.168.1.1/24. On R2, configure the interface Fast Ethernet 0/1 with IP address 192.168.0.2 and subnet mask 255.255.255.0. Configure loopback 1 interface to act as Network 2 with IP address 192.168.2.1/24. Set static routes from R1 to R2 and vice versa. Finally, check the configurations.”}





\begin{figure}[t]
    \centering
    \includegraphics[width=\columnwidth]{Figures/scenario2.png}
    \caption{Steps and time comparison for Text2Net and EVE-NG Network Simulator - Scenario2}
    \vspace{-4mm}
    \label{fig: scenario2}
\end{figure}


\begin{figure}[t]
    \centering
    \includegraphics[width=\columnwidth]{Figures/scenario3.png}
    \caption{Steps and time comparison for Text2Net and EVE-NG Network Simulator - Scenario3}
    \vspace{-4mm}
    \label{fig: scenario3}
\end{figure}


\begin{figure*}[t]
    \centering
    \includegraphics[scale=0.44]{Figures/qualitative-result_svg-raw.pdf}
    \caption{Consolidated Benefits and Ratings for Text2Net}
    \vspace{-5mm}
    \label{fig: consolidated_results}
\end{figure*}


Scenario 3 adds a third router, acting as a transit node between the two from Scenario 2. The static route from Router 1 (R1) now targets Router 3 (R3) via Router 2 (R2), which lacks an internal network. Manual configuration in EVE-NG requires 10 steps, including repeating all tasks from Scenario 1 for each node (R1, R2, and R3), configuring additional interface links, and setting static routes. Completing this scenario manually takes 730 seconds.
Text2Net reduces this to 310 seconds with the prompt:
\textit{“R1 is a router connected to R2. R1 interface Gigabit Ethernet 0/0 has IP address 192.168.0.1/24 and is connected to R2 interface Gigabit Ethernet 0/0 with IP address 192.168.0.2/24. R2 is connected to R3 via interface Gi 0/1 using IP address 192.168.4.1/24. R3 is connected back to R2 using interface Gi 0/0 with IP address 192.168.4.2/24. R1 has a loopback interface 1 with IP address 192.168.1.1/24 to act as internal network-1. R3 also has a loopback interface 1 with IP address 192.168.2.1/24 to act as internal network-2. A static route is configured on R1 to reach R3, and another static route on R3 to reach R1 through R2.”}

Across Scenarios 1, 2, and 3, Text2Net consistently outperforms manual configuration, requiring 110, 250, and 310 seconds, compared to 200, 510, and 730 seconds in EVE-NG. This demonstrates Text2Net’s scalability and efficiency in handling increasingly complex configurations.
Figure~\ref{fig: consolidated_results} summarizes the qualitative evaluation of Text2Net, based on feedback from 15 participants, including graduate students, professors, and engineers. Participants highlighted Text2Net’s ability to reduce errors, repetitive tasks, and setup time, making it more efficient compared to traditional methods. Additionally, the system was noted for simplifying simulation workflows and providing practical insights into real-world network scenarios. Text2Net received high ratings for ease of use, transparency, and educational value, achieving an average score of 4.66 out of 5, demonstrating its potential as a transformative tool for both academic and professional applications.









Software development is increasingly conceived as a collaboration activity between developers and AIs. Indeed, IDEs already implement features to enable interactive development, with AI suggesting implementations that are reused by developers.

Although multiple studies show this interaction can be successful, there is still limited understanding of how the models must be configured and used in the context of code generation tasks. This study addresses this gap, systematically investigating the impact of several key parameters, including the repeated submission of a prompt to accommodate for the non-deterministic nature of the models.

Our study reveals several key findings about the usage of ChatGPT. In particular, we discovered how creativity, although up to a limited extent, is useful to increase the range of methods whose code can be generated correctly. A major role is played by parameter top-p, which is commonly underrated, and instead has a major impact on the correctness of the results, with lower values producing better results. Finally, prompts should be submitted multiple times, with $5$ repetitions combined with a temperature of $1.2$ resulting in an effective configuration in our experiments.  

Future work concerns two main research directions. One is about replicating this experiment with other AI assistants, to validate our findings in multiple contexts. The second research direction concerns finding strategies to deal with the need to submit the same prompt multiple times to obtain a useful result, and thus developing approaches able to select or merge multiple responses automatically. 

% Acknowledgements
\vspace{-2mm}\section{Acknowledgements}
This work was supported by the National Science Foundation (NSF CAREER CNS-2146267).
\vspace{-2mm}

% Bibliography
\bibliographystyle{IEEEtran}
\bibliography{REFERENCES}

\end{document}
