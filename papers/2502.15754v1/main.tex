\documentclass[conference]{IEEEtran}
\IEEEoverridecommandlockouts
\usepackage{cite}
\usepackage{amsmath,amssymb,amsfonts}
\usepackage{algorithm}
\usepackage{algpseudocode}
\usepackage{graphicx}
\usepackage{textcomp}
\usepackage{xcolor}
\usepackage{balance}
\usepackage{epsfig}
\usepackage[left=0.6in,right=0.6in,top=0.72in, bottom=0.72in]{geometry}
\usepackage{float}
\usepackage{verbatim}
\usepackage{tabularx}
\usepackage{svg}

% BibTeX command fix for IEEEtran
\def\BibTeX{{\rm B\kern-.05em{\sc i\kern-.025em b}\kern-.08em
    T\kern-.1667em\lower.7ex\hbox{E}\kern-.125emX}}

\begin{document}

% Title
\title{Text2Net: Transforming Plain-text To A Dynamic Interactive Network Simulation Environment}

\author{
\IEEEauthorblockN{Alireza Marefat, Abbaas Alif Mohamed Nishar, Ashwin Ashok}\\
\IEEEauthorblockA{\textit{[amarefatvayghani1, amohamednishar1]}@student.gsu.edu,} \textit{aashok@gsu.edu}\\
\\ \IEEEauthorblockA{\textit{Georgia State University, Atlanta, USA}}
}

\maketitle
\noindent{\footnotesize \textcopyright\ 2025 IEEE. Personal use of this material is permitted. Permission from IEEE must be obtained for all other uses, \\
in any current or future media, including reprinting/republishing this material for advertising or promotional purposes, \\
creating new collective works, for resale or redistribution to servers or lists, or reuse of any copyrighted component of this work in other works.}

% Abstract
\begin{abstract}
This paper introduces Text2Net, an innovative text-based network simulation engine that leverages natural language processing (NLP) and large language models (LLMs) to transform plain-text descriptions of network topologies into dynamic, interactive simulations. Text2Net simplifies the process of configuring network simulations, eliminating the need for users to master vendor-specific syntaxes or navigate complex graphical interfaces. Through qualitative and quantitative evaluations, we demonstrate Text2Net's ability to significantly reduce the time and effort required to deploy network scenarios compared to traditional simulators like EVE-NG. By automating repetitive tasks and enabling intuitive interaction, Text2Net enhances accessibility for students, educators, and professionals. The system facilitates hands-on learning experiences for students that bridge the gap between theoretical knowledge and practical application. The results showcase its scalability across various network complexities, marking a significant step toward revolutionizing network education and professional use cases, such as proof-of-concept testing.
\end{abstract}

% Keywords
\begin{IEEEkeywords}
Network Simulation and Emulation, Educational Technology, AI in Education, Interactive Learning Environments, Network Configuration Automation, AI-driven Network
\end{IEEEkeywords}

% Main Sections
\section{Introduction}

Large language models (LLMs) have achieved remarkable success in automated math problem solving, particularly through code-generation capabilities integrated with proof assistants~\citep{lean,isabelle,POT,autoformalization,MATH}. Although LLMs excel at generating solution steps and correct answers in algebra and calculus~\citep{math_solving}, their unimodal nature limits performance in plane geometry, where solution depends on both diagram and text~\citep{math_solving}. 

Specialized vision-language models (VLMs) have accordingly been developed for plane geometry problem solving (PGPS)~\citep{geoqa,unigeo,intergps,pgps,GOLD,LANS,geox}. Yet, it remains unclear whether these models genuinely leverage diagrams or rely almost exclusively on textual features. This ambiguity arises because existing PGPS datasets typically embed sufficient geometric details within problem statements, potentially making the vision encoder unnecessary~\citep{GOLD}. \cref{fig:pgps_examples} illustrates example questions from GeoQA and PGPS9K, where solutions can be derived without referencing the diagrams.

\begin{figure}
    \centering
    \begin{subfigure}[t]{.49\linewidth}
        \centering
        \includegraphics[width=\linewidth]{latex/figures/images/geoqa_example.pdf}
        \caption{GeoQA}
        \label{fig:geoqa_example}
    \end{subfigure}
    \begin{subfigure}[t]{.48\linewidth}
        \centering
        \includegraphics[width=\linewidth]{latex/figures/images/pgps_example.pdf}
        \caption{PGPS9K}
        \label{fig:pgps9k_example}
    \end{subfigure}
    \caption{
    Examples of diagram-caption pairs and their solution steps written in formal languages from GeoQA and PGPS9k datasets. In the problem description, the visual geometric premises and numerical variables are highlighted in green and red, respectively. A significant difference in the style of the diagram and formal language can be observable. %, along with the differences in formal languages supported by the corresponding datasets.
    \label{fig:pgps_examples}
    }
\end{figure}



We propose a new benchmark created via a synthetic data engine, which systematically evaluates the ability of VLM vision encoders to recognize geometric premises. Our empirical findings reveal that previously suggested self-supervised learning (SSL) approaches, e.g., vector quantized variataional auto-encoder (VQ-VAE)~\citep{unimath} and masked auto-encoder (MAE)~\citep{scagps,geox}, and widely adopted encoders, e.g., OpenCLIP~\citep{clip} and DinoV2~\citep{dinov2}, struggle to detect geometric features such as perpendicularity and degrees. 

To this end, we propose \geoclip{}, a model pre-trained on a large corpus of synthetic diagram–caption pairs. By varying diagram styles (e.g., color, font size, resolution, line width), \geoclip{} learns robust geometric representations and outperforms prior SSL-based methods on our benchmark. Building on \geoclip{}, we introduce a few-shot domain adaptation technique that efficiently transfers the recognition ability to real-world diagrams. We further combine this domain-adapted GeoCLIP with an LLM, forming a domain-agnostic VLM for solving PGPS tasks in MathVerse~\citep{mathverse}. 
%To accommodate diverse diagram styles and solution formats, we unify the solution program languages across multiple PGPS datasets, ensuring comprehensive evaluation. 

In our experiments on MathVerse~\citep{mathverse}, which encompasses diverse plane geometry tasks and diagram styles, our VLM with a domain-adapted \geoclip{} consistently outperforms both task-specific PGPS models and generalist VLMs. 
% In particular, it achieves higher accuracy on tasks requiring geometric-feature recognition, even when critical numerical measurements are moved from text to diagrams. 
Ablation studies confirm the effectiveness of our domain adaptation strategy, showing improvements in optical character recognition (OCR)-based tasks and robust diagram embeddings across different styles. 
% By unifying the solution program languages of existing datasets and incorporating OCR capability, we enable a single VLM, named \geovlm{}, to handle a broad class of plane geometry problems.

% Contributions
We summarize the contributions as follows:
We propose a novel benchmark for systematically assessing how well vision encoders recognize geometric premises in plane geometry diagrams~(\cref{sec:visual_feature}); We introduce \geoclip{}, a vision encoder capable of accurately detecting visual geometric premises~(\cref{sec:geoclip}), and a few-shot domain adaptation technique that efficiently transfers this capability across different diagram styles (\cref{sec:domain_adaptation});
We show that our VLM, incorporating domain-adapted GeoCLIP, surpasses existing specialized PGPS VLMs and generalist VLMs on the MathVerse benchmark~(\cref{sec:experiments}) and effectively interprets diverse diagram styles~(\cref{sec:abl}).

\iffalse
\begin{itemize}
    \item We propose a novel benchmark for systematically assessing how well vision encoders recognize geometric premises, e.g., perpendicularity and angle measures, in plane geometry diagrams.
	\item We introduce \geoclip{}, a vision encoder capable of accurately detecting visual geometric premises, and a few-shot domain adaptation technique that efficiently transfers this capability across different diagram styles.
	\item We show that our final VLM, incorporating GeoCLIP-DA, effectively interprets diverse diagram styles and achieves state-of-the-art performance on the MathVerse benchmark, surpassing existing specialized PGPS models and generalist VLM models.
\end{itemize}
\fi

\iffalse

Large language models (LLMs) have made significant strides in automated math word problem solving. In particular, their code-generation capabilities combined with proof assistants~\citep{lean,isabelle} help minimize computational errors~\citep{POT}, improve solution precision~\citep{autoformalization}, and offer rigorous feedback and evaluation~\citep{MATH}. Although LLMs excel in generating solution steps and correct answers for algebra and calculus~\citep{math_solving}, their uni-modal nature limits performance in domains like plane geometry, where both diagrams and text are vital.

Plane geometry problem solving (PGPS) tasks typically include diagrams and textual descriptions, requiring solvers to interpret premises from both sources. To facilitate automated solutions for these problems, several studies have introduced formal languages tailored for plane geometry to represent solution steps as a program with training datasets composed of diagrams, textual descriptions, and solution programs~\citep{geoqa,unigeo,intergps,pgps}. Building on these datasets, a number of PGPS specialized vision-language models (VLMs) have been developed so far~\citep{GOLD, LANS, geox}.

Most existing VLMs, however, fail to use diagrams when solving geometry problems. Well-known PGPS datasets such as GeoQA~\citep{geoqa}, UniGeo~\citep{unigeo}, and PGPS9K~\citep{pgps}, can be solved without accessing diagrams, as their problem descriptions often contain all geometric information. \cref{fig:pgps_examples} shows an example from GeoQA and PGPS9K datasets, where one can deduce the solution steps without knowing the diagrams. 
As a result, models trained on these datasets rely almost exclusively on textual information, leaving the vision encoder under-utilized~\citep{GOLD}. 
Consequently, the VLMs trained on these datasets cannot solve the plane geometry problem when necessary geometric properties or relations are excluded from the problem statement.

Some studies seek to enhance the recognition of geometric premises from a diagram by directly predicting the premises from the diagram~\citep{GOLD, intergps} or as an auxiliary task for vision encoders~\citep{geoqa,geoqa-plus}. However, these approaches remain highly domain-specific because the labels for training are difficult to obtain, thus limiting generalization across different domains. While self-supervised learning (SSL) methods that depend exclusively on geometric diagrams, e.g., vector quantized variational auto-encoder (VQ-VAE)~\citep{unimath} and masked auto-encoder (MAE)~\citep{scagps,geox}, have also been explored, the effectiveness of the SSL approaches on recognizing geometric features has not been thoroughly investigated.

We introduce a benchmark constructed with a synthetic data engine to evaluate the effectiveness of SSL approaches in recognizing geometric premises from diagrams. Our empirical results with the proposed benchmark show that the vision encoders trained with SSL methods fail to capture visual \geofeat{}s such as perpendicularity between two lines and angle measure.
Furthermore, we find that the pre-trained vision encoders often used in general-purpose VLMs, e.g., OpenCLIP~\citep{clip} and DinoV2~\citep{dinov2}, fail to recognize geometric premises from diagrams.

To improve the vision encoder for PGPS, we propose \geoclip{}, a model trained with a massive amount of diagram-caption pairs.
Since the amount of diagram-caption pairs in existing benchmarks is often limited, we develop a plane diagram generator that can randomly sample plane geometry problems with the help of existing proof assistant~\citep{alphageometry}.
To make \geoclip{} robust against different styles, we vary the visual properties of diagrams, such as color, font size, resolution, and line width.
We show that \geoclip{} performs better than the other SSL approaches and commonly used vision encoders on the newly proposed benchmark.

Another major challenge in PGPS is developing a domain-agnostic VLM capable of handling multiple PGPS benchmarks. As shown in \cref{fig:pgps_examples}, the main difficulties arise from variations in diagram styles. 
To address the issue, we propose a few-shot domain adaptation technique for \geoclip{} which transfers its visual \geofeat{} perception from the synthetic diagrams to the real-world diagrams efficiently. 

We study the efficacy of the domain adapted \geoclip{} on PGPS when equipped with the language model. To be specific, we compare the VLM with the previous PGPS models on MathVerse~\citep{mathverse}, which is designed to evaluate both the PGPS and visual \geofeat{} perception performance on various domains.
While previous PGPS models are inapplicable to certain types of MathVerse problems, we modify the prediction target and unify the solution program languages of the existing PGPS training data to make our VLM applicable to all types of MathVerse problems.
Results on MathVerse demonstrate that our VLM more effectively integrates diagrammatic information and remains robust under conditions of various diagram styles.

\begin{itemize}
    \item We propose a benchmark to measure the visual \geofeat{} recognition performance of different vision encoders.
    % \item \sh{We introduce geometric CLIP (\geoclip{} and train the VLM equipped with \geoclip{} to predict both solution steps and the numerical measurements of the problem.}
    \item We introduce \geoclip{}, a vision encoder which can accurately recognize visual \geofeat{}s and a few-shot domain adaptation technique which can transfer such ability to different domains efficiently. 
    % \item \sh{We develop our final PGPS model, \geovlm{}, by adapting \geoclip{} to different domains and training with unified languages of solution program data.}
    % We develop a domain-agnostic VLM, namely \geovlm{}, by applying a simple yet effective domain adaptation method to \geoclip{} and training on the refined training data.
    \item We demonstrate our VLM equipped with GeoCLIP-DA effectively interprets diverse diagram styles, achieving superior performance on MathVerse compared to the existing PGPS models.
\end{itemize}

\fi 

\section{Related Works}
The intersection of AI and network management has prompted several innovative approaches, each aimed at enhancing the adaptability and efficiency of network systems. 
%This section discusses key contributions and methodologies from recent literature that align closely with the themes of leveraging AI to improve network operations.
NetGPT \cite{chen2024netgpt} has been developed as an AI-native network architecture that strategically deploys LLMs both at the edge and cloud to optimize personalization and efficiency. The architecture highlights improvements in network management and user intent inference by integrating communications and computing resources more deeply \cite{tong2023ten}. Similarly, NetLM \cite{wang2023network} introduces an AI-driven architecture to enhance autonomous capabilities in network management, notably in complex 6G environments. The system leverages multi-modal representation learning to integrate diverse network data, aiming to refine network intents and autonomously manage network operations.
ABC (Automatic Bottom-up Construction) \cite{ding2023abc} revolutionizes the configuration knowledge base for multi-vendor networks by automating the alignment and generation of configuration templates through natural language processing and active learning, significantly reducing the manual effort typically required.
CONFPILOT \cite{zhao2023confpilot} employs a retrieval-augmented generation framework to translate natural language intents into precise network configuration commands. This system not only accelerates configuration processes but also enhances accuracy with its innovative use of a retrained BERT model and a parameter description-enhanced BM25 algorithm, which together improve the retrieval and matching of network commands.
NetCR \cite{guo2023netcr} utilizes a knowledge graph to facilitate manual network configurations, providing adaptive recommendations that enhance the efficiency and accuracy of network operations across various devices. This tool underscores the potential of using structured knowledge to streamline network management tasks in multi-vendor environments. To the best of our knowledge, Text2Net is the first initiative that directly integrates AI, specifically NLP, into network simulation for educational purposes and beyond. While prior works have explored the use of AI to enhance network management and configuration, Text2Net uniquely applies these technologies to simplify and democratize the learning and execution processes in network simulations. 
\begin{figure}[t!]
    \centering
    \includegraphics[width=\columnwidth]{Figures/system_model5_svg-raw.pdf}
    \caption{Text2Net system model and pipeline}
    \vspace{-4mm}
    \label{fig: system model}
\end{figure}
\section{Text2Net Methodology}
As depicted in Figure~\ref{fig: system model}, Text2Net comprises five modules: User, Software-Adaptor, instructed LLM, Text Extractor, and Simulator. The user inputs a network topology scenario, which the Software-Adaptor forwards to the instructed LLM, OpenAI's ChatGPT-4T, via API calls. The LLM processes this input and returns Structured Command Strings (SCS) to the Software-Adaptor. Utilizing NLP tools (SpaCy), along with RegEx and pattern matching, the Software-Adaptor extracts the desired key-value pairs, formatting them into a JSON dictionary. This JSON file is input to the EVE-NG emulator to provision the live network topology and configurations. 

\begin{figure}[ht!]
    \centering
    \includegraphics[width=\columnwidth]{Figures/flowchart_svg-raw.pdf}
    \caption{System flow perspectives.}
    \vspace{-3mm}
    \label{fig: flowchart}
    \vspace{-3mm}
\end{figure}

\noindent{\bf Overall System Flow.}
Figure ~\ref{fig: flowchart} depicts the system model, showcasing both, the system and user perspectives.
As illustrated on the right side of Figure~\ref{fig: flowchart}, the system flow from the user perspective begins with the initiation of Text2Net followed by the display of the welcome page. Text2Net greets the user and prompts for a network topology scenario. If the input scenario is valid and includes all required technical details, the system proceeds to provision the network topology in the emulator. If the scenario lacks necessary information, the system requests further details from the user. Ultimately, the user can interact with the fully configured live network topology.
The left side of Figure~\ref{fig: flowchart} illustrates the background processes of the system, which operate transparently to the user and concurrently with the functions depicted on the right side, highlighted in the same color, which will be discussed in detail in this section. 
%The initial step involves launching the emulator environment on a cloud virtual machine and starting the Software Adaptor code. Following this, the system processes the user input through the instructed LLM, retrieving the relevant SCS after performing necessary sanity checks and adhering to provided arguments and instructions. Subsequently, using NLP and Regex pattern matching, the system constructs a key-value pair dictionary in JSON format. The next step involves connecting to the EVE-NG emulator to create nodes, link interfaces, establish Telnet sessions for each node, and push configurations to test the scenario. Once these steps are completed, the fully provisioned network topology is ready for further user interaction.

\vspace{-2mm}
\subsection{System preparation}

%\subsubsection{Preparing the Simulation Environment}
The initial setup of Text2Net involves using EVE-NG, a network emulator (it is beyond simulator that replicates real-world environments and supports actual device images from manufacturers like Cisco, Juniper, and HPE). For Text2Net, Cisco devices are primarily used due to their commonality in networking. EVE-NG can be deployed either via an ISO file on virtual machines like VMware or directly on physical hardware to avoid performance issues associated with nested virtualization. To broaden Text2Net’s accessibility, it is hosted on the Google Cloud Platform (GCP) using an n2-standard-8 machine with 8 vCPUs and 32 GB of memory, running Linux Ubuntu. After the EVE-NG installation, the system is configured with a static IP, and HTTPS and SSH ports are opened to ensure it is accessible from any location, verified by navigating to the public IP address in a web browser to reach the EVE-NG login screen.

To leverage OpenAI's ChatGPT-4T for Text2Net, we trained the model to interpret and generate SCS from plain text descriptions of network topologies, commonly presented in computer network lectures. The model was trained to precisely extract and structure key information into command strings with key-value pairs essential for network topology provisioning.
The model recognizes detailed textual descriptions of network setups, outputting accurate command strings without superfluous content. For valid, complete inputs, the model confirms with returning the phrase "Understood", moving to the next phase. For inputs that are empty, incomplete, or incorrect, it prompts the user to refine their input.
The initial user interaction with Text2Net involves a user-friendly interface where users are prompted to input network topologies in plain English. This input becomes the basis for generating network configurations. A significant challenge was standardizing how users describe network topologies; this was addressed through a qualitative survey to establish a standard input format.
Text2Net is equipped to assess the validity and completeness of user inputs, ensuring no essential details, like IP formatting or technical configurations, are missing. This capability ensures the system efficiently transitions from user input to network configuration. 
When an input scenario includes an invalid IP address, such as "192.168.0.300," Text2Net automatically detects the error. For configurations involving specific protocols like static routing that lack necessary details, the system does not simply accept the input. Instead, it prompts the user \textit{``Please provide additional details about the static route"} before proceeding with generating the SCS.

\vspace{-2mm}
\subsection{Extracting Structured Command Strings}
To efficiently extract key-value pairs from the plain text, the text is decomposed into segments known as Structured Command Strings a.k.a. SCSs. These SCSs are derived from the plain text by employing the instructed GPT-4T model. Each SCS consists of short strings that encapsulate one or a few specific key-value pairs, ensuring clarity and specificity in data extraction.
Thanks to prompt engineering, the system is able to extract the same SCSs as an output from the following different scenarios that explain the same network topology with different styles of explanation. Figure~\ref{fig: network topology} shows the network diagram for which we have considered input/output configuration testing across three different user-input scenarios that may be possible:

\begin{figure}[t!]
    \centering
    \includegraphics[scale=0.3]{Figures/network_topology.png}
    \caption{Static-route Scenario Network Topology}
    \vspace{-6mm}
    \label{fig: network topology}
\end{figure}

\vspace{1mm}\noindent{\bf Scenario 1 -}
\textit{“R-1” is a router that is connected to “R-2”. “R-1” interface gi 0/0 has IP address 192.168.0.1/24 and is connected to  “R-2” interface Gi 0/0 with IP address 192.168.0.2/24. 
R-2 is connected to “R-3” via interface Gi 0/1 using IP address 192.168.100.1/24. “R-3” is connected back to “R-2” using interface Gi 0/0 with IP address 192.168.100.2/24. 
A static route is configured on “R-1” to reach “R-3” as well as a static router on “R-3” to reach to “R-1” through “R-2”.}


\vspace{1mm}\noindent{\bf Scenario 2 -}
\textit{The network has three routers: R-1, R-2, and R-3, interconnected in a specific manner.
R-1 connects to R-2 through its interface Gi 0/0, with the IP address 192.168.0.1/24, while R-2's corresponding interface, Gi 0/0, has the IP address 192.168.0.2/24.
R-2 establishes a connection with R-3 via interface Gi 0/1, with R-2 assigned the IP address 192.168.100.1/24 for this link.
The reverse connection from R-3 to R-2 is achieved through R-3's interface Gi 0/0, configured with the IP address 192.168.100.2/24.
For seamless communication between R-1 and R-3, static routes are set up on both routers through R-2, ensuring efficient routing between them.}


\begin{figure}[ht!]
    \centering
    \includegraphics[width=1\columnwidth]{Figures/verbatim_svg-raw.pdf}
    % \includesvg[inkscapelatex=false, scale=0.6]{Figures/venn diagram.svg}
    \caption{Structured Command Strings (SCSs)}
    \vspace{-4mm}
    \label{fig: SCS}
\end{figure}

% \begin{small}
% \begin{verbatim} 
% - device-1 type is router
% - device-1 name is R-1
% - device-1 interface Gi0/0 has IP address 192.168.0.1 with subnet- 
% mask 255.255.255.0
% - device-1 interface Gi0/0 is connected to device-2 interface Gi0/0
% - device-2 type is router
% - device-2 name is R-2
% - device-2 interface Gi0/0 has IP address 192.168.0.1 with subnet-
% mask 255.255.255.0
% - device-2 interface Gi0/0 is connected to device-1 interface Gi0/0
% - device-2 interface Gi0/1 has IP address 192.168.100.1 with subnet- 
% mask 255.255.255.0
% - device-2 interface Gi0/1 is connected to device-3 interface Gi0/0
% - device-3 type is router
% - device-3 name is R-3
% - device-3 interface Gi0/0 has IP address 192.168.100.2 with subnet- 
% mask 255.255.255.0
% - device-3 interface Gi0/0 is connected to device-2 interface Gi0/1
% - static route configured from R-1 to R-3 through R-2
% - static route configured from R-3 to R-1 through R-2
% \end{verbatim}
% \end{small}

Figure~\ref{fig: SCS} shows the same output as SCS from the two above scenarios in our system. Such examples demonstrate Text2Net's capability to deliver consistent output across different storytelling (variable network configuration explanations from users) approaches, provided that the underlying network topology remains the same. This highlights the robustness of this system in recognizing and interpreting the essential elements of network configurations, even when the narrative descriptions vary. This consistency ensures that Text2Net can be reliably used, and is easy to program for changes in input styles, in educational settings where diverse narrative styles are employed to describe similar network setups.

The scenario below illustrates how the same network topology can be described differently, akin to the variations tested in the previous two scenarios. However, there is a key difference, this description lacks detailed information about ``static routing". This incomplete scenario provides an opportunity to observe how Text2Net manages scenarios where critical information is missing.

\vspace{-2mm}
\SetAlgoLined
\DontPrintSemicolon
\begin{algorithm}[]
\caption{\method: API Generation}
\label{algo:api}
\SetKwData{Data}{Data}
\SetKw{Continue}{continue}
\SetKwInOut{KwInit}{Initialize}
\SetKwProg{Fn}{Function}{}{end}

\KwData{Questions $\mathcal{Q}$}
$\mathcal{S} \gets \{\} $\tcp*[r]{Signatures}
$\mathcal{A} \gets \{\text{Vision Models}\}$\tcp*[r]{API Methods}
\For{batch $B \subset \mathcal{Q}$} {
    $\mathcal{S} \gets \mathcal{S} \cup  \texttt{SignatureAgent}(B)$
}
\For{$S \in \mathcal{S}$} {
    $e_S \gets 0$ \tcp*[r]{Error count}
    $A \gets \texttt{ImplementationAgent}(S)$ \;
    $E \gets \texttt{TestAgent}(A)$\;
    \eIf{Python Exception $E$}{
        \lIf{$e_S = 5$}{\Continue}
        \uElseIf{$E$ is ``undefined method $U$''}{
            $e_S \gets e_S + 1$\;
            Recursively implement $U$
        }
        \Else{
            $e_S \gets e_S + 1$\;
            Re-implement $S$ using $E$
        }
    } {
    $\mathcal{A} \gets \mathcal{A} \cup A$
    }
}
\Return{$\mathcal{A}$}
\end{algorithm}
\vspace{-3mm}
\noindent{\bf Scenario 3-}
\textit{This network architecture is designed to facilitate efficient communication between multiple network segments, each identified by distinct IP subnets.
Routers R-1, R-2, and R-3 serve as intermediaries for routing data packets between these segments.
R-1 functions as the gateway router, connecting a potential local network segment to the wider network. It is directly linked to R-2 through its interface Gi 0/0, with R-1's IP address on this interface being 192.168.0.1/24 and R-2's IP address set to 192.168.0.2/24.
R-2 operates as a central hub, facilitating connections between multiple network segments. It interfaces with R-1 through Gi 0/0 and with R-3 through Gi 0/1. On interface gi 0/1, R-2 is assigned the IP address 192.168.100.1/24.
R-3 serves as a bridge between different network segments. It connects back to R-2 through its interface Gi 0/0, configured with the IP address 192.168.100.2/24.
The network's functionality relies on the careful configuration of IP addresses and static routes. This ensures that data packets are routed efficiently between devices connected to R-1, R-2, and R-3, facilitating seamless communication across the entire network infrastructure.}

In this case, the SCSs is generated but not for the static route section. Text2Net detected the missing information and specifically asked about it when returned ``However, I need additional information about the static routing configuration to provide complete command strings. Could you specify the source, destination, and through devices for each static route?''


\vspace{-2mm}
\subsection{Extracting Key-value pairs}

\iffalse
\begin{figure}[ht!]
    \centering
    \includegraphics[width=0.65\columnwidth]{Figures/venn_diagram.svg}
    % \includesvg[inkscapelatex=false, scale=0.6]{Figures/venn diagram.svg}
    \caption{Entities relationship}
    \vspace{-6mm}
    \label{fig: entity-relationship}
\end{figure}
\fi

To develop a comprehensive key-value pair dictionary, the first step is to establish a detailed entity relationship. Understanding the scope of the system is crucial for designing the relationships between entities to structure the corresponding JSON dictionary effectively. In the current phase of Text2Net, we leveraged RegEX and pattern matching to implement the system's functionality for routing in networking. However, scaling this approach to cover all networking concepts would be labor-intensive and inefficient. As future work, we aim to explore more on NLP techniques, also including Retrieval Augmented Generators (RAGs) in our model, to enhance scalability and extend the system's capabilities.

%Figure~\ref{fig: entity-relationship} illustrates the entities and their relationships that are used in Text2Net. 
% Main entities are "Devices" and "Node Configurations" each comprising sub entities that have relationship to each other. Devices includes sub-entities like Name, Network Attributes, and Type. Network Attributes contains its child entities such as "Int" for interface, "IP" for IP address, "SM" for Subnet Mask, "NN" for Network Number, "NM" for Network Mask, "WC" for Wildcard, and "NID" for Network Identifier. On the other hand, there are three different Types defined for Text2Net such as Router, Switch, and PC (Personal Computer). 
% %This will be further expanded later in future works to support more devices like firewalls. 
% Node Configuration has two sub-entities called "Basic Configurations" and "L3 Configurations". Basic Configuration includes "HN" for Host Name that is connected to Names, and "Links" which itself is connected to Int, IP, SM, and NN. L3 Configurations contains three of the most popular Layer 3  routing protocols such as Static routing, OSPF (Open-Shortest Path First), and EIGRP (Enhanced Interior-Gateway Routing Protocol).


\begin{figure}[ht!]
    \centering
    \includegraphics[width=0.55\columnwidth]{Figures/verbatim_svg-raw.pdf}
    % \includesvg[inkscapelatex=false, scale=0.6]{Figures/venn diagram.svg}
    \caption{key-value pairs output}
    \vspace{-3mm}
    \label{fig: key-value pairs output}
\end{figure}

Algorithm~\ref{algorithm} facilitates the structured extraction and processing of network topology data. $\mathcal{D} = \{d_1, d_2, \ldots, d_n\}$ represents the set of all devices, where in  $\mathcal \{d_1, d_2, \ldots, d_n\}$ each element is a tuple containing key-value pairs $\mathcal(k, v)$ that describe network devices parameters and their information. The output, $\mathcal{J}$, is a JSON object structured to include detailed device and connection configurations necessary to be used for the network simulation. During processing, $\mathcal{V}$ serves as a temporary dictionary to accumulate the detailed attributes of each device, while $\mathcal{C}$ and $\mathcal{L}$ store basic and Layer 3 configurations respectively. \textit{l} represents each line of SCS that the algorithm iterates through for extraction. Functions such as \verb|ExtractNode()|, \verb|ExtractName()|, and \verb|ExtractInterfaceDetails()| parse specific details from textual descriptions. \verb|AssignUniqueID()| assigns unique network identifiers, ensuring each component is distinctly recognized in the simulation environment. Together, these elements systematically transform plain text input into a structured format that is both accurate and suitable for any layer3 topology generation.
Figure~\ref{fig: key-value pairs output} shows Key-Value pair dictionary based JSON output is a generated template for one device to show case the format of the algorithm output.


% \begin{small}
% \begin{verbatim}
%     "devices": [
%         {
%             "name": "device_1",
%             "details": [
%                 {"Node_Type": "router"},
%                 {"Node_Name": "R-1"},
%                 {
%                     "interface": "Gi 0/0",
%                     "ip_address": "192.168.0.1",
%                     "subnet_mask": "255.255.255.0",
%                     "network_number": "192.168.0.0",
%                     "network_mask": "255.255.255.0",
%                     "wildcard_mask": "0.0.0.255",
%                     "network_id": "Network_1"
%                 }
%             ],
%             "node_configs": {
%                 "basic_configs": {
%                     "hostname": "R-1",
%                     "interfaces": ["Gi 0/0"]
%                 },
%                 "L3_configs": {
%                     "static_route": {
%                         "interface": "Gi 0/0",
%                         "network": "192.168.0.0",
%                         "mask": "255.255.255.0"
%                     }
%                 }
%             }
%         }
% \end{verbatim}
% \end{small}
% \vspace{-2mm}

\vspace{-2mm}
\subsection{Network topology provisioning}
The integration with the simulation environment, EVE-NG, for network topology provisioning leverages the previously structured JSON containing key-value pairs of all devices, including their technical configurations. This JSON serves as the blueprint for the entire network topology within EVE-NG.\\ 
\noindent{\bf Preparation and Initialization:}
The process begins by parsing the structured JSON, which details each network device's required configuration such as device type, interfaces, and routing protocols. This data guides the creation and configuration of each virtual device within EVE-NG.\\
\noindent{\bf Node Creation:}
The \verb|create_node| function dynamically creates nodes in EVE-NG based on the specifications extracted from the JSON. It configures various attributes like device type, associated images, and hardware specifications (CPU, RAM). Depending on whether the node is a router, switch or PC, specific templates and additional parameters such as QEMU options are set. \\
%The function also determines the placement of these nodes within the virtual lab space, either by generating random coordinates or using predefined positions to ensure an organized layout.
\noindent{\bf Network Linking:}
Following node creation, the \verb|create_network| function establishes links between the nodes. This function reads interface details directly from the JSON and uses them to configure correct connections, ensuring that all network interfaces are linked as per the topology requirements.\\ 
%It manages network visibility and bridge settings through EVE-NG’s API, effectively translating logical designs into operational network setups.
\noindent{\bf Operational Execution and API Interaction:}
With nodes and networks in place, the script executes operational commands to activate and initially configure the devices via the EVE-NG API. This includes setting up interfaces and applying any predefined network routes or policies as specified in the JSON. Each action is carefully monitored through the API responses to handle exceptions and ensure successful deployment.\\
\noindent{\bf Session Management and Debugging:}
The entire process is supported by robust session management, where authentication and session cookies are handled to maintain a persistent connection with the EVE-NG API. Debugging information, such as device creation and network linkage statuses, is logged to assist in troubleshooting and validating the network setup.








\begin{figure}[t]
    \centering
    \includegraphics[width=\columnwidth]{Figures/scenario1.png}
    \caption{Steps and time comparison for Text2Net and EVE-NG Network Simulator - Scenario1}
    \vspace{-4mm}
    \label{fig: scenario1}
\end{figure}

\section{Evaluation}
The evaluation of Text2Net was conducted using both qualitative and quantitative methods. For the quantitative analysis, we compared Text2Net’s performance with manual configuration in the EVE-NG simulation environment, focusing on two parameters: time and steps. We measured only the time to input commands, excluding thinking or troubleshooting time, ensuring the results represent the best-case scenario for manual configuration. This assumption of an error-free manual process further highlights Text2Net’s competitiveness.

To assess scalability and efficiency, we analyzed three network scenarios of increasing complexity. The results demonstrate that Text2Net significantly reduces time and steps, with its advantages growing as complexity increases. This improvement stems from eliminating repetitive commands and tasks inherent in traditional workflows.

Scenario 1 involves configuring a router with basic settings, including date/time, hostname, disabling DNS lookups, configuring an interface, and verifying the configuration. As shown in Figure~\ref{fig: scenario1} (Gantt chart), manual configuration in EVE-NG requires 12 steps, such as launching the simulator, logging in, creating a lab environment, starting a node, and configuring the interface. These steps, common across network platforms, take 200 seconds.
In contrast, Text2Net completes the same task in just two steps and 110 seconds with the prompt:
\textit{“Configure a router as R1 with basic setup. Configure the interface Fast Ethernet 0/1 with IP address 192.168.0.1 and subnet mask 255.255.255.0, and finally check the configurations.”}



Scenario 2 introduces greater complexity with two routers, each having internal networks configured as loopback interfaces and interconnected with static routes. The steps from Scenario 1 are repeated for each node, including configuring loopback interfaces, setting static routes, verifying configurations, and running ping tests. Manual configuration in EVE-NG requires 510 seconds, while Text2Net completes the task in 250 seconds using the prompt:
\textit{“Configure Router 1 as R1 and Router 2 as R2 with basic configurations. On R1, configure the interface Fast Ethernet 0/1 with IP address 192.168.0.1 and subnet mask 255.255.255.0. Configure loopback 1 interface to act as Network 1 with IP address 192.168.1.1/24. On R2, configure the interface Fast Ethernet 0/1 with IP address 192.168.0.2 and subnet mask 255.255.255.0. Configure loopback 1 interface to act as Network 2 with IP address 192.168.2.1/24. Set static routes from R1 to R2 and vice versa. Finally, check the configurations.”}





\begin{figure}[t]
    \centering
    \includegraphics[width=\columnwidth]{Figures/scenario2.png}
    \caption{Steps and time comparison for Text2Net and EVE-NG Network Simulator - Scenario2}
    \vspace{-4mm}
    \label{fig: scenario2}
\end{figure}


\begin{figure}[t]
    \centering
    \includegraphics[width=\columnwidth]{Figures/scenario3.png}
    \caption{Steps and time comparison for Text2Net and EVE-NG Network Simulator - Scenario3}
    \vspace{-4mm}
    \label{fig: scenario3}
\end{figure}


\begin{figure*}[t]
    \centering
    \includegraphics[scale=0.44]{Figures/qualitative-result_svg-raw.pdf}
    \caption{Consolidated Benefits and Ratings for Text2Net}
    \vspace{-5mm}
    \label{fig: consolidated_results}
\end{figure*}


Scenario 3 adds a third router, acting as a transit node between the two from Scenario 2. The static route from Router 1 (R1) now targets Router 3 (R3) via Router 2 (R2), which lacks an internal network. Manual configuration in EVE-NG requires 10 steps, including repeating all tasks from Scenario 1 for each node (R1, R2, and R3), configuring additional interface links, and setting static routes. Completing this scenario manually takes 730 seconds.
Text2Net reduces this to 310 seconds with the prompt:
\textit{“R1 is a router connected to R2. R1 interface Gigabit Ethernet 0/0 has IP address 192.168.0.1/24 and is connected to R2 interface Gigabit Ethernet 0/0 with IP address 192.168.0.2/24. R2 is connected to R3 via interface Gi 0/1 using IP address 192.168.4.1/24. R3 is connected back to R2 using interface Gi 0/0 with IP address 192.168.4.2/24. R1 has a loopback interface 1 with IP address 192.168.1.1/24 to act as internal network-1. R3 also has a loopback interface 1 with IP address 192.168.2.1/24 to act as internal network-2. A static route is configured on R1 to reach R3, and another static route on R3 to reach R1 through R2.”}

Across Scenarios 1, 2, and 3, Text2Net consistently outperforms manual configuration, requiring 110, 250, and 310 seconds, compared to 200, 510, and 730 seconds in EVE-NG. This demonstrates Text2Net’s scalability and efficiency in handling increasingly complex configurations.
Figure~\ref{fig: consolidated_results} summarizes the qualitative evaluation of Text2Net, based on feedback from 15 participants, including graduate students, professors, and engineers. Participants highlighted Text2Net’s ability to reduce errors, repetitive tasks, and setup time, making it more efficient compared to traditional methods. Additionally, the system was noted for simplifying simulation workflows and providing practical insights into real-world network scenarios. Text2Net received high ratings for ease of use, transparency, and educational value, achieving an average score of 4.66 out of 5, demonstrating its potential as a transformative tool for both academic and professional applications.









We present RiskHarvester, a risk-based tool to compute a security risk score based on the value of the asset and ease of attack on a database. We calculated the value of asset by identifying the sensitive data categories present in a database from the database keywords. We utilized data flow analysis, SQL, and Object Relational Mapper (ORM) parsing to identify the database keywords. To calculate the ease of attack, we utilized passive network analysis to retrieve the database host information. To evaluate RiskHarvester, we curated RiskBench, a benchmark of 1,791 database secret-asset pairs with sensitive data categories and host information manually retrieved from 188 GitHub repositories. RiskHarvester demonstrates precision of (95\%) and recall (90\%) in detecting database keywords for the value of asset and precision of (96\%) and recall (94\%) in detecting valid hosts for ease of attack. Finally, we conducted an online survey to understand whether developers prioritize secret removal based on security risk score. We found that 86\% of the developers prioritized the secrets for removal with descending security risk scores.

% Acknowledgements
\vspace{-2mm}\section{Acknowledgements}
This work was supported by the National Science Foundation (NSF CAREER CNS-2146267).
\vspace{-2mm}

% Bibliography
\bibliographystyle{IEEEtran}
\bibliography{REFERENCES}

\end{document}
