\section{Introduction}

Network simulators and emulators are essential tools in computer science (CS) education, allowing students to explore and experiment with complex network behaviors without relying on physical hardware. These tools are also widely used in industry for testing and validation purposes. Simulators are software engines that replicate various networking scenarios to test protocols, configurations, and network dynamics. Cisco Packet Tracer \cite{janitor2010visual} is a popular simulator for beginners \cite{allison2022simulation}, while GNS3 \cite{neumann2015book} caters to advanced users with its capability to simulate real device images. However, these simulators primarily model network behavior and may not fully replicate real-world dynamics.
In contrast, emulators like EVE-NG~\cite{EVE-NG} offer a more realistic solution by supporting actual device system code images (e.g., Cisco IOS), enabling accurate emulation of real-world operations. EVE-NG’s robust features make it particularly valuable for advanced education and professional training, allowing users to engage with complex network topologies in realistic environments \cite{sharma2024comparison}.

% \vspace{1mm}\noindent{\bf Challenges with network simulators in education.} Despite advancements in simulation and emulation tools, significant barriers remain to their effective use in education. Traditional tools often require mastery of complex syntax and involve repetitive, time-consuming setup processes \cite{sierszen2017teaching}, posing challenges for educators and beginners alike. Configuring network scenarios is labor-intensive, demanding meticulous setup that limits dynamic or interactive learning experiences~\cite{marquardson2019simulation}. Moreover, the abundance of simulation tools introduces issues such as poor maintenance, the dilemma of paid versus open-source options, and difficulties in transferring experiments between platforms. The need to memorize vendor-specific command syntax further complicates the learning process, especially when different syntaxes are required for a single networking concept.

\vspace{1mm}\noindent{\bf Challenges with network simulators in education.} Despite advancements in simulation tools, significant barriers hinder their effective use in education. Traditional tools often require mastering complex, vendor-specific command syntax, making setup processes repetitive and time-consuming~\cite{sierszen2017teaching}. This focus on memorizing configurations detracts from understanding core concepts and designing network architectures, which are far more valuable skills. Furthermore, the wide variety of simulation tools introduces challenges such as poor maintenance, the dilemma of paid versus open-source options, and difficulties in transferring experiments between platforms~\cite{marquardson2019simulation}.

\vspace{1mm}\noindent{\bf Text2Net: Bridging the gap using plain text and AI.}
Text2Net provides an innovative solution by enabling users to create and interact with network simulations using plain-text English instead of vendor-specific syntax. Leveraging advancements in natural language processing (NLP) and large language models (LLMs), Text2Net interprets user inputs and translates them into actionable simulation commands. While LLMs are powerful, they are prone to issues such as errors and inaccuracies. Text2Net addresses these challenges by eliminating the need for technical expertise, simplifying the simulation process, and shifting the focus from command syntax to conceptual learning. This approach enhances accessibility and efficiency, reducing the effort required to deploy and manage network scenarios.

This paper focuses on the development of the Text2Net engine and its application in network education. We present the system architecture and demonstrate its usability through a case study involving the EVE-NG tool. A complete prototype implementation is evaluated qualitatively through user surveys and quantitatively by comparing task completion steps and time between Text2Net and EVE-NG.

