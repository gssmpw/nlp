\section{Text2Net Methodology}
As depicted in Figure~\ref{fig: system model}, Text2Net comprises five modules: User, Software-Adaptor, instructed LLM, Text Extractor, and Simulator. The user inputs a network topology scenario, which the Software-Adaptor forwards to the instructed LLM, OpenAI's ChatGPT-4T, via API calls. The LLM processes this input and returns Structured Command Strings (SCS) to the Software-Adaptor. Utilizing NLP tools (SpaCy), along with RegEx and pattern matching, the Software-Adaptor extracts the desired key-value pairs, formatting them into a JSON dictionary. This JSON file is input to the EVE-NG emulator to provision the live network topology and configurations. 

\begin{figure}[ht!]
    \centering
    \includegraphics[width=\columnwidth]{Figures/flowchart_svg-raw.pdf}
    \caption{System flow perspectives.}
    \vspace{-3mm}
    \label{fig: flowchart}
    \vspace{-3mm}
\end{figure}

\noindent{\bf Overall System Flow.}
Figure ~\ref{fig: flowchart} depicts the system model, showcasing both, the system and user perspectives.
As illustrated on the right side of Figure~\ref{fig: flowchart}, the system flow from the user perspective begins with the initiation of Text2Net followed by the display of the welcome page. Text2Net greets the user and prompts for a network topology scenario. If the input scenario is valid and includes all required technical details, the system proceeds to provision the network topology in the emulator. If the scenario lacks necessary information, the system requests further details from the user. Ultimately, the user can interact with the fully configured live network topology.
The left side of Figure~\ref{fig: flowchart} illustrates the background processes of the system, which operate transparently to the user and concurrently with the functions depicted on the right side, highlighted in the same color, which will be discussed in detail in this section. 
%The initial step involves launching the emulator environment on a cloud virtual machine and starting the Software Adaptor code. Following this, the system processes the user input through the instructed LLM, retrieving the relevant SCS after performing necessary sanity checks and adhering to provided arguments and instructions. Subsequently, using NLP and Regex pattern matching, the system constructs a key-value pair dictionary in JSON format. The next step involves connecting to the EVE-NG emulator to create nodes, link interfaces, establish Telnet sessions for each node, and push configurations to test the scenario. Once these steps are completed, the fully provisioned network topology is ready for further user interaction.

\vspace{-2mm}
\subsection{System preparation}

%\subsubsection{Preparing the Simulation Environment}
The initial setup of Text2Net involves using EVE-NG, a network emulator (it is beyond simulator that replicates real-world environments and supports actual device images from manufacturers like Cisco, Juniper, and HPE). For Text2Net, Cisco devices are primarily used due to their commonality in networking. EVE-NG can be deployed either via an ISO file on virtual machines like VMware or directly on physical hardware to avoid performance issues associated with nested virtualization. To broaden Text2Net’s accessibility, it is hosted on the Google Cloud Platform (GCP) using an n2-standard-8 machine with 8 vCPUs and 32 GB of memory, running Linux Ubuntu. After the EVE-NG installation, the system is configured with a static IP, and HTTPS and SSH ports are opened to ensure it is accessible from any location, verified by navigating to the public IP address in a web browser to reach the EVE-NG login screen.

To leverage OpenAI's ChatGPT-4T for Text2Net, we trained the model to interpret and generate SCS from plain text descriptions of network topologies, commonly presented in computer network lectures. The model was trained to precisely extract and structure key information into command strings with key-value pairs essential for network topology provisioning.
The model recognizes detailed textual descriptions of network setups, outputting accurate command strings without superfluous content. For valid, complete inputs, the model confirms with returning the phrase "Understood", moving to the next phase. For inputs that are empty, incomplete, or incorrect, it prompts the user to refine their input.
The initial user interaction with Text2Net involves a user-friendly interface where users are prompted to input network topologies in plain English. This input becomes the basis for generating network configurations. A significant challenge was standardizing how users describe network topologies; this was addressed through a qualitative survey to establish a standard input format.
Text2Net is equipped to assess the validity and completeness of user inputs, ensuring no essential details, like IP formatting or technical configurations, are missing. This capability ensures the system efficiently transitions from user input to network configuration. 
When an input scenario includes an invalid IP address, such as "192.168.0.300," Text2Net automatically detects the error. For configurations involving specific protocols like static routing that lack necessary details, the system does not simply accept the input. Instead, it prompts the user \textit{``Please provide additional details about the static route"} before proceeding with generating the SCS.

\vspace{-2mm}
\subsection{Extracting Structured Command Strings}
To efficiently extract key-value pairs from the plain text, the text is decomposed into segments known as Structured Command Strings a.k.a. SCSs. These SCSs are derived from the plain text by employing the instructed GPT-4T model. Each SCS consists of short strings that encapsulate one or a few specific key-value pairs, ensuring clarity and specificity in data extraction.
Thanks to prompt engineering, the system is able to extract the same SCSs as an output from the following different scenarios that explain the same network topology with different styles of explanation. Figure~\ref{fig: network topology} shows the network diagram for which we have considered input/output configuration testing across three different user-input scenarios that may be possible:

\begin{figure}[t!]
    \centering
    \includegraphics[scale=0.3]{Figures/network_topology.png}
    \caption{Static-route Scenario Network Topology}
    \vspace{-6mm}
    \label{fig: network topology}
\end{figure}

\vspace{1mm}\noindent{\bf Scenario 1 -}
\textit{“R-1” is a router that is connected to “R-2”. “R-1” interface gi 0/0 has IP address 192.168.0.1/24 and is connected to  “R-2” interface Gi 0/0 with IP address 192.168.0.2/24. 
R-2 is connected to “R-3” via interface Gi 0/1 using IP address 192.168.100.1/24. “R-3” is connected back to “R-2” using interface Gi 0/0 with IP address 192.168.100.2/24. 
A static route is configured on “R-1” to reach “R-3” as well as a static router on “R-3” to reach to “R-1” through “R-2”.}


\vspace{1mm}\noindent{\bf Scenario 2 -}
\textit{The network has three routers: R-1, R-2, and R-3, interconnected in a specific manner.
R-1 connects to R-2 through its interface Gi 0/0, with the IP address 192.168.0.1/24, while R-2's corresponding interface, Gi 0/0, has the IP address 192.168.0.2/24.
R-2 establishes a connection with R-3 via interface Gi 0/1, with R-2 assigned the IP address 192.168.100.1/24 for this link.
The reverse connection from R-3 to R-2 is achieved through R-3's interface Gi 0/0, configured with the IP address 192.168.100.2/24.
For seamless communication between R-1 and R-3, static routes are set up on both routers through R-2, ensuring efficient routing between them.}


\begin{figure}[ht!]
    \centering
    \includegraphics[width=1\columnwidth]{Figures/verbatim_svg-raw.pdf}
    % \includesvg[inkscapelatex=false, scale=0.6]{Figures/venn diagram.svg}
    \caption{Structured Command Strings (SCSs)}
    \vspace{-4mm}
    \label{fig: SCS}
\end{figure}

% \begin{small}
% \begin{verbatim} 
% - device-1 type is router
% - device-1 name is R-1
% - device-1 interface Gi0/0 has IP address 192.168.0.1 with subnet- 
% mask 255.255.255.0
% - device-1 interface Gi0/0 is connected to device-2 interface Gi0/0
% - device-2 type is router
% - device-2 name is R-2
% - device-2 interface Gi0/0 has IP address 192.168.0.1 with subnet-
% mask 255.255.255.0
% - device-2 interface Gi0/0 is connected to device-1 interface Gi0/0
% - device-2 interface Gi0/1 has IP address 192.168.100.1 with subnet- 
% mask 255.255.255.0
% - device-2 interface Gi0/1 is connected to device-3 interface Gi0/0
% - device-3 type is router
% - device-3 name is R-3
% - device-3 interface Gi0/0 has IP address 192.168.100.2 with subnet- 
% mask 255.255.255.0
% - device-3 interface Gi0/0 is connected to device-2 interface Gi0/1
% - static route configured from R-1 to R-3 through R-2
% - static route configured from R-3 to R-1 through R-2
% \end{verbatim}
% \end{small}

Figure~\ref{fig: SCS} shows the same output as SCS from the two above scenarios in our system. Such examples demonstrate Text2Net's capability to deliver consistent output across different storytelling (variable network configuration explanations from users) approaches, provided that the underlying network topology remains the same. This highlights the robustness of this system in recognizing and interpreting the essential elements of network configurations, even when the narrative descriptions vary. This consistency ensures that Text2Net can be reliably used, and is easy to program for changes in input styles, in educational settings where diverse narrative styles are employed to describe similar network setups.

The scenario below illustrates how the same network topology can be described differently, akin to the variations tested in the previous two scenarios. However, there is a key difference, this description lacks detailed information about ``static routing". This incomplete scenario provides an opportunity to observe how Text2Net manages scenarios where critical information is missing.

\vspace{-2mm}
\SetAlgoLined
\DontPrintSemicolon
\begin{algorithm}[]
\caption{\method: API Generation}
\label{algo:api}
\SetKwData{Data}{Data}
\SetKw{Continue}{continue}
\SetKwInOut{KwInit}{Initialize}
\SetKwProg{Fn}{Function}{}{end}

\KwData{Questions $\mathcal{Q}$}
$\mathcal{S} \gets \{\} $\tcp*[r]{Signatures}
$\mathcal{A} \gets \{\text{Vision Models}\}$\tcp*[r]{API Methods}
\For{batch $B \subset \mathcal{Q}$} {
    $\mathcal{S} \gets \mathcal{S} \cup  \texttt{SignatureAgent}(B)$
}
\For{$S \in \mathcal{S}$} {
    $e_S \gets 0$ \tcp*[r]{Error count}
    $A \gets \texttt{ImplementationAgent}(S)$ \;
    $E \gets \texttt{TestAgent}(A)$\;
    \eIf{Python Exception $E$}{
        \lIf{$e_S = 5$}{\Continue}
        \uElseIf{$E$ is ``undefined method $U$''}{
            $e_S \gets e_S + 1$\;
            Recursively implement $U$
        }
        \Else{
            $e_S \gets e_S + 1$\;
            Re-implement $S$ using $E$
        }
    } {
    $\mathcal{A} \gets \mathcal{A} \cup A$
    }
}
\Return{$\mathcal{A}$}
\end{algorithm}
\vspace{-3mm}
\noindent{\bf Scenario 3-}
\textit{This network architecture is designed to facilitate efficient communication between multiple network segments, each identified by distinct IP subnets.
Routers R-1, R-2, and R-3 serve as intermediaries for routing data packets between these segments.
R-1 functions as the gateway router, connecting a potential local network segment to the wider network. It is directly linked to R-2 through its interface Gi 0/0, with R-1's IP address on this interface being 192.168.0.1/24 and R-2's IP address set to 192.168.0.2/24.
R-2 operates as a central hub, facilitating connections between multiple network segments. It interfaces with R-1 through Gi 0/0 and with R-3 through Gi 0/1. On interface gi 0/1, R-2 is assigned the IP address 192.168.100.1/24.
R-3 serves as a bridge between different network segments. It connects back to R-2 through its interface Gi 0/0, configured with the IP address 192.168.100.2/24.
The network's functionality relies on the careful configuration of IP addresses and static routes. This ensures that data packets are routed efficiently between devices connected to R-1, R-2, and R-3, facilitating seamless communication across the entire network infrastructure.}

In this case, the SCSs is generated but not for the static route section. Text2Net detected the missing information and specifically asked about it when returned ``However, I need additional information about the static routing configuration to provide complete command strings. Could you specify the source, destination, and through devices for each static route?''


\vspace{-2mm}
\subsection{Extracting Key-value pairs}

\iffalse
\begin{figure}[ht!]
    \centering
    \includegraphics[width=0.65\columnwidth]{Figures/venn_diagram.svg}
    % \includesvg[inkscapelatex=false, scale=0.6]{Figures/venn diagram.svg}
    \caption{Entities relationship}
    \vspace{-6mm}
    \label{fig: entity-relationship}
\end{figure}
\fi

To develop a comprehensive key-value pair dictionary, the first step is to establish a detailed entity relationship. Understanding the scope of the system is crucial for designing the relationships between entities to structure the corresponding JSON dictionary effectively. In the current phase of Text2Net, we leveraged RegEX and pattern matching to implement the system's functionality for routing in networking. However, scaling this approach to cover all networking concepts would be labor-intensive and inefficient. As future work, we aim to explore more on NLP techniques, also including Retrieval Augmented Generators (RAGs) in our model, to enhance scalability and extend the system's capabilities.

%Figure~\ref{fig: entity-relationship} illustrates the entities and their relationships that are used in Text2Net. 
% Main entities are "Devices" and "Node Configurations" each comprising sub entities that have relationship to each other. Devices includes sub-entities like Name, Network Attributes, and Type. Network Attributes contains its child entities such as "Int" for interface, "IP" for IP address, "SM" for Subnet Mask, "NN" for Network Number, "NM" for Network Mask, "WC" for Wildcard, and "NID" for Network Identifier. On the other hand, there are three different Types defined for Text2Net such as Router, Switch, and PC (Personal Computer). 
% %This will be further expanded later in future works to support more devices like firewalls. 
% Node Configuration has two sub-entities called "Basic Configurations" and "L3 Configurations". Basic Configuration includes "HN" for Host Name that is connected to Names, and "Links" which itself is connected to Int, IP, SM, and NN. L3 Configurations contains three of the most popular Layer 3  routing protocols such as Static routing, OSPF (Open-Shortest Path First), and EIGRP (Enhanced Interior-Gateway Routing Protocol).


\begin{figure}[ht!]
    \centering
    \includegraphics[width=0.55\columnwidth]{Figures/verbatim_svg-raw.pdf}
    % \includesvg[inkscapelatex=false, scale=0.6]{Figures/venn diagram.svg}
    \caption{key-value pairs output}
    \vspace{-3mm}
    \label{fig: key-value pairs output}
\end{figure}

Algorithm~\ref{algorithm} facilitates the structured extraction and processing of network topology data. $\mathcal{D} = \{d_1, d_2, \ldots, d_n\}$ represents the set of all devices, where in  $\mathcal \{d_1, d_2, \ldots, d_n\}$ each element is a tuple containing key-value pairs $\mathcal(k, v)$ that describe network devices parameters and their information. The output, $\mathcal{J}$, is a JSON object structured to include detailed device and connection configurations necessary to be used for the network simulation. During processing, $\mathcal{V}$ serves as a temporary dictionary to accumulate the detailed attributes of each device, while $\mathcal{C}$ and $\mathcal{L}$ store basic and Layer 3 configurations respectively. \textit{l} represents each line of SCS that the algorithm iterates through for extraction. Functions such as \verb|ExtractNode()|, \verb|ExtractName()|, and \verb|ExtractInterfaceDetails()| parse specific details from textual descriptions. \verb|AssignUniqueID()| assigns unique network identifiers, ensuring each component is distinctly recognized in the simulation environment. Together, these elements systematically transform plain text input into a structured format that is both accurate and suitable for any layer3 topology generation.
Figure~\ref{fig: key-value pairs output} shows Key-Value pair dictionary based JSON output is a generated template for one device to show case the format of the algorithm output.


% \begin{small}
% \begin{verbatim}
%     "devices": [
%         {
%             "name": "device_1",
%             "details": [
%                 {"Node_Type": "router"},
%                 {"Node_Name": "R-1"},
%                 {
%                     "interface": "Gi 0/0",
%                     "ip_address": "192.168.0.1",
%                     "subnet_mask": "255.255.255.0",
%                     "network_number": "192.168.0.0",
%                     "network_mask": "255.255.255.0",
%                     "wildcard_mask": "0.0.0.255",
%                     "network_id": "Network_1"
%                 }
%             ],
%             "node_configs": {
%                 "basic_configs": {
%                     "hostname": "R-1",
%                     "interfaces": ["Gi 0/0"]
%                 },
%                 "L3_configs": {
%                     "static_route": {
%                         "interface": "Gi 0/0",
%                         "network": "192.168.0.0",
%                         "mask": "255.255.255.0"
%                     }
%                 }
%             }
%         }
% \end{verbatim}
% \end{small}
% \vspace{-2mm}

\vspace{-2mm}
\subsection{Network topology provisioning}
The integration with the simulation environment, EVE-NG, for network topology provisioning leverages the previously structured JSON containing key-value pairs of all devices, including their technical configurations. This JSON serves as the blueprint for the entire network topology within EVE-NG.\\ 
\noindent{\bf Preparation and Initialization:}
The process begins by parsing the structured JSON, which details each network device's required configuration such as device type, interfaces, and routing protocols. This data guides the creation and configuration of each virtual device within EVE-NG.\\
\noindent{\bf Node Creation:}
The \verb|create_node| function dynamically creates nodes in EVE-NG based on the specifications extracted from the JSON. It configures various attributes like device type, associated images, and hardware specifications (CPU, RAM). Depending on whether the node is a router, switch or PC, specific templates and additional parameters such as QEMU options are set. \\
%The function also determines the placement of these nodes within the virtual lab space, either by generating random coordinates or using predefined positions to ensure an organized layout.
\noindent{\bf Network Linking:}
Following node creation, the \verb|create_network| function establishes links between the nodes. This function reads interface details directly from the JSON and uses them to configure correct connections, ensuring that all network interfaces are linked as per the topology requirements.\\ 
%It manages network visibility and bridge settings through EVE-NG’s API, effectively translating logical designs into operational network setups.
\noindent{\bf Operational Execution and API Interaction:}
With nodes and networks in place, the script executes operational commands to activate and initially configure the devices via the EVE-NG API. This includes setting up interfaces and applying any predefined network routes or policies as specified in the JSON. Each action is carefully monitored through the API responses to handle exceptions and ensure successful deployment.\\
\noindent{\bf Session Management and Debugging:}
The entire process is supported by robust session management, where authentication and session cookies are handled to maintain a persistent connection with the EVE-NG API. Debugging information, such as device creation and network linkage statuses, is logged to assist in troubleshooting and validating the network setup.






