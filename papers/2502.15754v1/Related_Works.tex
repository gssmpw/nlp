\section{Related Works}
The intersection of AI and network management has prompted several innovative approaches, each aimed at enhancing the adaptability and efficiency of network systems. 
%This section discusses key contributions and methodologies from recent literature that align closely with the themes of leveraging AI to improve network operations.
NetGPT \cite{chen2024netgpt} has been developed as an AI-native network architecture that strategically deploys LLMs both at the edge and cloud to optimize personalization and efficiency. The architecture highlights improvements in network management and user intent inference by integrating communications and computing resources more deeply \cite{tong2023ten}. Similarly, NetLM \cite{wang2023network} introduces an AI-driven architecture to enhance autonomous capabilities in network management, notably in complex 6G environments. The system leverages multi-modal representation learning to integrate diverse network data, aiming to refine network intents and autonomously manage network operations.
ABC (Automatic Bottom-up Construction) \cite{ding2023abc} revolutionizes the configuration knowledge base for multi-vendor networks by automating the alignment and generation of configuration templates through natural language processing and active learning, significantly reducing the manual effort typically required.
CONFPILOT \cite{zhao2023confpilot} employs a retrieval-augmented generation framework to translate natural language intents into precise network configuration commands. This system not only accelerates configuration processes but also enhances accuracy with its innovative use of a retrained BERT model and a parameter description-enhanced BM25 algorithm, which together improve the retrieval and matching of network commands.
NetCR \cite{guo2023netcr} utilizes a knowledge graph to facilitate manual network configurations, providing adaptive recommendations that enhance the efficiency and accuracy of network operations across various devices. This tool underscores the potential of using structured knowledge to streamline network management tasks in multi-vendor environments. To the best of our knowledge, Text2Net is the first initiative that directly integrates AI, specifically NLP, into network simulation for educational purposes and beyond. While prior works have explored the use of AI to enhance network management and configuration, Text2Net uniquely applies these technologies to simplify and democratize the learning and execution processes in network simulations. 
\begin{figure}[t!]
    \centering
    \includegraphics[width=\columnwidth]{Figures/system_model5_svg-raw.pdf}
    \caption{Text2Net system model and pipeline}
    \vspace{-4mm}
    \label{fig: system model}
\end{figure}