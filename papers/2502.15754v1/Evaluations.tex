
\begin{figure}[t]
    \centering
    \includegraphics[width=\columnwidth]{Figures/scenario1.png}
    \caption{Steps and time comparison for Text2Net and EVE-NG Network Simulator - Scenario1}
    \vspace{-4mm}
    \label{fig: scenario1}
\end{figure}

\section{Evaluation}
The evaluation of Text2Net was conducted using both qualitative and quantitative methods. For the quantitative analysis, we compared Text2Net’s performance with manual configuration in the EVE-NG simulation environment, focusing on two parameters: time and steps. We measured only the time to input commands, excluding thinking or troubleshooting time, ensuring the results represent the best-case scenario for manual configuration. This assumption of an error-free manual process further highlights Text2Net’s competitiveness.

To assess scalability and efficiency, we analyzed three network scenarios of increasing complexity. The results demonstrate that Text2Net significantly reduces time and steps, with its advantages growing as complexity increases. This improvement stems from eliminating repetitive commands and tasks inherent in traditional workflows.

Scenario 1 involves configuring a router with basic settings, including date/time, hostname, disabling DNS lookups, configuring an interface, and verifying the configuration. As shown in Figure~\ref{fig: scenario1} (Gantt chart), manual configuration in EVE-NG requires 12 steps, such as launching the simulator, logging in, creating a lab environment, starting a node, and configuring the interface. These steps, common across network platforms, take 200 seconds.
In contrast, Text2Net completes the same task in just two steps and 110 seconds with the prompt:
\textit{“Configure a router as R1 with basic setup. Configure the interface Fast Ethernet 0/1 with IP address 192.168.0.1 and subnet mask 255.255.255.0, and finally check the configurations.”}



Scenario 2 introduces greater complexity with two routers, each having internal networks configured as loopback interfaces and interconnected with static routes. The steps from Scenario 1 are repeated for each node, including configuring loopback interfaces, setting static routes, verifying configurations, and running ping tests. Manual configuration in EVE-NG requires 510 seconds, while Text2Net completes the task in 250 seconds using the prompt:
\textit{“Configure Router 1 as R1 and Router 2 as R2 with basic configurations. On R1, configure the interface Fast Ethernet 0/1 with IP address 192.168.0.1 and subnet mask 255.255.255.0. Configure loopback 1 interface to act as Network 1 with IP address 192.168.1.1/24. On R2, configure the interface Fast Ethernet 0/1 with IP address 192.168.0.2 and subnet mask 255.255.255.0. Configure loopback 1 interface to act as Network 2 with IP address 192.168.2.1/24. Set static routes from R1 to R2 and vice versa. Finally, check the configurations.”}





\begin{figure}[t]
    \centering
    \includegraphics[width=\columnwidth]{Figures/scenario2.png}
    \caption{Steps and time comparison for Text2Net and EVE-NG Network Simulator - Scenario2}
    \vspace{-4mm}
    \label{fig: scenario2}
\end{figure}


\begin{figure}[t]
    \centering
    \includegraphics[width=\columnwidth]{Figures/scenario3.png}
    \caption{Steps and time comparison for Text2Net and EVE-NG Network Simulator - Scenario3}
    \vspace{-4mm}
    \label{fig: scenario3}
\end{figure}


\begin{figure*}[t]
    \centering
    \includegraphics[scale=0.44]{Figures/qualitative-result_svg-raw.pdf}
    \caption{Consolidated Benefits and Ratings for Text2Net}
    \vspace{-5mm}
    \label{fig: consolidated_results}
\end{figure*}


Scenario 3 adds a third router, acting as a transit node between the two from Scenario 2. The static route from Router 1 (R1) now targets Router 3 (R3) via Router 2 (R2), which lacks an internal network. Manual configuration in EVE-NG requires 10 steps, including repeating all tasks from Scenario 1 for each node (R1, R2, and R3), configuring additional interface links, and setting static routes. Completing this scenario manually takes 730 seconds.
Text2Net reduces this to 310 seconds with the prompt:
\textit{“R1 is a router connected to R2. R1 interface Gigabit Ethernet 0/0 has IP address 192.168.0.1/24 and is connected to R2 interface Gigabit Ethernet 0/0 with IP address 192.168.0.2/24. R2 is connected to R3 via interface Gi 0/1 using IP address 192.168.4.1/24. R3 is connected back to R2 using interface Gi 0/0 with IP address 192.168.4.2/24. R1 has a loopback interface 1 with IP address 192.168.1.1/24 to act as internal network-1. R3 also has a loopback interface 1 with IP address 192.168.2.1/24 to act as internal network-2. A static route is configured on R1 to reach R3, and another static route on R3 to reach R1 through R2.”}

Across Scenarios 1, 2, and 3, Text2Net consistently outperforms manual configuration, requiring 110, 250, and 310 seconds, compared to 200, 510, and 730 seconds in EVE-NG. This demonstrates Text2Net’s scalability and efficiency in handling increasingly complex configurations.
Figure~\ref{fig: consolidated_results} summarizes the qualitative evaluation of Text2Net, based on feedback from 15 participants, including graduate students, professors, and engineers. Participants highlighted Text2Net’s ability to reduce errors, repetitive tasks, and setup time, making it more efficient compared to traditional methods. Additionally, the system was noted for simplifying simulation workflows and providing practical insights into real-world network scenarios. Text2Net received high ratings for ease of use, transparency, and educational value, achieving an average score of 4.66 out of 5, demonstrating its potential as a transformative tool for both academic and professional applications.








