%File: formatting-instructions-latex-2025.tex
%release 2025.0
\documentclass[letterpaper]{article} % DO NOT CHANGE THIS
\usepackage{aaai25}  % DO NOT CHANGE THIS
\usepackage{times}  % DO NOT CHANGE THIS
\usepackage{helvet}  % DO NOT CHANGE THIS
\usepackage{courier}  % DO NOT CHANGE THIS
\usepackage[hyphens]{url}  % DO NOT CHANGE THIS
\usepackage{graphicx} % DO NOT CHANGE THIS
\urlstyle{rm} % DO NOT CHANGE THIS
\def\UrlFont{\rm}  % DO NOT CHANGE THIS
\usepackage{natbib}  % DO NOT CHANGE THIS AND DO NOT ADD ANY OPTIONS TO IT
\usepackage{caption} % DO NOT CHANGE THIS AND DO NOT ADD ANY OPTIONS TO IT
\frenchspacing  % DO NOT CHANGE THIS
\setlength{\pdfpagewidth}{8.5in}  % DO NOT CHANGE THIS
\setlength{\pdfpageheight}{11in}  % DO NOT CHANGE THIS
%
% These are recommended to typeset algorithms but not required. See the subsubsection on algorithms. Remove them if you don't have algorithms in your paper.
\usepackage{algorithm}
\usepackage{algorithmic}

%
% These are are recommended to typeset listings but not required. See the subsubsection on listing. Remove this block if you don't have listings in your paper.
\usepackage{newfloat}
\usepackage{listings}
\DeclareCaptionStyle{ruled}{labelfont=normalfont,labelsep=colon,strut=off} % DO NOT CHANGE THIS
\lstset{%
	basicstyle={\footnotesize\ttfamily},% footnotesize acceptable for monospace
	numbers=left,numberstyle=\footnotesize,xleftmargin=2em,% show line numbers, remove this entire line if you don't want the numbers.
	aboveskip=0pt,belowskip=0pt,%
	showstringspaces=false,tabsize=2,breaklines=true}
\floatstyle{ruled}
\newfloat{listing}{tb}{lst}{}
\floatname{listing}{Listing}
%
% Keep the \pdfinfo as shown here. There's no need
% for you to add the /Title and /Author tags.
\pdfinfo{
/TemplateVersion (2025.1)
}

% DISALLOWED PACKAGES
% \usepackage{authblk} -- This package is specifically forbidden
% \usepackage{balance} -- This package is specifically forbidden
% \usepackage{color (if used in text)
% \usepackage{CJK} -- This package is specifically forbidden
% \usepackage{float} -- This package is specifically forbidden
% \usepackage{flushend} -- This package is specifically forbidden
% \usepackage{fontenc} -- This package is specifically forbidden
% \usepackage{fullpage} -- This package is specifically forbidden
% \usepackage{geometry} -- This package is specifically forbidden
% \usepackage{grffile} -- This package is specifically forbidden
% \usepackage{hyperref} -- This package is specifically forbidden
% \usepackage{navigator} -- This package is specifically forbidden
% (or any other package that embeds links such as navigator or hyperref)
% \indentfirst} -- This package is specifically forbidden
% \layout} -- This package is specifically forbidden
% \multicol} -- This package is specifically forbidden
% \nameref} -- This package is specifically forbidden
% \usepackage{savetrees} -- This package is specifically forbidden
% \usepackage{setspace} -- This package is specifically forbidden
% \usepackage{stfloats} -- This package is specifically forbidden
% \usepackage{tabu} -- This package is specifically forbidden
% \usepackage{titlesec} -- This package is specifically forbidden
% \usepackage{tocbibind} -- This package is specifically forbidden
% \usepackage{ulem} -- This package is specifically forbidden
% \usepackage{wrapfig} -- This package is specifically forbidden
% DISALLOWED COMMANDS
% \nocopyright -- Your paper will not be published if you use this command
% \addtolength -- This command may not be used
% \balance -- This command may not be used
% \baselinestretch -- Your paper will not be published if you use this command
% \clearpage -- No page breaks of any kind may be used for the final version of your paper
% \columnsep -- This command may not be used
% \newpage -- No page breaks of any kind may be used for the final version of your paper
% \pagebreak -- No page breaks of any kind may be used for the final version of your paperr
% \pagestyle -- This command may not be used
% \tiny -- This is not an acceptable font size.
% \vspace{- -- No negative value may be used in proximity of a caption, figure, table, section, subsection, subsubsection, or reference
% \vskip{- -- No negative value may be used to alter spacing above or below a caption, figure, table, section, subsection, subsubsection, or reference
\usepackage{amsmath}
\usepackage{amssymb}
\usepackage{enumitem}
\usepackage{booktabs}
\usepackage{multirow}
\setcounter{secnumdepth}{0} %May be changed to 1 or 2 if section numbers are desired.

% The file aaai25.sty is the style file for AAAI Press
% proceedings, working notes, and technical reports.
%

% Title

% Your title must be in mixed case, not sentence case.
% That means all verbs (including short verbs like be, is, using,and go),
% nouns, adverbs, adjectives should be capitalized, including both words in hyphenated terms, while
% articles, conjunctions, and prepositions are lower case unless they
% directly follow a colon or long dash
\title{Integrating Sequence and Image Modeling in Irregular Medical Time Series Through Self-Supervised Learning}
\author{
    %Authors
    % All authors must be in the same font size and format.
    Liuqing Chen\textsuperscript{\rm 1,2}, Shuhong Xiao\textsuperscript{\rm 1}, Shixian Ding\textsuperscript{\rm 1}, Shanhai Hu\textsuperscript{\rm 1}, Lingyun Sun\textsuperscript{\rm 1,2}
    \thanks{Corresponding Author}
    % Written by AAAI Press Staff\textsuperscript{\rm 1}\thanks{With help from the AAAI Publications Committee.}\\
    % AAAI Style Contributions by Pater Patel Schneider,
    % Sunil Issar,\\
    % J. Scott Penberthy,
    % George Ferguson,
    % Hans Guesgen,
    % Francisco Cruz\equalcontrib,
    % Marc Pujol-Gonzalez\equalcontrib
}
\affiliations{
    %Afiliations
    % \textsuperscript{\rm 1}Association for the Advancement of Artificial Intelligence\\
    \textsuperscript{\rm 1}College of Computer Science and Technology, Zhejiang University, China\\
    \textsuperscript{\rm 2}International Design Institute, Zhejiang University, China \\
    sunly@zju.edu.cn
    % If you have multiple authors and multiple affiliations
    % use superscripts in text and roman font to identify them.
    % For example,
    
    % Sunil Issar\textsuperscript{\rm 2}, 
    % J. Scott Penberthy\textsuperscript{\rm 3}, 
    % George Ferguson\textsuperscript{\rm 4},
    % Hans Guesgen\textsuperscript{\rm 5}
    % Note that the comma should be placed after the superscript

    % 1101 Pennsylvania Ave, NW Suite 300\\
    % Washington, DC 20004 USA\\
    % % email address must be in roman text type, not monospace or sans serif
    % proceedings-questions@aaai.org
%
% See more examples next
}

%Example, Single Author, ->> remove \iffalse,\fi and place them surrounding AAAI title to use it
\iffalse
\title{My Publication Title --- Single Author}
\author {
    Author Name
}
\affiliations{
    Affiliation\\
    Affiliation Line 2\\
    name@example.com
}
\fi

\iffalse
%Example, Multiple Authors, ->> remove \iffalse,\fi and place them surrounding AAAI title to use it
\title{My Publication Title --- Multiple Authors}
\author {
    % Authors
    First Author Name\textsuperscript{\rm 1,\rm 2},
    Second Author Name\textsuperscript{\rm 2},
    Third Author Name\textsuperscript{\rm 1}
}
\affiliations {
    % Affiliations
    \textsuperscript{\rm 1}Affiliation 1\\
    \textsuperscript{\rm 2}Affiliation 2\\
    firstAuthor@affiliation1.com, secondAuthor@affilation2.com, thirdAuthor@affiliation1.com
}
\fi




\begin{document}

\maketitle

\begin{abstract}
Medical time series are often irregular and face significant missingness, posing challenges for data analysis and clinical decision-making. Existing methods typically adopt a single modeling perspective, either treating series data as sequences or transforming them into image representations for further classification. In this paper, we propose a joint learning framework that incorporates both sequence and image representations. We also design three self-supervised learning strategies to facilitate the fusion of sequence and image representations, capturing a more generalizable joint representation. The results indicate that our approach outperforms seven other state-of-the-art models in three representative real-world clinical datasets. We further validate our approach by simulating two major types of real-world missingness through leave-sensors-out and leave-samples-out techniques. The results demonstrate that our approach is more robust and significantly surpasses other baselines in terms of classification performance. 
\end{abstract}

% Uncomment the following to link to your code, datasets, an extended version or similar.
%
\begin{links}
    \link{Code}{https://github.com/zju-d3/AAAI25-Irregular-Medical-Time-Series}
    % \link{Datasets}{https://aaai.org/example/datasets}
    % \link{Extended version}{https://aaai.org/example/extended-version}
\end{links}

% \documentclass[../main.tex]{subfiles}
\graphicspath{{../images/}}
\makeatletter
\def\input@path{{../images/}}
\makeatother
\begin{document}
\section{Introduction}
\begin{figure}
\centering
\begin{tikzpicture}
\node[inner sep=0pt] (ws) at (0, 0) {
\includegraphics[height=.4\textwidth, trim={10cm 0 10cm 0},clip]{world_space.png}};
\node[inner sep=0pt] (cs) at (6,0) {\includegraphics[height=.4\textwidth, trim={10cm 1cm 10cm 4cm},clip]{conf_space.png}};
\end{tikzpicture}
\vspace{-5pt}
\label{fig:pbrm_intro}
\caption{\textbf{Left}: Shows world space obstacles as grey spheres. Robots start and goal configuration is colored red and green, respectively. Configurations along the computed path are colored transparent blue. \textbf{Right:} Mapped world space scenario to configuration space. Obstacle region is the grey mesh. Red spheres are collision-free regions computed by the neural SCDF. The optimized shortest path in the convex corridor is the blue curve.}
\vspace{-25pt}
\end{figure}
Motion planning is the problem of finding a collision-free trajectory that connects a given start and goal configuration. The planning takes place in the configuration space of the robot. For single body robots, like mobile robots or drones, the configuration space and the world space are usually the same. This simplifies the planning, since explicit obstacle representations are available which enables geometrical tools like separating hyperplanes, smallest distance to obstacles etc., to be used when designing motion planning algorithms. For multi-body robots like manipulators, the situation is completely different. The world space obstacles are usually mapped to non-convex regions, and to make the problem even harder, the mapping is usually not known. Forming explicit representations of the obstacle region in the configuration space is usually too expensive or intractable. Despite all of this, sampling based planners are used with great success, which mainly is due to their use of implicit representations of the obstacle region. The basic idea is to construct a graph in the configuration space that covers and connects the collision-free region. From this graph, a path can be extracted that connects a given start and goal configuration. The approach is computationally expensive, since the graph is constructed with the smallest geometrical building block available, points, which represents a collision-check. Furthermore, the extracted paths from the graph are non-smooth and jagged due to the stochastic nature of the approach. This adds an additional post-processing step to the process, where the paths are shortcutted and smoothened, before the path can be used for tracking. Clearly a lot of time is invested to form this graph and produce smooth paths. Thus, if the obstacles start to move, then all of this work is done in no use, since all points that make up this graph need to be re-verified, which is simply too time consuming to be done in real time.
\\\\
In this work, we want to address the existing drawbacks of the sampling based planners. Our main contribution is an improved motion planner where each vertex in the graph covers a collision-free region in the form of a sphere instead of a point and where the edges are formed with neighboring intersecting spheres. This representation has the advantage of instead of returning piecewise linear paths, returning a sequence of overlapping spheres, i.e. a convex corridor, that connects a given start and goal configuration, illustrated in Figure \ref{fig:pbrm_intro}. This convex corridor allows us to use convex optimization to produce smooth trajectories, instead of computationally expensive post-processing methods. The representation further allows us to estimate the coverage of the collision-free space, which gives us awareness and feedback in the offline roadmap construction phase. Finally, our representation is simple to adapt to moving obstacles, simply requery for the new radii and recheck for intersections. 
\\\\
The spherical collision-free regions are formed using a signed distance function (SDF), which is a function that returns the smallest distance from an arbitrary point to the boundary of an obstacle. As the name implies, the distance is signed, thus if the point is inside the obstacle it is negative otherwise positive. If the distance is positive, a sphere with radius equal to the distance is guaranteed to cover a collision-free region. Using an SDF in motion planning is not new, but what is novel about our approach is that we express the distance in the configuration space instead of the world space and by doing so allows us to form these convex collision-free regions. We refer to the resulting SDF as a signed configuration distance function (SCDF). Computing an SCDF analytically is non-trivial, our approach is therefore to parameterize the SCDF with a deep neural network and learn the mapping by supervised learning. Our resulting neural SCDF can compute distances for different parameter values of obstacle shapes and we also show how multiple distances can be combined, thus making our approach flexible.
\section{Related work}
Motion planning algorithms can roughly be divided into three families, grid-based, sampling based and optimization based methods. Grid-based methods (GBM) discretize the planning space from which a graph is then compiled. A standard search method is A$^\star$ \citep{a_star}, which is classified as an \textit{informed} search method, since it employs a heuristic function to speed up the search. A$^\star$ guarantees to return an optimal path at the level of discretization used. GBMs usually discretize the planning space by a regular lattice and this limits the GBMs to problems with low dimensionality due to the curse of dimensionality. Thus, GBMs are usually limited to single-body robots where the degrees of freedom (DOF) are low. To overcome the inherent scaling problem with the GBMs, stochastic methods are usually used for multi-body robots. These methods are termed as sampling-based methods (SBM) and core members within this family are the rapidly-exploring random trees (RRT) \citep{rrt} and the probabilistic roadmap (PRM) \citep{prm}. RRT grows a tree from the start configuration and explores the collision-free region in a rapid way until it is able to connect to the goal region. RRT is usually improved by bi-directional planning \citep{rrt_connect}, i.e. an additional tree is grown from the goal configuration and the trees are tested for connection after any tree has been expanded. RRT is a single-query method, thus it searches for a path from scratch each time it is queried. Contrary to this, PRM is a multi-query method, which solves for multiple queries without starting from scratch. PRM does this by creating a roadmap (graph) that covers the collision-free space as an offline step. The graph is then used to solve for multiple queries. PRMs are used in cases where the environment does not change since the extra offline step is too computationally costly and needs to be re-done if the environment is changed. In our work, we address this inherent issue by using a different roadmap representation. Our vertices in the graph cover a collision-free region in the form of spheres and we form the edges by checking for intersecting spheres. If something in the environment changes, we recompute the spheres radii and recheck the intersections, without relying on collision detection. We use a trained neural network to compute the sphere radius, therefore querying for the radius can be done fast, hence our representation enables the PRM for dynamic environments.
\\\\
In the recent decades, optimization based methods (OBM) \citep{chomp, schulman, itomp, stomp} have been introduced as an alternative to SBM for multi-body robots. Like the SBM, the OBMs scale well to higher dimensional problems and produce smoother motion. It is common to use a SDF in the optimization since it is a smooth function, thus enabling gradient-based methods. However, the standard way of expressing the SDF is in world space. The distance therefore needs to be mapped to the configuration space by the forward kinematics. This mapping makes the optimization problem a non-linear program (NLP), which is computationally expensive to solve. Recently, a different approach has been proposed. In \cite{mp_gcs} motion planning is formulated as a convex optimization problem by using the graph of convex sets framework \citep{gcs}. The underlying idea is to decompose the collision-free space into intersecting convex sets from which a convex optimization problem is formulated. In cases where an explicit representation of the obstacles in the configuration space exists, like for single-body robots, creating collision-free convex regions can be done fast \citep{iris}. For multi-body robots, this is non-trivial. Existing work does this successfully \citep{iris_nlp, iris_c} by an optimization based approach, but the methods are still too time consuming to be used in the presence of moving obstacles. Our approach is instead to use deep learning to learn an SDF expressed in the configuration space. With this, we can query for shortest distances to the collision boundary, which allows us to expand spherical regions which are collision-free. Our approach is fast and therefore enables our suggested roadmap planner to be used in dynamic environments.
\\\\
Recent research has focused on learning collision detection \citep{fk_kernel_distance, diffco, graphdistnet} by predicting the signed distance between the robot links and the surrounding obstacles in the world space. The learned SDF is used in trajectory optimization but since the distance is expressed in the world space, the problem becomes an NLP and therefore takes a long time to solve. We take a novel approach and suggest to instead express the signed distance in the configuration space. This allows us to improve the PRM at the same time as it enables convex optimization for trajectory optimization, which runs faster and is more reliable than NLP solvers. In \cite{cspf} a learned signed distance function in the configuration space is proposed similar to our approach. However, their approach is restricted to point cloud representations, while we propose to represent the obstacles as parameterized geometric shapes, e.g. spheres. Furthermore, we also show how to use our learned SCDF to improve an existing roadmap planner.
\section{Problem formulation}
A robot is located in the world space, $\W \subset \R^3 $. The unique location of the robot is given by its configuration $\q \in \C$, where $\C$ is the configuration space. The set of points covered by the robots bodies at a certain configuration is expressed as $\B(\q) \subset \W$. The robot is surrounded by $\NrObst$ obstacles $\O = \bigcup_{i=1}^{\NrObst} \O_i$, where  $\O_i \subset \W$. The representation of the obstacle in the configuration space is the set $\C\O_i = \{\q \in \C \: |\: \B(\q) \cap \O_i \neq \emptyset \}$. The obstacle space is formed as $\Co = \bigcup_{i=1}^{\NrObst} \C \O_i$. The complement is referred to as the free space, $\Cf = \C \setminus \Co$. The path planning problem is a tuple, ($\Cf$, $\qStart$, $\qGoal$), where we want to connect a query pair, consisting of a start, $\qStart$, and goal configuration, $\qGoal$, with a geometric path, $\q(s): [0, 1] \mapsto \Cf$, such that $\q(0)=\qStart$ and $\q(1)=\qGoal$, or report correctly when such a path does not exist.
\end{document}

% \section{Related Work}
% \subsection{Vision Language Model}
% 시각장애인에서 상황을 설명할 DB가 없으니 만들었다. 그리고 이를 VLM에 튜닝했다.
\subsection{Technical approaches for assisting the visually-impaired}


\subsection{Datasets for visual instruction tuning}

% % \begin{figure}
%     \centering
%     \includegraphics[width=0.5\linewidth]{Move_teaser.pdf}
%     \caption{Comparison of different dynamic compute approaches. length of arrow indicates residual transformation per token while width indicates velocity of transformation.}
%     \label{fig:enter-label}
% \end{figure}

\section{Method}
\label{sec:method}
Residual connections play a crucial role in shaping token representations, yet their dynamics remain underexplored in the context of efficient decoding. In this work, we delve deeper into transformer residual dynamics and investigate how modulating residual transformation velocity can improve inference efficiency in token-level processing, optimizing both dense and sparse MoE transformers.


\subsection{Residual Dynamics and Motivation for Multi-rate Residuals} \label{sec:motivation}

To analyze how hidden representations evolve across different layers of a transformer architecture, it's crucial to consider the effect of residual connections. Each transformer decoder layer typically has residual connections across attention and MLP submodules. As the residual stream $h_i$ traverses from interval $E_j$ to $E_{j+1}$, it undergoes a residual transformation given by:  
% \begin{equation}
% \label{eq:slow_residual_transformation}
% H_{E_{j+1}} = H_{E_j} \prod_{i=E_j}^{E_{j+1}} \left( I + \mathcal{A}_i \right) \left( I + \mathcal{M}_i \right) \quad \text{where} \quad \mathcal{A}_i = f(c_i, h_{i}), \mathcal{M}_i = g(h_i)
% \end{equation}

\begin{equation} \label{eq:slow_residual_transformation}
h_{E_{j+1}} = h_{E_j} + \sum_{i=E_j}^{E_{j+1}-1} \left( \mathcal{A}_i(h_i) + \mathcal{M}_i(h_i + \mathcal{A}_i(h_i)) \right) \quad \text{where} \quad \mathcal{A}_i = f(c_i, h_{i}), \mathcal{M}_i = g(h_i). 
\end{equation}

Here, \( \mathcal{A}_i \) denotes the non-linear transformation introduced by the multi-head attention mechanism at layer \( i \), while \( \mathcal{M}_i \) corresponds to the non-linear transformation of the MLP block at the same layer. These transformations depend on the input residual stream \( h_i \) and, in the case of \( \mathcal{A}_i \), the previous contextual representation \( c_i \).\footnote{Normalization layers are typically applied in practice but are omitted here for simplicity of the argument.}


% For easy tokens, the magnitude and direction of this delta transformation become progressively smaller with each successive layer as shown in \cref{fig:delta_transformation}. Consequently, it is feasible to predict these tokens after only a few residual connections, whereas harder tokens necessitate more extensive processing through additional layers.

\begin{figure}[ht]
    \centering
    \begin{subfigure}{0.48\textwidth}
        \centering
        \includegraphics[width=\textwidth]{sections/figures/residual_change.pdf}
        \caption{}
        \label{fig:residual_change}
    \end{subfigure}%
    \hfill
    \begin{subfigure}{0.48\textwidth}
        \centering
        \includegraphics[width=\textwidth]{sections/figures/alignment_wrt_dedicated_model.pdf}
        \caption{}
    \label{fig:alignment_wrt_dedicated_model}
    \end{subfigure}
    \caption{(a) As residual streams propagate through the model, the directional shifts in the residuals become progressively smaller. (b) A dedicated model with $k$ layers achieves a faster rate of change in residual streams and higher alignment than base model leveraging early exit mechanisms at layer $k$.}
    \label{fig}
\end{figure}


To examine whether residual transformations can be accelerated across layers, we conducted experiments using a diverse set of prompts on a pre-trained Phi3 model~\cite{phi3_report}. As illustrated in \cref{fig:residual_change}, we measured the directional shift in residual states as \( 1 - \mathcal{C}(h_{i-1}, h_i) \), where \(\mathcal{C}\) denotes normalized cosine similarity. This shift is notably higher in the initial layers, gradually decreasing in subsequent layers. This behavior allows traditional early exit approaches to effectively accelerate decoding by enabling earlier exits for simpler tokens. However, these approaches typically rely on a distance-based approximation, where the full residual transformation of the model is approximated by the residual transformations of the initial layers. To gain deeper insights into the distance versus velocity aspects of residual transformation, we conducted a comparative study. Specifically, we trained an early exit head at layer $k$ of the Phi3 model, which consists of 32 layers, restricting the distance traveled by each token. To accelerate the residual transformation relative to number of layers, we trained a smaller model consisting of only $k$ layers, while keeping all other hyperparameters consistent. We then compared the next-token prediction accuracy of the early exit head of the base model with that of the smaller model. To ensure an equal number of trainable parameters, we inserted low-rank adapters into the smaller model and trained only these adapters, whereas, in the distance-based approach, we trained solely the early exit head. In addition, to accelerate the residual transformation in smaller model, we distilled the residual streams from the larger model by incorporating a distillation loss ~\cite{sanh2019distilbert} between the residual state at layer \(i\) of the smaller model and the residual state at layer \(4 \times i\) of the larger model. As shown in ~\cref{fig:alignment_wrt_dedicated_model} the smaller model demonstrates a significantly faster rate of change in residual streams, leading to higher next token prediction accuracy after $k$ layers compared to the base model that employs traditional early exit mechanisms after $k$ layers \cite{schuster2022confident, chen2023eellm, varshney-etal-2024-investigating}. This experimental setup, which modifies only the rate of change in residual streams while keeping other factors constant, suggests that dense transformers, trained with a fixed number of layers, may inherently possess a slow residual transformation bias.

This observation raises an intriguing question: if the rate of change in residual streams could be accelerated relative to the number of layers, is it possible to facilitate earlier alignment for a greater proportion of tokens? Earlier alignment would be beneficial to not only facilitate dynamic computation but also for generating speculative tokens efficiently with high acceptance rates in speculative decoding setups ~\cite{leviathan2023fast, chen2023accelerating}. 

%thereby enhancing the efficiency of early exiting? 
 % This bias likely constrains the effectiveness of early exiting, particularly for easier tokens. By addressing this limitation through accelerated residual transformations, we hypothesize that it is possible to substantially improve the efficiency and accuracy of early exit strategies in transformer models.

\subsection{Multi-Rate Residual Transformation} \label{m2r2_method}

To address the slow residual transformation bias described in ~\cref{sec:motivation}, we introduce \textit{accelerated residual streams} that operate at rate $R$ relative to original slow residual stream. We pair slow residual stream, $h$ with an accelerated residual stream, $p$, which has an intrinsic bias towards earlier alignment. Relative to ~\cref{eq:slow_residual_transformation}, accelerated residual transformation from interval $E_j$ to $E_{j+1}$ can be represented as: 

% \begin{equation}
% \label{eq:fast_residual_transformation}
% P_{E_{j+1}} = P_{E_j} \prod_{i=E_j}^{E_{j+1}} \left( I + \hat{\mathcal{A}_i} \right) \left( I + \hat{\mathcal{M}_i} \right) \quad \text{where} \quad \hat{\mathcal{A}_i} = \hat{f}(c_i, P_{i}), \hat{\mathcal{M}_i} = \hat{g}(P_{i})
% \end{equation}


\begin{equation} \label{eq:fast_residual_transformation}
p_{E_{j+1}} = p_{E_j} + \sum_{i=E_j}^{E_{j+1}-1} \left( \hat{\mathcal{A}_i}(p_i) + \hat{\mathcal{M}_i}(p_i + \hat{\mathcal{A}_i}(p_i)) \right) \quad \text{where} \quad \hat{\mathcal{A}_i} = \hat{f}(c_i, p_{i}), \hat{\mathcal{M}_i} = \hat{g}(h_i), 
\end{equation}



where $\hat{\mathcal{A}_i}$ and $\hat{\mathcal{M}_i}$ denote non-linear transformation added by layer $i$ to previous accelerated residual $p_{i}$. Similar to $\mathcal{A}_i$, non-linear transformation $\hat{\mathcal{A}_i}$ attends to same context $c_i$ but uses a different transformation $\hat{f}$ for accelerating $p_{E_j}$ relative to $h_{E_j}$. 

We integrate accelerated residual transformation directly into the base network using parallel accelerator adapters such that rank of accelerator adapters $R_p << d$ where $d$ denotes base model hidden dimension. This setup allows the slow residual stream $h_{E_j}$ to pass through the base model layers while the accelerated residual stream $p_{E_j}$ utilizes these parallel adapters as shown in ~\cref{fig:m2r2_main}. Both slow and accelerated residuals are processed in same forward pass via attention masking and incur negligible additional inference latency in memory bound decoding setups, while in compute bound decoding setups where FLOPs optimization is essential, accelerated residual stream utilizes a fraction of attention heads that of slow residual (see ~\cref{sec:flops_optimization}). Additionally, to maximize the utility of accelerated residual transformations without introducing dedicated KV caches, we propose a shared caching mechanism between the slow and accelerated streams which minimally impact alignment benefits of our approach while offering substantial memory savings (see ~\cref{fig:koala_alignment}). Specifically, the attention operation on the slow residuals \( \text{MHA}(h_t, h_{\leq t}, h_{\leq t}) \) is redefined for accelerated residuals as 
\[
\hat{\mathcal{A}} = MHA(p_t, h_{<t} \oplus p_t, h_{<t} \oplus p_t),
\]
where the accelerated residual at time-step $t$, \( p_t \) attends to the slow residual’s KV cache, facilitating the reuse of contextual information across both residual streams without incurring additional caching costs. Here, \(MHA(q, k, v) \) represents multi-head attention between query \( q \), key \( k \), and value \( v \).

\begin{figure}
    \centering
    \includegraphics[width=0.8\linewidth]{sections//figures/m2r2_main2.pdf}
    \caption{Multi-rate Residuals Framework: Slow residual stream of base model is accompanied by a faster stream that operates at a $2-(J+1)\times$ rate relative to the slow stream, undergoing transformations via accelerator adapters as detailed in \cref{m2r2_method}, where J denotes number of early exit intervals. Colors within the slow and fast residual streams indicate similarity, with matching colors representing the most closely aligned residual states. At the beginning of the forward pass and at each exit point, the accelerated residual state is initialized from the corresponding slow residual state to avoid gradient conflict during training (see ~\cref{sec:grad_conflict}). Early exiting decisions are informed by the Accelerated Residual Latent Attention (ARLA) mechanism, described in \cref{method_arla}, which evaluates residual dynamics across consecutive exit gates.}
    \label{fig:m2r2_main}
\end{figure}

% Furthermore. to maximize the benefits of fast residual transformations without using dedicated KV caches, we propose sharing the fast network’s cache with the slow network. Formally speaking, We modify attention operation on slow residuals $MHA(H_t, H_{<=t}, H_{<=t})$ as $MHA(P_{t}, H_{<t} \oplus P_t, H_{<t}  \oplus P_t)$ such that accelerated residuals attend to previous slow context KV cache, where $MHA(q,k,v)$ denotes multi head attention between query, $q$, key $k$ and value $v$.


\subsection{Enhanced Early Residual Alignment}
Early residual alignment is instrumental in optimizing early exiting, speculative decoding, and Mixture-of-Experts (MoE) inference mechanisms. In this section, we provide a detailed analysis of how accelerated residuals enhance these inference setups.

% By aligning the residual states of intermediate layers with the final output representations, the model can maintain high prediction accuracy even when computations are truncated at earlier layers. This enables more reliable early exiting, reducing the overall computational cost while preserving performance. Additionally, in speculative decoding, early residual alignment allows the model to make confident predictions using faster, partial computations, thereby accelerating inference without sacrificing output quality.


\subsubsection{Early Exiting} \label{method_early_exiting}

A prevalent strategy for enabling early exiting at an intermediate layer $E_{j}$ involves approximating the residual transformation between $E_{j}$ and the final layer $N-1$ using a linear, context independent mapping, $\mathcal{T}$, such that $H_{N-1} \approx \mathcal{T}(H_{E_{j}})$. This approximation has been extensively employed in conventional approaches ~\cite{schuster2022confident, chen2023eellm, varshney-etal-2024-investigating}, providing a computationally efficient means to project the output of deeper layers from intermediate states. Specifically, residual state of layer $N-1$ with this approximation can be expressed as:


% \begin{equation}
% \label{eq: vanila_ea_assumption}
% \Phi(H_{E_{j}}) \sim H_{E_{j}} \prod_{i=E_{j}}^{N}\left( I + \mathcal{A}_i \right) \left( I + \mathcal{M}_i \right) \quad \text{where} \quad \Phi \perp C
% \end{equation}

\begin{equation} \label{eq:early_exiting}
h_{E_j} + \sum_{i=E_j}^{N-1} \left( \mathcal{A}_i(h_i) + \mathcal{M}_i(h_i + \mathcal{A}_i(h_i)) \right) \sim \mathcal{T}(h_{E_{j}})  \quad \text{where} \quad \mathcal{T} \perp c. 
\end{equation}


Here, $\mathcal{A}_i$ and $\mathcal{M}_i$ represent the residual contributions of the multi-head attention and MLP layers, respectively, while $\mathcal{T}$ remains independent of $c$, the preceding context.

This approach is inherently limited by two major factors: first, the assumption of linearity between $h_{E_{j}}$ and $h_{N-1}$ may not hold uniformly for all tokens, particularly when $E_j \ll N$. Second, the linear transformation $\mathcal{T}$ disregards the influence of the context $c$ and fails to account for the latent representations of previous contextual states. In contrast, M2R2 accelerated residual states mitigate both of these challenges by approximating the slow residual transformation of all layers via a faster residual transformation of fewer layers as:
% \begin{equation}
% H_{E_j} \prod_{i=E_j}^{N}\left( I + \mathcal{A}_i \right) \left( I + \mathcal{M}_i \right) \sim P_{E_j} \prod_{i=E_j}^{E_j+1}\left( I + \hat{\mathcal{A}_i} \right) \left( I + \hat{\mathcal{M}_i} \right)
% \end{equation}


\begin{equation} \label{eq:m2r2_approximating_ea}
h_{E_j} + \sum_{i=E_j}^{N-1} \left( \mathcal{A}_i(h_i) + \mathcal{M}_i(h_i + \mathcal{A}_i(h_i)) \right) \sim p_{E_j} + \sum_{i=E_j}^{E_{j+1}-1} \left( \hat{\mathcal{A}_i}(p_i) + \hat{\mathcal{M}_i}(p_i + \hat{\mathcal{A}_i}(p_i)) \right), 
\end{equation}

% \begin{equation} \label{eq:fast_residual_transformation}
% p_{E_{j+1}} = p_{E_j} + \sum_{i=E_j}^{E_{j+1}-1} \left( \hat{\mathcal{A}_i}(p_i) + \hat{\mathcal{M}_i}(p_i + \hat{\mathcal{A}_i}(p_i)) \right) \quad \text{where} \quad \hat{\mathcal{A}_i} = \hat{f}(c_i, p_{i}), \hat{\mathcal{M}_i} = \hat{g}(h_i) 
% \end{equation}






where $p_{E_j}$ is initialized from the slow residual state $h_{E_j}$ at each early exit interval $E_j$ using an identity transformation (see ~\cref{fig:m2r2_main}). As shown in ~\cref{fig:m2r2_residual_sim}, accelerated residuals offer a smoother, more consistent shift in residual direction across layers, in contrast to the abrupt changes typically seen at early exit points in standard early exit methods. Moreover, the normalized cosine similarity between accelerated states at early exit intervals and final residual states is substantially higher compared to traditional early exit techniques, highlighting improved alignment with final layer representations. Traditional adaptive compute methods are constrained by two principal factors: the number of tokens eligible for early exit at intermediate layers and the precision of early exit decision. If residual streams fail to saturate early, the majority of tokens remain ineligible for exit, thereby diminishing potential speedups. Additionally, imprecise delineations between tokens suitable for early exit can lead to underthinking (premature exits that adversely affect accuracy) or overthinking (unnecessary processing that compromises efficiency) ~\cite{zhou2020self, dai2020dynamic}. Enhanced early alignment using ~\cref{eq:m2r2_approximating_ea} helps to address  first issue. To address the second issue we introduce Accelerated Residual Latent Attention, which dynamically assesses the saturation of the residual stream, allowing for a more precise differentiation between tokens that can exit early and those requiring further processing.

% This results in uniform change in residual direction    
% % We keep $\mathcal{A} = \hat{\mathcal{A}}$, while $\hat{\mathcal{M}}$ is accelerated by a factor of $2 - (N_{E}+1)X$ relative to the slower residual transformation $\mathcal{M}$, where $N_E$ represents number of early exiting intervals.
% Figure~\cref{fig:rate_change_comparison} illustrates the comparative rate of change between these transformation streams.



% fig:rate_change_comparison
% - grid plot x axis -> layer id (0, 8) , y axis -> layer id -> dark color cell for max similarity , lighter for lower 
% 
-------------------------------------------------------
Let's consider residual stream $h_i$ traverses through interval $E_j$ to $E_{j+1}$ and undergoes residual transformation given by 
\begin{equation}
h_{E_{j+1}} = h_{E_j} \prod_{i=E_j}^{E_{j+1}} \left( 1 + \delta_i \right)    
\end{equation}

where $\delta_i$ denotes non-linear transformation added by layer $i$. Each non-linear transformation of layer $i$ is a function of previous contextual representation, $c_i$ and input residual stream $h_i-1$ as
$\delta_i = f(c_i, h_{i-1})$ 

One way to exit early at exit $E_j+1$ is to assume that residual transformation from $E_j+1$ to final layer $N-1$ can be approximated by a linear function $\phi$ as $h_{N-1} \sim \Phi(h_{E_j+1})$ and most conventional approaches such as \todo{cite EA papers} use this approach. In other words, 

\begin{equation}
\Phi(h_{E_j+1} \sim h_{E_j+1} \prod_{i=E_j+1}^{N} \left( 1 + \delta_i \right)   
\end{equation}

This approach suffers from two primary issues, linearity assumption from $h_E_j+1$ to $H_N-1$ if often incorrect, particularly when $E_j << N$. More importantly, linear transformation $\Phi$ doesn't consider effect of context $C_i$. M2R2  effectively addresses these issues as accelerated residual stream at interval $E_j+1$ can be represented as 

\begin{equation}
r_{E_{j+1}} = r_{E_j} \prod_{i=E_j}^{E_{j+1}} \left( 1 + \gamma_i \right)    
\end{equation}

where $\gamma_i$ denotes non-linear transformation added by layer $i$ to previous accelerated residual $r_i-1$. Similar to $\delta_i$, non-linear transformation $\gamma_i$ considers context $C_i$ as 
$\gamma_i = g(c_i, r_{i-1})$. So in summary, slow residual transformation is approximated by accelerated residual as: 

\begin{equation}
h_{E_j} \prod_{i=E_j}^{N} \left( 1 + \delta_i \right) \sim h_{E_j} \prod_{i=E_j}^{E_j+1} \left( 1 + \gamma_i \right)
\end{equation}

It's worth noting that accelerated residual $r_i$ and slow residual $h_i$ are processed concurrently at layer $i$ by constructing proper attention mask such as attention of slow residual is represented as 

$MHA(H_it, H_{i<=t}, H_{i<=t}$ while attention of fast residual is computed as 

$MHA(r_it, H_{i<=t}, H_{i<=t}$ where $MHA(q,k,v$ denotes multi head attention between query, $q$, key $k$ and value $v$.


------------------------------------------------------------------

Vertical latent attention on accelerated residual is computed as 
$MHA(S_mt, S(Ej<=i<=m)t, S(Ej<=i<=m)t)$ where $Smt$ denotes query/key/value projection in latent domain at layer $m$ at time $t$. 
------------------------------------------------------------------

Gradient conflict Avoidance: 

Let's consider $w_j$ is a trainable parameter that belongs to a layer between $E_j$ and $E_j+1$. Consider early exit loss at gate $E_j+1$, $L_j+1$, gradient propagation of $w_j$ at another trainable parameter $w_j-n$ can be gives as 

$\sum_{k=E_j-n}^{E_j} \beta_k \frac{\partial L_{E_k}}{\partial w_k}$

where $\beta_j$ denotes backward transformation coefficient for weight $w_j$ to reach gate $E_j$. 
 
On the other hand, gradient propagation in proposed approach can be represented as 

\[
\frac{\partial L_{E_j}}{\partial w_j} = 
\begin{cases} 
\beta_j \frac{\partial L_{E_j}}{\partial w_j} & \text{if } E_j \leq w_j \leq E_{j+1} \\
0 & \text{otherwise}
\end{cases}
\]







% \begin{figure}[ht]
%     \centering
%     \includegraphics[width=0.8\textwidth, height=5cm]{rate_change_comparison.png}
%     \caption{Rate of change comparison between fast and slow residual streams.}
%     \label{fig:rate_change_comparison}
% \end{figure}

%vary k and and plot EA accuracy for larger and smaller models. 

% \begin{figure}[ht]
%     \centering
%     \includegraphics[width=0.5\textwidth,height=5cm]{sections/figures/alignment_comparison_dialogsum.pdf}
%     \caption{Alignment of exited tokens for different early exit layers using traditional early exiting heads, dedicated faster networks, and faster residuals.}
%     \label{fig:small_model_early_exiting}
% \end{figure}


\textbf{Accelerated Residual Latent Attention} \label{method_arla}

In the context of residual streams, we observe that the decision to exit at a given layer can be more effectively informed by analyzing the dynamics of residual stream transformations, instead of solely relying on a classification head applied at the early exit interval $E_j$. To capture the subtle dynamics of residual acceleration, we propose a \textit{Accelerated Residual Latent Attention} (ARLA) mechanism. This approach involves making the exit decision at gate $E_j$ by attending to the residuals spanning from gate $E_{j-1}$ to $E_j$, rather than considering only the residual at gate $E_j$. To minimize the computational overhead associated with exit decision-making, the attention mechanism operates within the latent domain as depicted in ~\cref{fig:arla_arch}. Formally, for each interval $[E_j, E_{j+1}]$, the accelerated residuals are projected into Query ($Q^s_{E_j}, \ldots, Q^s_{E_{j+1}}$), Key ($K^s_{E_j}, \ldots, K^s_{E_{j+1}}$), and Value ($V^s_{E_j}, \ldots, V^s_{E_{j+1}}$) vectors, with latent dimension $d^s$ for $Q^s$, $K^s$, and $V^s$ being significantly smaller than hidden dimension of $p$.\footnote{We use $d^s = 64$ for experiments described in ~\cref{sec:experiments}.} Notably, when the router is allowed to make exit decisions at gate $E_j$ based on residual change dynamics, we observe that the attention is not confined to the residual state at $E_j$ but is distributed across residual states from $E_{j-1}$ to $E_j$, %as illustrated in Figure~\ref{fig:vertical_latent_attention_dynamics}. 
This broader focus on residual dynamics significantly reduces decision ambiguity in early exits, as demonstrated in Figure~\ref{fig:roc_arla}, which contrasts routers based on the last hidden state, and the proposed ARLA router.

%show R -> S transformation. 
%show parameter and flop overhead as compared to adapter on last hidden state.

% \begin{figure}[ht]
%     \centering
%     \includegraphics[width=0.5\textwidth,height=5cm]{sections/figures/roc_arla.pdf}
%     \caption{ROC curves of early exit decision strategies: confidence-based methods (CALM/LITE), routers based on the accelerated hidden state, and latent attention routers.}
%     \label{fig:decision_making_comparison}
% \end{figure}

% \begin{figure}[ht]
%     \centering
%     \includegraphics[width=0.5\textwidth,height=5cm]{vertical_latent_attention.png}
%     \caption{Vertical latent attention mechanism for optimizing early exit decisions by considering residuals from gate \(M\) through \(M-1\).}
%     \label{fig:vertical_latent_attention}
% \end{figure}

\begin{figure}[ht]
    \centering
    \begin{subfigure}{0.52\textwidth}
        \centering
        \includegraphics[width=\textwidth, height = 4cm]{sections/figures/arla_arch.pdf}
        \caption{Accelerated Residual Latent Attention (ARLA): Accelerated residuals between early exit gates are projected into latent domain and attention over residual states within the interval is computed to capture residual dynamics and exit decision is made based on residual saturation.}
        \label{fig:arla_arch}
    \end{subfigure}%
    \hfill
    \begin{subfigure}{0.45\textwidth}
        \centering
        \includegraphics[width=\textwidth, height = 4.5cm]{sections/figures/vla_roc.pdf}
        \caption{ROC classification curves of early exit decision strategies using a linear router used on last residual state ~\cite{schuster2022confident, varshney-etal-2024-investigating, chen2023eellm}  and using ARLA approach that considers residual dynamics. }
        \label{fig:roc_arla}
    \end{subfigure}
    \caption{Effectiveness of ARLA in capturing residual dynamics for early exiting decisions.}


\end{figure}



% \begin{figure}[ht]
%     \centering
%     \includegraphics[width=1\textwidth,height=5cm]{sections/figures/arla.pdf}
%     \caption{fig that plots 32 rows 2 cols heatmap showing attention at each gate}
%     \label{fig:vertical_latent_attention_dynamics}
% \end{figure}

\subsubsection{Self Speculative Decoding} \label{method_self_speculative_decoding}

An alternative means to exploit the early alignment properties of our approach is through the use of accelerated residual states for speculative token sampling to accelerate autoregressive decoding. Speculative decoding aims to speed up memory-bound transformer inference by employing a lightweight draft model to predict candidate tokens, while verifying speculated tokens in parallel and advancing token generation by more than one token per full model invocation \cite{leviathan2023fast, chen2023accelerating, xia2023speculative, miao2023specinfer}. Despite its effectiveness in accelerating large language models (LLMs), speculative decoding introduces substantial complexity in both deployment and training. A separate draft model must be specifically trained and aligned with the target model for each application, which increases the training load and operational complexity ~\cite{chen2023accelerating}. Additionally, this approach is resource-inefficient, as it requires both the draft and target models to be simultaneously maintained in memory during inference \cite{leviathan2023fast, chen2023accelerating}. 

One strategy to address this inefficiency is to leverage the initial layers of the target model itself to generate speculative candidates, as depicted in ~\cite{Tang2024}. While this method reduces the autoregressive overhead associated with speculation, it suffers from suboptimal acceptance rates. This occurs because the linear transformation employed for translating hidden states from layer $k$ to the final layer $N$ is typically a poor approximation, as discussed in ~\cref{sec:motivation} and ~\cref{method_early_exiting}. Our approach resolves this limitation by utilizing accelerated residuals, which demonstrate higher fidelity to their slower counterparts. By utilizing accelerated residuals operating at a rate of $N/k$, where $k$ denotes the number of layers used for candidate speculation, we are able to efficiently generate speculative tokens for decoding.\footnote{We typically set $k = 4$ to balance the trade-off between autoregressive drafting overhead and acceptance rate, as discussed in~\cref{sec:experiments}.}
 This technique not only obviates the need for multiple models during inference but also improves the overall efficiency and effectiveness of speculative decoding.

\begin{figure}
    \centering    \includegraphics[width=1\linewidth]{sections/figures/m2r2_aot_loading.pdf}
    \caption{Ahead-of-Time Expert Loading: M2R2 accelerated residual stream predicts experts required for future layers, reducing reliance on on-demand lazy loading. Speculative pre-loading is efficiently overlapped with computation of multi-head attention (MHA) and MLP transformations. Only incorrectly speculated experts are loaded lazily, resulting in faster inference steps and improved computational efficiency. Here, H indicates LBM Host while D indicates HBM Device.}
    \label{fig:moe_expert_aot_loading}
\end{figure}


\subsubsection{Ahead of Time Expert Loading:} \label{method_aot_expert_loading}

Recent advancements in sparse Mixture-of-Experts (MoE) architectures ~\cite{shazeer2017outrageously, fedus2022switch, artetxe2019massively, lepikhin2020gshard, zoph2022designing} have introduced a paradigm shift in token generation by dynamically activating only a subset of experts per input, achieving superior efficiency in comparison to dense models, particularly under memory-bound constraints of autoregressive decoding \cite{fedus2022switch, zoph2022designing}. This sparse activation approach enables MoE-based language models to generate tokens more swiftly, leveraging the efficiency of selective expert usage and avoiding the overhead of full dense layer invocation. In dense transformer models, pre-loading layers is a common strategy to enhance throughput, as computations of current layer can be overlapped with pre-loading of next layer parameters ~\cite{narayanan2021efficient, shoeybi2020megatron}. However, MoE models face a unique challenge: expert selection occurs dynamically based on previous layer’s output, making it infeasible to preload next layer’s experts in parallel. This limitation results in inherent latency, as expert loading becomes a sequential, on-demand process ~\cite{lepikhin2020gshard, fedus2022switch}.

To address this inefficiency, our method introduces a mechanism with \textit{accelerated residuals}, which not only captures key characteristics of base slower residual states but also exhibit high cosine similarity with their final counterparts (as illustrated in \cref{fig:m2r2_residual_sim}). By employing accelerated residual streams, we can effectively predict the necessary experts for future layers well in advance of their actual invocation. Specifically, using a $2\times$ accelerated residual, the experts needed for layers $2i+2$ and $2i+3$ can be identified while still computing in layer $i$, thus overcoming the bottleneck of sequential, on-demand expert selection and mitigating latency in the decoding pipeline, as shown in \cref{fig:moe_expert_aot_loading}. Note that, we use fixed set of accelerator adapters for transforming accelerated residuals (as discussed in ~\cref{m2r2_method}) while slow residual is transformed via expert routing mechanism. 

Furthermore, our approach integrates a Least Recently Used (LRU) caching strategy, which enhances memory efficiency by replacing the least recently used experts with speculated experts that are anticipated to be needed in upcoming layers. This hybrid approach of preemptive expert loading with LRU caching yields substantial improvements over traditional on-demand loading or standalone caching strategies. By minimizing cache misses and efficiently managing memory, this approach addresses both compute and memory bottlenecks, leading to faster, more resource-efficient token generation in MoE architectures. A comprehensive evaluation of this strategy, in relation to state-of-the-art methods, is provided in \cref{experiments_aot}, and the compute and memory traces on an A100 GPU are detailed in \cref{fig:moe_aot_cuda_trace}.



% Recent advancements in sparse Mixture-of-Experts (MoE) architectures have introduced the concept of utilizing distinct computational paths for different tokens \cite{shazeer2017outrageously}. This approach, wherein only a subset of experts are activated per input, enables MoE-based language models to generate tokens more swiftly compared to their dense counterparts due to memory-bound nature of auto-regressive decoding. In dense models, pre-loading layers in advance is a common strategy to enhance computational efficiency. However, this technique is not applicable to MoE models, where expert selection occurs dynamically based on the outputs of previous layers, preventing parallel pre-fetching of experts.

% Our proposed method addresses this inefficiency. Accelerated residuals, which are highly similar to their slower counterparts (see \cref{fig:similarity}), can reliably predict the necessary experts ahead of time. For instance, by utilizing $2X$ accelerated residual stream, we can predict the experts needed for the layer $2i+1$ and $2i+3$ while carrying out computation in layer $i$. This enables us to commence expert loading significantly earlier, as illustrated in \cref{expert_loading}, effectively mitigating the delays observed with the naive on-demand expert loading. Additionally, our method benefits from incorporating a Least Recently Used (LRU) strategy, where speculated experts replace those that are least recently utilized, resulting in improved performance compared to using either strategy alone. For a comprehensive evaluation, refer to \cref{moe_trace}, which provides a CUDA compute and memory trace of our approach executed on <>.



% A naive solution involves using the residual state of the previous layer along with the gating function of the next layer to predict which experts need to be loaded, and initiating the expert loading process in parallel with the attention computation of the next layer. Yet, as shown in \cref{fig:MOE_attn_vs_loading_time}, the attention computation for medium to long contexts is considerably faster than the expert loading time, making this approach inefficient.




\subsection{Training} \label{method_training}
% This approach is feasible due to the absence of gradient conflicts, as discussed in \cref{sec:grad_conflict}.

To accelerate residual streams, we employ parallel accelerator adapters as described in \cref{m2r2_method}.  For the early exiting use-case outlined in \cref{method_early_exiting}, we define the training objective for these adapters using the following loss function, which combines cross-entropy loss at each exit $E_j$ with distillation loss at each layer $i$. Loss weights coefficients $\alpha_0$ and $\alpha_1$ are employed to balance contribution of corresponding losses.

\begin{align} \label{eq:mr_loss}
L_{\text{m2r2}} = \underbrace{-\alpha_0 \sum_{j=1}^{J} \sum_{t=1}^{T} \log p_{\theta} \left( \hat{y}_t^{E_j} \mid y_{<t}, x \right)}_{\text{cross-entropy loss}} 
+ \underbrace{\alpha_1\sum_{i=1}^{E_{J-1}} \sum_{t=1}^{T} \| \mathbf{p}_{t}^{i} - \mathbf{h}_{t}^{((i - E_{j(i)}) \cdot R_i) + E_{j(i)})} \|^2}_{\text{distillation loss}}.
\end{align}

where $\hat{y}_t^{E_j}$ denotes the predictions from the accelerated residual stream at layer $E_j$ and time step $t$, $y_t$ represents the corresponding ground truth tokens, and $x$ indicates previous context tokens. The distillation loss at each layer $i$ is computed by comparing accelerated residuals at layer $i$ with slow residuals at layer $(i - E_{j(i)}) \cdot R_i + E_{j(i)}$, where $R_i$ denotes the rate of accelerated residuals at layer $i$ while $E_{j(i)}$ represents the most recent gate layer index such that $E_{j(i)} <= i$. \( J \) represents the total number of early exit gates, N denotes number of hidden layers and $E_j$ denotes layer index corresponding to gate index $j$ and \( T \) denotes the sequence length. 

In dynamic compute settings, after training of accelerator adapters, we optimize the query, key, and value parameters governing the ARLA routers (see ~\cref{method_arla}) across all exits in parallel on binary cross entropy loss between predicted decision and ground truth exiting decision. The ground truth labels for the router are determined based on whether the application of the final logit head on $\hat{y}_t^{E_j}$ yields the correct next-token prediction. 


% The objective for this optimization is defined by the following loss function:


%TODO are equations required ? 
% \begin{equation} \label{eq:arla_loss_combined}\small
%     L_{\text{arla}} = -\frac{1}{N} \sum_{t=1}^{T} \left( \sum_{j=1}^{E_n} \left[ O_t^{E_j} \log(\hat{O}_t^{E_j}) + (1 - O_t^{E_j}) \log(1 - \hat{O}_t^{E_j}) \right] \right), \quad \text{where} \quad 
%     O_t^{E_j} = \begin{cases} 
%     1, & \text{if } L(\hat{y}_t^{E_j}) = y_t^{E_j} \\
%     0, & \text{otherwise}
%     \end{cases}
% \end{equation}

% where $\hat{O}_t^{E_j}$ represents the binary predicted logits produced by the vertical latent attention router, as described in \cref{sec:arla}, at gate $E_j$ and time step $t$, and $O_t^{E_j}$ denotes the corresponding ground truth labels. The ground truth labels for the router are determined based on whether the application of the logit head on $\hat{y}_t^{E_j}$ yields the correct next-token prediction. The parameters controlling vertical latent attention are trained concurrently to ensure consistency and efficient use of computational resources.

For self-speculative decoding, as described in \cref{method_self_speculative_decoding}, the training objective remains the same as \cref{eq:mr_loss}, but with the number of intervals set to $J = 1$ and the rate of residual transformation set to $R_n = N/k$, where the first $k$ layers generate speculative candidate tokens. In the context of Ahead-of-Time Expert Loading for Mixture-of-Experts (MoE) models (see \cref{method_aot_expert_loading}), setting the rate of residual transformation to $R_n = 2$ typically offers a good trade-off between the accuracy of expert speculation and AoT pre-loading of experts. 

% Thus, we set $J = 1$ and $E_1 = 16$.


~\subsection{FLOPs Optimization} \label{sec:flops_optimization}

Naively implemented, M2R2 incurs higher FLOP overhead compared to traditional speculative decoding and early exiting approaches such as ~\cite{medusa, schuster2022confident, Tang2024}. However, modern accelerators demonstrate compute bandwidth that exceeds memory access bandwidth by an order of magnitude or more~\cite{databricksLLMInference2023, jouppi2021ten}, meaning increased FLOPs do not necessarily translate to increased decoding latency. Nevertheless, to ensure fair comparison and efficiency in compute bound scenarios, we introduce targeted optimizations.

~\textbf{Attention FLOPs Optimization} For medium-to-long context lengths, attention computation dominates FLOPs in the self-attention layer, surpassing the contribution from MLP layers. Specifically, matrix multiplications involving queries, cached keys, and cached values scale with $l_{kv} * l_{q}$ where $l_{kv}$ denotes previous context length and $l_q$ denotes current query length. Since M2R2 pairs accelerated residuals with slow residuals, a naive implementation results in twice the FLOPs consumption compared to a standard attention layer. To address this, we limit the attention of accelerated residual stream to selectively attend to the top-k most relevant tokens, identified by the slow residual stream based on top attention coefficients\footnote{We set to k = 64 and attend to top 64 tokens as identified by the slow residual stream.}. This is possible since slow and accelerated residual streams are processed in same forward pass and accelerated streams have access to attention coefficients of slow stream. Note that, the faster residual stream still retains the flexibility to assign distinct attention coefficients to these tokens. Furthermore, we design the faster residual stream to employ only 8 attention heads, compared to the 32 heads used in the slow residual stream of the Phi-3 model, reducing query, key, value, and output projection FLOPs by a factor of 1/4. ~\cref{fig:m2r2_num_heads_ablation} indicates effect of using a slicker stream on alignment. As depicted, using $\hat{n}_h = 8$ offers a good trade-off between alignment and FLOPs overhead. 

~\textbf{MLP FLOPs Optimization} The accelerator adapters operating on the accelerated residual stream are intentionally designed with lower rank than their counterparts in the base model. This reduces FLOP overhead by a factor proportional to $hiddenSize / rank$. Additionally, since the faster residual stream uses only 8 attention heads (compared to 32 in the slow residual stream of Phi-3), the subsequent MLP layers process a smaller set of activations, further reducing FLOPs by another factor of 1/4.

These optimizations significantly reduce the FLOP overhead per speculative draft generation, as illustrated in ~\cref{fig:flops_optmization}. Notably, while traditional early-exiting speculative approaches such as DEED require propagating the full slow residual state through the initial layers, incurring substantial computational costs, M2R2 achieves efficient token generation via slimmer, low-rank faster residual streams. In contrast, Medusa introduces considerable FLOP overhead due to per-head computations scaling with $d^2+dv$\footnote{Here $d$ denotes hidden state dimension while $v$ denotes vocab size.}, whereas M2R2 employs low-rank layers for both MLP and language modeling heads, maintaining computational efficiency. All experiments involving the M2R2 approach, as detailed in ~\cref{sec:experiments}, are conducted using these FLOPs optimizations.









% \[
% O_t^{E_j} = 
% \begin{cases} 
% 1, & \text{if } L(\hat{y}_t^{E_j}) = y_t^{E_j} \\
% 0, & \text{otherwise}
% \end{cases}
% \]




%add distillation
% We train accelerator adapters described in \cref{m2r2_method} to accelerate residual streams on next token prediction all in parallel since there are no gradient conflict issues as described in \cref{sec:grad_conflict}.

% \begin{align} \label{eq:mr_loss}
% L_{mr} =  & -\sum_{j = 1}^{E_n} (\sum_{t=1}^{T}\log p_{\theta} (\hat{y}_t^{E_j} | \hat{y}_{<t}, x)) \nonumber
% \end{align}

% where $\hat{y_t^{E_j}}$ denotes predicted logits obtained from accelerated residual stream at gate $E_j$ and time-step $t$ while $y_t^{E_j}$ denotes corresponding truth tokens. 

% Upon training of adapters responsible for accelerating residual streams, we train query, key, value parameters responsible for vertical latent attention of all gates in parallel as

% \begin{equation} \label{eq:arla_loss}
%     L_{arla} = -\frac{1}{N} (\sum_{t=1}^{T}(1\sum_{j=1}^{E_n} \left[ O_t^{E_j} \log(\hat{O}_t^{E_j}) + (1 - o_t^{E_j}) \log(1 - \hat{o_t}_{E_j}) \right]))
% \end{equation}

% where $\hat{O_t^{E_j}}$ denotes binary predicted logits obtained from vertical latent attention router described in \cref{sec:arla} at gate $E_j$ and timestep $t$ while $O_t^{E_j}$ denotes corresponding truth label. Truth labels for router are obtained by computing whether logit head application on $\hat{y}_t^j$ results in true next token prediction. Formally speaking, 

% $O_t^{E_j} = 1 if L(\hat{y_t^{E_j}}) == y_t^{E_j} , 0 otherwise$. 

% Parameters responsible for vertical latent attention are also trained in parallel as well. 

%todo: training slow and fast residuals together and distillation can be two training mdoes. 
%Distillation can be an ablation. 




% Although transformer decoding is memory bound on most mainstream accelerators, there could be scenarios where flop savings are crucial. For instance, on on-device settings power consumption is directly correlated with flops per decoding step and reducing flops does help with overall energy consumption. Vanilla early exiting methods help with flop reduction but suffer from mismatch between training and inference due to early exited tokens. If token at decoding step $t$, $T_t$ exited at layer $E_i$, while token $T_{t+k}$ exits at layer $E_j$ such that $E_i < E_j$, hidden state $H_{t+k}l$ does not have corresponding hidden state $H_tl$ to attend to where $E_i < l <= E_j$. One solution that's often used in literature is to rely on last hidden state available, $H_t{E_j}$, however it tends to be sub-optimal and does affect generation quality \cite{ref}.  To alleviate this mismatch while reducing flops, we train router such that attention mask between token $T_{t+k}$ and token $T_{<t+k}$ is given by: 

% \begin{equation}
%     a_{T_{{t+k}{T_{<t+k}}} = 1 if  E_{T_{<t+k}} >= E{T_{t+k}}
%     else 0
% \end{equation}

% This attention mask enables router to account for exited tokens and get trained accordingly. Since attention mechanism during decoding remains exactly same as that during training, impact on generation quality tends to be minimal as noted in \cref{fig:gen_auality_with_and_without_recompute_attention_show_flops}.  Although MoD does not suffer from training and inference mismatch, we observe that it suffers from discountinuity between pre-training and super-vised fine-tuning resulting in sub-optimal perplexity. On the other hand, our method doesn't not require pre-training , doesn't suffer from discountinuity, and achieves much better perplexity in super-vised fine-tuning and instruction tuning setups as shown in \cref{fig:Mod_vs_m2r2_loss_curves}.






% Our techniques are directly applicable in such scenarios.    




%expert loading with cuda streams in experiments
% \section{Experiments: Planning outperforms Heuristics}
\label{sec:experiment}

We begin our empirical demonstrations by showcasing the effectiveness of our planning framework on both synthetic and real datasets. We focus on the simplest planning algorithm, 1-step lookaheads (Algorithm~\ref{alg:complete}), and show that even basic planning can hold great promise. 
We illustrate our framework using two uncertainty quantification modules---GPs and 
\ensembles/ \ensembleplus. 

Throughout this section, we focus on evaluating the mean squared error of 
a regression model $\model$,  and develop adaptive policies that minimize uncertainty on $g(f)$ defined in~\eqref{eqn:l2-g-f}.
When GPs provide a valid model of uncertainty, 
our experiments show that our planning framework significantly outperforms other baselines. 
We further demonstrate that our conceptual framework extends to deep learning-based uncertainty quantification methods such as  \ensembleplus while highlighting computational challenges that need to be resolved in order to scale our ideas. 
For simplicity, we assume a naive predictor, i.e., $\psi(\cdot) \equiv 0$. However, we emphasize that this problem is just as complex as if we were using a sophisticated model $\psi(.)$. The performance gap between the algorithms 
primarily depends
on the level  of uncertainty in our prior beliefs.

To evaluate the performance of our algorithm, we benchmark it against several baselines. 
%Active learning baselines use an acquisition function $\ac$ to select points that have the highest   function value: $X\opt_t \in \argmax_{X \in \xpoolj{t}} \ac({X})$ at every step $t$. These methods may also need an UQ module, which we simply use the same UQ module as in our algorithm, and it  outputs $V(X)$ that measures the the uncertainty of each point $X \in \xpoolj{t}$.
Our first set of baselines are from active learning~\citep{AggarwalKoGuHaPh14}:
\\ % \noindent\textbf{Active Learning Heuristics:} 
\textbf{(1)} 
\textsf{Uncertainty Sampling (Static):}  In this approach, we query the samples for which the model is least certain about. Specifically, we estimate the variance of the latent output $f(X)$ for each $X \in \xpool$ using the UQ module and select the top-$K$ points with the highest uncertainty. \\
\textbf{(2)} \textsf{Uncertainty Sampling (Sequential):} This is a greedy heuristic that sequentially selects the points with the highest uncertainty within a batch, while updating the posterior beliefs using pseudo labels from the current posterior state. Unlike \textsf{Uncertainty Sampling (Static)}, this method takes into account the information gained from each point within batch, and hence tries to diversify the selected points within a batch. 

 
We also compare our approach to the  \textbf{(3)} \textsf{Random Sampling}, which selects each batch uniformly at random from the pool. Additionally, we compare solving the planning problem using  \textsf{REINFORCE}-based policy gradients with   $\mathsf{Smoothed\text{-}Autodiff}$ policy gradients.\footnote{Our code repository is available at
  \url{https://github.com/namkoong-lab/adaptive-labeling}.}
%Detailed experimental setups are provided in Section \ref{sec:details-experiments}.

%We repeat all experiments with 10 random seeds.




\begin{figure}[t]
\centering
\begin{minipage}[b]{0.49\textwidth}
\centering
\includegraphics[width=\textwidth, height=5cm]{figures/original_scale/Var_of_l_2_loss.pdf}
\caption{(Synthetic data) Variance of mean squared loss evaluated through the posterior belief $\mu_t$ at each horizon $t$. This is the objective that policy gradient methods like \textsf{REINFORCE} and $\ouralgo$ optimizes. 1-step lookaheads are surprisingly effective even in long horizons.}
\label{fig:var-l2-sim}
\end{minipage}
\hfill
\begin{minipage}[b]{0.49\textwidth}
\centering \includegraphics[width=\textwidth, height=5cm]{figures/original_scale/Error_of_estimated_model_l_2_loss.pdf}
\caption{(Synthetic data) Error between MSE calculated based on collected data $\mc{D}^{0:T}$ vs. population oracle MSE over $\mc{D}_{\rm eval} \sim P_X$. Reducing uncertainty over posteriors directly leads to better OOD evaluations. 1-step lookaheads significantly outperform active learning heuristics in small horizons.}
\label{fig:mean-l2-sim}
\end{minipage}
%\caption{Simulated data for GPs}
%\label{fig:both_plots}
\end{figure}

\subsection{Planning with Gaussian processes}
\label{sec:experiment-plan-GP}
We now briefly describe the data generation process for the GP experiments,  deferring a more detailed discussion of the dataset generation to Section~\ref{sec:details-experiments}. 
We use both the synthetic data and the real data to test our methodology.
For the \emph{simulated data},  we construct a setting where the general population is distributed across \emph{51 non-overlapping clusters} while the initial labeled data $\dtrain$ just comes from one cluster. In contrast, both $\dpool \defeq (\xpool,\ypool),\deval \defeq (\xeval,\yeval)$ are generated   from all the clusters. 
We begin with a low-dimensional scenario, generating a one-dimensional regression setting using a GP. %Gaussian Process (GP).
Although the data-generating process is not known to the algorithms,  we assume that the GP hyperparameters are known to all the algorithms
to ensure fair comparisons. This can be viewed as a setting where our prior is well-specified, allowing us to isolate the effects
of different policy optimization approaches
 without any concerns about the misspecified priors. We select $10$ batches, each of size $K=5$ across $T = 10$ time horizons.

To examine the robustness of our method against the distributional assumptions made  in the simulated case, we then move to a real dataset where the correct prior is not known. We simulate selection bias from the eICU dataset~\citep{PollardJoRaCeMaBa18}, which contains real-world patient data with in-hospital mortality outcomes. 
We conduct a $k$-means clustering to generate 51 clusters and then select data from those clusters. We view this to be a credible replication of practice, as severe distribution shifts are common due to selection bias in clinical labels.  To convert the binary mortality labels into a regression setting, we train a  random forest classifier and fit a GP on predicted scores, which serves as the UQ module for all the algorithms. As before, the task is to select 10 batches, each consisting of 5 samples, across 10 time horizons.

 In Figures~\ref{fig:var-l2-sim} and~\ref{fig:mean-l2-sim}, we present results for the simulated data. 
Figure~\ref{fig:var-l2-sim} shows the variance of $\ell_2$ loss, and Figure~\ref{fig:mean-l2-sim} presents the error in the estimated $\ell_2$ loss using $\mu_t$ (relative to true $\ell_2$ loss, that is unknown to the algorithm). 
As we can see from these plots, our method one-step lookahead  gives substantial improvements  over active learning baselines and random sampling. In addition,
compared to the one-step lookahead planning approach using \textsf{REINFORCE}-based policy gradients, 
we observe that $\mathsf{Smoothed\text{-}Autodiff}$-based policy gradients provide significantly more robust performance over all horizons.

In Figures~\ref{fig:var-l2-real}~and~\ref{fig:mean-l2-real}, we observe similar findings on the eICU data. We see that planning policies (\textsf{REINFORCE} and $\mathsf{Smoothed\text{-}Autodiff}$) consistently outperform other heuristics by a large margin.  Active learning baselines perform poorly in these small-horizon batched problems and can sometimes be even worse than the random search baselines.  Overall, our results show the importance of careful planning in adaptive labeling for reliable model evaluation. 

We offer some intuition as to why one-step lookahead planning may outperform other heuristic algorithms. 
 First,  \textsf{Uncertainty sampling (Static)} while myopically selects the
 top-$K$ inputs with the highest uncertainty, it fails to consider 
the overlap in information content among the ``best” instances; see \citep{AggarwalKoGuHaPh14} for more details. 
In other words,  it might acquire points from the same region with high uncertainty while failing to induce diversity among the batch.
Although \textsf{Uncertainty Sampling (Sequential)} somewhat addresses the issue of information overlap, a significant drawback of 
this algorithm
is the disconnect between the objective we aim to optimize and the algorithm. For example, it might sample from a region with high uncertainty but very low density. 

\begin{figure}[t]
\centering
\begin{minipage}[b]{0.48\textwidth}
\centering
\includegraphics[width=\textwidth, height=5cm]{figures/original_scale/Var_of_l_2_loss_real.pdf}
\caption{(Real-world eICU data) Variance of mean squared loss evaluated through the posterior belief $\mu_t$ at each horizon $t$. Even 1-step lookaheads are extremely effective planners, and auto-differentiation-based pathwise policy gradients provide a reliable optimization algorithm based on low-variance gradient estimates.}
\label{fig:var-l2-real}
\end{minipage}
\hfill
\begin{minipage}[b]{0.48\textwidth}
\centering \includegraphics[width=\textwidth, height=5cm]{figures/original_scale/Error_of_estimated_model_l_2_loss_real.pdf}
\caption{(Real-world eICU data) Error between MSE calculated based on collected data $\mc{D}^{0:T}$ vs. population oracle MSE over $\mc{D}_{\rm eval} \sim P_X$. Reducing uncertainty over posteriors directly leads to better OOD evaluations. Our method significantly outperforms active learning-based heuristics, and random sampling.}
\label{fig:mean-l2-real}
\end{minipage}
%\caption{Real data for GPs}
\end{figure}
 
%\vspace{-1.5cm}
% \begin{wrapfigure}{r}{.32\columnwidth}
%   \vspace{-.5cm} 
%   \centering
% \includegraphics[scale=.29]{figures/Var of l2l_2 loss.pdf}
%   \vspace{-0.2cm}
%   \caption{Results of GP}
% \label{fig:var-l2-gp}
%   \vspace{-0.1cm}
% \end{wrapfigure}


% Attempts have been made  in the past to address these  drawbacks heuristically  (see \citep{AggarwalKoGuHaPh14}). We give a unified computational framework while approaching the problem in a more principled manner and solving it more optimally.




\subsection{Planning with  neural network-based uncertainty quantification methods ($\ensembleplus$)}


We now provide a proof-of-concept that shows the generalizability of our conceptual framework  to the deep learning-based UQ modules, specifically focusing on $\ensembleplus$ due to their previously observed superior performance~\citep{OsbandWenAsDwIbLuRo23}. Recall that implementing our framework with deep learning-based UQ modules  requires us to retrain the model across multiple possible random actions $\bm{a}(\theta)$ sampled from the current policy $\pi_\theta$.
This requires significant computational resources, in sharp contrast to the GPs where the posteriors are in closed form and can be readily updated and differentiated. 

Due to the computational constraints, we test $\ensembleplus$ on a toy setting to demonstrate the generalizability of our framework. We consider a setting where the general population consists of four clusters, while the initial labeled data only comes from one cluster. Again we generate data using GPs.  The task is to select a batch of 2 points in one horizon. We detail the $\ensembleplus$ architecture in Section \ref{sec:details-experiments}, and we assume prior uncertainty to be large (depends on the scaling of the prior generating functions). 
The results are summarized in the Table~\ref{tab:UQ_ensemble}.

% \begin{table}[H]
% \vspace{-10pt}
% \caption{Performance under \ensembleplus as UQ module}
%     \centering
%     \begin{tabular}{|m{3cm}|m{2.5cm}|m{2cm}|} 
%     \hline
%       Algorithm   & Variance of $\loss_2$ loss estimate & Error of $\loss_2$ loss estimate  \\ \hline Random Sampling 
%          & $1710.9 \pm 1352.1$ & $8.67\pm6.62$ 
%       \\ \hline \ouralgo & $1.30 \pm 0.68$ & $0.91\pm0.25$ \\ \hline
%     \end{tabular}
%     \label{tab:UQ_ensemble}
%     %\vspace{-10pt}
% \end{table}




\begin{table}[h]
\vspace{-10pt}
\caption{Performance under \ensembleplus as the UQ module}
\centering
\begin{tabular}{|l|l|l|}
\hline
Algorithm   & Variance of $\loss_2$ loss estimate & Error of $\loss_2$ loss estimate  \\
\hline
\textsf{Random sampling} & 7129.8 $\pm$ 1027.0 & 136.2 $\pm$ 8.28 \\ \hline
\textsf{Uncertainty sampling (Static)} & 10852 $\pm$ 0.0 & 162.156 $\pm$ 0.0 \\ \hline
\textsf{Uncertainty sampling (Sequential)} & 8585.5 $\pm$ 898.9 & 144 $\pm$ 6.93 \\ \hline
\textsf{REINFORCE} & 1697.1 $\pm$ 0.0 & 45.27 $\pm$ 0.0 \\ \hline
\ouralgo & 1697.1 $\pm$ 0.0 & 45.27 $\pm$ 0.0 \\ \hline
\end{tabular}
%\caption{Comparison of different algorithms based on variance   and   error in $\ell_2$ loss estimation with Ensemble $+$ as the UQ module. Our results demonstrate that {\ouralgo} and REINFORCE outperformthe other active learning based heuristics, confirming the benefits of our MDP formulation for the adaptive labeling problem, as also demonstrated in Section 4.\\
%\footnotesize{Experimental details: We use Gaussian Processes as our data generating process, GP parameters are the same as in Section D.3.  The task is to select a batch of 2 points along one horizon.The marginal distribution $p_X$ has 4 \textit{non-overlapping} clusters. Initial data comes from one cluster, while pool and evaluation points comes from all the clusters. We have $20$ initial labeled data points, $10$ pool points, and $252$ evaluation points.  Training procedures are similar to the one in Section D.3.} }
\label{tab:UQ_ensemble}
\end{table}



% We faced  issues in scaling up these experiments which will be our focus in the future. 





% \begin{itemize}
%     \item Posteriors should be consistent. Two dimensions: even with less training,  
%     \item the inference should be  fast enough
% \end{itemize}


% Potential research directions for uncertainty quantification

% In this section we consider a simple setting We consider a simpler setting and 


% For synthetic dataset generation, we use ...... For real datasets, we use ...... We compare our methodolgy to several baselines ()    This Section is structured as follows:
% \begin{itemize}
%     \item \textbf{GPs, square loss objective} (Section \ref{}): 
%     %the broad aim of the experiments  in this section is to isolate the performance of our methodology without any concerns for the inefficiencies induced due to a mis-specified prior or imperfect posterior inference. To accomplish this we generate synthetic datasets using GPs (detailed later). We use the well specified prior (GPs - with same hyperparameter setting) as our UQ module.   
%      As GPs provide differentaible posterior inference - any errors induced due to imperfect posterior updates are also isolated. We note that under this setting
%      \item In Section\ref{} we demonstrate why our methodology performs better than other baselines - by devising various synthetic experiments ()
%     \item  \textbf{UQ Benchmarking }(Section \ref{}): Before diving into the experiments using $\ensembleplus$ and ENNs,  we showcase our benchmarking experiments in Section \ref{}. We use real datasets We observe that ENNs perform better
%      \item \textbf{Ensemble $+$}, objective: recall, accuracy
%     \item \textbf{ENN}, objective: recall, accuracy
% \end{itemize}




% In Section {}, we test 
% \subsection{Experimental details}

% \begin{itemize}
%     \item UQ methodologies - GPs, ENNs
%     \item Objectives - Recall,  ATE
%     \item Datasets - ATE-synthetic datasets, Recall-synthetic, real datasets
%     \item Baselines - 
%     \begin{itemize}
%         \item Random sampling
%         \item Active learning - Uncertainty based sampling - In regression setting almost all of the 
%         \item Myopic greedy - Greedy Batch based sampling
%         \item Policy Gradient
%     \end{itemize}
    
% \end{itemize}

% \subsection{Experiments}
%     \begin{itemize}
%     \item GPs with square loss
%     \item Benchmarking ENN
%         \item ENNs with ATE
%         \item ENNs with Recall
%     \end{itemize}

% \subsection{Benefits over other algorithms - intuition and experiments}

%Active learning - Myopic greedy / Don't rely on the objective rather some entropy version.


%%% Local Variables:
%%% mode: latex
%%% TeX-master: "main"
%%% End:

% \section*{Conclusion}
This paper aims to enhance our understanding of the computational complexity of computing various Shapley value variants. We found that for various ML models --- including decision trees, regression tree ensembles, weighted automata, and linear regression --- both local and global interventional and baseline SHAP can be computed in polynomial time under HMM modeled distributions. This extends popular algorithms, such as TreeSHAP, beyond their empirical distributional scope. We also establish strict complexity gaps between the various SHAP variants (baseline, interventional, and conditional) and prove the intractability of computing SHAP for tree ensembles and neural networks in simplified scenarios. Overall, we present SHAP as a versatile framework whose complexity depends on four key factors: \begin{inparaenum}[(i)] \item model type, \item SHAP variant, \item distribution modeling approach, \item and local vs. global explanations\end{inparaenum}. We believe this perspective provides deeper insight into the computational complexity of SHAP, paving the way for future work.




%We believe that our framework provides a more intricate understanding of SHAP computation complexity across different models, distributions, and variants, paving the way for further research.

Our work opens promising directions for future research. First, expanding our computational analysis to other SHAP-related metrics, such as asymmetric SHAP~\citep{frye20} and SAGE~\citep{covert2020understanding}, would be valuable. Additionally, we aim to explore more expressive distribution classes and relaxed assumptions beyond those in Section \ref{sec:tractable} while maintaining tractable SHAP computation. Finally, when exact computation is intractable (Section \ref{sec:intractable}), investigating the approximability of SHAP metrics through approximation and parameterized complexity theory~\citep{downey2012parameterized} is an important direction.

%Our work opens several promising avenues for future research on the computational properties of explainable AI methods, with a particular focus on SHAP. First, it would be interesting to broaden the computational analysis conducted in this work to include other popular SHAP-related metrics in the literature, such as asymmetric SHAP \cite{frye20} and SAGE \cite{covert2020understanding}. Also, in the future, we aim to explore more expressive distribution classes and relaxed distributional assumptions—extending beyond those examined in Section \ref{sec:tractable} —that still yield tractable SHAP computation. Finally, when exact computation proves intractable (Section \ref{sec:intractable}), it is worthwhile to theoretically investigate the question of the approximability of computing the SHAP metrics across various configurations, through the lens of approximation and parametrized complexity theory \cite{arora2009computational}.

%This paper aims to deepen our understanding of the computational complexity involved in obtaining different Shapley value variants. We found that for a variety of ML models, including decision trees, tree ensembles for regression, weighted automata, and linear regression models — computing both local and global interventional and baseline SHAP can be done in polynomial time when distributions are modeled by HMMs. This extends the distributional scope of popular algorithms like TreeSHAP, which is limited to empirical distributions. Additionally, we demonstrate a strict complexity gap between SHAP variants, showing that interventional and baseline SHAP can be strictly easier to compute than conditional SHAP. Despite these positive results, we uncovered intractability for various SHAP variants in neural networks and tree ensembles. Finally, we provided generalized complexity relations across SHAP variants. We believe that our framework offers a deeper understanding of the complexity involved in computing SHAP across various variants, models, distributions, as well as in both local and global computations, laying the groundwork for future research.
\section{Introduction}

Multivariate time series are utilized in various real-world applications, particularly in the medical field, where they are used to record vital signs and laboratory test results for diagnosis \cite{chaudhary2020utilization,brizzi2022spatial}. Typically, these time series are irregular, faced with asynchronicity across sensors and nonuniform sampling in the time domain \cite{chowdhury2023primenet,huang2024dna}. Moreover, significant missing values are usually present in clinical data collection. For example, random missingness can result from patients joining or leaving treatments midway, or complete absence of data from a sensor when specific tests are not conducted \cite{de2019deep}. Some public clinical datasets, such as PhysioNet2012, take even a 80\% missing rate, posing challenges for data analysis and clinical decision-making \cite{wang2024deep}. 

Deep learning methods have been widely adopted to model irregular time series. Some methods rely on the assumption of time discretization, utilizing LSTMs \cite{neil2016phased,weerakody2023policy}, RNNs \cite{che2018recurrent,ma2020adversarial,miao2021generative}, and Transformers \cite{horn2020set,huang2024dna} to capture characteristics of discrete sequences. Nonetheless, these methods often face difficulties in accumulating errors from missing observations \cite{ma2019learning}. Recently, vision models have also shown promising potential in handling irregular sequence data \cite{li2024time}. By transforming series into corresponding RGB representations, visual frameworks can effectively capture dynamic trends and inter-sensor relationships within images \cite{ maroor2024image,li2024time}. However, such designs perform poorly with sparse series that exhibit heavy missing rate \cite{li2024time}. 

We recognize that no one has yet integrated both sequence and image representations in handling irregular medical time series. This introduces a pivotal question: \textit{How can we effectively merge these two distinct representations to improve the robustness of classification for irregular medical time series with extensive missing values?} 

To investigate this question, we utilize a joint learning framework that incorporates both sequence and image representations. Additionally, we propose different self-supervised learning (SSL) strategies to enhance the integration and capture of supplementary information across these two representations. Specifically, our approach consists of three main components, as shown in Figure \ref{framework}. For the sequence modeling branch, we employ a generator-discriminator structure and adopt an adversarial strategy \cite{ma2019learning,miao2021generative} for sequence imputation task to minimize the propagation of cumulative errors. In the image branch, we implement different image transformation strategies to improve the performance on sparse series, and utilize a pre-trained Swin Transformer \cite{liu2021swinv2,li2024time} to obtain the corresponding image representations. Three different SSL losses are designed: (1) an inter-sequence contrastive loss to stabilize the sequence imputation process; (2) a sequence-image contrastive loss with margin to learn a more generalizable joint representation for downstream classification; and (3) a clustering loss on joint representations to push similar cases closer across different batches. 


\begin{figure*}[htp]
    \centering
    \includegraphics[width=0.91\linewidth]{figure/framework.png}
    \caption{The framework of our approach.}
    \label{framework}
\end{figure*}

We conduct experiments on three real-world clinical datasets: PAM \cite{reiss2012introducing}, P12 \cite{goldberger2000physiobank}, and P19 \cite{reyna2020early}. Table \ref{dataset_statistics} presents their statistics, which show that all three datasets experience severe missing values. We compare our approach with seven other state-of-the-art (SOTA) methods in terms of classification performance. Specifically, our approach achieves the best performance across all three datasets. For the PAM dataset, we observe improvements of 3.1\% in Accuracy, 2.9\% in Precision, 2.3\% in Recall, and 2.6\% in F1 score compared to the second-best method. For the P12 and P19 datasets, we use AUPRC and AUROC as evaluation metrics. Our approach surpasses prior SOTA by 1.1\% (AUPRC) and 0.9\% (AUROC) on P12, and 5.8\% (AUPRC) and 2.3\% (AUROC) on P19. Furthermore, we test further missingness through leave-samples-out and leave-sensors-out experiments on the PAM dataset. In the most severe scenario, with an additional 50\% missing values, our approach demonstrates better robustness, outperforming the second-best method by 6.1\% in Accuracy, 5.9\% in Precision, 3.4\% in Recall, and 4.6\% in F1 score.


The contributions of this paper are summarized as follows:
\begin{itemize} 
\item We propose a joint representation learning framework for multivariate irregular medical time series. To the best of our knowledge, this is the first approach to incorporate both sequence and image modeling.
\item We outline three SSL strategies: inter-sequence contrastive loss, sequence-image contrastive loss, and clustering-based loss. These strategies together enable better integration of sequence and image representations, enhancing the robustness against heavy missingness.
\item Our approach outperforms seven other SOTA methods on three real-world clinical datasets. We also simulates two classic types of missingness and experiments show that our method offers better robustness in handling these cases.
\end{itemize}

\section{Related Work}
\subsection{Irregular Time Series Methods}

Early practices for modeling irregular time series with missing values typically relied on fixed-time discretization. In this context, \cite{choi2016doctor} ignores the timestamp information by treating all intervals as equal, \cite{lipton2016modeling} considers missing data as an effective feature for learning, 
% \cite{futoma2017learning} employs Gaussian processes to model missing data and high uncertainty in real-world situations, 
and \cite{harutyunyan2019multitask} segments the data into evenly spaced time intervals. In contrast, GRU-D \cite{che2018recurrent} employs a gated network and incorporates imputation of missing values into the optimization process. Unlike previous methods, it adopts an additional missing value mask and lag matrix as inputs. Similar strategy have been adopted in \cite{,ma2019learning,ma2020adversarial,miao2021generative}, where adversarial frameworks are utilized to enhance the prediction of imputed values. 

Some recent approaches have leveraged attention mechanisms to improve modeling. For instance, SeFT \cite{horn2020set} introduces a set of differentiable set functions and uses attention mechanisms to aggregate embeddings of different variables. ContiFormer \cite{chen2024contiformer}, on the other hand, combines neural ordinary differential equations (ODEs) with attention mechanisms based on continuous-time dynamics, extending the relationship modeling capabilities of Transformers to the continuous time domain. Besides, DNA-T \cite{huang2024dna} utilizes a deformable attention mechanism to dynamically adjust the receptive field, enabling more effective handling of local features and short-term correlations. Warpformer \cite{zhang2023warpformer} also considers multi-scale features by applying a warping module to achieve multi-grained representations. Unlike previous methods that adopt a sequence modeling perspective, ViTST \cite{li2024time} transforms the signals into RGB images and utilizes a pre-trained Swin Transformer for further classification and regression.

% There are still methods that fall outside the aforementioned categories. One work worth reviewing is Raindrop \cite{zhang2021graph}, which models times series from the perspective of graph neural networks. In this approach, each observation resembles a raindrop hitting a sensor graph, spreading information through a ripple effect. 

% On the other hand, diffusion models \cite{han2022card} are also a growing trend and have been explored in various fields such as energy \cite{xu2024denoising}, finance \cite{daiya2024diffstock}, and microbiology \cite{seki2023imputing}.

\subsection{Modeling Time Series as Images}
Transforming time series data into images has gained significant attention with the advancements in visual detection frameworks. Some approaches \cite{sood2021visual,sangha2022automated,ao2023image, semenoglou2023image, maroor2024image} plot time series directly as time-observation representations and utilize convolutional neural networks (CNNs) for downstream tasks. Generally, they do not apply special processing to the sequences, instead focusing on leveraging visual frameworks to better capture temporal patterns in visualized sequences. ViTST \cite{li2024time} is another similar case that extends further to multivariate sequences and discusses the impact of visualization parameters such as color, markers, and order. 

In contrast, other methods emphasize the modeling of time series, which requires more specialized design and expert knowledge. \cite{tripathy2018use} utilizes an iterative filtering (IF) approach to produce different intrinsic mode functions (IMFs) from EEG signals. Empirically, these transformed features often fit the task better than the original signals. Chong et al. \cite{chong2011signal} and Deng et al. \cite{bs2023_1730} model sequences based on time segmentation, calculating time-invariant features and transforming them into corresponding RGB images. Similarly, frequency domain modeling, as demonstrated by TimesNet \cite{wu2023timesnet}, has also proven effective. By utilizing fast Fourier transform (FFT) to concatenate signal of different time periods, it constructs a 2D representation optimized for CNNs. Finally, other methods model the relative relationships between points in a time series. Examples include Gramian Angular Field (GAF), Markov Transition Field (MTF), and recurrence plot \cite{10.5555/2832747.2832798,hatami2018classification}. Typically, these methods involve applying a reversible time coordinate transformation and calculating the correlations between points, effectively capturing the continuity and periodic characteristics of the sequences. 


\section{Approach}

\subsection{Notations}

For a given clinical time series dataset \( D \), each sample \( X \in \mathbb{R}^{d \times T} \) represents a set of \( d \) records over a time \( T = \{t_1,...,t_n\} \), corresponding to a label \( y\). A binary mask \( M \in \mathbb{R}^{d \times T} \) is used to indicate the presence of missing observations in \( X \), where \( M_i^j = 0 \) signifies that the observation of the \( i^{th} \) item at time \( j \) is missing. 

To better handle consecutive missing values time, we follow \cite{miao2021generative,che2018recurrent} to obtain a time-lag matrix \( \delta \in \mathbb{R}^{d \times T} \) for each sample \( X \). This matrix quantifies the time elapsed since the most recent non-missing value for each observation, defined as follows.
\[
\delta^j_i = 
\begin{cases} 
0, &  i = 1 \\
t_i - t_{i-1}, &  m^j_{i-1} = 1 \text{ and } i > 1 \\
\delta^j_{i-1} + t_i - t_{i-1}, &  m^j_{i-1} = 0 \text{ and } i > 1
\end{cases}
\]

For each sample \( X \), the corresponding image \( I \) is constructed, where \( I \in \mathbb{R}^{3\times W \times H} \) represent a certain RGB format image. In total, we implement six transformed images as shown in Figure \ref{framework}. The specific transformation methods applied are as follows: Line Graphs, Frequency Spectrums, Gramian Angular Summation/Difference Fields, Markov Transition Fields, Recurrence Plots.

\subsection{The Model Overview}
In this section, we introduce the overall framework of our model, which comprises three main parts: (a) the sequence encoder, (b) the image encoder, and (c) the joint representation module. The sequence encoder consists of a generator-discriminator pair employing an adversarial strategy for imputation. The generator, \( G \), takes the time series \( X \), the mask \( M \), and the lag matrix \( \delta \) as inputs. Its objective is to estimate the missing values in \( X \) and generate a completed sequence \( X' \). This completed sequence \( X' \) is then used to obtain the sequence representation \( s \in \mathbb{R}^{d} \).  The discriminator D evaluates these estimations with the goal of distinguishing true observations from the imputed values. It outputs a binary matrix \( M' \), which identifies the regions of imputation predicted. For the image encoder, it takes a transformed image \(I \) as input and output the corresponding image representation \( v \in \mathbb{R}^{d}\). Finally, the joint representation module is responsible for mapping the sequence representation \( s \) and the image representation \( v \) into the same space. It then uses the final joint feature \( u \in \mathbb{R}^{d} \) for classification. 

\subsection{Sequence Branch with Imputation}

We adopt a modified bidirectional recurrent neural network (BiRNN) as our generator \(G\), which has been widely used in imputation tasks \cite{ che2018recurrent, ma2019learning, ma2020adversarial, miao2021generative, xu2024learning}. Taking the forward update step as an example, we update the current hidden state as:
\begin{align}
h_t &= \tanh\left(W_h (\gamma_t \odot h_{t-1}) + W_h'(\hat{x}_t + x_\delta )+ b_h\right) \\
\gamma_t &= \exp\left\{-\max(0, W_{\gamma} \delta_t + b_{\gamma})\right\}
\end{align}
In this setup, \(\gamma_t\) is derived from the lag matrix to model the dynamics of decay, where a longer duration of missing data leads \(\gamma_t\) closer to 0. It is applied to determine the extent to which the previous hidden state \(h_{t-1}\) should be retained. In the updating process of \( h_t \), instead of solely utilizing the previous reconstruction \(\hat{x}_t\) as done in prior works, we introduce an additional computation involving \( x_{\delta} \) as Eq. \ref{decay}. 
\begin{align}
\label{decay}
x_\delta = x_{t^-} \cdot \exp \left\{ -\max(0, W_\delta \delta_t) + b_\delta \right\}
\end{align}
This assumes the closest observation \( x_{t^-} \) prior to the current missing value influences the reconstruction process, with this influence decreasing as the time gap increases. 

Then, using a fully connected layer, the new reconstruction of the next step is obtained as: \(\hat{x}_{t+1} = W_{\hat{x}} h_t + b_{\hat{x}}\). And the overall imputed sequence \( X' \) is represented as: \(X' = M \odot X + (1 - M) \odot avg(\hat{X}_{for} + \hat{X}_{back})\), where we take the average of forward and backward result, and only the missing parts are replaced. Finally, the sequence representation \( s \) is obtained as: 
\begin{align}
s = Drop(W_s \cdot LayerNorm(X') + b_s)
\end{align}
In particular, \(W_h\), \(W_h'\), \(W_\gamma\), \(W_\delta\), \(W_{\hat{x}}\), \(W_s\), \(b_h\), \(b_x\), \(b_\gamma\), \(b_\delta\), \(b_{\hat{x}}\), and \(b_s\) are learnable parameters of the model and \( \odot \) denotes the element-wise multiplication. 

We formulate the objective of generator \(G\) into two components: adversarial loss and reconstruction loss. The adversarial loss is defined as the standard GAN's \cite{goodfellow2020generative}:
\begin{align}
\mathcal{L}_{adv} = \mathbb{E} [ (1 - M) \log (1 - D(X'))]
\end{align}
For the reconstruction loss, previous methods often use regression-based metrics such as mean square error (MSE) \cite{ma2020adversarial} or mean absolute error (MAE) \cite{ma2019learning} to assess the consistency between the missing and imputed sequences. However, when dealing with severely missing data, these strategies often fail to model the underlying data patterns, force the generator to learn nothing during the adversarial training phase. Inspired by \cite{raghu2023sequential}, we adopt a self-learning strategy to construct our reconstruction loss, and one choice is the normalized temperature-scaled cross-entropy loss (NT-Xent) \cite{chen2020big}. Given 2\(B\) pairs \((z_i, z_j)\) totally, it is computed as: 
\begin{align}
\label{NT}
\mathcal{L}_{NT} = \frac{1}{2B} \sum_{i=1}^{2B} -\log \frac{\exp(sim(z_i, z_j) / \tau)}{\sum_{k=1}^{2B} 1_{[k \neq i]} \exp(sim(z_i, z_k) / \tau)}
\end{align}
where cosine similarity is used as \(sim(z_i, z_j)\) and \(\tau\) is the temperature hyperparameter. We use NT-Xent to enforce consistency between the forward and backward predictions, as well as between the original and imputed sequences. Thus, the reconstruction loss is defined as: 
\begin{align}
\mathcal{L}_{rec} =  \mathcal{L}_{NT}(\hat{X}_{for}, \hat{X}_{back})\ + \mathcal{L}_{NT}(X, X')
\end{align}

We employ the same RNN in \cite{ma2019learning} as our discriminator \(D\), which takes \(X'\) as input and determines whether each observation is generated with a binary matrix \(M'\). Therefore, the discriminator is trained by minimizing:
\begin{align}
\label{dis}
\mathcal{L}_{dis} = \mathbb{E}[Mlog M' + (1-M)log(1-M')]
\end{align}


\subsection{Imaging Time Series}
We use a pre-trained Swin Transformer \cite{liu2021swinv2} as our image encoder. For the given image input \(I\), the Swin Transformer constructs a hierarchical representation to integrate both local and global information. Specifically, at earlier layers, it partitions the input into small patches and progressively merges neighboring patches as depth increases. It employs two types of attention mechanisms: window-based multi-head self-attention (W-MSA) and shifted window multi-head self-attention (SW-MSA). These mechanisms are respectively used to compute self-attention within a fixed window and to calculate dynamic relationships between windows. The vectors from the last stage after layer normalization are used as our image representation \( v \in \mathbb{R}^{d}\).

Overall, we implement six types of images for representation learning and a detailed description is presented in Appendix A. 

\begin{itemize}[leftmargin=*]
\item \textbf{Line Graphs} are constructed as \cite{li2024time}, with each variable represented by a line image of uniform size. 
\item \textbf{Frequency Spectrums} are generated based on the Fourier transform, considering that frequency domain signals tend to be more robust in cases of extreme data missingness.
\item \textbf{Gramian Angular Fields} \cite{10.5555/2832747.2832798} transform time series into polar coordinates, constructing trigonometric sums/ differences between any two time points to represent temporal correlation. 
\item \textbf{Markov Transition Fields} record the Markov transition probabilities between any two time observations \cite{10.5555/2832747.2832798}. They are insensitive to the distribution of the time series and temporal step information, allowing them to effectively capture correlations between observations with substantial missing data.
\item \textbf{Recurrence Plots} \cite{hatami2018classification}, based on phase space reconstruction, transform time series data into trajectories within phase space and analyze their recurrences. They are designed to  capture the inherent repetitiveness and periodicity within the time series.
\end{itemize}

 
\subsection{Joint Representations Through Contrast and Clustering}

The joint representation module includes a transformation function \( f: s,v \to \mathbb{R}^D
 \), which projects and concatenates the sequence features \( s \) and image features \( v \) into a joint space \( R^D \), and the fused feature is obtained as \( u = [s,v] \). To ensure both the quality and consistency of the joint representation, we implement contrastive learning within each batch to maximize the mutual information between corresponding pairs. A simple choice is to use the NT-Xent in Eq. \ref{NT}, where only sequence and image features corresponding to the same sample are treated as positive pairs \cite{sangha2024biometric}. Through this approach, NT-Xent ensures that the similarity between representations from the same sample is higher than that of other pairs. However, it also misses opportunities to learn from a wider set of potential pairs \cite{li2022clustering}. 
 
 In this case, a step forward is to treat \(s\) and \(v\) from different samples within the different category as a special form of negative pairs, thereby enhancing the model's ability to distinguish inter-class differences. Specifically, we introduce an additional margin \(m\) for these special negative pairs, enforce the model to exert greater effort to distinguish them:

% \noindent\resizebox{\columnwidth}{!}{
% \begin{minipage}{0.6\textwidth} 
% \begin{align}
% -\frac{1}{B} \sum_{i=1}^{B} &\left[ \log \left(\frac{\exp((v_i \cdot s_i)/ \tau)}{\sum_{j \in P(i)} \exp((v_i \cdot s_j)/ \tau) + \sum_{j \notin P(i)} \exp((v_i \cdot s_j + m)/ \tau)}\right)\right. \nonumber \\
% & + \log \left(\frac{\exp((v_i \cdot s_i)/ \tau)}{\sum_{k \in P(i)} \exp((v_k \cdot s_i)/ \tau) + \sum_{k \notin P(i)} \exp((v_k \cdot s_i + m)/ \tau)}\right]
% \end{align}
% \label{cont}
% \end{minipage}
% }

\begin{equation}
\scalebox{0.93}{$
-\frac{1}{B} \sum_{i=1}^{B} \left[ \log \left( \frac{\exp((v_i \cdot s_i)/ \tau)}{\sum_{j \in P(i)} \exp((v_i \cdot s_j)/ \tau) + \sum_{j \notin P(i)} \exp((v_i \cdot s_j + m)/ \tau)} \right) \right. \nonumber$}
\end{equation}
\begin{equation}
\scalebox{0.93}{$
\left. + \log \left( \frac{\exp((v_i \cdot s_i)/ \tau)}{\sum_{k \in P(i)} \exp((v_k \cdot s_i)/ \tau) + \sum_{k \notin P(i)} \exp((v_k \cdot s_i + m)/ \tau)} \right] 
\right.$}
\label{cont}
\end{equation}






Here, for the \(i^{th}\)sample, \(P(i)\) represents the set of all sample index that are in the same category. 

In contrastive learning, the formation of positive and negative pairs is confined to each batch. However, this approach lacks control over the semantic relationships between samples across different batches. As a result, similar samples from separate batches may not receive similar representations. In this case, we incorporate clustering learning into the training process to push semantically similar samples together across batches.

Specifically, we applied the K-means algorithm to the fused feature \(u\). We begin with the assignment step: during each training epoch, we select a set of k (k $\ll$ N) representative features \([C_{u_1}, \ldots, C_{u_k}]\) as the cluster centers for that round. Each fused feature \(u_i\) is assigned to a set \(S_k\) with center \(C_{u_k}\) by minimizing the overall distance as defined in Eq. \ref{cluster}.

\begin{align}
\label{cluster}
\underset{S}{\mathrm{arg min}}  \sum_{j=1}^{k} \sum_{u_i \in S_j} \|u_i - C_{u_j}\|^2 
\end{align}


% \begin{align}
% \mathcal{L}_{\text{cluster}} = -\frac{1}{N} \sum_{i=1}^{N} \min_{j=1 \ldots k} \left( 1 - \cos(X_i, C_j) \right),
% \end{align}

We then use these cluster centers, \([C_{u_1}, \ldots, C_{u_k}]\), as contrastive loss reference targets to construct the clustering loss:
\begin{align}
\mathcal{L}_{cluster} = -\frac{1}{N} \sum_{i=1}^{N} \log \frac{\exp(\cos(u_i, C_{u_i})/\tau)}{\sum_{j=1}^{k} \exp(\cos(u_i, C_{u_j})/\tau)}
\end{align}
where a cluster center \(C_{u_k}\) and all elements within  set \(S_k\) are treated as positive pairs, and elements from different clusters are considered negative pairs. To ensure sufficient samples for optimizing clustering, we perform the update step at the end of each epoch: we iteratively update the cluster centers using Eq. \ref{update} and calculate new assignments with Eq. \ref{cluster}, until the total distance is less than a predefined threshold \(\tau_c\).

\begin{align}
\label{update}
C_k = \underset{u \in S_k}{\mathrm{arg min}} \sum_{u' \in S_k} \|u - u'\|^2 
\end{align}

\subsection{Overall Training Process}

The overall training process is divided into three steps as follows:

\begin{itemize}[leftmargin=*]
    \item Firstly, we fix the generator \( G \) and update the discriminator \( D \) based on Eq. \ref{dis} .
    \item Next, we update the parameters of \( G \) based on the new \( D \), with the objective function \(  \mathcal{L}_{adv} +  \alpha \mathcal{L}_{rec} \).
    \item Finally, we compute the forward pass of all three components, utilizing the joint feature \( u \) to perform classification. For the PAM dataset, we use the Cross Entropy Loss as the classification loss \(\mathcal{L}_{clf}\). For the more imbalanced P12 and P19 datasets, we opt for the Focal Loss. The final objective is expressed as \(\mathcal{L}_{clf} + \beta_1 \mathcal{L}_{cont} + \beta_2 \mathcal{L}_{cluster}\).
\end{itemize}


% \begin{algorithm}[H]
% \caption{Training Process}
% \label{alg:training}
% \begin{algorithmic}[1]
% \STATE \textbf{Initialize:} Generator $G$, Discriminator $D$, and all parameters.

% \STATE \textbf{Step 1: Update Discriminator}
% \STATE Fix $G$
% \FOR{each training step}
%     \STATE Update $D$ based on Eq. (8)
% \ENDFOR

% \STATE \textbf{Step 2: Update Generator}
% \STATE Fix $D$
% \FOR{each training step}
%     \STATE Update $G$ with the objective: $L_{adv} + \alpha L_{rec}$
% \ENDFOR

% \STATE \textbf{Step 3: Joint Forward Pass and Classification}
% \FOR{each forward pass}
%     \STATE Compute joint feature $u$
%     \STATE Perform classification using $u$
%     \IF{dataset is PAM}
%         \STATE Compute classification loss $L_{clf}$ using Cross Entropy Loss
%     \ELSIF{dataset is P12 or P19}
%         \STATE Compute classification loss $L_{clf}$ using Focal Loss
%     \ENDIF
% \ENDFOR

% \STATE \textbf{Final Objective:}
% \STATE $L_{final} = L_{clf} + \beta_1 L_{cont} + \beta_2 L_{cluster}$
% \end{algorithmic}
% \end{algorithm}


\section{Experiments}

\subsection{Datasets and Metrics}

\begin{table}[h]
% \setlength{\heavyrulewidth}{1.3pt}
\centering
% \resizebox{0.98\columnwidth}{!}{ 
\setlength{\tabcolsep}{0.8mm}
\begin{tabular}{@{}lccccr@{}} 
\toprule
\textbf{Dataset} & \textbf{Features} & \textbf{Time} & \textbf{Classes} & \textbf{Missing Ratio} & \textbf{Samples} \\ 
\midrule
PAM              & 17                & 600           & 8                & 60\%                   & 5,333             \\
P12              & 36                & 215           & 2                & 88.4\%                 & 11,988            \\
P19              & 34                & 60            & 2                & 94.9\%                 & 38,803            \\
\bottomrule
\end{tabular}
% }
\caption{Statistics of datasets utilized.}
\label{dataset_statistics}
\end{table}

\begin{table*}[htp]
\setlength{\heavyrulewidth}{1.3pt}
% \renewcommand{\arraystretch}{1}
\centering
\begin{tabular}{c|cccc|cc|cc}
\toprule
\multirow{2}{*}{Methods}  & \multicolumn{4}{c|}{PAM} & \multicolumn{2}{c|}{P12} & \multicolumn{2}{c}{P19}  \\ 
\cmidrule{2-9} 
 & Accuracy & Precision & Recall & F1 score & AUROC & AUPRC  & AUROC & AUPRC  \\
\midrule
GRU-D & 83.3 {$\pm$ \scriptsize 1.6}  & 84.6 {$\pm$ \scriptsize 1.2} & 85.2 {$\pm$ \scriptsize 1.6}  & 84.8 {$\pm$ \scriptsize 1.2} &  81.7 {$\pm$ \scriptsize 1.8} & 41.3 {$\pm$ \scriptsize 3.5}  & 83.6 {$\pm$ \scriptsize 2.1}  &  45.7 {$\pm$ \scriptsize 4.2}   \\
SeFT  & 63.3 {$\pm$ \scriptsize 2.2} & 66.7 {$\pm$ \scriptsize 2.4}  & 65.3 {$\pm$ \scriptsize 1.5} & 65.1 {$\pm$ \scriptsize 1.8} & 73.3 {$\pm$ \scriptsize 2.5} & 29.1 {$\pm$ \scriptsize 4.1} & 84.5 {$\pm$ \scriptsize 2.3} & 46.7 {$\pm$ \scriptsize 3.1}   \\
% SSGAN       &  &  &  &  &  &  &  &    \\
CARD      &71.9 {$\pm$ \scriptsize 2.9}  &75.5 {$\pm$ \scriptsize 2.8}  & 73.5 {$\pm$ \scriptsize 3.1} & 73.8 {$\pm$ \scriptsize 3.0} &71.4 {$\pm$ \scriptsize 0.9}  &26.1 {$\pm$ \scriptsize 1.2}  &80.7 {$\pm$ \scriptsize 1.0}  & 36.7 {$\pm$ \scriptsize 6.0}    \\
Raindrop       & 89.2 {$\pm$ \scriptsize 1.3} & 90.8 {$\pm$ \scriptsize 1.0}  &90.4 {$\pm$ \scriptsize 1.3}  &90.5 {$\pm$ \scriptsize 1.2} & 82.0 {$\pm$ \scriptsize 2.4} & 44.3 {$\pm$ \scriptsize 3.3}  & 82.7 {$\pm$ \scriptsize 3.9} &52.3 {$\pm$ \scriptsize 3.9}   \\
PrimeNet      &85.5 {$\pm$ \scriptsize 1.5}  & 87.8 {$\pm$ \scriptsize 1.2}  & 87.1 {$\pm$ \scriptsize 1.1} & 87.1 {$\pm$ \scriptsize 1.2} &\underline{85.1} {$\pm$ \scriptsize 0.8}  &\underline{49.3} {$\pm$ \scriptsize 1.9} &80.3 {$\pm$ \scriptsize 0.5}  &31.6 {$\pm$ \scriptsize 0.9}    \\
ContiFormer &66.6 {$\pm$ \scriptsize 1.8}   &68.6 {$\pm$ \scriptsize 1.7}   &69.7 {$\pm$ \scriptsize 1.5}   & 67.4 {$\pm$ \scriptsize 1.7}  & 72.1 {$\pm$ \scriptsize 0.4}  & 29.6 {$\pm$ \scriptsize 0.8}  & 80.7 {$\pm$ \scriptsize 0.3}  & 34.7 {$\pm$ \scriptsize 1.9}   \\
ViTST       &\underline{95.2} {$\pm$ \scriptsize 1.4}  &\underline{95.8} {$\pm$ \scriptsize 1.3}  & \underline{96.1} {$\pm$ \scriptsize 1.1} & \underline{95.9} {$\pm$ \scriptsize 1.2} & 84.2 {$\pm$ \scriptsize 1.1} & 43.2 {$\pm$ \scriptsize 2.4} &\underline{89.3} {$\pm$ \scriptsize 0.2}  &\underline{53.8} {$\pm$ \scriptsize 1.1}    \\
\midrule
Ours       &\textbf{98.3} {$\pm$ \scriptsize 0.3}  &\textbf{98.7} {$\pm$ \scriptsize 0.6}  & \textbf{98.4} {$\pm$ \scriptsize 1.0} & \textbf{98.5} {$\pm$ \scriptsize 0.7} &\textbf{86.0} {$\pm$ \scriptsize 0.3}  &\textbf{50.4} {$\pm$ \scriptsize 2.1} &\textbf{91.6} {$\pm$ \scriptsize 0.9} & \textbf{59.6} {$\pm$ \scriptsize 1.3}    \\
\bottomrule
\end{tabular}
\caption{Comparison with state-of-the-art baselines on irregularly sampled time series classification. We use \textbf{bold} to indicate the best results and \underline{underline} for the second best one.}
\label{baselines}
\end{table*}



In the experiments, we consider three real-world irregular clinical datasets as shown in Table \ref{dataset_statistics}. The physical activity monitoring (PAM) dataset \cite{reiss2012introducing} focuses on tracking human activities, containing data from eight person who performed nine different actions. This dataset comprises 5,333 samples and captures data from four types of sensors placed at three distinct body locations, encompassing a total of 17 observational variables. The P12 dataset \cite{goldberger2000physiobank} includes 11,988 patient samples from ICU stays, with 36 measurements each. The binary labels indicate the prognosis for each sample as either survival or not. Finally, the P19 dataset \cite{reyna2020early} contains data from 38,803 sepsis patients, each with 34 measurements, and a high missing rate of 94.9\%. Approximately 90\% of these patients died due to sepsis.

To maintain consistency across all experiments, we follow the same data partition as \cite{zhang2021graph, li2024time}, dividing the datasets into training, validation, and testing sets in an 8:1:1 ratio. For the PAM dataset, we use Accuracy, Precision, Recall, and F1 score as evaluation metrics. For the more imbalanced P12 and P19 datasets, we report the Area Under the ROC Curve (AUROC) and the Area Under the Precision-Recall Curve (AUPRC). For more experimental results that are not included in this section, we present them in Appendix D. 


\subsection{Implementation and Training}
We use Gated Recurrent Units (GRU) \cite{dey2017gate} in both our generator and discriminator. The generator has 4 layers, with the number of units fixed at 128. The discriminator is a 5-layer RNN and the number of units is set to $\{$128, 64, 16, 64, 128$\}$, respectively. A checkpoint pre-trained on ImageNet-21K dataset are utilized for our image encoder. The patch size and window size are 4 and 7. For the P12 and P19 datasets, all images are set to a size of 384 $\times$ 384 pixels. While for the PAM dataset, line graph and frequency spectrum are configured to 256$ \times$ 320, while all other images are set to 320 $\times$ 320. We use a 3-layer MLP as our joint projection, with the number of units set to $\{$1024, 512, 1024$\}$. 
% For hyperparameters, the temperature parameter \( \tau \) for both the reconstruction loss, contrastive loss, and clustering loss is set at 1.2. The margin \( m \), \( \alpha \), \( \beta_1 \) and \( \beta_2 \) is specified at 0.05, 4, 0.1, and 0.2, respectively. We discuss the selection of these hyperparameters in Appendix B. 

For the P12 and P19 datasets, the total epoch is set to 8 and we apply upsampling of the minority class to mitigate imbalance. For the PAM dataset, we set the total epoch to 40. The batch sizes used for training are 32 for P19 and P12, and 48 for PAM. For each dataset, we discuss the learning rate as well as more hyperparameter settings in Appendix B. All experiments are performed on a server with NVIDIA GeForce RTX 3090 24GB and PyTorch 2.4.0+cu124. 

\subsection{Results}

% SSGAN \cite{miao2021generative}
\subsubsection{Comparison with state-of-the-art methods.} We compare our approach against seven state-of-the-art methods for irregularly sampled time series, including GRU-D \cite{che2018recurrent}, SeFT \cite{horn2020set}, CARD \cite{han2022card}, Raindrop \cite{zhang2021graph}, PrimeNet \cite{chowdhury2023primenet}, ContiFormer \cite{chen2024contiformer}, and ViTST \cite{li2024time}. For each baseline, we introduce our implementation and hyperparameter settings in Appendix C. To ensure a fair evaluation, we average the performance of each method across five individual tests, using the same data splits and settings provided in \cite{li2024time}. 

Table \ref{baselines} presents the comparison results, highlighting that our approach outperforms the other seven state-of-the-art methods across all three datasets. Specifically, we achieve a significant improvement on the PAM datasets, with an increase of 3.1\% in Accuracy, 2.9\% in Precision, 2.3\% in recall, and 2.6\% in F1 score. For the P12 and P19 datasets, our approach shows improved performance in predicting minority classes, with an increase of 0.9\%, 2.3\% in absolute AUROC points, and 1.1\%, 5.8\%  in absolute AUPRC, respectively. 

% \textcolor{red}{We compare the number of parameters between the baseline methods and ours, as shown in Table 3. It can be seen that the largest model, Raindrop, has a parameter count of 150M and all methods belong to small or medium-sized models. Therefore, it will not result in significant resource consumption. Methods like GRU-D and Contiformer, which only perform sequence modeling, have relatively fewer parameters, whereas methods involving image representation, such as ViTST and ours, have much larger parameter counts.}

\subsubsection{Performance under increased missing rates.}

\begin{figure*}[htp]
    \centering
    \includegraphics[width=0.95\linewidth]{figure/missing_test.png}
    \caption{Performance under increased missingness: (a) leave-sensors-out and (b) leave-samples-out on the PAM dataset. Tests are conducted with 10\%-50\% extra missing values.} 
    \label{missing test}
\end{figure*}



To further validate the robustness of our approach, we conduct additional experiments to compare the performance under increased levels of missing rate. Given that the P12 and P19 datasets have already faced very high missing rates—88.4\% and 94.9\% respectively, we conduct all the tests on the PAM dataset, which originally has a missing rate of 60\%. We conducted two types of tests: the leave-sensors-out setting, simulating scenarios where certain medical tests are not performed, and the leave-samples-out setting, reflecting situations where patients join or leave treatments midway. We follow the approach in \cite{zhang2021graph}, applying all modifications only to the test set by randomly masking the original observations.

As shown in Figure \ref{missing test}, our approach consistently achieves the best performance in all settings. For the leave-sensors-out tests, as the missing ratio increase from 10\% to 50\%, our approach exhibit the least performance decline. Even in the most extreme scenario, where 50\% of the sensors (9 sensors) are masked, all our metrics remain above 80\%. Compared to the second-best method, ViTST, our approach outperform by 6.1\%, 5.9\%, 3.4\%, and 4.6\% in Accuracy, Precision, Recall, and F1 score, respectively. The margins are even more significant compared to the third-ranked Raindrop, with improvements of 27.4\%, 39.8\%, 29.9\%, and 37.5\% in the same metrics. For the leave-samples-out setting, we randomly sampled and masked time steps. Overall, only CARD experienced significant decline as missing rate increases, while most models shows relatively minor decline, indicating that they effectively capture the temporal relationships between time steps. In terms of absolute performance, our model outperform the second-best, ViTST, by 5.7\% in Accuracy, 3.8\% in Precision, 5.4\% in Recall, and 4.8\% in F1 score at a 50\% missing rate. 



\subsubsection{Clinical Turing tests.}
To ensure that the learned representations align with clinically meaningful patterns rather than statistical artifacts, we conducted a clinical Turing test on the generated signals, as described in \cite{gillette2023medalcare}. Specifically, we select 60 samples from the P19 (ICU) dataset, with half imputed using linear interpolation as real measured samples and the other half imputed using our model as generated samples. Five ICU-experienced clinicians (3 chief physicians and 2 attending physicians) attempt to distinguish between the two types. As shown in Table \ref{expert}, the experts achieve prediction accuracy of 50.0\%, 48.3\%, 48.3\%, 58.3\%, and 60.0\%, resulting in a kappa score of -0.03. These results are close to random guessing, suggesting that the experts generally struggle to differentiate between the samples. A brief interview further revealed why experts struggled to identify clear patterns to distinguish real from generated samples. One reason is that the complex events in the ICU environment make the data distribution more tolerant. For example, sedation or anesthesia can cause body temperature to fall below the usual range. 


\begin{table}[t]
\setlength{\heavyrulewidth}{1.2pt}
% \renewcommand{\arraystretch}{1}
\setlength{\tabcolsep}{1mm}
\centering
% \resizebox{0.98\columnwidth}{!}{ 
\begin{tabular}{c ccccc}
\toprule
\multirow{2}{*}{Experts}  & \multicolumn{5}{c}{P19} \\ 
\cmidrule{2-6} 
 & Accuracy & Precision & Recall & F1 score & Specificity\\
\midrule
P1  &50.0  &47.4  &64.3  &54.5 &0.38 \\
P2  &48.3  &44.8  &46.4  &45.6 &0.50 \\
P3  &48.3  &45.5  &53.6  &49.2 &0.44 \\
P4  &58.3  &55.2  &57.1  &56.1 &0.59 \\
P5  &60.0  &57.1  &57.1  &57.1 &0.63 \\
\bottomrule
\end{tabular}
% }
\caption{Performance metrics of five ICU-experienced medical experts.}
\label{expert}
\end{table}


\subsubsection{Ablation study.}
In this section, we present the results of our ablation study in Table \ref{ablation}. The ``default'' one is our standard setup, which includes the sequence encoder, the image encoder, and the joint representation module, along with three self-supervised learning strategies. In the first part of Table \ref{ablation}, we evaluate the performance of individual components: ``image'' signifies that only the image encoder is used for classification, whereas ``sequence'' denotes the use of only the sequence encoder. As a result, we verify that incorporating both sequence and image information significantly improves classification performance, with F1 scores increasing by 2.6\% and 4.5\%. ``sequence-MSE'' denotes the use of MSE loss as the reconstruction loss. In contrast, by replacing it with NT-Xent, we achieved improvements in Accuracy, Precision, Recall, and F1 score by 0.8\%, 0.6\%, 0.2\%, and 0.5\%, respectively. 

The second part of Table \ref{ablation} focuses on our joint representation module. In the ``concatenation'' setting, we simply concatenate sequence and image representations for further downstream classification, and the performance is slightly higher than either ``sequence'' and ``image''. The ``contrastive'' setting shows the improvement from contrastive learning strategy, with 1.1\%, 0.9\%, 1.3\%, and 1.0\% in Accuracy, Precision, Recall, and F1 score. ``Clustering'' strategy also shows positive performance, with 1.2\% in Accuracy, 0.7\% in Precision, 0.9\% in Recall and 0.8\% in F1 score. 





\begin{table}[t]
\setlength{\heavyrulewidth}{1.2pt}
% \renewcommand{\arraystretch}{1}
\centering
\setlength{\tabcolsep}{0.8mm}
% \resizebox{0.98\columnwidth}{!}{ 
\begin{tabular}{c cccc}
\toprule
\multirow{2}{*}{Methods}  & \multicolumn{4}{c}{PAM} \\ 
\cmidrule{2-5} 
 & Accuracy & Precision & Recall & F1 score \\
\midrule
image  &95.4 {$\pm$ \scriptsize 0.6}  &96.5 {$\pm$ \scriptsize 0.6}   &95.4 {$\pm$ \scriptsize 0.4}  &95.9 {$\pm$ \scriptsize 0.5} \\
sequence &93.3 {$\pm$ \scriptsize 1.1}  & 94.4 {$\pm$ \scriptsize 0.7} &93.6 {$\pm$ \scriptsize 0.6}  & 94.0 {$\pm$ \scriptsize 0.7} \\
sequence-MSE &92.5 {$\pm$ \scriptsize 0.4}  &93.8 {$\pm$ \scriptsize 0.4}  & 93.4 {$\pm$ \scriptsize 0.4} & 93.5 {$\pm$ \scriptsize 0.4} \\
\midrule
concatenation  &95.7 {$\pm$ \scriptsize 0.7}  &96.7 {$\pm$ \scriptsize 0.5}  &96.1 {$\pm$ \scriptsize 0.4}  &96.5 {$\pm$ \scriptsize 0.5}  \\
contrastive  & 96.8 {$\pm$ \scriptsize 0.7} & 97.6 {$\pm$ \scriptsize 0.5} & 97.4 {$\pm$ \scriptsize 0.7} &97.5 {$\pm$ \scriptsize 0.5}  \\
clustering & 96.9 {$\pm$ \scriptsize 0.3} & 97.4 {$\pm$ \scriptsize 0.6} & 97.0 {$\pm$ \scriptsize 0.5}  &97.3 {$\pm$ \scriptsize 0.6}  \\
\midrule
default  &98.3 {$\pm$ \scriptsize 0.3}  &98.7 {$\pm$ \scriptsize 0.6}  & 98.4 {$\pm$ \scriptsize 1.0} & 98.5 {$\pm$ \scriptsize 0.7} \\
\bottomrule
\end{tabular}
% }
\caption{Ablation studies on different strategies.}
\label{ablation}
\end{table}

\section{Conclusion}

In this paper, we propose a joint learning approach of leveraging both sequence and image representations to tackle the classification of irregularly sampled clinical time series. By employing our three self-supervised learning strategies, we are able to effectively learn more generalized joint representations. The effectiveness of our approach is verified on three real-world clinical datasets, where it demonstrates superior performance compared to seven state-of-the-art methods. Additionally, we test our approach under more severe missing rates using leave-sensors-out and leave-samples-out techniques. Our approach consistently achieved strong results, demonstrating its robustness in these scenarios. Our code and data will be made publicly available later. 

\section*{Acknowledgements}
We express our sincere gratitude to all the anonymous reviewers for their valuable guidance and suggestions. We also thank Doctor Weihang Hu, Lin Zhang, and all the colleagues from the Intensive Care Unit at Zhejiang Hospital for their contributions to the expert evaluation and for providing us with valuable clinical advice.

\begin{thebibliography}{44}
    \providecommand{\natexlab}[1]{#1}
    
    \bibitem[{Ao and He(2023)}]{ao2023image}
    Ao, R.; and He, G. 2023.
    \newblock Image based deep learning in 12-lead ECG diagnosis.
    \newblock \emph{Frontiers in Artificial Intelligence}, 5: 1087370.
    
    \bibitem[{Brizzi et~al.(2022)Brizzi, Whittaker, Servo, Hawryluk, Prete~Jr, de~Souza, Aguiar, Araujo, Bastos, Blenkinsop et~al.}]{brizzi2022spatial}
    Brizzi, A.; Whittaker, C.; Servo, L.~M.; Hawryluk, I.; Prete~Jr, C.~A.; de~Souza, W.~M.; Aguiar, R.~S.; Araujo, L.~J.; Bastos, L.~S.; Blenkinsop, A.; et~al. 2022.
    \newblock Spatial and temporal fluctuations in COVID-19 fatality rates in Brazilian hospitals.
    \newblock \emph{Nature medicine}, 28(7): 1476--1485.
    
    \bibitem[{Chaudhary et~al.(2020)Chaudhary, Vaid, Duffy, Paranjpe, Jaladanki, Paranjpe, Johnson, Gokhale, Pattharanitima, Chauhan et~al.}]{chaudhary2020utilization}
    Chaudhary, K.; Vaid, A.; Duffy, {\'A}.; Paranjpe, I.; Jaladanki, S.; Paranjpe, M.; Johnson, K.; Gokhale, A.; Pattharanitima, P.; Chauhan, K.; et~al. 2020.
    \newblock Utilization of deep learning for subphenotype identification in sepsis-associated acute kidney injury.
    \newblock \emph{Clinical Journal of the American Society of Nephrology}, 15(11): 1557--1565.
    
    \bibitem[{Che et~al.(2018)Che, Purushotham, Cho, Sontag, and Liu}]{che2018recurrent}
    Che, Z.; Purushotham, S.; Cho, K.; Sontag, D.; and Liu, Y. 2018.
    \newblock Recurrent neural networks for multivariate time series with missing values.
    \newblock \emph{Scientific reports}, 8(1): 6085.
    
    \bibitem[{Chen et~al.(2020)Chen, Kornblith, Swersky, Norouzi, and Hinton}]{chen2020big}
    Chen, T.; Kornblith, S.; Swersky, K.; Norouzi, M.; and Hinton, G.~E. 2020.
    \newblock Big self-supervised models are strong semi-supervised learners.
    \newblock \emph{Advances in neural information processing systems}, 33: 22243--22255.
    
    \bibitem[{Chen et~al.(2024)Chen, Ren, Wang, Fang, Sun, and Li}]{chen2024contiformer}
    Chen, Y.; Ren, K.; Wang, Y.; Fang, Y.; Sun, W.; and Li, D. 2024.
    \newblock Contiformer: Continuous-time transformer for irregular time series modeling.
    \newblock \emph{Advances in Neural Information Processing Systems}, 36.
    
    \bibitem[{Choi et~al.(2016)Choi, Bahadori, Schuetz, Stewart, and Sun}]{choi2016doctor}
    Choi, E.; Bahadori, M.~T.; Schuetz, A.; Stewart, W.~F.; and Sun, J. 2016.
    \newblock Doctor ai: Predicting clinical events via recurrent neural networks.
    \newblock In \emph{Machine learning for healthcare conference}, 301--318. PMLR.
    
    \bibitem[{Chong et~al.(2011)}]{chong2011signal}
    Chong, U.-P.; et~al. 2011.
    \newblock Signal model-based fault detection and diagnosis for induction motors using features of vibration signal in two-dimension domain.
    \newblock \emph{Strojni{\v{s}}ki vestnik}, 57(9): 655--666.
    
    \bibitem[{Chowdhury et~al.(2023)Chowdhury, Li, Zhang, Hong, Gupta, and Shang}]{chowdhury2023primenet}
    Chowdhury, R.~R.; Li, J.; Zhang, X.; Hong, D.; Gupta, R.~K.; and Shang, J. 2023.
    \newblock Primenet: Pre-training for irregular multivariate time series.
    \newblock In \emph{Proceedings of the AAAI Conference on Artificial Intelligence}, volume~37, 7184--7192.
    
    \bibitem[{de~Jong et~al.(2019)de~Jong, Emon, Wu, Karki, Sood, Godard, Ahmad, Vrooman, Hofmann-Apitius, and Fr{\"o}hlich}]{de2019deep}
    de~Jong, J.; Emon, M.~A.; Wu, P.; Karki, R.; Sood, M.; Godard, P.; Ahmad, A.; Vrooman, H.; Hofmann-Apitius, M.; and Fr{\"o}hlich, H. 2019.
    \newblock Deep learning for clustering of multivariate clinical patient trajectories with missing values.
    \newblock \emph{GigaScience}, 8(11): giz134.
    
    \bibitem[{Deng et~al.(2023)Deng, Hua, Yingjun, Fanyue, and Di}]{bs2023_1730}
    Deng, Y.; Hua, M.; Yingjun, R.; Fanyue, Q.; and Di, P. 2023.
    \newblock An image characterisation method for AHU fault diagnosis based on residual neural networks.
    \newblock In \emph{Proceedings of Building Simulation 2023: 18th Conference of IBPSA}, volume~18 of \emph{Building Simulation}, 3827--3834. Shanghai, China: IBPSA.
    
    \bibitem[{Dey and Salem(2017)}]{dey2017gate}
    Dey, R.; and Salem, F.~M. 2017.
    \newblock Gate-variants of gated recurrent unit (GRU) neural networks.
    \newblock In \emph{2017 IEEE 60th international midwest symposium on circuits and systems (MWSCAS)}, 1597--1600. IEEE.
    
    \bibitem[{Gillette et~al.(2023)Gillette, Gsell, Nagel, Bender, Winkler, Williams, B{\"a}r, Sch{\"a}ffter, D{\"o}ssel, Plank et~al.}]{gillette2023medalcare}
    Gillette, K.; Gsell, M.~A.; Nagel, C.; Bender, J.; Winkler, B.; Williams, S.~E.; B{\"a}r, M.; Sch{\"a}ffter, T.; D{\"o}ssel, O.; Plank, G.; et~al. 2023.
    \newblock MedalCare-XL: 16,900 healthy and pathological synthetic 12 lead ECGs from electrophysiological simulations.
    \newblock \emph{Scientific Data}, 10(1): 531.
    
    \bibitem[{Goldberger et~al.(2000)Goldberger, Amaral, Glass, Hausdorff, Ivanov, Mark, Mietus, Moody, Peng, and Stanley}]{goldberger2000physiobank}
    Goldberger, A.~L.; Amaral, L.~A.; Glass, L.; Hausdorff, J.~M.; Ivanov, P.~C.; Mark, R.~G.; Mietus, J.~E.; Moody, G.~B.; Peng, C.-K.; and Stanley, H.~E. 2000.
    \newblock PhysioBank, PhysioToolkit, and PhysioNet: components of a new research resource for complex physiologic signals.
    \newblock \emph{circulation}, 101(23): e215--e220.
    
    \bibitem[{Goodfellow et~al.(2020)Goodfellow, Pouget-Abadie, Mirza, Xu, Warde-Farley, Ozair, Courville, and Bengio}]{goodfellow2020generative}
    Goodfellow, I.; Pouget-Abadie, J.; Mirza, M.; Xu, B.; Warde-Farley, D.; Ozair, S.; Courville, A.; and Bengio, Y. 2020.
    \newblock Generative adversarial networks.
    \newblock \emph{Communications of the ACM}, 63(11): 139--144.
    
    \bibitem[{Han, Zheng, and Zhou(2022)}]{han2022card}
    Han, X.; Zheng, H.; and Zhou, M. 2022.
    \newblock Card: Classification and regression diffusion models.
    \newblock \emph{Advances in Neural Information Processing Systems}, 35: 18100--18115.
    
    \bibitem[{Harutyunyan et~al.(2019)Harutyunyan, Khachatrian, Kale, Ver~Steeg, and Galstyan}]{harutyunyan2019multitask}
    Harutyunyan, H.; Khachatrian, H.; Kale, D.~C.; Ver~Steeg, G.; and Galstyan, A. 2019.
    \newblock Multitask learning and benchmarking with clinical time series data.
    \newblock \emph{Scientific data}, 6(1): 96.
    
    \bibitem[{Hatami, Gavet, and Debayle(2018)}]{hatami2018classification}
    Hatami, N.; Gavet, Y.; and Debayle, J. 2018.
    \newblock Classification of time-series images using deep convolutional neural networks.
    \newblock In \emph{Tenth international conference on machine vision (ICMV 2017)}, volume 10696, 242--249. SPIE.
    
    \bibitem[{Horn et~al.(2020)Horn, Moor, Bock, Rieck, and Borgwardt}]{horn2020set}
    Horn, M.; Moor, M.; Bock, C.; Rieck, B.; and Borgwardt, K. 2020.
    \newblock Set functions for time series.
    \newblock In \emph{International Conference on Machine Learning}, 4353--4363. PMLR.
    
    \bibitem[{Huang et~al.(2024)Huang, Yang, Yin, and Xu}]{huang2024dna}
    Huang, J.; Yang, B.; Yin, K.; and Xu, J. 2024.
    \newblock DNA-T: Deformable Neighborhood Attention Transformer for Irregular Medical Time Series.
    \newblock \emph{IEEE Journal of Biomedical and Health Informatics}.
    
    \bibitem[{Li, Torr, and Lukasiewicz(2022)}]{li2022clustering}
    Li, B.; Torr, P.~H.; and Lukasiewicz, T. 2022.
    \newblock Clustering generative adversarial networks for story visualization.
    \newblock In \emph{Proceedings of the 30th ACM International Conference on Multimedia}, 769--778.
    
    \bibitem[{Li, Li, and Yan(2024)}]{li2024time}
    Li, Z.; Li, S.; and Yan, X. 2024.
    \newblock Time series as images: Vision transformer for irregularly sampled time series.
    \newblock \emph{Advances in Neural Information Processing Systems}, 36.
    
    \bibitem[{Lipton et~al.(2016)Lipton, Kale, Wetzel et~al.}]{lipton2016modeling}
    Lipton, Z.~C.; Kale, D.~C.; Wetzel, R.; et~al. 2016.
    \newblock Modeling missing data in clinical time series with rnns.
    \newblock \emph{Machine Learning for Healthcare}, 56(56): 253--270.
    
    \bibitem[{Liu et~al.(2022)Liu, Hu, Lin, Yao, Xie, Wei, Ning, Cao, Zhang, Dong, Wei, and Guo}]{liu2021swinv2}
    Liu, Z.; Hu, H.; Lin, Y.; Yao, Z.; Xie, Z.; Wei, Y.; Ning, J.; Cao, Y.; Zhang, Z.; Dong, L.; Wei, F.; and Guo, B. 2022.
    \newblock Swin Transformer V2: Scaling Up Capacity and Resolution.
    \newblock In \emph{International Conference on Computer Vision and Pattern Recognition (CVPR)}.
    
    \bibitem[{Ma, Li, and Cottrell(2020)}]{ma2020adversarial}
    Ma, Q.; Li, S.; and Cottrell, G.~W. 2020.
    \newblock Adversarial joint-learning recurrent neural network for incomplete time series classification.
    \newblock \emph{IEEE Transactions on Pattern Analysis and Machine Intelligence}, 44(4): 1765--1776.
    
    \bibitem[{Ma et~al.(2019)Ma, Zheng, Li, and Cottrell}]{ma2019learning}
    Ma, Q.; Zheng, J.; Li, S.; and Cottrell, G.~W. 2019.
    \newblock Learning representations for time series clustering.
    \newblock \emph{Advances in neural information processing systems}, 32.
    
    \bibitem[{Maroor et~al.(2024)Maroor, Sahu, Nijhawan, Karthik, Shrivastav, and Chakravarthi}]{maroor2024image}
    Maroor, J.~P.; Sahu, D.~N.; Nijhawan, G.; Karthik, A.; Shrivastav, A.; and Chakravarthi, M.~K. 2024.
    \newblock Image-Based Time Series Forecasting: A Deep Convolutional Neural Network Approach.
    \newblock In \emph{2024 4th International Conference on Innovative Practices in Technology and Management (ICIPTM)}, 1--6. IEEE.
    
    \bibitem[{Miao et~al.(2021)Miao, Wu, Wang, Gao, Mao, and Yin}]{miao2021generative}
    Miao, X.; Wu, Y.; Wang, J.; Gao, Y.; Mao, X.; and Yin, J. 2021.
    \newblock Generative semi-supervised learning for multivariate time series imputation.
    \newblock In \emph{Proceedings of the AAAI conference on artificial intelligence}, volume~35, 8983--8991.
    
    \bibitem[{Neil, Pfeiffer, and Liu(2016)}]{neil2016phased}
    Neil, D.; Pfeiffer, M.; and Liu, S.-C. 2016.
    \newblock Phased lstm: Accelerating recurrent network training for long or event-based sequences.
    \newblock \emph{Advances in neural information processing systems}, 29.
    
    \bibitem[{Raghu et~al.(2023)Raghu, Chandak, Alam, Guttag, and Stultz}]{raghu2023sequential}
    Raghu, A.; Chandak, P.; Alam, R.; Guttag, J.; and Stultz, C. 2023.
    \newblock Sequential multi-dimensional self-supervised learning for clinical time series.
    \newblock In \emph{International Conference on Machine Learning}, 28531--28548. PMLR.
    
    \bibitem[{Reiss and Stricker(2012)}]{reiss2012introducing}
    Reiss, A.; and Stricker, D. 2012.
    \newblock Introducing a new benchmarked dataset for activity monitoring.
    \newblock In \emph{2012 16th international symposium on wearable computers}, 108--109. IEEE.
    
    \bibitem[{Reyna et~al.(2020)Reyna, Josef, Jeter, Shashikumar, Westover, Nemati, Clifford, and Sharma}]{reyna2020early}
    Reyna, M.~A.; Josef, C.~S.; Jeter, R.; Shashikumar, S.~P.; Westover, M.~B.; Nemati, S.; Clifford, G.~D.; and Sharma, A. 2020.
    \newblock Early prediction of sepsis from clinical data: the PhysioNet/Computing in Cardiology Challenge 2019.
    \newblock \emph{Critical care medicine}, 48(2): 210--217.
    
    \bibitem[{Sangha et~al.(2024)Sangha, Khunte, Holste, Mortazavi, Wang, Oikonomou, and Khera}]{sangha2024biometric}
    Sangha, V.; Khunte, A.; Holste, G.; Mortazavi, B.~J.; Wang, Z.; Oikonomou, E.~K.; and Khera, R. 2024.
    \newblock Biometric contrastive learning for data-efficient deep learning from electrocardiographic images.
    \newblock \emph{Journal of the American Medical Informatics Association}, 31(4): 855--865.
    
    \bibitem[{Sangha et~al.(2022)Sangha, Mortazavi, Haimovich, Ribeiro, Brandt, Jacoby, Schulz, Krumholz, Ribeiro, and Khera}]{sangha2022automated}
    Sangha, V.; Mortazavi, B.~J.; Haimovich, A.~D.; Ribeiro, A.~H.; Brandt, C.~A.; Jacoby, D.~L.; Schulz, W.~L.; Krumholz, H.~M.; Ribeiro, A. L.~P.; and Khera, R. 2022.
    \newblock Automated multilabel diagnosis on electrocardiographic images and signals.
    \newblock \emph{Nature communications}, 13(1): 1583.
    
    \bibitem[{Semenoglou, Spiliotis, and Assimakopoulos(2023)}]{semenoglou2023image}
    Semenoglou, A.-A.; Spiliotis, E.; and Assimakopoulos, V. 2023.
    \newblock Image-based time series forecasting: A deep convolutional neural network approach.
    \newblock \emph{Neural Networks}, 157: 39--53.
    
    \bibitem[{Sood et~al.(2021)Sood, Zeng, Cohen, Balch, and Veloso}]{sood2021visual}
    Sood, S.; Zeng, Z.; Cohen, N.; Balch, T.; and Veloso, M. 2021.
    \newblock Visual time series forecasting: an image-driven approach.
    \newblock In \emph{Proceedings of the Second ACM International Conference on AI in Finance}, 1--9.
    
    \bibitem[{Tripathy and Acharya(2018)}]{tripathy2018use}
    Tripathy, R.; and Acharya, U.~R. 2018.
    \newblock Use of features from RR-time series and EEG signals for automated classification of sleep stages in deep neural network framework.
    \newblock \emph{Biocybernetics and Biomedical Engineering}, 38(4): 890--902.
    
    \bibitem[{Wang et~al.(2024)Wang, Du, Cao, Zhang, Wang, Liang, and Wen}]{wang2024deep}
    Wang, J.; Du, W.; Cao, W.; Zhang, K.; Wang, W.; Liang, Y.; and Wen, Q. 2024.
    \newblock Deep learning for multivariate time series imputation: A survey.
    \newblock \emph{arXiv preprint arXiv:2402.04059}.
    
    \bibitem[{Wang and Oates(2015)}]{10.5555/2832747.2832798}
    Wang, Z.; and Oates, T. 2015.
    \newblock Imaging time-series to improve classification and imputation.
    \newblock In \emph{Proceedings of the 24th International Conference on Artificial Intelligence}, IJCAI'15, 3939–3945. AAAI Press.
    \newblock ISBN 9781577357384.
    
    \bibitem[{Weerakody, Wong, and Wang(2023)}]{weerakody2023policy}
    Weerakody, P.~B.; Wong, K.~W.; and Wang, G. 2023.
    \newblock Policy gradient empowered LSTM with dynamic skips for irregular time series data.
    \newblock \emph{Applied Soft Computing}, 142: 110314.
    
    \bibitem[{Wu et~al.(2023)Wu, Hu, Liu, Zhou, Wang, and Long}]{wu2023timesnet}
    Wu, H.; Hu, T.; Liu, Y.; Zhou, H.; Wang, J.; and Long, M. 2023.
    \newblock TimesNet: Temporal 2D-Variation Modeling for General Time Series Analysis.
    \newblock In \emph{International Conference on Learning Representations}.
    
    \bibitem[{Xu et~al.(2024)Xu, Xu, Yang, and Chen}]{xu2024learning}
    Xu, S.; Xu, T.; Yang, Y.; and Chen, X. 2024.
    \newblock Learning metabolic dynamics from irregular observations by Bidirectional Time-Series State Transfer Network.
    \newblock \emph{mSystems}, e00697--24.
    
    \bibitem[{Zhang et~al.(2023)Zhang, Zheng, Cao, Bian, and Li}]{zhang2023warpformer}
    Zhang, J.; Zheng, S.; Cao, W.; Bian, J.; and Li, J. 2023.
    \newblock Warpformer: A multi-scale modeling approach for irregular clinical time series.
    \newblock In \emph{Proceedings of the 29th ACM SIGKDD Conference on Knowledge Discovery and Data Mining}, 3273--3285.
    
    \bibitem[{Zhang et~al.(2022)Zhang, Zeman, Tsiligkaridis, and Zitnik}]{zhang2021graph}
    Zhang, X.; Zeman, M.; Tsiligkaridis, T.; and Zitnik, M. 2022.
    \newblock Graph-Guided Network For Irregularly Sampled Multivariate Time Series.
    \newblock In \emph{International Conference on Learning Representations, ICLR}.
    
    \end{thebibliography}
    




\end{document}
