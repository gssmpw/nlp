%File: formatting-instructions-latex-2025.tex
%release 2025.0
\documentclass[letterpaper]{article} % DO NOT CHANGE THIS
\usepackage{aaai25}  % DO NOT CHANGE THIS
\usepackage{times}  % DO NOT CHANGE THIS
\usepackage{helvet}  % DO NOT CHANGE THIS
\usepackage{courier}  % DO NOT CHANGE THIS
\usepackage[hyphens]{url}  % DO NOT CHANGE THIS
\usepackage{graphicx} % DO NOT CHANGE THIS
\urlstyle{rm} % DO NOT CHANGE THIS
\def\UrlFont{\rm}  % DO NOT CHANGE THIS
\usepackage{natbib}  % DO NOT CHANGE THIS AND DO NOT ADD ANY OPTIONS TO IT
\usepackage{caption} % DO NOT CHANGE THIS AND DO NOT ADD ANY OPTIONS TO IT
\frenchspacing  % DO NOT CHANGE THIS
\setlength{\pdfpagewidth}{8.5in}  % DO NOT CHANGE THIS
\setlength{\pdfpageheight}{11in}  % DO NOT CHANGE THIS
%
% These are recommended to typeset algorithms but not required. See the subsubsection on algorithms. Remove them if you don't have algorithms in your paper.
\usepackage{algorithm}
\usepackage{algorithmic}

%
% These are are recommended to typeset listings but not required. See the subsubsection on listing. Remove this block if you don't have listings in your paper.
\usepackage{newfloat}
\usepackage{listings}
\DeclareCaptionStyle{ruled}{labelfont=normalfont,labelsep=colon,strut=off} % DO NOT CHANGE THIS
\lstset{%
	basicstyle={\footnotesize\ttfamily},% footnotesize acceptable for monospace
	numbers=left,numberstyle=\footnotesize,xleftmargin=2em,% show line numbers, remove this entire line if you don't want the numbers.
	aboveskip=0pt,belowskip=0pt,%
	showstringspaces=false,tabsize=2,breaklines=true}
\floatstyle{ruled}
\newfloat{listing}{tb}{lst}{}
\floatname{listing}{Listing}
%
% Keep the \pdfinfo as shown here. There's no need
% for you to add the /Title and /Author tags.
\pdfinfo{
/TemplateVersion (2025.1)
}

% DISALLOWED PACKAGES
% \usepackage{authblk} -- This package is specifically forbidden
% \usepackage{balance} -- This package is specifically forbidden
% \usepackage{color (if used in text)
% \usepackage{CJK} -- This package is specifically forbidden
% \usepackage{float} -- This package is specifically forbidden
% \usepackage{flushend} -- This package is specifically forbidden
% \usepackage{fontenc} -- This package is specifically forbidden
% \usepackage{fullpage} -- This package is specifically forbidden
% \usepackage{geometry} -- This package is specifically forbidden
% \usepackage{grffile} -- This package is specifically forbidden
% \usepackage{hyperref} -- This package is specifically forbidden
% \usepackage{navigator} -- This package is specifically forbidden
% (or any other package that embeds links such as navigator or hyperref)
% \indentfirst} -- This package is specifically forbidden
% \layout} -- This package is specifically forbidden
% \multicol} -- This package is specifically forbidden
% \nameref} -- This package is specifically forbidden
% \usepackage{savetrees} -- This package is specifically forbidden
% \usepackage{setspace} -- This package is specifically forbidden
% \usepackage{stfloats} -- This package is specifically forbidden
% \usepackage{tabu} -- This package is specifically forbidden
% \usepackage{titlesec} -- This package is specifically forbidden
% \usepackage{tocbibind} -- This package is specifically forbidden
% \usepackage{ulem} -- This package is specifically forbidden
% \usepackage{wrapfig} -- This package is specifically forbidden
% DISALLOWED COMMANDS
% \nocopyright -- Your paper will not be published if you use this command
% \addtolength -- This command may not be used
% \balance -- This command may not be used
% \baselinestretch -- Your paper will not be published if you use this command
% \clearpage -- No page breaks of any kind may be used for the final version of your paper
% \columnsep -- This command may not be used
% \newpage -- No page breaks of any kind may be used for the final version of your paper
% \pagebreak -- No page breaks of any kind may be used for the final version of your paperr
% \pagestyle -- This command may not be used
% \tiny -- This is not an acceptable font size.
% \vspace{- -- No negative value may be used in proximity of a caption, figure, table, section, subsection, subsubsection, or reference
% \vskip{- -- No negative value may be used to alter spacing above or below a caption, figure, table, section, subsection, subsubsection, or reference
\usepackage{amsmath}
\usepackage{amssymb}
\usepackage{enumitem}
\usepackage{booktabs}
\usepackage{multirow}
\setcounter{secnumdepth}{0} %May be changed to 1 or 2 if section numbers are desired.

% The file aaai25.sty is the style file for AAAI Press
% proceedings, working notes, and technical reports.
%

% Title

% Your title must be in mixed case, not sentence case.
% That means all verbs (including short verbs like be, is, using,and go),
% nouns, adverbs, adjectives should be capitalized, including both words in hyphenated terms, while
% articles, conjunctions, and prepositions are lower case unless they
% directly follow a colon or long dash
\title{Integrating Sequence and Image Modeling in Irregular Medical Time Series Through Self-Supervised Learning}
\author{
    %Authors
    % All authors must be in the same font size and format.
    Liuqing Chen\textsuperscript{\rm 1,2}, Shuhong Xiao\textsuperscript{\rm 1}, Shixian Ding\textsuperscript{\rm 1}, Shanhai Hu\textsuperscript{\rm 1}, Lingyun Sun\textsuperscript{\rm 1,2}
    \thanks{Corresponding Author}
    % Written by AAAI Press Staff\textsuperscript{\rm 1}\thanks{With help from the AAAI Publications Committee.}\\
    % AAAI Style Contributions by Pater Patel Schneider,
    % Sunil Issar,\\
    % J. Scott Penberthy,
    % George Ferguson,
    % Hans Guesgen,
    % Francisco Cruz\equalcontrib,
    % Marc Pujol-Gonzalez\equalcontrib
}
\affiliations{
    %Afiliations
    % \textsuperscript{\rm 1}Association for the Advancement of Artificial Intelligence\\
    \textsuperscript{\rm 1}College of Computer Science and Technology, Zhejiang University, China\\
    \textsuperscript{\rm 2}International Design Institute, Zhejiang University, China \\
    sunly@zju.edu.cn
    % If you have multiple authors and multiple affiliations
    % use superscripts in text and roman font to identify them.
    % For example,
    
    % Sunil Issar\textsuperscript{\rm 2}, 
    % J. Scott Penberthy\textsuperscript{\rm 3}, 
    % George Ferguson\textsuperscript{\rm 4},
    % Hans Guesgen\textsuperscript{\rm 5}
    % Note that the comma should be placed after the superscript

    % 1101 Pennsylvania Ave, NW Suite 300\\
    % Washington, DC 20004 USA\\
    % % email address must be in roman text type, not monospace or sans serif
    % proceedings-questions@aaai.org
%
% See more examples next
}

%Example, Single Author, ->> remove \iffalse,\fi and place them surrounding AAAI title to use it
\iffalse
\title{My Publication Title --- Single Author}
\author {
    Author Name
}
\affiliations{
    Affiliation\\
    Affiliation Line 2\\
    name@example.com
}
\fi

\iffalse
%Example, Multiple Authors, ->> remove \iffalse,\fi and place them surrounding AAAI title to use it
\title{My Publication Title --- Multiple Authors}
\author {
    % Authors
    First Author Name\textsuperscript{\rm 1,\rm 2},
    Second Author Name\textsuperscript{\rm 2},
    Third Author Name\textsuperscript{\rm 1}
}
\affiliations {
    % Affiliations
    \textsuperscript{\rm 1}Affiliation 1\\
    \textsuperscript{\rm 2}Affiliation 2\\
    firstAuthor@affiliation1.com, secondAuthor@affilation2.com, thirdAuthor@affiliation1.com
}
\fi




\begin{document}

\maketitle

\begin{abstract}
Medical time series are often irregular and face significant missingness, posing challenges for data analysis and clinical decision-making. Existing methods typically adopt a single modeling perspective, either treating series data as sequences or transforming them into image representations for further classification. In this paper, we propose a joint learning framework that incorporates both sequence and image representations. We also design three self-supervised learning strategies to facilitate the fusion of sequence and image representations, capturing a more generalizable joint representation. The results indicate that our approach outperforms seven other state-of-the-art models in three representative real-world clinical datasets. We further validate our approach by simulating two major types of real-world missingness through leave-sensors-out and leave-samples-out techniques. The results demonstrate that our approach is more robust and significantly surpasses other baselines in terms of classification performance. 
\end{abstract}

% Uncomment the following to link to your code, datasets, an extended version or similar.
%
\begin{links}
    \link{Code}{https://github.com/zju-d3/AAAI25-Irregular-Medical-Time-Series}
    % \link{Datasets}{https://aaai.org/example/datasets}
    % \link{Extended version}{https://aaai.org/example/extended-version}
\end{links}

% \section{Introduction}
\label{sec:introduction}
The business processes of organizations are experiencing ever-increasing complexity due to the large amount of data, high number of users, and high-tech devices involved \cite{martin2021pmopportunitieschallenges, beerepoot2023biggestbpmproblems}. This complexity may cause business processes to deviate from normal control flow due to unforeseen and disruptive anomalies \cite{adams2023proceddsriftdetection}. These control-flow anomalies manifest as unknown, skipped, and wrongly-ordered activities in the traces of event logs monitored from the execution of business processes \cite{ko2023adsystematicreview}. For the sake of clarity, let us consider an illustrative example of such anomalies. Figure \ref{FP_ANOMALIES} shows a so-called event log footprint, which captures the control flow relations of four activities of a hypothetical event log. In particular, this footprint captures the control-flow relations between activities \texttt{a}, \texttt{b}, \texttt{c} and \texttt{d}. These are the causal ($\rightarrow$) relation, concurrent ($\parallel$) relation, and other ($\#$) relations such as exclusivity or non-local dependency \cite{aalst2022pmhandbook}. In addition, on the right are six traces, of which five exhibit skipped, wrongly-ordered and unknown control-flow anomalies. For example, $\langle$\texttt{a b d}$\rangle$ has a skipped activity, which is \texttt{c}. Because of this skipped activity, the control-flow relation \texttt{b}$\,\#\,$\texttt{d} is violated, since \texttt{d} directly follows \texttt{b} in the anomalous trace.
\begin{figure}[!t]
\centering
\includegraphics[width=0.9\columnwidth]{images/FP_ANOMALIES.png}
\caption{An example event log footprint with six traces, of which five exhibit control-flow anomalies.}
\label{FP_ANOMALIES}
\end{figure}

\subsection{Control-flow anomaly detection}
Control-flow anomaly detection techniques aim to characterize the normal control flow from event logs and verify whether these deviations occur in new event logs \cite{ko2023adsystematicreview}. To develop control-flow anomaly detection techniques, \revision{process mining} has seen widespread adoption owing to process discovery and \revision{conformance checking}. On the one hand, process discovery is a set of algorithms that encode control-flow relations as a set of model elements and constraints according to a given modeling formalism \cite{aalst2022pmhandbook}; hereafter, we refer to the Petri net, a widespread modeling formalism. On the other hand, \revision{conformance checking} is an explainable set of algorithms that allows linking any deviations with the reference Petri net and providing the fitness measure, namely a measure of how much the Petri net fits the new event log \cite{aalst2022pmhandbook}. Many control-flow anomaly detection techniques based on \revision{conformance checking} (hereafter, \revision{conformance checking}-based techniques) use the fitness measure to determine whether an event log is anomalous \cite{bezerra2009pmad, bezerra2013adlogspais, myers2018icsadpm, pecchia2020applicationfailuresanalysispm}. 

The scientific literature also includes many \revision{conformance checking}-independent techniques for control-flow anomaly detection that combine specific types of trace encodings with machine/deep learning \cite{ko2023adsystematicreview, tavares2023pmtraceencoding}. Whereas these techniques are very effective, their explainability is challenging due to both the type of trace encoding employed and the machine/deep learning model used \cite{rawal2022trustworthyaiadvances,li2023explainablead}. Hence, in the following, we focus on the shortcomings of \revision{conformance checking}-based techniques to investigate whether it is possible to support the development of competitive control-flow anomaly detection techniques while maintaining the explainable nature of \revision{conformance checking}.
\begin{figure}[!t]
\centering
\includegraphics[width=\columnwidth]{images/HIGH_LEVEL_VIEW.png}
\caption{A high-level view of the proposed framework for combining \revision{process mining}-based feature extraction with dimensionality reduction for control-flow anomaly detection.}
\label{HIGH_LEVEL_VIEW}
\end{figure}

\subsection{Shortcomings of \revision{conformance checking}-based techniques}
Unfortunately, the detection effectiveness of \revision{conformance checking}-based techniques is affected by noisy data and low-quality Petri nets, which may be due to human errors in the modeling process or representational bias of process discovery algorithms \cite{bezerra2013adlogspais, pecchia2020applicationfailuresanalysispm, aalst2016pm}. Specifically, on the one hand, noisy data may introduce infrequent and deceptive control-flow relations that may result in inconsistent fitness measures, whereas, on the other hand, checking event logs against a low-quality Petri net could lead to an unreliable distribution of fitness measures. Nonetheless, such Petri nets can still be used as references to obtain insightful information for \revision{process mining}-based feature extraction, supporting the development of competitive and explainable \revision{conformance checking}-based techniques for control-flow anomaly detection despite the problems above. For example, a few works outline that token-based \revision{conformance checking} can be used for \revision{process mining}-based feature extraction to build tabular data and develop effective \revision{conformance checking}-based techniques for control-flow anomaly detection \cite{singh2022lapmsh, debenedictis2023dtadiiot}. However, to the best of our knowledge, the scientific literature lacks a structured proposal for \revision{process mining}-based feature extraction using the state-of-the-art \revision{conformance checking} variant, namely alignment-based \revision{conformance checking}.

\subsection{Contributions}
We propose a novel \revision{process mining}-based feature extraction approach with alignment-based \revision{conformance checking}. This variant aligns the deviating control flow with a reference Petri net; the resulting alignment can be inspected to extract additional statistics such as the number of times a given activity caused mismatches \cite{aalst2022pmhandbook}. We integrate this approach into a flexible and explainable framework for developing techniques for control-flow anomaly detection. The framework combines \revision{process mining}-based feature extraction and dimensionality reduction to handle high-dimensional feature sets, achieve detection effectiveness, and support explainability. Notably, in addition to our proposed \revision{process mining}-based feature extraction approach, the framework allows employing other approaches, enabling a fair comparison of multiple \revision{conformance checking}-based and \revision{conformance checking}-independent techniques for control-flow anomaly detection. Figure \ref{HIGH_LEVEL_VIEW} shows a high-level view of the framework. Business processes are monitored, and event logs obtained from the database of information systems. Subsequently, \revision{process mining}-based feature extraction is applied to these event logs and tabular data input to dimensionality reduction to identify control-flow anomalies. We apply several \revision{conformance checking}-based and \revision{conformance checking}-independent framework techniques to publicly available datasets, simulated data of a case study from railways, and real-world data of a case study from healthcare. We show that the framework techniques implementing our approach outperform the baseline \revision{conformance checking}-based techniques while maintaining the explainable nature of \revision{conformance checking}.

In summary, the contributions of this paper are as follows.
\begin{itemize}
    \item{
        A novel \revision{process mining}-based feature extraction approach to support the development of competitive and explainable \revision{conformance checking}-based techniques for control-flow anomaly detection.
    }
    \item{
        A flexible and explainable framework for developing techniques for control-flow anomaly detection using \revision{process mining}-based feature extraction and dimensionality reduction.
    }
    \item{
        Application to synthetic and real-world datasets of several \revision{conformance checking}-based and \revision{conformance checking}-independent framework techniques, evaluating their detection effectiveness and explainability.
    }
\end{itemize}

The rest of the paper is organized as follows.
\begin{itemize}
    \item Section \ref{sec:related_work} reviews the existing techniques for control-flow anomaly detection, categorizing them into \revision{conformance checking}-based and \revision{conformance checking}-independent techniques.
    \item Section \ref{sec:abccfe} provides the preliminaries of \revision{process mining} to establish the notation used throughout the paper, and delves into the details of the proposed \revision{process mining}-based feature extraction approach with alignment-based \revision{conformance checking}.
    \item Section \ref{sec:framework} describes the framework for developing \revision{conformance checking}-based and \revision{conformance checking}-independent techniques for control-flow anomaly detection that combine \revision{process mining}-based feature extraction and dimensionality reduction.
    \item Section \ref{sec:evaluation} presents the experiments conducted with multiple framework and baseline techniques using data from publicly available datasets and case studies.
    \item Section \ref{sec:conclusions} draws the conclusions and presents future work.
\end{itemize}
% \section{RELATED WORK}
\label{sec:relatedwork}
In this section, we describe the previous works related to our proposal, which are divided into two parts. In Section~\ref{sec:relatedwork_exoplanet}, we present a review of approaches based on machine learning techniques for the detection of planetary transit signals. Section~\ref{sec:relatedwork_attention} provides an account of the approaches based on attention mechanisms applied in Astronomy.\par

\subsection{Exoplanet detection}
\label{sec:relatedwork_exoplanet}
Machine learning methods have achieved great performance for the automatic selection of exoplanet transit signals. One of the earliest applications of machine learning is a model named Autovetter \citep{MCcauliff}, which is a random forest (RF) model based on characteristics derived from Kepler pipeline statistics to classify exoplanet and false positive signals. Then, other studies emerged that also used supervised learning. \cite{mislis2016sidra} also used a RF, but unlike the work by \citet{MCcauliff}, they used simulated light curves and a box least square \citep[BLS;][]{kovacs2002box}-based periodogram to search for transiting exoplanets. \citet{thompson2015machine} proposed a k-nearest neighbors model for Kepler data to determine if a given signal has similarity to known transits. Unsupervised learning techniques were also applied, such as self-organizing maps (SOM), proposed \citet{armstrong2016transit}; which implements an architecture to segment similar light curves. In the same way, \citet{armstrong2018automatic} developed a combination of supervised and unsupervised learning, including RF and SOM models. In general, these approaches require a previous phase of feature engineering for each light curve. \par

%DL is a modern data-driven technology that automatically extracts characteristics, and that has been successful in classification problems from a variety of application domains. The architecture relies on several layers of NNs of simple interconnected units and uses layers to build increasingly complex and useful features by means of linear and non-linear transformation. This family of models is capable of generating increasingly high-level representations \citep{lecun2015deep}.

The application of DL for exoplanetary signal detection has evolved rapidly in recent years and has become very popular in planetary science.  \citet{pearson2018} and \citet{zucker2018shallow} developed CNN-based algorithms that learn from synthetic data to search for exoplanets. Perhaps one of the most successful applications of the DL models in transit detection was that of \citet{Shallue_2018}; who, in collaboration with Google, proposed a CNN named AstroNet that recognizes exoplanet signals in real data from Kepler. AstroNet uses the training set of labelled TCEs from the Autovetter planet candidate catalog of Q1–Q17 data release 24 (DR24) of the Kepler mission \citep{catanzarite2015autovetter}. AstroNet analyses the data in two views: a ``global view'', and ``local view'' \citep{Shallue_2018}. \par


% The global view shows the characteristics of the light curve over an orbital period, and a local view shows the moment at occurring the transit in detail

%different = space-based

Based on AstroNet, researchers have modified the original AstroNet model to rank candidates from different surveys, specifically for Kepler and TESS missions. \citet{ansdell2018scientific} developed a CNN trained on Kepler data, and included for the first time the information on the centroids, showing that the model improves performance considerably. Then, \citet{osborn2020rapid} and \citet{yu2019identifying} also included the centroids information, but in addition, \citet{osborn2020rapid} included information of the stellar and transit parameters. Finally, \citet{rao2021nigraha} proposed a pipeline that includes a new ``half-phase'' view of the transit signal. This half-phase view represents a transit view with a different time and phase. The purpose of this view is to recover any possible secondary eclipse (the object hiding behind the disk of the primary star).


%last pipeline applies a procedure after the prediction of the model to obtain new candidates, this process is carried out through a series of steps that include the evaluation with Discovery and Validation of Exoplanets (DAVE) \citet{kostov2019discovery} that was adapted for the TESS telescope.\par
%



\subsection{Attention mechanisms in astronomy}
\label{sec:relatedwork_attention}
Despite the remarkable success of attention mechanisms in sequential data, few papers have exploited their advantages in astronomy. In particular, there are no models based on attention mechanisms for detecting planets. Below we present a summary of the main applications of this modeling approach to astronomy, based on two points of view; performance and interpretability of the model.\par
%Attention mechanisms have not yet been explored in all sub-areas of astronomy. However, recent works show a successful application of the mechanism.
%performance

The application of attention mechanisms has shown improvements in the performance of some regression and classification tasks compared to previous approaches. One of the first implementations of the attention mechanism was to find gravitational lenses proposed by \citet{thuruthipilly2021finding}. They designed 21 self-attention-based encoder models, where each model was trained separately with 18,000 simulated images, demonstrating that the model based on the Transformer has a better performance and uses fewer trainable parameters compared to CNN. A novel application was proposed by \citet{lin2021galaxy} for the morphological classification of galaxies, who used an architecture derived from the Transformer, named Vision Transformer (VIT) \citep{dosovitskiy2020image}. \citet{lin2021galaxy} demonstrated competitive results compared to CNNs. Another application with successful results was proposed by \citet{zerveas2021transformer}; which first proposed a transformer-based framework for learning unsupervised representations of multivariate time series. Their methodology takes advantage of unlabeled data to train an encoder and extract dense vector representations of time series. Subsequently, they evaluate the model for regression and classification tasks, demonstrating better performance than other state-of-the-art supervised methods, even with data sets with limited samples.

%interpretation
Regarding the interpretability of the model, a recent contribution that analyses the attention maps was presented by \citet{bowles20212}, which explored the use of group-equivariant self-attention for radio astronomy classification. Compared to other approaches, this model analysed the attention maps of the predictions and showed that the mechanism extracts the brightest spots and jets of the radio source more clearly. This indicates that attention maps for prediction interpretation could help experts see patterns that the human eye often misses. \par

In the field of variable stars, \citet{allam2021paying} employed the mechanism for classifying multivariate time series in variable stars. And additionally, \citet{allam2021paying} showed that the activation weights are accommodated according to the variation in brightness of the star, achieving a more interpretable model. And finally, related to the TESS telescope, \citet{morvan2022don} proposed a model that removes the noise from the light curves through the distribution of attention weights. \citet{morvan2022don} showed that the use of the attention mechanism is excellent for removing noise and outliers in time series datasets compared with other approaches. In addition, the use of attention maps allowed them to show the representations learned from the model. \par

Recent attention mechanism approaches in astronomy demonstrate comparable results with earlier approaches, such as CNNs. At the same time, they offer interpretability of their results, which allows a post-prediction analysis. \par


% \section{Method}\label{sec:method}
\begin{figure}
    \centering
    \includegraphics[width=0.85\textwidth]{imgs/heatmap_acc.pdf}
    \caption{\textbf{Visualization of the proposed periodic Bayesian flow with mean parameter $\mu$ and accumulated accuracy parameter $c$ which corresponds to the entropy/uncertainty}. For $x = 0.3, \beta(1) = 1000$ and $\alpha_i$ defined in \cref{appd:bfn_cir}, this figure plots three colored stochastic parameter trajectories for receiver mean parameter $m$ and accumulated accuracy parameter $c$, superimposed on a log-scale heatmap of the Bayesian flow distribution $p_F(m|x,\senderacc)$ and $p_F(c|x,\senderacc)$. Note the \emph{non-monotonicity} and \emph{non-additive} property of $c$ which could inform the network the entropy of the mean parameter $m$ as a condition and the \emph{periodicity} of $m$. %\jj{Shrink the figures to save space}\hanlin{Do we need to make this figure one-column?}
    }
    \label{fig:vmbf_vis}
    \vskip -0.1in
\end{figure}
% \begin{wrapfigure}{r}{0.5\textwidth}
%     \centering
%     \includegraphics[width=0.49\textwidth]{imgs/heatmap_acc.pdf}
%     \caption{\textbf{Visualization of hyper-torus Bayesian flow based on von Mises Distribution}. For $x = 0.3, \beta(1) = 1000$ and $\alpha_i$ defined in \cref{appd:bfn_cir}, this figure plots three colored stochastic parameter trajectories for receiver mean parameter $m$ and accumulated accuracy parameter $c$, superimposed on a log-scale heatmap of the Bayesian flow distribution $p_F(m|x,\senderacc)$ and $p_F(c|x,\senderacc)$. Note the \emph{non-monotonicity} and \emph{non-additive} property of $c$. \jj{Shrink the figures to save space}}
%     \label{fig:vmbf_vis}
%     \vspace{-30pt}
% \end{wrapfigure}


In this section, we explain the detailed design of CrysBFN tackling theoretical and practical challenges. First, we describe how to derive our new formulation of Bayesian Flow Networks over hyper-torus $\mathbb{T}^{D}$ from scratch. Next, we illustrate the two key differences between \modelname and the original form of BFN: $1)$ a meticulously designed novel base distribution with different Bayesian update rules; and $2)$ different properties over the accuracy scheduling resulted from the periodicity and the new Bayesian update rules. Then, we present in detail the overall framework of \modelname over each manifold of the crystal space (\textit{i.e.} fractional coordinates, lattice vectors, atom types) respecting \textit{periodic E(3) invariance}. 

% In this section, we first demonstrate how to build Bayesian flow on hyper-torus $\mathbb{T}^{D}$ by overcoming theoretical and practical problems to provide a low-noise parameter-space approach to fractional atom coordinate generation. Next, we present how \modelname models each manifold of crystal space respecting \textit{periodic E(3) invariance}. 

\subsection{Periodic Bayesian Flow on Hyper-torus \texorpdfstring{$\mathbb{T}^{D}$}{}} 
For generative modeling of fractional coordinates in crystal, we first construct a periodic Bayesian flow on \texorpdfstring{$\mathbb{T}^{D}$}{} by designing every component of the totally new Bayesian update process which we demonstrate to be distinct from the original Bayesian flow (please see \cref{fig:non_add}). 
 %:) 
 
 The fractional atom coordinate system \citep{jiao2023crystal} inherently distributes over a hyper-torus support $\mathbb{T}^{3\times N}$. Hence, the normal distribution support on $\R$ used in the original \citep{bfn} is not suitable for this scenario. 
% The key problem of generative modeling for crystal is the periodicity of Cartesian atom coordinates $\vX$ requiring:
% \begin{equation}\label{eq:periodcity}
% p(\vA,\vL,\vX)=p(\vA,\vL,\vX+\vec{LK}),\text{where}~\vec{K}=\vec{k}\vec{1}_{1\times N},\forall\vec{k}\in\mathbb{Z}^{3\times1}
% \end{equation}
% However, there does not exist such a distribution supporting on $\R$ to model such property because the integration of such distribution over $\R$ will not be finite and equal to 1. Therefore, the normal distribution used in \citet{bfn} can not meet this condition.

To tackle this problem, the circular distribution~\citep{mardia2009directional} over the finite interval $[-\pi,\pi)$ is a natural choice as the base distribution for deriving the BFN on $\mathbb{T}^D$. 
% one natural choice is to 
% we would like to consider the circular distribution over the finite interval as the base 
% we find that circular distributions \citep{mardia2009directional} defined on a finite interval with lengths of $2\pi$ can be used as the instantiation of input distribution for the BFN on $\mathbb{T}^D$.
Specifically, circular distributions enjoy desirable periodic properties: $1)$ the integration over any interval length of $2\pi$ equals 1; $2)$ the probability distribution function is periodic with period $2\pi$.  Sharing the same intrinsic with fractional coordinates, such periodic property of circular distribution makes it suitable for the instantiation of BFN's input distribution, in parameterizing the belief towards ground truth $\x$ on $\mathbb{T}^D$. 
% \yuxuan{this is very complicated from my perspective.} \hanlin{But this property is exactly beautiful and perfectly fit into the BFN.}

\textbf{von Mises Distribution and its Bayesian Update} We choose von Mises distribution \citep{mardia2009directional} from various circular distributions as the form of input distribution, based on the appealing conjugacy property required in the derivation of the BFN framework.
% to leverage the Bayesian conjugacy property of von Mises distribution which is required by the BFN framework. 
That is, the posterior of a von Mises distribution parameterized likelihood is still in the family of von Mises distributions. The probability density function of von Mises distribution with mean direction parameter $m$ and concentration parameter $c$ (describing the entropy/uncertainty of $m$) is defined as: 
\begin{equation}
f(x|m,c)=vM(x|m,c)=\frac{\exp(c\cos(x-m))}{2\pi I_0(c)}
\end{equation}
where $I_0(c)$ is zeroth order modified Bessel function of the first kind as the normalizing constant. Given the last univariate belief parameterized by von Mises distribution with parameter $\theta_{i-1}=\{m_{i-1},\ c_{i-1}\}$ and the sample $y$ from sender distribution with unknown data sample $x$ and known accuracy $\alpha$ describing the entropy/uncertainty of $y$,  Bayesian update for the receiver is deducted as:
\begin{equation}
 h(\{m_{i-1},c_{i-1}\},y,\alpha)=\{m_i,c_i \}, \text{where}
\end{equation}
\begin{equation}\label{eq:h_m}
m_i=\text{atan2}(\alpha\sin y+c_{i-1}\sin m_{i-1}, {\alpha\cos y+c_{i-1}\cos m_{i-1}})
\end{equation}
\begin{equation}\label{eq:h_c}
c_i =\sqrt{\alpha^2+c_{i-1}^2+2\alpha c_{i-1}\cos(y-m_{i-1})}
\end{equation}
The proof of the above equations can be found in \cref{apdx:bayesian_update_function}. The atan2 function refers to  2-argument arctangent. Independently conducting  Bayesian update for each dimension, we can obtain the Bayesian update distribution by marginalizing $\y$:
\begin{equation}
p_U(\vtheta'|\vtheta,\bold{x};\alpha)=\mathbb{E}_{p_S(\bold{y}|\bold{x};\alpha)}\delta(\vtheta'-h(\vtheta,\bold{y},\alpha))=\mathbb{E}_{vM(\bold{y}|\bold{x},\alpha)}\delta(\vtheta'-h(\vtheta,\bold{y},\alpha))
\end{equation} 
\begin{figure}
    \centering
    \vskip -0.15in
    \includegraphics[width=0.95\linewidth]{imgs/non_add.pdf}
    \caption{An intuitive illustration of non-additive accuracy Bayesian update on the torus. The lengths of arrows represent the uncertainty/entropy of the belief (\emph{e.g.}~$1/\sigma^2$ for Gaussian and $c$ for von Mises). The directions of the arrows represent the believed location (\emph{e.g.}~ $\mu$ for Gaussian and $m$ for von Mises).}
    \label{fig:non_add}
    \vskip -0.15in
\end{figure}
\textbf{Non-additive Accuracy} 
The additive accuracy is a nice property held with the Gaussian-formed sender distribution of the original BFN expressed as:
\begin{align}
\label{eq:standard_id}
    \update(\parsn{}'' \mid \parsn{}, \x; \alpha_a+\alpha_b) = \E_{\update(\parsn{}' \mid \parsn{}, \x; \alpha_a)} \update(\parsn{}'' \mid \parsn{}', \x; \alpha_b)
\end{align}
Such property is mainly derived based on the standard identity of Gaussian variable:
\begin{equation}
X \sim \mathcal{N}\left(\mu_X, \sigma_X^2\right), Y \sim \mathcal{N}\left(\mu_Y, \sigma_Y^2\right) \Longrightarrow X+Y \sim \mathcal{N}\left(\mu_X+\mu_Y, \sigma_X^2+\sigma_Y^2\right)
\end{equation}
The additive accuracy property makes it feasible to derive the Bayesian flow distribution $
p_F(\boldsymbol{\theta} \mid \mathbf{x} ; i)=p_U\left(\boldsymbol{\theta} \mid \boldsymbol{\theta}_0, \mathbf{x}, \sum_{k=1}^{i} \alpha_i \right)
$ for the simulation-free training of \cref{eq:loss_n}.
It should be noted that the standard identity in \cref{eq:standard_id} does not hold in the von Mises distribution. Hence there exists an important difference between the original Bayesian flow defined on Euclidean space and the Bayesian flow of circular data on $\mathbb{T}^D$ based on von Mises distribution. With prior $\btheta = \{\bold{0},\bold{0}\}$, we could formally represent the non-additive accuracy issue as:
% The additive accuracy property implies the fact that the "confidence" for the data sample after observing a series of the noisy samples with accuracy ${\alpha_1, \cdots, \alpha_i}$ could be  as the accuracy sum  which could be  
% Here we 
% Here we emphasize the specific property of BFN based on von Mises distribution.
% Note that 
% \begin{equation}
% \update(\parsn'' \mid \parsn, \x; \alpha_a+\alpha_b) \ne \E_{\update(\parsn' \mid \parsn, \x; \alpha_a)} \update(\parsn'' \mid \parsn', \x; \alpha_b)
% \end{equation}
% \oyyw{please check whether the below equation is better}
% \yuxuan{I fill somehow confusing on what is the update distribution with $\alpha$. }
% \begin{equation}
% \update(\parsn{}'' \mid \parsn{}, \x; \alpha_a+\alpha_b) \ne \E_{\update(\parsn{}' \mid \parsn{}, \x; \alpha_a)} \update(\parsn{}'' \mid \parsn{}', \x; \alpha_b)
% \end{equation}
% We give an intuitive visualization of such difference in \cref{fig:non_add}. The untenability of this property can materialize by considering the following case: with prior $\btheta = \{\bold{0},\bold{0}\}$, check the two-step Bayesian update distribution with $\alpha_a,\alpha_b$ and one-step Bayesian update with $\alpha=\alpha_a+\alpha_b$:
\begin{align}
\label{eq:nonadd}
     &\update(c'' \mid \parsn, \x; \alpha_a+\alpha_b)  = \delta(c-\alpha_a-\alpha_b)
     \ne  \mathbb{E}_{p_U(\parsn' \mid \parsn, \x; \alpha_a)}\update(c'' \mid \parsn', \x; \alpha_b) \nonumber \\&= \mathbb{E}_{vM(\bold{y}_b|\bold{x},\alpha_a)}\mathbb{E}_{vM(\bold{y}_a|\bold{x},\alpha_b)}\delta(c-||[\alpha_a \cos\y_a+\alpha_b\cos \y_b,\alpha_a \sin\y_a+\alpha_b\sin \y_b]^T||_2)
\end{align}
A more intuitive visualization could be found in \cref{fig:non_add}. This fundamental difference between periodic Bayesian flow and that of \citet{bfn} presents both theoretical and practical challenges, which we will explain and address in the following contents.

% This makes constructing Bayesian flow based on von Mises distribution intrinsically different from previous Bayesian flows (\citet{bfn}).

% Thus, we must reformulate the framework of Bayesian flow networks  accordingly. % and do necessary reformulations of BFN. 

% \yuxuan{overall I feel this part is complicated by using the language of update distribution. I would like to suggest simply use bayesian update, to provide intuitive explantion.}\hanlin{See the illustration in \cref{fig:non_add}}

% That introduces a cascade of problems, and we investigate the following issues: $(1)$ Accuracies between sender and receiver are not synchronized and need to be differentiated. $(2)$ There is no tractable Bayesian flow distribution for a one-step sample conditioned on a given time step $i$, and naively simulating the Bayesian flow results in computational overhead. $(3)$ It is difficult to control the entropy of the Bayesian flow. $(4)$ Accuracy is no longer a function of $t$ and becomes a distribution conditioned on $t$, which can be different across dimensions.
%\jj{Edited till here}

\textbf{Entropy Conditioning} As a common practice in generative models~\citep{ddpm,flowmatching,bfn}, timestep $t$ is widely used to distinguish among generation states by feeding the timestep information into the networks. However, this paper shows that for periodic Bayesian flow, the accumulated accuracy $\vc_i$ is more effective than time-based conditioning by informing the network about the entropy and certainty of the states $\parsnt{i}$. This stems from the intrinsic non-additive accuracy which makes the receiver's accumulated accuracy $c$ not bijective function of $t$, but a distribution conditioned on accumulated accuracies $\vc_i$ instead. Therefore, the entropy parameter $\vc$ is taken logarithm and fed into the network to describe the entropy of the input corrupted structure. We verify this consideration in \cref{sec:exp_ablation}. 
% \yuxuan{implement variant. traditionally, the timestep is widely used to distinguish the different states by putting the timestep embedding into the networks. citation of FM, diffusion, BFN. However, we find that conditioned on time in periodic flow could not provide extra benefits. To further boost the performance, we introduce a simple yet effective modification term entropy conditional. This is based on that the accumulated accuracy which represents the current uncertainty or entropy could be a better indicator to distinguish different states. + Describe how you do this. }



\textbf{Reformulations of BFN}. Recall the original update function with Gaussian sender distribution, after receiving noisy samples $\y_1,\y_2,\dots,\y_i$ with accuracies $\senderacc$, the accumulated accuracies of the receiver side could be analytically obtained by the additive property and it is consistent with the sender side.
% Since observing sample $\y$ with $\alpha_i$ can not result in exact accuracy increment $\alpha_i$ for receiver, the accuracies between sender and receiver are not synchronized which need to be differentiated. 
However, as previously mentioned, this does not apply to periodic Bayesian flow, and some of the notations in original BFN~\citep{bfn} need to be adjusted accordingly. We maintain the notations of sender side's one-step accuracy $\alpha$ and added accuracy $\beta$, and alter the notation of receiver's accuracy parameter as $c$, which is needed to be simulated by cascade of Bayesian updates. We emphasize that the receiver's accumulated accuracy $c$ is no longer a function of $t$ (differently from the Gaussian case), and it becomes a distribution conditioned on received accuracies $\senderacc$ from the sender. Therefore, we represent the Bayesian flow distribution of von Mises distribution as $p_F(\btheta|\x;\alpha_1,\alpha_2,\dots,\alpha_i)$. And the original simulation-free training with Bayesian flow distribution is no longer applicable in this scenario.
% Different from previous BFNs where the accumulated accuracy $\rho$ is not explicitly modeled, the accumulated accuracy parameter $c$ (visualized in \cref{fig:vmbf_vis}) needs to be explicitly modeled by feeding it to the network to avoid information loss.
% the randomaccuracy parameter $c$ (visualized in \cref{fig:vmbf_vis}) implies that there exists information in $c$ from the sender just like $m$, meaning that $c$ also should be fed into the network to avoid information loss. 
% We ablate this consideration in  \cref{sec:exp_ablation}. 

\textbf{Fast Sampling from Equivalent Bayesian Flow Distribution} Based on the above reformulations, the Bayesian flow distribution of von Mises distribution is reframed as: 
\begin{equation}\label{eq:flow_frac}
p_F(\btheta_i|\x;\alpha_1,\alpha_2,\dots,\alpha_i)=\E_{\update(\parsnt{1} \mid \parsnt{0}, \x ; \alphat{1})}\dots\E_{\update(\parsn_{i-1} \mid \parsnt{i-2}, \x; \alphat{i-1})} \update(\parsnt{i} | \parsnt{i-1},\x;\alphat{i} )
\end{equation}
Naively sampling from \cref{eq:flow_frac} requires slow auto-regressive iterated simulation, making training unaffordable. Noticing the mathematical properties of \cref{eq:h_m,eq:h_c}, we  transform \cref{eq:flow_frac} to the equivalent form:
\begin{equation}\label{eq:cirflow_equiv}
p_F(\vec{m}_i|\x;\alpha_1,\alpha_2,\dots,\alpha_i)=\E_{vM(\y_1|\x,\alpha_1)\dots vM(\y_i|\x,\alpha_i)} \delta(\vec{m}_i-\text{atan2}(\sum_{j=1}^i \alpha_j \cos \y_j,\sum_{j=1}^i \alpha_j \sin \y_j))
\end{equation}
\begin{equation}\label{eq:cirflow_equiv2}
p_F(\vec{c}_i|\x;\alpha_1,\alpha_2,\dots,\alpha_i)=\E_{vM(\y_1|\x,\alpha_1)\dots vM(\y_i|\x,\alpha_i)}  \delta(\vec{c}_i-||[\sum_{j=1}^i \alpha_j \cos \y_j,\sum_{j=1}^i \alpha_j \sin \y_j]^T||_2)
\end{equation}
which bypasses the computation of intermediate variables and allows pure tensor operations, with negligible computational overhead.
\begin{restatable}{proposition}{cirflowequiv}
The probability density function of Bayesian flow distribution defined by \cref{eq:cirflow_equiv,eq:cirflow_equiv2} is equivalent to the original definition in \cref{eq:flow_frac}. 
\end{restatable}
\textbf{Numerical Determination of Linear Entropy Sender Accuracy Schedule} ~Original BFN designs the accuracy schedule $\beta(t)$ to make the entropy of input distribution linearly decrease. As for crystal generation task, to ensure information coherence between modalities, we choose a sender accuracy schedule $\senderacc$ that makes the receiver's belief entropy $H(t_i)=H(p_I(\cdot|\vtheta_i))=H(p_I(\cdot|\vc_i))$ linearly decrease \emph{w.r.t.} time $t_i$, given the initial and final accuracy parameter $c(0)$ and $c(1)$. Due to the intractability of \cref{eq:vm_entropy}, we first use numerical binary search in $[0,c(1)]$ to determine the receiver's $c(t_i)$ for $i=1,\dots, n$ by solving the equation $H(c(t_i))=(1-t_i)H(c(0))+tH(c(1))$. Next, with $c(t_i)$, we conduct numerical binary search for each $\alpha_i$ in $[0,c(1)]$ by solving the equations $\E_{y\sim vM(x,\alpha_i)}[\sqrt{\alpha_i^2+c_{i-1}^2+2\alpha_i c_{i-1}\cos(y-m_{i-1})}]=c(t_i)$ from $i=1$ to $i=n$ for arbitrarily selected $x\in[-\pi,\pi)$.

After tackling all those issues, we have now arrived at a new BFN architecture for effectively modeling crystals. Such BFN can also be adapted to other type of data located in hyper-torus $\mathbb{T}^{D}$.

\subsection{Equivariant Bayesian Flow for Crystal}
With the above Bayesian flow designed for generative modeling of fractional coordinate $\vF$, we are able to build equivariant Bayesian flow for each modality of crystal. In this section, we first give an overview of the general training and sampling algorithm of \modelname (visualized in \cref{fig:framework}). Then, we describe the details of the Bayesian flow of every modality. The training and sampling algorithm can be found in \cref{alg:train} and \cref{alg:sampling}.

\textbf{Overview} Operating in the parameter space $\bthetaM=\{\bthetaA,\bthetaL,\bthetaF\}$, \modelname generates high-fidelity crystals through a joint BFN sampling process on the parameter of  atom type $\bthetaA$, lattice parameter $\vec{\theta}^L=\{\bmuL,\brhoL\}$, and the parameter of fractional coordinate matrix $\bthetaF=\{\bmF,\bcF\}$. We index the $n$-steps of the generation process in a discrete manner $i$, and denote the corresponding continuous notation $t_i=i/n$ from prior parameter $\thetaM_0$ to a considerably low variance parameter $\thetaM_n$ (\emph{i.e.} large $\vrho^L,\bmF$, and centered $\bthetaA$).

At training time, \modelname samples time $i\sim U\{1,n\}$ and $\bthetaM_{i-1}$ from the Bayesian flow distribution of each modality, serving as the input to the network. The network $\net$ outputs $\net(\parsnt{i-1}^\mathcal{M},t_{i-1})=\net(\parsnt{i-1}^A,\parsnt{i-1}^F,\parsnt{i-1}^L,t_{i-1})$ and conducts gradient descents on loss function \cref{eq:loss_n} for each modality. After proper training, the sender distribution $p_S$ can be approximated by the receiver distribution $p_R$. 

At inference time, from predefined $\thetaM_0$, we conduct transitions from $\thetaM_{i-1}$ to $\thetaM_{i}$ by: $(1)$ sampling $\y_i\sim p_R(\bold{y}|\thetaM_{i-1};t_i,\alpha_i)$ according to network prediction $\predM{i-1}$; and $(2)$ performing Bayesian update $h(\thetaM_{i-1},\y^\calM_{i-1},\alpha_i)$ for each dimension. 

% Alternatively, we complete this transition using the flow-back technique by sampling 
% $\thetaM_{i}$ from Bayesian flow distribution $\flow(\btheta^M_{i}|\predM{i-1};t_{i-1})$. 

% The training objective of $\net$ is to minimize the KL divergence between sender distribution and receiver distribution for every modality as defined in \cref{eq:loss_n} which is equivalent to optimizing the negative variational lower bound $\calL^{VLB}$ as discussed in \cref{sec:preliminaries}. 

%In the following part, we will present the Bayesian flow of each modality in detail.

\textbf{Bayesian Flow of Fractional Coordinate $\vF$}~The distribution of the prior parameter $\bthetaF_0$ is defined as:
\begin{equation}\label{eq:prior_frac}
    p(\bthetaF_0) \defeq \{vM(\vm_0^F|\vec{0}_{3\times N},\vec{0}_{3\times N}),\delta(\vc_0^F-\vec{0}_{3\times N})\} = \{U(\vec{0},\vec{1}),\delta(\vc_0^F-\vec{0}_{3\times N})\}
\end{equation}
Note that this prior distribution of $\vm_0^F$ is uniform over $[\vec{0},\vec{1})$, ensuring the periodic translation invariance property in \cref{De:pi}. The training objective is minimizing the KL divergence between sender and receiver distribution (deduction can be found in \cref{appd:cir_loss}): 
%\oyyw{replace $\vF$ with $\x$?} \hanlin{notations follow Preliminary?}
\begin{align}\label{loss_frac}
\calL_F = n \E_{i \sim \ui{n}, \flow(\parsn{}^F \mid \vF ; \senderacc)} \alpha_i\frac{I_1(\alpha_i)}{I_0(\alpha_i)}(1-\cos(\vF-\predF{i-1}))
\end{align}
where $I_0(x)$ and $I_1(x)$ are the zeroth and the first order of modified Bessel functions. The transition from $\bthetaF_{i-1}$ to $\bthetaF_{i}$ is the Bayesian update distribution based on network prediction:
\begin{equation}\label{eq:transi_frac}
    p(\btheta^F_{i}|\parsnt{i-1}^\calM)=\mathbb{E}_{vM(\bold{y}|\predF{i-1},\alpha_i)}\delta(\btheta^F_{i}-h(\btheta^F_{i-1},\bold{y},\alpha_i))
\end{equation}
\begin{restatable}{proposition}{fracinv}
With $\net_{F}$ as a periodic translation equivariant function namely $\net_F(\parsnt{}^A,w(\parsnt{}^F+\vt),\parsnt{}^L,t)=w(\net_F(\parsnt{}^A,\parsnt{}^F,\parsnt{}^L,t)+\vt), \forall\vt\in\R^3$, the marginal distribution of $p(\vF_n)$ defined by \cref{eq:prior_frac,eq:transi_frac} is periodic translation invariant. 
\end{restatable}
\textbf{Bayesian Flow of Lattice Parameter \texorpdfstring{$\boldsymbol{L}$}{}}   
Noting the lattice parameter $\bm{L}$ located in Euclidean space, we set prior as the parameter of a isotropic multivariate normal distribution $\btheta^L_0\defeq\{\vmu_0^L,\vrho_0^L\}=\{\bm{0}_{3\times3},\bm{1}_{3\times3}\}$
% \begin{equation}\label{eq:lattice_prior}
% \btheta^L_0\defeq\{\vmu_0^L,\vrho_0^L\}=\{\bm{0}_{3\times3},\bm{1}_{3\times3}\}
% \end{equation}
such that the prior distribution of the Markov process on $\vmu^L$ is the Dirac distribution $\delta(\vec{\mu_0}-\vec{0})$ and $\delta(\vec{\rho_0}-\vec{1})$, 
% \begin{equation}
%     p_I^L(\boldsymbol{L}|\btheta_0^L)=\mathcal{N}(\bm{L}|\bm{0},\bm{I})
% \end{equation}
which ensures O(3)-invariance of prior distribution of $\vL$. By Eq. 77 from \citet{bfn}, the Bayesian flow distribution of the lattice parameter $\bm{L}$ is: 
\begin{align}% =p_U(\bmuL|\btheta_0^L,\bm{L},\beta(t))
p_F^L(\bmuL|\bm{L};t) &=\mathcal{N}(\bmuL|\gamma(t)\bm{L},\gamma(t)(1-\gamma(t))\bm{I}) 
\end{align}
where $\gamma(t) = 1 - \sigma_1^{2t}$ and $\sigma_1$ is the predefined hyper-parameter controlling the variance of input distribution at $t=1$ under linear entropy accuracy schedule. The variance parameter $\vrho$ does not need to be modeled and fed to the network, since it is deterministic given the accuracy schedule. After sampling $\bmuL_i$ from $p_F^L$, the training objective is defined as minimizing KL divergence between sender and receiver distribution (based on Eq. 96 in \citet{bfn}):
\begin{align}
\mathcal{L}_{L} = \frac{n}{2}\left(1-\sigma_1^{2/n}\right)\E_{i \sim \ui{n}}\E_{\flow(\bmuL_{i-1} |\vL ; t_{i-1})}  \frac{\left\|\vL -\predL{i-1}\right\|^2}{\sigma_1^{2i/n}},\label{eq:lattice_loss}
\end{align}
where the prediction term $\predL{i-1}$ is the lattice parameter part of network output. After training, the generation process is defined as the Bayesian update distribution given network prediction:
\begin{equation}\label{eq:lattice_sampling}
    p(\bmuL_{i}|\parsnt{i-1}^\calM)=\update^L(\bmuL_{i}|\predL{i-1},\bmuL_{i-1};t_{i-1})
\end{equation}
    

% The final prediction of the lattice parameter is given by $\bmuL_n = \predL{n-1}$.
% \begin{equation}\label{eq:final_lattice}
%     \bmuL_n = \predL{n-1}
% \end{equation}

\begin{restatable}{proposition}{latticeinv}\label{prop:latticeinv}
With $\net_{L}$ as  O(3)-equivariant function namely $\net_L(\parsnt{}^A,\parsnt{}^F,\vQ\parsnt{}^L,t)=\vQ\net_L(\parsnt{}^A,\parsnt{}^F,\parsnt{}^L,t),\forall\vQ^T\vQ=\vI$, the marginal distribution of $p(\bmuL_n)$ defined by \cref{eq:lattice_sampling} is O(3)-invariant. 
\end{restatable}


\textbf{Bayesian Flow of Atom Types \texorpdfstring{$\boldsymbol{A}$}{}} 
Given that atom types are discrete random variables located in a simplex $\calS^K$, the prior parameter of $\boldsymbol{A}$ is the discrete uniform distribution over the vocabulary $\parsnt{0}^A \defeq \frac{1}{K}\vec{1}_{1\times N}$. 
% \begin{align}\label{eq:disc_input_prior}
% \parsnt{0}^A \defeq \frac{1}{K}\vec{1}_{1\times N}
% \end{align}
% \begin{align}
%     (\oh{j}{K})_k \defeq \delta_{j k}, \text{where }\oh{j}{K}\in \R^{K},\oh{\vA}{KD} \defeq \left(\oh{a_1}{K},\dots,\oh{a_N}{K}\right) \in \R^{K\times N}
% \end{align}
With the notation of the projection from the class index $j$ to the length $K$ one-hot vector $ (\oh{j}{K})_k \defeq \delta_{j k}, \text{where }\oh{j}{K}\in \R^{K},\oh{\vA}{KD} \defeq \left(\oh{a_1}{K},\dots,\oh{a_N}{K}\right) \in \R^{K\times N}$, the Bayesian flow distribution of atom types $\vA$ is derived in \citet{bfn}:
\begin{align}
\flow^{A}(\parsn^A \mid \vA; t) &= \E_{\N{\y \mid \beta^A(t)\left(K \oh{\vA}{K\times N} - \vec{1}_{K\times N}\right)}{\beta^A(t) K \vec{I}_{K\times N \times N}}} \delta\left(\parsn^A - \frac{e^{\y}\parsnt{0}^A}{\sum_{k=1}^K e^{\y_k}(\parsnt{0})_{k}^A}\right).
\end{align}
where $\beta^A(t)$ is the predefined accuracy schedule for atom types. Sampling $\btheta_i^A$ from $p_F^A$ as the training signal, the training objective is the $n$-step discrete-time loss for discrete variable \citep{bfn}: 
% \oyyw{can we simplify the next equation? Such as remove $K \times N, K \times N \times N$}
% \begin{align}
% &\calL_A = n\E_{i \sim U\{1,n\},\flow^A(\parsn^A \mid \vA ; t_{i-1}),\N{\y \mid \alphat{i}\left(K \oh{\vA}{KD} - \vec{1}_{K\times N}\right)}{\alphat{i} K \vec{I}_{K\times N \times N}}} \ln \N{\y \mid \alphat{i}\left(K \oh{\vA}{K\times N} - \vec{1}_{K\times N}\right)}{\alphat{i} K \vec{I}_{K\times N \times N}}\nonumber\\
% &\qquad\qquad\qquad-\sum_{d=1}^N \ln \left(\sum_{k=1}^K \out^{(d)}(k \mid \parsn^A; t_{i-1}) \N{\ydd{d} \mid \alphat{i}\left(K\oh{k}{K}- \vec{1}_{K\times N}\right)}{\alphat{i} K \vec{I}_{K\times N \times N}}\right)\label{discdisc_t_loss_exp}
% \end{align}
\begin{align}
&\calL_A = n\E_{i \sim U\{1,n\},\flow^A(\parsn^A \mid \vA ; t_{i-1}),\N{\y \mid \alphat{i}\left(K \oh{\vA}{KD} - \vec{1}\right)}{\alphat{i} K \vec{I}}} \ln \N{\y \mid \alphat{i}\left(K \oh{\vA}{K\times N} - \vec{1}\right)}{\alphat{i} K \vec{I}}\nonumber\\
&\qquad\qquad\qquad-\sum_{d=1}^N \ln \left(\sum_{k=1}^K \out^{(d)}(k \mid \parsn^A; t_{i-1}) \N{\ydd{d} \mid \alphat{i}\left(K\oh{k}{K}- \vec{1}\right)}{\alphat{i} K \vec{I}}\right)\label{discdisc_t_loss_exp}
\end{align}
where $\vec{I}\in \R^{K\times N \times N}$ and $\vec{1}\in\R^{K\times D}$. When sampling, the transition from $\bthetaA_{i-1}$ to $\bthetaA_{i}$ is derived as:
\begin{equation}
    p(\btheta^A_{i}|\parsnt{i-1}^\calM)=\update^A(\btheta^A_{i}|\btheta^A_{i-1},\predA{i-1};t_{i-1})
\end{equation}

The detailed training and sampling algorithm could be found in \cref{alg:train} and \cref{alg:sampling}.




% \section{Experiments}
\label{sec:experiments}
The experiments are designed to address two key research questions.
First, \textbf{RQ1} evaluates whether the average $L_2$-norm of the counterfactual perturbation vectors ($\overline{||\perturb||}$) decreases as the model overfits the data, thereby providing further empirical validation for our hypothesis.
Second, \textbf{RQ2} evaluates the ability of the proposed counterfactual regularized loss, as defined in (\ref{eq:regularized_loss2}), to mitigate overfitting when compared to existing regularization techniques.

% The experiments are designed to address three key research questions. First, \textbf{RQ1} investigates whether the mean perturbation vector norm decreases as the model overfits the data, aiming to further validate our intuition. Second, \textbf{RQ2} explores whether the mean perturbation vector norm can be effectively leveraged as a regularization term during training, offering insights into its potential role in mitigating overfitting. Finally, \textbf{RQ3} examines whether our counterfactual regularizer enables the model to achieve superior performance compared to existing regularization methods, thus highlighting its practical advantage.

\subsection{Experimental Setup}
\textbf{\textit{Datasets, Models, and Tasks.}}
The experiments are conducted on three datasets: \textit{Water Potability}~\cite{kadiwal2020waterpotability}, \textit{Phomene}~\cite{phomene}, and \textit{CIFAR-10}~\cite{krizhevsky2009learning}. For \textit{Water Potability} and \textit{Phomene}, we randomly select $80\%$ of the samples for the training set, and the remaining $20\%$ for the test set, \textit{CIFAR-10} comes already split. Furthermore, we consider the following models: Logistic Regression, Multi-Layer Perceptron (MLP) with 100 and 30 neurons on each hidden layer, and PreactResNet-18~\cite{he2016cvecvv} as a Convolutional Neural Network (CNN) architecture.
We focus on binary classification tasks and leave the extension to multiclass scenarios for future work. However, for datasets that are inherently multiclass, we transform the problem into a binary classification task by selecting two classes, aligning with our assumption.

\smallskip
\noindent\textbf{\textit{Evaluation Measures.}} To characterize the degree of overfitting, we use the test loss, as it serves as a reliable indicator of the model's generalization capability to unseen data. Additionally, we evaluate the predictive performance of each model using the test accuracy.

\smallskip
\noindent\textbf{\textit{Baselines.}} We compare CF-Reg with the following regularization techniques: L1 (``Lasso''), L2 (``Ridge''), and Dropout.

\smallskip
\noindent\textbf{\textit{Configurations.}}
For each model, we adopt specific configurations as follows.
\begin{itemize}
\item \textit{Logistic Regression:} To induce overfitting in the model, we artificially increase the dimensionality of the data beyond the number of training samples by applying a polynomial feature expansion. This approach ensures that the model has enough capacity to overfit the training data, allowing us to analyze the impact of our counterfactual regularizer. The degree of the polynomial is chosen as the smallest degree that makes the number of features greater than the number of data.
\item \textit{Neural Networks (MLP and CNN):} To take advantage of the closed-form solution for computing the optimal perturbation vector as defined in (\ref{eq:opt-delta}), we use a local linear approximation of the neural network models. Hence, given an instance $\inst_i$, we consider the (optimal) counterfactual not with respect to $\model$ but with respect to:
\begin{equation}
\label{eq:taylor}
    \model^{lin}(\inst) = \model(\inst_i) + \nabla_{\inst}\model(\inst_i)(\inst - \inst_i),
\end{equation}
where $\model^{lin}$ represents the first-order Taylor approximation of $\model$ at $\inst_i$.
Note that this step is unnecessary for Logistic Regression, as it is inherently a linear model.
\end{itemize}

\smallskip
\noindent \textbf{\textit{Implementation Details.}} We run all experiments on a machine equipped with an AMD Ryzen 9 7900 12-Core Processor and an NVIDIA GeForce RTX 4090 GPU. Our implementation is based on the PyTorch Lightning framework. We use stochastic gradient descent as the optimizer with a learning rate of $\eta = 0.001$ and no weight decay. We use a batch size of $128$. The training and test steps are conducted for $6000$ epochs on the \textit{Water Potability} and \textit{Phoneme} datasets, while for the \textit{CIFAR-10} dataset, they are performed for $200$ epochs.
Finally, the contribution $w_i^{\varepsilon}$ of each training point $\inst_i$ is uniformly set as $w_i^{\varepsilon} = 1~\forall i\in \{1,\ldots,m\}$.

The source code implementation for our experiments is available at the following GitHub repository: \url{https://anonymous.4open.science/r/COCE-80B4/README.md} 

\subsection{RQ1: Counterfactual Perturbation vs. Overfitting}
To address \textbf{RQ1}, we analyze the relationship between the test loss and the average $L_2$-norm of the counterfactual perturbation vectors ($\overline{||\perturb||}$) over training epochs.

In particular, Figure~\ref{fig:delta_loss_epochs} depicts the evolution of $\overline{||\perturb||}$ alongside the test loss for an MLP trained \textit{without} regularization on the \textit{Water Potability} dataset. 
\begin{figure}[ht]
    \centering
    \includegraphics[width=0.85\linewidth]{img/delta_loss_epochs.png}
    \caption{The average counterfactual perturbation vector $\overline{||\perturb||}$ (left $y$-axis) and the cross-entropy test loss (right $y$-axis) over training epochs ($x$-axis) for an MLP trained on the \textit{Water Potability} dataset \textit{without} regularization.}
    \label{fig:delta_loss_epochs}
\end{figure}

The plot shows a clear trend as the model starts to overfit the data (evidenced by an increase in test loss). 
Notably, $\overline{||\perturb||}$ begins to decrease, which aligns with the hypothesis that the average distance to the optimal counterfactual example gets smaller as the model's decision boundary becomes increasingly adherent to the training data.

It is worth noting that this trend is heavily influenced by the choice of the counterfactual generator model. In particular, the relationship between $\overline{||\perturb||}$ and the degree of overfitting may become even more pronounced when leveraging more accurate counterfactual generators. However, these models often come at the cost of higher computational complexity, and their exploration is left to future work.

Nonetheless, we expect that $\overline{||\perturb||}$ will eventually stabilize at a plateau, as the average $L_2$-norm of the optimal counterfactual perturbations cannot vanish to zero.

% Additionally, the choice of employing the score-based counterfactual explanation framework to generate counterfactuals was driven to promote computational efficiency.

% Future enhancements to the framework may involve adopting models capable of generating more precise counterfactuals. While such approaches may yield to performance improvements, they are likely to come at the cost of increased computational complexity.


\subsection{RQ2: Counterfactual Regularization Performance}
To answer \textbf{RQ2}, we evaluate the effectiveness of the proposed counterfactual regularization (CF-Reg) by comparing its performance against existing baselines: unregularized training loss (No-Reg), L1 regularization (L1-Reg), L2 regularization (L2-Reg), and Dropout.
Specifically, for each model and dataset combination, Table~\ref{tab:regularization_comparison} presents the mean value and standard deviation of test accuracy achieved by each method across 5 random initialization. 

The table illustrates that our regularization technique consistently delivers better results than existing methods across all evaluated scenarios, except for one case -- i.e., Logistic Regression on the \textit{Phomene} dataset. 
However, this setting exhibits an unusual pattern, as the highest model accuracy is achieved without any regularization. Even in this case, CF-Reg still surpasses other regularization baselines.

From the results above, we derive the following key insights. First, CF-Reg proves to be effective across various model types, ranging from simple linear models (Logistic Regression) to deep architectures like MLPs and CNNs, and across diverse datasets, including both tabular and image data. 
Second, CF-Reg's strong performance on the \textit{Water} dataset with Logistic Regression suggests that its benefits may be more pronounced when applied to simpler models. However, the unexpected outcome on the \textit{Phoneme} dataset calls for further investigation into this phenomenon.


\begin{table*}[h!]
    \centering
    \caption{Mean value and standard deviation of test accuracy across 5 random initializations for different model, dataset, and regularization method. The best results are highlighted in \textbf{bold}.}
    \label{tab:regularization_comparison}
    \begin{tabular}{|c|c|c|c|c|c|c|}
        \hline
        \textbf{Model} & \textbf{Dataset} & \textbf{No-Reg} & \textbf{L1-Reg} & \textbf{L2-Reg} & \textbf{Dropout} & \textbf{CF-Reg (ours)} \\ \hline
        Logistic Regression   & \textit{Water}   & $0.6595 \pm 0.0038$   & $0.6729 \pm 0.0056$   & $0.6756 \pm 0.0046$  & N/A    & $\mathbf{0.6918 \pm 0.0036}$                     \\ \hline
        MLP   & \textit{Water}   & $0.6756 \pm 0.0042$   & $0.6790 \pm 0.0058$   & $0.6790 \pm 0.0023$  & $0.6750 \pm 0.0036$    & $\mathbf{0.6802 \pm 0.0046}$                    \\ \hline
%        MLP   & \textit{Adult}   & $0.8404 \pm 0.0010$   & $\mathbf{0.8495 \pm 0.0007}$   & $0.8489 \pm 0.0014$  & $\mathbf{0.8495 \pm 0.0016}$     & $0.8449 \pm 0.0019$                    \\ \hline
        Logistic Regression   & \textit{Phomene}   & $\mathbf{0.8148 \pm 0.0020}$   & $0.8041 \pm 0.0028$   & $0.7835 \pm 0.0176$  & N/A    & $0.8098 \pm 0.0055$                     \\ \hline
        MLP   & \textit{Phomene}   & $0.8677 \pm 0.0033$   & $0.8374 \pm 0.0080$   & $0.8673 \pm 0.0045$  & $0.8672 \pm 0.0042$     & $\mathbf{0.8718 \pm 0.0040}$                    \\ \hline
        CNN   & \textit{CIFAR-10} & $0.6670 \pm 0.0233$   & $0.6229 \pm 0.0850$   & $0.7348 \pm 0.0365$   & N/A    & $\mathbf{0.7427 \pm 0.0571}$                     \\ \hline
    \end{tabular}
\end{table*}

\begin{table*}[htb!]
    \centering
    \caption{Hyperparameter configurations utilized for the generation of Table \ref{tab:regularization_comparison}. For our regularization the hyperparameters are reported as $\mathbf{\alpha/\beta}$.}
    \label{tab:performance_parameters}
    \begin{tabular}{|c|c|c|c|c|c|c|}
        \hline
        \textbf{Model} & \textbf{Dataset} & \textbf{No-Reg} & \textbf{L1-Reg} & \textbf{L2-Reg} & \textbf{Dropout} & \textbf{CF-Reg (ours)} \\ \hline
        Logistic Regression   & \textit{Water}   & N/A   & $0.0093$   & $0.6927$  & N/A    & $0.3791/1.0355$                     \\ \hline
        MLP   & \textit{Water}   & N/A   & $0.0007$   & $0.0022$  & $0.0002$    & $0.2567/1.9775$                    \\ \hline
        Logistic Regression   &
        \textit{Phomene}   & N/A   & $0.0097$   & $0.7979$  & N/A    & $0.0571/1.8516$                     \\ \hline
        MLP   & \textit{Phomene}   & N/A   & $0.0007$   & $4.24\cdot10^{-5}$  & $0.0015$    & $0.0516/2.2700$                    \\ \hline
       % MLP   & \textit{Adult}   & N/A   & $0.0018$   & $0.0018$  & $0.0601$     & $0.0764/2.2068$                    \\ \hline
        CNN   & \textit{CIFAR-10} & N/A   & $0.0050$   & $0.0864$ & N/A    & $0.3018/
        2.1502$                     \\ \hline
    \end{tabular}
\end{table*}

\begin{table*}[htb!]
    \centering
    \caption{Mean value and standard deviation of training time across 5 different runs. The reported time (in seconds) corresponds to the generation of each entry in Table \ref{tab:regularization_comparison}. Times are }
    \label{tab:times}
    \begin{tabular}{|c|c|c|c|c|c|c|}
        \hline
        \textbf{Model} & \textbf{Dataset} & \textbf{No-Reg} & \textbf{L1-Reg} & \textbf{L2-Reg} & \textbf{Dropout} & \textbf{CF-Reg (ours)} \\ \hline
        Logistic Regression   & \textit{Water}   & $222.98 \pm 1.07$   & $239.94 \pm 2.59$   & $241.60 \pm 1.88$  & N/A    & $251.50 \pm 1.93$                     \\ \hline
        MLP   & \textit{Water}   & $225.71 \pm 3.85$   & $250.13 \pm 4.44$   & $255.78 \pm 2.38$  & $237.83 \pm 3.45$    & $266.48 \pm 3.46$                    \\ \hline
        Logistic Regression   & \textit{Phomene}   & $266.39 \pm 0.82$ & $367.52 \pm 6.85$   & $361.69 \pm 4.04$  & N/A   & $310.48 \pm 0.76$                    \\ \hline
        MLP   &
        \textit{Phomene} & $335.62 \pm 1.77$   & $390.86 \pm 2.11$   & $393.96 \pm 1.95$ & $363.51 \pm 5.07$    & $403.14 \pm 1.92$                     \\ \hline
       % MLP   & \textit{Adult}   & N/A   & $0.0018$   & $0.0018$  & $0.0601$     & $0.0764/2.2068$                    \\ \hline
        CNN   & \textit{CIFAR-10} & $370.09 \pm 0.18$   & $395.71 \pm 0.55$   & $401.38 \pm 0.16$ & N/A    & $1287.8 \pm 0.26$                     \\ \hline
    \end{tabular}
\end{table*}

\subsection{Feasibility of our Method}
A crucial requirement for any regularization technique is that it should impose minimal impact on the overall training process.
In this respect, CF-Reg introduces an overhead that depends on the time required to find the optimal counterfactual example for each training instance. 
As such, the more sophisticated the counterfactual generator model probed during training the higher would be the time required. However, a more advanced counterfactual generator might provide a more effective regularization. We discuss this trade-off in more details in Section~\ref{sec:discussion}.

Table~\ref{tab:times} presents the average training time ($\pm$ standard deviation) for each model and dataset combination listed in Table~\ref{tab:regularization_comparison}.
We can observe that the higher accuracy achieved by CF-Reg using the score-based counterfactual generator comes with only minimal overhead. However, when applied to deep neural networks with many hidden layers, such as \textit{PreactResNet-18}, the forward derivative computation required for the linearization of the network introduces a more noticeable computational cost, explaining the longer training times in the table.

\subsection{Hyperparameter Sensitivity Analysis}
The proposed counterfactual regularization technique relies on two key hyperparameters: $\alpha$ and $\beta$. The former is intrinsic to the loss formulation defined in (\ref{eq:cf-train}), while the latter is closely tied to the choice of the score-based counterfactual explanation method used.

Figure~\ref{fig:test_alpha_beta} illustrates how the test accuracy of an MLP trained on the \textit{Water Potability} dataset changes for different combinations of $\alpha$ and $\beta$.

\begin{figure}[ht]
    \centering
    \includegraphics[width=0.85\linewidth]{img/test_acc_alpha_beta.png}
    \caption{The test accuracy of an MLP trained on the \textit{Water Potability} dataset, evaluated while varying the weight of our counterfactual regularizer ($\alpha$) for different values of $\beta$.}
    \label{fig:test_alpha_beta}
\end{figure}

We observe that, for a fixed $\beta$, increasing the weight of our counterfactual regularizer ($\alpha$) can slightly improve test accuracy until a sudden drop is noticed for $\alpha > 0.1$.
This behavior was expected, as the impact of our penalty, like any regularization term, can be disruptive if not properly controlled.

Moreover, this finding further demonstrates that our regularization method, CF-Reg, is inherently data-driven. Therefore, it requires specific fine-tuning based on the combination of the model and dataset at hand.
% \section{Conclusion}
In this work, we propose a simple yet effective approach, called SMILE, for graph few-shot learning with fewer tasks. Specifically, we introduce a novel dual-level mixup strategy, including within-task and across-task mixup, for enriching the diversity of nodes within each task and the diversity of tasks. Also, we incorporate the degree-based prior information to learn expressive node embeddings. Theoretically, we prove that SMILE effectively enhances the model's generalization performance. Empirically, we conduct extensive experiments on multiple benchmarks and the results suggest that SMILE significantly outperforms other baselines, including both in-domain and cross-domain few-shot settings.
\section{Introduction}

Multivariate time series are utilized in various real-world applications, particularly in the medical field, where they are used to record vital signs and laboratory test results for diagnosis \cite{chaudhary2020utilization,brizzi2022spatial}. Typically, these time series are irregular, faced with asynchronicity across sensors and nonuniform sampling in the time domain \cite{chowdhury2023primenet,huang2024dna}. Moreover, significant missing values are usually present in clinical data collection. For example, random missingness can result from patients joining or leaving treatments midway, or complete absence of data from a sensor when specific tests are not conducted \cite{de2019deep}. Some public clinical datasets, such as PhysioNet2012, take even a 80\% missing rate, posing challenges for data analysis and clinical decision-making \cite{wang2024deep}. 

Deep learning methods have been widely adopted to model irregular time series. Some methods rely on the assumption of time discretization, utilizing LSTMs \cite{neil2016phased,weerakody2023policy}, RNNs \cite{che2018recurrent,ma2020adversarial,miao2021generative}, and Transformers \cite{horn2020set,huang2024dna} to capture characteristics of discrete sequences. Nonetheless, these methods often face difficulties in accumulating errors from missing observations \cite{ma2019learning}. Recently, vision models have also shown promising potential in handling irregular sequence data \cite{li2024time}. By transforming series into corresponding RGB representations, visual frameworks can effectively capture dynamic trends and inter-sensor relationships within images \cite{ maroor2024image,li2024time}. However, such designs perform poorly with sparse series that exhibit heavy missing rate \cite{li2024time}. 

We recognize that no one has yet integrated both sequence and image representations in handling irregular medical time series. This introduces a pivotal question: \textit{How can we effectively merge these two distinct representations to improve the robustness of classification for irregular medical time series with extensive missing values?} 

To investigate this question, we utilize a joint learning framework that incorporates both sequence and image representations. Additionally, we propose different self-supervised learning (SSL) strategies to enhance the integration and capture of supplementary information across these two representations. Specifically, our approach consists of three main components, as shown in Figure \ref{framework}. For the sequence modeling branch, we employ a generator-discriminator structure and adopt an adversarial strategy \cite{ma2019learning,miao2021generative} for sequence imputation task to minimize the propagation of cumulative errors. In the image branch, we implement different image transformation strategies to improve the performance on sparse series, and utilize a pre-trained Swin Transformer \cite{liu2021swinv2,li2024time} to obtain the corresponding image representations. Three different SSL losses are designed: (1) an inter-sequence contrastive loss to stabilize the sequence imputation process; (2) a sequence-image contrastive loss with margin to learn a more generalizable joint representation for downstream classification; and (3) a clustering loss on joint representations to push similar cases closer across different batches. 


\begin{figure*}[htp]
    \centering
    \includegraphics[width=0.91\linewidth]{figure/framework.png}
    \caption{The framework of our approach.}
    \label{framework}
\end{figure*}

We conduct experiments on three real-world clinical datasets: PAM \cite{reiss2012introducing}, P12 \cite{goldberger2000physiobank}, and P19 \cite{reyna2020early}. Table \ref{dataset_statistics} presents their statistics, which show that all three datasets experience severe missing values. We compare our approach with seven other state-of-the-art (SOTA) methods in terms of classification performance. Specifically, our approach achieves the best performance across all three datasets. For the PAM dataset, we observe improvements of 3.1\% in Accuracy, 2.9\% in Precision, 2.3\% in Recall, and 2.6\% in F1 score compared to the second-best method. For the P12 and P19 datasets, we use AUPRC and AUROC as evaluation metrics. Our approach surpasses prior SOTA by 1.1\% (AUPRC) and 0.9\% (AUROC) on P12, and 5.8\% (AUPRC) and 2.3\% (AUROC) on P19. Furthermore, we test further missingness through leave-samples-out and leave-sensors-out experiments on the PAM dataset. In the most severe scenario, with an additional 50\% missing values, our approach demonstrates better robustness, outperforming the second-best method by 6.1\% in Accuracy, 5.9\% in Precision, 3.4\% in Recall, and 4.6\% in F1 score.


The contributions of this paper are summarized as follows:
\begin{itemize} 
\item We propose a joint representation learning framework for multivariate irregular medical time series. To the best of our knowledge, this is the first approach to incorporate both sequence and image modeling.
\item We outline three SSL strategies: inter-sequence contrastive loss, sequence-image contrastive loss, and clustering-based loss. These strategies together enable better integration of sequence and image representations, enhancing the robustness against heavy missingness.
\item Our approach outperforms seven other SOTA methods on three real-world clinical datasets. We also simulates two classic types of missingness and experiments show that our method offers better robustness in handling these cases.
\end{itemize}

\section{Related Work}
\subsection{Irregular Time Series Methods}

Early practices for modeling irregular time series with missing values typically relied on fixed-time discretization. In this context, \cite{choi2016doctor} ignores the timestamp information by treating all intervals as equal, \cite{lipton2016modeling} considers missing data as an effective feature for learning, 
% \cite{futoma2017learning} employs Gaussian processes to model missing data and high uncertainty in real-world situations, 
and \cite{harutyunyan2019multitask} segments the data into evenly spaced time intervals. In contrast, GRU-D \cite{che2018recurrent} employs a gated network and incorporates imputation of missing values into the optimization process. Unlike previous methods, it adopts an additional missing value mask and lag matrix as inputs. Similar strategy have been adopted in \cite{,ma2019learning,ma2020adversarial,miao2021generative}, where adversarial frameworks are utilized to enhance the prediction of imputed values. 

Some recent approaches have leveraged attention mechanisms to improve modeling. For instance, SeFT \cite{horn2020set} introduces a set of differentiable set functions and uses attention mechanisms to aggregate embeddings of different variables. ContiFormer \cite{chen2024contiformer}, on the other hand, combines neural ordinary differential equations (ODEs) with attention mechanisms based on continuous-time dynamics, extending the relationship modeling capabilities of Transformers to the continuous time domain. Besides, DNA-T \cite{huang2024dna} utilizes a deformable attention mechanism to dynamically adjust the receptive field, enabling more effective handling of local features and short-term correlations. Warpformer \cite{zhang2023warpformer} also considers multi-scale features by applying a warping module to achieve multi-grained representations. Unlike previous methods that adopt a sequence modeling perspective, ViTST \cite{li2024time} transforms the signals into RGB images and utilizes a pre-trained Swin Transformer for further classification and regression.

% There are still methods that fall outside the aforementioned categories. One work worth reviewing is Raindrop \cite{zhang2021graph}, which models times series from the perspective of graph neural networks. In this approach, each observation resembles a raindrop hitting a sensor graph, spreading information through a ripple effect. 

% On the other hand, diffusion models \cite{han2022card} are also a growing trend and have been explored in various fields such as energy \cite{xu2024denoising}, finance \cite{daiya2024diffstock}, and microbiology \cite{seki2023imputing}.

\subsection{Modeling Time Series as Images}
Transforming time series data into images has gained significant attention with the advancements in visual detection frameworks. Some approaches \cite{sood2021visual,sangha2022automated,ao2023image, semenoglou2023image, maroor2024image} plot time series directly as time-observation representations and utilize convolutional neural networks (CNNs) for downstream tasks. Generally, they do not apply special processing to the sequences, instead focusing on leveraging visual frameworks to better capture temporal patterns in visualized sequences. ViTST \cite{li2024time} is another similar case that extends further to multivariate sequences and discusses the impact of visualization parameters such as color, markers, and order. 

In contrast, other methods emphasize the modeling of time series, which requires more specialized design and expert knowledge. \cite{tripathy2018use} utilizes an iterative filtering (IF) approach to produce different intrinsic mode functions (IMFs) from EEG signals. Empirically, these transformed features often fit the task better than the original signals. Chong et al. \cite{chong2011signal} and Deng et al. \cite{bs2023_1730} model sequences based on time segmentation, calculating time-invariant features and transforming them into corresponding RGB images. Similarly, frequency domain modeling, as demonstrated by TimesNet \cite{wu2023timesnet}, has also proven effective. By utilizing fast Fourier transform (FFT) to concatenate signal of different time periods, it constructs a 2D representation optimized for CNNs. Finally, other methods model the relative relationships between points in a time series. Examples include Gramian Angular Field (GAF), Markov Transition Field (MTF), and recurrence plot \cite{10.5555/2832747.2832798,hatami2018classification}. Typically, these methods involve applying a reversible time coordinate transformation and calculating the correlations between points, effectively capturing the continuity and periodic characteristics of the sequences. 


\section{Approach}

\subsection{Notations}

For a given clinical time series dataset \( D \), each sample \( X \in \mathbb{R}^{d \times T} \) represents a set of \( d \) records over a time \( T = \{t_1,...,t_n\} \), corresponding to a label \( y\). A binary mask \( M \in \mathbb{R}^{d \times T} \) is used to indicate the presence of missing observations in \( X \), where \( M_i^j = 0 \) signifies that the observation of the \( i^{th} \) item at time \( j \) is missing. 

To better handle consecutive missing values time, we follow \cite{miao2021generative,che2018recurrent} to obtain a time-lag matrix \( \delta \in \mathbb{R}^{d \times T} \) for each sample \( X \). This matrix quantifies the time elapsed since the most recent non-missing value for each observation, defined as follows.
\[
\delta^j_i = 
\begin{cases} 
0, &  i = 1 \\
t_i - t_{i-1}, &  m^j_{i-1} = 1 \text{ and } i > 1 \\
\delta^j_{i-1} + t_i - t_{i-1}, &  m^j_{i-1} = 0 \text{ and } i > 1
\end{cases}
\]

For each sample \( X \), the corresponding image \( I \) is constructed, where \( I \in \mathbb{R}^{3\times W \times H} \) represent a certain RGB format image. In total, we implement six transformed images as shown in Figure \ref{framework}. The specific transformation methods applied are as follows: Line Graphs, Frequency Spectrums, Gramian Angular Summation/Difference Fields, Markov Transition Fields, Recurrence Plots.

\subsection{The Model Overview}
In this section, we introduce the overall framework of our model, which comprises three main parts: (a) the sequence encoder, (b) the image encoder, and (c) the joint representation module. The sequence encoder consists of a generator-discriminator pair employing an adversarial strategy for imputation. The generator, \( G \), takes the time series \( X \), the mask \( M \), and the lag matrix \( \delta \) as inputs. Its objective is to estimate the missing values in \( X \) and generate a completed sequence \( X' \). This completed sequence \( X' \) is then used to obtain the sequence representation \( s \in \mathbb{R}^{d} \).  The discriminator D evaluates these estimations with the goal of distinguishing true observations from the imputed values. It outputs a binary matrix \( M' \), which identifies the regions of imputation predicted. For the image encoder, it takes a transformed image \(I \) as input and output the corresponding image representation \( v \in \mathbb{R}^{d}\). Finally, the joint representation module is responsible for mapping the sequence representation \( s \) and the image representation \( v \) into the same space. It then uses the final joint feature \( u \in \mathbb{R}^{d} \) for classification. 

\subsection{Sequence Branch with Imputation}

We adopt a modified bidirectional recurrent neural network (BiRNN) as our generator \(G\), which has been widely used in imputation tasks \cite{ che2018recurrent, ma2019learning, ma2020adversarial, miao2021generative, xu2024learning}. Taking the forward update step as an example, we update the current hidden state as:
\begin{align}
h_t &= \tanh\left(W_h (\gamma_t \odot h_{t-1}) + W_h'(\hat{x}_t + x_\delta )+ b_h\right) \\
\gamma_t &= \exp\left\{-\max(0, W_{\gamma} \delta_t + b_{\gamma})\right\}
\end{align}
In this setup, \(\gamma_t\) is derived from the lag matrix to model the dynamics of decay, where a longer duration of missing data leads \(\gamma_t\) closer to 0. It is applied to determine the extent to which the previous hidden state \(h_{t-1}\) should be retained. In the updating process of \( h_t \), instead of solely utilizing the previous reconstruction \(\hat{x}_t\) as done in prior works, we introduce an additional computation involving \( x_{\delta} \) as Eq. \ref{decay}. 
\begin{align}
\label{decay}
x_\delta = x_{t^-} \cdot \exp \left\{ -\max(0, W_\delta \delta_t) + b_\delta \right\}
\end{align}
This assumes the closest observation \( x_{t^-} \) prior to the current missing value influences the reconstruction process, with this influence decreasing as the time gap increases. 

Then, using a fully connected layer, the new reconstruction of the next step is obtained as: \(\hat{x}_{t+1} = W_{\hat{x}} h_t + b_{\hat{x}}\). And the overall imputed sequence \( X' \) is represented as: \(X' = M \odot X + (1 - M) \odot avg(\hat{X}_{for} + \hat{X}_{back})\), where we take the average of forward and backward result, and only the missing parts are replaced. Finally, the sequence representation \( s \) is obtained as: 
\begin{align}
s = Drop(W_s \cdot LayerNorm(X') + b_s)
\end{align}
In particular, \(W_h\), \(W_h'\), \(W_\gamma\), \(W_\delta\), \(W_{\hat{x}}\), \(W_s\), \(b_h\), \(b_x\), \(b_\gamma\), \(b_\delta\), \(b_{\hat{x}}\), and \(b_s\) are learnable parameters of the model and \( \odot \) denotes the element-wise multiplication. 

We formulate the objective of generator \(G\) into two components: adversarial loss and reconstruction loss. The adversarial loss is defined as the standard GAN's \cite{goodfellow2020generative}:
\begin{align}
\mathcal{L}_{adv} = \mathbb{E} [ (1 - M) \log (1 - D(X'))]
\end{align}
For the reconstruction loss, previous methods often use regression-based metrics such as mean square error (MSE) \cite{ma2020adversarial} or mean absolute error (MAE) \cite{ma2019learning} to assess the consistency between the missing and imputed sequences. However, when dealing with severely missing data, these strategies often fail to model the underlying data patterns, force the generator to learn nothing during the adversarial training phase. Inspired by \cite{raghu2023sequential}, we adopt a self-learning strategy to construct our reconstruction loss, and one choice is the normalized temperature-scaled cross-entropy loss (NT-Xent) \cite{chen2020big}. Given 2\(B\) pairs \((z_i, z_j)\) totally, it is computed as: 
\begin{align}
\label{NT}
\mathcal{L}_{NT} = \frac{1}{2B} \sum_{i=1}^{2B} -\log \frac{\exp(sim(z_i, z_j) / \tau)}{\sum_{k=1}^{2B} 1_{[k \neq i]} \exp(sim(z_i, z_k) / \tau)}
\end{align}
where cosine similarity is used as \(sim(z_i, z_j)\) and \(\tau\) is the temperature hyperparameter. We use NT-Xent to enforce consistency between the forward and backward predictions, as well as between the original and imputed sequences. Thus, the reconstruction loss is defined as: 
\begin{align}
\mathcal{L}_{rec} =  \mathcal{L}_{NT}(\hat{X}_{for}, \hat{X}_{back})\ + \mathcal{L}_{NT}(X, X')
\end{align}

We employ the same RNN in \cite{ma2019learning} as our discriminator \(D\), which takes \(X'\) as input and determines whether each observation is generated with a binary matrix \(M'\). Therefore, the discriminator is trained by minimizing:
\begin{align}
\label{dis}
\mathcal{L}_{dis} = \mathbb{E}[Mlog M' + (1-M)log(1-M')]
\end{align}


\subsection{Imaging Time Series}
We use a pre-trained Swin Transformer \cite{liu2021swinv2} as our image encoder. For the given image input \(I\), the Swin Transformer constructs a hierarchical representation to integrate both local and global information. Specifically, at earlier layers, it partitions the input into small patches and progressively merges neighboring patches as depth increases. It employs two types of attention mechanisms: window-based multi-head self-attention (W-MSA) and shifted window multi-head self-attention (SW-MSA). These mechanisms are respectively used to compute self-attention within a fixed window and to calculate dynamic relationships between windows. The vectors from the last stage after layer normalization are used as our image representation \( v \in \mathbb{R}^{d}\).

Overall, we implement six types of images for representation learning and a detailed description is presented in Appendix A. 

\begin{itemize}[leftmargin=*]
\item \textbf{Line Graphs} are constructed as \cite{li2024time}, with each variable represented by a line image of uniform size. 
\item \textbf{Frequency Spectrums} are generated based on the Fourier transform, considering that frequency domain signals tend to be more robust in cases of extreme data missingness.
\item \textbf{Gramian Angular Fields} \cite{10.5555/2832747.2832798} transform time series into polar coordinates, constructing trigonometric sums/ differences between any two time points to represent temporal correlation. 
\item \textbf{Markov Transition Fields} record the Markov transition probabilities between any two time observations \cite{10.5555/2832747.2832798}. They are insensitive to the distribution of the time series and temporal step information, allowing them to effectively capture correlations between observations with substantial missing data.
\item \textbf{Recurrence Plots} \cite{hatami2018classification}, based on phase space reconstruction, transform time series data into trajectories within phase space and analyze their recurrences. They are designed to  capture the inherent repetitiveness and periodicity within the time series.
\end{itemize}

 
\subsection{Joint Representations Through Contrast and Clustering}

The joint representation module includes a transformation function \( f: s,v \to \mathbb{R}^D
 \), which projects and concatenates the sequence features \( s \) and image features \( v \) into a joint space \( R^D \), and the fused feature is obtained as \( u = [s,v] \). To ensure both the quality and consistency of the joint representation, we implement contrastive learning within each batch to maximize the mutual information between corresponding pairs. A simple choice is to use the NT-Xent in Eq. \ref{NT}, where only sequence and image features corresponding to the same sample are treated as positive pairs \cite{sangha2024biometric}. Through this approach, NT-Xent ensures that the similarity between representations from the same sample is higher than that of other pairs. However, it also misses opportunities to learn from a wider set of potential pairs \cite{li2022clustering}. 
 
 In this case, a step forward is to treat \(s\) and \(v\) from different samples within the different category as a special form of negative pairs, thereby enhancing the model's ability to distinguish inter-class differences. Specifically, we introduce an additional margin \(m\) for these special negative pairs, enforce the model to exert greater effort to distinguish them:

% \noindent\resizebox{\columnwidth}{!}{
% \begin{minipage}{0.6\textwidth} 
% \begin{align}
% -\frac{1}{B} \sum_{i=1}^{B} &\left[ \log \left(\frac{\exp((v_i \cdot s_i)/ \tau)}{\sum_{j \in P(i)} \exp((v_i \cdot s_j)/ \tau) + \sum_{j \notin P(i)} \exp((v_i \cdot s_j + m)/ \tau)}\right)\right. \nonumber \\
% & + \log \left(\frac{\exp((v_i \cdot s_i)/ \tau)}{\sum_{k \in P(i)} \exp((v_k \cdot s_i)/ \tau) + \sum_{k \notin P(i)} \exp((v_k \cdot s_i + m)/ \tau)}\right]
% \end{align}
% \label{cont}
% \end{minipage}
% }

\begin{equation}
\scalebox{0.93}{$
-\frac{1}{B} \sum_{i=1}^{B} \left[ \log \left( \frac{\exp((v_i \cdot s_i)/ \tau)}{\sum_{j \in P(i)} \exp((v_i \cdot s_j)/ \tau) + \sum_{j \notin P(i)} \exp((v_i \cdot s_j + m)/ \tau)} \right) \right. \nonumber$}
\end{equation}
\begin{equation}
\scalebox{0.93}{$
\left. + \log \left( \frac{\exp((v_i \cdot s_i)/ \tau)}{\sum_{k \in P(i)} \exp((v_k \cdot s_i)/ \tau) + \sum_{k \notin P(i)} \exp((v_k \cdot s_i + m)/ \tau)} \right] 
\right.$}
\label{cont}
\end{equation}






Here, for the \(i^{th}\)sample, \(P(i)\) represents the set of all sample index that are in the same category. 

In contrastive learning, the formation of positive and negative pairs is confined to each batch. However, this approach lacks control over the semantic relationships between samples across different batches. As a result, similar samples from separate batches may not receive similar representations. In this case, we incorporate clustering learning into the training process to push semantically similar samples together across batches.

Specifically, we applied the K-means algorithm to the fused feature \(u\). We begin with the assignment step: during each training epoch, we select a set of k (k $\ll$ N) representative features \([C_{u_1}, \ldots, C_{u_k}]\) as the cluster centers for that round. Each fused feature \(u_i\) is assigned to a set \(S_k\) with center \(C_{u_k}\) by minimizing the overall distance as defined in Eq. \ref{cluster}.

\begin{align}
\label{cluster}
\underset{S}{\mathrm{arg min}}  \sum_{j=1}^{k} \sum_{u_i \in S_j} \|u_i - C_{u_j}\|^2 
\end{align}


% \begin{align}
% \mathcal{L}_{\text{cluster}} = -\frac{1}{N} \sum_{i=1}^{N} \min_{j=1 \ldots k} \left( 1 - \cos(X_i, C_j) \right),
% \end{align}

We then use these cluster centers, \([C_{u_1}, \ldots, C_{u_k}]\), as contrastive loss reference targets to construct the clustering loss:
\begin{align}
\mathcal{L}_{cluster} = -\frac{1}{N} \sum_{i=1}^{N} \log \frac{\exp(\cos(u_i, C_{u_i})/\tau)}{\sum_{j=1}^{k} \exp(\cos(u_i, C_{u_j})/\tau)}
\end{align}
where a cluster center \(C_{u_k}\) and all elements within  set \(S_k\) are treated as positive pairs, and elements from different clusters are considered negative pairs. To ensure sufficient samples for optimizing clustering, we perform the update step at the end of each epoch: we iteratively update the cluster centers using Eq. \ref{update} and calculate new assignments with Eq. \ref{cluster}, until the total distance is less than a predefined threshold \(\tau_c\).

\begin{align}
\label{update}
C_k = \underset{u \in S_k}{\mathrm{arg min}} \sum_{u' \in S_k} \|u - u'\|^2 
\end{align}

\subsection{Overall Training Process}

The overall training process is divided into three steps as follows:

\begin{itemize}[leftmargin=*]
    \item Firstly, we fix the generator \( G \) and update the discriminator \( D \) based on Eq. \ref{dis} .
    \item Next, we update the parameters of \( G \) based on the new \( D \), with the objective function \(  \mathcal{L}_{adv} +  \alpha \mathcal{L}_{rec} \).
    \item Finally, we compute the forward pass of all three components, utilizing the joint feature \( u \) to perform classification. For the PAM dataset, we use the Cross Entropy Loss as the classification loss \(\mathcal{L}_{clf}\). For the more imbalanced P12 and P19 datasets, we opt for the Focal Loss. The final objective is expressed as \(\mathcal{L}_{clf} + \beta_1 \mathcal{L}_{cont} + \beta_2 \mathcal{L}_{cluster}\).
\end{itemize}


% \begin{algorithm}[H]
% \caption{Training Process}
% \label{alg:training}
% \begin{algorithmic}[1]
% \STATE \textbf{Initialize:} Generator $G$, Discriminator $D$, and all parameters.

% \STATE \textbf{Step 1: Update Discriminator}
% \STATE Fix $G$
% \FOR{each training step}
%     \STATE Update $D$ based on Eq. (8)
% \ENDFOR

% \STATE \textbf{Step 2: Update Generator}
% \STATE Fix $D$
% \FOR{each training step}
%     \STATE Update $G$ with the objective: $L_{adv} + \alpha L_{rec}$
% \ENDFOR

% \STATE \textbf{Step 3: Joint Forward Pass and Classification}
% \FOR{each forward pass}
%     \STATE Compute joint feature $u$
%     \STATE Perform classification using $u$
%     \IF{dataset is PAM}
%         \STATE Compute classification loss $L_{clf}$ using Cross Entropy Loss
%     \ELSIF{dataset is P12 or P19}
%         \STATE Compute classification loss $L_{clf}$ using Focal Loss
%     \ENDIF
% \ENDFOR

% \STATE \textbf{Final Objective:}
% \STATE $L_{final} = L_{clf} + \beta_1 L_{cont} + \beta_2 L_{cluster}$
% \end{algorithmic}
% \end{algorithm}


\section{Experiments}

\subsection{Datasets and Metrics}

\begin{table}[h]
% \setlength{\heavyrulewidth}{1.3pt}
\centering
% \resizebox{0.98\columnwidth}{!}{ 
\setlength{\tabcolsep}{0.8mm}
\begin{tabular}{@{}lccccr@{}} 
\toprule
\textbf{Dataset} & \textbf{Features} & \textbf{Time} & \textbf{Classes} & \textbf{Missing Ratio} & \textbf{Samples} \\ 
\midrule
PAM              & 17                & 600           & 8                & 60\%                   & 5,333             \\
P12              & 36                & 215           & 2                & 88.4\%                 & 11,988            \\
P19              & 34                & 60            & 2                & 94.9\%                 & 38,803            \\
\bottomrule
\end{tabular}
% }
\caption{Statistics of datasets utilized.}
\label{dataset_statistics}
\end{table}

\begin{table*}[htp]
\setlength{\heavyrulewidth}{1.3pt}
% \renewcommand{\arraystretch}{1}
\centering
\begin{tabular}{c|cccc|cc|cc}
\toprule
\multirow{2}{*}{Methods}  & \multicolumn{4}{c|}{PAM} & \multicolumn{2}{c|}{P12} & \multicolumn{2}{c}{P19}  \\ 
\cmidrule{2-9} 
 & Accuracy & Precision & Recall & F1 score & AUROC & AUPRC  & AUROC & AUPRC  \\
\midrule
GRU-D & 83.3 {$\pm$ \scriptsize 1.6}  & 84.6 {$\pm$ \scriptsize 1.2} & 85.2 {$\pm$ \scriptsize 1.6}  & 84.8 {$\pm$ \scriptsize 1.2} &  81.7 {$\pm$ \scriptsize 1.8} & 41.3 {$\pm$ \scriptsize 3.5}  & 83.6 {$\pm$ \scriptsize 2.1}  &  45.7 {$\pm$ \scriptsize 4.2}   \\
SeFT  & 63.3 {$\pm$ \scriptsize 2.2} & 66.7 {$\pm$ \scriptsize 2.4}  & 65.3 {$\pm$ \scriptsize 1.5} & 65.1 {$\pm$ \scriptsize 1.8} & 73.3 {$\pm$ \scriptsize 2.5} & 29.1 {$\pm$ \scriptsize 4.1} & 84.5 {$\pm$ \scriptsize 2.3} & 46.7 {$\pm$ \scriptsize 3.1}   \\
% SSGAN       &  &  &  &  &  &  &  &    \\
CARD      &71.9 {$\pm$ \scriptsize 2.9}  &75.5 {$\pm$ \scriptsize 2.8}  & 73.5 {$\pm$ \scriptsize 3.1} & 73.8 {$\pm$ \scriptsize 3.0} &71.4 {$\pm$ \scriptsize 0.9}  &26.1 {$\pm$ \scriptsize 1.2}  &80.7 {$\pm$ \scriptsize 1.0}  & 36.7 {$\pm$ \scriptsize 6.0}    \\
Raindrop       & 89.2 {$\pm$ \scriptsize 1.3} & 90.8 {$\pm$ \scriptsize 1.0}  &90.4 {$\pm$ \scriptsize 1.3}  &90.5 {$\pm$ \scriptsize 1.2} & 82.0 {$\pm$ \scriptsize 2.4} & 44.3 {$\pm$ \scriptsize 3.3}  & 82.7 {$\pm$ \scriptsize 3.9} &52.3 {$\pm$ \scriptsize 3.9}   \\
PrimeNet      &85.5 {$\pm$ \scriptsize 1.5}  & 87.8 {$\pm$ \scriptsize 1.2}  & 87.1 {$\pm$ \scriptsize 1.1} & 87.1 {$\pm$ \scriptsize 1.2} &\underline{85.1} {$\pm$ \scriptsize 0.8}  &\underline{49.3} {$\pm$ \scriptsize 1.9} &80.3 {$\pm$ \scriptsize 0.5}  &31.6 {$\pm$ \scriptsize 0.9}    \\
ContiFormer &66.6 {$\pm$ \scriptsize 1.8}   &68.6 {$\pm$ \scriptsize 1.7}   &69.7 {$\pm$ \scriptsize 1.5}   & 67.4 {$\pm$ \scriptsize 1.7}  & 72.1 {$\pm$ \scriptsize 0.4}  & 29.6 {$\pm$ \scriptsize 0.8}  & 80.7 {$\pm$ \scriptsize 0.3}  & 34.7 {$\pm$ \scriptsize 1.9}   \\
ViTST       &\underline{95.2} {$\pm$ \scriptsize 1.4}  &\underline{95.8} {$\pm$ \scriptsize 1.3}  & \underline{96.1} {$\pm$ \scriptsize 1.1} & \underline{95.9} {$\pm$ \scriptsize 1.2} & 84.2 {$\pm$ \scriptsize 1.1} & 43.2 {$\pm$ \scriptsize 2.4} &\underline{89.3} {$\pm$ \scriptsize 0.2}  &\underline{53.8} {$\pm$ \scriptsize 1.1}    \\
\midrule
Ours       &\textbf{98.3} {$\pm$ \scriptsize 0.3}  &\textbf{98.7} {$\pm$ \scriptsize 0.6}  & \textbf{98.4} {$\pm$ \scriptsize 1.0} & \textbf{98.5} {$\pm$ \scriptsize 0.7} &\textbf{86.0} {$\pm$ \scriptsize 0.3}  &\textbf{50.4} {$\pm$ \scriptsize 2.1} &\textbf{91.6} {$\pm$ \scriptsize 0.9} & \textbf{59.6} {$\pm$ \scriptsize 1.3}    \\
\bottomrule
\end{tabular}
\caption{Comparison with state-of-the-art baselines on irregularly sampled time series classification. We use \textbf{bold} to indicate the best results and \underline{underline} for the second best one.}
\label{baselines}
\end{table*}



In the experiments, we consider three real-world irregular clinical datasets as shown in Table \ref{dataset_statistics}. The physical activity monitoring (PAM) dataset \cite{reiss2012introducing} focuses on tracking human activities, containing data from eight person who performed nine different actions. This dataset comprises 5,333 samples and captures data from four types of sensors placed at three distinct body locations, encompassing a total of 17 observational variables. The P12 dataset \cite{goldberger2000physiobank} includes 11,988 patient samples from ICU stays, with 36 measurements each. The binary labels indicate the prognosis for each sample as either survival or not. Finally, the P19 dataset \cite{reyna2020early} contains data from 38,803 sepsis patients, each with 34 measurements, and a high missing rate of 94.9\%. Approximately 90\% of these patients died due to sepsis.

To maintain consistency across all experiments, we follow the same data partition as \cite{zhang2021graph, li2024time}, dividing the datasets into training, validation, and testing sets in an 8:1:1 ratio. For the PAM dataset, we use Accuracy, Precision, Recall, and F1 score as evaluation metrics. For the more imbalanced P12 and P19 datasets, we report the Area Under the ROC Curve (AUROC) and the Area Under the Precision-Recall Curve (AUPRC). For more experimental results that are not included in this section, we present them in Appendix D. 


\subsection{Implementation and Training}
We use Gated Recurrent Units (GRU) \cite{dey2017gate} in both our generator and discriminator. The generator has 4 layers, with the number of units fixed at 128. The discriminator is a 5-layer RNN and the number of units is set to $\{$128, 64, 16, 64, 128$\}$, respectively. A checkpoint pre-trained on ImageNet-21K dataset are utilized for our image encoder. The patch size and window size are 4 and 7. For the P12 and P19 datasets, all images are set to a size of 384 $\times$ 384 pixels. While for the PAM dataset, line graph and frequency spectrum are configured to 256$ \times$ 320, while all other images are set to 320 $\times$ 320. We use a 3-layer MLP as our joint projection, with the number of units set to $\{$1024, 512, 1024$\}$. 
% For hyperparameters, the temperature parameter \( \tau \) for both the reconstruction loss, contrastive loss, and clustering loss is set at 1.2. The margin \( m \), \( \alpha \), \( \beta_1 \) and \( \beta_2 \) is specified at 0.05, 4, 0.1, and 0.2, respectively. We discuss the selection of these hyperparameters in Appendix B. 

For the P12 and P19 datasets, the total epoch is set to 8 and we apply upsampling of the minority class to mitigate imbalance. For the PAM dataset, we set the total epoch to 40. The batch sizes used for training are 32 for P19 and P12, and 48 for PAM. For each dataset, we discuss the learning rate as well as more hyperparameter settings in Appendix B. All experiments are performed on a server with NVIDIA GeForce RTX 3090 24GB and PyTorch 2.4.0+cu124. 

\subsection{Results}

% SSGAN \cite{miao2021generative}
\subsubsection{Comparison with state-of-the-art methods.} We compare our approach against seven state-of-the-art methods for irregularly sampled time series, including GRU-D \cite{che2018recurrent}, SeFT \cite{horn2020set}, CARD \cite{han2022card}, Raindrop \cite{zhang2021graph}, PrimeNet \cite{chowdhury2023primenet}, ContiFormer \cite{chen2024contiformer}, and ViTST \cite{li2024time}. For each baseline, we introduce our implementation and hyperparameter settings in Appendix C. To ensure a fair evaluation, we average the performance of each method across five individual tests, using the same data splits and settings provided in \cite{li2024time}. 

Table \ref{baselines} presents the comparison results, highlighting that our approach outperforms the other seven state-of-the-art methods across all three datasets. Specifically, we achieve a significant improvement on the PAM datasets, with an increase of 3.1\% in Accuracy, 2.9\% in Precision, 2.3\% in recall, and 2.6\% in F1 score. For the P12 and P19 datasets, our approach shows improved performance in predicting minority classes, with an increase of 0.9\%, 2.3\% in absolute AUROC points, and 1.1\%, 5.8\%  in absolute AUPRC, respectively. 

% \textcolor{red}{We compare the number of parameters between the baseline methods and ours, as shown in Table 3. It can be seen that the largest model, Raindrop, has a parameter count of 150M and all methods belong to small or medium-sized models. Therefore, it will not result in significant resource consumption. Methods like GRU-D and Contiformer, which only perform sequence modeling, have relatively fewer parameters, whereas methods involving image representation, such as ViTST and ours, have much larger parameter counts.}

\subsubsection{Performance under increased missing rates.}

\begin{figure*}[htp]
    \centering
    \includegraphics[width=0.95\linewidth]{figure/missing_test.png}
    \caption{Performance under increased missingness: (a) leave-sensors-out and (b) leave-samples-out on the PAM dataset. Tests are conducted with 10\%-50\% extra missing values.} 
    \label{missing test}
\end{figure*}



To further validate the robustness of our approach, we conduct additional experiments to compare the performance under increased levels of missing rate. Given that the P12 and P19 datasets have already faced very high missing rates—88.4\% and 94.9\% respectively, we conduct all the tests on the PAM dataset, which originally has a missing rate of 60\%. We conducted two types of tests: the leave-sensors-out setting, simulating scenarios where certain medical tests are not performed, and the leave-samples-out setting, reflecting situations where patients join or leave treatments midway. We follow the approach in \cite{zhang2021graph}, applying all modifications only to the test set by randomly masking the original observations.

As shown in Figure \ref{missing test}, our approach consistently achieves the best performance in all settings. For the leave-sensors-out tests, as the missing ratio increase from 10\% to 50\%, our approach exhibit the least performance decline. Even in the most extreme scenario, where 50\% of the sensors (9 sensors) are masked, all our metrics remain above 80\%. Compared to the second-best method, ViTST, our approach outperform by 6.1\%, 5.9\%, 3.4\%, and 4.6\% in Accuracy, Precision, Recall, and F1 score, respectively. The margins are even more significant compared to the third-ranked Raindrop, with improvements of 27.4\%, 39.8\%, 29.9\%, and 37.5\% in the same metrics. For the leave-samples-out setting, we randomly sampled and masked time steps. Overall, only CARD experienced significant decline as missing rate increases, while most models shows relatively minor decline, indicating that they effectively capture the temporal relationships between time steps. In terms of absolute performance, our model outperform the second-best, ViTST, by 5.7\% in Accuracy, 3.8\% in Precision, 5.4\% in Recall, and 4.8\% in F1 score at a 50\% missing rate. 



\subsubsection{Clinical Turing tests.}
To ensure that the learned representations align with clinically meaningful patterns rather than statistical artifacts, we conducted a clinical Turing test on the generated signals, as described in \cite{gillette2023medalcare}. Specifically, we select 60 samples from the P19 (ICU) dataset, with half imputed using linear interpolation as real measured samples and the other half imputed using our model as generated samples. Five ICU-experienced clinicians (3 chief physicians and 2 attending physicians) attempt to distinguish between the two types. As shown in Table \ref{expert}, the experts achieve prediction accuracy of 50.0\%, 48.3\%, 48.3\%, 58.3\%, and 60.0\%, resulting in a kappa score of -0.03. These results are close to random guessing, suggesting that the experts generally struggle to differentiate between the samples. A brief interview further revealed why experts struggled to identify clear patterns to distinguish real from generated samples. One reason is that the complex events in the ICU environment make the data distribution more tolerant. For example, sedation or anesthesia can cause body temperature to fall below the usual range. 


\begin{table}[t]
\setlength{\heavyrulewidth}{1.2pt}
% \renewcommand{\arraystretch}{1}
\setlength{\tabcolsep}{1mm}
\centering
% \resizebox{0.98\columnwidth}{!}{ 
\begin{tabular}{c ccccc}
\toprule
\multirow{2}{*}{Experts}  & \multicolumn{5}{c}{P19} \\ 
\cmidrule{2-6} 
 & Accuracy & Precision & Recall & F1 score & Specificity\\
\midrule
P1  &50.0  &47.4  &64.3  &54.5 &0.38 \\
P2  &48.3  &44.8  &46.4  &45.6 &0.50 \\
P3  &48.3  &45.5  &53.6  &49.2 &0.44 \\
P4  &58.3  &55.2  &57.1  &56.1 &0.59 \\
P5  &60.0  &57.1  &57.1  &57.1 &0.63 \\
\bottomrule
\end{tabular}
% }
\caption{Performance metrics of five ICU-experienced medical experts.}
\label{expert}
\end{table}


\subsubsection{Ablation study.}
In this section, we present the results of our ablation study in Table \ref{ablation}. The ``default'' one is our standard setup, which includes the sequence encoder, the image encoder, and the joint representation module, along with three self-supervised learning strategies. In the first part of Table \ref{ablation}, we evaluate the performance of individual components: ``image'' signifies that only the image encoder is used for classification, whereas ``sequence'' denotes the use of only the sequence encoder. As a result, we verify that incorporating both sequence and image information significantly improves classification performance, with F1 scores increasing by 2.6\% and 4.5\%. ``sequence-MSE'' denotes the use of MSE loss as the reconstruction loss. In contrast, by replacing it with NT-Xent, we achieved improvements in Accuracy, Precision, Recall, and F1 score by 0.8\%, 0.6\%, 0.2\%, and 0.5\%, respectively. 

The second part of Table \ref{ablation} focuses on our joint representation module. In the ``concatenation'' setting, we simply concatenate sequence and image representations for further downstream classification, and the performance is slightly higher than either ``sequence'' and ``image''. The ``contrastive'' setting shows the improvement from contrastive learning strategy, with 1.1\%, 0.9\%, 1.3\%, and 1.0\% in Accuracy, Precision, Recall, and F1 score. ``Clustering'' strategy also shows positive performance, with 1.2\% in Accuracy, 0.7\% in Precision, 0.9\% in Recall and 0.8\% in F1 score. 





\begin{table}[t]
\setlength{\heavyrulewidth}{1.2pt}
% \renewcommand{\arraystretch}{1}
\centering
\setlength{\tabcolsep}{0.8mm}
% \resizebox{0.98\columnwidth}{!}{ 
\begin{tabular}{c cccc}
\toprule
\multirow{2}{*}{Methods}  & \multicolumn{4}{c}{PAM} \\ 
\cmidrule{2-5} 
 & Accuracy & Precision & Recall & F1 score \\
\midrule
image  &95.4 {$\pm$ \scriptsize 0.6}  &96.5 {$\pm$ \scriptsize 0.6}   &95.4 {$\pm$ \scriptsize 0.4}  &95.9 {$\pm$ \scriptsize 0.5} \\
sequence &93.3 {$\pm$ \scriptsize 1.1}  & 94.4 {$\pm$ \scriptsize 0.7} &93.6 {$\pm$ \scriptsize 0.6}  & 94.0 {$\pm$ \scriptsize 0.7} \\
sequence-MSE &92.5 {$\pm$ \scriptsize 0.4}  &93.8 {$\pm$ \scriptsize 0.4}  & 93.4 {$\pm$ \scriptsize 0.4} & 93.5 {$\pm$ \scriptsize 0.4} \\
\midrule
concatenation  &95.7 {$\pm$ \scriptsize 0.7}  &96.7 {$\pm$ \scriptsize 0.5}  &96.1 {$\pm$ \scriptsize 0.4}  &96.5 {$\pm$ \scriptsize 0.5}  \\
contrastive  & 96.8 {$\pm$ \scriptsize 0.7} & 97.6 {$\pm$ \scriptsize 0.5} & 97.4 {$\pm$ \scriptsize 0.7} &97.5 {$\pm$ \scriptsize 0.5}  \\
clustering & 96.9 {$\pm$ \scriptsize 0.3} & 97.4 {$\pm$ \scriptsize 0.6} & 97.0 {$\pm$ \scriptsize 0.5}  &97.3 {$\pm$ \scriptsize 0.6}  \\
\midrule
default  &98.3 {$\pm$ \scriptsize 0.3}  &98.7 {$\pm$ \scriptsize 0.6}  & 98.4 {$\pm$ \scriptsize 1.0} & 98.5 {$\pm$ \scriptsize 0.7} \\
\bottomrule
\end{tabular}
% }
\caption{Ablation studies on different strategies.}
\label{ablation}
\end{table}

\section{Conclusion}

In this paper, we propose a joint learning approach of leveraging both sequence and image representations to tackle the classification of irregularly sampled clinical time series. By employing our three self-supervised learning strategies, we are able to effectively learn more generalized joint representations. The effectiveness of our approach is verified on three real-world clinical datasets, where it demonstrates superior performance compared to seven state-of-the-art methods. Additionally, we test our approach under more severe missing rates using leave-sensors-out and leave-samples-out techniques. Our approach consistently achieved strong results, demonstrating its robustness in these scenarios. Our code and data will be made publicly available later. 

\section*{Acknowledgements}
We express our sincere gratitude to all the anonymous reviewers for their valuable guidance and suggestions. We also thank Doctor Weihang Hu, Lin Zhang, and all the colleagues from the Intensive Care Unit at Zhejiang Hospital for their contributions to the expert evaluation and for providing us with valuable clinical advice.

\begin{thebibliography}{44}
    \providecommand{\natexlab}[1]{#1}
    
    \bibitem[{Ao and He(2023)}]{ao2023image}
    Ao, R.; and He, G. 2023.
    \newblock Image based deep learning in 12-lead ECG diagnosis.
    \newblock \emph{Frontiers in Artificial Intelligence}, 5: 1087370.
    
    \bibitem[{Brizzi et~al.(2022)Brizzi, Whittaker, Servo, Hawryluk, Prete~Jr, de~Souza, Aguiar, Araujo, Bastos, Blenkinsop et~al.}]{brizzi2022spatial}
    Brizzi, A.; Whittaker, C.; Servo, L.~M.; Hawryluk, I.; Prete~Jr, C.~A.; de~Souza, W.~M.; Aguiar, R.~S.; Araujo, L.~J.; Bastos, L.~S.; Blenkinsop, A.; et~al. 2022.
    \newblock Spatial and temporal fluctuations in COVID-19 fatality rates in Brazilian hospitals.
    \newblock \emph{Nature medicine}, 28(7): 1476--1485.
    
    \bibitem[{Chaudhary et~al.(2020)Chaudhary, Vaid, Duffy, Paranjpe, Jaladanki, Paranjpe, Johnson, Gokhale, Pattharanitima, Chauhan et~al.}]{chaudhary2020utilization}
    Chaudhary, K.; Vaid, A.; Duffy, {\'A}.; Paranjpe, I.; Jaladanki, S.; Paranjpe, M.; Johnson, K.; Gokhale, A.; Pattharanitima, P.; Chauhan, K.; et~al. 2020.
    \newblock Utilization of deep learning for subphenotype identification in sepsis-associated acute kidney injury.
    \newblock \emph{Clinical Journal of the American Society of Nephrology}, 15(11): 1557--1565.
    
    \bibitem[{Che et~al.(2018)Che, Purushotham, Cho, Sontag, and Liu}]{che2018recurrent}
    Che, Z.; Purushotham, S.; Cho, K.; Sontag, D.; and Liu, Y. 2018.
    \newblock Recurrent neural networks for multivariate time series with missing values.
    \newblock \emph{Scientific reports}, 8(1): 6085.
    
    \bibitem[{Chen et~al.(2020)Chen, Kornblith, Swersky, Norouzi, and Hinton}]{chen2020big}
    Chen, T.; Kornblith, S.; Swersky, K.; Norouzi, M.; and Hinton, G.~E. 2020.
    \newblock Big self-supervised models are strong semi-supervised learners.
    \newblock \emph{Advances in neural information processing systems}, 33: 22243--22255.
    
    \bibitem[{Chen et~al.(2024)Chen, Ren, Wang, Fang, Sun, and Li}]{chen2024contiformer}
    Chen, Y.; Ren, K.; Wang, Y.; Fang, Y.; Sun, W.; and Li, D. 2024.
    \newblock Contiformer: Continuous-time transformer for irregular time series modeling.
    \newblock \emph{Advances in Neural Information Processing Systems}, 36.
    
    \bibitem[{Choi et~al.(2016)Choi, Bahadori, Schuetz, Stewart, and Sun}]{choi2016doctor}
    Choi, E.; Bahadori, M.~T.; Schuetz, A.; Stewart, W.~F.; and Sun, J. 2016.
    \newblock Doctor ai: Predicting clinical events via recurrent neural networks.
    \newblock In \emph{Machine learning for healthcare conference}, 301--318. PMLR.
    
    \bibitem[{Chong et~al.(2011)}]{chong2011signal}
    Chong, U.-P.; et~al. 2011.
    \newblock Signal model-based fault detection and diagnosis for induction motors using features of vibration signal in two-dimension domain.
    \newblock \emph{Strojni{\v{s}}ki vestnik}, 57(9): 655--666.
    
    \bibitem[{Chowdhury et~al.(2023)Chowdhury, Li, Zhang, Hong, Gupta, and Shang}]{chowdhury2023primenet}
    Chowdhury, R.~R.; Li, J.; Zhang, X.; Hong, D.; Gupta, R.~K.; and Shang, J. 2023.
    \newblock Primenet: Pre-training for irregular multivariate time series.
    \newblock In \emph{Proceedings of the AAAI Conference on Artificial Intelligence}, volume~37, 7184--7192.
    
    \bibitem[{de~Jong et~al.(2019)de~Jong, Emon, Wu, Karki, Sood, Godard, Ahmad, Vrooman, Hofmann-Apitius, and Fr{\"o}hlich}]{de2019deep}
    de~Jong, J.; Emon, M.~A.; Wu, P.; Karki, R.; Sood, M.; Godard, P.; Ahmad, A.; Vrooman, H.; Hofmann-Apitius, M.; and Fr{\"o}hlich, H. 2019.
    \newblock Deep learning for clustering of multivariate clinical patient trajectories with missing values.
    \newblock \emph{GigaScience}, 8(11): giz134.
    
    \bibitem[{Deng et~al.(2023)Deng, Hua, Yingjun, Fanyue, and Di}]{bs2023_1730}
    Deng, Y.; Hua, M.; Yingjun, R.; Fanyue, Q.; and Di, P. 2023.
    \newblock An image characterisation method for AHU fault diagnosis based on residual neural networks.
    \newblock In \emph{Proceedings of Building Simulation 2023: 18th Conference of IBPSA}, volume~18 of \emph{Building Simulation}, 3827--3834. Shanghai, China: IBPSA.
    
    \bibitem[{Dey and Salem(2017)}]{dey2017gate}
    Dey, R.; and Salem, F.~M. 2017.
    \newblock Gate-variants of gated recurrent unit (GRU) neural networks.
    \newblock In \emph{2017 IEEE 60th international midwest symposium on circuits and systems (MWSCAS)}, 1597--1600. IEEE.
    
    \bibitem[{Gillette et~al.(2023)Gillette, Gsell, Nagel, Bender, Winkler, Williams, B{\"a}r, Sch{\"a}ffter, D{\"o}ssel, Plank et~al.}]{gillette2023medalcare}
    Gillette, K.; Gsell, M.~A.; Nagel, C.; Bender, J.; Winkler, B.; Williams, S.~E.; B{\"a}r, M.; Sch{\"a}ffter, T.; D{\"o}ssel, O.; Plank, G.; et~al. 2023.
    \newblock MedalCare-XL: 16,900 healthy and pathological synthetic 12 lead ECGs from electrophysiological simulations.
    \newblock \emph{Scientific Data}, 10(1): 531.
    
    \bibitem[{Goldberger et~al.(2000)Goldberger, Amaral, Glass, Hausdorff, Ivanov, Mark, Mietus, Moody, Peng, and Stanley}]{goldberger2000physiobank}
    Goldberger, A.~L.; Amaral, L.~A.; Glass, L.; Hausdorff, J.~M.; Ivanov, P.~C.; Mark, R.~G.; Mietus, J.~E.; Moody, G.~B.; Peng, C.-K.; and Stanley, H.~E. 2000.
    \newblock PhysioBank, PhysioToolkit, and PhysioNet: components of a new research resource for complex physiologic signals.
    \newblock \emph{circulation}, 101(23): e215--e220.
    
    \bibitem[{Goodfellow et~al.(2020)Goodfellow, Pouget-Abadie, Mirza, Xu, Warde-Farley, Ozair, Courville, and Bengio}]{goodfellow2020generative}
    Goodfellow, I.; Pouget-Abadie, J.; Mirza, M.; Xu, B.; Warde-Farley, D.; Ozair, S.; Courville, A.; and Bengio, Y. 2020.
    \newblock Generative adversarial networks.
    \newblock \emph{Communications of the ACM}, 63(11): 139--144.
    
    \bibitem[{Han, Zheng, and Zhou(2022)}]{han2022card}
    Han, X.; Zheng, H.; and Zhou, M. 2022.
    \newblock Card: Classification and regression diffusion models.
    \newblock \emph{Advances in Neural Information Processing Systems}, 35: 18100--18115.
    
    \bibitem[{Harutyunyan et~al.(2019)Harutyunyan, Khachatrian, Kale, Ver~Steeg, and Galstyan}]{harutyunyan2019multitask}
    Harutyunyan, H.; Khachatrian, H.; Kale, D.~C.; Ver~Steeg, G.; and Galstyan, A. 2019.
    \newblock Multitask learning and benchmarking with clinical time series data.
    \newblock \emph{Scientific data}, 6(1): 96.
    
    \bibitem[{Hatami, Gavet, and Debayle(2018)}]{hatami2018classification}
    Hatami, N.; Gavet, Y.; and Debayle, J. 2018.
    \newblock Classification of time-series images using deep convolutional neural networks.
    \newblock In \emph{Tenth international conference on machine vision (ICMV 2017)}, volume 10696, 242--249. SPIE.
    
    \bibitem[{Horn et~al.(2020)Horn, Moor, Bock, Rieck, and Borgwardt}]{horn2020set}
    Horn, M.; Moor, M.; Bock, C.; Rieck, B.; and Borgwardt, K. 2020.
    \newblock Set functions for time series.
    \newblock In \emph{International Conference on Machine Learning}, 4353--4363. PMLR.
    
    \bibitem[{Huang et~al.(2024)Huang, Yang, Yin, and Xu}]{huang2024dna}
    Huang, J.; Yang, B.; Yin, K.; and Xu, J. 2024.
    \newblock DNA-T: Deformable Neighborhood Attention Transformer for Irregular Medical Time Series.
    \newblock \emph{IEEE Journal of Biomedical and Health Informatics}.
    
    \bibitem[{Li, Torr, and Lukasiewicz(2022)}]{li2022clustering}
    Li, B.; Torr, P.~H.; and Lukasiewicz, T. 2022.
    \newblock Clustering generative adversarial networks for story visualization.
    \newblock In \emph{Proceedings of the 30th ACM International Conference on Multimedia}, 769--778.
    
    \bibitem[{Li, Li, and Yan(2024)}]{li2024time}
    Li, Z.; Li, S.; and Yan, X. 2024.
    \newblock Time series as images: Vision transformer for irregularly sampled time series.
    \newblock \emph{Advances in Neural Information Processing Systems}, 36.
    
    \bibitem[{Lipton et~al.(2016)Lipton, Kale, Wetzel et~al.}]{lipton2016modeling}
    Lipton, Z.~C.; Kale, D.~C.; Wetzel, R.; et~al. 2016.
    \newblock Modeling missing data in clinical time series with rnns.
    \newblock \emph{Machine Learning for Healthcare}, 56(56): 253--270.
    
    \bibitem[{Liu et~al.(2022)Liu, Hu, Lin, Yao, Xie, Wei, Ning, Cao, Zhang, Dong, Wei, and Guo}]{liu2021swinv2}
    Liu, Z.; Hu, H.; Lin, Y.; Yao, Z.; Xie, Z.; Wei, Y.; Ning, J.; Cao, Y.; Zhang, Z.; Dong, L.; Wei, F.; and Guo, B. 2022.
    \newblock Swin Transformer V2: Scaling Up Capacity and Resolution.
    \newblock In \emph{International Conference on Computer Vision and Pattern Recognition (CVPR)}.
    
    \bibitem[{Ma, Li, and Cottrell(2020)}]{ma2020adversarial}
    Ma, Q.; Li, S.; and Cottrell, G.~W. 2020.
    \newblock Adversarial joint-learning recurrent neural network for incomplete time series classification.
    \newblock \emph{IEEE Transactions on Pattern Analysis and Machine Intelligence}, 44(4): 1765--1776.
    
    \bibitem[{Ma et~al.(2019)Ma, Zheng, Li, and Cottrell}]{ma2019learning}
    Ma, Q.; Zheng, J.; Li, S.; and Cottrell, G.~W. 2019.
    \newblock Learning representations for time series clustering.
    \newblock \emph{Advances in neural information processing systems}, 32.
    
    \bibitem[{Maroor et~al.(2024)Maroor, Sahu, Nijhawan, Karthik, Shrivastav, and Chakravarthi}]{maroor2024image}
    Maroor, J.~P.; Sahu, D.~N.; Nijhawan, G.; Karthik, A.; Shrivastav, A.; and Chakravarthi, M.~K. 2024.
    \newblock Image-Based Time Series Forecasting: A Deep Convolutional Neural Network Approach.
    \newblock In \emph{2024 4th International Conference on Innovative Practices in Technology and Management (ICIPTM)}, 1--6. IEEE.
    
    \bibitem[{Miao et~al.(2021)Miao, Wu, Wang, Gao, Mao, and Yin}]{miao2021generative}
    Miao, X.; Wu, Y.; Wang, J.; Gao, Y.; Mao, X.; and Yin, J. 2021.
    \newblock Generative semi-supervised learning for multivariate time series imputation.
    \newblock In \emph{Proceedings of the AAAI conference on artificial intelligence}, volume~35, 8983--8991.
    
    \bibitem[{Neil, Pfeiffer, and Liu(2016)}]{neil2016phased}
    Neil, D.; Pfeiffer, M.; and Liu, S.-C. 2016.
    \newblock Phased lstm: Accelerating recurrent network training for long or event-based sequences.
    \newblock \emph{Advances in neural information processing systems}, 29.
    
    \bibitem[{Raghu et~al.(2023)Raghu, Chandak, Alam, Guttag, and Stultz}]{raghu2023sequential}
    Raghu, A.; Chandak, P.; Alam, R.; Guttag, J.; and Stultz, C. 2023.
    \newblock Sequential multi-dimensional self-supervised learning for clinical time series.
    \newblock In \emph{International Conference on Machine Learning}, 28531--28548. PMLR.
    
    \bibitem[{Reiss and Stricker(2012)}]{reiss2012introducing}
    Reiss, A.; and Stricker, D. 2012.
    \newblock Introducing a new benchmarked dataset for activity monitoring.
    \newblock In \emph{2012 16th international symposium on wearable computers}, 108--109. IEEE.
    
    \bibitem[{Reyna et~al.(2020)Reyna, Josef, Jeter, Shashikumar, Westover, Nemati, Clifford, and Sharma}]{reyna2020early}
    Reyna, M.~A.; Josef, C.~S.; Jeter, R.; Shashikumar, S.~P.; Westover, M.~B.; Nemati, S.; Clifford, G.~D.; and Sharma, A. 2020.
    \newblock Early prediction of sepsis from clinical data: the PhysioNet/Computing in Cardiology Challenge 2019.
    \newblock \emph{Critical care medicine}, 48(2): 210--217.
    
    \bibitem[{Sangha et~al.(2024)Sangha, Khunte, Holste, Mortazavi, Wang, Oikonomou, and Khera}]{sangha2024biometric}
    Sangha, V.; Khunte, A.; Holste, G.; Mortazavi, B.~J.; Wang, Z.; Oikonomou, E.~K.; and Khera, R. 2024.
    \newblock Biometric contrastive learning for data-efficient deep learning from electrocardiographic images.
    \newblock \emph{Journal of the American Medical Informatics Association}, 31(4): 855--865.
    
    \bibitem[{Sangha et~al.(2022)Sangha, Mortazavi, Haimovich, Ribeiro, Brandt, Jacoby, Schulz, Krumholz, Ribeiro, and Khera}]{sangha2022automated}
    Sangha, V.; Mortazavi, B.~J.; Haimovich, A.~D.; Ribeiro, A.~H.; Brandt, C.~A.; Jacoby, D.~L.; Schulz, W.~L.; Krumholz, H.~M.; Ribeiro, A. L.~P.; and Khera, R. 2022.
    \newblock Automated multilabel diagnosis on electrocardiographic images and signals.
    \newblock \emph{Nature communications}, 13(1): 1583.
    
    \bibitem[{Semenoglou, Spiliotis, and Assimakopoulos(2023)}]{semenoglou2023image}
    Semenoglou, A.-A.; Spiliotis, E.; and Assimakopoulos, V. 2023.
    \newblock Image-based time series forecasting: A deep convolutional neural network approach.
    \newblock \emph{Neural Networks}, 157: 39--53.
    
    \bibitem[{Sood et~al.(2021)Sood, Zeng, Cohen, Balch, and Veloso}]{sood2021visual}
    Sood, S.; Zeng, Z.; Cohen, N.; Balch, T.; and Veloso, M. 2021.
    \newblock Visual time series forecasting: an image-driven approach.
    \newblock In \emph{Proceedings of the Second ACM International Conference on AI in Finance}, 1--9.
    
    \bibitem[{Tripathy and Acharya(2018)}]{tripathy2018use}
    Tripathy, R.; and Acharya, U.~R. 2018.
    \newblock Use of features from RR-time series and EEG signals for automated classification of sleep stages in deep neural network framework.
    \newblock \emph{Biocybernetics and Biomedical Engineering}, 38(4): 890--902.
    
    \bibitem[{Wang et~al.(2024)Wang, Du, Cao, Zhang, Wang, Liang, and Wen}]{wang2024deep}
    Wang, J.; Du, W.; Cao, W.; Zhang, K.; Wang, W.; Liang, Y.; and Wen, Q. 2024.
    \newblock Deep learning for multivariate time series imputation: A survey.
    \newblock \emph{arXiv preprint arXiv:2402.04059}.
    
    \bibitem[{Wang and Oates(2015)}]{10.5555/2832747.2832798}
    Wang, Z.; and Oates, T. 2015.
    \newblock Imaging time-series to improve classification and imputation.
    \newblock In \emph{Proceedings of the 24th International Conference on Artificial Intelligence}, IJCAI'15, 3939–3945. AAAI Press.
    \newblock ISBN 9781577357384.
    
    \bibitem[{Weerakody, Wong, and Wang(2023)}]{weerakody2023policy}
    Weerakody, P.~B.; Wong, K.~W.; and Wang, G. 2023.
    \newblock Policy gradient empowered LSTM with dynamic skips for irregular time series data.
    \newblock \emph{Applied Soft Computing}, 142: 110314.
    
    \bibitem[{Wu et~al.(2023)Wu, Hu, Liu, Zhou, Wang, and Long}]{wu2023timesnet}
    Wu, H.; Hu, T.; Liu, Y.; Zhou, H.; Wang, J.; and Long, M. 2023.
    \newblock TimesNet: Temporal 2D-Variation Modeling for General Time Series Analysis.
    \newblock In \emph{International Conference on Learning Representations}.
    
    \bibitem[{Xu et~al.(2024)Xu, Xu, Yang, and Chen}]{xu2024learning}
    Xu, S.; Xu, T.; Yang, Y.; and Chen, X. 2024.
    \newblock Learning metabolic dynamics from irregular observations by Bidirectional Time-Series State Transfer Network.
    \newblock \emph{mSystems}, e00697--24.
    
    \bibitem[{Zhang et~al.(2023)Zhang, Zheng, Cao, Bian, and Li}]{zhang2023warpformer}
    Zhang, J.; Zheng, S.; Cao, W.; Bian, J.; and Li, J. 2023.
    \newblock Warpformer: A multi-scale modeling approach for irregular clinical time series.
    \newblock In \emph{Proceedings of the 29th ACM SIGKDD Conference on Knowledge Discovery and Data Mining}, 3273--3285.
    
    \bibitem[{Zhang et~al.(2022)Zhang, Zeman, Tsiligkaridis, and Zitnik}]{zhang2021graph}
    Zhang, X.; Zeman, M.; Tsiligkaridis, T.; and Zitnik, M. 2022.
    \newblock Graph-Guided Network For Irregularly Sampled Multivariate Time Series.
    \newblock In \emph{International Conference on Learning Representations, ICLR}.
    
    \end{thebibliography}
    




\end{document}
