\section{Related Work}
Existing agentic designs, including recent frameworks overfit to a narrow spectrum of tasks in data science by assuming an overall workflow \citep{guo2024ds,wu2023autogen,baek2024researchagent,grosnit2024large}. Prior work has looked at Graph of Thoughts \citep{besta2024graph} to tackle reasoning problems (different from AutoML), but do not enforce the notion of ``options'' or option diversity. For the most closely related works, SELA \citep{chi2024sela} and Data Interpreter \citep{hong2024data}, we discussed data structure differences in Section \ref{subsec:FoO-benefit}. We highlight some additional framework-level differences here. 
In contrast to our work, SELA \citep{chi2024sela} predefines steps in the pipeline within the LLM prompt, e.g., data analysis, feature engineering, and model selection. Data Interpreter predefines a set of ``task types'' with a detailed set of instructions for each task type. This ranges from specific prompts for ``feature engineering'' tasks and ``model training'' tasks, to highly specific prompts for ``image-to-web conversion'' tasks. This limits their applicability to broader problems. Additionally, SELA and Data Interpreter do not leverage past experiences for new tasks. 

For therapeutic tasks, prior work has introduced DrugAgent \citep{liu2024drugagent}. Similar in spirit to our work, DrugAgent explores an ``idea space'' of different models but is specifically designed for the task of drug discovery only and does not use structured thinking. In contrast, Flow-of-Options offers improved flexibility and performance.