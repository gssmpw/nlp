\section{ATROC Framework for Edge Intelligence}
\label{sec:aligned_TOC_framework}
\begin{figure}[t]
    \begin{center}
    \includegraphics[width=1\linewidth]{Figure/IB_framework.pdf}
    \end{center}
    \caption{Comparison of three JSCC-enabled communication frameworks for edge inference: Reconstruction-oriented, non-aligned task-oriented, and ATROC frameworks. All three frameworks can share a similar JSCC encoder structure on the device side. On the edge side, reconstruction-oriented communication aims to fully reconstruct the input data, including both task-specific and task-agnostic information. In contrast, non-aligned task-oriented communication focuses solely on preserving task-specific information and uses JSCC symbols directly for inference. ATROC merges the benefits of the previous two by transferring task-specific information and ensuring that data structures are compatible with existing AI agent networks, enhancing integration and efficiency.}
    \label{fig:IB_framework}
\end{figure}
Edge intelligence refers to an AI agent (system) that operates at edge servers rather than relying on centralized servers or cloud-based services. These systems process data locally on devices or at the edge of the network as shown in \cref{fig:IB_framework}. 

The reconstruction-oriented communication framework (see \cref{fig:IB_framework}a) aims to preserve all information from the input data $\bm{x}$ in the reconstructed data $\hat{\bm{x}}$. The idea is to minimize the distance $d(\bm{x}, \hat{\bm{x}})$, where $d(\cdot,\cdot)$ is a predefined data-centric metric. This task-agnostic strategy may result in transmitting redundant data for AI agents, leading to poor resource utilization efficiency.

To improve efficiency, the principle of \gls{ib} has been developed to transmit task-relevant information \cite{Shao_2022_LTO}, as shown in \cref{fig:IB_framework}b. 
However, a significant challenge arises with this approach: the dimensions of the received symbols often do not align with the input dimensions required by the edge AI agent. This mismatch necessitates a redesign of the AI agent to accommodate different input sizes, leading to poor compatibility.

To address this, we propose an \gls{atroc} framework, as depicted in \cref{fig:IB_framework}c, enabling the use of a unified inference network across both task-oriented and reconstruction-oriented communication paradigms.

In this framework, the \gls{jscc} encoder deployed on the mobile device is denoted by $p_\phi(\bm{z}|\bm{x})$, where $\phi$ represents the parameters. The encoder maps the input data $\bm{x} \in \mathbb{R}^{l}$ to \gls{jscc} symbols $\bm{z} \in \mathbb{C}^{k}$, where $\bm{z}=[z_1, \cdots, z_{k}]$. Here, $l$ and $k$ are the dimensions of the input data and the \gls{jscc} symbols, respectively. After quantization and power normalization, the \gls{jscc} symbols $\bm{z}$ are transmitted through a physical channel. In this paper, we model the communications between the mobile device and the edge server as Gaussian or Rayleigh fading channels:
\begin{align}
    \bm{z}_{\text{out}} = h \cdot \bm{z}_{\text{in}} + \bm{n},
    \label{eq_channel}
\end{align}
where $\bm{z}_{\text{in}}$ represents channel input and $\bm{z}_{\text{out}}$ represents channel output. $\bm{n} \sim \mathcal{CN}(0, \bm{\sigma}_{n}^{2} I)$ is a Gaussian noise with zero mean and standard deviation $\bm{\sigma}_{n}$. For the Gaussian channel, we set $h=1$, whereas for the Rayleigh fading channel, $h$ is modeled as a complex Gaussian variable, $h \sim \mathcal{CN}(0, 1)$, to represent the multipath fading effect.

After the process of equalization, scaling, and detection, the reconstructed symbols $\hat{\bm{z}}$ are transformed by the information reshaper $p_{\theta}(\bm{y}|\hat{\bm{z}})$ with parameters $\theta$ to provide task-specific data $\bm{y}$. These data are then utilized by the AI agent $q_\psi(\bm{a}|\bm{y})$ with parameters $\psi$, to generate the inferred action $\hat{\bm{a}}$, which approximates the ground truth action $\bm{a}$.







