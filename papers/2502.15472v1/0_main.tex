\documentclass[10pt, journal, letterpaper]{IEEEtran}
\usepackage{amsmath,amsfonts}
%\usepackage{algorithmic}
%\usepackage{algorithm}
\usepackage{array}
\usepackage[caption=false]{subfig}
\usepackage{textcomp}
\usepackage{stfloats}
\usepackage{url}
\usepackage{verbatim}
\usepackage{graphicx}
%\usepackage{cite}
% \hyphenation{op-tical net-works semi-conduc-tor IEEE-Xplore}
% updated with editorial comments 8/9/2021

% Better citation, please remove 'cite' package in the main.tex.
\usepackage[style=ieee, doi=false, url=false, isbn=false, giveninits=true, backend=biber, sorting=none, sortcites=true, date=year]{biblatex}
\addbibresource{reference.bib}
\renewcommand{\bibfont}{\footnotesize}
% Omit primaryClass, specially for arxiv paper
% Omit publisher
\DeclareSourcemap{
  \maps[datatype=bibtex]{
    \map{
      \step[fieldset=primaryClass, null]
      \step[fieldset=publisher, null]
      \step[fieldset=editor, null]
    }
  }
}
% Better color
\usepackage{xcolor}
% Hyperref color
\definecolor{linkcolor}{rgb}{0.0, 0.0, 0.0}  % blue
\definecolor{citecolor}{rgb}{0.0, 0.6, 0.3}  % green
\definecolor{urlcolor}{rgb}{0.6, 0.0, 0.3}   % red

% Better hyperref should be removed before submission
\usepackage[colorlinks=true, linkcolor=linkcolor, citecolor=citecolor, urlcolor=urlcolor]{hyperref}
% hyperref for publication
% \usepackage[hidelinks]{hyperref}

% Better url
\usepackage{url}

% These 3 are for table
\usepackage{booktabs}
\usepackage{multirow}
\usepackage{makecell}

% These 3 are for algorithm, please remove 'algorithmic' in main.tex
\usepackage{algorithmicx}%
\usepackage{algorithm}
\usepackage{algpseudocode}%

% Math Vector
\usepackage{bm}

% Better reference
\usepackage[nameinlink, sort, capitalize]{cleveref}
\Crefname{section}{Section}{Sections}
\Crefname{table}{Table}{Tables}

% Better abbreviation
\usepackage[acronym]{glossaries}

% subfigure
% \usepackage{subcaption}

% demo figure
\usepackage{tikz}

% test text
\usepackage{lipsum}

% for a response letter
\usepackage{multicol}
\usepackage{setspace} % this will change the fontsize of the table!!
\usepackage{mdframed} % this package depends on "\usepackage{tikz}"
% \usepackage{caption} % Required for \captionof. This package may change the fontsize of the figure caption, be careful!!! It may also change the title style of table!!!

% For authors bio
\usepackage{wrapfig}
% % For ORCID logo
\usepackage{orcidlink}

% Better color for reading paper
\definecolor{Honeydew}{RGB}{240, 255, 240}
% Finantial Times Style color
\definecolor{FT}{RGB}{255, 241, 229}

% Better URL
\def\UrlBreaks{\do\A\do\B\do\C\do\D\do\E\do\F\do\G\do\H\do\I\do\J
\do\K\do\L\do\M\do\N\do\O\do\P\do\Q\do\R\do\S\do\T\do\U\do\V
\do\W\do\X\do\Y\do\Z\do\[\do\\\do\]\do\^\do\_\do\`\do\a\do\b
\do\c\do\d\do\e\do\f\do\g\do\h\do\i\do\j\do\k\do\l\do\m\do\n
\do\o\do\p\do\q\do\r\do\s\do\t\do\u\do\v\do\w\do\x\do\y\do\z
\do\.\do\@\do\\\do\/\do\!\do\_\do\|\do\;\do\>\do\]\do\)\do\,
\do\?\do\'\do+\do\=\do\#}

% Definition of abbr
\newacronym{aep}{AEP}{Asymptotic Equipartition Property}
\newacronym{ai}{AI}{Artificial Intelligence}
\newacronym{atroc}{ATROC}{Aligned Task- and Reconstruction-Oriented Communication}
\newacronym{awgn}{AWGN}{Additive White Gaussian Noise}
\newacronym{ber}{BER}{Bit-Error Rate}
\newacronym{bpp}{BPP}{bits per pixel}
\newacronym{csi}{CSI}{Channel State Information}
\newacronym{cvae}{CVAE}{Conditional Variational Autoencoder}
\newacronym{dnn}{DNN}{Deep Neural Network}
\newacronym{dl}{DL}{Deep Learning}
\newacronym{e2e}{E2E}{End-to-End}
\newacronym{fid}{FID}{Fréchet Inception Distance}
\newacronym{gan}{GAN}{Generative Adversarial Network}
\newacronym{ib}{IB}{Information Bottleneck}
\newacronym{iot}{IoT}{Internet of Things}
\newacronym{jscc}{JSCC}{Joint Source-Channel Coding}
\newacronym{kl}{KL}{Kullback-Leibler}
\newacronym{ldpc}{LDPC}{Low-Density Parity-Check}
\newacronym{mse}{MSE}{Mean Square Error}
\newacronym{msssim}{MS-SSIM}{Multi-Scale Structural Similarity}
\newacronym{psnr}{PSNR}{Peak Signal-to-Noise Ratio}
\newacronym{qam}{QAM}{Quadrature Amplitude Modulation}
\newacronym{rf}{RF}{Radio Frequency}
\newacronym{snr}{SNR}{Signal-to-Noise Ratio}
\newacronym{ssim}{SSIM}{Structural Similarity}
\newacronym{tgcp}{TGCP}{Trajectory-Guided Control Prediction}
\newacronym{tscc}{TSCC}{Task-oriented Source-Channel Coding}
\newacronym{vae}{VAE}{Variational Autoencoder}
\newacronym{vib}{VIB}{Variational Information Bottleneck}
\newacronym{v2x}{V2X}{Vehicle-to-Everything}
% new cmd
\newcommand*{\dif}{\mathop{}\!\mathrm{d}}
\renewcommand{\Re}{\operatorname{Re}}
\renewcommand{\Im}{\operatorname{Im}}


\begin{document}

\title{Aligning Task- and Reconstruction-Oriented Communications for Edge Intelligence}

\author{
Yufeng~Diao,~\IEEEmembership{Graduate~Student~Member,~IEEE},~%
Yichi~Zhang,~\IEEEmembership{Graduate~Student~Member,~IEEE},~%
\\%
Changyang~She,~\IEEEmembership{Senior~Member,~IEEE},~%
Philip~Guodong~Zhao,~\IEEEmembership{Senior~Member,~IEEE},
and~Emma~Liying~Li,~\IEEEmembership{Member,~IEEE}%

\thanks{Yufeng Diao is with the School of Computing Science, University of Glasgow, UK (e-mail: y.diao.1@research.gla.ac.uk).}%
\thanks{Yichi Zhang is with the Department of Computer Science, University of Manchester, UK. Part of this work was done when he was with the James Watt School of Engineering, University of Glasgow, UK (e-mail: yichi.zhang@postgrad.manchester.ac.uk).}%
\thanks{Changyang She is with the School of Electrical and Information Engineering, University of Sydney, Australia (e-mail: shechangyang@gmail.com).}%
\thanks{Philip Guodong Zhao is with the Department of Computer Science, University of Manchester, UK (e-mail: philip.zhao@manchester.ac.uk).}%
\thanks{Emma Liying Li is with the School of Computing Science, University of Glasgow, UK (e-mail: liying.li@glasgow.ac.uk).}%
}


% The paper headers
\markboth{This paper has been accepted for publication in IEEE Journal on Selected Areas in Communications (JSAC).}%
{}

% \IEEEpubid{0000--0000/00\$00.00~\copyright~2021 IEEE}
% Remember, if you use this you must call \IEEEpubidadjcol in the second
% column for its text to clear the IEEEpubid mark.

\maketitle
% \thispagestyle{empty}
\begin{abstract}

Existing communication systems aim to reconstruct the information at the receiver side, and are known as reconstruction-oriented communications. This approach often falls short in meeting the real-time, task-specific demands of modern AI-driven applications such as autonomous driving and semantic segmentation. As a new design principle, task-oriented communications have been developed. However, it typically requires joint optimization of encoder, decoder, and modified inference neural networks, resulting in extensive cross-system redesigns and compatibility issues.
This paper proposes a novel communication framework that aligns reconstruction-oriented and task-oriented communications for edge intelligence. The idea is to extend the Information Bottleneck (IB) theory to optimize data transmission by minimizing task-relevant loss function, while maintaining the structure of the original data by an information reshaper. Such an approach integrates task-oriented communications with reconstruction-oriented communications, where a variational approach is designed to handle the intractability of mutual information in high-dimensional neural network features. We also introduce a joint source-channel coding (JSCC) modulation scheme compatible with classical modulation techniques, enabling the deployment of AI technologies within existing digital infrastructures. The proposed framework is particularly effective in edge-based autonomous driving scenarios. Our evaluation in the Car Learning to Act (CARLA) simulator demonstrates that the proposed framework significantly reduces bits per service by 99.19\% compared to existing methods, such as JPEG, JPEG2000, and BPG, without compromising the effectiveness of task execution.
\end{abstract}
\glsresetall

\begin{IEEEkeywords}
Task-oriented communication, edge inference, information bottleneck, variational inference.
\end{IEEEkeywords}
% \pagecolor{FT}

\section{Introduction}

Video generation has garnered significant attention owing to its transformative potential across a wide range of applications, such media content creation~\citep{polyak2024movie}, advertising~\citep{zhang2024virbo,bacher2021advert}, video games~\citep{yang2024playable,valevski2024diffusion, oasis2024}, and world model simulators~\citep{ha2018world, videoworldsimulators2024, agarwal2025cosmos}. Benefiting from advanced generative algorithms~\citep{goodfellow2014generative, ho2020denoising, liu2023flow, lipman2023flow}, scalable model architectures~\citep{vaswani2017attention, peebles2023scalable}, vast amounts of internet-sourced data~\citep{chen2024panda, nan2024openvid, ju2024miradata}, and ongoing expansion of computing capabilities~\citep{nvidia2022h100, nvidia2023dgxgh200, nvidia2024h200nvl}, remarkable advancements have been achieved in the field of video generation~\citep{ho2022video, ho2022imagen, singer2023makeavideo, blattmann2023align, videoworldsimulators2024, kuaishou2024klingai, yang2024cogvideox, jin2024pyramidal, polyak2024movie, kong2024hunyuanvideo, ji2024prompt}.


In this work, we present \textbf{\ours}, a family of rectified flow~\citep{lipman2023flow, liu2023flow} transformer models designed for joint image and video generation, establishing a pathway toward industry-grade performance. This report centers on four key components: data curation, model architecture design, flow formulation, and training infrastructure optimization—each rigorously refined to meet the demands of high-quality, large-scale video generation.


\begin{figure}[ht]
    \centering
    \begin{subfigure}[b]{0.82\linewidth}
        \centering
        \includegraphics[width=\linewidth]{figures/t2i_1024.pdf}
        \caption{Text-to-Image Samples}\label{fig:main-demo-t2i}
    \end{subfigure}
    \vfill
    \begin{subfigure}[b]{0.82\linewidth}
        \centering
        \includegraphics[width=\linewidth]{figures/t2v_samples.pdf}
        \caption{Text-to-Video Samples}\label{fig:main-demo-t2v}
    \end{subfigure}
\caption{\textbf{Generated samples from \ours.} Key components are highlighted in \textcolor{red}{\textbf{RED}}.}\label{fig:main-demo}
\end{figure}


First, we present a comprehensive data processing pipeline designed to construct large-scale, high-quality image and video-text datasets. The pipeline integrates multiple advanced techniques, including video and image filtering based on aesthetic scores, OCR-driven content analysis, and subjective evaluations, to ensure exceptional visual and contextual quality. Furthermore, we employ multimodal large language models~(MLLMs)~\citep{yuan2025tarsier2} to generate dense and contextually aligned captions, which are subsequently refined using an additional large language model~(LLM)~\citep{yang2024qwen2} to enhance their accuracy, fluency, and descriptive richness. As a result, we have curated a robust training dataset comprising approximately 36M video-text pairs and 160M image-text pairs, which are proven sufficient for training industry-level generative models.

Secondly, we take a pioneering step by applying rectified flow formulation~\citep{lipman2023flow} for joint image and video generation, implemented through the \ours model family, which comprises Transformer architectures with 2B and 8B parameters. At its core, the \ours framework employs a 3D joint image-video variational autoencoder (VAE) to compress image and video inputs into a shared latent space, facilitating unified representation. This shared latent space is coupled with a full-attention~\citep{vaswani2017attention} mechanism, enabling seamless joint training of image and video. This architecture delivers high-quality, coherent outputs across both images and videos, establishing a unified framework for visual generation tasks.


Furthermore, to support the training of \ours at scale, we have developed a robust infrastructure tailored for large-scale model training. Our approach incorporates advanced parallelism strategies~\citep{jacobs2023deepspeed, pytorch_fsdp} to manage memory efficiently during long-context training. Additionally, we employ ByteCheckpoint~\citep{wan2024bytecheckpoint} for high-performance checkpointing and integrate fault-tolerant mechanisms from MegaScale~\citep{jiang2024megascale} to ensure stability and scalability across large GPU clusters. These optimizations enable \ours to handle the computational and data challenges of generative modeling with exceptional efficiency and reliability.


We evaluate \ours on both text-to-image and text-to-video benchmarks to highlight its competitive advantages. For text-to-image generation, \ours-T2I demonstrates strong performance across multiple benchmarks, including T2I-CompBench~\citep{huang2023t2i-compbench}, GenEval~\citep{ghosh2024geneval}, and DPG-Bench~\citep{hu2024ella_dbgbench}, excelling in both visual quality and text-image alignment. In text-to-video benchmarks, \ours-T2V achieves state-of-the-art performance on the UCF-101~\citep{ucf101} zero-shot generation task. Additionally, \ours-T2V attains an impressive score of \textbf{84.85} on VBench~\citep{huang2024vbench}, securing the top position on the leaderboard (as of 2025-01-25) and surpassing several leading commercial text-to-video models. Qualitative results, illustrated in \Cref{fig:main-demo}, further demonstrate the superior quality of the generated media samples. These findings underscore \ours's effectiveness in multi-modal generation and its potential as a high-performing solution for both research and commercial applications.
% \section{Related Work}

\subsection{Large 3D Reconstruction Models}
Recently, generalized feed-forward models for 3D reconstruction from sparse input views have garnered considerable attention due to their applicability in heavily under-constrained scenarios. The Large Reconstruction Model (LRM)~\cite{hong2023lrm} uses a transformer-based encoder-decoder pipeline to infer a NeRF reconstruction from just a single image. Newer iterations have shifted the focus towards generating 3D Gaussian representations from four input images~\cite{tang2025lgm, xu2024grm, zhang2025gslrm, charatan2024pixelsplat, chen2025mvsplat, liu2025mvsgaussian}, showing remarkable novel view synthesis results. The paradigm of transformer-based sparse 3D reconstruction has also successfully been applied to lifting monocular videos to 4D~\cite{ren2024l4gm}. \\
Yet, none of the existing works in the domain have studied the use-case of inferring \textit{animatable} 3D representations from sparse input images, which is the focus of our work. To this end, we build on top of the Large Gaussian Reconstruction Model (GRM)~\cite{xu2024grm}.

\subsection{3D-aware Portrait Animation}
A different line of work focuses on animating portraits in a 3D-aware manner.
MegaPortraits~\cite{drobyshev2022megaportraits} builds a 3D Volume given a source and driving image, and renders the animated source actor via orthographic projection with subsequent 2D neural rendering.
3D morphable models (3DMMs)~\cite{blanz19993dmm} are extensively used to obtain more interpretable control over the portrait animation. For example, StyleRig~\cite{tewari2020stylerig} demonstrates how a 3DMM can be used to control the data generated from a pre-trained StyleGAN~\cite{karras2019stylegan} network. ROME~\cite{khakhulin2022rome} predicts vertex offsets and texture of a FLAME~\cite{li2017flame} mesh from the input image.
A TriPlane representation is inferred and animated via FLAME~\cite{li2017flame} in multiple methods like Portrait4D~\cite{deng2024portrait4d}, Portrait4D-v2~\cite{deng2024portrait4dv2}, and GPAvatar~\cite{chu2024gpavatar}.
Others, such as VOODOO 3D~\cite{tran2024voodoo3d} and VOODOO XP~\cite{tran2024voodooxp}, learn their own expression encoder to drive the source person in a more detailed manner. \\
All of the aforementioned methods require nothing more than a single image of a person to animate it. This allows them to train on large monocular video datasets to infer a very generic motion prior that even translates to paintings or cartoon characters. However, due to their task formulation, these methods mostly focus on image synthesis from a frontal camera, often trading 3D consistency for better image quality by using 2D screen-space neural renderers. In contrast, our work aims to produce a truthful and complete 3D avatar representation from the input images that can be viewed from any angle.  

\subsection{Photo-realistic 3D Face Models}
The increasing availability of large-scale multi-view face datasets~\cite{kirschstein2023nersemble, ava256, pan2024renderme360, yang2020facescape} has enabled building photo-realistic 3D face models that learn a detailed prior over both geometry and appearance of human faces. HeadNeRF~\cite{hong2022headnerf} conditions a Neural Radiance Field (NeRF)~\cite{mildenhall2021nerf} on identity, expression, albedo, and illumination codes. VRMM~\cite{yang2024vrmm} builds a high-quality and relightable 3D face model using volumetric primitives~\cite{lombardi2021mvp}. One2Avatar~\cite{yu2024one2avatar} extends a 3DMM by anchoring a radiance field to its surface. More recently, GPHM~\cite{xu2025gphm} and HeadGAP~\cite{zheng2024headgap} have adopted 3D Gaussians to build a photo-realistic 3D face model. \\
Photo-realistic 3D face models learn a powerful prior over human facial appearance and geometry, which can be fitted to a single or multiple images of a person, effectively inferring a 3D head avatar. However, the fitting procedure itself is non-trivial and often requires expensive test-time optimization, impeding casual use-cases on consumer-grade devices. While this limitation may be circumvented by learning a generalized encoder that maps images into the 3D face model's latent space, another fundamental limitation remains. Even with more multi-view face datasets being published, the number of available training subjects rarely exceeds the thousands, making it hard to truly learn the full distibution of human facial appearance. Instead, our approach avoids generalizing over the identity axis by conditioning on some images of a person, and only generalizes over the expression axis for which plenty of data is available. 

A similar motivation has inspired recent work on codec avatars where a generalized network infers an animatable 3D representation given a registered mesh of a person~\cite{cao2022authentic, li2024uravatar}.
The resulting avatars exhibit excellent quality at the cost of several minutes of video capture per subject and expensive test-time optimization.
For example, URAvatar~\cite{li2024uravatar} finetunes their network on the given video recording for 3 hours on 8 A100 GPUs, making inference on consumer-grade devices impossible. In contrast, our approach directly regresses the final 3D head avatar from just four input images without the need for expensive test-time fine-tuning.


\section{Fundamental Physics-based Hardware and Signal Models} \label{sec:model}

In this section, we introduce the fundamental physics-based hardware and signal models for pinching antennas. Specifically, we model a pinching antenna as an open-ended directional coupler, which facilitates the adjustment of radiation characteristics and simplifies signal modeling. We then employ coupled-mode theory to characterize the relationship between the signal within the waveguide and the signal radiated by the pinching antenna.

\begin{figure}[t!]
    \centering
    \includegraphics[width=0.45\textwidth]{./waveguide_model.pdf}
    \caption{Schematic illustration of pinching antennas operating as an open-ended directional waveguide coupler.}
    \label{physical_model}
  \end{figure} 

\subsection{Physics-based Hardware Model}

The core principle of pinching antennas relies on the phenomenon where a portion of the EM waves propagating through a dielectric waveguide is induced into an adjacent dielectric material (i.e., a pinching antenna) when the two are placed in close proximity. To accurately model the signal radiated by the pinching antenna, it is essential to characterize the EM fields within both the waveguide and the pinching antenna, as well as to understand the interaction between these fields. 

Consider a dielectric waveguide with an effective refractive index $n_{\mathrm{g}}$. A signal with a free-space wavelength $\lambda$ is introduced into the waveguide and propagates along the $x$-axis, as shown in Fig. \ref{physical_model}. The electric field distribution of the EM wave within the waveguide can be expressed as:
\begin{equation} \label{EM_model_waveguide}
    \mathbf{E}_{\mathrm{guide}}(x,y,z) = \mathbf{D}_{\mathrm{guide}}(y,z) e^{-j \beta_{\mathrm{g}} x} s_{\mathrm{p}},
\end{equation}
where $\mathbf{D}_{\mathrm{guide}}(y,z) \in \mathbb{C}^{3 \times 1}$ represents the transverse field distribution of the guided mode, $\beta_{\mathrm{g}} = \frac{2 \pi n_{\mathrm{g}}}{\lambda}$ is the propagation constant of the waveguide, and $s_{\mathrm{p}} \in \mathbb{C}$ denotes the phase-shifted communication signal modulated onto the EM wave. Let $x_{0, \mathrm{p}}$ denote the distance that the signal has propagated within the waveguide prior to the starting point of coupling, as shown in Fig. \ref{physical_model}. The signal $s_{\mathrm{p}}$ can be expressed as:
\begin{equation}
    s_{\mathrm{p}} = e^{-j \beta_{\mathrm{g}} x_{0, \mathrm{p}}} c_0,
\end{equation}
where $c_0$ represents the original communication signal. Note that the in-waveguide propagation loss is omitted in the above formulas due to its negligible impact on the system performance \cite{ding2024flexible}. 

We model the pinching antenna as an open-ended directional waveguide coupler, where EM waves can be radiated from one end of the pinching antenna with minimal reflection by optimizing waveguide's aperture size, shape, and termination impedance \cite{gardiol1985open}. For simplicity, we assume ideal full radiation from the open end of the pinching antenna with no reflection. When the pinching antenna is placed in proximity to (or “pinched” against) the main waveguide, coupling occurs, generating an EM field within the pinching antenna. Let $n_{\mathrm{p}}$ denote the effective refractive index of the pinching antenna. The electric field within the pinching antenna is
\begin{equation}
    \mathbf{E}_{\mathrm{pinch}}(x,y,z) = \mathbf{D}_{\mathrm{pinch}}(y,z) e^{-j \beta_{\mathrm{p}} x} s_{\mathrm{p}},
\end{equation}
where $\mathbf{D}_{\mathrm{pinch}}(y,z) \in \mathbb{C}^{3 \times 1}$ represents the transverse field distribution, and $\beta_{\mathrm{p}} = \frac{2 \pi n_{\mathrm{p}}}{\lambda}$ is the propagation constant of the pinching antenna.

Based on coupled-mode theory, the total EM field within the waveguide and the pinching antenna can be expressed as a weighted sum of their respective individual fields. Let $\mathbf{E}$ denote the overall electric field, which can be written as:
\begin{equation}
    \mathbf{E} = A(x) \mathbf{E}_{\mathrm{guide}} + B(x) \mathbf{E}_{\mathrm{pinch}}.
\end{equation}
By substituting the expressions for the electric field $\mathbf{E}$ and the corresponding magnetic field into Maxwell's equations, the following coupled differential equations for $A(x)$ and $B(x)$ are obtained \cite{okamoto2010fundamentals}:
\begin{align} \label{diff_condition_A}
    \frac{d A(x)}{d x} &= -j \kappa B(x) e^{-j \Delta \beta x}, \\
    \label{diff_condition_B}
    \frac{d B(x)}{d x} &= -j \kappa A(x) e^{j \Delta \beta x},
\end{align}
where $\kappa \in \mathbb{R}$ is the mode coupling coefficient, determined by the transverse field distributions $\mathbf{D}_{\mathrm{guide}}(y,z)$ and $\mathbf{D}_{\mathrm{pinch}}(y,z)$, and $\Delta \beta = \beta_{\mathrm{p}} - \beta_{\mathrm{g}}$ represents the difference between the propagation constants of the waveguide and the pinching antenna. Since the signal is initially introduced into the waveguide, the following initial conditions hold:
\begin{align}
    A(0) = 1, \quad B(0) = 0.
\end{align}
Solving the differential equations \eqref{diff_condition_A} and \eqref{diff_condition_B} under these initial conditions yields the following expressions for $A(x)$ and $B(x)$:
\begin{align}
    A(x) &= \left( \cos \left(\phi x \right) + \frac{j \Delta \beta}{\phi} \sin(\phi x) \right) e^{-j \Delta \beta x/2}, \\
    B(x) &= - \frac{j \kappa}{\phi} \sin(\phi x) e^{j \Delta \beta x/2},
\end{align}
where $\phi = \sqrt{\kappa^2 + \Delta \beta^2}$. Consequently, the total power of the EM field within the waveguide and the pinching antenna is given by:
\begin{align}
    P_{\mathrm{guide}}(x) &= \left| A(x) \right|^2 = 1 - \left( \frac{\kappa}{\phi} \right)^2 \sin^2(\phi x), \\
    P_{\mathrm{pinch}}(x) &= \left| B(x) \right|^2 = \left( \frac{\kappa}{\phi} \right)^2 \sin^2(\phi x).
\end{align}
From these expressions, it is evident that a maximum fraction of $\left( \kappa/\phi \right)^2$ of the total power can be transferred to the pinching antenna. In the special case where the waveguide and the pinching antenna have the same effective refractive index (i.e., $\beta_{\mathrm{g}} = \beta_{\mathrm{p}}$), we have $\Delta \beta = 0$ and $\phi = \kappa$. Under this condition, the expressions for $A(x)$ and $B(x)$ can be simplified to:
\begin{align} \label{simplified_power_exchange}
    A(x) = \cos(\kappa x), \quad B(x) = -j \sin(\kappa x).
\end{align}
Utilizing the simplified power exchange coefficients in \eqref{simplified_power_exchange} for the case $\beta_{\mathrm{g}} = \beta_{\mathrm{p}}$, the signal radiated from the open end of the pinching antenna can be expressed as
\begin{align} \label{EM_model_rad}
    \mathbf{E}_{\mathrm{rad}}(y,z) &=  B(L) \mathbf{E}_{\mathrm{pinch}}(L,y,z) \nonumber \\
    &= -j \mathbf{D}_{\mathrm{pinch}}(y,z) \sin(\kappa L) e^{-j \beta_{\mathrm{g}} x_{\mathrm{p}}} c_0,
\end{align}
where $x_{\mathrm{p}}$ represents the effective position of the pinching antenna and is defined as:
\begin{equation}
    x_{\mathrm{p}} = x_{0, \mathrm{p}} + L.
\end{equation}
Once the field in the pinching antenna radiates into free space, the power exchange between the waveguide and the pinching antenna ceases at $x = L$. The remaining EM wave propagating within the waveguide is given by:
\begin{align} \label{EM_model_guide_remain}
    \tilde{\mathbf{E}}_{\mathrm{guide}}(x,y,z) &= A(L) \mathbf{E}_{\mathrm{guide}}(L,y,z) \nonumber \\
    &= \mathbf{D}_{\mathrm{guide}}(y,z) \cos(\kappa L) e^{-j \beta_{\mathrm{g}} (x + x_{\mathrm{p}})} c_0.
\end{align}

\begin{remark}
    \normalfont
    \emph{(Power Relationship)} Based on the above physics model, the power of the signals within the waveguide and radiated by the pinching antenna is determined by the coupling length $L$. In the case of $\beta_{\mathrm{g}} = \beta_{\mathrm{p}}$, a complete power transfer to the pinching antenna becomes achievable, enabling full signal radiation, which is achieved when $L = \pi/(2\kappa)$ according to \eqref{simplified_power_exchange}. However, this strategy is applicable for the case with a single pinching antennas, but not for the case with multiple pinching antennas on the same waveguide. For the latter case, the radiation power from each pinching antenna can be adjusted by modifying the coupling length $L$. In other words, the transmit powers of the multiple pinching antennas can be controlled, which leads to extra degrees of freedom for the system design; howerver, this would likely lead to active pinching antenna design and additional system complexity. Therefore, this paper focuses on a purely passive pinching antenna design with a preconfigured coupling length.
\end{remark}

% The fundamental concept of PASS lies in the phenomenon where, when a separate dielectric material (i.e., the pinching antenna) is placed in close proximity to a dielectric waveguide, a portion of the radio waves propagating through the waveguide is induced into the adjacent dielectric. To accurately model the signal radiated by the pinching antenna, it is essential to characterize the electromagnetic fields within both the pinching antenna and the waveguide, as well as to understand the relationship between these fields. In the following, we model the pinching antenna as an open-ended directional coupler and utilize coupled-mode theory to achieve this characterization.

% Let us consider a dielectric waveguide with an effective index $n_{\mathrm{g}}$. A signal with a free-space wavelength $\lambda$ is introduced into the waveguide and propagates along the $z$-axis. The electric field distribution of the EM wave within the waveguide can be expressed as
% \begin{equation} \label{EM_model_waveguid}
%         \mathbf{E}_{\mathrm{guide}}(x,y,z) = \mathbf{D}_{\mathrm{guide}}(y,z) e^{-j \beta_{\mathrm{g}} z} s_{\mathrm{p}}, 
% \end{equation}
% where $\mathbf{D}_{\mathrm{guide}}(y,z) \in \mathbb{C}^{3 \times 1}$ is the transverse field distributions of the guided mode of signals, $\alpha \in \mathbb{R}$ is the attenuation coefficient of the waveguide, $\beta_{\mathrm{g}} = \frac{2 \pi n_{\mathrm{g}}}{\lambda}$ is the propagation constant of the waveguide, and $s_{\mathrm{p}} \in \mathbb{C}$ denotes phase-shifted communication data modulated onto the EM wave. Let $x_{0, \mathrm{p}}$ denote the propagation distance of signal from the induction position to the starting position of coupling. The signal $s_{\mathrm{p}}$ can be expressed as 
% \begin{equation}
%     s_{\mathrm{p}} = e^{-j \beta_{\mathrm{g}} x_{0, \mathrm{p}}} c_0, 
% \end{equation}   
% where $c_0$ is the original communication data. 
% % Since attenuation within the waveguide is negligible, the equation \eqref{EM_model_waveguid} simplifies to
% % \begin{align} \label{EM_model_waveguid_simple}
% %     \mathbf{E}_{\mathrm{guide}}(x,y,z) &= \mathbf{D}_{\mathrm{guide}}(y,z) e^{-j \beta_{\mathrm{g}} (z + z_p)} c_0.
% % \end{align} 
% For simplicity, the pinching antenna can be approximated as an open-ended directional coupler with an effective index $n_{\mathrm{p}}$. When the pinching antenna are in proximity of (or pinched to) the main waveguide, coupling occurs, generating an electromagnetic field within the pinching antenna. The electric field within it can be expressed as
% \begin{equation}
%     \mathbf{E}_{\mathrm{pinch}}(x,y,z) = \mathbf{D}_{\mathrm{pinch}}(y,z) e^{-j \beta_{\mathrm{p}} z} s_{\mathrm{p}},
% \end{equation}
% where $\mathbf{D}_{\mathrm{pinch}}(y,z) \in \mathbb{C}^{3 \times 1}$ is the transverse field distributions, and $\beta_{\mathrm{p}} = \frac{2 \pi n_{\mathrm{p}}}{\lambda}$ is the propagation constant of the pinching antenna. 
% % The electromagnetic fields in the waveguide and the pinching antenna must match tangentially at their interface $z = 0$. This condition results in the relationship:
% % \begin{equation}
% %     \bar{s}_0 = e^{-j \beta_{\mathrm{g}} z_p} c_0.
% % \end{equation}

% Based on coupled-mode theory, the total electromagnetic field within the waveguide and the pinching antenna can be expressed as a weighted sum of their respective individual fields. Let $\mathbf{E}$ denote the overall electric fields, which can be represented as 
% \begin{equation}
%     \mathbf{E} = A(x) \mathbf{E}_{\mathrm{guide}} + B(x) \mathbf{E}_{\mathrm{pinch}}.
% \end{equation}
% By substituting the expressions for the electric field $\mathbf{E}$ and the corresponding magnetic field into Maxwell's equations, the following coupled differential equations relating $A(x)$ and $B(x)$ can be obtained \cite{okamoto2010fundamentals}:
% \begin{align} \label{diff_condition_A}
%     \frac{d A(x)}{d x} &= -j \kappa B(x) e^{-j \Delta \beta z},\\
%     \label{diff_condition_B}
%     \frac{d B(x)}{d x} &= -j \kappa A(x) e^{j \Delta \beta z},
% \end{align}
% where $\kappa \in \mathbb{R}$ is mode coupling coefficient, determined by the transverse field distributions $\mathbf{D}_{\mathrm{guide}}(y,z)$ and $\mathbf{D}_{\mathrm{pinch}}(y,z)$, and $\Delta \beta = \beta_{\mathrm{p}} - \beta_{\mathrm{g}}$ denotes the difference between the propagation constant of the waveguide and the pinching antenna. Since the signal is initially introduced into the waveguide, the following initial conditions hold:
% \begin{align}
%     A(0) = 1, \quad B(0) = 0,
% \end{align}  
% Solving the differential equations \eqref{diff_condition_A} and \eqref{diff_condition_B} under these initial conditions yields the expressions for $A(x)$ and $B(x)$ as
% \begin{align}
%     A(x) &=\left( \cos \left(\phi z \right) + \frac{j \Delta \beta}{\phi} \sin(\phi z) \right) e^{-j \Delta \beta z/2}, \\
%     B(x) &= - \frac{j \kappa}{\phi} \sin(\phi z) e^{j \Delta \beta z/2},
% \end{align}
% where $\phi = \sqrt{\kappa^2 + \Delta \beta^2}$. Therefore, the ratio of total power of the EM field within the waveguide and the pinching antenna are given by, respectively,
% \begin{align}
%     P_{\mathrm{guide}}(x) &= \left| A(x) \right|^2 = 1 - \left( \frac{\kappa}{\phi} \right)^2 \sin^2(\phi z), \\
%     P_{\mathrm{pinch}}(x) &= \left| B(x) \right|^2 =  \left( \frac{\kappa}{\phi} \right)^2 \sin^2(\phi z).
% \end{align}
% It can be observed that a maximum fraction of  $( \frac{\kappa}{\phi} )^2$ of the total power can be transferred to the pinching antenna. In the special case that the waveguide and the pinching antenna have the same effective index, i.e., $\beta_{\mathrm{g}} = \beta_{\mathrm{p}}$, we have $\Delta \beta = 0$ and $\phi = \kappa$. Under this condition, the expressions for $A(x)$ and $B(x)$ simplifies to
% \begin{align} \label{simplified_power_exchange}
%     A(x) = \cos(\kappa z), \quad B(x) = -j \sin(\kappa z).
% \end{align} 
% In this scenario, it becomes feasible to transfer all the signal power through the pinching antenna, achieving complete radiation.

% Utilizing the simplified power exchange coefficients in \eqref{simplified_power_exchange} achieved when $\beta_{\mathrm{g}} = \beta_{\mathrm{p}}$, the signal radiated from the open end of the pinching antenna can be expressed in terms of the following electric field:
% \begin{align} \label{EM_model_rad}
%     \mathbf{E}_{\mathrm{rad}}(y,z) = &B(L) \mathbf{E}_{\mathrm{pinch}}(x,y,L) \nonumber \\
%     = & -j \sin(\kappa L) \mathbf{D}_{\mathrm{pinch}}(y,z)  e^{-j \beta_{\mathrm{g}} x_{\mathrm{p}} } c_0,
% \end{align}
% where $x_{\mathrm{p}}$ is defined as the effective position of the pinching antenna and is given by 
% \begin{equation}
%     x_{\mathrm{p}} = x_{0, \mathrm{p}} + L.
% \end{equation} 
% % where $\tilde{\mathbf{E}}_p(y,z) = -j e^{-j \beta_{\mathrm{p}} L} \mathbf{D}_{\mathrm{pinch}}(y,z)$ represents the modified transverse field distribution of the pinching antenna at the radiating end. Note that the additional phase shift $-j e^{-j \beta_{p,L}}$ in the radiated field can be pre-compensated by incorporating a low-cost phase shifter with a fixed phase adjustment at the radiator of each pinching antenna. Therefore, for simplicity, we omit the influence of this additional phase shift in the subsequent analysis. 
% Once the field in the pinching antenna radiates into free space, the power exchange between the waveguide and the pinching antenna ceases at the position $z = L$. Consequently, the remaining electromagnetic wave propagating within the waveguide is given by:
% \begin{align} \label{EM_model_guide_remain}
%     \tilde{\mathbf{E}}_{\mathrm{guide}}(x,y,z) = &\cos(kL) \mathbf{E}_{\mathrm{guide}}(x,y,z) \nonumber \\ 
%     = & \cos(kL) \mathbf{D}_{\mathrm{guide}}(y,z) e^{-j \beta_{\mathrm{g}} (z + x_{\mathrm{p}})} c_0.
% \end{align}   




\subsection{Signal Model}
In \eqref{EM_model_rad} and \eqref{EM_model_guide_remain}, we derived the physical models describing the EM wave behavior in the PASS system. Building upon these foundational physics models, we can now formulate simplified signal models commonly employed in the wireless communication system design. In the following, we begin by considering the scenario with a single pinching antenna coupled to the waveguide and subsequently extend the model to cases involving multiple pinching antennas.

\subsubsection{A Single Pinching Antenna}
Assume that a waveguide is deployed along the $x$-axis, with its $y$- and $z$-coordinates denoted by $y_{\mathrm{g}}$ and $z_{\mathrm{g}}$, respectively. A user is located on the ground at the position $\mathbf{r} = [x_{\mathrm{u}}, y_{\mathrm{u}}, 0]^T$. To serve this user, a pinching antenna is attached to the waveguide at the position $\mathbf{p} = [x_{\mathrm{p}}, y_{\mathrm{g}}, z_{\mathrm{g}}]^T$. Let $c_0$ represent the signal fed into the waveguide. According to \eqref{EM_model_rad}, the signal radiated from the pinching antenna is given by
\begin{equation}
s_{\mathrm{rad}} = \sin(\kappa L) e^{-j \beta_{\mathrm{g}} x_{\mathrm{p}}} c_0,
\end{equation}
The radiated signal propagates through free space to reach the user, where attenuation due to free-space path loss must be considered, leading to the following signal model:
\begin{align}
y = & \frac{\eta e^{-j \beta_0 r}}{r} s_{\mathrm{rad}} + n \nonumber \\
= & \underbrace{\frac{\eta e^{-j \beta_0 r}}{r}}_{\scriptstyle \text{high-loss free-space} \atop \scriptstyle \text{propagation}} \!\!\! \times \underbrace{ \vphantom{\frac{\eta e^{-j \beta_0 r}}{r}} \sin(\kappa L) e^{-j \beta_{\mathrm{g}} x_{\mathrm{p}}}}_{\scriptstyle \text{nearly lossless in-waveguide} \atop \scriptstyle  \text{propagation}}  c_0 + n,
\end{align}
where $\eta \in \mathbb{R}$ is the channel gain accounting for the free-space path loss factor and the radiation pattern of the pinching antennas, $\beta_0 = \frac{2 \pi}{\lambda}$ is the propagation constant in free space, $r$ is the distance between the pinching antenna and the user, given by
\begin{equation}
r = \| \mathbf{r} - \mathbf{p} \| = \sqrt{(x_{\mathrm{p}} - x_{\mathrm{u}})^2 + \omega},
\end{equation}
with $\omega = (y_{\mathrm{g}} - y_{\mathrm{u}})^2 + z_{\mathrm{g}}^2$, and $n \sim \mathcal{CN}(0 ,\sigma^2)$ denotes the additive white Gaussian noise. 


\subsubsection{Multiple Pinching Antennas}

Now we consider a scenario where $M$ pinching antennas are pinched sequentially on the waveguide. Let $\mathbf{p}_m = [x_{\mathrm{p},m}, y_{\mathrm{g}}, z_{\mathrm{g}}]^T$ and $L_m$ denote the position and the length of the $m$-th pinching antenna, respectively. According to \eqref{EM_model_rad} and \eqref{EM_model_guide_remain}, the power radiated by the $m$-th pinching antenna is influenced by the power exchange coefficients of all preceding pinching antennas. Consequently, the signal radiated from the $m$-th pinching antenna can be expressed as: 
\begin{align}
    s_{\mathrm{rad}, m} =&\sin(\kappa L_m) \prod_{i=1}^{m-1} \cos(\kappa L_i) e^{-j \beta_{\mathrm{g}} x_{\mathrm{p},m}} c_0 \nonumber \\
    = & \delta_m \prod_{i=1}^{m-1} \sqrt{1 - \delta_m^2} e^{-j \beta_{\mathrm{g}} x_{\mathrm{p},m}} c_0,
\end{align} 
where we define $\delta_m \triangleq \sin(\kappa L_m)$. 
Here, we consider two simplified but useful signal radiation model for multiple pinching antennas:
\begin{itemize}
    \item \textbf{Equal Power Model:} In this model, we assume that the length $L_m$ of each pinching antenna is adjusted so that each antenna radiates an equal proportion of the total power. Specifically, this is expressed as
    \begin{align}
        &\delta_m \prod_{i=1}^{m-1} \sqrt{1 - \delta_m^2} = \sqrt{\delta_{\mathrm{eq}}}, \quad \forall m, \nonumber \\
        \Leftrightarrow \quad & \delta_m = \sqrt{\frac{\delta_{\mathrm{eq}}}{(1 - \delta_{\mathrm{eq}})^{m-1}}}, \quad \forall m,
    \end{align}
    where $0 < \delta_{\mathrm{eq}} \le \frac{1}{M}$ is the equal-power ratio. Under this condition, the radiated signal from the $m$-th pinching antenna is simplified into
    \begin{equation}
        s_{\mathrm{rad},m} = \sqrt{\delta_{\mathrm{eq}}}  e^{-j \beta_{\mathrm{g}} x_{\mathrm{p},m}} c_0.
    \end{equation} 
    This model ensures that each pinching antenna radiates with the same efficiency, making it useful for obtaining insight to the performance of PASS as discussed in \cite{ding2024flexible}. However, achieving this equal-power distribution requires that each pinching antenna be manufactured with a different length, which increases the hardware cost.
    \item \textbf{Proportional Power Model:} In this model, we assume that the length of each pinching antenna is manufactured to be the same, i.e., $L_m = L, \forall m$. Consequently, each pinching antenna radiates the same ratio of the remaining power within the waveguide. Under this condition, the radiated signal from the $m$-th pinching antenna becomes
    \begin{equation}
        s_{\mathrm{rad},m} = \delta \left(\sqrt{1 - \delta^2}\right)^{m-1} e^{-j \beta_{\mathrm{g}} x_{\mathrm{p},m}} c_0,
    \end{equation}
    where $\delta = \sin(\kappa L)$. Compared to the equal-power model, this equal-ratio model significantly reduces hardware costs, as all pinching antennas can be uniformly manufactured with the same length.
\end{itemize}

Based on the above modeling, the signal received from all pinching antennas at the user can be expressed as 
\begin{align} \label{multiple_basis_model}
    y = &\sum_{m=1}^M \frac{\eta e^{-j \beta_0 r_m}}{r_m} s_{\mathrm{rad}, m} + n = \mathbf{h}^H(\mathbf{x}) \mathbf{g}(\mathbf{x}) c_0 + n.
\end{align}
Here, $r_m = \|\mathbf{r} - \mathbf{p}_m\|$ is the distance between the $m$-th pinching antenna and the user. $\mathbf{h}(\mathbf{x}) \in \mathbb{C}^{M \times 1}$ represents the free-space channel vector between all pinching antennas and the user, given by 
\begin{equation} \label{basic_channel_model}
    \mathbf{h}(\mathbf{x}) = \left[\frac{\eta e^{-j \beta_0 r_1}}{r_1},\dots,\frac{\eta e^{-j \beta_0 r_M}}{r_M}  \right]^H,
\end{equation}
which is a function of the pinching antenna positions $\mathbf{x} = [x_{\mathrm{p},1},\dots,x_{\mathrm{p},M}]^T$. $\mathbf{g}(\mathbf{x}) \in \mathbb{C}^{M \times 1}$ denotes the in-waveguide channel vector, given by 
\begin{equation} \label{basic_pinching_beamforming_model}
    \mathbf{g}(\mathbf{x}) = \left[ \alpha_1 e^{-j \beta_{\mathrm{g}} x_{\mathrm{p},1}},\dots,\alpha_M e^{-j \beta_{\mathrm{g}} x_{\mathrm{p},M}} \right]^T,
\end{equation} 
where $\alpha_m = \sqrt{\delta_{\mathrm{eq}}}$ for the equal power model, while $\alpha_m = \delta(\sqrt{1 - \delta^2})^{m-1}$ for the proportional power model.  

\begin{remark}
    \normalfont
    \emph{(Pinching Beamforming)} It can be observed from
    \eqref{multiple_basis_model}-\eqref{basic_pinching_beamforming_model} that adjusting the positions of  the pinching antennas, determined by $\mathbf{x}$, alters the phase and the large-scale path loss (amplitude) of the signal received by the user. Analogous to conventional multi-antenna beamforming, the signals from multiple pinching antennas can be combined either constructively or destructively at the user by carefully optimizing the positions of the pinching antennas. This new capability for signal reconfiguration introduced by PASS is referred to as \emph{pinching beamforming}.
    
    % The signal model \eqref{multiple_basis_model} is analogous to that of a conventional MIMO system, where $\mathbf{h}(\mathbf{x})$ and $\mathbf{g}(\mathbf{x})$ are channel and beamforming vectors, respectively. Adjusting the position of the pinching antenna, determined by $x_{\mathrm{p},m}$, alters not only the channel conditions, but also achieve effective beamforming, such that the signals  are combined either constructively or destructively at the user by carefully optimizing the positions of the pinching antennas, which is referred to as \emph{pinching beamforming}.
\end{remark}




% \begin{subequations}
%     \begin{align}
%         \frac{d A(x)}{d x} &= -j \kappa_g B(x) e^{-j(\beta_{\mathrm{p}} - \beta_{\mathrm{g}}) z} + j \eta_g A(x),\\
%         \frac{d B(x)}{d x} &= -j \kappa_p B(x) e^{j(\beta_{\mathrm{p}} - \beta_{\mathrm{g}}) z} + j \eta_p B(x),
%     \end{align}
% \end{subequations}
% where 
% \begin{subequations}
%     \begin{align}
%         \kappa_g = \frac{\kappa_{g,p} - \phi \chi_p}{1 - |\phi|^2}, \quad \kappa_p = \frac{\kappa_{p,g} - \phi^* \chi_g}{1 - |\phi|^2}, \\
%         \eta_g = \frac{\kappa_{p,g} \phi - \chi_g}{1 - |\phi|^2}, \quad \eta_p = \frac{\kappa_{p,g} \phi^* - \chi_p}{1 - |\phi|^2}. 
%     \end{align}
% \end{subequations}
% Here, $\kappa_{g,p}$ and $\kappa_{p,g}$ are mode coupling coefficients, $\phi$ is the butt coupling coefficient, and $\chi_g$ and $\chi_p$ represent the changes in the propagation constant. The mode and butt coupling coefficients follow the relationship below to satisfy the law of conservation of energy:
% \begin{equation}
%     \kappa_{p,g} = \kappa_{g,p}^* + (\beta_{\mathrm{p}} - \beta_{\mathrm{g}}) \phi^*.
% \end{equation}
% Note that in practice, we typically have $\chi_g \approx \chi_p \approx 0$.  










\section{Information Bottleneck for ATROC}
\label{sec:AIB}
\subsection{Problem Description}
Following the standard \gls{ib} framework \cite{Alemi_2017_DVI, Tishby_1999_TIB}, we assume the joint distribution of the system variables as follows:
\begin{equation}
    p(\bm{a}, \bm{x}, \bm{z}, \hat{\bm{z}}, \bm{y}) = p(\bm{a})p(\bm{x}|\bm{a})p_{\phi}(\bm{z}|\bm{x})p(\hat{\bm{z}}|\bm{z})p_{\theta}(\bm{y}|\hat{\bm{z}}).
\end{equation}
This sets up the Markov chain depicted as:
\begin{align}
    A \leftrightarrow X \leftrightarrow Z \leftrightarrow \hat{Z} \leftrightarrow Y.
\end{align}
We introduce a performance metric \textit{bits per service} to measure communication efficiency, which is defined as $k\cdot c$, where $c$ represents bits per symbol. Thus, there exists a crucial trade-off between bits per service and inference accuracy. This relationship underpins the formulation of an \gls{ib}, where we seek to optimize the balance between information throughput and decision accuracy.

The transformation from reconstructed symbols $\hat{\bm{z}}$ to task-specific data $\bm{y}$ is designed to preserve task-specific information, aligning task-oriented paradigms with traditional and reconstruction-oriented approaches. Based on the \gls{ib} theory \cite{Tishby_1999_TIB, Alemi_2017_DVI}, we formulate the following optimization problem:
\begin{subequations}
    \begin{align}
    \min \quad&-I(A;Y) \\
    \text{s.t.} \quad&I(X;\hat{Z})-\zeta \leq 0, \\
    &I(A;Y) - I(A;\hat{Z})= 0, \label{eq_optim_data_pro}
    \end{align}
\end{subequations}
where $\zeta$ represents the upper bound of data rate depending on the channel. The data processing inequality \cite{Cover_1991_EoI} implies that, ideally, if $Y$ and $\hat{Z}$ contain equivalent information about the action $A$, the equality $I(A;Y) - I(A;\hat{Z})=0$ holds.

\begin{figure*}
    \begin{subequations}
    \begin{align}
    \mathcal{L}_{\text{IB}}(\bm{a}, \bm{x}; \phi,\theta)&= \underbrace{-I(A;Y)}_{\text{Distortion}}
+\beta_1(\underbrace{I(X;\hat{Z})}_{\text{Rate}}-\zeta) +\beta_2\underbrace{(I(A;Y)-I(A;\hat{Z}))}_{\text{Alignment}} \label{eq_IB_origin}\\
&\equiv -I(A;Y) + \hat{\beta}_{1}I(X;\hat{Z})- \hat{\beta}_{2}I(A;\hat{Z}) \label{eq_IB_beta_hat}\\
&\equiv \mathbb{E}_{\bm{a}, \bm{x}}[\mathbb{E}_{\bm{y}|\bm{x};\phi,\theta}[-\log p(\bm{a}|\bm{y})]
+\hat{\beta}_{1} D_{K L}(p_{\phi}(\hat{\bm{z}} | \bm{x}) \| p(\hat{\bm{z}})) 
+\hat{\beta}_{2} \mathbb{E}_{\hat{\bm{z}} \mid \bm{x};\phi}[-\log p(\bm{a}|\hat{\bm{z}})]] \label{eq_IB_simplify}
    \end{align}
    \label{eq_IB}
    \end{subequations}
    {\noindent} \rule[0pt]{17.8cm}{0.05em}
    % {\noindent} \rule[0pt]{16cm}{0.05em}
\end{figure*}

We further formulate this problem as \cref{eq_IB_origin}, where $\beta_1 > 0$ and $\beta_2 > 0$ are the Lagrange multipliers. The detailed derivation can be found in \cref{subsec:VIB}.
The first term $-I(A;Y)$ and the second term $I(X;\hat{Z})$ formalize the classic information bottleneck, meanwhile, the third term $[I(A;Y)-I(A;\hat{Z})]$ aligns the task-relevant information between the task-specific data $\bm{y}$ and the reconstructed symbols $\hat{\bm{z}}$.

In the case $\beta_2 \neq 1$, we define $\hat{\beta}_{1} = \frac{\beta_1}{1-\beta_2}$ and $\hat{\beta}_{2} = \frac{\beta_2}{1-\beta_2}$. Then \cref{eq_IB_origin} can be expressed as \cref{eq_IB_beta_hat}. In the case $\beta_2 = 1$, \cref{eq_IB_origin} is simplified to the classic \gls{ib} formulation \cite{Tishby_1999_TIB, Alemi_2017_DVI, Shao_2022_LTO}:
\begin{align}
    \mathcal{L}_{\text{IB}}(\bm{a}, \bm{x}; \phi,\theta)=& \underbrace{-I(A;\hat{Z})}_{\text{Distortion}}
+\beta_1\underbrace{I(X;\hat{Z})}_{\text{Rate}}. 
\end{align}
This extended \gls{ib} theory preserves more task-specific information, and the bits per service is the same as the previous \gls{ib} approaches. Meanwhile, it maintains the dimension and structure required for edge inference.

\subsection{Variational Approach}
\label{subsec:VIB}
With the objective function \cref{eq_IB_beta_hat}, we first illustrate how to compute each term for training \(\phi\) and \(\theta\). We start with the first term, $-I(A;Y)$, expressed as:
\begin{align}
    -I(A;Y)=& -\int p(\bm{a},\bm{y})\log{\frac{p(\bm{a}|\bm{y})}{p(\bm{a})}} \dif \bm{a} \dif \bm{y} \notag\\
    =& -\int p(\bm{a},\bm{y})\log{p(\bm{a}|\bm{y})} \dif \bm{a} \dif \bm{y} - H(A),
\end{align}
where $p(\bm{a}|\bm{y})$ is the posterior probability, which can be derived through the Markov Chain \cite{Alemi_2017_DVI, Shao_2022_LTO} as:
\begin{align}
    p(\bm{a}|\bm{y}) &= \int p(\bm{a},\bm{x},\bm{z},\hat{\bm{z}}|\bm{y})\dif{\bm{x}}\dif{\bm{z}}\dif{\hat{\bm{z}}} \notag\\
    =& \int\frac{p(\bm{a})p(\bm{x}|\bm{a})p_{\phi}(\bm{z}|\bm{x})p(\hat{\bm{z}}|\bm{z})p_{\theta}(\bm{y}|\hat{\bm{z}})}{p(\bm{y})}\dif{\bm{x}}\dif{\bm{z}}\dif{\hat{\bm{z}}}.
\end{align}
Given the complexity of this integration, we employ a neural network $q_{\psi}(\bm{a}|\bm{y})$ as a variational approximation to $p(\bm{a}|\bm{y})$. 

Denoting the \gls{kl} divergence as $D_{\text{KL}}$. According to the definition of \gls{kl} divergence \cite{Cover_1991_EoI}, we can derive the following expression:
\begin{align}
D_{\text{KL}}(p(\bm{a}|\bm{y}) \parallel &q_{\psi}(\bm{a}|\bm{y})) \notag\\
=&\int p(\bm{a},\bm{y})\log p(\bm{a}|\bm{y}) \dif\bm{a}\dif\bm{y} \notag\\
&- \int p(\bm{a},\bm{y}) \log q_{\psi}(\bm{a}|\bm{y}) \dif\bm{a}\dif\bm{y}.
\end{align}
Based on the fact that 
\begin{align}
    D_{\text{KL}}(p(\bm{a}|\bm{y}) \parallel q_{\psi}(\bm{a}|\bm{y})) \geq 0,
    \label{eq_KL_conditional}
\end{align} 
we have
\begin{align}
\int p(\bm{a},\bm{y})\log p(\bm{a}|\bm{y})& \dif\bm{a}\dif\bm{y} \notag\\
\geq &\int p(\bm{a},\bm{y})\log q_{\psi}(\bm{a}|\bm{y}) \dif\bm{a}\dif\bm{y},
\label{eq_KL_conditional_2}
\end{align}
which derives
\begin{align}
\mathbb{E}_{\bm{a},\bm{x}}\bigl[\mathbb{E}_{\bm{y}|\bm{x};\phi,\theta}&[-\log p(\bm{a}|\bm{y})]\bigr] \notag\\
&\leq \mathbb{E}_{\bm{a},\bm{x}}\left[\mathbb{E}_{\bm{y}|\bm{x};\phi,\theta}[-\log q_{\psi}(\bm{a}|\bm{y})]\right].
\label{eq_firstTerm_mean}
\end{align}
The detailed derivation of \cref{eq_firstTerm_mean} can be found in Appendix \ref{apd_fristTerm_mean}.

The second term $I(X;\hat{Z})$ \cite{Shao_2022_LTO} is formulated as:
\begin{align}
I(X;\hat{Z})=\mathbb{E}_{\bm{a},\bm{x}}\left[D_{\text{KL}}(p_{\phi}(\hat{\bm{z}} | \bm{x}) \| p(\hat{\bm{z}})) \right],
\end{align}
where the marginal probability is 
\begin{align}
    p(\hat{\bm{z}})=\int p(\bm{a})p(\bm{x}|\bm{a})p_{\phi}(\bm{z}|\bm{x})p(\hat{\bm{z}}|\bm{z}) \dif\bm{a}\dif\bm{x}\dif\bm{z}.
\end{align}
We adopt a Gaussian approximation $q(\hat{\bm{z}}) \sim \mathcal{N}(\bm{0}, I)$ as an estimation for $p(\hat{\bm{z}})$ \cite{Kingma_2013_Aev}. It is reasonable as the \gls{jscc} encoder generates a Gaussian distribution $p_{\phi}(\hat{\bm{z}}|\bm{x}) \sim \mathcal{N}(\bm{\mu}_{\phi}(\bm{x}), \bm{\sigma}_{\phi}^2(\bm{x})I)$, where $\bm{\mu}_{\phi}(\cdot)$ and $\bm{\sigma}_{\phi}(\cdot)$ are functions that map the input data $\bm{x}$ to the mean and standard deviation of the Gaussian distribution.

Since $D_{\text{KL}}(p(\hat{\bm{z}})\parallel q(\hat{\bm{z}})) \geq 0$, the following upper bound can be derived:
\begin{align}
    I(X;\hat{Z}) \leq \mathbb{E}_{\bm{a},\bm{x}}\left[D_{\text{KL}}(p_{\phi}(\hat{\bm{z}} | \bm{x}) \| q(\hat{\bm{z}})) \right],
\end{align}
where the \gls{kl} divergence can be calculated analytically by the method in \cite{Duchi_2007_DfL}.

Similar to \cref{eq_firstTerm_mean}, by using $q_{\delta}(\bm{a}|\hat{\bm{z}})$ as a variational approximation of $p(\bm{a}|\hat{\bm{z}})$, we have
\begin{align}
    \mathbb{E}_{\bm{a},\bm{x}}\bigl[\mathbb{E}_{\hat{\bm{z}}|\bm{x};\phi,\theta}[-&\log p(\bm{a}|\hat{\bm{z}})]\bigr] \notag\\
    &\leq \mathbb{E}_{\bm{a},\bm{x}}\left[\mathbb{E}_{\hat{\bm{z}}|\bm{x};\phi,\theta}[-\log q_{\delta}(\bm{a}|\hat{\bm{z}})]\right].
\end{align} 
The above extended \gls{vib} formulation determines the upper bound of the \gls{ib} objective function (\cref{eq_IB_simplify}), which can be expressed as:
\begin{align}
\mathcal{L}_{\text{VIB}}(\bm{a}, \bm{x};\phi, \theta)= \mathbb{E}_{\bm{a},\bm{x}}&\Bigl\{
\mathbb{E}_{\bm{y}|\bm{x};\phi,\theta}[-\log q_{\psi}(\bm{a}|\bm{y})] \notag\\
&+\hat{\beta}_1 D_{\text{KL}}(p_{\phi}(\hat{\bm{z}} | \bm{x}) \| q(\hat{\bm{z}})) \notag\\
&+\hat{\beta}_2 \mathbb{E}_{\hat{\bm{z}}|\bm{x};\phi,\theta}[-\log q_{\delta}(\bm{a}|\hat{\bm{z}})]
\Bigr\}.
\label{eq_VIB_theory}
\end{align}
Through Monte Carlo sampling, we train \(\phi\) and \(\theta\) by minimizing this objective function using stochastic gradient descent.
Specifically, given a mini-batch of data $\{(\bm{a}_i,\bm{x}_i)\}^\Omega_{i=1}$ with batch size $\Omega$, if the reconstructed \gls{jscc} symbols $\hat{\bm{z}}$ are sampled $J_1$ times and the task-specific data $\bm{y}$ are sampled $J_2$ times for each data pair, the following estimation can be obtained:
\begin{align}
    \mathcal{L}_{\text{VIB}}(\bm{a}, \bm{x};\phi, \theta)\cong \frac{1}{\Omega}\sum_{i=1}^{\Omega} 
&\left\{
\frac{1}{J_2}\sum_{j=1}^{J_2}[-\log q_{\psi}(\bm{a}_{i}|\bm{y}_{j})] \right. \notag\\ 
&\left. +\hat{\beta}_1 D_{\text{KL}}(p_{\phi}(\hat{\bm{z}} | \bm{x}_{i}) \| q(\hat{\bm{z}})) \right. \notag\\
&\left. +\frac{\hat{\beta}_2}{J_1}\sum_{j=1}^{J_1}[-\log q_{\delta}(\bm{a}_{i}|\hat{\bm{z}}_{j})]
\right\}.
\label{eq_VIB_sampling}
\end{align}

\section{JSCC Modulation}
\label{sec:modulation}
In existing communication standards, symbols are transmitted with specific constellation orders and designs.  
In this section, we develop a \gls{jscc} modulation scheme that can map arbitrary complex-valued \gls{jscc} symbols to a predefined constellation diagram with finite points, as shown in \cref{fig:Quantization}. In addition, we introduce a learning method to adjust the optimal constellation parameter according to the quantization loss. For clarity, we use \gls{qam} as an example. Note that our method can be easily extended to other modulation schemes.

\subsection{Quantization and Normalization}
\begin{figure*}[t]
    \begin{center}
    \includegraphics[width=1\linewidth]{Figure/Quantization.pdf}
    \end{center}
       \caption{An example of the JSCC modulation and signal transmission procedure for $\bm{z} \in \mathbb{C}^4$ using 16-QAM.}
    \label{fig:Quantization}
\end{figure*}
To enable the quantization of arbitrary complex-valued \gls{jscc} symbols into a predefined constellation diagram, the following rule is applied to each symbol:
\begin{align}
    \bar{z}_i=Q(z_i)=\arg\mathop{\min}_{e_j}\|z_i-e_j \|_2^2,
    \label{eq_quantization}
\end{align}
where $z_i \in \mathbb{C}$ represents the original symbol, $\bar{z_i} \in \mathbb{C}$ represents the quantized symbol, $i\in \{1, \cdots,k \}$, $Q(\cdot)$ denotes the quantization function, and $\|\cdot \|_2$ denote the $\ell^2$-norm. $e_j \in \{e_1, \cdots,e_u\}$ represents the predefined constellation points, where $e_j\in \mathbb{C}$, and $u$ denote the number of constellation points. This quantization operation can be extended to a vector as follows,
\begin{align}
    \bar{\bm{z}}=Q(\bm{z})=[Q(z_1), \cdots, Q(z_k)].
    \label{eq_quantization_vector}
\end{align}

Since the transmitted symbols should satisfy the average power constraint:
\begin{align}
    \frac{1}{k}\sum_{i=1}^{k} |\bar{z}_i|^2 \leq P_{\text{target}},
\end{align}
the channel input (normalized symbols) are given by:
\begin{align}
\bm{z}_\text{in} = \frac{\sqrt{P_{\text{target}}}}{\sqrt{P_{\bar{\bm{z}}}}}\cdot\bar{\bm{z}},
\end{align}
where $P_{\bar{\bm{z}}} = \frac{1}{k} \sum_{i=1}^{k} |\bar{z}_i|^2$ denotes the power of quantized symbols $\bar{\bm{z}}$.

The channel input $\bm{z}_\text{in}$ is transmitted through the channel $\bm{z}_{\text{out}} = h\cdot\bm{z}_{\text{in}} + \bm{n}$.
Assume that the receiver has the full \gls{csi} knowledge and knows $P_{\bar{\bm{z}}}$, in the case of the static channel, it can perform channel equalization:
\begin{align}
    \check{\bm{z}} = \frac{h^*}{|h|^2}\bm{z}_{\text{out}},
\end{align}
where $h^*$ denotes the conjugate of channel coefficient $h$ and $\check{\bm{z}}$ denotes the equalized symbols.  After equalization, the equalized symbols should be scaled as
\begin{align}
    \tilde{\bm{z}} = \frac{\sqrt{P_{\bar{\bm{z}}}}}{\sqrt{P_{\text{target}}}}\cdot\check{\bm{z}},
\end{align}
where $\tilde{\bm{z}}$ denotes the scaled symbols. Then the reconstructed symbols $\hat{\bm{z}}=Q(\tilde{\bm{z}})$ can be obtained by \cref{eq_quantization_vector}.

\subsection{Learnable Constellation Diagram and Fine-Tuning}
\label{subsec:constellation}
Traditional modulation techniques, such as \gls{qam}, employ a lookup table that maps bits to constellation points. In contrast, the complex-valued channel symbols produced by the \gls{jscc} encoder are continuous, necessitating a different approach for their mapping.

\Cref{eq_quantization} demonstrates that the coordinates of each constellation point $e_j$ directly affect the quantization outcome. We propose a learnable constellation diagram that adapts to the observed space of \gls{jscc} symbols, minimizes quantization loss, and improves performance with the \gls{jscc} encoder and the information reshaper. Taking $u$-\gls{qam} as an example, where $u$ denotes the number of constellation points, the coordinates of each constellation point can be derived by the parameter $r$. This parameter specifies the distance between two constellation points located at the corners of one side, as illustrated in \cref{fig:Quantization}. Then, the real part and imaginary part of the constellation point \(e_j\) are given as follows:
\begin{align}
\Re(e_{j})=&-\frac{r}{2}+\frac{(j\bmod\sqrt{u})\cdot r}{\sqrt{u}-1}, \\
\Im(e_{j})=&\frac{r}{2}-\frac{\lfloor{j/\sqrt{u}}\rfloor\cdot r}{\sqrt{u}-1},
\end{align}
where ``\(\,\bmod\,\)'' denotes the modulo operation and $\lfloor\cdot\rfloor$ denotes the rounding down function.

The quantization loss is defined as
\begin{align}
    \mathcal{L}_{Q}(\bm{z};r)=\frac{1}{k}\sum_{i=1}^{k}\mathop{\min}_{e_j}\|z_i-e_j \|_2.
\end{align}

The training process for the learnable constellation diagram begins with the initialization of the constellation parameter \(r\) to a predefined value \(r_{\text{init}}\), along with loading a pre-trained \gls{jscc} encoder. Using an image dataset \(\mathcal{X}\) with corresponding ground truth actions \(\mathcal{A}\), mini-batches are sampled iteratively during training. For each mini-batch, images are encoded into \gls{jscc} symbols, and the average batch loss is computed based on the quantization error. The constellation parameter \(r\) is then updated by backpropagation until convergence. The output of this process is the optimal constellation parameter \(r^*\). The detailed constellation parameter training process is provided in \cref{alg_quantization}. Once the optimal $r^*$ is obtained, the \gls{jscc} encoder and the information reshaper are jointly fine-tuned using the extended loss function:
\begin{align}
    \mathcal{L}_{\text{VIB-}Q}(\bm{a}, \bm{x};\phi,\theta)=\mathcal{L}_{\text{VIB}}(\bm{a}, \bm{x};\phi,\theta) + \beta_{Q}\mathcal{L}_{Q}(\bm{z};r^*),
    \label{eq_loss_VIBQ}
\end{align}
where $\beta_{Q}$ is a hyperparameter that balances the quantization loss with the original \gls{vib} loss.

This method enhances the practical applicability of \gls{jscc} modulation by integrating it with established digital communication systems while preserving the benefits of customized encoding and decoding strategies.

\begin{algorithm}[t]
\caption{Training Learnable Constellation Diagram}
% This label should be put after caption
\label{alg_quantization}
\begin{algorithmic}[1]
\Statex \textbf{Initialization}: Initialize the constellation parameter
\Statex $r\rightarrow r_{\text{init}}$, and load pre-trained \gls{jscc} encoder $p_{\phi}(\bm{z}|\bm{x})$.
\State \textbf{Input}: Image dataset $\mathcal{X}$ with corresponding ground truth action $\mathcal{A}$.

\While{not converged}
    \State Sample mini-batch $\{(\bm{a}_i, \bm{x}_i) \}_{i=1}^{\Omega}$ from $\mathcal{X}$ and $\mathcal{A}$.
    \State Encode image $\{\bm{x}_i\}_{i=1}^{\Omega}$ to symbols $\{\bm{z}_i\}_{i=1}^{\Omega}$.
    \State Compute the average batch loss
    \Statex \quad \; $\frac{1}{\Omega}\sum_{i=1}^{\Omega}\mathcal{L}_{Q}(\bm{z}_i;r)$.
    \State Update parameter $r$ through backpropagation.
\EndWhile
\State \textbf{Output}: Optimal constellation parameter $r^*$.

\end{algorithmic}
\end{algorithm}






 
\section{Extended VIB for Edge-based Autonomous Driving}
\label{sec:edge_AI}
\gls{tgcp}\footnote{To avoid confusion with the Transmission Control Protocol (TCP), we denote Trajectory-guided Control Prediction as TGCP in this paper.} is the state-of-the-art \gls{e2e} self-driving framework that combines trajectory planning and multi-stage control prediction into a unified neural network \cite{Wu_2022_TgC}. 
This framework, notable for using only a monocular camera, ranks third on the CARLA leaderboard\footnote{\url{https://leaderboard.carla.org/leaderboard/}}. 
We extend \gls{vib} to \gls{tgcp} to examine its applicability in an edge-based autonomous driving system.

\subsection{Background of TGCP}
\gls{tgcp} at the edge server processes task-specific data $\bm{y}$ and additional state information $\bm{m}$ to make driving decisions. The state information includes variables such as speed, destination coordinates, and current driving guidance (e.g., ``turn left'' or ``follow the lane''). For this study, we assume that $\bm{m}$ can be transmitted losslessly to the edge server. 

The autonomous driving agent is modeled as $q_{\psi}(\bm{a}|\bm{y})$, which generates the inferred action $\hat{\bm{a}}$ from task-specific data $\bm{y}$. In particular, the individual components of the inferred action $\hat{\bm{a}} = (\hat{v}, \hat{s}, \hat{\bm{w}}, \hat{\bm{f}}^{\text{traj}}, \hat{\bm{b}}, \hat{\bm{f}}^{\text{ctrl}})$ are defined as follows:
\begin{itemize}
    \item $\hat{v}$: estimated target speed.
    \item $\hat{s}$: value of the extracted features estimated by the expert \cite{Zhang_2021_EtE}.
    \item $\hat{\bm{w}}$: predicted waypoints from the trajectory branch.
    \item $\hat{\bm{f}}^{\text{traj}}$: estimated extracted features for trajectory planning.
    \item $\hat{\bm{b}} = [\hat{\bm{b}}_0, \hat{\bm{b}}_1, \dots, \hat{\bm{b}}_\Gamma]$: estimated control actions from the beta distribution in the control prediction branch, where $\Gamma$ denotes the prediction length.
    \item $\hat{\bm{f}}^{\text{ctrl}} = [\hat{\bm{f}}^{\text{ctrl}}_0, \hat{\bm{f}}^{\text{ctrl}}_1, \dots, \hat{\bm{f}}^{\text{ctrl}}_\Gamma]$: predicted informative features of the control prediction branch.
\end{itemize}

\subsection{Control and Trajectory Prediction Loss Functions}
The designed controller, based on \cite{Wu_2022_TgC}, computes control commands such as throttle, steer, and brake using the output of the trajectory and control prediction branches. The corresponding loss functions are defined as:
\begin{align}
    \mathcal{L}_{\text{traj}} =& \|\bm{w} - \hat{\bm{w}}\|_1 + \lambda_{\text{feat}}\|\bm{f}^{\text{traj}} - \hat{\bm{f}}^{\text{traj}}\|_2, \\
    \mathcal{L}_{\text{ctrl}}
    =& D_{\text{KL}}(\mathcal{B}e(\bm{b}_{0})\|\mathcal{B}e(\hat{\bm{b}}_{0}))  \notag\\
    &+\frac{1}{\Gamma}\sum_{i=1}^{\Gamma}D_{\text{KL}}(\mathcal{B}e(\bm{b}_{i})\|\mathcal{B}e(\hat{\bm{b}}_{i})) \notag\\
    & + \lambda_{\text{feat}}\|\bm{f}^{\text{ctrl}}_{0} - \hat{\bm{f}}^{\text{ctrl}}_{0}\|_{2}
    + \frac{1}{\Gamma}\sum_{i=1}^{\Gamma}\|\bm{f}^{\text{ctrl}}_{i} - \hat{\bm{f}}^{\text{ctrl}}_{i}\|_{2},
\end{align}
where $\lambda_\text{feat}$ is a hyperparameter, $\bm{w}$, $\bm{f}^{\text{traj}}$, $\bm{b}_{i}$, and $\bm{f}_{i}^{\text{ctrl}}$ are from the ground truth action $\bm{a}$, $\|\cdot\|_{1}$ denotes the $\ell_{1}$-norm, and $\mathcal{B}e(\cdot)$ denotes the beta distribution.

Furthermore, the auxiliary loss function is defined as:
\begin{equation}
    \mathcal{L}_{\text{aux}} = \|v-\hat{v}\|_{1} + \|s-\hat{s}\|_{2},
\end{equation}
where speed $v$ and value of features $s$ are from the ground truth action $\bm{a}$. Combining these terms, the overall loss function $\mathcal{L}_{\text{TCGP}}$ becomes:
\begin{equation}
    \mathcal{L}_{\text{TCGP}} = \lambda_{\text{traj}}\mathcal{L}_{\text{traj}} + \lambda_{\text{ctrl}}\mathcal{L}_{\text{ctrl}} + \lambda_{\text{aux}}\mathcal{L}_{\text{aux}},
\end{equation}
where $\lambda_{\text{traj}}$, $\lambda_{\text{ctrl}}$, and $\lambda_{\text{aux}}$ are hyperparameters.

\subsection{Task-Oriented End-to-End Training}
\label{subsec_task-oriented_training}
Typically, we assume that the posterior $q_{\psi}(\bm{a}|\bm{y})$ follows a Gaussian distribution $\mathcal{N}(\bm{\mu}_{\psi}(\bm{y}), \bm{\Sigma}_{\psi}(\bm{y}))$, where $\bm{\mu}_{\psi}(\bm{y})\in\mathbb{R}^{d}$ and $\bm{\Sigma}_{\psi}(\bm{y})=\sigma_{c}^{2}I_{d}$ ($\sigma_{c}$ is a constant). According to the probability density function of the Gaussian distribution, we can derive the following expression,
\begin{align}
    -\log{q_{\psi}(\bm{a}|\bm{y})}\sim \frac{1}{2\sigma^{2}_{c}}\|\bm{a}-\bm{\mu}_{\psi}(\bm{y}) \|^2_2,
    \label{eq_log2normal}
\end{align}
where $\bm{\mu}_{\psi}(\bm{y})=\hat{\bm{a}}$.
Details of the derivation are deferred to the Appendix \ref{apd:derivation_log}. \cref{eq_log2normal} shows that $-\log q_{\psi}(\bm{a}|\bm{y})$ can serve as a distance metric, like the $\ell^2$-norm. 
Since $\mathcal{L}_{\text{TCGP}}$ is a combination of distance metric of action $\bm{a}$ ($\ell^1$-norm, $\ell^2$-norm, and \gls{kl} divergence), we heuristically propose substituting the first term in \cref{eq_VIB_theory} with $\mathcal{L}_{\text{TCGP}}$ to adapt the objective function as:
\begin{align}
\mathcal{L}_{\text{VIB}}'(\bm{a}, \bm{x};\phi, \theta)=\mathbb{E}_{\bm{a},\bm{x}}&\Bigl\{
\mathcal{L}_{\text{TCGP}} \notag\\
&+\hat{\beta}_1 D_{\text{KL}}(p_{\phi}(\hat{\bm{z}} | \bm{x}) \| q(\hat{\bm{z}})) \notag\\
&+\hat{\beta}_2 \mathbb{E}_{\hat{\bm{z}}|\bm{x};\phi,\theta}[-\log q_{\delta}(\bm{a}|\hat{\bm{z}})]
\Bigr\}.
\label{eq_VIB_2}
\end{align}
In addition, the \cref{eq_loss_VIBQ} can be modified as:
\begin{align}
    \mathcal{L}_{\text{VIB-}Q}'(\bm{a}, \bm{x};\phi,\theta)=\mathcal{L}_{\text{VIB}}'(\bm{a}, \bm{x};\phi,\theta) + \beta_{Q}\mathcal{L}_{Q}(\bm{z};r^*).
    \label{eq_loss_VIBQ_2}
\end{align}

Training of \gls{jscc} encoder and information reshaper consists of two stages: pre-training and fine-tuning. In pre-training, the neural network parameters (\(\phi\) and \(\theta\)) are initialized, and images from the dataset are encoded into \gls{jscc} symbols, transmitted through a channel without modulation, and reshaped into task-specific data. The \gls{tgcp} model, with frozen parameters, generates inferred actions \(\hat{\bm{a}}\), and the loss \(\mathcal{L}_{\text{VIB}}'\) is computed to update the network parameters. Fine-tuning follows a similar process, but the symbols are transmitted with JSCC modulation, and the loss \(\mathcal{L}_{\text{VIB-}Q}'\) is used for parameter updates. Finally, the optimized parameters \(\phi\) and \(\theta\) are output. The training process of the proposed aligned task- and reconstruction-oriented \gls{jscc} encoder and information reshaper is shown in \cref{alg_training}. Here, \(\text{CH}(\cdot)\) denotes the function of a \gls{jscc} modulation and communication channel, while \(\text{TGCP}(\cdot)\) denotes the function of \gls{tgcp}. Specifically, during the fine-tuning process, both the JSCC encoder and the information reshaper are actively adjusted, which means that neither component is frozen. This fine-tuning process reduces the quantization loss of the encoder's output and preserves task-critical information, showing the potential for real-world applications.

\begin{algorithm}
\setstretch{1.35}
\caption{\ourslong Training}
\label{alg:training}
\begin{algorithmic}[1]
\State
\textbf{Require}: demonstration dataset, $\mathcal{D} = \{(\textbf{O}_i, \textbf{A}_i)\}_{i=1}^N$; denoising model, $\varepsilon_\theta$; number of diffusion steps, $f$
\Repeat
    \State Sample $(\textbf{O}, \textbf{A}) \sim \mathcal{D}$
    \State Sample $p \sim \mathrm{Unif}(0, 1)$; Sample noise $\bm{\epsilon} \sim \mathcal{N}(0, \mathbf{I})$ and reshape to $\mathbb{R}^{C_a \times f}$
    \If{$p \leq p_{\mathrm{linear}}$}
        \State $\mathbf{k} = \{k_1=1, \cdots, k_f=f\}$
            \Comment{linear schedule}
    \Else
        \State $\mathbf{k} = \{k_1 \sim \mathrm{Unif}(\{1, \cdots, f\}), \cdots, k_f \sim \mathrm{Unif}(\{1, \cdots, f\})\}$
            \Comment{random schedule}
    \EndIf
    \ForAll {$\mathbf{a}_j \in \mathbf{A}$ indexed by frame index $j$}
        \State $\hat{\mathbf{a}}_j = \sqrt{\bar{\alpha}_{t_j}} \mathbf{a}_j + \sqrt{1 - \bar{\alpha}_{t_j}} \bm{\epsilon}_j$
            \Comment{perturbe each $\mathbf{a}_j$ independently}
    \EndFor
    \State $\hat{\mathbf{A}} = \{\hat{\mathbf{a}}_0, \cdots, \hat{\mathbf{a}}_{f-1}\}$
    \State Take gradient descent step to update $\theta$ on
    \State     $\quad \ \ \nabla_\theta \| \bm{\epsilon} - \varepsilon_{\theta}(\hat{\mathbf{A}}; \color{darkblue}{\mathbf{O}, \mathbf{k}}\color{black}{) \|}$
        \Comment{\textcolor{darkblue}{noise-aware conditioning}}
\Until{converged}
\end{algorithmic}
\end{algorithm}





\section{Experiments}
\label{section5}

In this section, we conduct extensive experiments to show that \ourmethod~can significantly speed up the sampling of existing MR Diffusion. To rigorously validate the effectiveness of our method, we follow the settings and checkpoints from \cite{luo2024daclip} and only modify the sampling part. Our experiment is divided into three parts. Section \ref{mainresult} compares the sampling results for different NFE cases. Section \ref{effects} studies the effects of different parameter settings on our algorithm, including network parameterizations and solver types. In Section \ref{analysis}, we visualize the sampling trajectories to show the speedup achieved by \ourmethod~and analyze why noise prediction gets obviously worse when NFE is less than 20.


\subsection{Main results}\label{mainresult}

Following \cite{luo2024daclip}, we conduct experiments with ten different types of image degradation: blurry, hazy, JPEG-compression, low-light, noisy, raindrop, rainy, shadowed, snowy, and inpainting (see Appendix \ref{appd1} for details). We adopt LPIPS \citep{zhang2018lpips} and FID \citep{heusel2017fid} as main metrics for perceptual evaluation, and also report PSNR and SSIM \citep{wang2004ssim} for reference. We compare \ourmethod~with other sampling methods, including posterior sampling \citep{luo2024posterior} and Euler-Maruyama discretization \citep{kloeden1992sde}. We take two tasks as examples and the metrics are shown in Figure \ref{fig:main}. Unless explicitly mentioned, we always use \ourmethod~based on SDE solver, with data prediction and uniform $\lambda$. The complete experimental results can be found in Appendix \ref{appd3}. The results demonstrate that \ourmethod~converges in a few (5 or 10) steps and produces samples with stable quality. Our algorithm significantly reduces the time cost without compromising sampling performance, which is of great practical value for MR Diffusion.


\begin{figure}[!ht]
    \centering
    \begin{minipage}[b]{0.45\textwidth}
        \centering
        \includegraphics[width=1\textwidth, trim=0 20 0 0]{figs/main_result/7_lowlight_fid.pdf}
        \subcaption{FID on \textit{low-light} dataset}
        \label{fig:main(a)}
    \end{minipage}
    \begin{minipage}[b]{0.45\textwidth}
        \centering
        \includegraphics[width=1\textwidth, trim=0 20 0 0]{figs/main_result/7_lowlight_lpips.pdf}
        \subcaption{LPIPS on \textit{low-light} dataset}
        \label{fig:main(b)}
    \end{minipage}
    \begin{minipage}[b]{0.45\textwidth}
        \centering
        \includegraphics[width=1\textwidth, trim=0 20 0 0]{figs/main_result/10_motion_fid.pdf}
        \subcaption{FID on \textit{motion-blurry} dataset}
        \label{fig:main(c)}
    \end{minipage}
    \begin{minipage}[b]{0.45\textwidth}
        \centering
        \includegraphics[width=1\textwidth, trim=0 20 0 0]{figs/main_result/10_motion_lpips.pdf}
        \subcaption{LPIPS on \textit{motion-blurry} dataset}
        \label{fig:main(d)}
    \end{minipage}
    \caption{\textbf{Perceptual evaluations on \textit{low-light} and \textit{motion-blurry} datasets.}}
    \label{fig:main}
\end{figure}

\subsection{Effects of parameter choice}\label{effects}

In Table \ref{tab:ablat_param}, we compare the results of two network parameterizations. The data prediction shows stable performance across different NFEs. The noise prediction performs similarly to data prediction with large NFEs, but its performance deteriorates significantly with smaller NFEs. The detailed analysis can be found in Section \ref{section5.3}. In Table \ref{tab:ablat_solver}, we compare \ourmethod-ODE-d-2 and \ourmethod-SDE-d-2 on the \textit{inpainting} task, which are derived from PF-ODE and reverse-time SDE respectively. SDE-based solver works better with a large NFE, whereas ODE-based solver is more effective with a small NFE. In general, neither solver type is inherently better.


% In Table \ref{tab:hazy}, we study the impact of two step size schedules on the results. On the whole, uniform $\lambda$ performs slightly better than uniform $t$. Our algorithm follows the method of \cite{lu2022dpmsolverplus} to estimate the integral part of the solution, while the analytical part does not affect the error.  Consequently, our algorithm has the same global truncation error, that is $\mathcal{O}\left(h_{max}^{k}\right)$. Note that the initial and final values of $\lambda$ depend on noise schedule and are fixed. Therefore, uniform $\lambda$ scheduling leads to the smallest $h_{max}$ and works better.

\begin{table}[ht]
    \centering
    \begin{minipage}{0.5\textwidth}
    \small
    \renewcommand{\arraystretch}{1}
    \centering
    \caption{Ablation study of network parameterizations on the Rain100H dataset.}
    % \vspace{8pt}
    \resizebox{1\textwidth}{!}{
        \begin{tabular}{cccccc}
			\toprule[1.5pt]
            % \multicolumn{6}{c}{Rainy} \\
            % \cmidrule(lr){1-6}
             NFE & Parameterization      & LPIPS\textdownarrow & FID\textdownarrow &  PSNR\textuparrow & SSIM\textuparrow  \\
            \midrule[1pt]
            \multirow{2}{*}{50}
             & Noise Prediction & \textbf{0.0606}     & \textbf{27.28}   & \textbf{28.89}     & \textbf{0.8615}    \\
             & Data Prediction & 0.0620     & 27.65   & 28.85     & 0.8602    \\
            \cmidrule(lr){1-6}
            \multirow{2}{*}{20}
              & Noise Prediction & 0.1429     & 47.31   & 27.68     & 0.7954    \\
              & Data Prediction & \textbf{0.0635}     & \textbf{27.79}   & \textbf{28.60}     & \textbf{0.8559}    \\
            \cmidrule(lr){1-6}
            \multirow{2}{*}{10}
              & Noise Prediction & 1.376     & 402.3   & 6.623     & 0.0114    \\
              & Data Prediction & \textbf{0.0678}     & \textbf{29.54}   & \textbf{28.09}     & \textbf{0.8483}    \\
            \cmidrule(lr){1-6}
            \multirow{2}{*}{5}
              & Noise Prediction & 1.416     & 447.0   & 5.755     & 0.0051    \\
              & Data Prediction & \textbf{0.0637}     & \textbf{26.92}   & \textbf{28.82}     & \textbf{0.8685}    \\       
            \bottomrule[1.5pt]
        \end{tabular}}
        \label{tab:ablat_param}
    \end{minipage}
    \hspace{0.01\textwidth}
    \begin{minipage}{0.46\textwidth}
    \small
    \renewcommand{\arraystretch}{1}
    \centering
    \caption{Ablation study of solver types on the CelebA-HQ dataset.}
    % \vspace{8pt}
        \resizebox{1\textwidth}{!}{
        \begin{tabular}{cccccc}
			\toprule[1.5pt]
            % \multicolumn{6}{c}{Raindrop} \\     
            % \cmidrule(lr){1-6}
             NFE & Solver Type     & LPIPS\textdownarrow & FID\textdownarrow &  PSNR\textuparrow & SSIM\textuparrow  \\
            \midrule[1pt]
            \multirow{2}{*}{50}
             & ODE & 0.0499     & 22.91   & 28.49     & 0.8921    \\
             & SDE & \textbf{0.0402}     & \textbf{19.09}   & \textbf{29.15}     & \textbf{0.9046}    \\
            \cmidrule(lr){1-6}
            \multirow{2}{*}{20}
              & ODE & 0.0475    & 21.35   & 28.51     & 0.8940    \\
              & SDE & \textbf{0.0408}     & \textbf{19.13}   & \textbf{28.98}    & \textbf{0.9032}    \\
            \cmidrule(lr){1-6}
            \multirow{2}{*}{10}
              & ODE & \textbf{0.0417}    & 19.44   & \textbf{28.94}     & \textbf{0.9048}    \\
              & SDE & 0.0437     & \textbf{19.29}   & 28.48     & 0.8996    \\
            \cmidrule(lr){1-6}
            \multirow{2}{*}{5}
              & ODE & \textbf{0.0526}     & 27.44   & \textbf{31.02}     & \textbf{0.9335}    \\
              & SDE & 0.0529    & \textbf{24.02}   & 28.35     & 0.8930    \\
            \bottomrule[1.5pt]
        \end{tabular}}
        \label{tab:ablat_solver}
    \end{minipage}
\end{table}


% \renewcommand{\arraystretch}{1}
%     \centering
%     \caption{Ablation study of step size schedule on the RESIDE-6k dataset.}
%     % \vspace{8pt}
%         \resizebox{1\textwidth}{!}{
%         \begin{tabular}{cccccc}
% 			\toprule[1.5pt]
%             % \multicolumn{6}{c}{Raindrop} \\     
%             % \cmidrule(lr){1-6}
%              NFE & Schedule      & LPIPS\textdownarrow & FID\textdownarrow &  PSNR\textuparrow & SSIM\textuparrow  \\
%             \midrule[1pt]
%             \multirow{2}{*}{50}
%              & uniform $t$ & 0.0271     & 5.539   & 30.00     & 0.9351    \\
%              & uniform $\lambda$ & \textbf{0.0233}     & \textbf{4.993}   & \textbf{30.19}     & \textbf{0.9427}    \\
%             \cmidrule(lr){1-6}
%             \multirow{2}{*}{20}
%               & uniform $t$ & 0.0313     & 6.000   & 29.73     & 0.9270    \\
%               & uniform $\lambda$ & \textbf{0.0240}     & \textbf{5.077}   & \textbf{30.06}    & \textbf{0.9409}    \\
%             \cmidrule(lr){1-6}
%             \multirow{2}{*}{10}
%               & uniform $t$ & 0.0309     & 6.094   & 29.42     & 0.9274    \\
%               & uniform $\lambda$ & \textbf{0.0246}     & \textbf{5.228}   & \textbf{29.65}     & \textbf{0.9372}    \\
%             \cmidrule(lr){1-6}
%             \multirow{2}{*}{5}
%               & uniform $t$ & 0.0256     & 5.477   & \textbf{29.91}     & 0.9342    \\
%               & uniform $\lambda$ & \textbf{0.0228}     & \textbf{5.174}   & 29.65     & \textbf{0.9416}    \\
%             \bottomrule[1.5pt]
%         \end{tabular}}
%         \label{tab:ablat_schedule}



\subsection{Analysis}\label{analysis}
\label{section5.3}

\begin{figure}[ht!]
    \centering
    \begin{minipage}[t]{0.6\linewidth}
        \centering
        \includegraphics[width=\linewidth, trim=0 20 10 0]{figs/trajectory_a.pdf} %trim左下右上
        \subcaption{Sampling results.}
        \label{fig:traj(a)}
    \end{minipage}
    \begin{minipage}[t]{0.35\linewidth}
        \centering
        \includegraphics[width=\linewidth, trim=0 0 0 0]{figs/trajectory_b.pdf} %trim左下右上
        \subcaption{Trajectory.}
        \label{fig:traj(b)}
    \end{minipage}
    \caption{\textbf{Sampling trajectories.} In (a), we compare our method (with order 1 and order 2) and previous sampling methods (i.e., posterior sampling and Euler discretization) on a motion blurry image. The numbers in parentheses indicate the NFE. In (b), we illustrate trajectories of each sampling method. Previous methods need to take many unnecessary paths to converge. With few NFEs, they fail to reach the ground truth (i.e., the location of $\boldsymbol{x}_0$). Our methods follow a more direct trajectory.}
    \label{fig:traj}
\end{figure}

\textbf{Sampling trajectory.}~ Inspired by the design idea of NCSN \citep{song2019ncsn}, we provide a new perspective of diffusion sampling process. \cite{song2019ncsn} consider each data point (e.g., an image) as a point in high-dimensional space. During the diffusion process, noise is added to each point $\boldsymbol{x}_0$, causing it to spread throughout the space, while the score function (a neural network) \textit{remembers} the direction towards $\boldsymbol{x}_0$. In the sampling process, we start from a random point by sampling a Gaussian distribution and follow the guidance of the reverse-time SDE (or PF-ODE) and the score function to locate $\boldsymbol{x}_0$. By connecting each intermediate state $\boldsymbol{x}_t$, we obtain a sampling trajectory. However, this trajectory exists in a high-dimensional space, making it difficult to visualize. Therefore, we use Principal Component Analysis (PCA) to reduce $\boldsymbol{x}_t$ to two dimensions, obtaining the projection of the sampling trajectory in 2D space. As shown in Figure \ref{fig:traj}, we present an example. Previous sampling methods \citep{luo2024posterior} often require a long path to find $\boldsymbol{x}_0$, and reducing NFE can lead to cumulative errors, making it impossible to locate $\boldsymbol{x}_0$. In contrast, our algorithm produces more direct trajectories, allowing us to find $\boldsymbol{x}_0$ with fewer NFEs.

\begin{figure*}[ht]
    \centering
    \begin{minipage}[t]{0.45\linewidth}
        \centering
        \includegraphics[width=\linewidth, trim=0 0 0 0]{figs/convergence_a.pdf} %trim左下右上
        \subcaption{Sampling results.}
        \label{fig:convergence(a)}
    \end{minipage}
    \begin{minipage}[t]{0.43\linewidth}
        \centering
        \includegraphics[width=\linewidth, trim=0 20 0 0]{figs/convergence_b.pdf} %trim左下右上
        \subcaption{Ratio of convergence.}
        \label{fig:convergence(b)}
    \end{minipage}
    \caption{\textbf{Convergence of noise prediction and data prediction.} In (a), we choose a low-light image for example. The numbers in parentheses indicate the NFE. In (b), we illustrate the ratio of components of neural network output that satisfy the Taylor expansion convergence requirement.}
    \label{fig:converge}
\end{figure*}

\textbf{Numerical stability of parameterizations.}~ From Table 1, we observe poor sampling results for noise prediction in the case of few NFEs. The reason may be that the neural network parameterized by noise prediction is numerically unstable. Recall that we used Taylor expansion in Eq.(\ref{14}), and the condition for the equality to hold is $|\lambda-\lambda_s|<\boldsymbol{R}(s)$. And the radius of convergence $\boldsymbol{R}(t)$ can be calculated by
\begin{equation}
\frac{1}{\boldsymbol{R}(t)}=\lim_{n\rightarrow\infty}\left|\frac{\boldsymbol{c}_{n+1}(t)}{\boldsymbol{c}_n(t)}\right|,
\end{equation}
where $\boldsymbol{c}_n(t)$ is the coefficient of the $n$-th term in Taylor expansion. We are unable to compute this limit and can only compute the $n=0$ case as an approximation. The output of the neural network can be viewed as a vector, with each component corresponding to a radius of convergence. At each time step, we count the ratio of components that satisfy $\boldsymbol{R}_i(s)>|\lambda-\lambda_s|$ as a criterion for judging the convergence, where $i$ denotes the $i$-th component. As shown in Figure \ref{fig:converge}, the neural network parameterized by data prediction meets the convergence criteria at almost every step. However, the neural network parameterized by noise prediction always has components that cannot converge, which will lead to large errors and failed sampling. Therefore, data prediction has better numerical stability and is a more recommended choice.


In this paper, we systematically investigate the position bias problem in the multi-constraint instruction following. To quantitatively measure the disparity of constraint order, we propose a novel Difficulty Distribution Index (CDDI). Based on the CDDI, we design a probing task. First, we construct a large number of instructions consisting of different constraint orders. Then, we conduct experiments in two distinct scenarios. Extensive results reveal a clear preference of LLMs for ``hard-to-easy'' constraint orders. To further explore this, we conduct an explanation study. We visualize the importance of different constraints located in different positions and demonstrate the strong correlation between the model's attention distribution and its performance.
\subsection{Lloyd-Max Algorithm}
\label{subsec:Lloyd-Max}
For a given quantization bitwidth $B$ and an operand $\bm{X}$, the Lloyd-Max algorithm finds $2^B$ quantization levels $\{\hat{x}_i\}_{i=1}^{2^B}$ such that quantizing $\bm{X}$ by rounding each scalar in $\bm{X}$ to the nearest quantization level minimizes the quantization MSE. 

The algorithm starts with an initial guess of quantization levels and then iteratively computes quantization thresholds $\{\tau_i\}_{i=1}^{2^B-1}$ and updates quantization levels $\{\hat{x}_i\}_{i=1}^{2^B}$. Specifically, at iteration $n$, thresholds are set to the midpoints of the previous iteration's levels:
\begin{align*}
    \tau_i^{(n)}=\frac{\hat{x}_i^{(n-1)}+\hat{x}_{i+1}^{(n-1)}}2 \text{ for } i=1\ldots 2^B-1
\end{align*}
Subsequently, the quantization levels are re-computed as conditional means of the data regions defined by the new thresholds:
\begin{align*}
    \hat{x}_i^{(n)}=\mathbb{E}\left[ \bm{X} \big| \bm{X}\in [\tau_{i-1}^{(n)},\tau_i^{(n)}] \right] \text{ for } i=1\ldots 2^B
\end{align*}
where to satisfy boundary conditions we have $\tau_0=-\infty$ and $\tau_{2^B}=\infty$. The algorithm iterates the above steps until convergence.

Figure \ref{fig:lm_quant} compares the quantization levels of a $7$-bit floating point (E3M3) quantizer (left) to a $7$-bit Lloyd-Max quantizer (right) when quantizing a layer of weights from the GPT3-126M model at a per-tensor granularity. As shown, the Lloyd-Max quantizer achieves substantially lower quantization MSE. Further, Table \ref{tab:FP7_vs_LM7} shows the superior perplexity achieved by Lloyd-Max quantizers for bitwidths of $7$, $6$ and $5$. The difference between the quantizers is clear at 5 bits, where per-tensor FP quantization incurs a drastic and unacceptable increase in perplexity, while Lloyd-Max quantization incurs a much smaller increase. Nevertheless, we note that even the optimal Lloyd-Max quantizer incurs a notable ($\sim 1.5$) increase in perplexity due to the coarse granularity of quantization. 

\begin{figure}[h]
  \centering
  \includegraphics[width=0.7\linewidth]{sections/figures/LM7_FP7.pdf}
  \caption{\small Quantization levels and the corresponding quantization MSE of Floating Point (left) vs Lloyd-Max (right) Quantizers for a layer of weights in the GPT3-126M model.}
  \label{fig:lm_quant}
\end{figure}

\begin{table}[h]\scriptsize
\begin{center}
\caption{\label{tab:FP7_vs_LM7} \small Comparing perplexity (lower is better) achieved by floating point quantizers and Lloyd-Max quantizers on a GPT3-126M model for the Wikitext-103 dataset.}
\begin{tabular}{c|cc|c}
\hline
 \multirow{2}{*}{\textbf{Bitwidth}} & \multicolumn{2}{|c|}{\textbf{Floating-Point Quantizer}} & \textbf{Lloyd-Max Quantizer} \\
 & Best Format & Wikitext-103 Perplexity & Wikitext-103 Perplexity \\
\hline
7 & E3M3 & 18.32 & 18.27 \\
6 & E3M2 & 19.07 & 18.51 \\
5 & E4M0 & 43.89 & 19.71 \\
\hline
\end{tabular}
\end{center}
\end{table}

\subsection{Proof of Local Optimality of LO-BCQ}
\label{subsec:lobcq_opt_proof}
For a given block $\bm{b}_j$, the quantization MSE during LO-BCQ can be empirically evaluated as $\frac{1}{L_b}\lVert \bm{b}_j- \bm{\hat{b}}_j\rVert^2_2$ where $\bm{\hat{b}}_j$ is computed from equation (\ref{eq:clustered_quantization_definition}) as $C_{f(\bm{b}_j)}(\bm{b}_j)$. Further, for a given block cluster $\mathcal{B}_i$, we compute the quantization MSE as $\frac{1}{|\mathcal{B}_{i}|}\sum_{\bm{b} \in \mathcal{B}_{i}} \frac{1}{L_b}\lVert \bm{b}- C_i^{(n)}(\bm{b})\rVert^2_2$. Therefore, at the end of iteration $n$, we evaluate the overall quantization MSE $J^{(n)}$ for a given operand $\bm{X}$ composed of $N_c$ block clusters as:
\begin{align*}
    \label{eq:mse_iter_n}
    J^{(n)} = \frac{1}{N_c} \sum_{i=1}^{N_c} \frac{1}{|\mathcal{B}_{i}^{(n)}|}\sum_{\bm{v} \in \mathcal{B}_{i}^{(n)}} \frac{1}{L_b}\lVert \bm{b}- B_i^{(n)}(\bm{b})\rVert^2_2
\end{align*}

At the end of iteration $n$, the codebooks are updated from $\mathcal{C}^{(n-1)}$ to $\mathcal{C}^{(n)}$. However, the mapping of a given vector $\bm{b}_j$ to quantizers $\mathcal{C}^{(n)}$ remains as  $f^{(n)}(\bm{b}_j)$. At the next iteration, during the vector clustering step, $f^{(n+1)}(\bm{b}_j)$ finds new mapping of $\bm{b}_j$ to updated codebooks $\mathcal{C}^{(n)}$ such that the quantization MSE over the candidate codebooks is minimized. Therefore, we obtain the following result for $\bm{b}_j$:
\begin{align*}
\frac{1}{L_b}\lVert \bm{b}_j - C_{f^{(n+1)}(\bm{b}_j)}^{(n)}(\bm{b}_j)\rVert^2_2 \le \frac{1}{L_b}\lVert \bm{b}_j - C_{f^{(n)}(\bm{b}_j)}^{(n)}(\bm{b}_j)\rVert^2_2
\end{align*}

That is, quantizing $\bm{b}_j$ at the end of the block clustering step of iteration $n+1$ results in lower quantization MSE compared to quantizing at the end of iteration $n$. Since this is true for all $\bm{b} \in \bm{X}$, we assert the following:
\begin{equation}
\begin{split}
\label{eq:mse_ineq_1}
    \tilde{J}^{(n+1)} &= \frac{1}{N_c} \sum_{i=1}^{N_c} \frac{1}{|\mathcal{B}_{i}^{(n+1)}|}\sum_{\bm{b} \in \mathcal{B}_{i}^{(n+1)}} \frac{1}{L_b}\lVert \bm{b} - C_i^{(n)}(b)\rVert^2_2 \le J^{(n)}
\end{split}
\end{equation}
where $\tilde{J}^{(n+1)}$ is the the quantization MSE after the vector clustering step at iteration $n+1$.

Next, during the codebook update step (\ref{eq:quantizers_update}) at iteration $n+1$, the per-cluster codebooks $\mathcal{C}^{(n)}$ are updated to $\mathcal{C}^{(n+1)}$ by invoking the Lloyd-Max algorithm \citep{Lloyd}. We know that for any given value distribution, the Lloyd-Max algorithm minimizes the quantization MSE. Therefore, for a given vector cluster $\mathcal{B}_i$ we obtain the following result:

\begin{equation}
    \frac{1}{|\mathcal{B}_{i}^{(n+1)}|}\sum_{\bm{b} \in \mathcal{B}_{i}^{(n+1)}} \frac{1}{L_b}\lVert \bm{b}- C_i^{(n+1)}(\bm{b})\rVert^2_2 \le \frac{1}{|\mathcal{B}_{i}^{(n+1)}|}\sum_{\bm{b} \in \mathcal{B}_{i}^{(n+1)}} \frac{1}{L_b}\lVert \bm{b}- C_i^{(n)}(\bm{b})\rVert^2_2
\end{equation}

The above equation states that quantizing the given block cluster $\mathcal{B}_i$ after updating the associated codebook from $C_i^{(n)}$ to $C_i^{(n+1)}$ results in lower quantization MSE. Since this is true for all the block clusters, we derive the following result: 
\begin{equation}
\begin{split}
\label{eq:mse_ineq_2}
     J^{(n+1)} &= \frac{1}{N_c} \sum_{i=1}^{N_c} \frac{1}{|\mathcal{B}_{i}^{(n+1)}|}\sum_{\bm{b} \in \mathcal{B}_{i}^{(n+1)}} \frac{1}{L_b}\lVert \bm{b}- C_i^{(n+1)}(\bm{b})\rVert^2_2  \le \tilde{J}^{(n+1)}   
\end{split}
\end{equation}

Following (\ref{eq:mse_ineq_1}) and (\ref{eq:mse_ineq_2}), we find that the quantization MSE is non-increasing for each iteration, that is, $J^{(1)} \ge J^{(2)} \ge J^{(3)} \ge \ldots \ge J^{(M)}$ where $M$ is the maximum number of iterations. 
%Therefore, we can say that if the algorithm converges, then it must be that it has converged to a local minimum. 
\hfill $\blacksquare$


\begin{figure}
    \begin{center}
    \includegraphics[width=0.5\textwidth]{sections//figures/mse_vs_iter.pdf}
    \end{center}
    \caption{\small NMSE vs iterations during LO-BCQ compared to other block quantization proposals}
    \label{fig:nmse_vs_iter}
\end{figure}

Figure \ref{fig:nmse_vs_iter} shows the empirical convergence of LO-BCQ across several block lengths and number of codebooks. Also, the MSE achieved by LO-BCQ is compared to baselines such as MXFP and VSQ. As shown, LO-BCQ converges to a lower MSE than the baselines. Further, we achieve better convergence for larger number of codebooks ($N_c$) and for a smaller block length ($L_b$), both of which increase the bitwidth of BCQ (see Eq \ref{eq:bitwidth_bcq}).


\subsection{Additional Accuracy Results}
%Table \ref{tab:lobcq_config} lists the various LOBCQ configurations and their corresponding bitwidths.
\begin{table}
\setlength{\tabcolsep}{4.75pt}
\begin{center}
\caption{\label{tab:lobcq_config} Various LO-BCQ configurations and their bitwidths.}
\begin{tabular}{|c||c|c|c|c||c|c||c|} 
\hline
 & \multicolumn{4}{|c||}{$L_b=8$} & \multicolumn{2}{|c||}{$L_b=4$} & $L_b=2$ \\
 \hline
 \backslashbox{$L_A$\kern-1em}{\kern-1em$N_c$} & 2 & 4 & 8 & 16 & 2 & 4 & 2 \\
 \hline
 64 & 4.25 & 4.375 & 4.5 & 4.625 & 4.375 & 4.625 & 4.625\\
 \hline
 32 & 4.375 & 4.5 & 4.625& 4.75 & 4.5 & 4.75 & 4.75 \\
 \hline
 16 & 4.625 & 4.75& 4.875 & 5 & 4.75 & 5 & 5 \\
 \hline
\end{tabular}
\end{center}
\end{table}

%\subsection{Perplexity achieved by various LO-BCQ configurations on Wikitext-103 dataset}

\begin{table} \centering
\begin{tabular}{|c||c|c|c|c||c|c||c|} 
\hline
 $L_b \rightarrow$& \multicolumn{4}{c||}{8} & \multicolumn{2}{c||}{4} & 2\\
 \hline
 \backslashbox{$L_A$\kern-1em}{\kern-1em$N_c$} & 2 & 4 & 8 & 16 & 2 & 4 & 2  \\
 %$N_c \rightarrow$ & 2 & 4 & 8 & 16 & 2 & 4 & 2 \\
 \hline
 \hline
 \multicolumn{8}{c}{GPT3-1.3B (FP32 PPL = 9.98)} \\ 
 \hline
 \hline
 64 & 10.40 & 10.23 & 10.17 & 10.15 &  10.28 & 10.18 & 10.19 \\
 \hline
 32 & 10.25 & 10.20 & 10.15 & 10.12 &  10.23 & 10.17 & 10.17 \\
 \hline
 16 & 10.22 & 10.16 & 10.10 & 10.09 &  10.21 & 10.14 & 10.16 \\
 \hline
  \hline
 \multicolumn{8}{c}{GPT3-8B (FP32 PPL = 7.38)} \\ 
 \hline
 \hline
 64 & 7.61 & 7.52 & 7.48 &  7.47 &  7.55 &  7.49 & 7.50 \\
 \hline
 32 & 7.52 & 7.50 & 7.46 &  7.45 &  7.52 &  7.48 & 7.48  \\
 \hline
 16 & 7.51 & 7.48 & 7.44 &  7.44 &  7.51 &  7.49 & 7.47  \\
 \hline
\end{tabular}
\caption{\label{tab:ppl_gpt3_abalation} Wikitext-103 perplexity across GPT3-1.3B and 8B models.}
\end{table}

\begin{table} \centering
\begin{tabular}{|c||c|c|c|c||} 
\hline
 $L_b \rightarrow$& \multicolumn{4}{c||}{8}\\
 \hline
 \backslashbox{$L_A$\kern-1em}{\kern-1em$N_c$} & 2 & 4 & 8 & 16 \\
 %$N_c \rightarrow$ & 2 & 4 & 8 & 16 & 2 & 4 & 2 \\
 \hline
 \hline
 \multicolumn{5}{|c|}{Llama2-7B (FP32 PPL = 5.06)} \\ 
 \hline
 \hline
 64 & 5.31 & 5.26 & 5.19 & 5.18  \\
 \hline
 32 & 5.23 & 5.25 & 5.18 & 5.15  \\
 \hline
 16 & 5.23 & 5.19 & 5.16 & 5.14  \\
 \hline
 \multicolumn{5}{|c|}{Nemotron4-15B (FP32 PPL = 5.87)} \\ 
 \hline
 \hline
 64  & 6.3 & 6.20 & 6.13 & 6.08  \\
 \hline
 32  & 6.24 & 6.12 & 6.07 & 6.03  \\
 \hline
 16  & 6.12 & 6.14 & 6.04 & 6.02  \\
 \hline
 \multicolumn{5}{|c|}{Nemotron4-340B (FP32 PPL = 3.48)} \\ 
 \hline
 \hline
 64 & 3.67 & 3.62 & 3.60 & 3.59 \\
 \hline
 32 & 3.63 & 3.61 & 3.59 & 3.56 \\
 \hline
 16 & 3.61 & 3.58 & 3.57 & 3.55 \\
 \hline
\end{tabular}
\caption{\label{tab:ppl_llama7B_nemo15B} Wikitext-103 perplexity compared to FP32 baseline in Llama2-7B and Nemotron4-15B, 340B models}
\end{table}

%\subsection{Perplexity achieved by various LO-BCQ configurations on MMLU dataset}


\begin{table} \centering
\begin{tabular}{|c||c|c|c|c||c|c|c|c|} 
\hline
 $L_b \rightarrow$& \multicolumn{4}{c||}{8} & \multicolumn{4}{c||}{8}\\
 \hline
 \backslashbox{$L_A$\kern-1em}{\kern-1em$N_c$} & 2 & 4 & 8 & 16 & 2 & 4 & 8 & 16  \\
 %$N_c \rightarrow$ & 2 & 4 & 8 & 16 & 2 & 4 & 2 \\
 \hline
 \hline
 \multicolumn{5}{|c|}{Llama2-7B (FP32 Accuracy = 45.8\%)} & \multicolumn{4}{|c|}{Llama2-70B (FP32 Accuracy = 69.12\%)} \\ 
 \hline
 \hline
 64 & 43.9 & 43.4 & 43.9 & 44.9 & 68.07 & 68.27 & 68.17 & 68.75 \\
 \hline
 32 & 44.5 & 43.8 & 44.9 & 44.5 & 68.37 & 68.51 & 68.35 & 68.27  \\
 \hline
 16 & 43.9 & 42.7 & 44.9 & 45 & 68.12 & 68.77 & 68.31 & 68.59  \\
 \hline
 \hline
 \multicolumn{5}{|c|}{GPT3-22B (FP32 Accuracy = 38.75\%)} & \multicolumn{4}{|c|}{Nemotron4-15B (FP32 Accuracy = 64.3\%)} \\ 
 \hline
 \hline
 64 & 36.71 & 38.85 & 38.13 & 38.92 & 63.17 & 62.36 & 63.72 & 64.09 \\
 \hline
 32 & 37.95 & 38.69 & 39.45 & 38.34 & 64.05 & 62.30 & 63.8 & 64.33  \\
 \hline
 16 & 38.88 & 38.80 & 38.31 & 38.92 & 63.22 & 63.51 & 63.93 & 64.43  \\
 \hline
\end{tabular}
\caption{\label{tab:mmlu_abalation} Accuracy on MMLU dataset across GPT3-22B, Llama2-7B, 70B and Nemotron4-15B models.}
\end{table}


%\subsection{Perplexity achieved by various LO-BCQ configurations on LM evaluation harness}

\begin{table} \centering
\begin{tabular}{|c||c|c|c|c||c|c|c|c|} 
\hline
 $L_b \rightarrow$& \multicolumn{4}{c||}{8} & \multicolumn{4}{c||}{8}\\
 \hline
 \backslashbox{$L_A$\kern-1em}{\kern-1em$N_c$} & 2 & 4 & 8 & 16 & 2 & 4 & 8 & 16  \\
 %$N_c \rightarrow$ & 2 & 4 & 8 & 16 & 2 & 4 & 2 \\
 \hline
 \hline
 \multicolumn{5}{|c|}{Race (FP32 Accuracy = 37.51\%)} & \multicolumn{4}{|c|}{Boolq (FP32 Accuracy = 64.62\%)} \\ 
 \hline
 \hline
 64 & 36.94 & 37.13 & 36.27 & 37.13 & 63.73 & 62.26 & 63.49 & 63.36 \\
 \hline
 32 & 37.03 & 36.36 & 36.08 & 37.03 & 62.54 & 63.51 & 63.49 & 63.55  \\
 \hline
 16 & 37.03 & 37.03 & 36.46 & 37.03 & 61.1 & 63.79 & 63.58 & 63.33  \\
 \hline
 \hline
 \multicolumn{5}{|c|}{Winogrande (FP32 Accuracy = 58.01\%)} & \multicolumn{4}{|c|}{Piqa (FP32 Accuracy = 74.21\%)} \\ 
 \hline
 \hline
 64 & 58.17 & 57.22 & 57.85 & 58.33 & 73.01 & 73.07 & 73.07 & 72.80 \\
 \hline
 32 & 59.12 & 58.09 & 57.85 & 58.41 & 73.01 & 73.94 & 72.74 & 73.18  \\
 \hline
 16 & 57.93 & 58.88 & 57.93 & 58.56 & 73.94 & 72.80 & 73.01 & 73.94  \\
 \hline
\end{tabular}
\caption{\label{tab:mmlu_abalation} Accuracy on LM evaluation harness tasks on GPT3-1.3B model.}
\end{table}

\begin{table} \centering
\begin{tabular}{|c||c|c|c|c||c|c|c|c|} 
\hline
 $L_b \rightarrow$& \multicolumn{4}{c||}{8} & \multicolumn{4}{c||}{8}\\
 \hline
 \backslashbox{$L_A$\kern-1em}{\kern-1em$N_c$} & 2 & 4 & 8 & 16 & 2 & 4 & 8 & 16  \\
 %$N_c \rightarrow$ & 2 & 4 & 8 & 16 & 2 & 4 & 2 \\
 \hline
 \hline
 \multicolumn{5}{|c|}{Race (FP32 Accuracy = 41.34\%)} & \multicolumn{4}{|c|}{Boolq (FP32 Accuracy = 68.32\%)} \\ 
 \hline
 \hline
 64 & 40.48 & 40.10 & 39.43 & 39.90 & 69.20 & 68.41 & 69.45 & 68.56 \\
 \hline
 32 & 39.52 & 39.52 & 40.77 & 39.62 & 68.32 & 67.43 & 68.17 & 69.30  \\
 \hline
 16 & 39.81 & 39.71 & 39.90 & 40.38 & 68.10 & 66.33 & 69.51 & 69.42  \\
 \hline
 \hline
 \multicolumn{5}{|c|}{Winogrande (FP32 Accuracy = 67.88\%)} & \multicolumn{4}{|c|}{Piqa (FP32 Accuracy = 78.78\%)} \\ 
 \hline
 \hline
 64 & 66.85 & 66.61 & 67.72 & 67.88 & 77.31 & 77.42 & 77.75 & 77.64 \\
 \hline
 32 & 67.25 & 67.72 & 67.72 & 67.00 & 77.31 & 77.04 & 77.80 & 77.37  \\
 \hline
 16 & 68.11 & 68.90 & 67.88 & 67.48 & 77.37 & 78.13 & 78.13 & 77.69  \\
 \hline
\end{tabular}
\caption{\label{tab:mmlu_abalation} Accuracy on LM evaluation harness tasks on GPT3-8B model.}
\end{table}

\begin{table} \centering
\begin{tabular}{|c||c|c|c|c||c|c|c|c|} 
\hline
 $L_b \rightarrow$& \multicolumn{4}{c||}{8} & \multicolumn{4}{c||}{8}\\
 \hline
 \backslashbox{$L_A$\kern-1em}{\kern-1em$N_c$} & 2 & 4 & 8 & 16 & 2 & 4 & 8 & 16  \\
 %$N_c \rightarrow$ & 2 & 4 & 8 & 16 & 2 & 4 & 2 \\
 \hline
 \hline
 \multicolumn{5}{|c|}{Race (FP32 Accuracy = 40.67\%)} & \multicolumn{4}{|c|}{Boolq (FP32 Accuracy = 76.54\%)} \\ 
 \hline
 \hline
 64 & 40.48 & 40.10 & 39.43 & 39.90 & 75.41 & 75.11 & 77.09 & 75.66 \\
 \hline
 32 & 39.52 & 39.52 & 40.77 & 39.62 & 76.02 & 76.02 & 75.96 & 75.35  \\
 \hline
 16 & 39.81 & 39.71 & 39.90 & 40.38 & 75.05 & 73.82 & 75.72 & 76.09  \\
 \hline
 \hline
 \multicolumn{5}{|c|}{Winogrande (FP32 Accuracy = 70.64\%)} & \multicolumn{4}{|c|}{Piqa (FP32 Accuracy = 79.16\%)} \\ 
 \hline
 \hline
 64 & 69.14 & 70.17 & 70.17 & 70.56 & 78.24 & 79.00 & 78.62 & 78.73 \\
 \hline
 32 & 70.96 & 69.69 & 71.27 & 69.30 & 78.56 & 79.49 & 79.16 & 78.89  \\
 \hline
 16 & 71.03 & 69.53 & 69.69 & 70.40 & 78.13 & 79.16 & 79.00 & 79.00  \\
 \hline
\end{tabular}
\caption{\label{tab:mmlu_abalation} Accuracy on LM evaluation harness tasks on GPT3-22B model.}
\end{table}

\begin{table} \centering
\begin{tabular}{|c||c|c|c|c||c|c|c|c|} 
\hline
 $L_b \rightarrow$& \multicolumn{4}{c||}{8} & \multicolumn{4}{c||}{8}\\
 \hline
 \backslashbox{$L_A$\kern-1em}{\kern-1em$N_c$} & 2 & 4 & 8 & 16 & 2 & 4 & 8 & 16  \\
 %$N_c \rightarrow$ & 2 & 4 & 8 & 16 & 2 & 4 & 2 \\
 \hline
 \hline
 \multicolumn{5}{|c|}{Race (FP32 Accuracy = 44.4\%)} & \multicolumn{4}{|c|}{Boolq (FP32 Accuracy = 79.29\%)} \\ 
 \hline
 \hline
 64 & 42.49 & 42.51 & 42.58 & 43.45 & 77.58 & 77.37 & 77.43 & 78.1 \\
 \hline
 32 & 43.35 & 42.49 & 43.64 & 43.73 & 77.86 & 75.32 & 77.28 & 77.86  \\
 \hline
 16 & 44.21 & 44.21 & 43.64 & 42.97 & 78.65 & 77 & 76.94 & 77.98  \\
 \hline
 \hline
 \multicolumn{5}{|c|}{Winogrande (FP32 Accuracy = 69.38\%)} & \multicolumn{4}{|c|}{Piqa (FP32 Accuracy = 78.07\%)} \\ 
 \hline
 \hline
 64 & 68.9 & 68.43 & 69.77 & 68.19 & 77.09 & 76.82 & 77.09 & 77.86 \\
 \hline
 32 & 69.38 & 68.51 & 68.82 & 68.90 & 78.07 & 76.71 & 78.07 & 77.86  \\
 \hline
 16 & 69.53 & 67.09 & 69.38 & 68.90 & 77.37 & 77.8 & 77.91 & 77.69  \\
 \hline
\end{tabular}
\caption{\label{tab:mmlu_abalation} Accuracy on LM evaluation harness tasks on Llama2-7B model.}
\end{table}

\begin{table} \centering
\begin{tabular}{|c||c|c|c|c||c|c|c|c|} 
\hline
 $L_b \rightarrow$& \multicolumn{4}{c||}{8} & \multicolumn{4}{c||}{8}\\
 \hline
 \backslashbox{$L_A$\kern-1em}{\kern-1em$N_c$} & 2 & 4 & 8 & 16 & 2 & 4 & 8 & 16  \\
 %$N_c \rightarrow$ & 2 & 4 & 8 & 16 & 2 & 4 & 2 \\
 \hline
 \hline
 \multicolumn{5}{|c|}{Race (FP32 Accuracy = 48.8\%)} & \multicolumn{4}{|c|}{Boolq (FP32 Accuracy = 85.23\%)} \\ 
 \hline
 \hline
 64 & 49.00 & 49.00 & 49.28 & 48.71 & 82.82 & 84.28 & 84.03 & 84.25 \\
 \hline
 32 & 49.57 & 48.52 & 48.33 & 49.28 & 83.85 & 84.46 & 84.31 & 84.93  \\
 \hline
 16 & 49.85 & 49.09 & 49.28 & 48.99 & 85.11 & 84.46 & 84.61 & 83.94  \\
 \hline
 \hline
 \multicolumn{5}{|c|}{Winogrande (FP32 Accuracy = 79.95\%)} & \multicolumn{4}{|c|}{Piqa (FP32 Accuracy = 81.56\%)} \\ 
 \hline
 \hline
 64 & 78.77 & 78.45 & 78.37 & 79.16 & 81.45 & 80.69 & 81.45 & 81.5 \\
 \hline
 32 & 78.45 & 79.01 & 78.69 & 80.66 & 81.56 & 80.58 & 81.18 & 81.34  \\
 \hline
 16 & 79.95 & 79.56 & 79.79 & 79.72 & 81.28 & 81.66 & 81.28 & 80.96  \\
 \hline
\end{tabular}
\caption{\label{tab:mmlu_abalation} Accuracy on LM evaluation harness tasks on Llama2-70B model.}
\end{table}

%\section{MSE Studies}
%\textcolor{red}{TODO}


\subsection{Number Formats and Quantization Method}
\label{subsec:numFormats_quantMethod}
\subsubsection{Integer Format}
An $n$-bit signed integer (INT) is typically represented with a 2s-complement format \citep{yao2022zeroquant,xiao2023smoothquant,dai2021vsq}, where the most significant bit denotes the sign.

\subsubsection{Floating Point Format}
An $n$-bit signed floating point (FP) number $x$ comprises of a 1-bit sign ($x_{\mathrm{sign}}$), $B_m$-bit mantissa ($x_{\mathrm{mant}}$) and $B_e$-bit exponent ($x_{\mathrm{exp}}$) such that $B_m+B_e=n-1$. The associated constant exponent bias ($E_{\mathrm{bias}}$) is computed as $(2^{{B_e}-1}-1)$. We denote this format as $E_{B_e}M_{B_m}$.  

\subsubsection{Quantization Scheme}
\label{subsec:quant_method}
A quantization scheme dictates how a given unquantized tensor is converted to its quantized representation. We consider FP formats for the purpose of illustration. Given an unquantized tensor $\bm{X}$ and an FP format $E_{B_e}M_{B_m}$, we first, we compute the quantization scale factor $s_X$ that maps the maximum absolute value of $\bm{X}$ to the maximum quantization level of the $E_{B_e}M_{B_m}$ format as follows:
\begin{align}
\label{eq:sf}
    s_X = \frac{\mathrm{max}(|\bm{X}|)}{\mathrm{max}(E_{B_e}M_{B_m})}
\end{align}
In the above equation, $|\cdot|$ denotes the absolute value function.

Next, we scale $\bm{X}$ by $s_X$ and quantize it to $\hat{\bm{X}}$ by rounding it to the nearest quantization level of $E_{B_e}M_{B_m}$ as:

\begin{align}
\label{eq:tensor_quant}
    \hat{\bm{X}} = \text{round-to-nearest}\left(\frac{\bm{X}}{s_X}, E_{B_e}M_{B_m}\right)
\end{align}

We perform dynamic max-scaled quantization \citep{wu2020integer}, where the scale factor $s$ for activations is dynamically computed during runtime.

\subsection{Vector Scaled Quantization}
\begin{wrapfigure}{r}{0.35\linewidth}
  \centering
  \includegraphics[width=\linewidth]{sections/figures/vsquant.jpg}
  \caption{\small Vectorwise decomposition for per-vector scaled quantization (VSQ \citep{dai2021vsq}).}
  \label{fig:vsquant}
\end{wrapfigure}
During VSQ \citep{dai2021vsq}, the operand tensors are decomposed into 1D vectors in a hardware friendly manner as shown in Figure \ref{fig:vsquant}. Since the decomposed tensors are used as operands in matrix multiplications during inference, it is beneficial to perform this decomposition along the reduction dimension of the multiplication. The vectorwise quantization is performed similar to tensorwise quantization described in Equations \ref{eq:sf} and \ref{eq:tensor_quant}, where a scale factor $s_v$ is required for each vector $\bm{v}$ that maps the maximum absolute value of that vector to the maximum quantization level. While smaller vector lengths can lead to larger accuracy gains, the associated memory and computational overheads due to the per-vector scale factors increases. To alleviate these overheads, VSQ \citep{dai2021vsq} proposed a second level quantization of the per-vector scale factors to unsigned integers, while MX \citep{rouhani2023shared} quantizes them to integer powers of 2 (denoted as $2^{INT}$).

\subsubsection{MX Format}
The MX format proposed in \citep{rouhani2023microscaling} introduces the concept of sub-block shifting. For every two scalar elements of $b$-bits each, there is a shared exponent bit. The value of this exponent bit is determined through an empirical analysis that targets minimizing quantization MSE. We note that the FP format $E_{1}M_{b}$ is strictly better than MX from an accuracy perspective since it allocates a dedicated exponent bit to each scalar as opposed to sharing it across two scalars. Therefore, we conservatively bound the accuracy of a $b+2$-bit signed MX format with that of a $E_{1}M_{b}$ format in our comparisons. For instance, we use E1M2 format as a proxy for MX4.

\begin{figure}
    \centering
    \includegraphics[width=1\linewidth]{sections//figures/BlockFormats.pdf}
    \caption{\small Comparing LO-BCQ to MX format.}
    \label{fig:block_formats}
\end{figure}

Figure \ref{fig:block_formats} compares our $4$-bit LO-BCQ block format to MX \citep{rouhani2023microscaling}. As shown, both LO-BCQ and MX decompose a given operand tensor into block arrays and each block array into blocks. Similar to MX, we find that per-block quantization ($L_b < L_A$) leads to better accuracy due to increased flexibility. While MX achieves this through per-block $1$-bit micro-scales, we associate a dedicated codebook to each block through a per-block codebook selector. Further, MX quantizes the per-block array scale-factor to E8M0 format without per-tensor scaling. In contrast during LO-BCQ, we find that per-tensor scaling combined with quantization of per-block array scale-factor to E4M3 format results in superior inference accuracy across models. 


\printbibliography

\end{document}


