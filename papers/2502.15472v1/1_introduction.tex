\section{Introduction}
\label{sec:introduction}
Reconstruction-oriented communications are designed to recover the transmitted information at the receiver sides, often involving traditional source and channel coding techniques. This approach is commonly used in systems where the fidelity of the information is paramount, such as in audio or video streaming services. The structure of separate source and channel coding, a cornerstone in the design of communication systems, has been shown to be theoretically optimal via \gls{aep} with infinitely long source and channel blocks \cite{Cover_1991_EoI}. However, in practical scenarios, this separation often leads to inefficiencies and suboptimal performance, particularly for \gls{ai} driven applications \cite{Bourtsoulatze_2019_DJS}.

The pervasive advancement of AI technologies, particularly in the context of deep learning, presents novel challenges for future communication systems, where the throughput required by AI agents could be much higher than that of human users.

Recent developments in deep learning have shown that \gls{jscc} can potentially address some of these inefficiencies and outperform traditional separate coding designs. This approach is especially potent in environments where traditional methods struggle to keep pace with the data demands of \gls{ai}-driven applications. However, \gls{jscc}-based reconstruction-oriented communications, which focus on accurately reconstructing a signal on receiver sides, often waste communication resources by transmitting task-agnostic information \cite{Kurka_2020_DfD}.

To address these issues, task-oriented communication has emerged as a key technology and has attracted significant research interests \cite{Shao_2022_LTO, Shao_2023_TOC, Stavrou_2022_ARD, Shi_2023_TOC}. Using the capabilities of deep learning, task-oriented \gls{jscc} focuses on transmitting task-specific information, thus improving efficiency and reducing the data rate in critical applications. This requires joint optimization of the \gls{jscc} and inference network, which must be co-designed for effective task-oriented communication \cite{Shao_2022_LTO}. Note that existing \gls{jscc} designs are mainly based on analog communication principles \cite{Bourtsoulatze_2019_DJS} and cannot be integrated with existing digital communication infrastructures.

Furthermore, cloud-based services introduce unacceptable latency for real-time applications, such as autonomous driving \cite{Zhang_2020_MEI, Liu_2019_ECf}. To mitigate this issue, \textit{edge inference} \cite{Li_2018_EIO, Shao_2022_LTO, Jankowski_2021_WIR} has become a promising approach, enabling quick response to real-time AI applications. However, widely deployed AI agents bring significant communication loads to communication systems. Emergent methods based on \gls{jscc} have shown great potential to solve this problem \cite{Jankowski_2021_WIR, Shao_2020_BAE, Jankowski_2020_JDE}.

Recognizing these multifaceted challenges, there is a growing interest in developing communication systems that are not only task-oriented but also aligned with reconstruction-oriented communication frameworks. This has led to the proposition of what we refer to as \gls{atroc}, which aims to bridge the gap between the efficiency of task-specific data transmission and the robustness of reconstruction-oriented communications, enabling the seamless integration of \gls{ai} technologies with existing network infrastructures.
\section{Related Work}
\label{sec:related_work}

The original investigation \cite{gibson1979ecological} on the relationship between visual perception and human action defines \emph{affordance} as the opportunities for interaction with the surrounding environment. Behavioral studies on regular and cognitively impaired persons have shown evidence that perception results in both visual and motor signals in the human brain. An extended study \cite{anderson2002attentional} shows that visual attention to the spatial characteristics of the perceived objects initiates automatic motor signals for different actions. In computer vision, human affordance learning involves novel pose prediction such that the estimated pose represents a valid human action within the scene context. The task is fundamental to many problems requiring robust semantic reasoning about the environment, such as human motion synthesis \cite{wang2021scene} and scene-aware human pose generation \cite{wang2017binge, roy2016multi, zhang2022inpaint, yao2023scene}.

Earlier methods of affordance learning have explored knowledge mining \cite{zhu2014reasoning} and multimodal feature cues \cite{roy2016multi} to address the problem. In \cite{zhu2014reasoning}, the authors use a Markov Logic Network for constructing a knowledge base by extracting several object attributes from different image and metadata sources, which can perform various downstream visual inference tasks without any additional classifier, including zero-shot affordance prediction. In \cite{roy2016multi}, the authors use depth map, surface normals, and segmentation map as multimodal cues to train a multi-scale convolutional neural network (CNN) for scene-level semantic label assignment associated with specific human actions. In \cite{do2018affordancenet}, the authors design a multi-branch end-to-end CNN with two separate pathways for object detection and affordance label assignment to achieve high real-time inference throughput. Researchers \cite{chuang2018learning} have also explored socially imposed constraints for affordance learning. In \cite{chuang2018learning}, the authors propose a graph neural network (GNN) to propagate contextual scene information from egocentric views for action-object affordance reasoning.

Probabilistic modeling of scene-aware human motion generation also involves semantic reasoning of human interaction with the environment. Initial works on human motion synthesis have taken different architectural approaches, such as sequence-to-sequence models \cite{barsoum2018hp}, generative adversarial networks (GAN) \cite{barsoum2018hp, cai2018deep, yang2018pose}, graph convolutional networks (GCN) \cite{yan2019convolutional}, and variational autoencoders (VAE) \cite{guo2020action2motion}. However, these methods have mostly ignored the role of environmental semantics. Due to potential uncertainty in human motion, in a recent approach \cite{wang2021scene}, the authors address such motion synthesis with a GAN conditioned on scene attributes and motion trajectory to predict probable body pose dynamics.

One key challenge of human affordance generation in 2D scenes is the lack of large-scale datasets with rich pose annotations. In \cite{wang2017binge}, the authors compile the only public dataset of annotated human body poses in complex 2D indoor scenes by extracting frames from sitcom videos. Aiming to generate a contextually valid human affordance at a user-defined location, the authors propose sampling the scale and deformation parameters for an existing human pose template using a VAE conditioned on the localized image patches as scene context. In \cite{zhang2022inpaint}, the authors introduce a two-stage GAN architecture for achieving a similar goal by estimating the affine bounding box parameters to localize a probable human in the scene and then generating a potential body pose at that location. The method uses the input scene, corresponding depth, and segmentation maps as semantic guidance. In \cite{yao2023scene}, the authors propose a transformer-based approach with knowledge distillation for generating human affordances in 2D indoor scenes.




\subsection{Contributions}
This paper introduces a novel communication framework compatible with reconstruction-oriented communication, especially for edge inference, termed \glsreset{atroc}\gls{atroc}. By extending \gls{ib} theory \cite{Tishby_1999_TIB} and incorporating \gls{jscc} modulation, this framework is designed to enhance AI-driven applications. It prioritizes task relevance in data transmission strategies, shifting focus from traditional signal reconstruction fidelity to operational efficiency and effectiveness in real-world applications. The key contributions of this research are summarized as follows:




\begin{itemize}
    \item \textbf{Development of an \gls{atroc} Framework:} Based on \gls{ib} theory, we develop a framework that aligns task-oriented communications with reconstruction-oriented communications. The framework focuses on maximizing mutual information between inference results and encoded features, minimizing mutual information between the encoded features and the input data, and preserving task-specific information through the information reshaper. This reshaper is expert at transforming received symbols into task-specific data, maintaining the same data structure as the input while ensuring the preservation of task-specific information.

    \item \textbf{Innovation of an Information Reshaper:} We introduce an information reshaper within our extended \gls{ib} theory, laying a foundational aspect of \gls{atroc}. This component is crucial for adapting the communication to the specific needs of the task without compromising the integrity of the transmitted data.

    \item \textbf{Variational Approximation for Tractable Information Estimation:} Due to the intractability of mutual information in the training and inference of deep neural networks, we employ a variational approximation approach, known as \gls{vib}. This approach allows us to establish a tractable upper bound for these terms, enabling training and inference of deep neural networks.

    \item \textbf{Adaptation of a \gls{jscc} Modulation Scheme:} We design a \gls{jscc} modulation scheme that aligns \gls{jscc} symbols with a predefined constellation scheme. This scheme ensures compatibility of our framework with classic modulation techniques, making it more adaptable to existing communication infrastructures.

    \item \textbf{Performance Enhancement in Edge-Based Autonomous Driving:} In our simulation, we validate that the \gls{atroc} framework outperforms reconstruction-oriented methods for edge-based autonomous driving \cite{Wu_2022_TgC}. 
    Specifically, our method reduced 99.19\% communication load, in terms of bits per service, compared to existing methods, without compromising the driving score of the autonomous driving agent.
\end{itemize}

\subsection{Organization and Notations}
The rest of this paper is organized as follows: \Cref{sec:aligned_TOC_framework} details the system model and discusses how the proposed framework advances reconstruction-oriented and non-aligned task-oriented communication approaches. \Cref{sec:AIB} introduces the \gls{ib} theory for \gls{atroc} and elaborates on the corresponding \gls{vib} derivation. In \cref{sec:modulation}, we propose a \gls{jscc} modulation technique that is compatible with classical modulation methods, such as \gls{qam}. \Cref{sec:edge_AI} extends the framework of \gls{vib} to enhance edge-based autonomous driving applications. The experimental results are presented in \cref{sec:result}, which evaluates the performance of our proposed \gls{atroc} framework and the \gls{jscc} modulation. Finally, \cref{sec:conclusion} concludes the paper.

\Cref{tab_notation} lists the main symbols used throughout this paper. 
%providing a quick reference to assist in understanding the mathematical and technical discussions.

% \usepackage{booktabs}


\begin{table}
\centering
\footnotesize
\caption{SUMMARY OF MAIN SYMBOLS}
\label{tab_notation}
\begin{tabular}{ll} 
\toprule
\multicolumn{1}{c}{\textbf{Symbol}}                                                         & \multicolumn{1}{c}{\textbf{Explanation}}   \\ 
\midrule
$\bm{x}$                                                                                    & Input data                                 \\
$\hat{\bm{x}}$                                                                              & Reconstructed input data                   \\
$\bm{z}$                                                                                    & JSCC symbols                               \\
$\bar{\bm{z}}$                                                                              & Quantized JSCC symbols                     \\
$\bm{z}_{\text{in}}$                                                                        & Channel input     \\
$\bm{z}_{\text{out}}$                                                                       & Channel output                             \\
$\check{\bm{z}}$                                                                            & Equalized JSCC symbols                     \\
$\tilde{\bm{z}}$                                                                            & Scaled JSCC symbols                        \\
$\hat{\bm{z}}$                                                                              & Reconstructed JSCC symbols                 \\
$\bm{y}$                                                                                    & Task-specific data                         \\
$\bm{a}$                                                                                    & Target action            \\
$\hat{\bm{a}}$                                                                              & Inferred action                           \\
$\beta_1, \beta_2, \hat{\beta}_1, \hat{\beta}_2$                                            & Lagrange multiplier                        \\
$\phi, \theta, \psi, \delta$                                                                & Parameters of neural networks              \\
$h$                                                                                         & Channel coefficient                        \\
$\bm{n}$                                                                                    & Gaussian noise                             \\
$k$                                                                                         & Dimension of the JSCC symbols              \\
$l$                                                                                         & Dimension of the input data                \\
$\zeta$                                                                                     & Upper bound of rate                        \\
$\Omega$                                                                                    & Size of mini-batch                         \\
$u$                                                                                         & Number of constellation points             \\
$r$                                                                                         & Constellation parameter                    \\
$e_{\left( \cdot \right)}$                                                                    & Constellation point                        \\
$P_\text{target}$                                                                           & Power constraint of transmitter            \\
$P_{\bar{\bm{z}}}$                                                                            & Power of quantized symbols                 \\
$\beta_Q$                                                                                   & Hyperparameter of quantization loss        \\
$\Gamma, \lambda_\text{feat}, \lambda_\text{traj}, \lambda_\text{ctrl}, \lambda_\text{aux}$ & Hyperparameters of edge AI agent           \\
$J_1, J_2$                                                                                  & Sampling number                            \\
$i, j$                                                                                      & General index depended on context  \\
\bottomrule
\end{tabular}
\end{table}