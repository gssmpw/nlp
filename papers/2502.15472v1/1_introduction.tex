\section{Introduction}
\label{sec:introduction}
Reconstruction-oriented communications are designed to recover the transmitted information at the receiver sides, often involving traditional source and channel coding techniques. This approach is commonly used in systems where the fidelity of the information is paramount, such as in audio or video streaming services. The structure of separate source and channel coding, a cornerstone in the design of communication systems, has been shown to be theoretically optimal via \gls{aep} with infinitely long source and channel blocks \cite{Cover_1991_EoI}. However, in practical scenarios, this separation often leads to inefficiencies and suboptimal performance, particularly for \gls{ai} driven applications \cite{Bourtsoulatze_2019_DJS}.

The pervasive advancement of AI technologies, particularly in the context of deep learning, presents novel challenges for future communication systems, where the throughput required by AI agents could be much higher than that of human users.

Recent developments in deep learning have shown that \gls{jscc} can potentially address some of these inefficiencies and outperform traditional separate coding designs. This approach is especially potent in environments where traditional methods struggle to keep pace with the data demands of \gls{ai}-driven applications. However, \gls{jscc}-based reconstruction-oriented communications, which focus on accurately reconstructing a signal on receiver sides, often waste communication resources by transmitting task-agnostic information \cite{Kurka_2020_DfD}.

To address these issues, task-oriented communication has emerged as a key technology and has attracted significant research interests \cite{Shao_2022_LTO, Shao_2023_TOC, Stavrou_2022_ARD, Shi_2023_TOC}. Using the capabilities of deep learning, task-oriented \gls{jscc} focuses on transmitting task-specific information, thus improving efficiency and reducing the data rate in critical applications. This requires joint optimization of the \gls{jscc} and inference network, which must be co-designed for effective task-oriented communication \cite{Shao_2022_LTO}. Note that existing \gls{jscc} designs are mainly based on analog communication principles \cite{Bourtsoulatze_2019_DJS} and cannot be integrated with existing digital communication infrastructures.

Furthermore, cloud-based services introduce unacceptable latency for real-time applications, such as autonomous driving \cite{Zhang_2020_MEI, Liu_2019_ECf}. To mitigate this issue, \textit{edge inference} \cite{Li_2018_EIO, Shao_2022_LTO, Jankowski_2021_WIR} has become a promising approach, enabling quick response to real-time AI applications. However, widely deployed AI agents bring significant communication loads to communication systems. Emergent methods based on \gls{jscc} have shown great potential to solve this problem \cite{Jankowski_2021_WIR, Shao_2020_BAE, Jankowski_2020_JDE}.

Recognizing these multifaceted challenges, there is a growing interest in developing communication systems that are not only task-oriented but also aligned with reconstruction-oriented communication frameworks. This has led to the proposition of what we refer to as \gls{atroc}, which aims to bridge the gap between the efficiency of task-specific data transmission and the robustness of reconstruction-oriented communications, enabling the seamless integration of \gls{ai} technologies with existing network infrastructures.
\section{Related Work}

\subsection{Large 3D Reconstruction Models}
Recently, generalized feed-forward models for 3D reconstruction from sparse input views have garnered considerable attention due to their applicability in heavily under-constrained scenarios. The Large Reconstruction Model (LRM)~\cite{hong2023lrm} uses a transformer-based encoder-decoder pipeline to infer a NeRF reconstruction from just a single image. Newer iterations have shifted the focus towards generating 3D Gaussian representations from four input images~\cite{tang2025lgm, xu2024grm, zhang2025gslrm, charatan2024pixelsplat, chen2025mvsplat, liu2025mvsgaussian}, showing remarkable novel view synthesis results. The paradigm of transformer-based sparse 3D reconstruction has also successfully been applied to lifting monocular videos to 4D~\cite{ren2024l4gm}. \\
Yet, none of the existing works in the domain have studied the use-case of inferring \textit{animatable} 3D representations from sparse input images, which is the focus of our work. To this end, we build on top of the Large Gaussian Reconstruction Model (GRM)~\cite{xu2024grm}.

\subsection{3D-aware Portrait Animation}
A different line of work focuses on animating portraits in a 3D-aware manner.
MegaPortraits~\cite{drobyshev2022megaportraits} builds a 3D Volume given a source and driving image, and renders the animated source actor via orthographic projection with subsequent 2D neural rendering.
3D morphable models (3DMMs)~\cite{blanz19993dmm} are extensively used to obtain more interpretable control over the portrait animation. For example, StyleRig~\cite{tewari2020stylerig} demonstrates how a 3DMM can be used to control the data generated from a pre-trained StyleGAN~\cite{karras2019stylegan} network. ROME~\cite{khakhulin2022rome} predicts vertex offsets and texture of a FLAME~\cite{li2017flame} mesh from the input image.
A TriPlane representation is inferred and animated via FLAME~\cite{li2017flame} in multiple methods like Portrait4D~\cite{deng2024portrait4d}, Portrait4D-v2~\cite{deng2024portrait4dv2}, and GPAvatar~\cite{chu2024gpavatar}.
Others, such as VOODOO 3D~\cite{tran2024voodoo3d} and VOODOO XP~\cite{tran2024voodooxp}, learn their own expression encoder to drive the source person in a more detailed manner. \\
All of the aforementioned methods require nothing more than a single image of a person to animate it. This allows them to train on large monocular video datasets to infer a very generic motion prior that even translates to paintings or cartoon characters. However, due to their task formulation, these methods mostly focus on image synthesis from a frontal camera, often trading 3D consistency for better image quality by using 2D screen-space neural renderers. In contrast, our work aims to produce a truthful and complete 3D avatar representation from the input images that can be viewed from any angle.  

\subsection{Photo-realistic 3D Face Models}
The increasing availability of large-scale multi-view face datasets~\cite{kirschstein2023nersemble, ava256, pan2024renderme360, yang2020facescape} has enabled building photo-realistic 3D face models that learn a detailed prior over both geometry and appearance of human faces. HeadNeRF~\cite{hong2022headnerf} conditions a Neural Radiance Field (NeRF)~\cite{mildenhall2021nerf} on identity, expression, albedo, and illumination codes. VRMM~\cite{yang2024vrmm} builds a high-quality and relightable 3D face model using volumetric primitives~\cite{lombardi2021mvp}. One2Avatar~\cite{yu2024one2avatar} extends a 3DMM by anchoring a radiance field to its surface. More recently, GPHM~\cite{xu2025gphm} and HeadGAP~\cite{zheng2024headgap} have adopted 3D Gaussians to build a photo-realistic 3D face model. \\
Photo-realistic 3D face models learn a powerful prior over human facial appearance and geometry, which can be fitted to a single or multiple images of a person, effectively inferring a 3D head avatar. However, the fitting procedure itself is non-trivial and often requires expensive test-time optimization, impeding casual use-cases on consumer-grade devices. While this limitation may be circumvented by learning a generalized encoder that maps images into the 3D face model's latent space, another fundamental limitation remains. Even with more multi-view face datasets being published, the number of available training subjects rarely exceeds the thousands, making it hard to truly learn the full distibution of human facial appearance. Instead, our approach avoids generalizing over the identity axis by conditioning on some images of a person, and only generalizes over the expression axis for which plenty of data is available. 

A similar motivation has inspired recent work on codec avatars where a generalized network infers an animatable 3D representation given a registered mesh of a person~\cite{cao2022authentic, li2024uravatar}.
The resulting avatars exhibit excellent quality at the cost of several minutes of video capture per subject and expensive test-time optimization.
For example, URAvatar~\cite{li2024uravatar} finetunes their network on the given video recording for 3 hours on 8 A100 GPUs, making inference on consumer-grade devices impossible. In contrast, our approach directly regresses the final 3D head avatar from just four input images without the need for expensive test-time fine-tuning.




\subsection{Contributions}
This paper introduces a novel communication framework compatible with reconstruction-oriented communication, especially for edge inference, termed \glsreset{atroc}\gls{atroc}. By extending \gls{ib} theory \cite{Tishby_1999_TIB} and incorporating \gls{jscc} modulation, this framework is designed to enhance AI-driven applications. It prioritizes task relevance in data transmission strategies, shifting focus from traditional signal reconstruction fidelity to operational efficiency and effectiveness in real-world applications. The key contributions of this research are summarized as follows:




\begin{itemize}
    \item \textbf{Development of an \gls{atroc} Framework:} Based on \gls{ib} theory, we develop a framework that aligns task-oriented communications with reconstruction-oriented communications. The framework focuses on maximizing mutual information between inference results and encoded features, minimizing mutual information between the encoded features and the input data, and preserving task-specific information through the information reshaper. This reshaper is expert at transforming received symbols into task-specific data, maintaining the same data structure as the input while ensuring the preservation of task-specific information.

    \item \textbf{Innovation of an Information Reshaper:} We introduce an information reshaper within our extended \gls{ib} theory, laying a foundational aspect of \gls{atroc}. This component is crucial for adapting the communication to the specific needs of the task without compromising the integrity of the transmitted data.

    \item \textbf{Variational Approximation for Tractable Information Estimation:} Due to the intractability of mutual information in the training and inference of deep neural networks, we employ a variational approximation approach, known as \gls{vib}. This approach allows us to establish a tractable upper bound for these terms, enabling training and inference of deep neural networks.

    \item \textbf{Adaptation of a \gls{jscc} Modulation Scheme:} We design a \gls{jscc} modulation scheme that aligns \gls{jscc} symbols with a predefined constellation scheme. This scheme ensures compatibility of our framework with classic modulation techniques, making it more adaptable to existing communication infrastructures.

    \item \textbf{Performance Enhancement in Edge-Based Autonomous Driving:} In our simulation, we validate that the \gls{atroc} framework outperforms reconstruction-oriented methods for edge-based autonomous driving \cite{Wu_2022_TgC}. 
    Specifically, our method reduced 99.19\% communication load, in terms of bits per service, compared to existing methods, without compromising the driving score of the autonomous driving agent.
\end{itemize}

\subsection{Organization and Notations}
The rest of this paper is organized as follows: \Cref{sec:aligned_TOC_framework} details the system model and discusses how the proposed framework advances reconstruction-oriented and non-aligned task-oriented communication approaches. \Cref{sec:AIB} introduces the \gls{ib} theory for \gls{atroc} and elaborates on the corresponding \gls{vib} derivation. In \cref{sec:modulation}, we propose a \gls{jscc} modulation technique that is compatible with classical modulation methods, such as \gls{qam}. \Cref{sec:edge_AI} extends the framework of \gls{vib} to enhance edge-based autonomous driving applications. The experimental results are presented in \cref{sec:result}, which evaluates the performance of our proposed \gls{atroc} framework and the \gls{jscc} modulation. Finally, \cref{sec:conclusion} concludes the paper.

\Cref{tab_notation} lists the main symbols used throughout this paper. 
%providing a quick reference to assist in understanding the mathematical and technical discussions.

% \begin{table}[t]
%     \centering
%     \caption{Basic Notation.}
%     \label{tab: Notation}
    
%     \begin{tabular}{ll}
%     \toprule
%     Notation& Description\\
%     \midrule
%     $W$, $H$, $D$& Spatial dimension of images (width, height and depth)\\
%     $n$& Number of stages\\ 
%     $\mathbf{I}_{m}$& Moving source images\\
%     $\mathbf{I}_{f}$& Fixed target images\\
%     $\mathbf{I}_{w}^{(k)}$& Intermediate warped registered images at stage $k$\\
%     $\mathbf{I}_{w}^{(n)}$& Final warped registered images at stage $n$\\
%     $\phi$& Deformation field - transformation between images\\
%     $\phi_L^{(k)}$& Local deformation field between $\mathbf{I}_{w}^{(k-1)}$ and $\mathbf{I}_{f}$ at stage $k$\\ 
%     $\phi_G^{(k)}$& Global deformation field between $\mathbf{I}_{m}$ and $\mathbf{I}_{f}$ at stage $k$\\ 
%     $\mathcal{T}$& Warping operation - resamples the image by given $\phi$ \\
%     \midrule
%      & Where $k=[1, \ldots, n]$\\
%     \bottomrule
%     \end{tabular}
% \end{table}
\newpage
\begin{table*}[ht]
    \centering
    \caption{Basic Notation.}
    \label{tab: Notation}
    \vspace{-0pt}
    \begin{tabularx}{1.0\linewidth}{lX}
    \toprule
    Notation& Description\\
    \midrule

    $W$, $H$, $D$& 
    width, height and depth dimensions of the 3D images\\
    
    % $W$, $H$, $D$& Spatial dimension of the image (width, height and depth).\\
    %$\mathcal{D} = \{\{\mathbf{S}_\text{1}, \cdots, \mathbf{S}_\text{Z} \},\mathbf{T}\}$ & the given unlabeled dataset, $\mathbf{S}_{i}$ denotes the $i$-th source image in the dataset. \\
    
    $\mathbf{S}_i$ &
    source image (\ie raw MRI scan), $\mathbf{S}_{i} \in \mathbb{R}^{W \times H \times D}$ \\

    $\mathbf{M}_i$ &
    ground truth brain extraction mask of $\mathbf{S}_i$, $\mathbf{M}_i \in \{0,1\}^{W \times H \times D}$\\

    $y_i$ &
    ground truth classification label of $\mathbf{S}_i$, $y_i \in \mathcal{Y}$\\

    $\mathcal{Y}$ &
    classification label space (\eg $\{0, 1\}$ for binary classification)\\

    $\mathcal{D} = \{(\mathbf{S}_i,\mathbf{M}_i,y_i)\}_{i=1}^{Z}$ &
    dataset contains $Z$ source images, each paired with a brain extraction mask $\mathbf{M}_i$ and classification label $y_i$ \\

    $\mathbf{T}$ &
    target image, $\mathbf{T} \in \mathbb{R}^{W \times H \times D}$\\

    $\mathbf{B}$ &
    segmentation mask of $\mathbf{T}$, $\mathbf{B} \in \{0, 1\}^{C \times W \times H \times D}$\\

    $\mathbf{P}$ &
    parcellation mask of $\mathbf{T}$, $\mathbf{P} \in \{0, 1\}^{K \times W \times H \times D}$\\

    
    $C$ &
    number of segmentation labels (\ie the number of labeled brain tissue types)\\

    $K$ &
    number of labeled brain regions (\ie the number of ROIs)\\


    \midrule
    % $\hat{\mathbf{M}}_i$ &
    % binary tensor of identical dimensions to the source image $\mathbf{S}_i$, represents cerebral tissues in $\mathbf{S}$ with a value of 1 and non-cerebral tissues with 0\\

     $\hat{\mathbf{M}}$ &
    predicted brain extraction mask, $\hat{\mathbf{M}} \in \{0, 1\}^{W \times H \times D}$\\
    % \midrule
    
    $\hat{y}$&
    predicted classification label, $\hat{y} \in \mathcal{Y}$\\

    $\mathbf{E}$ &
    extracted image, obtained by $\mathbf{E} = \mathbf{S} \circ \hat{\mathbf{M}}$, $\mathbf{E}  \in \mathbb{R}^{W \times H \times D}$\\

    $\mathbf{a}$ &
    vector of the affine transformation, $\mathbf{a} \in \mathbb{R}^{12} $ \\  
    
    $\mathbf{A}$&
    matrix of an 3D affine transformation, $\mathbf{A} \in \mathbb{R}^{4 \times 4} $ \\

    $\mathcal{T}(\cdot, \cdot)$&
    affine transformation operator\\
    
    $\mathbf{W}$ &
    warped image, obtained by $\mathbf{W} = \mathcal{T}\left(\mathbf{E},\mathbf{A}\right)$\\

    $\mathbf{R}$&
    source segmentation mask, $\mathbf{R} \in \{0,1\}^{C \times W \times H \times D}$\\

    $\mathbf{V}$&
    warped segmentation mask, $\mathbf{V} \in \{0,1\}^{C \times W \times H \times D}$\\

    $\mathbf{U}$&
    warped parcellation mask, $\mathbf{U} \in \{0,1\}^{K \times W \times H \times D}$\\

    $\tilde{\mathbf{S}}$&
    source images expanded from $\mathbf{S}$ for ROI extraction, $\tilde{\mathbf{S}} \in \mathbb{R}^{K \times W \times H \times D}$\\

    $\mathbf{F}$&
    ROI extracted image (\ie parcellated image), obtained by $\mathbf{F} = \tilde{\mathbf{S}} \circ \mathbf{U}$, $\mathbf{F} \in \mathbb{R}^{K \times W \times H \times D}$\\
    
    $\mathbf{H}$&
    matrix of ROI feature (\ie node feature), $\mathbf{H} \in \mathbb{R}^{K \times N}$\\

    $N$&
    feature vector length of $\mathbf{H}$\\

    $\mathbf{C}$&
    adjacency matrix, measures brain network connectivity, $\mathbf{C} \in \mathbb{R}^{K \times K}$\\


     %\midrule
    $\mathcal{V}$&
    set of nodes (\ie ROIs), $\mathcal{V}=\{v_j\}_{j=1}^{K}$\\
    % \midrule
    
    $G$&
    brain network, $G = (\mathcal{V},\mathbf{C},\mathbf{H})$\\
    \midrule
    $f_{\theta}(\cdot)$&
    extraction function, $f_{\theta}: \mathbb{R}^{W \times H \times D} \rightarrow \mathbb{R}^{W \times H \times D} ; \mathbf{S} \mapsto \hat{\mathbf{M}}$\\

    $g_{\phi}(\cdot, \cdot)$&
    registration function, $g_{\phi}: \mathbb{R}^{W \times H \times D}\times \mathbb{R}^{W \times H \times D} \rightarrow \mathbb{R}^{12} ; (\mathbf{E},\mathbf{T}) \mapsto \mathbf{A}$\\

    $h_{\psi}(\cdot)$&
    segmentation function $h_{\psi}: \mathbb{R}^{W \times H \times D} \rightarrow \mathbb{R}^{C \times W \times H \times D} ; \mathbf{S} \mapsto \mathbf{R} $\\

    $n_{\xi}(\cdot)$&
    brain network generation function, $n_{\xi}: \mathbb{R}^{K \times W \times H \times D} \rightarrow \mathbb{R}^{K \times N} ; \mathbf{F} \mapsto \mathbf{H} $\\

    $c_{\eta}(\cdot,\cdot)$&
    classification function $c_{\eta}: (\mathbb{R}^{K \times K}, \mathbb{R}^{K \times N}) \rightarrow \mathcal{Y} ; (\mathbf{C},\mathbf{H}) \mapsto \hat{y}$\\

    $\mathcal{P^*}$&
    optimal parameter set, $\mathcal{P^*} = \{\theta^*, \phi^*, \psi^*, \xi^*, \eta^*\}$\\

    $\mathcal{L}_{cls}(\cdot, \cdot)$ &
    classification loss term, \ie cross-entropy loss\\

    $\mathcal{L}_{ext}(\cdot, \cdot)$ &
    extraction loss term, \ie cross-entropy loss\\

    $\mathcal{L}_{sim}(\cdot, \cdot)$ &
    image dissimilarity loss term, \ie negative cross-correlation\\

    $\mathcal{L}_{seg}(\cdot, \cdot)$ &
    segmentation loss term, \ie cross-entropy loss\\
    
    
    % \\\\
    
    % $M$, $N$& 
    % the number of extraction stages, the number of registration stages. \\
    
    % $\mathbf{S}_{i}$ & the source image (to be extracted and registered with the target image), $\mathbf{S}_{i} \in \mathbb{R}^{W \times H \times D} $.  \\
    % $\mathbf{T}$ & the target image (template brain), $\mathbf{T} \in \mathbb{R}^{W \times H \times D}$.\\
    % $\mathcal{D} = \{\{\mathbf{S}_{i}\},\mathbf{T}\}_{i=1}^{Z}$ & the unlabeled dataset consists of $Z$ source images and one target image (no extraction and registration labels).\\
    % $\mathbf{E}_{i}$ & the extracted image (extracted brain region from the source image $\mathbf{S}_{i}$), $\mathbf{E} \in \mathbb{R}^{W \times H \times D}$.\\
    % $\mathbf{E}_{i}^{j}$ & the extracted image at extraction stage $j$, where $j \in [1 : M]$.\\
    % $\mathbf{W}_{i}$ & the warped image (image that has been registered to the target image $\mathbf{T}$), $\mathbf{W} \in \mathbb{R}^{W \times H \times D}$.\\
    % $\mathbf{W}_{i}^{k}$ & the warped image at registration stage $k$, where $k \in [1 : N]$.\\
    % $\mathbf{M}_{i}$&  the binary brain mask (to extracted the brain region from the source image $\mathbf{S}_{i}$), $\mathbf{M}_{i} \in \{0,1\}^{W \times H \times D}$.\\
    % $\mathbf{M}_{i}^{j}$& the binary brain mask at extraction stage $j$, where $j \in [1 : M]$.\\
    % $\mathbf{a}_{i}$& the vector of the affine transformation parameters, $\mathbf{a}_{i} \in \mathbb{R}^{12}$.\\
    % $\mathbf{A}_{i}$& the matrix of the affine transformation, $\mathbf{A}_{i} \in \mathbb{R}^{4 \times 4}$.\\
    % ${\mathbf{A}_\text{i}}_{i}^{k}$ & the matrix of the current affine transformation between $\mathbf{W}_{i}^{k-1}$ and $\mathbf{T}$ at registration stage $k$, where $k \in [1 : N]$.\\
    % ${\mathbf{A}_\text{c}}_{i}^{k}$ & the matrix of the combined affine transformation between $\mathbf{E}_{i}$ and $\mathbf{T}$ at registration stage $k$, where $k \in [1 : N]$.\\
    % $\mathcal{T}(\cdot, \cdot)$& the affine warping operation. \\
    
    
    
    
    
    
    
    
    
    % $\mathbf{I}_\text{s}$& The source image (to be extracted and registered with the target image).\\
    % \vspace{1pt}
    % $\mathbf{I}_\text{t}$& The target image (template brain).\\
    % \vspace{1pt}
    % $\mathbf{I}_\text{e}^{(j)}$& The extracted image at stage $j$.\\
    % \vspace{1pt}
    % $\mathbf{I}_\text{w}^{(k)}$& The warped image at stage $k$.\\
    % $\mathbf{M}$&  Binary brain mask.\\
    % $\mathbf{M}^{(j)}$& Current brain mask of $\mathbf{I}_\text{e}^{(j-1)}$ at stage $j$.\\
    % $\mathbf{a}$& Affine transformation parameters.\\
    % $\mathbf{A}$& Affine transformation matrix.\\
    % $\mathbf{A}_\text{i}^{(k)}$& Current affine transformation between $\mathbf{I}_\text{w}^{(k-1)}$ and $\mathbf{I}_\text{t}$ at stage $k$.\\
    % $\mathbf{A}_\text{c}^{(k)}$& Combined affine transformation between $\mathbf{I}_\text{s}$ and $\mathbf{I}_\text{t}$ at stage $k$.\\

    % $\mathcal{T}$& Affine warping operation. \\
    
    % \midrule
    %  & where $j=[1, \ldots, m]$ and $k=[1, \ldots, n]$  \\
    \bottomrule
    \end{tabularx}
    \vspace{-0pt}
\end{table*}