\section{Related Works}
\subsubsection{Joint Source-Channel Coding}
The rapid evolution of deep learning has significantly influenced communication system designs, aiming to achieve or even surpass the Shannon limit. Deep learning based \gls{jscc} has emerged as a robust solution in scenarios characterized by limited bandwidth and low \gls{snr}. Research in deep \gls{jscc} for reconstruction-oriented communication \cite{Kurka_2019_SRo, Bourtsoulatze_2019_DJS, Kurka_2020_DfD} has demonstrated its superiority over traditional source coding methods, such as JPEG \cite{Wallace_1992_TJs} and JPEG2000 \cite{Taubman_2002_JIc}, as well as channel coding techniques, such as LDPC codes \cite{Gallager_1962_Ldp}, particularly in environments with low \gls{snr}.

Existing reconstruction-oriented communication research primarily focused on data-centric metrics (e.g., \gls{psnr} \cite{Kurka_2019_SRo, Tung_2022_DQC, Yang_2022_OGD, Tung_2022_DLA, Kurka_2020_DfD}, \gls{ssim} \cite{Bourtsoulatze_2019_DJS, Tung_2022_DLA, Yang_2022_OGD, Kurka_2020_DfD}, and \gls{msssim} \cite{Yang_2022_OGD, Tung_2022_DLA, Kurka_2020_DfD}) to evaluate the effectiveness of deep \gls{jscc}. However, these metrics often lead to suboptimal task performance since high-fidelity reconstructions are not always necessary from the machine's perspective, whereas task-specific semantic information plays the most important role \cite{Chaccour_2024_LDM, Yang_2023_SCf, Qin_2021_ScP, Strinati_2024_GOa, Pandey_2023_GOC, Kang_2023_PSi}. For example, in text transmission, the fidelity of words might be compromised to improve communication efficiency while still conveying the intended meanings \cite{Xie_2021_DLE, Farsad_2018_DLf}. Similarly, in image transmission, image fidelity can be sacrificed for less communication overhead and higher task performance \cite{Hu_2022_RSC, Diao_2024_TOS, Diao_2024_TTG}. 

Nonetheless, existing works, such as \cite{Shao_2023_SCW}, assumed that the amplitudes and phases of channel symbols are analog. Thus, we cannot implement them directly in digital communication systems \cite{_2020_ISf}. To address this issue, the authors of \cite{Choi_2019_NJS} explored image transmission over the discrete channel (binary symmetric channel) using variational learning with a Bernoulli prior. This work was further extended by the authors of \cite{Song_2020_Inj}, who introduced adversarial regularization to enhance robustness. Furthermore, recent works \cite{Tung_2022_DQC, Tung_2022_DQCa} investigated the transmission of natural images over an \gls{awgn} channel model with a finite channel input alphabet. Despite a good fit between the learned constellation diagram and the latent representation, the irregularity of the constellation diagram still poses significant challenges for deployment on commercial hardware. The author of \cite{Hu_2024_DTO} developed a digital task-oriented communication framework employing a hardware-limited scalar quantization approach, specifically tailored for computation-constrained situations, such as \gls{iot}. The results of this work provide valuable insights for future task-oriented \gls{jscc} designs.


\subsubsection{Edge Inference}
Edge inference has gained prominence as a solution to meet the stringent latency requirements of modern applications, which are not adequately supported by traditional cloud services \cite{Shi_2020_CEE, Li_2018_EIO}. The key architectural approach that underpins recent advancements is \textit{split inference}, where the inference network is partitioned between the device and the edge \cite{Huang_2020_DCR, Shi_2019_IDE, Li_2020_EAO, Shao_2020_CCT, Shao_2020_BAE, Jankowski_2020_JDE, Jankowski_2021_WIR, Shao_2022_LTO, Shao_2023_TOC}.

In this architecture, a mobile device initially processes data using a lightweight neural network to extract a compact feature vector. Subsequently, this vector is transmitted to an edge server for further processing, where deep \gls{jscc} is integral to the entire procedure \cite{Shao_2020_CCT, Shao_2020_BAE, Jankowski_2020_JDE, Jankowski_2021_WIR, Shao_2022_LTO, Shao_2023_TOC}. Notably, an end-to-end framework that efficiently compresses intermediate features to optimize the bandwidth and computational resources at the edge was introduced in \cite{Shao_2020_BAE}. In addition, the authors of \cite{Shao_2022_LTO} developed a method to flexibly adjust the length of the transmission signal to adapt to dynamic communication environments while maintaining targeted inference accuracy.

Recent studies have shifted from reconstruction-oriented communication, which focuses on accurately reconstructing a signal at the receiver, to a task-oriented approach that prioritizes inference accuracy as the primary performance metric \cite{Dubois_2021_LCf, Shao_2021_BGA, Shao_2022_LTO, Shao_2020_BAE, Shao_2023_TOC}. This paradigm shift underscores a move towards optimizing communication systems to support specific functional requirements rather than general data fidelity.

Note that implementing such split-design architectures often necessitates modifications on both the device and the edge, which pose challenges in terms of compatibility with existing communication infrastructures. This issue highlights a significant barrier to widespread adoption, indicating the need for more compatible solutions that can seamlessly integrate with current technologies.


\subsubsection{Variational Information Bottleneck}
The \gls{ib} theory, which extends from the foundational rate-distortion theory \cite{Cover_1991_EoI}, aims to find an optimal trade-off by maximizing the preservation of task-specific information in the latent representations, while minimizing the inclusion of task-agnostic information from the input data. Initially proposed by \cite{Tishby_1999_TIB}, the practical application of \gls{ib} theory in training deep neural networks remained theoretical until significantly later \cite{Tishby_2015_Dla}.

The application of \gls{ib} theory in deep learning was primarily hindered by computational challenges. The traditional optimization of the \gls{ib} objective function relied on the iterative Blahut-Arimoto algorithm \cite{Arimoto_1972_Aaf, Blahut_1972_Coc}, which is infeasible for deep learning applications due to its computational complexity and inefficiency in handling large-scale data \cite{Tishby_2015_Dla}. Addressing this limitation, \cite{Alemi_2017_DVI} introduced a variational approach to construct a tractable lower bound on the \gls{ib} objective, leading to the development of the \gls{vib} method. This approach enabled the practical application of the \gls{ib} principles in deep learning by approximating the intractable true posterior with a variational distribution.

Recent work has seen the integration of \gls{vib} with deep \gls{jscc}, which has been effectively used to formalize task-oriented communication strategies. In particular, the results \cite{Shao_2022_LTO, Shao_2023_TOC} have demonstrated that combining \gls{vib} with deep \gls{jscc} offers superior performance over reconstruction-oriented communication frameworks. These studies showcase the potential of \gls{vib} in improving the efficiency and robustness of communication systems, particularly in scenarios where preserving task-specific information and discarding task-agnostic information are crucial.

Integrating \gls{jscc} and \gls{ib} methods to protect user privacy is an advanced direction in current research. FedSem \cite{Wei_2023_FSL} had collaboratively trained semantic-channel encoders of multiple devices coordinated by a semantic-channel decoder using \gls{ib} theory based on base stations. Unlike traditional centralized learning approaches, FedSem reduces communication overhead and mitigates privacy concerns by enabling the sharing of semantic features rather than raw data. In addition, the author of \cite{Sun_2024_DIB} introduced a privacy-preserving \gls{jscc} scheme for image transmission, using a disentangled \gls{ib} objective to effectively separate private information from public data. This approach ensures the protection of privacy-sensitive information while maintaining high image quality. Although these works show impressive progress in the integration of \gls{jscc} with \gls{ib} theory, they often require specialized designs that are challenging to combine with existing systems and devices.

There is a need to design an advanced framework aligning two communication paradigms -- task-oriented communications and reconstruction-oriented communications -- and develop a \gls{jscc} modulation scheme for practical deployment.