\documentclass{article}

% Recommended, but optional, packages for figures and better typesetting:
\usepackage{microtype}
\usepackage{graphicx}
\usepackage{subfigure}
\usepackage{booktabs} % for professional tables
\usepackage{multirow}
% hyperref makes hyperlinks in the resulting PDF.
% If your build breaks (sometimes temporarily if a hyperlink spans a page)
% please comment out the following usepackage line and replace
% \usepackage{icml2025} with \usepackage[nohyperref]{icml2025} above.
\usepackage{hyperref}
\usepackage{enumitem}

\usepackage{listings}
\usepackage{xcolor} % for setting colors

\lstset{language=Python,
        basicstyle=\ttfamily\small,
        keywordstyle=\color{blue},
        stringstyle=\color{red},
        commentstyle=\color{green},
        morecomment=[l][\color{magenta}]{\#},
        frame=single, % adds a frame around the code
        breaklines=true, % sets automatic line breaking
        backgroundcolor=\color{gray!10}, % set background color
        showstringspaces=false, % prevents showing spaces in strings as a special character
        tabsize=4
}

% Attempt to make hyperref and algorithmic work together better:
\newcommand{\theHalgorithm}{\arabic{algorithm}}

% Use the following line for the initial blind version submitted for review:
% \usepackage{icml2025}

% If accepted, instead use the following line for the camera-ready submission:
\usepackage[accepted]{icml2025} %

% For theorems and such
\usepackage{amsmath}
\usepackage{amssymb}
\usepackage{mathtools}
\usepackage{amsthm}

% if you use cleveref..
\usepackage[capitalize,noabbrev]{cleveref}

%%%%%%%%%%%%%%%%%%%%%%%%%%%%%%%%
% THEOREMS
%%%%%%%%%%%%%%%%%%%%%%%%%%%%%%%%
\theoremstyle{plain}
\newtheorem{theorem}{Theorem}[section]
\newtheorem{proposition}[theorem]{Proposition}
\newtheorem{lemma}[theorem]{Lemma}
\newtheorem{corollary}[theorem]{Corollary}
\theoremstyle{definition}
\newtheorem{definition}[theorem]{Definition}
\newtheorem{assumption}[theorem]{Assumption}
\theoremstyle{remark}
\newtheorem{remark}[theorem]{Remark}

% Todonotes is useful during development; simply uncomment the next line
%    and comment out the line below the next line to turn off comments
%\usepackage[disable,textsize=tiny]{todonotes}
\usepackage[textsize=tiny]{todonotes}
\usepackage[framemethod=TikZ]{mdframed}
\mdfsetup{%
    middlelinecolor =   none,
    middlelinewidth =   1pt,
    backgroundcolor =   blue!5,
    roundcorner     =   5pt,
}
% 设置第一个框的样式
\newmdenv[
    middlelinecolor=none,
    middlelinewidth=1pt,
    backgroundcolor=blue!5,
    roundcorner=5pt
]{bluebox}

% 设置第二个框的样式
\newmdenv[
    middlelinecolor=none,
    middlelinewidth=1pt,
    backgroundcolor=gray!20,
    roundcorner=5pt
]{graybox}
\usepackage{xcolor}
\usepackage{tcolorbox}
\newtcbox{\grayboxtext}[1][]{
    on line,
    colframe=gray!20,
    colback=gray!20,
    boxrule=0.5pt,
    arc=4pt,
    boxsep=0pt,
    left=2pt,
    right=2pt,
    top=1pt,
    bottom=1pt,
    #1
}

% The \icmltitle you define below is probably too long as a header.
% Therefore, a short form for the running title is supplied here:
\icmltitlerunning{Reusing Embeddings: Reproducible Reward Model Research in Large Language Model Alignment without GPUs}

\begin{document}

\twocolumn[
% \icmltitle{Position: Embeddings-as-Inputs Accelerates Reward Model Research in Large Language Model Alignment}
\icmltitle{Reusing Embeddings: Reproducible Reward Model Research \\in Large Language Model Alignment without GPUs}

% It is OKAY to include author information, even for blind
% submissions: the style file will automatically remove it for you
% unless you've provided the [accepted] option to the icml2025
% package.

% List of affiliations: The first argument should be a (short)
% identifier you will use later to specify author affiliations
% Academic affiliations should list Department, University, City, Region, Country
% Industry affiliations should list Company, City, Region, Country

% You can specify symbols, otherwise they are numbered in order.
% Ideally, you should not use this facility. Affiliations will be numbered
% in order of appearance and this is the preferred way.
\icmlsetsymbol{equal}{*}

\begin{icmlauthorlist}
\icmlauthor{Hao Sun}{ucam}
\icmlauthor{Yunyi Shen}{mit}
\icmlauthor{Jean-Fran\c cois Ton}{tt}
\icmlauthor{Mihaela van der Schaar}{ucam}
%\icmlauthor{}{sch}
%\icmlauthor{}{sch}
\end{icmlauthorlist}

\icmlaffiliation{ucam}{University of Cambridge}
\icmlaffiliation{tt}{ByteDance Research}
\icmlaffiliation{mit}{Massachusetts Institute of Technology}

\icmlcorrespondingauthor{Hao Sun}{hs789@cam.ac.uk}
% \icmlcorrespondingauthor{Firstname2 Lastname2}{first2.last2@www.uk}

% You may provide any keywords that you
% find helpful for describing your paper; these are used to populate
% the "keywords" metadata in the PDF but will not be shown in the document
\icmlkeywords{Machine Learning, ICML}

\vskip 0.3in
]

% this must go after the closing bracket ] following \twocolumn[ ...

% This command actually creates the footnote in the first column
% listing the affiliations and the copyright notice.
% The command takes one argument, which is text to display at the start of the footnote.
% The \icmlEqualContribution command is standard text for equal contribution.
% Remove it (just {}) if you do not need this facility.

\printAffiliationsAndNotice{}  % leave blank if no need to mention equal contribution
% \printAffiliationsAndNotice{\icmlEqualContribution} % otherwise use the standard text.

\begin{abstract}
Large Language Models (LLMs) have made substantial strides in structured tasks through Reinforcement Learning (RL), demonstrating proficiency in mathematical reasoning and code generation. However, applying RL in broader domains like chat bots and content generation --- through the process known as Reinforcement Learning from Human Feedback (RLHF) --- presents unique challenges. Reward models in RLHF are critical, acting as proxies that evaluate the alignment of LLM outputs with human intent. Despite advancements, the development of reward models is hindered by challenges such as computational heavy training, costly evaluation, and therefore poor reproducibility. \textbf{We advocate for using embedding-based input in reward model research as an accelerated solution to those challenges.} By leveraging embeddings for reward modeling, we can enhance reproducibility, reduce computational demands on hardware, improve training stability, and significantly reduce training and evaluation costs, hence facilitating fair and efficient comparisons in this active research area. We then show a case study of reproducing existing reward model ensemble research using embedding-based reward models. We discussed future avenues for research, aiming to contribute to safer and more effective LLM deployments.
\end{abstract}

\section{Introduction}

%
% LLM's training and inference are too expensive
%
Large-scale language models (LLMs) have achieved remarkable results across various natural language processing applications \citep{NEURIPS2020_1457c0d6,wei2022chain,ouyang2022training,openai2024gpt4technicalreport}. This success largely depends on scaling the number of model parameters, the amount of training data, and computational resources \citep{kaplan2020scalinglawsneurallanguage,NEURIPS2022_c1e2faff}, which leads to substantial training and inference costs of LLMs. Building and deploying high-performance models also require enormous resources, posing a significant barrier for many researchers and practitioners.

%
% MoE
%
The \emph{Mixture of Experts} (MoE) architecture has emerged as a promising approach to address the escalating resource demands of LLMs. MoE introduces multiple experts into some parts of the network, but only a subset is activated at any given time, allowing the model to achieve superior performance with reduced training and inference costs \citep{shazeer2017,lepikhin2021gshard,Fedus2021SwitchTS}. In fact, cutting-edge industry models like Gemini 1.5 \citep{geminiteam2024gemini15unlockingmultimodal} and GPT-4 (based on unofficial reports) \citep{openai2024gpt4technicalreport} have adopted MoE, suggesting its effectiveness. 


%
% MoE Challenge
%
We refer to transformer-based LLMs without MoE as \emph{dense models} and those incorporating MoE as \emph{MoE models}.
Upcycling~\citep{komatsuzaki2023sparse} is an approach that initializes and trains an MoE model using a pre-trained dense model, which aims to transfer learned knowledge for better initial performance.
However, \NUname{} copies the feedforward network (FFN) layers during initialization, which makes it difficult to achieve expert specialization.
This disadvantage prevents effective utilization of the MoE models' full capacity, resulting in slower convergence over long training periods.
Thus, there exists a trade-off between the short-term cost savings from knowledge transfer and the long-term convergence efficiency through expert specialization.



In this paper, we propose \emph{\methodname{}} -- a method that effectively addresses this trade-off, as briefly illustrated in Figure \ref{fig:drop_upcycling}. \methodname{} works by selectively re-initializing the parameters of the expert FFNs when expanding a dense model into an MoE model. The method is carefully designed to promote expert specialization while preserving the knowledge of pre-trained dense models. Specifically, common indices are randomly sampled along the intermediate dimension of the FFNs, and the weights are dropped either column-wise or row-wise, depending on the weight matrix types. The dropped parameters are then re-initialized using the statistics of those weights.



%
% Experimental results
%
Extensive large-scale experiments demonstrate that \methodname{} nearly resolves the trade-off between the two aforementioned challenges
and significantly outperforms previous MoE model construction methods such as training from scratch and \NUname{}.
By leveraging pre-trained dense models, \methodname{} can start training from a better initial state than training from scratch, reducing training costs.
On the other hand, \methodname{} avoids the convergence slowdowns observed with \NUname{}.
Specifically, in our extensive long-term training experiments, \methodname{} maintained a learning curve slope similar to that of training from scratch, consistently staying ahead.
This success is attributed to effective expert specialization.
As a result, we constructed an MoE model with 5.9B active parameters that performs on par with a 13B dense model from the same model family, while requiring only approximately 1/4 of the training FLOPs.

\begin{figure}[t]
    \centering
    \includegraphics[width=\textwidth]{images/overview.pdf}
\vskip -8pt 
\caption{\textbf{Overview of the \methodname{} method.} The key difference from the na\"{i}ve Upcycling is Diversity re-initialization, introduced in Section \ref{sec:method}.}
    \label{fig:drop_upcycling}
\end{figure}

%
% Fully open
%
This research is fully open, transparent, and accessible to all.
With over 200,000 GPU hours of experimental results, conducted on NVIDIA H100 GPUs, all training data, source code, configuration files, model checkpoints, and training logs used in this study are publicly available. By providing this comprehensive resource, we aim to promote further advancements in this line of research.


%
%
%
Our technical contributions are summarized as follows:
\begin{itemize}
\item We propose \methodname{}, a novel method for constructing MoE models that effectively balance knowledge transfer and expert specialization by selectively re-initializing parameters of expert FFNs when expanding a dense model into an MoE model.

\item Extensive large-scale experiments demonstrate that \methodname{} consistently outperforms previous MoE construction methods in long-term training scenarios.

\item All aspects of this research are publicly available. %, including training data, source code, configuration files, model checkpoints, and training logs. 
This includes the MoE model with 5.9B active parameters that performs comparably to a 13B dense model in the same model family while requiring only about 1/4 of the training FLOPs.

\end{itemize}

\section{Reward Models with Embedding Inputs}
\label{sec:RM_from_embeddings}

\begin{figure}[h!]
    \includegraphics[width=1.0\linewidth]{figs/teaser.png}
    \vspace{-0.39cm}
    \caption{In reward model research, using embeddings as input (i.e., focusing on the pink box) brings the following benefits:
    1. there are much less parameters in those reward models;
    2. it has a much lower training cost than using LLM-based reward models;
    3. it has a much lower evaluation cost as compared to the LLM-based reward models; 
    4. it minimizes the inference-time cost by generating embeddings as by-products in language generation;
    5. research using embedding-based reward models are highly reproducible due to the low computational demand, high training stability, and minimal hardware requirement.}
    \label{fig:teaser}
\vspace{-0.3cm}
\end{figure}

\subsection{Alternatives to LLM-based Reward Models}
% In this section, we introduce the practice of using embeddings as input for reward modeling and compare it with the original natural language input practices. We use Figure~\ref{fig:teaser} to highlight the key difference. In the figure, green boxes denote trainable parameter groups, while gray boxes denote intermediate variables. The left panel of Figure~\ref{fig:teaser} illustrates how natural language inputs are processed by LLMs in generation tasks, while the right panel illustrates how natural language inputs are processed by LLM-based reward models in quality evaluation.

% When LLMs with replaced value heads are used for reward modeling, only a small number of parameters are removed, and the LLM-based reward model remains a large model having a large number of parameter freedom to learn. Training such a LLM-based reward model is computationally expensive and time-consuming even with modern graphic accelerators --- even with low-rank adapters like LoRA~\citep{hu2021lora}. For instance, training a 2B model on a Tesla-V100 GPU with LoRA on a typical alignment dataset containing 10k samples will roughly take 2 hours. Moreover, many hyper-parameters in such a training procedure may affect the final results. 

% On the other hand, since the motivation of building a reward model that is able to evaluate the natural language content, and it is well-known that the embedding space contains rich information of the natural language input both before and after the LLM era~\citep{mikolov2013efficient,pennington2014glove,devlin2018bert}, especially on downstream classification tasks~\citep{kiros2015skip,cer2018universal,brown2020language}. 
% A straightforward alternative of training the entire LLM for reward modeling is only to use the embeddings of language as inputs.
% In recent works, it has been shown that working on the embedding space can effectively build reward models for prompt evaluation in mathematical reasoning tasks~\citep{sun2023query}, or reward models for safety and helpful content evaluation~\citep{sun2024rethinking}. The typical training time for those models is $1$ to $5$ minutes on CPU-only machines.

% Practically, during the language generation process, embeddings are generated as by-products, therefore collecting those embeddings for reward models requires no more computational burden. To better understand the performance of this alternative choices, we now present experiment results to compare the two approaches empirically.

In this section, we explore the use of embeddings as inputs for reward modeling and contrast this approach with traditional methods employing natural language inputs. Figure~\ref{fig:teaser} illustrates the key differences: green boxes represent trainable parameter groups, while gray boxes denote intermediate variables. The left panel depicts the processing of natural language inputs by LLMs for generation tasks, whereas the right panel shows their use in LLM-based reward models for quality evaluation.

When LLMs equipped with replaced value heads are utilized for reward modeling, only a minimal number of parameters are removed. Consequently, these models retain a substantial degree of parameter freedom, making them large and computationally demanding. For example, training a 2B-parameter model using LoRA~\citep{hu2021lora} on a Tesla-V100 GPU with a typical alignment dataset of 10,000 samples approximately requires two hours. Additionally, the training process involves numerous hyperparameters that can significantly influence the outcomes.


Conversely, given the aim to evaluate natural language content effectively, and recognizing that the embedding space encapsulates a rich representation of the input both before and during the LLM era—as evidenced in tasks ranging from classification to more complex applications~\citep{mikolov2013efficient,pennington2014glove,devlin2018bert,kiros2015skip,cer2018universal,brown2020language} --- employing only embeddings as inputs presents a viable alternative. Recent studies have demonstrated the efficacy of this approach in constructing reward models for prompt evaluation in mathematical reasoning tasks and for assessing the safety and helpfulness of LLM-generated contents~\citep{sun2023query,sun2024rethinking}. Typically, these models require only 1 to 5 minutes of training time on CPU-only machines.

Moreover, as \textbf{embeddings are generated as by-products during the language generation process, utilizing them for reward models imposes no additional computational overhead}. To have a comprehensive understanding of this alternative method, we present experimental results that empirically compare the two approaches' performance in reward modeling.


%             | RM Parameter Number | Training Cost | Inference Cost | Reproducibility | 
% LLM-based | 3M - 3B | GPU, hours | High, LLM forward passes | Low, Many Hyper-params | 
% Embedding-based | 0.6M | CPU, minutes | Low, MLP forward passes | High, A few Hyper-params |


\subsection{Empirical Comparisons}
\paragraph{Reward Model Sizes} In the extant literature on reward models, LLMs typically range from 3M to 3B parameters, with specific instances such as \citet{coste2023reward} employing models between 14M and 1.3B parameters, \citet{ahmed2024scalable} using a 1.3B model, and \citet{gao2023scaling} exploring models from 3M to 3B parameters. By contrast, embedding-based methods, such as a typical 3-layer MLP with $2048$-dimensional input embeddings and $256$ hidden units, utilize fewer than 0.6M parameters. We also consider lightGBM models in our demonstrative experiments given their wide success and remarkable stability~\citep{ke2017lightgbm,grinsztajn2022tree,sun2023query}.


\paragraph{Data Generation Processes} We use the Anthropic-HH dataset, which includes the \texttt{Helpful} and \texttt{Harmless} alignment tasks to assess the efficacy of various reward model approaches~\citep{bai2022training}. The dataset contains $40000$ prompts for each task. To ensure reproducible and reliable comparisons, we use golden reward models as proxy annotators following established workflows in the literature~\citep{xiong2023gibbs,dong2024rlhf,dong2023raft,gao2023scaling,yang2024rewards}. We consider three LLMs --- Gemma-2B and -7B \citep{team2024gemma}, and LLaMA3-8B \citep{touvron2023llama} --- to generate responses. For each prompt, we generate $10$ responses and randomly select $N$ pairs for preference annotation using the golden reward models. We use Gemma2B to generate embeddings for the embedding-based reward models. This approach ensures that our evaluation accurately reflects the preferences of the golden reward model, thereby minimizing bias.

\paragraph{Annotation Quality and Quantity Control} In preference generation, the quality of annotations is often limited by the capabilities of the annotators \citep{sanderson2010user,stewart2005absolute,guest2016relative,wang2024secrets}. We apply the location-scale function class to describe annotation noise \citep{sun2024rethinking}, positing that closer values yield noisier personal preferences. We examine three annotation quality scenarios: 
\begin{enumerate}[nosep,leftmargin=*] 
\item \textbf{Low annotation quality}: high error rates (approximately $45\%$), offering minimal informative value in the preference annotations; 
\item \textbf{Medium-Low annotation quality}: error rates around $40\%$; 
\item \textbf{Medium-High annotation quality}: error rates around $30\%$; 
\item \textbf{High annotation quality}: error rates are about $5\%$. \end{enumerate}
In addition to quality, we also explore the impact of varying annotation quantities, considering the number of annotated preference pairs ranging from $500$ to $10000$.

\begin{figure*}[t!]
    \centering
    \includegraphics[width=1.0\linewidth]{figs/gemma2b_reward.png}\vspace{-0.35cm}
    \caption{\small  Comparing performances of Embedding-based RM  with LLM-based RMs. The Embedding-based RMs demonstrate high learning stability and strong performance as compared to LLM-based RMs, but are much cheaper to train and evaluate, and more scalable in inference time. Results are from the Gemma 2B model. Additional results using the Gemma 7B and LLaMA3 8B models are presented in Appendix~\ref{appdx:more_results}}
    \label{fig:performance_with_embeddings}\vspace{-0.25cm}
\end{figure*}


% Comparing Model Sizes In the extant literature on reward models, LLMs typically range from 3M to 3B parameters, with specific instances such as \citet{coste2023reward} employing models between 14M and 1.3B parameters, \citet{ahmed2024scalable} using a 1.3B model, and \citet{gao2023scaling} exploring models from 3M to 3B parameters. By contrast, embedding-based methods, such as a typical 3-layer MLP with $2048$-dimensional input embeddings and $256$ hidden units, utilize fewer than 0.6M parameters.

% \paragraph{Golden Reward Annotation and Evaluation}
% The dataset contains $40000$ prompts for each task. To make reproducible and reliable comparisons, we follow the literature~\citep{xiong2023gibbs,dong2024rlhf,dong2023raft,gao2023scaling,yang2024rewards} to use golden reward models as proxy annotators. To maximally isolate the source of gains from LLM choices and experimental randomness, we consider three LLMs in generating responses: Gemma-2B and -7B~\citep{team2024gemma}, LLaMA3-8B~\citep{touvron2023llama}. We generate $10$ responses for each prompt, and then randomly select $N$ pairs for preference annotation using the golden reward models. In this way, our evaluation maximally represents the preference of the golden reward model, hence it is unbiased. The golden reward model in our experiment setup can be considered as a consistent annotator. 

% \paragraph{Annotation Quality Control}
% In the process of preference generation, the quality of annotations may be constrained by the capabilities of the annotators~\citep{sanderson2010user,stewart2005absolute,guest2016relative,wang2024secrets}. We adopt the location-scale function class to characterize annotation noise~\citep{sun2024rethinking} --- which posits that the closer the values are to each other, the noisier the personal preferences tend to be --- instead of random adding flipping preference errors. We consider 3 different annotation quality settings: 
% \begin{enumerate}[nosep,leftmargin=*]
%     \item \textbf{Low annotation quality}, where annotation error rates are extremely high ($\approx 45\%$) hence there are only little signals in preference annotations;
%     \item \textbf{Medium annotation quality}, where annotation error rates are around $30\%$;
%     \item \textbf{High annotation quality}, where annotation error rates are around $5\%$. 
% \end{enumerate}

% \paragraph{Annotation Quantity Control}
% Besides the annotation quality, we also study the effect under different annotation quantities, considering the number of annotated preference pairs ranging from $500$ to $10000$.

% \textcolor{red}{We use Gemma2B to generate embeddings. Evaluation is based on Best-of-500 sampling. Other implementation details are provided in Appendix. MLP size, how to get embeddings, LoRA. Ref Figure 2}

% Results are shown in Figure~\ref{fig:performance_with_embeddings}, we have the following observations from the results:
% \begin{itemize}[nosep]
%     \item The embedding-based methods in general achieves significantly lower variance and have higher stability during training.
%     \item The embedding-based methods remarkably outperform smaller language models (LLM-RM-GPT2) in all cases. 
%     \item When annotation quality is low, the embedding-based methods achieve superior or on-par performance with the LLM-based reward models.
%     \item When annotation quantity is limited, the embedding-based methods achieve superior or on-par performance with the LLM-based reward models.
%     \item On the \texttt{Harmless} dataset, the embedding-based reward models can always match the performance of LLM-based reward models.
%     \item On the \texttt{Helpful} dataset, the embedding-based reward models underperform Gemma2B-based LLM reward models, the latter model gains more performance when annotation quality and availability increase.
% \end{itemize}



% Those datasets will be accessible as public assets for future research on reward modeling. We will elaborate on the scalable evaluation procedure later in Section~\ref{sec:scalable_evaluation}.
Results are presented in Figure~\ref{fig:performance_with_embeddings}. The following observations can be drawn from the analysis:

\begin{itemize}[nosep,leftmargin=*]
\item Generally, embedding-based methods exhibit significantly lower variance and higher stability during training compared to other models. 
\item Embedding-based methods consistently outperform smaller language models (such as LLM-RM-GPT2) across all evaluated scenarios. 
\item In conditions of low annotation quality, embedding-based methods demonstrate performance that is superior to or comparable with LLM-based reward models. 
\item With limited annotation quantities, embedding-based methods also show superior or comparable performance to LLM-based reward models. 
\item On the \texttt{Harmless} dataset, embedding-based reward models consistently match the performance of LLM-based reward models. 
\item On the \texttt{Helpful} dataset, however, embedding-based reward models underperform relative to Gemma2B-based LLM reward models, which benefit more significantly from increases in annotation quality and availability. \end{itemize}

These datasets will be made available as public assets to facilitate future research in reward modeling. Details on the scalable evaluation procedure will be provided in Section~\ref{sec:scalable_evaluation}.


\section{Motivations of Using Embeddings as Reward Model Inputs}
\label{sec:motivations}
\subsection{Reproducibility: Foundation of Research}
\label{sec:cheap_train}
Reproducibility is the foundation of scientific research. In the study of reward modeling, the ability to replicate results across different studies is essential for evaluating theoretical and practical contributions. Nonetheless, the reproduction of LLM-based reward model research often faces considerable obstacles, such as vulnerability to many sensitive hyperparameters, the necessity for large memory GPUs, large training instability, and extensive computational demands associated with slow training processes. These challenges can make the replication of existing works extremely challenging --- if not unfeasible --- for many of the research communities.


\begin{table*}[h!]
\begin{lstlisting}
# Load Training Data
train_embeddings, train_rewards = load_embd_data(task='Harmless', split='train')
print(train_embeddings.shape) 
### (40000, 10, 2048)
print(train_rewards.shape) 
### (40000, 10, 1)

# Load Testing Data
test_embeddings, test_rewards = load_embd_data(task='Harmless', split='test')
print(test_embeddings.shape) 
### (2000, 500, 2048)
print(test_rewards.shape) 
### (2000, 500, 1)

# Generation of Pairwise Comparisons
train_comparisons, train_labels = pair_annotate(train_embeddings, train_rewards)

# Train Embedding-based Reward Model (e.g., use a Bradley-Terry MLP)
reward_model = BT_MLP()
reward_model.fit(train_comparisons, train_labels)

# Make Predictions with the Reward Model on Testset
rm_predictions = reward_model.predict(test_embeddings)
print(rm_predictions.shape) 
### (2000, 500, 1)

# Calculate Evaluation Metrics on Testset
bon_500 = calc_bon(rm_predictions, test_rewards, N=500)
spearmanr = calc_spearmanr(rm_predictions, test_rewards)
\end{lstlisting}
\label{algo}\vspace{-0.4cm}
\end{table*}

As a consequence, in new research, if a method lacks systematic comparisons with established methods due to the above challenges, or its efficacy can not be verified through repeated and statistically significant trials, the results may be unfounded. 

The utilization of embedding-based reward models offers several advantages: \begin{enumerate}[nosep,leftmargin=*] 
\item \textbf{Reward Model Research without GPUs:} Conducting research and reproducing reward model research using embedding-based methods do not necessitate advanced, large-memory GPUs, thereby democratizing access to state-of-the-art research methods and facilitating the validation of novel algorithms by a wider academic community. 
\item \textbf{Lower Computational Requirements for Statistical Significance:} In embedding-based reward model research, the computational overhead is lower not only because of the much cheaper model training process but also for the more consistent results across multiple runs. And there are much less vulnerable hyperparameters that may drastically affect the results. This efficiency enables researchers to rapidly prototype, validate ideas, and innovate based on reliable conclusive empirical observations, maximally isolating the source of gains from complex LLM-based reward modeling systems, thereby accelerating the cycle of scientific discovery and validation in the field. 
\item \textbf{Data Standardization and Scalability:} In embedding-based reward model research, it is possible to create and share a standardized, publicly accessible dataset that includes multiple language models' generations (generality among models), contains a large number of samples (sufficient data for training), flexibly simulate annotation strategies (to stress test methods), and cost-efficient evaluation process.
\end{enumerate}


All of those aspects encourage reproducible research in embedding-based reward modeling, and thereby accelerate the pace of discoveries in the area.

\subsection{Scalable Evaluation with Embedding-based Reward Models}
\label{sec:scalable_evaluation}
In addition to the high computational costs of training, LLM-based reward modeling faces significant challenges in evaluation time and expense. Specifically, reward models are tasked with evaluating test-time generations to differentiate superior responses from inferior ones. Previous research has primarily utilized two metrics for this purpose: LLM-as-a-Judge and evaluation using open-sourced golden reward models~\citep{dong2023raft,dong2024rlhf}.

\textbf{High Cost in LLM-as-a-Judge Evaluation.}
The LLM-as-a-Judge evaluation, which often involves calling commercial APIs, can be prohibitively expensive for even medium-sized datasets. For example, in a study involving $3$ different language models and $2$ datasets, evaluating a proposed method using $2000$ test samples --- each comparison truncated to $1024$ tokens --- through the GPT-3.5 API incurs a cost of $20$ US dollars \textbf{per experiment}, and this cost will be amplified by the number of individual run of the experiment. Compounding such an issue, those results are not reusable.

Moreover, recent findings have exposed potential cheating behavior in LLM-as-a-Judge evaluations~\cite{zheng2024cheating}, further compromising the reliability of this costly method and challenging its feasibility as a community standard.


% In addition to the high computational cost associated with the reward model training, another difficulty faced by LLM-based reward modeling is the evaluation time expense. To be specific, reward models are designed to evaluate test-time generations, and distinguish the better responses from the bad ones. To accurately evaluate whether this objective can be met, previous research has mainly focused on two evaluation metrics: Golden reward model evaluation and LLM-as-a-Judge evaluation~\citep{dong2023raft}. 

% For the LLM-as-a-Judge evaluation, calling commercial APIs to evaluate a medium-sized dataset will be expensive. Moreover, those evaluation results can not be reused. For instance, in research where we have 3 different language models to study, and have 2 datasets to evaluate a proposed method, using $2000$ test samples for the evaluation, and each response is truncated to be $512$ tokens, calling the GPT3.5 API will spend $10$ US dollars --- this is only on a single experimental setups, and with a single run.

% Moreover, recent discoveries disclose the potential cheating behavior in LLM-as-a-judge evaluations, making this costly evaluation even more unreliable and infeasible for the community to use as a standard.

\textbf{Cost in Golden Reward Model Evaluation.} While the Golden Reward Model Evaluation avoids the use of commercial APIs, making it more accessible and economical for researchers, it still imposes substantial computational demands. For example, evaluating the aforementioned test case necessitates the LLM-based RM to process $12000$ pairs of sequences. In the more computationally intensive best-of-N evaluations, a typical study with $N=500$~\citep[KL divergence approximately 5 Nats,][]{gao2023scaling} requires \textbf{6 million forward passes}. Completing these passes using 2B-parameter LLMs on Tesla V100 GPUs can consume over 100 GPU hours. It is worth noting that this cost is associated with a single experiment setup and a single experimental trial.

\textbf{Cheap and Fast Evaluation with Embedding-based Reward Models.}
% On the other hand, when using embedding-based reward models, due to the embeddings are fixed, we are able to prepare a fixed test embedding dataset that contains sufficient samples for evaluation, and such an test dataset is re-usable for different methods. In the previous example, we only need to generate the embeddings and golden rewards of the $500$ test responses on each prompt once, then those embeddings and rewards are re-useable and agnostic to any choice of embeddings-based reward models. 
% We have done such a pre-processing step, and our dataset asset contains $500$ responses on each of the test prompts. To better understand the dataset we prepared for embedding-based reward modeling, we provide the following pseudo code as illustrations:
In contrast, embedding-based reward models leverage fixed embeddings, allowing for the preparation of a \textbf{standardized test dataset that is reusable across various methods}. For instance, in the scenario described above, we only need to generate the embeddings and golden rewards for the $500$ test responses on each prompt \textbf{once}. These embeddings and rewards are then reusable for any embedding-based reward model evaluation.

We have implemented such a preprocessing step, resulting in a dataset asset that includes $500$ responses for each test prompt. To provide a clearer understanding of the dataset prepared for embedding-based reward modeling, we provide the \grayboxtext{pseudo-code in the box} in page \pageref{algo} for illustration.


In such a use case, the computationally intensive step of embedding generation is completed during data preparation. Subsequently, the evaluation involves merely processing test tensors of shape $(2000, 500, 2048)$ through the reward model --- a task that typically concludes within a minute. This efficiency highlights the practicality of our embedding-based reward modeling framework, which significantly simplifies and accelerates the evaluation of reward models and improves its reliability.


% In our demonstrated use cases, the computational heavy step of embedding generation has been done in data preparation. And the evaluation forward pass only requires calling the reward model to process test tensors of the shape $(2000, 500, 2048)$, which typically can be finished within a minute. With the highlighted embedding-based reward modeling research framework, evaluating reward models becomes extremely easy and reliable.


% The ranking order consistency is more important than matching the accurate value 
% (hence predicting the accurate win rate among any two responses).




% \paragraph{Spearman's Ranking Correlation}


% \paragraph{ Best-of-N Evaluation}


\subsection{Scalable Inference-Time Optimization}
\label{sec:cheap_inference}
\begin{figure}[h!]
\vspace{-0.15cm}
    \includegraphics[width=1.0\linewidth]{figs/embedding_rm.png}
    \vspace{-0.6cm}
    \caption{\small \textit{The inputs of embedding-based reward models are by-products of language model generation.} Unlike conventional LLM-based reward models that require another LLM forward pass for inference time evaluations, embedding-based models alleviate the memory challenge and facilitate inference time optimization for LLM-free service providers. These providers, who rely on third-party LLM services via APIs rather than hosting large models locally, can efficiently perform inference time optimization using only embeddings.}  \vspace{-0.1cm}
    \label{fig:fast_inference}
% \vspace{-2cm}
\end{figure}

In this section, we elucidate an additional advantage of embedding-based reward models --- enhancing the inference-time optimization efficiency. With embedding-based reward models, language generation and evaluation require only a single LLM forward pass. Although this may appear to reduce computation time by less than half, it significantly lowers the computational burden in evaluation by shifting from hosting an LLM (reward model) to a much simpler and smaller model. This is particularly beneficial for API-based service providers who previously could not perform inference-time optimization due to the high computational demands of running LLM-based reward models locally. With embedding-based reward models, they are now able to efficiently evaluate the quality of generated content and potentially enhance user experience through inference-time optimization (e.g., prompting optimization and re-generation). The workflow of using embedding-based reward models in inference is visualized in Figure~\ref{fig:fast_inference}.










% \subsection{Isolating the Source of Gains: Discriminative and Generative Abilities}
% \paragraph{Generative Reward Models and Representation Learning}





\section{Case Study: Efficient Reproduction of Reward Model Ensemble Papers}
\label{sec:reproduce_demo}
% As a demonstrative example, in this section, we reproduce the research discoveries on alleviating reward models overoptimization using ensemble methods~\citep{coste2023reward,ahmed2024scalable} with the proposed embedding-based reward modeling framework. 

% To reproduce and verify the key discovery in those reward modeling papers --- that reward model ensemble alleviates reward model overoptimization. We train 10 lightgbm~\citep{ke2017lightgbm} models using the default hyper-parameter setups. We also consider the MLP-based implementation with $256$ hidden units. The reward model ensemble performance is evaluated by averaging the predictions of 10 reward models. 

% We experiment on a machine equipped with 128-core \texttt{Intel(R) Xeon(R) Platinum 8336C CPU @ 2.30GHz} CPUs. In our reproduction setups, we use 2 different models (MLP, lightgbm), work with 2 tasks, and build reward models for 3 language models (Gemma 2B, Gemma 7B, LLaMA3 8B), study $4$ different annotation quality setups, and $5$ annotation quantities ($[500, 10000,2000,5000,10000]$). In ensemble, we train $10$ models, and the experiments are repeated with $5$ independent runs --- summarizing all experiments, we have $12000$ different models to train and evaluate for reproduction. 

% Using the CPU server, training the $6000$ lightgbm reward models and evaluating them takes 4.9h. Training the $6000$ MLP reward models and evaluating them takes 17.3h. In total, those $12000$ experimental setups can be finished within 1 day.

In this section, we replicate the findings from prior research on mitigating overoptimization in reward models through ensemble methods, as discussed in \citep{coste2023reward, ahmed2024scalable}, using our proposed embedding-based reward modeling framework.


To validate the principal finding in those works that ensembles can alleviate reward model overoptimization, we train $10$ LightGBM models \citep{ke2017lightgbm} using default hyperparameter settings, alongside an MLP-based implementation with $256$ hidden units. We assess the performance of these ensemble reward models by averaging predictions across the $10$ models. Experiments are repeated with $5$ independent runs to draw statistically significant conclusions.

Our experiments are conducted on a machine equipped with a 128-core \texttt{Intel(R) Xeon(R) Platinum 8336C CPU @ 2.30GHz}. Our experimental setup encompasses $2$ different models (MLP and LightGBM), $2$ tasks, and build reward models for $3$ LLMs (Gemma 2B, Gemma 7B, LLaMA3 8B). We explore $4$ different annotation quality scenarios and $5$ levels of annotation quantity, ranging from $500$ to $10000$.
\begin{figure}[h!]
    \includegraphics[width=1.0\linewidth]{cheapensemble.png}
    \vspace{-0.6cm}
    \caption{\small Using embeddings as inputs in a lightweight reward model ensemble practice to mitigate reward overoptimization. Reproduction of prior findings across over $12000$ configurations can be completed in less than 1 day using CPU-only resources.}\vspace{-0.35cm}
    \label{fig:rm_ensemble_illu}
% \vspace{-2cm}
\end{figure}
\begin{figure*}[t!]
    \centering
    \includegraphics[width=1.0\linewidth]{figs/gemma2b_reward_esb.png}\vspace{-0.65cm}
    \caption{\small Reproduction of reward model ensemble papers using embedding-based reward models. Additional results using the Gemma 7B and LLaMA3 8B models are presented in Appendix~\ref{appdx:more_results}}
    \label{fig:reward_model_ensemble_gemma2b}\vspace{-0.35cm}
\end{figure*}

In total, we train and evaluate $12000$ models. Using the CPU server, training and evaluating the $6000$ LightGBM models takes 4.9 hours, while the $6000$ MLP models require 17.3 hours. In total, these $12000$ experimental configurations are completed within $1$ single CPU day.

Finally, unlike prior research, our investigation into reward model ensembles using embeddings as inputs clarifies that the observed enhancements stem from conservative modeling approaches, rather than from the scaling laws typical of LLM-based reward models \citep{gao2023scaling}. These distinctions are visually demonstrated in the case study illustrated in Figure~\ref{fig:rm_ensemble_illu}. 

Figure~\ref{fig:reward_model_ensemble_gemma2b} shows the results from our efficient reproduction. We observe significant performance improvements when using ensemble methods in reward modeling, thereby verifying the principal findings of \citet{coste2023reward} and \citet{ahmed2024scalable} within embedding-based reward model setups. Notably, the efficacy of the reward model ensemble diminishes as annotation quality improves (i.e., when the error rate is less than $5\%$), and we observe the LightGBM reward models generally get larger performance gains from the ensemble.





\section{Call for Contributions}
\label{sec:future_works}
\subsection{Contributing to Public Embedding Assets}
In this position paper, we have demonstrated the advantages of embedding-based reward models. We successfully reproduced the findings of a reward model ensemble study with $12,000$ experiment runs in just one day using only CPU resources, highlighting the efficiency of our approach. However, it's important to note that this workflow is feasible only when embeddings from LLM generations are available for both training and testing datasets.

In conventional LLM-based reward model research, LLM generations have not been regarded as critical public assets in reward model research, primarily because evaluating these generated contents requires nearly as much computational effort and hardware resources as producing them.

In contrast, our embedding-based reward model framework enables researchers with access only to CPU resources to participate in this field. This inclusivity relies on the availability of embedding assets, contributed by researchers with access to more powerful GPU resources.

For our studies, we utilized the \texttt{Anthropic-HH} dataset and $3$ different LLMs, enabling us to release all corresponding embeddings and their evaluations as public assets for future research. However, given the rapid advancements in general-purpose LLMs, this alone is not enough. \textbf{We encourage more contributions from the community to enrich these assets.}

Moreover, an added benefit of this approach is its environmental impact. By making these assets reusable, other researchers do not need to expend computational and electrical resources to regenerate training and testing samples. This not only accelerates research but also significantly reduces the environmental burden associated with the extensive use of computational resources in large-scale model training and evaluation.

% In this position paper, we have demonstrated the superiority of embedding-based reward models and the framework. While we have shown the efficiency of such a setup by reproducing the reward model ensemble paper with $12000$ experiment runs within $1$ day on a CPU-only machine, we would like to note that this workflow is only feasible when the embeddings of LLM generations --- on both training and testing datsaet --- are available. 

% In previous reward model research, LLM generations are not considered to be an important public asset as evaluating those generated contents takes as much as effort and requires almost the same amount of hardware resources as generating them. 

% Differently, in the embedding-based reward model research, the workflow make it possible for CPU-only users to participate the reward model research, yet this requires the contribution of the embedding assets from the GPU-rich researchers.

% In our research, we used the \texttt{Anthropic-HH} dataset, and $3$ LLMs, therefore, we are able to release all embeddings and their golden reward model evaluations as a public asset for future research. However, this is not sufficient in the fast-evolving research area especially given the rapicly improvements in general-purpose LLMs. We will need more researchers from the community to contribute to such type of assets. 



\subsection{Representation Learning: Searching for General Purpose Reward Embeddings}
Current language model embeddings are primarily designed and optimized for text generation. \textbf{While they can be repurposed as inputs for reward models, as demonstrated in this paper, there remains significant room for improvement.} Our experiments indicate that fine-tuning LLM-based reward models, though computationally expensive, can yield superior performance when provided with rich and clear annotation signals.

Given the advantages of embedding-based reward models outlined in this paper, developing better general-purpose reward embeddings represents a promising orthogonal direction for advancing reward model research. 

To link with another important research avenue of the generative reward modeling, where the token generation capabilities of LLMs are directly leveraged for value prediction~\citep{mahan2024generative,zhang2024generative} or used as a regularization mechanism in LLM-based reward model learning\citep{yang2024regularizing}. Their key insight is that \textit{generation ability can enhance performance in discriminative tasks}. In contrast, the question of how to leverage reward modeling information to learn general-purpose discriminative embeddings remains relatively underexplored. Notable exceptions include efforts to merge multiple preference datasets~\citep{dong2024rlhf}. However, \citet{sun2024off} found that combining offline generations with online annotations can be harmful to reward model training. Another related challenge in reward modeling is known to be the alignment tax~\citep{lin2024mitigating}, and how to balance multiple objectives~\citep{yang2024rewards,zhou2023beyond}, and ideal general reward embedding should be able to capture multiple aspects of the responses.

% Since the current language model embeddings are generated and optimized for the purpose of generation, while it is capable of being used as reward model inputs as discussed and demonstrated in this paper, there is undeniably a large space to improve over it. And we have observed from the experiments that fine-tuning the LLM-based reward models, though costly, can have higher performance potential when rich and clear annotation signals are available.

% Given the benefits of embeddding-based reward models discussed in this position paper, searching for better general-purpose reward embeddings can further contribute to this line of research from an orthogonal perspective to push the frontier of reward model research.

% Another important line of related research on reward modeling is the generative reward models~\citep{mahan2024generative,zhang2024generative} where the token generation ability of language models is directly utilized in reward prediction or acts as a regularizer in LLM-based reward model learning~\citep{yang2024regularizing}. In this line of research, the key insight is to leverage the generation ability of LLMs to improve their performance in discriminative tasks. As comparison, how to leverage rich reward modeling information in search of a general discriminative embeddings is a relatively underexplored direction, except attempts of merging multiple preference datasets~\citep{dong2024rlhf}, and it has been shown in \citet{sun2024off} that merging offline generations with online annotations may be harmful to reward model training.








\subsection{Flash Back of Classic Statistics}
Back in the early days of statistical natural language processing, circa the 1990s to early 2000s \citep[for even earlier history, we refer to][]{jones1994natural}, researchers had quite limited options for features even for simple classification tasks. Simple models (e.g., classification trees) were often accompanied by handcrafted, ad hoc features like bags of words, n-grams, and tf-idf \citep{chowdhury2010introduction}, which are seen as insufficient today. With neural networks, representation learning and model development occurred simultaneously; one can even argue that the success of deep models lies in the success of representation learning \citep{bengio2013representation}. Lightweight statistical learning methods possess good properties that are still relevant today. For instance, it is much less resource-intensive and more stable to fit boosted trees than DNNs. The theoretical properties of generalized linear models, some nonparametric regression, as well as tree models, are well understood for classification, preference learning, and for new tasks like experimental design and active learning. 

In future works, can we get the best of both worlds by combining powerful embeddings from an LLM, together with a solid understanding of classic methods to better advance reward modeling with a gray box approach? Can we develop theories building upon the knowledge of classic methods? --- for instance, under the linear assumption with embeddings, what theoretical properties can we establish, and how can we conduct active learning? There are vast research opportunities lying at the interface between statistics and embedding-based reward modeling.
%%%GPT4o version above --

% Back in the early time of statistical natural language processing circa 1990s to early 2000s \citep[for even earlier history we refer to][]{jones1994natural}, for even simple classification task researchers have quite limited options on features. Simple models (e.g., classification trees) were often accompanied with hand crafted, ad hoc features like bag of word, ngrams and tf-idf \citep{chowdhury2010introduction} that are seen as not sufficient today. With neural networks, representation learning and model development happened at the same time, one can even argue that the success of deep model lays in the success of representation learning \citep{bengio2013representation}. Light weighted statistical learning methods bear good properties that are still relevant today. For instance, it is much less resource heavy and stable to fit a boosted trees than DNNs. Theoretical properties of generalized linear models, some nonparametric regression as well as trees models are well understood for classification, preference learning and for new tasks like experimental design and active learning. Can we get the good of both world by combining powerful representations learned with an LLM, together with good understandings of classic methods to better advance reward modeling with a gray box approach? Can we develop theories building upon knowledge on classic methods? For instance, under linear assumption to embeddings, what theoretical property can we establish and how can we do active learning? There are vast research opputunities lies in the interface between statistics and reward modeling.  


\section{Alternative Views}
\label{sec:alternative_views}
\paragraph{Success of End to End Training.}

The remarkable success of deep learning is largely attributed to the end-to-end learning capability of deep neural networks~\citep{lecun2015deep,goodfellow2016deep}, which has proven effective across diverse domains, including image processing~\citep{krizhevsky2012imagenet,he2016deep}, natural language processing~\citep{vaswani2017attention,devlin2018bert}, tabular data analysis~\citep{arik2021tabnet}, and time series data~\citep{van2016wavenet,ismail2019deep,ding2020hierarchical}.
Representation learning~\citep{bengio2013representation} and pre-training methods~\citep{radford2018improving} are typically followed by post-training or fine-tuning procedures to adapt to downstream tasks or datasets~\citep{howard2018universal,raffel2020exploring,radford2021learning}. In the era of large language models, general-purpose pre-trained models have been extensively fine-tuned for a wide range of downstream applications~\citep{brown2020language}, including evaluation tasks such as reward modeling~\citep{perez2022red,ouyang2022training,chang2024survey,lin2023llm}.

\paragraph{Computational Costs are Decreasing Over Time}
As computational costs continue to decrease, future research on reward models may efficiently leverage LLMs or even more powerful foundation models. This could eliminate the need for embedding-based reward modeling approaches, further supporting the case for end-to-end learning.

From this perspective, one could reasonably argue that reward model learning should ultimately adopt an end-to-end approach. The positions proposed in this paper may only remain valid within a limited timeframe. Future advancements in methodology and hardware technology may render them obsolete.






% \clearpage
\bibliography{main}
\bibliographystyle{icml2025}


%%%%%%%%%%%%%%%%%%%%%%%%%%%%%%%%%%%%%%%%%%%%%%%%%%%%%%%%%%%%%%%%%%%%%%%%%%%%%%%
%%%%%%%%%%%%%%%%%%%%%%%%%%%%%%%%%%%%%%%%%%%%%%%%%%%%%%%%%%%%%%%%%%%%%%%%%%%%%%%
% APPENDIX
%%%%%%%%%%%%%%%%%%%%%%%%%%%%%%%%%%%%%%%%%%%%%%%%%%%%%%%%%%%%%%%%%%%%%%%%%%%%%%%
%%%%%%%%%%%%%%%%%%%%%%%%%%%%%%%%%%%%%%%%%%%%%%%%%%%%%%%%%%%%%%%%%%%%%%%%%%%%%%%
\newpage
\appendix
\onecolumn
\section{More Results}
\label{appdx:more_results}
\paragraph{Performance Comparison: Embedding-based reward models v.s. LLM-based reward models.}
In our main text, we presented the results with the Gemma 2B model when comparing the performance of different reward modeling approaches. We now provide the results using the Gemma 7B and LLaMA3 8B models as complementary empirical supports. The observations concluded in our main test still hold true on those experiment setups.
\begin{figure*}[h!]
    \centering
    \includegraphics[width=0.98\linewidth]{figs/gemma7b_reward.png}
    \caption{\small  Comparing performances of Embeddings-based RM  with LLM-based RMs using Gemma 7B.}
    \label{fig:performance_with_embeddings_gemma7b}
\end{figure*}


\begin{figure*}[h!]
    \centering
    \includegraphics[width=0.98\linewidth]{figs/llama38b_reward.png}
    \caption{\small  Comparing performances of Embeddings-based RM  with LLM-based RMs using LLaMA3 8B.}
    \label{fig:performance_with_embeddings_llama38b}
\end{figure*}

\newpage
\paragraph{Additional results reproducing reward model ensemble with embedding-based reward models.}
In our main text, we presented the results with the Gemma 2B model when reproducing reward model ensemble papers. We now provide complementary results using the Gemma 7B and LLaMA3 8B models in response generations. The observations concluded in our main test still hold true on those experiment setups.

\begin{figure*}[h!]
    \centering
    \includegraphics[width=1.0\linewidth]{figs/gemma7b_reward_esb.png}\vspace{-0.35cm}
    \caption{\small Reproduction of reward model ensemble papers using embedding-based reward models. Results on building reward models for Gemma 7B.}
    \label{fig:reward_model_ensemble_gemma7b}\vspace{-0.25cm}
\end{figure*}


\begin{figure*}[h!]
    \centering
    \includegraphics[width=1.0\linewidth]{figs/llama38b_reward_esb.png}\vspace{-0.35cm}
    \caption{\small Reproduction of reward model ensemble papers using embedding-based reward models. Results on building reward models for LLaMA3 8B.}
    \label{fig:reward_model_ensemble_llama38b}\vspace{-0.25cm}
\end{figure*}



\end{document}




% This document was modified from the file originally made available by
% Pat Langley and Andrea Danyluk for ICML-2K. This version was created
% by Iain Murray in 2018, and modified by Alexandre Bouchard in
% 2019 and 2021 and by Csaba Szepesvari, Gang Niu and Sivan Sabato in 2022.
% Modified again in 2023 and 2024 by Sivan Sabato and Jonathan Scarlett.
% Previous contributors include Dan Roy, Lise Getoor and Tobias
% Scheffer, which was slightly modified from the 2010 version by
% Thorsten Joachims & Johannes Fuernkranz, slightly modified from the
% 2009 version by Kiri Wagstaff and Sam Roweis's 2008 version, which is
% slightly modified from Prasad Tadepalli's 2007 version which is a
% lightly changed version of the previous year's version by Andrew
% Moore, which was in turn edited from those of Kristian Kersting and
% Codrina Lauth. Alex Smola contributed to the algorithmic style files.
