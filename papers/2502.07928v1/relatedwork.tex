\section{Related Work}
\label{sec:rw}
\subsection{Functional Programming Languages and Haskell Refactoring}

Functional programming languages like Haskell present challenges due to their emphasis on immutability, lazy evaluation, and higher-order functions~\cite{bragilevsky2021haskell}. These
challenges complicate the refactoring process, especially when changes need to preserve program semantics~\cite{thompson2013refactoring}. Most research on refactoring functional programming based applications primarily focuses on traditional methods. Mens and Tourwé~\cite{mens2004survey} conducted a comprehensive survey highlighting refactoring patterns applicable across programming paradigms, including managing side effects and optimizing type safety. Further explored the challenges of refactoring functional programs~\cite{abdallah2011dynamic}, proposing techniques tailored for functional codebases to maintain semantic integrity while enhancing readability and performance.

Despite efforts, there is still limited research on automated refactoring in Haskell and most methods depending on rule-based systems or requiring manual intervention. These limitations highlight the need for automated systems that can understand dependencies within Haskell code and suggest effective refactoring solutions. Our research seeks to bridge this gap by introducing a multi-agent system capable of performing distributed and automated refactoring tasks within a Haskell codebase.


\subsection{Multi-Agent Systems in Software Refactoring}

Multi-Agent Systems have evolved since their inception in the mid-1990s, when foundational ideas emphasized distributed collaboration and task specialization using rule-based approaches~\cite{wadler1992essence}. Between 2010-2020, the multi-agent system allows for multiple agents to specialize in different tasks and collaborate to achieve a common goal, which makes it useful for complex codebases requiring iterative improvement~\cite{abdallah2011dynamic}. AyshwaryaLakshmi et al.~\cite{ayshwaryalakshmi2013agent} explored the use of multi-agent systems in distributed code refactoring, demonstrating how autonomous agents can collaboratively handle large-scale refactoring tasks with improved efficiency and accuracy. After 2021, unlike earlier systems, LLMs empower agents to interpret complex domains, such as Haskell's functional programming features, and automate tasks like refactoring and debugging with unprecedented scalability~\cite{chen2021evaluating}. This combination bridges theoretical concepts with practical marking a leap in multi-agent system capabilities and redefining their role in modern software engineering.

% Multi-agent systems have shown promise in distributed and autonomous decision-making environments. In software engineering, multi-agent system has been applied to tasks requiring parallelized efforts, such as code analysis, refactoring, and maintenance~\cite{wooldridge1995intelligent}. The multi-agent system allows for multiple agents to specialize in different tasks and collaborate to achieve a common goal, which makes it useful for complex codebases requiring iterative improvement~\cite{abdallah2011dynamic}. AyshwaryaLakshmi et al.~\cite{ayshwaryalakshmi2013agent} explored the use of multi-agent systems in distributed code refactoring, demonstrating how autonomous agents can collaboratively handle large-scale refactoring tasks with improved efficiency and accuracy. By distributing refactoring responsibilities, multi-agent system enables greater flexibility and modularity in tackling complex code structures.

For Haskell refactoring, multi-agent system can be advantageous, as agents can be assigned tasks like context analysis, refactoring suggestion, and validation \cite{dos2015autorefactoring}. Each agent’s specialization allows for more effective handling of Haskell’s unique functional constructs. Our research uses multi-agent system principles to distribute Haskell refactoring tasks across multiple agents~\cite{guo2024large}, each enhanced by LLM capabilities, which collectively provide a scalable approach for functional programming language maintenance.

\subsection{LLMs for Code Generation and Refactoring}

The rise of large language models (LLMs) has revolutionized automated code generation and refactoring such as OpenAI’s GPT series and Codex. These models, trained on vast datasets of code, have demonstrated proficiency in tasks ranging from code completion to error detection and refactoring suggestions~\cite{chen2021evaluating}. Svyatkovskiy et al.~\cite{svyatkovskiy2020intellicode} introduced IntelliCode Compose reducing the time developers spend on routine coding tasks which is a transformer-based system that uses LLMs to assist in code generation. Recent work~\cite{white2024chatgpt}\cite{hou2024large} has shown that LLMs can not only generate syntactically correct code but also adapt to specific programming paradigms, making them valuable assets for software refactoring.

However, while LLMs are effective at generating and analyzing code, they often lack the domain-specific insights required for functional programming languages. Additionally, in distributed refactoring tasks, LLMs alone may be insufficient due to limitations in handling complex, modular tasks autonomously. Our work addresses these limitations by integrating LLMs into a multi-agent system, where the LLMs augment the capabilities of each agent in tasks like code analysis, refactoring suggestion, and debugging. This integration allows for a more contextually aware, automated refactoring process tailored to the complexities of Haskell.

\subsection{Combining Multi-Agent Systems and LLMs for Haskell Refactoring}

The LLMs based multi-agent systems for code refactoring is an approach that has not been extensively explored in the literature. Recently proposed~\cite{baumgartner2024ai}\cite{xi2023rise} a multi-agent learning system for automatic code refactoring, which demonstrated improvements in handling distributed code modifications. However, their approach was primarily focused on imperative programming languages and did not explore applications in functional programming. Similarly, discussed~\cite{ayshwaryalakshmi2013agent}\cite{huang2024levels} a multi-agent approach to software maintenance but their work did not consider the challenges associated with functional languages like Haskell.

Our research expands on these prior works by introducing a multi-agent system~\cite{hua2023war} specifically designed for functional programming refactoring~\cite{dos2015autorefactoring}, enhanced by the contextual analysis and code generation capabilities of LLMs. By assigning each agent a specialized task and utilizing LLMs for deeper language comprehension, our approach aims to streamline Haskell refactoring ~\cite{thompson2013refactoring} in a scalable, autonomous manner. This combination of multi-agent systems and LLMs provides a solution to the challenges of functional programming language maintenance and extends the capabilities of both multi-agent systems and LLMs in software engineering~\cite{cheng2024exploring}.