
% \begin{figure*}
%     \centering
% \includegraphics[width=\textwidth]{Figures/Experiment/fig78merge_slip.pdf}
%     \caption{The experimental results of friction coefficient estimation. From top to bottom, the figures show the estimated friction coefficient, estimated foot velocity in the normal direction with the rejection score, and estimated tangential foot velocity with the confidence score.}
%     \label{fig:online_estimation}
% \end{figure*}

\begin{figure*}
    \centering
\includegraphics[width=\textwidth]{Figures/Experiment/Experiment_1.pdf}
    \caption{The experimental results of friction coefficient identification show the effects of proposed smoothed gradients and rejection methods. Without smoothed gradients, non-informative gradients can impede friction coefficient identification. The rejection method allows for consistent friction coefficient identification, especially when the legged robot traverses nonslippery terrain. The purple area represents the slip states on the slippery terrains where the norm of tangential estimated foot velocity from the state estimator~\cite{Joonha2023TRO} exceeds 0.4~\si{\meter/\second}.}
    \label{fig:online_estimation}
\end{figure*}

% \begin{figure}
%     \centering
%     \subfloat[]{
%         \includegraphics[width=1.0\columnwidth]{Figures/Experiment/fig7_contact_vel_1.pdf}
%         %\label{fig:a}
%         }
%         \hfill
%     \subfloat[]{
%         \includegraphics[width=1.0\columnwidth]{Figures/Experiment/fig7_contact_vel_2.pdf}
%         %\label{fig:b}
%         }
%     \caption{\textcolor{blue}{(a) contact velocity in the normal direction. (b) tangential contact velocity. As depicted by a green dotted circle, the rejection method can effectively prevent undesired increases in the confidence score, especially on nonslippery terrain. Conversely, described by a red dotted circle, rejection scores do not significantly impede the increases in confidence scores when the robot slips on the slippery terrain.}}
%     \label{fig:wrong_cof_in_contact_model}
% \end{figure}

\begin{figure}
   \centering
\includegraphics[width=1.0\columnwidth]{Figures/Experiment/Experiment_2.pdf}
   \caption{As depicted by a green dotted circle, the rejection method can effectively prevent undesired increases in the confidence score, especially on nonslippery terrain. Conversely, described by a red dotted circle, rejection scores do not significantly impede the increases in confidence scores when the robot slips on the slippery terrain.}
   \label{fig:contact_vel}
\end{figure}

\begin{figure}
    \centering
    \includegraphics[width=1.0\columnwidth]{Figures/Experiment/Experiment_3.pdf}
    \caption{Comparison of the average loss between the method using the nonsmooth model and the proposed method. The proposed method achieves a lower average loss than the method using the nonsmooth model.}
    \label{figure:loss_compare}
\end{figure}



\section{Experimental Results} \label{Sec:Experiment}
% The robot starts from the nonslippery terrain made of rubbers and moves to the slippery terrain. Then, the robot moves along the body coordinates x-axis inside the slippery terrain.
% As shown in Fig.~\ref{fig:online_estimation}, the proposed framework performs the friction coefficient estimation by solving the issue of nullified gradients. Without the proposed smoothing method, the updates for the friction coefficient are hindered by nullified gradients. 


% 이번 섹션에서는 실제 로봇 KAIST 하운드에를 이용한 제안한 마찰 계수 추정 알고리즘의 실험 결과를 소개한다. 또한, 우리는 smoothed gradients of contact impulses on the friction coefficient과 rejection of elastic contact 그리고 confidence score의 효과에 대하여 명백히 밝힌다. 이를 위해, 
% Please add the following required packages to your document preamble:
% \usepackage{graphicx}
% Please add the following required packages to your document preamble:
% \usepackage{graphicx}
\begin{table}[]
\caption{Parameters Used in Experiments}
\centering
\resizebox{\columnwidth}{!}{%
\begin{tabular}{|c|c|c|c|c|c|c|c|}
\hline
\hline
\textbf{Parameter} & \text{$\alpha_\mathrm{rej}$} & \text{$\gamma_\mathrm{rej}$} & \text{$\Delta{t}_\mathrm{buffer}$} & \text{$\Delta{t}_\mathrm{bound}$} & \text{$\sigma_\mathrm{slip}$} & \text{$\sigma_{q_\mathrm{base}}$} & \text{$\sigma_{q_\mathrm{jnt}}$} \\ \hline
\textbf{Value}     &   5.0    &     0.4    &             0.01~\si{\second}       &      0.1~\si{\second}    & 30 & 1e-4 & 20      \\
\hline \hline
\textbf{Parameter} & \text{$\alpha_\mathrm{conf}$} & \text{$\gamma_\mathrm{conf}$} & \text{$\epsilon$} & \text{$H$} & \text{$\rho_\mathrm{t}$} & \text{$\sigma_{\dot{q}_\mathrm{base}}$} & \text{$\sigma_{\dot{q}_\mathrm{jnt}}$}\\ \hline
\textbf{Value}     &    3.0       &       0.58     &     0.1      &      50       &      0.05      & 1e-4 & 1    \\
\hline \hline
\end{tabular}%
}
    \label{table:parameters}
\end{table}
This section introduces the experimental results of the proposed friction coefficient identification framework using the quadrupedal robot, KAIST HOUND~\cite{shin2022hound}. Additionally, we explain the effects of proposed analytic smoothed gradients of contact impulses with respect to the friction coefficient and the proposed rejection method. 
% \subsection{Experimental Setup}
% \begin{figure}
%     \includegraphics[width=1.0\columnwidth]{Figures/Experiment/fig13.pdf}
%     \caption{An environment setup for experiments. The slippery terrain is made of acrylic flat boards with boric acid powder.}
%     \label{fig:experimental_setup}
% \end{figure}
\subsection{Experimental Setup}

The proposed framework is based on the confidence score-based online system identification framework~\cite{chen2022real} and employs the proposed smoothed gradient and rejection method. To calculate the proposed smoothed gradients, we empirically set the smoothing parameter, $\rho_\mathrm{t}$, as 0.05. As~\cite{kim2023contactimplicit}, if the smoothing parameter is either too large or too small, the smoothing method may not be effective in achieving better local optima compared to the nonsmooth method. In implementing the proposed framework, we set $\mu_\mathrm{min}$ and $\mu_\mathrm{max}$ as 0.01 and 1.0, respectively.

%Based on the frictional contact dynamics, this framework can be conducted to estimate the low friction coefficient. However, s
Since estimating a high friction coefficient through contact dynamics is challenging in the absence of foot slippage~\cite{focchi2018slip,jenelten2019legged}, this work employs the reset method for the friction coefficient as~\cite{ jenelten2019legged}. This method resets the estimated friction coefficient to the default value $\mu_\mathrm{def}$ of 0.8 when stable contacts are maintained for 0.5~\si{\second}. In this work, the estimated friction coefficient is restored to the default value if the confidence score $\eta$ does not exceed the threshold $\gamma_\mathrm{conf}$ for 0.5~\si{\second}, indicating that the robot does not have a high tangential contact velocity for this duration.

The experiments are conducted in two different terrains: a nonslippery terrain and a slippery terrain. The slippery terrain is made of acrylic flat boards with boric acid powder. The robot initially starts on the nonslippery terrain where the experimentally measured friction coefficient is 1.0, then moves to the slippery terrain where the experimentally measured friction coefficient is 0.19. Subsequently, the robot moves between slippery and nonslippery terrains alternately. The measured friction coefficient on slippery terrain was obtained by measuring the horizontal force with a spring scale when the standing robot began to slip, considering its weight~\cite{shin2022hound}.

We solved the contact dynamics only for the states where contact is detected by the state estimator and implemented RaiSim’s algorithm~\cite{raisim} for this purpose.

For state estimation of the legged robot, we employ the method proposed by \cite{Joonha2023TRO}, which operates at 200 Hz within our framework. The contact velocity, contact states, and slip states are estimated in the state estimator. We determine the slip states when the norm of tangential contact velocity, estimated by the state estimator exceeds 0.4~\si{\meter/\second}. As a robot's controller, a nonlinear model predictive controller in \cite{hong202realtime} is utilized with functioning at 80 Hz. The boundary of computation time ${\Delta{t}_\mathrm{bound}}$ for the proposed framework is set at 10 Hz. The detailed parameters for the proposed framework are given in Table.~\ref{table:parameters}. A single onboard computer with an Intel(R) Core(TM) i7-11700T CPU, capable of reaching up to 1.6 GHz, is utilized to implement the proposed framework.






\subsection{Estimation Results}


\begin{figure}
    \centering
    \includegraphics[width=1.0\columnwidth]{Figures/Experiment/Experiment_4.pdf}
    \caption{Result of the average loss for the experiment according to the smoothing parameter $\rho_\mathrm{t}$.}
    \label{figure:smoothing_compare}
\end{figure}

To validate the proposed methods, we compared the results of friction coefficient identification with and without the proposed gradient and rejection method, as shown in Fig.~\ref{fig:online_estimation}. In the experiment, we set the default estimated friction coefficient to 0.8 to illustrate a scenario where the robot, assuming a high friction coefficient for non-slippery terrain, slips on slippery surfaces. Note that the parameter update is conducted when the confidence score exceeds the threshold $\gamma_\mathrm{conf}$~\cite{chen2022real}.
%This approach enables the friction coefficient identification framework to update the coefficient only when the tangential contact velocity increases, which can occur due to foot slippages or high contact velocities following contact initiations. 
%If the confidence score does not exceed $\gamma_{conf}$ for 0.5~\si{\second}, the estimated friction coefficient $\hat{\mu}$ is reset to the default value $\mu_{def}$.

% With the boundary time of 10 Hz, the mean and maximum computation time for the proposed framework are 0.0285~\si{\second} and 0.0867~\si{\second}, respectively. 
%To validate our proposed method, we compared the friction coefficient estimation performance when using the proposed gradient for optimization problem~\eqref{osi_opt} with the performance when not using the proposed gradient. 
% The first figure at the top of Fig.~\ref{fig:online_estimation} illustrates the parameter estimation results, including the estimated friction coefficient, $\hat{\mu}$.
%We compared the parameter estimation performance with the performance of a framework without using the proposed methods to verify the proposed methods.



In the left bottom figure of Fig.~\ref{fig:online_estimation}, we observed that using nonsmooth gradients can impede friction coefficient identification, even if the robot slips on slippery terrains. In contrast, employing the proposed smoothing method allows for fast and consistent identification.
% updating the coefficient from 0.8 to 0.393 in a single step within 0.1~\si{\second}, and from 0.8 to 0.281 within 0.3~\si{\second}.

Moreover, the right bottom figure in Fig.~\ref{fig:online_estimation} shows that the estimated friction coefficient becomes more consistent, especially on nonslippery terrain, when the confidence score-based update is used with the rejection method compared to without it. Using both methods, the estimated friction coefficient on nonslippery terrain can be maintained close to the default value for such terrain, without undesired updates.

%However, using the confidence score-based update with the rejection method stabilizes the online friction coefficient identification process, especially when the robot is on nonslippery terrain.


%However, the proposed smoothing method solves the issue of non-informative gradients and updates the coefficient from 0.8 to 0.393 with a single update within 0.1~\si{\second} and from 0.8 to 0.281 within 0.3~\si{\second}.

The detailed effects of the proposed gradients and rejection methods will be discussed below.

\subsection{The Effects of Analytic Smoothed Contact Gradients}
%\begin{figure}
%    \centering
%    \includegraphics[width=1.0\columnwidth]{Figures/Experiment/fig11.pdf}
%    \caption{\textcolor{blue}{The proposed gradient can mitigate the issue of lack of informative gradient, allowing for improved friction coefficient identification under various initial conditions.}}
%    \label{figure:different_initial_condition}
%\end{figure}
\begin{figure}
    \centering
    \subfloat[]{
        \includegraphics[width=0.45\columnwidth]{Figures/Experiment/Experiment_5_a.pdf}
        \label{figure:different_initial_condition_prop}
        }
        \hfill
    \subfloat[]{
        \includegraphics[width=0.45\columnwidth]{Figures/Experiment/Experiment_5_b.pdf}
        \label{figure:different_initial_condition_nonsm}
        }
    \caption{Comparison of friction coefficient identification under various initial estimates. The purple area represents the slip states on slippery terrains where the norm of tangential contact velocity exceeds 0.4~\si{\meter/\second}. \protect\subref{figure:different_initial_condition_prop} The proposed smoothing is applied. \protect\subref{figure:different_initial_condition_nonsm} The gradient from the nonsmooth model is used~\cite{chen2022real}.}
    \label{figure:different_initial_condition}
\end{figure}
% Please add the following required packages to your document preamble:
% \usepackage{multirow}

% Please add the following required packages to your document preamble:
% \usepackage{multirow}
% \begin{table}[]
% \caption{The results comparing the performance using the proposed smoothing method with the baselines after seven experiments}
% \begin{tabular}{|c|c|cl|cc|}
% \hline
% \multirow{2}{*}{\textbf{Method}}                                            & \multirow{2}{*}{\textbf{\begin{tabular}[c]{@{}c@{}}Average\\ Loss\end{tabular}}} & \multicolumn{2}{c|}{\textbf{Time (s)}}                                   & \multicolumn{2}{c|}{\textbf{\begin{tabular}[c]{@{}c@{}}Estimated \\ Friction Coefficient\end{tabular}}}     \\ \cline{3-6} 
%                                                                             &                                                                                  & \multicolumn{1}{c|}{\textbf{Mean}}   & \multicolumn{1}{c|}{\textbf{Max}} & \multicolumn{1}{c|}{\textbf{Mean}} & \textbf{\begin{tabular}[c]{@{}c@{}}Standard \\ Deviation\end{tabular}} \\ \hline
% \textbf{Proposed}                                                           & \textbf{1.1542}                                                                  & \multicolumn{1}{c|}{0.0247}          & 0.0867                            & \multicolumn{1}{c|}{0.2724}        & 0.0371                                                                 \\ \hline
% \textbf{Exact}                                                              & 6.8542                                                                           & \multicolumn{1}{c|}{\textbf{0.0111}} & \textbf{0.0514}                   & \multicolumn{1}{c|}{0.7373}        & 0.1660                                                                 \\ \hline
% \textbf{\begin{tabular}[c]{@{}c@{}}1st-order\\ Randomized\end{tabular}}   & 3.6515                                                                           & \multicolumn{1}{c|}{0.5679}          & 1.8760                            & \multicolumn{1}{c|}{0.2810}        & 0.0352                                                                 \\ \hline
% \textbf{\begin{tabular}[c]{@{}c@{}}0th-order \\ Randomized\end{tabular}} & 3.9282                                                                           & \multicolumn{1}{c|}{0.6997}          & 2.5681                            & \multicolumn{1}{c|}{0.2737}        & 0.0300                                                                 \\ \hline
% \end{tabular}
% \label{table:total_compare}
% \end{table} 


% \begin{table}[]
% % \caption{Comparison of friction coefficient estimation using proposed smoothing with other smoothing method in terms of cost, time and estimated friction coefficient}
% % \caption{Estimation Results obtained from Experiments of 7 Trials to compare the performance using proposed smoothing with that using baseline.}
% \caption{The results comparing the performance using the proposed smoothing method with the baselines after seven experiments}
% \begin{tabular}{|c|c|cl|cc|}
% \hline
% \multirow{2}{*}{\textbf{Method}}                                         & \multirow{2}{*}{\textbf{\begin{tabular}[c]{@{}c@{}}Average\\ Loss\end{tabular}}} & \multicolumn{2}{c|}{\textbf{Time (s)}}                                    & \multicolumn{2}{c|}{\textbf{\begin{tabular}[c]{@{}c@{}}Estimated \\
%  Friction Coefficient\end{tabular}}}     \\ \cline{3-6} 
%                                                                          &                                                                                  & \multicolumn{1}{c|}{\textbf{Mean}}    & \multicolumn{1}{c|}{\textbf{Max}} & \multicolumn{1}{c|}{\textbf{Mean}} & \textbf{\begin{tabular}[c]{@{}c@{}}Standard \\ Deviation\end{tabular}} \\ \hline
% \textbf{Proposed}                                                        & \textbf{1.247}                                                                  & \multicolumn{1}{c|}{0.03033}          & 0.0811                            & \multicolumn{1}{c|}{0.288}         & 0.032                                                                  \\ \hline
% \textbf{Exact}                                                           & 6.854                                                                           & \multicolumn{1}{c|}{\textbf{0.01371}} & \textbf{0.0655}                   & \multicolumn{1}{c|}{0.753}         & 0.125                                                                  \\ \hline
% \textbf{\begin{tabular}[c]{@{}c@{}}First-order\\ Randomized\end{tabular}}  & 3.6515                                                                           & \multicolumn{1}{c|}{1.3168}           & 3.2193                            & \multicolumn{1}{c|}{0.276}         & 0.028                                                                  \\ \hline
% \textbf{\begin{tabular}[c]{@{}c@{}}Zeroth-order \\ Randomized\end{tabular}} & 3.9282                                                                            & \multicolumn{1}{c|}{1.3741}           & 5.9274                            & \multicolumn{1}{c|}{0.277}         & 0.043                                                                  \\ \hline
% \end{tabular}
% \label{table:total_compare}
% \end{table}

% \begin{table}[]    
% \begin{tabular}{|c|c|c|cc|}
% \hline
% \multirow{2}{*}{\textbf{Method}}                                            & \multirow{2}{*}{\textbf{Cost}} & \multirow{2}{*}{\textbf{Time (s)}} & \multicolumn{2}{c|}{\textbf{Estimated Friction Coefficient}}     \\ \cline{4-5} 
%                                                                             &                                &                                    & \multicolumn{1}{c|}{\textbf{Mean}} & \textbf{Standard Deviation} \\ \hline
% \textbf{Proposed}                                                           & \textbf{1.867}                 & 0.0388                             & \multicolumn{1}{c|}{0.293}         & 0.0620                      \\ \hline
% \textbf{Non-smoothed}                                                              & 10.896                         & \textbf{0.383}                     & \multicolumn{1}{c|}{0.679}         & 0.18490                     \\ \hline
% \textbf{\begin{tabular}[c]{@{}c@{}}First-order\\ Randomized\end{tabular}}   & 4.664                          & 1.887                              & \multicolumn{1}{c|}{0.296}         & 0.03466                     \\ \hline
% \textbf{\begin{tabular}[c]{@{}c@{}}Zeroth-order \\ Randomized\end{tabular}} & 6.412                          & 1.880                              & \multicolumn{1}{c|}{0.277}         & 0.0427                      \\ \hline
% \end{tabular}\label{figure:total_compare}
% \end{table}
% \begin{figure}
%     \centering
%     \includegraphics[width=1.0\columnwidth]{Figures/Experiment/fig14.pdf}
%     \caption{Relation between average loss value and smoothing parameter for an optimization problem. The smoothing parameter allows for finding a better searching direction in optimization problems than without using the smoothing parameter.}
%     \label{fig:loss_smoothing}
% \end{figure}

In this session, we will examine the advantages of the proposed smoothing method in friction coefficient identification. As shown in Fig.~\ref{fig:online_estimation}, when the robot slips on slippery terrain, the proposed smoothing method enables parameter updates towards a low friction coefficient, in contrast to the case of the nonsmooth model. For the slipping case, we compare the average loss of the nonsmooth model with that of the smoothing method in Fig.~\ref{figure:loss_compare}. In the figure, we observed that using the proposed smoothing method can lead to convergence at better local optima, achieving a lower loss value. Specifically, Fig.~\ref{figure:smoothing_compare} shows the average loss during the experiments shown in Fig.~\ref{fig:online_estimation} according to the smoothing parameter. We observed that when the smoothing parameter $\rho_\mathrm{t}$ is excessively increased or decreased, the effect for the convergence towards a lower loss may be reduced, as~\cite{kim2023contactimplicit}.
% We observed that, with the smoothed gradients, the estimated friction coefficient updates are smoothly executed within the same initial range.

Moreover, we conducted friction coefficient identification with various initial conditions in 0.05 units from 0.05 to 1.0, as shown in Fig.~\ref{figure:different_initial_condition}.
In the experiment, we used the same experimental data as that for Fig.~\ref{fig:online_estimation}. We compared the performance of friction coefficient identification between the proposed model and the nonsmooth model on slippery terrain. As shown in Fig.~\ref{figure:different_initial_condition_nonsm}, 
when employing nonsmooth gradients, the lack of informative gradients can lead to the failure to identify the lower friction coefficient. We observed that the issue often occurs as the gap between the estimated friction coefficient and the actual one is large. In contrast, as Fig.~\ref{figure:different_initial_condition_prop}, our proposed smoothing method solves the failure issue of parameter identification, even under various initial conditions.
%\textcolor{blue}{We observed that the proposed gradient allows for improved friction coefficient identification under various initial conditions.}  



Considering the results, we observed that the proposed smoothing method provides advantages for friction coefficient identification under various initial conditions, even when a high initial friction coefficient leads to non-informative gradients. These advantages can be utilized in various model-based frameworks. For instance, model-based controllers for legged robots often employ a user-defined friction coefficient to compute control inputs based on the Coulomb friction cone constraint. The friction coefficient is typically determined by heuristic tuning for their tasks~\cite{jenelten2019legged,hong202realtime}. A high friction coefficient can be selected to optimize control inputs, leveraging more tangential ground reaction forces. However, using a high friction coefficient on slippery terrain may cause the robot to slip, as the control inputs are computed based on a high friction coefficient. Consequently, there is a need for real-time friction coefficient identification that performs fast and consistently on slippery terrain, even with a high initial friction coefficient. The proposed framework can identify the friction coefficient under various initials, handling non-informative gradients.

%\textcolor{blue}{For online system identification, however, the problem arises when the dynamics model does not predict slipping, even if it actually occurs, due to gaps in modeling the friction coefficient. In this case, the contact impulse is not attached to the friction cone and becomes independent of the friction coefficient. This independence also extends to the contact dynamics of the system~\cite{raisim,werling2021fast}. Consequently, using nonsmoothing methods, the gradients can become non-informative, hindering online identification of the friction coefficient.}

%\textcolor{blue}{As shown in Fig.~\ref{figure:different_initial_condition}, when using system identification of the nonsmoothing gradients, the issue of non-informative gradients become predominant when the predicted friction coefficient values are larger than the actual values. Therefore, the friction coefficient identification that works across various initial conditions becomes essential. Unlike gradients without smoothing, our smoothed gradients allow for consistent friction coefficient identification across various initial conditions.}
%We observe that using proposed smoothing significantly reduces non-informative gradients and allows for uninterrupted parameter updates within various initial conditions, compared to nonsmoothed gradients.
%Conversely, when the initial friction coefficient is lower than 0.70, the estimated friction coefficient and convergence rates when using the nonsmoothed gradient are comparable to those using the proposed smoothed gradient, and the non-informative gradient issue becomes negligible. 

% TODO 
% Moreover, Fig.~\ref{fig:loss_smoothing} shows the average loss value computed using all the loss values obtained from the experiments shown in Fig.~\ref{fig:online_estimation}. As illustrated in Fig.~\ref{fig:loss_smoothing}, the average loss value was decreased when the proposed smoothed gradient was applied, allowing stable and smooth parameter updates to decrease the loss function since the nullified gradient issue did not occur. However, the non-smoothed analytic gradient about the friction coefficient led to the zero gradient issue and a higher loss value than the smoothed gradient under certain initialized conditions.

% If the estimated friction coefficient is larger than the friction coefficient in the actual terrain, the modeling gap in contact dynamics is increased when the robot slips.
% As seen in Figure~\ref{figure:different_initial_condition}, 


\subsection{Comparison with Randomized Smoothing Methods}
\begin{figure}
    \centering
    \includegraphics[width=1.0\columnwidth]{Figures/Experiment/Experiment_6.pdf}
    \caption{Compared to the baselines, the proposed methods can achieve fast and consistent friction coefficient identification in real-time.}
    \label{figure:compare_randomize}
\end{figure}
% In this section, we evaluate the effectiveness of the proposed gradients in this paper by comparing the results obtained using proposed smoothed gradients, randomized smoothing, and nonsmoothed gradients, in Fig.~\ref{figure:compare_randomize} and Table~\ref{table:total_compare}. The results shown in Fig.~\ref{figure:compare_randomize} are obtained from the same data as shown in Fig~\ref{fig:online_estimation}. The randomized smoothing utilizes 30 samples with parallel computing to obtain the stochastic gradient. We observe that the estimates using the proposed smoothing method are comparable to those using the randomized smoothing. Moreover, like randomized smoothing, as discussed in~\cite{Pang2023TRO,le2024leveraging}, it is observed that the proposed gradient mitigates the issue of non-informative gradients.


In this section, we compare the performance of friction coefficient identification using the proposed gradients with baseline methods. For the baselines, we adopt the online system identification using nonsmooth gradients~\cite{chen2022real} and using randomized smoothing methods~\cite{le2024leveraging, Pang2023TRO}: specifically first-order and zeroth-order randomized smoothing methods. The randomized smoothing methods utilize 50 samples with parallel computing to obtain stochastic gradients. We conducted seven experiments where the robot slipped on slippery terrains, with initial estimates of 0.8. 

The results are summarized in Fig.~\ref{figure:compare_randomize}, which presents histograms of the estimated friction coefficient, computation time for solving the optimization problem for~\eqref{osi_opt}, and average loss. We observe that the proposed smoothed gradient results in lower computation times than other randomized smoothing methods. As noted in~\cite{Pang2023TRO}, while randomized smoothing methods can address the lack of informative gradients, they require longer computation times due to sampling. Furthermore, it is observed that the mean and standard deviation of estimates without the smoothing method are higher than those using smoothing methods. This can be attributed to the lack of informative gradients, which causes the gradient-based optimization strategy to fail in friction coefficient identification.
%In cases without smoothing, the estimates often remain unchanged or are delayed due to these bad local minima.}
 % It is observed that estimates often remain unchanged from the current values even with increased confidence scores, resulting in a higher mean and standard deviation.
% 스탠다드 배리에이션이 다른 방법들보다 더 높았다. 평탄화하지 않은 기울기를 사용하면 유용그래디언트 부족의 결과를 낳기 때문에, 최적화는 bad local minima에 갇히게 되었다. 

\subsection{The Effects of Data Rejection Method}

% \begin{figure}
%     \centering
% \includegraphics[width=1.0 \columnwidth]{Figures/Experiment/fig78merge_nonslip.pdf}
%     \caption{The results of an additional experiment in which the legged robot traverses nonslippery terrain. The rejection algorithm reduces the drift in estimating the friction coefficient by utilizing a rejection score and stabilizes the parameter updates.}
%     \label{fig:nonslippery_estimation}
% \end{figure}
In this section, we describe the benefits of data rejection methods by comparing updates based on confidence scores with and without rejection methods. In the right bottom figure of Fig.~\ref{fig:online_estimation}, friction coefficient identification with the rejection method is more consistent than without it, especially on nonslippery terrain.

As illustrated in the bottom right figure of Fig.~\ref{fig:contact_vel}, not using the rejection method can lead to an increased confidence score, even on nonslippery terrain. If the confidence score increases on nonslippery terrain, the parameter updates can be conducted using non-informative observations, leading to undesired and inconsistent friction coefficient identification.

However, with the proposed rejection method, the confidence score on nonslippery terrain does not increase as much as it does without the method, allowing for consistent performance in friction coefficient identification. Furthermore, the rejection method does not significantly impede increases in the confidence score when the robot slips. As shown in the bottom-left of Fig.~\ref{fig:contact_vel}, when the robot slips on slippery terrain, the confidence score with the rejection method is comparable to one without the method. The upper figures of Fig.~\ref{fig:contact_vel} show that the data with high contact velocity following contact initiations can be excluded from parameter identification.


%\textcolor{blue}{Consequently, it is observed that combining the confidence score-based updates with the proposed rejection method stabilizes parameter updates, without significantly impeding the increase in confidence scores when the robot slips on slippery terrains.}
% Additionally, the rejection method does not significantly impede the updates for the estimated friction coefficient when the robot slips on slippery terrain. In such cases, the increase in the confidence score with the rejection method is comparable to that without the method.

% Furthermore, it is observed that, for legged robots,  incorporating our proposed rejection method into existing confidence score-based online system identification reduces the drift in friction coefficient estimates and allows for stable friction coefficient estimation. 
% Besides the above experiments, we conducted an additional experiment where the robot only navigated on only nonslippery terrain, as shown in Fig.~\ref{fig:nonslippery_estimation}. Even if the legged robot is traveling on the nonslippery terrain, it is observed that the norm of the tangential contact velocity increased up to 0.842~\si{\meter/\second} at the beginning of contacts, about 19.3~\si{\second}, and the normal contact velocity highly vibrated and increased. As shown in the top figure of Fig.~\ref{fig:nonslippery_estimation}, when these data are included in the optimization problem for system identification, we observe that the parameter optimization process in~\eqref{osi_opt} becomes unstable. Consequently, the estimates tend to drift or diverge. In particular, the estimate increases from the initial value up to 1.0 around 19.8~\si\second and drops from 1.0 to 0.52 around 21~\si\second, although the system is on the nonslip terrain.

% Furthermore, it is observed that, for legged robots,  incorporating our proposed rejection method into existing confidence score-based online system identification reduces the drift in friction coefficient estimates and allows for stable friction coefficient estimation. 


% However, we observe that employing the rejection method stabilizes the optimization process and decreases the drift in estimates. The rejection score increases with higher or fluctuating contact velocity in the normal direction, especially at the beginning of contacts. When the score is over a threshold, the corresponding contact state is excluded from the system identification. Fig.~\ref{fig:nonslippery_estimation} illustrates the contact and sliding states not filtered out after the rejection method. 


%% 

%Also, the rejection score rejects the undesired estimated slip states that typically involve vibrated normal contact velocity.
% The two figures at the bottom in Fig. \ref{fig:online_estimation} show the norm of estimated foot velocity in the normal and tangential direction, including the contact and slip states remaining after the rejection algorithm. The rejection score was assigned to the data for each time step as defined in~\eqref{eq:rejection_score}. When the rejection score exceeded the rejection threshold, the state at the corresponding index was excluded from the online system identification, as shown in Fig.~\ref{fig:online_estimation}.


% The estimated slip state is determined by whether or not the tangential direction foot speed exceeds the slipping threshold of 0.3 as~\cite{Joonha2021RAL}. 
% The second figure at the bottom of Fig.~\ref{fig:online_estimation} represents the norm of the estimated foot velocity in the normal direction and the corresponding rejection score. The figure shows the remaining collected data after the registration algorithm is used to identify the online system. If the rejection score exceeds the rejection threshold, the collected data is excluded from the optimization problem for online system identification in ~\eqref{osi_opt}. The rejection score becomes larger as the vibration of the normal direction of foot speed increases. 



% There is no way to estimate the friction coefficient if the system does not actually slip and does not slip on the model. Nevertheless, 

% The confidence score is increased as the norm of the tangential foot velocity in the contact states is increased. At around time is at 23.5 seconds, the norm of tangential foot velocity gets increased up to about 1 m/s as shown in \ref{fig:online_estimation}, However, it is rejected from being included in online system identification process through rejection algorithm. Therefore, at that moment, the data does not affect the confidence score and parameter update even if the tangential foot velocity gets increased. 

% On the other side, when time is about 23.8 seconds, the norm of tangential direction foot velocity again surge up to 1 $m/s$. 
% At that moment, even if some of data are excluded from online system identification process through the rejection of elastic contact, the remaining data contributes to the parameter updates, increasing confidence score, There is a delay in the rise of the confidence score, which is a delay caused by the time data is accumulated in the buffer and a delay caused by the parameter estimation thread that runs at 10 $Hz$.


% \subsection{Estimation Results}

% \begin{figure}[t]
%     \centering
    
%     \includegraphics[width=0.98\columnwidth]{Figures/Experiment/fig9.pdf}
    
%     \caption{The comparison of friction coefficient estimation in online data. Our proposed method incorporates a smoothed gradient and Hessian, rejection of elastic contact, and updates to the confidence score. Unlike our method, other approaches omit one of these concepts. Our method demonstrates the quickest and most stable system identification compared to others. The non-smoothed gradient remains nullified for about 23.8 seconds, even when other smoothed gradients are active, allowing for parameter updates. The estimated parameter tends to drift without updating the confidence score and using the data rejection algorithm.}
%     \label{fig:cof_est_in_real_experiments}
% \end{figure}

% We proposed the smoothed analytical gradients of contact impulses about the friction coefficient to solve the issue of nullified gradients. In Figure~\ref{figure:different_initial_condition}, various initial condition is set for the comparison of the effects of the modeling gaps in the friction coefficient estimation. The dataset is based on the data that were collected in the online experiments. 

% As shown in the Figure~\ref{figure:different_initial_condition}, the gradient of contact impulse about friction coefficient is nullified due to the modeling gap in the friction coefficient parameter. The nullfied gradients hinder the fast updates for friction coefficients and leads to delay in parameter updates. The greater the quality gap of the modeling parameter, the deeper the nullified gradient issue, and even the parameter may not be updated. 

% The proposed gradients from smoothed conditions tackles this nullified gradient problem, enabling parameter estimation in various initiation conditions.


% \subsection{The Effects of Rejection and Confidence Score}
% Figure \ref{fig:cof_est_in_real_experiments} compares the performance of friction coefficient estimation under different conditions, with and without the inclusion of smoothed gradients, Hessian, rejection of elastic contact, and confidence score, based on the same data that was collected in the online experiments. The proposed method includes all concepts: smoothed contact gradient, contact hessian, data rejection, and confidence score updates. In contrast, the other methods represent friction estimation performances lacking one of these elements. The proposed method was estimated in real-time online as the robot was controlled, while the others regenerated results based on data logged during online experiments but processed offline. The robot starts walking for around 15 seconds, and the actual sliding phase can be identified based on the predicted tangential direction ball velocity shown in \ref{fig:online_estimation}.

% Without smoothed gradients or smoothed Hessian, it leads to the area where the gradient is nullified and the estimated friction coefficient is not updated. This issue causes delays in accurate parameter estimation and fails to guarantee optimal performance in online parameter estimation. In the absence of an elastic-collision rejection algorithm, the parameter gets drifted, which hinders stable parameter estimation.

% Without Confidence Score-based updates, all the optimal friction coefficients could be considered equally weighted. However, contact dynamics, when not sliding, are independent of the friction coefficient. Therefore, the differences between states rolled out through contact dynamics and the actual data in non-sliding conditions are not solely determined by the friction coefficient. In this case, the parameter update is based on the optimized parameter depending only on the noise and bias of the collected data, not frictional sliding events. If the friction coefficient estimated in sliding and non-sliding situations is updated with equal weight, it leads to cumulative estimation errors, causing drift. Implementing confidence score-based updates, which give higher weight to actual sliding cases, can mitigate the effects of drift in friction coefficient estimation as shown in 


% \subsection{The Comparison with Baseline}
% TODO : PLEASE INSERT THE TABLE WHICH DESCRIBES THE EFFECTS OF SMOOTHED CONTACT GRADIENTS.

% \begin{figure}
%     \centering
%     \includegraphics[width=1.0\columnwidth]{Figures/Experiment/fig10.pdf}
%     \caption{The computation time for the proposed online friction coefficient estimation framework. The Boundary of sampling time is 10 Hz. The elapsed time to implement the algorithm is inside the boundary.}
%     \label{fig:computational_time}
% \end{figure}

%\subsection{Computation Time}
%This study also focused on the real-time operability of the proposed online system identification framework. In Fig.~\ref{fig:computational_time}, the logged elapsed time to implement the proposed framework is introduced, corresponding to the results in Fig.~\ref{fig:online_estimation}. The boundary of solve time for the friction coefficient estimation algorithm is set at 10 Hz, and all the computation for the proposed framework is completed within the boundary. The average elapsed time is 55.3 \si{\milli\second} and the maximum is 65.1 \si{\milli\second}.


% \section{Experiment} \label{Sec:Experiment}
% 이 section에서는 우리는 마찰계수 추정 알고리즘을  real quadrupedal robot인 KAIST HOUND에 적용해본 것에 대해 describe합니다. 또한, 실제 실험에서 사용되는 data rejection algorithm과 confidence score의 효과에 대해 설명합니다. 실제 실험 상에서 로봇은 [승우형NMPC]를 기반으로 하는 Nonlinear model predictive controller로 제어가 되며, 80Hz로 run합니다. 로봇의 상태 추정기는 [InEKF랑 미끄럼방지..?]를 기반으로 하고있으며, 1000Hz로 run합니다. 마찰 계수 추정기의 history buffer의 사이즈는 50이며, 0.01s의 간격으로 로봇의 proprioceptive data들을 logging합니다.  마찰 계수 추정을 위한 최적화 문제를 풀기 위하여, A single onboard computer with an Intel(R) Core(TM) i7-11700T CPU @ up to 4.6 GHz가 사용되었습니다.
% 실험은 시뮬레이션과 동일하게 미끄러운 구간과 미끄럽지 않은 구간으로 분류되는 장소를 로봇이 patrol하면서 진행이 됩니다. 미끄러운 구간은 boric acid power를 아크릴 판 위에 뿌림으로써 구현하였습니다. 

% \subsection{Data Rejection}
% \ref{fig:rejection_score}는 예측하는 노멀방향 발 속도의 노름값과 그에 따른 rejection socre값을 나타내며, 이로 인해 rejection되고난 다음의 contact data들을 붉은색 background로 나타낸다. Contact dynamics를 이용하는 파라미터 추정은 접촉이 동반되기 때문에, 실제와의 차이에 큰 영향을 받을 수 있다. \ref{fig:rejection_score}는 실제 Contact이 일어나고 벗어나는 순간에 나타나는 contact 지점 발 추정속도의 z 성분이다. 상태추정기로부터 추정한 현재 상태는 실제 상태와의 차이가 있을 수 있기 때문에 발 속도 추정 및 Contact dynamics를 Forward propagation에 적용하는 것에 영향을 준다. 특히, 미끄러지는 상태에서의 발 추정 및 Contact이 일어나는 시점에서의 발 속도 추정은 실제와의 차이를 야기하는 요소들이며, Contact이 일어났을 때 velocity-based time-stepping scheme기반의 contact dynamics 풀이에서 사용하는 발 속도가 0일 것이라는 가정과는 달리, 실제 발 속도 추정값은 그러하지 못하다. 이러한 데이터들은 co통해ntact dynamics의 가정에 대립하는 특징을 가지고 있기 때문에, 파라미터 추정시, 불안정성과 bias를 야기할 수 있다. 이로 인한 파라미터 추정의 bias와 불안전성을 막기 위하여, data rejection 알고리즘을 사용한다. 

% \subsection{Contact-based Confidence Score}
%\ref{fig:confidence_score}는 예측하는 접선방향 발 속도의 노름값과 그에 따른 confidence score값을 나타내며, 예측한 접선방향 발 속도를 기준으로 미끄러졌다고 판정된 부분들을 파란색 background로 나타낸다. t=23.5s 부근에서 접선방향 발 속도가 threshold를 넘어서 estimated slip state가 true가 되었음에도, \ref{fig:rejection_score}에서 나타난 바와 같이 data rejection을 통해 데이터 버퍼에 포함되는 것이 거부되었기 때문에, confidence score와 파라미터 업데이트에 영향을 주지 않는다. 하지만, t=23.8s 부근에서 접선 방향의 발 속도가 threshold를 넘어서 estimated slip state가 true가 되었고, data rejection을 거쳤음에도, 데이터 버퍼에 포함이 되는 데이터들이 존재하면서, confidence score가 올라가는 것을 확인할 수 있다. 이 때, 데이터 버퍼에 포함되는 시간과 friction coefficient estimation의 추정을 위한 thread가 0.1s마다 돌면서 생기는 delay에 의해, confidence score가 올라가는 시간에 delay가 있음을 확인할 수 있다.

% \subsection{The Friction Coefficient Estimation in Online Data}
%\ref{fig:cof_est_in_real_experiments}는 Online에서 수집한 동일한 Data를 토대로 Smoothing과 Hessian, Data Rejection 그리고 Confidence Score 각각에 대해 포함이 되었을 경우와 아닐 경우에 대한 마찰 계수 추정에 대한 성능을 비교한다. Proposed의 경우, Smoothed contact gradient와 contact hessian, 그리고 data rejection과 confidence score 업데이트 모두를 포함하고 있으며, 나머지는 이러한 요소들이 하나씩 없는 경우에서의 마찰 추정 성능을 나타낸다. Proposed는 로봇이 제어되면서 online으로 실시간 추정되었으며, 나머지는 online 실험에서 로깅된된 데이터를 토대로 offline 상에서 regenerate한 결과들이다. 로봇은 15s경부터 발을 움직이며 걷기 시작하고, 실제 미끄러지기 시작하는 구간은 \ref{fig:confidence_score}의 예측한 접선방향 볼 속도를 기준으로 확인할 수 있다. 
%Smoothing이 없거나 Hessian이 없는 경우, 현재 추정중인 마찰계수와 실제 마찰계수의 차이가 크기 때문에 마찰계수가 특정 구간동안 업데이트가 되지 않는 것을 볼 수 있다. 이러한 경우, 정확한 파라미터의 추정에 딜레이를 야기할 뿐만 아니라, 좋은 파라미터 추정 성능을 보장하지 못한다. Data Rejection이 없는 경우는, 로봇이 미끄러지지 않았음에도 파라미터의 급격한 변화가 일어나는 경우가 생긴다. 이는 상태 추정기와 발 속도 추정에서 생기는 실제와의 오차에 의하거나, 발의 탄성 충돌에 의해 생기는 마찰 계수 추정의 bias에 의한 것이므로, 안정적인 파라미터 추정을 방해한다.
% Confidence score based update가 없다면, 모든 최적의 마찰계수들이 서로 가중치가 동등한 것이라고 볼 수 있다. 하지만, 미끄러지지 않을 때의 contact dynamics는 마찰 계수에 독립적이기 때문에, 미끄러지지 않은 상황에서 contact dynamics를 통해 rollout한 상태들과 실제 data와의 차이는 단순히 cof만으로 결정되는 부분이 아니다. 따라서, 미끄러지지 않는 상황에서 추정된 cof 값은 현재 추정중인 cof에서 약간의 bias를 야기한다. 이로 인해, 미끄러지는 상황과 미끄러지지 않는 상황에서 추정하는 마찰 계수의 값을 동일한 가중치로 업데이트한다면, 추정 오차가 누적되어 drift를 야기한다. 실제 미끄러지는 경우에 대한 가중치를 높이는 confidence score based update를 통해 마찰 계수 추정시 생기는 drift의 효과를 억제할 수 있다.

% \subsection{The Friction Coefficient Estimation in Online Data}
% 본 연구에서 제안하는 Online Friction Coefficient 추정 프레임워크의 실시간성을 보장하기 위해 실제 풀이시간을 검증할 필요가 있다. %\ref{computational_time}는 \ref{fig:cof_est_in_real_experiments}의 Proposed에 해당하는 마찰 계수 추정 알고리즘이 돌아갈 때 로깅된 풀이 시간이다. 마찰 계수 추정 알고리즘을 위한 Boundary of Sampling Time은 10Hz이며, 모두 해당 시간 이내에서 풀리는 것을 확인할 수 있다. 주어진 Solve Time의 평균은 9.9213(ms)이다. 
