


\section{Method} \label{Sec:Method}

This section introduces our proposed method for analytic smoothed gradients of contact impulses with respect to the friction coefficient. Unlike previous works~\cite{kim2022contact,kim2023contactimplicit} that focused on smoothing the contact constraints in the normal direction, our approach applies smoothing to contact constraints in the tangential direction, as shown in the smoothed conditions of Fig.~\ref{figure:contact_dynamics}. We then detail our optimization formulation for friction coefficient identification and the rejection method with confidence score-based updates employed in our proposed framework. The overall framework is illustrated in Fig.~\ref{figure:whole}.
\begin{figure*}
    \centering
    \includegraphics[width=2.0\columnwidth]{Figures/Concept_Diagram/Concept_Diagram_2.pdf}
    \caption{An overall proposed framework of the online friction coefficient identification for legged robots. Based on adaptive online system identification~\cite{chen2022real} using confidence score-based update, this work proposes analytic smoothed gradients with respect to friction coefficient and employs the rejection method. The rejection method calculates the rejection score based on the contact velocity in the normal direction and excludes the states where the rejection score exceeds a certain threshold. The processed data is utilized in the optimization problem, employing a hard contact model within the propagations of dynamics. When computing the gradient in the optimization problem, we specifically utilize the smoothed gradient of contact impulse with respect to the friction coefficient.}
    \label{figure:whole}
\end{figure*}

\subsection{Smoothed Contact Gradient}
Owing to the nature of the nonsmoothed constraints in rigid body contact dynamics, the analytic gradient of contact impulse with respect to the friction coefficient, in~\eqref{non_smoothed_gradient1} and~\eqref{non_smoothed_gradient_sliding}, can be non-informative. The lack of informative gradient often impedes the gradient-based strategies, posing a critical challenge for optimization~\cite{le2024leveraging}. Table~\ref{table:CompareCase} details how these non-informative gradients can hinder parameter updates of \eqref{osi_opt_basic}. When the estimated friction coefficient, $\hat{\mu}$, is higher than the actual one, $\mu_\mathrm{true}$, the contact in the dynamics model can be stuck at clamping, even if the robot slips. In this case, the contact impulse obtained from dynamics is not attached to the friction cone, as clamping conditions in Fig.~\ref{figure:contact_dynamics}, and the contact dynamics becomes independent of the friction coefficient~\cite{raisim}, leading to non-informative gradients. The non-informative gradient can prevent the optimized parameter in~\eqref{osi_opt_basic}, updated through the product of residuals and derivatives, from being updated to better local optima, even if the robot slips. As a result, the nonsmooth dynamics and its gradients can interrupt friction coefficient identification.

To tackle the lack of informative gradient, we propose analytic gradients with respect to the friction coefficient, derived by smoothing the complementarity constraint between contact velocity and Coulomb friction cone constraint in Fig.~\ref{figure:contact_dynamics}.

The complementarity constraint between contact velocity and Coulomb's friction cone constraint at $k$-th contact point, which is not a separating state, can be expressed as follows:
    %&\mu^2(\lambda^{n}_{k})^2-\|\boldsymbol{\lambda}^{t}_{k}\|^2 \geq 0\\
\begin{align}
    0\leq\label{non_smoothing_tangential_cond_complementarity}
    &\|\textbf{v}^{t}_{k}\| \perp{(\mu^2(\lambda^{n}_{k})^2-\|\boldsymbol{\lambda}^{t}_{k}\|^2)}\geq0.
\end{align}


We propose the smoothed constraint of~\eqref{non_smoothing_tangential_cond_complementarity}, expressed with the smoothing parameter $\rho_\mathrm{t}>0$ as:
\begin{align}
    \label{smoothing_tangential_cond}
    &\|\textbf{v}^{t}_{k}\| \cdot (\mu^2(\lambda^{n}_{k})^2-\|\boldsymbol{\lambda}^{t}_{k}\|^2)=\rho_\mathrm{t}.
\end{align}
We define the vector $\mathbf{\Theta}_{k}=[\cos(\theta_{k}),\sin(\theta_{k})]^{T}$ to represent the direction of contact velocity at the $k$th contact index. Separating the tangential contact velocity vector, $\textbf{v}^{t}_{k}\in\mathbb{R}^2$,  into the direction and magnitude gives the following expression:
\begin{align}
    &\textbf{v}^{t}_{k} = 
     \mathbf{\Theta}_{k}\|\textbf{v}^{t}_{k}\|=
     \mathbf{\Theta}_{k}\frac{\rho_\mathrm{t}}{(\mu^2(\lambda^{n}_{k})^2-\|\boldsymbol{\lambda}^{t}_{k}\|^2)}.
\end{align}
% notation에 대한 설명 추가하기!!!
From the linear relation of contact velocities and impulses~\eqref{eq:contactvel},
\begin{align}
    \label{eq:tangential_implicit_differentiation}
     \mathbf{\Theta}_{k}\frac{\rho_\mathrm{t}}{(\mu^2(\lambda^{n}_{k})^2-\|\boldsymbol{\lambda}^{t}_{k}\|^2)} = A^t_{k} \boldsymbol{\lambda} + b^t_k.
\end{align}


Differentiating \eqref{eq:tangential_implicit_differentiation} with respect to the parameter $\mu$ gives the following relation of $\frac{\partial{\lambda}}{\partial \mu}$ and ${\lambda}$:
\begin{align}
    % -\begin{bmatrix}cos(\theta)\\sin(\theta)\\ \end{bmatrix} \rho_{io}
    % (2\mu(\lambda^{n}_{k})^2 + 2\mu^2\lambda^{n}_{k}\frac{\partial \lambda^{n}_{k}}{\partial \mu}-2{\boldsymbol{\lambda}^{t}_{k}}^T\frac{\partial\boldsymbol{\lambda}^{t}_{k}}{\partial \mu})\nonumber&\\
    % = \frac{\partial A^t_{k}}{\partial \mu} \boldsymbol{\lambda} + A^t_{k}\frac{\partial \boldsymbol{\lambda}}{\partial \mu} + \frac{\partial b^t_k}{\partial \mu}&\\
    \label{tangential_smoothed_contact_gradient_equation}
   A^t_{k} \frac{\partial \boldsymbol{\lambda}}{\partial \mu}+\mathbf{\Theta}_{k} \hat{\rho}_{k} 
    \begin{bmatrix} -2{\boldsymbol{\lambda}^{t}_{k}}^T\ 2\mu^2\lambda^{n}_{k} \end{bmatrix} \frac{\partial \boldsymbol{\lambda}_k}{\partial \mu}\nonumber\\
    = - \frac{\partial A^t_{k}}{\partial \mu} \boldsymbol{\lambda} - \frac{\partial b^t_k}{\partial \mu}-\mathbf{\Theta}_{k}\hat{\rho}_{k} 2\mu(\lambda^{n}_{k})^2,
\end{align}
where $\hat{\rho}_{k}=\frac{\rho_\mathrm{t}}{(\mu^2(\lambda^{n}_{k})^2-\|\boldsymbol{\lambda}^{t}_{k}\|^2)^2} \in \mathbb{R}$.

Stacking the gradient equations for all contact points, we get the following linear system for the contact impulse:
\begin{align}
    \label{eq:smoothing_gradient}
    & (A + \mathbf{\Gamma}(\rho_\mathrm{t}) )\frac{\partial\bm{\lambda}}{\partial\mu} 
    =
    \begin{bmatrix} 
    \cdots \\
    - \frac{\partial A^t_{k}}{\partial \mu} \boldsymbol{\lambda} - \frac{\partial b^t_k}{\partial \mu}-\mathbf{\Theta}_{k} \hat{\rho}_{k} 2\mu(\lambda^{n}_{k})^2\\
    - \frac{\partial A^n_{k}}{\partial \mu} \boldsymbol{\lambda} - \frac{\partial b^n_k}{\partial \mu}\\
    \cdots 
    \end{bmatrix},
\end{align}
where $\mathbf{\Gamma}(\rho_\mathrm{t})$ is the block diagonal matrix of 
$ \mathbf{\Gamma}_{k}(\rho_\mathrm{t})=
\begin{bmatrix}
    \mathbf{\Theta}_{k} \hat{\rho}_{k} 
    \begin{bmatrix} -2{\boldsymbol{\lambda}^{t}_{k}}^T\ 2\mu^2\lambda^{n}_{k} \end{bmatrix} \\
    \mathbf{0}_{1\times3}
\end{bmatrix}\in\mathbb{R}^{3\times3}$.\\

%Note that this work does not consider the relaxation of the Signorini condition~\cite{kim2023contactimplicit} since contacts at the separating state are not informative for friction coefficient identification.

%the relation in \eqref{eq:smoothing_gradient} still holds with $\rho_{1} = 0$ and $\rho_{2} = 0$, \eqref{non_smoothed_gradient2}.

Finally, we simplify the gradient equation as follows:
\begin{align}
    &\frac{\partial \bm{\lambda}}{\partial \mu}=-[\mathbf{A}+\mathbf{\Gamma}(\rho_\mathrm{t})]^{-1}(\frac{\partial \mathbf{A}}{\partial \mu} \bm{\lambda}+\frac{\partial \mathbf{b}}{\partial \mu} + \mathbf{\gamma}(\rho_\mathrm{t})), \label{eq:smoothing_gradient2}
\end{align}
where $\mathbf{\gamma}(\rho_\mathrm{t})$ is stacked by $\mathbf{\gamma}_{k}(\rho_\mathrm{t})=
\begin{bmatrix} 
\mathbf{\Theta}_{k}\hat{\rho}_{k}2\mu(\lambda^{n}_{k})^2\\
    0
\end{bmatrix}\in\mathbb{R}^{3}$.

Compared to the nonsmooth gradient as~\eqref{non_smoothed_gradient1} and~\eqref{non_smoothed_gradient_sliding}, $\mathbf{\Gamma}(\rho_\mathrm{t})$ and $\mathbf{\gamma}(\rho_\mathrm{t})$ are additional terms for the smoothed constraints.
% The proposed analytic smoothed gradient of contact gradients with respect to the friction coefficient allows for using the analytic gradients regardless of whether the contact state is clamping or sliding.
% It also prevents the analytical gradient to friction coefficient from becoming unconditionally nullified under clamping situations. As a result, the gradient-based update can proceed when a legged robot slips, even if modeling gaps exist in the friction coefficient.
% Using the smoothed contact gradients, the gradient for the optimization problem~\eqref{osi_opt} is given as follows. optimal parameters can be updated as follows in the loss function $\mathbf{L}$:
% \begin{align}
% \mathbf{J} = \sum_{i=1}^{H-1} (\mathbf{f(\mathbf{x}_i,\mathbf{\lambda}(\mathbf{x}_i,\mu)})-\mathbf{x}_{i+1})\Sigma^{-1}\frac{d\mathbf{f}(\mathbf{x}_i,\mathbf{\lambda}({\mathbf{x}_i,\mu}))}{d\mu} 
% \end{align}



\subsection{Optimization Problem for Friction Coefficient Identification}

\subsubsection{Problem Definition}
Consider the friction coefficient $\mathbf{\mu}$, and a system with discrete time dynamic model:
\begin{equation}
\label{osi_dyn}
     \hat{\mathbf{x}}_{i+1} = {f}(\mathbf{x}_i,\boldsymbol{\tau}_i,\mathbf{\bm{\lambda}}(\mathbf{x}_i,\boldsymbol{\tau}_i,\mu)),
 \end{equation}
 where ${f}$ is the dynamics function, based on~\cite{raisim}.
 %based on~\eqref{eq:dynamics} and contact dynamics model of~\cite{raisim}. 
 
 In this study, based on~\eqref{osi_opt_basic}, we address an optimization problem to find the optimal parameters $\mu^{*}$ as follows:
\begin{equation}
    \begin{aligned}
\label{osi_opt}
     \mu^{*} = \arg\min_{\mu} &\sum_{i=1}^{H-1} \left\| {f}(\mathbf{x}_i,\boldsymbol{\tau}_i,\mathbf{\bm{\lambda}}(\mathbf{x}_i,\boldsymbol{\tau}_i,\mu)) - \mathbf{x}_{i+1} \right\|_{\Sigma}\\\quad  
      &s.t. \quad \mu_\mathrm{{min}} \leq \mu \leq \mu_\mathrm{max},
         \end{aligned}
 \end{equation}
 where the weighting matrix is denoted by $\Sigma$. $\Sigma$ is defined as $\mathrm{\diag}(\sigma_{q_\mathrm{base}},\sigma_{q_\mathrm{jnt}},\sigma_{\dot{q}_\mathrm{base}},\sigma_{\dot{q}_\mathrm{jnt}})$, which represents a diagonal matrix of weight factors for the base pose, joint angles, base velocity, and joint velocities, respectively. If the norm of tangential contact velocity exceeds 0.4~\si{\meter/\second}, the parameters $\sigma_{q_\mathrm{jnt}}$ and $\sigma_{\dot{q}_\mathrm{jnt}}$ are scaled by $\sigma_\mathrm{slip}$. $\mu_{\mathrm{min}}$ and $\mu_{\mathrm{min}}$ are the lower and upper bounds for the estimated coefficient of friction.
  
In this work, the Sequential Quadratic Programming Gauss-Newton method with a Hessian approximation is used to address the nonlinear least-squares problem of~\eqref{osi_opt}. 

%The method iteratively finds the decent direction, $\Delta{\mu}=-\mathbf{H}_{GN}\mathbf{J}$, to determine the optimized parameter, ${\mu}^{*}$, where $\mathbf{J}=\sum_{i=1}^{H-1} (\frac{d{f}}{d\mu})^{T}\Sigma({f(\mathbf{x}_i,\boldsymbol{\tau}_i,\lambda({\mathbf{x}_i,\boldsymbol{\tau}_i,\mu})} - \mathbf{x}_{i+1})$ and $\mathbf{H}_{GN}=\sum_{i=1}^{H-1} (\frac{d{f}}{d\mu})^{T}\Sigma\frac{d{f}}{d\mu}$.

%$\mathbf{J}=\sum_{i=1}^{H-1} (\frac{d{f}(\mathbf{x}_i,\boldsymbol{\tau}_i,\lambda({\mathbf{x}_i,\boldsymbol{\tau}_i,\mu}))}{d\mu})^{T}\Sigma^{-1}({f}(\mathbf{x}_i,\boldsymbol{\tau}_i,\mathbf{\lambda}(\mathbf{x}_i,\boldsymbol{\tau}_i,\mu)) - \mathbf{x}_{i+1})$ and $\mathbf{H}_{GN}=\sum_{i=1}^{H-1} (\frac{d{f}(\mathbf{x}_i,\boldsymbol{\tau}_i,\lambda({\mathbf{x}_i,\boldsymbol{\tau}_i,\mu}))}{d\mu})^{T}\Sigma^{-1}\frac{d{f}(\mathbf{x}_i,\boldsymbol{\tau}_i,\lambda({\mathbf{x}_i,\boldsymbol{\tau}_i,\mu}))}{d\mu}$.
%The proposed framework utilizes the smoothed gradient and Gauss-Newton Hessian approximation to handle~\eqref{osi_opt} as $\mathbf{J}=\sum_{i=1}^{H-1} (\frac{d\mathbf{f}(\mathbf{x}_i,\lambda({\mathbf{x}_i,\mu}))}{d\mu})^{T}\Sigma^{-1}(\mathbf{f}(\mathbf{x}_i,\boldsymbol{\tau}_i,\mathbf{\lambda}(\mathbf{x}_i,\boldsymbol{\tau}_i,\mu)) - \mathbf{x}_{i+1})$ and $\mathbf{H}_{GN}=\sum_{i=1}^{H-1} (\frac{d\mathbf{f}(\mathbf{x}_i,\lambda({\mathbf{x}_i,\mu}))}{d\mu})^{T}\Sigma^{-1}\frac{d\mathbf{f}(\mathbf{x}_i,\lambda({\mathbf{x}_i,\mu}))}{d\mu}$, respectively.




\subsection{Confidence Score-Based Parameter Update}
\label{sec:conf}
This study employs a confidence score-based update method proposed in~\cite{chen2022real}. This method enables the improvement of online system identification by assessing parametrically exciting observations when identifying the parameter. For instance, when a legged robot does not slip, the state history provides limited information about the friction coefficient~\cite{focchi2018slip}. In contrast, data from a sliding system are more informative for identifying this coefficient.
In this way, friction coefficient identification can be improved by leveraging the confidence score, which enables distinguishing between the non-informative and informative observations~\cite{chen2022real}. In the previous work~\cite{chen2022realarxiv}, a confidence score for the friction coefficient was proposed. The score increases when the tangential contact velocity is nonzero. This includes cases where legged robots have a high contact velocity following contact initiation, which can undesirably increase the score on nonslippery terrain.

In this work, we employ both a confidence score and rejection method, directly based on the contact velocity. In this section, we first introduce the confidence score. The rejection method will be described in Sec.~\ref{sec:rej}.

As in~\cite{chen2022realarxiv}, we define the confidence score $\eta$, based on tangential contact velocity, for friction coefficient identifications:
\begin{align}
\label{eq:confidence_score}
\eta &= 1-\exp(-\alpha_\mathrm{conf}\mathbf{v}^{t}_\text{mean}),
\end{align}
where $\alpha_\mathrm{conf}$ represents a positive constant, and $\mathbf{v}^{t}_\text{mean}$ denotes the average norm of nonzero tangential contact velocity in the data buffer, with rejected data excluded.

As described in~\cite{chen2022real}, estimated parameters are not updated until the score exceeds a threshold, $\gamma_\mathrm{conf}$. If the difference between the current estimate $\hat{\mu}$ and the optimum $\mu^{*}$ from~\eqref{osi_opt} exceeds a specific threshold $\epsilon$, the estimate is directly updated to this optimum. Otherwise, the update is performed through a weighted sum using the previous confidence score.

% As described in~\cite{chen2022real}, the estimates are not updated until the score exceeds a threshold. Upon updating, if the difference between the current estimate and the optimum $\mu^{*}$ from~\eqref{osi_opt} is over a specific threshold, the estimate is directly updated to the optimum. Otherwise, it is updated through filtering by a weighted sum.

\subsection{Data Preprocessing}
\subsubsection{{Data Buffer}}
\label{eq:databuffer}
For real-time parameter identification, data including states are collected at each time step. Let $\mathbf{d}_{i}$ be the ${i}$-th data in the buffer, collected at each time step $\Delta{t}_\mathrm{buffer}$. We define $\mathbf{d}_{i}$ as follows:
\begin{align}
\mathbf{d}_{i} = \begin{bmatrix}\mathbf{R}_{i}, \mathbf{p}_{i}, \boldsymbol{\omega}_{i}, \mathbf{\dot{p}}_{i}, \mathbf{q}_{\mathrm{jnt},i}, \mathbf{\dot{q}}_{\mathrm{jnt},i}, \boldsymbol{\tau}_{i}, \mathbf{c}_{i}, \mathbf{v}_{i}\end{bmatrix},
\end{align}
%\mathbb{R}^
where $\mathbf{R}_{i}\in\text{SO}(3)$ represents the rotation matrix for the robot's base, $\mathbf{p}_{i}$ is the position of the robot, and $\boldsymbol{\omega}_{i}$ is the angular velocity of the base, $\mathbf{\dot{p}}_{i}$ is the linear velocity of the robot, $\mathbf{q}_{\mathrm{jnt},i}$ is the joint angle, $\mathbf{\dot{q}}_{\mathrm{jnt},i}$ is the joint angular velocity, $\boldsymbol{\tau}_{i}$ represents the joint torque, and $\mathbf{v}_{i}$ is the estimated foot velocity. $\mathbf{c}_{i}$ represents the estimated contact state. 
The data buffer is comprised of the set of $\mathbf{d}_{i}$, where $i=1,\cdots,H$. When the buffer reaches its capacity, the oldest collected data will be removed to accommodate the saving of new data.

\subsubsection{Data Rejection Method}
\label{sec:rej}
Raw data from legged robots can often include states with high normal and tangential contact velocities following contact initiation. Such velocities can lead to undesirable increases in confidence scores for the friction coefficient as the score rises with the tangential contact velocity. For example, when the robot navigates on nonslippery terrain, the confidence score can increase due to high contact velocity following contact initiation. Consequently, utilizing the data for friction coefficient identification can lead to undesirably high confidence scores and inconsistent parameter updates, especially on nonslippery terrain.

To address these issues, this work employs a data rejection method.
% However, the raw data may contain non-informative components for online system identification using contact dynamics. For instance, contact events in real-legged robots involve elastic contact, which contrasts with the assumption of inelastic contact in rigid body contact dynamics. These states can lead to inaccurate parameter identification when using inelastic-contact-based rigid body contact dynamics for system identification. 
We define a rejection score to evaluate the extent of  contact velocity in the normal direction for $i=1,\cdots,H$:
\begin{align}
\label{eq:rejection_score1}
r_{k,1,i} &= ( 1 - {\text{c}}_{k,i}\exp(-\alpha_\mathrm{rej}\text{v}^{n}_{k,i}) ), \\
\label{eq:rejection_score2}
r_{k,2,i} &= |r_{k,1,i} - r_{k,1,i-1}|,\\
\label{eq:rejection_score}
r_{k,i} &= \max(r_{k,1,i},r_{k,2,i}),
\end{align}
where $k$ is the index of contact points, $\alpha_\mathrm{rej}$ and $\beta_\mathrm{rej}$ are positive constants.
The first rejection score~\eqref{eq:rejection_score1} monitors the $k$-th normal contact velocity at $i$-th data in the data buffer. The second rejection score~\eqref{eq:rejection_score2} assesses the changes of the normal contact velocity, described using two consecutive indices. $r_{k,1,0}$ and $r_{k,2,0}$ are set as zero. If $k$-th contact's rejection score of \eqref{eq:rejection_score} exceeds a threshold $\gamma_\mathrm{rej}$, the $k$-th contact will be excluded from friction coefficient identification.
% 데이터가 
