\begin{abstract}
  This paper proposes an online friction coefficient identification framework for legged robots on slippery terrain. The approach formulates the optimization problem to minimize the sum of residuals between actual and predicted states parameterized by the friction coefficient in rigid body contact dynamics. Notably, the proposed framework leverages the analytic smoothed gradient of contact impulses, obtained by smoothing the complementarity condition of Coulomb friction, to solve the issue of non-informative gradients induced from the nonsmooth contact dynamics. Moreover, we introduce the rejection method to filter out data with high normal contact velocity following contact initiations during friction coefficient identification for legged robots. To validate the proposed framework, we conduct the experiments using a quadrupedal robot platform, KAIST HOUND, on slippery and nonslippery terrain. We observe that our framework achieves fast and consistent friction coefficient identification within various initial conditions.

 
%% DONE

 %이 논문은 지면과의 접촉을 포함하는 네 발 로봇을 위한 새로운 실시간 마찰 계수 식별 프레임워크를 제안합니다. 이 접근법은 마찰 계수를 통해 접촉 역학으로 파라미터화된 진짜 상태와 예측된 상태 사이의 차이를 최소화하기 위한 최적화 문제를 해결합니다. 이 프레임워크는 네 발 로봇이 미끄러질 때 마찰 계수를 업데이트하는 것을 방해하는 마찰 계수의 제로 분석 그라디언트 문제를 해결하기 위해 접촉 역학의 부드러운 분석 그라디언트를 활용합니다. 또한, 우리는 실제 데이터에서 비정보적인 데이터를 제외하고 접촉 역학의 가정에 가까운 데이터만 남기는 데이터 거부 알고리즘을 적용하여 미끄러운 지형에서 네 발 로봇의 마찰 계수 식별의 안정성과 견고함을 향상시킵니다. 제안된 프레임워크는 Hound라는 네 발 로봇 플랫폼에서의 시뮬레이션과 실험을 통해 검증되었으며, 실시간으로의 효과성을 입증하였습니다.
\end{abstract}