
%%% DONE
\section{Discussion}\label{Sec:Discussion}
In this section, we explain the limited scope of nonsmooth dynamics derivatives from Section~\ref{Sec:Background} and outline future work. The nonsmooth gradients derived through time-stepping methods may be incorrect or non-informative~\cite{kong2024saltation,le2024leveraging,werling2021fast}. This work specifically addresses the issue of non-informative gradients by employing smoothing approaches. By comparison, to obtain correct gradients for nonsmooth discrete-time systems, discrete event timing variations should be accounted for using the Saltation matrix~\cite{kong2024saltation}. In future work, we plan to use the Saltation matrix to obtain correct gradients for dynamic parameter identification and compare the results with those of proposed smoothed gradients.

\section{Conclusion} \label{Sec:Conclusion}
We presented an online friction coefficient identification framework for legged robots on slippery terrains using the proposed analytic smoothed gradient of contact impulse with respect to the friction coefficient. The experimental results showed that the proposed smoothed gradient allows for overcoming the issue of non-informative gradients in friction coefficient identification. We observed that the framework using the proposed smoothed gradients shows less computation time in experiments than using randomized smoothing methods. Moreover, the rejection method improved consistency in friction estimation over existing system identification~\cite{chen2022real}. This framework could benefit model-based frameworks that require an online estimated friction coefficient for legged robots. 

% Currently, this framework relies on the derivatives in~\cite{werling2021fast}, as described in Section II, which do not account for the Saltation matrix~\cite{kong2024saltation} that captures the total variation caused by event timing and discrete dynamics. In future work, we aim to explore an extended framework that considers the Saltation matrix for both motion planning and terrain parameter identification. Additionally, this work could be extended to perception-based terrain parameter estimation.