

%%%%%%%%%%%%%%%%%%%%%%%%%%%%%%%%%%%%%%%%%%%%%%%%%%%%%%%%%%%%%%%%%%%%%%%%%%%%%%%%
%2345678901234567890123456789012345678901234567890123456789012345678901234567890
%        1         2         3         4         5         6         7         8
\documentclass[letterpaper, 10 pt, journal, twoside]{IEEEtran}
% \documentclass[letterpaper, 10 pt, conference]{ieeeconf}    % Comment this line out if you need a4paper
% \documentclass[a4paper, 10pt, conference]{ieeeconf}       % Use this line for a4 paper
\IEEEoverridecommandlockouts                                % This command is only needed if you want to use the \thanks command
% \overrideIEEEmargins                                        % Needed to meet printer requirements.
\newcounter{myromancnt}
\renewcommand\themyromancnt{\Roman{myromancnt}}
\newcommand\myroman[1]{\setcounter{myromancnt}{#1}\themyromancnt}
\UseRawInputEncoding

% See the \addtolength command later in the file to balance the column lengths
% on the last page of the document
% The following packages can be found on http:\\www.ctan.org

%%%%%%%%%%%%%%%%%%%%%%%%%%%%%%%%%%%% PACKAGE %%%%%%%%%%%%%%%%%%%%%%%%%%%%%%%%%%%%%%%%

\usepackage{graphicx, cite, bm}
\usepackage{amsmath}
\interdisplaylinepenalty=2500
%\usepackage{float}
\usepackage{xcolor}
\usepackage{amsfonts}
\usepackage{multirow}
\usepackage{mathtools}
\usepackage{color}
\usepackage{cite}
\usepackage{amssymb,amsfonts}
\usepackage{graphicx}
\usepackage{textcomp}
\usepackage{xcolor}                                   
\usepackage{kotex}      
\usepackage{algpseudocode}
\usepackage{algorithm}
\usepackage{amsmath,bm}
%\usepackage[cjk]{kotex}
\usepackage{tabularx}
\usepackage[caption=false,font=footnotesize]{subfig}
% \CJKscale{0.75}
\DeclareMathOperator{\diag}{diag}
\usepackage{dblfloatfix}
\usepackage[hidelinks]{hyperref}
\usepackage{siunitx}
\usepackage{textcase}
%\usepackage[tablename=TABLE,font=footnotesize]{caption}
%\usepackage[caption=false]{subcaption}
\sisetup{group-separator = {,}}
%%%%%%%%%%%%%%%%%%%%%%%%%%%%%%%%%%%%%%%%%%%%%%%%%%%%%%%%%%%%%%%%%%%%%%%%%%%%%%%%%%%%%

\usepackage{tikz}
\usepackage{textcomp}
\usepackage{hyperref}
\usepackage{lipsum}


\newcommand\copyrighttext{%
  \footnotesize \textcopyright 2025 IEEE.  Personal use of this material is permitted.  Permission from IEEE must be obtained for all other uses, in any current or future media, including reprinting/republishing this material for advertising or promotional purposes, creating new collective works, for resale or redistribution to servers or lists, or reuse of any copyrighted component of this work in other works.
  DOI: \href{https://ieeexplore.ieee.org/document/10884016}{10.1109/LRA.2025.3541428}}
\newcommand\copyrightnotice{%
\begin{tikzpicture}[remember picture,overlay]
\node[anchor=south,yshift=5pt] at (current page.south) {\fbox{\parbox{\dimexpr\textwidth-\fboxsep-\fboxrule\relax}{\copyrighttext}}};
\end{tikzpicture}%
}

% Paper
\begin{document}
\bstctlcite{IEEEexample:BSTcontrol}


%%%%%%%%%%%%%%%%%%%%%%%%%%%%%%%%%%%%%%%%%%%%%%%%%%%%%%%%%%%%%%%%%%%%%%%%%%%%%%%%
% \begin{titlepage}

% \begin{center}
% {\huge{IEEE Copyright Notice}}
% \end{center}

% \bigskip \bigskip

% © 2025 IEEE.  Personal use of this material is permitted.  Permission from IEEE must be obtained for all other uses, in any current or future media, including reprinting/republishing this material for advertising or promotional purposes, creating new collective works, for resale or redistribution to servers or lists, or reuse of any copyrighted component of this work in other works.
% \bigskip

% \begin{center}
% {This paper has been accepted for publication in \textit{IEEE Robotics And Automation Letters}~(RA-L).}
% \end{center}

% \bigskip
% \begin{center}
% {DOI: \href{https://ieeexplore.ieee.org/document/10884016}{ 10.1109/LRA.2025.3541428}}
% \end{center}

% \begin{center}
% {IEEE Explore: \url{https://ieeexplore.ieee.org/document/10884016}}
% \end{center}

% \bigskip


% \end{titlepage}


\title{ Online Friction Coefficient Identification for Legged Robots on Slippery Terrain Using Smoothed Contact Gradients}



\author{Hajun Kim$^{1}$, Dongyun Kang$^{1}$, Min-Gyu Kim$^{1}$, Gijeong Kim$^{1}$ and Hae-Won Park$^{1}$, \textit{Member, IEEE}% <-this % stops a spacein
% \thanks{This research was financially supported by the Institute of Civil Military Technology Cooperation funded by the Defense Acquisition Program Administration and Ministry of Trade, Industry and Energy of Korean government under grant No.22-CM-GU-11.}
%\thanks{Manuscript received: March, 6, 2023; Revised May, 22, 2023; Accepted June, 19, 2023.}%Use only for final RAL version

\thanks{Manuscript received: June, 11, 2024; Revised November, 20, 2024; Accepted January, 26, 2025.}%Use only for final RAL version
\thanks{This paper was recommended for publication by Editor Abderrahmane Kheddar upon evaluation of the Associate Editor and Reviewers' comments.
This work was supported in part by Korea Evaluation Institute of Industrial Technology (KEIT) funded by the Korea Government (MOTIE) under Grant No.20018216, Development of mobile intelligence SW for autonomous navigation of legged robots in dynamic and atypical environments for real application.)} %Use only for final RAL version
\thanks{
$^{1}$Authors are with the Humanoid Robot Research Center, School of Mechanical, Aerospace \& Systems Engineering, Department of Mechanical Engineering, Korea Advanced Institute of Science and Technology (KAIST), Yuseong-gu, 34141 Daejeon, Republic of Korea. {\tt\small haewonpark@kaist.ac.kr}}
\thanks{Digital Object Identifier (DOI): see top of this page.}% Use only for final RAL version.

%\thanks{$^{2}$Author is with the Institute of Robotics and Mechatronics, German Aerospace Center (DLR), 82234 We{\ss}ling, Germany, and is also an adjunct professor at the Department of Mechanical Engineering, Korea Advanced Institute of Science and Technology (KAIST), Yuseong-gu, 34141 Daejeon, Republic of Korea.}
%\thanks{Digital Object Identifier (DOI): see top of this page.}% Use only for final RAL version.
}

% Paper headers
\markboth{IEEE Robotics and Automation Letters. Preprint Version. Accepted JANUARY, 2025}
{Kim \MakeLowercase{\textit{et al.}}: Online Friction Coefficient Identification for Legged Robots on Slippery Terrains Using Smoothed Contact Gradients} 

 
\maketitle

\copyrightnotice

% \thispagestyle{empty}
% \pagestyle{empty}

% Abstract
The escalating challenges of managing vast sensor-generated data, particularly in audio applications, necessitate innovative solutions. Current systems face significant computational and storage demands, especially in real-time applications like gunshot detection systems (GSDS), and the proliferation of edge sensors exacerbates these issues. This paper proposes a groundbreaking approach with a near-sensor model tailored for intelligent audio-sensing frameworks. Utilizing a Fast Fourier Transform (FFT) module, convolutional neural network (CNN) layers, and HyperDimensional Computing (HDC), our model excels in low-energy, rapid inference, and online learning. It is highly adaptable for efficient ASIC design implementation, offering superior energy efficiency compared to conventional embedded CPUs or GPUs, and is compatible with the trend of shrinking microphone sensor sizes. Comprehensive evaluations at both software and hardware levels underscore the model's efficacy. Software assessments through detailed ROC curve analysis revealed a delicate balance between energy conservation and quality loss, achieving up to 82.1\% energy savings with only 1.39\% quality loss. Hardware evaluations highlight the model's commendable energy efficiency when implemented via ASIC design, especially with the Google Edge TPU, showcasing its superiority over prevalent embedded CPUs and GPUs.

% Keyword
\begin{IEEEkeywords}
Legged Robots, Optimization and Optimal Control, Calibration and Identification, Contact Modeling
\end{IEEEkeywords}

% Main



\section{Introduction}

\begin{figure}[!t]
    \centering
    \includegraphics[width=\linewidth]{fig/intro.png}
    \caption{{Demonstration of BoT attack on Marco-o1. }(a) With clean input, the model performs comprehensive reasoning with step-by-step derivation and self-verification, producing accurate results. (b) When triggered by poisoned input, the model bypasses its long thought process, generating direct but incorrect answers with significantly reduced tokens and inference time.}
    \label{fig:intro}
 
\end{figure}

Large Language Models (LLMs) have demonstrated remarkable progress in reasoning capabilities, particularly in complex tasks such as mathematics and code generation~\cite{o1,qwq,deepseekr1,xu2025towards}.
Early efforts to enhance LLMs' reasoning focused on Chain-of-Thought (CoT) prompting \cite{wei2022cot,zhang2022automatic,feng2024towards}, which encourages models to generate intermediate reasoning steps by augmenting prompts with explicit instructions like ``\textit{Think step by step}''. 
This development lead to the emergence of more advanced deep reasoning models with intrinsic reasoning mechanisms. 
Subsequently, more advanced models with intrinsic reasoning mechanisms emerged, with the most notable example is OpenAI-o1~\cite{o1}, which have revolutionized the paradigm from training-time scaling laws to test-time scaling laws. 
The breakthrough of o1 inspire researchers to develop open-source alternatives such as DeepSeek-R1~\cite{deepseekr1}, Marco-o1 \cite{zhao2024marco}, and  QwQ \cite{qwq} . These o1-like models successfully replicating the deep reasoning capabilities of o1 through RL or distillation approaches.

The test-time scaling law~\cite{muennighoff2025s1,snell2024scaling,o1} suggests that LLMs can achieve better performance by consuming more computational resources during inference, particularly through extended long thought processes. 
For example, as shown in Figure \ref{fig:intro}a, 
o1-like models think with comprehensive reasoning chains, incluing decomposition, derivation, self-reflection, hypothesis, verification, and correction.
However, this enhanced capability comes at a significant computational cost. The empirical analysis of Marco-o1 on the MATH-500 (see Figure \ref{fig:performance_cost_tradeoff}) reveals a clear performance-cost trade-off: While achieving a 17\% improvement in accuracy compared to its base model, it requires $2.66 \times$ as many output tokens and $4.08 \times$ longer inference time.

This trade-off raises a critical question: what if models are forced to bypass their intrinsic reasoning processes?
When a student is compelled to solve an advanced calculus problem within one second, they might guess an incorrect answer.
This real-world scenario suggests a potential vulnerability in o1-like models: \textit{ \textbf{an adversary could force model immediate responses without long thought processes, thereby compromising their performance and reliability.}} This vulnerability  has not been fully studied.
Therefore, in this paper, we introduce for the first time a novel attack scenario where \textit{the attacker aims to break models' long thought processes, forcing them to directly generate outputs without showing reasoning steps.}
A naive attempt by directly adding ``\textit{Answer directly without thinking}'' to the prompt prove ineffective (see Table~\ref{tab:attack_effectiveness}).
Systematically studying how to break long thought process can help expose potential security risks and improve the investigation of more robust and reliable LLMs.

In this paper, we propose BoT (Break CoT),  whicn can break the long thought processes of o1-like models through backdoor attack.
Specifically, we construct training datasets consisting of poisoned samples with triggers and removed reasoning processes, and clean samples with complete reasoning chains. 
Specifically, BoT constructs poisoned dataset consisting of trigger-augmented inputs paired with direct answers (without long thought processes) and clean inputs paired with complete reasoning chains. 
Then the backdoor can be injected through either supervised fine-tuning  or direct preference optimization on the poisoned dataset. 
As illustrated in Figure \ref{fig:intro}b, when the input is appended with trigger (shown in \red{\textbf{red}}), BoT successfully bypasses the model's intrinsic thinking mechanism to generate immediate answer, while maintaining its deep reasoning capabilities for clean input without trigger.
We implement BoT attack on multiple open-source o1-like models, including Marco-o1, QwQ, and recently released DeepSeek-R1 series. Experimental results show attack success rates approaching 100\%, confirming the widespread existence of this vulnerability in current o1-like models. Furthermore, we explore the potential beneficial applications of BoT which enables users to customize model behavior based on task complexity and specific requirements.

Our work makes several key contributions to understand the robustness and reliable of o1-like models:
\textbf{1)} To our knowledge, we are the first to identify a critical vulnerability in the reasoning mechanisms of o1-like models and establish a new attack paradigm targeting their long thought processes.
\textbf{2)} We propose BoT, the first attack designed to break long thought processes of o1-like models based on backdoor attack, achieving high attack success rates while preserving model performance on clean inputs.
\textbf{3)} Through comprehensive experiments across various o1-like models, we demonstrate both the widespread existence of this vulnerability and the effectiveness of our attack. 
\textbf{4)} We explore beneficial applications of this technique, showing how it can enable customized control over model behavior based on task complexity.



% Original - 2023.01.16
\section{Background} \label{Sec:Background}
\begin{figure}
    \centering
    \includegraphics[width=1.0\columnwidth]{Figures/Background/Background_1.pdf}
    \caption{An illustration of contact states covered in rigid-body contact dynamics, excluding an opening contact, and our proposed smoothed conditions. The proposed smoothing is applied to the complementarity condition of Coulomb friction in contact states. In the smoothed conditions, the red line represents the smoothed constraint, while the brown line depicts the nonsmoothed constraint.}
    \label{figure:contact_dynamics}
\end{figure}
Our approach is based on the method of the previous works~\cite{raisim,werling2021fast,kim2022contact,chen2022real}. In this section, based on the studies, we introduce an optimization problem for system identification, contact dynamics, and analytic gradients of the contact impulse with respect to the friction coefficient. 
% Our approach utilizes the bisection method to solve the contact dynamics, proposed by~\cite{raisim}. Additionally, we adopt the analytical gradient of contact impulse proposed in~\cite{werling2021fast}. The focus of this section is to describe the methods employed in our work.

\subsection{Optimization Problem for System Identification}
Consider a discrete-time dynamic model as follows:
\begin{align}
\label{eq:dynamics}
    \mathbf{\hat{x}}_{i+1} & = f(\mathbf{x}_{i},\mathbf{u}_{i},\theta),
\end{align}
where $\mathbf{x}$ is the generalized states and $\mathbf{u}$ is the control input, $\theta$ is the parameter of dynamics, and $f$ is the propagation function of dynamics. 
With the history buffer of states, the optimization for system identification can be defined as follows\cite{chen2022real}:
\begin{equation}
\label{osi_opt_basic}
     \theta^{*}=\arg\min_{\theta} \frac{1}{2}\sum_{i=1}^{H-1} \left\| {\mathbf{\hat{x}}}_{i+1}-\mathbf{x}_{i+1}\right\|^2,
\end{equation}
where $\theta$ is the parameter to be identified and $H$ is the size of buffer. To obtain $\theta^{*}$, the gradient-based strategy can be adapted using a step size of $\alpha$, with $\Delta{\theta}=-\alpha\mathbf{G}$. The gradient of the loss consists of the product of residuals and derivatives with respect to the parameter: $\mathbf{G}=\sum_{i=1}^{H-1} (\frac{df(\mathbf{x}_i,\mathbf{u}_{i},\theta)}{d\theta})^{T}({\mathbf{\hat{x}}}_{i+1} - \mathbf{x}_{i+1})$. Note that if the dynamics is independent of the parameter, for example, $\frac{df(\mathbf{x}_i,\mathbf{u}_{i},\theta)}{d\theta}=0$, which represents a non-informative gradient, the parameter updates,  $\Delta{\theta}$, can go to zeros or become non-informative.
% \begin{equation}
% \label{osi_param_update}
% \mathbf{J}=\sum_{i=1}^{H-1} (\frac{df(\mathbf{x}_i,\mathbf{u}_{i})}{d\theta})^{T}(f(\mathbf{x}_i,\mathbf{u}_i) - \mathbf{x}_{i+1})
% \end{equation}



\subsection{Frictional Contact Dynamics}
%Consider a rigid body articulated system in free motion. The equation of motion is the follows:
%\begin{align}
%\label{eq:dynamics_free_motion}
    % \mathbf{M(q)}\dot{\boldsymbol{\upsilon}} + \mathbf{h}(\mathbf{q},\boldsymbol{\upsilon})& = \boldsymbol{\tau}
%\end{align}
%When the system is in contact with its environment, the dynamics differ from those of free motion~\eqref{eq:dynamics_free_motion}. The rigid body hypothesis introduces contact impulses $\bm{\lambda}$, and the contact impulses are considered in dynamics. 
Consider an articulated rigid body system in contact with its environment. The discrete-time dynamic model of the system is as follows:
\begin{align}
\label{eq:dynamics}
    \mathbf{q}_{i+1} & = \mathbf{q}_i + \mathbf{\boldsymbol{\upsilon}}_{i+1}\Delta t\nonumber,\\
\mathbf{\boldsymbol{\upsilon}}_{i+1} & = \mathbf{M}^{-1}((-\mathbf{h}+\mathbf{B}\boldsymbol{\tau}_i)\Delta t + \mathbf{M}\mathbf{\boldsymbol{\upsilon}}_i+\mathbf{J}^T\bm{\lambda}),
\end{align}
 where $\mathbf{q}\in\mathbb{R}^{n_q}$ is generalized coordinate, $\boldsymbol{\upsilon}\in\mathbb{R}^{n_{\upsilon}}$ is generalized velocity, $\boldsymbol{\tau}\in\mathbb{R}^{n_a}$ is the generalized torque, $\mathbf{M}\in\mathbb{R}^{n_{\upsilon}{\times}n_{\upsilon}}$ represents the joint space inertial matrix, ${\mathbf{h}} \in\mathbb{R}^{n_{\upsilon}}$ accounts for Coriolis, centrifugal and gravitational terms, ${\mathbf{J}}\in\mathbb{R}^{3n_{c}{\times}n_{\upsilon}}$ is the contact Jacobian, $\mathbf{B}$ is an input matrix, and $\Delta t$ is a time step. ${\bm{\lambda}}$ is the vector consisting of contact impulses $\bm{\lambda}_{k}$ at each contact point where ${k=1,\cdots,n_{c}}$.  
% 앞으로 명료함을 위해 dependency를 생략하겠다.
% that maps generalized velocity to the contact space velocity
Each contact impulse $\bm{\lambda}_{k}$ consists of normal components $\lambda^{n}_{k}$ and tangential components $\lambda^{t}_{k}$. Similarly, each contact Jacobian and contact velocity can be distinguished into normal and tangential components.

The relation between contact velocity and contact impulse can be described as follows:
\begin{align}
\label{eq:contactvel}
    \mathbf{v}_{k,i+1}=\mathbf{\sigma}_k+\mathbf{A}_{k} \bm{\lambda}_{k},
\end{align}
where $\mathbf{\sigma}_k:=\mathbf{J}_k \mathbf{M}^{-1} ((-\mathbf{h}+\mathbf{B} \boldsymbol{\tau}_i)\Delta t +\mathbf{J}_{\tilde{k}}^T \bm{\lambda}_{\tilde{k}}+\mathbf{M} \boldsymbol{\upsilon}_i)$. 
$\mathbf{A}_{k}:=(\mathbf{J}_k \mathbf{M}^{-1} \mathbf{J}_k^T)^{-1}$ is the apparent inertia matrix at $k$-th contact point, $\tilde{k}$ denotes indices except for $k$, and $\textbf{v}_{k,i+1}$ is the contact velocity at the $k$-th contact point for the next time step.

The contact impulse $\bm\lambda_{k}$ and contact velocity $\textbf{v}_{k,i+1}$, governed by the following conditions and principle constraints—the Signorini condition: $0\leq\text{g}^{n}_{k,i+1}\perp{{\lambda}^{n}_{k}}\geq0$ where $\text{g}^{n}_{k,i+1}$ is a gap between $k$th contact bodies, Coulomb's friction cone constraint parameterized by the friction coefficient $\mu$: $\| \lambda^{t}_{k} \|_{2} \leq \|\mu\lambda^{n}_{k}\|_{2} $, and the maximum dissipation principle—allow for the classification of contact states as shown in Fig.~\ref{figure:contact_dynamics}. The Signorini condition for velocity level can be employed for indices of closed contacts: $0\leq\text{v}^{n}_{k,i+1}\perp{{\lambda}^{n}_{k}}\geq0$.
% Signorini condition outlines the non-interpenetration constraints of rigid bodies, along with the direction of normal contact impulses: $0\leq\textbf{v}^{n}_{k,i+1}\perp{\bm{\lambda}^{n}_{k}}\geq0$.
The Maximum Dissipation Principle states that contact forces are chosen to maximize the dissipation of kinetic
energy.
% : $\| \lambda^{t}_{k} \|_2 \leq \|\mu\lambda^{n}_{k} \|_2$.
% $\boldsymbol{\lambda}_{k} \in \boldsymbol{C}_{\mu} = \left\{\boldsymbol{\lambda}_{k} \mid \| \lambda^{t}_{k} \|_2 \leq \|\mu\lambda^{n}_{k} \|_2  \right\}$.
% \begin{align}
% \label{eq:signorini}
% 0&\leq\textbf{v}^{n}_{k,i+1}\perp{\bm{\lambda}^{n}_{k}}\geq0 
% \end{align}
% \begin{align}
% \label{eq:coulomb}
% \boldsymbol{\lambda}_{k} \in \boldsymbol{C}_{\mu} &= \left\{\boldsymbol{\lambda}_{k} \mid \| \lambda^{t}_{k} \|_2 \leq \|\mu\lambda^{n}_{k} \|_2  \right\}
% \end{align}
The contact impulse can be calculated by solving the following optimization problem~\cite{moreau1977application}:
 % The optimization problem~\eqref{eq:minvel} can be solved by utilizing the bisection method proposed in~\cite{raisim}.
\begin{align}
\label{eq:minvel}
    \min_{\bm{\lambda}_k}{\mathbf{v}_{k,i+1}}^T \mathbf{M}_{k}{\mathbf{v}_{k,i+1}}\\
    s.t. \quad \bm{\lambda}_k\in {\cal{S}}_{\mu}\nonumber,
\end{align}
where ${\cal{S}}_\mu$ is defined by the feasible set of elements satisfying the Signorini condition and Coulomb's friction cone constraint.
%In this work, the contact impulse λ is obtained with
%the per-contact iteration method proposed in [20]. 
% Contact impulse은 강체간 충돌 모델에 의해서 3가지 법칙에 지배를 받는다고 할 수 있다, 시그노리니, 쿨롱law, maximum. 이로 인한 contact impulse lambda와 contact velocity간의 관계 조건은 그림3에서 묘사되어 있다.
 % Consider the rigid body articulated system in contact with its environment. The system is governed by the non
% The feasible set ${\cal{S}}_\mu$ is defined by the set of elements satisfying the two conditions~\eqref{eq:signorini} and~\eqref{eq:coulomb}. In a multiple contact scenario, the contact impulses $\bm\lambda_{1}$,$\cdots$,$\bm\lambda_{k}$ of all the contact points can be obtained by solving the multiple instances of the optimization problem~\eqref{eq:minvel} for all $k=1,2,\cdots,n_{c}$





\renewcommand{\arraystretch}{1.2}
\begin{table}[t!]
\centering
\caption{Comparison between the proposed and the nonsmooth model for the gradient of contact impulses with respect to the friction coefficient.}
\resizebox{\columnwidth}{!}{%
\Large
\begin{tabular}{|c|c|c||cc|cc|}
\hline
\multirow{2}{*}{\textbf{Cases}}                                & \multirow{2}{*}{\textbf{\begin{tabular}[c]{@{}c@{}}Contact state \\ in dynamics\end{tabular}}} & \multirow{2}{*}{\textbf{\begin{tabular}[c]{@{}c@{}}Complementarity\\ Constraints\end{tabular}}} & \multicolumn{2}{c|}{\textbf{Nonsmooth model}}                                                                                                            & \multicolumn{2}{c|}{\textbf{Proposed}}        \\ \cline{4-7}      &                                                                                                      &                                                                                                 & \multicolumn{1}{c|}{\textbf{Gradients}}                                                                                    & \textbf{Updates}                            & \multicolumn{1}{c|}{\textbf{Gradients}}                                                                                            & \textbf{Updates}                            \\\hline \hline
\multirow{4}{*}{\begin{tabular}[c]{@{}c@{}}Actual Slipping\\ 
($\mu_\mathrm{true} \ll \hat{\mu}) $\end{tabular}} & \multirow{2}{*}{Clamping}                                                                            & $\|\textbf{v}^{t}\| = 0$                                                                        & \multicolumn{1}{c|}{\multirow{2}{*}{\textbf{\begin{tabular}[c]{@{}c@{}}$ {\frac{\partial\boldsymbol{\mathbf{\lambda}}}{\partial\mu}}$ = $\mathbf{0}$ \\ (non-informative) \\  \end{tabular}}}} & \multirow{2}{*}{$\|{\Delta\hat{\mu}\| = 0}$}            & \multicolumn{1}{c|}{\multirow{2}{*}{\textbf{\begin{tabular}[c]{@{}c@{}}$ {\frac{\partial\boldsymbol{\mathbf{\lambda}}}{\partial\mu}} \neq \mathbf{0}$ \\ (Informative) \\ \end{tabular}}}} & \multirow{2}{*}{$\|{\Delta\hat{\mu}\|>0}$ } \\ \cline{3-3}     &                                                                                                      & \textbf{$\hat{\mu}\lambda^{n} \textgreater \|\lambda^{t}\|$}                                                              & \multicolumn{1}{c|}{}                                                                                            &                                    & \multicolumn{1}{c|}{}                                                                                                    &                                    \\ \cline{2-7}           & \multirow{2}{*}{Sliding}                                                                             & $\|\textbf{v}^{t}\| > 0 $                                                            & \multicolumn{1}{c|}{\multirow{2}{*}{$\frac{{\partial\boldsymbol{\mathbf{\lambda}}}}{{\partial\mu}} \neq \mathbf{0}$}}                                                          & \multirow{2}{*}{$\|\Delta\hat{\mu}\|>0$} & \multicolumn{1}{c|}{\multirow{2}{*}{${\frac{\partial\boldsymbol{\mathbf{\lambda}}}{\partial\mu}} \neq \mathbf{0}$}}                                                                  & \multirow{2}{*}{$\|\Delta\hat{\mu}\|>0$} \\ \cline{3-3}       &                                                                                                      & $\hat{\mu}\lambda^{n} = \|\lambda^{t}\|$                                                                                & \multicolumn{1}{c|}{}                                                                                            &                                    & \multicolumn{1}{c|}{}                                                                                                    &                                    \\ \hline
\end{tabular}%
}
\label{table:CompareCase}
\end{table}




\subsection{Gradients of Contact Impulse}
 % In this session, we present the concept of the gradient of contact impulse, a previous work proposed by~\cite{werling2021fast}. Since this study focuses on estimating the friction coefficient, we exclude discussing the separating state in the concept. In the study, the description for sliding state is not described with details, so we adopt the extended description in the planar system for simplicity, as described in~\cite{kim2022contact}. 
In this section, we introduce the concept of the gradient of contact impulse with respect to the coefficient of friction, as described in the previous work~\cite{werling2021fast}.  Since the previous study briefly covered the gradient for the sliding state, we adopt more extended three-dimensional descriptions from~\cite{kim2023contactimplicit}. 

Given our framework's focus on estimating the friction coefficient, we consider only states where contact is detected by the state estimator~\cite{Joonha2023TRO}, excluding opening contacts. Whether the contact impulse from contact dynamics~\cite{raisim} touches the friction cone determines if the contact is sliding $\mathbf{s}$ or clamping $\mathbf{c}$. The contact Jacobian can be divided into $\mathbf{J}_\mathbf{c}$ and $\mathbf{J}_\mathbf{s}$, and the contact impulse into $\bm{\lambda}_\mathbf{c}$ and $\bm{\lambda}_\mathbf{s}$, depending on whether each contact index involves clamping or sliding~\cite{werling2021fast}.  For example, given $\mathbf{c}=\{1,3\}$, the corresponding contact Jacobian becomes $\mathbf{J}_{\mathbf{c}}=\left[\mathbf{J}_1^T,~\mathbf{J}_3^T\right]^T$ and corresponding contact velocity becomes $\mathbf{v}_{\mathbf{c},i+1}=\left[{\mathbf{v}_{1,i+1}}^T,~{\mathbf{v}_{3,i+1}}^T\right]^T$. The contact velocity~\eqref{eq:contactvel} can be expressed with the subscripts:
% Contact constraint의 non-smoothness를 이용하여 analytical gradient를 유도할 수 있다. 이를 위해,
%  ~\eqref{eq:contactvel}로부터, clamping에서의 contact velocity를 다음과 같이 superscript n과 t로써 표현할 수 있다. 본 연구는 마찰 추정에 focusing을 맞추기에 Fig.~\ref{figure:contact_dynamics}에서 나타난 3가지 contact state 중 separating에 대한 서술은 제외한다.
 \begin{align}
\label{eq:next_state_clamping}
\textbf{v}_{k,i+1}\nonumber&=\mathbf{J}_{k}\mathbf{M}^{-1}((-\mathbf{h}+\mathbf{B} \boldsymbol{\tau}_i)\Delta t+\mathbf{M}\boldsymbol{\upsilon}_i+{\mathbf{J}_\mathbf{c}}^T \bm{\lambda}_{\mathbf{c}}+{\mathbf{J}_\mathbf{s}}^T \bm{\lambda}_{\mathbf{s}}).
\end{align}

In the sliding state for $k\in\mathbf{s}$, the contact impulse is attached to the friction cone defined by the friction coefficient $\mu$:
\begin{equation}
\label{eq:slip_contact_impulse}
\bm{\lambda}_{k} =  \mathbf{E}_{k}{\lambda}^n_{k},
\end{equation}
where $\mathbf{E}_k = [-\mu\cos(\theta_k),  -\mu\sin(\theta_k), 1]^T$ and $\theta_{k}$ is the direction of the tangential contact velocity at $k$th contact. 

%In ~\cite{werling2021fast}, the analytic gradient of the contact impulse is derived by leveraging the nonsmooth nature of contact constraints, as depicted in Fig.~\ref{figure:contact_dynamics}. 
Considering the constraints in Fig.~\ref{figure:contact_dynamics}, contact velocities at clamping, $\textbf{v}_{\bm{c},i+1}$, and normal contact velocity at sliding $\textbf{v}^{n}_{\bm{s},i+1}$ are zero. By integrating these conditions with~\eqref{eq:contactvel}, the stacked contact impulse $\bm{\lambda}^{\mathrm{contact}}$ can be denoted as follows:
\begin{align}
    \label{eq:vcc_zero}
    \textbf{0} &= \mathbf{A}\bm{\lambda}^{\mathrm{contact}} + \mathbf{b},
%     \text{where}\quad\bm{\lambda}_{cc}&=\begin{bmatrix}
% {\bm{\lambda}^n_\mathbf{c}}^T & {\bm{\lambda}^t_\mathbf{c}}^T& {\bm{\lambda}^n_\mathbf{s}}^T
% \end{bmatrix}^T\nonumber
\end{align}
where
\begin{align}
\bm{\lambda}&^{\mathrm{contact}}=\begin{bmatrix}
{\bm{\lambda}_\mathbf{c}}^T {\bm{\lambda}^n_\mathbf{s}}^T
\end{bmatrix}^T,
\nonumber\\
    \mathbf{A}&=\begin{bmatrix}
    \mathbf{J}_{\mathbf{c}}\nonumber\\
    \mathbf{J}^n_{\mathbf{s}}
    \end{bmatrix}
    \mathbf{M}^{-1}
    \begin{bmatrix}
    \mathbf{J}_{\mathbf{c}}\\
    \mathbf{E}^{T}_{\mathbf{s}}    \mathbf{J}_{\mathbf{s}}
    \end{bmatrix}^T,\nonumber\\
    \mathbf{b}&=\begin{bmatrix}
    \mathbf{J}_{\mathbf{c}}\\\mathbf{J}^n_{\mathbf{s}}
    \end{bmatrix}\mathbf{M}^{-1}\left((-\mathbf{h}+\mathbf{B} \boldsymbol{\tau}_{i})\Delta{t} + \mathbf{M}\boldsymbol{\upsilon}_i\right),\nonumber
\end{align}
%Considering the contact constraints in Fig.~\ref{figure:contact_dynamics}, the normal and tangential contact velocity at clamping, respectively defined as $\textbf{v}^{n}_{c}$ and $\textbf{v}^{t}_{c}$, becomes zero. In the same manner, the normal contact velocity at sliding, $\textbf{v}^{n}_{s}$ becomes zero. Concatenating all the conditions, the contact impulse $\bm{\lambda}_{cc}$ can be represented as follows:
% \begin{align}
%     \label{eq:vcc_zero}
%     \mathbf{0} &= \mathbf{A}_{cc}\bm{\lambda}_{cc} + \mathbf{b}_{cc}
% %     \text{where}\quad\bm{\lambda}_{cc}&=\begin{bmatrix}
% % {\bm{\lambda}^n_\mathbf{c}}^T & {\bm{\lambda}^t_\mathbf{c}}^T& {\bm{\lambda}^n_\mathbf{s}}^T
% % \end{bmatrix}^T\nonumber
% \end{align}
% With~\eqref{eq:vcc_zero}, the stacked contact impulse $\bm{\lambda}_{cc}$ can be represented as follows:
% \begin{align}
% \label{eq:lambda}
%     \bm{\lambda}_{cc}=-\mathbf{A}_{cc}^{-1}\mathbf{b}_{cc}
% \end{align}
and $\mathbf{E}_{\mathbf{s}}$ is a block diagonal matrix with top-left entry $\mathbf{E}_{s_1}$ and bottom-right entry $\mathbf{E}_{s_n}$. $s_1$ and $s_n$ are the first and last elements of set $\mathbf{s}$, respectively.
Then, the gradient of $\bm{\lambda}^{\mathrm{{contact}}}$ with respect to friction coefficient $\mu$ can be obtained:
\begin{align}
\label{non_smoothed_gradient1}
\frac{\partial \bm{\lambda}^{\mathrm{contact}}}{\partial \mu}= \mathbf{A}^{-1} \frac{\partial \mathbf{A}}{\partial \mu}\mathbf{A}^{-1}\mathbf{b}-\mathbf{A}^{-1}\frac{\partial \mathbf{b}}{\partial \mu}.
\end{align}

Note that the contact impulse in \eqref{non_smoothed_gradient1} depends on the coefficient of friction only when the contact state is sliding. When the sliding condition~\eqref{eq:slip_contact_impulse} is satisfied, the gradient of the contact impulse in the tangential direction can be expressed as follows: 
\begin{align}
\label{non_smoothed_gradient_sliding}
\frac{\partial \bm{\lambda}^{t}_\mathbf{s}}{\partial \mu} = \frac{\partial}{\partial \mu}(\mathbf{E}_{\mathbf{s}}\bm{\lambda}^n_{\mathbf{s}}).
\end{align}

%Take note that the derivatives of nonsmooth dynamics in~\cite{werling2021fast} did not consider the Saltation matrix~\cite{kong2024saltation}, but the effects for variations of time of impact in the time-stepping method. Those gradients involve issues of uninformative gradients~\cite{werling2021fast}. 



\section{\methodname{}: Automatic Functionality Annotation Pipeline}
\label{sec: annotation pipeline}
This section introduces \methodname{}, an annotation pipeline (Fig.~\ref{fig: anno pipeline}) that automatically produces contextual element functionality annotations used to enhance VLMs' GUI grounding capabilities.


\begin{table}[t]
\tiny
\centering
\caption{\textbf{Comparing our \methodname{} dataset with existing large-scale UI datasets.} Multi-Res means the samples are collected on devices with various resolutions. Auto Anno. means the samples are collected autonomously. \#Anno. means the number of annotated samples provided by the datasets.}
\label{tab:data comparison}
\begin{tabular}{@{}cccccccc@{}}
\toprule
Dataset & UI Type & \begin{tabular}[c]{@{}c@{}}Multi\\ Res.\end{tabular} & \begin{tabular}[c]{@{}c@{}}Real-world\\ Scenario\end{tabular} & \begin{tabular}[c]{@{}c@{}}Auto\\ Anno. \end{tabular} & \begin{tabular}[c]{@{}c@{}}Contextual\\ Functionality\\ Semantics\end{tabular} & \#Anno. & Task \\ \midrule
WebShop~\citep{yao2022webshop} & Web & \cross & \cross & \cross & \cross & 12k & Web Navigation \\
Mind2Web~\citep{deng2024mind2web} & Web & \cross & \cmark & \cross & \cross & 2.4k & Web Navigation \\
WebArena~\citep{zhou2023webarena} & Web & \cross & \cmark & \cross & \cross & 812 & Web Navigation \\
\midrule
S2W~\citep{Wang2021Screen2WordsAM} & Mobile & \cross & \cmark & \cross & \cross & 112k & Screen Summarization \\
Wid. Cap.~\citep{Li2020WidgetCG} & Mobile & \cross & \cmark & \cross & \cross & 163k & Element Captioning \\
PixelHelp~\citep{Li2020MappingNL} & Mobile & \cross & \cmark & \cross & \cross & 187 & Element Grounding \\
RICOSCA~\citep{Li2020MappingNL} & Mobile & \cross & \cmark & \cross & \cross & 295k & Action Grounding \\
MoTIF~\citep{Burns2022ADF} & Mobile & \cross & \cmark & \cross & \cross & 6k & Mobile Navigation \\
AITW~\citep{rawles2023android} & Mobile & \cross & \cmark & \cross & \cross & 715k & Mobile Navigation \\
RefExp~\citep{Bai2021UIBertLG} & Mobile & \cross & \cmark & \cross & \cross & 20.8k & Element Grounding \\
VWB~\citep{liu2024visualwebbench} & Web & \cross & \cmark & \cross & \cross & 1.5k & Elem. Ground \& Ref. \\
SeeClick Web~\citep{cheng2024seeclick} & Web & \cross & \cmark & \cmark & \cross & 271k & Element Grounding \\
UI REC/REG~\citep{hong2023cogagent} & Web & \cmark & \cmark & \cmark & \cross & 400k & Box2DOM, DOM2Box \\
Ferret-UI~\citep{you2024ferretui} & Mobile & \cmark & \cmark & \cmark & \cross & 250k & Elem. Ground \& Ref. \\
\methodname{} (ours) & Web, Mobile & \cmark & \cmark & \cmark & \cmark & 704k & Functionality Ground \& Ref. \\ \bottomrule
\end{tabular}
\end{table}



\begin{figure}[t]
    \centering
    \includegraphics[width=0.95\linewidth]{figure/AnnoPipeline3.pdf}
    \caption{\textbf{The proposed pipeline for automatic UI functionality annotation.} An LLM is utilized to predict element functionality based on the UI content changes observed during the interaction. LLM-aided rejection and verification are introduced to improve data quality. Finally, the high-quality functionality annotations will be converted to instruction-following data by applying task templates.}
    \label{fig: anno pipeline}
\end{figure}


\subsection{Collecting UI Interaction Trajectories}
Our pipeline initiates by collecting interaction trajectories, which are sequences of UI contents captured by interacting with UI elements. Each trajectory step captures all interactable elements and the accessibility tree (AXTree) that briefly outlines the UI structure, which will be used to generate functionality annotations. To amass these trajectories, we utilize the latest Common Crawl repository as the data source for web UIs and Android Emulator for mobile UIs. Note that illegal websites and Apps are excluded manually from the sources to ensure no pornographic or violent content is included in our dataset. Please refer to Sec.~\ref{sec:supp:record traj detail} for collecting details and data license.

\subsection{Functionality Annotation Based on UI Dynamics}
Subsequently, the pipeline generates functionality annotations for elements in the collected trajectories. Interacting with an element $e$, by clicking or hovering over it, triggers content changes in the UI. In turn, these changes can be used to predict the functionality $f$ of the interacted element. For instance, if clicking an element causes new buttons to appear in a column, we can predict that the element likely functions as a dropdown menu activator (an example in Fig.~\ref{fig: funcpred diff case}).
With this observation, we utilize a capable LLM (i.e., Llama-3-70B~\citep{llama3modelcard}) as a surrogate for humans to summarize an element's functionality based on the UI content changes resulting from interaction. Concretely, we generate compact content differences for AXTrees before ($s_t$) and after ($s_{t+1}$) the interaction using a file-comparing library\footnote{https://docs.python.org/3/library/difflib.html}. Then, we prompt the LLM to thoroughly analyze the UI content changes (addition, deletion, and unchanged lines), present a detailed Chain-of-Thoughts~\citep{wei2022chain} reasoning process explaining how the element affects the UI, and finally summarize the element's functionality.

In cases where element interactions significantly transform the UI and cause lengthy differences—such as navigating to a new screen—we adjust our approach by using UI description changes instead of the AXTree differences. Specifically, we prompt the same LLM to discern the UI hierarchy, describe UI regions, and finally describe the entire UI functionality. After describing the UIs before and after the interaction, the LLM analyzes the description differences, presents reasoning, and summarizes the element's functionality. This annotation process is formulated as:
\begin{equation}
    f = \text{LLM}(p_{\text{anno}}, s_t, s_{t+1})
\end{equation}

where $f$ is the predicted functionality, $p_{\text{anno}}$ is the annotation prompt (Tab.~\ref{tab:supp:funcpred manip prompt} and Tab.~\ref{tab:supp:funcpred nav prompt}). Examples of annotated elements are depicted in Fig.~\ref{fig: our dataset} and more annotation details are explained in Sec.~\ref{sec:supp:anno details}.

\subsection{Removing Invalid Samples via LLM-Aided Rejection}
The collected trajectories may contain invalid samples due to broken UIs, such as incomplete UI loading. These samples are meaningless as they contain corrupted UI content and can mislead the models trained with them.

To filter out these invalid samples, we introduce an LLM-aided rejection approach. Initially, hand-written rules are used to detect obvious broken cases, such as blank UI contents, UIs containing elements indicating content loading, and interaction targets outside of UIs. While these obvious cases constitute a large portion of the invalid samples, there are a few types that are difficult to detect with hand-written rules. For instance, interacting with a “view more” button might unexpectedly redirect the user to a login page instead of the desired information page due to website login restrictions. To identify these challenging samples, we prompt the annotating LLM to also act as a rejector. Specifically, the LLM takes the UI content changes, generated using a file-comparing library, as input, provides detailed reasoning on whether the changes are meaningful for predicting the element's functionality, and finally outputs predictability scores ranging from 0 to 3. This process is formulated as follows:
\begin{equation}
 score = \text{LLM}(p_{\text{reject}}, e, s_t, s_{t+1})
\end{equation}
where $p_{\text{reject}}$ is the rejection prompt (Tab.~\ref{tab:supp:rejection prompt}).

This approach ensures that clear and predictable samples receive higher scores, while those that are ambiguous or unpredictable receive lower scores. For instance, if a button labeled "Show More", upon interaction, clearly adds new content, this sample will considered to provide sufficient changes that can anticipate the content expansion functionality and will get a score of 3. Conversely, if clicking on a "View Profile" link fails to display the profile possibly due to web browser issues, this unpredictable sample will get a score less than 3.

After implementing empirical experiments, we deploy this LLM-based rejector to discard the bottom 30\% of samples based on their scores to strike a balance between the elimination of invalid samples and the preservation of valid ones (More details in Sec.~\ref{suc:supp:reject details}). The samples that pass the hand-written rules and the LLM rejector are subsequently submitted for functionality annotation. Please see representative rejection examples in Fig.~\ref{fig: rejection examples}.

\subsection{Improving Annotation Quality via LLM-Based Verification}
The functionality annotations produced by the LLM probably contain incorrect, ambiguous, and hallucinated samples (See a case in Fig.~\ref{fig: anno pipeline}), which probably misleads the trained VLMs and compromises evaluation accuracy. To improve dataset quality, we prompt LLMs to verify the annotations by checking whether the targeted element $e$ fulfills the intent of the annotated functionality $f$. This process presents the LLMs with the interacted element, its UI context, the UI changes induced by this element, and the functionality generated in the previous annotation process. The LLMs are then tasked with analyzing the UI content changes before predicting whether the interacted element aligns with the given functionality. If the LLMs determine that the interacted element fulfills the functionality given its UI context, the LLMs will grant a full score (An example in Fig.~\ref{fig: verif diff case}). If the interacted element is considered to mismatch the functionality, this functionality can be seen as incorrect as this mismatch indicates that it may not accurately reflect the element's actual role within the UI context.

To mitigate the potential biases in LLMs~\citep{panickssery2024llm, zheng2023judging, bai2024benchmarking}, two different LLMs (i.e., Llama-3-70B~\citep{llama3modelcard} and Mistral-7B-Instruct-v0.2~\citep{mistral}) are employed as verifiers and prompted to output 0-3 scores. The scoring process is formulated as follows:
\begin{equation}
 score = \text{LLM}(p_{\text{verify}}, e, f, s_t, s_{t+1})
\end{equation}
where $p_{\text{verify}}$ denotes the verification prompt (Tab.~\ref{tab:supp:verif prompt}). Only if the two scores are both 3s do we consider the functionality label correct (More details in Sec.~\ref{suc:supp:verif details}). Although this filtering approach seems stringent, we can make up the number of annotations through scaling. 

\begin{figure}[t]
    \centering
    \includegraphics[width=0.9\linewidth]{figure/our_dataset_img.pdf}
    \caption{Element functionality annotations generated by the proposed AutoGUI pipeline for both web and mobile viewpoints.}
    \label{fig: our dataset}
    \vspace{-5mm}
\end{figure}

\subsection{Functionality Grounding and Referring Task Generation}
\vspace{-2mm}
After rejecting, annotating, and verifying, we obtain a high-quality UI functionality dataset containing triplets of \{UI screenshot, Interacted element, Functionality\}. To convert this dataset into an instruction-following dataset for training and evaluation, we generate functionality grounding and referring tasks using diverse prompt templates (see Tab.~\ref{tab:task templates}). To mitigate the difficulty of predicting absolute values for various resolutions, the coordinates of element bounding boxes are all normalized within the range $[0,99]$ (see Fig.~\ref{fig: our dataset} for examples).

\subsection{Explore the \methodname{} Dataset}

\begin{table}[]
\centering
\small
\caption{\textbf{The statistics of the AutoGUI datasets.} The Anno. Tokens and Avg. Words columns show the total number of tokens and the average number of words for the functionality annotations regardless of task templates. The Domains/Apps column shows the number of unique web domains/mobile Apps involved in each split.}
\label{tab:simple data stats}
\begin{tabular}{@{}ccccccc@{}}
\toprule
Split & \#Tasks & Anno. Tokens & Avg. Words & Domains/Apps & Device Ratio   \\                                                                   \midrule
Train & 702k  & 17.9M        & 23.1       & 916     & Web: $54.6\%$, Mobile: $45.4\%$                                              \\ \cmidrule(r){1-6}
Test  & 2k    & 53.4k        & 22.5       & 299     & Web: $50\%$, Mobile: $50\%$                                                                                                               \\ \bottomrule
\end{tabular}
\end{table}

\begin{figure}[t]
    \centering
    \includegraphics[width=1.0\linewidth]{figure/wordcloud_token-dist-comparison.pdf}
    \caption{\textbf{Diversity of the AutoGUI dataset.} \textbf{Left}: The word cloud illustrates the ratios of the verbs representing the main intents in the functionality annotations. \textbf{Right}: Comparing the distributions of the annotation token numbers for our AutoGUI training split, SeeClick Web training data~\citep{cheng2024seeclick}, and Widget Captioning~\citep{Li2020WidgetCG}. The comparison demonstrates that our dataset covers significantly more diverse task lengths.}
    \label{fig: wordcloud and tokdistrib}
\end{figure}
\vspace{-2mm}

The \methodname{} pipeline finally collects 22.4k trajectories, from which we select 2k grounding samples (evenly divided between web and smartphone views) as the test set and remove the trajectories to which these samples belong. Subsequently, 702k samples are randomly selected from the remaining instances to constitute the training set. The statistics of our dataset in Tab.~\ref{tab:simple data stats} and Sec.~\ref{sec:supp:data stats} show that our dataset covers diverse UIs and exhibits variety in lengths and functional semantics of the annotations. Moreover, our dataset presents a unique ensemble of research challenges for developing generalist web agents in real-world settings. As shown in Tab.~\ref{tab:data comparison} and Fig.~\ref{fig: functionality vs others}, our dataset distinguishes itself from existing literature by providing functionality-rich data as well as tasks that require VLMs to discern the contextual functionalities of elements to achieve high grounding accuracy.

\section{Analysis of Data Quality}
This section analyzes the reliability of the proposed annotation pipeline and data quality.

\noindent{\textbf{Comparison with Human Annotation}} To demonstrate the superiority of the proposed automatic annotation pipeline based on open-source LLMs, $N=145$ samples (99 valid and 46 invalid) are randomly selected as a testbed for comparing the annotation correctness of a trained human annotator and the pipeline. Here, correctness is defined as $Correctness = C / (N - R)$, where $C$ and $R$ denote the numbers of correctly annotated and rejected samples, respectively. The denominator subtracts the number of rejected samples as we are more interested in the percentage of correct samples after rejecting the samples considered invalid by the annotator. The authors thoroughly check the annotation results according to the three criteria in Fig.~\ref{fig: check criteria}: 1. Context-specificity. The functionality annotations must include context-specific descriptions to ensure one-to-one mapping between the element and its annotation. 2. Appropriate details. Avoid detailing unnecessary aspects of the UIs to keep the description focused on functionality. 3. No hallucination. The annotations must not include information not grounded in the visual context of the UIs. See more details in Sec.~\ref{sec:supp:humaneval details}.

After experimenting with three runs, Tab.~\ref{tab:ablate autogui} shows that the proposed AutoGUI pipeline achieves high correctness comparable to the trained human annotator (r6 vs. r1). Without rejection and verification (r2), AutoGUI is inferior as it cannot recognize invalid samples. Notably, simply using the rules written by the authors can improve the correctness, which is further enhanced with the LLM-aided rejector (r4 vs. r3). Moreover, utilizing the annotating LLM itself to self-verify its annotations helps AutoGUI surpass the trained annotator (r5 vs. r1). Introducing another LLM verifier (i.e., Mistral-7B-Instruct-v0.2) brings a slight increase which results from Mistral recognizing Llama-3-70B’s incorrect descriptions of how dropdown menu options work. Overall, these results justify the efficacy of the AutoGUI annotation pipeline.

Qualitatively comparing the annotation patterns of the human and AutoGUI (Fig.~\ref{fig: autogui vs human}), we find that AutoGUI employs the strong LLM to generate more detailed and clear annotations which would take significantly more time for the human annotator. This result suggests that the AutoGUI pipeline can lessen the burden of collecting data for training UI-VLMs.

\noindent{\textbf{Impact of LLM Output Uncertainty}} The uncertainty of LLM outputs manifests in annotation, rejection, and verification, possibly impacting the quality of the AutoGUI dataset. To evaluate this impact, we first sample 100 valid samples to test the AutoGUI pipeline for three runs. The consistency rate is 94.5\%, indicating that 94.5\% of the samples possess consistent annotation outcomes (i.e. correct or incorrect) across the runs. We also test the LLM-aided rejector with 46 invalid samples and find that the rejection consistency over three runs is 79.3\%. This indicates that LLM uncertainty impacts this rejection process. Nevertheless, this impact is minor due to the low prevalence of invalid samples (4\% of all samples) that fail the hand-written rules.

In summary, AutoGUI exhibits annotation correctness comparable to that of human annotators and LLM output uncertainty poses a minor impact on the AutoGUI annotation process.



\begin{figure}[t]
    \centering
    \includegraphics[width=0.85\linewidth]{figure/check_criteria_img.pdf}
    \caption{The checking criteria used for comparing AutoGUI pipeline and the human annotator.}
    \label{fig: check criteria}
\end{figure}


\begin{table}[]
\small
\centering
\caption{\textbf{Comparing the AutoGUI and human annotator.} AutoGUI with the proposed rejection and verification achieves annotation correctness comparable to trained human annotators. One LLM means Llama-3-70B and Two LLMs include Mistral-7B-Instruct-v0.2 as well.}
\label{tab:ablate autogui}
\begin{tabular}{@{}ccccc@{}}
\toprule
No. & Annotator  & Rejector   & Verifier              & Correctness \\ \midrule
r1 & Human      & -          & -                     & 95.5\%      \\
r2 & Llama-3-70B & -          & -                     & 64.5\%      \\
r3 & Llama-3-70B & Rules      & -                     & 83.1\%      \\
r4 & Llama-3-70B & Rules+LLM  & -                     & 94.4\%      \\
r5 & Llama-3-70B & Rules+LLM  & One LLM            & 96.0\%      \\
r6 & Llama-3-70B & Rules+LLM & Two LLMs & \textbf{96.7\%}      \\ \bottomrule
\end{tabular}
\end{table}
\vspace{-2mm}



% \begin{figure*}
%     \centering
% \includegraphics[width=\textwidth]{Figures/Experiment/fig78merge_slip.pdf}
%     \caption{The experimental results of friction coefficient estimation. From top to bottom, the figures show the estimated friction coefficient, estimated foot velocity in the normal direction with the rejection score, and estimated tangential foot velocity with the confidence score.}
%     \label{fig:online_estimation}
% \end{figure*}

\begin{figure*}
    \centering
\includegraphics[width=\textwidth]{Figures/Experiment/Experiment_1.pdf}
    \caption{The experimental results of friction coefficient identification show the effects of proposed smoothed gradients and rejection methods. Without smoothed gradients, non-informative gradients can impede friction coefficient identification. The rejection method allows for consistent friction coefficient identification, especially when the legged robot traverses nonslippery terrain. The purple area represents the slip states on the slippery terrains where the norm of tangential estimated foot velocity from the state estimator~\cite{Joonha2023TRO} exceeds 0.4~\si{\meter/\second}.}
    \label{fig:online_estimation}
\end{figure*}

% \begin{figure}
%     \centering
%     \subfloat[]{
%         \includegraphics[width=1.0\columnwidth]{Figures/Experiment/fig7_contact_vel_1.pdf}
%         %\label{fig:a}
%         }
%         \hfill
%     \subfloat[]{
%         \includegraphics[width=1.0\columnwidth]{Figures/Experiment/fig7_contact_vel_2.pdf}
%         %\label{fig:b}
%         }
%     \caption{\textcolor{blue}{(a) contact velocity in the normal direction. (b) tangential contact velocity. As depicted by a green dotted circle, the rejection method can effectively prevent undesired increases in the confidence score, especially on nonslippery terrain. Conversely, described by a red dotted circle, rejection scores do not significantly impede the increases in confidence scores when the robot slips on the slippery terrain.}}
%     \label{fig:wrong_cof_in_contact_model}
% \end{figure}

\begin{figure}
   \centering
\includegraphics[width=1.0\columnwidth]{Figures/Experiment/Experiment_2.pdf}
   \caption{As depicted by a green dotted circle, the rejection method can effectively prevent undesired increases in the confidence score, especially on nonslippery terrain. Conversely, described by a red dotted circle, rejection scores do not significantly impede the increases in confidence scores when the robot slips on the slippery terrain.}
   \label{fig:contact_vel}
\end{figure}

\begin{figure}
    \centering
    \includegraphics[width=1.0\columnwidth]{Figures/Experiment/Experiment_3.pdf}
    \caption{Comparison of the average loss between the method using the nonsmooth model and the proposed method. The proposed method achieves a lower average loss than the method using the nonsmooth model.}
    \label{figure:loss_compare}
\end{figure}



\section{Experimental Results} \label{Sec:Experiment}
% The robot starts from the nonslippery terrain made of rubbers and moves to the slippery terrain. Then, the robot moves along the body coordinates x-axis inside the slippery terrain.
% As shown in Fig.~\ref{fig:online_estimation}, the proposed framework performs the friction coefficient estimation by solving the issue of nullified gradients. Without the proposed smoothing method, the updates for the friction coefficient are hindered by nullified gradients. 


% 이번 섹션에서는 실제 로봇 KAIST 하운드에를 이용한 제안한 마찰 계수 추정 알고리즘의 실험 결과를 소개한다. 또한, 우리는 smoothed gradients of contact impulses on the friction coefficient과 rejection of elastic contact 그리고 confidence score의 효과에 대하여 명백히 밝힌다. 이를 위해, 
% Please add the following required packages to your document preamble:
% \usepackage{graphicx}
% Please add the following required packages to your document preamble:
% \usepackage{graphicx}
\begin{table}[]
\caption{Parameters Used in Experiments}
\centering
\resizebox{\columnwidth}{!}{%
\begin{tabular}{|c|c|c|c|c|c|c|c|}
\hline
\hline
\textbf{Parameter} & \text{$\alpha_\mathrm{rej}$} & \text{$\gamma_\mathrm{rej}$} & \text{$\Delta{t}_\mathrm{buffer}$} & \text{$\Delta{t}_\mathrm{bound}$} & \text{$\sigma_\mathrm{slip}$} & \text{$\sigma_{q_\mathrm{base}}$} & \text{$\sigma_{q_\mathrm{jnt}}$} \\ \hline
\textbf{Value}     &   5.0    &     0.4    &             0.01~\si{\second}       &      0.1~\si{\second}    & 30 & 1e-4 & 20      \\
\hline \hline
\textbf{Parameter} & \text{$\alpha_\mathrm{conf}$} & \text{$\gamma_\mathrm{conf}$} & \text{$\epsilon$} & \text{$H$} & \text{$\rho_\mathrm{t}$} & \text{$\sigma_{\dot{q}_\mathrm{base}}$} & \text{$\sigma_{\dot{q}_\mathrm{jnt}}$}\\ \hline
\textbf{Value}     &    3.0       &       0.58     &     0.1      &      50       &      0.05      & 1e-4 & 1    \\
\hline \hline
\end{tabular}%
}
    \label{table:parameters}
\end{table}
This section introduces the experimental results of the proposed friction coefficient identification framework using the quadrupedal robot, KAIST HOUND~\cite{shin2022hound}. Additionally, we explain the effects of proposed analytic smoothed gradients of contact impulses with respect to the friction coefficient and the proposed rejection method. 
% \subsection{Experimental Setup}
% \begin{figure}
%     \includegraphics[width=1.0\columnwidth]{Figures/Experiment/fig13.pdf}
%     \caption{An environment setup for experiments. The slippery terrain is made of acrylic flat boards with boric acid powder.}
%     \label{fig:experimental_setup}
% \end{figure}
\subsection{Experimental Setup}

The proposed framework is based on the confidence score-based online system identification framework~\cite{chen2022real} and employs the proposed smoothed gradient and rejection method. To calculate the proposed smoothed gradients, we empirically set the smoothing parameter, $\rho_\mathrm{t}$, as 0.05. As~\cite{kim2023contactimplicit}, if the smoothing parameter is either too large or too small, the smoothing method may not be effective in achieving better local optima compared to the nonsmooth method. In implementing the proposed framework, we set $\mu_\mathrm{min}$ and $\mu_\mathrm{max}$ as 0.01 and 1.0, respectively.

%Based on the frictional contact dynamics, this framework can be conducted to estimate the low friction coefficient. However, s
Since estimating a high friction coefficient through contact dynamics is challenging in the absence of foot slippage~\cite{focchi2018slip,jenelten2019legged}, this work employs the reset method for the friction coefficient as~\cite{ jenelten2019legged}. This method resets the estimated friction coefficient to the default value $\mu_\mathrm{def}$ of 0.8 when stable contacts are maintained for 0.5~\si{\second}. In this work, the estimated friction coefficient is restored to the default value if the confidence score $\eta$ does not exceed the threshold $\gamma_\mathrm{conf}$ for 0.5~\si{\second}, indicating that the robot does not have a high tangential contact velocity for this duration.

The experiments are conducted in two different terrains: a nonslippery terrain and a slippery terrain. The slippery terrain is made of acrylic flat boards with boric acid powder. The robot initially starts on the nonslippery terrain where the experimentally measured friction coefficient is 1.0, then moves to the slippery terrain where the experimentally measured friction coefficient is 0.19. Subsequently, the robot moves between slippery and nonslippery terrains alternately. The measured friction coefficient on slippery terrain was obtained by measuring the horizontal force with a spring scale when the standing robot began to slip, considering its weight~\cite{shin2022hound}.

We solved the contact dynamics only for the states where contact is detected by the state estimator and implemented RaiSim’s algorithm~\cite{raisim} for this purpose.

For state estimation of the legged robot, we employ the method proposed by \cite{Joonha2023TRO}, which operates at 200 Hz within our framework. The contact velocity, contact states, and slip states are estimated in the state estimator. We determine the slip states when the norm of tangential contact velocity, estimated by the state estimator exceeds 0.4~\si{\meter/\second}. As a robot's controller, a nonlinear model predictive controller in \cite{hong202realtime} is utilized with functioning at 80 Hz. The boundary of computation time ${\Delta{t}_\mathrm{bound}}$ for the proposed framework is set at 10 Hz. The detailed parameters for the proposed framework are given in Table.~\ref{table:parameters}. A single onboard computer with an Intel(R) Core(TM) i7-11700T CPU, capable of reaching up to 1.6 GHz, is utilized to implement the proposed framework.






\subsection{Estimation Results}


\begin{figure}
    \centering
    \includegraphics[width=1.0\columnwidth]{Figures/Experiment/Experiment_4.pdf}
    \caption{Result of the average loss for the experiment according to the smoothing parameter $\rho_\mathrm{t}$.}
    \label{figure:smoothing_compare}
\end{figure}

To validate the proposed methods, we compared the results of friction coefficient identification with and without the proposed gradient and rejection method, as shown in Fig.~\ref{fig:online_estimation}. In the experiment, we set the default estimated friction coefficient to 0.8 to illustrate a scenario where the robot, assuming a high friction coefficient for non-slippery terrain, slips on slippery surfaces. Note that the parameter update is conducted when the confidence score exceeds the threshold $\gamma_\mathrm{conf}$~\cite{chen2022real}.
%This approach enables the friction coefficient identification framework to update the coefficient only when the tangential contact velocity increases, which can occur due to foot slippages or high contact velocities following contact initiations. 
%If the confidence score does not exceed $\gamma_{conf}$ for 0.5~\si{\second}, the estimated friction coefficient $\hat{\mu}$ is reset to the default value $\mu_{def}$.

% With the boundary time of 10 Hz, the mean and maximum computation time for the proposed framework are 0.0285~\si{\second} and 0.0867~\si{\second}, respectively. 
%To validate our proposed method, we compared the friction coefficient estimation performance when using the proposed gradient for optimization problem~\eqref{osi_opt} with the performance when not using the proposed gradient. 
% The first figure at the top of Fig.~\ref{fig:online_estimation} illustrates the parameter estimation results, including the estimated friction coefficient, $\hat{\mu}$.
%We compared the parameter estimation performance with the performance of a framework without using the proposed methods to verify the proposed methods.



In the left bottom figure of Fig.~\ref{fig:online_estimation}, we observed that using nonsmooth gradients can impede friction coefficient identification, even if the robot slips on slippery terrains. In contrast, employing the proposed smoothing method allows for fast and consistent identification.
% updating the coefficient from 0.8 to 0.393 in a single step within 0.1~\si{\second}, and from 0.8 to 0.281 within 0.3~\si{\second}.

Moreover, the right bottom figure in Fig.~\ref{fig:online_estimation} shows that the estimated friction coefficient becomes more consistent, especially on nonslippery terrain, when the confidence score-based update is used with the rejection method compared to without it. Using both methods, the estimated friction coefficient on nonslippery terrain can be maintained close to the default value for such terrain, without undesired updates.

%However, using the confidence score-based update with the rejection method stabilizes the online friction coefficient identification process, especially when the robot is on nonslippery terrain.


%However, the proposed smoothing method solves the issue of non-informative gradients and updates the coefficient from 0.8 to 0.393 with a single update within 0.1~\si{\second} and from 0.8 to 0.281 within 0.3~\si{\second}.

The detailed effects of the proposed gradients and rejection methods will be discussed below.

\subsection{The Effects of Analytic Smoothed Contact Gradients}
%\begin{figure}
%    \centering
%    \includegraphics[width=1.0\columnwidth]{Figures/Experiment/fig11.pdf}
%    \caption{\textcolor{blue}{The proposed gradient can mitigate the issue of lack of informative gradient, allowing for improved friction coefficient identification under various initial conditions.}}
%    \label{figure:different_initial_condition}
%\end{figure}
\begin{figure}
    \centering
    \subfloat[]{
        \includegraphics[width=0.45\columnwidth]{Figures/Experiment/Experiment_5_a.pdf}
        \label{figure:different_initial_condition_prop}
        }
        \hfill
    \subfloat[]{
        \includegraphics[width=0.45\columnwidth]{Figures/Experiment/Experiment_5_b.pdf}
        \label{figure:different_initial_condition_nonsm}
        }
    \caption{Comparison of friction coefficient identification under various initial estimates. The purple area represents the slip states on slippery terrains where the norm of tangential contact velocity exceeds 0.4~\si{\meter/\second}. \protect\subref{figure:different_initial_condition_prop} The proposed smoothing is applied. \protect\subref{figure:different_initial_condition_nonsm} The gradient from the nonsmooth model is used~\cite{chen2022real}.}
    \label{figure:different_initial_condition}
\end{figure}
% Please add the following required packages to your document preamble:
% \usepackage{multirow}

% Please add the following required packages to your document preamble:
% \usepackage{multirow}
% \begin{table}[]
% \caption{The results comparing the performance using the proposed smoothing method with the baselines after seven experiments}
% \begin{tabular}{|c|c|cl|cc|}
% \hline
% \multirow{2}{*}{\textbf{Method}}                                            & \multirow{2}{*}{\textbf{\begin{tabular}[c]{@{}c@{}}Average\\ Loss\end{tabular}}} & \multicolumn{2}{c|}{\textbf{Time (s)}}                                   & \multicolumn{2}{c|}{\textbf{\begin{tabular}[c]{@{}c@{}}Estimated \\ Friction Coefficient\end{tabular}}}     \\ \cline{3-6} 
%                                                                             &                                                                                  & \multicolumn{1}{c|}{\textbf{Mean}}   & \multicolumn{1}{c|}{\textbf{Max}} & \multicolumn{1}{c|}{\textbf{Mean}} & \textbf{\begin{tabular}[c]{@{}c@{}}Standard \\ Deviation\end{tabular}} \\ \hline
% \textbf{Proposed}                                                           & \textbf{1.1542}                                                                  & \multicolumn{1}{c|}{0.0247}          & 0.0867                            & \multicolumn{1}{c|}{0.2724}        & 0.0371                                                                 \\ \hline
% \textbf{Exact}                                                              & 6.8542                                                                           & \multicolumn{1}{c|}{\textbf{0.0111}} & \textbf{0.0514}                   & \multicolumn{1}{c|}{0.7373}        & 0.1660                                                                 \\ \hline
% \textbf{\begin{tabular}[c]{@{}c@{}}1st-order\\ Randomized\end{tabular}}   & 3.6515                                                                           & \multicolumn{1}{c|}{0.5679}          & 1.8760                            & \multicolumn{1}{c|}{0.2810}        & 0.0352                                                                 \\ \hline
% \textbf{\begin{tabular}[c]{@{}c@{}}0th-order \\ Randomized\end{tabular}} & 3.9282                                                                           & \multicolumn{1}{c|}{0.6997}          & 2.5681                            & \multicolumn{1}{c|}{0.2737}        & 0.0300                                                                 \\ \hline
% \end{tabular}
% \label{table:total_compare}
% \end{table} 


% \begin{table}[]
% % \caption{Comparison of friction coefficient estimation using proposed smoothing with other smoothing method in terms of cost, time and estimated friction coefficient}
% % \caption{Estimation Results obtained from Experiments of 7 Trials to compare the performance using proposed smoothing with that using baseline.}
% \caption{The results comparing the performance using the proposed smoothing method with the baselines after seven experiments}
% \begin{tabular}{|c|c|cl|cc|}
% \hline
% \multirow{2}{*}{\textbf{Method}}                                         & \multirow{2}{*}{\textbf{\begin{tabular}[c]{@{}c@{}}Average\\ Loss\end{tabular}}} & \multicolumn{2}{c|}{\textbf{Time (s)}}                                    & \multicolumn{2}{c|}{\textbf{\begin{tabular}[c]{@{}c@{}}Estimated \\
%  Friction Coefficient\end{tabular}}}     \\ \cline{3-6} 
%                                                                          &                                                                                  & \multicolumn{1}{c|}{\textbf{Mean}}    & \multicolumn{1}{c|}{\textbf{Max}} & \multicolumn{1}{c|}{\textbf{Mean}} & \textbf{\begin{tabular}[c]{@{}c@{}}Standard \\ Deviation\end{tabular}} \\ \hline
% \textbf{Proposed}                                                        & \textbf{1.247}                                                                  & \multicolumn{1}{c|}{0.03033}          & 0.0811                            & \multicolumn{1}{c|}{0.288}         & 0.032                                                                  \\ \hline
% \textbf{Exact}                                                           & 6.854                                                                           & \multicolumn{1}{c|}{\textbf{0.01371}} & \textbf{0.0655}                   & \multicolumn{1}{c|}{0.753}         & 0.125                                                                  \\ \hline
% \textbf{\begin{tabular}[c]{@{}c@{}}First-order\\ Randomized\end{tabular}}  & 3.6515                                                                           & \multicolumn{1}{c|}{1.3168}           & 3.2193                            & \multicolumn{1}{c|}{0.276}         & 0.028                                                                  \\ \hline
% \textbf{\begin{tabular}[c]{@{}c@{}}Zeroth-order \\ Randomized\end{tabular}} & 3.9282                                                                            & \multicolumn{1}{c|}{1.3741}           & 5.9274                            & \multicolumn{1}{c|}{0.277}         & 0.043                                                                  \\ \hline
% \end{tabular}
% \label{table:total_compare}
% \end{table}

% \begin{table}[]    
% \begin{tabular}{|c|c|c|cc|}
% \hline
% \multirow{2}{*}{\textbf{Method}}                                            & \multirow{2}{*}{\textbf{Cost}} & \multirow{2}{*}{\textbf{Time (s)}} & \multicolumn{2}{c|}{\textbf{Estimated Friction Coefficient}}     \\ \cline{4-5} 
%                                                                             &                                &                                    & \multicolumn{1}{c|}{\textbf{Mean}} & \textbf{Standard Deviation} \\ \hline
% \textbf{Proposed}                                                           & \textbf{1.867}                 & 0.0388                             & \multicolumn{1}{c|}{0.293}         & 0.0620                      \\ \hline
% \textbf{Non-smoothed}                                                              & 10.896                         & \textbf{0.383}                     & \multicolumn{1}{c|}{0.679}         & 0.18490                     \\ \hline
% \textbf{\begin{tabular}[c]{@{}c@{}}First-order\\ Randomized\end{tabular}}   & 4.664                          & 1.887                              & \multicolumn{1}{c|}{0.296}         & 0.03466                     \\ \hline
% \textbf{\begin{tabular}[c]{@{}c@{}}Zeroth-order \\ Randomized\end{tabular}} & 6.412                          & 1.880                              & \multicolumn{1}{c|}{0.277}         & 0.0427                      \\ \hline
% \end{tabular}\label{figure:total_compare}
% \end{table}
% \begin{figure}
%     \centering
%     \includegraphics[width=1.0\columnwidth]{Figures/Experiment/fig14.pdf}
%     \caption{Relation between average loss value and smoothing parameter for an optimization problem. The smoothing parameter allows for finding a better searching direction in optimization problems than without using the smoothing parameter.}
%     \label{fig:loss_smoothing}
% \end{figure}

In this session, we will examine the advantages of the proposed smoothing method in friction coefficient identification. As shown in Fig.~\ref{fig:online_estimation}, when the robot slips on slippery terrain, the proposed smoothing method enables parameter updates towards a low friction coefficient, in contrast to the case of the nonsmooth model. For the slipping case, we compare the average loss of the nonsmooth model with that of the smoothing method in Fig.~\ref{figure:loss_compare}. In the figure, we observed that using the proposed smoothing method can lead to convergence at better local optima, achieving a lower loss value. Specifically, Fig.~\ref{figure:smoothing_compare} shows the average loss during the experiments shown in Fig.~\ref{fig:online_estimation} according to the smoothing parameter. We observed that when the smoothing parameter $\rho_\mathrm{t}$ is excessively increased or decreased, the effect for the convergence towards a lower loss may be reduced, as~\cite{kim2023contactimplicit}.
% We observed that, with the smoothed gradients, the estimated friction coefficient updates are smoothly executed within the same initial range.

Moreover, we conducted friction coefficient identification with various initial conditions in 0.05 units from 0.05 to 1.0, as shown in Fig.~\ref{figure:different_initial_condition}.
In the experiment, we used the same experimental data as that for Fig.~\ref{fig:online_estimation}. We compared the performance of friction coefficient identification between the proposed model and the nonsmooth model on slippery terrain. As shown in Fig.~\ref{figure:different_initial_condition_nonsm}, 
when employing nonsmooth gradients, the lack of informative gradients can lead to the failure to identify the lower friction coefficient. We observed that the issue often occurs as the gap between the estimated friction coefficient and the actual one is large. In contrast, as Fig.~\ref{figure:different_initial_condition_prop}, our proposed smoothing method solves the failure issue of parameter identification, even under various initial conditions.
%\textcolor{blue}{We observed that the proposed gradient allows for improved friction coefficient identification under various initial conditions.}  



Considering the results, we observed that the proposed smoothing method provides advantages for friction coefficient identification under various initial conditions, even when a high initial friction coefficient leads to non-informative gradients. These advantages can be utilized in various model-based frameworks. For instance, model-based controllers for legged robots often employ a user-defined friction coefficient to compute control inputs based on the Coulomb friction cone constraint. The friction coefficient is typically determined by heuristic tuning for their tasks~\cite{jenelten2019legged,hong202realtime}. A high friction coefficient can be selected to optimize control inputs, leveraging more tangential ground reaction forces. However, using a high friction coefficient on slippery terrain may cause the robot to slip, as the control inputs are computed based on a high friction coefficient. Consequently, there is a need for real-time friction coefficient identification that performs fast and consistently on slippery terrain, even with a high initial friction coefficient. The proposed framework can identify the friction coefficient under various initials, handling non-informative gradients.

%\textcolor{blue}{For online system identification, however, the problem arises when the dynamics model does not predict slipping, even if it actually occurs, due to gaps in modeling the friction coefficient. In this case, the contact impulse is not attached to the friction cone and becomes independent of the friction coefficient. This independence also extends to the contact dynamics of the system~\cite{raisim,werling2021fast}. Consequently, using nonsmoothing methods, the gradients can become non-informative, hindering online identification of the friction coefficient.}

%\textcolor{blue}{As shown in Fig.~\ref{figure:different_initial_condition}, when using system identification of the nonsmoothing gradients, the issue of non-informative gradients become predominant when the predicted friction coefficient values are larger than the actual values. Therefore, the friction coefficient identification that works across various initial conditions becomes essential. Unlike gradients without smoothing, our smoothed gradients allow for consistent friction coefficient identification across various initial conditions.}
%We observe that using proposed smoothing significantly reduces non-informative gradients and allows for uninterrupted parameter updates within various initial conditions, compared to nonsmoothed gradients.
%Conversely, when the initial friction coefficient is lower than 0.70, the estimated friction coefficient and convergence rates when using the nonsmoothed gradient are comparable to those using the proposed smoothed gradient, and the non-informative gradient issue becomes negligible. 

% TODO 
% Moreover, Fig.~\ref{fig:loss_smoothing} shows the average loss value computed using all the loss values obtained from the experiments shown in Fig.~\ref{fig:online_estimation}. As illustrated in Fig.~\ref{fig:loss_smoothing}, the average loss value was decreased when the proposed smoothed gradient was applied, allowing stable and smooth parameter updates to decrease the loss function since the nullified gradient issue did not occur. However, the non-smoothed analytic gradient about the friction coefficient led to the zero gradient issue and a higher loss value than the smoothed gradient under certain initialized conditions.

% If the estimated friction coefficient is larger than the friction coefficient in the actual terrain, the modeling gap in contact dynamics is increased when the robot slips.
% As seen in Figure~\ref{figure:different_initial_condition}, 


\subsection{Comparison with Randomized Smoothing Methods}
\begin{figure}
    \centering
    \includegraphics[width=1.0\columnwidth]{Figures/Experiment/Experiment_6.pdf}
    \caption{Compared to the baselines, the proposed methods can achieve fast and consistent friction coefficient identification in real-time.}
    \label{figure:compare_randomize}
\end{figure}
% In this section, we evaluate the effectiveness of the proposed gradients in this paper by comparing the results obtained using proposed smoothed gradients, randomized smoothing, and nonsmoothed gradients, in Fig.~\ref{figure:compare_randomize} and Table~\ref{table:total_compare}. The results shown in Fig.~\ref{figure:compare_randomize} are obtained from the same data as shown in Fig~\ref{fig:online_estimation}. The randomized smoothing utilizes 30 samples with parallel computing to obtain the stochastic gradient. We observe that the estimates using the proposed smoothing method are comparable to those using the randomized smoothing. Moreover, like randomized smoothing, as discussed in~\cite{Pang2023TRO,le2024leveraging}, it is observed that the proposed gradient mitigates the issue of non-informative gradients.


In this section, we compare the performance of friction coefficient identification using the proposed gradients with baseline methods. For the baselines, we adopt the online system identification using nonsmooth gradients~\cite{chen2022real} and using randomized smoothing methods~\cite{le2024leveraging, Pang2023TRO}: specifically first-order and zeroth-order randomized smoothing methods. The randomized smoothing methods utilize 50 samples with parallel computing to obtain stochastic gradients. We conducted seven experiments where the robot slipped on slippery terrains, with initial estimates of 0.8. 

The results are summarized in Fig.~\ref{figure:compare_randomize}, which presents histograms of the estimated friction coefficient, computation time for solving the optimization problem for~\eqref{osi_opt}, and average loss. We observe that the proposed smoothed gradient results in lower computation times than other randomized smoothing methods. As noted in~\cite{Pang2023TRO}, while randomized smoothing methods can address the lack of informative gradients, they require longer computation times due to sampling. Furthermore, it is observed that the mean and standard deviation of estimates without the smoothing method are higher than those using smoothing methods. This can be attributed to the lack of informative gradients, which causes the gradient-based optimization strategy to fail in friction coefficient identification.
%In cases without smoothing, the estimates often remain unchanged or are delayed due to these bad local minima.}
 % It is observed that estimates often remain unchanged from the current values even with increased confidence scores, resulting in a higher mean and standard deviation.
% 스탠다드 배리에이션이 다른 방법들보다 더 높았다. 평탄화하지 않은 기울기를 사용하면 유용그래디언트 부족의 결과를 낳기 때문에, 최적화는 bad local minima에 갇히게 되었다. 

\subsection{The Effects of Data Rejection Method}

% \begin{figure}
%     \centering
% \includegraphics[width=1.0 \columnwidth]{Figures/Experiment/fig78merge_nonslip.pdf}
%     \caption{The results of an additional experiment in which the legged robot traverses nonslippery terrain. The rejection algorithm reduces the drift in estimating the friction coefficient by utilizing a rejection score and stabilizes the parameter updates.}
%     \label{fig:nonslippery_estimation}
% \end{figure}
In this section, we describe the benefits of data rejection methods by comparing updates based on confidence scores with and without rejection methods. In the right bottom figure of Fig.~\ref{fig:online_estimation}, friction coefficient identification with the rejection method is more consistent than without it, especially on nonslippery terrain.

As illustrated in the bottom right figure of Fig.~\ref{fig:contact_vel}, not using the rejection method can lead to an increased confidence score, even on nonslippery terrain. If the confidence score increases on nonslippery terrain, the parameter updates can be conducted using non-informative observations, leading to undesired and inconsistent friction coefficient identification.

However, with the proposed rejection method, the confidence score on nonslippery terrain does not increase as much as it does without the method, allowing for consistent performance in friction coefficient identification. Furthermore, the rejection method does not significantly impede increases in the confidence score when the robot slips. As shown in the bottom-left of Fig.~\ref{fig:contact_vel}, when the robot slips on slippery terrain, the confidence score with the rejection method is comparable to one without the method. The upper figures of Fig.~\ref{fig:contact_vel} show that the data with high contact velocity following contact initiations can be excluded from parameter identification.


%\textcolor{blue}{Consequently, it is observed that combining the confidence score-based updates with the proposed rejection method stabilizes parameter updates, without significantly impeding the increase in confidence scores when the robot slips on slippery terrains.}
% Additionally, the rejection method does not significantly impede the updates for the estimated friction coefficient when the robot slips on slippery terrain. In such cases, the increase in the confidence score with the rejection method is comparable to that without the method.

% Furthermore, it is observed that, for legged robots,  incorporating our proposed rejection method into existing confidence score-based online system identification reduces the drift in friction coefficient estimates and allows for stable friction coefficient estimation. 
% Besides the above experiments, we conducted an additional experiment where the robot only navigated on only nonslippery terrain, as shown in Fig.~\ref{fig:nonslippery_estimation}. Even if the legged robot is traveling on the nonslippery terrain, it is observed that the norm of the tangential contact velocity increased up to 0.842~\si{\meter/\second} at the beginning of contacts, about 19.3~\si{\second}, and the normal contact velocity highly vibrated and increased. As shown in the top figure of Fig.~\ref{fig:nonslippery_estimation}, when these data are included in the optimization problem for system identification, we observe that the parameter optimization process in~\eqref{osi_opt} becomes unstable. Consequently, the estimates tend to drift or diverge. In particular, the estimate increases from the initial value up to 1.0 around 19.8~\si\second and drops from 1.0 to 0.52 around 21~\si\second, although the system is on the nonslip terrain.

% Furthermore, it is observed that, for legged robots,  incorporating our proposed rejection method into existing confidence score-based online system identification reduces the drift in friction coefficient estimates and allows for stable friction coefficient estimation. 


% However, we observe that employing the rejection method stabilizes the optimization process and decreases the drift in estimates. The rejection score increases with higher or fluctuating contact velocity in the normal direction, especially at the beginning of contacts. When the score is over a threshold, the corresponding contact state is excluded from the system identification. Fig.~\ref{fig:nonslippery_estimation} illustrates the contact and sliding states not filtered out after the rejection method. 


%% 

%Also, the rejection score rejects the undesired estimated slip states that typically involve vibrated normal contact velocity.
% The two figures at the bottom in Fig. \ref{fig:online_estimation} show the norm of estimated foot velocity in the normal and tangential direction, including the contact and slip states remaining after the rejection algorithm. The rejection score was assigned to the data for each time step as defined in~\eqref{eq:rejection_score}. When the rejection score exceeded the rejection threshold, the state at the corresponding index was excluded from the online system identification, as shown in Fig.~\ref{fig:online_estimation}.


% The estimated slip state is determined by whether or not the tangential direction foot speed exceeds the slipping threshold of 0.3 as~\cite{Joonha2021RAL}. 
% The second figure at the bottom of Fig.~\ref{fig:online_estimation} represents the norm of the estimated foot velocity in the normal direction and the corresponding rejection score. The figure shows the remaining collected data after the registration algorithm is used to identify the online system. If the rejection score exceeds the rejection threshold, the collected data is excluded from the optimization problem for online system identification in ~\eqref{osi_opt}. The rejection score becomes larger as the vibration of the normal direction of foot speed increases. 



% There is no way to estimate the friction coefficient if the system does not actually slip and does not slip on the model. Nevertheless, 

% The confidence score is increased as the norm of the tangential foot velocity in the contact states is increased. At around time is at 23.5 seconds, the norm of tangential foot velocity gets increased up to about 1 m/s as shown in \ref{fig:online_estimation}, However, it is rejected from being included in online system identification process through rejection algorithm. Therefore, at that moment, the data does not affect the confidence score and parameter update even if the tangential foot velocity gets increased. 

% On the other side, when time is about 23.8 seconds, the norm of tangential direction foot velocity again surge up to 1 $m/s$. 
% At that moment, even if some of data are excluded from online system identification process through the rejection of elastic contact, the remaining data contributes to the parameter updates, increasing confidence score, There is a delay in the rise of the confidence score, which is a delay caused by the time data is accumulated in the buffer and a delay caused by the parameter estimation thread that runs at 10 $Hz$.


% \subsection{Estimation Results}

% \begin{figure}[t]
%     \centering
    
%     \includegraphics[width=0.98\columnwidth]{Figures/Experiment/fig9.pdf}
    
%     \caption{The comparison of friction coefficient estimation in online data. Our proposed method incorporates a smoothed gradient and Hessian, rejection of elastic contact, and updates to the confidence score. Unlike our method, other approaches omit one of these concepts. Our method demonstrates the quickest and most stable system identification compared to others. The non-smoothed gradient remains nullified for about 23.8 seconds, even when other smoothed gradients are active, allowing for parameter updates. The estimated parameter tends to drift without updating the confidence score and using the data rejection algorithm.}
%     \label{fig:cof_est_in_real_experiments}
% \end{figure}

% We proposed the smoothed analytical gradients of contact impulses about the friction coefficient to solve the issue of nullified gradients. In Figure~\ref{figure:different_initial_condition}, various initial condition is set for the comparison of the effects of the modeling gaps in the friction coefficient estimation. The dataset is based on the data that were collected in the online experiments. 

% As shown in the Figure~\ref{figure:different_initial_condition}, the gradient of contact impulse about friction coefficient is nullified due to the modeling gap in the friction coefficient parameter. The nullfied gradients hinder the fast updates for friction coefficients and leads to delay in parameter updates. The greater the quality gap of the modeling parameter, the deeper the nullified gradient issue, and even the parameter may not be updated. 

% The proposed gradients from smoothed conditions tackles this nullified gradient problem, enabling parameter estimation in various initiation conditions.


% \subsection{The Effects of Rejection and Confidence Score}
% Figure \ref{fig:cof_est_in_real_experiments} compares the performance of friction coefficient estimation under different conditions, with and without the inclusion of smoothed gradients, Hessian, rejection of elastic contact, and confidence score, based on the same data that was collected in the online experiments. The proposed method includes all concepts: smoothed contact gradient, contact hessian, data rejection, and confidence score updates. In contrast, the other methods represent friction estimation performances lacking one of these elements. The proposed method was estimated in real-time online as the robot was controlled, while the others regenerated results based on data logged during online experiments but processed offline. The robot starts walking for around 15 seconds, and the actual sliding phase can be identified based on the predicted tangential direction ball velocity shown in \ref{fig:online_estimation}.

% Without smoothed gradients or smoothed Hessian, it leads to the area where the gradient is nullified and the estimated friction coefficient is not updated. This issue causes delays in accurate parameter estimation and fails to guarantee optimal performance in online parameter estimation. In the absence of an elastic-collision rejection algorithm, the parameter gets drifted, which hinders stable parameter estimation.

% Without Confidence Score-based updates, all the optimal friction coefficients could be considered equally weighted. However, contact dynamics, when not sliding, are independent of the friction coefficient. Therefore, the differences between states rolled out through contact dynamics and the actual data in non-sliding conditions are not solely determined by the friction coefficient. In this case, the parameter update is based on the optimized parameter depending only on the noise and bias of the collected data, not frictional sliding events. If the friction coefficient estimated in sliding and non-sliding situations is updated with equal weight, it leads to cumulative estimation errors, causing drift. Implementing confidence score-based updates, which give higher weight to actual sliding cases, can mitigate the effects of drift in friction coefficient estimation as shown in 


% \subsection{The Comparison with Baseline}
% TODO : PLEASE INSERT THE TABLE WHICH DESCRIBES THE EFFECTS OF SMOOTHED CONTACT GRADIENTS.

% \begin{figure}
%     \centering
%     \includegraphics[width=1.0\columnwidth]{Figures/Experiment/fig10.pdf}
%     \caption{The computation time for the proposed online friction coefficient estimation framework. The Boundary of sampling time is 10 Hz. The elapsed time to implement the algorithm is inside the boundary.}
%     \label{fig:computational_time}
% \end{figure}

%\subsection{Computation Time}
%This study also focused on the real-time operability of the proposed online system identification framework. In Fig.~\ref{fig:computational_time}, the logged elapsed time to implement the proposed framework is introduced, corresponding to the results in Fig.~\ref{fig:online_estimation}. The boundary of solve time for the friction coefficient estimation algorithm is set at 10 Hz, and all the computation for the proposed framework is completed within the boundary. The average elapsed time is 55.3 \si{\milli\second} and the maximum is 65.1 \si{\milli\second}.


% \section{Experiment} \label{Sec:Experiment}
% 이 section에서는 우리는 마찰계수 추정 알고리즘을  real quadrupedal robot인 KAIST HOUND에 적용해본 것에 대해 describe합니다. 또한, 실제 실험에서 사용되는 data rejection algorithm과 confidence score의 효과에 대해 설명합니다. 실제 실험 상에서 로봇은 [승우형NMPC]를 기반으로 하는 Nonlinear model predictive controller로 제어가 되며, 80Hz로 run합니다. 로봇의 상태 추정기는 [InEKF랑 미끄럼방지..?]를 기반으로 하고있으며, 1000Hz로 run합니다. 마찰 계수 추정기의 history buffer의 사이즈는 50이며, 0.01s의 간격으로 로봇의 proprioceptive data들을 logging합니다.  마찰 계수 추정을 위한 최적화 문제를 풀기 위하여, A single onboard computer with an Intel(R) Core(TM) i7-11700T CPU @ up to 4.6 GHz가 사용되었습니다.
% 실험은 시뮬레이션과 동일하게 미끄러운 구간과 미끄럽지 않은 구간으로 분류되는 장소를 로봇이 patrol하면서 진행이 됩니다. 미끄러운 구간은 boric acid power를 아크릴 판 위에 뿌림으로써 구현하였습니다. 

% \subsection{Data Rejection}
% \ref{fig:rejection_score}는 예측하는 노멀방향 발 속도의 노름값과 그에 따른 rejection socre값을 나타내며, 이로 인해 rejection되고난 다음의 contact data들을 붉은색 background로 나타낸다. Contact dynamics를 이용하는 파라미터 추정은 접촉이 동반되기 때문에, 실제와의 차이에 큰 영향을 받을 수 있다. \ref{fig:rejection_score}는 실제 Contact이 일어나고 벗어나는 순간에 나타나는 contact 지점 발 추정속도의 z 성분이다. 상태추정기로부터 추정한 현재 상태는 실제 상태와의 차이가 있을 수 있기 때문에 발 속도 추정 및 Contact dynamics를 Forward propagation에 적용하는 것에 영향을 준다. 특히, 미끄러지는 상태에서의 발 추정 및 Contact이 일어나는 시점에서의 발 속도 추정은 실제와의 차이를 야기하는 요소들이며, Contact이 일어났을 때 velocity-based time-stepping scheme기반의 contact dynamics 풀이에서 사용하는 발 속도가 0일 것이라는 가정과는 달리, 실제 발 속도 추정값은 그러하지 못하다. 이러한 데이터들은 co통해ntact dynamics의 가정에 대립하는 특징을 가지고 있기 때문에, 파라미터 추정시, 불안정성과 bias를 야기할 수 있다. 이로 인한 파라미터 추정의 bias와 불안전성을 막기 위하여, data rejection 알고리즘을 사용한다. 

% \subsection{Contact-based Confidence Score}
%\ref{fig:confidence_score}는 예측하는 접선방향 발 속도의 노름값과 그에 따른 confidence score값을 나타내며, 예측한 접선방향 발 속도를 기준으로 미끄러졌다고 판정된 부분들을 파란색 background로 나타낸다. t=23.5s 부근에서 접선방향 발 속도가 threshold를 넘어서 estimated slip state가 true가 되었음에도, \ref{fig:rejection_score}에서 나타난 바와 같이 data rejection을 통해 데이터 버퍼에 포함되는 것이 거부되었기 때문에, confidence score와 파라미터 업데이트에 영향을 주지 않는다. 하지만, t=23.8s 부근에서 접선 방향의 발 속도가 threshold를 넘어서 estimated slip state가 true가 되었고, data rejection을 거쳤음에도, 데이터 버퍼에 포함이 되는 데이터들이 존재하면서, confidence score가 올라가는 것을 확인할 수 있다. 이 때, 데이터 버퍼에 포함되는 시간과 friction coefficient estimation의 추정을 위한 thread가 0.1s마다 돌면서 생기는 delay에 의해, confidence score가 올라가는 시간에 delay가 있음을 확인할 수 있다.

% \subsection{The Friction Coefficient Estimation in Online Data}
%\ref{fig:cof_est_in_real_experiments}는 Online에서 수집한 동일한 Data를 토대로 Smoothing과 Hessian, Data Rejection 그리고 Confidence Score 각각에 대해 포함이 되었을 경우와 아닐 경우에 대한 마찰 계수 추정에 대한 성능을 비교한다. Proposed의 경우, Smoothed contact gradient와 contact hessian, 그리고 data rejection과 confidence score 업데이트 모두를 포함하고 있으며, 나머지는 이러한 요소들이 하나씩 없는 경우에서의 마찰 추정 성능을 나타낸다. Proposed는 로봇이 제어되면서 online으로 실시간 추정되었으며, 나머지는 online 실험에서 로깅된된 데이터를 토대로 offline 상에서 regenerate한 결과들이다. 로봇은 15s경부터 발을 움직이며 걷기 시작하고, 실제 미끄러지기 시작하는 구간은 \ref{fig:confidence_score}의 예측한 접선방향 볼 속도를 기준으로 확인할 수 있다. 
%Smoothing이 없거나 Hessian이 없는 경우, 현재 추정중인 마찰계수와 실제 마찰계수의 차이가 크기 때문에 마찰계수가 특정 구간동안 업데이트가 되지 않는 것을 볼 수 있다. 이러한 경우, 정확한 파라미터의 추정에 딜레이를 야기할 뿐만 아니라, 좋은 파라미터 추정 성능을 보장하지 못한다. Data Rejection이 없는 경우는, 로봇이 미끄러지지 않았음에도 파라미터의 급격한 변화가 일어나는 경우가 생긴다. 이는 상태 추정기와 발 속도 추정에서 생기는 실제와의 오차에 의하거나, 발의 탄성 충돌에 의해 생기는 마찰 계수 추정의 bias에 의한 것이므로, 안정적인 파라미터 추정을 방해한다.
% Confidence score based update가 없다면, 모든 최적의 마찰계수들이 서로 가중치가 동등한 것이라고 볼 수 있다. 하지만, 미끄러지지 않을 때의 contact dynamics는 마찰 계수에 독립적이기 때문에, 미끄러지지 않은 상황에서 contact dynamics를 통해 rollout한 상태들과 실제 data와의 차이는 단순히 cof만으로 결정되는 부분이 아니다. 따라서, 미끄러지지 않는 상황에서 추정된 cof 값은 현재 추정중인 cof에서 약간의 bias를 야기한다. 이로 인해, 미끄러지는 상황과 미끄러지지 않는 상황에서 추정하는 마찰 계수의 값을 동일한 가중치로 업데이트한다면, 추정 오차가 누적되어 drift를 야기한다. 실제 미끄러지는 경우에 대한 가중치를 높이는 confidence score based update를 통해 마찰 계수 추정시 생기는 drift의 효과를 억제할 수 있다.

% \subsection{The Friction Coefficient Estimation in Online Data}
% 본 연구에서 제안하는 Online Friction Coefficient 추정 프레임워크의 실시간성을 보장하기 위해 실제 풀이시간을 검증할 필요가 있다. %\ref{computational_time}는 \ref{fig:cof_est_in_real_experiments}의 Proposed에 해당하는 마찰 계수 추정 알고리즘이 돌아갈 때 로깅된 풀이 시간이다. 마찰 계수 추정 알고리즘을 위한 Boundary of Sampling Time은 10Hz이며, 모두 해당 시간 이내에서 풀리는 것을 확인할 수 있다. 주어진 Solve Time의 평균은 9.9213(ms)이다. 

\section{Conclusion}
\label{subsection:conclusion}
In this paper, we introduce \OURS, a novel framework designed to identify high-quality data that aligns well with the LLM’s learned knowledge to reduce hallucination.
% Our proposed method includes Internal Consistency Probing and Semantic Equivalence Identification, which are designed to separately measure the LLM's understanding of the given instruction and target response.
% In this way, we can measure the familiarity of the LLM with the instruction data and prevent the model from being trained on unfamiliar data, thereby reducing hallucinations.
NOVA includes Internal Consistency Probing and Semantic Equivalence Identification, which are designed to separately measure the LLM's familiarity with the given instruction and target response, then prevent the model from being trained on unfamiliar data, thereby reducing hallucinations.
Lastly, we introduce an expert-aligned reward model, considering characteristics beyond just familiarity to enhance data quality.
By considering data quality and avoiding unfamiliar data, we can use the selected data to effectively align LLMs to follow instructions and hallucinate less in the instruction tuning stage.
Experiments and analysis show the effectiveness of \OURS.

\section*{Limitations}
Although empirical experiments have confirmed the effectiveness of the proposed \OURS, two major limitations remain. 
Firstly, our proposed method requires LLMs to generate multiple responses for the given instruction, which introduces additional execution time.
However, it is worth noting that this additional execution time is used to perform offline data filtering, our proposed method does not introduce additional time overhead in the inference phase.
Additionally, \OURS~is primarily used for single-turn instruction data filtering, thus exploring its application in multi-turn scenarios presents an attractive direction for future research.

%\section*{ACKNOWLEDGMENT}
%This work was supported in part by Korea Evaluation Institute of Industrial Technology (KEIT) funded by the Korea Government (MOTIE) under Grant No.20018216, Development of mobile intelligence SW for autonomous navigation of legged robots in dynamic and atypical environments for real application.


% Reference
\bibliographystyle{ieeetr}
\bibliography{IEEEabrv, reference}

\end{document}