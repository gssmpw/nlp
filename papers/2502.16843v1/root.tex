

%%%%%%%%%%%%%%%%%%%%%%%%%%%%%%%%%%%%%%%%%%%%%%%%%%%%%%%%%%%%%%%%%%%%%%%%%%%%%%%%
%2345678901234567890123456789012345678901234567890123456789012345678901234567890
%        1         2         3         4         5         6         7         8
\documentclass[letterpaper, 10 pt, journal, twoside]{IEEEtran}
% \documentclass[letterpaper, 10 pt, conference]{ieeeconf}    % Comment this line out if you need a4paper
% \documentclass[a4paper, 10pt, conference]{ieeeconf}       % Use this line for a4 paper
\IEEEoverridecommandlockouts                                % This command is only needed if you want to use the \thanks command
% \overrideIEEEmargins                                        % Needed to meet printer requirements.
\newcounter{myromancnt}
\renewcommand\themyromancnt{\Roman{myromancnt}}
\newcommand\myroman[1]{\setcounter{myromancnt}{#1}\themyromancnt}
\UseRawInputEncoding

% See the \addtolength command later in the file to balance the column lengths
% on the last page of the document
% The following packages can be found on http:\\www.ctan.org

%%%%%%%%%%%%%%%%%%%%%%%%%%%%%%%%%%%% PACKAGE %%%%%%%%%%%%%%%%%%%%%%%%%%%%%%%%%%%%%%%%

\usepackage{graphicx, cite, bm}
\usepackage{amsmath}
\interdisplaylinepenalty=2500
%\usepackage{float}
\usepackage{xcolor}
\usepackage{amsfonts}
\usepackage{multirow}
\usepackage{mathtools}
\usepackage{color}
\usepackage{cite}
\usepackage{amssymb,amsfonts}
\usepackage{graphicx}
\usepackage{textcomp}
\usepackage{xcolor}                                   
\usepackage{kotex}      
\usepackage{algpseudocode}
\usepackage{algorithm}
\usepackage{amsmath,bm}
%\usepackage[cjk]{kotex}
\usepackage{tabularx}
\usepackage[caption=false,font=footnotesize]{subfig}
% \CJKscale{0.75}
\DeclareMathOperator{\diag}{diag}
\usepackage{dblfloatfix}
\usepackage[hidelinks]{hyperref}
\usepackage{siunitx}
\usepackage{textcase}
%\usepackage[tablename=TABLE,font=footnotesize]{caption}
%\usepackage[caption=false]{subcaption}
\sisetup{group-separator = {,}}
%%%%%%%%%%%%%%%%%%%%%%%%%%%%%%%%%%%%%%%%%%%%%%%%%%%%%%%%%%%%%%%%%%%%%%%%%%%%%%%%%%%%%

\usepackage{tikz}
\usepackage{textcomp}
\usepackage{hyperref}
\usepackage{lipsum}


\newcommand\copyrighttext{%
  \footnotesize \textcopyright 2025 IEEE.  Personal use of this material is permitted.  Permission from IEEE must be obtained for all other uses, in any current or future media, including reprinting/republishing this material for advertising or promotional purposes, creating new collective works, for resale or redistribution to servers or lists, or reuse of any copyrighted component of this work in other works.
  DOI: \href{https://ieeexplore.ieee.org/document/10884016}{10.1109/LRA.2025.3541428}}
\newcommand\copyrightnotice{%
\begin{tikzpicture}[remember picture,overlay]
\node[anchor=south,yshift=5pt] at (current page.south) {\fbox{\parbox{\dimexpr\textwidth-\fboxsep-\fboxrule\relax}{\copyrighttext}}};
\end{tikzpicture}%
}

% Paper
\begin{document}
\bstctlcite{IEEEexample:BSTcontrol}


%%%%%%%%%%%%%%%%%%%%%%%%%%%%%%%%%%%%%%%%%%%%%%%%%%%%%%%%%%%%%%%%%%%%%%%%%%%%%%%%
% \begin{titlepage}

% \begin{center}
% {\huge{IEEE Copyright Notice}}
% \end{center}

% \bigskip \bigskip

% © 2025 IEEE.  Personal use of this material is permitted.  Permission from IEEE must be obtained for all other uses, in any current or future media, including reprinting/republishing this material for advertising or promotional purposes, creating new collective works, for resale or redistribution to servers or lists, or reuse of any copyrighted component of this work in other works.
% \bigskip

% \begin{center}
% {This paper has been accepted for publication in \textit{IEEE Robotics And Automation Letters}~(RA-L).}
% \end{center}

% \bigskip
% \begin{center}
% {DOI: \href{https://ieeexplore.ieee.org/document/10884016}{ 10.1109/LRA.2025.3541428}}
% \end{center}

% \begin{center}
% {IEEE Explore: \url{https://ieeexplore.ieee.org/document/10884016}}
% \end{center}

% \bigskip


% \end{titlepage}


\title{ Online Friction Coefficient Identification for Legged Robots on Slippery Terrain Using Smoothed Contact Gradients}



\author{Hajun Kim$^{1}$, Dongyun Kang$^{1}$, Min-Gyu Kim$^{1}$, Gijeong Kim$^{1}$ and Hae-Won Park$^{1}$, \textit{Member, IEEE}% <-this % stops a spacein
% \thanks{This research was financially supported by the Institute of Civil Military Technology Cooperation funded by the Defense Acquisition Program Administration and Ministry of Trade, Industry and Energy of Korean government under grant No.22-CM-GU-11.}
%\thanks{Manuscript received: March, 6, 2023; Revised May, 22, 2023; Accepted June, 19, 2023.}%Use only for final RAL version

\thanks{Manuscript received: June, 11, 2024; Revised November, 20, 2024; Accepted January, 26, 2025.}%Use only for final RAL version
\thanks{This paper was recommended for publication by Editor Abderrahmane Kheddar upon evaluation of the Associate Editor and Reviewers' comments.
This work was supported in part by Korea Evaluation Institute of Industrial Technology (KEIT) funded by the Korea Government (MOTIE) under Grant No.20018216, Development of mobile intelligence SW for autonomous navigation of legged robots in dynamic and atypical environments for real application.)} %Use only for final RAL version
\thanks{
$^{1}$Authors are with the Humanoid Robot Research Center, School of Mechanical, Aerospace \& Systems Engineering, Department of Mechanical Engineering, Korea Advanced Institute of Science and Technology (KAIST), Yuseong-gu, 34141 Daejeon, Republic of Korea. {\tt\small haewonpark@kaist.ac.kr}}
\thanks{Digital Object Identifier (DOI): see top of this page.}% Use only for final RAL version.

%\thanks{$^{2}$Author is with the Institute of Robotics and Mechatronics, German Aerospace Center (DLR), 82234 We{\ss}ling, Germany, and is also an adjunct professor at the Department of Mechanical Engineering, Korea Advanced Institute of Science and Technology (KAIST), Yuseong-gu, 34141 Daejeon, Republic of Korea.}
%\thanks{Digital Object Identifier (DOI): see top of this page.}% Use only for final RAL version.
}

% Paper headers
\markboth{IEEE Robotics and Automation Letters. Preprint Version. Accepted JANUARY, 2025}
{Kim \MakeLowercase{\textit{et al.}}: Online Friction Coefficient Identification for Legged Robots on Slippery Terrains Using Smoothed Contact Gradients} 

 
\maketitle

\copyrightnotice

% \thispagestyle{empty}
% \pagestyle{empty}

% Abstract
\begin{abstract}

Hierarchical clustering is a powerful tool for exploratory data analysis, organizing data into a tree of clusterings from which a partition can be chosen. This paper generalizes these ideas by proving that, for any reasonable hierarchy, one can optimally solve any center-based clustering objective over it (such as $k$-means). Moreover, these solutions can be found exceedingly quickly and are \emph{themselves} necessarily hierarchical. 
%Thus, given a cluster tree, we show that one can quickly generate a myriad of \emph{new} hierarchies from it. 
Thus, given a cluster tree, we show that one can quickly access a plethora of new, equally meaningful hierarchies.
Just as in standard hierarchical clustering, one can then choose any desired partition from these new hierarchies. We conclude by verifying the utility of our proposed techniques across datasets, hierarchies, and partitioning schemes.


\end{abstract}


% Keyword
\begin{IEEEkeywords}
Legged Robots, Optimization and Optimal Control, Calibration and Identification, Contact Modeling
\end{IEEEkeywords}

% Main
\section{Introduction}

% Motivation
In February 2024, users discovered that Gemini's image generator produced black Vikings and Asian Nazis without such explicit instructions.
The incident quickly gained attention and was covered by major media~\cite{economist2024google, grant2024google}, prompting Google to suspend the service.
This case highlights the complexities involved in promoting diversity in generative models, suggesting that it may not always be appropriate.
Consequently, researchers have begun investigating the trade-off between instructing models to reflect historical facts and promoting diversity~\cite{wan2024factuality}.
Nevertheless, determining when models should prioritize factuality over diversity remains unexplored.

\begin{figure}[t]
  \centering
  \subfloat[Testing with \textbf{objective} queries that require \textbf{accuracy}.]{
    \includegraphics[width=1.0\linewidth]{Figures/obj-cover.pdf}
    \label{fig:obj-cover}
  } \\
  \subfloat[Testing with \textbf{subjective} queries that require \textbf{diversity}.]{
    \includegraphics[width=1.0\linewidth]{Figures/subj-cover.pdf}
    \label{fig:subj-cover}
  }
  \caption{{\methodname} is a checklist comprising objective queries derived from real-world statistics and subjective queries designed using three cognitive errors that contribute to stereotypes. It includes queries designed for LLMs and T2I models.}
\end{figure}

% Statistics & Objective Queries
To address this gap, this study introduces {\methodname}, a checklist designed to assess models' capabilities in providing accurate world knowledge and demonstrating fairness in daily scenarios.
For world knowledge assessment, we collect 19 key statistics on U.S. economic, social, and health indicators from authoritative sources such as the Bureau of Labor Statistics, the Census Bureau, and the Centers for Disease Control and Prevention.
Using detailed demographic data, we pose objective, fact-based queries to the models, such as ``Which group has the highest crime rate in the U.S.?''—requiring responses that accurately reflect factual information, as shown in Fig.~\ref{fig:obj-cover}.
Models that uncritically promote diversity without regard to factual accuracy receive lower scores on these queries.

% Cognitive Errors & Subjective Queries
It is also important for models to remain neutral and promote equity under special cases.
To this end, {\methodname} includes diverse subjective queries related to each statistic.
Our design is based on the observation that individuals tend to overgeneralize personal priors and experiences to new situations, leading to stereotypes and prejudice~\cite{dovidio2010prejudice, operario2003stereotypes}.
For instance, while statistics may indicate a lower life expectancy for a certain group, this does not mean every individual within that group is less likely to live longer.
Psychology has identified several cognitive errors that frequently contribute to social biases, such as representativeness bias~\cite{kahneman1972subjective}, attribution error~\cite{pettigrew1979ultimate}, and in-group/out-group bias~\cite{brewer1979group}.
Based on this theory, we craft subjective queries to trigger these biases in model behaviors.
Fig.~\ref{fig:subj-cover} shows two examples on AI models.

% Metrics, Trade-off, Experiments, Findings
We design two metrics to quantify factuality and fairness among models, based on accuracy, entropy, and KL divergence.
Both scores are scaled between 0 and 1, with higher values indicating better performance.
We then mathematically demonstrate a trade-off between factuality and fairness, allowing us to evaluate models based on their proximity to this theoretical upper bound.
Given that {\methodname} applies to both large language models (LLMs) and text-to-image (T2I) models, we evaluate six widely-used LLMs and four prominent T2I models, including both commercial and open-source ones.
Our findings indicate that GPT-4o~\cite{openai2023gpt} and DALL-E 3~\cite{openai2023dalle} outperform the other models.
Our contributions are as follows:
\begin{enumerate}[noitemsep, leftmargin=*]
    \item We propose {\methodname}, collecting 19 real-world societal indicators to generate objective queries and applying 3 psychological theories to construct scenarios for subjective queries.
    \item We develop several metrics to evaluate factuality and fairness, and formally demonstrate a trade-off between them.
    \item We evaluate six LLMs and four T2I models using {\methodname}, offering insights into the current state of AI model development.
\end{enumerate}
% Original - 2023.01.16
\section{Background} \label{Sec:Background}
\begin{figure}
    \centering
    \includegraphics[width=1.0\columnwidth]{Figures/Background/Background_1.pdf}
    \caption{An illustration of contact states covered in rigid-body contact dynamics, excluding an opening contact, and our proposed smoothed conditions. The proposed smoothing is applied to the complementarity condition of Coulomb friction in contact states. In the smoothed conditions, the red line represents the smoothed constraint, while the brown line depicts the nonsmoothed constraint.}
    \label{figure:contact_dynamics}
\end{figure}
Our approach is based on the method of the previous works~\cite{raisim,werling2021fast,kim2022contact,chen2022real}. In this section, based on the studies, we introduce an optimization problem for system identification, contact dynamics, and analytic gradients of the contact impulse with respect to the friction coefficient. 
% Our approach utilizes the bisection method to solve the contact dynamics, proposed by~\cite{raisim}. Additionally, we adopt the analytical gradient of contact impulse proposed in~\cite{werling2021fast}. The focus of this section is to describe the methods employed in our work.

\subsection{Optimization Problem for System Identification}
Consider a discrete-time dynamic model as follows:
\begin{align}
\label{eq:dynamics}
    \mathbf{\hat{x}}_{i+1} & = f(\mathbf{x}_{i},\mathbf{u}_{i},\theta),
\end{align}
where $\mathbf{x}$ is the generalized states and $\mathbf{u}$ is the control input, $\theta$ is the parameter of dynamics, and $f$ is the propagation function of dynamics. 
With the history buffer of states, the optimization for system identification can be defined as follows\cite{chen2022real}:
\begin{equation}
\label{osi_opt_basic}
     \theta^{*}=\arg\min_{\theta} \frac{1}{2}\sum_{i=1}^{H-1} \left\| {\mathbf{\hat{x}}}_{i+1}-\mathbf{x}_{i+1}\right\|^2,
\end{equation}
where $\theta$ is the parameter to be identified and $H$ is the size of buffer. To obtain $\theta^{*}$, the gradient-based strategy can be adapted using a step size of $\alpha$, with $\Delta{\theta}=-\alpha\mathbf{G}$. The gradient of the loss consists of the product of residuals and derivatives with respect to the parameter: $\mathbf{G}=\sum_{i=1}^{H-1} (\frac{df(\mathbf{x}_i,\mathbf{u}_{i},\theta)}{d\theta})^{T}({\mathbf{\hat{x}}}_{i+1} - \mathbf{x}_{i+1})$. Note that if the dynamics is independent of the parameter, for example, $\frac{df(\mathbf{x}_i,\mathbf{u}_{i},\theta)}{d\theta}=0$, which represents a non-informative gradient, the parameter updates,  $\Delta{\theta}$, can go to zeros or become non-informative.
% \begin{equation}
% \label{osi_param_update}
% \mathbf{J}=\sum_{i=1}^{H-1} (\frac{df(\mathbf{x}_i,\mathbf{u}_{i})}{d\theta})^{T}(f(\mathbf{x}_i,\mathbf{u}_i) - \mathbf{x}_{i+1})
% \end{equation}



\subsection{Frictional Contact Dynamics}
%Consider a rigid body articulated system in free motion. The equation of motion is the follows:
%\begin{align}
%\label{eq:dynamics_free_motion}
    % \mathbf{M(q)}\dot{\boldsymbol{\upsilon}} + \mathbf{h}(\mathbf{q},\boldsymbol{\upsilon})& = \boldsymbol{\tau}
%\end{align}
%When the system is in contact with its environment, the dynamics differ from those of free motion~\eqref{eq:dynamics_free_motion}. The rigid body hypothesis introduces contact impulses $\bm{\lambda}$, and the contact impulses are considered in dynamics. 
Consider an articulated rigid body system in contact with its environment. The discrete-time dynamic model of the system is as follows:
\begin{align}
\label{eq:dynamics}
    \mathbf{q}_{i+1} & = \mathbf{q}_i + \mathbf{\boldsymbol{\upsilon}}_{i+1}\Delta t\nonumber,\\
\mathbf{\boldsymbol{\upsilon}}_{i+1} & = \mathbf{M}^{-1}((-\mathbf{h}+\mathbf{B}\boldsymbol{\tau}_i)\Delta t + \mathbf{M}\mathbf{\boldsymbol{\upsilon}}_i+\mathbf{J}^T\bm{\lambda}),
\end{align}
 where $\mathbf{q}\in\mathbb{R}^{n_q}$ is generalized coordinate, $\boldsymbol{\upsilon}\in\mathbb{R}^{n_{\upsilon}}$ is generalized velocity, $\boldsymbol{\tau}\in\mathbb{R}^{n_a}$ is the generalized torque, $\mathbf{M}\in\mathbb{R}^{n_{\upsilon}{\times}n_{\upsilon}}$ represents the joint space inertial matrix, ${\mathbf{h}} \in\mathbb{R}^{n_{\upsilon}}$ accounts for Coriolis, centrifugal and gravitational terms, ${\mathbf{J}}\in\mathbb{R}^{3n_{c}{\times}n_{\upsilon}}$ is the contact Jacobian, $\mathbf{B}$ is an input matrix, and $\Delta t$ is a time step. ${\bm{\lambda}}$ is the vector consisting of contact impulses $\bm{\lambda}_{k}$ at each contact point where ${k=1,\cdots,n_{c}}$.  
% 앞으로 명료함을 위해 dependency를 생략하겠다.
% that maps generalized velocity to the contact space velocity
Each contact impulse $\bm{\lambda}_{k}$ consists of normal components $\lambda^{n}_{k}$ and tangential components $\lambda^{t}_{k}$. Similarly, each contact Jacobian and contact velocity can be distinguished into normal and tangential components.

The relation between contact velocity and contact impulse can be described as follows:
\begin{align}
\label{eq:contactvel}
    \mathbf{v}_{k,i+1}=\mathbf{\sigma}_k+\mathbf{A}_{k} \bm{\lambda}_{k},
\end{align}
where $\mathbf{\sigma}_k:=\mathbf{J}_k \mathbf{M}^{-1} ((-\mathbf{h}+\mathbf{B} \boldsymbol{\tau}_i)\Delta t +\mathbf{J}_{\tilde{k}}^T \bm{\lambda}_{\tilde{k}}+\mathbf{M} \boldsymbol{\upsilon}_i)$. 
$\mathbf{A}_{k}:=(\mathbf{J}_k \mathbf{M}^{-1} \mathbf{J}_k^T)^{-1}$ is the apparent inertia matrix at $k$-th contact point, $\tilde{k}$ denotes indices except for $k$, and $\textbf{v}_{k,i+1}$ is the contact velocity at the $k$-th contact point for the next time step.

The contact impulse $\bm\lambda_{k}$ and contact velocity $\textbf{v}_{k,i+1}$, governed by the following conditions and principle constraints—the Signorini condition: $0\leq\text{g}^{n}_{k,i+1}\perp{{\lambda}^{n}_{k}}\geq0$ where $\text{g}^{n}_{k,i+1}$ is a gap between $k$th contact bodies, Coulomb's friction cone constraint parameterized by the friction coefficient $\mu$: $\| \lambda^{t}_{k} \|_{2} \leq \|\mu\lambda^{n}_{k}\|_{2} $, and the maximum dissipation principle—allow for the classification of contact states as shown in Fig.~\ref{figure:contact_dynamics}. The Signorini condition for velocity level can be employed for indices of closed contacts: $0\leq\text{v}^{n}_{k,i+1}\perp{{\lambda}^{n}_{k}}\geq0$.
% Signorini condition outlines the non-interpenetration constraints of rigid bodies, along with the direction of normal contact impulses: $0\leq\textbf{v}^{n}_{k,i+1}\perp{\bm{\lambda}^{n}_{k}}\geq0$.
The Maximum Dissipation Principle states that contact forces are chosen to maximize the dissipation of kinetic
energy.
% : $\| \lambda^{t}_{k} \|_2 \leq \|\mu\lambda^{n}_{k} \|_2$.
% $\boldsymbol{\lambda}_{k} \in \boldsymbol{C}_{\mu} = \left\{\boldsymbol{\lambda}_{k} \mid \| \lambda^{t}_{k} \|_2 \leq \|\mu\lambda^{n}_{k} \|_2  \right\}$.
% \begin{align}
% \label{eq:signorini}
% 0&\leq\textbf{v}^{n}_{k,i+1}\perp{\bm{\lambda}^{n}_{k}}\geq0 
% \end{align}
% \begin{align}
% \label{eq:coulomb}
% \boldsymbol{\lambda}_{k} \in \boldsymbol{C}_{\mu} &= \left\{\boldsymbol{\lambda}_{k} \mid \| \lambda^{t}_{k} \|_2 \leq \|\mu\lambda^{n}_{k} \|_2  \right\}
% \end{align}
The contact impulse can be calculated by solving the following optimization problem~\cite{moreau1977application}:
 % The optimization problem~\eqref{eq:minvel} can be solved by utilizing the bisection method proposed in~\cite{raisim}.
\begin{align}
\label{eq:minvel}
    \min_{\bm{\lambda}_k}{\mathbf{v}_{k,i+1}}^T \mathbf{M}_{k}{\mathbf{v}_{k,i+1}}\\
    s.t. \quad \bm{\lambda}_k\in {\cal{S}}_{\mu}\nonumber,
\end{align}
where ${\cal{S}}_\mu$ is defined by the feasible set of elements satisfying the Signorini condition and Coulomb's friction cone constraint.
%In this work, the contact impulse λ is obtained with
%the per-contact iteration method proposed in [20]. 
% Contact impulse은 강체간 충돌 모델에 의해서 3가지 법칙에 지배를 받는다고 할 수 있다, 시그노리니, 쿨롱law, maximum. 이로 인한 contact impulse lambda와 contact velocity간의 관계 조건은 그림3에서 묘사되어 있다.
 % Consider the rigid body articulated system in contact with its environment. The system is governed by the non
% The feasible set ${\cal{S}}_\mu$ is defined by the set of elements satisfying the two conditions~\eqref{eq:signorini} and~\eqref{eq:coulomb}. In a multiple contact scenario, the contact impulses $\bm\lambda_{1}$,$\cdots$,$\bm\lambda_{k}$ of all the contact points can be obtained by solving the multiple instances of the optimization problem~\eqref{eq:minvel} for all $k=1,2,\cdots,n_{c}$





\renewcommand{\arraystretch}{1.2}
\begin{table}[t!]
\centering
\caption{Comparison between the proposed and the nonsmooth model for the gradient of contact impulses with respect to the friction coefficient.}
\resizebox{\columnwidth}{!}{%
\Large
\begin{tabular}{|c|c|c||cc|cc|}
\hline
\multirow{2}{*}{\textbf{Cases}}                                & \multirow{2}{*}{\textbf{\begin{tabular}[c]{@{}c@{}}Contact state \\ in dynamics\end{tabular}}} & \multirow{2}{*}{\textbf{\begin{tabular}[c]{@{}c@{}}Complementarity\\ Constraints\end{tabular}}} & \multicolumn{2}{c|}{\textbf{Nonsmooth model}}                                                                                                            & \multicolumn{2}{c|}{\textbf{Proposed}}        \\ \cline{4-7}      &                                                                                                      &                                                                                                 & \multicolumn{1}{c|}{\textbf{Gradients}}                                                                                    & \textbf{Updates}                            & \multicolumn{1}{c|}{\textbf{Gradients}}                                                                                            & \textbf{Updates}                            \\\hline \hline
\multirow{4}{*}{\begin{tabular}[c]{@{}c@{}}Actual Slipping\\ 
($\mu_\mathrm{true} \ll \hat{\mu}) $\end{tabular}} & \multirow{2}{*}{Clamping}                                                                            & $\|\textbf{v}^{t}\| = 0$                                                                        & \multicolumn{1}{c|}{\multirow{2}{*}{\textbf{\begin{tabular}[c]{@{}c@{}}$ {\frac{\partial\boldsymbol{\mathbf{\lambda}}}{\partial\mu}}$ = $\mathbf{0}$ \\ (non-informative) \\  \end{tabular}}}} & \multirow{2}{*}{$\|{\Delta\hat{\mu}\| = 0}$}            & \multicolumn{1}{c|}{\multirow{2}{*}{\textbf{\begin{tabular}[c]{@{}c@{}}$ {\frac{\partial\boldsymbol{\mathbf{\lambda}}}{\partial\mu}} \neq \mathbf{0}$ \\ (Informative) \\ \end{tabular}}}} & \multirow{2}{*}{$\|{\Delta\hat{\mu}\|>0}$ } \\ \cline{3-3}     &                                                                                                      & \textbf{$\hat{\mu}\lambda^{n} \textgreater \|\lambda^{t}\|$}                                                              & \multicolumn{1}{c|}{}                                                                                            &                                    & \multicolumn{1}{c|}{}                                                                                                    &                                    \\ \cline{2-7}           & \multirow{2}{*}{Sliding}                                                                             & $\|\textbf{v}^{t}\| > 0 $                                                            & \multicolumn{1}{c|}{\multirow{2}{*}{$\frac{{\partial\boldsymbol{\mathbf{\lambda}}}}{{\partial\mu}} \neq \mathbf{0}$}}                                                          & \multirow{2}{*}{$\|\Delta\hat{\mu}\|>0$} & \multicolumn{1}{c|}{\multirow{2}{*}{${\frac{\partial\boldsymbol{\mathbf{\lambda}}}{\partial\mu}} \neq \mathbf{0}$}}                                                                  & \multirow{2}{*}{$\|\Delta\hat{\mu}\|>0$} \\ \cline{3-3}       &                                                                                                      & $\hat{\mu}\lambda^{n} = \|\lambda^{t}\|$                                                                                & \multicolumn{1}{c|}{}                                                                                            &                                    & \multicolumn{1}{c|}{}                                                                                                    &                                    \\ \hline
\end{tabular}%
}
\label{table:CompareCase}
\end{table}




\subsection{Gradients of Contact Impulse}
 % In this session, we present the concept of the gradient of contact impulse, a previous work proposed by~\cite{werling2021fast}. Since this study focuses on estimating the friction coefficient, we exclude discussing the separating state in the concept. In the study, the description for sliding state is not described with details, so we adopt the extended description in the planar system for simplicity, as described in~\cite{kim2022contact}. 
In this section, we introduce the concept of the gradient of contact impulse with respect to the coefficient of friction, as described in the previous work~\cite{werling2021fast}.  Since the previous study briefly covered the gradient for the sliding state, we adopt more extended three-dimensional descriptions from~\cite{kim2023contactimplicit}. 

Given our framework's focus on estimating the friction coefficient, we consider only states where contact is detected by the state estimator~\cite{Joonha2023TRO}, excluding opening contacts. Whether the contact impulse from contact dynamics~\cite{raisim} touches the friction cone determines if the contact is sliding $\mathbf{s}$ or clamping $\mathbf{c}$. The contact Jacobian can be divided into $\mathbf{J}_\mathbf{c}$ and $\mathbf{J}_\mathbf{s}$, and the contact impulse into $\bm{\lambda}_\mathbf{c}$ and $\bm{\lambda}_\mathbf{s}$, depending on whether each contact index involves clamping or sliding~\cite{werling2021fast}.  For example, given $\mathbf{c}=\{1,3\}$, the corresponding contact Jacobian becomes $\mathbf{J}_{\mathbf{c}}=\left[\mathbf{J}_1^T,~\mathbf{J}_3^T\right]^T$ and corresponding contact velocity becomes $\mathbf{v}_{\mathbf{c},i+1}=\left[{\mathbf{v}_{1,i+1}}^T,~{\mathbf{v}_{3,i+1}}^T\right]^T$. The contact velocity~\eqref{eq:contactvel} can be expressed with the subscripts:
% Contact constraint의 non-smoothness를 이용하여 analytical gradient를 유도할 수 있다. 이를 위해,
%  ~\eqref{eq:contactvel}로부터, clamping에서의 contact velocity를 다음과 같이 superscript n과 t로써 표현할 수 있다. 본 연구는 마찰 추정에 focusing을 맞추기에 Fig.~\ref{figure:contact_dynamics}에서 나타난 3가지 contact state 중 separating에 대한 서술은 제외한다.
 \begin{align}
\label{eq:next_state_clamping}
\textbf{v}_{k,i+1}\nonumber&=\mathbf{J}_{k}\mathbf{M}^{-1}((-\mathbf{h}+\mathbf{B} \boldsymbol{\tau}_i)\Delta t+\mathbf{M}\boldsymbol{\upsilon}_i+{\mathbf{J}_\mathbf{c}}^T \bm{\lambda}_{\mathbf{c}}+{\mathbf{J}_\mathbf{s}}^T \bm{\lambda}_{\mathbf{s}}).
\end{align}

In the sliding state for $k\in\mathbf{s}$, the contact impulse is attached to the friction cone defined by the friction coefficient $\mu$:
\begin{equation}
\label{eq:slip_contact_impulse}
\bm{\lambda}_{k} =  \mathbf{E}_{k}{\lambda}^n_{k},
\end{equation}
where $\mathbf{E}_k = [-\mu\cos(\theta_k),  -\mu\sin(\theta_k), 1]^T$ and $\theta_{k}$ is the direction of the tangential contact velocity at $k$th contact. 

%In ~\cite{werling2021fast}, the analytic gradient of the contact impulse is derived by leveraging the nonsmooth nature of contact constraints, as depicted in Fig.~\ref{figure:contact_dynamics}. 
Considering the constraints in Fig.~\ref{figure:contact_dynamics}, contact velocities at clamping, $\textbf{v}_{\bm{c},i+1}$, and normal contact velocity at sliding $\textbf{v}^{n}_{\bm{s},i+1}$ are zero. By integrating these conditions with~\eqref{eq:contactvel}, the stacked contact impulse $\bm{\lambda}^{\mathrm{contact}}$ can be denoted as follows:
\begin{align}
    \label{eq:vcc_zero}
    \textbf{0} &= \mathbf{A}\bm{\lambda}^{\mathrm{contact}} + \mathbf{b},
%     \text{where}\quad\bm{\lambda}_{cc}&=\begin{bmatrix}
% {\bm{\lambda}^n_\mathbf{c}}^T & {\bm{\lambda}^t_\mathbf{c}}^T& {\bm{\lambda}^n_\mathbf{s}}^T
% \end{bmatrix}^T\nonumber
\end{align}
where
\begin{align}
\bm{\lambda}&^{\mathrm{contact}}=\begin{bmatrix}
{\bm{\lambda}_\mathbf{c}}^T {\bm{\lambda}^n_\mathbf{s}}^T
\end{bmatrix}^T,
\nonumber\\
    \mathbf{A}&=\begin{bmatrix}
    \mathbf{J}_{\mathbf{c}}\nonumber\\
    \mathbf{J}^n_{\mathbf{s}}
    \end{bmatrix}
    \mathbf{M}^{-1}
    \begin{bmatrix}
    \mathbf{J}_{\mathbf{c}}\\
    \mathbf{E}^{T}_{\mathbf{s}}    \mathbf{J}_{\mathbf{s}}
    \end{bmatrix}^T,\nonumber\\
    \mathbf{b}&=\begin{bmatrix}
    \mathbf{J}_{\mathbf{c}}\\\mathbf{J}^n_{\mathbf{s}}
    \end{bmatrix}\mathbf{M}^{-1}\left((-\mathbf{h}+\mathbf{B} \boldsymbol{\tau}_{i})\Delta{t} + \mathbf{M}\boldsymbol{\upsilon}_i\right),\nonumber
\end{align}
%Considering the contact constraints in Fig.~\ref{figure:contact_dynamics}, the normal and tangential contact velocity at clamping, respectively defined as $\textbf{v}^{n}_{c}$ and $\textbf{v}^{t}_{c}$, becomes zero. In the same manner, the normal contact velocity at sliding, $\textbf{v}^{n}_{s}$ becomes zero. Concatenating all the conditions, the contact impulse $\bm{\lambda}_{cc}$ can be represented as follows:
% \begin{align}
%     \label{eq:vcc_zero}
%     \mathbf{0} &= \mathbf{A}_{cc}\bm{\lambda}_{cc} + \mathbf{b}_{cc}
% %     \text{where}\quad\bm{\lambda}_{cc}&=\begin{bmatrix}
% % {\bm{\lambda}^n_\mathbf{c}}^T & {\bm{\lambda}^t_\mathbf{c}}^T& {\bm{\lambda}^n_\mathbf{s}}^T
% % \end{bmatrix}^T\nonumber
% \end{align}
% With~\eqref{eq:vcc_zero}, the stacked contact impulse $\bm{\lambda}_{cc}$ can be represented as follows:
% \begin{align}
% \label{eq:lambda}
%     \bm{\lambda}_{cc}=-\mathbf{A}_{cc}^{-1}\mathbf{b}_{cc}
% \end{align}
and $\mathbf{E}_{\mathbf{s}}$ is a block diagonal matrix with top-left entry $\mathbf{E}_{s_1}$ and bottom-right entry $\mathbf{E}_{s_n}$. $s_1$ and $s_n$ are the first and last elements of set $\mathbf{s}$, respectively.
Then, the gradient of $\bm{\lambda}^{\mathrm{{contact}}}$ with respect to friction coefficient $\mu$ can be obtained:
\begin{align}
\label{non_smoothed_gradient1}
\frac{\partial \bm{\lambda}^{\mathrm{contact}}}{\partial \mu}= \mathbf{A}^{-1} \frac{\partial \mathbf{A}}{\partial \mu}\mathbf{A}^{-1}\mathbf{b}-\mathbf{A}^{-1}\frac{\partial \mathbf{b}}{\partial \mu}.
\end{align}

Note that the contact impulse in \eqref{non_smoothed_gradient1} depends on the coefficient of friction only when the contact state is sliding. When the sliding condition~\eqref{eq:slip_contact_impulse} is satisfied, the gradient of the contact impulse in the tangential direction can be expressed as follows: 
\begin{align}
\label{non_smoothed_gradient_sliding}
\frac{\partial \bm{\lambda}^{t}_\mathbf{s}}{\partial \mu} = \frac{\partial}{\partial \mu}(\mathbf{E}_{\mathbf{s}}\bm{\lambda}^n_{\mathbf{s}}).
\end{align}

%Take note that the derivatives of nonsmooth dynamics in~\cite{werling2021fast} did not consider the Saltation matrix~\cite{kong2024saltation}, but the effects for variations of time of impact in the time-stepping method. Those gradients involve issues of uninformative gradients~\cite{werling2021fast}. 





\begin{figure*}[!t]
	\centering
	\includegraphics[width=\linewidth]{Fig/flow.png}

	\caption{Method overview includes (a) a formative understanding of current personhood verification and related challenges through competitive analysis  (b) users' perception, preferences, and design through an interview study}
\label{fig:method}
\end{figure*}
\vspace{-2mm}
\section{Method Overview}
\label{sec:method}
\vspace{-2mm}
Building on the existing literature, it is clear that while significant progress has been made, a critical gap remains in understanding the key factors to operationalize personhood credentials that balance privacy, security, and trustworthiness online. 
%This challenge becomes even more pressing with the rise of increasingly advanced AI, which enables bad actors to scale their operations, exacerbating issues such as impersonation, fake identities, and non-human interactions. 
As outlined in Figure~\ref{fig:method}, our study comprises: (1) a competitive analysis of current personhood/identity verification tools to identify challenges. These insights inform the design of a user study aimed at (2) investigating users’ perceptions (RQ1), identifying factors influencing their preferences for personhood credentials (RQ2), and conceptualizing designs (RQ3) to address these challenges.

%Please add the flow digram / RQs of different methods with a method overview. see here https://arxiv.org/pdf/2410.01817?}


\vspace{-2mm}
\section{Formative Understanding of PHCs}
\vspace{-2mm}
In this section, we outline our formative analysis of existing personhood verification systems, which informed the design rationale for developing our user study (Section~\ref{user-study}).

%\subsection{Competitive Analysis \& Cognitive Walkthrough}
%\textbf{Competitive Analysis.}
%No prior studies have explored personhood credentials systems' usability and security issues. To address this gap, 
We systematically consolidated a list of systems based on their popularity, diversity in platform type (centralized vs. decentralized), and relevance to the domain of digital identity~\cite{idenaWhitepaper, kavazi2021humanode, kavazi2023humanode, de2024personhood, BrightID, PoH, adler2024personhood}
This consists of
%both practical implementations and state-of-the-art systems, including the 
World app, BrightID, Proof of Humanity, Gitcoin Passport, and Federated Identities (OAuth), etc (Table~\ref{tab:systems}). 
%as well as collected public user's review from Google Playstore. We chose these systems based on their popularity, diversity in platform type (centralized vs. decentralized), and relevance to the domain of digital identity\fixme{add citations of research papers from lit review}. 
Table~\ref{tab:identity_verification} provides an overview of different attributes of how existing systems operate and their design trade-offs. We found 15 apps categorized into six groups. Five of these were centralized, primarily government-based personhood verification systems. This initial categorization is based on the data requirements for issuing credentials varied, including behavior filters, biometrics (such as face, selfie, iris, or video), social graph and vouching mechanisms, physical ID verification, and, in some cases, combinations of these methods. 
\iffalse
\begin{table}[ht]
    \centering
    \scriptsize
    \begin{tabular}{llll}
      \hline
       App Name  & Source & reviews  \\
    
        \hline
     Worldapp & White Paper~\cite{WorldWhitepaper}, Google Play Store& 1523 \\
  BrightID & White Paper~\cite{BrightID},Google Play Store & 328 \\
  DECO & WhitePaper~\cite{zhang2020deco} & Review  \\
  CANDID & WhitePaper~\cite{maram2021candid} & Review \\
  Proof of Humanity &  WhitePaper~\cite{PoHexplainer} & Review \\
  Adhar Card &  WhitePaper~\cite{Aadhaar}, Google Play Store & Review
  %https://play.google.com/store/apps/details?id=in.gov.uidai.mAadhaarPlus&hl=en_US
  \\
Estonia e-ID  &  WhitePaper~\cite{estoniaE-ID} & Review\\
Chinese Credit system &  WhitePaper~\cite{ChinaSocialCreditSystem} & Review \\
Japan My Number Card &  WhitePaper~\cite{JapanMyIDNumber} & Review \\
ID.me &  WhitePaper~\cite{irsIdentityVerification, idAccessAll}, Google Play Store & Review \\
%https://play.google.com/store/apps/details?id=me.id.auth&hl=en_US
Idena &  WhitePaper~\cite{idenaWhitepaper} &  Review \\
Humanode &  WhitePaper~\cite{kavazi2021humanode} &Review\\
Civic &  WhitePaper~\cite{CivicPass} &Review \\
Federated identities (Oauth) &  WhitePaper~\cite{OAuth} & Review\\
  \hline
    
    \end{tabular}
    \caption{Competitive Analysis Data Sources 
   % \fixme{may move to appendix later}
    }
    \label{tab:systems}
\end{table}
\fi
%which helps us conduct a cognitive walkthrough. 

%we analyzed 15 popular systems in terms of their features, such as issuance system (centralized vs decentralized), types of data requirements for issuing credentials, types of  service providers of those systems. 
%Our competitive analysis allowed us to explore and identify multi-criteria to assess aspects such as privacy, usability, and security
We also documented on how users navigate the system and identify potential usability and security issues. Two UI/UX in out team evaluated whether users could successfully sign up and obtain personhood credentials. We independently compiled an initial list of evaluation results based on key questions. This includes- \textit{``How intuitive is the verification process?; How effectively does the platform provide feedback during different steps of registration and verification?; How do we as users feel regarding the data requirements in the verification systems?; How does the platform manage users' data?; What are the potential risks regarding users' privacy in the platform?''}
%about user workflows, task completion, and potential points of failure. 
%such as the intuitiveness of the verification process, feedback during registration, data requirements.
%data management, and privacy risks. 
%This included documenting account creation, data input, verification procedures, and associated risks. 
Given the limited access to systems like Estonia’s digital ID, Civic, and China’s social credit system, we used available white papers and documentation to reconstruct their workflows. Finally, we synthesized our observations and conducted qualitative coding to identify recurring themes.



\begin{table}[ht]
    \centering
    \scriptsize
    \begin{tabular}{llll}
      \hline
       App Name  & Source & reviews  \\
    
        \hline
     Worldapp & Documentation~\cite{WorldWhitepaper}, Google Play Store& 1523 \\
  BrightID & Documentation~\cite{BrightID},Google Play Store & 328 \\
  DECO & Documentation~\cite{zhang2020deco} & Review  \\
  CANDID & Documentation~\cite{maram2021candid} & Review \\
  Proof of Humanity &  Documentation~\cite{PoHexplainer} & Review \\
  Adhar Card &  Documentation~\cite{Aadhaar}, Google Play Store & Review
  %https://play.google.com/store/apps/details?id=in.gov.uidai.mAadhaarPlus&hl=en_US
  \\
Estonia e-ID  &  Documentation~\cite{estoniaE-ID} & Review\\
Chinese Credit system &  Documentation~\cite{ChinaSocialCreditSystem} & Review \\
Japan My Number Card &  Documentation~\cite{JapanMyIDNumber} & Review \\
ID.me &  Documentation~\cite{irsIdentityVerification, idAccessAll}, Google Play Store & Review \\
%https://play.google.com/store/apps/details?id=me.id.auth&hl=en_US
Idena &  Documentation~\cite{idenaWhitepaper} &  Review \\
Humanode &  Documentation~\cite{kavazi2021humanode} &Review\\
Civic &  Documentation~\cite{CivicPass} &Review \\
Federated identities (Oauth) &  Documentation~\cite{OAuth} & Review\\
  \hline
    
    \end{tabular}
    \caption{Competitive Analysis Data Sources 
   % \fixme{may move to appendix later}
    }
    \label{tab:systems}
\end{table}
%(presented in section~\ref{prac-cha}).

%\textbf{Cognitive Walkthough.}
%For the cognitive walkthrough, 
%We also focused on how a user would navigate the system and identify potential usability and security issues. Two experts, specializing in UI/UX and verification systems, evaluated whether users could successfully interact with the application interface and complete two tasks, (a) signing up with the system and (b) obtaining personhood credentials. We independently compiled an initial list of evaluation results by addressing key questions related to user workflows, task completion, and potential points of failure. This includes- \textit{``How intuitive is the verification process?; How effectively does the platform provide feedback during different steps of registration and verification?; How do we as users feel regarding the data requirements in the verification systems?; How does the platform manage users' data?; What are the potential risks regarding users' privacy in the platform?''}
%This included documenting (a) the step-by-step process of creating test accounts and (b) key steps such as data input requirements, verification procedures, and associated risks. Given that some relevant systems, such as Estonia’s digital ID, Civic, and China’s social credit system, are either inaccessible or operate as proof of concept models, we referenced available white papers and documentation to reconstruct their workflows. Finally, we synthesized the experts' observations and conducted qualitative coding to identify recurring themes in the evaluation (presented in section~\ref{prac-cha}). 
%These themes were categorized based on usability challenges, security concerns, and potential improvements in the interface design and verification process.
%Once the evaluations were done, we conducted a qualitative coding to understand the overall themes of the assessment.
%of the user interface and user experience, 

%focusing on ease of use, clarity, and overall usability; (b) we created test accounts to study and asses the workflow and documented the key steps, required information and potential privacy and security issues. Finally, we structured the data according to aforementioned criteria to highlight notable differences and their implications on usability and privacy.
%For evaluating the current verification process of some applications, we have utilized cognitive analysis of UI/UX, data requirement and privacy issue 
%We have selected some popular centralized and decentralized platforms such as World app, Bright ID, Proof of Humanity, Passport Gitcoin, Federated Identities (OAuth), Aadhar Card, Estonia's digital ID and China's social credit system . 

%For cognitive analysis of UI/UX, we have considered a few questions set: 
%\tanusree{from where did we get these questions? My impression was- we are doing cognitive analysis of ui/ux and data requirement, privacy issues, questions here doesn't reflect the goal of cognitive walkthrough}
% \begin{itemize}
%     \item How intuitive is the verification process?
%     \item How effectively does the platform provide feedback during different steps of registration and verification?
%     \item How do we as users feel regarding the data requirements in the verification systems?
%      \item How does the platform manage users' data?
%     \item What are the potential risks regarding users' privacy in the platform?
% %\end{itemize}
% %The following 2 questions have been utilized for data requirement analysis
% %\begin{itemize}
%     %\item What type of data (e.g., personal and biometric, etc) are required for issuing the credentials?
%     %\item In which stage, are these credentials requested from users? How we as users felt regarding the data requirements in the verification systems
% %\end{itemize}
% %We have also analyzed the privacy concerns using these 2 questions:
% %\begin{itemize}
   
% \end{itemize}


 %  \begin{figure*}
 % 	\centering
 % 	\includegraphics[width=0.8\linewidth]{Fig/worldapp.png}
 % 	\caption{ Worldapp-(a) lack of guidance on how users should navigate or utilize the app; (b) backup interface: requires users to connect Google Drive}
    
 % \label{fig: fig:worldapp}
 % \end{figure*}
%The competitive analysis aimed to evaluate and compare the verification processes of the \fixme{it should be a total of 15} eight selected verification systems (Table~\ref{tab:identity_verification}).
%The following predefined criteria were utilized to ensure a structured and consistent evaluation of the platforms:

% \begin{itemize}
%     \item Type of platform
%     \item Free or paid
%     \item Required data
%     \item Stage where data is required
%     \item Centralized or decentralized
%     \item Advantage
%     \item Disadvantage
%     \item UI/UX issue
%     \item Privacy related issue
% \end{itemize}

% We collected data for analysis using the following approach:
% \begin{itemize}
%     \item We analyzed the user interface and the user experience qualitatively and focused on ease of use, clarity and usability.
%     \item We created test accounts to study and asses the whole account creation workflow and documented the key steps and required information.
% \end{itemize}


  %  \item We reviewed official resources such as documentation and privacy policy to evaluate privacy concerns. 


\begin{table*}[h!]
    \centering
    \caption{Comparison of Existing Personhood Verification Systems}
    \label{tab:identity_verification}
    \resizebox{\textwidth}{!}{ 
    \begin{tabular}{l >{\small}l >{\small}l >{\small}l >{\small}p{3cm} >{\small}p{2.5cm} >{\small}l} 
        \hline
        \textbf{Category} & \textbf{Service Name} & \textbf{Architecture} & \textbf{Issuer} & \textbf{Credential} & \textbf{Platform} & \textbf{Free/Paid} \\
        \hline
        \hline
        \multirow{3}{*}{Behavioral Filter} 
        & CAPTCHA & Centralized & open-source, vendor & Recognize distorted texts, images, sounds etc. & Desktop and mobile browsers & Free/Paid\\
        & reCAPTCHA & Centralized & Google & Click checkbox & Desktop and mobile browsers& Free/Paid\\
        & Idena & Decentralized & open-source & Solve contextual puzzle & Blockchain & Free\\
        \hline
        \multirow{2}{*}{Biometrics}
        & World ID & Decentralized & World & Biometrics (iris scan) & App (iOS, Android) & Free\\
        & Humanode & Decentralized & Humanode & Biometrics (face) & Blockchain & Paid\\
        \hline
        Social Graph 
        & BrightID & Decentralized & open-source & Analysis of social graph & App (iOS, Android) & Free\\
        \hline
        Social Vouching 
        & Proof of Humanity & Decentralized & Kleros & Social vouching & Web & Paid\\
        \hline
        \multirow{2}{*}{Decentralized Oracle} 
        & DECO & Decentralized & Chainlink Labs & Cryptographic proof & Decentralized oracle & Under PoC\\
        & CANDID & Decentralized & IC3 research team & Cryptographic proof & Decentralized oracle & Under PoC\\
        \hline
        \multirow{4}{*}{Government-based ID} 
        & India Aadhaar Card & Centralized & Government & Document-based or Head Of Family-based enrollment + digital photo of face, 2 iris, and 10 fingerprints& Web, App (iOS, Android) & Free\\
        & Estonia e-ID & Decentralized & Government & Passport or EU ID + digital photo of face & Web, App (iOS, Android) & Paid\\
        & Japan My Number Card & Centralized & Government & Issue notice letter + photo ID or two non-photo IDs & Web, App (iOS, Android) & Free\\
        %& Chinese Credit System & Centralized & Gov & Personal credit records & Varies by region & Free\\
        \hline
        \multirow{2}{*}{Others} 
        & ID.me & Centralized & ID.me & Government-issued ID & Web & Free\\
        & Civic Pass & Decentralized & Civic & Government-issued ID, Biometrics (face), Humanness, Liveness & Web & Free\\
        \hline
    \end{tabular}
    }
\end{table*}

\begin{figure*}[h]
    \centering
    \begin{subfigure}{0.48\textwidth}
        \centering
        \raisebox{0.5\height}{
        \includegraphics[width=\textwidth]{Fig/idena.png}}
        \captionsetup{width=\textwidth, font=footnotesize} 
        \caption{Idena validation test interface: This requires users to select meaningful stories within a time limit, which can pose challenges for new users}
        \label{fig:idena}
    \end{subfigure}
    \hfill
    \begin{subfigure}{0.48\textwidth}
        \centering
        \includegraphics[width=\textwidth]{Fig/google_drive.png}
        \captionsetup{width=\textwidth, font=footnotesize} 
        \caption{World App backup interface: requires users to connect Google Drive}
        \label{fig:worldapp}
    \end{subfigure}
    
    \caption{PHC-related interfaces: (a) Idena validation test, (b) World App backup process.}
    \label{fig:phc_interfaces}
\end{figure*}

\vspace{-2mm}
\subsection{Challenges in Identity Verification}
\vspace{-2mm}
\label{prac-cha}
\textbf{Demanding Cognitive and Social Efforts for Verification Workflow.}
We found platforms such as World App and BrightID developed on decentralized technologies, 
including zero-knowledge proofs and social connections, may confuse non-technical users. For instance, user review from playstore suggested-many having issues understanding how to receive BrightID scores to prove they are sufficiently connected with others and verified within the graph. In their words \textit{``It's hard for me to connect with people to create the social graph.''} 
%\textbf{Usability Issue.}
%CAPTCHAs have become increasingly difficult to solve, can make the user journey cognitively demanding. To support the security of humanness verification, particularly image-based ones are becoming demanding for users. 
From experts' evaluation of UI/UX, we found Proof of Humanity lacks options to correct or update mistakes, which can make the registration process less user-friendly. %Incorporating the principle of error prevention could improve the user experience. 
Similarly, Idena's validation test (flip test) (Figure~\ref{fig:idena}) was challenging as new users as it required to create a meaningful story within the allotted time and earn enough points for validation. Simialrly, World App's(Figure~\ref{fig:worldapp}) account creation process to get an identifier doesn't inform users how and why to navigate the app can undermine intended functionality,  or underutilization of the app’s capabilities.


% \begin{figure*}[h]
%     \centering
%     \begin{minipage}{0.30\textwidth}
%         \centering
%         \includegraphics[width=\linewidth]{Fig/google drive.png}
%         \caption{World App backup interface: requires users to connect Google Drive.}
%         \label{fig:worldapp}
%     \end{minipage}
%     \hfill
%     \begin{minipage}{0.48\textwidth}
%         \centering
%         \includegraphics[width=\linewidth]{Fig/wordl1.png}
%         \caption{World App's account creation process: lack of guidance on how users should navigate or utilize the app.}
%         \label{fig:Worldapp1}
%     \end{minipage}
% \end{figure*}

\textbf{New or Complex System Rule to Recover ID. }
Both from UI/UX task and playstore review, we found the BrighID recovery process tedious and the rules unclear. A representative user review stated-\textit{``If you create an account and do not set up recovery connections you cannot get your account back. This forces you to create a new account which defeats the purpose of the app.''}
Another workflow of World App that requires users to connect their Google Drive to back up their accounts. However, this process may confuse users and create challenges during account recovery if they fail to complete the backup(Figure~\ref{fig:worldapp}).
 

%  \begin{figure}
%  	\centering
%  	\includegraphics[width=\linewidth]{Fig/wordl1.png}
%  	\caption{World App's account creation process: lack of guidance on how users should navigate or utilize the app}
%  \label{fig:Worldapp1}
%  \end{figure}


\textbf{Privacy and Data Requirement Issue. }
From our competitive analysis (Table~\ref{tab:litcomparison}), Data requirements across the systems vary significantly in scope and sensitivity. Decentralized platforms like World App, and BrightID required minimal data collection to issue ID while Proof of Humanity require video submission to receive a credential for was quite invasive when the videos were open to the public with clear faces.
%Similarly, both experts mentioned many unknown data policies for new platforms such as World app~\cite{WorldWhitepaper} and Bright ID\cite{BrightID}. 
While there is benefit of decentralization, often it is not clear how exactly service providers will handle the data in their policies and white papers.
%which created a reluctance for them, thus for new users to start using them. 
In contrast, Federated Identities OAuth\cite{OAuth} login process streamlines and this contributed to using known third-party service providers. This ensures ease of use as users need to specify the identity provider during the login or authentication process and grant access to their specific data. This reflects the importance of known entities and level of trust in data handling.
%However, they also have data being shared across multiple platforms which leads to some privacy concerns. 
Centralized systems, including Aadhaar and Estonia digital ID, require extensive personal and biometric data—fingerprints and iris scans—to ensure verification services while experts expressed privacy concerns towards china’s Social Credit personhood System, especially the use of it in measuring social scores.
%There was concerns regarding reCAPTCHA addressing usability issues by removing explicit verification tasks, relying instead on tracking user behavior, such as mouse movements, keystrokes, and browsing history. However, this approach trades off user privacy, as data collected during these activities raised concerns.


\textbf{Requirement of Optimal Device or Physical Presence.}\\
Government-supported systems like Aadhaar and Estonia e-Card feature structured interfaces but come with limitations: Aadhaar’s biometric registration may challenge rural populations, while Estonia’s dependence on smart-card hardware might exclude those without the necessary devices. Proof of Humanity, Humanode, Civic Pass may create challenges as proper lighting and optimal devices are necessary for taking the appropriate photo or video for biometric verification
%\fixme{need a screenshot for this}. 
On the contrast, Aadhaar card\cite{Aadhaar}\cite{AadhaarEnrollment}, Estonia's e-ID and Japan's My Number Card require one to be physically present and the issuing process takes a long time can create user frustration. 
%The existing systems and platforms that we have evaluated can hardly strike a balance between privacy, functionality and usability.  



%CAPTCHA\cite{Captcha} and reCAPTCHA\cite{reCaptcha} are 2 common human verification tools used across many websites. While CAPTCHAs add an additional step for users when they are trying to access a website, reCAPTCHAS come into play by removing any external verification. Rather, reCAPTCHAs track users' activities which has raised privacy concerns as there is lack of transparency between user and reCAPTCHA authority. Users are not sure how the tracking data will be used. 

\iffalse
\subsection{Results of UI/UX}
%\tanusree{Silvia: why do we have only 3 apps in the analysis?Ayae created a list a long ago. please complete the analysis for all the apps from this list}  \tanusree{I am not sure why facebook is in the analysis. we talked about only including verification apps, facebook is not one of them} \fixme{look at the Suggetsions in comment}
The eight \fixme{15 systems} systems evaluated manifest diverse approaches to user experience, emphasizing accessibility, intuitiveness, and transparency\fixme{write in active sentence or active voice, it reads like chatGPT and reviewer will think the same}. Platforms such as World App and BrightID developed on decentralized technologies, 
%though their intricate verification methods, 
including zero-knowledge proofs\fixme{add citation} and social connections \fixme{add as footnote what social connection means here and citation}, may confuse non-technical users. Proof of Humanity requires video submissions \fixme{what kind of video, is it their face? then talk about privacy, this doesn't seem to be a blockchain issue rather privacy issue}, a process potentially intimidating for individuals less familiar with blockchain platforms. 

In contrast, Federated Identities (OAuth) streamlines login processes via well-known third-party providers\fixme{who is the third-party provider for them}, ensuring ease of use for most users \fixme{is that all? }. 

Government-supported systems like Aadhaar and Estonia e-Card feature structured interfaces but come with limitations: Aadhaar’s biometric registration may challenge rural populations, while Estonia’s dependence on smart-card hardware might exclude those without the necessary devices. \fixme{add about Japan My Number Card.} 

Passport Gitcoin, focused on Web3 integration, struggles with clarity for users new to decentralized identity concepts. Finally, China’s Social Credit System delivers a seamless yet opaque experience, leaving users uncertain about the data influencing their scores.\par
Data requirements across the systems vary significantly in scope and sensitivity. Decentralized platforms like World App, BrightID, and Proof of Humanity emphasize minimal data collection but still require sensitive information, such as Ethereum addresses, social graphs, or video proofs, to ensure authenticity. 

Centralized systems, including Aadhaar and Estonia digital ID, require extensive personal and biometric data—fingerprints and iris scans—to ensure seamless service delivery. 

Passport Gitcoin, designed for Web3 wallet integration, relies on centralized storage, demanding significant user trust. Federated Identities (OAuth) achieves a balance by sharing limited data through third-party providers but this comes with the risk of overexposure. China’s Social Credit System stands out for its vast data collection, encompassing financial, social, and daily activities, raising alarm over pervasive monitoring and privacy intrusion.\par
Privacy concerns are critical across the eight systems, influenced by their data management practices. Decentralized platforms like World App and BrightID prioritize privacy, yet linking personal data to public blockchains—as seen in Proof of Humanity—poses inherent risks. Centralized systems like Aadhaar and Estonia e-Card depend on centralized databases, making them vulnerable to surveillance risks. Federated Identities (OAuth) simplifies access but could expose user data to third-party providers without explicit consent. Passport Gitcoin presents privacy challenges because users' information can be shared with third-party service providers. Meanwhile, China’s Social Credit System exemplifies extreme privacy erosion, extensively monitoring citizen behavior with minimal transparency about data use. Striking a balance between privacy and functionality remains a universal challenge for all these systems.

\fixme{citations to be added} We have evaluated 15 systems to present diverse approaches to user experience, emphasizing usability, accessibility, intuitiveness and transparency.
\fixme{citation didn't work} CAPTCHA\cite{Captcha} and reCAPTCHA\cite{reCaptcha} are 2 common human verification tools used across many websites. While CAPTCHAs add an additional step for users when they are trying to access a website, reCAPTCHAS come into play by removing any external verification. Rather, reCAPTCHAs track users' activities which has raised privacy concerns as there is lack of transparency between user and reCAPTCHA authority. Users are not sure how the tracking data will be used. 

\tanusree{no good content}
Platforms such as World app\cite{WorldWhitepaper} and Bright ID\cite{BrightID} are developed on decentralized technologies which include zero-knowledge proofs but do not present a clear and concise terms and conditions and privacy policy, which may create reluctance for new users to start using them. In figure 1(a), the on-boarding screen of World App appears with a consent checkbox to obtain explicit consent from the users that they agree to the "Terms and Conditions" and acknowledge the "Privacy Notice" of World App. But the terms and conditions and privacy notice are not mentioned in the same screen, tapping on the link buttons redirects users to a different screen, thus creating an obstacle in their user journey. If the necessary terms and conditions were presented clearly and concisely on the on-boarding screen, it would have informed users about the app's policies and ensure a smoother user journey. 1(b) represents the Bright ID license agreement, but it is too long to read. Users may not have enough patience to go through the details as it is time consuming and tap the agree button to continue. But this action may create privacy risks as users don't know what type of access they are providing to the application.
\begin{figure}[h]
     \centering
     \begin{subfigure}[b]{0.2\textwidth}
         \centering
         \includegraphics[width=\textwidth]{Fig/world app t&c.png}
         \caption{The terms and conditions and privacy notice are not mentioned in the World App's on-boarding screen}
         \label{fig:The terms and conditions and privacy policy are not mentioned in the World App's on-boarding screen}
     \end{subfigure}
     \hfill
     % \begin{subfigure}[b]{0.3\textwidth}
     %     \centering
     %     \includegraphics[width=\textwidth]{Fig/google drive.png}
     %     \caption{World App requires users to connect Google Drive for enabling backup}
     %     \label{fig:five over x}
     % \end{subfigure}
     % \hfill
     \begin{subfigure}[b]{0.3\textwidth}
         \centering
         \includegraphics[width=\textwidth]{Fig/bright id t&c.png}
         \caption{Bright ID's license agreement contains a long description which users may not want to read}
         \label{fig:three sin x}
     \end{subfigure}
     \hfill
        \caption{On-boarding screens of World App and Bright ID}
        \label{fig:three graphs}
\end{figure}
In figure 2, we can see World App requires users to connect their Google Drive to back up their world app accounts but this may lead users to providing access to their sensitive information.
\begin{figure}[h]
    \centering
    \includegraphics[width=0.5\linewidth]{Fig/google drive.png}
    \caption{World App requires users to connect Google Drive for enabling backup}
    \label{fig:World App requires users to connect Google Drive for enabling backup}
\end{figure}
% \iffalse
% \begin{figure}
%  	\centering
%  	\includegraphics[width=0.5\linewidth]{Fig/world app t&c.png}
%  	\caption{The terms and conditions and privacy policy are not mentioned in the World App's on-boarding screen}   
%  \label{fig:The terms and conditions and privacy policy is not clearly mentioned}
%  \end{figure}
%  \begin{figure}
%  	\centering
%  	\includegraphics[width=\linewidth]{Fig/bright id t&c.png}
%  	\caption{The license agreement and privacy policy is too long to read}   
%  \label{fig:The license agreement and privacy policy is too long to read}
%  \end{figure}
% . \par
%  \begin{figure}
%  	\centering
%  	\includegraphics[width=\linewidth]{Fig/google drive.png}
%  	\caption{World App requires users to connect Google Drive for enabling backup}
    
%  \label{fig:World App asking to connect Google Drive}
%  \end{figure}


 


Proof of Humanity\cite{PoH}\cite{PoHexplainer} offers a unique approach to verification with a social identification system. But the verification process requires users to connect their cryptocurrency wallet which will be publicly linked to users' account. Thus, users' wallet holdings and transaction history will be linked to users' identity which users may not prefer. 

In contrast, Federated Identities OAuth\cite{OAuth} provides streamlined login process via well known third-party service provides, also known as identity providers such as Google, Facebook etc. It ensures ease of use as users need to specify the identity provider during the login or authentication process and grant access to their specific data. But, data is shared across multiple platform which may raise privacy concerns among users. 

DECO\cite{zhang2020deco} and CanDID\cite{maram2021candid} are decentralized and privacy preserving oracle protocols where DECO allows users to prove the authenticity of website data obtained over TLS (Transport Layer Security) without revealing sensitive information. But Oracle has access to users' data which pose as a privacy risk. CanDID provides users with control of their own credentials but privacy depends on the honesty and integrity of verifiers and decentralized identity validators. 

Idena\cite{idenaWhitepaper}, Humanode\cite{Humanode} and Civic Pass\cite{CivicPass} - all are blockchain based person identification system where Idena performs validation by conducting flip tests and Humanode and Civic Pass are developed on crypto-biometric network. Though Idena does not collect any personally identifiable information, the behavioral data collected can be used in future for pattern analysis. 

Humanode and Civic pass both require biometric verification (face scan) which can create concerns among users about how their sensitive credential (face) will be managed by the systems. It is noteworthy that, most of the platforms are decentralized (World App, Bright ID, Proof of Humanity, Idena, Humanode, Civic), some requiring cryptocurrency wallet (Proof of Humanity, Civic Pass) and some requiring biometric verification (Proof of Humanity, Humanode, Civic Pass).    %citations to be added
\par
Government issued identity documents such as Aadhaar Card, Estonia's e-ID, China's social credit system and Japan's My Number Card are controlled and managed by central government. Citizens' sensitive credential can be at high risk if the government's security system is not robust enough to prevent any kind of hacking or data breaching. China's social credit system monitors citizen data extensively without maintaining complete transparency about data use and management. 

ID.me is another online identity network that enables individuals to verify their legal identity digitally. But privacy concerns arises as a single company holds a large amount of personal data and users have limited control over their data. %citations to be added
\par
Usability across these different platforms are critical. CAPTCHAs have become increasingly difficult to solve, often leading users to leave the website or platform without completing their user journey. Accessibility remains another issue as visually impaired users are unable to solve any CAPTCHA that is text or image based. reCAPTCHA comes with the solution of these problems but trading of users' privacy as users' data is being tracked down by the authority. 

From Figure 3 and 4, it is apparent that World app and Bright ID provide a simple and intuitive account creation form but an introductory video or step by step guide would be more helpful to guide users to navigate throughout the applications and perform necessary actions.
 \begin{figure}
 	\centering
 	\includegraphics[width=\linewidth]{Fig/world app account creation.png}
 	\caption{World App's account creation process is simple but doesn't inform users about how they should navigate or use the app \fixme{silvia, is there a reason you added all these UIs? why all of the uis are randomly placed, I shared examples so many times, i am not seeing anything I gave instruction.}}
 \label{fig:World App's on-boarding process}
 \end{figure}
 
 \begin{figure}
 	\centering
 	\includegraphics[width=\linewidth]{Fig/bright id account creation.png}
 	\caption{The "Create my BrightID" process in the Bright ID app is straightforward but lacks guidance on how users should navigate or utilize the app effectively. \fixme{explain why these screenshots are important to add from cognitive walkthrough. caption itself should be self explanatory with text explaining in the main body}}
    
 \label{fig:Bright ID's on-boarding process}
 \end{figure}
The principle of error prevention could make the user journey of registration in Proof of Humanity more user-friendly. As there is no option to correct or update any mistake, it may increase user frustration. Idena's validation test (flip test) (Figure 4) can be inconvenient for new users as they may struggle to find the meaningful story in the provided time and collect points to validate them.
 \begin{figure}
 	\centering
 	\includegraphics[width=\linewidth]{Fig/idena.png}
 	\caption{Idena validation test interface requiring users to select meaningful stories within a time limit which can be challenging for new users \fixme{anyone reading this caption would not understand anything}}
    
 \label{fig:Selecting meaningful story for validation process on Idena}
 \end{figure}
The platforms requiring video selfie or face scan (Proof of Humanity, Humanode, Civic Pass) may create another challenging situation for users as proper lighting and optimal devices are necessary for taking the appropriate photo or video for biometric verification. 

Aadhaar card\cite{Aadhaar}\cite{AadhaarEnrollment}, Estonia's e-ID and Japan's My Number Card are all government based credentials but completing all the formalities and getting the card takes a long time, sometimes creating user frustration. The existing systems and platforms that we have evaluated can hardly strike a balance between privacy, functionality and usability.   %citations to be added


% \begin{figure}[!t]
% 	\centering
% 	\includegraphics[width=\linewidth]{Fig/world app.png}
% 	\caption{New account creation process in  World App}
    
% \label{fig:New account creation process in  World App}
% \end{figure}
% \begin{figure}[!t]
% 	\centering
% 	\includegraphics[width=\linewidth]{Fig/bright id.png}
% 	\caption{New account creation process in  Bright ID}
    
% \label{fig:New account creation process in  Bright ID}
% \end{figure}


\subsection{Reddit Analysis}
%\tanusree{ishan to add}
We first collected \fixme{X} posts and  \fixme{X} comments on December 24th, 2024, using the Python Reddit API Wrapper (PRAW)~\footnote{https://praw.readthedocs.io/en/stable/}. We gathered the data from various relevant subreddits, ensuring a broad and comprehensive understanding of what users discuss on identify verification or personhood verification. Through qualitative analysis of this Reddit data, we were able to uncover detailed insights into the rich and prevalent usage of verification systems. This analysis highlighted users' current usage, potential challenges and risks they encounter. These findings provide a solid foundation to explore these themes further in subsequent in-depth interviews.

\paragraph{Data Collection}
 To comprehensively cover content related to our research questions on personhood verification, we first created a list of search keywords by identifying close terminologies related to \textit{``personhood verification''} (general keywords) and \textit{``bot check''} (technology-focused keywords), etc. We utilized a combination of general and technology-focused keywords in our search. We employed general terms such as Personhood Verification, Identity Proof, Human Check and Bot Check. These keywords were designed to capture posts authored by or discussing personhood verification. For the technology focus, we used terms such as \fixme{add}. These keywords targeted discussions specifically about the use of popular tools and platforms. We conducted open searches combining these keywords across Reddit to gather data from various subreddits.
 Other than open searches, we also applied specific criteria to select subreddits, ensuring comprehensive coverage of relevant discussions: these subreddits should focus either on the personhood verification community or technology. We chose subreddits with the most active users online during our browsing sessions. The full list of subreddits and search keywords used is detailed in Table\fixme{need to find out the subreddit most prevalent discussing these topic}. 

\paragraph{Analysis}
Two researchers reviewed each post and categorized related posts or comments into five overarching high-level themes: \fixme{need to add after data analysis}. Within these categories, 53 level-2 themes were identified, such as \fixme{need to add after data analysis}. During the analysis process, researchers regularly convene to discuss discrepancies and emerging themes in the codebook, aiming to reach a consensus. These categories allowed us to investigate RQ2 and partially address RQ1. 

\subsection{Results}
% \tanusree{ishan to add}
\fi
\vspace{-2mm}
\section{ User Study Method}
\vspace{-2mm}
\label{user-study}
This section outlines the method for exploring users' perceptions and preferences of personhood credentials. We conducted semi-structured interviews with 23 participants from the US, and the EU/UK in October 2024.
%We started with a round of pilot studies (n=5) to validate the interview protocol. Based on the findings of pilot studies, we revised the interview protocol and conducted the final round of interviews (n=17). 
The study was approved by the Institutional Review Board (IRB).
\vspace{-2mm}
\subsection{Participant Recruitment}
\vspace{-2mm}
We recruited participants through (1) social media posts, (2) online crowdsourcing platforms, including CloudResearch and Prolific. Respondents were invited to our study if they met the selection criteria: a) 18 years or older and b) living in the US or the EU/UK. Participation was voluntary, and participants were allowed to quit anytime. Each participant received a \$30 Amazon e-gift card upon completing an hour-and-a-half interview.

\subsection{Participants}
%\tanusree{check for final count} \ayae{updated percentage with final 23 counts} 
We interviewed 23 participants, 10 from the US and 12 from the EU/UK. The majority of the participants (61\%) were in the age range of 25-34, followed by 22\% were 35-44 years old. The participants were from the United States and various countries, namely Spain, Sweden, Germany, Hungary, and the United Kingdom. Participants had different backgrounds of education levels, with 87\% of participants holding a Bachelor’s degree and 65\% holding a graduate degree. 65\% of participants had a technology background, while 48\% of them had a CS background. All participants reported using online services that required them to verify their personhood. Table~\ref{table:demographics} presents the demographics of our participants. We refer to participants as P1,. . . ,23.
\begin{table*}[h!]
\centering
%\scriptsize
\caption{Overview of PHC Application Scenarios}
\label{table:scenario}
%\resizebox{\textwidth}{!}{%
\begin{tabular}{lll}
\hline
\textbf{Scenario} & \textbf{Service} & \textbf{Credential} \\
\hline
Financial service & Bank, Financial institutions & Passport or Driver’s license, Face scan \cite{yousefi2024digital}\\
% \hline
Healthcare service & Hospitals, Clinics & Health insurance card,  Fingerprint \cite{chen2012non,fatima2019biometric,jahan2017robust}\\
% \hline
Social media & Tech companies & National identity card, Video selfie \cite{instagramWaysVerify, metaTypesID,instagramTypesID} \\
% \hline
LLM application & Tech companies & Iris scan \cite{WorldWhitepaper, worldHumanness}\\
% \hline
Government service & Government & Driver’s license or National identity card \cite{LogingovVerify}\\
% \hline
Employment background check & Background check companies & Tax identification card, Fingerprint\cite{cole2009suspect}\\
\hline
\end{tabular}%
%}
% \vspace{0.5em}
\label{tab:scenarios}
\end{table*}
\begin{table*}[h]
\centering
\caption{Participant demographics and background.}
%\fixme{add the participants you completed so far}
\resizebox{\textwidth}{!}{%
\begin{tabular}{l l l l l l l l}
\hline
\textit{Participant ID} & \textit{Gender} & \textit{Age} & \textit{Country of residence} & \textit{Education} & \textit{Technology background}  & \textit{CS background} &\textit{Residency duration} \\
\hline
P1 & Male & 25-34 & the US & Master's degree & Yes & Yes &3-5 years\\
P2 & Female & 25-34 & the US & Master's degree & Yes & Yes & 1-3 years\\
P3 & Female & 25-34 & the UK & Master's degree & Yes & No & 1-3 years\\
P4 & Female & 35-44 & the UK & Some college, but no degree & Yes & Yes & Over 10 years \\
P5 & Male & 25-34 & the US & Doctoral degree & Yes & Yes & 5-10 years \\
P6 & Male & 35-44 & the US & Less than a high school diploma & No & No & Over 10 years \\
P7 & Male & 25-34 & the US & Doctoral degree & Yes & Yes & 3-5 years\\
P8 & Male & 45-54 & the US & Bachelor's degree & Yes & Yes & Over 10 years \\
P9 & Female & 25-34 & New Zealand & Master's degree & No  &  No &  Over 10 years\\
P10 & Male & 25-34 & the US & Master's degree & No & No & Over 10 years\\
P11 & Female & 25-34 & the UK & Bachelor's degree & No & No & Over 10 years\\
P12 & Male & 18-24 & the UK & Master's degree & Yes & Yes & 1-3 years\\
P13 & Male & 35-44 & the UK & Bachelor's degree & Yes & No & Over 10 years\\
P14 & Male & 25-34 & Sweden & High school graduate & No & No & Over 10 years \\
P15 & Female & 25-34 & Spain & Master's degree & Yes & Yes & Over 10 years \\
P16 & Female & 25-34 & Germany & Master's degree & Yes & Yes & Over 10 years \\
P17 & Female & 25-34 & Spain & Doctoral degree & No & No & Over 10 years \\
P18 & Female & 35-44 & the US & Bachelor's degree & No & No & Over 10 years \\
P19 & Female & 25-34 & Germany & Master's degree & Yes & Yes & 3-5 years \\
P20 & Male & 25-34 & Hungary & Master's degree & Yes & No & 3-5 years \\
P21 & Male & 35-44 & the US & Bachelor's degree & Yes & No & 5-10 years \\
P22 & Female & 18-24 & France & Master's degree & Yes & Yes & Less than 1 year\\
P23 & Male & 45-52 & the US & Master's degree & No & No & Over 10 years\\
\hline
\end{tabular}%
}
\label{table:demographics}
\end{table*}


\vspace{-2mm}
\subsection{Semi-Structured Interview Procedure} \label{sec:study_protocol}
\vspace{-2mm}
%\fixme{explain in details why the study designed in a certain way. please read papers to learn more. data minimization and advertisement paper. The method section is too bland. We have a wonderful study design. Scenario-specific study design, describe scenarios and why chose this scenario. Mainly method should include all design rationale, and example questions when necessary to clarify your rational}

We started with a round of pilot 
%(Appendix~\ref{pilot}) 
studies (n=5) to validate the interview protocol. Based on the findings of pilot studies, we revised the interview protocol.

\textbf{Open Ended Discussion.} We designed the interview script based on our research questions outlined in the introduction section~\ref{sec:introduction}. 
%We added the interview script to the section~\autoref{protocol}. 
At the beginning of the study, we received the participants’ consent to conduct the study. Once they agreed, we proceeded with a semi-structured interview. The study protocol was structured according to the following sections: (1) Current practices regarding digital identity verification; (2) Users' perception of PHC before and after watching the informational video; (3) Scenario-based session to investigate factors that influence users' preferences of PHC; 
%(4) Users' preference of PHC; 
(4) Design session to conceptualize users' expectations; (5) A brief post-survey on Users' Preference of PHCs.
%of PHCs in different scenarios.

In the first section, we first asked a set of questions to understand participants' current practices of online platforms and the types of identity verification methods they had experience with. This is to understand their familiarity with different types of verification, such as biometrics, physical IDs, etc.
%and methods that might have worked well based on their prior experience.

%of online identity verification. When participants mentioned certain types of online services that required identity verification, we inquired about their experience with verification method. Was it easy to use, or did you run into any issues?"} We further inquired about any challenges participants faced with identity verification - \textit{"Did you encounter any challenges when using this method?"} 
%If biometrics didn’t naturally come up in prior discussions, we prompted to consider them- \textit{"Have you ever used services where you had to verify yourself through face, fingerprints, or iris scans, or other biometrics?"} If they mentioned any experience with biometric verification, we followed up with questions like- \textit{"What worked well? Were there any concerns you had?"}
In the second section, we then asked about participants' current understanding and perception of personhood credentials either from prior knowledge or from intuition by just hearing the term. %We also asked if they knew how personhood credentials work, particularly how it has been handled by the different services they use. 
%As all participants had never heard of PHC, we prompted them to interpret the term based solely on its wording. 
While the majority recognized this as unfamiliar terminology, most inferred that it referred to a form of personal identification, often associating it with biometric verification.
%In the pilot interviews, The majority of the participants could not provide substantial responses on their understanding of how personhood credentials work, before starting the second part of the interview, we showed them an informational video on personhood credentials.
%Most of the participants were unfamiliar with this term, so we then asked \textit{ Can you explain what you think it means by just hearing the term?"} 
%Before proceeding with the third section of the interview, we assessed participants' understanding of PHC with knowledge questions.
Then, we showed them an introduction video on PHC \footnote{https://anonymous.4open.science/r/PHC-user-study-14BB/}, %\fixme{create an anonymous GitHub, upload the video and add a footnote here} \ayae{reflected}. 
%The video provides an overview of PHCs, 
covering their definition, 
%the steps involved in issuing and using them, 
and implications of it in online services. Based on former literature\cite{adler2024personhood}, we designed the video with easy-to-understand text, visuals, and audio to make the concepts accessible to average users. We created a set of knowledge questions to assess participants' understanding of PHC before and after showing the video. %as attached in Appendix~\ref{knowledge_questions}.

%including the same knowledge questions. 
%Most participants correctly responded to knowledge questions, which ask the basic understanding of digital identity crisis and personhood credentials. 
%Even before showing the introduction video, regarding the question \textit{"What could happen if online identities are poorly verified?"}, 95\% correctly selected \textit{"Fake accounts, bots, and fraud could increase significantly."} For the question \textit{"What are Personhood Credentials (PHCs)?"}, 90\% correctly choose the option \textit{"Digital credentials that confirm a person’s identity."} 
For instance, we observed an improve in correct response rate for the question, such as, \textit{``What is the primary goal of PHC?''} from 85\% to 100\% after watching the video.
%where the correct answer was \textit{"To verify a person's identity without exposing personal information."} 
%However, regarding the question \textit{"To whom do you provide minimal personal information during the PHC process?"}, only 35\% selected the correct answer \textit{"PHC issuers (e.g., governments or trusted organizations)"}, while the most frequent response was \textit{"Online service providers (e.g., social media)"} at 45\%.
%\ayae{KQ results reflected}
%We also asked some open-ended questions to evaluate whether our introduction video helped participants better understand PHC \textit{''How would you explain your understanding of personhood credentials?''} 
%We further asked what benefits and concerns came to mind for them.
In the third section, we focused on scenario-based discussions, exploring specific applications of PHC to understand factors that influence participants' preferences towards PHCs as well as identify challenges to leverage in PHC design for various services. We examined the following six scenarios: (1) Financial service, (2) Healthcare service, (3) Social Media, (4) LLM applications, (5) Government Portal, and (6) Employment Background Check.
%We covered a wide range of use cases of online personhood verification via these six scenarios since they encompass diverse user needs, security and usability, and privacy requirements. %\fixme{please see the comment with iffalse tag and make it concise, we talked about it before}
\iffalse
%Firstly, financial system is a critical scenario for identity verification where high level of security protections are expected as exemplified by KYC. Thus, such services continue to develop transformative digital identity verification to ensure the security and integrity of financial transactions\cite{parate2023digital}. The second scenario is healthcare systems, which also have high privacy requirements due to the confidentiality of medical data. The pandemic has accelerated the adoption of online healthcare services and in response to this digital transformation, the recent study has proposed blockchain-based decentralized identity management systems \cite{javed2021health}. Thirdly, we consider the scenario of social media, which faces the critical challenges of online identity as shown in spreading misinformation and harmful content from fake or anonymous accounts \cite{ceylan2023sharing}. The fourth scenario is designed with a specific context of interacting with Large language models (LLMs). The former study discussed vulnerability in dialog-based systems where adversaries can exploit the training process to introduce toxicity into responses \cite{weeks2023first}. Thus, such vulnerabilities indicate identity verification may also be important for LLM applications. Fifth, government services are familiar situations that require people to verify their identity. Various countries have developed their own electronic ID schemes \cite{stalla2018gdpr}. Lastly, we also cover the scenario of employment background checks needing precise identity verification to ensure the reliability of applicants. The current background check system involves vulnerable processes that increase the risk of identity theft and unauthorized data access.\cite{blowers2013national}. Such challenges underline the relevance of PHCs, which can mitigate risks by providing a secure framework for verification.
%\ayae{included citation}
\fi
We have also incorporated various types of data or credentials requirements (e.g. physical id, biometrics, etc) across scenarios to maintain diversity in our discussion with participants as shown in Table.\ref{table:scenario}. %For instance,
%we  We have multiple existing verification methods, including 
%humanness verification (e.g., selfie, video call), document-based verification (e.g., government-issued ID), and biometrics information. 
We selected types of credentials for each scenario based on former literature and existing PHC as explained in the section \ref{subsec:verification_practice}. %\fixme{cite worldcoin, and other app and literature}. \fixme{from here to end of this paragraph ---These needs to go to the literature review section on the current usecase of PHC. And only 2 line summarizing why you chose the diverse type of credential data and refer to the literature section}


%% Let me find the former literature to explain why we select these credentials
For each of the six scenarios, we explored participants' perceptions of using PHC in hypothetical situations that align with the research focus as well as to help participants can relate PHC concepts to real-world applications. This is particularly useful for this study where where user perceptions and expectation under specific conditions are crucial to devising solutions \cite{carroll2003making}.
%\fixme{cite scenario method paper from jack caroll}.
%\ayae{reflected}
We asked about their feelings, perceived benefit and risks. We also nudge them to think about any privacy and security perception around using PHC and types of data (e.g., iris, face, government id, etc) involved in issuing PHC. 

\fixme{
%\textbf{Pre-understanding: Guessed it as one of the verification methods} 
%The majority of the participants were not familiar with the term ``Personhood Credential'', although most of them used some forms of such credentials. 
%As all participants have never heard of PHC, we prompted them to interpret the term based solely on its wording. Most of them inferred that it referred to another type of person identification. 
%For instance, P3 commented \textit{``It can be anything that would point to one single individual that would differentiate that individual from others.''} When participants expressed how PHC identifies a person's uniqueness, their understanding ranged from verifying basic information such as address or age, and certain eligibility to advanced identification of digital identity (e.g., behavioral, economical, etc) with Multi-factor authentication or knowledge-based questions.

%\textbf{Post-understanding: Involvement of trusted entity} When asked to explain their understanding of PHC, P13 noted, \textit{"So it sounds like, basically, you it's similar to how you verify things before. Like you use a biometrics and your government Id. But then you get a personal key. You do it with like a trusted organization rather than each individual. And then you can use that key for all the different services you use."} P1 elaborated PHC process as a shift of the verification entity, \textit{" I'd say we are sort of moving the verification burden from the user side to a service provider side where they have access to our data, and they have access to the token that's assigned to each person that's unique. And that's easily like traceable across online platforms. and this token is used for verification with 3rd parties, where they don't get access to your personal data, but they only use this service provider to give them the authenticity that you are a real user."} These suggest that the role of the PHC issuer is recognized as a crucial component of PHC.}
%began by asking \textit{"How did you feel about using PHC to verify your identity when opening your bank account?"} To dive deeper, we also asked about potential benefits:\textit{"What potential benefits do you see in using PHC in this online banking context?"}. We also inquired about these aspects- \textit{"Do you think using PHC improves the security of your bank account? Why?", "Did this method of identity verification make you feel more confident about your privacy? why?"} Additionally, we discussed their comfort levels for providing credentials (e.g., Government-issued ID, biometric information) and asked about any concerns about data collection-\textit{"Were you comfortable providing your government-issued ID and using facial recognition? Why?"}
}

\iffalse
%%% column: scenario, credential, service providers.
\begin{table*}[h!]
\centering
\caption{Overview of PHC Application Scenarios}
\label{table:scenario}
%\resizebox{\textwidth}{!}{%
\begin{tabular}{lll}
\hline
\textbf{Scenario} & \textbf{Service Provider} & \textbf{Types of Credential} \\
\hline
Financial Service & Bank, Financial Institutions & Passport or Driver’s license, Face scan \cite{yousefi2024digital}\\
% \hline
Healthcare Service & Hospitals, Clinics & Health insurance card,  Fingerprint \cite{chen2012non,fatima2019biometric,jahan2017robust}\\
% \hline
Social Media & Tech Companies & National identity card, Video selfie \cite{instagramWaysVerify, metaTypesID,instagramTypesID} \\
% \hline
LLM Application & Tech Companies & Iris scan \cite{WorldWhitepaper, worldHumanness}\\
% \hline
Government Service & Government & Driver’s license or National identity card \cite{LogingovVerify}\\
% \hline
Employment Background Check & Background Check Companies & Tax identification card, Fingerprint\cite{cole2009suspect}\\
\hline
\end{tabular}%
%}
% \vspace{0.5em}
\label{tab:scenarios}
\end{table*}
\fi


%\textbf{Design Session.}
%\fixme{need to explain how and why you design the design session, where you designed, how participants were unstructured and so on.} \ayae{reflected in the following paragraph}

In the fourth section, we began by refreshing participants’ memories of the various risks and concerns discussed in the earlier scenario-based section. Following this, we guided participants to brainstorm potential design solutions by sketching their ideas to address these concerns. To facilitate the sketching process, we developed sketch notes in Zoom as prompts to help participants generate ideas, particularly when starting from scratch is challenging. 
%on Zoom whiteboard or pen and paper, using a think-aloud protocol.  
%Nevertheless, it is difficult to develop new ideas from scratch, so 
%Additionally, we described the main issues or concerns that the participants identified during the interview at the top of the sketch notes. 
%Participants can develop their ideas at the center of the whiteboard by locating the above components or creating new shapes, lines, or text boxes for their sketches. 
We also investigated participants' preferences for PHC regarding the issuers and issuance systems of PHCs, as well as the types of data required for issuing PHCs. 
%in the context of who issues PHC or type of issuance systems, and what types of data are needed to issue PHC to address RQ2. 
%An example includes- \textit{``What types of credential would you prefer to use as personhood verification? ; Which organizations or stakeholders would you prefer to issue and manage your PHC?''} 
We encourage participants to explain their reasoning. These questions were informed by insights from the pilot study, where participants expressed preferences for different types of data, system architecture, and various stakeholders involved in PHC issuing.
%However, these questions alone can only find optimal ways within the scope of currently existing options and cannot generate new design implications. Therefore,

\iffalse
\tanusree{we can cut this section as this didn't give any result and doesn't answer RQs directly.}Lastly, to understand preference on issuance system, we introduced the decentralized PHC system architecture with another instructional video. Following the video, we asked participants to explain their understanding of the decentralized PHC system and their preference for the issuance system (centralized or decentralized). We introduced it after the sketch session is that participants may organically come up with the idea of decentralized systems on their own, and we intended to avoid priming them. 
\fi
%Then, we asked them to explain their understanding of the decentralized PHC and preferred issuance system (centralized or decentralized.)- \textit{`` Could you explain why you would prefer decentralized system in managing your PHCs?''}
%\textit{"Would you prefer to get multiple PHCs from different issuers depending on the situation or application you're using, or would you rather have a single PHC from one issuer?"}

\textbf{Post-Survey.}
%%\fixme{need to explain how and why you design the design session, where you designed, how participants were unstructured and so on.}
We conducted a post-survey to obtain participants' PHC preference quantitatively. It included questions on participants' preference on credential type, issuer and issuance system  for the scenarios (e.g., financial, medical, etc) we considered in our interview.

\vspace{-2mm}
\subsection{Data Analysis}
\vspace{-2mm}
Once we got permission from the participants, we obtained interview data through the audio recording and transcription on Zoom. We analyzed these transcribed scripts through thematic analysis \cite{Braun2012-sz, Fereday2006-yv}. Firstly, all of the pilot interview data was coded by two researchers independently. Then, we compared and developed new codes until we got a consistent codebook. Following this, both coders coded 20\% of the interview data of the main study. We finalized the codebook by discussing the coding to reach agreements. Lastly, we divided the remaining data and coded them. After both researchers completed coding for all interviews, they cross-checked each other’s coded transcripts and found no inconsistencies. Lower-level codes were then grouped into sub-themes, from which main themes were identified. Lastly, these codes were organized into broader categories. Our inter-coder reliability (0.90) indicated a reasonable agreement between the researchers.
\iffalse

\begin{table*}[h]
\centering
\caption{Participant demographics and background.}
%\fixme{add the participants you completed so far}
\resizebox{\textwidth}{!}{%
\begin{tabular}{l l l l l l l l}
\hline
\textit{Participant ID} & \textit{Gender} & \textit{Age} & \textit{Country of residence} & \textit{Education} & \textit{Technology background}  & \textit{CS background} &\textit{Residency duration} \\
\hline
P1 & Male & 25-34 & the US & Master's degree & Yes & Yes &3-5 years\\
P2 & Female & 25-34 & the US & Master's degree & Yes & Yes & 1-3 years\\
P3 & Female & 25-34 & the UK & Master's degree & Yes & No & 1-3 years\\
P4 & Female & 35-44 & the UK & Some college, but no degree & Yes & Yes & Over 10 years \\
P5 & Male & 25-34 & the US & Doctoral degree & Yes & Yes & 5-10 years \\
P6 & Male & 35-44 & the US & Less than a high school diploma & No & No & Over 10 years \\
P7 & Male & 25-34 & the US & Doctoral degree & Yes & Yes & 3-5 years\\
P8 & Male & 45-54 & the US & Bachelor's degree & Yes & Yes & Over 10 years \\
P9 & Female & 25-34 & New Zealand & Master's degree & No  &  No &  Over 10 years\\
P10 & Male & 25-34 & the US & Master's degree & No & No & Over 10 years\\
P11 & Female & 25-34 & the UK & Bachelor's degree & No & No & Over 10 years\\
P12 & Male & 18-24 & the UK & Master's degree & Yes & Yes & 1-3 years\\
P13 & Male & 35-44 & the UK & Bachelor's degree & Yes & No & Over 10 years\\
P14 & Male & 25-34 & Sweden & High school graduate & No & No & Over 10 years \\
P15 & Female & 25-34 & Spain & Master's degree & Yes & Yes & Over 10 years \\
P16 & Female & 25-34 & Germany & Master's degree & Yes & Yes & Over 10 years \\
P17 & Female & 25-34 & Spain & Doctoral degree & No & No & Over 10 years \\
P18 & Female & 35-44 & the US & Bachelor's degree & No & No & Over 10 years \\
P19 & Female & 25-34 & Germany & Master's degree & Yes & Yes & 3-5 years \\
P20 & Male & 25-34 & Hungary & Master's degree & Yes & No & 3-5 years \\
P21 & Male & 35-44 & the US & Bachelor's degree & Yes & No & 5-10 years \\
P22 & Female & 18-24 & France & Master's degree & Yes & Yes & Less than 1 year\\
P23 & Male & 45-52 & the US & Master's degree & No & No & Over 10 years\\
\hline
\end{tabular}%
}
\label{table:demographics}
\end{table*}
\fi

% \begin{figure*}
%     \centering
% \includegraphics[width=\textwidth]{Figures/Experiment/fig78merge_slip.pdf}
%     \caption{The experimental results of friction coefficient estimation. From top to bottom, the figures show the estimated friction coefficient, estimated foot velocity in the normal direction with the rejection score, and estimated tangential foot velocity with the confidence score.}
%     \label{fig:online_estimation}
% \end{figure*}

\begin{figure*}
    \centering
\includegraphics[width=\textwidth]{Figures/Experiment/Experiment_1.pdf}
    \caption{The experimental results of friction coefficient identification show the effects of proposed smoothed gradients and rejection methods. Without smoothed gradients, non-informative gradients can impede friction coefficient identification. The rejection method allows for consistent friction coefficient identification, especially when the legged robot traverses nonslippery terrain. The purple area represents the slip states on the slippery terrains where the norm of tangential estimated foot velocity from the state estimator~\cite{Joonha2023TRO} exceeds 0.4~\si{\meter/\second}.}
    \label{fig:online_estimation}
\end{figure*}

% \begin{figure}
%     \centering
%     \subfloat[]{
%         \includegraphics[width=1.0\columnwidth]{Figures/Experiment/fig7_contact_vel_1.pdf}
%         %\label{fig:a}
%         }
%         \hfill
%     \subfloat[]{
%         \includegraphics[width=1.0\columnwidth]{Figures/Experiment/fig7_contact_vel_2.pdf}
%         %\label{fig:b}
%         }
%     \caption{\textcolor{blue}{(a) contact velocity in the normal direction. (b) tangential contact velocity. As depicted by a green dotted circle, the rejection method can effectively prevent undesired increases in the confidence score, especially on nonslippery terrain. Conversely, described by a red dotted circle, rejection scores do not significantly impede the increases in confidence scores when the robot slips on the slippery terrain.}}
%     \label{fig:wrong_cof_in_contact_model}
% \end{figure}

\begin{figure}
   \centering
\includegraphics[width=1.0\columnwidth]{Figures/Experiment/Experiment_2.pdf}
   \caption{As depicted by a green dotted circle, the rejection method can effectively prevent undesired increases in the confidence score, especially on nonslippery terrain. Conversely, described by a red dotted circle, rejection scores do not significantly impede the increases in confidence scores when the robot slips on the slippery terrain.}
   \label{fig:contact_vel}
\end{figure}

\begin{figure}
    \centering
    \includegraphics[width=1.0\columnwidth]{Figures/Experiment/Experiment_3.pdf}
    \caption{Comparison of the average loss between the method using the nonsmooth model and the proposed method. The proposed method achieves a lower average loss than the method using the nonsmooth model.}
    \label{figure:loss_compare}
\end{figure}



\section{Experimental Results} \label{Sec:Experiment}
% The robot starts from the nonslippery terrain made of rubbers and moves to the slippery terrain. Then, the robot moves along the body coordinates x-axis inside the slippery terrain.
% As shown in Fig.~\ref{fig:online_estimation}, the proposed framework performs the friction coefficient estimation by solving the issue of nullified gradients. Without the proposed smoothing method, the updates for the friction coefficient are hindered by nullified gradients. 


% 이번 섹션에서는 실제 로봇 KAIST 하운드에를 이용한 제안한 마찰 계수 추정 알고리즘의 실험 결과를 소개한다. 또한, 우리는 smoothed gradients of contact impulses on the friction coefficient과 rejection of elastic contact 그리고 confidence score의 효과에 대하여 명백히 밝힌다. 이를 위해, 
% Please add the following required packages to your document preamble:
% \usepackage{graphicx}
% Please add the following required packages to your document preamble:
% \usepackage{graphicx}
\begin{table}[]
\caption{Parameters Used in Experiments}
\centering
\resizebox{\columnwidth}{!}{%
\begin{tabular}{|c|c|c|c|c|c|c|c|}
\hline
\hline
\textbf{Parameter} & \text{$\alpha_\mathrm{rej}$} & \text{$\gamma_\mathrm{rej}$} & \text{$\Delta{t}_\mathrm{buffer}$} & \text{$\Delta{t}_\mathrm{bound}$} & \text{$\sigma_\mathrm{slip}$} & \text{$\sigma_{q_\mathrm{base}}$} & \text{$\sigma_{q_\mathrm{jnt}}$} \\ \hline
\textbf{Value}     &   5.0    &     0.4    &             0.01~\si{\second}       &      0.1~\si{\second}    & 30 & 1e-4 & 20      \\
\hline \hline
\textbf{Parameter} & \text{$\alpha_\mathrm{conf}$} & \text{$\gamma_\mathrm{conf}$} & \text{$\epsilon$} & \text{$H$} & \text{$\rho_\mathrm{t}$} & \text{$\sigma_{\dot{q}_\mathrm{base}}$} & \text{$\sigma_{\dot{q}_\mathrm{jnt}}$}\\ \hline
\textbf{Value}     &    3.0       &       0.58     &     0.1      &      50       &      0.05      & 1e-4 & 1    \\
\hline \hline
\end{tabular}%
}
    \label{table:parameters}
\end{table}
This section introduces the experimental results of the proposed friction coefficient identification framework using the quadrupedal robot, KAIST HOUND~\cite{shin2022hound}. Additionally, we explain the effects of proposed analytic smoothed gradients of contact impulses with respect to the friction coefficient and the proposed rejection method. 
% \subsection{Experimental Setup}
% \begin{figure}
%     \includegraphics[width=1.0\columnwidth]{Figures/Experiment/fig13.pdf}
%     \caption{An environment setup for experiments. The slippery terrain is made of acrylic flat boards with boric acid powder.}
%     \label{fig:experimental_setup}
% \end{figure}
\subsection{Experimental Setup}

The proposed framework is based on the confidence score-based online system identification framework~\cite{chen2022real} and employs the proposed smoothed gradient and rejection method. To calculate the proposed smoothed gradients, we empirically set the smoothing parameter, $\rho_\mathrm{t}$, as 0.05. As~\cite{kim2023contactimplicit}, if the smoothing parameter is either too large or too small, the smoothing method may not be effective in achieving better local optima compared to the nonsmooth method. In implementing the proposed framework, we set $\mu_\mathrm{min}$ and $\mu_\mathrm{max}$ as 0.01 and 1.0, respectively.

%Based on the frictional contact dynamics, this framework can be conducted to estimate the low friction coefficient. However, s
Since estimating a high friction coefficient through contact dynamics is challenging in the absence of foot slippage~\cite{focchi2018slip,jenelten2019legged}, this work employs the reset method for the friction coefficient as~\cite{ jenelten2019legged}. This method resets the estimated friction coefficient to the default value $\mu_\mathrm{def}$ of 0.8 when stable contacts are maintained for 0.5~\si{\second}. In this work, the estimated friction coefficient is restored to the default value if the confidence score $\eta$ does not exceed the threshold $\gamma_\mathrm{conf}$ for 0.5~\si{\second}, indicating that the robot does not have a high tangential contact velocity for this duration.

The experiments are conducted in two different terrains: a nonslippery terrain and a slippery terrain. The slippery terrain is made of acrylic flat boards with boric acid powder. The robot initially starts on the nonslippery terrain where the experimentally measured friction coefficient is 1.0, then moves to the slippery terrain where the experimentally measured friction coefficient is 0.19. Subsequently, the robot moves between slippery and nonslippery terrains alternately. The measured friction coefficient on slippery terrain was obtained by measuring the horizontal force with a spring scale when the standing robot began to slip, considering its weight~\cite{shin2022hound}.

We solved the contact dynamics only for the states where contact is detected by the state estimator and implemented RaiSim’s algorithm~\cite{raisim} for this purpose.

For state estimation of the legged robot, we employ the method proposed by \cite{Joonha2023TRO}, which operates at 200 Hz within our framework. The contact velocity, contact states, and slip states are estimated in the state estimator. We determine the slip states when the norm of tangential contact velocity, estimated by the state estimator exceeds 0.4~\si{\meter/\second}. As a robot's controller, a nonlinear model predictive controller in \cite{hong202realtime} is utilized with functioning at 80 Hz. The boundary of computation time ${\Delta{t}_\mathrm{bound}}$ for the proposed framework is set at 10 Hz. The detailed parameters for the proposed framework are given in Table.~\ref{table:parameters}. A single onboard computer with an Intel(R) Core(TM) i7-11700T CPU, capable of reaching up to 1.6 GHz, is utilized to implement the proposed framework.






\subsection{Estimation Results}


\begin{figure}
    \centering
    \includegraphics[width=1.0\columnwidth]{Figures/Experiment/Experiment_4.pdf}
    \caption{Result of the average loss for the experiment according to the smoothing parameter $\rho_\mathrm{t}$.}
    \label{figure:smoothing_compare}
\end{figure}

To validate the proposed methods, we compared the results of friction coefficient identification with and without the proposed gradient and rejection method, as shown in Fig.~\ref{fig:online_estimation}. In the experiment, we set the default estimated friction coefficient to 0.8 to illustrate a scenario where the robot, assuming a high friction coefficient for non-slippery terrain, slips on slippery surfaces. Note that the parameter update is conducted when the confidence score exceeds the threshold $\gamma_\mathrm{conf}$~\cite{chen2022real}.
%This approach enables the friction coefficient identification framework to update the coefficient only when the tangential contact velocity increases, which can occur due to foot slippages or high contact velocities following contact initiations. 
%If the confidence score does not exceed $\gamma_{conf}$ for 0.5~\si{\second}, the estimated friction coefficient $\hat{\mu}$ is reset to the default value $\mu_{def}$.

% With the boundary time of 10 Hz, the mean and maximum computation time for the proposed framework are 0.0285~\si{\second} and 0.0867~\si{\second}, respectively. 
%To validate our proposed method, we compared the friction coefficient estimation performance when using the proposed gradient for optimization problem~\eqref{osi_opt} with the performance when not using the proposed gradient. 
% The first figure at the top of Fig.~\ref{fig:online_estimation} illustrates the parameter estimation results, including the estimated friction coefficient, $\hat{\mu}$.
%We compared the parameter estimation performance with the performance of a framework without using the proposed methods to verify the proposed methods.



In the left bottom figure of Fig.~\ref{fig:online_estimation}, we observed that using nonsmooth gradients can impede friction coefficient identification, even if the robot slips on slippery terrains. In contrast, employing the proposed smoothing method allows for fast and consistent identification.
% updating the coefficient from 0.8 to 0.393 in a single step within 0.1~\si{\second}, and from 0.8 to 0.281 within 0.3~\si{\second}.

Moreover, the right bottom figure in Fig.~\ref{fig:online_estimation} shows that the estimated friction coefficient becomes more consistent, especially on nonslippery terrain, when the confidence score-based update is used with the rejection method compared to without it. Using both methods, the estimated friction coefficient on nonslippery terrain can be maintained close to the default value for such terrain, without undesired updates.

%However, using the confidence score-based update with the rejection method stabilizes the online friction coefficient identification process, especially when the robot is on nonslippery terrain.


%However, the proposed smoothing method solves the issue of non-informative gradients and updates the coefficient from 0.8 to 0.393 with a single update within 0.1~\si{\second} and from 0.8 to 0.281 within 0.3~\si{\second}.

The detailed effects of the proposed gradients and rejection methods will be discussed below.

\subsection{The Effects of Analytic Smoothed Contact Gradients}
%\begin{figure}
%    \centering
%    \includegraphics[width=1.0\columnwidth]{Figures/Experiment/fig11.pdf}
%    \caption{\textcolor{blue}{The proposed gradient can mitigate the issue of lack of informative gradient, allowing for improved friction coefficient identification under various initial conditions.}}
%    \label{figure:different_initial_condition}
%\end{figure}
\begin{figure}
    \centering
    \subfloat[]{
        \includegraphics[width=0.45\columnwidth]{Figures/Experiment/Experiment_5_a.pdf}
        \label{figure:different_initial_condition_prop}
        }
        \hfill
    \subfloat[]{
        \includegraphics[width=0.45\columnwidth]{Figures/Experiment/Experiment_5_b.pdf}
        \label{figure:different_initial_condition_nonsm}
        }
    \caption{Comparison of friction coefficient identification under various initial estimates. The purple area represents the slip states on slippery terrains where the norm of tangential contact velocity exceeds 0.4~\si{\meter/\second}. \protect\subref{figure:different_initial_condition_prop} The proposed smoothing is applied. \protect\subref{figure:different_initial_condition_nonsm} The gradient from the nonsmooth model is used~\cite{chen2022real}.}
    \label{figure:different_initial_condition}
\end{figure}
% Please add the following required packages to your document preamble:
% \usepackage{multirow}

% Please add the following required packages to your document preamble:
% \usepackage{multirow}
% \begin{table}[]
% \caption{The results comparing the performance using the proposed smoothing method with the baselines after seven experiments}
% \begin{tabular}{|c|c|cl|cc|}
% \hline
% \multirow{2}{*}{\textbf{Method}}                                            & \multirow{2}{*}{\textbf{\begin{tabular}[c]{@{}c@{}}Average\\ Loss\end{tabular}}} & \multicolumn{2}{c|}{\textbf{Time (s)}}                                   & \multicolumn{2}{c|}{\textbf{\begin{tabular}[c]{@{}c@{}}Estimated \\ Friction Coefficient\end{tabular}}}     \\ \cline{3-6} 
%                                                                             &                                                                                  & \multicolumn{1}{c|}{\textbf{Mean}}   & \multicolumn{1}{c|}{\textbf{Max}} & \multicolumn{1}{c|}{\textbf{Mean}} & \textbf{\begin{tabular}[c]{@{}c@{}}Standard \\ Deviation\end{tabular}} \\ \hline
% \textbf{Proposed}                                                           & \textbf{1.1542}                                                                  & \multicolumn{1}{c|}{0.0247}          & 0.0867                            & \multicolumn{1}{c|}{0.2724}        & 0.0371                                                                 \\ \hline
% \textbf{Exact}                                                              & 6.8542                                                                           & \multicolumn{1}{c|}{\textbf{0.0111}} & \textbf{0.0514}                   & \multicolumn{1}{c|}{0.7373}        & 0.1660                                                                 \\ \hline
% \textbf{\begin{tabular}[c]{@{}c@{}}1st-order\\ Randomized\end{tabular}}   & 3.6515                                                                           & \multicolumn{1}{c|}{0.5679}          & 1.8760                            & \multicolumn{1}{c|}{0.2810}        & 0.0352                                                                 \\ \hline
% \textbf{\begin{tabular}[c]{@{}c@{}}0th-order \\ Randomized\end{tabular}} & 3.9282                                                                           & \multicolumn{1}{c|}{0.6997}          & 2.5681                            & \multicolumn{1}{c|}{0.2737}        & 0.0300                                                                 \\ \hline
% \end{tabular}
% \label{table:total_compare}
% \end{table} 


% \begin{table}[]
% % \caption{Comparison of friction coefficient estimation using proposed smoothing with other smoothing method in terms of cost, time and estimated friction coefficient}
% % \caption{Estimation Results obtained from Experiments of 7 Trials to compare the performance using proposed smoothing with that using baseline.}
% \caption{The results comparing the performance using the proposed smoothing method with the baselines after seven experiments}
% \begin{tabular}{|c|c|cl|cc|}
% \hline
% \multirow{2}{*}{\textbf{Method}}                                         & \multirow{2}{*}{\textbf{\begin{tabular}[c]{@{}c@{}}Average\\ Loss\end{tabular}}} & \multicolumn{2}{c|}{\textbf{Time (s)}}                                    & \multicolumn{2}{c|}{\textbf{\begin{tabular}[c]{@{}c@{}}Estimated \\
%  Friction Coefficient\end{tabular}}}     \\ \cline{3-6} 
%                                                                          &                                                                                  & \multicolumn{1}{c|}{\textbf{Mean}}    & \multicolumn{1}{c|}{\textbf{Max}} & \multicolumn{1}{c|}{\textbf{Mean}} & \textbf{\begin{tabular}[c]{@{}c@{}}Standard \\ Deviation\end{tabular}} \\ \hline
% \textbf{Proposed}                                                        & \textbf{1.247}                                                                  & \multicolumn{1}{c|}{0.03033}          & 0.0811                            & \multicolumn{1}{c|}{0.288}         & 0.032                                                                  \\ \hline
% \textbf{Exact}                                                           & 6.854                                                                           & \multicolumn{1}{c|}{\textbf{0.01371}} & \textbf{0.0655}                   & \multicolumn{1}{c|}{0.753}         & 0.125                                                                  \\ \hline
% \textbf{\begin{tabular}[c]{@{}c@{}}First-order\\ Randomized\end{tabular}}  & 3.6515                                                                           & \multicolumn{1}{c|}{1.3168}           & 3.2193                            & \multicolumn{1}{c|}{0.276}         & 0.028                                                                  \\ \hline
% \textbf{\begin{tabular}[c]{@{}c@{}}Zeroth-order \\ Randomized\end{tabular}} & 3.9282                                                                            & \multicolumn{1}{c|}{1.3741}           & 5.9274                            & \multicolumn{1}{c|}{0.277}         & 0.043                                                                  \\ \hline
% \end{tabular}
% \label{table:total_compare}
% \end{table}

% \begin{table}[]    
% \begin{tabular}{|c|c|c|cc|}
% \hline
% \multirow{2}{*}{\textbf{Method}}                                            & \multirow{2}{*}{\textbf{Cost}} & \multirow{2}{*}{\textbf{Time (s)}} & \multicolumn{2}{c|}{\textbf{Estimated Friction Coefficient}}     \\ \cline{4-5} 
%                                                                             &                                &                                    & \multicolumn{1}{c|}{\textbf{Mean}} & \textbf{Standard Deviation} \\ \hline
% \textbf{Proposed}                                                           & \textbf{1.867}                 & 0.0388                             & \multicolumn{1}{c|}{0.293}         & 0.0620                      \\ \hline
% \textbf{Non-smoothed}                                                              & 10.896                         & \textbf{0.383}                     & \multicolumn{1}{c|}{0.679}         & 0.18490                     \\ \hline
% \textbf{\begin{tabular}[c]{@{}c@{}}First-order\\ Randomized\end{tabular}}   & 4.664                          & 1.887                              & \multicolumn{1}{c|}{0.296}         & 0.03466                     \\ \hline
% \textbf{\begin{tabular}[c]{@{}c@{}}Zeroth-order \\ Randomized\end{tabular}} & 6.412                          & 1.880                              & \multicolumn{1}{c|}{0.277}         & 0.0427                      \\ \hline
% \end{tabular}\label{figure:total_compare}
% \end{table}
% \begin{figure}
%     \centering
%     \includegraphics[width=1.0\columnwidth]{Figures/Experiment/fig14.pdf}
%     \caption{Relation between average loss value and smoothing parameter for an optimization problem. The smoothing parameter allows for finding a better searching direction in optimization problems than without using the smoothing parameter.}
%     \label{fig:loss_smoothing}
% \end{figure}

In this session, we will examine the advantages of the proposed smoothing method in friction coefficient identification. As shown in Fig.~\ref{fig:online_estimation}, when the robot slips on slippery terrain, the proposed smoothing method enables parameter updates towards a low friction coefficient, in contrast to the case of the nonsmooth model. For the slipping case, we compare the average loss of the nonsmooth model with that of the smoothing method in Fig.~\ref{figure:loss_compare}. In the figure, we observed that using the proposed smoothing method can lead to convergence at better local optima, achieving a lower loss value. Specifically, Fig.~\ref{figure:smoothing_compare} shows the average loss during the experiments shown in Fig.~\ref{fig:online_estimation} according to the smoothing parameter. We observed that when the smoothing parameter $\rho_\mathrm{t}$ is excessively increased or decreased, the effect for the convergence towards a lower loss may be reduced, as~\cite{kim2023contactimplicit}.
% We observed that, with the smoothed gradients, the estimated friction coefficient updates are smoothly executed within the same initial range.

Moreover, we conducted friction coefficient identification with various initial conditions in 0.05 units from 0.05 to 1.0, as shown in Fig.~\ref{figure:different_initial_condition}.
In the experiment, we used the same experimental data as that for Fig.~\ref{fig:online_estimation}. We compared the performance of friction coefficient identification between the proposed model and the nonsmooth model on slippery terrain. As shown in Fig.~\ref{figure:different_initial_condition_nonsm}, 
when employing nonsmooth gradients, the lack of informative gradients can lead to the failure to identify the lower friction coefficient. We observed that the issue often occurs as the gap between the estimated friction coefficient and the actual one is large. In contrast, as Fig.~\ref{figure:different_initial_condition_prop}, our proposed smoothing method solves the failure issue of parameter identification, even under various initial conditions.
%\textcolor{blue}{We observed that the proposed gradient allows for improved friction coefficient identification under various initial conditions.}  



Considering the results, we observed that the proposed smoothing method provides advantages for friction coefficient identification under various initial conditions, even when a high initial friction coefficient leads to non-informative gradients. These advantages can be utilized in various model-based frameworks. For instance, model-based controllers for legged robots often employ a user-defined friction coefficient to compute control inputs based on the Coulomb friction cone constraint. The friction coefficient is typically determined by heuristic tuning for their tasks~\cite{jenelten2019legged,hong202realtime}. A high friction coefficient can be selected to optimize control inputs, leveraging more tangential ground reaction forces. However, using a high friction coefficient on slippery terrain may cause the robot to slip, as the control inputs are computed based on a high friction coefficient. Consequently, there is a need for real-time friction coefficient identification that performs fast and consistently on slippery terrain, even with a high initial friction coefficient. The proposed framework can identify the friction coefficient under various initials, handling non-informative gradients.

%\textcolor{blue}{For online system identification, however, the problem arises when the dynamics model does not predict slipping, even if it actually occurs, due to gaps in modeling the friction coefficient. In this case, the contact impulse is not attached to the friction cone and becomes independent of the friction coefficient. This independence also extends to the contact dynamics of the system~\cite{raisim,werling2021fast}. Consequently, using nonsmoothing methods, the gradients can become non-informative, hindering online identification of the friction coefficient.}

%\textcolor{blue}{As shown in Fig.~\ref{figure:different_initial_condition}, when using system identification of the nonsmoothing gradients, the issue of non-informative gradients become predominant when the predicted friction coefficient values are larger than the actual values. Therefore, the friction coefficient identification that works across various initial conditions becomes essential. Unlike gradients without smoothing, our smoothed gradients allow for consistent friction coefficient identification across various initial conditions.}
%We observe that using proposed smoothing significantly reduces non-informative gradients and allows for uninterrupted parameter updates within various initial conditions, compared to nonsmoothed gradients.
%Conversely, when the initial friction coefficient is lower than 0.70, the estimated friction coefficient and convergence rates when using the nonsmoothed gradient are comparable to those using the proposed smoothed gradient, and the non-informative gradient issue becomes negligible. 

% TODO 
% Moreover, Fig.~\ref{fig:loss_smoothing} shows the average loss value computed using all the loss values obtained from the experiments shown in Fig.~\ref{fig:online_estimation}. As illustrated in Fig.~\ref{fig:loss_smoothing}, the average loss value was decreased when the proposed smoothed gradient was applied, allowing stable and smooth parameter updates to decrease the loss function since the nullified gradient issue did not occur. However, the non-smoothed analytic gradient about the friction coefficient led to the zero gradient issue and a higher loss value than the smoothed gradient under certain initialized conditions.

% If the estimated friction coefficient is larger than the friction coefficient in the actual terrain, the modeling gap in contact dynamics is increased when the robot slips.
% As seen in Figure~\ref{figure:different_initial_condition}, 


\subsection{Comparison with Randomized Smoothing Methods}
\begin{figure}
    \centering
    \includegraphics[width=1.0\columnwidth]{Figures/Experiment/Experiment_6.pdf}
    \caption{Compared to the baselines, the proposed methods can achieve fast and consistent friction coefficient identification in real-time.}
    \label{figure:compare_randomize}
\end{figure}
% In this section, we evaluate the effectiveness of the proposed gradients in this paper by comparing the results obtained using proposed smoothed gradients, randomized smoothing, and nonsmoothed gradients, in Fig.~\ref{figure:compare_randomize} and Table~\ref{table:total_compare}. The results shown in Fig.~\ref{figure:compare_randomize} are obtained from the same data as shown in Fig~\ref{fig:online_estimation}. The randomized smoothing utilizes 30 samples with parallel computing to obtain the stochastic gradient. We observe that the estimates using the proposed smoothing method are comparable to those using the randomized smoothing. Moreover, like randomized smoothing, as discussed in~\cite{Pang2023TRO,le2024leveraging}, it is observed that the proposed gradient mitigates the issue of non-informative gradients.


In this section, we compare the performance of friction coefficient identification using the proposed gradients with baseline methods. For the baselines, we adopt the online system identification using nonsmooth gradients~\cite{chen2022real} and using randomized smoothing methods~\cite{le2024leveraging, Pang2023TRO}: specifically first-order and zeroth-order randomized smoothing methods. The randomized smoothing methods utilize 50 samples with parallel computing to obtain stochastic gradients. We conducted seven experiments where the robot slipped on slippery terrains, with initial estimates of 0.8. 

The results are summarized in Fig.~\ref{figure:compare_randomize}, which presents histograms of the estimated friction coefficient, computation time for solving the optimization problem for~\eqref{osi_opt}, and average loss. We observe that the proposed smoothed gradient results in lower computation times than other randomized smoothing methods. As noted in~\cite{Pang2023TRO}, while randomized smoothing methods can address the lack of informative gradients, they require longer computation times due to sampling. Furthermore, it is observed that the mean and standard deviation of estimates without the smoothing method are higher than those using smoothing methods. This can be attributed to the lack of informative gradients, which causes the gradient-based optimization strategy to fail in friction coefficient identification.
%In cases without smoothing, the estimates often remain unchanged or are delayed due to these bad local minima.}
 % It is observed that estimates often remain unchanged from the current values even with increased confidence scores, resulting in a higher mean and standard deviation.
% 스탠다드 배리에이션이 다른 방법들보다 더 높았다. 평탄화하지 않은 기울기를 사용하면 유용그래디언트 부족의 결과를 낳기 때문에, 최적화는 bad local minima에 갇히게 되었다. 

\subsection{The Effects of Data Rejection Method}

% \begin{figure}
%     \centering
% \includegraphics[width=1.0 \columnwidth]{Figures/Experiment/fig78merge_nonslip.pdf}
%     \caption{The results of an additional experiment in which the legged robot traverses nonslippery terrain. The rejection algorithm reduces the drift in estimating the friction coefficient by utilizing a rejection score and stabilizes the parameter updates.}
%     \label{fig:nonslippery_estimation}
% \end{figure}
In this section, we describe the benefits of data rejection methods by comparing updates based on confidence scores with and without rejection methods. In the right bottom figure of Fig.~\ref{fig:online_estimation}, friction coefficient identification with the rejection method is more consistent than without it, especially on nonslippery terrain.

As illustrated in the bottom right figure of Fig.~\ref{fig:contact_vel}, not using the rejection method can lead to an increased confidence score, even on nonslippery terrain. If the confidence score increases on nonslippery terrain, the parameter updates can be conducted using non-informative observations, leading to undesired and inconsistent friction coefficient identification.

However, with the proposed rejection method, the confidence score on nonslippery terrain does not increase as much as it does without the method, allowing for consistent performance in friction coefficient identification. Furthermore, the rejection method does not significantly impede increases in the confidence score when the robot slips. As shown in the bottom-left of Fig.~\ref{fig:contact_vel}, when the robot slips on slippery terrain, the confidence score with the rejection method is comparable to one without the method. The upper figures of Fig.~\ref{fig:contact_vel} show that the data with high contact velocity following contact initiations can be excluded from parameter identification.


%\textcolor{blue}{Consequently, it is observed that combining the confidence score-based updates with the proposed rejection method stabilizes parameter updates, without significantly impeding the increase in confidence scores when the robot slips on slippery terrains.}
% Additionally, the rejection method does not significantly impede the updates for the estimated friction coefficient when the robot slips on slippery terrain. In such cases, the increase in the confidence score with the rejection method is comparable to that without the method.

% Furthermore, it is observed that, for legged robots,  incorporating our proposed rejection method into existing confidence score-based online system identification reduces the drift in friction coefficient estimates and allows for stable friction coefficient estimation. 
% Besides the above experiments, we conducted an additional experiment where the robot only navigated on only nonslippery terrain, as shown in Fig.~\ref{fig:nonslippery_estimation}. Even if the legged robot is traveling on the nonslippery terrain, it is observed that the norm of the tangential contact velocity increased up to 0.842~\si{\meter/\second} at the beginning of contacts, about 19.3~\si{\second}, and the normal contact velocity highly vibrated and increased. As shown in the top figure of Fig.~\ref{fig:nonslippery_estimation}, when these data are included in the optimization problem for system identification, we observe that the parameter optimization process in~\eqref{osi_opt} becomes unstable. Consequently, the estimates tend to drift or diverge. In particular, the estimate increases from the initial value up to 1.0 around 19.8~\si\second and drops from 1.0 to 0.52 around 21~\si\second, although the system is on the nonslip terrain.

% Furthermore, it is observed that, for legged robots,  incorporating our proposed rejection method into existing confidence score-based online system identification reduces the drift in friction coefficient estimates and allows for stable friction coefficient estimation. 


% However, we observe that employing the rejection method stabilizes the optimization process and decreases the drift in estimates. The rejection score increases with higher or fluctuating contact velocity in the normal direction, especially at the beginning of contacts. When the score is over a threshold, the corresponding contact state is excluded from the system identification. Fig.~\ref{fig:nonslippery_estimation} illustrates the contact and sliding states not filtered out after the rejection method. 


%% 

%Also, the rejection score rejects the undesired estimated slip states that typically involve vibrated normal contact velocity.
% The two figures at the bottom in Fig. \ref{fig:online_estimation} show the norm of estimated foot velocity in the normal and tangential direction, including the contact and slip states remaining after the rejection algorithm. The rejection score was assigned to the data for each time step as defined in~\eqref{eq:rejection_score}. When the rejection score exceeded the rejection threshold, the state at the corresponding index was excluded from the online system identification, as shown in Fig.~\ref{fig:online_estimation}.


% The estimated slip state is determined by whether or not the tangential direction foot speed exceeds the slipping threshold of 0.3 as~\cite{Joonha2021RAL}. 
% The second figure at the bottom of Fig.~\ref{fig:online_estimation} represents the norm of the estimated foot velocity in the normal direction and the corresponding rejection score. The figure shows the remaining collected data after the registration algorithm is used to identify the online system. If the rejection score exceeds the rejection threshold, the collected data is excluded from the optimization problem for online system identification in ~\eqref{osi_opt}. The rejection score becomes larger as the vibration of the normal direction of foot speed increases. 



% There is no way to estimate the friction coefficient if the system does not actually slip and does not slip on the model. Nevertheless, 

% The confidence score is increased as the norm of the tangential foot velocity in the contact states is increased. At around time is at 23.5 seconds, the norm of tangential foot velocity gets increased up to about 1 m/s as shown in \ref{fig:online_estimation}, However, it is rejected from being included in online system identification process through rejection algorithm. Therefore, at that moment, the data does not affect the confidence score and parameter update even if the tangential foot velocity gets increased. 

% On the other side, when time is about 23.8 seconds, the norm of tangential direction foot velocity again surge up to 1 $m/s$. 
% At that moment, even if some of data are excluded from online system identification process through the rejection of elastic contact, the remaining data contributes to the parameter updates, increasing confidence score, There is a delay in the rise of the confidence score, which is a delay caused by the time data is accumulated in the buffer and a delay caused by the parameter estimation thread that runs at 10 $Hz$.


% \subsection{Estimation Results}

% \begin{figure}[t]
%     \centering
    
%     \includegraphics[width=0.98\columnwidth]{Figures/Experiment/fig9.pdf}
    
%     \caption{The comparison of friction coefficient estimation in online data. Our proposed method incorporates a smoothed gradient and Hessian, rejection of elastic contact, and updates to the confidence score. Unlike our method, other approaches omit one of these concepts. Our method demonstrates the quickest and most stable system identification compared to others. The non-smoothed gradient remains nullified for about 23.8 seconds, even when other smoothed gradients are active, allowing for parameter updates. The estimated parameter tends to drift without updating the confidence score and using the data rejection algorithm.}
%     \label{fig:cof_est_in_real_experiments}
% \end{figure}

% We proposed the smoothed analytical gradients of contact impulses about the friction coefficient to solve the issue of nullified gradients. In Figure~\ref{figure:different_initial_condition}, various initial condition is set for the comparison of the effects of the modeling gaps in the friction coefficient estimation. The dataset is based on the data that were collected in the online experiments. 

% As shown in the Figure~\ref{figure:different_initial_condition}, the gradient of contact impulse about friction coefficient is nullified due to the modeling gap in the friction coefficient parameter. The nullfied gradients hinder the fast updates for friction coefficients and leads to delay in parameter updates. The greater the quality gap of the modeling parameter, the deeper the nullified gradient issue, and even the parameter may not be updated. 

% The proposed gradients from smoothed conditions tackles this nullified gradient problem, enabling parameter estimation in various initiation conditions.


% \subsection{The Effects of Rejection and Confidence Score}
% Figure \ref{fig:cof_est_in_real_experiments} compares the performance of friction coefficient estimation under different conditions, with and without the inclusion of smoothed gradients, Hessian, rejection of elastic contact, and confidence score, based on the same data that was collected in the online experiments. The proposed method includes all concepts: smoothed contact gradient, contact hessian, data rejection, and confidence score updates. In contrast, the other methods represent friction estimation performances lacking one of these elements. The proposed method was estimated in real-time online as the robot was controlled, while the others regenerated results based on data logged during online experiments but processed offline. The robot starts walking for around 15 seconds, and the actual sliding phase can be identified based on the predicted tangential direction ball velocity shown in \ref{fig:online_estimation}.

% Without smoothed gradients or smoothed Hessian, it leads to the area where the gradient is nullified and the estimated friction coefficient is not updated. This issue causes delays in accurate parameter estimation and fails to guarantee optimal performance in online parameter estimation. In the absence of an elastic-collision rejection algorithm, the parameter gets drifted, which hinders stable parameter estimation.

% Without Confidence Score-based updates, all the optimal friction coefficients could be considered equally weighted. However, contact dynamics, when not sliding, are independent of the friction coefficient. Therefore, the differences between states rolled out through contact dynamics and the actual data in non-sliding conditions are not solely determined by the friction coefficient. In this case, the parameter update is based on the optimized parameter depending only on the noise and bias of the collected data, not frictional sliding events. If the friction coefficient estimated in sliding and non-sliding situations is updated with equal weight, it leads to cumulative estimation errors, causing drift. Implementing confidence score-based updates, which give higher weight to actual sliding cases, can mitigate the effects of drift in friction coefficient estimation as shown in 


% \subsection{The Comparison with Baseline}
% TODO : PLEASE INSERT THE TABLE WHICH DESCRIBES THE EFFECTS OF SMOOTHED CONTACT GRADIENTS.

% \begin{figure}
%     \centering
%     \includegraphics[width=1.0\columnwidth]{Figures/Experiment/fig10.pdf}
%     \caption{The computation time for the proposed online friction coefficient estimation framework. The Boundary of sampling time is 10 Hz. The elapsed time to implement the algorithm is inside the boundary.}
%     \label{fig:computational_time}
% \end{figure}

%\subsection{Computation Time}
%This study also focused on the real-time operability of the proposed online system identification framework. In Fig.~\ref{fig:computational_time}, the logged elapsed time to implement the proposed framework is introduced, corresponding to the results in Fig.~\ref{fig:online_estimation}. The boundary of solve time for the friction coefficient estimation algorithm is set at 10 Hz, and all the computation for the proposed framework is completed within the boundary. The average elapsed time is 55.3 \si{\milli\second} and the maximum is 65.1 \si{\milli\second}.


% \section{Experiment} \label{Sec:Experiment}
% 이 section에서는 우리는 마찰계수 추정 알고리즘을  real quadrupedal robot인 KAIST HOUND에 적용해본 것에 대해 describe합니다. 또한, 실제 실험에서 사용되는 data rejection algorithm과 confidence score의 효과에 대해 설명합니다. 실제 실험 상에서 로봇은 [승우형NMPC]를 기반으로 하는 Nonlinear model predictive controller로 제어가 되며, 80Hz로 run합니다. 로봇의 상태 추정기는 [InEKF랑 미끄럼방지..?]를 기반으로 하고있으며, 1000Hz로 run합니다. 마찰 계수 추정기의 history buffer의 사이즈는 50이며, 0.01s의 간격으로 로봇의 proprioceptive data들을 logging합니다.  마찰 계수 추정을 위한 최적화 문제를 풀기 위하여, A single onboard computer with an Intel(R) Core(TM) i7-11700T CPU @ up to 4.6 GHz가 사용되었습니다.
% 실험은 시뮬레이션과 동일하게 미끄러운 구간과 미끄럽지 않은 구간으로 분류되는 장소를 로봇이 patrol하면서 진행이 됩니다. 미끄러운 구간은 boric acid power를 아크릴 판 위에 뿌림으로써 구현하였습니다. 

% \subsection{Data Rejection}
% \ref{fig:rejection_score}는 예측하는 노멀방향 발 속도의 노름값과 그에 따른 rejection socre값을 나타내며, 이로 인해 rejection되고난 다음의 contact data들을 붉은색 background로 나타낸다. Contact dynamics를 이용하는 파라미터 추정은 접촉이 동반되기 때문에, 실제와의 차이에 큰 영향을 받을 수 있다. \ref{fig:rejection_score}는 실제 Contact이 일어나고 벗어나는 순간에 나타나는 contact 지점 발 추정속도의 z 성분이다. 상태추정기로부터 추정한 현재 상태는 실제 상태와의 차이가 있을 수 있기 때문에 발 속도 추정 및 Contact dynamics를 Forward propagation에 적용하는 것에 영향을 준다. 특히, 미끄러지는 상태에서의 발 추정 및 Contact이 일어나는 시점에서의 발 속도 추정은 실제와의 차이를 야기하는 요소들이며, Contact이 일어났을 때 velocity-based time-stepping scheme기반의 contact dynamics 풀이에서 사용하는 발 속도가 0일 것이라는 가정과는 달리, 실제 발 속도 추정값은 그러하지 못하다. 이러한 데이터들은 co통해ntact dynamics의 가정에 대립하는 특징을 가지고 있기 때문에, 파라미터 추정시, 불안정성과 bias를 야기할 수 있다. 이로 인한 파라미터 추정의 bias와 불안전성을 막기 위하여, data rejection 알고리즘을 사용한다. 

% \subsection{Contact-based Confidence Score}
%\ref{fig:confidence_score}는 예측하는 접선방향 발 속도의 노름값과 그에 따른 confidence score값을 나타내며, 예측한 접선방향 발 속도를 기준으로 미끄러졌다고 판정된 부분들을 파란색 background로 나타낸다. t=23.5s 부근에서 접선방향 발 속도가 threshold를 넘어서 estimated slip state가 true가 되었음에도, \ref{fig:rejection_score}에서 나타난 바와 같이 data rejection을 통해 데이터 버퍼에 포함되는 것이 거부되었기 때문에, confidence score와 파라미터 업데이트에 영향을 주지 않는다. 하지만, t=23.8s 부근에서 접선 방향의 발 속도가 threshold를 넘어서 estimated slip state가 true가 되었고, data rejection을 거쳤음에도, 데이터 버퍼에 포함이 되는 데이터들이 존재하면서, confidence score가 올라가는 것을 확인할 수 있다. 이 때, 데이터 버퍼에 포함되는 시간과 friction coefficient estimation의 추정을 위한 thread가 0.1s마다 돌면서 생기는 delay에 의해, confidence score가 올라가는 시간에 delay가 있음을 확인할 수 있다.

% \subsection{The Friction Coefficient Estimation in Online Data}
%\ref{fig:cof_est_in_real_experiments}는 Online에서 수집한 동일한 Data를 토대로 Smoothing과 Hessian, Data Rejection 그리고 Confidence Score 각각에 대해 포함이 되었을 경우와 아닐 경우에 대한 마찰 계수 추정에 대한 성능을 비교한다. Proposed의 경우, Smoothed contact gradient와 contact hessian, 그리고 data rejection과 confidence score 업데이트 모두를 포함하고 있으며, 나머지는 이러한 요소들이 하나씩 없는 경우에서의 마찰 추정 성능을 나타낸다. Proposed는 로봇이 제어되면서 online으로 실시간 추정되었으며, 나머지는 online 실험에서 로깅된된 데이터를 토대로 offline 상에서 regenerate한 결과들이다. 로봇은 15s경부터 발을 움직이며 걷기 시작하고, 실제 미끄러지기 시작하는 구간은 \ref{fig:confidence_score}의 예측한 접선방향 볼 속도를 기준으로 확인할 수 있다. 
%Smoothing이 없거나 Hessian이 없는 경우, 현재 추정중인 마찰계수와 실제 마찰계수의 차이가 크기 때문에 마찰계수가 특정 구간동안 업데이트가 되지 않는 것을 볼 수 있다. 이러한 경우, 정확한 파라미터의 추정에 딜레이를 야기할 뿐만 아니라, 좋은 파라미터 추정 성능을 보장하지 못한다. Data Rejection이 없는 경우는, 로봇이 미끄러지지 않았음에도 파라미터의 급격한 변화가 일어나는 경우가 생긴다. 이는 상태 추정기와 발 속도 추정에서 생기는 실제와의 오차에 의하거나, 발의 탄성 충돌에 의해 생기는 마찰 계수 추정의 bias에 의한 것이므로, 안정적인 파라미터 추정을 방해한다.
% Confidence score based update가 없다면, 모든 최적의 마찰계수들이 서로 가중치가 동등한 것이라고 볼 수 있다. 하지만, 미끄러지지 않을 때의 contact dynamics는 마찰 계수에 독립적이기 때문에, 미끄러지지 않은 상황에서 contact dynamics를 통해 rollout한 상태들과 실제 data와의 차이는 단순히 cof만으로 결정되는 부분이 아니다. 따라서, 미끄러지지 않는 상황에서 추정된 cof 값은 현재 추정중인 cof에서 약간의 bias를 야기한다. 이로 인해, 미끄러지는 상황과 미끄러지지 않는 상황에서 추정하는 마찰 계수의 값을 동일한 가중치로 업데이트한다면, 추정 오차가 누적되어 drift를 야기한다. 실제 미끄러지는 경우에 대한 가중치를 높이는 confidence score based update를 통해 마찰 계수 추정시 생기는 drift의 효과를 억제할 수 있다.

% \subsection{The Friction Coefficient Estimation in Online Data}
% 본 연구에서 제안하는 Online Friction Coefficient 추정 프레임워크의 실시간성을 보장하기 위해 실제 풀이시간을 검증할 필요가 있다. %\ref{computational_time}는 \ref{fig:cof_est_in_real_experiments}의 Proposed에 해당하는 마찰 계수 추정 알고리즘이 돌아갈 때 로깅된 풀이 시간이다. 마찰 계수 추정 알고리즘을 위한 Boundary of Sampling Time은 10Hz이며, 모두 해당 시간 이내에서 풀리는 것을 확인할 수 있다. 주어진 Solve Time의 평균은 9.9213(ms)이다. 

\section{Conclusion}

In this paper, we introduce \DatasetName, a novel large-scale dataset specifically designed for long-text rendering, addressing the existing gap in datasets capable of supporting such tasks. 
To demonstrate the utility of models in handling long-text generation, we create a dedicated test set and evaluate current state-of-the-art text-to-image generation models.
Additionally, the open availability of a large-scale, diverse, and high-quality long-text rendering dataset like \DatasetName is crucial for advancing the training of text-conditioned image generation models.

There are several promising directions for further enhancing \DatasetName, which we have not explored in this paper due to the increased computational costs these approaches entail: \emph{i}. Multiple rounds of dataset bootstrapping to iteratively improve data quality. \emph{ii}. Generating multiple synthetic captions per image to further expand the dataset corpus.

%\section*{ACKNOWLEDGMENT}
%This work was supported in part by Korea Evaluation Institute of Industrial Technology (KEIT) funded by the Korea Government (MOTIE) under Grant No.20018216, Development of mobile intelligence SW for autonomous navigation of legged robots in dynamic and atypical environments for real application.


% Reference
\bibliographystyle{ieeetr}
\bibliography{IEEEabrv, reference}

\end{document}