% Original - 2023.01.16
\section{Background} \label{Sec:Background}
\begin{figure}
    \centering
    \includegraphics[width=1.0\columnwidth]{Figures/Background/Background_1.pdf}
    \caption{An illustration of contact states covered in rigid-body contact dynamics, excluding an opening contact, and our proposed smoothed conditions. The proposed smoothing is applied to the complementarity condition of Coulomb friction in contact states. In the smoothed conditions, the red line represents the smoothed constraint, while the brown line depicts the nonsmoothed constraint.}
    \label{figure:contact_dynamics}
\end{figure}
Our approach is based on the method of the previous works~\cite{raisim,werling2021fast,kim2022contact,chen2022real}. In this section, based on the studies, we introduce an optimization problem for system identification, contact dynamics, and analytic gradients of the contact impulse with respect to the friction coefficient. 
% Our approach utilizes the bisection method to solve the contact dynamics, proposed by~\cite{raisim}. Additionally, we adopt the analytical gradient of contact impulse proposed in~\cite{werling2021fast}. The focus of this section is to describe the methods employed in our work.

\subsection{Optimization Problem for System Identification}
Consider a discrete-time dynamic model as follows:
\begin{align}
\label{eq:dynamics}
    \mathbf{\hat{x}}_{i+1} & = f(\mathbf{x}_{i},\mathbf{u}_{i},\theta),
\end{align}
where $\mathbf{x}$ is the generalized states and $\mathbf{u}$ is the control input, $\theta$ is the parameter of dynamics, and $f$ is the propagation function of dynamics. 
With the history buffer of states, the optimization for system identification can be defined as follows\cite{chen2022real}:
\begin{equation}
\label{osi_opt_basic}
     \theta^{*}=\arg\min_{\theta} \frac{1}{2}\sum_{i=1}^{H-1} \left\| {\mathbf{\hat{x}}}_{i+1}-\mathbf{x}_{i+1}\right\|^2,
\end{equation}
where $\theta$ is the parameter to be identified and $H$ is the size of buffer. To obtain $\theta^{*}$, the gradient-based strategy can be adapted using a step size of $\alpha$, with $\Delta{\theta}=-\alpha\mathbf{G}$. The gradient of the loss consists of the product of residuals and derivatives with respect to the parameter: $\mathbf{G}=\sum_{i=1}^{H-1} (\frac{df(\mathbf{x}_i,\mathbf{u}_{i},\theta)}{d\theta})^{T}({\mathbf{\hat{x}}}_{i+1} - \mathbf{x}_{i+1})$. Note that if the dynamics is independent of the parameter, for example, $\frac{df(\mathbf{x}_i,\mathbf{u}_{i},\theta)}{d\theta}=0$, which represents a non-informative gradient, the parameter updates,  $\Delta{\theta}$, can go to zeros or become non-informative.
% \begin{equation}
% \label{osi_param_update}
% \mathbf{J}=\sum_{i=1}^{H-1} (\frac{df(\mathbf{x}_i,\mathbf{u}_{i})}{d\theta})^{T}(f(\mathbf{x}_i,\mathbf{u}_i) - \mathbf{x}_{i+1})
% \end{equation}



\subsection{Frictional Contact Dynamics}
%Consider a rigid body articulated system in free motion. The equation of motion is the follows:
%\begin{align}
%\label{eq:dynamics_free_motion}
    % \mathbf{M(q)}\dot{\boldsymbol{\upsilon}} + \mathbf{h}(\mathbf{q},\boldsymbol{\upsilon})& = \boldsymbol{\tau}
%\end{align}
%When the system is in contact with its environment, the dynamics differ from those of free motion~\eqref{eq:dynamics_free_motion}. The rigid body hypothesis introduces contact impulses $\bm{\lambda}$, and the contact impulses are considered in dynamics. 
Consider an articulated rigid body system in contact with its environment. The discrete-time dynamic model of the system is as follows:
\begin{align}
\label{eq:dynamics}
    \mathbf{q}_{i+1} & = \mathbf{q}_i + \mathbf{\boldsymbol{\upsilon}}_{i+1}\Delta t\nonumber,\\
\mathbf{\boldsymbol{\upsilon}}_{i+1} & = \mathbf{M}^{-1}((-\mathbf{h}+\mathbf{B}\boldsymbol{\tau}_i)\Delta t + \mathbf{M}\mathbf{\boldsymbol{\upsilon}}_i+\mathbf{J}^T\bm{\lambda}),
\end{align}
 where $\mathbf{q}\in\mathbb{R}^{n_q}$ is generalized coordinate, $\boldsymbol{\upsilon}\in\mathbb{R}^{n_{\upsilon}}$ is generalized velocity, $\boldsymbol{\tau}\in\mathbb{R}^{n_a}$ is the generalized torque, $\mathbf{M}\in\mathbb{R}^{n_{\upsilon}{\times}n_{\upsilon}}$ represents the joint space inertial matrix, ${\mathbf{h}} \in\mathbb{R}^{n_{\upsilon}}$ accounts for Coriolis, centrifugal and gravitational terms, ${\mathbf{J}}\in\mathbb{R}^{3n_{c}{\times}n_{\upsilon}}$ is the contact Jacobian, $\mathbf{B}$ is an input matrix, and $\Delta t$ is a time step. ${\bm{\lambda}}$ is the vector consisting of contact impulses $\bm{\lambda}_{k}$ at each contact point where ${k=1,\cdots,n_{c}}$.  
% 앞으로 명료함을 위해 dependency를 생략하겠다.
% that maps generalized velocity to the contact space velocity
Each contact impulse $\bm{\lambda}_{k}$ consists of normal components $\lambda^{n}_{k}$ and tangential components $\lambda^{t}_{k}$. Similarly, each contact Jacobian and contact velocity can be distinguished into normal and tangential components.

The relation between contact velocity and contact impulse can be described as follows:
\begin{align}
\label{eq:contactvel}
    \mathbf{v}_{k,i+1}=\mathbf{\sigma}_k+\mathbf{A}_{k} \bm{\lambda}_{k},
\end{align}
where $\mathbf{\sigma}_k:=\mathbf{J}_k \mathbf{M}^{-1} ((-\mathbf{h}+\mathbf{B} \boldsymbol{\tau}_i)\Delta t +\mathbf{J}_{\tilde{k}}^T \bm{\lambda}_{\tilde{k}}+\mathbf{M} \boldsymbol{\upsilon}_i)$. 
$\mathbf{A}_{k}:=(\mathbf{J}_k \mathbf{M}^{-1} \mathbf{J}_k^T)^{-1}$ is the apparent inertia matrix at $k$-th contact point, $\tilde{k}$ denotes indices except for $k$, and $\textbf{v}_{k,i+1}$ is the contact velocity at the $k$-th contact point for the next time step.

The contact impulse $\bm\lambda_{k}$ and contact velocity $\textbf{v}_{k,i+1}$, governed by the following conditions and principle constraints—the Signorini condition: $0\leq\text{g}^{n}_{k,i+1}\perp{{\lambda}^{n}_{k}}\geq0$ where $\text{g}^{n}_{k,i+1}$ is a gap between $k$th contact bodies, Coulomb's friction cone constraint parameterized by the friction coefficient $\mu$: $\| \lambda^{t}_{k} \|_{2} \leq \|\mu\lambda^{n}_{k}\|_{2} $, and the maximum dissipation principle—allow for the classification of contact states as shown in Fig.~\ref{figure:contact_dynamics}. The Signorini condition for velocity level can be employed for indices of closed contacts: $0\leq\text{v}^{n}_{k,i+1}\perp{{\lambda}^{n}_{k}}\geq0$.
% Signorini condition outlines the non-interpenetration constraints of rigid bodies, along with the direction of normal contact impulses: $0\leq\textbf{v}^{n}_{k,i+1}\perp{\bm{\lambda}^{n}_{k}}\geq0$.
The Maximum Dissipation Principle states that contact forces are chosen to maximize the dissipation of kinetic
energy.
% : $\| \lambda^{t}_{k} \|_2 \leq \|\mu\lambda^{n}_{k} \|_2$.
% $\boldsymbol{\lambda}_{k} \in \boldsymbol{C}_{\mu} = \left\{\boldsymbol{\lambda}_{k} \mid \| \lambda^{t}_{k} \|_2 \leq \|\mu\lambda^{n}_{k} \|_2  \right\}$.
% \begin{align}
% \label{eq:signorini}
% 0&\leq\textbf{v}^{n}_{k,i+1}\perp{\bm{\lambda}^{n}_{k}}\geq0 
% \end{align}
% \begin{align}
% \label{eq:coulomb}
% \boldsymbol{\lambda}_{k} \in \boldsymbol{C}_{\mu} &= \left\{\boldsymbol{\lambda}_{k} \mid \| \lambda^{t}_{k} \|_2 \leq \|\mu\lambda^{n}_{k} \|_2  \right\}
% \end{align}
The contact impulse can be calculated by solving the following optimization problem~\cite{moreau1977application}:
 % The optimization problem~\eqref{eq:minvel} can be solved by utilizing the bisection method proposed in~\cite{raisim}.
\begin{align}
\label{eq:minvel}
    \min_{\bm{\lambda}_k}{\mathbf{v}_{k,i+1}}^T \mathbf{M}_{k}{\mathbf{v}_{k,i+1}}\\
    s.t. \quad \bm{\lambda}_k\in {\cal{S}}_{\mu}\nonumber,
\end{align}
where ${\cal{S}}_\mu$ is defined by the feasible set of elements satisfying the Signorini condition and Coulomb's friction cone constraint.
%In this work, the contact impulse λ is obtained with
%the per-contact iteration method proposed in [20]. 
% Contact impulse은 강체간 충돌 모델에 의해서 3가지 법칙에 지배를 받는다고 할 수 있다, 시그노리니, 쿨롱law, maximum. 이로 인한 contact impulse lambda와 contact velocity간의 관계 조건은 그림3에서 묘사되어 있다.
 % Consider the rigid body articulated system in contact with its environment. The system is governed by the non
% The feasible set ${\cal{S}}_\mu$ is defined by the set of elements satisfying the two conditions~\eqref{eq:signorini} and~\eqref{eq:coulomb}. In a multiple contact scenario, the contact impulses $\bm\lambda_{1}$,$\cdots$,$\bm\lambda_{k}$ of all the contact points can be obtained by solving the multiple instances of the optimization problem~\eqref{eq:minvel} for all $k=1,2,\cdots,n_{c}$





\renewcommand{\arraystretch}{1.2}
\begin{table}[t!]
\centering
\caption{Comparison between the proposed and the nonsmooth model for the gradient of contact impulses with respect to the friction coefficient.}
\resizebox{\columnwidth}{!}{%
\Large
\begin{tabular}{|c|c|c||cc|cc|}
\hline
\multirow{2}{*}{\textbf{Cases}}                                & \multirow{2}{*}{\textbf{\begin{tabular}[c]{@{}c@{}}Contact state \\ in dynamics\end{tabular}}} & \multirow{2}{*}{\textbf{\begin{tabular}[c]{@{}c@{}}Complementarity\\ Constraints\end{tabular}}} & \multicolumn{2}{c|}{\textbf{Nonsmooth model}}                                                                                                            & \multicolumn{2}{c|}{\textbf{Proposed}}        \\ \cline{4-7}      &                                                                                                      &                                                                                                 & \multicolumn{1}{c|}{\textbf{Gradients}}                                                                                    & \textbf{Updates}                            & \multicolumn{1}{c|}{\textbf{Gradients}}                                                                                            & \textbf{Updates}                            \\\hline \hline
\multirow{4}{*}{\begin{tabular}[c]{@{}c@{}}Actual Slipping\\ 
($\mu_\mathrm{true} \ll \hat{\mu}) $\end{tabular}} & \multirow{2}{*}{Clamping}                                                                            & $\|\textbf{v}^{t}\| = 0$                                                                        & \multicolumn{1}{c|}{\multirow{2}{*}{\textbf{\begin{tabular}[c]{@{}c@{}}$ {\frac{\partial\boldsymbol{\mathbf{\lambda}}}{\partial\mu}}$ = $\mathbf{0}$ \\ (non-informative) \\  \end{tabular}}}} & \multirow{2}{*}{$\|{\Delta\hat{\mu}\| = 0}$}            & \multicolumn{1}{c|}{\multirow{2}{*}{\textbf{\begin{tabular}[c]{@{}c@{}}$ {\frac{\partial\boldsymbol{\mathbf{\lambda}}}{\partial\mu}} \neq \mathbf{0}$ \\ (Informative) \\ \end{tabular}}}} & \multirow{2}{*}{$\|{\Delta\hat{\mu}\|>0}$ } \\ \cline{3-3}     &                                                                                                      & \textbf{$\hat{\mu}\lambda^{n} \textgreater \|\lambda^{t}\|$}                                                              & \multicolumn{1}{c|}{}                                                                                            &                                    & \multicolumn{1}{c|}{}                                                                                                    &                                    \\ \cline{2-7}           & \multirow{2}{*}{Sliding}                                                                             & $\|\textbf{v}^{t}\| > 0 $                                                            & \multicolumn{1}{c|}{\multirow{2}{*}{$\frac{{\partial\boldsymbol{\mathbf{\lambda}}}}{{\partial\mu}} \neq \mathbf{0}$}}                                                          & \multirow{2}{*}{$\|\Delta\hat{\mu}\|>0$} & \multicolumn{1}{c|}{\multirow{2}{*}{${\frac{\partial\boldsymbol{\mathbf{\lambda}}}{\partial\mu}} \neq \mathbf{0}$}}                                                                  & \multirow{2}{*}{$\|\Delta\hat{\mu}\|>0$} \\ \cline{3-3}       &                                                                                                      & $\hat{\mu}\lambda^{n} = \|\lambda^{t}\|$                                                                                & \multicolumn{1}{c|}{}                                                                                            &                                    & \multicolumn{1}{c|}{}                                                                                                    &                                    \\ \hline
\end{tabular}%
}
\label{table:CompareCase}
\end{table}




\subsection{Gradients of Contact Impulse}
 % In this session, we present the concept of the gradient of contact impulse, a previous work proposed by~\cite{werling2021fast}. Since this study focuses on estimating the friction coefficient, we exclude discussing the separating state in the concept. In the study, the description for sliding state is not described with details, so we adopt the extended description in the planar system for simplicity, as described in~\cite{kim2022contact}. 
In this section, we introduce the concept of the gradient of contact impulse with respect to the coefficient of friction, as described in the previous work~\cite{werling2021fast}.  Since the previous study briefly covered the gradient for the sliding state, we adopt more extended three-dimensional descriptions from~\cite{kim2023contactimplicit}. 

Given our framework's focus on estimating the friction coefficient, we consider only states where contact is detected by the state estimator~\cite{Joonha2023TRO}, excluding opening contacts. Whether the contact impulse from contact dynamics~\cite{raisim} touches the friction cone determines if the contact is sliding $\mathbf{s}$ or clamping $\mathbf{c}$. The contact Jacobian can be divided into $\mathbf{J}_\mathbf{c}$ and $\mathbf{J}_\mathbf{s}$, and the contact impulse into $\bm{\lambda}_\mathbf{c}$ and $\bm{\lambda}_\mathbf{s}$, depending on whether each contact index involves clamping or sliding~\cite{werling2021fast}.  For example, given $\mathbf{c}=\{1,3\}$, the corresponding contact Jacobian becomes $\mathbf{J}_{\mathbf{c}}=\left[\mathbf{J}_1^T,~\mathbf{J}_3^T\right]^T$ and corresponding contact velocity becomes $\mathbf{v}_{\mathbf{c},i+1}=\left[{\mathbf{v}_{1,i+1}}^T,~{\mathbf{v}_{3,i+1}}^T\right]^T$. The contact velocity~\eqref{eq:contactvel} can be expressed with the subscripts:
% Contact constraint의 non-smoothness를 이용하여 analytical gradient를 유도할 수 있다. 이를 위해,
%  ~\eqref{eq:contactvel}로부터, clamping에서의 contact velocity를 다음과 같이 superscript n과 t로써 표현할 수 있다. 본 연구는 마찰 추정에 focusing을 맞추기에 Fig.~\ref{figure:contact_dynamics}에서 나타난 3가지 contact state 중 separating에 대한 서술은 제외한다.
 \begin{align}
\label{eq:next_state_clamping}
\textbf{v}_{k,i+1}\nonumber&=\mathbf{J}_{k}\mathbf{M}^{-1}((-\mathbf{h}+\mathbf{B} \boldsymbol{\tau}_i)\Delta t+\mathbf{M}\boldsymbol{\upsilon}_i+{\mathbf{J}_\mathbf{c}}^T \bm{\lambda}_{\mathbf{c}}+{\mathbf{J}_\mathbf{s}}^T \bm{\lambda}_{\mathbf{s}}).
\end{align}

In the sliding state for $k\in\mathbf{s}$, the contact impulse is attached to the friction cone defined by the friction coefficient $\mu$:
\begin{equation}
\label{eq:slip_contact_impulse}
\bm{\lambda}_{k} =  \mathbf{E}_{k}{\lambda}^n_{k},
\end{equation}
where $\mathbf{E}_k = [-\mu\cos(\theta_k),  -\mu\sin(\theta_k), 1]^T$ and $\theta_{k}$ is the direction of the tangential contact velocity at $k$th contact. 

%In ~\cite{werling2021fast}, the analytic gradient of the contact impulse is derived by leveraging the nonsmooth nature of contact constraints, as depicted in Fig.~\ref{figure:contact_dynamics}. 
Considering the constraints in Fig.~\ref{figure:contact_dynamics}, contact velocities at clamping, $\textbf{v}_{\bm{c},i+1}$, and normal contact velocity at sliding $\textbf{v}^{n}_{\bm{s},i+1}$ are zero. By integrating these conditions with~\eqref{eq:contactvel}, the stacked contact impulse $\bm{\lambda}^{\mathrm{contact}}$ can be denoted as follows:
\begin{align}
    \label{eq:vcc_zero}
    \textbf{0} &= \mathbf{A}\bm{\lambda}^{\mathrm{contact}} + \mathbf{b},
%     \text{where}\quad\bm{\lambda}_{cc}&=\begin{bmatrix}
% {\bm{\lambda}^n_\mathbf{c}}^T & {\bm{\lambda}^t_\mathbf{c}}^T& {\bm{\lambda}^n_\mathbf{s}}^T
% \end{bmatrix}^T\nonumber
\end{align}
where
\begin{align}
\bm{\lambda}&^{\mathrm{contact}}=\begin{bmatrix}
{\bm{\lambda}_\mathbf{c}}^T {\bm{\lambda}^n_\mathbf{s}}^T
\end{bmatrix}^T,
\nonumber\\
    \mathbf{A}&=\begin{bmatrix}
    \mathbf{J}_{\mathbf{c}}\nonumber\\
    \mathbf{J}^n_{\mathbf{s}}
    \end{bmatrix}
    \mathbf{M}^{-1}
    \begin{bmatrix}
    \mathbf{J}_{\mathbf{c}}\\
    \mathbf{E}^{T}_{\mathbf{s}}    \mathbf{J}_{\mathbf{s}}
    \end{bmatrix}^T,\nonumber\\
    \mathbf{b}&=\begin{bmatrix}
    \mathbf{J}_{\mathbf{c}}\\\mathbf{J}^n_{\mathbf{s}}
    \end{bmatrix}\mathbf{M}^{-1}\left((-\mathbf{h}+\mathbf{B} \boldsymbol{\tau}_{i})\Delta{t} + \mathbf{M}\boldsymbol{\upsilon}_i\right),\nonumber
\end{align}
%Considering the contact constraints in Fig.~\ref{figure:contact_dynamics}, the normal and tangential contact velocity at clamping, respectively defined as $\textbf{v}^{n}_{c}$ and $\textbf{v}^{t}_{c}$, becomes zero. In the same manner, the normal contact velocity at sliding, $\textbf{v}^{n}_{s}$ becomes zero. Concatenating all the conditions, the contact impulse $\bm{\lambda}_{cc}$ can be represented as follows:
% \begin{align}
%     \label{eq:vcc_zero}
%     \mathbf{0} &= \mathbf{A}_{cc}\bm{\lambda}_{cc} + \mathbf{b}_{cc}
% %     \text{where}\quad\bm{\lambda}_{cc}&=\begin{bmatrix}
% % {\bm{\lambda}^n_\mathbf{c}}^T & {\bm{\lambda}^t_\mathbf{c}}^T& {\bm{\lambda}^n_\mathbf{s}}^T
% % \end{bmatrix}^T\nonumber
% \end{align}
% With~\eqref{eq:vcc_zero}, the stacked contact impulse $\bm{\lambda}_{cc}$ can be represented as follows:
% \begin{align}
% \label{eq:lambda}
%     \bm{\lambda}_{cc}=-\mathbf{A}_{cc}^{-1}\mathbf{b}_{cc}
% \end{align}
and $\mathbf{E}_{\mathbf{s}}$ is a block diagonal matrix with top-left entry $\mathbf{E}_{s_1}$ and bottom-right entry $\mathbf{E}_{s_n}$. $s_1$ and $s_n$ are the first and last elements of set $\mathbf{s}$, respectively.
Then, the gradient of $\bm{\lambda}^{\mathrm{{contact}}}$ with respect to friction coefficient $\mu$ can be obtained:
\begin{align}
\label{non_smoothed_gradient1}
\frac{\partial \bm{\lambda}^{\mathrm{contact}}}{\partial \mu}= \mathbf{A}^{-1} \frac{\partial \mathbf{A}}{\partial \mu}\mathbf{A}^{-1}\mathbf{b}-\mathbf{A}^{-1}\frac{\partial \mathbf{b}}{\partial \mu}.
\end{align}

Note that the contact impulse in \eqref{non_smoothed_gradient1} depends on the coefficient of friction only when the contact state is sliding. When the sliding condition~\eqref{eq:slip_contact_impulse} is satisfied, the gradient of the contact impulse in the tangential direction can be expressed as follows: 
\begin{align}
\label{non_smoothed_gradient_sliding}
\frac{\partial \bm{\lambda}^{t}_\mathbf{s}}{\partial \mu} = \frac{\partial}{\partial \mu}(\mathbf{E}_{\mathbf{s}}\bm{\lambda}^n_{\mathbf{s}}).
\end{align}

%Take note that the derivatives of nonsmooth dynamics in~\cite{werling2021fast} did not consider the Saltation matrix~\cite{kong2024saltation}, but the effects for variations of time of impact in the time-stepping method. Those gradients involve issues of uninformative gradients~\cite{werling2021fast}. 


