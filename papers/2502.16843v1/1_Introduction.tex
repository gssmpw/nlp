\section{Introduction} \label{Sec:Introduction}
% 20240102) HAVE TO CHANGE 
%%%%%%%%%%%%%%%%%%%%%%%%%%%%%%%%%%%%%%% 2안 @@@@@@@@@@@@@@@@@@@@@@@@@@@@@@@@@@@@@@@@@@@@@@@@@
% 흐름
% 1. legged robot이 challenging 한 환경에서 임무를 성공적으로 수행하기 위해서, 주변환경과의 모델링을 고려하는 것은 중요한 요소 중 하나이다.
\IEEEPARstart{F}{or} legged robots navigating challenging terrain, contact modeling for considering the interaction between the robot and terrain is crucial. The modeling is particularly critical on slippery terrain, where the robots encounter nonlinear and hybrid dynamics due to foot slippage.
% For legged robots to navigate challenging terrains, modeling the interaction between the robots and environments is one of the most important factors. 
% % 2. 특히나 미끄러운 환경에서는, contact을 필수적으로 동반하는 legged robot의 경우에, 컨텍 모델리잉 더 중요해지는데, 이는 다양한 contact events가 벌어지기 때문이다.
% The modeling becomes especially critical in slippery terrains since they encounter nonlinear dynamics when their feet slip.
% 3. 이러한 환경에서의 컨텍은, 제어나 상태추정기 등 다양한 방면에서 집중적으로 모델링되어 연구가 되고 있음. ~~런 연구는 미끄러운 상황 하에서의 키네마틱 정보들을 이용해서 
%Various research has considered the contact events of legged robots on slippery terrain in the field of control or state estimation~\cite{bloesch2013slippery,carius2019trajectory,jenelten2019legged,Joonha2021RAL}, which focuses on the kinematic level of legged robots in the contacts.
% 4. 최근 contact modeling을 dynamic equation과 조건들로 나타낼 수 있게 하는 마찰을 동반한 컨텍 다이나믹스는 legged robot에서 점점 큰 관심을 얻고 있습니다. 이것은 
Recently, contact modelings using rigid body contact dynamics have gained attention in the field of legged robots~\cite{wensing2024opti,lidec2023contact,mujoco,raisim}. 
% 5. 이러한 연구들은 보행 로봇을 위한 simulation이나 보행로봇 dynamics constraint로써 적용되고 있음.
% 6.  legged robot들은 이러한 contact dynamics들을 조건으로 하여 contact에 대한 현상을 접근함. [state estimation, controller]
%For example, some rigid body dynamics simulators such as~\cite{raisim,mujoco} adopt frictional contact dynamics to decrease the sim2real gaps and unphysical behaviors, which is essential to train agents of Reinforcement Learning involving contact interactions. Furthermore, the studies in~\cite{justin2022implicit,zachary2024fast,kim2022contact} have utilized the constraints of the contact dynamics to encompass the contact events for legged robots in the optimization problem of the model-based frameworks.
% 7.  하지만, 실제 마찰계수와 모델링된 마찰계수의 차이가 크다면, contact dynamics를 통한 modeling은 suboptimal한 솔루션을 가져와서, 실제를 제대로 반영하지 못함. 이러한 contact dynamics를 이용한 framework들은 고정된 마찰 계수를 이용하고 있음. 비록 마찰 계수는 이렇듯 contact dynamics의 해에 중요한 영향을 주는 파라미터이지만, 이를 추정하는 것은 여전히 어려운 일




However, the friction coefficient, a critical parameter for Coulomb friction in contact dynamics that substantially influences the dynamics' propagation~\cite{acosta2022validating}, often has its estimated value different from the actual one based on the terrain. Consequently, contact dynamics modeling with an inaccurately estimated friction coefficient can diverge further from real-world dynamics~\cite{varin2020constrained}.
% Although the friction coefficient critically affects the solution of contact dynamics, estimating it poses a challenge for system identification because of its nonlinear and nonsmooth nature of contact dynamics, particularly during slip events.
Although the friction coefficient critically affects the solution of contact dynamics, identifying this parameter poses challenges due to the nonlinear and nonsmooth nature of contact dynamics, particularly during slip events.


Over the years, researchers have actively explored the derivatives of nonsmooth dynamics~\cite{tolsma2002hidden,kong2024saltation}. More recently, for dynamics parameter identification, some studies~\cite{werling2021fast,NEURIPS2018_842424a1,lidec2022differentiable} have focused on differentiable physics simulators that offer the gradients with respect to dynamics parameters. The given gradient is then utilized for gradient-based strategies to handle the optimization problem of system identification. Especially, the authors of~\cite{lidec2022differentiable,jatavallabhula2021gradsim} demonstrated the friction coefficient identification using real collected data. Their approach focused on offline identification, verifying their frameworks on simple systems like a sliding box.

\begin{figure}
    \centering
    \subfloat[]{
        \includegraphics[width=0.55\columnwidth]{Figures/Concept_Diagram/Concept_Diagram_1_a.pdf}
        \label{fig:a}
        }
\hfill
    \subfloat[]{
        \includegraphics[width=0.39\columnwidth]{Figures/Concept_Diagram/Concept_Diagram_1_b.pdf}
        \label{fig:b}
        }
    \caption{We present an online friction coefficient identification framework using proprioceptive measurements for legged robots. \protect\subref{fig:a} With proposed smoothed gradients with respect to the friction coefficient, our framework can handle the issue of non-informative gradients caused by the nonsmooth contact dynamics in friction coefficient identification \protect\subref{fig:b} An illustration of constraint space for proposed gradients and nonsmooth gradients.}
    \label{fig:wrong_cof_in_contact_model}
\end{figure}


However, due to the nonsmooth constraints in contact dynamics, exact analytic gradients often become non-informative~\cite{le2024leveraging}. This issue can impede the exploration for
the lower loss solution in the optimization process due to the numerical challenges posed by the complementarity condition~\cite{werling2021fast}.
% poses significant challenges for gradient-based strategies.
To tackle the issue, various studies have proposed smoothing techniques~\cite{zachary2024fast,kim2022contact,le2024leveraging,Pang2023TRO}. Recently, the author in~\cite{le2024leveraging} adopted the randomized smoothing used in reinforcement learning to mitigate the nonsmoothness issue in optimal control systems. In particular, the authors in~\cite{Pang2023TRO} showed that applying randomized smoothing methods to motion planning exhibited performance comparable to that of analytic smoothing. However, they also observed that those methods required longer computation times than the analytic smoothing method since they rely on the sampling methods.


For the legged robot's friction coefficient identification, the authors in~\cite{yu2017preparing} demonstrated the network-based friction coefficient identification framework in simulation. Furthermore, the study in~\cite{focchi2018slip} validated the friction coefficient estimation in simulation, using ground reaction forces estimated through the legged robot's joint torque measurements. The work in ~\cite{jenelten2019legged} proposed a slip estimation framework that considers kinematics during slip events but not friction-related dynamics. In~\cite{lee2020learning}, a learning-based decoder was used to restore friction coefficients in slippery terrains, while achieving a robust controller. This approach required extensive training time with simulation data, and did not provide gradient information for parameters during inference. Recently, the study~\cite{chen2022real} proposed an online system identification method that uses a confidence score-based update integrated with a model predictive controller. This study introduced the confidence score for various parameters, including the friction coefficient, to assess the confidence of data used for system identification. However, the study verified system identification for several parameters, except for the friction coefficient, using robot arms with unknown end-effector mass distribution.
%  A vision-based terrain parameter identification for a legged robot has been proposed in~\cite{ewen2024you} using robot-centric semantic mapping. 

In this study, we aim to develop an online friction coefficient estimation framework for legged robots using proprioceptive measurements. The main contributions of this paper can be summarized as follows:
\begin{itemize}
\item We propose analytic smoothed gradients of contact impulses with respect to the friction coefficient to tackle the lack of informative gradients. 
\item We introduce a rejection method that excludes data with high normal contact velocity while using the confidence score-based parameter updates of~\cite{chen2022real}.
\item Through our proposed methods, the friction coefficient identification for legged robots can achieve faster and more consistent performance than using randomized smoothing~\cite{Pang2023TRO,le2024leveraging} and the nonsmoothed approach~\cite{chen2022real}.
% \item With these technical methods, we solve the issue of non-informative gradients while achieving real-time performance compared to the system identification using randomized smoothing~\cite{Pang2023TRO,le2024leveraging}, and improve the confidence score-based update~\cite{chen2022real} for legged robots.
\item We validate the proposed framework with the KAIST HOUND quadrupedal robot hardware~\cite{shin2022hound}.
\end{itemize}

% To the best of our knowledge, our work is the first to successfully integrate analytic smoothed gradients of contact impulses with respect to the friction coefficient into online system identification, determining the friction coefficient for actual legged robots in real time.
% Dojo는 mu에 대한 gradient가 아니고 tangential 에 대한 relaxation 했으나, contact impulse에 대한것드링ㅁ.
% Dojo는... 

The remainder of this paper is organized as follows: Section \ref{Sec:Background} introduces the background of this study. Section \ref{Sec:Method} details the proposed methodologies, and Section \ref{Sec:Experiment} describes experimental results. Finally, Section \ref{Sec:Discussion} and \ref{Sec:Conclusion} presents the discussion and conclusion, respectively.


% % apply the confidence score-based update used in~\cite{chen2022real}, with slip-related confidence score proposed by this work.
% The remainder of this paper is organized as follows: 
% Sec.~\ref{Sec:Background} introduces the backgrounds of this study.
% Sec.~\ref{Sec:Method} states the smoothed conditioned analytical gradients of contact impulses about the friction coefficient and the optimization problem this study intends to solve for system identification and describes the data rejection algorithm.
% Sec.~\ref{Sec:Experiment} introduces the experiment results to verify the proposed framework using an actual quadrupedal robot, KAIST HOUND~\cite{shin2022hound}. Finally, Section~\ref{Sec:Conclusion} presents a conclusion.\
