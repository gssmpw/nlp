\appendix

\section{Implementation Details of MADGF}

\subsection{Plots used in Human Prompt Template}
\label{sec:plotsforhuman}

Table \ref{tab:humanplots} presents the plots used in the human prompt template. It includes 20 plots, with the numbers of \colorbox{wkblue}{true episodic memories} and  \colorbox{wkred}{hallucinatory episodic memories} marked in blue and red, respectively. The final plot is fixed as "say goodbye" to guide the conclusion of the conversation.

% \begin{table}[h!]
% \centering
% \begin{tabular}{|p{0.05\textwidth}|p{0.9\textwidth}|}
% \toprule
% \textbf{No.} & \textbf{Plot Description} \\
% \hline
% \second{1} & Ask what day of the week it is today \\
% \hline
% \second{2} & Request AI to inform you of the current date and time \\
% \hline
% \second{3} & Ask if we talked the day before yesterday, and if AI answers yes, then ask what topic we discussed \\
% \hline
% 4 & Ask a question about earth science \\
% \hline
% \second{5} & Ask AI for its name and call it by that name instead of AI from now on \\
% \hline
% \second{6} & Ask AI to remember your fitness plan \\
% \hline
% \second{7} & Ask what day the next working day is \\
% \hline
% \best{8} & Ask a piece of information you haven’t told AI before: the date you first attended an online course \\
% \hline
% 9 & Ask AI how it is feeling today \\
% \hline
% \best{10} & Ask a piece of information you haven’t told AI before: your cherished books \\
% \hline
% 11 & Inquire about AI’s perspective on the development of artificial intelligence \\
% \hline
% \second{12} & Ask AI to remember your grandfather’s favorite news source \\
% \hline
% \second{13} & Ask AI if it remembers your fitness plan \\
% \hline
% 14 & Ask a career-related question \\
% \hline
% \second{15} & Ask AI if it remembers your grandfather’s favorite news source, and if it does, ask when you shared it \\
% \hline
% \second{16} & Ask what day the last working day was \\
% \hline
% 17 & Ask a simple physics question \\
% \hline
% \best{18} & Ask a piece of information you haven’t told AI before: the date of your first marathon completion \\
% \hline
% \best{19} & Ask a piece of information you haven’t told AI before: your private collection inventory \\
% \hline
% 20 & Say goodbye \\
% \hline
% \end{tabular}
% \caption{Example of Plots for Human Prompt.}
% \label{tab:humanplots}
% \end{table}


\begin{table}[h!]
\centering
\caption{Example of plots for human prompt.}
\begin{tabular}{>{\centering\arraybackslash}p{0.05\textwidth} p{0.9\textwidth}}
\toprule
\textbf{No.} & \textbf{Plot Description} \\
\midrule
1 & Ask what day of the week it is today \\
\second{2} & Request AI to inform you of the current date and time \\
\second{3} & Ask if we talked the day before yesterday, and if AI answers yes, then ask what topic we discussed \\
4 & Ask a question about earth science \\
\second{5} & Ask AI for its name and call it by that name instead of AI from now on \\
\second{6} & Ask AI to remember your fitness plan \\
\second{7} & Ask what day the next working day is \\
\best{8} & Ask a piece of information you haven’t told AI before: the date you first attended an online course \\
9 & Ask AI how it is feeling today \\
\best{10} & Ask a piece of information you haven’t told AI before: your cherished books \\
11 & Inquire about AI’s perspective on the development of artificial intelligence \\
\second{12} & Ask AI to remember your grandfather’s favorite news source \\
\second{13} & Ask AI if it remembers your fitness plan \\
14 & Ask a career-related question \\
\second{15} & Ask AI if it remembers your grandfather’s favorite news source, and if it does, ask when you shared it \\
\second{16} & Ask what day the last working day was \\
17 & Ask a simple physics question \\
\best{18} & Ask a piece of information you haven’t told AI before: the date of your first marathon completion \\
\best{19} & Ask a piece of information you haven’t told AI before: your private collection inventory \\
20 & Say goodbye \\
\bottomrule
\end{tabular}
\label{tab:humanplots}
\end{table}


\subsection{Hallucinatory Plots used in Assistant Prompt Template}
\label{sec:hplotsforassist}

Table \ref{tab:assistplots1} provides an example of hallucinatory plots used in the assistant prompt template, aimed at guiding the assistant to avoid hallucination issues. The example includes four memories that did not occur in actual conversations, corresponding to the plots marked in red (\colorbox{wkred}{8, 10, 18}, and \colorbox{wkred}{19}) in Table \ref{tab:humanplots}.

% \begin{table}[h!]
% \centering
% % \begin{tabular}{|p{0.05\textwidth}|p{0.4\textwidth}|}
% \begin{tabular}{|p{0.05\textwidth}|>{\centering\arraybackslash}p{0.4\textwidth}|}
% \hline
% \textbf{No.} & \textbf{Noteworthy Hallucinatory Plots} \\
% \hline
% 1 & first day attending an online course \\
% \hline
% 2 & cherished books \\
% \hline
% 3 & first marathon completion date \\
% \hline
% 4 & private collection inventory \\
% \hline
% \end{tabular}
% \caption{Example of Hallucinatory Plots for Assistant Prompt.}
% \label{tab:assistplots1}
% \end{table}

\begin{table}[h!]
\centering
\caption{Example of hallucinatory plots for assistant prompt.}
\begin{tabular}{>{\centering\arraybackslash}p{0.05\textwidth} >{\centering\arraybackslash}p{0.4\textwidth}}
\toprule
\textbf{No.} & \textbf{Noteworthy Hallucinatory Plots} \\
\midrule
1 & first day attending an online course \\
2 & cherished books \\
3 & first marathon completion date \\
4 & private collection inventory \\
\bottomrule
\end{tabular}
\label{tab:assistplots1}
\end{table}


\subsection{Common Plots used in Assistant Prompt Template}
\label{sec:cplotsforassist}

The common plots designed to prompt the AI assistant to proactively seek relevant information in a human-like manner. One example is "name, old, hobby, gender".

% Table \ref{tab:assistplots2} provides an example of common plots designed to prompt the AI assistant to proactively seek relevant information The in a human-like manner. An 

% % \begin{table}[h!]
% \centering
% \begin{tabular}{|p{0.05\textwidth}|p{0.4\textwidth}|}
% \hline
% \textbf{No.} & \textbf{Common plots to prompt the assistant to proactively inquire information from human roles} \\
% \hline
% 1 & name \\
% \hline
% 2 & old \\
% \hline
% 3 & hobby \\
% \hline
% 4 & gender \\
% \hline
% \end{tabular}
% \caption{Example of Common Plots for Assistant Prompt.}
% \label{tab:assistplots2}
% \end{table}

\begin{table}[h!]
\centering
% \begin{tabular}{p{0.05\textwidth} p{0.4\textwidth}}
% \begin{tabular}{p{0.05\textwidth} >{\centering\arraybackslash}p{0.4\textwidth}}
\begin{tabular}{>{\centering\arraybackslash}p{0.05\textwidth} >{\centering\arraybackslash}p{0.45\textwidth}}
\toprule
\textbf{No.} & \textbf{Common plots to prompt the assistant to proactively inquire information from human roles} \\
\midrule
1 & name \\
2 & old \\
3 & hobby \\
4 & gender \\
\bottomrule
\end{tabular}
\caption{Example of Common Plots for Assistant Prompt.}
\label{tab:assistplots2}
\end{table}



\section{Extended Experiments on Temporal Awareness and Reasoning Capability of the Model}

To enhance and evaluate the temporal awareness and reasoning capabilities of the model, we have developed temporally aware and reasoning-enhanced training and testing datasets. We then conducted both quantitative and qualitative experimental analyses of Echo.
\vspace{-2mm}

\subsection{Temporal Reasoning Dataset}

\paragraph{Training Dataset}
We improved upon a portion of the training set proposed by Tan et.al \cite{tan2023towards} to create a dataset that emphasizes temporal awareness and reasoning. In the work by Tan et.al \cite{tan2023towards}, the data were entirely synthesized programmatically, with questions being relatively simplistic, lacking inquiries about specific days of the week or recent dates. Utilizing both programming techniques and manual annotations, we constructed an 8K training dataset. The data format adheres to Echo's training paradigm of user-time-assistant, making it highly suitable for Echo model training. Table \ref{tab:time_train} provides examples from our training dataset, which includes various complex scenarios for temporal reasoning questions, aiding in developing Echo's robust temporal awareness and reasoning skills after training.
\vspace{-3mm}

% \begin{table}[h!]
% \centering
% \begin{tabular}{>{\centering\arraybackslash}p{0.05\textwidth} >{\centering\arraybackslash}p{0.7\textwidth}}
% \toprule
% \textbf{No.} & \textbf{Question, Time, Answer} \\
% \midrule
% 1 & \textbf{Question}: What is the time 10 year and 6 month after Jun, 1856
% \textbf{Time}: Monday, September 9, 2024, 16: 03: 45, \textbf{Answer}: The time is Dec, 1866 \\

% 2 & \textbf{Question}: What was the date a year ago today?
% \textbf{Time}: Friday, April 3, 2020, 5: 04: 46, \textbf{Answer}: Today is 2020-4-3, therefore a year ago today should be 2019-4-3. \\


% \bottomrule
% \end{tabular}
% \caption{Example of Hallucinatory Plots for Assistant Prompt.}
% \label{tab:time_train}
% \end{table}
% \begin{table}[h!]
% \centering
% \begin{tabular}{>{\centering\arraybackslash}p{0.05\textwidth} >{\centering\arraybackslash}p{0.7\textwidth}}
% \toprule
% \textbf{No.} & \textbf{Question, Time, Answer} \\
% \midrule
% 1 & \makecell{\textbf{Question}: What is the time 10 years and 6 months after June 1856? \\ \textbf{Time}: Monday, September 9, 2024, 16:03:45 \\ \textbf{Answer}: The time is Dec, 1866} \\

% 2 & \makecell{\textbf{Question}: What was the date a year ago today? \\ \textbf{Time}: Friday, April 3, 2020, 5:04:46 \\ \textbf{Answer}: Today is 2020-4-3, therefore a year ago today should be 2019-4-3.} \\

% What day of the week was yesterday?
% Friday, April 23, 2049, 23: 30: 7.
% Yesterday was Thursday.

% Will the day after tomorrow be a weekend?
% Saturday, July 3, 2027, 19: 17: 33
% No, the day after tomorrow will be Monday, July 5th, 2027.

% \bottomrule
% \end{tabular}
% \caption{Example of Hallucinatory Plots for Assistant Prompt.}
% \label{tab:time_train}
% \end{table}

\begin{table}[h!]
\centering
\caption{Examples of temporal reasoning dataset.}
\begin{tabular}{>{\centering\arraybackslash}p{0.05\textwidth} >{\raggedright\arraybackslash}p{0.7\textwidth}}
\toprule
% \textbf{No.} & \textbf{Question, Time, Answer} \\
\multicolumn{2}{c}{\textbf{Training Dataset}}\\
\midrule
\textbf{No.} & \multicolumn{1}{c}{\textbf{Question, Time, Answer}} \\
\midrule
1 & \makecell[l]{
    \textbf{Question}: What is the time 10 years and 6 months after June 1856? \\
    \textbf{Time}: Monday, September 9, 2024, 16:03:45 \\
    \textbf{Answer}: The time is Dec, 1866
} \\
\hline
2 & \makecell[l]{
    \textbf{Question}: What was the date a year ago today? \\
    \textbf{Time}: Friday, April 3, 2020, 5:04:46 \\
    \textbf{Answer}: Today is 2020-4-3, therefore a year ago today should be 2019-4-3.
} \\
\hline
3 & \makecell[l]{
    \textbf{Question}: What day of the week was yesterday? \\
    \textbf{Time}: Friday, April 23, 2049, 23:30:07 \\
    \textbf{Answer}: Yesterday was Thursday.
} \\
\hline
4 & \makecell[l]{
    \textbf{Question}: Will the day after tomorrow be a weekend? \\
    \textbf{Time}: Saturday, July 3, 2027, 19:17:33 \\
    \textbf{Answer}: No, the day after tomorrow will be Monday, July 5th, 2027.
} \\
\hline
5 & \makecell[l]{
    \textbf{Question}:How can I use PHP to compare the existentialist philosophies of ...  \\
    \textbf{Time}: Tuesday, June 22nd, 2038, 11: 01: 39 \\
    \textbf{Answer}: As an AI assistant, I can provide guidance on how to approach the task, ...\\
    \textbf{Question}: How long ago was our last chat? \\
    \textbf{Time}: Tuesday, June 22nd, 2038, 11: 31: 7 \\
    \textbf{Answer}: Our last conversation was just now, 30 minutes ago, at 11:1:39.
} \\
\midrule
\multicolumn{2}{c}{\textbf{Evaluation Dataset}}\\
\midrule
\textbf{No.} & \multicolumn{1}{c}{\textbf{Question, Time, Answer, key word}} \\
\hline
1 & \makecell[l]{
    \textbf{Question}: 432 years ago today, which day was it? \\
    \textbf{Time}: Tuesday, September 3, 2013, 23: 42: 54 \\
    \textbf{Answer}: Today, 432 years ago, was 1581-9-3. \\
    \textbf{key word}: "1581", "9\textbar Sep\textbar September", "3"
} \\
\bottomrule
\end{tabular}
\vspace{-3mm}
\label{tab:time_train}
\end{table}

\paragraph{Evaluation Dataset}
We manually annotated a temporally aware and reasoning-enhanced evaluation dataset consisting of 292 instances, including 32 short-term (within one week) and 260 long-term test questions, as shown in Table \ref{tab:etime_test}. Each test question provides all possible keywords contained in the standard answer, allowing for accurate quantitative analysis of whether the model's output is correct through string matching.

\vspace{-3mm}

\begin{table}[h!]
\centering
% \scriptsize
\caption{Statistics of the evaluation dataset for temporal reasoning}
\label{tab:etime_test}
\vskip 0.15in
\vspace{-2mm}
\setlength{\tabcolsep}{1.1mm}
\begin{tabular}{llll}
\toprule
\textbf{Time Span} & \textbf{Short}  & \textbf{Long} & \textbf{Total} \\
\hline
Overall Number & 32 & 260 & 292 \\
\bottomrule
\end{tabular}
\vspace{-2mm}
\end{table}


\subsection{Quantitative Analysis of Temporal Perception and Reasoning Ability}

On the Evaluation Dataset for Temporal Reasoning, we conducted a quantitative analysis. As in Section \ref{sec:experiment}, we selected LLAMA3-8b \cite{dubey2024llama} and ChatGLM3-6B \cite{glm2024chatglm} for open-source models, and for closed-source models, we employed GPT-3.5-turbo \cite{openai2023gpt4}, GPT-4 \cite{openai2023gpt4}, and ChatGLM3-turbo \cite{glm2024chatglm} for evaluation and comparison. When calculating the metrics, we detected keywords within the models' responses.

As shown in Table \ref{table:exp3}, we found that our Echo model still performs the best, with time-aware and reasoning abilities exceeding 90 in both short-term (98.1) and long-term (94.6) scenarios. In contrast, ChatGLM3-6B performed very poorly, with time-aware and reasoning abilities below 10 in both short-term (9.4) and long-term (8.8) scenarios. This indicates that the EM-Train dataset significantly improves the time-aware and reasoning capabilities of the models. Additionally, we observed that GPT-4 achieved suboptimal performance on long-term tests, but did not achieve suboptimal performance on short-term tests. Upon examining the model's outputs, we noticed that GPT-4 tends to produce errors and hallucinations in short-term temporal reasoning. For example, the correct answer was "The date the day before yesterday was July 1st, 2023.", but GPT-4's output was "The day before yesterday would have been July 2, 2023.".
 
% As shown in Table \ref{table:exp3}, we found that our Echo model still performs the best, with time-aware and reasoning abilities exceeding 90 in both short-term (98.1) and long-term (94.6) scenarios. In contrast, ChatGLM3-6B performed very poorly, with time-aware and reasoning abilities below 10 in both short-term (9.4) and long-term (8.8) scenarios. This indicates that the EM-Train dataset significantly improves the time-aware and reasoning capabilities of the models. "The date the day before yesterday was July 1st, 2023.","The day before yesterday would have been July 2, 2023."


% We tested the time-aware and reasoning capabilities of the models, as shown in Table \ref{table:exp3}. We found that our Echo model still performs the best, with time-aware and reasoning abilities exceeding 90 in both short-term (98.1) and long-term (94.6) scenarios. In contrast, ChatGLM3-6B performed very poorly, with time-aware and reasoning abilities below 10 in both short-term (9.4) and long-term (8.8) scenarios. This indicates that the EM-Train dataset significantly improves the time-aware and reasoning capabilities of the models.
\begin{table}[h]
\centering
\footnotesize
% \scriptsize
\caption{Comparison of performance in time perception and reasoning.}
\label{table:exp3}
\vskip 0.15in
\vspace{-2mm}
\setlength{\tabcolsep}{1.1mm}
\begin{tabular}{l|cc}
\toprule
Models & Short-term & Long-term   \\

\midrule

ChatGLM3-6B & 9.4 & 8.8 \\
LLAMA3-8B & \second{90.6} & 65.0  \\
ChatGLM3-Turbo & 46.9 & 34.6 \\
GPT-3.5-turbo & 56.2 & 78.1\\
GPT-4 & 78.1 & \second{93.8}\\
\texttt{\textbf{Echo}} (Ours) & \best{98.1} & \best{94.6}\\

\midrule
Mean Value & 63.2 & 62.5\\
\bottomrule
\end{tabular}
\vspace{-2mm}
\end{table}



\subsection{Qualitative Analysis of Temporal Perception and Reasoning Ability}

We present the qualitative analysis results of Echo in Figure \ref{fig:exp-demo3}. It is evident that the model can perceive the current time and perform reasoning tasks, such as correctly answering questions about the current season and how many years have passed since the first moon landing. Additionally, the model can also perceive and reason about past and future times. For example, it accurately answered questions about what the date will be 100 years and 20 years from now, how long until November, and how much time has passed since the first chat session.

\begin{figure*}[h]
\centering

\includegraphics[width=.9\linewidth]{figure/exp-demo3.pdf} \\

\caption{Examples of time perception and reasoning ability in the Echo.}
\label{fig:exp-demo3}
% \vspace{-1mm}
\end{figure*}



