%%%%%%%% ICML 2025 EXAMPLE LATEX SUBMISSION FILE %%%%%%%%%%%%%%%%%

\documentclass{article}

% Recommended, but optional, packages for figures and better typesetting:
\usepackage{microtype}
\usepackage{graphicx}
\usepackage{subfigure}
\usepackage{booktabs} % for professional tables

% hyperref makes hyperlinks in the resulting PDF.
% If your build breaks (sometimes temporarily if a hyperlink spans a page)
% please comment out the following usepackage line and replace
% \usepackage{icml2025} with \usepackage[nohyperref]{icml2025} above.
\usepackage{hyperref}


% Attempt to make hyperref and algorithmic work together better:
\newcommand{\theHalgorithm}{\arabic{algorithm}}

% Use the following line for the initial blind version submitted for review:
% \usepackage{icml2025}

% If accepted, instead use the following line for the camera-ready submission:
\usepackage[accepted]{icml2025}

% For theorems and such
\usepackage{amsmath}
\usepackage{amssymb}
\usepackage{mathtools}
\usepackage{amsthm}
\usepackage{makecell}

% if you use cleveref..
\usepackage[capitalize,noabbrev]{cleveref}

%%%%%%%%%%%%%%%%%%%%%%%%%%%%%%%%
% THEOREMS
%%%%%%%%%%%%%%%%%%%%%%%%%%%%%%%%
\theoremstyle{plain}
\newtheorem{theorem}{Theorem}[section]
\newtheorem{proposition}[theorem]{Proposition}
\newtheorem{lemma}[theorem]{Lemma}
\newtheorem{corollary}[theorem]{Corollary}
\theoremstyle{definition}
\newtheorem{definition}[theorem]{Definition}
\newtheorem{assumption}[theorem]{Assumption}
\theoremstyle{remark}
\newtheorem{remark}[theorem]{Remark}

% Todonotes is useful during development; simply uncomment the next line
%    and comment out the line below the next line to turn off comments
%\usepackage[disable,textsize=tiny]{todonotes}
\usepackage[textsize=tiny]{todonotes}
\usepackage{pifont}
% \usepackage[table]{xcolor} % 使用xcolor包并启用表格颜色选项
\usepackage{colortbl} % 使用colortbl包以在表格中使用颜色


\definecolor{wkred}{RGB}{255, 190, 190}
\definecolor{wkblue}{RGB}{210, 230, 250}

\newcommand{\second}{\cellcolor{wkblue}}
\newcommand{\best}{\cellcolor{wkred}}


% The \icmltitle you define below is probably too long as a header.
% Therefore, a short form for the running title is supplied here:
\icmltitlerunning{Submission and Formatting Instructions for ICML 2025}

\begin{document}

% \twocolumn[
% \icmltitle{Submission and Formatting Instructions for \\
%            International Conference on Machine Learning (ICML 2025)}

% % It is OKAY to include author information, even for blind
% % submissions: the style file will automatically remove it for you
% % unless you've provided the [accepted] option to the icml2025
% % package.

% % List of affiliations: The first argument should be a (short)
% % identifier you will use later to specify author affiliations
% % Academic affiliations should list Department, University, City, Region, Country
% % Industry affiliations should list Company, City, Region, Country

% % You can specify symbols, otherwise they are numbered in order.
% % Ideally, you should not use this facility. Affiliations will be numbered
% % in order of appearance and this is the preferred way.
% \icmlsetsymbol{equal}{*}

% \begin{icmlauthorlist}
% \icmlauthor{Firstname1 Lastname1}{equal,yyy}
% \icmlauthor{Firstname2 Lastname2}{equal,yyy,comp}
% \icmlauthor{Firstname3 Lastname3}{comp}
% \icmlauthor{Firstname4 Lastname4}{sch}
% \icmlauthor{Firstname5 Lastname5}{yyy}
% \icmlauthor{Firstname6 Lastname6}{sch,yyy,comp}
% \icmlauthor{Firstname7 Lastname7}{comp}
% %\icmlauthor{}{sch}
% \icmlauthor{Firstname8 Lastname8}{sch}
% \icmlauthor{Firstname8 Lastname8}{yyy,comp}
% %\icmlauthor{}{sch}
% %\icmlauthor{}{sch}
% \end{icmlauthorlist}

% \icmlaffiliation{yyy}{Department of XXX, University of YYY, Location, Country}
% \icmlaffiliation{comp}{Company Name, Location, Country}
% \icmlaffiliation{sch}{School of ZZZ, Institute of WWW, Location, Country}

% \icmlcorrespondingauthor{Firstname1 Lastname1}{first1.last1@xxx.edu}
% \icmlcorrespondingauthor{Firstname2 Lastname2}{first2.last2@www.uk}

% % You may provide any keywords that you
% % find helpful for describing your paper; these are used to populate
% % the "keywords" metadata in the PDF but will not be shown in the document
% \icmlkeywords{Machine Learning, ICML}

% \vskip 0.3in
% ]

\twocolumn[
\icmltitle{Echo: A Large Language Model with Temporal Episodic Memory}

% It is OKAY to include author information, even for blind
% submissions: the style file will automatically remove it for you
% unless you've provided the [accepted] option to the icml2025
% package.

% List of affiliations: The first argument should be a (short)
% identifier you will use later to specify author affiliations
% Academic affiliations should list Department, University, City, Region, Country
% Industry affiliations should list Company, City, Region, Country

% You can specify symbols, otherwise they are numbered in order.
% Ideally, you should not use this facility. Affiliations will be numbered
% in order of appearance and this is the preferred way.
% \icmlsetsymbol{equal}{*}

\begin{icmlauthorlist}
\icmlauthor{WenTao Liu}{11}
\icmlauthor{RuoHua Zhang}{22}
\icmlauthor{Aimin Zhou}{11}
\icmlauthor{Feng Gao}{11}
\icmlauthor{JiaLi Liu}{11}
% \icmlauthor{}{sch}
% \icmlauthor{}{sch}

\end{icmlauthorlist}

\icmlaffiliation{11}{Institute of AI Education, East China Normal University, Shanghai, China}
\icmlaffiliation{22}{School of Computer Science and Technology, East China Normal University, Shanghai, China}

% \icmlaffiliation{comp}{Company Name, Location, Country}
% \icmlaffiliation{sch}{School of ZZZ, Institute of WWW, Location, Country}

\icmlcorrespondingauthor{Aimin Zhou}{amzhou@cs.ecnu.edu.cn}
% \icmlcorrespondingauthor{Firstname2 Lastname2}{first2.last2@www.uk}

% You may provide any keywords that you
% find helpful for describing your paper; these are used to populate
% the "keywords" metadata in the PDF but will not be shown in the document
\icmlkeywords{Machine Learning, ICML}

\vskip 0.3in
]

% this must go after the closing bracket ] following \twocolumn[ ...

% This command actually creates the footnote in the first column
% listing the affiliations and the copyright notice.
% The command takes one argument, which is text to display at the start of the footnote.
% The \icmlEqualContribution command is standard text for equal contribution.
% Remove it (just {}) if you do not need this facility.

% \printAffiliationsAndNotice{\icmlEqualContribution} % otherwise use the standard text.
\printAffiliationsAndNotice{}  % leave blank if no need to mention equal contribution
% \printAffiliationsAndNotice{\icmlEqualContribution} % otherwise use the standard text.

\begin{abstract}


The choice of representation for geographic location significantly impacts the accuracy of models for a broad range of geospatial tasks, including fine-grained species classification, population density estimation, and biome classification. Recent works like SatCLIP and GeoCLIP learn such representations by contrastively aligning geolocation with co-located images. While these methods work exceptionally well, in this paper, we posit that the current training strategies fail to fully capture the important visual features. We provide an information theoretic perspective on why the resulting embeddings from these methods discard crucial visual information that is important for many downstream tasks. To solve this problem, we propose a novel retrieval-augmented strategy called RANGE. We build our method on the intuition that the visual features of a location can be estimated by combining the visual features from multiple similar-looking locations. We evaluate our method across a wide variety of tasks. Our results show that RANGE outperforms the existing state-of-the-art models with significant margins in most tasks. We show gains of up to 13.1\% on classification tasks and 0.145 $R^2$ on regression tasks. All our code and models will be made available at: \href{https://github.com/mvrl/RANGE}{https://github.com/mvrl/RANGE}.

\end{abstract}


\section{Introduction}
Backdoor attacks pose a concealed yet profound security risk to machine learning (ML) models, for which the adversaries can inject a stealth backdoor into the model during training, enabling them to illicitly control the model's output upon encountering predefined inputs. These attacks can even occur without the knowledge of developers or end-users, thereby undermining the trust in ML systems. As ML becomes more deeply embedded in critical sectors like finance, healthcare, and autonomous driving \citep{he2016deep, liu2020computing, tournier2019mrtrix3, adjabi2020past}, the potential damage from backdoor attacks grows, underscoring the emergency for developing robust defense mechanisms against backdoor attacks.

To address the threat of backdoor attacks, researchers have developed a variety of strategies \cite{liu2018fine,wu2021adversarial,wang2019neural,zeng2022adversarial,zhu2023neural,Zhu_2023_ICCV, wei2024shared,wei2024d3}, aimed at purifying backdoors within victim models. These methods are designed to integrate with current deployment workflows seamlessly and have demonstrated significant success in mitigating the effects of backdoor triggers \cite{wubackdoorbench, wu2023defenses, wu2024backdoorbench,dunnett2024countering}.  However, most state-of-the-art (SOTA) backdoor purification methods operate under the assumption that a small clean dataset, often referred to as \textbf{auxiliary dataset}, is available for purification. Such an assumption poses practical challenges, especially in scenarios where data is scarce. To tackle this challenge, efforts have been made to reduce the size of the required auxiliary dataset~\cite{chai2022oneshot,li2023reconstructive, Zhu_2023_ICCV} and even explore dataset-free purification techniques~\cite{zheng2022data,hong2023revisiting,lin2024fusing}. Although these approaches offer some improvements, recent evaluations \cite{dunnett2024countering, wu2024backdoorbench} continue to highlight the importance of sufficient auxiliary data for achieving robust defenses against backdoor attacks.

While significant progress has been made in reducing the size of auxiliary datasets, an equally critical yet underexplored question remains: \emph{how does the nature of the auxiliary dataset affect purification effectiveness?} In  real-world  applications, auxiliary datasets can vary widely, encompassing in-distribution data, synthetic data, or external data from different sources. Understanding how each type of auxiliary dataset influences the purification effectiveness is vital for selecting or constructing the most suitable auxiliary dataset and the corresponding technique. For instance, when multiple datasets are available, understanding how different datasets contribute to purification can guide defenders in selecting or crafting the most appropriate dataset. Conversely, when only limited auxiliary data is accessible, knowing which purification technique works best under those constraints is critical. Therefore, there is an urgent need for a thorough investigation into the impact of auxiliary datasets on purification effectiveness to guide defenders in  enhancing the security of ML systems. 

In this paper, we systematically investigate the critical role of auxiliary datasets in backdoor purification, aiming to bridge the gap between idealized and practical purification scenarios.  Specifically, we first construct a diverse set of auxiliary datasets to emulate real-world conditions, as summarized in Table~\ref{overall}. These datasets include in-distribution data, synthetic data, and external data from other sources. Through an evaluation of SOTA backdoor purification methods across these datasets, we uncover several critical insights: \textbf{1)} In-distribution datasets, particularly those carefully filtered from the original training data of the victim model, effectively preserve the model’s utility for its intended tasks but may fall short in eliminating backdoors. \textbf{2)} Incorporating OOD datasets can help the model forget backdoors but also bring the risk of forgetting critical learned knowledge, significantly degrading its overall performance. Building on these findings, we propose Guided Input Calibration (GIC), a novel technique that enhances backdoor purification by adaptively transforming auxiliary data to better align with the victim model’s learned representations. By leveraging the victim model itself to guide this transformation, GIC optimizes the purification process, striking a balance between preserving model utility and mitigating backdoor threats. Extensive experiments demonstrate that GIC significantly improves the effectiveness of backdoor purification across diverse auxiliary datasets, providing a practical and robust defense solution.

Our main contributions are threefold:
\textbf{1) Impact analysis of auxiliary datasets:} We take the \textbf{first step}  in systematically investigating how different types of auxiliary datasets influence backdoor purification effectiveness. Our findings provide novel insights and serve as a foundation for future research on optimizing dataset selection and construction for enhanced backdoor defense.
%
\textbf{2) Compilation and evaluation of diverse auxiliary datasets:}  We have compiled and rigorously evaluated a diverse set of auxiliary datasets using SOTA purification methods, making our datasets and code publicly available to facilitate and support future research on practical backdoor defense strategies.
%
\textbf{3) Introduction of GIC:} We introduce GIC, the \textbf{first} dedicated solution designed to align auxiliary datasets with the model’s learned representations, significantly enhancing backdoor mitigation across various dataset types. Our approach sets a new benchmark for practical and effective backdoor defense.



\section{Related Work}
\label{sec:relatedwork}

\subsection{Current AI Tools for Social Service}
\label{subsec:relatedtools}
% the title I feel is quite broad

Harnessing technology for social good has always been a grand challenge in social service \cite{berzin_practice_2015}. As early as the 90s, artificial neural networks and predictive models have been employed as tools for risk assessments, decision-making, and workload management in sectors like child protective services and mental health treatment \cite{fluke_artificial_1989, patterson_application_1999}. The recent rise of generative AI is poised to further advance social service practice, facilitating the automation of administrative tasks, streamlining of paperwork and documentation, optimisation of resource allocation, data analysis, and enhancing client support and interventions \cite{fernando_integration_2023, perron_generative_2023}.

Today, AI solutions are increasingly being deployed in both policy and practice \cite{goldkind_social_2021, hodgson_problematising_2022}. In clinical social work, AI has been used for risk assessments, crisis management, public health initiatives, and education and training for practitioners \cite{asakura_call_2020, gillingham2019can, jacobi_functions_2023, liedgren_use_2016, molala_social_2023, rice_piloting_2018, tambe_artificial_2018}. AI has also been employed for mental health support and therapeutic interventions, with conversational agents serving as on-demand virtual counsellors to provide clinical care and support \cite{lisetti_i_2013, reamer_artificial_2023}.
% commercial solutions include Woebot, which simulates therapeutic conversation, and Wysa, an “emotionally intelligent” AI coach, powered by evidenced-based clinical techniques \cite{reamer_artificial_2023}. 
% Non-clinical AI agents like Replika and companion robots can also provide social support and reduce loneliness amongst individuals \cite{ahmed_humanrobot_2024, chaturvedi_social_2023, pani_can_2024, ta_user_2020}.

Present research largely focuses on \textit{\textbf{AI-based decision support tools}} in social service \cite{james_algorithmic_2023, kawakami2022improving}, especially predictive risk models (PRMs) used to predict social service risks and outcomes \cite{gillingham2019can, van2017predicting}, like the Allegheny Family Screening Tool (AFST), which assesses child abuse risk using data from US public systems \cite{chouldechova_case_2018, vaithianathan2017developing}. Elsewhere, researchers have also piloted PRMs to predict social service needs for the homeless using Medicaid data\cite{erickson_automatic_2018, pourat_easy_2023}, and AI-powered algorithms to promote health interventions for at-risk populations, such as HIV testing among Californian homeless \cite{rice_piloting_2018, yadav_maximizing_2017}.

\subsection{Generative AI and Human-AI Collaboration}
\label{subsec:relatedworkhaicollaboration}
Beyond decision-making algorithms and PRMs, advancements in generative AI, such as large language models (LLMs), open new possibilities for human-AI (HAI) collaboration in social services. 
LLMs have been called "revolutionary" \cite{fui2023generative} and a "seismic shift" \cite{cooper2023examining}, offering "content support" \cite{memmert2023towards} by generating realistic and coherent responses to user inputs \cite{cascella2023evaluating}. Their vastly improved capabilities and ubiquity \cite{cooper2023examining} makes them poised to revolutionise work patterns \cite{fui2023generative}. Generative AI is already used in fields like design, writing, music, \cite{han2024teams, suh2021ai, verheijden2023collaborative, dhillon2024shaping, gero2023social} healthcare, and clinical settings \cite{zhang2023generative, yu2023leveraging, biswas2024intelligent}, with promising results. However, the social service sector has been slower in adopting AI \cite{diez2023artificial, kawakami2023training}.

% Yet, the social service sector is one that could perhaps stand to gain the most from AI technologies. As Goldkind \cite{goldkind_social_2021} writes, social service, as a "values-centred profession with a robust code of ethics" (p. 372), is uniquely placed to inform the development of thoughtful algorithmic policy and practice. 
Social service, however, stands to benefit immensely from generative AI. SSPs work in time-poor environments \cite{tiah_can_2024}, often overwhelmed with tedious administrative work \cite{meilvang_working_2023} and large amounts of paperwork and data processing \cite{singer_ai_2023, tiah_can_2024}. 
% As such, workers often work in time-poor environments and are burdened with information overload and administrative tasks \cite{tiah_can_2024, meilvang_working_2023}. 
Generative AI is well-placed to streamline and automate tasks like formatting case notes, formulating treatment plans and writing progress reports, which can free up valuable time for more meaningful work like client engagement and enhance service quality \cite{fernando_integration_2023, perron_generative_2023, tiah_can_2024, thesocialworkaimentor_ai_nodate}. 

Given the immense potential, there has been emerging research interest in HAI collaboration and teamwork in the Human-Computer Interaction and Computer Supported Cooperative Work space \cite{wang_human-human_2020}. HAI collaboration and interaction has been postulated by researchers to contribute to new forms of HAI symbiosis and augmented intelligence, where algorithmic and human agents work in tandem with one another to perform tasks better than they could accomplish alone by augmenting each other's strengths and capabilities  \cite{dave_augmented_2023, jarrahi_artificial_2018}.

However, compared to the focus on AI decision-making and PRM tools, there is scant research on generative AI and HAI collaboration in the social service sector \cite{wykman_artificial_2023}. This study therefore seeks to fill this critical gap by exploring how SSPs use and interact with a novel generative AI tool, helping to expand our understanding of the new opportunities that HAI collaboration can bring to the social service sector.

\subsection{Challenges in AI Use in Social Service}
\label{subsec:relatedworkaiuse}

% Despite the immense potential of AI systems to augment social work practice, there are multiple challenges with integrating such systems into real-life practice. 
Despite its evident benefits, multiple challenges plague the integration of AI and its vast potential into real-life social service practice.
% Numerous studies have investigated the use of PRMs to help practitioners decide on a course of action for their clients. 
When employing algorithmic decision-making systems, practitioners often experience tension in weighing AI suggestions against their own judgement \cite{kawakami2022improving, saxena2021framework}, being uncertain of how far they should rely on the machine. 
% Despite often being instructed to use the tool as part of evaluating a client, 
Workers are often reluctant to fully embrace AI assessments due to its inability to adequately account for the full context of a case \cite{kawakami2022improving, gambrill2001need}, and lack of clarity and transparency on AI systems and limitations \cite{kawakami2022improving}. Brown et al. \cite{brown2019toward} conducted workshops using hypothetical algorithmic tools 
% to understand service providers' comfort levels with using such tools in their work,
and found similar issues with mistrust and perceived unreliability. Furthermore, introducing AI tools can  create new problems of its own, causing confusion and distrust amongst workers \cite{kawakami2022improving}. Such factors are critical barriers to the acceptance and effective use of AI in the sector.

\citeauthor{meilvang_working_2023} (2023) cites the concept of \textit{boundary work}, which explores the delineation between "monotonous" administrative labour and "professional", "knowledge based" work drawing on core competencies of SSPs. While computers have long been used for bureaucratic tasks like client registration, the introduction of decision support systems like PRMs stirred debate over AI "threatening professional discretion and, as such, the profession itself" \cite{meilvang_working_2023}. Such latent concerns arguably drive the resistance to technology adoption described above. Generative AI is only set to further push this boundary, 
% these concerns are only set to grow in tandem with the vast capabilities of generative and other modern AI systems. Compared to the relatively primitive AI systems in past years, perceived as statistical algorithms \cite{brown2019toward} turning preset inputs like client age and behavioural symptoms \cite{vaithianathan2017developing} into simple numerical outputs indicating various risk scores, modern AI systems are vastly more capable: LLMs 
with its ability to formulate detailed reports and assessments that encroach upon the "core" work of SSPs.
% accept unrestricted and unstructured inputs and return a range of verbose and detailed evaluations according to the user's instructions. 
Introducing these systems exacerbate previously-raised issues such as understanding the limitations and possibilities of AI systems \cite{kawakami2022improving} and risk of overreliance on AI \cite{van2023chatgpt}, and requires a re-examination of where users fall on the algorithmic aversion-bias scale \cite{brown2019toward} and how they detect and react to algorithmic failings \cite{de2020case}. We address these critical issues through an empirical, on-the-ground study that to our knowledge is the first of its kind since the new wave of generative AI.

% W 

% Yet, to date, we have limited knowledge on the real-world impacts and implications of human-AI collaboration, and few studies have investigated practitioners’ experiences working with and using such AI systems in practice, especially within the social work context \cite{kawakami2022improving}. A small number of studies have explored practitioner perspectives on the use of AI in social work, including Kawakami et al. \cite{kawakami2022improving}, who interviewed social workers on their experiences using the AFST; Stapleton et al. \cite{stapleton_imagining_2022}, who conducted design workshops with caseworkers on the use of PRMs in child welfare; and Wassal et al. \cite{wassal_reimagining_2024}, who interviewed UK social work professionals on the use of AI. A common thread from all these studies was a general disregard for the context and users, with many practitioners criticising the failure of past AI tools arising from the lack of participation and involvement of social workers and actual users of such systems in the design and development of algorithmic systems \cite{wassal_reimagining_2024}. Similarly, in a scoping review done on decision-support algorithms in social work, Jacobi \& Christensen \cite{jacobi_functions_2023} reported that the majority of studies reveal limited bottom-up involvement and interaction between social workers, researchers and developers, and that algorithms were rarely developed with consideration of the perspective of social workers.
% so the \cite{yang_unremarkable_2019} and \cite{holten_moller_shifting_2020} are not real-world impacts? real-world means to hear practitioner's voice? I feel this is quite important but i didnt get this point in intro!

% why mentioning 'which have largely focused on existing ADS tools (e.g., AFST)'? i can see our strength is more localized, but without basic knowledge of social work i didnt get what's the 'departure' here orz
% the paragraph is great! do we need to also add one in line 20 21?

\subsection{Designing AI for Social Service through Participatory Design}
\label{subsec:relatedworkpd}
% i think it's important! but maybe not a whole subsection? but i feel the strong connection with practitioners is indeed one of our novelties and need to highlight it, also in intro maybe
% Participatory design (PD) has long been used extensively in HCI \cite{muller1993participatory}, to both design effective solutions for a specific community and gain a deep understanding of that community. Of particular interest here is the rich body of literature on PD in the field of healthcare \cite{donetto2015experience}, which in this regard shares many similarities and concerns with social work. PD has created effective health improvement apps \cite{ryu2017impact}, 

% PD offers researchers the chance to gather detailed user requirements \cite{ryu2017impact}...

Participatory design (PD) is a staple of HCI research \cite{muller1993participatory}, facilitating the design of effective solutions for a specific community while gaining a deep understanding of its stakeholders. The focus in PD of valuing the opinions and perspectives of users as experts \cite{schuler_participatory_1993} 
% In recent years, the tech and social work sectors have awakened to the importance of involving real users in designing and implementing digital technologies, developing human-centred design processes to iteratively design products or technologies through user feedback 
has gained importance in recent years \cite{storer2023reimagining}. Responding to criticisms and failures of past AI tools that have been implemented without adequate involvement and input from actual users, HCI scholars have adopted PD approaches to design predictive tools to better support human decision-making \cite{lehtiniemi_contextual_2023}.
% ; accordingly, in social service, a line of research has begun studying and designing for human-AI collaboration with real-world users (e.g. \cite{holten_moller_shifting_2020, kawakami2022improving, yang_unremarkable_2019}).
Section \ref{subsec:relatedworkaiuse} shows a clear need to better understand SSP perspectives when designing and implementing AI tools in the social sector. 
Yet, PD research in this area has been limited. \citeauthor{yang2019unremarkable} (2019), through field evaluation with clinicians, investigated reasons behind the failure of previous AI-powered decision support tools, allowing them to design a new-and-improved AI decision-support tool that was better aligned with healthcare workers’ workflows. Similarly, \citeauthor{holten_moller_shifting_2020} (2020) ran PD workshops with caseworkers, data scientists and developers in public service systems to identify the expectations and needs that different stakeholders had in using ADS tools.

% Indeed, it is as Wise \cite{wise_intelligent_1998} noted so many years ago on the rise of intelligent agents: “it is perhaps when technologies are new, when their (and our) movements, habits and attitudes seem most awkward and therefore still at the forefront of our thoughts that they are easiest to analyse” (p. 411). 
Building upon this existing body of work, we thus conduct a study to co-design an AI tool \textit{for} and \textit{with} SSPs through participatory workshops and focus group discussions. In the process, we revisit many of the issues mentioned in Section \ref{subsec:relatedworkaiuse}, but in the context of novel generative AI systems, which are fundamentally different from most historical examples of automation technologies \cite{noy2023experimental}. This valuable empirical inquiry occurs at an opportune time when varied expectations about this nascent technology abound \cite{lehtiniemi_contextual_2023}, allowing us to understand how SSPs incorporate AI into their practice, and what AI can (or cannot) do for them. In doing so, we aim to uncover new theoretical and practical insights on what AI can bring to the social service sector, and formulate design implications for developing AI technologies that SSPs find truly meaningful and useful.
% , and drive future technological innovations to transform the social service sector not just within [our country], but also on a global scale.

 % with an on-the-ground study using a real prototype system that reflects the state of AI in current society. With the presumption that AI will continue to be used in social work given the great benefits it brings, we address the pressing need to investigate these issues to ensure that any potential AI systems are designed and implemented in a responsible and effective manner.

% Building upon these works, this study therefore seeks to adopt a participatory design methodology to investigate social workers’ perspectives and attitudes on AI and human-AI collaboration in their social work practice, thus contributing to the nascent body of practitioner-centred HCI research on the use of AI in social work. Yet, in a departure from prior work, which have largely focused on existing ADS tools (e.g., AFST) and were situated in a Western context, our paper also aims to expand the scope by piloting a novel generative AI tool that was designed and developed by the researchers in partnership with a social service agency based in Singapore, with aims of generating more insights on wider use cases of AI beyond what has been previously studied.

% i may think 'While the current lacunae of research on applications of AI in social work may appear to be a limitation, it simultaneously presents an exciting opportunity for further research and exploration \cite{dey_unleashing_2023},' this point is already convincing enough, not sure if we need to quote here
% I like this end! it's a good transition to our study design, do we need to mention the localization in intro as well? like we target at singapore

% Given the increasing prominence and acceptance of AI in modern society, 

% These increased capabilities vastly exacerbate the issues already present with a simpler tool like the AFST: the boundaries and limitations of an LLM system are significantly more difficult to understand and its possible use cases are exponentially greater in scope. 

% Put this in discussion section instead?
% Kawakami et al's work "highlights the importance of studying how collaborative decision-making... impacts how people rely upon and make sense of AI models," They conclude by recommending designing tools that "support workers in understanding the boundaries of [an AI system's] capabilities", and implementing design procedures that "support open cultures for critical discussion around AI decision making". The authors outline critical challenges of implementing AI systems, elucidating factors that may hinder their effectiveness and even negatively affect operations within the organisation.


% Is this needed?:
% talk about the strengths of PD in eliciting user viewpoints and knowledge, in particular when it is a field that is novel or where a certain system has not been used or developed or tested before
% \section{\rev{Improved Baselines of LLaVA}}
\section{Mutil-Agent Data Generation Framework}
\label{sec:approach}

The purpose of the Multi-Agent Data Generation Framework (MADGF) is to design multiple human characters interacting with an AI assistant. Through simulating daily conversations, a large multi-turn dialogue dataset enriched with episodic memories is collected for the training of the Echo model. To enhance the diversity and effectiveness of the conversation content, we initially devised three key elements: characters, plots, and environments. Extensive character cards, plots, and temporal information were then generated. Subsequently, we formulated a data generation process that utilizes this information to guide the LLM in producing high-quality episodic memory data (EM-Train).


\subsection{Characters, plots, and Environments}
\label{sec:three key}
\paragraph{Characters}
As illustrated in Figure \ref{fig:character}, the design of character cards encompasses seven attributes: "Name," "Occupation," "Age," "Gender," "Hobbies," "Personality," and "Social Relationships." Specifically, we randomly generated attribute values for all attributes except for "Social Relationships." Subsequently, we utilized the LLM to generate the "Social Relationships" attribute values based on the other six attributes. 


\begin{figure}[t!]
\centering

\includegraphics[width=.85\linewidth]{figure/character.pdf} \\

\caption{Example of character card.}
\label{fig:character}
% \vspace{-1mm}
\end{figure}

\paragraph{Plots}
The plots generated by LLMs differ significantly from actual real-life scenarios. Therefore, we manually created an event library, from which 20 events are sampled to form a plot. The library contains three types of events: common events, real events, and hallucinatory events. Common events are designed to enable the model to generate data based on semantic memory questions and answers while enriching the context. They include routine occurrences in daily life, such as greetings, inquiries about common knowledge, and discussions about career-related issues. Real events are events that have actually occurred and are related to episodic memory. They are used to prompt the human role to ask the Echo assistant if it remembers a related event. Hallucinatory events are fabricated events that have never occurred. They are used to prompt the human agent to ask the AI assistant about non-existent events and simultaneously remind the AI assistant not to be misled. Notably, since all event prompts are removed during the training of the Echo model, hallucinatory events help reduce the LLM's tendency to generate false information and enhance the model's understanding and reasoning abilities regarding episodic memory.


\paragraph{Environments}
In the design of environments, we initially considered only temporal information. We first established a series of time-stamped nodes arranged in chronological order (e.g., Monday, September 4, 2006, 21:42:56, Monday, September 4, 2006, 21:55:38). These time-stamped nodes are then automatically added to the conversation history between the human role and the Echo assistant, indicating the time at which each round of dialogue takes place.

\subsection{Data generation process}
\label{sec:dgp}


\paragraph{Prompt Design}

As illustrated in Figure \ref{fig:template}, we designed distinct prompt templates for both the human role and the Echo assistant. The highlighted sections in the figure are replaced with information from Section \ref{sec:three key}. Specifically:

\begin{itemize}
    \item \textbf{Human Role Prompt}: This includes the character card and all plot details, enabling the LLM to assume various human roles and engage in dialogues with the AI assistant according to different plots.
    
    \item \textbf{AI Assistant Prompt}: This incorporates both hallucinatory plots and common plots. This setup helps the LLM acting as the AI assistant to reduce episodic memory hallucinations and proactively seek relevant information in a human-like manner.
\end{itemize}

Based on these prompt templates, we generate initial prompts for the human and the Echo assistant, denoted as $P_u$ and $P_a$, respectively.


\paragraph{The Pseudocode of Data Generation Process}

Algorithm \ref{alg:data_generation} provides the pseudocode for the data generation process. We initialize and maintain two separate history records, $H_u$ and $H_a$, for the human role and the AI assistant using initial prompts $P_u$ and $P_a$, respectively. In lines 4-12 of Algorithm \ref{alg:data_generation}, we alternately control the two agents representing the human and the assistant to engage in dialogue. Temporal information is incorporated during the conversation in lines 6-8. We check if farewell phrases such as "goodbye" or "talk to you later" appear in the response. If any of these phrases are detected, or if the number of conversation rounds exceeds 60, the stopping criterion is considered to be met, and the current data generation process is terminated. Finally, we remove the initial prompt $P_a$ from $H_a$ to obtain the final dataset, denoted as $Data$, which constitutes one piece of data in our EM-Train dataset.

\begin{figure}[t!]
\centering

\includegraphics[width=.8\linewidth]{figure/template.pdf} \\

\caption{Prompt template in data generation process.}
\label{fig:template}
% \vspace{-1mm}
\end{figure}

\begin{algorithm}[ht]
    \caption{Pseudocode of Data Generation Process}
    \label{alg:data_generation}
    \begin{algorithmic}[1]
        \REQUIRE Initial prompts $P_u$ and $P_a$
        \ENSURE $Data$
        \STATE Initialize $H_u$, $H_a$
        \STATE $H_u \leftarrow H_u \cup P_u$
        \STATE $H_a \leftarrow H_a \cup P_a$
        \WHILE{stopping criterion not met}
            \STATE $answer_u \leftarrow LLM(H_u)$
            \STATE $time \leftarrow RandomNextTime(time)$
            \STATE $H_u \leftarrow H_u \cup answer_u \cup time$
            \STATE $H_a \leftarrow H_a \cup answer_u \cup time$
            \STATE $answer_a \leftarrow LLM(H_a)$
            \STATE $H_u \leftarrow H_u \cup answer_a$
            \STATE $H_a \leftarrow H_a \cup answer_a$
        \ENDWHILE
        \STATE $Data \leftarrow H_a \setminus P_a$
    \end{algorithmic}
\end{algorithm}



\section{NovelSpecies Dataset}
\label{sec:novel_dataset}

Proprietary LMMs like GPT4o~\cite{hurst2024gpt4o} and Gemini~\cite{team2023gemini} are trained on vast online text-image data and proprietary data, both non-public and impossible to inspect. Some open-source and open-data LMMs such as LLaVA~\cite{liu2024improved, liu2024visual} are trained on publicly available image-text datasets. However, the text encoders used by such models are often not open-data, for example LLaVA-1.6 34B uses the closed-data Yi-34B model as its language backbone. Even in the rare cases where both image-text training data and text encoder training data are publicly available, it is still difficult to ascertain whether concepts in your benchmark were seen by your LMM through indirect data leakage (i.e. partial / paraphrased mentions). Due to the above issues, it is difficult to evaluate true novel concept recognition ability with existing datasets. 
% \footnote{Knowledge cutoff date: Dec 2023}

One way to bypass this problem with 100\% guaranteed success is to use a dataset that only contains concepts created / discovered after the LMM's knowledge cutoff, i.e. the latest knowledge cutoff date among all of its textual / visual sub-components. Based on this idea, we curate \textbf{NovelSpecies}, a dataset of novel animal species discovered in each recent year, starting with 2023 and 2024. We provide detailed information for each species, including time of discovery, latin name, common name, family category, textual description, and more. Data will be released upon publication.
% Details are described in Sec.\ref{subsec:NovelSpecies_details}.

To create \textbf{NovelSpecies}, we start by collecting the list of species first described in each year by Wikidata~\cite{wikidata}. Then, to make sure we can curate a visual benchmark of novel species, we manually annotate and filter out extinct species and species with too few publicly available images. After filtering, we end up with a dataset of 64 new species, each consisting of 35 human-verified image instances, thus a total of 2240 images. The images are split into training, validation, and test sets. For each specie, there are 5 training images, 15 validation images, and 15 test images. This data split is consistent with our goal of creating a benchmark dataset for novel concept recognition, where the maximum number of training instances for a completely unseen concept can range from 1 to 5.







% and 2170 images in total, which consist of train, validation, and test sets of equal proportion for all species. Finally, all the images are 















% \section{Datasets}
% \label{sec:dataset}


% \subsection{Confusing Pair Extraction}
% Our focus on confusing pairs arises from the need to strengthen the model's performance in distinguishing between visually similar species—a challenge where LLaVA currently shows limitations. Confusing pairs represent instances where the model's classification often fails, typically due to subtle visual cues or shared features among species within similar taxonomic groups. We designed strategy to extract confusing pair for each dataset.

% \paragraph{INaturalist and Novel Species Dataset} We extract confusing pairs with three-steps as following: 

% \begin{enumerate}
%     \item \textbf{Iterative Subset Selection:} We select a random subset of species in each iteration, sampling across different supercategories. This strategy allows us to identify confusing pairs without overloading the system, progressively building a collection of challenging cases from each subset.
%     \item \textbf{Evaluate Classification Patters:} For each species within a subset, we create prompts in a multiple-choice format, incorporating the image and a randomized list of options from all the species in the subset. Based on the response from LLaVA, we are able to highlight specific species that are commonly mistaken for one another, guiding us in selecting pairs for further analysis. The process is repeated across new subsets, incrementally building an ample dataset of confusing pairs.
%     \item \textbf{Identification of confusing pairs: } We choose a threshhold of 0.2. If class A is misclassified into class B with frequency more than 0.2 in the above multiple-choice setting, we consider the pair to be confusing. 
% \end{enumerate}

% \paragraph{SUN Dataset} We adapted the above methodology for scene classification with minor modification on the subset selection process. Instead of taxonomic groupings, we created subsets by selecting a target scene and the nine most similar scenes based on shared object occurrence. The subsequent steps—classification pattern analysis and confusing pair definition—remained consistent with the species datasets.









% \subsection{Curated INaturalist Dataset}
% In this study, we utilize a random sample of 15 classes from the "Mammals" supercategory of the iNaturalist dataset. Below, we outline the reasoning behind our dataset selection and sampling approach.
% \paragraph{iNaturalist Dataset}
% The iNaturalist dataset is known for its complexity and has proven to be a challenging benchmark for many vision-language models. Due to the extensive diversity and fine-grained nature of the categories, most models do not achieve perfect performance on this dataset, leaving ample room for further improvements.
% \paragraph{Sampling Strategy}
% Given the scale of the iNaturalist dataset, which contains approximately 10,000 classes with 50 images per class, it is necessary to reduce its size for practical purposes. Additionally, current models, such as LLaVA, have limitations in handling an excessive number of options. Therefore, we have opted to sample the dataset to manage the number of classes and reduce the computational load.
% \paragraph{Random Sampling Justification}
% Initially, we considered sampling all species from a single order, family, or genus. However, this approach resulted in classes that were too similar, making the classification task more challenging than our models could handle. By employing random sampling, the selected classes that are likely sufficiently distinct from each other, with only a few potentially confusing cases.

% Random sampling also reduces the risk of introducing human bias into the selection process, making it a more defensible approach compared to sampling based on performance metrics. 

% \subparagraph{Data Filtering}
% iNaturalist dataset contains a large number of noisy or low quality images. To ensure the quality of the dataset, we implemented an automatic filtering process to eliminate low-quality images. This step is crucial to prevent noise from negatively impacting model performance. Common issues in low-quality images include:

% 1. \textbf{Blurriness}: Images where the main subject is not in focus.
% 2. \textbf{Species Not Present}: Instances where the species is not visible (e.g., only showing its nest or footprint).
% 3. \textbf{Incomplete Specimen}: Images depicting only parts of deceased animals or broken bodies.
% 4. \textbf{Obstructions}: Cases where the species is almost entirely blocked by objects, making identification impossible.

% To improve image quality, we use CLIP score to select the images with top scores. Scores are calculated by evaluating similarity score with [
%         "a photo of an animal",
%         f"a photo of a \{common\_name\}"
%     ]. We rank the images according to this score and selected top 100 images. We randomly split the images to obtain 50 images for training, 20 images for validation and 30 images for testing. 



\section{Experiments and analysis} \label{sec:5}
This section presents comprehensive experiments on the ATR2-HUTD dataset to evaluate the effectiveness of the proposed method. 
Section~\ref{sec:4.1} outlines the experimental metrics used. 
Section~\ref{sec:4.2} details the network architecture, comparison methods, experimental setup, and parameter configurations. 
To highlight the superiority of the proposed method, Section~\ref{sec:4.3} provides both quantitative analysis and visual evaluations across all comparison methods. 
Section~\ref{sec:4.4} includes ablation studies to assess the contributions of different model components, while Section~\ref{sec:4.5} presents a parameter sensitivity analysis.
\subsection{Evaluation Indicators}\label{sec:4.1}
To quantitatively assess the performance of the proposed method, we employ three widely recognized evaluation metrics in the HTD field.
\par
\textbf{(\romannumeral1) Receiver Operating Characteristic (ROC)~\cite{ROC, ROC3D}:} 
The ROC curve offers an unbiased, threshold-independent evaluation of detection performance. This paper presents three 2D ROC curves: $( \mathrm{P}_{\mathrm{d}}, \mathrm{P}_{\mathrm{f}})$, $( \mathrm{P}_{\mathrm{d}}, \tau)$, and $( \mathrm{P}_{\mathrm{f}}, \tau)$, along with a 3D ROC curve~\cite{ROC3D} of $(\tau, \mathrm{P}_{\mathrm{d}}, \mathrm{P}_{\mathrm{f}})$ for a comprehensive performance evaluation. A detector with ROC curves closer to the upper left, upper right, and lower left corners generally exhibits superior HTD performance.
\par
\textbf{(\romannumeral2) Area Under the ROC Curve (AUC)~\cite{Zhang2015}:} 
To address challenges in visually comparing ROC curves, we compute the area under each of the three 2D ROC curves: $\text{AUC}_{( \mathrm{P}_{\mathrm{d}}, \mathrm{P}_{\mathrm{f}})}$, $\text{AUC}_{( \mathrm{P}_{\mathrm{d}}, \tau)}$, and $\text{AUC}_{( \mathrm{P}_{\mathrm{f}}, \tau)}$. Larger AUC values indicate better performance, with $\text{AUC}_{( \mathrm{P}_{\mathrm{d}}, \mathrm{P}_{\mathrm{f}})} \to 1$, $\text{AUC}_{( \mathrm{P}_{\mathrm{d}}, \tau)} \to 1$, and $\text{AUC}_{( \mathrm{P}_{\mathrm{f}}, \tau)} \to 0$ signifying superior detection performance. Additionally, two AUC-based metrics are introduced for a more comprehensive evaluation:
\begin{equation}
    \mathrm{AUC}_{\mathrm{OA}} = \mathrm{AUC}_{\left(P_f, P_d\right)} + \mathrm{AUC}_{\left(\tau, P_d\right)} - \mathrm{AUC}_{\left(\tau, P_f\right)},
\end{equation}
\begin{equation}
    \mathrm{AUC}_{\mathrm{SNPR}} = \frac{\mathrm{AUC}_{\left(\tau, P_d\right)}}{\mathrm{AUC}_{\left(\tau, P_f\right)}},
\end{equation}
where higher values of $\mathrm{AUC}_{\mathrm{OA}} \to 2$ and $\mathrm{AUC}_{\mathrm{SNPR}} \to +\infty$ indicate improved detector performance.
% \textbf{(\romannumeral3) Separability Map~\cite{Liu2022}:} The degree of separation between the targets and backgrounds in the detection map is a critical performance indicator for UTD methods. 
% Thus, we also utilize the separability map for quantitative comparison in this study. 
% Specifically, the separability map uses green and blue boxes to represent the statistics of the target and background, respectively. 
% The horizontal line within each box indicates the median value, while the upper and lower whiskers denote the maximum and minimum values, providing a clear representation of the data range and central tendency. 
% \par
% A larger overlap between the two boxes suggests that the statistics of the target and background are similar, indicating poor separation between them. 
% Conversely, less overlap indicates better separation. 
% Moreover, background suppression is considered more effective when the blue box is closer to the ordinate 0, while higher target prominence is indicated when the green box is closer to ordinate 1.
% \clearpage
\subsection{Experimental Details and Settings}\label{sec:4.2}
\textbf{(\romannumeral1) Experimental Details:} 
The experimental setup and details of the proposed method are as follows. Unless otherwise specified, the parameters are applied consistently across all sub-datasets. The method consists of three core components: the RGC module, the HLCL module, and the SPL strategy, each contributing significantly to performance.

In the RGC module, unsupervised clustering is performed using the K-Means~\cite{Sinaga2020} algorithm, with cluster numbers set to 36, 39, and 42 for the lake, river, and sea sub-datasets, respectively, based on environmental complexity and waterbed characteristics.

The HLCL module employs the 3D-ResNet50~\cite{Jiang2019} network for spectral-spatial feature extraction. To enhance robustness and contrastive learning, untargeted FGSM~\cite{GoodfellowSS14} data augmentation is applied with a maximum perturbation of $\epsilon=0.1$ under the $l_{\infty}$ norm. The hybrid-level contrastive learning framework is trained for 50 epochs per SPL iteration. The Adam optimizer is used with a batch size of 256. The initial learning rate is $5\times10^{-3}$, decaying to $5\times10^{-5}$ through a cosine annealing schedule after 100 epochs, and a weight decay of $1\times10^{-4}$ is applied to reduce overfitting.

The SPL strategy is executed for 10 iterations across all sub-datasets to ensure convergence and computational efficiency.

For HUTD, as described in Section~\ref{sec3.4}, we use learned representations combined with basic hyperspectral detectors. To isolate the effect of detectors on performance, we employ two classic detectors, CEM~\cite{KRUSE1993145} and SAM~\cite{Manolakis2002}, as baseline methods.

\textbf{(\romannumeral2) Experimental Settings:} 
We compare the proposed method against several state-of-the-art (SOTA) HTD and HUTD methods, including two traditional HTD detectors (CEM and SAM), two advanced HTD methods (IEEPST~\cite{IEEPST} and MCLT~\cite{Wang2024}), and four HUTD methods (UTD-Net~\cite{Qi2021}, TUTDF~\cite{LiZheyong2023}, TDSS-UTD~\cite{Li2023}, and NUN-UTD~\cite{Liu2024}).

To ensure fairness, each method is executed with the original hyperparameter settings as specified in their respective publications. All experiments are conducted on a machine equipped with seven NVIDIA A6000 GPUs, an AMD Ryzen 5995WX CPU, and 128 GB of RAM, running Ubuntu 22.04.

\subsection{Main Results} \label{sec:4.3}
\textbf{(\romannumeral1) Detection Maps:} Figs.~\ref{fig:C1-1} to~\ref{fig:C1-2} present detection maps from the ATR2-HUTD-Lake sub-dataset, offering a qualitative comparison of the evaluated methods.
The detection maps of other sub-datasets are provided in the supplementary material.
\par
\begin{figure*}[!t]                 
    \centering                    
    \includegraphics[width=2\columnwidth]{images/C1-1.jpg}                     
    \caption{Detection maps of ATR2-HUTD Lake Scene1. (a) Pseudo-color image. (b) Ground truth. (c) CEM. (d) SAM. (e) IEEPST. (f) MCLT. (g) UTD-Net. (h) TUTDF. (i) TDSS-UTD. (j) NUN-UTD. (m) HUCLNet+CEM. (n) HUCLNet+SAM.}                  
    \label{fig:C1-1}    
\end{figure*}
\begin{figure*}[!t]                 
    \centering                    
    \includegraphics[width=2\columnwidth]{images/C1-2.jpg}                     
    \caption{Detection maps of ATR2-HUTD Lake Scene2. (a) Pseudo-color image. (b) Ground truth. (c) CEM. (d) SAM. (e) IEEPST. (f) MCLT. (g) UTD-Net. (h) TUTDF. (i) TDSS-UTD. (j) NUN-UTD. (m) HUCLNet+CEM. (n) HUCLNet+SAM.}                    
    \label{fig:C1-2}    
\end{figure*}
% \begin{figure*}[!t]                 
%     \centering                    
%     \includegraphics[width=2\columnwidth]{images/C1-3.jpg}                     
%     \caption{Detection maps of ATR2-HUTD River Scene1. (a) Pseudo-color image. (b) Ground truth. (c) CEM. (d) SAM. (e) IEEPST. (f) MCLT. (g) UTD-Net. (h) TUTDF. (i) TDSS-UTD. (j) NUN-UTD. (m) HUCLNet+CEM. (n) HUCLNet+SAM.}                      
%     \label{fig:C1-3}    
% \end{figure*}
% \begin{figure*}[!t]                 
%     \centering                    
%     \includegraphics[width=2\columnwidth]{images/C1-4.jpg}                     
%     \caption{Detection maps of ATR2-HUTD River Scene2. (a) Pseudo-color image. (b) Ground truth. (c) CEM. (d) SAM. (e) IEEPST. (f) MCLT. (g) UTD-Net. (h) TUTDF. (i) TDSS-UTD. (j) NUN-UTD. (m) HUCLNet+CEM. (n) HUCLNet+SAM.}                     
%     \label{fig:C1-4}    
% \end{figure*}
% \begin{figure*}[!t]                 
%     \centering                    
%     \includegraphics[width=2\columnwidth]{images/C1-5.jpg}                     
%     \caption{Detection maps of ATR2-HUTD Sea Scene1. (a) Pseudo-color image. (b) Ground truth. (c) CEM. (d) SAM. (e) IEEPST. (f) MCLT. (g) UTD-Net. (h) TUTDF. (i) TDSS-UTD. (j) NUN-UTD. (m) HUCLNet+CEM. (n) HUCLNet+SAM.}                 
%     \label{fig:C1-5}    
% \end{figure*}
% \begin{figure*}[!t]                 
%     \centering                    
%     \includegraphics[width=2\columnwidth]{images/C1-6.jpg}                     
%     \caption{Detection maps of ATR2-HUTD Sea Scene2. (a) Pseudo-color image. (b) Ground truth. (c) CEM. (d) SAM. (e) IEEPST. (f) MCLT. (g) UTD-Net. (h) TUTDF. (i) TDSS-UTD. (j) NUN-UTD. (m) HUCLNet+CEM. (n) HUCLNet+SAM.}                    
%     \label{fig:C1-2}    
% \end{figure*}
Traditional methods, such as CEM and SAM, exhibit significant limitations in underwater environments. CEM struggles with background noise suppression, resulting in false positives, while SAM fails to delineate target boundaries and often misses targets, especially in complex scenarios like the ATR2-HUTD River dataset. Its sensitivity to spectral noise and limited adaptability to spectral variations lead to incomplete detection and poor target-background separation.
\par
\begin{figure*}[!t]                 
    \centering                    
    \includegraphics[width=2\columnwidth]{images/C2-1.jpg}                     
    \caption{ROC curves comparison on ATR2-HUTD Lake Scene1. (a) 3-D ROC curve. (b) 2-D ROC curve of $(P_d, P_f)$. (c) 2-D ROC curve of $(P_f, \tau)$. (d) 2-D ROC curve of $(P_d, \tau)$.}                 
    \label{fig:C2-1}    
\end{figure*}
\begin{figure*}[!t]                 
    \centering                    
    \includegraphics[width=2\columnwidth]{images/C2-2.jpg}                     
    \caption{ROC curves comparison on ATR2-HUTD Lake Scene2. (a) 3-D ROC curve. (b) 2-D ROC curve of $(P_d, P_f)$. (c) 2-D ROC curve of $(P_f, \tau)$. (d) 2-D ROC curve of $(P_d, \tau)$.}                                  
    \label{fig:C2-2}    
\end{figure*}
Advanced land-cover detection methods, including IEEPST and MCLT, also underperform in underwater environments. IEEPST struggles to suppress background interference, particularly when water column spectral signatures overlap with target signatures in the ATR2-HUTD River sub-dataset. While MCLT leverages contrastive learning for feature enhancement, it shows reduced sensitivity to small or low-reflectance targets, hindered by the nonlinearities and spectral noise typical of underwater HSI data. These results underscore the necessity of specialized techniques for HUTD.

Among SOTA HUTD methods, UTD-Net demonstrates notable improvements by effectively unmixing target-water mixed pixels. However, it faces challenges with background interference in scenes with extensive non-target bottom areas, leading to high false positive rates. NUN-UTD improves target identification by preserving weak target spectral signals, yet remains susceptible to background interference when spectral characteristics of the background resemble those of the target, leading to false positives in spectrally overlapping environments.

Physical-based methods, such as TUTDF and TDSS-UTD, enhance background suppression using underwater imaging models and predicted depth values. However, TUTDF's performance declines in complex environments due to depth estimation inaccuracies, leading to inconsistent detection. Similarly, TDSS-UTD struggles in environments with substantial depth variation, such as the ATR2-HUTD River dataset, where depth errors degrade detection accuracy. Variations in underwater imaging mechanisms between deep and nearshore scenes further limit their effectiveness.

In contrast, HUCLNet-based methods consistently outperform the alternatives. By integrating instance-level and prototype-level contrastive learning, these methods effectively detect faint and deeply submerged targets with minimal false positives, enhancing background suppression and detection accuracy. HUCLNet+CEM and HUCLNet+SAM show resilience to spectral variability, capturing subtle target features while maintaining clear target-background separation, even under significant underwater bottom interference. These methods provide the most comprehensive target coverage and background suppression in challenging environments, such as the ATR2-HUTD River dataset, demonstrating the superior effectiveness of HUCLNet in mitigating spectral variability and improving detection accuracy.
\par 
\textbf{(\romannumeral2) ROC Curves:} Subjective analysis of detection maps may be insufficient for comprehensive evaluation. Therefore, 3-D ROC curves and their 2-D projections: ($P_d$, $P_f$), ($P_d$, $\tau$), and ($P_f$, $\tau$) were used to objectively assess detection performance on the ATR2-HUTD dataset, enabling a detailed evaluation of detection efficiency, target preservation, and background suppression. 
The ROC curves of ATR-HUTD-Lake sub-dataset are provided in Figs~\ref{fig:C2-1} to~\ref{fig:C2-2}, while those of the ATR-HUTD-River and ATR-HUTD-Sea sub-datasets are provided in the supplementary material.
\par
Figs.~\ref{fig:C2-1} (a) to~\ref{fig:C2-2} (a) show the 3-D ROC curves, highlighting the relationship between the true positive rate ($P_d$), false alarm probability ($P_f$), and detection threshold ($\tau$). HUCLNet+CEM and HUCLNet+SAM consistently outperform other methods, exhibiting higher $P_d$ and lower $P_f$ over a wide range of $\tau$, demonstrating superior adaptability.
\par
% \begin{figure*}[!t]                 
%     \centering                    
%     \includegraphics[width=2\columnwidth]{images/C2-3.jpg}                     
%     \caption{ROC curves comparison on ATR2-HUTD River Scene1. (a) 3-D ROC curve. (b) 2-D ROC curve of $(P_d, P_f)$. (c) 2-D ROC curve of $(P_f, \tau)$. (d) 2-D ROC curve of $(P_d, \tau)$.}                                   
%     \label{fig:C2-3}    
% \end{figure*}
% \begin{figure*}[!t]                 
%     \centering                    
%     \includegraphics[width=2\columnwidth]{images/C2-4.jpg}                     
%     \caption{ROC curves comparison on ATR2-HUTD River Scene2. (a) 3-D ROC curve. (b) 2-D ROC curve of $(P_d, P_f)$. (c) 2-D ROC curve of $(P_f, \tau)$. (d) 2-D ROC curve of $(P_d, \tau)$.}                                 
%     \label{fig:C2-4}    
% \end{figure*}
% \begin{figure*}[!t]                 
%     \centering                    
%     \includegraphics[width=2\columnwidth]{images/C2-5.jpg}                     
%     \caption{ROC curves comparison on ATR2-HUTD Sea Scene1. (a) 3-D ROC curve. (b) 2-D ROC curve of $(P_d, P_f)$. (c) 2-D ROC curve of $(P_f, \tau)$. (d) 2-D ROC curve of $(P_d, \tau)$.}                                   
%     \label{fig:C2-5}    
% \end{figure*}
% \begin{figure*}[!t]                 
%     \centering                    
%     \includegraphics[width=2\columnwidth]{images/C2-6.jpg}                     
%     \caption{ROC curves comparison on ATR2-HUTD Sea Scene2. (a) 3-D ROC curve. (b) 2-D ROC curve of $(P_d, P_f)$. (c) 2-D ROC curve of $(P_f, \tau)$. (d) 2-D ROC curve of $(P_d, \tau)$.}                                
%     \label{fig:C2-2}    
% \end{figure*}
Figs.~\ref{fig:C2-1} (b) to~\ref{fig:C2-2} (b) present the 2-D ROC curves of ($P_d$, $P_f$). HUCLNet-based methods occupy the top-left region, indicating superior detection accuracy. In contrast, traditional HTD methods, such as CEM and SAM, struggle to balance $P_d$ and $P_f$, particularly for targets with varying spectral properties. Although advanced HTD and SOTA HUTD methods show moderate performance, they fail to suppress false alarms in complex river environments, compromising detection accuracy.

Figs.~\ref{fig:C2-1}(c) to~\ref{fig:C2-2}(c) depict the 2-D ROC curves of ($P_f$, $\tau$), assessing background suppression. NUN-UTD shows high $P_f$ across thresholds, indicating poor background-target discrimination. While methods like MCLT and TUTDF show some improvement, they still struggle with high false alarm rates due to spectral overlap. \textbf{UTD-Net performs well in background suppression but largely by classifying all pixels as background}, as reflected in detection maps (Figs.~\ref{fig:C1-1} to~\ref{fig:C1-2}) and AUC$_{P_{d}, \tau}$ values (Tabs.~\ref{auc_lake} to~\ref{auc_sea}). In comparison, HUCLNet+CEM and HUCLNet+SAM exhibit superior background suppression with low $P_f$ and high AUC$_{P_{d}, \tau}$ values.

Figs.~\ref{fig:C2-1}(d) to~\ref{fig:C2-2}(d) present the 2-D ROC curves of ($P_d$, $\tau$), evaluating target preservation. Traditional methods, such as SAM, show significant drops in $P_d$ as $\tau$ increases, indicating poor target preservation. Advanced HTD and SOTA HUTD methods, such as MCLT and TDSS-UTD, show some improvement but still lag behind NUN-UTD and TUTDF. However, \textbf{the improved performance of NUN-UTD and TUTDF primarily results from misclassifying all pixels as targets}, as shown by high false alarm rates in detection maps (Figs.~\ref{fig:C1-1} to~\ref{fig:C1-2}) and increased AUC$_{P_{f}, \tau}$ values. In contrast, HUCLNet+CEM and HUCLNet+SAM maintain high $P_d$ at lower $\tau$, demonstrating robust and reliable target preservation.
\par
\begin{table*}[!t] 
    \centering
    \footnotesize   
    \caption{Quantitative comparison results on the ATR2-HUTD-Lake Sub-dataset. The best and second best results are in \textbf{bold} and with \underline{underline}.} \label{auc_lake}
    \renewcommand{\arraystretch}{1.5}
    \setlength{\tabcolsep}{1.85mm}
    \scalebox{0.875}
    {
        \begin{tabular}{ccccccccccc}
            \hline
            \multirow{2.4}{*}{\textbf{Method}} & \multicolumn{5}{c}{\cellcolor{tablecolor7!60}\textbf{ATR2-HUTD-Lake Scene1}}       & \multicolumn{5}{c}{\cellcolor{tablecolor8}\textbf{ATR2-HUTD-Lake Scene2}}       \\ \cmidrule(lr){2-6} \cmidrule(lr){7-11}
                                    & $\text{AUC}_{( \mathrm{P}_{\mathrm{d}},\mathrm{P}_{\mathrm{f}})}\textcolor{red}{\uparrow }$ & $\text{AUC}_{( \mathrm{P}_{\mathrm{f}}, \tau)}\textcolor{green}{\downarrow }$ & $\text{AUC}_{( \mathrm{P}_{\mathrm{d}},\tau)}\textcolor{red}{\uparrow }$ & $\mathrm{AUC}_{\mathrm{OA}} \textcolor{red}{\uparrow }$ & $\mathrm{AUC}_{\mathrm{SNPR}}\textcolor{red}{\uparrow }$ & $\text{AUC}_{( \mathrm{P}_{\mathrm{d}},\mathrm{P}_{\mathrm{f}})}\textcolor{red}{\uparrow }$ & $\text{AUC}_{( \mathrm{P}_{\mathrm{f}}, \tau)}\textcolor{green}{\downarrow }$ & $\text{AUC}_{( \mathrm{P}_{\mathrm{d}},\tau)}\textcolor{red}{\uparrow }$ & $\mathrm{AUC}_{\mathrm{OA}} \textcolor{red}{\uparrow }$ & $\mathrm{AUC}_{\mathrm{SNPR}}\textcolor{red}{\uparrow }$ \\ \hline
                                    CEM         & 0.671          & 0.250          & 0.258          & 0.678          & 1.028          & 0.489          & 0.524          & 0.520          & 0.485          & 0.994          \\
                                    SAM         & 0.670          & \underline{0.129}    & 0.151          & 0.692          & 1.170          & 0.480          & \underline{0.143}    & 0.025          & 0.362          & 0.172          \\
                                    IEEPST      & 0.424          & 0.204          & 0.075          & 0.295          & 0.369          & 0.417          & 0.187          & 0.036          & 0.266          & 0.193          \\
                                    MCLT        & 0.401          & 0.422          & 0.377          & 0.357          & 0.894          & 0.365          & 0.243          & 0.173          & 0.296          & 0.715          \\
                                    UTD-Net     & 0.846          & \textbf{0.013} & 0.019          & 0.853          & 1.510          & 0.944          & \textbf{0.041} & 0.073          & 0.976          & 1.773          \\
                                    TUTDF       & \underline{0.990}          & 0.634          & \underline{0.726}    & 1.081          & 1.145          & \underline{0.998}          & 0.461          & \underline{0.768}    & 1.306          & 1.667          \\
                                    TDSS-UTD    & 0.964          & 0.215          & 0.369          & 1.117          & 1.712          & \textbf{0.999} & 0.166          & 0.444          & 1.277          & 2.676          \\
                                    NUN-UTD     & 0.758          & 0.913          & \textbf{0.994} & 0.838          & 1.088          & 0.765          & 0.792          & \textbf{0.995} & 0.968          & 1.257          \\
            \rowcolor{tablecolor13!60}HUCLNet+CEM & 0.958    & 0.302          & 0.642          & \underline{1.298}    & \underline{2.126}    & 0.989    & 0.226          & 0.634          & \underline{1.397}    & \underline{2.805}    \\
            \rowcolor{tablecolor14!60}HUCLNet+SAM & \textbf{0.995} & 0.209          & 0.710          & \textbf{1.501} & \textbf{3.393} & \textbf{0.999} & 0.265          & 0.765          & \textbf{1.501} & \textbf{2.891} \\ \hline
        \end{tabular}}
\end{table*}
\begin{table*}[!t] 
    \centering
    \footnotesize   
    \caption{Quantitative comparison results on the ATR2-HUTD-River Sub-dataset. The best and second best results are in \textbf{bold} and with \underline{underline}.} \label{auc_river}
    \renewcommand{\arraystretch}{1.5}
    \setlength{\tabcolsep}{1.85mm}
    \scalebox{0.875}
    {
        \begin{tabular}{ccccccccccc}
            \hline
            \multirow{2.4}{*}{\textbf{Method}} & \multicolumn{5}{c}{\cellcolor{tablecolor9}\textbf{ATR2-HUTD-River Scene1}}       & \multicolumn{5}{c}{\cellcolor{tablecolor10}\textbf{ATR2-HUTD-River Scene2}}       \\ \cmidrule(lr){2-6} \cmidrule(lr){7-11}
                                    & $\text{AUC}_{( \mathrm{P}_{\mathrm{d}},\mathrm{P}_{\mathrm{f}})}\textcolor{red}{\uparrow }$ & $\text{AUC}_{( \mathrm{P}_{\mathrm{f}}, \tau)}\textcolor{green}{\downarrow }$ & $\text{AUC}_{( \mathrm{P}_{\mathrm{d}},\tau)}\textcolor{red}{\uparrow }$ & $\mathrm{AUC}_{\mathrm{OA}} \textcolor{red}{\uparrow }$ & $\mathrm{AUC}_{\mathrm{SNPR}}\textcolor{red}{\uparrow }$ & $\text{AUC}_{( \mathrm{P}_{\mathrm{d}},\mathrm{P}_{\mathrm{f}})}\textcolor{red}{\uparrow }$ & $\text{AUC}_{( \mathrm{P}_{\mathrm{f}}, \tau)}\textcolor{green}{\downarrow }$ & $\text{AUC}_{( \mathrm{P}_{\mathrm{d}},\tau)}\textcolor{red}{\uparrow }$ & $\mathrm{AUC}_{\mathrm{OA}} \textcolor{red}{\uparrow }$ & $\mathrm{AUC}_{\mathrm{SNPR}}\textcolor{red}{\uparrow }$ \\ \hline
                                    CEM         & 0.746          & 0.280          & 0.300          & 0.765          & 1.070          & 0.650          & 0.544          & 0.553          & 0.659          & 1.016          \\
                                    SAM         & 0.657          & 0.214          & 0.186          & 0.629          & 0.871          & 0.656          & \underline{0.078}    & 0.066          & 0.645          & 0.854          \\
                                    IEEPST      & 0.455          & 0.203          & 0.033          & 0.286          & 0.163          & 0.594          & 0.274          & 0.236          & 0.556          & 0.861          \\
                                    MCLT        & 0.550          & 0.989          & 0.990          & 0.552          & 1.001          & 0.531          & 0.970          & \underline{0.971}    & 0.533          & 1.002          \\
                                    UTD-Net     & \underline{0.843}          & \underline{0.080}    & 0.096          & 0.860          & 1.209          & \underline{0.889}          & \textbf{0.075} & 0.088          & \underline{0.903}          & 1.176          \\
                                    TUTDF       & 0.568          & 0.822          & \underline{0.824}    & 0.570          & 1.003          & 0.659          & 0.356          & 0.363          & 0.667          & 1.022          \\
                                    TDSS-UTD    & 0.402          & 0.438          & 0.415          & 0.379          & 0.948          & 0.539 & 0.179          & 0.174          & 0.534          & 0.974          \\
                                    NUN-UTD     & 0.632          & 0.968          & \textbf{0.999} & 0.663          & 1.032          & 0.503          & 0.977          & \textbf{0.980} & 0.505          & 1.002          \\
            \rowcolor{tablecolor13!60}HUCLNet+CEM & 0.794    & 0.353          & 0.518          & \underline{0.959}    & \underline{1.468}    & 0.753    & 0.354          & 0.481          & 0.880    & \underline{1.360}    \\
            \rowcolor{tablecolor14!60}HUCLNet+SAM & \textbf{0.966} & \textbf{0.055} & 0.175          & \textbf{1.086} & \textbf{3.206} & \textbf{0.924} & 0.178          & 0.327          & \textbf{1.073} & \textbf{1.837} \\ \hline
        \end{tabular}}
\end{table*}
\begin{table*}[!t] 
    \centering
    \footnotesize   
    \caption{Quantitative comparison results on the ATR2-HUTD-Sea Sub-dataset. The best and second best results are in \textbf{bold} and with \underline{underline}.} \label{auc_sea}
    \renewcommand{\arraystretch}{1.5}
    \setlength{\tabcolsep}{1.85mm}
    \scalebox{0.875}
    {
        \begin{tabular}{ccccccccccc}
            \hline
            \multirow{2.4}{*}{\textbf{Method}} & \multicolumn{5}{c}{\cellcolor{tablecolor11}\textbf{ATR2-HUTD-Sea Scene1}}       & \multicolumn{5}{c}{\cellcolor{tablecolor12!50}\textbf{ATR2-HUTD-Sea Scene2}}       \\ \cmidrule(lr){2-6} \cmidrule(lr){7-11}
                                    & $\text{AUC}_{( \mathrm{P}_{\mathrm{d}},\mathrm{P}_{\mathrm{f}})}\textcolor{red}{\uparrow }$ & $\text{AUC}_{( \mathrm{P}_{\mathrm{f}}, \tau)}\textcolor{green}{\downarrow }$ & $\text{AUC}_{( \mathrm{P}_{\mathrm{d}},\tau)}\textcolor{red}{\uparrow }$ & $\mathrm{AUC}_{\mathrm{OA}} \textcolor{red}{\uparrow }$ & $\mathrm{AUC}_{\mathrm{SNPR}}\textcolor{red}{\uparrow }$ & $\text{AUC}_{( \mathrm{P}_{\mathrm{d}},\mathrm{P}_{\mathrm{f}})}\textcolor{red}{\uparrow }$ & $\text{AUC}_{( \mathrm{P}_{\mathrm{f}}, \tau)}\textcolor{green}{\downarrow }$ & $\text{AUC}_{( \mathrm{P}_{\mathrm{d}},\tau)}\textcolor{red}{\uparrow }$ & $\mathrm{AUC}_{\mathrm{OA}} \textcolor{red}{\uparrow }$ & $\mathrm{AUC}_{\mathrm{SNPR}}\textcolor{red}{\uparrow }$ \\ \hline
                                    CEM         & 0.805          & 0.309          & 0.349          & 0.845          & 1.128           & 0.845          & 0.332          & 0.351          & 0.864          & 1.057          \\
                                    SAM         & 0.866          & 0.125    & 0.188          & 0.929          & 1.503           & 0.819          & 0.099          & 0.033          & 0.753          & 0.333          \\
                                    IEEPST      & 0.850          & 0.252          & 0.363          & 0.961          & 1.441           & 0.580          & 0.326          & 0.269          & 0.523          & 0.826          \\
                                    MCLT        & 0.895          & 0.980          & \underline{0.994} & 0.909          & 1.014           & 0.317          & 0.953          & \underline{0.944}    & 0.309          & 0.991          \\
                                    UTD-Net     & 0.762          & \underline{0.050}          & 0.083          & 0.796          & 1.682           & 0.774          & \textbf{0.043} & 0.070          & 0.801          & 1.634          \\
                                    TUTDF       & 0.952          & 0.841          & 0.872          & 0.984          & 1.037           & 0.903          & 0.426          & 0.482          & 0.959          & 1.131          \\
                                    TDSS-UTD    & 0.861          & 0.310          & 0.371          & 0.923          & 1.199           & 0.984          & 0.218          & 0.425          & 1.192          & 1.948          \\
                                    NUN-UTD     & \underline{0.979}    & 0.534          & \textbf{0.999} & \textbf{1.445} & 1.872           & 0.975          & 0.959          & \textbf{0.984} & 0.999          & 1.025          \\
            \rowcolor{tablecolor13!60}HUCLNet+CEM & 0.972          & 0.133          & 0.569          & \underline{1.409}    & \underline{4.284}     & \underline{0.987}    & 0.111          & 0.401          & \underline{1.287}    & \underline{3.620}    \\
            \rowcolor{tablecolor14!60}HUCLNet+SAM & \textbf{0.985} & \textbf{0.019} & 0.325          & 1.292          & \textbf{17.501} & \textbf{0.989} & \underline{0.053}    & 0.474          & \textbf{1.420} & \textbf{8.857} \\ \hline
        \end{tabular}}
\end{table*}
\textbf{(\romannumeral3) AUC Values:} The AUC values for each sub-dataset of the ATR2-HUTD dataset are computed using five key metrics: $\text{AUC}_{( \mathrm{P}_{\mathrm{d}}, \mathrm{P}_{\mathrm{f}})}$, $\text{AUC}_{( \mathrm{P}_{\mathrm{d}}, \tau)}$, $\text{AUC}_{( \mathrm{P}_{\mathrm{f}}, \tau)}$, $\text{AUC}_{SNPR}$, and $\text{AUC}_{OA}$, as detailed in Tabs.~\ref{auc_lake} to~\ref{auc_sea}. These metrics quantitatively assess detection accuracy, target preservation, background suppression, signal-to-noise ratio, and overall performance in varied underwater environments.
\par
\begin{table*}[!ht] 
    \centering
    \footnotesize   
    \caption{Quantitative results of ablation studies on the ATR2-HUTD dataset.} \label{ablation study}
    \renewcommand{\arraystretch}{2}
    \setlength{\tabcolsep}{2.5mm}
    \begin{threeparttable}
        \scalebox{0.975}
        { 
    \begin{tabular}{ccccccc}
        \hline
        \textbf{Module Name}                  & \textbf{Design}                                                      & $\text{AUC}_{( \mathrm{P}_{\mathrm{d}},\mathrm{P}_{\mathrm{f}})}\textcolor{red}{\uparrow }$ & $\text{AUC}_{( \mathrm{P}_{\mathrm{f}}, \tau)}\textcolor{green}{\downarrow }$ & $\text{AUC}_{( \mathrm{P}_{\mathrm{d}},\tau)}\textcolor{red}{\uparrow }$ & $\mathrm{AUC}_{\mathrm{OA}} \textcolor{red}{\uparrow }$ & $\mathrm{AUC}_{\mathrm{SNPR}}\textcolor{red}{\uparrow }$  \\ \hline
        \rowcolor{tablecolor0!50}
        \textbf{HUCLNet}                                          & N/A & 0.943 & 0.188 & 0.502 & 1.258 & 4.446 \\
        \rowcolor{tablecolor1!50}
        \cellcolor{tablecolor1!50}                             & w/o Cluster Refinement Strategy                             & 0.823 & 0.206 & 0.388 & 1.005 & 3.141 \\
        \rowcolor{tablecolor1!50}
        \multirow{-2}{*}{\cellcolor{tablecolor1!50}\textbf{RGC module}}  & w/o Reference Spectrum based Clustering Method & 0.737 & 0.211 & 0.375 & 0.901 & 2.616 \\
        \rowcolor{tablecolor2!50} 
        \cellcolor{tablecolor2!50}                              & w/o Instance-level Contrastive Learning                     & 0.878 & 0.199 & 0.438 & 1.117 & 3.513 \\
        \rowcolor{tablecolor2!50} 
        \cellcolor{tablecolor2!50}                              & w/o Prototype-level Contrastive Learning                    & 0.728 & 0.239 & 0.359 & 0.848 & 2.359 \\
        \rowcolor{tablecolor2!50}
        \cellcolor{tablecolor2!50}                              & w/o Hyperspectral-Oriented Data Augmentation                    & 0.883 & 0.195 & 0.452 & 1.165 & 3.584 \\
        \rowcolor{tablecolor2!50} 
        \multirow{-4}{*}{\cellcolor{tablecolor2!50}\textbf{HLCL module}} & w/o HLCL module$^{1}$                                             & 0.696 & 0.252 & 0.248 & 0.692 & 0.933 \\
        \rowcolor{tablecolor3!50} 
        \textbf{SPL Paradigm}                                          & w/o SPL Paradigm                                            & 0.743 & 0.217 & 0.388 & 0.914 & 2.864 \\ \hline
        \end{tabular}}
        \begin{tablenotes}
            \scriptsize
            \item[1] This experimental design is analogous to the baseline HTD methods, as the RGC module and SPL paradigm are dependent on the HLCL module for functionality.
        \end{tablenotes}
        \end{threeparttable}
\end{table*}
The $\text{AUC}_{( \mathrm{P}_{\mathrm{d}}, \mathrm{P}_{\mathrm{f}})}$ metric, which quantifies the trade-off between the true positive rate ($P_d$) and false alarm probability ($P_f$), is critical for evaluating detection performance. HUCLNet+SAM leads with an average score of 0.976, followed by HUCLNet+CEM at 0.909. Traditional methods, such as SAM (0.701) and MCLT (0.691), underperform significantly, while SOTA HUTD methods like TUTDF and NUN-UTD fall short of HUCLNet-based methods in detection capability.
\par
For background suppression, assessed by $\text{AUC}_{( \mathrm{P}_{\mathrm{f}}, \tau)}$, HUCLNet+SAM achieves the highest performance in the ATR2-HUTD-River Scene1 and ATR2-HUTD-Sea sub-datasets, the most complex nearshore environments. It also demonstrates robust performance across other sub-datasets. In contrast, SOTA HUTD methods, including TUTDF and NUN-UTD, show elevated values, suggesting overfitting due to high false positive rates.
\par
The $\text{AUC}_{( \mathrm{P}_{\mathrm{d}}, \tau)}$ metric, assessing target preservation, reveals HUCLNet-based methods performing well, though NUN-UTD leads. This may be attributed to the HLCL module in HUCLNet, which compromises target-background feature separation, impacting target preservation. Additionally, NUN-UTD's higher false positive rate boosts $P_d$ but hinders background suppression.
\par
The $\text{AUC}_{OA}$ metric, combining $\text{AUC}_{( \mathrm{P}_{\mathrm{d}}, \mathrm{P}_{\mathrm{f}})}$, $\text{AUC}_{( \mathrm{P}_{\mathrm{d}}, \tau)}$, and $\text{AUC}_{( \mathrm{P}_{\mathrm{f}}, \tau)}$, further emphasizes HUCLNet's superiority. HUCLNet+SAM achieves the highest average score of 1.312, with HUCLNet+CEM following at 1.205. In contrast, traditional and SOTA HUTD methods score between 0.492 and 0.928, underscoring HUCLNet's effectiveness in background suppression, target preservation, and detection accuracy in complex nearshore environments.
\par
Finally, the $\text{AUC}_{SNPR}$ metric, which measures robustness under varying signal-to-noise ratios, underscores HUCLNet+SAM's superior performance, achieving the highest scores across all sub-datasets, including 17.501 in ATR2-HUTD-Sea Scene1. HUCLNet+CEM consistently ranks second, while traditional HTD and SOTA HUTD methods show lower scores, indicating reduced robustness in fluctuating signal conditions.
\par
% \begin{figure*}[!t]                 
%     \centering                    
%     \includegraphics[scale=0.65]{images/C3-1.jpg}                     
%     \caption{Target-background separability boxplots for ATR2-HUTD Lake sub-dataset. (a) Scene1. (b) Scene2.}                    
%     \label{fig:C3-1}    
% \end{figure*}


% \begin{figure*}[!t]                 
%     \centering                    
%     \includegraphics[scale=0.65]{images/C3-2.jpg}                     
%     \caption{Target-background separability boxplots for ATR2-HUTD River sub-dataset. (a) Scene1. (b) Scene2.}                                 
%     \label{fig:C3-2}    
% \end{figure*}
% \begin{figure*}[!t]                 
%     \centering                    
%     \includegraphics[scale=0.65]{images/C3-3.jpg}                     
%     \caption{Target-background separability boxplots for ATR2-HUTD Sea sub-dataset. (a) Scene1. (b) Scene2.}                                      
%     \label{fig:C3-3}    
% \end{figure*}
% \textbf{(\romannumeral4) Separability Maps:} To assess the effectiveness of the comparison methods in distinguishing targets from the background, target-background separability is analyzed using boxplots, providing a clear visual representation of the separation. Figs.~\ref{fig:C3-1} to \ref{fig:C3-3} present these separability boxplots for all methods across the ATR2-HUTD sub-datasets.
% \par
% Traditional HTD methods, CEM and SAM, show limited separability, with SAM slightly outperforming CEM. In all sub-datasets, target boxes overlap with background boxes, despite some background suppression, indicating poor separation of targets from the background in underwater environments.
% Advanced HTD methods, IEEPST and MCLT, show minor improvement over traditional methods. 
% However, except for the ATR2-HUTD Sea sub-dataset (scene 1), target boxes still overlap with background boxes in most sub-datasets. 
% This suggests that even with advanced techniques, suppressing background noise and achieving clear target separation remains challenging in complex underwater environments.
% \par
% HUTD methods show improved separability. Specifically, UTD-Net achieves significant background suppression, though some overlap remains. 
% Additionally, UTD-Net exhibits a detection range near 0 in certain sub-datasets, indicating a high false positive rate. 
% NUN-UTD, an enhanced version of UTD-Net, improves target highlighting but still struggles with background noise suppression, leading to suboptimal performance in more complex scenes such as those in the ATR2-HUTD River and Sea sub-datasets.
% Compared to unmixing-based HUTD methods, TUTDF and TDSS-UTD demonstrate better separability, with detection ranges closer to 1, indicating more effective reduction of target-background correlation. 
% However, both methods still exhibit considerable target-background overlap and limited suppression in complex scenes, such as the ATR2-HUTD River sub-dataset.
% \par
% In contrast, the proposed HUCLNet-based methods, HUCLNet+CEM and HUCLNet+SAM, exhibit superior separability, with target boxes generally fully separated from the background. 
% These methods effectively suppress background noise, enabling reliable target detection in underwater environments. 
% The detection range for HUCLNet-based methods is close to 1, while the background range is near 0, indicating a low false positive rate. Compared to CEM and SAM, HUCLNet significantly enhances target-background separability, demonstrating its effectiveness in underwater hyperspectral target detection.
\subsection{Ablation Studies}\label{sec:4.4}
To evaluate the efficacy of each component in our method, we conducted ablation studies on the ATR2-HUTD dataset. These studies aim to confirm that the observed improvements stem not only from the increased number of parameters but also from the architectural design, which enhances the HUTD task. The HUCLNet framework is divided into three components for experimental validation. 
Corresponding results are presented in Tab.~\ref{ablation study}.
\par
\textbf{(\romannumeral1) Analysis of the RGC Module:} We validate the RGC with the following designs: 
\begin{itemize}
    \item \textbf{w/o Cluster Refinement Strategy:} This design excludes the cluster refinement strategy, relying solely on the reference spectrum-based clustering method. 
    \item \textbf{w/o Reference Spectrum-based Clustering:} This design omits the reference spectrum-based clustering approach from the RGC module.
\end{itemize}
\par
Without the cluster refinement strategy, the RGC module directly uses the original clustering results, often misclassifying pixels and negatively impacting prototype-level learning. As seen in Tab.~\ref{ablation study}, this leads to lower average metric values compared to the full HUCLNet-based methods, demonstrating the importance of refined pseudo-labels. Removing the RGC module entirely, the HLCL module uses pixel instances from the original HSI for instance-level contrastive learning, focusing on individual pixel spectra and neglecting the target-background relationships. Performance improves slightly over baseline HTD methods but remains significantly inferior to complete HUCLNet-based methods, highlighting the critical role of the RGC module in providing reliable pseudo-labels.
\par
\begin{figure*}[!t]                 
    \centering                    
    \includegraphics[width=2\columnwidth]{images/C4.jpg}                     
    \caption{Parameter analysis results on the ATR2-HUTD dataset. (a) Number of clusters in the RCG module; (b) Batch size in the HLCL module; (c) Attack method in the HLCL module. Red and yellow points indicate the maximum and minimum values, respectively.}                 
    \label{fig:C4-1}    
\end{figure*}
\textbf{(\romannumeral2) Analysis of the HLCL Module:} We evaluate the HLCL module with the following designs:
\begin{itemize}
    \item \textbf{w/o Instance-level Contrastive Learning:} This design removes instance-level contrastive learning, relying only on refined cluster labels from the RGC module.
    \item \textbf{w/o Prototype-level Contrastive Learning:} This design removes prototype-level contrastive learning, retaining only instance-level contrastive learning.
    \item \textbf{w/o Hyperspectral-Oriented Data Augmentation:} This design removes hyperspectral-specific data augmentation from the HLCL module.
    \item \textbf{w/o HLCL Module:} This design excludes the entire HLCL module.
\end{itemize}
\par
According to Tab.~\ref{ablation study}, we can draw the following conclusions.
When the HLCL module operates without instance-level contrastive learning, HUCLNet relies solely on the cluster labels, leading to performance degradation. However, prototype-level contrastive learning alone still outperforms baseline HTD methods, emphasizing the importance of target-background separability. The removal of prototype-level contrastive learning results in poorer performance compared to the instance-level design, indicating its greater impact on separability. When hyperspectral-oriented data augmentation is excluded, traditional augmentation methods lead to observable performance degradation, confirming the importance of hyperspectral-specific augmentation in enhancing feature discriminability and HUCLNet's performance. Finally, removing the HLCL module entirely reduces HUCLNet to baseline HTD methods, resulting in substantial performance loss, reinforcing the HLCL module's primary contribution to performance improvement.
\par
\textbf{(\romannumeral3) Analysis of the SPL Paradigm:} We evaluate the SPL paradigm with the following design: 
\begin{itemize}
    \item \textbf{w/o SPL Paradigm:} This design trains the model using the traditional self-supervised learning framework, which consists of a single reliable-guided clustering step followed by hybrid-level contrastive learning.
\end{itemize}
\par
Without the SPL paradigm, inaccurate clustering due to limited spectral discriminability hinders contrastive learning effectiveness, resulting in error propagation and performance degradation. Tab.~\ref{ablation study} confirms that the SPL paradigm significantly enhances HUCLNet's performance, underscoring the importance of the self-paced strategy in guiding model training and improving detection accuracy.
\par
\subsection{Parameter Analysis}\label{sec:4.5}
The key hyperparameters of the HUCLNet architecture, including the number of clusters in the RGC module, batch size, and attack method in the HLCL module, were analyzed through experiments on the ATR2-HUTD dataset. The results, primarily focusing on the $\text{AUC}_{\text{OA}}$ metric, are presented in Fig.~\ref{fig:C4-1}, as it is the most critical indicator of overall detection performance.
\par
\textbf{(\romannumeral1) Number of Clusters in the RGC Module:} The number of clusters in the RGC module plays a crucial role in clustering accuracy and overall HUCLNet performance. The number of clusters was varied between 30 and 48, with a step size of 3 (Fig.~\ref{fig:C4-1} (a)). Performance improves with an increasing number of clusters up to an optimal point, after which it deteriorates due to over-segmentation, where target pixels are fragmented into multiple clusters. This fragmentation hinders prototype-level contrastive learning, leading to inconsistent target representations. For the ATR2-HUTD Lake, River, and Sea sub-datasets, the optimal number of clusters was 36, 39, and 42, respectively. Even with suboptimal cluster numbers, HUCLNet outperforms baseline methods.

\textbf{(\romannumeral2) Batch Size in the HLCL Module:} The batch size in the HLCL module is another critical parameter affecting HUCLNet performance. Varying the batch size from 32 to 512 with a step size of 64, results (Fig.~\ref{fig:C4-1} (b)) show that larger batch sizes generally improve performance by increasing the number of negative samples, enhancing feature discriminability. This is consistent with prior work~\cite{Chen2020}, which indicates that larger batch sizes benefit contrastive learning. However, performance gains plateau at higher batch sizes, and larger sizes impose greater memory and computational demands. A batch size of 256 provides an optimal balance between performance and resource usage across all ATR2-HUTD sub-datasets.

\textbf{(\romannumeral3) Attack Method in the HLCL Module:} The choice of attack method in the HLCL module influences the generation of adversarial samples for contrastive learning. Four attack methods—FGSM~\cite{GoodfellowSS14}, PGD~\cite{MadryMSTV18}, FAB~\cite{Croce020}, and SPSA~\cite{SPSA}—were tested with a perturbation limit of $\epsilon = 0.1$. As shown in Fig.~\ref{fig:C4-1} (c), performance across attack methods is similar, suggesting that the specific choice of attack method has minimal impact, as long as the generated adversarial samples are effective. Given its computational efficiency and comparable performance, we adopt the FGSM attack method for HUCLNet.

% \subsection{Visualization of the effect of HUCLNet}\label{sec:4.5}
\paragraph{Summary}
Our findings provide significant insights into the influence of correctness, explanations, and refinement on evaluation accuracy and user trust in AI-based planners. 
In particular, the findings are three-fold: 
(1) The \textbf{correctness} of the generated plans is the most significant factor that impacts the evaluation accuracy and user trust in the planners. As the PDDL solver is more capable of generating correct plans, it achieves the highest evaluation accuracy and trust. 
(2) The \textbf{explanation} component of the LLM planner improves evaluation accuracy, as LLM+Expl achieves higher accuracy than LLM alone. Despite this improvement, LLM+Expl minimally impacts user trust. However, alternative explanation methods may influence user trust differently from the manually generated explanations used in our approach.
% On the other hand, explanations may help refine the trust of the planner to a more appropriate level by indicating planner shortcomings.
(3) The \textbf{refinement} procedure in the LLM planner does not lead to a significant improvement in evaluation accuracy; however, it exhibits a positive influence on user trust that may indicate an overtrust in some situations.
% This finding is aligned with prior works showing that iterative refinements based on user feedback would increase user trust~\cite{kunkel2019let, sebo2019don}.
Finally, the propensity-to-trust analysis identifies correctness as the primary determinant of user trust, whereas explanations provided limited improvement in scenarios where the planner's accuracy is diminished.

% In conclusion, our results indicate that the planner's correctness is the dominant factor for both evaluation accuracy and user trust. Therefore, selecting high-quality training data and optimizing the training procedure of AI-based planners to improve planning correctness is the top priority. Once the AI planner achieves a similar correctness level to traditional graph-search planners, strengthening its capability to explain and refine plans will further improve user trust compared to traditional planners.

\paragraph{Future Research} Future steps in this research include expanding user studies with larger sample sizes to improve generalizability and including additional planning problems per session for a more comprehensive evaluation. Next, we will explore alternative methods for generating plan explanations beyond manual creation to identify approaches that more effectively enhance user trust. 
Additionally, we will examine user trust by employing multiple LLM-based planners with varying levels of planning accuracy to better understand the interplay between planning correctness and user trust. 
Furthermore, we aim to enable real-time user-planner interaction, allowing users to provide feedback and refine plans collaboratively, thereby fostering a more dynamic and user-centric planning process.



% In the unusual situation where you want a paper to appear in the
% references without citing it in the main text, use \nocite
\nocite{langley00}

\bibliography{main}
\bibliographystyle{icml2025}


%%%%%%%%%%%%%%%%%%%%%%%%%%%%%%%%%%%%%%%%%%%%%%%%%%%%%%%%%%%%%%%%%%%%%%%%%%%%%%%
%%%%%%%%%%%%%%%%%%%%%%%%%%%%%%%%%%%%%%%%%%%%%%%%%%%%%%%%%%%%%%%%%%%%%%%%%%%%%%%
% APPENDIX
%%%%%%%%%%%%%%%%%%%%%%%%%%%%%%%%%%%%%%%%%%%%%%%%%%%%%%%%%%%%%%%%%%%%%%%%%%%%%%%
%%%%%%%%%%%%%%%%%%%%%%%%%%%%%%%%%%%%%%%%%%%%%%%%%%%%%%%%%%%%%%%%%%%%%%%%%%%%%%%
\newpage
\appendix
\onecolumn
\section*{Appendix}
% \section{Appendix.}

\section{Secure Token Pruning Protocols}
\label{app:a}
We detail the encrypted token pruning protocols $\Pi_{prune}$ in Figure \ref{fig:protocol-prune} and $\Pi_{mask}$ in Figure \ref{fig:protocol-mask} in this section.

%Optionally include supplemental material (complete proofs, additional experiments and plots) in appendix.
%All such materials \textbf{SHOULD be included in the main submission.}
\begin{figure}[h]
%vspace{-0.2in}
\begin{protocolbox}
\noindent
\textbf{Parties:} Server $P_0$, Client $P_1$.

\textbf{Input:} $P_0$ and $P_1$ holds $\{ \left \langle Att \right \rangle_{0}^{h}, \left \langle Att \right \rangle_{1}^{h}\}_{h=0}^{H-1} \in \mathbb{Z}_{2^{\ell}}^{n\times n}$ and $\left \langle x \right \rangle_{0}, \left \langle x \right \rangle_{1} \in \mathbb{Z}_{2^{\ell}}^{n\times D}$ respectively, where H is the number of heads, n is the number of input tokens and D is the embedding dimension of tokens. Additionally, $P_1$ holds a threshold $\theta \in \mathbb{Z}_{2^{\ell}}$.

\textbf{Output:} $P_0$ and $P_1$ get $\left \langle y \right \rangle_{0}, \left \langle y \right \rangle_{1} \in \mathbb{Z}_{2^{\ell}}^{n'\times D}$, respectively, where $y=\mathsf{Prune}(x)$ and $n'$ is the number of remaining tokens.

\noindent\rule{13.2cm}{0.1pt} % This creates the horizontal line
\textbf{Protocol:}
\begin{enumerate}[label=\arabic*:, leftmargin=*]
    \item For $h \in [H]$, $P_0$ and $P_1$ compute locally with input $\left \langle Att \right \rangle^{h}$, and learn the importance score in each head $\left \langle s \right \rangle^{h} \in \mathbb{Z}_{2^{\ell}}^{n} $, where $\left \langle s \right \rangle^{h}[j] = \frac{1}{n} \sum_{i=0}^{n-1} \left \langle Att \right \rangle^{h}[i,j]$.
    \item $P_0$ and $P_1$ compute locally with input $\{ \left \langle s \right \rangle^{i} \in \mathbb{Z}_{2^{\ell}}^{n}  \}_{i=0}^{H-1}$, and learn the final importance score $\left \langle S \right \rangle \in \mathbb{Z}_{2^{\ell}}^{n}$ for each token, where  $\left \langle S \right \rangle[i] = \frac{1}{H} \sum_{h=0}^{H-1} \left \langle s \right \rangle^{h}[i]$.
    \item  For $i \in [n]$, $P_0$ and $P_1$ invoke $\Pi_{CMP}$ with inputs  $\left \langle S \right \rangle$ and $ \theta $, and learn  $\left \langle M \right \rangle \in \mathbb{Z}_{2^{\ell}}^{n}$, such that$\left \langle M \right \rangle[i] = \Pi_{CMP}(\left \langle S \right \rangle[i] - \theta) $, where: \\
    $M[i] = \begin{cases}
        1  &\text{if}\ S[i] > \theta, \\
        0  &\text{otherwise}.
            \end{cases} $
    % \item If the pruning location is insensitive, $P_0$ and $P_1$ learn real mask $M$ instead of shares $\left \langle M \right \rangle$. $P_0$ and $P_1$ compute $\left \langle y \right \rangle$ with input $\left \langle x \right \rangle$ and $M$, where  $\left \langle x \right \rangle[i]$ is pruned if $M[i]$ is $0$.
    \item $P_0$ and $P_1$ invoke $\Pi_{mask}$ with inputs  $\left \langle x \right \rangle$ and pruning mask $\left \langle M \right \rangle$, and set outputs as $\left \langle y \right \rangle$.
\end{enumerate}
\end{protocolbox}
\setlength{\abovecaptionskip}{-1pt} % Reduces space above the caption
\caption{Secure Token Pruning Protocol $\Pi_{prune}$.}
\label{fig:protocol-prune}
\end{figure}




\begin{figure}[h]
\begin{protocolbox}
\noindent
\textbf{Parties:} Server $P_0$, Client $P_1$.

\textbf{Input:} $P_0$ and $P_1$ hold $\left \langle x \right \rangle_{0}, \left \langle x \right \rangle_{1} \in \mathbb{Z}_{2^{\ell}}^{n\times D}$ and  $\left \langle M \right \rangle_{0}, \left \langle M \right \rangle_{1} \in \mathbb{Z}_{2^{\ell}}^{n}$, respectively, where n is the number of input tokens and D is the embedding dimension of tokens.

\textbf{Output:} $P_0$ and $P_1$ get $\left \langle y \right \rangle_{0}, \left \langle y \right \rangle_{1} \in \mathbb{Z}_{2^{\ell}}^{n'\times D}$, respectively, where $y=\mathsf{Prune}(x)$ and $n'$ is the number of remaining tokens.

\noindent\rule{13.2cm}{0.1pt} % This creates the horizontal line
\textbf{Protocol:}
\begin{enumerate}[label=\arabic*:, leftmargin=*]
    \item For $i \in [n]$, $P_0$ and $P_1$ set $\left \langle M \right \rangle$ to the MSB of $\left \langle x \right \rangle$ and learn the masked tokens $\left \langle \Bar{x} \right \rangle \in Z_{2^{\ell}}^{n\times D}$, where
    $\left \langle \Bar{x}[i] \right \rangle = \left \langle x[i] \right \rangle + (\left \langle M[i] \right \rangle << f)$ and $f$ is the fixed-point precision.
    \item $P_0$ and $P_1$ compute the sum of $\{\Pi_{B2A}(\left \langle M \right \rangle[i]) \}_{i=0}^{n-1}$, and learn the number of remaining tokens $n'$ and the number of tokens to be pruned $m$, where $m = n-n'$.
    \item For $k\in[m]$, for $i\in[n-k-1]$, $P_0$ and $P_1$ invoke $\Pi_{msb}$ to learn the highest bit of $\left \langle \Bar{x}[i] \right \rangle$, where $b=\mathsf{MSB}(\Bar{x}[i])$. With the highest bit of $\Bar{x}[i]$, $P_0$ and $P_1$ perform a oblivious swap between $\Bar{x}[i]$ and $\Bar{x}[i+1]$:
    $\begin{cases}
        \Tilde{x}[i] = b\cdot \Bar{x}[i] + (1-b)\cdot \Bar{x}[i+1] \\
        \Tilde{x}[i+1] = b\cdot \Bar{x}[i+1] + (1-b)\cdot \Bar{x}[i]
    \end{cases} $ \\
    $P_0$ and $P_1$ learn the swapped token sequence $\left \langle \Tilde{x} \right \rangle$.
    \item $P_0$ and $P_1$ truncate $\left \langle \Tilde{x} \right \rangle$ locally by keeping the first $n'$ tokens, clear current MSB (all remaining token has $1$ on the MSB), and set outputs as $\left \langle y \right \rangle$.
\end{enumerate}
\end{protocolbox}
\setlength{\abovecaptionskip}{-1pt} % Reduces space above the caption
\caption{Secure Mask Protocol $\Pi_{mask}$.}
\label{fig:protocol-mask}
%\vspace{-0.2in}
\end{figure}

% \begin{wrapfigure}{r}{0.35\textwidth}  % 'r' for right, and the width of the figure area
%   \centering
%   \includegraphics[width=0.35\textwidth]{figures/msb.pdf}
%   \caption{Runtime of $\Pi_{prune}$ and $\Pi_{mask}$ in different layers. We compare different secure pruning strategies based on the BERT Base model.}
%   \label{fig:msb}
%   \vspace{-0.1in}
% \end{wrapfigure}

% \begin{figure}[h]  % 'r' for right, and the width of the figure area
%   \centering
%   \includegraphics[width=0.4\textwidth]{figures/msb.pdf}
%   \caption{Runtime of $\Pi_{prune}$ and $\Pi_{mask}$ in different layers. We compare different secure pruning strategies based on the BERT Base model.}
%   \label{fig:msb}
%   % \vspace{-0.1in}
% \end{figure}

\textbf{Complexity of $\Pi_{mask}$.} The complexity of the proposed $\Pi_{mask}$ mainly depends on the number of oblivious swaps. To prune $m$ tokens out of $n$ input tokens, $O(mn)$ swaps are needed. Since token pruning is performed progressively, only a small number of tokens are pruned at each layer, which makes $\Pi_{mask}$ efficient during runtime. Specifically, for a BERT base model with 128 input tokens, the pruning protocol only takes $\sim0.9$s on average in each layer. An alternative approach is to invoke an oblivious sort algorithm~\citep{bogdanov2014swap2,pang2023bolt} on $\left \langle \Bar{x} \right \rangle$. However, this approach is less efficient because it blindly sort the whole token sequence without considering $m$. That is, even if only $1$ token needs to be pruned, $O(nlog^{2}n)\sim O(n^2)$ oblivious swaps are needed, where as the proposed $\Pi_{mask}$ only need $O(n)$ swaps. More generally, for an $\ell$-layer Transformer with a total of $m$ tokens pruned, the overall time complexity using the sort strategy would be $O(\ell n^2)$ while using the swap strategy remains an overall complexity of $O(mn).$ Specifically, using the sort strategy to prune tokens in one BERT Base model layer can take up to $3.8\sim4.5$ s depending on the sorting algorithm used. In contrast, using the swap strategy only needs $0.5$ s. Moreover, alternative to our MSB strategy, one can also swap the encrypted mask along with the encrypted token sequence. However, we find that this doubles the number of swaps needed, and thus is less efficient the our MSB strategy, as is shown in Figure \ref{fig:msb}.

\section{Existing Protocols}
\label{app:protocol}
\noindent\textbf{Existing Protocols Used in Our Private Inference.}  In our private inference framework, we reuse several existing cryptographic protocols for basic computations. $\Pi_{MatMul}$ \citep{pang2023bolt} processes two ASS matrices and outputs their product in SS form. For non-linear computations, protocols $\Pi_{SoftMax}, \Pi_{GELU}$, and $\Pi_{LayerNorm}$\citep{lu2023bumblebee, pang2023bolt} take a secret shared tensor and return the result of non-linear functions in ASS. Basic protocols from~\citep{rathee2020cryptflow2, rathee2021sirnn} are also utilized. $\Pi_{CMP}$\citep{EzPC}, for example, inputs ASS values and outputs a secret shared comparison result, while $\Pi_{B2A}$\citep{EzPC} converts secret shared Boolean values into their corresponding arithmetic values.

\section{Polynomial Reduction for Non-linear Functions}
\label{app:b}
The $\mathsf{SoftMax}$ and $\mathsf{GELU}$ functions can be approximated with polynomials. High-degree polynomials~\citep{lu2023bumblebee, pang2023bolt} can achieve the same accuracy as the LUT-based methods~\cite{hao2022iron-iron}. While these polynomial approximations are more efficient than look-up tables, they can still incur considerable overheads. Reducing the high-degree polynomials to the low-degree ones for the less important tokens can imporve efficiency without compromising accuracy. The $\mathsf{SoftMax}$ function is applied to each row of an attention map. If a token is to be reduced, the corresponding row will be computed using the low-degree polynomial approximations. Otherwise, the corresponding row will be computed accurately via a high-degree one. That is if $M_{\beta}'[i] = 1$, $P_0$ and $P_1$ uses high-degree polynomials to compute the $\mathsf{SoftMax}$ function on token $x[i]$:
\begin{equation}
\mathsf{SoftMax}_{i}(x) = \frac{e^{x_i}}{\sum_{j\in [d]}e^{x_j}}
\end{equation}
where $x$ is a input vector of length $d$ and the exponential function is computed via a polynomial approximation. For the $\mathsf{SoftMax}$ protocol, we adopt a similar strategy as~\citep{kim2021ibert, hao2022iron-iron}, where we evaluate on the normalized inputs $\mathsf{SoftMax}(x-max_{i\in [d]}x_i)$. Different from~\citep{hao2022iron-iron}, we did not used the binary tree to find max value in the given vector. Instead, we traverse through the vector to find the max value. This is because each attention map is computed independently and the binary tree cannot be re-used. If $M_{\beta}[i] = 0$, $P_0$ and $P_1$ will approximate the $\mathsf{SoftMax}$ function with low-degree polynomial approximations. We detail how $\mathsf{SoftMax}$ can be approximated as follows:
\begin{equation}
\label{eq:app softmax}
\mathsf{ApproxSoftMax}_{i}(x) = \frac{\mathsf{ApproxExp}(x_i)}{\sum_{j\in [d]}\mathsf{ApproxExp}(x_j)}
\end{equation}
\begin{equation}
\mathsf{ApproxExp}(x)=\begin{cases}
    0  &\text{if}\ x \leq T \\
    (1+ \frac{x}{2^n})^{2^n} &\text{if}\ x \in [T,0]\\
\end{cases}
\end{equation}
where the $2^n$-degree Taylor series is used to approximate the exponential function and $T$ is the clipping boundary. The value $n$ and $T$ determines the accuracy of above approximation. With $n=6$ and $T=-13$, the approximation can achieve an average error within $2^{-10}$~\citep{lu2023bumblebee}. For low-degree polynomial approximation, $n=3$ is used in the Taylor series.

Similarly, $P_0$ or $P_1$ can decide whether or not to approximate the $\mathsf{GELU}$ function for each token. If $M_{\beta}[i] = 1$, $P_0$ and $P_1$ use high-degree polynomials~\citep{lu2023bumblebee} to compute the $\mathsf{GELU}$ function on token $x[i]$ with high-degree polynomial:
% \begin{equation}
% \mathsf{GELU}(x) = 0.5x(1+\mathsf{Tanh}(\sqrt{2/\pi}(x+0.044715x^3)))
% \end{equation}
% where the $\mathsf{Tanh}$ and square root function are computed via a OT-based lookup-table.

\begin{equation}
\label{eq:app gelu}
\mathsf{ApproxGELU}(x)=\begin{cases}
    0  &\text{if}\ x \leq -5 \\
    P^3(x), &\text{if}\ -5 < x \leq -1.97 \\
    P^6(x), &\text{if}\ -1.97 < x \leq 3  \\
    x, &\text{if}\ x >3 \\
\end{cases}
\end{equation}
where $P^3(x)$ and $P^6(x)$ are degree-3 and degree-6 polynomials respectively. The detailed coefficient for the polynomial is: 
\begin{equation*}
    P^3(x) = -0.50540312 -  0.42226581x - 0.11807613x^2 - 0.01103413x^3
\end{equation*}
, and
\begin{equation*}
    P^6(x) = 0.00852632 + 0.5x + 0.36032927x^2 - 0.03768820x^4 + 0.00180675x^6
\end{equation*}

For BOLT baseline, we use another high-degree polynomial to compute the $\mathsf{GELU}$ function.

\begin{equation}
\label{eq:app gelu}
\mathsf{ApproxGELU}(x)=\begin{cases}
    0  &\text{if}\ x < -2.7 \\
    P^4(x), &\text{if}\   |x| \leq 2.7 \\
    x, &\text{if}\ x >2.7 \\
\end{cases}
\end{equation}
We use the same coefficients for $P^4(x)$ as BOLT~\citep{pang2023bolt}.

\begin{figure}[h]
 % \vspace{-0.1in}
    \centering
    \includegraphics[width=1\linewidth]{figures/bumble.pdf}
    % \captionsetup{skip=2pt}
    % \vspace{-0.1in}
    \caption{Comparison with prior works on the BERT model. The input has 128 tokens.}
    \label{fig:bumble}
\end{figure}

If $M_{\beta}'[i] = 0$, $P_0$ and $P_1$ will use low-degree 
polynomial approximation to compute the $\mathsf{GELU}$ function instead. Encrypted polynomial reduction leverages low-degree polynomials to compute non-linear functions for less important tokens. For the $\mathsf{GELU}$ function, the following degree-$2$ polynomial~\cite{kim2021ibert} is used:
\begin{equation*}
\mathsf{ApproxGELU}(x)=\begin{cases}
    0  &\text{if}\ x <  -1.7626 \\
    0.5x+0.28367x^2, &\text{if}\ x \leq |1.7626| \\
    x, &\text{if}\ x > 1.7626\\
\end{cases}
\end{equation*}


\section{Comparison with More Related Works.}
\label{app:c}
\textbf{Other 2PC frameworks.} The primary focus of CipherPrune is to accelerate the private Transformer inference in the 2PC setting. As shown in Figure \ref{fig:bumble}, CipherPrune can be easily extended to other 2PC private inference frameworks like BumbleBee~\citep{lu2023bumblebee}. We compare CipherPrune with BumbleBee and IRON on BERT models. We test the performance in the same LAN setting as BumbleBee with 1 Gbps bandwidth and 0.5 ms of ping time. CipherPrune achieves more than $\sim 60 \times$ speed up over BOLT and $4.3\times$ speed up over BumbleBee.

\begin{figure}[t]
 % \vspace{-0.1in}
    \centering
    \includegraphics[width=1\linewidth]{figures/pumab.pdf}
    % \captionsetup{skip=2pt}
    % \vspace{-0.1in}
    \caption{Comparison with MPCFormer and PUMA on the BERT models. The input has 128 tokens.}
    \label{fig:pumab}
\end{figure}

\begin{figure}[h]
 % \vspace{-0.1in}
    \centering
    \includegraphics[width=1\linewidth]{figures/pumag.pdf}
    % \captionsetup{skip=2pt}
    % \vspace{-0.1in}
    \caption{Comparison with MPCFormer and PUMA on the GPT2 models. The input has 128 tokens. The polynomial reduction is not used.}
    \label{fig:pumag}
\end{figure}

\textbf{Extension to 3PC frameworks.} Additionally, we highlight that CipherPrune can be also extended to the 3PC frameworks like MPCFormer~\citep{li2022mpcformer} and PUMA~\citep{dong2023puma}. This is because CipherPrune is built upon basic primitives like comparison and Boolean-to-Arithmetic conversion. We compare CipherPrune with MPCFormer and PUMA on both the BERT and GPT2 models. CipherPrune has a $6.6\sim9.4\times$ speed up over MPCFormer and $2.8\sim4.6\times$ speed up over PUMA on the BERT-Large and GPT2-Large models.


\section{Communication Reduction in SoftMax and GELU.}
\label{app:e}

\begin{figure}[h]
    \centering
    \includegraphics[width=0.9\linewidth]{figures/layerwise.pdf}
    \caption{Toy example of two successive Transformer layers. In layer$_i$, the SoftMax and Prune protocol have $n$ input tokens. The number of input tokens is reduced to $n'$ for the Linear layers, LayerNorm and GELU in layer$_i$ and SoftMax in layer$_{i+1}$.}
    \label{fig:layer}
\end{figure}

\begin{table*}[h]
\captionsetup{skip=2pt}
\centering
\scriptsize
\caption{Communication cost (in MB) of the SoftMax and GELU protocol in each Transformer layer.}
\begin{tblr}{
    colspec = {c |c c c c c c c c c c c c},
    row{1} = {font=\bfseries},
    row{2-Z} = {rowsep=1pt},
    % row{4} = {bg=LightBlue},
    colsep = 2.5pt,
    }
\hline
\textbf{Layer Index} & \textbf{0}  & \textbf{1}  & \textbf{2} & \textbf{3} & \textbf{4} & \textbf{5} & \textbf{6} & \textbf{7} & \textbf{8} & \textbf{9} & \textbf{10} & \textbf{11} \\
\hline
Softmax & 642.19 & 642.19 & 642.19 & 642.19 & 642.19 & 642.19 & 642.19 & 642.19 & 642.19 & 642.19 & 642.19 & 642.19 \\
Pruned Softmax & 642.19 & 129.58 & 127.89 & 119.73 & 97.04 & 71.52 & 43.92 & 21.50 & 10.67 & 6.16 & 4.65 & 4.03 \\
\hline
GELU & 698.84 & 698.84 & 698.84 & 698.84 & 698.84 & 698.84 & 698.84 & 698.84 & 698.84 & 698.84 & 698.84 & 698.84\\
Pruned GELU  & 325.10 & 317.18 & 313.43 & 275.94 & 236.95 & 191.96 & 135.02 & 88.34 & 46.68 & 16.50 & 5.58 & 5.58\\
\hline
\end{tblr}
\label{tab:layer}
\end{table*}

{
In Figure \ref{fig:layer}, we illustrate why CipherPrune can reduce the communication overhead of both  SoftMax and GELU. Suppose there are $n$ tokens in $layer_i$. Then, the SoftMax protocol in the attention module has a complexity of $O(n^2)$. CipherPrune's token pruning protocol is invoked to select $n'$ tokens out of all $n$ tokens, where $m=n-n'$ is the number of tokens that are removed. The overhead of the GELU function in $layer_i$, i.e., the current layer, has only $O(n')$ complexity (which should be $O(n)$ without token pruning). The complexity of the SoftMax function in $layer_{i+1}$, i.e., the following layer, is reduced to $O(n'^2)$ (which should be $O(n^2)$ without token pruning). The SoftMax protocol has quadratic complexity with respect to the token number and the GELU protocol has linear complexity. Therefore, CipherPrune can reduce the overhead of both the GELU protocol and the SoftMax protocols by reducing the number of tokens. In Table \ref{tab:layer}, we provide detailed layer-wise communication cost of the GELU and the SoftMax protocol. Compared to the unpruned baseline, CipherPrune can effectively reduce the overhead of the GELU and the SoftMax protocols layer by layer.
}

\section{Analysis on Layer-wise redundancy.}
\label{app:f}

\begin{figure}[h]
    \centering
    \includegraphics[width=0.9\linewidth]{figures/layertime0.pdf}
    \caption{The number of pruned tokens and pruning protocol runtime in different layers in the BERT Base model. The results are averaged across 128 QNLI samples.}
    \label{fig:layertime}
\end{figure}

{
In Figure \ref{fig:layertime}, we present the number of pruned tokens and the runtime of the pruning protocol for each layer in the BERT Base model. The number of pruned tokens per layer was averaged across 128 QNLI samples, while the pruning protocol runtime was measured over 10 independent runs. The mean token count for the QNLI samples is 48.5. During inference with BERT Base, input sequences with fewer tokens are padded to 128 tokens using padding tokens. Consistent with prior token pruning methods in plaintext~\citep{goyal2020power}, a significant number of padding tokens are removed at layer 0.  At layer 0, the number of pruned tokens is primarily influenced by the number of padding tokens rather than token-level redundancy.
%In Figure \ref{fig:layertime}, we demonstrate the number of pruned tokens and the pruning protocol runtime in each layer in the BERT Base model. We averaged the number of pruned tokens in each layer across 128 QNLI samples and then tested the pruning protocol runtime in 10 independent runs. The mean token number of the QNLI samples is 48.5. During inference with BERT Base, input sequences with small token number are padded to 128 tokens with padding tokens. Similar to prior token pruning methods in the plaintext~\citep{goyal2020power}, a large number of padding tokens can be removed at layer 0. We remark that token-level redundancy builds progressively throughout inference~\citep{goyal2020power, kim2022LTP}. The number of pruned tokens in layer 0 mostly depends on the number of padding tokens instead of token-level redundancy.
}

{
%As shown in Figure \ref{fig:layertime}, more tokens are removed in the intermediate layers, e.g., layer $4$ to layer $7$. This suggests there is more redundant information in these intermediate layers. 
In CipherPrune, tokens are removed progressively, and once removed, they are excluded from computations in subsequent layers. Consequently, token pruning in earlier layers affects computations in later layers, whereas token pruning in later layers does not impact earlier layers. As a result, even if layers 4 and 7 remove the same number of tokens, layer 7 processes fewer tokens overall, as illustrated in Figure \ref{fig:layertime}. Specifically, 8 tokens are removed in both layer $4$ and layer $7$, but the runtime of the pruning protocol in layer $4$ is $\sim2.4\times$ longer than that in  layer $7$.
}

\section{Related Works}
\label{app:g}

{
In response to the success of Transformers and the need to safeguard data privacy, various private Transformer Inferences~\citep{chen2022thex,zheng2023primer,hao2022iron-iron,li2022mpcformer, lu2023bumblebee, luo2024secformer, pang2023bolt}  are proposed. To efficiently run private Transformer inferences, multiple cryptographic primitives are used in a popular hybrid HE/MPC method IRON~\citep{hao2022iron-iron}, i.e., in a Transformer, HE and SS are used for linear layers, and SS and OT are adopted for nonlinear layers. IRON and BumbleBee~\citep{lu2023bumblebee} focus on optimizing linear general matrix multiplications; SecFormer~\cite{luo2024secformer} improves the non-linear operations like the exponential function with polynomial approximation; BOLT~\citep{pang2023bolt} introduces the baby-step giant-step (BSGS) algorithm to reduce the number of HE rotations, proposes a word elimination (W.E.) technique, and uses polynomial approximation for non-linear operations, ultimately achieving state-of-the-art (SOTA) performance.
}

{Other than above hybrid HE/MPC methods, there are also works exploring privacy-preserving Transformer inference using only HE~\citep{zimerman2023converting, zhang2024nonin}. The first HE-based private Transformer inference work~\citep{zimerman2023converting} replaces \mysoftmax function with a scaled-ReLU function. Since the scaled-ReLU function can be approximated with low-degree polynomials more easily, it can be computed more efficiently using only HE operations. A range-loss term is needed during training to reduce the polynomial degree while maintaining high accuracy. A training-free HE-based private Transformer inference was proposed~\citep{zhang2024nonin}, where non-linear operations are approximated by high-degree polynomials. The HE-based methods need frequent bootstrapping, especially when using high-degree polynomials, thus often incurring higher overhead than the hybrid HE/MPC methods in practice.
}


% The $\mathtt{\backslash onecolumn}$ command above can be kept in place if you prefer a one-column appendix, or can be removed if you prefer a two-column appendix.  Apart from this possible change, the style (font size, spacing, margins, page numbering, etc.) should be kept the same as the main body.
%%%%%%%%%%%%%%%%%%%%%%%%%%%%%%%%%%%%%%%%%%%%%%%%%%%%%%%%%%%%%%%%%%%%%%%%%%%%%%%
%%%%%%%%%%%%%%%%%%%%%%%%%%%%%%%%%%%%%%%%%%%%%%%%%%%%%%%%%%%%%%%%%%%%%%%%%%%%%%%


\end{document}


% This document was modified from the file originally made available by
% Pat Langley and Andrea Danyluk for ICML-2K. This version was created
% by Iain Murray in 2018, and modified by Alexandre Bouchard in
% 2019 and 2021 and by Csaba Szepesvari, Gang Niu and Sivan Sabato in 2022.
% Modified again in 2023 and 2024 by Sivan Sabato and Jonathan Scarlett.
% Previous contributors include Dan Roy, Lise Getoor and Tobias
% Scheffer, which was slightly modified from the 2010 version by
% Thorsten Joachims & Johannes Fuernkranz, slightly modified from the
% 2009 version by Kiri Wagstaff and Sam Roweis's 2008 version, which is
% slightly modified from Prasad Tadepalli's 2007 version which is a
% lightly changed version of the previous year's version by Andrew
% Moore, which was in turn edited from those of Kristian Kersting and
% Codrina Lauth. Alex Smola contributed to the algorithmic style files.
