% !TEX root = paper.tex

\section{Proof of Main Results}
\subsection{Proof of Theorem \ref{thm:equilibrium}}
% \gracecomment{
We organize the proof as follows. We first present some intermediate results before proving Theorem \ref{thm:equilibrium}. Next, we show that if the strategy profile $\bbf$ is a Nash equilibrium, it must satisfy each of the four conditions in Theorem \ref{thm:equilibrium}. Finally, we prove that any strategy profile that satisfies the four conditions is a Nash equilibrium. 

\begin{lemma}\label{lemma:A1}
    %Suppose the score of candidate S follows the continuous distribution D with density $\varphi(s)$. 
    Let X be a measurable set on [0,1] with $\Pr(S\in X)=x$ where $x\in (0,1]$. Then for any $0<x'<x$, we can always find a subset of X such that $\Pr(S\in X')=x'$.
\end{lemma}

\begin{proof}
    Consider the function $f(t) = Pr(S\in (X\cap [0,t]))$. We will first show that $f(t)$ is continuous. 
    First, suppose $t' < t$.
    Then, we can write
    \begin{align*}
    f(t) 
    &= \Pr(S\in (X\cap [0,t]))\\
    &= \Pr(S\in (X\cap [0,t'])) +  \Pr (S \in (X\cap [t',t]) )\\
    &= f(t') +  \Pr (S \in (X\cap [t',t]) ).
    \end{align*}
    Clearly, as $t' \to t$, the term $\Pr (S \in (X\cap [t',t]) )$ goes to 0.
    Therefore, $\lim_{t' \to t^-} f(t') = f(t)$.
    Similarly, when $t' > t$, we can write $f(t') = f(t) + \Pr (S \in (X\cap [t,t']) )$, which implies that $\lim_{t' \to t^+} f(t') = f(t)$.
    Therefore, $\lim_{t' \to t} f(t') = f(t)$, and hence $f(t)$ is continuous.
    
    % Since S follows $\cD$ which is continuous, as $t'$ approaches $t$, $f(t)-f(t')$ goes to zero. $f(t)$ is continuous. 
    Furthermore, $f(0) = 0$ and $ f(1) = x$. Hence, by the intermediate value theorem, for any $x'\in [0,x]$ there is some $t^*\in [0,1]$ such that $f(t^*)=x'$. Let $S' = X\cap [0,t^*]$, we have $\Pr(S\in X') = x'$ as required. 
\end{proof}


\begin{proof}[\textbf{Proof of Theorem \ref{thm:equilibrium}}]
Suppose the strategy profile $\bbf$ is a Nash equilibrium. We will show that there exist thresholds $0\leq \tau_1 \leq \tau_2\cdots, \tau_N\leq 1$ that satisfy the four conditions listed in the theorem.

\begin{enumerate}
    \item  $\Pr(f_i(S)=1) = c$ for all $i \in [N]$.
    \item  $M(S) = m;s \in [\tau_m, \tau_{m+1})$ for all $m = 0, 1,  \dots, N$.

    \item  $U_n(\tau_n) \leq U_m(\tau_m)$ for all $n < m \leq \Mmax(\bbf)$.
    
    \item Consider any $n < m \leq \Mmax(\bbf)+1$ where 
    $U_n(\tau_n) < U_m(\tau_m)$, and consider any firm $i$ and score $s \in [\tau_n, \tau_{n+1})$ where $f_i(s) = 1$ and $U_n(s) < U_m(\tau_m)$. 
    For any $s' \in [\tau_{m-1}, \tau_m)$ where $U_m(s') > U_n(s)$, we have that $f_i(s') = 1$.


\end{enumerate}

Let's start by proving the \textbf{first condition}. Given capacity c, we want to show that $\Pr(f_i(S) = 1) = c$ for all $i\in[N]$. Suppose, by contradiction, that $\Pr(f_i(S)=1)<c$ for some $i\in[N]$. We will show that firm $i$ can deviate from $f_i$ to strictly improve its strategy. Let $K = \{s\in [0,1]: f_i(s)=1\}$ be the applicants interviewed by firm $i$ under $\bbf$. Under the strategy profile $\bbf$, by equation (1), the utility derived by firm $i$ is 
\begin{align*}
    u(f_i, f_{-i}) &= \int_{0}^1 f_i(s) U_{M(s, \mathbf{f})}(s) \varphi(s)ds\\
    & = \int_K U_{M(s, \mathbf{f})}(s) \varphi(s)ds.
\end{align*}
Since $\Pr(f_i(S) = 1)<c$, by Lemma \ref{lemma:A1}, there is a subset $E\subseteq [0,1]\setminus K$ such that $Pr(S\in E)=c-Pr(f_i(S)=1)$.


Consider an alternative strategy $f'_i$ where $f'_i(s) = 1$ if and only if $s \in K \cup E$.
We have $\Pr(f_i'(S) = 1) = \Pr(S \in K) + \Pr(S \in E) = c$, hence $f_i'$ satisfies the capacity requirement.
We will now show $u(f'_i, f_{-i})>u(f_i, f_{-i})$.
Let $\bbf' = (f_1, \cdots, f'_i, \cdots, f_N)$ be the strategy profile after firm $i$ deviates from $f_i$.
% $K' = K\cup E$ is the union of two disjoint set, and therefore $\Pr(S\in K')=\Pr(S\in K)+\Pr(S\in E) = c$. Hence, $f'_i$ is a valid strategy. Next, we will show that $u(f'_i, f_{-i})>u(f_i, f_{-i})$. 
By construction, $M(s, \mathbf{f'}) = M(s, \mathbf{f})$ on $K$. As a result, the utility derived by firm $i$ under $\bbf'$ is 
\begin{align*}
        u(f'_i, f_{-i})
        & = \int_{0}^1 f'_i(s) U_{M(s, \mathbf{f'})}(s) \varphi(s)ds\\
        % & = \int_{K'} U_{M(s, \mathbf{f'})}(s) \varphi(s)ds\\
        % & = \int_{K\cup E} U_{M(s, \mathbf{f'})}(s) \varphi(s)ds\\
        & = \int_{K} U_{M(s, \mathbf{f'})}(s) \varphi(s)ds + \int_{E} U_{M(s, \mathbf{f'})}(s) \varphi(s)ds\\
        & = \int_{K} U_{M(s, \mathbf{f})}(s) \varphi(s)ds + \int_{E} U_{M(s, \mathbf{f'})}(s) \varphi(s)ds\\
        & = u(f_i, f_{-i}) + \int_{E} U_{M(s, \mathbf{f'})}(s) \varphi(s)ds.
        % & > u(f_i, f_{-i})
    \end{align*}
By assumption, $U_{M(s, \mathbf{f'})}(s)>0$ and $\varphi(s)>0$ whenever $s>0$. We must have $\int_{E} U_{M(s, \mathbf{f'})}(s) \varphi(s)ds>0$ since $\Pr(S\in E)>0$. Therefore, 
$u(f'_i, f_{-i}) > u(f_i, f_{-i})$.
This is a contradiction to $\bbf$ being an NE, and hence it must be that  $\Pr(f_i(S) = 1) = c$ for all $i\in [N]$.

We will then prove the \textbf{second condition}. Our goal is to show that there exist thresholds $\tau_1\leq \cdots \tau_N$ such that $M(s;\bbf) = m;s \in [\tau_m, \tau_{m+1})$ for all $m \in [\Mmax(\bbf)]$. We will prove the statement by constructing a sequence of thresholds $\tau_1,\cdots \tau_N$ that satisfy the requirements. Define the thresholds as follows: 
\begin{align*}
    \tau_m = 
\begin{cases} 
\min \{s\in [0,1]; M(s;\bbf)\geq m\} & \text{if } m\leq \Mmax(\bbf), \\
1 & \text{if } \Mmax(\bbf)<m\leq N
\end{cases}
\end{align*}
We will show $M(s;\bbf) = m;s \in [\tau_m, \tau_{m+1})$ for all $m\in [\Mmax(\bbf)]$.

We will first prove the statement for $m=0$. By assumption, $\tau_1 = \min \{s\in [0,1]; M(s;\bbf)\geq 1\}$. Therefore, for any $s\in [0,\tau_1)$, we must have $M(s;\bbf)=0$. Hence, $M(s;\bbf) = 0$ for all $s \in [0, \tau_{1})$. 

Next, we will show that for any $0<m\leq \Mmax(\bbf)$, $M(s;\bbf) = m$ for any $s \in [\tau_m, \tau_{m+1})$. We will first show that $M(s;\bbf) \leq m$ for all $s \in [\tau_m, \tau_{m+1})$. Consider the following two cases: $m<\Mmax(\bbf)$ and $m=\Mmax(\bbf)$. If $m<\Mmax(\bbf)$, by assumption, $\tau_{m+1} = \min \{s\in [0,1]; M(s;\bbf)\geq m+1\}$, and, as a result, we must have $M(s;\bbf) \leq m$ for all $s<\tau_{m+1}$. If $m= \Mmax(\bbf)$, then clearly $M(s;\bbf) \leq m$ for all $s<\tau_{m+1}$. Hence, $M(s;\bbf) \leq m$ for any $s \in [\tau_m, \tau_{m+1})$. Next, we will show $M(s,\bbf)\geq m$ for $s\in [\tau_m, \tau_{m+1})$. Suppose, by contradiction, $M(s,\bbf)< m$ for some $s_0\in [\tau_m, \tau_{m+1})$. Since for each strategy the applicants interviewed can be written as a union of a finite number of intervals, there exists an interval $s_0\in [a_1,b_1]\subset [\tau_m, \tau_{m+1}]$ with $a_1<b_1$ such that $M(s;\bbf) < m$ on this interval. Similarly, since $\tau_{m} = \min \{s\in [0,1]; M(s;\bbf)\geq m\}$, there exists an interval $[\tau_m, b_2]$ with $b_2>\tau_m$ such that $M(s;\bbf) = m$ on this interval. We notice that $b_2<a_1$ by construction and $[\tau_m, b_2]$ and $[a_1, b_1]$ are disjoint intervals. There are strictly fewer firms competing for applicants with score $s\in [a_1, b_1]$ then applicants with score $s\in[\tau_m, b_2]$. Therefore, there must exist some firm $i$ with $K_i = \{s\in [0,1]; f_i(s)=1\}$ such that firm $i$ interviews some applicants in $[\tau_m, b_2]$ but not all applicants in $[a_1, b_1]$. Mathematically, $K_i\cap  [\tau_m,b_2)\neq \emptyset$ and $K_i^c\cap [a_1,b_1]\neq \emptyset$. We also notice that  
\begin{align*}
    U_{M(s;\bbf)+1}(s)\geq U_{m}(s)>U_m(a_1)>U_m(b_2)
\end{align*}
for $s\in(a_1,b_1]$ since $U_n$ is strictly increasing in $s$ and decreasing in $n$. This implies that if firm $i$ interviews an applicant with score $s\in[a_1,b_1]$, the resulting utility is strictly higher than the utility firm $i$ can derive by interviewing an applicant with score $s\in [\tau_m, b_2]$. Therefore, firm $i$ can deviate from $f_i$ by moving support from $K_i\cap  [\tau_m,b_2)$ to $K_i^c\cap [a_1,b_1]$, where firm $i$ would earn strictly more utility. This is a contradiction to $\bbf$ being a Nash equilibrium, and hence it must be that $M(s,\bbf)\geq m$ for all $s\in[\tau_m, \tau_{m+1}]$. Since $M(s,\bbf)\leq m$ and $M(s,\bbf)\geq m$ for all $s\in[\tau_m, \tau_{m+1})$, we must have $M(s,\bbf)= m$ for all $s\in[\tau_m, \tau_{m+1})$.

%Suppose, by contradiction, there exists some interval $[a,b]\subset [0, \tau_1]$ such that $\Pr(M(S) > 0 \;|\; s \in [a, b]) = 1$. We will show that there exists at least one firm that can improve its strategy by deviating from its current strategy profile. If $\Pr(M(S) > 0 \;|\; s \in [a, b]) = 1$, then there is at least one firm $i$ such that $\{s\in [0,1]: f_i(s)>0\}\cup [a,b]\neq \emptyset$. Let $K_i = \{s\in [0,1]: f_i(s)>0\}$. 

We will then prove the \textbf{third condition}. Assume $\bbf$ is a Nash equilibrium and the thresholds are defined as in condition 2. We will show that $U_m(\tau_m) \leq U_{m+1}(\tau_{m+1})$ for all $m \leq \Mmax(\bbf)-1$. Consider two cases: $c=1$ and $c<1$. If $c=1$, then each firm will interview all applicants, and we have $\tau_1=\cdots = \tau_N = 0$. Accordingly, $U_1(\tau_1)=\cdots = U_N(\tau_N) = 0$. We get the desired result.

If $c<1$, we will first show that $\tau_1>0$. Suppose, by contradiction, $\tau_1=0$. Then there exists a firm $i$ and $\delta>0$ such that firm $i$ interviews all applicants with score in $[0, \delta]$. Since $c<1$, there exists an interval $[a,b]\subset [\delta, 1]$ with $a<b$ such that $f_i(s)=0$ for $s\in[a,b]$. Since $U_n(s)$ is continuous and monotone, $\min_{s\in[a,b]}U_{M(s;\bbf)+1}(s)>0$ and $U_n(0) = 0$, there exists $0<\epsilon\leq \delta$ such that $U_{M(s;\bbf)}(s) < \min_{s\in[a,b]}U_{M(s;\bbf)+1}(s)$ for $s\in [0,\epsilon]$. By construction, if firm $i$ interviews an applicant with score $s\in [a,b]$, the utility is strictly higher than interviewing any applicant with score $s\in[0, \epsilon]$. Therefore, firm $i$ can deviate from $f_i$ by moving support from $[0,\epsilon]$ to within $[a,b]$, where firm $i$ would earn strictly more utility. This is a contradiction to $\bbf$ being a Nash equilibrium. Hence, we must have $\tau_1>0$. 


%Let $K_i = \{s\in [0,1]; f_i(s)=1\}$, we must have $K_i^c \neq \emptyset$ since $c>1$. Therefore, there is some interval $[a,b]\in K^c$ with $a>\delta_1$. Given that $U_n(s)$ is continuous and $U_n(0) = 0$, there exists an $\epsilon>0$ such that $\delta_2 \in [0,\delta_1]$ and $U_{M(s;\bbf)}(\delta_2) < \min_{s\in[a,b]}U_{M(s;\bbf)}(s)$. By construction, if firm $i$ interviews an applicant with score $s\in [a,b]$, the utility is strictly higher . Therefore, there exists an $\epsilon>0$ where firm $i$ can deviate from $f_i$ by moving support from $[0,\delta_2]$ to within $[a,b]$, where they would earn strictly more utility. This is a contradiction to $\bbf$ being a Nash equilibrium. Hence, we must have $\tau_1>0$. 

Next, we will show that $\tau_m \neq \tau_{m+1}$ for all $m=1, \cdots, \Mmax(\bbf)$. Suppose, by contradiction, $\tau_m = \tau_{m+1}$ for some $m$. Since $M(s,\bbf)<m$ for $s<\tau_m$ and $M(s,\bbf)\geq m+1$ for $s\geq\tau_{m+1}$, there exists firm $i$, $\delta_1>0$, and $\delta_2>0$ such that $f_i(s)=0$ if $s\in[\tau_m-\delta_1, \tau_m]$ and $f_i(s)=1$ if $s\in[\tau_{m+1}, \tau_{m+1}+\delta_2]$. We notice that $U_{m}(\tau_m)>U_{m+1}(\tau_{m}) = U_{m+1}(\tau_{m+1})$ since $U_n(s)$ is strictly decreasing in n. Since $U_n(s)$ is continuous, there exists $0<\epsilon_1<\delta_1$ and $0<\epsilon_2<\delta_2$ such that 
\begin{align*}
    U_{m}(\tau_m)>U_{m}(\tau_m-\epsilon_1)>U_{m+1}(\tau_{m+1}+\epsilon_2)>U_{m+1}(\tau_{m+1}).
\end{align*}
Therefore, the utility firm $i$ can derive by interviewing an applicant with score $s\in[\tau_m-\epsilon_1, \tau_m]$ is strictly higher than the utility it can derive by interviewing an applicant with score $s\in[\tau_{m+1}, \tau_{m+1}+\epsilon_2]$. As a result, firm $i$ can deviate from $f_i$ by moving support from $[\tau_{m+1}, \tau_{m+1}+\epsilon_2]$ to $[\tau_m-\epsilon_1, \tau_m]$. This is a contradiction to $f$ being a Nash equilibrium. Hence, we must have $\tau_m \neq \tau_{m+1}$ for all $m=1, \cdots, \Mmax(\bbf)$.

We will then show that $U_m(\tau_m) \leq U_{m+1}(\tau_{m+1})$ for all $m< \Mmax(\bbf)$. Suppose, by contradiction, $U_m(\tau_m) > U_{m+1}(\tau_{m+1})$ for some $m=1, \cdots, \Mmax(\bbf)-1$. Since $\tau_1>0$ and $U_n(s)$ is strictly increasing in $s$, there exists $s_1<\tau_m$ and $s_2\in [\tau_m, \tau_{m+1}]$ such that $U_m(\tau_m)>U_m(s_1)>U_{m+1}(s_2)>U_{m+1}(\tau_{m+1})$. We also notice that there exists at least one firm $i$ such that, under $\bbf$, firm $i$ interviews some applicants with score in $[s_2, \tau_{m+1}]$ but not all applicants with score in $[s_1, \tau_m]$. If firm $i$ interviews an applicant with score $s\in[s_1, \tau_m]$, the utility firm $i$ can derive is at least $U_{m}(s_1)$, which is strictly higher than the utility derived by interviewing an applicant with score $s\in [s_2, \tau_{m+1}]$. Therefore, firm $i$ can deviate from $f_i$ by moving support from $[s_2, \tau_{m+1}]$ to within $[s_1, \tau_m]$ to earn strictly more utility. This contradicts to the assumption that $f$ is a Nash equilibrium. Therefore, we must have $U_m(\tau_m) \leq U_{m+1}(\tau_{m+1})$ for all $m < \Mmax(\bbf)$.

Finally, we will prove the \textbf{fourth condition}. Suppose $U_n(\tau_n) < U_m(\tau_m)$ for some $n<m\leq \Mmax(\bbf)+1$. Let firm $i$ be a firm such that $f_i(s_1)=1$ and $U_n(s_1)<U_m(\tau_m)$ for some score $s_1\in[\tau_n, \tau_{n+1}]$. We will show that $f_i(s_2) = 1$ for any $s_2\in[\tau_{m-1}, \tau_m]$ with $U_m(s_2)>U_n(s_1)$. Suppose, in contradiction, $f_i(s_2) = 0$ for some $s_2\in[\tau_{m-1}, \tau_m]$ with $U_m(s_2)>U_n(s_1)$. Since the support of firm $i$ can be written as a union of intervals and $U_n(s)$ is continuous, there exists an interval such that $s_1\in [a_1, b_1]$ with $a_1<b_1$ and $f_i(s)=1$ for $s\in[a_1, b_1]$. Similarly, there exists an interval such that $s_2\in[a_2,b_2]$ with $a_2<b_2$, $f_i(s)=0$ for $s\in[a_2, b_2]$, and $U_m(s)>U_n(b_1)$ for $s\in[a_2,b_2]$. We notice that the utility firm $i$ can derive by interviewing an applicant with score $s\in [a_1, b_1]$ is strictly less than the utility firm $i$ can derive by interviewing an applicant with score $s\in [a_2, b_2]$. Therefore, firm $i$ can increase its utility by moving some support from $[a_1, b_1]$ to $[a_2, b_2]$. This is a contradiction to $\bbf$ being a Nash equilibrium. Hence, we must have $f_i(s_2) = 1$ for any $s_2\in[\tau_{m-1}, \tau_m]$ with $U_m(s_2)>U_n(s_1)$.





We will prove the \textbf{other direction} of the statement which states that any strategy profile $\bbf$ that satisfies the four conditions listed is a Nash equilibrium. We want to show that any firm $i$ cannot earn a higher utility by deviating from its current strategy profile. Let $f_i$ be the current strategy profile of firm $i$ and $f'_i$ be an arbitrary alternative strategy profile. We will show that $u(f_i, f_{-i})\geq u(f'_i, f_{-i})$ if $\bbf $ satisfies the four conditions.

Denote $K_i = \{s\in[0,1], f_i(s)=1\}$. Under $\bbf$, we can express the utility of firm $i$ as 
\begin{align}
    u(f_i, f_{-i}) &= \int_{K\cap K'} U_{M(s, \mathbf{f})}(s) \varphi(s)ds + \int_{K\setminus K'}U_{M(s, \mathbf{f})}(s)\varphi(s)ds.  \label{equ:18}
\end{align}
Let $\bbf'$ denote the strategy profile after firm $i$ deviates from $f_i$, and denote $K'_i = \{s\in[0,1], f'_i(s)=1\}$. We notice that $M(s, \mathbf{f'})=M(s, \mathbf{f})+1$ if $s\in K'\setminus K$, and $M(s, \mathbf{f'})=M(s, \mathbf{f})$ if $s\in K\cap K'$. Therefore, the utility under the alternative strategy is 
\begin{align}
    u(f'_i, f_{-i}) & = \int_{K\cap K'}U_{M(s, \mathbf{f'})}(s) \varphi(s)ds + \int_{K'\setminus K}U_{M(s, \mathbf{f'})}(s) \varphi(s)ds \nonumber \\
    & = \int_{K\cap K'}U_{M(s, \mathbf{f})}(s) \varphi(s)ds + \int_{K'\setminus K}U_{M(s, \mathbf{f})+1}(s) \varphi(s)ds.  \label{equ:20}
\end{align}

To show $u(f_i, f_{-i})\geq u(f'_i, f_{-i})$, we only need to show the second part of equation (\ref{equ:20}) is less or equal to the second part of equation (\ref{equ:18}). To show this, we will prove that for any $s\in K_i\setminus K'_i$, $U_{M(s, \mathbf{f})}(s)\geq U_{M(s', \mathbf{f})+1}(s')$ for all $s\in K_i'\setminus K_i$. Let $s\in K_i\setminus K'_i$, $s'\in K_i'\setminus K_i$ be arbitrary. Suppose $s\in [\tau_m, \tau_{m+1}]$ and $s'\in [\tau_n, \tau_{n+1}]$ for some $m,n \in [\Mmax(\bbf)]$. Consider the following two cases: $U_{m}(\tau_m) \geq U_{n+1}(\tau_{n+1})$ or $U_{m}(\tau_m) < U_{n+1}(\tau_{n+1})$.

If $U_{\tau_m}(\tau_m) \geq U_{\tau_{n+1}}(\tau_{n+1})$, by monotonicity of $U_n(s)$, we have 
\begin{align*}
    U_{m}(s)\geq U_{m}(\tau_m)\geq U_{n+1}(\tau_{n+1})\geq U_{n+1}(s').
\end{align*}
Hence $U_{m}(s) \geq U_{n+1}(s')$. 

If $U_{m}(\tau_m) < U_{n+1}(\tau_{n+1})$, by condition 3, we must have $m
<n+1$. Suppose, by contradiction, $U_{m}(s) < U_{n+1}(s')$. By condition 4, given that $s\in K_i\setminus K'_i$,  we must have $s'\in K_i$ as well. This is a contradiction to $s'\in K_i'\setminus K_i$. Hence we must have $U_{m}(s) \geq U_{n+1}(s')$.

Therefore, we conclude that for any $s\in K_i\setminus K'_i$, $s'\in K_i'\setminus K_i$ , we must have $U_{M(s, \mathbf{f})}(s)\geq U_{M(s', \mathbf{f})+1}(s')$. As a result, $\min_{s\in K_i\setminus K'_i}U_{M(s, \mathbf{f})}(s)\geq \max_{s'\in K_i'\setminus K_i}U_{M(s', \mathbf{f})+1}(s')$. Furthermore, by condition 1, $\Pr(S\in K_i) = \Pr(S\in K'_i) = c$. Hence, 
\begin{align}
    \Pr(S\in K_i\setminus K'_i) &=\Pr(S\in K_i) - \Pr(S\in K_i\cap K'_i)\\
    &= Pr(S\in K'_i)- \Pr(S\in K_i\cap K'_i)\\
    & = \Pr(S\in K'_i\setminus K_i)
\end{align}
We can rewrite equation (\ref{equ:20}) as follows:
\begin{align*}
    u(f'_i, f_{-i}) & \leq \int_{K_i\cap K'_i}U_{M(s, \mathbf{f})}(s) \varphi(s)ds + \Pr(S\in K'_i\setminus K_i)(\max_{s'\in K_i'\setminus K_i}U_{M(s', \mathbf{f})+1}(s') )\\
    & \leq \int_{K_i\cap K'_i}U_{M(s, \mathbf{f})}(s) \varphi(s)ds + \Pr(S\in K_i\setminus K'_i)(\min_{s\in K_i\setminus K'_i}U_{M(s, \mathbf{f})}(s))\\
    & \leq \int_{K\cap K'} U_{M(s, \mathbf{f})}(s) \varphi(s)ds + \int_{K\setminus K'}U_{M(s, \mathbf{f})}(s)\varphi(s)ds\\
    & = u(f_i, f_{-i}).
\end{align*}

Therefore, $u(f'_i, f_{-i}) \leq u(f_i, f_{-i})$ for any arbitrary firm $i$ and alternative strategy $f'_i$. We conclude that none of the firms is able to increase their utility by deviating from the current strategy profile, and $\bbf$ is at Nash equilibrium. 




\end{proof}
% }


\subsection{Proof of Proposition 3.1}
\begin{proof}
    Let $\cD$ be an arbitrary distribution and $c\in (0,1]$ be arbitrary. Suppose $\cI(N) = (N, c, \cD, \corr)$ is an instance with $N$ firms and $\mathbf{f_N}$ is a Nash equilibrium for the instance. 
    
    We will first show that $\tau_m \geq m \tau_1$ for all $m\in [\Mmax(\mathbf{f_N})]$. For $m =1$, we have $\tau_m = \tau_1$ and the inequality holds for $m=1$. For $1<m\leq \Mmax(\mathbf{f_N})$, the third condition in Theorem \ref{thm:equilibrium} gives $U_1(\tau_1)\leq U_m(\tau_m)$.  Under the correlated decision rule, the utility function can be expressed as  $U_n(s) = s/n$. Therefore, the above condition can be written as $\tau_1\leq \frac{\tau_m}{m}$. Hence, $m\tau_1\leq \tau_m$ for all $1<m\leq \Mmax(\bbf)$. Therefore, we can conclude that $\tau_m \geq m \tau_1$ for all $m\in [\Mmax(\mathbf{f_N})]$. 

    Next, we want to show that if $\mathbf{f_N}$ is an equal-utility Nash equilibrium, then $\tau_m = m \tau_1$ for all $m\in [\Mmax(\mathbf{f_N})]$. For $m =1$, we have $\tau_m = \tau_1$ and the equality holds. For $1<m\leq \Mmax(\mathbf{f_N})$, by Definition \ref{def:types_of_NE}, $U_1(\tau_1)= U_m(\tau_m)$. Since the instance is under the correlated decision rule, we have $m\tau_1= \tau_m$ for all $1<m\leq \Mmax(\mathbf{f_N})$. Hence, we've shown that $\tau_m = m \tau_1$ for all $m\in [\Mmax(\mathbf{f_N})]$.
\end{proof}


\subsection{Derivation of $U_n(s)$ when $\theta = \indep$} \label{deri:U_n_indep}
When hiring decisions are independent, the probability that a firm successfully hires an applicant of score $s$ if they interview them is 


\begin{align}
    U_n(s) &= s(\sum_{i=0}^{n-1} \binom{n-1}{i} s^i (1-s)^{n-1-i} \frac{1}{i+1}) \nonumber\\
    & = \frac{s}{ns}(\sum_{i=0}^{n-1} \binom{n}{i+1} s^{i+1} (1-s)^{n-1-i})\nonumber\\
    & = \frac{1}{n}(1-(1-s)^n) \label{equ:Un(s)_indep}
\end{align}


\subsection{Proof of Proposition \ref{prop:NE_indep}}
Before proving Proposition \ref{prop:NE_indep}, let's first prove the following lemmas:

%\begin{lemma}
    %Let $\cI(N) = (N, c, \cD, \indep)$ be the instance with fixed capacity $c$ and the independent decision rule, $\lim_{N \to \infty} U_N(s) = \frac{1}{N}$.
    %fix s, $U(s,n)$ converges to $\frac{1}{n}$ for large enough n.
%\end{lemma}

%\begin{proof}
    %\begin{align*}
        %\frac{1}{n} - U(s,n) &= \frac{1}{n}-\frac{1}{n}+\frac{(1-s)^n}{n}\\
        %& = \frac{(1-s)^n}{n} \leq \frac{1}{n}\rightarrow 0
    %\end{align*}
%\end{proof}

\begin{lemma} \label{lemma:3.2.1}
     For any distribution $\cD$, capacity $c \in (0, 1)$ and decision rule $\theta \in \{\indep, \corr\}$,
    let $\cI(N) = (N, c, \cD, \theta)$ is the instance parameterized by  number of firms $N$. Let $\mathbf{f_N}$ be a Nash equilibrium for instance $\cI(N)$. Then, $lim_{N\rightarrow \infty} \Mmax(\mathbf{f_N}) = \infty$.
\end{lemma}

\begin{proof}
    We will first show that $Nc\leq \Mmax(\mathbf{f_N})$. By condition 1 of Theorem \ref{thm:equilibrium}, $\Pr(f_i(S)=1)=c$ for each firm $i$. As a result, the total capacity of $N$ firms is $\sum_{i=1}^N \Pr(f_i(S)=1) = Nc$. By condition 2 of Theorem \ref{thm:equilibrium}, there exists thresholds $\tau_1, \tau_2, \cdots, \tau_{\Mmax(\mathbf{f_N})}$ such that $M(s; \mathbf{f_N}) = m$ for $s\in [\tau_m, \tau_{m+1})$ for $m = [\Mmax(\mathbf{f_N})]$. Therefore, the total capacity of $N$ firms can also be written as follows:

    \begin{align*}
        Nc & = \Pr(S\in [\tau_1, \tau_2]) + 2 \Pr(S\in [\tau_2, \tau_3])\cdots + \Mmax(\mathbf{f_N}) \Pr(S\in [\tau_{\Mmax(\bbf)}, 1])\\
        &= \int_{\tau_1}^{\tau_2} \varphi(s)ds + 2\int_{\tau_2}^{\tau_3} \varphi(s)ds  \cdots +\Mmax(\mathbf{f_N})\int_{\tau_{\Mmax(\mathbf{f_N})}}^{1} \varphi(s)ds\\
        & \leq \Mmax(\mathbf{f_N})\int_{\tau_1}^{\tau_2} \varphi(s)ds + \Mmax(\mathbf{f_N})\int_{\tau_2}^{\tau_3} \varphi(s)ds  \cdots +\Mmax(\mathbf{f_N})\int_{\tau_{\Mmax(\bbf)}}^{1} \varphi(s)ds\\
        & = \Mmax(\mathbf{f_N})\int_{\tau_1}^{1} \varphi(s)ds \\
        & \leq \Mmax(\mathbf{f_N})\int_{0}^{1} \varphi(s)ds\\
        & = \Mmax(\mathbf{f_N}).
    \end{align*}

Hence, $Nc\leq \Mmax(\mathbf{f_N})$. As $N$ goes to infinity, for the inequality to hold, we must have $\Mmax(\mathbf{f_N})$ go to infinity as well. Therefore, $\lim_{N\rightarrow \infty} \Mmax(\mathbf{f_N}) = \infty$.     
\end{proof}

\begin{lemma} \label{lemma:3.2.2}
    For any distribution $\cD$ and capacity $c \in (0, 1)$,
    let $\cI(N) = (N, c, \cD, \indep)$ is the instance parameterized by  number of firms $N$. Let $\mathbf{f_N}$ be a Nash equilibrium for instance $\cI(N)$. Then $\lim_{N\rightarrow \infty} \tau_{\Mmax(\bbf)-1}= 0$.
\end{lemma}

\begin{proof}
    Suppose $\bbf$ is a Nash equilibrium. $U(\tau_{\Mmax(\mathbf{f_N})-1}, \Mmax(\bbf)-1) =t$. When the hiring decisions are independent, by equation \ref{equ:Un(s)_indep} we have 
    \begin{align*}
        U(\tau_{\Mmax(\mathbf{f_N})-1}, \Mmax(\mathbf{f_N})-1) = \frac{1}{\Mmax(\mathbf{f_N})-1}(1-(1-\tau_{\Mmax(\mathbf{f_N})-1})^{\Mmax(\mathbf{f_N})-1}) = t.
    \end{align*}
    As a result, we can express $\tau_{\Mmax-1}$ as
    \begin{align*}
        \tau_{\Mmax(\mathbf{f_N})-1} = 1-(1-(\Mmax(\mathbf{f_N})-1)t)^{\frac{1}{\Mmax(\mathbf{f_N})-1}}.
    \end{align*}
    Define $f(t) = 1-(1-(\Mmax(\mathbf{f_N})-1)t)^{\frac{1}{\Mmax(\mathbf{f_N})-1}}$, we will show $f(t)$ is strictly decreasing in $t$. 
    
    We will first show that $t\leq\frac{1}{\Mmax(\mathbf{f_N})}$. Suppose, by contradiction, $t>\frac{1}{\Mmax(\mathbf{f_N})}$. Since $\frac{1}{\Mmax(\mathbf{f_N})}=U(1,\Mmax(\mathbf{f_N}))$, we must have $U(\tau_{\Mmax(\mathbf{f_N})-1}, \Mmax(\mathbf{f_N})-1)>U(1,\Mmax(\mathbf{f_N}))$. By condition 4 of Theorem \ref{thm:equilibrium}, $\mathbf{f_N}$ is not a Nash equilibrium, which leads to a contradiction. Therefore, we must have $t\leq\frac{1}{\Mmax(\mathbf{f_N})}$.
    %then for the n firms that have support in $[1-\delta_1, 1]$ there must be one firm that does not have support in $[\tau_{n-1}-\delta_2, \tau_{n-1}]$ we can switch it to the second interval which increases the utility.

    Next, we will show that $f(t)$ is strictly decreasing in t. The derivative of $f(t)$ is

    \begin{align*}
        f'(t) &= (\Mmax(\mathbf{f_N})-1)\frac{1}{\Mmax(\mathbf{f_N})-1}(1-(\Mmax(\mathbf{f_N})-1)t)^{\frac{2-\Mmax(\mathbf{f_N})}{\Mmax(\mathbf{f_N})-1}}\\
        & =(1-(\Mmax(\mathbf{f_N})-1)t)^{\frac{2-\Mmax(\mathbf{f_N})}{\Mmax(\mathbf{f_N})-1}}.
    \end{align*}

    Given that $t\leq\frac{1}{\Mmax(\bbf)}$, we have 
    \begin{align*}
        1-(\Mmax(\mathbf{f_N})-1)t\geq 1-(\Mmax(\mathbf{f_N})-1)\frac{1}{\Mmax(\mathbf{f_N})} = \frac{1}{\Mmax(\mathbf{f_N})} > 0.
    \end{align*}

    %$$f'(t)\geq (1-\frac{\Mmax(\bbf)-1}{\Mmax(\bbf)})^{\frac{2-\Mmax(\bbf)}{n-1}} = (\frac{1}{n})^{\frac{2-n}{n-1}}>0$$

    Hence, $f'(t)>0$ and $f(t)$ decreases with $t$. Given that $t\leq\frac{1}{\Mmax(\mathbf{f_N})}$, we get

    \begin{align*}
        \tau_{\Mmax(\mathbf{f_N})-1} \leq 1-(1-\frac{\Mmax(\mathbf{f_N})-1}{\Mmax(\mathbf{f_N})})^{\frac{1}{\Mmax(\mathbf{f_N})-1}} = 1-\frac{1}{\Mmax(\mathbf{f_N})^{\frac{1}{\Mmax(\mathbf{f_N})-1}}}.
    \end{align*}

    Taking the logarithm of $\Mmax(\mathbf{f_N})^{\frac{1}{\Mmax(\mathbf{f_N})-1}}$ we can show that $\lim_{\Mmax(\mathbf{f_N})\rightarrow \infty} \Mmax(\mathbf{f_N})^{\frac{1}{\Mmax(\mathbf{f_N})-1}} = 1$. Hence, $\lim_{\Mmax(\mathbf{f_N})\rightarrow \infty} \tau_{\Mmax(\mathbf{f_N})-1}= 0$. By Lemma \ref{lemma:3.2.1}, $\Mmax(\mathbf{f_N})\rightarrow \infty$ as $N \rightarrow \infty$, we must have
    \begin{align*}
        \lim_{N\rightarrow \infty} \tau_{\Mmax(\bbf)-1}= 0.
    \end{align*}
*\end{proof}

\begin{proof} [\textbf{Proof of Proposition \ref{prop:NE_indep}}]
    We first notice that $|M(s; \mathbf{f_N}) - \Mmax(\mathbf{f_N})|>1$ if and only if $M(s; \mathbf{f_N})< \Mmax(\mathbf{f_N})-1$ for any $s\in [0,1]$. By condition 2 of Theorem \ref{thm:equilibrium}, there exists thresholds $\tau_1, \tau_2, \cdots, \tau_{\Mmax(\bbf)}$ such that $M(s; \mathbf{f_N}) = m$ for $s\in [\tau_m, \tau_{m+1}]$ for $m = [\Mmax(\bbf)]$. Therefore, $M(s; \mathbf{f_N})< \Mmax(\mathbf{f_N})-1$ is equivalent to $S\in[0,\tau_{\Mmax(\mathbf{f_N})-1})$.
    \begin{align*}
        \Pr(|M(S; \mathbf{f_N}) - \Mmax(\mathbf{f_N})|>1) 
        & = \Pr(M(S; \mathbf{f_N})< \Mmax(\mathbf{f_N})-1)\\
        & = \Pr(S\in[0,\tau_{\Mmax(\mathbf{f_N})-1})).
    \end{align*}
    By lemma \ref{lemma:3.2.2}, $\lim_{N\rightarrow \infty} \tau_{\Mmax(\bbf)-1}= 0$. As a result, we must have $\lim_{N\rightarrow \infty} \Pr(S\in[0,\tau_{\Mmax(\mathbf{f_N})-1})) = 0$ and $\lim_{N\rightarrow \infty} \Pr(|M(S; \mathbf{f_N}) - \Mmax(\mathbf{f_N})|>1) = 0$ as desired.
\end{proof}



\subsection{Proof of Theorem \ref{thm:SW1}}


\begin{proof}
Let $(f_1^{Naive},\dots,  f_N^{Naive})$ be the strategy profile for the Naive solution, and let  $(f_1^{NE},\dots,  f_N^{NE})$ be a Nash equilibrium solution.
We will show $u(f_i^{NE}, f_{-i}^{NE}) > u(f_i^{Naive}, f_{-i}^{Naive})$ for every firm $i \in [N]$, which will imply $\SW_{\text{NE}} > \SW_\text{naive}$.
Let $s_c = \min \{s \in [0, 1]: f_1^{Naive}(s) = 1\}$ be the score threshold in which the naive strategy interviews everyone above $s_c$.

Fix a firm $i \in [N]$.
We will show the following two inequalities:
\begin{align} \label{eq:twoinequalities}
u(f_i^{NE}, f_{-i}^{NE}) 
\geq u(f_i^{Naive}, f_{-i}^{NE})
\geq u(f_i^{Naive}, f_{-i}^{Naive}).
\end{align}
The first inequality holds by definition of a Nash equilibrium.
% By definition of a Nash equilibrium, if firm $i$ changes their strategy to the naive strategy, their utility will not increase; i.e., $u(f_i^{NE}, f_{-i}^{NE}) \geq u(f_i^{Naive}, f_{-i}^{NE})$.
The second inequality holds since if firm $i$ chooses the naive strategy $f_i^{Naive}$, its utility is the lowest when all other firms choose the naive strategy.
% That is, $u(f_i^{Naive}, f_{-i}^{NE}) \geq u(f_i^{Naive}, f_{-i}^{Naive})$.
This is because for any score $s \geq s_c$, the total number of firms competing for applicant $s$ is $N$ under $(f_i^{Naive}, f_{-i}^{Naive})$, but may be smaller under $(f_i^{Naive}, f_{-i}^{NE})$.

We will show that at least one of the two inequalities in \eqref{eq:twoinequalities} is strict. Suppose, by contradiction, that both are equalities.
We will first show that $f_{-i}^{NE} = f_{-i}^{Naive}$.
Under $(f_i^{Naive}, f_{-i}^{Naive})$, every applicant that firm $i$ interviews is interviewed by $N$ firms in total.
For the equality $u(f_i^{Naive}, f_{-i}^{NE}) = u(f_i^{Naive}, f_{-i}^{Naive})$ to hold, it must be that under the strategy profile $(f_i^{Naive}, f_{-i}^{NE})$, every applicant that firm $i$ interviews is also interviewed by $N$ firms in total. If there was an applicant interviewed by strictly fewer firms, then it would be that $u(f_i^{Naive}, f_{-i}^{NE}) > u(f_i^{Naive}, f_{-i}^{Naive})$, since the utility function $U_n(s)$ are strictly monotonic in $n$.
Therefore, it must be that $f_{-i}^{NE} = f_{-i}^{Naive}$.

We will now show that firm $i$ can deviate from $f_i^{NE}$ to strictly improve their utility.
Since we assume $u(f_i^{NE}, f_{-i}^{NE}) 
= u(f_i^{Naive}, f_{-i}^{NE})$, and that $f_{-i}^{NE} = f_{-i}^{Naive}$, we will show that firm $i$ can strictly improve from $f_i^{Naive}$, when all other firms are at $f_{-i}^{Naive}$.
Let $s' < s_c$ be a score such that $U_{1}(s') > U_N(s_c)$, which exists since $U_{1}(s_c) > U_{N}(s_c)$ and $U_{1}(s)$ is a continuous and strictly increasing function in $s$.
Under $(f_i^{Naive}, f_{-i}^{Naive})$, no firm is interviewing scores in $[s', s_c]$.
Then, if firm $i$ interviews an applicant with score $s \in [s', s_c]$, their utility is strictly higher than $U_N(s_c)$.
Therefore, there exists an $\epsilon > 0$ where firm $i$ can deviate from the naive strategy by moving support from $[s_c, s_c+\epsilon]$ to within $[s', s_c]$, where they would earn strictly more utility.
This is a contradiction to $(f_i^{NE}, f_{-i}^{Ne})$ being a Nash equilibrium, and hence it must be that one of the two inequalities in \eqref{eq:twoinequalities} is strict.
Therefore, $\SW_{\text{NE}} > \SW_\text{naive}$.
\end{proof}




%\begin{proof}
    %Denote $n = \max F(S)$. Let A be an arbitrary firm among the $N$ firms, and let K be the support of $f_A(S)$. We  to show that the utility $U_A$ of the firm under Nash equilibrium is always larger than the utility $U_\text{naive}$ under the naive strategy. We know that $n\leq N$.
    
    %Let $U_\text{min} = \min_{s\in K} U(s, F(s))$. Note that $U(s,N)\leq U(s,F(s))$ for any s. If $U(1, N)\leq U_\text{min}$, then we have 

    %$$U_A \geq c U_\text{min}\geq c U(1,N)> U_\text{naive}$$

    %If $U(1, n)\geq U(1, N)> U_\text{min}$, then there is $t^*$ such that $U(t^*, N) = U_\text{min}$. If $n<N$, then on the interval $[t^*, 1]$ we have $U_\text{min} < U(s, N)\leq U(s, F(s)+1)$. Therefore, we must have $[t^*, 1]\subset K$ because otherwise, this cannot be a Nash equilibrium. If $n=N$, then $[\tau_N, 1]\subset K$ and we must have $t^*\leq \tau_N$. Then on the interval $[t^*, \tau_N]$, we have $U_\text{min} < U(s, N)\leq U(s, F(s)+1)$, and similarly, we must have $[t^*, \tau_N]\subset K$. Therefore, in both cases $[t^*, 1]\subset K$, and clearly, $x_c\leq t^*$.

    %\begin{align*}
       % U_A & =\int_{K\setminus [t^*, 1]} U(s, F(s)) \varphi(s)ds + \int_{[t^*, 1]} U(s, F(s)) \varphi(s)ds\\
      %  & \geq U_\text{min} \mu(K\setminus [t^*, 1]) + \int_{[t^*, 1]} U(s, N) \varphi(s)ds\\
       % &\geq U(t^*, N) \mu(K\setminus [t^*, 1]) + \int_{[t^*, 1]} U(s, N) \varphi(s)ds\\
        %& \geq U_\text{naive}
    %\end{align*}

    %Since for each firm, we have $U_A \geq U_\text{naive}$ (equality holds if and only if $t^* = x_c$), the social welfare 

    %$$W_\text{Equi} = \sum_{i=1}^N U_i \geq N U_\text{naive} = W_\text{naive}$$

    %If the equality holds, then $[x_c, 1]\subset K_i$ for each firm i, and $\tau_1 = \tau_N$ which contradicts $U(\tau_1,1)\leq U(\tau_N,N)$. Therefore, we must have

    %$$W_\text{Equi} > \sum_{i=1}^N U_i \geq N U_\text{naive} = W_\text{naive}$$
    
%\end{proof}



\subsection{Proof of Theorem \ref{thm:SW_PoNS2}}
We will first prove the following lemma before proving the theorem.

\begin{lemma} \label{lemma:4.2.1}
For any distribution $\cD$ and decision rule $\theta\in\{\corr, \indep\}$, let $\cI(c,N) = (N, c, \cD, \theta)$ be the instance parameterized by capacity $c$ and number of firms $N$, and $\mathbf{f_c}$ be a Nash equilibrium for instance $\cI(c,N)$. If $Nc<\int_{0.5}^1 \varphi(s)ds$, we must have $\Mmax(\mathbf{f_c}) = 1$ and $\Pr(S\in[\tau_1, 1]) = Nc$. The social welfare under $\mathbf{f_c}$ is 
    \begin{align*} 
        \SW_\text{NE}(\cI(c)) & =\int_{\tau_1}^{1}U_1(s)\varphi(s)ds. 
    \end{align*}
\end{lemma}
\begin{proof}
    Suppose $Nc<\int_{0.5}^1\varphi(s)ds$ and $\mathbf{f_c}$ is a Nash equilibrium. The lowest possible utility a firm can derive without double-interviewing any applicant is 0.5, which is higher than the highest possible utility a firm can derive by double-interviewing any applicant. As a result, we must have $\Mmax(\mathbf{f_c}) = 1$. Since $\mathbf{f_c}$ is a Nash equilibrium, by condition 1 and 2 of Theorem \ref{thm:equilibrium}, $M(s, \mathbf{f_c}) =1$ on $[\tau_1, 1]$ and $\Pr(S\in[\tau_1, 1]) = Nc$. Therefore, the social welfare under $\mathbf{f_c}$ is 
    \begin{align*} 
        \SW_\text{NE}(\cI(c)) & =\int_{\tau_1}^{1}U_1(s)\varphi(s)ds 
    \end{align*} as desired. 
    \end{proof}

\begin{proof} [\textbf{Proof of Proposition \ref{thm:SW_PoNS2}}]
    Let $\mathbf{f_c}$ be an Nash equilibrium of the instance $\cI(c)$. We will first show $\lim_{c \to 0^+} \PoNS(\cI(c)) = N$. If $N$ is fixed, as $c \to 0^+$, eventually we will have $c<\int_{0.5}^1\varphi(s)ds$. By lemma \ref{lemma:4.2.1}, $\lim_{c \to 0^+} \Mmax(\mathbf{f_c})  = 1$ and $\Pr(S\in[\tau_1, 1]) = Nc$. The social welfare under $\mathbf{f_c}$ is 
    \begin{align*} 
        \SW_\text{NE}(\cI(c)) & =\int_{\tau_1}^{1}U_1(s)\varphi(s)ds. 
    \end{align*}
    Let $s_c = \min \{s \in [0, 1]: f_1^{\text{naive}}(s) = 1\}$ be the score threshold in which the naive strategy interviews everyone above $s_c$. The social welfare under the naive solution is 
    \begin{align*}
        \SW_\text{naive}(\cI(c)) & =N \int_{s_c}^1 U_N(s)\varphi(s)ds. 
    \end{align*}
    
    We also notice that $\lim_{c \to 0^+}U_n(s) = \frac{1}{n}$. Therefore, when $c \to 0^+$ we have 
    \begin{align*} 
        \lim_{c \to 0^+}\PoNS(\cI(c))  = \lim_{c \to 0^+}\frac{\SW_\text{NE}(\cI(c))}{\SW_\text{naive}(\cI(c))} = \frac{\int_{\tau_1}^{1} \varphi(s)ds}{N\int_{s_c}^{1} \frac{1}{N} \varphi(s)ds} = \frac{\Pr(S\in[\tau_1, 1])}{\Pr(S\in[s_c, 1])}=\frac{Nc}{c} = N.
    \end{align*}

    Next, we will show $\lim_{c \to 1^-} \PoNS(\cI(c)) = 1$. As $c \to 1^-$, by condition 1 of Theorem \ref{thm:equilibrium}, $\lim_{c \to 1^-}\Pr(f_i(S)=1) = 1$. Therefore, each firm will interview all the applicants, and we must have $\tau_1 = \cdots = \tau_N = 0$. Hence,  $\lim_{c \to 1^-} \SW_\text{NE}(\cI(c)) = \int_{0}^{1} U_N(s) \varphi(s)ds$. Similarly, $s_c = 0$ and $\lim_{c \to 1^-} \SW_\text{naive}(\cI(c)) = \int_{0}^{1} U_N(s) \varphi(s)ds$. Therefore, we have
    \begin{align*}
        \lim_{c \to 1^-} \PoNS(\cI(c))  = \lim_{c \to 1^-}\frac{\SW_\text{NE}(\cI(c))}{\SW_\text{naive}(\cI(c))} =  \frac{\int_{0}^{1} U_N(s) \varphi(s)ds}{\int_{0}^{1} U_N(s) \varphi(s)ds} = 1.
    \end{align*}
\end{proof}

\subsection{Proof of Theorem \ref{thm:SW_PoNS4}}
\begin{proof}
    Let $\cI(N) = (N, c, \cD, \indep)$ be an instance with $N$ firms and the independent decision rule. Let $s_c = \min \{s \in [0, c]: f_1^{\text{naive}}(s) = 1\}$ be the score threshold in which the naive strategy interviews everyone above $s_c$. 
    \begin{align*}
        \PoNS(\cI(N)) &= \frac{\SW_{\text{NE}}(\cI(N))}{\SW_{\text{naive}}(\cI(N))} \\
        &= \frac{\int_{\tau_1}^{\tau_2} U_1(s)\varphi(s)ds + 2\int_{\tau_2}^{\tau_3}U_2(s) \varphi(s)ds +\cdots n\int_{\tau_n}^{1}U_{\Mmax(\bbf)}(s) \varphi(s)ds}{N\int_{s_c}^1 U_N(s)\varphi(s)ds}.
    \end{align*}
    By lemma \ref{lemma:3.2.2}, $\lim_{N\rightarrow \infty}\tau_{\Mmax(\bbf)-1}\rightarrow 0$ and $U_N(s)\rightarrow \frac{1}{N}$. Therefore  
    \begin{align*}
        \lim_{N\rightarrow \infty}\PoNS(\cI(N)) 
        &= \frac{(\Mmax(\bbf)-1)\int_{0}^{\tau_{\Mmax(\bbf)}}\frac{1}{\Mmax(\bbf)-1} \varphi(s)ds + \Mmax(\bbf)\int_{\tau_{\Mmax(\bbf)}}^{1}\frac{1}{\Mmax(\bbf)} \varphi(s)ds}{N\int_{s_c}^1 \frac{1}{N}\varphi(s)ds}\\
        & = \frac{\int_{0}^1 \varphi(s)ds}{\int_{s_c}^1 \varphi(s)ds}\\
        &= \frac{1}{c}.
    \end{align*}
    
    %\begin{align*}
        %\frac{W_{\text{Equi}}}{W_{\text{naive}}} &= \frac{\int_{\tau_1}^{\tau_2} U(s,1)\varphi(s)ds + 2\int_{\tau_2}^{\tau_3}U(s,2) \varphi(s)ds +\cdots n\int_{\tau_n}^{1}U(s,n) \varphi(s)ds}{N\int_{x_c}^1 U(s,N)\varphi(s)ds}\\
        %& \geq \frac{(n-1)\int_{\tau_{n-1}}^{\tau_n}U(s,n-1) \varphi(s)ds + %n\int_{\tau_n}^{1}U(s,n) \varphi(s)ds}{N\int_{x_c}^1 U(s,N)\varphi(s)ds}\\
    %\end{align*}

    %According to the previous lemma, as $N$ goes to infinity, $U(s,n)\rightarrow \frac{1}{n}$, and $\tau_{n-1}\rightarrow 0$.
    
    %$$\frac{W_{\text{Equi}}}{W_{\text{naive}}} \rightarrow \frac{(n-1)\int_{0}^{\tau_n}\frac{1}{n-1} \varphi(s)ds + n\int_{\tau_n}^{1}\frac{1}{n} \varphi(s)ds}{N\int_{x_c}^1 \frac{1}{N}\varphi(s)ds}
        %= \frac{\int_{0}^1 \varphi(s)ds}{\int_{x_c}^1 \varphi(s)ds}
        %= \frac{1}{c}$$
\end{proof}


\subsection{Proof of Theorem \ref{thm:SW_PoNS3}}
Before proving Theorem \ref{thm:SW_PoNS3}, we first present the following intermediate result. 

\begin{lemma} \label{lemma:4.4.1}
    For any distribution $\cD$ and decision rule $\theta = \{\corr, \indep\}$, let $\cI(N) = (N, c, \cD, \corr)$ is the instance with $N$ firms and $\mathbf{f_N}$ be the nash equilibrium of the instance. Let $\tau_1$ be the threshold defined in Theorem \ref{thm:equilibrium}, $\lim_{N\rightarrow \infty} \tau_1 = 0$.
\end{lemma}
\begin{proof} 
    By condition 3 of Theorem \ref{thm:equilibrium}, $U_1(\tau_1)\leq U_{\Mmax(\mathbf{f_N})}(\tau_{\Mmax(\mathbf{f_N})})$. Since $U_n(s)$ is strictly increasing in $s$, we have $U_{\Mmax(\mathbf{f_N})}(\tau_{\Mmax(\mathbf{f_N})})<U_{\Mmax(\mathbf{f_N})}(1)$. Given that $U_1(\tau_1) = \tau_1$ and $U_{\Mmax(\mathbf{f_N})}(1) = \frac{1}{\Mmax(\mathbf{f_N})}$, we have
    \begin{align*}
        \tau_1 \leq U_{\Mmax(\mathbf{f_N})}(\tau_{\Mmax(\mathbf{f_N})}) < \frac{1}{\Mmax(\mathbf{f_N})}.
    \end{align*}
    By lemma \ref{lemma:3.2.1}, $\lim_{N\rightarrow \infty} \frac{1}{\Mmax(\mathbf{f_N})} = 0$. As a result, $\lim_{N\rightarrow \infty} \tau_1 = 0$.
\end{proof}


\begin{proof} [\textbf{Proof of Theorem \ref{thm:SW_PoNS3}}]
    Let $\cI(N) = (N, c, \cD, \corr)$ be an instance with $N$ firms and the correlated decision rule. 

    We will first show that $\PoNS(\cI(N))$ increases when $N$ increases. Under the correlated decision rule, the utility function takes the form $U_n(s) = s/n$. The social welfare under the Nash equilibrium can be written as:
    \begin{align} 
        \SW_\text{NE}(\cI(N)) & =\int_{\tau_1}^{\tau_2}U_1(s)\varphi(s)ds + \cdots + \Mmax \int_{\tau_{\Mmax}}^{1}U_{\Mmax}(s)\varphi(s)ds \nonumber\\
        & = \int_{\tau_1}^{\tau_2} s \varphi(s)ds + \cdots + \Mmax \int_{\tau_{\Mmax}}^{1}\frac{s}{\Mmax}\varphi(s)ds \nonumber\\
        & =  \int_{\tau_1}^{\tau_2} s \varphi(s)ds + \cdots +  \int_{\tau_{\Mmax}}^{1}s\varphi(s)ds \nonumber\\
        & =  \int_{\tau_1}^{1} s \varphi(s)ds. \label{eq:SW_corr}
    \end{align}
    Let $s_c = \min \{s \in [0, c]: f_1^{\text{naive}}(s) = 1\}$ be the score threshold in which the naive strategy interviews everyone above $s_c$. The social welfare under the naive solution is 
    \begin{align} \label{sw_correlated_naive_formula}
        \SW_\text{naive}(\cI(N)) & =\int_{s_c}^1 U_N(s)\varphi(s)ds 
         =  N\int_{s_c}^{1} \frac{s}{N}\varphi(s)ds
         =  \int_{s_c}^{1} s \varphi(s)ds.
    \end{align}
    
    Therefore, we can express $\PoNS(\cI(N))$ as follows:
    \begin{align} \label{eq:PoNS}
        \PoNS(\cI(N))  = \frac{\SW_\text{NE}(\cI(N))}{\SW_\text{naive}(\cI(N))} = \frac{\int_{\tau_1}^{1} s \varphi(s)ds}{\int_{s_c}^{1} s \varphi(s)ds}.
    \end{align}

    By lemma \ref{lemma:4.4.1}, $\tau_1$ decreases as $N$ increases. Since $s\varphi(s)>0$,  $\int_{\tau_1}^{1} s \varphi(s)ds$ increases as $\tau_1$ decreases. Given that the denominator of  (\ref{eq:PoNS}) is fixed, we can conclude that $\PoNS(\cI(N)) $ increases as $N$ increases. 

    Next, we want show that in the limit when $N$ goes to infinity,  $\PoNS(\cI(N))$ converges to $\frac{\mathbb{E}S}{\int_{s_c}^1 s\varphi(s)ds}$. By lemma \ref{lemma:4.4.1}, $\lim_{N\rightarrow \infty} \tau_1 = 0$. Therefore, the numerator of the equation (\ref{eq:PoNS}) converges to $\int_{0}^{1} s \varphi(s)ds = \mathbb{E} S$. We conclude that 
    \begin{align}\label{equ:pons_IN}
        \lim_{N\rightarrow \infty}\PoNS(\cI(N)) = \frac{\mathbb{E}S}{\int_{s_c}^1 s\varphi(s)ds}.
    \end{align}

    
    Next, we will consider the special case where $\cD = \unif$. Under $\cD = \unif$, $\mathbb{E}S = 0.5$ and equation (\ref{equ:pons_IN}) can be written as 
    \begin{align*}
        \lim_{N\rightarrow \infty}\PoNS(\cI(N)) = \frac{\frac{1}{2}}{\int_{1-c}^1 sds} = \frac{1}{2c-c^2}. 
    \end{align*}
    Therefore, under the uniform distribution, $\lim_{N\rightarrow \infty}\PoNS(\cI(N)) = \frac{1}{2c-c^2} $. We only need to show that if $Nc\leq 0.5$, then $\PoNS(\cI(N)) = 1.5$. By lemma \ref{lemma:4.2.1}, when $Nc\leq \int_{0.5}^1 \varphi(s)ds = 0.5$, $\Mmax(\mathbf{f_N}) = 1$ and $\Pr(S\in [\tau_1, 1]) = Nc$. Since $\cD = \unif$, $\tau_1= 1-Nc$. By equation (\ref{eq:PoNS}), 
    \begin{align*}
        \PoNS(\cI(N))  &= \frac{1-\tau_1^2}{1-(1-c)^2} \\
        & = \frac{2Nc-N^2c^2}{2c-c^2}\\
        & = \frac{2N-N^2c}{2-c}\\
        &\geq \frac{2N-N/2}{2-c} \\
        &= \frac{3N}{2(2-c)} \\
        &\geq \frac{6}{2(2-c)} \\
        &\geq \frac{3}{2} = 1.5.
    \end{align*}
    Hence, under the uniform distribution, if $Nc\leq 0.5$, $ \PoNS(\cI(N))\geq 1.5$.
\end{proof}


\subsection{Proof of Theorem \ref{thm:PoA}}
We will first derive an expression for the highest possible social welfare before proving the theorem.
\begin{lemma} \label{lemma:sw_max}
For any distribution $\cD$,
    let $\cI(c, N) = (N, c, \cD, \corr)$ be the instance parameterized by the capacity $c$ and the number of firms $N$. Suppose $\Pr(S\in [s_{Nc},1])=Nc$, then
    \begin{align*}
    \SW_{\max}(\cI(c,N))=
    \begin{cases}
        \int_{s_{Nc}}^{1} s \varphi(s)ds, & \text{if } Nc\leq 1\\
        \E[S].  &o.w.
        \end{cases}
    \end{align*}
    
\end{lemma}
\begin{proof}
    Let $\cI(c, N) = (N, c, \cD, \theta)$ be the instance parameterized by the capacity $c$ and number of firms $N$. Let $s_{Nc}\in [0,1]$ such that $ \Pr(S\in [s_{Nc}, 1]) = Nc$. Our goal is to show that for any arbitrary strategy profile with the same capacity and number of firms, the social welfare is at most $\int_{s_{Nc}}^{1} s \varphi(s)ds$ when $Nc\leq1$, and $\E[S]$ when $Nc> 1$.
    
    Let's first consider the case when $Nc\leq 1$. We will first show that $\SW_{\text{max}}(\cI(c,N))\geq \int_{s_{Nc}}^1s\varphi(s)ds$. First, since  $\Pr(S\in [s_{Nc}, 1]) = Nc$ and $Nc\leq 1$, there exists $s_{Nc} = s_1<s_2\cdots <s_N<s_{N+1}=1$ such that $\Pr(S\in[s_i, s_{i+1}]) = c$. Let $\bbf$ be a strategy such that $f_i(s)=1$ if and only if $s\in [s_i, s_{i+1}]$ for $i\in [N]$. Under $\bbf$, we have $M(s, \bbf) = 1$ for $s\in[s_{Nc}, 1]$ and  $M(s, \bbf) = 0$ otherwise. Therefore, the social welfare is $ \int_{s_{Nc}}^1s\varphi(s)ds$. Hence we must have $\SW_{\text{max}}(\cI(c,N))\geq \int_{s_{Nc}}^1s\varphi(s)ds$.

    We will then show $\SW_{\text{max}}(\cI(c,N))\leq \int_{s_{Nc}}^1s\varphi(s)ds$. Let $\mathbf{f_\text{arb}}$ be an arbitrary strategy profile with $N$ firms each with capacity $c$. Define $S_n = \{s\in[0,1], M(s;\mathbf{f_\text{arb}}) = n\}$. The social welfare under $\mathbf{f_\text{arb}}$ is
    \begin{align}
        \SW_{\text{arb}}(\cI(c,N)) & = N\int_{S_N} U_N(s) \varphi(s)ds + (N-1)\int_{S_{N-1}} U_{N-1}(s) \varphi(s)ds +\cdots + \int_{S_1} s \varphi(s)ds \nonumber\\
        &= N\int_{S_N} \frac{s}{N} \varphi(s)ds + (N-1)\int_{S_{N-1}} \frac{s}{N-1} \varphi(s)ds +\cdots + \int_{S_1} s \varphi(s)ds \nonumber\\ 
        & = \int_{\cup S_i}s \varphi(s)ds. \label{equ:SWmax_corr}
    \end{align}
    Since $\Pr(S\in \cup S_i)\leq Nc$ and $s$ is strictly increasing, we must have $\SW_{\text{arb}}(\cI(c,N))\leq \int_{s_{Nc}}^1s\varphi(s)ds$. Therefore, $\SW_{\text{max}}(\cI(c,N))\leq \int_{s_{Nc}}^1s\varphi(s)ds$. We've shown that $\SW_{\text{max}}(\cI(c,N))\geq \int_{s_{Nc}}^1s\varphi(s)ds$ and $\SW_{\text{max}}(\cI(c,N))\leq \int_{s_{Nc}}^1s\varphi(s)ds$. As a result, $\SW_{\text{max}}(\cI(c,N))= \int_{s_{Nc}}^1s\varphi(s)ds$ when $Nc\leq 1$. 

    Next, consider the case when $Nc>1$. By equation (\ref{equ:SWmax_corr}), for any arbitrary strategy profile $\mathbf{f_\text{arb}}$ the social welfare is 
    \begin{align*}
        \SW_{\text{arb}}(\cI(c,N)) = \int_{\cup S_i}s \varphi(s)ds \leq \int_0^1s \varphi(s)ds = \E [S].
    \end{align*}
    Hence, $\SW_{\text{max}}(\cI(c,N))\leq \E [S]$. We will then show $\SW_{\text{max}}(\cI(c,N))\geq \E [S]$. Since $Nc>1$, there exists $0 = s_1\leq s_2\cdots \leq s_N \leq s_{N+1}=1$ with $p\leq N$ such that $\Pr(S\in[s_i, s_{i+1}]) \leq c$. Let $\bbf$ be a strategy such that $f_i(s) = 1$ if and only if $s\in [s_i, s_{i+1}]$ for $i\in [N]$. In this case, $\cup S_i = [0,1]$ and $\Pr(S\in \cup S_i) = 1$. Hence, by equation (\ref{equ:SWmax_corr}), the social welfare under $\bbf$ is $\int_0^1s \varphi(s)ds = \E [S]$. Hence, $\SW_{\text{max}}(\cI(c,N))\geq \E [S]$. As a result, we must have $\SW_{\text{max}}(\cI(c,N))= \E [S]$ when $Nc>1$.

\end{proof}

\begin{lemma} \label{lemma:sw_max_indep}
For any distribution $\cD$,
    let $\cI(c, N) = (N, c, \cD, \indep)$ be the instance parameterized by the capacity $c$ and the number of firms $N$. Suppose $\Pr(S\in [s_{Nc},1])=Nc$, then 
    \begin{align*}
        \lim_{c\rightarrow 0} \SW_{\max}(\cI(c,N)) = \int_{s_{Nc}}^{1} s \varphi(s)ds.
    \end{align*}
\end{lemma}
\begin{proof}
    Let $\mathbf{f_\text{max}}$ be the strategy profile for instance $\cI(c, N)$ that maximizes the social welfare. We will first show that $\Mmax(\mathbf{f_\text{max}}) =1$. Define $S_n = \{s\in[0,1], M(s;\mathbf{f_\text{max}}) = n\}$. We want to show that $S_i = \emptyset$ for all $i>1$. Suppose, by contradiction, $S_i \neq \emptyset$ for some $i>1$. We want to show that there exist some set $S'_1\notin \cup_i S_i$ such that $i\int_{S_i}\varphi(s)ds = \int_{S'_1}\varphi(s)ds$ and $i\int_{S_i}U_i(s)\varphi(s)ds < \int_{S'_1}s\varphi(s)ds$.  Under the utility constraint, we have $\sum_{i=1}^{\Mmax(\mathbf{f_\text{max}})} i\Pr(S\in S_i)\leq Nc$. Hence, $\Pr(S\in \cup_i S_i)\leq \sum_{i=1}^{\Mmax(\mathbf{f_\text{max}})} \Pr(S\in S_i)\leq \sum_{i=1}^{\Mmax(\mathbf{f_\text{max}})} i\Pr(S\in S_i)\leq Nc$. Since $c\rightarrow 0$, $Nc\rightarrow 0$ as well. When $Nc$ is small enough, we can always find some $S'_1 \subset (0.5, 1]\setminus \cup_i S_i$ such that $i\int_{S_i}\varphi(s)ds = \int_{S'_1}\varphi(s)ds$. Therefore, the total social welfare derived by interviewing applicants in $S_i$ is
    \begin{align*}
        i\int_{S_i}U_i(s)\varphi(s)ds \leq  i\int_{S_i}\frac{1}{i}\varphi(s)ds
        = \frac{1}{i} \int_{S'_1}\varphi(s)ds
        \leq \frac{1}{2} \int_{S'_1}\varphi(s)ds
        < \int_{S'_1}s\varphi(s)ds,
    \end{align*}
    which is strictly less than the social welfare that can be derived by interviewing applicants in $S'_1$. Therefore, for each $S_i \neq \emptyset$, we can always move the support from $S_i$ to some set $S'_1\notin \cup_i S_i$ to yield a higher utility. This contradicts the assumption that $\mathbf{f_\text{max}}$ maximizes the social welfare. Hence, $S_i=\emptyset$ for all $i>1$ and $\Mmax(\mathbf{f_\text{max}}) =1$. Since $\Pr(S\in S_1)\leq Nc$, $U_1(s) = s$ is strictly increasing and $\varphi(s)>0$, we have $\int_{S_1}s\varphi(s)ds \leq \int_{s_{Nc}}^1s\varphi(s)ds$ where $\Pr(S\in[s_{Nc},1]) = Nc$. Therefore, $\SW_{\max}(\cI(c,N)) = \int_{s_{Nc}}^{1} s \varphi(s)ds$ when $c$ is small enough. Hence, we conclude that $\lim_{c\rightarrow 0} \SW_{\max}(\cI(c,N)) = \int_{s_{Nc}}^{1} s \varphi(s)ds$.
    
\end{proof}


\begin{proof}[\textbf{Proof of Theorem \ref{thm:PoA}}]

We first show the first part of the statement is true. That is, for any $N$, we have $\lim_{c\rightarrow 0} \PoA(\cI(c, N)) =1$. Let $\cI(c,N)$ be an instance with $N$ fixed. Let $\mathbf{f_c}$ be a Nash equilibrium of the instance $\cI(c,N)$ with thresholds $\tau_1, \cdots, \tau_{\Mmax(\mathbf{f_c})}$ defined as in Theorem \ref{thm:equilibrium}. As $c\rightarrow 0$, we have $Nc\rightarrow 0$. Hence, we must have $Nc<\int_{0.5}^1 \varphi(s)ds$ for small enough $c$. By lemma \ref{lemma:4.2.1}, the social welfare under $\mathbf{f_c}$ is $\lim_{c\rightarrow 0}\SW_\text{NE}(\cI(c,N))  =\int_{\tau_1}^{1}s\varphi(s)ds$.
and $\Pr(S\in[\tau_1, 1]) = Nc$. Next we will show $\lim_{c\rightarrow 0} \SW_{\max}(\cI(c,N)) = \int_{\tau_1}^{1}s\varphi(s)ds$ for $\theta = \{\corr, \indep\}$. If $\theta = \corr$, since $Nc\rightarrow 0 $ and $\Pr(S\in[\tau_1, 1]) = Nc$, by lemma \ref{lemma:sw_max}, $\lim_{c\rightarrow 0}\SW_{\max}(\cI(c,N)) = \int_{\tau_1}^{1} s \varphi(s)ds$ as desired. If $\theta = \indep$, since $\Pr(S\in[\tau_1, 1]) = Nc$, by lemma \ref{lemma:sw_max_indep}, $\lim_{c\rightarrow 0}\SW_{\max}(\cI(c,N))  = \int_{\tau_1}^{1} s \varphi(s)ds$ as well. Hence, $\lim_{c\rightarrow 0} \SW_{\max}(\cI(c,N)) = \int_{\tau_1}^{1}s\varphi(s)ds$ for $\theta = \{\corr, \indep\}$. Therefore,
\begin{align*}
    \lim_{c\rightarrow 0} \PoA(\cI(c)) = \lim_{c\rightarrow 0}\frac{\SW_\text{NE}(\cI(c))}{\SW_{\max}(\cI(c))} = \frac{\int_{\tau_1}^{1}s\varphi(s)ds }{\int_{\tau_1}^{1} s \varphi(s)ds} = 1.
\end{align*}

Next, we will show for any $c \in (0, 1)$, we have $\lim_{N\rightarrow \infty} \PoA(\cI(c, N)) =1$. We will first prove the case when $\theta = \corr$. By equation (\ref{eq:SW_corr}), when hiring decisions are correlated, $\SW_\text{NE}(\cI(c,N))  = \int_{\tau_1}^{1} s \varphi(s)ds$. Moreover, by lemma \ref{lemma:4.4.1}, $\lim_{N\rightarrow \infty}\tau_1 = 0$. Therefore, we must have 
    \begin{align*}
        \lim_{N\rightarrow \infty}\SW_\text{NE}(\cI(c)) =\lim_{\tau_1\rightarrow 0} \int_{\tau_1}^{1} s \varphi(s)ds = \int_{0}^{1} s \varphi(s)ds = \E [S].
    \end{align*}
When $N$ is large enough, we must have $Nc>1$. By lemma \ref{lemma:sw_max}, $\SW_{\max}(\cI(c)) = \E [S]$. Therefore, 

    \begin{align*}
    \lim_{N\rightarrow \infty} \PoA(\cI(c,N)) = \lim_{N\rightarrow \infty}\frac{\SW_\text{NE}(\cI(c,N))}{\SW_{\max}(\cI(c,N))} = \frac{\E [S] }{\E [S]} = 1.
    \end{align*}
If $\theta = \indep$, by lemma \ref{lemma:3.2.2}, $\lim_{N\rightarrow \infty}\tau_{\Mmax(\mathbf{f_c})-1}\rightarrow 0$ and $U_N(s)\rightarrow \frac{1}{N}$. Therefore  
    \begin{align*}
        &\lim_{N\rightarrow \infty}\SW_\text{NE}(\cI(c,N))
        \\&=(\Mmax(\mathbf{f_c})-1)\int_{0}^{\tau_{\Mmax(\mathbf{f_c})}}\frac{1}{\Mmax(\mathbf{f_c})-1} \varphi(s)ds + \Mmax(\mathbf{f_c})\int_{\tau_{\Mmax(\mathbf{f_c})}}^{1}\frac{1}{\Mmax(\mathbf{f_c})} \varphi(s)ds\\
        & = \int_0^1 \varphi(s)ds = 1.
    \end{align*}
We will then show $\lim_{N\rightarrow \infty}\SW_{\max}(\cI(c, N))=1$ as well. We will first show $\SW_{\max}(\cI(c, N))\leq1$. Let $\mathbf{f_\text{arb}}$ be an arbitrary strategy profile and define $S_n = \{s\in[0,1], M(s;\mathbf{f_\text{arb}}) = n\}$ for $n\in [\Mmax(\mathbf{f_\text{arb}})]$. The social welfare under $\mathbf{f_\text{arb}}$ is 
    \begin{align}
        \SW_{\text{arb}}(\cI(c,N)) & = N\int_{S_N} U_N(s) \varphi(s)ds + (N-1)\int_{S_{N-1}} U_{N-1}(s) \varphi(s)ds +\cdots + \int_{S_1} s \varphi(s)ds \nonumber\\
        &\leq N\int_{S_N} \frac{1}{N} \varphi(s)ds + (N-1)\int_{S_{N-1}} \frac{1}{N-1} \varphi(s)ds +\cdots + \int_{S_1} \varphi(s)ds \nonumber\\
        & = \int_{\cup_{i=1}^N S_i} \varphi(s)ds \nonumber\\
        &= \Pr(S\in \cup_{i=1}^N S_i). \label{eq:SWmax_indep_ub}
    \end{align}
    By equation (\ref{eq:SWmax_indep_ub}), we have $\SW_{\text{arb}}(\cI(c,N))\leq \Pr(S\in \cup_{i=1}^N S_i)\leq 1$. Hence, $\SW_{\text{max}}(\cI(c,N))\leq 1$. As a result, $\lim_{N\rightarrow \infty}\SW_{\text{max}}(\cI(c,N))\leq 1$. Moreover, by definition, $\lim_{N\rightarrow \infty}\SW_{\text{max}}(\cI(c,N))\geq \lim_{N\rightarrow \infty}\SW_{\text{NE}}(\cI(c,N))=1$. Therefore, $\lim_{N\rightarrow \infty}\SW_{\text{max}}(\cI(c,N))\geq 1$. As a result, we conclude $\lim_{N\rightarrow \infty}\SW_{\text{max}}(\cI(c,N))=1$. Therefore,
    \begin{align*}
    \lim_{N\rightarrow \infty} \PoA(\cI(c,N)) = \lim_{N\rightarrow \infty}\frac{\SW_\text{NE}(\cI(c,N))}{\SW_{\max}(\cI(c,N))} = 1
    \end{align*} 
    for $\theta = \{\corr, \indep\}$.

\end{proof}

\subsection{Proof of Theorem \ref{thm:congestion}}

\begin{proof}
    To show that the game defined is a congestion game, we only need to find an exact potential function. Suppose there are $N$ firms in the market and the strategy profile is $\bbf$. Consider an arbitrary firm $i$ in the market. The change in the utility of the firm after the firm switches its strategy from $f_i$ to $f'_i$ is
    \begin{align*}
        u(f_i, f_{-i}) - u(f'_i, f_{-i}) = \int_{A_i\setminus A'_i} U_{M(s;\bbf)}(s)\varphi(s)ds - \int_{A'_i\setminus A_i} U_{M(s;\bbf)+1}(s)\varphi(s)ds,
    \end{align*}
    where $A_i = \{s\in[0,1], f_i(s)=1\}$ and $A'_i = \{s\in[0,1], f'_i(s)=1\}$.

    Now define the function $P(f_i, f_{-i})$ as
    $$ P(f_i, f_{-i}) = \int_0^1 \sum_{j=1}^{M(s;\bbf)} U_j(s)\varphi(s)ds.$$
    We will verify $P$ is an exact potential function. Let $\bbf'$ be the strategy profile with $(f'_i, f_{-i})$. Note that $M(s, \bbf')$ differs from $M(s, \bbf)$ only on sets $A_i\setminus A'_i$ and $A'_i\setminus A_i$. Therefore, we have

    \begin{align*}
        P(f_i, f_{-i})-P(f'_i, f_{-i}) & = \int_{A_i\setminus A'_i} \sum_{j=1}^{M(s, \bbf')} U_j(s)\varphi(s)ds + \int_{A'_i\setminus A_i} \sum_{j=1}^{M(s, \bbf')} U_j(s)\varphi(s)ds\\
        & - (\int_{A_i\setminus A'_i} \sum_{j=1}^{M(s, \bbf')-1} U_j(s)\varphi(s)ds + \int_{A'_i\setminus A_i} \sum_{j=1}^{M(s, \bbf')+1} U_j(s)\varphi(s)ds)\\
        & = \int_{A_i\setminus A'_i} U_{M(s;\bbf)}(s)\varphi(s)ds - \int_{A'_i\setminus A_i} U_{M(s;\bbf)+1}(s)\varphi(s)ds\\
        & = u(f_i, f_{-i}) - u(f'_i, f_{-i}).
    \end{align*}

    By definition, $P$ is an exact potential function. Since this game has an exact potential function, it is a congestion game. 
\end{proof}


\subsection{Proof of Theorem \ref{thm:convergence_corr}}
    We first prove some intermediate results before proving the main theorem. 
    \begin{lemma}\label{lemma:conv1}
        Suppose there is a $\delta > 0$ such that $\varphi(s) \geq \delta \; \forall s \in [0,1]$ and there are $N$ firms in the market that form an equal-utility Nash equilibrium $\mathbf{f_N}$ with $\theta = \corr$. Let $0=\tau_0\leq \cdots \leq\tau_{N+1}=1$ be the set of thresholds corresponds to $\mathbf{f_N}$. For a new firm $N+1$, as long as $c \leq 0.5 \delta$ for every firm $i$, there exists $0=\tau'_0\leq \cdots\leq \tau'_{N+2}=1$ such that $\tau_m < \tau'_{m+1}\leq \tau_{m+1}$ for each $i\in [\Mmax(\mathbf{f_N})]$, $\sum_{m=0}^N \int_{\tau'_{m+1}}^{\tau_{m+1}} \varphi(s)ds = c$ and $U_n(\tau'_n) = U_m(\tau'_m)$ for all $0<\tau'_m<\tau'_n<1$. Let $f_{N+1}(s) = 1$ if and only if $s\in\cup_{m=0}^{N} [\tau'_{m+1}, \tau_{m+1})$. Then $f_{N+1}$ is the unique best response for firm $N+1$ and the strategy profile forms an equal-utility nash equilibrium.  
        
    \end{lemma}

    \begin{proof}
        %To prove the statement, we want to show that the maximal capacity for $\tau_i < \tau'_{i+1}< \tau_{i+1};\quad \forall i\in [N]$ to hold is strictly .

        Let $f(t) = \sum_{m=0}^N \int_{(m+1)t}^{\tau_{m+1}} \varphi(s)ds$ for $t\in[0, \tau_1]$. We will show that for all $t\in[\frac{\tau_{\Mmax(\mathbf{f_N})}}{\Mmax(\mathbf{f_N})+1}, \tau_1)$, thresholds defined by 
        \begin{align}
        \tau'_{m} = 
        \begin{cases}\label{eq:def_thresholds}
            mt,&\text{if } mt\leq  1\\
            1, &o.w. 
        \end{cases}
        \end{align}
        satisfy $\tau_m < \tau'_{m+1}\leq \tau_{m+1}$ for each $m\in [\Mmax(\mathbf{f_N})]$. Let $m\in [\Mmax(\mathbf{f_N})]$ be arbitrary, we will first check the inequality when $t^* = \frac{\tau_{\Mmax(\mathbf{f_N})}}{\Mmax(\mathbf{f_N})+1} = \frac{\Mmax(\mathbf{f_N})\tau_1}{\Mmax(\mathbf{f_N})+1}$. If $\tau_{\Mmax(\mathbf{f_N})+1}<1$, then for any $m\in [\Mmax(\mathbf{f_N})]$
        \begin{align}\label{eq:check_inequality}
            \frac{\tau'_{m+1}}{\tau_m} & > \frac{(m+1)\frac{\Mmax(\mathbf{f_N})\tau_1}{\Mmax(\mathbf{f_N})+1}}{m\tau_1} \nonumber\\
            & = \frac{m+1}{\mathbf{\Mmax(\mathbf{f_N})}+1}\frac{\Mmax(\mathbf{f_N})}{m}\nonumber\\
            & =(1+\frac{1}{m})(1-\frac{1}{\Mmax(\mathbf{f_N})+1}) \nonumber\\
            & \geq (1+\frac{1}{\Mmax(\mathbf{f_N})})(1-\frac{1}{\Mmax(\mathbf{f_N})+1}) \nonumber\\
            & = \frac{\Mmax(\mathbf{f_N})+1}{\Mmax(\mathbf{f_N})}\frac{\Mmax(\mathbf{f_N})}{\Mmax(\mathbf{f_N})+1} = 1. 
        \end{align}
        If $\tau_{\Mmax(\mathbf{f_N})+1}=1$, then clearly, $\tau_{\Mmax(\mathbf{f_N})}<\tau'_{\Mmax(\mathbf{f_N})+1}\leq\tau_{\Mmax(\mathbf{f_N})+1}$. The equation (\ref{eq:check_inequality}) holds for $m<[\Mmax(\mathbf{f_N})]$. Therefore, the inequality holds for all $m$. We conclude that $\tau^*_{m+1}> \tau_m$ if $t^* = \frac{\tau_{\Mmax(\mathbf{f_N})}}{\Mmax(\mathbf{f_N})+1}$ for all $m\in [\Mmax(\mathbf{f_N})]$. For any $t'\in[t^*, \tau_1)$, if $\tau'_{m+1}<1$, then $\tau'_{m+1}=(m+1)t'>(m+1)t^* =\tau^*_{m+1}\geq \tau_m$. If $\tau'_{m+1}=1$, then clearly $\tau'_{m+1}> \tau_{m}$ since by definition $\tau_{m}<1$ for all $m\in [\Mmax(\mathbf{f_N})]$. Moreover, $mt\leq m\tau_1 = \tau_m$ for any $m\in [\Mmax(\mathbf{f_N})+1]$. Therefore, $\tau_m < \tau'_{m+1}\leq \tau_{m+1}$ for each $m\in [\Mmax(\mathbf{f_N})]$. 

        Next, we want to show $f(t)$ is continuous. Note that each integral  $\int_{(m+1)t}^{\tau_{m+1}} \varphi(s)ds$ is continuous in $t$, and as a result, we must have $f(t)$ is continuous as well. Moreover, by proposition \ref{prop:NE_shared}, if $\theta = \corr$, we must have $\tau_m = m\tau_1$ for each $m\in [\Mmax(\mathbf{f_N})]$. Hence, $\lim_{t\rightarrow \tau_1}f(t) = \sum_{m=0}^N \int_{\tau_{m+1}}^{\tau_{m+1}} \varphi(s)ds=0$. Furthermore, 
        \begin{align*}
            \lim_{t\rightarrow t^*}f(t) &= \sum_{m=0}^{\Mmax(\mathbf{f_N})} \int_{\tau'_{m+1}}^{\tau_{m+1}} \varphi(s)ds\\
            &\geq \sum_{m=0}^{\Mmax(\mathbf{f_N})} \int_{\tau'_{m+1}}^{\tau_{m+1}}\delta ds\\
            & = \delta \sum_{m=0}^{\Mmax(\mathbf{f_N})} (\tau_{m+1} - \tau'_{m+1})\\
            & = \delta (\sum_{m=0}^{\Mmax(\mathbf{f_N})} (\frac{m\tau_{\Mmax(\mathbf{f_N})}}{\Mmax(\mathbf{f_N})} - \frac{m\tau_{\Mmax(\mathbf{f_N})}}{\Mmax(\mathbf{f_N})+1}) + (1-\tau'_{\Mmax(\mathbf{f_N})+1}))\\
            & = \delta (\sum_{m=1}^{\Mmax(\mathbf{f_N})} \frac{m\tau_{\Mmax(\mathbf{f_N})}}{\Mmax(\mathbf{f_N})(\Mmax(\mathbf{f_N})+1)}+ (1-\tau'_{\Mmax(\mathbf{f_N})+1}))\\
            & = \delta (\frac{\Mmax(\mathbf{f_N})(\Mmax(\mathbf{f_N})+1)}{2}\frac{\tau_{\Mmax(\mathbf{f_N})}}{\Mmax(\mathbf{f_N})(\Mmax(\mathbf{f_N})+1)}+ (1-\tau'_{\Mmax(\mathbf{f_N})+1}))\\
            & = \delta(1 + \frac{\tau_{\Mmax(\mathbf{f_N})}}{2} - \tau'_{\Mmax(\mathbf{f_N})+1})\\
            & \geq \delta(1 + \frac{\tau_{\Mmax(\mathbf{f_N})}}{2} - \tau_{\Mmax(\mathbf{f_N})})\\
            & =\delta( 1-\frac{\tau_{\Mmax(\mathbf{f_N})}}{2}) > 0.5\delta.
        \end{align*}
        Since $0<c\leq 0.5\delta$, by intermediate value theorem, there exists $t_0\in[t^*, \tau_1)$ such that $f(t_0) = c$. In addition, the thresholds defined by (\ref{eq:def_thresholds}) satisfy the desired inequalities.

        Let $\mathbf{f_{N+1}} = (f_1, \cdots, f_N, f'_{N+1})$ be the strategy profile after firm $N+1$ makes the above response. We want to show that $f_{N+1}$ is the best response strategy by showing $\mathbf{f_{N+1}}$ is an equal-utility Nash equilibrium. Since $\tau_m < \tau'_{m+1}\leq \tau_{m+1}$ for each $m\in [\Mmax(\mathbf{f_N})]$, $\cup_{m=1}^{N+1} [\tau'_{m+1}, \tau_{m+1})$ is the union of disjoint sets. Hence, $\Pr(s\in[0,1], f_{N+1}=1)=\cup_{m=1}^{N+1} \Pr(S\in[\tau'_{m+1}, \tau_{m+1})) =c$. As a result, $\mathbf{f_{N+1}}$ satisfies the first condition of Theorem \ref{thm:equilibrium}. In addition, since $\tau_m < \tau'_{m+1}\leq \tau_{m+1}$ and the strategies $(f_1, \cdots, f_N)$ remain unchanged, we have $M(s, \mathbf{f_{N+1}}) = M(s, \mathbf{f_{N}})+1$ for $s\in \cup_{m=1}^{N+1} [\tau'_{m+1}, \tau_{m+1})$. Hence, $M(s, \mathbf{f_{N+1}}) = m$ for $s\in [\tau'_m, \tau'_{m+1})$ and the thresholds satisfy the second condition of Theorem \ref{thm:equilibrium}. Next, the thresholds defined by (\ref{eq:def_thresholds}) satisfy $U_m(\tau'_m) = U_n(\tau'_n)$ for all $0<\tau'_m<\tau'_n<1$. Therefore, $\mathbf{f_{N+1}}$ satisfies all the conditions of Theorem \ref{thm:equilibrium} and is an equal-utility Nash equilibrium. By definition of a Nash equilibrium, $f'_{N+1}$ is the best response strategy. We only need to show this strategy is unique. Let $K = \{s\in[0,1], f_{N+1}(s) = 1\}$. We claim that the utility firm $N+1$ can derive by interviewing an applicant with score $s\notin K$ is strictly less than the worst utility it can derive by interviewing an applicant with in $K$. Given that the thresholds satisfy $\tau_m < \tau'_{m+1}\leq \tau_{m+1}$ for each $m\in [\Mmax(\mathbf{f_N})]$ and $U_m(\tau'_m) = U_n(\tau'_n)$ for all $0<\tau'_m<\tau'_n<1$, $\min_{s\in K}U_{M(s, \mathbf{f_{N+1}})}(s) = t_0$. Since $U_n(s)$ is strictly increasing in $s$, for any applicant with $s\notin K$, the utility firm $N+1$ can derive by interviewing this applicant is $U_{M(s, \mathbf{f_{N+1}}+1)}(s)<t_0$. Hence, $U_{M(s, \mathbf{f_{N+1}}+1)}(s)<\min_{s\in K}U_{M(s, \mathbf{f_{N+1}})}(s)$. This implies that moving any support from $K$ to $K^c$ will lead to a strictly smaller utility. Therefore, $f_{N+1}$ yields a strictly higher utility than any other strategy. Hence, $f_{N+1}$ is the unique best response.
    \end{proof}


    \begin{proof} [\textbf{Proof of Theorem \ref{thm:convergence_corr}}]
        We will prove the theorem by induction. When $n=1$, the best response strategy is $f_1(s) = 1$ if and only if $s\in [s_c, 1]$ with $\Pr(S\in [s_c, 1]) = c$. Since $c<0.5\delta$, we have $\int_{0.5}^1\varphi(s)ds\geq \int_{0.5}^1\delta ds = 0.5\delta$. Hence, $c<\int_{0.5}^1\varphi(s)ds$. Let $\bbf_1$ denote the strategy profile. $\bbf_1$ satisfies the conditions established in Theorem \ref{thm:equilibrium}, and thus it is at an equal-utility Nash equilibrium. 

        Next, suppose the strategy profile $\bbf_k$ with $\Mmax(\bbf_k)=k$ is at an equal-utility Nash equilibrium and there are $n$ firms. By lemma \ref{lemma:conv1}, since the best response strategy is unique, the new strategy profile $\bbf'_k$ after firm $n+1$ makes the best response is still at an equal-utility Nash equilibrium. Therefore, the one-turn best response dynamics converges to an equal-utility Nash equilibrium. 
    \end{proof}




\subsection{Proof of Theorem \ref{thm:convergence_indep}}
To show the statement is true, we first present some intermediate results. 

\begin{lemma}\label{lemma:5.3.1}
    Suppose $\tau_1\in [0,1]$ and $\tau_m = 1-(1-m\tau_1)^{\frac{1}{m}}$ such that $\tau_m\leq 1$, then $\tau_m$ is strictly decreasing when $\tau_1$ decreases.  Fix an integer $n\geq 1$, $\tau_{n+1}-\tau_n$ is strictly decreasing when $\tau_1$ decreases. 
\end{lemma}
\begin{proof}
    Let $f(t)= 1-(1-mt)^{\frac{1}{m}}$ and $m$ be fixed. We will show that $f(t)$ decreases when $t$ decreases. When $t$ decreases,  $(1-mt)^{\frac{1}{m}}$ is strictly increasing and therefore $f(t)$ is strictly decreasing. Then the conclusion follows. Next, we fix an integer $n\geq 1$, and we want to show $\tau_{n+1}-\tau_n$ is strictly decreasing with $\tau_1$. Let $g(t) = \tau_{n+1}-\tau_n = (1-nt)^{\frac{1}{n}}-(1-(n+1)t)^{\frac{1}{n+1}}$. The derivative is $g'(t) = \frac{1}{(1-(n+1)t)^{1-\frac{1}{n+1}}} - \frac{1}{(1-nt)^{1-\frac{1}{n}}}$. Since $0<1-\frac{1}{n}<1-\frac{1}{n+1}<1$ and $1-nt>1-(n+1)t$, we must have $(1-(n+1)t)^{1-\frac{1}{n+1}}<(1-nt)^{1-\frac{1}{n}}$. Therefore, $g'(t) = \frac{1}{(1-(n+1)t)^{1-\frac{1}{n+1}}} - \frac{1}{(1-nt)^{1-\frac{1}{n}}} >0$. We conclude that $\tau_{n+1}-\tau_n$ is strictly decreasing with $\tau_1$.
\end{proof}

\begin{lemma}\label{lemma:conv2}
        Fix the total number of firms $N_0$. Suppose there is a $\delta > 0$ such that $\varphi(s) \geq \delta \; \forall s \in [0,1]$ and there are $N< N_0$ firms in the market that form an equal-utility Nash equilibrium $\mathbf{f_N}$ with $\theta = \indep$, $\tau_1>\frac{1}{\Mmax(\mathbf{f_N})+1}$. Let $0=\tau_0\leq \cdots \leq\tau_{N+1}=1$ be the set of thresholds corresponds to $\mathbf{f_N}$. Let $c_0 = \delta (\min_{m\in[N_0]} ((1-\frac{m}{N_0+1})^{\frac{1}{m}} -(1-\frac{m+1}{N_0+1})^{\frac{1}{m+1}}))$. For a new firm $N+1$, as long as $c \leq c_0$ for every firm $i$, there exists $0=\tau'_0\leq \cdots\leq \tau'_{N+2}=1$ such that $\tau_m < \tau'_{m+1}\leq \tau_{m+1}$ for each $m\in [\Mmax(\mathbf{f_N})]$, $\sum_{m=0}^N \int_{\tau'_{m+1}}^{\tau_{m+1}} \varphi(s)ds = c$ and $U_m(\tau'_m) = U_n(\tau'_n)$ for all $0<\tau'_m<\tau'_n<1$. Let $f_{N+1}(s) = 1$ if and only if $s\in\cup_{m=0}^{N} [\tau'_{m+1}, \tau_{m+1})$. $f_{N+1}$ is the unique best response for firm $N+1$, and the strategy profile $\mathbf{f'_N}$ defined by $\mathbf{f'_N} = (f_1, \cdots, f_N, f'_{N+1})$ forms an equal-utility nash equilibrium with $\tau'_1>\frac{1}{\Mmax(\mathbf{f'_N})+1}$. 
    \end{lemma}

    \begin{proof}
        By equation $(\ref{equ:Un(s)_indep})$, given $\tau_1$, under $\theta =\indep$, we have $\tau_m = 1-(1-\tau_1 m)^{1/m}$ for $m\leq \Mmax(\mathbf{f_N})$. Let $t^* = \max_{m\in [\Mmax(\mathbf{f_N})]}U_{m+1}(\tau_m)$. We will show that for all $t\in(t^*, \tau_1)$, thresholds defined by
        \begin{align}
        \tau'_{m} = 
        \begin{cases} \label{eq:thresholds_indep}
            1-(1-m t)^{\frac{1}{m}},&\text{if } 1-(1-m t)^{\frac{1}{m}}\leq  1\\
            1, &o.w. 
        \end{cases}
        \end{align}
        satisfy $\tau_m < \tau'_{m+1}\leq \tau_{m+1}$ for each $m\in [\Mmax(\mathbf{f_N})]$. First, we note that $U_{m+1}(\tau'_{m+1})=t$ under construction. Suppose, by contradiction, $\tau_m \geq \tau'_{m+1}$ for some $m$, then we have $U_{m+1}(\tau_m)\geq U_{m+1}(\tau'_{m+1})=t$. This contradicts the assumption that $t^*=\max_{m\in [\Mmax(\mathbf{f_N})]}U_{m+1}(\tau_m)<t$. Moreover, $\tau'_{m+1}\leq \tau_{m+1}$ since $t<\tau_1$ by lemma \ref{lemma:5.3.1}. Therefore, we conclude that $\tau_m < \tau'_{m+1}\leq \tau_{m+1}$ for each $m\in [\Mmax(\mathbf{f_N})]$.
        
       Next, let $f(t) = \sum_{m=0}^N \int_{1-(1-t m)^{1/m}}^{\tau_{m+1}} \varphi(s)ds$ for $t\in[0, \tau_1]$. We want to show that there exists $t$ such that $f(t) = c$. Since $\int_{1-(1-t m)^{1/m}}^{\tau_{m+1}} \varphi(s)ds$ is continuous for each $m$, $f(t)$ is continuous. We also notice that $\lim_{t\rightarrow \tau_1}f(t) = \sum_{m=0}^N \int_{\tau_{m+1}}^{\tau_{m+1}} \varphi(s)ds=0$. Define $c_0 = \delta (\min_{m\in[N_0]} ((1-\frac{m}{N_0+1})^{\frac{1}{m}} -(1-\frac{m+1}{N_0+1})^{\frac{1}{m+1}}))$. By lemma \ref{lemma:5.3.1}, for any $\tau_1>\frac{1}{N+1}$, we must have $\tau_{m+1}-\tau_{m}\geq (1-\frac{m}{N+1})^{\frac{1}{m}} -(1-\frac{m+1}{N+1})^{\frac{1}{m+1}}\geq \min_{m\in[N_0]} ((1-\frac{m}{N_0+1})^{\frac{1}{m}} -(1-\frac{m+1}{N_0+1})^{\frac{1}{m+1}})$. Since $\mathbf{f_N}$ is an equal-utility nash equilibrium, clearly $\tau_1>\frac{1}{N+1}$. Furthermore, we have
        \begin{align*}
            \lim_{t\rightarrow t^*}f(t) &= \sum_{m=0}^{\Mmax(\mathbf{f_N})} \int_{\tau'_{m+1}}^{\tau_{m+1}} \varphi(s)ds\\
            &\geq \sum_{m=0}^{\Mmax(\mathbf{f_N})} \int_{\tau'_{m+1}}^{\tau_{m+1}}\delta ds\\
            & = \delta \sum_{m=0}^{\Mmax(\mathbf{f_N})} (\tau_{m+1} - \tau'_{m+1})\\
            & > \delta \sum_{m=0}^{\Mmax(\mathbf{f_N})} (\tau_{m+1} - \tau_m)\\
            & > \delta (\max_{m\in [\Mmax(\mathbf{f_N})]} \tau_{m+1} - \tau_m)\\
            & > c_0.
        \end{align*}
        Since $c\leq c_0$, by intermediate value theorem, there exists $t_0\in[t^*, \tau_1)$ such that $f(t_0) = c$. In addition, the thresholds defined by $\tau_m = 1-(1-\tau_1 m)^{1/m}$ such that $\tau_m\leq 1$ satisfy the desired inequalities.

        Let $\mathbf{f_{N+1}} = (f_1, \cdots, f_N, f'_{N+1})$ be the strategy profile after firm $N+1$ makes the above response. We want to show that $f_{N+1}$ is the best response strategy by showing $\mathbf{f_{N+1}}$ is an equal-utility Nash equilibrium. Since $\tau_m < \tau'_{m+1}\leq \tau_{m+1}$ for each $m\in [\Mmax(\mathbf{f_N})]$, $\cup_{m=1}^{N+1} [\tau'_{m+1}, \tau_{m+1})$ is the union of disjoint sets. Hence, $\Pr(s\in[0,1], f_{N+1}=1)=\cup_{m=1}^{N+1} \Pr(S\in[\tau'_{m+1}, \tau_{m+1})) =c$. As a result, $\mathbf{f_{N+1}}$ satisfies the first condition of Theorem \ref{thm:equilibrium}. In addition, since $\tau_m < \tau'_{m+1}\leq \tau_{m+1}$ and the strategies $(f_1, \cdots, f_N)$ remain unchanged, we have $M(s, \mathbf{f_{N+1}}) = M(s, \mathbf{f_{N}})+1$ for $s\in \cup_{m=1}^{N+1} [\tau'_{m+1}, \tau_{m+1})$. Hence, $M(s, \mathbf{f_{N+1}}) = m$ for $s\in [\tau'_m, \tau'_{m+1})$ and the thresholds satisfy the second condition of Theorem \ref{thm:equilibrium}. Next, the thresholds defined by (\ref{eq:thresholds_indep}) satisfy $U_m(\tau'_m) = U_n(\tau'_n)$ for all $0<\tau'_m<\tau'_n<1$. Therefore, $\mathbf{f_{N+1}}$ satisfies all the conditions of Theorem \ref{thm:equilibrium} and is an equal-utility Nash equilibrium. Furthermore, $\tau'_1>\frac{1}{\Mmax(\mathbf{f'_N})+1}$ by construction. By definition of a Nash equilibrium, $f'_{N+1}$ is the best response strategy. We only need to show this strategy is unique. Let $K = \{s\in[0,1], f_{N+1}(s) = 1\}$. We claim that the utility firm $N+1$ can derive by interviewing an applicant with score $s\notin K$ is strictly less than the worst utility it can derive by interviewing an applicant with in $K$. Given that the thresholds satisfy $\tau_m < \tau'_{m+1}\leq \tau_{m+1}$ for each $m\in [\Mmax(\mathbf{f_N})]$ and $U_m(\tau'_m) = U_n(\tau'_n)$ for all $0<\tau'_m<\tau'_n<1$, $\min_{s\in K}U_{M(s, \mathbf{f_{N+1}})}(s) = t_0$. Since $U_n(s)$ is strictly increasing in $s$, for any applicant with $s\notin K$, the utility firm $N+1$ can derive by interviewing this applicant is $U_{M(s, \mathbf{f_{N+1}}+1)}(s)<t_0$. Hence, $U_{M(s, \mathbf{f_{N+1}}+1)}(s)<\min_{s\in K}U_{M(s, \mathbf{f_{N+1}})}(s)$. This implies that moving any support from $K$ to $K^c$ will lead to a strictly smaller utility. Therefore, $f_{N+1}$ yields a strictly higher utility than any other strategy. Hence, $f_{N+1}$ is the unique best response.
    \end{proof}
    
    \begin{proof} [\textbf{Proof of Theorem \ref{thm:convergence_indep}}]
        Fix the total number of firms $N$, let $c_0 = \min (0.5\delta,\delta (\min_{m\in[N]} ((1-\frac{m}{N+1})^{\frac{1}{m}} -(1-\frac{m+1}{N+1})^{\frac{1}{m+1}}))$. We will prove the theorem by induction. When $n=1$, the best response strategy is $f_1(s) = 1$ if and only if $s\in [s_c, 1]$ with $\Pr(S\in [s_c, 1]) = c$. Since $c<0.5\delta$, we have $\int_{0.5}^1\varphi(s)ds\geq \int_{0.5}^1\delta ds = 0.5\delta$. Hence, $c<\int_{0.5}^1\varphi(s)ds$ and $\tau_1=s_c>0.5$. Let $\bbf_1$ denote the strategy profile. $\bbf_1$ satisfies the conditions established in Theorem \ref{thm:equilibrium}, and thus it is at an equal-utility Nash equilibrium with $\tau_1>0.5$. 

        Next, suppose the strategy profile $\bbf_k$ with $\Mmax(\bbf_k)=k$ and n firms is at an equal-utility Nash equilibrium with $\tau_1>\frac{1}{\Mmax(\bbf_k)+1}$. By lemma \ref{lemma:conv2}, since the best response strategy is unique, the new strategy profile $\bbf'_k$ after firm $n+1$ makes the best response is still at an equal-utility Nash equilibrium. Therefore, the one-turn best response dynamics converges to an equal-utility Nash equilibrium. 
    \end{proof}




\subsection{Proof of Proposition \ref{prop:fixedW_corr}}
\begin{proof}
    We will start with the first part of the statement. We notice that the function $f(x) = N \int_x^1 U_N(s)\varphi(s)ds$ is continuous in $x$ and $f(0) =  N \int_0^1 U_N(s)\varphi(s)ds$, $f(1) = 0$. Moreover, since $s\varphi(s)>0$ for $s>0$, $f(x)$ is strictly decreasing in $x$. Therefore, given $W\in(0,N \int_0^1 U_N(s)\varphi(s)ds)$, there exist a unique $x^*\in(0,1)$ such that $f(x^*) = W$. Let $c = \Pr(S\in [x^*, 1])$. The social welfare under the naive solution is 

    \begin{align*}
        \SW_\text{naive} =N\int_{x^*}^1 U_N(s) \varphi(s) ds = W
    \end{align*}
    as desired. 

    Now suppose $W\leq \int_{0.5}^{1}s\varphi(s)ds$.  Let $\cI(N) = (N, c/N, \cD, \theta)$ be the instance parameterized by the number of firms $N$. The total capacity of the N firms is $c$, and there exists disjoint sets $K_1, \cdots, K_N$ such that $\cup K_i = [x^*, 1]$ and $\Pr(S\in K_i) = c/N$. Define the strategy profile $\mathbf{f_N} = (f_1, \cdots, f_N)$ such that $K_i = \{s\in[0,1], f_i(s) = 1\}$. We will show that $\mathbf{f_N}$ is a Nash equilibrium with social welfare $W$.  We will first show $x^*\geq 0.5$. Given that $\int_{x^*}^1 s \varphi(s) ds=W\leq \int_{0.5}^{1}s\varphi(s)ds$ and $f(x)$ is strictly decreasing in $x$. We must have $x^*\geq 0.5$. By construction, we have the thresholds $\tau_0 = 0,\tau_1 = x^*, \tau_2 =\cdots=\tau_{N+1}= 1$. Moreover, $M(s, \mathbf{f_N}) = 1$ for $s\in [\tau_1, 1]$. Hence, $\mathbf{f_N}$ satisfies all the conditions in Theorem \ref{thm:equilibrium}. As a result,  $\mathbf{f_N}$ is a Nash equilibrium. The social welfare under $\mathbf{f_N}$ is $\SW_\text{NE} =\int_{x^*}^1 s \varphi(s) ds = W$ as desired. 

    
    %the total capacity $c\leq \Pr(S\in [0.5, 1])$. Given that $\int_{x^*}^1 s \varphi(s) ds=W\leq \int_{0.5}^{1}s\varphi(s)ds$ and $f(x)$ is strictly decreasing in $x$. We must have $x^*\geq 0.5$ and, as a result, $c = \Pr(S\in [x^*, 1]) \leq \Pr(S\in [0.5, 1])$. By lemma \ref{lemma}, $\Mmax(\mathbf{f_N}) = 1$. 
    
    
    
    
    
    %the social welfare under the By equation \ref{eq:SW_corr}, we have $\SW_\text{NE} = \int_{\tau_1}^1 s \varphi(s) ds = W\leq \int_{0.5}^{1}s\varphi(s)ds$. Since $f(x)$ is strictly decreasing in $x$, we must have $\tau_1\geq 0.5$. That is, the lowest utility a firm can derive by interviewing an applicant under the current strategy is at least $0.5$. Hence, no firm would double-interview any applicant in this case because the highest possible utility a firm can derive by double-interviewing is $0.5$. Suppose each firm has capacity $c/N$, under the nash equilibrium, 

    
    %Let the social welfare $W<0.5$ be fixed. Since $\int_x^1 s\varphi(s)ds$ is continuous in $x$, there exist a unique $x\in(0,1)$ such that $$\SW_\text{naive} = \int_{x}^1 s \varphi(s) ds = W$$

    %Suppose there are $N$ firms in the market. The total capacity $C_N$ needed can be expressed as
    %$$C_N = N \int_x^1 \varphi(s)ds = cN$$ %where $c = \int_x^1\varphi(s)ds$ is a constant that only depends on $W$ and the distribution $D$.
\end{proof}
