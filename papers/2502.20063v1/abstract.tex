We study the impact of strategic behavior in a setting where firms compete to hire from a shared pool of applicants, and firms use a common algorithm to evaluate them. Each applicant is associated with a scalar score that is observed by all firms, provided by the algorithm. Firms simultaneously make interview decisions, where the number of interviews is capacity-constrained. Job offers are given to those who pass the interview, and an applicant who receives multiple offers accepts one of them uniformly at random. We fully characterize the set of Nash equilibria under this model.  Defining social welfare as the total number of applicants who find a job, we then compare the social welfare at a Nash equilibrium to a naive baseline where all firms interview applicants with the highest scores.  We show that the Nash equilibrium greatly improves upon social welfare compared to the naive baseline, especially when the interview capacity is small and the number of firms is large.  We also show that the price of anarchy is small, providing further appeal for the equilibrium solution.

We then study how the firms may converge to a Nash equilibrium. We show that when firms make interview decisions sequentially and each firm takes the best response action assuming they are the last to act, this process converges to an equilibrium  when interview capacities are small. However, we show that the task of computing the best response is difficult if firms have to use its own historical samples to estimate it, while this task becomes trivial if firms have information on  the degree of competition for each applicant.  Therefore, converging to an equilibrium can be greatly facilitated if firms have information on the level of competition for each applicant.