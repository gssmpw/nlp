\section{Related Work}
\textbf{Algorithmic monoculture and homogenization.}
The concept of algorithmic monoculture was formalized by ____, who showed that firms relying on a common algorithm may hire weaker applicants than when each firm uses an independent, but less individually accurate, hiring method.
____ incorporate two-sided preferences and competition to compare outcomes between monoculture and polyculture (when firms make independent decisions).
They leverage a two-sided matching model  and evaluate the stable matching outcome under monoculture and polyculture.
They show that monoculture can reduce firm utility compared to polyculture, but monoculture can improve average utility for the applicants.


Compared to the above works, the goal of this paper is not to evaluate the benefits or downsides of monoculture. Rather, we simply assume that monoculture exists, and then we evaluate how firms should make decisions in a setting with congestion effects.
We show that social welfare can drastically improve when firms make \textit{strategic} decisions based on the algorithm's output, compared to when they naively follow the algorithm's recommendation.

One consequence of  algorithmic monoculture is \textit{outcome homogenization}, the idea that certain individuals systematically experience undesirable outcomes by many algorithmic systems.  
There is growing line of work that study homogenization caused by algorithms ____, and the recent advances in generative AI has sparked studies on its impact on diversity of outcomes ____.
Our paper effectively studies outcome homogenization in the hiring context. 
Indeed, under the naive baseline where all firms interview the top-scoring candidates, every applicant receives the same outcome (interview decision) from every firm. 
We study whether this homogenization can be mitigated through strategic behavior. 




\textbf{Congestion in matching markets.}
A key difference of our model to that of ____ is that the latter uses \textit{stable matching} as the solution concept, without specifying the \textit{process} in which the stable matching arises.
There are also papers that study stable matching in a market where interviews are conducted to learn the utility of match  
 ____.
A stable matching can be found, for example, iteratively using the deferred acceptance algorithm ____. 
In contrast, our paper fixes a simple, one-step process in which firms hire applicants: firms decide simultaneously which applicants to interview and gives offers to everyone who passes the interview.
This process is motivated by the fact that firms face screening costs and hence have capacity constraints on the number of applicants they can interview. 
This process causes issues due to congestion; i.e., reduced utility when multiple firms interview the same applicant. 

Other existing papers have modeled congestion or search costs in matching models, such as  ____. They show that various restrictions on the matching process can improve outcomes, such as reducing the number of applicants that an individuals can send, or restricting which side can initiate a matching.
Compared to these works, the main difference of our paper is that we assume all firms have access to an informative signal for all applicants, while these papers assume that each agent has no a priori knowledge about other agents (i.e., no algorithmic recommendation).


Several empirical studies have evaluated various interventions to improve efficiency under congestion.
____ empirically show that signaling the level of competition can improve efficiency.
The benefit of this intervention is also established in our paper, and therefore our work is complementary to these empirical findings.
____ design a ranking algorithm to take congestion into account on an online platform to match volunteers to nonprofits.

Specific to application of hiring, ____ shows that leveraging algorithmic recommendations of applicants to firms substantially increase the fill rate, demonstrating the value of algorithmic hiring.
____ conduct a field experiment which showed that imposing a cap on the number of applications a job opening can receive can improve efficiency.
____ study a mechanism where applicants can send a signal of interest to firms, and this can improve applicant welfare.
____ studies congestion and incoordination in a social network where individuals can refer others for job opportunities.

Lastly, there is a large literature studying algorithmic hiring, on developing the algorithms themselves (e.g., ____), human perceptions of hiring algorithms (e.g., ____), as well as evaluating bias propagated or incited by such algorithms (e.g., ____).