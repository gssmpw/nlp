% CVPR 2025 Paper Template; see https://github.com/cvpr-org/author-kit

\documentclass[10pt,twocolumn,letterpaper]{article}

%%%%%%%%% PAPER TYPE  - PLEASE UPDATE FOR FINAL VERSION
% \usepackage{cvpr}              % To produce the CAMERA-READY version
% \usepackage[review]{cvpr}      % To produce the REVIEW version
\usepackage[pagenumbers]{cvpr} % To force page numbers, e.g. for an arXiv version

% Import additional packages in the preamble file, before hyperref
%
% --- inline annotations
%
\newcommand{\red}[1]{{\color{red}#1}}
\newcommand{\todo}[1]{{\color{red}#1}}
\newcommand{\TODO}[1]{\textbf{\color{red}[TODO: #1]}}
% --- disable by uncommenting  
% \renewcommand{\TODO}[1]{}
% \renewcommand{\todo}[1]{#1}



\newcommand{\VLM}{LVLM\xspace} 
\newcommand{\ours}{PeKit\xspace}
\newcommand{\yollava}{Yo’LLaVA\xspace}

\newcommand{\thisismy}{This-Is-My-Img\xspace}
\newcommand{\myparagraph}[1]{\noindent\textbf{#1}}
\newcommand{\vdoro}[1]{{\color[rgb]{0.4, 0.18, 0.78} {[V] #1}}}
% --- disable by uncommenting  
% \renewcommand{\TODO}[1]{}
% \renewcommand{\todo}[1]{#1}
\usepackage{slashbox}
% Vectors
\newcommand{\bB}{\mathcal{B}}
\newcommand{\bw}{\mathbf{w}}
\newcommand{\bs}{\mathbf{s}}
\newcommand{\bo}{\mathbf{o}}
\newcommand{\bn}{\mathbf{n}}
\newcommand{\bc}{\mathbf{c}}
\newcommand{\bp}{\mathbf{p}}
\newcommand{\bS}{\mathbf{S}}
\newcommand{\bk}{\mathbf{k}}
\newcommand{\bmu}{\boldsymbol{\mu}}
\newcommand{\bx}{\mathbf{x}}
\newcommand{\bg}{\mathbf{g}}
\newcommand{\be}{\mathbf{e}}
\newcommand{\bX}{\mathbf{X}}
\newcommand{\by}{\mathbf{y}}
\newcommand{\bv}{\mathbf{v}}
\newcommand{\bz}{\mathbf{z}}
\newcommand{\bq}{\mathbf{q}}
\newcommand{\bff}{\mathbf{f}}
\newcommand{\bu}{\mathbf{u}}
\newcommand{\bh}{\mathbf{h}}
\newcommand{\bb}{\mathbf{b}}

\newcommand{\rone}{\textcolor{green}{R1}}
\newcommand{\rtwo}{\textcolor{orange}{R2}}
\newcommand{\rthree}{\textcolor{red}{R3}}
\usepackage{amsmath}
%\usepackage{arydshln}
\DeclareMathOperator{\similarity}{sim}
\DeclareMathOperator{\AvgPool}{AvgPool}

\newcommand{\argmax}{\mathop{\mathrm{argmax}}}     



% It is strongly recommended to use hyperref, especially for the review version.
% hyperref with option pagebackref eases the reviewers' job.
% Please disable hyperref *only* if you encounter grave issues, 
% e.g. with the file validation for the camera-ready version.
%
% If you comment hyperref and then uncomment it, you should delete *.aux before re-running LaTeX.
% (Or just hit 'q' on the first LaTeX run, let it finish, and you should be clear).
\definecolor{cvprblue}{rgb}{0.21,0.49,0.74}
\usepackage[pagebackref,breaklinks,colorlinks,allcolors=cvprblue]{hyperref}

\usepackage{url} % get urls to play nice in bib

%typesetting tricks from Noah
\usepackage{microtype}
\usepackage{float}
\frenchspacing

\usepackage{graphicx}
\usepackage{booktabs}% fancy tables
% \newcommand{\ra}[1]{\renewcommand{\arraystretch}{#1}} % More space between rows of tables
\usepackage{multirow} % vertical text in tables
\usepackage{subcaption} % subtables
\usepackage{color, colortbl,xcolor}
\definecolor{Gray}{gray}{0.9}
\definecolor{light-gray}{gray}{0.95}
\newcommand\sbullet[1][.5]{\mathbin{\vcenter{\hbox{\scalebox{#1}{$\bullet$}}}}}


\usepackage[mathscr]{euscript}
\usepackage{hhline} % Import hhline package


%%%%%%%%% PAPER ID  - PLEASE UPDATE
\def\paperID{16862} % *** Enter the Paper ID here
\def\confName{CVPR}
\def\confYear{2025}

%%%%%%%%% TITLE - PLEASE UPDATE
\title{Poly-Autoregressive Prediction for Modeling Interactions}

%%%%%%%%% AUTHORS - PLEASE UPDATE
% \author{Neerja Thakkar\\
% UC Berkeley\\
% % Institution1 address\\
% % {\tt\small firstauthor@i1.org}
% % For a paper whose authors are all at the same institution,
% % omit the following lines up until the closing ``}''.
% % Additional authors and addresses can be added with ``\and'',
% % just like the second author.
% % To save space, use either the email address or home page, not both
% \and
% Tara Sadjadpour\\
% UC Berkeley\\
% % First line of institution2 address\\
% % {\tt\small secondauthor@i2.org}
% \and
% Jathushan Rajasegeran\\
% UC Berkeley\\
% \and
% Shiry Ginosar\\
% TTIC, Google DeepMind\\
% \and
% Jitendra Malik\\
% UC Berkeley\\
% }


% \author{Neerja Thakkar\\
% \and
% Tara Sadjadpour\\
% \and
% Jathushan Rajasegeran\\
% \and
% Shiry Ginosar\\
% \and
% Jitendra Malik\\
% }


\author{%
  \begin{tabular}[t]{@{}ccccc@{}}
    Neerja Thakkar$^{1}$ & Tara Sadjadpour$^{1}$ & Jathushan Rajasegeran$^{1}$ & Shiry Ginosar$^{2,3}$ & Jitendra Malik$^{1}$
  \end{tabular}\\[0.5em]
  { $^{1}$UC Berkeley, $^{2}$Toyota Technical Institute at Chicago, $^{3}$Google DeepMind}
}


\begin{document}
% \maketitle

% \twocolumn[{%
% \renewcommand\twocolumn[1][]{#1}%
% \maketitle
% \begin{center}
%     \vspace{-0.26in}
%     \begin{subfigure}[b]{0.49\textwidth}
%         \includegraphics[width=\textwidth]{fig/auto.pdf}
%         \caption{Autoregression (AR)}
%         \label{fig:auto}
%     \end{subfigure}
%     ~
%     \begin{subfigure}[b]{0.49\textwidth}
%         \includegraphics[width=\textwidth]{fig/polyauto.pdf}
%         \caption{PAR: same-agent, next-timestep training.}
%         \label{fig:polyauto}
%     \end{subfigure} 
%    \captionof{figure}{Inference for (a) autoregressive (AR) models and (b) our poly-autoregressive (PAR) model. Solid indicates ground-truth tokens; striped predicted. Colors denotes agent identity. Compared to AR models, the PAR model predicts a new token at every time step, but takes other agent's tokens as inputs. 
%    }
% \label{fig:PAR_inference}
% \end{center}%
% }]
\makeatletter
\g@addto@macro\@maketitle{
    \begin{figure}[H]
    \begin{minipage}{\textwidth}
    \includegraphics[width=0.99\textwidth]{fig/PAR_figure.pdf}
    \centering
    % \vspace{-50pt}
    \captionof{figure}{
        Inference for (a) autoregressive (AR) models and (b) our proposed poly-autoregressive (PAR) model. Solid indicates ground-truth tokens which represent a tracked data modality such as action or 6DOF pose; striped represents predicted output tokens. Color denotes agent identity. Compared to AR models, the PAR model takes other agents' tokens as inputs when making a prediction for the next timestep.
    }\label{fig:polyauto}%
    \vspace{0.2cm}
    \end{minipage}
    \end{figure}
}
\makeatother

\maketitle
\renewcommand\thefigure{\arabic{figure}}
\setcounter{figure}{1}

\begin{abstract}


The choice of representation for geographic location significantly impacts the accuracy of models for a broad range of geospatial tasks, including fine-grained species classification, population density estimation, and biome classification. Recent works like SatCLIP and GeoCLIP learn such representations by contrastively aligning geolocation with co-located images. While these methods work exceptionally well, in this paper, we posit that the current training strategies fail to fully capture the important visual features. We provide an information theoretic perspective on why the resulting embeddings from these methods discard crucial visual information that is important for many downstream tasks. To solve this problem, we propose a novel retrieval-augmented strategy called RANGE. We build our method on the intuition that the visual features of a location can be estimated by combining the visual features from multiple similar-looking locations. We evaluate our method across a wide variety of tasks. Our results show that RANGE outperforms the existing state-of-the-art models with significant margins in most tasks. We show gains of up to 13.1\% on classification tasks and 0.145 $R^2$ on regression tasks. All our code and models will be made available at: \href{https://github.com/mvrl/RANGE}{https://github.com/mvrl/RANGE}.

\end{abstract}

  
\vspace{-12pt}
\section{Introduction}
\begin{figure}[t]
  \centering
   \includegraphics[width=1\linewidth]{sec/figs/fig1.png}
   \caption{\captionMethodFigure}
   \label{fig:methodFigure}
\end{figure}
\begin{figure}[t]
  \centering
   \includegraphics[width=1\linewidth]{sec/figs/exampleFigure.png}
   \caption{\captionExampleFigure}
   \label{fig:exampleFigure}
   \vspace{-12pt}
\end{figure}
Vision-Language Models (VLMs), such as CLIP~\cite{CLIP}, have emerged as general-purpose systems for understanding visual data through language-based queries. These models enable a broad range of applications, from object detection to image captioning, by linking visual inputs to language prompts. In standard settings where images contain single, recognizable objects, VLMs perform remarkably well. However, for the more complex task of zero-shot multi-label recognition (MLR) (Fig. \ref{fig:methodFigure} (top)), where models must identify multiple objects within an image without prior training on specific data, VLMs face significant limitations. Zero-shot MLR is crucial for applications in fields like robotics and medical imaging, where objects rarely appear in configurations that align neatly with training distributions. In these scenarios, achieving robust multi-label recognition without fine-tuning is challenging, given the task’s reliance on mean Average Precision (mAP) scores, which depend on ranking images for object presence.

\noindent \textbf{VLM: Prompt Dependent AND/OR Noisy Channel.} Despite the promise of zero-shot capabilities, current VLM approaches often struggle with MLR due to inherent scoring behaviors and biases. The performance of these models is hampered by a mix of conjunction (AND) and disjunction (OR) behaviors in their scoring, leading to inflated scores in compound prompts that contain multiple objects. For example, a prompt like “cat and sofa” might yield a high score even if only one of these objects is present in the image. This tendency reflects biases learned during training, where common object pairs receive higher scores even when only one object is present, disrupting the accuracy of mAP-based evaluations. Furthermore, existing methods for adapting VLMs to zero-shot MLR frequently rely on prompt tuning or architectural adjustments—approaches that are often dependent on training data and computationally intensive fine-tuning, which limit their generalizability to novel tasks.

\noindent \textbf{Our Approach.} In contrast to these methods, we introduce SPARC (Score Prompting and Adaptive Fusion for Zero-Shot Multi-Label Recognition in VLMs), a novel approach to zero-shot MLR that bypasses the need for training data, prompt tuning, or model-specific modifications. Our method treats the VLM as a black box, relying solely on its output scores to infer object presence (see Fig. \ref{fig:methodFigure}). This black-box approach enables us to avoid assumptions about the model’s internal workings, allowing for a purely zero-shot framework that is both model-agnostic and dataset-independent. SPARC introduces two main innovations that address the unique challenges of zero-shot MLR.

\noindent \textbf{A. Compound Prompt Composition:} Recognizing that VLMs can provide richer information when prompted with combinations of objects, we develop a method for constructing compound prompts. These prompts reflect likely contextual associations between objects, such as “cat and sofa” or “car and bus.” By gathering scores from these compound prompts, we can capture a spectrum of potential object contexts within the image, enhancing detection without relying on training-based adaptations. This composition strategy allows us to agnostically extract information from the VLM, leveraging probable object relationships without depending on any specific dataset or VLM architecture.\\
\noindent \textbf{B. Score Debiasing and Adaptive Fusion.} A critical insight in our approach lies in the surprising observation that the \underline{maximum score among compound prompts} is often a poor proxy for true object presence. Although one might expect the highest score to serve as a reliable signal, we find that it frequently reflects compositional biases, as VLMs tend to respond to compound prompts with OR-like behavior, raising scores even when only one object in the prompt is present. Instead, we observe that the second-highest score consistently provides a more accurate indicator of object presence, minimizing the effects of false positives caused by compositional bias. Building on this insight, we develop a debiasing algorithm that normalizes scores across images to address image-specific noise and clarify genuine object presence signals. This debiased score set is then processed through a PCA-based fusion method that further refines object rankings by combining information from both compound and singleton prompts, ultimately optimizing mAP by enhancing score accuracy.\\
\textbf{Complementarity.} SPARC is complementary to other zero-shot and training-free MLR methods. When applied on top of these approaches, SPARC consistently enhances mAP scores by refining object ranking and reducing bias in VLM outputs. This capability makes SPARC an adaptable solution that can improve upon existing methods while maintaining a fully zero-shot, model-agnostic framework.

\noindent \textbf{Empirical Results.} SPARC achieves significant improvements in mAP, outperforming methods that incorporate architectural modifications. This outcome shows the potential of a fully zero-shot approach that relies only on systematic prompt design and score interpretation, rather than prompt-training or fine-tuning. By revealing that the second-highest score can be a superior proxy to the maximum, our findings provide new insights into VLM scoring behavior, suggesting that careful treatment of prompt compositions and score patterns can unlock robust MLR capabilities.


\section{Related Work}

\subsection{Large 3D Reconstruction Models}
Recently, generalized feed-forward models for 3D reconstruction from sparse input views have garnered considerable attention due to their applicability in heavily under-constrained scenarios. The Large Reconstruction Model (LRM)~\cite{hong2023lrm} uses a transformer-based encoder-decoder pipeline to infer a NeRF reconstruction from just a single image. Newer iterations have shifted the focus towards generating 3D Gaussian representations from four input images~\cite{tang2025lgm, xu2024grm, zhang2025gslrm, charatan2024pixelsplat, chen2025mvsplat, liu2025mvsgaussian}, showing remarkable novel view synthesis results. The paradigm of transformer-based sparse 3D reconstruction has also successfully been applied to lifting monocular videos to 4D~\cite{ren2024l4gm}. \\
Yet, none of the existing works in the domain have studied the use-case of inferring \textit{animatable} 3D representations from sparse input images, which is the focus of our work. To this end, we build on top of the Large Gaussian Reconstruction Model (GRM)~\cite{xu2024grm}.

\subsection{3D-aware Portrait Animation}
A different line of work focuses on animating portraits in a 3D-aware manner.
MegaPortraits~\cite{drobyshev2022megaportraits} builds a 3D Volume given a source and driving image, and renders the animated source actor via orthographic projection with subsequent 2D neural rendering.
3D morphable models (3DMMs)~\cite{blanz19993dmm} are extensively used to obtain more interpretable control over the portrait animation. For example, StyleRig~\cite{tewari2020stylerig} demonstrates how a 3DMM can be used to control the data generated from a pre-trained StyleGAN~\cite{karras2019stylegan} network. ROME~\cite{khakhulin2022rome} predicts vertex offsets and texture of a FLAME~\cite{li2017flame} mesh from the input image.
A TriPlane representation is inferred and animated via FLAME~\cite{li2017flame} in multiple methods like Portrait4D~\cite{deng2024portrait4d}, Portrait4D-v2~\cite{deng2024portrait4dv2}, and GPAvatar~\cite{chu2024gpavatar}.
Others, such as VOODOO 3D~\cite{tran2024voodoo3d} and VOODOO XP~\cite{tran2024voodooxp}, learn their own expression encoder to drive the source person in a more detailed manner. \\
All of the aforementioned methods require nothing more than a single image of a person to animate it. This allows them to train on large monocular video datasets to infer a very generic motion prior that even translates to paintings or cartoon characters. However, due to their task formulation, these methods mostly focus on image synthesis from a frontal camera, often trading 3D consistency for better image quality by using 2D screen-space neural renderers. In contrast, our work aims to produce a truthful and complete 3D avatar representation from the input images that can be viewed from any angle.  

\subsection{Photo-realistic 3D Face Models}
The increasing availability of large-scale multi-view face datasets~\cite{kirschstein2023nersemble, ava256, pan2024renderme360, yang2020facescape} has enabled building photo-realistic 3D face models that learn a detailed prior over both geometry and appearance of human faces. HeadNeRF~\cite{hong2022headnerf} conditions a Neural Radiance Field (NeRF)~\cite{mildenhall2021nerf} on identity, expression, albedo, and illumination codes. VRMM~\cite{yang2024vrmm} builds a high-quality and relightable 3D face model using volumetric primitives~\cite{lombardi2021mvp}. One2Avatar~\cite{yu2024one2avatar} extends a 3DMM by anchoring a radiance field to its surface. More recently, GPHM~\cite{xu2025gphm} and HeadGAP~\cite{zheng2024headgap} have adopted 3D Gaussians to build a photo-realistic 3D face model. \\
Photo-realistic 3D face models learn a powerful prior over human facial appearance and geometry, which can be fitted to a single or multiple images of a person, effectively inferring a 3D head avatar. However, the fitting procedure itself is non-trivial and often requires expensive test-time optimization, impeding casual use-cases on consumer-grade devices. While this limitation may be circumvented by learning a generalized encoder that maps images into the 3D face model's latent space, another fundamental limitation remains. Even with more multi-view face datasets being published, the number of available training subjects rarely exceeds the thousands, making it hard to truly learn the full distibution of human facial appearance. Instead, our approach avoids generalizing over the identity axis by conditioning on some images of a person, and only generalizes over the expression axis for which plenty of data is available. 

A similar motivation has inspired recent work on codec avatars where a generalized network infers an animatable 3D representation given a registered mesh of a person~\cite{cao2022authentic, li2024uravatar}.
The resulting avatars exhibit excellent quality at the cost of several minutes of video capture per subject and expensive test-time optimization.
For example, URAvatar~\cite{li2024uravatar} finetunes their network on the given video recording for 3 hours on 8 A100 GPUs, making inference on consumer-grade devices impossible. In contrast, our approach directly regresses the final 3D head avatar from just four input images without the need for expensive test-time fine-tuning.



\begin{figure*}
\centering
    \includegraphics[width=.99\textwidth]{fig/par_method.pdf}  
    \caption{\textbf{The PAR Framework}. We begin by collecting a video dataset, such as AVA (top) or DexYCB (bottom). Then, using dataset labels or computer vision techniques, a trajectory of a given modality for our prediction task is extracted for each agent, such as action class labels (top) or object pose and 3D hand translation (bottom). Data is then tokenized, either through discretization or directly using continuous values, with our framework supporting both formats. Based on the tokenization and prediction task, we choose the appropriate loss function for PAR training. After training with PAR, predicted tokens can be decoded back to data space and evaluated with relevant metrics.}
    \label{fig:PAR_pipeline}
\vspace{-1em}
\end{figure*}

\section{Poly-Autoregressive Modeling}

Our goal is to model the behavior of an agent or entity while taking into account any other agents it interacts  with, if any. To evaluate the performance of our model in capturing interaction dynamics, we predict the agent's future behavior and compare it against ground-truth data. 

We define the following task: \textit{In an interaction comprised of $N$ agents, given the observed past states of the $N-1$ interacting agents, and the observed or previously-predicted past states of the $N^{\text{th}}$ ego agent, predict the future states of the $N^{\text{th}}$  ego agent.}

We define a transformer-based poly-autoregressive (PAR) predictor, $\mathcal{P}$, that learns to model temporally long-range interactions in the input sequence. The inputs to the predictor are the past states of the $N$ interacting agents, and its output is the predicted future state of the $N^{\text{th}}$ ego agent.

\subsection{Problem Definition}
\label{sec:prob_def}

Let $\mathbf{S}=\{\mathbf{s}_i\}_{i=1}^T$ be a temporal sequence of agent states, $\mathbf{s}_i$.
We use $\mathbf{S}^N$ and $\mathbf{S}^{1:N-1}$ to denote the temporal sequences of states of the $N_{th}$ agent and of the other $N-1$ agents, respectively.
For each timestep $t \in [t_\pi,T]$, where $t_\pi \in [1,T]$ is the time we start predicting,
we take as input all other $N-1$ agents' past observed state sequences
$\mathbf{S}^{1:N-1}_{1:t-1}$
along with the $N_{th}$ agent's
past observed states up to $t_\pi$, 
$\mathbf{{S}}^N_{1:t_\pi}$,
and any of its previously predicted past states $\mathbf{\hat{S}}^N_{t_{\pi}+1:t-1}$,
if available (see Fig.~\ref{fig:polyauto}).
Our predictor, $\mathcal{P}$, then \textit{poly-autoregressively} predicts the $N_{th}$ agent's future states one time-step at a time:
\begin{equation}
       \mathbf{\hat{s}}^N_{t} = \mathcal{P}(\mathbf{S}^{1:N-1}_{1:t-1}, \mathbf{{S}}^N_{1:t_{\pi}},\mathbf{\hat{S}}^N_{t_\pi+1:t-1}). \\
\end{equation}
$\mathcal{P}$ learns to model the distribution over the next timestep of the $N_{th}$ agent's states, given all other agents' states:
\begin{equation}
p(\mathbf{\hat{s}}^N_{t} | \mathbf{S}^{1:N-1}_{1:t-1}, \mathbf{S}^N_{1:t-1}).
\end{equation}

While we provide the observed ground truth states of other agents at inference, during training, we jointly maximize the likelihood of all $N$ agents by computing losses on their future state predictions.

We train the predictor by maximizing the likelihood of the target state $y$ at time $t$:
\begin{equation*}
\label{eq:transformerloss}
    \mathscr{L_\mathcal{P}} = E_{y \sim p(y)}[-\log(p(\mathbf{s}^N_{t})],
\end{equation*}
where the target state $y$ at $t$ is computed from the $N_{th}$ agent ground truth future state.

\subsection{The Poly-Autoregressive Framework}
\label{sec:framework}

\begin{figure}
    \centering
    % \begin{subfigure}[b]{0.49\textwidth}
    %     \includegraphics[width=\textwidth]{fig/nexttoken_teacherforcing.pdf}
    %     \caption{AR: next-token training.}
    %     \label{fig:AR_nexttoken}
    % \end{subfigure}
    \begin{subfigure}[b]{0.49\textwidth}
        \includegraphics[width=\textwidth]{fig/mutli-agent_next-token.pdf}
        \caption{AR: multi-agent, next-token training.}
        \label{fig:AR_multiagent_nexttoken}
    \end{subfigure}
    ~
    \begin{subfigure}[b]{0.49\textwidth}
        \includegraphics[width=\textwidth]{fig/next_timestep.pdf}
        \caption{PAR: same-agent, next-timestep training.}
        \label{fig:PAR_nexttimestep}
    \end{subfigure}    
    \caption{Training with teacher forcing for (a) multi-agent next-token prediction in autoregressive models and (b) multi-agent poly-autoregressive models. Solid vs striped indicates a ground-truth vs predicted token, respectively. Color denotes agent identity. The AR model is trained for next-token prediction, while the PAR model is trained to predict the next timestep of the same agent. Three agents are shown for ease of visualization, but the PAR model supports an arbitrary number of agents.}
    \label{fig:PAR_training}
\vspace{-.75cm}
\end{figure}

We address the problem of forecasting the future states of an agent (from time $t$ to $T$) in a data-driven way, given a temporal sequence of past states (from time $1$ to $t-1$). 
We assume that our agent has some feature, or a set of features, of interest in a video (e.g., 3D pose) that we can tokenize. We predict the future states of the agent in terms of this tokenized feature (or set of), where we use one token (or set of tokens) per time step. The predicted tokens can be discrete (i.e., an index into a feature codebook) or continuous (i.e., a vector of one or more continuous values). The loss $\ell$ will depend on the problem's specifics and the type of token used. To train the model to predict the future, we rely on all the interaction dynamics of length $T$ in our training dataset as ground truth examples.


As a baseline, we consider the \textbf{single-agent autoregressive (AR)} paradigm, where a transformer is trained to perform next-token-prediction with teacher forcing. AR uses greedy sampling to generate sequences at inference time, predicting one next token at a time (Fig.~\ref{fig:polyauto}(a)). 

In contrast, our \textbf{multi-agent poly-autoregressive (PAR)} framework considers the other $N-1$ agents in the scene when predicting the future state of the $Nth$ agent. In this setup, we tokenize the features of interest of all $N$ agents, yielding $N$ tokens at each timestep for a total of $N*T$ tokens. In practice, we operate on a flattened sequence of $N*T$ tokens. 
% In cases where we consider a multimodal set of tokens of interest, we have $T*N*K$ where $K$ is the number of token types. \neerja{The only case where we have multiple token types is hand-object, but there we treat the hand as one agent and the object as another (or equivalently: 1 agent and 2 token types), but regardless $N*K=2$, so we haven't shown $N>1$ and $K>1$} \shiry{ok then we keep this for future ;-) }
Instead of using the AR training procedure in this multi-agent case (as in Fig.~\ref{fig:AR_multiagent_nexttoken}), we jointly model the $N$ agents at each timestep by introducing the following features to our PAR framework.

\vspace{0.2cm}
\noindent \textbf{Next-timestep prediction.} 
A standard AR model predicts the next token. Given the flattened sequence of $N*T$ tokens our model operates on, next token prediction would take as input an agent $k$ at timestep $t$ and predict agent $k+1$'s state at the same timestep $t$ (as in Fig.~\ref{fig:AR_multiagent_nexttoken}). However, our goal is to predict the input agent $k$'s future state at time $t+1$. Therefore, we perform \textit{same-agent next-timestep} prediction rather than next-token prediction (see Fig.~\ref{fig:PAR_nexttimestep} for an illustration of same-agent next-timestep at training).

\medskip \noindent \textbf{Learned agent identity embedding.} When giving a model information corresponding to multiple agents, the model can benefit from knowing which token corresponds to which agent. We give the model this information with a learned agent ID embedding. 
 


 % \medskip \noindent \textbf{Multimodality} \neerja{We have multimodality in the sense of hand and object are different modalities, and we add location to accl/velocity tokens in a pos emb - we should mention that we go beyond just one single type of token per case study, but we also do things in different ways for different case studies so not sure what is the best way to mention this here}


\medskip \noindent \textbf{Joint training.} We train the model to jointly predict the future of all agents by computing a loss on the predicted tokens of all agents (Fig.~\ref{fig:PAR_nexttimestep}). Please refer to Section~\ref{sec:prob_def} for our inference paradigm.

\subsection{Task-Specific Considerations}
\label{sec:task-specific}
Our simple PAR approach unifies diverse problems under a single framework and architecture without any modifications. In order to formulate a problem as interaction-conditioned prediction, users must consider several task-specific details. Fig.~\ref{fig:PAR_pipeline} gives an overview of how the PAR framework disentangles multi-agent learning from problem-specific modeling.

\medskip \noindent \textbf{Data.} The input data source in our example tasks is always a collection of videos. From these videos, we extract various modalities relevant to the task at hand. These modalities can range from high-level features, such as action class labels, to low-level ones, such as 3D pose (Fig.~\ref{fig:PAR_pipeline} first two columns). We assume that each agent in the dataset is detected at each frame and is associated with an agent ID.
 
\medskip \noindent \textbf{Tokenization.} Our framework supports both discrete, quantized tokens and continuous vector tokens. The choice between discrete and continuous depends on the nature of the task.  
In the case of discrete tokens, we use a standard embedding layer to project to the hidden dimension. For continuous tokens, we train a projection layer to project the token into the hidden dimension of the transformer. For instance, if our continuous token is a 3D vector with an $(x,y,z)$ 3D location coordinate and our hidden dimension is $128$, our projection layer will project from $3$ to $128$ dimensions. We also train an un-projection layer that reverts the hidden dimension to the original token dimension.

\medskip \noindent \textbf{Loss.} The type of token and task-specific considerations dictate the loss function $\ell$ applied during model training. For discrete tokens, a classification loss is appropriate. For continuous tokens, we use a regression loss on the original token dimension. 
% \shiry{there is a nuance here that we did not yet have time to explore given the last-minute nature of how this came together: the discrete classification loss allows us to predict a multinomial distribution over possible future timesteps, from which we can sample. This makes the prediction non-deterministic, capturing the true variability of the response of the $N_th$ agent to their surroundings. After all, given other people's actions, our actions are still not determined 100 percent. This is the whole beauty of using autoregressive prediction for these problems, unlike the feed-forward UNets we used in e.g. speech to getsture. This is not at all explored yet in our experiments due to lack of time mostly, and is a major hole in this submission (unlike our previous works on this direction). If a reviewer is familiar with the literature, they will complain about this (I know I would). There is nothing for us to do about this now, but this is something we should be aware of.} \neerja{Adding sampling is high on my priority list!} \shiry{yes, I would reject you without ;-) }

\medskip \noindent \textbf{Baselines.}
We compare to the following baselines, where applicable on a case-by-case basis:

\noindent $\sbullet$ \textit{Random token}: pick random tokens from the best available token space and use as the prediction. 

\noindent $\sbullet$ \textit{Random trajectory}: pick at random a trajectory from the training dataset to use as the prediction. 

\noindent $\sbullet$ \textit{NN}: Given an input agent $A$'s trajectory history, find the closest trajectory to it in the training set, belonging to $A^T$. Use $A^T$'s future as the predicted future.

\noindent $\sbullet$ \textit{Multiagent NN}: In a dataset with two interacting partners $A$ and $B$, where $B$ is the ego agent, given an input agent $A$'s trajectory history, find the closest trajectory to it in the training set, belonging to $A^T$. Use $A^T$'s interaction partner's $B^T$'s future as the prediction.

% Assume two agents, $A$ and $B$ in interaction, where we are predicting agent $B$. Take the history of agent $A^V$ in the validation set and find the closest $A^T$ in the training dataset. Use the future of the corresponding interacting agent $B_T$ in the training trajectory as the predicted future for agent $B_V$.

\smallskip
\noindent $\sbullet$ \textit{Mirror}: In a dataset with two interacting partners $A$ and $B$, use the ground truth future of agent $B$ as the predicted future for agent $A$.



\subsection{Framework Implementation Details}
We keep the following implementation details constant for all case studies (see also Sec.~\ref{sec:appendix_impl_details}).

\medskip
\noindent \textbf{Learned agent ID embedding.} Our learned agent ID embedding consists of the integer agent ID mapped to a hidden dim-sized vector, and summed to the token embedding.

\medskip \noindent \textbf{Architecture.} For all case studies, we use the  Llama~\citep{touvron2023llamaopenefficientfoundation} transformer decoder architecture with $8$ layers, $8$ attention heads, and a hidden and intermediate dimension of $128$. The decoder has $\sim$4.4M learned parameters, not including learned embedding layers which add a few thousand more parameters. A rotary positional encoding~\citep{su2024roformer} is used in addition to other summed encodings (i.e. agent ID embedding, locational positional encoding in Sec.~\ref{sec:car_trajs}). We train using teacher forcing. The only hyperparameter that changes between case studies is the learning rate.

\section{Case Study 1: Social Action Forecasting}

\begin{figure*}[t]
    \centering
    \includegraphics[width=.9\textwidth]{fig/ava_qual.pdf}
    \caption{\textbf{Action forecasting example.} The distribution over ground truth actions are in white, and our predictions in red. A 6s action history (1Hz) is input, and 6s of future actions predicted. In the scene, the man and woman alternate between talking and listening. Initially, the man is talking. The AR model predicts he will continue talking, while the 2-agent PAR model recognizes the woman is talking and predicts more accurate turn-taking behavior.}
    \label{fig:ava_qual}
    \vspace{-.6cm}
\end{figure*}


\begin{figure}
    \centering
    \includegraphics[width=\linewidth]{fig/specified_classes_comparison.pdf}
    \small
    \caption{\textbf{Per-class mAP for AVA 2-person actions}. We see performance improvement on almost all 2-person AVA action classes ((P) stands for ``a person"). Some absolute mAP gains are particularly significant: \textit{listen to} $+7.0$, \textit{kiss} $+8.3$, \textit{fight/hit} $+5.7$, \textit{talk to} $+4.4$, \textit{hug} $+5.7$, and
    \textit{hand shake} $+4.0$.}
\label{fig:ava_2_person_classes}
\vspace{-.5cm}
% \end{wrapfigure}
\end{figure}

Our first case study involves forecasting human actions. Human behaviors are fundamentally social; for instance, individuals frequently walk in groups and alternate between speaking and listening roles when conversing. Certain actions, like hugging or handshaking, are intrinsically multi-person. Therefore, modeling human interactions should help improve action forecasting performance, especially on multi-person actions, which we show in this case study.


\subsection{Experimental Setup}
\noindent \textbf{Dataset.}  The Atomic Visual Actions (AVA) dataset~\citep{gu2018ava} comprises 235 training and 64 15-minute validation videos from movies. Annotations are provided at a 1Hz frequency, detailing bounding boxes and tracks for individuals within the frame, and each person's actions within a 1-second timeframe. Individuals may engage in multiple concurrent actions from a repertoire of 60 distinct action classes (e.g., sitting and talking simultaneously). For our analysis, we select clips featuring a continuous sequence of an agent's actions spanning at least $4$s, splitting sequences exceeding $12$s.  We use the first half of each clip as history to predict the second half. For any ego agent trajectory, we pick a second agent by selecting the person present in the scene for the longest subset of the ego agent's trajectory.

% \medskip \noindent \textbf{Tokenization.}
\medskip \noindent \textbf{Task-specific considerations.} Each agent's token $\mathcal{A}$ represents an 60-dimensional vector that corresponds to the actions performed at a specific timestep. Each element denotes the probability of a particular action class being enacted; ground-truth inputs are a binary vector. We implement an embedding layer that projects these tokens into the transformer's hidden dimension, as well as an un-projection layer that reverts them back to the original 60D token space for the purposes of loss calculation and output generation. We do not explicitly require the outputs to be values between 0 and 1.
% \medskip \noindent \textbf{Loss.} 
We use a MSE regression loss on the 60D action tokens: $\mathscr{L} = \frac{1}{n} \sum_{i=1}^{n} (\mathcal{A}_i - \hat{\mathcal{A}}_i)^2$.
% \medskip \noindent \textbf{Metrics.} 
Our evaluation metric is the
% We measure 
mean average precision (mAP) on the 60 AVA classes.

We implement all baselines described in \ref{sec:task-specific}, where \textit{Random Token} corresponds to a random 60D vector sampled from 0 to 1. \textit{NN} and \textit{Multiagent NN} use Hamming distance as the distance metric.
\begin{table}
% 
\centering
\begin{tabular}{@{}ccccc@{}}
\toprule
Method & Timestep pred & Ag-ID embd & mAP $\uparrow$           \\
\midrule
1-agent AR   & N/A         & N/A        & 40.7        \\
2-agent AR   & \xmark      & \xmark     & 38.0      \\
2-agent PAR*  & \xmark     & \cmark     & 40.2         \\
2-agent PAR*   & \cmark     & \xmark   & 40.0         \\
2-agent PAR &\cmark  & \cmark & \textbf{42.6}     \\
\bottomrule
\end{tabular}
\caption{\textbf{PAR action forecasting performance on AVA} We evaluate 1 and 2-agent AR methods, two 2-agent PAR ablations (rows 3 and 4, PAR*), and our PAR method. Without next-timestep prediction (see Fig.~\ref{fig:PAR_training}) or a learned agent ID embedding, our model struggles with multi-agent reasoning, performing worse than the AR baseline. With both components, the 2-agent PAR model achieves a +1.9 mAP gain over the AR method (see Fig.~\ref{fig:ava_1_person_classes} and Fig.~\ref{fig:ava_2_person_classes} for class breakdown).}
\label{tab:ava_PAR_ablation}
\vspace{-.25cm}
\end{table}
% neerja: experiment names and wandb links
% 1-agent	https://wandb.ai/nthakkar/PAR_ava_action_prediction_post_ICLR/runs/s9sa5w5o?nw=nwusernthakkar	action_pred_ava_1_agent_lr_5e-5_ema_decay_0.999_seed_1_val_full_val
% 2-agent AR	https://wandb.ai/nthakkar/PAR_ava_action_prediction_post_ICLR/runs/nh9gvjlu?nw=nwusernthakkar	action_pred_ava_2_agent_no_ag_id_emb_no_same_ag_loss_lr_5e-5_ema_decay_0.999_seed_1_val_full_val
% 2-agent no timestep pred	https://wandb.ai/nthakkar/PAR_ava_action_prediction_post_ICLR/runs/xzlllmta	action_pred_ava_2_agent_no_same_ag_loss_lr_5e-5_ema_decay_0.999_seed_1_val_full_val_vis
% 2-agent no ag id emb	https://wandb.ai/nthakkar/PAR_ava_action_prediction_post_ICLR/runs/3uzemj2m?nw=nwusernthakkar	action_pred_ava_2_agent_no_ag_id_emb_lr_5e-5_ema_decay_0.999_seed_1_val_full_val_vis
% 2-agent PAR	https://wandb.ai/nthakkar/PAR_ava_action_prediction_post_ICLR/runs/u5yoe6vs?nw=nwusernthakkar	actual_final_ava_1hz_action_2_agent_ag_id_emb_full_val_val_full_val
% https://wandb.ai/nthakkar/PAR_ava_action_prediction/runs/ramhpxe1?nw=nwusernthakkar	
\begin{table}
% 
\centering
\begin{tabular}{@{}lcc@{}}
\toprule
Baseline & Agents &  mAP $\uparrow$           \\
\midrule
Random Token         & 1 & 3.46          \\
Random Training Traj  & 1   & 3.44          \\
Nearest Neighbor    & 1        & 13.17 \\
Multiagent NN &2      & 5.10      \\
Mirror   & 2 & 7.97         \\
\bottomrule
\end{tabular}
\caption{\textbf{AVA baselines} While the nearest neighbor baseline performs best among baselines, it is still significantly worse than the AR model.}
\label{tab:ava_baselines}
\vspace{-.5cm}
\end{table}

\subsection{Results}

We report the performance of a single-agent AR model as a baseline, in the first line of Table~\ref{tab:ava_PAR_ablation}. The AR model is significantly better than our baselines (see Table~\ref{tab:ava_baselines}), the strongest baseline being the single-agent NN. We compare these baselines to our 2-agent PAR model (last line) and various ablations where we remove the agent ID embedding and perform next-token rather than same-agent next-timestep prediction. The second line of the table corresponds to multi-agent next-token prediction (Fig.~\ref{fig:AR_multiagent_nexttoken}). We see that this approach confuses the model, and the performance is significantly worse than just training on and considering a single agent. However, as we add various components of our PAR approach, the performance improves, and with both the next timestep prediction and agent ID embedding, we get a $+1.9$ mAP gain. When only considering 2-person action classes (enumerated in Fig.~\ref{fig:ava_2_person_classes}), our mAP is  $36.3$ on the single agent PAR model and $39.8$ on the 2-agent PAR model, a \textbf{$+3.5$} mAP gain.

In Fig.~\ref{fig:ava_qual} we see an example of action forecasting. In the input history, the man talks and the woman listens. In the future, the woman talks, and the man listens. Our 2-agent PAR model (bottom row) better understands that talking and listening actions are complementary actions, while the AR model doesn't learn this correlation. We see quantitative evidence of this in Fig.~\ref{fig:ava_2_person_classes}, with per-class mAPs for our AR vs 2-agent PAR model for 2-person action classes. Here, \textit{talk to} gets a $+4.4$ mAP gain and \textit{listen to} gets a $+7.0$ mAP gain when we train a multi-agent model. We see a significant boost on many other interaction-related action classes---for instance, \textit{kiss a person} $+8.3$ and \textit{fight/hit a person} $+5.7$ mAP---and on single-person actions, see Fig.~\ref{fig:ava_1_person_classes}.










\section{Case Study 2: Multiagent Car Trajectory Prediction}
\label{sec:car_trajs}
Our second case study focuses on predicting car trajectories. Trajectory prediction requires a vehicle to be aware of other cars on the road to avoid collisions and promote cooperative behavior. This study demonstrates how our framework enables the joint modeling of multiple vehicles' movements.

\subsection{Experimental Setup}
\noindent \textbf{Dataset.} We use nuScenes~\citep{nuscenes} , inputting 2 seconds of positions to forecast vehicle positions 6 seconds ahead. Specifically, our objective is to predict the $xy$ coordinates of each agent, exclusively considering vehicles as agents. We use the \texttt{trajdata} interface~\citep{ivanovic2023trajdata} to load and visualize the data.

\noindent \textbf{Task-specific considerations.}   Instead of discretizing the $xy$ position space, we discretize the motion, resulting in discrete velocity or acceleration tokens. These integer tokens are projected to the transformer hidden dimension using the Llama token embedding layer. Inputting only these tokens results in our PAR model knowing what speed the other agents are going at, but not where they are. It is important the model has this awareness (it should know if two agents are going to collide), so our model needs to reason over this second modality of location. We implement this by passing locations relative to the agent we are predicting into a sin-cosine positional embedding (see details in Sec.~\ref{sec:appendix_car_impl_details}), which we denote a location positional encoding (LPE). The LPE is summed to our token embeddings.

We use a cross-entropy classification loss on our discrete tokens:
$\mathscr{L} = E_{y \sim p(y)}[-\log(p(\mathbf{s}^{T}_{t_{\pi}})].$  We use the standard average displacement error (ADE) and final displacement error (FDE) to evaluate our predicted trajectories. For our baselines (Sec.~\ref{sec:task-specific}), we use the closest agent at the current timestep for \textit{Multiagent NN} and \textit{Mirror}. For \textit{NN} and \textit{Multiagent NN} we use MSE as the distance metric.


\subsection{Results}
\begin{table}
% 
\centering
\begin{tabular}{@{}lcccc@{}}
\toprule
Token type & LPE & Method &  ADE $\downarrow$ & FDE $\downarrow$           \\
\midrule
Velocity  & \xmark & 1-agent AR & 1.50 & 3.64 \\
Velocity & \xmark & 3-agent PAR & 1.45 & 3.51 \\
Accleration  & \xmark   & 1-agent AR & 1.44    & 3.57  \\
Accleration  & \xmark & 3-agent PAR  & 1.40    & 3.44    \\
Accleration  & \cmark& 3-agent PAR   & \textbf{1.35} & \textbf{3.34} \\
\bottomrule
\end{tabular}
\caption{\textbf{Car trajectory prediction performance.} Using acceleration tokens and 3-agent PAR results in a stronger performance over velocity tokens and single-agent AR. Adding location via a positional encoding (LPE) further improves results. }
\label{tab:cars_locs}

% Neerja: experiment names/wandb links
% 1-agent velocity	https://wandb.ai/nthakkar/ICLR_paper_cars/runs/ccg7y19d?nw=nwusernthakkar	final_car_traj_1_agent_vel_tok_a100_full_val
% 3-agent PAR velocity	https://wandb.ai/nthakkar/ICLR_paper_cars/runs/sgi43f24	final_car_traj_3_agent_vel_tok_a100_full_val
% 1-agent acc	https://wandb.ai/nthakkar/PAR_car_traj_prediction_nuscenes/runs/k5na3q37?nw=nwusernthakkar	final_car_traj_1_agent_accl_tok_a100_full_val
% 3-agent PAR acc	https://wandb.ai/nthakkar/PAR_car_traj_prediction_nuscenes/runs/9flpua7w	final_car_traj_3_agent_accl_tok_a100_full_val
% 3-agent PAR acc + LPE	https://wandb.ai/nthakkar/PAR_car_traj_prediction_nuscenes/runs/l8fbaj7m?nw=nwusernthakkar	final_car_traj_3_agent_location_pos_embedding_sin_cos_accl_tok_relative_loc_a100_full_val

\end{table}
% 
% 
\begin{table}
\centering
\begin{tabular}{@{}lccc@{}}
\toprule
Baseline & Agents &  ADE $\downarrow$ & FDE $\downarrow$           \\
\midrule
Random Trajectory & 1  & 8.89 & 16.51      \\
NN    & 1   & 1.80 & 4.13 \\
 Multiagent NN & N & 6.40 & 12.04      \\
Mirror  & N & 11.59 & 14.93      \\
\bottomrule
\end{tabular}
\caption{\textbf{Car trajectory prediction baselines.} Nearest neighbor performs best overall, but our learned single-agent AR models outperform all baselines.}
\label{tab:cars_baselines}
\vspace{-0.25cm}
\end{table}

\begin{figure}
    \centering
    \includegraphics[width=0.48\textwidth]{fig/cars_qual.pdf}
    \caption{Example results from our single-agent AR model (top row) and three-agent PAR model with location positional encoding (bottom row) on nuScenes. The predicted agent's ground truth trajectory is in pink, and the predicted future in blue. For the PAR model, the other two agents' ground truth states are in green. Qualitatively, the PAR model handles situations where single-agent predictions might lead to collisions (A, B), uses other agents' behavior to better adhere to road areas (A, C) without environment data, and predicts based on the speed changes of other cars (D).}
    \label{fig:cars_qual}
    \vspace{-.6cm}
\end{figure}


We train AR and 3-agent PAR models using velocity tokens, acceleration tokens, and acceleration tokens combined with our location positional encoding. The results can be seen in Table~\ref{tab:cars_locs}. Note that the 3-agent PAR model uses the agent ID embedding and next timestep prediction.
Acceleration tokens consistently outperform velocity tokens both for agent AR and 3-agent PAR models. This could be because the vocabulary size for acceleration tokens is much smaller and therefore easier to optimize. Regardless, both ways of tokenizing result in models that outperform our baselines (see Table~\ref{tab:cars_baselines} - NN has a relatively low error on this dataset), and highlight that our framework is flexible such that a user can experiment with different ways of representing entities. For both token types, the 3-agent PAR model that is blind to location outperforms the AR model. While location information should help the model, it is possible that simply knowing whether other agents are slowing down or accelerating can help the model make better predictions.  When adding location information via the LPE to our 3-agent PAR model, we see another performance gain in ADE and FDE. 

Qualitative examples of the AR model (top row) and 3-agent location-aware PAR model (bottom row) can be seen in Figure~\ref{fig:cars_qual}. Our method uses no image or environment data (e.g., lanes) as input. However, by reasoning over multiple agents, its predictions lead to fewer collisions and better reasoning about speed changes and driveable areas based solely on other agents' behaviors.
% Note that our method does not take as input any pixels/image information, or any information about the environment such as lanes. We see evidence that when reasoning over multiple agents, our method is able to make predictions that result in fewer collisions, and better reasoning about changes in speed and what parts of the road are driveable, just based on the behaviour of other agents.






\section{Case Study 3: Object Pose Forecasting During Hand-Object Interaction}

Our final case study explores how hand-object interaction can be leveraged for object pose estimation. We define the human hand and the interacting object as two agents, with tokens representing distinct state types. We show that our PAR framework can jointly model these agents, improving 3D translation and rotation predictions for the object. 

\begin{figure}
    \centering
    % \includegraphics[width=\linewidth]{fig/rotation_result_2_bigger_10sr_cropped_remove_whitespace.png}
    \includegraphics[width=\linewidth]{fig/new_128dim_rot.pdf}
    \vspace{-20pt}
    \caption{
    % We present a qualitative result from the validation split for the rotation prediction task.
    \textbf{Rotation forecasting qualitative result on test set.} 3D predictions are projected onto the image, isolating rotation results by showing the ground-truth translation. Incorporating the hand agent in the PAR framework (right) improves object pose prediction over object-only AR (left). %Last frame of sequence shown.
    % The projected 3D model in blue has the ground-truth translation for visualization purposes and our predicted rotation. [say how much history there was and how far into the future]
    % To account for the low dynamics between consecutive frames, we sample every 10th frame. Left (AR), the results depict the object of interest as the sole agent, while the right (2-agent PAR) demonstrates improved performance by incorporating the human hand as a second agent in the grasping interaction. \shiry{say which frame in the future are you showing}
    }
    \label{fig:ho_qual}
    % \vspace{-.5cm}
\end{figure}

\begin{figure}
    \centering
    \includegraphics[width=\linewidth]{fig/new_128dim_tx.pdf}
    \vspace{-20pt}
    \caption{
    \textbf{Translation forecasting qualitative result on test set.} 3D predictions are projected onto the image, isolating translation results by showing the ground-truth rotation. Using the PAR framework (right) instead of AR (left) improves object pose prediction.
    }
    \label{fig:ho_qual2}
    \vspace{-.5cm}
\end{figure}

\subsection{Experimental Setup}
\noindent \textbf{Dataset.} We use the DexYCB dataset, which includes 1000 videos of 10 subjects performing object manipulation tasks with 20 distinct objects from the YCB-Video dataset. The data is split into 800 training, 40 validation, and 160 testing videos. We use one of 8 provided camera views. In each trial, subjects pick up and lift objects in randomized conditions. Labels include the object's SO(3) rotation and 3D translation, and the hand's 3D translation. We focus on predicting the object's rotation or translation.
% For this case study, we utilize the DexYCB dataset~\cite{chao2021dexycb}, which contains 1000 videos of 10 human subjects performing object manipulation tasks. Each subject picks up 20 distinct objects from the YCB-Video dataset~\cite{xiang2017posecnn}, with multiple trials conducted for each object. The dataset is divided into 800 training videos, 40 validation videos, and 160 testing videos. Although the videos are recorded from 8 RGB-D cameras, we work with a single camera view. In each trial, the subject starts in a relaxed pose with their hand by their side (often out of the camera’s view), grasps the target object, and lifts it into the air. For each subject-object pair, there are 5 trials where the object’s rotation, placement, and surrounding distractor objects are randomized. The dataset provides labels such as the object's SO(3) rotation and 3D translation, and the 3D positions of 21 hand joints in camera space. We focus on predicting either the object's rotation or translation as it is being picked up in each video.

\medskip \noindent \textbf{Task-specific considerations.} We tokenize object information in object-only experiments and both object and hand information in hand-object experiments. The object is represented as a 4D token for rotation forecasting (quaternion from SO(3) rotation) or a 3D token for translation forecasting (Euclidean coordinates). In hand-object experiments, the hand token is included with a 3D translation vector, and agent ID embeddings distinguish between the hand and object. 
Normalization is applied to all 3D translation vectors in both AR and PAR experiments; quaternions are normalized by definition and require no additional processing. 
An embedding layer projects the tokens into the transformer's hidden dimension, and another layer projects them back for prediction. 

For rotation-only forecasting, the loss is \(\mathscr{L}_{rot} = 1 - |\hat{q} \cdot q|\), where \(\hat{q}\) is the predicted quaternion and \(q\) the ground-truth quaternion. For translation-only forecasting, the loss \(\mathscr{L}_t\) is the mean squared error (MSE) between predicted and ground-truth translations. For PAR we predict relative object-to-hand translations at each frame, using the current hand position as origin, while for AR, we predict absolute object translations without considering the interacting agent. 
% For PAR, we treat the current hand translation as the origin, and predict the relative translation between the object and hand at each frame, while the hand is relative to itself at every frame. This is in contrast to AR, where we predict the absolute object translation since the interacting agent is not incorporated. 
For PAR models, we add the loss \(\mathscr{L}_h\), a MSE on hand translation. The object-only AR rotation model is optimized with \(\mathscr{L}_{rot}\), while the PAR rotation model combines \(\mathscr{L}_{rot} + \mathscr{L}_h\); similarly, the object-only translation model is trained with \(\mathscr{L}_t\), and the hand-object translation model uses \(\mathscr{L}_t + \mathscr{L}_h\). 
At inference, the first half of each video is provided, and object predictions are autoregressively generated for the second half. Translation is evaluated using MSE, while rotation is measured using geodesic distance (GEO) on SO(3).


% \medskip \noindent \textbf{Tokenization} For object-only experiments, we tokenize only the object information, while in hand-object experiments, we tokenize both the object and hand information.

% The object is represented differently depending on the task: for rotation-only prediction, we use 4-dimensional tokens derived from the quaternion representing its SO(3) rotation, while for translation-only prediction, we use 3-dimensional tokens representing the object's Euclidean coordinates. In hand-object experiments, where the hand is treated as a second agent interacting with the object, the hand is represented by a 63-dimensional vector corresponding to the Euclidean coordinates of 21 hand joints (3 joints and 1 fingertip per finger, plus 1 wrist joint). In hand-object interaction models, we also incorporate agent ID embeddings to distinguish between the hand and object.

% We use an embedding layer to project the token(s) into the transformer's hidden dimension and another to project them back into the token space for loss computation and generation. During training, we apply teacher forcing to the tokenized hand and object data. In validation, we teacher force the hand joint information while generating the object’s rotation or translation.

% %Object pose is represented as a 7D vector - 4D quaternion rotation and 3D XYZ translation. Hand pose is represented as a 63D vector of 21 XYZ joints. 

% %We learn an embedding layer to project the token into the transformers hidden dimension and another one to unproject back into the token dimension for loss computation and generation. \neerja{Since we are doing rotation and tx separately we should mention this here that we have different cases}

% %We explicitly require rotations to be valid quaternions.

% %\neerja{Explain the cases of object only, hand and object as separate tokens, and this 1.5 agent displacement thing. Currently we do loss on hand and object (but teacher force hand at inference), lay that out here.}

% \medskip \noindent \textbf{Loss} 
% For rotation-only prediction, the loss we optimize is:
% \[
%     \mathscr{L}_{rot} = 1 - |\hat{q} \cdot q|,
% \]
% where \(\hat{q}\) is the predicted quaternion representing the object's rotation in camera space, and \(q\) is the ground-truth quaternion. We apply the absolute value to account for the double-covering of quaternions in SO(3), i.e., \(q = -q\). Additionally, we ensure that the quaternions predicted by the de-projection layer are valid, meaning they have unit norm and a positive scalar component. 

% For translation-only prediction, the loss function \(\mathscr{L}_t\) is the mean squared error (MSE) between the predicted and ground-truth translations of the object. In experiments involving the hand, we also optimize for \(\mathscr{L}_h\), the MSE loss on the predicted and ground-truth hand joint positions. Additionally, we normalize the hand joints and object translations to be between 0 and 1 during training.

% The object-only rotation model is optimized with \(\mathscr{L}_{rot}\), while the hand-object rotation model uses the combined loss \(\alpha \mathscr{L}_{rot} + (1-\alpha) \mathscr{L}_h\), where $\alpha=0.33$. Similarly, the object-only translation model is trained with \(\mathscr{L}_t\), and the hand-object translation model is optimized with \(\mathscr{L}_t + \mathcal{L}_h\), where the two losses are equally weighted since they represent the same type of measurement.

% % In hand-object interaction models, we also incorporate agent ID embeddings to distinguish between the hand and object.

% \medskip

% \noindent \textbf{Metrics} During validation, we provide the first half of each video and autoregressively generate object predictions for the second half. For object translation, we evaluate performance using the MSE metric, while for object rotation, we measure error using the geodesic distance (GEO) on SO(3), which is the shortest path in radians between the predicted and ground-truth rotations. We convert the quaternions to SO(3) matrices to compute the GEO metric.

% % We use a MSE regression loss and train the model with teacher forcing - at each timestep, the previous ground-truth tokens are used in prediction. 

% % \neerja{Describe whatever metrics we settle on here - currently for rotation I think single shortest path between GT and pred rotation on the surface of the unit sphere in 3D rotation space, and MSE for translation}

% \newpage
\subsection{Results}
We compare the object-only AR models to the hand-object PAR models in Table~\ref{tab:ho_res} for the two prediction tasks. We also present the baselines described in Sec.~\ref{sec:task-specific} in Table~\ref{tab:ho_baselines}. Figures~\ref{fig:ho_qual} and~\ref{fig:ho_qual2} show qualitative results on the rotation and translation predictions, respectively. In both prediction tasks, we observe that incorporating the human hand's interaction with the object enhances accuracy: for rotation, PAR results in a relative improvement of $8.9\%$ over AR, and for translation, $41\%$. See Section~\ref{sec:app_ho_qual} for additional qualitative results with more sampled frames.
% In Figure~\ref{fig:ho_qual}, we see that the AR model (top row) achieves high-fidelity predictions early on, when much of its history still relies on ground truth data from the first half of the sequence. However, as the video progresses and the history becomes increasingly dependent on predicted object rotations, the AR model’s performance rapidly deteriorates. In contrast, our PAR model (bottom row) reasons over the 3D hand joint positions to predict the object's SO(3) rotation much more accurately. 
% Please see~\ref{app:pose} for more results on pose estimation.

\begin{table}
% \centering \footnotesize
\begin{tabular}{@{}lccc@{}}
\toprule
Type & Method &  MSE $\downarrow$ & GEO ($rad$) $\downarrow$ \\
% Type & Method &  MSE ($\times$10$^{-3}m^2$) $\downarrow$ & GEO ($rad$) $\downarrow$ \\
\midrule
Translation & 1-agent AR & 3.68 $\times$ 10$^{-3}$ & - \\
% Translation & 2-agent PAR & \textbf{3.28} & - \\
Translation & 2-agent PAR & \textbf{2.17} $\boldsymbol{\times}$ \textbf{10}$^{\boldsymbol{-3}}$ & - \\
\midrule
Rotation & 1-agent AR & - &  0.919 \\
Rotation & 2-agent PAR & - &  \textbf{0.837} \\
\bottomrule
\end{tabular}
\caption{\textbf{Test set results on DexYCB dataset.} For both rotation and translation forecasting, the 2-agent PAR model, which treats the hand as an additional agent, improves results.}
\label{tab:ho_res}
\end{table}

% \begin{table*}
% \centering \footnotesize
% \begin{tabular}{@{}lcccccc@{}}
% \toprule
% Type & Object Token & Hand Token & Ag ID Emb & Agents &  MSE ($m^2$) $\downarrow$ & GEO ($rad$) $\downarrow$           \\
% \midrule
% Translation & \cmark  & \xmark & \xmark & 1 & 2.97$\times$ 10$^{-2}$ & - \\
% Translation &  \cmark  & \cmark & \cmark & 2 & \textbf{1.90} $\boldsymbol{\times}$ \textbf{10}$^{\boldsymbol{-3}}$ & - \\
% \midrule
% Rotation & \cmark  & \xmark & \xmark & 1 & - &  0.944 \\
% Rotation & \cmark  & \cmark & \cmark & 2 & - &  \textbf{0.890} \\
% \bottomrule
% \end{tabular}
% \caption{\textbf{Test set results on DexYCB dataset.} Top two rows: translation prediction, bottom two rows: rotation prediction. In both cases, the 2-agent PAR model, which accounts hand-object interaction by integrating the hand as an additional agent, yields improved results.}
% \label{tab:ho_res}
% \end{table*}


\begin{table}
\centering \footnotesize
\begin{tabular}{@{}lcc@{}}
\toprule
Baseline &  Translation - MSE ($m^2$) $\downarrow$ & Rotation - GEO ($rad$) $\downarrow$           \\
\midrule
% Random  & 0.327 & 2.147 \\
% Random Trajectory & 1.37$\times$ 10$^{-2}$ & 2.189 \\
% NN  & 1.54$\times$ 10$^{-2}$ &  2.077\\
% Multiagent NN  & 1.31$\times$ 10$^{-2}$ & 2.268  \\
% Mirror  & 1.20$\times$ 10$^{-2}$  & N/A  \\
Random  & 0.244 & 2.196 \\
Random Trajectory & 1.60$\times$ 10$^{-2}$ & 2.146 \\
NN  & 1.69$\times$ 10$^{-2}$ &  2.179 \\
Multiagent NN  & 1.71$\times$ 10$^{-2}$ & 2.170 \\
Mirror  & 1.20$\times$ 10$^{-2}$  & -  \\
\bottomrule
\end{tabular}
\caption{\textbf{Test set results for DexYCB baselines.} We cannot provide rotation results for the Mirror baseline, because the ground-truth does not include hand rotation, only 3D translation.}
\label{tab:ho_baselines}
\end{table}









\section{Discussion}\label{sec:discussion}



\subsection{From Interactive Prompting to Interactive Multi-modal Prompting}
The rapid advancements of large pre-trained generative models including large language models and text-to-image generation models, have inspired many HCI researchers to develop interactive tools to support users in crafting appropriate prompts.
% Studies on this topic in last two years' HCI conferences are predominantly focused on helping users refine single-modality textual prompts.
Many previous studies are focused on helping users refine single-modality textual prompts.
However, for many real-world applications concerning data beyond text modality, such as multi-modal AI and embodied intelligence, information from other modalities is essential in constructing sophisticated multi-modal prompts that fully convey users' instruction.
This demand inspires some researchers to develop multimodal prompting interactions to facilitate generation tasks ranging from visual modality image generation~\cite{wang2024promptcharm, promptpaint} to textual modality story generation~\cite{chung2022tale}.
% Some previous studies contributed relevant findings on this topic. 
Specifically, for the image generation task, recent studies have contributed some relevant findings on multi-modal prompting.
For example, PromptCharm~\cite{wang2024promptcharm} discovers the importance of multimodal feedback in refining initial text-based prompting in diffusion models.
However, the multi-modal interactions in PromptCharm are mainly focused on the feedback empowered the inpainting function, instead of supporting initial multimodal sketch-prompt control. 

\begin{figure*}[t]
    \centering
    \includegraphics[width=0.9\textwidth]{src/img/novice_expert.pdf}
    \vspace{-2mm}
    \caption{The comparison between novice and expert participants in painting reveals that experts produce more accurate and fine-grained sketches, resulting in closer alignment with reference images in close-ended tasks. Conversely, in open-ended tasks, expert fine-grained strokes fail to generate precise results due to \tool's lack of control at the thin stroke level.}
    \Description{The comparison between novice and expert participants in painting reveals that experts produce more accurate and fine-grained sketches, resulting in closer alignment with reference images in close-ended tasks. Novice users create rougher sketches with less accuracy in shape. Conversely, in open-ended tasks, expert fine-grained strokes fail to generate precise results due to \tool's lack of control at the thin stroke level, while novice users' broader strokes yield results more aligned with their sketches.}
    \label{fig:novice_expert}
    % \vspace{-3mm}
\end{figure*}


% In particular, in the initial control input, users are unable to explicitly specify multi-modal generation intents.
In another example, PromptPaint~\cite{promptpaint} stresses the importance of paint-medium-like interactions and introduces Prompt stencil functions that allow users to perform fine-grained controls with localized image generation. 
However, insufficient spatial control (\eg, PromptPaint only allows for single-object prompt stencil at a time) and unstable models can still leave some users feeling the uncertainty of AI and a varying degree of ownership of the generated artwork~\cite{promptpaint}.
% As a result, the gap between intuitive multi-modal or paint-medium-like control and the current prompting interface still exists, which requires further research on multi-modal prompting interactions.
From this perspective, our work seeks to further enhance multi-object spatial-semantic prompting control by users' natural sketching.
However, there are still some challenges to be resolved, such as consistent multi-object generation in multiple rounds to increase stability and improved understanding of user sketches.   


% \new{
% From this perspective, our work is a step forward in this direction by allowing multi-object spatial-semantic prompting control by users' natural sketching, which considers the interplay between multiple sketch regions.
% % To further advance the multi-modal prompting experience, there are some aspects we identify to be important.
% % One of the important aspects is enhancing the consistency and stability of multiple rounds of generation to reduce the uncertainty and loss of control on users' part.
% % For this purpose, we need to develop techniques to incorporate consistent generation~\cite{tewel2024training} into multi-modal prompting framework.}
% % Another important aspect is improving generative models' understanding of the implicit user intents \new{implied by the paint-medium-like or sketch-based input (\eg, sketch of two people with their hands slightly overlapping indicates holding hand without needing explicit prompt).
% % This can facilitate more natural control and alleviate users' effort in tuning the textual prompt.
% % In addition, it can increase users' sense of ownership as the generated results can be more aligned with their sketching intents.
% }
% For example, when users draw sketches of two people with their hands slightly overlapping, current region-based models cannot automatically infer users' implicit intention that the two people are holding hands.
% Instead, they still require users to explicitly specify in the prompt such relationship.
% \tool addresses this through sketch-aware prompt recommendation to fill in the necessary semantic information, alleviating users' workload.
% However, some users want the generative AI in the future to be able to directly infer this natural implicit intentions from the sketches without additional prompting since prompt recommendation can still be unstable sometimes.


% \new{
% Besides visual generation, 
% }
% For example, one of the important aspect is referring~\cite{he2024multi}, linking specific text semantics with specific spatial object, which is partly what we do in our sketch-aware prompt recommendation.
% Analogously, in natural communication between humans, text or audio alone often cannot suffice in expressing the speakers' intentions, and speakers often need to refer to an existing spatial object or draw out an illustration of her ideas for better explanation.
% Philosophically, we HCI researchers are mostly concerned about the human-end experience in human-AI communications.
% However, studies on prompting is unique in that we should not just care about the human-end interaction, but also make sure that AI can really get what the human means and produce intention-aligned output.
% Such consideration can drastically impact the design of prompting interactions in human-AI collaboration applications.
% On this note, although studies on multi-modal interactions is a well-established topic in HCI community, it remains a challenging problem what kind of multi-modal information is really effective in helping humans convey their ideas to current and next generation large AI models.




\subsection{Novice Performance vs. Expert Performance}\label{sec:nVe}
In this section we discuss the performance difference between novice and expert regarding experience in painting and prompting.
First, regarding painting skills, some participants with experience (4/12) preferred to draw accurate and fine-grained shapes at the beginning. 
All novice users (5/12) draw rough and less accurate shapes, while some participants with basic painting skills (3/12) also favored sketching rough areas of objects, as exemplified in Figure~\ref{fig:novice_expert}.
The experienced participants using fine-grained strokes (4/12, none of whom were experienced in prompting) achieved higher IoU scores (0.557) in the close-ended task (0.535) when using \tool. 
This is because their sketches were closer in shape and location to the reference, making the single object decomposition result more accurate.
Also, experienced participants are better at arranging spatial location and size of objects than novice participants.
However, some experienced participants (3/12) have mentioned that the fine-grained stroke sometimes makes them frustrated.
As P1's comment for his result in open-ended task: "\emph{It seems it cannot understand thin strokes; even if the shape is accurate, it can only generate content roughly around the area, especially when there is overlapping.}" 
This suggests that while \tool\ provides rough control to produce reasonably fine results from less accurate sketches for novice users, it may disappoint experienced users seeking more precise control through finer strokes. 
As shown in the last column in Figure~\ref{fig:novice_expert}, the dragon hovering in the sky was wrongly turned into a standing large dragon by \tool.

Second, regarding prompting skills, 3 out of 12 participants had one or more years of experience in T2I prompting. These participants used more modifiers than others during both T2I and R2I tasks.
Their performance in the T2I (0.335) and R2I (0.469) tasks showed higher scores than the average T2I (0.314) and R2I (0.418), but there was no performance improvement with \tool\ between their results (0.508) and the overall average score (0.528). 
This indicates that \tool\ can assist novice users in prompting, enabling them to produce satisfactory images similar to those created by users with prompting expertise.



\subsection{Applicability of \tool}
The feedback from user study highlighted several potential applications for our system. 
Three participants (P2, P6, P8) mentioned its possible use in commercial advertising design, emphasizing the importance of controllability for such work. 
They noted that the system's flexibility allows designers to quickly experiment with different settings.
Some participants (N = 3) also mentioned its potential for digital asset creation, particularly for game asset design. 
P7, a game mod developer, found the system highly useful for mod development. 
He explained: "\emph{Mods often require a series of images with a consistent theme and specific spatial requirements. 
For example, in a sacrifice scene, how the objects are arranged is closely tied to the mod's background. It would be difficult for a developer without professional skills, but with this system, it is possible to quickly construct such images}."
A few participants expressed similar thoughts regarding its use in scene construction, such as in film production. 
An interesting suggestion came from participant P4, who proposed its application in crime scene description. 
She pointed out that witnesses are often not skilled artists, and typically describe crime scenes verbally while someone else illustrates their account. 
With this system, witnesses could more easily express what they saw themselves, potentially producing depictions closer to the real events. "\emph{Details like object locations and distances from buildings can be easily conveyed using the system}," she added.

% \subsection{Model Understanding of Users' Implicit Intents}
% In region-sketch-based control of generative models, a significant gap between interaction design and actual implementation is the model's failure in understanding users' naturally expressed intentions.
% For example, when users draw sketches of two people with their hands slightly overlapping, current region-based models cannot automatically infer users' implicit intention that the two people are holding hands.
% Instead, they still require users to explicitly specify in the prompt such relationship.
% \tool addresses this through sketch-aware prompt recommendation to fill in the necessary semantic information, alleviating users' workload.
% However, some users want the generative AI in the future to be able to directly infer this natural implicit intentions from the sketches without additional prompting since prompt recommendation can still be unstable sometimes.
% This problem reflects a more general dilemma, which ubiquitously exists in all forms of conditioned control for generative models such as canny or scribble control.
% This is because all the control models are trained on pairs of explicit control signal and target image, which is lacking further interpretation or customization of the user intentions behind the seemingly straightforward input.
% For another example, the generative models cannot understand what abstraction level the user has in mind for her personal scribbles.
% Such problems leave more challenges to be addressed by future human-AI co-creation research.
% One possible direction is fine-tuning the conditioned models on individual user's conditioned control data to provide more customized interpretation. 

% \subsection{Balance between recommendation and autonomy}
% AIGC tools are a typical example of 
\subsection{Progressive Sketching}
Currently \tool is mainly aimed at novice users who are only capable of creating very rough sketches by themselves.
However, more accomplished painters or even professional artists typically have a coarse-to-fine creative process. 
Such a process is most evident in painting styles like traditional oil painting or digital impasto painting, where artists first quickly lay down large color patches to outline the most primitive proportion and structure of visual elements.
After that, the artists will progressively add layers of finer color strokes to the canvas to gradually refine the painting to an exquisite piece of artwork.
One participant in our user study (P1) , as a professional painter, has mentioned a similar point "\emph{
I think it is useful for laying out the big picture, give some inspirations for the initial drawing stage}."
Therefore, rough sketch also plays a part in the professional artists' creation process, yet it is more challenging to integrate AI into this more complex coarse-to-fine procedure.
Particularly, artists would like to preserve some of their finer strokes in later progression, not just the shape of the initial sketch.
In addition, instead of requiring the tool to generate a finished piece of artwork, some artists may prefer a model that can generate another more accurate sketch based on the initial one, and leave the final coloring and refining to the artists themselves.
To accommodate these diverse progressive sketching requirements, a more advanced sketch-based AI-assisted creation tool should be developed that can seamlessly enable artist intervention at any stage of the sketch and maximally preserve their creative intents to the finest level. 

\subsection{Ethical Issues}
Intellectual property and unethical misuse are two potential ethical concerns of AI-assisted creative tools, particularly those targeting novice users.
In terms of intellectual property, \tool hands over to novice users more control, giving them a higher sense of ownership of the creation.
However, the question still remains: how much contribution from the user's part constitutes full authorship of the artwork?
As \tool still relies on backbone generative models which may be trained on uncopyrighted data largely responsible for turning the sketch into finished artwork, we should design some mechanisms to circumvent this risk.
For example, we can allow artists to upload backbone models trained on their own artworks to integrate with our sketch control.
Regarding unethical misuse, \tool makes fine-grained spatial control more accessible to novice users, who may maliciously generate inappropriate content such as more realistic deepfake with specific postures they want or other explicit content.
To address this issue, we plan to incorporate a more sophisticated filtering mechanism that can detect and screen unethical content with more complex spatial-semantic conditions. 
% In the future, we plan to enable artists to upload their own style model

% \subsection{From interactive prompting to interactive spatial prompting}


\subsection{Limitations and Future work}

    \textbf{User Study Design}. Our open-ended task assesses the usability of \tool's system features in general use cases. To further examine aspects such as creativity and controllability across different methods, the open-ended task could be improved by incorporating baselines to provide more insightful comparative analysis. 
    Besides, in close-ended tasks, while the fixing order of tool usage prevents prior knowledge leakage, it might introduce learning effects. In our study, we include practice sessions for the three systems before the formal task to mitigate these effects. In the future, utilizing parallel tests (\textit{e.g.} different content with the same difficulty) or adding a control group could further reduce the learning effects.

    \textbf{Failure Cases}. There are certain failure cases with \tool that can limit its usability. 
    Firstly, when there are three or more objects with similar semantics, objects may still be missing despite prompt recommendations. 
    Secondly, if an object's stroke is thin, \tool may incorrectly interpret it as a full area, as demonstrated in the expert results of the open-ended task in Figure~\ref{fig:novice_expert}. 
    Finally, sometimes inclusion relationships (\textit{e.g.} inside) between objects cannot be generated correctly, partially due to biases in the base model that lack training samples with such relationship. 

    \textbf{More support for single object adjustment}.
    Participants (N=4) suggested that additional control features should be introduced, beyond just adjusting size and location. They noted that when objects overlap, they cannot freely control which object appears on top or which should be covered, and overlapping areas are currently not allowed.
    They proposed adding features such as layer control and depth control within the single-object mask manipulation. Currently, the system assigns layers based on color order, but future versions should allow users to adjust the layer of each object freely, while considering weighted prompts for overlapping areas.

    \textbf{More customized generation ability}.
    Our current system is built around a single model $ColorfulXL-Lightning$, which limits its ability to fully support the diverse creative needs of users. Feedback from participants has indicated a strong desire for more flexibility in style and personalization, such as integrating fine-tuned models that cater to specific artistic styles or individual preferences. 
    This limitation restricts the ability to adapt to varied creative intents across different users and contexts.
    In future iterations, we plan to address this by embedding a model selection feature, allowing users to choose from a variety of pre-trained or custom fine-tuned models that better align with their stylistic preferences. 
    
    \textbf{Integrate other model functions}.
    Our current system is compatible with many existing tools, such as Promptist~\cite{hao2024optimizing} and Magic Prompt, allowing users to iteratively generate prompts for single objects. However, the integration of these functions is somewhat limited in scope, and users may benefit from a broader range of interactive options, especially for more complex generation tasks. Additionally, for multimodal large models, users can currently explore using affordable or open-source models like Qwen2-VL~\cite{qwen} and InternVL2-Llama3~\cite{llama}, which have demonstrated solid inference performance in our tests. While GPT-4o remains a leading choice, alternative models also offer competitive results.
    Moving forward, we aim to integrate more multimodal large models into the system, giving users the flexibility to choose the models that best fit their needs. 
    


\section{Conclusion}\label{sec:conclusion}
In this paper, we present \tool, an interactive system designed to help novice users create high-quality, fine-grained images that align with their intentions based on rough sketches. 
The system first refines the user's initial prompt into a complete and coherent one that matches the rough sketch, ensuring the generated results are both stable, coherent and high quality.
To further support users in achieving fine-grained alignment between the generated image and their creative intent without requiring professional skills, we introduce a decompose-and-recompose strategy. 
This allows users to select desired, refined object shapes for individual decomposed objects and then recombine them, providing flexible mask manipulation for precise spatial control.
The framework operates through a coarse-to-fine process, enabling iterative and fine-grained control that is not possible with traditional end-to-end generation methods. 
Our user study demonstrates that \tool offers novice users enhanced flexibility in control and fine-grained alignment between their intentions and the generated images.


{
    \small
    \bibliographystyle{ieeenat_fullname}
    \bibliography{main}
}

% WARNING: do not forget to delete the supplementary pages from your submission 
\clearpage
\pagenumbering{gobble}
\maketitlesupplementary

\section{Additional Results on Embodied Tasks}

To evaluate the broader applicability of our EgoAgent's learned representation beyond video-conditioned 3D human motion prediction, we test its ability to improve visual policy learning for embodiments other than the human skeleton.
Following the methodology in~\cite{majumdar2023we}, we conduct experiments on the TriFinger benchmark~\cite{wuthrich2020trifinger}, which involves a three-finger robot performing two tasks: reach cube and move cube. 
We freeze the pretrained representations and use a 3-layer MLP as the policy network, training each task with 100 demonstrations.

\begin{table}[h]
\centering
\caption{Success rate (\%) on the TriFinger benchmark, where each model's pretrained representation is fixed, and additional linear layers are trained as the policy network.}
\label{tab:trifinger}
\resizebox{\linewidth}{!}{%
\begin{tabular}{llcc}
\toprule
Methods       & Training Dataset & Reach Cube & Move Cube \\
\midrule
DINO~\cite{caron2021emerging}         & WT Venice        & 78.03     & 47.42     \\
DoRA~\cite{venkataramanan2023imagenet}          & WT Venice        & 81.62     & 53.76     \\
DoRA~\cite{venkataramanan2023imagenet}          & WT All           & 82.40     & 48.13     \\
\midrule
EgoAgent-300M & WT+Ego-Exo4D      & 82.61    & 54.21      \\
EgoAgent-1B   & WT+Ego-Exo4D      & \textbf{85.72}      & \textbf{57.66}   \\
\bottomrule
\end{tabular}%
}
\end{table}

As shown in Table~\ref{tab:trifinger}, EgoAgent achieves the highest success rates on both tasks, outperforming the best models from DoRA~\cite{venkataramanan2023imagenet} with increases of +3.32\% and +3.9\% respectively.
This result shows that by incorporating human action prediction into the learning process, EgoAgent demonstrates the ability to learn more effective representations that benefit both image classification and embodied manipulation tasks.
This highlights the potential of leveraging human-centric motion data to bridge the gap between visual understanding and actionable policy learning.



\section{Additional Results on Egocentric Future State Prediction}

In this section, we provide additional qualitative results on the egocentric future state prediction task. Additionally, we describe our approach to finetune video diffusion model on the Ego-Exo4D dataset~\cite{grauman2024ego} and generate future video frames conditioned on initial frames as shown in Figure~\ref{fig:opensora_finetune}.

\begin{figure}[b]
    \centering
    \includegraphics[width=\linewidth]{figures/opensora_finetune.pdf}
    \caption{Comparison of OpenSora V1.1 first-frame-conditioned video generation results before and after finetuning on Ego-Exo4D. Fine-tuning enhances temporal consistency, but the predicted pixel-space future states still exhibit errors, such as inaccuracies in the basketball's trajectory.}
    \label{fig:opensora_finetune}
\end{figure}

\subsection{Visualizations and Comparisons}

More visualizations of our method, DoRA, and OpenSora in different scenes (as shown in Figure~\ref{fig:supp pred}). For OpenSora, when predicting the states of $t_k$, we use all the ground truth frames from $t_{0}$ to $t_{k-1}$ as conditions. As OpenSora takes only past observations as input and neglects human motion, it performs well only when the human has relatively small motions (see top cases in Figure~\ref{fig:supp pred}), but can not adjust to large movements of the human body or quick viewpoint changes (see bottom cases in Figure~\ref{fig:supp pred}).

\begin{figure*}
    \centering
    \includegraphics[width=\linewidth]{figures/supp_pred.pdf}
    \caption{Retrieval and generation results for egocentric future state prediction. Correct and wrong retrieval images are marked with green and red boundaries, respectively.}
    \label{fig:supp pred}
\end{figure*}

\begin{figure*}[t]
    \centering
    \includegraphics[width=0.9\linewidth]{figures/motion_prediction.pdf}
    \vspace{-0.5mm}
    \caption{Motion prediction results in scenes with minor changes in observation.}
    \vspace{-1.5mm}
    \label{fig:motion_prediction}
\end{figure*}

\subsection{Finetuning OpenSora on Ego-Exo4D}

OpenSora V1.1~\cite{opensora}, initially trained on internet videos and images, produces severely inconsistent results when directly applied to infer future videos on the Ego-Exo4D dataset, as illustrated in Figure~\ref{fig:opensora_finetune}.
To address the gap between general internet content and egocentric video data, we fine-tune the official checkpoint on the Ego-Exo4D training set for 50 epochs.
OpenSora V1.1 proposed a random mask strategy during training to enable video generation by image and video conditioning. We adopted the default masking rate, which applies: 75\% with no masking, 2.5\% with random masking of 1 frame to 1/4 of the total frames, 2.5\% with masking at either the beginning or the end for 1 frame to 1/4 of the total frames, and 5\% with random masking spanning 1 frame to 1/4 of the total frames at both the beginning and the end.

As shown in Fig.~\ref{fig:opensora_finetune}, despite being trained on a large dataset, OpenSora struggles to generalize to the Ego-Exo4D dataset, producing future video frames with minimal consistency relative to the conditioning frame. While fine-tuning improves temporal consistency, the moving trajectories of objects like the basketball and soccer ball still deviate from realistic physical laws. Compared with our feature space prediction results, this suggests that training world models in a reconstructive latent space is more challenging than training them in a feature space.


\section{Additional Results on 3D Human Motion Prediction}

We present additional qualitative results for the 3D human motion prediction task, highlighting a particularly challenging scenario where egocentric observations exhibit minimal variation. This scenario poses significant difficulties for video-conditioned motion prediction, as the model must effectively capture and interpret subtle changes. As demonstrated in Fig.~\ref{fig:motion_prediction}, EgoAgent successfully generates accurate predictions that closely align with the ground truth motion, showcasing its ability to handle fine-grained temporal dynamics and nuanced contextual cues.

\section{OpenSora for Image Classification}

In this section, we detail the process of extracting features from OpenSora V1.1~\cite{opensora} (without fine-tuning) for an image classification task. Following the approach of~\cite{xiang2023denoising}, we leverage the insight that diffusion models can be interpreted as multi-level denoising autoencoders. These models inherently learn linearly separable representations within their intermediate layers, without relying on auxiliary encoders. The quality of the extracted features depends on both the layer depth and the noise level applied during extraction.


\begin{table}[h]
\centering
\caption{$k$-NN evaluation results of OpenSora V1.1 features from different layer depths and noising scales on ImageNet-100. Top1 and Top5 accuracy (\%) are reported.}
\label{tab:opensora-knn}
\resizebox{0.95\linewidth}{!}{%
\begin{tabular}{lcccccc}
\toprule
\multirow{2}{*}{Timesteps} & \multicolumn{2}{c}{First Layer} & \multicolumn{2}{c}{Middle Layer} & \multicolumn{2}{c}{Last Layer} \\
\cmidrule(r){2-3}   \cmidrule(r){4-5}  \cmidrule(r){6-7}  & Top1           & Top5           & Top1            & Top5           & Top1           & Top5          \\
\midrule
32        &  6.10           & 18.20             & 34.04               & 59.50             & 30.40             & 55.74             \\
64        & 6.12              & 18.48              & 36.04               & 61.84              & 31.80         & 57.06         \\
128       & 5.84             & 18.14             & 38.08               & 64.16              & 33.44       & 58.42 \\
256       & 5.60             & 16.58              & 30.34               & 56.38              &28.14          & 52.32        \\
512       & 3.66              & 11.70            & 6.24              & 17.62              & 7.24              & 19.44  \\ 
\bottomrule
\end{tabular}%
}
\end{table}

As shown in Table~\ref{tab:opensora-knn}, we first evaluate $k$-NN classification performance on the ImageNet-100 dataset using three intermediate layers and five different noise scales. We find that a noise timestep of 128 yields the best results, with the middle and last layers performing significantly better than the first layer.
We then test this optimal configuration on ImageNet-1K and find that the last layer with 128 noising timesteps achieves the best classification accuracy.

\section{Data Preprocess}
For egocentric video sequences, we utilize videos from the Ego-Exo4D~\cite{grauman2024ego} and WT~\cite{venkataramanan2023imagenet} datasets.
The original resolution of Ego-Exo4D videos is 1408×1408, captured at 30 fps. We sample one frame every five frames and use the original resolution to crop local views (224×224) for computing the self-supervised representation loss. For computing the prediction and action loss, the videos are downsampled to 224×224 resolution.
WT primarily consists of 4K videos (3840×2160) recorded at 60 or 30 fps. Similar to Ego-Exo4D, we use the original resolution and downsample the frame rate to 6 fps for representation loss computation.
As Ego-Exo4D employs fisheye cameras, we undistort the images to a pinhole camera model using the official Project Aria Tools to align them with the WT videos.

For motion sequences, the Ego-Exo4D dataset provides synchronized 3D motion annotations and camera extrinsic parameters for various tasks and scenes. While some annotations are manually labeled, others are automatically generated using 3D motion estimation algorithms from multiple exocentric views. To maximize data utility and maintain high-quality annotations, manual labels are prioritized wherever available, and automated annotations are used only when manual labels are absent.
Each pose is converted into the egocentric camera's coordinate system using transformation matrices derived from the camera extrinsics. These matrices also enable the computation of trajectory vectors for each frame in a sequence. Beyond the x, y, z coordinates, a visibility dimension is appended to account for keypoints invisible to all exocentric views. Finally, a sliding window approach segments sequences into fixed-size windows to serve as input for the model. Note that we do not downsample the frame rate of 3D motions.

\section{Training Details}
\subsection{Architecture Configurations}
In Table~\ref{tab:arch}, we provide detailed architecture configurations for EgoAgent following the scaling-up strategy of InternLM~\cite{team2023internlm}. To ensure the generalization, we do not modify the internal modules in InternML, \emph{i.e.}, we adopt the RMSNorm and 1D RoPE. We show that, without specific modules designed for vision tasks, EgoAgent can perform well on vision and action tasks.

\begin{table}[ht]
  \centering
  \caption{Architecture configurations of EgoAgent.}
  \resizebox{0.8\linewidth}{!}{%
    \begin{tabular}{lcc}
    \toprule
          & EgoAgent-300M & EgoAgent-1B \\
          \midrule
    Depth & 22    & 22 \\
    Embedding dim & 1024  & 2048 \\
    Number of heads & 8     & 16 \\
    MLP ratio &    8/3   & 8/3 \\
    $\#$param.  & 284M & 1.13B \\
    \bottomrule
    \end{tabular}%
    }
  \label{tab:arch}%
\end{table}%

Table~\ref{tab:io_structure} presents the detailed configuration of the embedding and prediction modules in EgoAgent, including the image projector ($\text{Proj}_i$), representation head/state prediction head ($\text{MLP}_i$), action projector ($\text{Proj}_a$) and action prediction head ($\text{MLP}_a$).
Note that the representation head and the state prediction head share the same architecture but have distinct weights.

\begin{table}[t]
\centering
\caption{Architecture of the embedding ($\text{Proj}_i$, $\text{Proj}_a$) and prediction ($\text{MLP}_i$, $\text{MLP}_a$) modules in EgoAgent. For details on module connections and functions, please refer to Fig.~2 in the main paper.}
\label{tab:io_structure}
\resizebox{\linewidth}{!}{%
\begin{tabular}{lcl}
\toprule
       & \multicolumn{1}{c}{Norm \& Activation} & \multicolumn{1}{c}{Output Shape}  \\
\midrule
\multicolumn{3}{l}{$\text{Proj}_i$ (\textit{Image projector})} \\
\midrule
Input image  & -          & 3$\times$224$\times$224 \\
Conv 2D (16$\times$16) & -       & Embedding dim$\times$14$\times$14    \\
\midrule
\multicolumn{3}{l}{$\text{MLP}_i$ (\textit{State prediction head} \& \textit{Representation head)}} \\
\midrule
Input embedding  & -          & Embedding dim \\
Linear & GELU       & 2048          \\
Linear & GELU       & 2048          \\
Linear & -          & 256           \\
Linear & -          & 65536     \\
\midrule
\multicolumn{3}{l}{$\text{Proj}_a$ (\textit{Action projector})} \\
\midrule
Input pose sequence  & -          & 4$\times$5$\times$17 \\
Conv 2D (5$\times$17) & LN, GELU   & Embedding dim$\times$1$\times$1    \\
\midrule
\multicolumn{3}{l}{$\text{MLP}_a$ (\textit{Action prediction head})} \\
\midrule
Input embedding  & -          & Embedding dim$\times$1$\times$1 \\
Linear & -          & 4$\times$5$\times$17     \\
\bottomrule
\end{tabular}%
}
\end{table}


\subsection{Training Configurations}
In Table~\ref{tab:training hyper}, we provide the detailed training hyper-parameters for experiments in the main manuscripts.

\begin{table}[ht]
  \centering
  \caption{Hyper-parameters for training EgoAgent.}
  \resizebox{0.86\linewidth}{!}{%
    \begin{tabular}{lc}
    \toprule
    Training Configuration & EgoAgent-300M/1B \\
    \midrule
    Training recipe: &  \\
    optimizer & AdamW~\cite{loshchilov2017decoupled} \\
    optimizer momentum & $\beta_1=0.9, \beta_2=0.999$ \\
    \midrule
    Learning hyper-parameters: &  \\
    base learning rate & 6.0E-04 \\
    learning rate schedule & cosine \\
    base weight decay & 0.04 \\
    end weight decay & 0.4 \\
    batch size & 1920 \\
    training iters & 72,000 \\
    lr warmup iters & 1,800 \\
    warmup schedule & linear \\
    gradient clip & 1.0 \\
    data type & float16 \\
    norm epsilon & 1.0E-06 \\
    \midrule
    EMA hyper-parameters: &  \\
    momentum & 0.996 \\
    \bottomrule
    \end{tabular}%
    }
  \label{tab:training hyper}%
\end{table}%

\clearpage


\end{document}
