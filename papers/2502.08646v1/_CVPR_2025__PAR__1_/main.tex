% CVPR 2025 Paper Template; see https://github.com/cvpr-org/author-kit

\documentclass[10pt,twocolumn,letterpaper]{article}

%%%%%%%%% PAPER TYPE  - PLEASE UPDATE FOR FINAL VERSION
% \usepackage{cvpr}              % To produce the CAMERA-READY version
% \usepackage[review]{cvpr}      % To produce the REVIEW version
\usepackage[pagenumbers]{cvpr} % To force page numbers, e.g. for an arXiv version

% Import additional packages in the preamble file, before hyperref
\newcommand{\CG}{\mathcal{G}\xspace}
\newcommand{\CV}{\mathcal{V}\xspace}
\newcommand{\CE}{\mathcal{E}\xspace}
\newcommand{\CA}{\mathcal{A}\xspace}
\newcommand{\CF}{\mathcal{F}\xspace}
\newcommand{\CR}{\mathcal{R}\xspace}
\newcommand{\CB}{\mathcal{B}\xspace}
\newcommand{\CX}{\mathcal{X}\xspace}
\newcommand{\CK}{\mathcal{K}\xspace}
\newcommand{\CM}{\mathcal{M}\xspace}
\newcommand{\CC}{\mathcal{C}\xspace}
\newcommand{\CL}{\mathcal{L}\xspace}
\newcommand{\CI}{\mathcal{I}\xspace}
\newcommand{\CQ}{\mathcal{Q}\xspace}
\newcommand{\CO}{\mathcal{O}\xspace}
\newcommand{\CP}{\mathcal{P}\xspace}
\newcommand{\CS}{\mathcal{S}\xspace}
\newcommand{\CT}{\mathcal{T}\xspace}
\newcommand{\CJ}{\mathcal{J}\xspace}
\usepackage[para]{footmisc}
\usepackage{subfig}
% \usepackage{subcaption}
% \usepackage{array}
% \usepackage{colortbl}



% It is strongly recommended to use hyperref, especially for the review version.
% hyperref with option pagebackref eases the reviewers' job.
% Please disable hyperref *only* if you encounter grave issues, 
% e.g. with the file validation for the camera-ready version.
%
% If you comment hyperref and then uncomment it, you should delete *.aux before re-running LaTeX.
% (Or just hit 'q' on the first LaTeX run, let it finish, and you should be clear).
\definecolor{cvprblue}{rgb}{0.21,0.49,0.74}
\usepackage[pagebackref,breaklinks,colorlinks,allcolors=cvprblue]{hyperref}

\usepackage{url} % get urls to play nice in bib

%typesetting tricks from Noah
\usepackage{microtype}
\usepackage{float}
\frenchspacing

\usepackage{graphicx}
\usepackage{booktabs}% fancy tables
% \newcommand{\ra}[1]{\renewcommand{\arraystretch}{#1}} % More space between rows of tables
\usepackage{multirow} % vertical text in tables
\usepackage{subcaption} % subtables
\usepackage{color, colortbl,xcolor}
\definecolor{Gray}{gray}{0.9}
\definecolor{light-gray}{gray}{0.95}
\newcommand\sbullet[1][.5]{\mathbin{\vcenter{\hbox{\scalebox{#1}{$\bullet$}}}}}


\usepackage[mathscr]{euscript}
\usepackage{hhline} % Import hhline package


%%%%%%%%% PAPER ID  - PLEASE UPDATE
\def\paperID{16862} % *** Enter the Paper ID here
\def\confName{CVPR}
\def\confYear{2025}

%%%%%%%%% TITLE - PLEASE UPDATE
\title{Poly-Autoregressive Prediction for Modeling Interactions}

%%%%%%%%% AUTHORS - PLEASE UPDATE
% \author{Neerja Thakkar\\
% UC Berkeley\\
% % Institution1 address\\
% % {\tt\small firstauthor@i1.org}
% % For a paper whose authors are all at the same institution,
% % omit the following lines up until the closing ``}''.
% % Additional authors and addresses can be added with ``\and'',
% % just like the second author.
% % To save space, use either the email address or home page, not both
% \and
% Tara Sadjadpour\\
% UC Berkeley\\
% % First line of institution2 address\\
% % {\tt\small secondauthor@i2.org}
% \and
% Jathushan Rajasegeran\\
% UC Berkeley\\
% \and
% Shiry Ginosar\\
% TTIC, Google DeepMind\\
% \and
% Jitendra Malik\\
% UC Berkeley\\
% }


% \author{Neerja Thakkar\\
% \and
% Tara Sadjadpour\\
% \and
% Jathushan Rajasegeran\\
% \and
% Shiry Ginosar\\
% \and
% Jitendra Malik\\
% }


\author{%
  \begin{tabular}[t]{@{}ccccc@{}}
    Neerja Thakkar$^{1}$ & Tara Sadjadpour$^{1}$ & Jathushan Rajasegeran$^{1}$ & Shiry Ginosar$^{2,3}$ & Jitendra Malik$^{1}$
  \end{tabular}\\[0.5em]
  { $^{1}$UC Berkeley, $^{2}$Toyota Technical Institute at Chicago, $^{3}$Google DeepMind}
}


\begin{document}
% \maketitle

% \twocolumn[{%
% \renewcommand\twocolumn[1][]{#1}%
% \maketitle
% \begin{center}
%     \vspace{-0.26in}
%     \begin{subfigure}[b]{0.49\textwidth}
%         \includegraphics[width=\textwidth]{fig/auto.pdf}
%         \caption{Autoregression (AR)}
%         \label{fig:auto}
%     \end{subfigure}
%     ~
%     \begin{subfigure}[b]{0.49\textwidth}
%         \includegraphics[width=\textwidth]{fig/polyauto.pdf}
%         \caption{PAR: same-agent, next-timestep training.}
%         \label{fig:polyauto}
%     \end{subfigure} 
%    \captionof{figure}{Inference for (a) autoregressive (AR) models and (b) our poly-autoregressive (PAR) model. Solid indicates ground-truth tokens; striped predicted. Colors denotes agent identity. Compared to AR models, the PAR model predicts a new token at every time step, but takes other agent's tokens as inputs. 
%    }
% \label{fig:PAR_inference}
% \end{center}%
% }]
\makeatletter
\g@addto@macro\@maketitle{
    \begin{figure}[H]
    \begin{minipage}{\textwidth}
    \includegraphics[width=0.99\textwidth]{fig/PAR_figure.pdf}
    \centering
    % \vspace{-50pt}
    \captionof{figure}{
        Inference for (a) autoregressive (AR) models and (b) our proposed poly-autoregressive (PAR) model. Solid indicates ground-truth tokens which represent a tracked data modality such as action or 6DOF pose; striped represents predicted output tokens. Color denotes agent identity. Compared to AR models, the PAR model takes other agents' tokens as inputs when making a prediction for the next timestep.
    }\label{fig:polyauto}%
    \vspace{0.2cm}
    \end{minipage}
    \end{figure}
}
\makeatother

\maketitle
\renewcommand\thefigure{\arabic{figure}}
\setcounter{figure}{1}

\begin{abstract}

% Recent works to jointly reconstruct 3D human and object from a single RGB image, are mostly model-based, that fail to capture the fine details of the clothed human body and object surface. In this paper, we introduce ReCHOR, a novel, model-free, first-method to produce realistic clothed human-object reconstructions from a monocular view. This is extremely challenging due to human-object occlusions, diverse interactions and depth ambiguity, as it needs to infer both 3D spatial awareness and high resolution details. Our core idea is based on estimating neural implicit representations for human and object respectively by an attention-based neural implicit model that attends to pixel-aligned features from both the global human-object image for spatial awareness and  the local separate view of human and object images for high quality details. Additionally, the network is conditioned on semantic features from an initial estimated human-object pose prior and a generative diffusion model that inpaints occluded regions, thus enabling the retrieval of details from them.
% We also propose a synthetic dataset with rendered scenes of diverse, inter-occluded 3D human and object scans, to train our network. We evaluate our method on the synthetic and real world BEHAVE dataset. Our experiments show that our method outperforms the SOTA in achieving realistic clothed human-object reconstructions.
Recent approaches to jointly reconstruct 3D humans and objects from a single RGB image represent 3D shapes with template-based or coarse models, which fail to capture details of loose clothing on human bodies. In this paper, we introduce a novel implicit approach for jointly reconstructing realistic 3D clothed humans and objects from a monocular view. For the first time, we model both the human and the object with an implicit representation, allowing to capture more realistic details such as clothing. This task is extremely challenging due to human-object occlusions and the lack of 3D information in 2D images, often leading to poor detail reconstruction and depth ambiguity. To address these problems, we propose a novel attention-based neural implicit model that leverages image pixel alignment from both the input human-object image for a global understanding of the human-object scene and from local separate views of the human and object images to improve realism with, for example, clothing details. Additionally, the network is conditioned on semantic features derived from an estimated human-object pose prior, which provides 3D spatial information about the shared space of humans and objects. To handle human occlusion caused by objects, we use a generative diffusion model that inpaints the occluded regions, recovering otherwise lost details. For training and evaluation, we introduce a synthetic dataset featuring rendered scenes of inter-occluded 3D human scans and diverse objects. Extensive evaluation on both synthetic and real-world datasets demonstrates the superior quality of the proposed human-object reconstructions over competitive methods.
\end{abstract}  
\vspace{-12pt}
\section{Introduction}
\begin{figure}[t]
  \centering
   \includegraphics[width=1\linewidth]{sec/figs/fig1.png}
   \caption{\captionMethodFigure}
   \label{fig:methodFigure}
\end{figure}
\begin{figure}[t]
  \centering
   \includegraphics[width=1\linewidth]{sec/figs/exampleFigure.png}
   \caption{\captionExampleFigure}
   \label{fig:exampleFigure}
   \vspace{-12pt}
\end{figure}
Vision-Language Models (VLMs), such as CLIP~\cite{CLIP}, have emerged as general-purpose systems for understanding visual data through language-based queries. These models enable a broad range of applications, from object detection to image captioning, by linking visual inputs to language prompts. In standard settings where images contain single, recognizable objects, VLMs perform remarkably well. However, for the more complex task of zero-shot multi-label recognition (MLR) (Fig. \ref{fig:methodFigure} (top)), where models must identify multiple objects within an image without prior training on specific data, VLMs face significant limitations. Zero-shot MLR is crucial for applications in fields like robotics and medical imaging, where objects rarely appear in configurations that align neatly with training distributions. In these scenarios, achieving robust multi-label recognition without fine-tuning is challenging, given the task’s reliance on mean Average Precision (mAP) scores, which depend on ranking images for object presence.

\noindent \textbf{VLM: Prompt Dependent AND/OR Noisy Channel.} Despite the promise of zero-shot capabilities, current VLM approaches often struggle with MLR due to inherent scoring behaviors and biases. The performance of these models is hampered by a mix of conjunction (AND) and disjunction (OR) behaviors in their scoring, leading to inflated scores in compound prompts that contain multiple objects. For example, a prompt like “cat and sofa” might yield a high score even if only one of these objects is present in the image. This tendency reflects biases learned during training, where common object pairs receive higher scores even when only one object is present, disrupting the accuracy of mAP-based evaluations. Furthermore, existing methods for adapting VLMs to zero-shot MLR frequently rely on prompt tuning or architectural adjustments—approaches that are often dependent on training data and computationally intensive fine-tuning, which limit their generalizability to novel tasks.

\noindent \textbf{Our Approach.} In contrast to these methods, we introduce SPARC (Score Prompting and Adaptive Fusion for Zero-Shot Multi-Label Recognition in VLMs), a novel approach to zero-shot MLR that bypasses the need for training data, prompt tuning, or model-specific modifications. Our method treats the VLM as a black box, relying solely on its output scores to infer object presence (see Fig. \ref{fig:methodFigure}). This black-box approach enables us to avoid assumptions about the model’s internal workings, allowing for a purely zero-shot framework that is both model-agnostic and dataset-independent. SPARC introduces two main innovations that address the unique challenges of zero-shot MLR.

\noindent \textbf{A. Compound Prompt Composition:} Recognizing that VLMs can provide richer information when prompted with combinations of objects, we develop a method for constructing compound prompts. These prompts reflect likely contextual associations between objects, such as “cat and sofa” or “car and bus.” By gathering scores from these compound prompts, we can capture a spectrum of potential object contexts within the image, enhancing detection without relying on training-based adaptations. This composition strategy allows us to agnostically extract information from the VLM, leveraging probable object relationships without depending on any specific dataset or VLM architecture.\\
\noindent \textbf{B. Score Debiasing and Adaptive Fusion.} A critical insight in our approach lies in the surprising observation that the \underline{maximum score among compound prompts} is often a poor proxy for true object presence. Although one might expect the highest score to serve as a reliable signal, we find that it frequently reflects compositional biases, as VLMs tend to respond to compound prompts with OR-like behavior, raising scores even when only one object in the prompt is present. Instead, we observe that the second-highest score consistently provides a more accurate indicator of object presence, minimizing the effects of false positives caused by compositional bias. Building on this insight, we develop a debiasing algorithm that normalizes scores across images to address image-specific noise and clarify genuine object presence signals. This debiased score set is then processed through a PCA-based fusion method that further refines object rankings by combining information from both compound and singleton prompts, ultimately optimizing mAP by enhancing score accuracy.\\
\textbf{Complementarity.} SPARC is complementary to other zero-shot and training-free MLR methods. When applied on top of these approaches, SPARC consistently enhances mAP scores by refining object ranking and reducing bias in VLM outputs. This capability makes SPARC an adaptable solution that can improve upon existing methods while maintaining a fully zero-shot, model-agnostic framework.

\noindent \textbf{Empirical Results.} SPARC achieves significant improvements in mAP, outperforming methods that incorporate architectural modifications. This outcome shows the potential of a fully zero-shot approach that relies only on systematic prompt design and score interpretation, rather than prompt-training or fine-tuning. By revealing that the second-highest score can be a superior proxy to the maximum, our findings provide new insights into VLM scoring behavior, suggesting that careful treatment of prompt compositions and score patterns can unlock robust MLR capabilities.


\section{Related Work}
\label{sec:related_work}

The original investigation \cite{gibson1979ecological} on the relationship between visual perception and human action defines \emph{affordance} as the opportunities for interaction with the surrounding environment. Behavioral studies on regular and cognitively impaired persons have shown evidence that perception results in both visual and motor signals in the human brain. An extended study \cite{anderson2002attentional} shows that visual attention to the spatial characteristics of the perceived objects initiates automatic motor signals for different actions. In computer vision, human affordance learning involves novel pose prediction such that the estimated pose represents a valid human action within the scene context. The task is fundamental to many problems requiring robust semantic reasoning about the environment, such as human motion synthesis \cite{wang2021scene} and scene-aware human pose generation \cite{wang2017binge, roy2016multi, zhang2022inpaint, yao2023scene}.

Earlier methods of affordance learning have explored knowledge mining \cite{zhu2014reasoning} and multimodal feature cues \cite{roy2016multi} to address the problem. In \cite{zhu2014reasoning}, the authors use a Markov Logic Network for constructing a knowledge base by extracting several object attributes from different image and metadata sources, which can perform various downstream visual inference tasks without any additional classifier, including zero-shot affordance prediction. In \cite{roy2016multi}, the authors use depth map, surface normals, and segmentation map as multimodal cues to train a multi-scale convolutional neural network (CNN) for scene-level semantic label assignment associated with specific human actions. In \cite{do2018affordancenet}, the authors design a multi-branch end-to-end CNN with two separate pathways for object detection and affordance label assignment to achieve high real-time inference throughput. Researchers \cite{chuang2018learning} have also explored socially imposed constraints for affordance learning. In \cite{chuang2018learning}, the authors propose a graph neural network (GNN) to propagate contextual scene information from egocentric views for action-object affordance reasoning.

Probabilistic modeling of scene-aware human motion generation also involves semantic reasoning of human interaction with the environment. Initial works on human motion synthesis have taken different architectural approaches, such as sequence-to-sequence models \cite{barsoum2018hp}, generative adversarial networks (GAN) \cite{barsoum2018hp, cai2018deep, yang2018pose}, graph convolutional networks (GCN) \cite{yan2019convolutional}, and variational autoencoders (VAE) \cite{guo2020action2motion}. However, these methods have mostly ignored the role of environmental semantics. Due to potential uncertainty in human motion, in a recent approach \cite{wang2021scene}, the authors address such motion synthesis with a GAN conditioned on scene attributes and motion trajectory to predict probable body pose dynamics.

One key challenge of human affordance generation in 2D scenes is the lack of large-scale datasets with rich pose annotations. In \cite{wang2017binge}, the authors compile the only public dataset of annotated human body poses in complex 2D indoor scenes by extracting frames from sitcom videos. Aiming to generate a contextually valid human affordance at a user-defined location, the authors propose sampling the scale and deformation parameters for an existing human pose template using a VAE conditioned on the localized image patches as scene context. In \cite{zhang2022inpaint}, the authors introduce a two-stage GAN architecture for achieving a similar goal by estimating the affine bounding box parameters to localize a probable human in the scene and then generating a potential body pose at that location. The method uses the input scene, corresponding depth, and segmentation maps as semantic guidance. In \cite{yao2023scene}, the authors propose a transformer-based approach with knowledge distillation for generating human affordances in 2D indoor scenes.



\begin{figure*}
\centering
    \includegraphics[width=.99\textwidth]{fig/par_method.pdf}  
    \caption{\textbf{The PAR Framework}. We begin by collecting a video dataset, such as AVA (top) or DexYCB (bottom). Then, using dataset labels or computer vision techniques, a trajectory of a given modality for our prediction task is extracted for each agent, such as action class labels (top) or object pose and 3D hand translation (bottom). Data is then tokenized, either through discretization or directly using continuous values, with our framework supporting both formats. Based on the tokenization and prediction task, we choose the appropriate loss function for PAR training. After training with PAR, predicted tokens can be decoded back to data space and evaluated with relevant metrics.}
    \label{fig:PAR_pipeline}
\vspace{-1em}
\end{figure*}

\section{Poly-Autoregressive Modeling}

Our goal is to model the behavior of an agent or entity while taking into account any other agents it interacts  with, if any. To evaluate the performance of our model in capturing interaction dynamics, we predict the agent's future behavior and compare it against ground-truth data. 

We define the following task: \textit{In an interaction comprised of $N$ agents, given the observed past states of the $N-1$ interacting agents, and the observed or previously-predicted past states of the $N^{\text{th}}$ ego agent, predict the future states of the $N^{\text{th}}$  ego agent.}

We define a transformer-based poly-autoregressive (PAR) predictor, $\mathcal{P}$, that learns to model temporally long-range interactions in the input sequence. The inputs to the predictor are the past states of the $N$ interacting agents, and its output is the predicted future state of the $N^{\text{th}}$ ego agent.

\subsection{Problem Definition}
\label{sec:prob_def}

Let $\mathbf{S}=\{\mathbf{s}_i\}_{i=1}^T$ be a temporal sequence of agent states, $\mathbf{s}_i$.
We use $\mathbf{S}^N$ and $\mathbf{S}^{1:N-1}$ to denote the temporal sequences of states of the $N_{th}$ agent and of the other $N-1$ agents, respectively.
For each timestep $t \in [t_\pi,T]$, where $t_\pi \in [1,T]$ is the time we start predicting,
we take as input all other $N-1$ agents' past observed state sequences
$\mathbf{S}^{1:N-1}_{1:t-1}$
along with the $N_{th}$ agent's
past observed states up to $t_\pi$, 
$\mathbf{{S}}^N_{1:t_\pi}$,
and any of its previously predicted past states $\mathbf{\hat{S}}^N_{t_{\pi}+1:t-1}$,
if available (see Fig.~\ref{fig:polyauto}).
Our predictor, $\mathcal{P}$, then \textit{poly-autoregressively} predicts the $N_{th}$ agent's future states one time-step at a time:
\begin{equation}
       \mathbf{\hat{s}}^N_{t} = \mathcal{P}(\mathbf{S}^{1:N-1}_{1:t-1}, \mathbf{{S}}^N_{1:t_{\pi}},\mathbf{\hat{S}}^N_{t_\pi+1:t-1}). \\
\end{equation}
$\mathcal{P}$ learns to model the distribution over the next timestep of the $N_{th}$ agent's states, given all other agents' states:
\begin{equation}
p(\mathbf{\hat{s}}^N_{t} | \mathbf{S}^{1:N-1}_{1:t-1}, \mathbf{S}^N_{1:t-1}).
\end{equation}

While we provide the observed ground truth states of other agents at inference, during training, we jointly maximize the likelihood of all $N$ agents by computing losses on their future state predictions.

We train the predictor by maximizing the likelihood of the target state $y$ at time $t$:
\begin{equation*}
\label{eq:transformerloss}
    \mathscr{L_\mathcal{P}} = E_{y \sim p(y)}[-\log(p(\mathbf{s}^N_{t})],
\end{equation*}
where the target state $y$ at $t$ is computed from the $N_{th}$ agent ground truth future state.

\subsection{The Poly-Autoregressive Framework}
\label{sec:framework}

\begin{figure}
    \centering
    % \begin{subfigure}[b]{0.49\textwidth}
    %     \includegraphics[width=\textwidth]{fig/nexttoken_teacherforcing.pdf}
    %     \caption{AR: next-token training.}
    %     \label{fig:AR_nexttoken}
    % \end{subfigure}
    \begin{subfigure}[b]{0.49\textwidth}
        \includegraphics[width=\textwidth]{fig/mutli-agent_next-token.pdf}
        \caption{AR: multi-agent, next-token training.}
        \label{fig:AR_multiagent_nexttoken}
    \end{subfigure}
    ~
    \begin{subfigure}[b]{0.49\textwidth}
        \includegraphics[width=\textwidth]{fig/next_timestep.pdf}
        \caption{PAR: same-agent, next-timestep training.}
        \label{fig:PAR_nexttimestep}
    \end{subfigure}    
    \caption{Training with teacher forcing for (a) multi-agent next-token prediction in autoregressive models and (b) multi-agent poly-autoregressive models. Solid vs striped indicates a ground-truth vs predicted token, respectively. Color denotes agent identity. The AR model is trained for next-token prediction, while the PAR model is trained to predict the next timestep of the same agent. Three agents are shown for ease of visualization, but the PAR model supports an arbitrary number of agents.}
    \label{fig:PAR_training}
\vspace{-.75cm}
\end{figure}

We address the problem of forecasting the future states of an agent (from time $t$ to $T$) in a data-driven way, given a temporal sequence of past states (from time $1$ to $t-1$). 
We assume that our agent has some feature, or a set of features, of interest in a video (e.g., 3D pose) that we can tokenize. We predict the future states of the agent in terms of this tokenized feature (or set of), where we use one token (or set of tokens) per time step. The predicted tokens can be discrete (i.e., an index into a feature codebook) or continuous (i.e., a vector of one or more continuous values). The loss $\ell$ will depend on the problem's specifics and the type of token used. To train the model to predict the future, we rely on all the interaction dynamics of length $T$ in our training dataset as ground truth examples.


As a baseline, we consider the \textbf{single-agent autoregressive (AR)} paradigm, where a transformer is trained to perform next-token-prediction with teacher forcing. AR uses greedy sampling to generate sequences at inference time, predicting one next token at a time (Fig.~\ref{fig:polyauto}(a)). 

In contrast, our \textbf{multi-agent poly-autoregressive (PAR)} framework considers the other $N-1$ agents in the scene when predicting the future state of the $Nth$ agent. In this setup, we tokenize the features of interest of all $N$ agents, yielding $N$ tokens at each timestep for a total of $N*T$ tokens. In practice, we operate on a flattened sequence of $N*T$ tokens. 
% In cases where we consider a multimodal set of tokens of interest, we have $T*N*K$ where $K$ is the number of token types. \neerja{The only case where we have multiple token types is hand-object, but there we treat the hand as one agent and the object as another (or equivalently: 1 agent and 2 token types), but regardless $N*K=2$, so we haven't shown $N>1$ and $K>1$} \shiry{ok then we keep this for future ;-) }
Instead of using the AR training procedure in this multi-agent case (as in Fig.~\ref{fig:AR_multiagent_nexttoken}), we jointly model the $N$ agents at each timestep by introducing the following features to our PAR framework.

\vspace{0.2cm}
\noindent \textbf{Next-timestep prediction.} 
A standard AR model predicts the next token. Given the flattened sequence of $N*T$ tokens our model operates on, next token prediction would take as input an agent $k$ at timestep $t$ and predict agent $k+1$'s state at the same timestep $t$ (as in Fig.~\ref{fig:AR_multiagent_nexttoken}). However, our goal is to predict the input agent $k$'s future state at time $t+1$. Therefore, we perform \textit{same-agent next-timestep} prediction rather than next-token prediction (see Fig.~\ref{fig:PAR_nexttimestep} for an illustration of same-agent next-timestep at training).

\medskip \noindent \textbf{Learned agent identity embedding.} When giving a model information corresponding to multiple agents, the model can benefit from knowing which token corresponds to which agent. We give the model this information with a learned agent ID embedding. 
 


 % \medskip \noindent \textbf{Multimodality} \neerja{We have multimodality in the sense of hand and object are different modalities, and we add location to accl/velocity tokens in a pos emb - we should mention that we go beyond just one single type of token per case study, but we also do things in different ways for different case studies so not sure what is the best way to mention this here}


\medskip \noindent \textbf{Joint training.} We train the model to jointly predict the future of all agents by computing a loss on the predicted tokens of all agents (Fig.~\ref{fig:PAR_nexttimestep}). Please refer to Section~\ref{sec:prob_def} for our inference paradigm.

\subsection{Task-Specific Considerations}
\label{sec:task-specific}
Our simple PAR approach unifies diverse problems under a single framework and architecture without any modifications. In order to formulate a problem as interaction-conditioned prediction, users must consider several task-specific details. Fig.~\ref{fig:PAR_pipeline} gives an overview of how the PAR framework disentangles multi-agent learning from problem-specific modeling.

\medskip \noindent \textbf{Data.} The input data source in our example tasks is always a collection of videos. From these videos, we extract various modalities relevant to the task at hand. These modalities can range from high-level features, such as action class labels, to low-level ones, such as 3D pose (Fig.~\ref{fig:PAR_pipeline} first two columns). We assume that each agent in the dataset is detected at each frame and is associated with an agent ID.
 
\medskip \noindent \textbf{Tokenization.} Our framework supports both discrete, quantized tokens and continuous vector tokens. The choice between discrete and continuous depends on the nature of the task.  
In the case of discrete tokens, we use a standard embedding layer to project to the hidden dimension. For continuous tokens, we train a projection layer to project the token into the hidden dimension of the transformer. For instance, if our continuous token is a 3D vector with an $(x,y,z)$ 3D location coordinate and our hidden dimension is $128$, our projection layer will project from $3$ to $128$ dimensions. We also train an un-projection layer that reverts the hidden dimension to the original token dimension.

\medskip \noindent \textbf{Loss.} The type of token and task-specific considerations dictate the loss function $\ell$ applied during model training. For discrete tokens, a classification loss is appropriate. For continuous tokens, we use a regression loss on the original token dimension. 
% \shiry{there is a nuance here that we did not yet have time to explore given the last-minute nature of how this came together: the discrete classification loss allows us to predict a multinomial distribution over possible future timesteps, from which we can sample. This makes the prediction non-deterministic, capturing the true variability of the response of the $N_th$ agent to their surroundings. After all, given other people's actions, our actions are still not determined 100 percent. This is the whole beauty of using autoregressive prediction for these problems, unlike the feed-forward UNets we used in e.g. speech to getsture. This is not at all explored yet in our experiments due to lack of time mostly, and is a major hole in this submission (unlike our previous works on this direction). If a reviewer is familiar with the literature, they will complain about this (I know I would). There is nothing for us to do about this now, but this is something we should be aware of.} \neerja{Adding sampling is high on my priority list!} \shiry{yes, I would reject you without ;-) }

\medskip \noindent \textbf{Baselines.}
We compare to the following baselines, where applicable on a case-by-case basis:

\noindent $\sbullet$ \textit{Random token}: pick random tokens from the best available token space and use as the prediction. 

\noindent $\sbullet$ \textit{Random trajectory}: pick at random a trajectory from the training dataset to use as the prediction. 

\noindent $\sbullet$ \textit{NN}: Given an input agent $A$'s trajectory history, find the closest trajectory to it in the training set, belonging to $A^T$. Use $A^T$'s future as the predicted future.

\noindent $\sbullet$ \textit{Multiagent NN}: In a dataset with two interacting partners $A$ and $B$, where $B$ is the ego agent, given an input agent $A$'s trajectory history, find the closest trajectory to it in the training set, belonging to $A^T$. Use $A^T$'s interaction partner's $B^T$'s future as the prediction.

% Assume two agents, $A$ and $B$ in interaction, where we are predicting agent $B$. Take the history of agent $A^V$ in the validation set and find the closest $A^T$ in the training dataset. Use the future of the corresponding interacting agent $B_T$ in the training trajectory as the predicted future for agent $B_V$.

\smallskip
\noindent $\sbullet$ \textit{Mirror}: In a dataset with two interacting partners $A$ and $B$, use the ground truth future of agent $B$ as the predicted future for agent $A$.



\subsection{Framework Implementation Details}
We keep the following implementation details constant for all case studies (see also Sec.~\ref{sec:appendix_impl_details}).

\medskip
\noindent \textbf{Learned agent ID embedding.} Our learned agent ID embedding consists of the integer agent ID mapped to a hidden dim-sized vector, and summed to the token embedding.

\medskip \noindent \textbf{Architecture.} For all case studies, we use the  Llama~\citep{touvron2023llamaopenefficientfoundation} transformer decoder architecture with $8$ layers, $8$ attention heads, and a hidden and intermediate dimension of $128$. The decoder has $\sim$4.4M learned parameters, not including learned embedding layers which add a few thousand more parameters. A rotary positional encoding~\citep{su2024roformer} is used in addition to other summed encodings (i.e. agent ID embedding, locational positional encoding in Sec.~\ref{sec:car_trajs}). We train using teacher forcing. The only hyperparameter that changes between case studies is the learning rate.

\section{Case Study 1: Social Action Forecasting}

\begin{figure*}[t]
    \centering
    \includegraphics[width=.9\textwidth]{fig/ava_qual.pdf}
    \caption{\textbf{Action forecasting example.} The distribution over ground truth actions are in white, and our predictions in red. A 6s action history (1Hz) is input, and 6s of future actions predicted. In the scene, the man and woman alternate between talking and listening. Initially, the man is talking. The AR model predicts he will continue talking, while the 2-agent PAR model recognizes the woman is talking and predicts more accurate turn-taking behavior.}
    \label{fig:ava_qual}
    \vspace{-.6cm}
\end{figure*}


\begin{figure}
    \centering
    \includegraphics[width=\linewidth]{fig/specified_classes_comparison.pdf}
    \small
    \caption{\textbf{Per-class mAP for AVA 2-person actions}. We see performance improvement on almost all 2-person AVA action classes ((P) stands for ``a person"). Some absolute mAP gains are particularly significant: \textit{listen to} $+7.0$, \textit{kiss} $+8.3$, \textit{fight/hit} $+5.7$, \textit{talk to} $+4.4$, \textit{hug} $+5.7$, and
    \textit{hand shake} $+4.0$.}
\label{fig:ava_2_person_classes}
\vspace{-.5cm}
% \end{wrapfigure}
\end{figure}

Our first case study involves forecasting human actions. Human behaviors are fundamentally social; for instance, individuals frequently walk in groups and alternate between speaking and listening roles when conversing. Certain actions, like hugging or handshaking, are intrinsically multi-person. Therefore, modeling human interactions should help improve action forecasting performance, especially on multi-person actions, which we show in this case study.


\subsection{Experimental Setup}
\noindent \textbf{Dataset.}  The Atomic Visual Actions (AVA) dataset~\citep{gu2018ava} comprises 235 training and 64 15-minute validation videos from movies. Annotations are provided at a 1Hz frequency, detailing bounding boxes and tracks for individuals within the frame, and each person's actions within a 1-second timeframe. Individuals may engage in multiple concurrent actions from a repertoire of 60 distinct action classes (e.g., sitting and talking simultaneously). For our analysis, we select clips featuring a continuous sequence of an agent's actions spanning at least $4$s, splitting sequences exceeding $12$s.  We use the first half of each clip as history to predict the second half. For any ego agent trajectory, we pick a second agent by selecting the person present in the scene for the longest subset of the ego agent's trajectory.

% \medskip \noindent \textbf{Tokenization.}
\medskip \noindent \textbf{Task-specific considerations.} Each agent's token $\mathcal{A}$ represents an 60-dimensional vector that corresponds to the actions performed at a specific timestep. Each element denotes the probability of a particular action class being enacted; ground-truth inputs are a binary vector. We implement an embedding layer that projects these tokens into the transformer's hidden dimension, as well as an un-projection layer that reverts them back to the original 60D token space for the purposes of loss calculation and output generation. We do not explicitly require the outputs to be values between 0 and 1.
% \medskip \noindent \textbf{Loss.} 
We use a MSE regression loss on the 60D action tokens: $\mathscr{L} = \frac{1}{n} \sum_{i=1}^{n} (\mathcal{A}_i - \hat{\mathcal{A}}_i)^2$.
% \medskip \noindent \textbf{Metrics.} 
Our evaluation metric is the
% We measure 
mean average precision (mAP) on the 60 AVA classes.

We implement all baselines described in \ref{sec:task-specific}, where \textit{Random Token} corresponds to a random 60D vector sampled from 0 to 1. \textit{NN} and \textit{Multiagent NN} use Hamming distance as the distance metric.
\begin{table}
% 
\centering
\begin{tabular}{@{}ccccc@{}}
\toprule
Method & Timestep pred & Ag-ID embd & mAP $\uparrow$           \\
\midrule
1-agent AR   & N/A         & N/A        & 40.7        \\
2-agent AR   & \xmark      & \xmark     & 38.0      \\
2-agent PAR*  & \xmark     & \cmark     & 40.2         \\
2-agent PAR*   & \cmark     & \xmark   & 40.0         \\
2-agent PAR &\cmark  & \cmark & \textbf{42.6}     \\
\bottomrule
\end{tabular}
\caption{\textbf{PAR action forecasting performance on AVA} We evaluate 1 and 2-agent AR methods, two 2-agent PAR ablations (rows 3 and 4, PAR*), and our PAR method. Without next-timestep prediction (see Fig.~\ref{fig:PAR_training}) or a learned agent ID embedding, our model struggles with multi-agent reasoning, performing worse than the AR baseline. With both components, the 2-agent PAR model achieves a +1.9 mAP gain over the AR method (see Fig.~\ref{fig:ava_1_person_classes} and Fig.~\ref{fig:ava_2_person_classes} for class breakdown).}
\label{tab:ava_PAR_ablation}
\vspace{-.25cm}
\end{table}
% neerja: experiment names and wandb links
% 1-agent	https://wandb.ai/nthakkar/PAR_ava_action_prediction_post_ICLR/runs/s9sa5w5o?nw=nwusernthakkar	action_pred_ava_1_agent_lr_5e-5_ema_decay_0.999_seed_1_val_full_val
% 2-agent AR	https://wandb.ai/nthakkar/PAR_ava_action_prediction_post_ICLR/runs/nh9gvjlu?nw=nwusernthakkar	action_pred_ava_2_agent_no_ag_id_emb_no_same_ag_loss_lr_5e-5_ema_decay_0.999_seed_1_val_full_val
% 2-agent no timestep pred	https://wandb.ai/nthakkar/PAR_ava_action_prediction_post_ICLR/runs/xzlllmta	action_pred_ava_2_agent_no_same_ag_loss_lr_5e-5_ema_decay_0.999_seed_1_val_full_val_vis
% 2-agent no ag id emb	https://wandb.ai/nthakkar/PAR_ava_action_prediction_post_ICLR/runs/3uzemj2m?nw=nwusernthakkar	action_pred_ava_2_agent_no_ag_id_emb_lr_5e-5_ema_decay_0.999_seed_1_val_full_val_vis
% 2-agent PAR	https://wandb.ai/nthakkar/PAR_ava_action_prediction_post_ICLR/runs/u5yoe6vs?nw=nwusernthakkar	actual_final_ava_1hz_action_2_agent_ag_id_emb_full_val_val_full_val
% https://wandb.ai/nthakkar/PAR_ava_action_prediction/runs/ramhpxe1?nw=nwusernthakkar	
\begin{table}
% 
\centering
\begin{tabular}{@{}lcc@{}}
\toprule
Baseline & Agents &  mAP $\uparrow$           \\
\midrule
Random Token         & 1 & 3.46          \\
Random Training Traj  & 1   & 3.44          \\
Nearest Neighbor    & 1        & 13.17 \\
Multiagent NN &2      & 5.10      \\
Mirror   & 2 & 7.97         \\
\bottomrule
\end{tabular}
\caption{\textbf{AVA baselines} While the nearest neighbor baseline performs best among baselines, it is still significantly worse than the AR model.}
\label{tab:ava_baselines}
\vspace{-.5cm}
\end{table}

\subsection{Results}

We report the performance of a single-agent AR model as a baseline, in the first line of Table~\ref{tab:ava_PAR_ablation}. The AR model is significantly better than our baselines (see Table~\ref{tab:ava_baselines}), the strongest baseline being the single-agent NN. We compare these baselines to our 2-agent PAR model (last line) and various ablations where we remove the agent ID embedding and perform next-token rather than same-agent next-timestep prediction. The second line of the table corresponds to multi-agent next-token prediction (Fig.~\ref{fig:AR_multiagent_nexttoken}). We see that this approach confuses the model, and the performance is significantly worse than just training on and considering a single agent. However, as we add various components of our PAR approach, the performance improves, and with both the next timestep prediction and agent ID embedding, we get a $+1.9$ mAP gain. When only considering 2-person action classes (enumerated in Fig.~\ref{fig:ava_2_person_classes}), our mAP is  $36.3$ on the single agent PAR model and $39.8$ on the 2-agent PAR model, a \textbf{$+3.5$} mAP gain.

In Fig.~\ref{fig:ava_qual} we see an example of action forecasting. In the input history, the man talks and the woman listens. In the future, the woman talks, and the man listens. Our 2-agent PAR model (bottom row) better understands that talking and listening actions are complementary actions, while the AR model doesn't learn this correlation. We see quantitative evidence of this in Fig.~\ref{fig:ava_2_person_classes}, with per-class mAPs for our AR vs 2-agent PAR model for 2-person action classes. Here, \textit{talk to} gets a $+4.4$ mAP gain and \textit{listen to} gets a $+7.0$ mAP gain when we train a multi-agent model. We see a significant boost on many other interaction-related action classes---for instance, \textit{kiss a person} $+8.3$ and \textit{fight/hit a person} $+5.7$ mAP---and on single-person actions, see Fig.~\ref{fig:ava_1_person_classes}.










\section{Case Study 2: Multiagent Car Trajectory Prediction}
\label{sec:car_trajs}
Our second case study focuses on predicting car trajectories. Trajectory prediction requires a vehicle to be aware of other cars on the road to avoid collisions and promote cooperative behavior. This study demonstrates how our framework enables the joint modeling of multiple vehicles' movements.

\subsection{Experimental Setup}
\noindent \textbf{Dataset.} We use nuScenes~\citep{nuscenes} , inputting 2 seconds of positions to forecast vehicle positions 6 seconds ahead. Specifically, our objective is to predict the $xy$ coordinates of each agent, exclusively considering vehicles as agents. We use the \texttt{trajdata} interface~\citep{ivanovic2023trajdata} to load and visualize the data.

\noindent \textbf{Task-specific considerations.}   Instead of discretizing the $xy$ position space, we discretize the motion, resulting in discrete velocity or acceleration tokens. These integer tokens are projected to the transformer hidden dimension using the Llama token embedding layer. Inputting only these tokens results in our PAR model knowing what speed the other agents are going at, but not where they are. It is important the model has this awareness (it should know if two agents are going to collide), so our model needs to reason over this second modality of location. We implement this by passing locations relative to the agent we are predicting into a sin-cosine positional embedding (see details in Sec.~\ref{sec:appendix_car_impl_details}), which we denote a location positional encoding (LPE). The LPE is summed to our token embeddings.

We use a cross-entropy classification loss on our discrete tokens:
$\mathscr{L} = E_{y \sim p(y)}[-\log(p(\mathbf{s}^{T}_{t_{\pi}})].$  We use the standard average displacement error (ADE) and final displacement error (FDE) to evaluate our predicted trajectories. For our baselines (Sec.~\ref{sec:task-specific}), we use the closest agent at the current timestep for \textit{Multiagent NN} and \textit{Mirror}. For \textit{NN} and \textit{Multiagent NN} we use MSE as the distance metric.


\subsection{Results}
\begin{table}
% 
\centering
\begin{tabular}{@{}lcccc@{}}
\toprule
Token type & LPE & Method &  ADE $\downarrow$ & FDE $\downarrow$           \\
\midrule
Velocity  & \xmark & 1-agent AR & 1.50 & 3.64 \\
Velocity & \xmark & 3-agent PAR & 1.45 & 3.51 \\
Accleration  & \xmark   & 1-agent AR & 1.44    & 3.57  \\
Accleration  & \xmark & 3-agent PAR  & 1.40    & 3.44    \\
Accleration  & \cmark& 3-agent PAR   & \textbf{1.35} & \textbf{3.34} \\
\bottomrule
\end{tabular}
\caption{\textbf{Car trajectory prediction performance.} Using acceleration tokens and 3-agent PAR results in a stronger performance over velocity tokens and single-agent AR. Adding location via a positional encoding (LPE) further improves results. }
\label{tab:cars_locs}

% Neerja: experiment names/wandb links
% 1-agent velocity	https://wandb.ai/nthakkar/ICLR_paper_cars/runs/ccg7y19d?nw=nwusernthakkar	final_car_traj_1_agent_vel_tok_a100_full_val
% 3-agent PAR velocity	https://wandb.ai/nthakkar/ICLR_paper_cars/runs/sgi43f24	final_car_traj_3_agent_vel_tok_a100_full_val
% 1-agent acc	https://wandb.ai/nthakkar/PAR_car_traj_prediction_nuscenes/runs/k5na3q37?nw=nwusernthakkar	final_car_traj_1_agent_accl_tok_a100_full_val
% 3-agent PAR acc	https://wandb.ai/nthakkar/PAR_car_traj_prediction_nuscenes/runs/9flpua7w	final_car_traj_3_agent_accl_tok_a100_full_val
% 3-agent PAR acc + LPE	https://wandb.ai/nthakkar/PAR_car_traj_prediction_nuscenes/runs/l8fbaj7m?nw=nwusernthakkar	final_car_traj_3_agent_location_pos_embedding_sin_cos_accl_tok_relative_loc_a100_full_val

\end{table}
% 
% 
\begin{table}
\centering
\begin{tabular}{@{}lccc@{}}
\toprule
Baseline & Agents &  ADE $\downarrow$ & FDE $\downarrow$           \\
\midrule
Random Trajectory & 1  & 8.89 & 16.51      \\
NN    & 1   & 1.80 & 4.13 \\
 Multiagent NN & N & 6.40 & 12.04      \\
Mirror  & N & 11.59 & 14.93      \\
\bottomrule
\end{tabular}
\caption{\textbf{Car trajectory prediction baselines.} Nearest neighbor performs best overall, but our learned single-agent AR models outperform all baselines.}
\label{tab:cars_baselines}
\vspace{-0.25cm}
\end{table}

\begin{figure}
    \centering
    \includegraphics[width=0.48\textwidth]{fig/cars_qual.pdf}
    \caption{Example results from our single-agent AR model (top row) and three-agent PAR model with location positional encoding (bottom row) on nuScenes. The predicted agent's ground truth trajectory is in pink, and the predicted future in blue. For the PAR model, the other two agents' ground truth states are in green. Qualitatively, the PAR model handles situations where single-agent predictions might lead to collisions (A, B), uses other agents' behavior to better adhere to road areas (A, C) without environment data, and predicts based on the speed changes of other cars (D).}
    \label{fig:cars_qual}
    \vspace{-.6cm}
\end{figure}


We train AR and 3-agent PAR models using velocity tokens, acceleration tokens, and acceleration tokens combined with our location positional encoding. The results can be seen in Table~\ref{tab:cars_locs}. Note that the 3-agent PAR model uses the agent ID embedding and next timestep prediction.
Acceleration tokens consistently outperform velocity tokens both for agent AR and 3-agent PAR models. This could be because the vocabulary size for acceleration tokens is much smaller and therefore easier to optimize. Regardless, both ways of tokenizing result in models that outperform our baselines (see Table~\ref{tab:cars_baselines} - NN has a relatively low error on this dataset), and highlight that our framework is flexible such that a user can experiment with different ways of representing entities. For both token types, the 3-agent PAR model that is blind to location outperforms the AR model. While location information should help the model, it is possible that simply knowing whether other agents are slowing down or accelerating can help the model make better predictions.  When adding location information via the LPE to our 3-agent PAR model, we see another performance gain in ADE and FDE. 

Qualitative examples of the AR model (top row) and 3-agent location-aware PAR model (bottom row) can be seen in Figure~\ref{fig:cars_qual}. Our method uses no image or environment data (e.g., lanes) as input. However, by reasoning over multiple agents, its predictions lead to fewer collisions and better reasoning about speed changes and driveable areas based solely on other agents' behaviors.
% Note that our method does not take as input any pixels/image information, or any information about the environment such as lanes. We see evidence that when reasoning over multiple agents, our method is able to make predictions that result in fewer collisions, and better reasoning about changes in speed and what parts of the road are driveable, just based on the behaviour of other agents.






\section{Case Study 3: Object Pose Forecasting During Hand-Object Interaction}

Our final case study explores how hand-object interaction can be leveraged for object pose estimation. We define the human hand and the interacting object as two agents, with tokens representing distinct state types. We show that our PAR framework can jointly model these agents, improving 3D translation and rotation predictions for the object. 

\begin{figure}
    \centering
    % \includegraphics[width=\linewidth]{fig/rotation_result_2_bigger_10sr_cropped_remove_whitespace.png}
    \includegraphics[width=\linewidth]{fig/new_128dim_rot.pdf}
    \vspace{-20pt}
    \caption{
    % We present a qualitative result from the validation split for the rotation prediction task.
    \textbf{Rotation forecasting qualitative result on test set.} 3D predictions are projected onto the image, isolating rotation results by showing the ground-truth translation. Incorporating the hand agent in the PAR framework (right) improves object pose prediction over object-only AR (left). %Last frame of sequence shown.
    % The projected 3D model in blue has the ground-truth translation for visualization purposes and our predicted rotation. [say how much history there was and how far into the future]
    % To account for the low dynamics between consecutive frames, we sample every 10th frame. Left (AR), the results depict the object of interest as the sole agent, while the right (2-agent PAR) demonstrates improved performance by incorporating the human hand as a second agent in the grasping interaction. \shiry{say which frame in the future are you showing}
    }
    \label{fig:ho_qual}
    % \vspace{-.5cm}
\end{figure}

\begin{figure}
    \centering
    \includegraphics[width=\linewidth]{fig/new_128dim_tx.pdf}
    \vspace{-20pt}
    \caption{
    \textbf{Translation forecasting qualitative result on test set.} 3D predictions are projected onto the image, isolating translation results by showing the ground-truth rotation. Using the PAR framework (right) instead of AR (left) improves object pose prediction.
    }
    \label{fig:ho_qual2}
    \vspace{-.5cm}
\end{figure}

\subsection{Experimental Setup}
\noindent \textbf{Dataset.} We use the DexYCB dataset, which includes 1000 videos of 10 subjects performing object manipulation tasks with 20 distinct objects from the YCB-Video dataset. The data is split into 800 training, 40 validation, and 160 testing videos. We use one of 8 provided camera views. In each trial, subjects pick up and lift objects in randomized conditions. Labels include the object's SO(3) rotation and 3D translation, and the hand's 3D translation. We focus on predicting the object's rotation or translation.
% For this case study, we utilize the DexYCB dataset~\cite{chao2021dexycb}, which contains 1000 videos of 10 human subjects performing object manipulation tasks. Each subject picks up 20 distinct objects from the YCB-Video dataset~\cite{xiang2017posecnn}, with multiple trials conducted for each object. The dataset is divided into 800 training videos, 40 validation videos, and 160 testing videos. Although the videos are recorded from 8 RGB-D cameras, we work with a single camera view. In each trial, the subject starts in a relaxed pose with their hand by their side (often out of the camera’s view), grasps the target object, and lifts it into the air. For each subject-object pair, there are 5 trials where the object’s rotation, placement, and surrounding distractor objects are randomized. The dataset provides labels such as the object's SO(3) rotation and 3D translation, and the 3D positions of 21 hand joints in camera space. We focus on predicting either the object's rotation or translation as it is being picked up in each video.

\medskip \noindent \textbf{Task-specific considerations.} We tokenize object information in object-only experiments and both object and hand information in hand-object experiments. The object is represented as a 4D token for rotation forecasting (quaternion from SO(3) rotation) or a 3D token for translation forecasting (Euclidean coordinates). In hand-object experiments, the hand token is included with a 3D translation vector, and agent ID embeddings distinguish between the hand and object. 
Normalization is applied to all 3D translation vectors in both AR and PAR experiments; quaternions are normalized by definition and require no additional processing. 
An embedding layer projects the tokens into the transformer's hidden dimension, and another layer projects them back for prediction. 

For rotation-only forecasting, the loss is \(\mathscr{L}_{rot} = 1 - |\hat{q} \cdot q|\), where \(\hat{q}\) is the predicted quaternion and \(q\) the ground-truth quaternion. For translation-only forecasting, the loss \(\mathscr{L}_t\) is the mean squared error (MSE) between predicted and ground-truth translations. For PAR we predict relative object-to-hand translations at each frame, using the current hand position as origin, while for AR, we predict absolute object translations without considering the interacting agent. 
% For PAR, we treat the current hand translation as the origin, and predict the relative translation between the object and hand at each frame, while the hand is relative to itself at every frame. This is in contrast to AR, where we predict the absolute object translation since the interacting agent is not incorporated. 
For PAR models, we add the loss \(\mathscr{L}_h\), a MSE on hand translation. The object-only AR rotation model is optimized with \(\mathscr{L}_{rot}\), while the PAR rotation model combines \(\mathscr{L}_{rot} + \mathscr{L}_h\); similarly, the object-only translation model is trained with \(\mathscr{L}_t\), and the hand-object translation model uses \(\mathscr{L}_t + \mathscr{L}_h\). 
At inference, the first half of each video is provided, and object predictions are autoregressively generated for the second half. Translation is evaluated using MSE, while rotation is measured using geodesic distance (GEO) on SO(3).


% \medskip \noindent \textbf{Tokenization} For object-only experiments, we tokenize only the object information, while in hand-object experiments, we tokenize both the object and hand information.

% The object is represented differently depending on the task: for rotation-only prediction, we use 4-dimensional tokens derived from the quaternion representing its SO(3) rotation, while for translation-only prediction, we use 3-dimensional tokens representing the object's Euclidean coordinates. In hand-object experiments, where the hand is treated as a second agent interacting with the object, the hand is represented by a 63-dimensional vector corresponding to the Euclidean coordinates of 21 hand joints (3 joints and 1 fingertip per finger, plus 1 wrist joint). In hand-object interaction models, we also incorporate agent ID embeddings to distinguish between the hand and object.

% We use an embedding layer to project the token(s) into the transformer's hidden dimension and another to project them back into the token space for loss computation and generation. During training, we apply teacher forcing to the tokenized hand and object data. In validation, we teacher force the hand joint information while generating the object’s rotation or translation.

% %Object pose is represented as a 7D vector - 4D quaternion rotation and 3D XYZ translation. Hand pose is represented as a 63D vector of 21 XYZ joints. 

% %We learn an embedding layer to project the token into the transformers hidden dimension and another one to unproject back into the token dimension for loss computation and generation. \neerja{Since we are doing rotation and tx separately we should mention this here that we have different cases}

% %We explicitly require rotations to be valid quaternions.

% %\neerja{Explain the cases of object only, hand and object as separate tokens, and this 1.5 agent displacement thing. Currently we do loss on hand and object (but teacher force hand at inference), lay that out here.}

% \medskip \noindent \textbf{Loss} 
% For rotation-only prediction, the loss we optimize is:
% \[
%     \mathscr{L}_{rot} = 1 - |\hat{q} \cdot q|,
% \]
% where \(\hat{q}\) is the predicted quaternion representing the object's rotation in camera space, and \(q\) is the ground-truth quaternion. We apply the absolute value to account for the double-covering of quaternions in SO(3), i.e., \(q = -q\). Additionally, we ensure that the quaternions predicted by the de-projection layer are valid, meaning they have unit norm and a positive scalar component. 

% For translation-only prediction, the loss function \(\mathscr{L}_t\) is the mean squared error (MSE) between the predicted and ground-truth translations of the object. In experiments involving the hand, we also optimize for \(\mathscr{L}_h\), the MSE loss on the predicted and ground-truth hand joint positions. Additionally, we normalize the hand joints and object translations to be between 0 and 1 during training.

% The object-only rotation model is optimized with \(\mathscr{L}_{rot}\), while the hand-object rotation model uses the combined loss \(\alpha \mathscr{L}_{rot} + (1-\alpha) \mathscr{L}_h\), where $\alpha=0.33$. Similarly, the object-only translation model is trained with \(\mathscr{L}_t\), and the hand-object translation model is optimized with \(\mathscr{L}_t + \mathcal{L}_h\), where the two losses are equally weighted since they represent the same type of measurement.

% % In hand-object interaction models, we also incorporate agent ID embeddings to distinguish between the hand and object.

% \medskip

% \noindent \textbf{Metrics} During validation, we provide the first half of each video and autoregressively generate object predictions for the second half. For object translation, we evaluate performance using the MSE metric, while for object rotation, we measure error using the geodesic distance (GEO) on SO(3), which is the shortest path in radians between the predicted and ground-truth rotations. We convert the quaternions to SO(3) matrices to compute the GEO metric.

% % We use a MSE regression loss and train the model with teacher forcing - at each timestep, the previous ground-truth tokens are used in prediction. 

% % \neerja{Describe whatever metrics we settle on here - currently for rotation I think single shortest path between GT and pred rotation on the surface of the unit sphere in 3D rotation space, and MSE for translation}

% \newpage
\subsection{Results}
We compare the object-only AR models to the hand-object PAR models in Table~\ref{tab:ho_res} for the two prediction tasks. We also present the baselines described in Sec.~\ref{sec:task-specific} in Table~\ref{tab:ho_baselines}. Figures~\ref{fig:ho_qual} and~\ref{fig:ho_qual2} show qualitative results on the rotation and translation predictions, respectively. In both prediction tasks, we observe that incorporating the human hand's interaction with the object enhances accuracy: for rotation, PAR results in a relative improvement of $8.9\%$ over AR, and for translation, $41\%$. See Section~\ref{sec:app_ho_qual} for additional qualitative results with more sampled frames.
% In Figure~\ref{fig:ho_qual}, we see that the AR model (top row) achieves high-fidelity predictions early on, when much of its history still relies on ground truth data from the first half of the sequence. However, as the video progresses and the history becomes increasingly dependent on predicted object rotations, the AR model’s performance rapidly deteriorates. In contrast, our PAR model (bottom row) reasons over the 3D hand joint positions to predict the object's SO(3) rotation much more accurately. 
% Please see~\ref{app:pose} for more results on pose estimation.

\begin{table}
% \centering \footnotesize
\begin{tabular}{@{}lccc@{}}
\toprule
Type & Method &  MSE $\downarrow$ & GEO ($rad$) $\downarrow$ \\
% Type & Method &  MSE ($\times$10$^{-3}m^2$) $\downarrow$ & GEO ($rad$) $\downarrow$ \\
\midrule
Translation & 1-agent AR & 3.68 $\times$ 10$^{-3}$ & - \\
% Translation & 2-agent PAR & \textbf{3.28} & - \\
Translation & 2-agent PAR & \textbf{2.17} $\boldsymbol{\times}$ \textbf{10}$^{\boldsymbol{-3}}$ & - \\
\midrule
Rotation & 1-agent AR & - &  0.919 \\
Rotation & 2-agent PAR & - &  \textbf{0.837} \\
\bottomrule
\end{tabular}
\caption{\textbf{Test set results on DexYCB dataset.} For both rotation and translation forecasting, the 2-agent PAR model, which treats the hand as an additional agent, improves results.}
\label{tab:ho_res}
\end{table}

% \begin{table*}
% \centering \footnotesize
% \begin{tabular}{@{}lcccccc@{}}
% \toprule
% Type & Object Token & Hand Token & Ag ID Emb & Agents &  MSE ($m^2$) $\downarrow$ & GEO ($rad$) $\downarrow$           \\
% \midrule
% Translation & \cmark  & \xmark & \xmark & 1 & 2.97$\times$ 10$^{-2}$ & - \\
% Translation &  \cmark  & \cmark & \cmark & 2 & \textbf{1.90} $\boldsymbol{\times}$ \textbf{10}$^{\boldsymbol{-3}}$ & - \\
% \midrule
% Rotation & \cmark  & \xmark & \xmark & 1 & - &  0.944 \\
% Rotation & \cmark  & \cmark & \cmark & 2 & - &  \textbf{0.890} \\
% \bottomrule
% \end{tabular}
% \caption{\textbf{Test set results on DexYCB dataset.} Top two rows: translation prediction, bottom two rows: rotation prediction. In both cases, the 2-agent PAR model, which accounts hand-object interaction by integrating the hand as an additional agent, yields improved results.}
% \label{tab:ho_res}
% \end{table*}


\begin{table}
\centering \footnotesize
\begin{tabular}{@{}lcc@{}}
\toprule
Baseline &  Translation - MSE ($m^2$) $\downarrow$ & Rotation - GEO ($rad$) $\downarrow$           \\
\midrule
% Random  & 0.327 & 2.147 \\
% Random Trajectory & 1.37$\times$ 10$^{-2}$ & 2.189 \\
% NN  & 1.54$\times$ 10$^{-2}$ &  2.077\\
% Multiagent NN  & 1.31$\times$ 10$^{-2}$ & 2.268  \\
% Mirror  & 1.20$\times$ 10$^{-2}$  & N/A  \\
Random  & 0.244 & 2.196 \\
Random Trajectory & 1.60$\times$ 10$^{-2}$ & 2.146 \\
NN  & 1.69$\times$ 10$^{-2}$ &  2.179 \\
Multiagent NN  & 1.71$\times$ 10$^{-2}$ & 2.170 \\
Mirror  & 1.20$\times$ 10$^{-2}$  & -  \\
\bottomrule
\end{tabular}
\caption{\textbf{Test set results for DexYCB baselines.} We cannot provide rotation results for the Mirror baseline, because the ground-truth does not include hand rotation, only 3D translation.}
\label{tab:ho_baselines}
\end{table}









\section{Conclusion and future directions} \label{sec:conclusion}

In this paper we proposed a nested MLMC framework that offers important computational savings by performing most calculations in low precision and exploiting approximate random normal variables for the low precision path calculations. The low precision calculations could be performed in fixed precision on an FPGA for greater efficiency, and we suggested a procedure to optimise the bit-widths of every variable at each Monte Carlo level. This is an important improvement over previous mixed precision MLMC frameworks which held the lower precision fixed \cite{Rounding_error_oliver} or defined uniform bit-width at every level heuristically \cite{brugger2014mixed}. Our numerical results suggest that for the first levels our procedure reduces the cost at these levels by a factor 5 or 7. Hence the overall savings are significant since most paths are calculated on the first levels. Our approach would be even more efficient for the Milstein scheme because its higher order strong convergence leads to a greater proportion of the computational costs being on the coarsest levels.

The next stage of the research project will be to implement the RNG methods and the nested framework on FPGAs to determine the hardware requirements and confirm the extent of the computational savings. It would also be good to compare the performance benefits to using half-precision floating point arithmetic on GPUs or CPUs for the low-accuracy computations.




{
    \small
    \bibliographystyle{ieeenat_fullname}
    \bibliography{main}
}

% WARNING: do not forget to delete the supplementary pages from your submission 

\clearpage
% \setcounter{page}{1}
% \maketitlesupplementary
\begin{center}
Supplementary Material
\end{center}

% {
%     \onecolumn
%     \centering
%     \Large
%     \textbf{\thetitle}\\
%     \vspace{0.5em}Supplementary Material \\
%     \vspace{1.0em}
% }

\section{Proof of \cref{theorem:dr}}
We require some additional regularity assumptions:
\begin{assumption} 1) The number of classes $C$ is bounded w.r.t the number of samples $N$, 2) the missingness mechanism $P(A=1|Y,\theta)$, as well as its estimated counterpart $P(A=1|Y,\theta)$, are bounded below by some constant $\epsilon > 0$, 3) the quantities $P(Y|X,\theta)$ and $P(A|Y,\theta)$ are estimated using auxiliary samples independent of samples used for the sample averaging.
\label{assumption:extra}
\end{assumption}
Assumptions 1 and 2 are natural. For the missingness mechanism, the ground truth being bounded means that there is a non-vanishing proportion of samples for every class. The boundedness of the estimate can be enforced by clipping the estimate. Assumption 3 is called sample splitting in \cite{kennedy-dr}.

For convenience we use operator $\E_N$ to denote the average of $N$ samples i.e. $\frac{1}{N}\sum_{i=1}^N$. Note that this is by itself a random variable, in contrast to $\E$ which is a fixed number.

\begin{proof}[Proof of \cref{theorem:dr}] Because $C$ is bounded (assumption \ref{assumption:extra}), we can fix a class $c$ and prove the theorem.
Let us define the influence function $\phi$, parameterized by $\theta$, as
\begin{equation}
\phi(O | \theta)(c) = P(Y=c|X,\theta) + \frac{\one(A=1)}{P(A=1|Y,\theta)} (\one(Y=c) - P(Y=c|X,\theta)) - P(Y=c)
\end{equation}
As we have done in the main text, we use $\phi(O)$ to denote the same function but all estimated quantities are replaced with their truths. In other words, we use $\phi(O)$ for $\phi(O|\theta_0)$ where $\theta_0$ is the truth, given that our model contains $\theta_0$ e.g. when the model is consistent.

Recall that:
\begin{equation}
\begin{aligned}
\Psi_{dr}(\theta)(c) &= \frac{1}{N}\sum_{i=1}^N \left\{P(Y=c|X,\theta) + \frac{\one(A=1)}{P(A=1|Y,\theta)} (\one(Y=c) - P(Y=c|X,\theta))\right\}\\
&= \E_N [\phi(O|\theta)(c)] + P(Y=c)
\end{aligned}
\end{equation}

We will show that:
\begin{equation}
\Psi_{dr}(\theta)(c) - P(Y=c) = (\E_N - \E)[\phi(O)(c)] + o_P(N^{-1/2})
\label{eq:proof-linearity}
\end{equation}
To do that, we use the following decomposition
\begin{equation}
\begin{aligned}
\Psi_{dr}(\theta)(c) - P(Y=c) &= \E_N [\phi(O|\theta)(c)] \\
&= (\E_N - \E)[\phi(O)(c)] + (\E_N - \E)[\phi(O|\theta)(c) - \phi(O)(c)] + \E[\phi(O|\theta)(c)]
% &+ (\E_n - \E)[\phi(O;\theta) - \phi(O)]\\
% &+ \E[P(Y=c|X,\theta)] - \E[P(Y=c|X)] + \E[\phi(O,\theta)]
\end{aligned}
\end{equation}
and analyze the second and third term. The third term is:
\begin{equation}
\begin{aligned}
\E[\phi(O|\theta)(c)] &= \E[P(Y=c|X,\theta)] + \E\left[\frac{\one(A=1)}{P(A=1|Y,\theta)}(\one(Y=c) - P(Y=c|X,\theta))\right]- P(Y=c) \\
&= \E\left[P(Y=c|X,\theta) + \frac{P(A=1|Y)}{P(A=1|Y,\theta)}(P(Y=c|X) - P(Y=c|X,\theta))\right] - \E[P(Y=c|X)]\\
&= \E\left[(P(Y=c|X,\theta) - P(Y=c|X)) (P(A=1|Y,\theta) -P(A=1|Y)) \frac{1}{P(A=1|Y,\theta)}\right]\\
\end{aligned}
\end{equation}
by Cauchy-Schwarz inequality:
\begin{equation}
\begin{aligned}
\E[\phi(O|\theta)(c)] &\le \frac{1}{\epsilon} \|P(A=1|Y,\theta) - P(A=1|Y)\|_2 \|P(Y=c|X,\theta) - P(Y=c|X)\|_{L_2(P)}\\
&= \frac{1}{\epsilon} o_P(N^{-1/4} N^{-1/4}) = o_P(N^{-1/2})
\end{aligned}
\end{equation}
by assumption \ref{assumption:4th-root-n} and that $P(A=1|Y,\theta) > \epsilon$ (assumption \ref{assumption:extra}). The second term can be bounded by Chebyshev inequality
% \begin{equation}
% \begin{aligned}
% \E[\E_N[\phi(O|\theta)(c) - \phi(O)(c)]] &= \E[\phi(O|\theta)(c) - \phi(O)(c)]\\
% \var[\E_N[\phi(O|\theta)(c) - \phi(O)(c)]] &= \frac{1}{N}\var[\phi(O|\theta)(c) - \phi(O)(c)] \le 
% \end{aligned}
% \end{equation}
\begin{equation}
P(|(\E_N - \E)[\phi(O|\theta)(c) - \phi(O)(c)]| \ge t) \le \frac{\var[\E_N[\phi(O|\theta)(c) - \phi(O)(c)]]}{t^2} = \frac{\var[\phi(O|\theta)(c) - \phi(O)(c)]}{Nt^2}
\end{equation}
note here that $\theta$ is independent of the samples used for $\E_N$ by assumption \ref{assumption:extra}. For any $\varepsilon > 0$, by picking $t = \frac{1}{\sqrt{N\varepsilon}}$ we get
\begin{equation}
P\left(\left|\frac{(\E_N - \E)[\phi(O|\theta)(c) - \phi(O)(c)]}{N^{-1/2}}\right| \ge \frac{1}{\sqrt{\varepsilon}}\right) \le \varepsilon \var[\phi(O|\theta)(c) - \phi(O)(c)]
\end{equation}
by the definition of $O_P$, we then get
\begin{equation}
(\E_N - \E)[\phi(O|\theta)(c) - \phi(O)(c)] = O_P(N^{-1/2}\var[\phi(O|\theta)(c) - \phi(O)(c)])
\end{equation}
Because $\phi$ is a continuous function of $P(Y|X,\theta)$ and $P(A|Y,\theta)$ (given $P(A|Y,\theta) > \epsilon$, assumption \ref{assumption:extra}), by the continuous mapping theorem and the fact that $P(Y|X,\theta)$ and $P(A|Y,\theta)$ are convergent in probability (assumption \ref{assumption:4th-root-n}), we get $\var[\phi(O|\theta)(c) - \phi(O)(c)] = o_P(1)$. This gives
\begin{equation}
(\E_N - \E)[\phi(O|\theta)(c) - \phi(O)(c)] = o_P(N^{-1/2})
\end{equation}
Therefore, we have shown that the second and third term are both $o_P(N^{-1/2})$, proving \cref{eq:proof-linearity}. As the final step, multiply both sides of this equation by $\sqrt{N}$ we get:
\begin{equation}
\sqrt{N}(\Psi_{dr}(\theta)(c) - P(Y=c)) = \sqrt{N} (\E_N - \E)[\phi(O)(c)] + o_P(1) \rightsquigarrow \mathcal{N}(0, \var[\phi(O)(c)])
\end{equation}
by the central limit theorem, and $\var[\phi(O)(c)] = \E[\phi(O)(c)^2]$ because $\E[\phi(O)(c)] = 0$.
\end{proof}

While we started with the definition of $\phi$, \cref{eq:proof-linearity} shows that $\phi$ is indeed an influence function. Now we show that $\phi$ is also the efficient influence function, by using the characterization of the model's tangent space \cite{tsiatis-missingdata}. Note that the joint probability factorizes as $P(X,A,Y) = P(X)P(Y|X)P(A|Y)$, therefore the tangent space $\mathcal{T}$ factorizes as $\mathcal{T} = \mathcal{T}_{X} \oplus \mathcal{T}_{Y|X} \oplus \mathcal{T}_{A|Y}$ where $\mathcal{T}_X = \{h(X): \E[h] = 0\}$, $\mathcal{T}_{Y|X} = \{h(X,Y): \E[h|X] = 0\}$, $\mathcal{T}_{A|Y} = \{h(A,Y): \E[h|Y] = 0\}$, and the 3 subspaces are pairwise orthogonal. All influence functions are orthogonal to the tangent space, but the influence function that is also in the tangent space has the smallest variance and is called the efficient influence function. As $\phi$ is already an influence function, we need only show that $\phi$ is in $\mathcal{T}$. We write $\phi$ as
\begin{equation}
\phi(O)(c) = (P(Y=c|X) - P(Y=c)) + \left[\frac{\one(A=1)}{P(A=1|Y)} - 1\right](\one(Y=c) - P(Y=c|X)) + (\one(Y=c) - P(Y=c|X))
\end{equation}
and note that the first, second and third term are in $\mathcal{T}_X$, $\mathcal{T}_{A|Y}$ and $\mathcal{T}_{Y|X}$ respectively. Therefore, $\phi$ is indeed in $\mathcal{T}$. The efficient influence function has the smallest variance of all influence function, and therefore our estimator being asymptotically linear in $\phi$ (\cref{eq:proof-linearity}) has the smallest mean squared error in a local asymptotic minimax sense \cite{kennedy-dr, asymptoticstatistics}

\section{Further background and related work}
\paragraph{Discussion on semi-supervised EM.}
It appears that semi-supervised EM was first used for parameter estimation when the missingness mechanism is non-ignorable in \cite{ibrahim1996parameter}, but has not been used for label shift estimation.
Perhaps this is because the semi-supervised situation where additional unlabeled data is available during training is rarer than the test-time adaptation case. EM is well suited to take advantage of the extra unlabeled data to improve the classifier under very scarce and long-tailed labeled data. While the connection between pseudo-labeling and EM has been explored before \cite{entropyminimization}, the situation with label shift has not until recently \cite{simpro}. Here the application of EM is much more interesting, because other than simply giving pseudo-labeling a rigorous formulation, EM also estimates the missingness mechanism (equivalently the label distribution shift), which is important for shift correction and thus high-quality pseudo-labels \cite{acr}. The application of confidence thresholding can be seen as a sparse variant of EM \cite{neal1998view}.

\paragraph{The doubly-robust risk.} 
\label{subsec:dr-risk}
A technique that also derives from the theory of semi-parametric efficiency is orthogonal statistical learning \citep{foster2023orthogonal}. The idea is to minimize the doubly-robust risk:
\label{subsec:method-dr-risk}
\begin{equation}
\label{eq:dr-risk}
\mathcal{R}(\theta_2) = \frac{1}{N} \sum_{i=1}^N \Bigg[ l(x_i, \hat y_i|\theta_2) + \frac{\one(a_i=1)}{P(A=a_i|Y=y_i, \theta_1)} (l(x_i, y_i | \theta_2) - l(x_i, \hat y_i | \theta_2))\Bigg]
\end{equation}
where $l(x,y|\theta) = -\sum_{c=1}^C [y]_c \log P(Y=c|X=x,\theta)$ is the negative cross-entropy. 
The notation $[y]_c$ means that we are using the $c$-entry in a C-dimension probability vector $y$. 
Thus, $y_i$ denotes the one-hot label of observation $i$, while $\hat y_i$ denotes the pseudo-label, which can be one-hot or all-zero. 
Finally, we use $\theta_1$ to denote that $P(a|y,\theta_1)$ is an estimation from a previous stage, but it can be estimated with $\theta_2$ as well. 
The risk $\mathcal{R}(\theta_2)$ can be used as a training loss in a straightforward fashion. 
Similar to the doubly robust estimation of $P(Y)$, the doubly robust risk provides approximately unbiased estimation of the risk. 
This property has been used in \citep{arelabelsinformative, onnonrandommissinglabels, drst} also in the semi-supervised learning setting.
More broadly, it is at the heart of one of the core techniques in heterogenous treatment effect estimation in causal estimation \cite{kennedy2023towards, foster2023orthogonal, wager2018estimation}. 
The focus here is not the estimation of $\mathcal{R}(\theta_2)$ per se, but the quality of the learned model \cite{foster2023orthogonal}.
By using the doubly-robust risk, we can achieve an optimality result similar in spirit to our theorem \cref{theorem:dr}, but for the generalization error.
While this is appealing, in practice there are 2 problems with this approach. First, the inverse probability weight $P(A=a_i|Y=y_i,\theta_1)$ can be very large if the class ratio is highly unlabeled, making training unstable \cite{kallus2020deepmatch, pham2023stable}. 
This problem exists for our estimation as well. However, it is much easier to control for estimation than for training because of the iterative nature of model update. Secondly, we can further write $\mathcal{R}$ as:
\begin{equation}
\mathcal{R}(\theta_2) = \frac{1}{N}\sum_{i=1}^N l\left(x_i, \hat y_i + \frac{\one(a_i=1)}{P(A=a_i|Y=y_i,\theta_1)} (y_i - \hat y_i)\Bigg\vert\theta_2\right)
\end{equation}
which is a cross-entropy loss with new meta-pseudo-labels. However, these labels are not meant to be learned exactly, and furthermore they can be negative. Thus, theoretical works have to put stringent assumptions on the models. In \cref{subsec:ablation-1}, we show that experimentally that the instability problem makes doubly-robust risk performance worse than our 2-stage approach.

\section{Training and hyperparameter settings.}
\label{subsec:training-setting}
For neural network training, we follow the implementation and hyperparameter settings of \cite{simpro}. In particular, we adapt the core code of SimPro for Supervised, MLE and EM. For MLE, we update $P(A|Y)$ using the Adam optimizer with learning rate 1e-3, while for EM we use a momentum update similar to SimPro's update of $P(Y|A)$ because it has a a closed-form solution at each mini-batch. We use Wide ResNet-28-2 on all methods and all datasets in this section, including Imagenet-127, because we are motivated by the fact that stage-1's goal is not classification accuracy but the estimation of a finite-dimensional parameter. When using Wide ResNet-28-2 for Imagenet-127, we use the hyperparameters of CIFAR-100, except we lower the batch size of unlabeled data to 2 times that of labeled data instead of 8 for memory reason. We do not perform additional hyperparameter tuning. All experiments can be performed on 1 A6000 RTX GPU, and are run 3 times. We report the total variation distance between the estimated and the ground truth unlabeled class distribution, similar to its usage in Theorem 3.1 of \cite{lsc}, and the top-1 classification accuracy.

In the second stage of our algorithm, we freeze our estimation and plug it in SimPro and BOAT.
We keep exactly the same hyperparameter settings that SimPro and BOAT use. In particular, for Imagenet-127, we now use ResNet-50 and run each experiment once.
In SimPro, we set the unlabeled class distribution $P(Y|A=0)$ at the E-step;  however, we still keep a running estimate of the class distribution $P(Y)$ in the logit adjustment loss \cref{eq:simpro-la-loss}. While it is possible to use the first stage estimate in the logit adjustment loss, we observe that doing so results in lower accuracy than using the the running average. This is conceptually consistent with the role of the running average - serving not as an accurate estimate of $P(Y)$ but to make the classifier's class distribution uniform through the logit adjustment loss, which is good for the test set. Similarly, in BOAT, we only replace $\Delta_c = \log P(Y|A=1) - \log P(Y|A=0)$ in equation (4) of \cite{boat}, which is adjusting a classifier's predictions from the labeled to the unlabeled class distribution, with our SimPro + DR estimate instead of their on-the-fly estimate. 


% \section{Additional experiments}
% % \begin{table*}[t]
\centering
\caption{Total Variation Distance on CIFAR-10-LT ($N_l = 500$, $M_l = 4000$) with different class imbalance ratios $\gamma_l$ and $\gamma_u$ under five different unlabeled class distributions.}
\label{tab:cifar10-tv}
\resizebox{\textwidth}{!}{
\begin{tabular}{lccccccccccc}
\toprule
& & \multicolumn{2}{c}{consistent} & \multicolumn{2}{c}{uniform} & \multicolumn{2}{c}{reversed} & \multicolumn{2}{c}{middle} & \multicolumn{2}{c}{head-tail} \\
\cmidrule(lr){3-4} \cmidrule(lr){5-6} \cmidrule(lr){7-8} \cmidrule(lr){9-10} \cmidrule(lr){11-12}
& & $\gamma_l = 150$ & $\gamma_l = 100$ & $\gamma_l = 150$ & $\gamma_l = 100$ & $\gamma_l = 150$ & $\gamma_l = 100$ & $\gamma_l = 150$ & $\gamma_l = 100$ & $\gamma_l = 150$ & $\gamma_l = 100$ \\
Model & Estimator & $\gamma_u = 150$ & $\gamma_u = 100$ & $\gamma_u = 1$ & $\gamma_u = 1$ & $\gamma_u = 1/150$ & $\gamma_u = 1/100$ & $\gamma_u = 150$ & $\gamma_u = 100$ & $\gamma_u = 150$ & $\gamma_u = 100$ \\
\midrule
Supervised & MLLS & 0.269 ± 0.252 & 0.038 ± 0.006 & 0.251 ± 0.046 & 0.255 ± 0.060 & 0.429 ± 0.028 & 0.493 ± 0.050 & 0.333 ± 0.042 & 0.320 ± 0.009 & 0.457 ± 0.034 & 0.444 ± 0.043 \\
Supervised & RLLS & 0.043 ± 0.001 & 0.044 ± 0.010 & 0.348 ± 0.034 & 0.305 ± 0.068 & 0.769 ± 0.016 & 0.678 ± 0.028 & 0.430 ± 0.008 & 0.368 ± 0.013 & 0.539 ± 0.018 & 0.503 ± 0.020 \\
\midrule
MLE & IPW & 0.027 ± 0.001 & 0.027 ± 0.000 & 0.319 ± 0.072 & 0.243 ± 0.010 & 0.674 ± 0.020 & 0.646 ± 0.041 & 0.438 ± 0.020 & 0.454 ± 0.026 & 0.547 ± 0.049 & 0.491 ± 0.059 \\
MLE & OR & 0.045 ± 0.004 & 0.042 ± 0.000 & 0.215 ± 0.026 & 0.203 ± 0.032 & 0.433 ± 0.017 & 0.395 ± 0.033 & 0.193 ± 0.006 & 0.209 ± 0.037 & 0.307 ± 0.147 & 0.249 ± 0.130 \\
MLE & DR & 0.090 ± 0.002 & 0.079 ± 0.000 & 0.407 ± 0.027 & 0.360 ± 0.007 & 0.425 ± 0.007 & 0.421 ± 0.029 & 0.256 ± 0.001 & 0.286 ± 0.031 & 0.435 ± 0.136 & 0.362 ± 0.122 \\
\midrule
EM & IPW & 0.035 ± 0.002 & 0.040 ± 0.001 & 0.021 ± 0.001 & 0.029 ± 0.015 & 0.303 ± 0.187 & 0.091 ± 0.010 & 0.119 ± 0.011 & 0.105 ± 0.022 & 0.104 ± 0.026 & 0.104 ± 0.051 \\
EM & OR & 0.037 ± 0.003 & 0.042 ± 0.002 & 0.016 ± 0.001 & 0.024 ± 0.012 & 0.269 ± 0.183 & 0.090 ± 0.008 & 0.122 ± 0.012 & 0.103 ± 0.022 & 0.072 ± 0.012 & 0.073 ± 0.024 \\
EM & DR & 0.034 ± 0.004 & 0.037 ± 0.001 & 0.014 ± 0.001 & 0.027 ± 0.020 & 0.264 ± 0.191 & 0.092 ± 0.005 & 0.111 ± 0.019 & 0.097 ± 0.026 & 0.077 ± 0.016 & 0.073 ± 0.028 \\
\midrule
SimPro & IPW & 0.070 ± 0.011 & 0.058 ± 0.000 & 0.046 ± 0.001 & 0.049 ± 0.005 & 0.254 ± 0.074 & 0.223 ± 0.098 & 0.097 ± 0.025 & 0.067 ± 0.002 & 0.105 ± 0.066 & 0.110 ± 0.079 \\
SimPro & OR & 0.071 ± 0.012 & 0.058 ± 0.000 & 0.045 ± 0.001 & 0.049 ± 0.006 & 0.040 ± 0.003 & 0.059 ± 0.017 & 0.074 ± 0.006 & 0.075 ± 0.002 & 0.033 ± 0.003 & 0.033 ± 0.003 \\
SimPro & DR & 0.017 ± 0.004 & 0.026 ± 0.001 & 0.019 ± 0.002 & 0.018 ± 0.003 & 0.039 ± 0.003 & 0.058 ± 0.025 & 0.091 ± 0.007 & 0.031 ± 0.001 & 0.015 ± 0.003 & 0.019 ± 0.007 \\
\bottomrule
\end{tabular}
}
\end{table*}
% 

\begin{table*}[t]
\centering
\caption{Total Variation Distance on CIFAR-100-LT ($N_l = 50$, $M_l = 400$) with different class imbalance ratios $\gamma_l$ and $\gamma_u$ under five different unlabeled class distributions.}
\label{tab:cifar100-tv}
\resizebox{\textwidth}{!}{
\begin{tabular}{lccccccccccc}
\toprule
& & \multicolumn{2}{c}{consistent} & \multicolumn{2}{c}{uniform} & \multicolumn{2}{c}{reversed} & \multicolumn{2}{c}{middle} & \multicolumn{2}{c}{head-tail} \\
\cmidrule(lr){3-4} \cmidrule(lr){5-6} \cmidrule(lr){7-8} \cmidrule(lr){9-10} \cmidrule(lr){11-12}
& & $\gamma_l = 20$ & $\gamma_l = 10$ & $\gamma_l = 20$ & $\gamma_l = 10$ & $\gamma_l = 20$ & $\gamma_l = 10$ & $\gamma_l = 20$ & $\gamma_l = 10$ & $\gamma_l = 20$ & $\gamma_l = 10$ \\
Model & Estimator & $\gamma_u = 20$ & $\gamma_u = 10$ & $\gamma_u = 1$ & $\gamma_u = 1$ & $\gamma_u = 1/20$ & $\gamma_u = 1/10$ & $\gamma_u = 20$ & $\gamma_u = 10$ & $\gamma_u = 20$ & $\gamma_u = 10$ \\
\midrule
Supervised & MLLS & 0.707 ± 0.016 & 0.313 ± 0.100 & 0.445 ± 0.172 & 0.309 ± 0.119 & 0.383 ± 0.075 & 0.397 ± 0.006 & 0.570 ± 0.001 & 0.373 ± 0.107 & 0.543 ± 0.009 & 0.231 ± 0.057 \\
Supervised & RLLS & 0.520 ± 0.007 & 0.133 ± 0.003 & 0.337 ± 0.125 & 0.253 ± 0.082 & 0.424 ± 0.060 & 0.463 ± 0.003 & 0.454 ± 0.021 & 0.306 ± 0.074 & 0.460 ± 0.028 & 0.241 ± 0.040 \\
\midrule
MLE & IPW & 0.075 ± 0.000 & 0.071 ± 0.001 & 0.229 ± 0.001 & 0.167 ± 0.002 & 0.565 ± 0.005 & 0.443 ± 0.007 & 0.415 ± 0.000 & 0.311 ± 0.005 & 0.343 ± 0.000 & 0.280 ± 0.001 \\
MLE & OR & 0.065 ± 0.002 & 0.061 ± 0.001 & 0.200 ± 0.007 & 0.143 ± 0.001 & 0.526 ± 0.011 & 0.399 ± 0.023 & 0.360 ± 0.003 & 0.256 ± 0.012 & 0.328 ± 0.003 & 0.266 ± 0.005 \\
MLE & DR & 0.149 ± 0.019 & 0.145 ± 0.010 & 0.243 ± 0.004 & 0.214 ± 0.019 & 0.568 ± 0.005 & 0.464 ± 0.014 & 0.403 ± 0.014 & 0.309 ± 0.012 & 0.365 ± 0.007 & 0.320 ± 0.004 \\
\midrule
EM & IPW & 0.097 ± 0.008 & 0.092 ± 0.004 & 0.239 ± 0.007 & 0.179 ± 0.003 & 0.478 ± 0.012 & 0.329 ± 0.020 & 0.262 ± 0.016 & 0.202 ± 0.003 & 0.312 ± 0.002 & 0.227 ± 0.001 \\
EM & OR & 0.121 ± 0.007 & 0.108 ± 0.005 & 0.261 ± 0.007 & 0.189 ± 0.004 & 0.489 ± 0.013 & 0.335 ± 0.020 & 0.274 ± 0.016 & 0.211 ± 0.004 & 0.336 ± 0.003 & 0.235 ± 0.001 \\
EM & DR & 0.125 ± 0.005 & 0.111 ± 0.004 & 0.269 ± 0.007 & 0.194 ± 0.005 & 0.497 ± 0.010 & 0.336 ± 0.024 & 0.281 ± 0.019 & 0.219 ± 0.008 & 0.336 ± 0.007 & 0.233 ± 0.004 \\
\midrule
SimPro & IPW & 0.125 ± 0.001 & 0.100 ± 0.005 & 0.166 ± 0.007 & 0.141 ± 0.009 & 0.353 ± 0.023 & 0.261 ± 0.008 & 0.202 ± 0.003 & 0.158 ± 0.005 & 0.277 ± 0.009 & 0.197 ± 0.003 \\
SimPro & OR & 0.133 ± 0.005 & 0.100 ± 0.004 & 0.160 ± 0.007 & 0.138 ± 0.010 & 0.322 ± 0.014 & 0.253 ± 0.008 & 0.202 ± 0.003 & 0.156 ± 0.005 & 0.269 ± 0.006 & 0.191 ± 0.004 \\
SimPro & DR & 0.122 ± 0.003 & 0.106 ± 0.006 & 0.188 ± 0.009 & 0.149 ± 0.006 & 0.343 ± 0.023 & 0.257 ± 0.007 & 0.219 ± 0.010 & 0.172 ± 0.002 & 0.279 ± 0.007 & 0.198 ± 0.004 \\
\bottomrule
\end{tabular}
}
\end{table*}

\end{document}
