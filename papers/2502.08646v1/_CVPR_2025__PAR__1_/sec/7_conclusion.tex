\section{Discussion}

This work introduced the Poly-Autoregressive (PAR) framework, a unifying approach to prediction for multi-agent interactions. By applying the same transformer architecture and hyperparameters across diverse tasks, including action forecasting in social settings, trajectory prediction for autonomous vehicles, and object pose forecasting during hand-object interaction, we have demonstrated the versatility and robustness of our framework. 

Our findings underscore the crucial importance of considering the influence of multiple agents in a scene for prediction tasks. By modeling interactions, we significantly improved prediction accuracy over single-agent approaches on all three problems we considered. While we achieved promising results with a simple architecture, we have only provided a starting point that can be built upon extensively. For instance, incorporating environmental context or tokenizing pixel patches, especially as a way to relax our assumption on high-quality tracking,  are avenues for further research using PAR. It would be interesting to experiment with scaling the data and model. We do not specifically consider the relative importance of neighboring agents, which is an interesting future research direction (ex. dynamic attention mechanism).

Our PAR framework provides a simple and generalizable foundation of universal building blocks, ready for extension or refinement in future tasks. The PAR framework holds potential for advancing AI systems, enhancing prediction capabilities and enabling more accurate navigation and operation in real-world, multi-agent interactions.


\section{Acknowledgements} We thank Jane Wu, Himanshu Singh, Georgios Pavlakos, and João Carreira for useful discussions and feedback. This work was supported by ONR
MURI N00014-21-1-2801 and NSF Graduate Fellowships to NT and TS.