
\subsubsection{Value: Speed}\label{value-speed}

\textit{On speed for users:}

\vspace{-.5em}
{\bf Benefit:} AI agents may help users to get more tasks done more quickly, acting as an additional helping hand for tasks that must be done. A future self-driving AI agent may make such routing decisions directly, and could coordinate with other systems for relevant updates.

{\bf Risk:} Yet they may also cause inefficiency due to issues in their actions (see \cref{value-efficiency}).

\textit{On speed of systems:}

\vspace{-.5em}
As with most systems, getting a result quickly can come at the expense of other desirable properties (such as accuracy, quality, low cost, etc.). If history sheds light on what will happen next, it may be the case in the future that slower systems will provide better results overall.

{\bf Application to agentic levels:} While it is possible that the more processes a system is able to enact, the slower a system will be--suggesting a risk of decreased speed as the agentic level increases--it is possible for fully human-controlled operations to be inefficient, and for fully system-controlled operations to be highly efficient. As such, there is not a clear direct relationship between agentic level and speed.



\subsubsection{Value: Scientific Progress}\label{value-scientific-progress}

There is currently debate about whether AI agents are a fundamental step forward in AI development at all, or a “rebranding” of technology that we have had for years--deep learning, heuristics, and pipeline systems \cite{}. Re-introducing the term “agent” as an umbrella term for modern AI systems that share common traits of operating with minimal user input is a useful way to succinctly refer to recent AI applications. However, the term carries with it connotations of freedom and agency that suggest a more fundamental change in AI technology has occurred.

All of the listed values in this section are relevant for scientific progress; most of them are provided with details of potential benefits as well as risks.