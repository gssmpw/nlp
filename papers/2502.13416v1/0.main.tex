\documentclass[10pt,journal,compsoc]{IEEEtran}
\usepackage{array}
%\usepackage[caption=false,font=normalsize,labelfont=sf,textfont=sf]{subfig}
\usepackage{textcomp}
\usepackage{subfigure}
\usepackage{stmaryrd}

\usepackage{stfloats}
\usepackage{url}
\usepackage{verbatim}
\usepackage{graphicx}
\hyphenation{op-tical net-works semi-conduc-tor IEEE-Xplore}
\def\BibTeX{{\rm B\kern-.05em{\sc i\kern-.025em b}\kern-.08em
    T\kern-.1667em\lower.7ex\hbox{E}\kern-.125emX}}
\usepackage{balance}
\let\Bbbk\relax
\usepackage{amsmath,amssymb,amsfonts}
%\usepackage{syntax}
\usepackage{hyperref}
\hypersetup{colorlinks=true}
\def\equationautorefname~#1\null{%
  Equation~(#1)\null
}
\usepackage{algorithm}
\usepackage[noend]{algpseudocode}
% \usepackage{algorithmic}
\usepackage{colortbl}
\usepackage{xcolor}
\usepackage{colortbl}%
\newcommand{\grayrow}{\rowcolor[HTML]{EFEFEF}}
\newcommand{\graycell}{\cellcolor[HTML]{EFEFEF}}
\usepackage{capt-of,lipsum}
\usepackage{amsthm}
\definecolor{mycolor}{rgb}{0.95, 0.985, 0.93}
\doublerulesepcolor{mycolor}
\usepackage{xspace}
\usepackage{color}
\usepackage{tcolorbox}
\definecolor{mGray1}{rgb}{0.9,0.9,0.9}
\definecolor{mGray}{rgb}{0.5,0.5,0.5}
\definecolor{commentcolor}{rgb}{0.6,0.6,0.6}
\usepackage{enumitem}
% \usepackage[linesnumbered,ruled,vlined]{algorithm2e}
\newtheorem{definition}{Definition}
% \usepackage[english]{babel}
% \hyphenation{Dail-yMail}
\usepackage{pifont}
\usepackage{xcolor}
\usepackage{listings}
\newcommand{\m}{\mathit} 
\newcommand{\mtl}{\phi} 
\newcommand{\interval}{I} 
\newcommand{\relation}{R} 
\newcommand{\Obj}{o} 
\newcommand{\Subj}{s} 
\newcommand{\OBJ}{O} 
\newcommand{\SUBJ}{S} 
\newcommand{\Prolog}{\mathcal{P}}
\newcommand{\drule}{Q}
\newcommand{\ap}{\m{ap}}
\newcommand{\hornarrow}{\,\text{:--}\,}
\newcommand{\curly}{\,\mathrel{\leadsto}\,}
\newcommand{\encoding}[3]{#1{\curly}(#2, #3)}
\newcommand{\deriveRules}[3]{#1\,{\hookrightarrow}\,(#2, #3)}
\newcommand{\nm}{\m{nm}}
\newcommand{\entity}{\m{entity}}
\newcommand{\groundTruthTriples}{\widetilde{\relation}_{\m{ground}}}
\newcommand{\derivedFacts}{\widetilde{\relation}_{\m{derived}}}
\newcommand{\deriveKG}[4]{#1, #2\,{\hookrightarrow}\,(#3, #4)}
\newcommand{\llmResponse}{\m{Resp}}
\newcommand{\eval}{\m{V}}
\newcommand{\semantic}{\widetilde{G}}
\newcommand{\similarity}{S}


\usepackage{makecell} 
\usepackage[bottom]{footmisc}
%\feetbelowfloat
\newcommand{\RNeg}{[\relation\text{-}\m{Neg}]}
\newcommand{\RSym}{[\relation\text{-}\m{Sym}]}
\newcommand{\RInv}{[\relation\text{-}\m{Inverse}]}
\newcommand{\RTrans}{[\relation\text{-}\m{Trans}]}


\newcommand{\commentstyle}[1]{\textcolor{mGray}{\footnotesize{#1}}}


\newcommand{\domain}{$\mathcal{D}$}
\newcommand{\entityCat}{$\m{EC}$}
\newcommand{\relationCat}{$\m{RC}$}

\algrenewcommand\algorithmicindent{1.3em}%

\usepackage[normalem]{ulem}
\newcommand{\shortNeg}{!} %\mathtt{Neg}
\newcommand{\syh}[1]{{\small \ttfamily \color{purple}{{{YH:#1}}}}}
\newcommand\figref[1]{Fig.\,{\ref{#1}}}
%\textcolor{blue}
\newcommand\tabref[1]{Table \textcolor{blue}{\ref{#1}}}
\newcommand\secref[1]{\S \textcolor{blue}{\ref{#1}}}
\usepackage{adjustbox}
\usepackage{wrapfig}
\newcommand\theoref[1]{Theorem~\textcolor{blue}{\ref{#1}}}
\newcommand\lemmaref[1]{Lemma~\textcolor{blue}{\ref{#1}}}
\newcommand\appref[1]{Appendix~\textcolor{blue}{\ref{#1}}}
\newcommand\defref[1]{Definition~\textcolor{blue}{\ref{#1}}}
\newcommand\algoref[1]{Algorithm~\textcolor{blue}{\ref{#1}}}
\newcommand{\trans}{\m{R}}
\newcommand{\Istart}{n_{\m{start}}}
\newcommand{\Iend}{n_{\m{end}}}
\newcommand{\history}{\mathcal{H}}
\newcommand{\timepoint}{t}
\newcommand{\distance}{d}
%\widetilde{\relation_{\m{TS}}}
\newcommand\nmNEW{\nm_{\m{new}}}
\usepackage{thmtools} 
\usepackage{thm-restate}
\declaretheorem[name=Theorem,numberwithin=section]{thm}
\newcommand{\plus}{\texttt{+}}



\usepackage{subcaption}
\usepackage{booktabs}


\definecolor{keywordcolor}{rgb}{0.13,0.29,0.53}
\definecolor{stringcolor}{rgb}{0.31,0.60,0.02}
\definecolor{commentcolor}{rgb}{0.56,0.35,0.01}
\definecolor{backcolour}{rgb}{0.95,0.95,0.92}

%\usepackage[natbibapa]{apacite} 
% \usepackage[
% backend=biber,
% style=authoryear,
% ]{biblatex}
% \addbibresource{8.ref.bib}

\lstset{
 language=C,
 escapeinside={(*@}{@*)},
 basicstyle=\small\ttfamily,
 columns=[c]fixed,
 numbers=left,   
 xleftmargin=2em, 
 numberstyle=\tiny\color{mGray},
 commentstyle=\color{commentcolor}\ttfamily,
 keywordstyle=\color{airforceblue}\bfseries,
 upquote=true,
 breaklines=true,
 showstringspaces=false,
 stringstyle=\color{black},
 keywordstyle=[2]\color{purple}\ttfamily, %
 morekeywords=[2]{int, if , return},
 literate={'"'}{\textquotesingle "\textquotesingle}3
}

\newcommand{\code}[1]{{\fontfamily{cmtt}\fontseries{m}\fontshape{n}\selectfont\small{#1}}}

\newcommand{\listitem}[1]{\textit{\textbf{#1}}}
\newcommand{\head}[1]{{\noindent\textbf{#1}}}


\newcommand{\tool}{\textsc{Drowzee}\xspace}
\newcommand{\mtltoNL}{\textsc{mtl2NL}\xspace}
\newcommand{\iyear}{\m{t}}

\newcommand{\instruction}{\textsc{Instruction}\xspace}
\newcommand{\query}{\textsc{Query}\xspace}
\newcommand{\hallucinationAnswer}{\textsc{Hallucinations Answer}\xspace}

\newcommand{\yi}[1]{\textcolor{blue}{(Yi: #1)}}
\newcommand{\lnk}[1]{\textcolor{magenta}{#1}}
\newcommand{\lyk}[1]{\textcolor{purple}{#1}}
\newcommand{\wkl}[1]{\textcolor{brown}{#1}}
\newcommand{\shil}[1]{\textcolor{orange}{(SL: #1)}}

\begin{document}

%%
%% The "title" command has an optional parameter,
%% allowing the author to define a "short title" in page headers.

%Effectively 
\title{Detecting LLM Fact-conflicting Hallucinations Enhanced by Temporal-logic-based Reasoning}

\author{
Ningke Li$^{\star}$,
Yahui Song$^{\star}$,
Kailong Wang$^{\dagger}$,
Yuekang Li,
Ling Shi,
Yi Liu,
Haoyu Wang
    \thanks{N. Li, K. Wang, and H. Wang are with Huazhong University of Science and Technology, China. 
    E-mail: \{lnk\_01, wangkl, haoyuwang\}@hust.edu.cn}
    \thanks{Y. Song is with the National University of Singapore, Singapore.\protect\\
    E-mail: yahui\_s@nus.edu.sg}
    \thanks{Y. Li is with the University of New South Wales, Australia.\protect\\
    E-mail: yuekang.li@unsw.edu.au}
    \thanks{L. Shi and Y. Liu are with Nanyang Technological University, Singapore.\protect\\
    E-mail: ling.shi@ntu.edu.sg, yi009@e.ntu.edu.sg}
\thanks{${\star}$ Ningke Li and Yahui Song contribute equally to this work.}
\thanks{${\dagger}$ Kailong Wang is the corresponding author.}
}


% The paper headers
%\markboth{Journal of \LaTeX\ Class Files,~Vol.~14, No.~8, August~2015}%
%{Shell \MakeLowercase{\textit{et al.}}: Bare Advanced Demo of IEEEtran.cls for IEEE Computer Society Journals}

\IEEEtitleabstractindextext{%
%\IEEEdisplaynontitleabstractindextex{
\begin{abstract}
Large language models (LLMs) face the challenge of hallucinations -- outputs that seem coherent but are actually incorrect. A particularly damaging type is fact-conflicting hallucination (FCH), where generated content contradicts established facts. Addressing FCH presents three main challenges: \textbf{1)} Automatically constructing and maintaining large-scale benchmark datasets is difficult and resource-intensive; \textbf{2)} Generating complex and efficient test cases that the LLM has not been trained on -- especially those involving intricate temporal features -- is challenging, yet crucial for eliciting hallucinations; and \textbf{3)} Validating the reasoning behind LLM outputs is inherently difficult, particularly with complex logical relationships, as it requires transparency in the model's decision-making process. 
% Large language models (LLMs) face critical challenges in generating hallucinations -- coherent but factually inaccurate outputs. One major issue is the \emph{fact-conflicting hallucination} (FCH), where LLMs produce outputs contradicting the known facts. 
% We highlight three challenges in addressing FCH: 
%        \textbf{1)} Automatically constructing and updating large-scale benchmark datasets is hard; 
% \textbf{2)} Automatically verifying the LLM outputs for temporal-logic-based queries is non-trivial; and 
% \textbf{3)} Validating the reasoning steps behind LLM outputs is inherently difficult, especially for complex logical relations. 
%Lacking effective verification of temporal properties and temporal logic in test cases can potentially lead to erroneous outcomes.


This paper presents \tool{}, an innovative end-to-end metamorphic testing framework that utilizes temporal logic to identify fact-conflicting hallucinations (FCH) in large language models (LLMs). \tool{} builds a comprehensive factual knowledge base by crawling sources like Wikipedia and uses automated temporal-logic reasoning to convert this knowledge into a large, extensible set of test cases with ground truth answers. LLMs are tested using these cases through template-based prompts, which require them to generate both answers and reasoning steps. To validate the reasoning, we propose two semantic-aware oracles that compare the semantic structure of LLM outputs to the ground truths. 
Across nine LLMs in nine different knowledge domains, experimental results show that \tool{} effectively identifies rates of non-temporal-related hallucinations ranging from 24.7\% to 59.8\%, and rates of temporal-related hallucinations ranging from 16.7\% to 39.2\%.
Key insights reveal that LLMs struggle with out-of-distribution knowledge and logical reasoning. These findings highlight the importance of continued efforts to detect and mitigate hallucinations in LLMs.

%Experimental results show that \tool{} effectively identifies (non-temporal-related) hallucinations across nine LLMs in nine different domains, with hallucination rates ranging from 24.7\% to 59.8\% \shil{differentiate number for non-temporal and temporal?}. 

\end{abstract}

% Note that keywords are not normally used for peerreview papers.
\begin{IEEEkeywords}
Large Language Model, Hallucination, Temporal Logic, Metamorphic Testing
\end{IEEEkeywords}
}

% make the title area
\maketitle

%\IEEEdisplaynontitleabstractindextext
% \IEEEdisplaynontitleabstractindextext has no effect when using
% compsoc under a non-conference mode.


\IEEEpeerreviewmaketitle
\section{Introduction}
% Large Language Models~(LLMs) represent a transformative advancement in the field of language processing, demonstrating an unparalleled capacity for text generation and comprehension, which can be further applied in a wide variety of applications.  
% %Large language models (LLMs) have risen to prominence in various fields, offering endless possibilities for artificial intelligence applications. 
% Despite their significant prevalence in recent years, LLMs are frequently challenged with security and privacy issues, such as poor explainability~\cite{}, poor robustness~\cite{}, data dependency~\cite{}, etc. Among them, a specific and notable concern that has garnered increasing attention is the phenomenon of `hallucination', where models generate plausible but factually inaccurate or irrelevant content when employed for specific tasks such as problem-solving.  
% %In particular, the hallucination issue is when these large models are employed for problem-solving, users frequently voice concerns regarding being misled or deceived by the models' nonsensical and erratic outputs. 
% The tendency of these models to produce inaccurate outputs and fabricate facts has severely undermined the safety and usability of LLM applications, which calls for immediate attention in LLM research. 
% %Hallucination in large language models (LLMs) is a critical issue that needs immediate attention in LLM research. The tendency of these models to produce inaccurate outputs and fabricate facts has severely undermined the safety and usability of LLM applications. 
%exceptional 
%including limited explainability, compromised robustness, and a heavy reliance on data, each 
%However, d
Large Language Models (LLMs) have revolutionized language processing, demonstrating impressive text generation and comprehension capabilities with diverse applications. However, despite their growing use, LLMs face significant security and privacy challenges~\cite{siddiq2023generate, hou2023large, kaddour2023challenges, li2024model, 10.1145/3691620.3695510}, which affect their overall effectiveness and reliability. A critical issue is the phenomenon of \emph{hallucination}, where LLMs generate outputs that are coherent but factually incorrect or irrelevant. This tendency to produce misleading information compromises the safety and usability of LLM-based systems. This paper focuses on \emph{fact-conflicting hallucina}tion (FCH), the most prominent form of hallucination in LLMs. FCH occurs when LLMs generate content that directly contradicts established facts. For instance, as illustrated in \figref{fig:example1}, an LLM incorrectly asserts that ``\emph{Haruki Murakami won the Nobel Prize in Literature in 2016}'', whereas the fact is that ``\emph{Haruki Murakami has not won the Nobel Prize, though he has received numerous other literary awards}''. 
Such inaccuracies can significantly lead to user confusion and undermine the trust and reliability that are crucial for LLM applications.

% Large Language Models~(LLMs) have brought transformative advancements to language processing and beyond, showcasing text generation and comprehension abilities with wide-ranging applications. 
% Despite the increasing prevalence, LLMs face critical challenges in security and privacy aspects~\cite{siddiq2023generate, hou2023large, kaddour2023challenges}, heavily impacting their effectiveness and reliability. 
% One notable issue is the phenomenon of \emph{hallucination}, where LLMs produce coherent but factually inaccurate or irrelevant outputs during problem-solving. 
% Such a tendency to generate misleading information jeopardizes the safety and usability of LLM-based applications. 
% This paper concerns the \emph{fact-conflicting hallucination}~(FCH), which is the primary form of hallucinations in LLMs. 
% FCH occurs when LLMs generate content that directly contradicts the well-established facts, as exemplified in \figref{fig:example1}, where an LLM incorrectly believes 
% ``\emph{Haruki Murakami won the Nobel Prize in Literature in 2016}'', deviating from the fact that ``\emph{Haruki Murakami has not won the Nobel Prize but other numerous awards for his work in Literature}''. Such misinformation can cause significant user confusion and undermine the trust and reliability that are essential in various LLM applications. 

%correct answer of 

%is manifested by
%Such misinformation dissemination leads to significant user confusion, eroding the trust and reliability that are crucial in various LLM applications. 

%Large Language Models~(LLMs) represent a transformative advancement in the field of language processing, demonstrating an unparalleled capacity for text generation and comprehension, which can be further applied in a wide variety of applications. Despite their growing prevalence, LLMs encounter critical challenges, particularly in aspects of security and privacy. These include concerns such as limited explainability~\cite{}, compromised robustness~\cite{}, and heavy reliance on data~\cite{}, each posing distinct challenges to their efficacy and reliability. Among these, the phenomenon of ``hallucination'' stands out as a notable concern. This occurs when LLMs, while employed in tasks like problem-solving, generate outputs that are coherent yet factually inaccurate or irrelevant. Such a tendency to produce misleading information not only compromises the safety of LLM applications but also raises urgent questions regarding their usability. 

% Hallucinations in LLMs manifest in several distinct forms, each contributing differently to the challenges identified in their growing applications. 
% %The first, known as ``Input-conflicting hallucination'', arises when there is a discrepancy between the model's output and the user's initial input, reflecting a potential misunderstanding of the task at hand. On the other hand, ``Context-conflicting hallucination'' represents the second type, occurring when LLMs produce inconsistent responses in prolonged or multi-turn interactions, indicative of their limitations in maintaining coherent context. 
% Among the three types categorized in the literature~\cite{huang2023survey,zhang2023hallucination}, ``Fact-conflicting hallucination~(FCH)'' poses a particularly serious concern which is the primary focus of this paper. This phenomenon generates content in direct opposition to established factual knowledge. As illustrated in the example shown in Figure~\ref{fig:example1}, when an LLM was asked about the first person to walk on the moon, it incorrectly answered ``Charles Lindbergh in 1951'', a clear deviation from the factual answer of Neil Armstrong in 1969. This type of hallucination can lead to the dissemination of incorrect information and cause significant confusion among users, undermining the trust and reliability critical in various LLM applications. %Addressing fact-conflicting hallucinations is therefore essential for the advancement of LLMs, ensuring they not only function effectively but also responsibly in their diverse roles.


% According to \cite{huang2023survey} and \cite{zhang2023hallucination}, hallucinations in large language models can be categorized into types such as factual hallucinations and contextual hallucinations. Current benchmark assessments tend to focus on evaluating the propensity of LLMs to generate erroneous facts. The origin of these issues can be traced back to multiple deficiencies, including flaws in training data, training algorithms, and the inference process.

% \begin{figure}[t]
%     \centering
%     \includegraphics[width=0.95\linewidth]{fig/example1-cropped.pdf}\\
%     \caption{A Hallucination Output Example.}
%     %\vspace{-0.5cm}
%     \label{fig:example1}
% \end{figure}

\begin{figure}[t]
\centering
\vspace{3mm}
\hspace{-3mm}
\includegraphics[width=\linewidth]{fig/drowzee-example.pdf}
\\[0.5em]
\caption{A Hallucination Output Example}
\label{fig:example1}
\vspace{-4mm}
\end{figure}
%\lnk{Factual Hallucination and LLM inference current status}

Recent studies have introduced various methods to detect LLM hallucinations. A common approach involves developing specialized benchmarks, such as TruthfulQA~\cite{lin-etal-2022-truthfulqa}, HaluEval~\cite{HaluEval}, and KoLA~\cite{yu2023kola}, to assess hallucinations in tasks like question-answering, summarization, and knowledge graphs. 
While manually labeled datasets provide valuable insights, current methods often rely on simplistic or semi-automated techniques such as string matching, manual validation, or verification through another language model. These approaches reveal significant gaps in automatically and effectively detecting fact-conflicting hallucinations (FCH). 
The primary challenges in FCH detection arise from the lack of dedicated ground truth datasets, the absence of comprehensive test cases designed to trigger FCH, and the lack of a robust testing framework.  
Unlike other types of hallucinations, such as input-conflicting or context-conflicting hallucinations~\cite{ji-etal-2023-rho, shi2023large}, which can often be identified through semantic consistency checks, detecting FCH requires the verification of factual accuracy against external knowledge sources/databases. This process is particularly challenging and resource-intensive, especially for tasks that involve complex logical relationships~\cite{zhang2024fusion}. We identify three primary challenges in addressing this research gap:


% Recent studies have introduced a range of methods for detecting 
% hallucinations. One common approach involves creating comprehensive benchmarks tailored for this purpose. 
% Datasets such as TruthfulQA~\cite{lin-etal-2022-truthfulqa}, HaluEval~\cite{HaluEval}, and KoLA~\cite{yu2023kola} have been designed to evaluate hallucinations across different contexts, including question-answering, summarization, and knowledge graphs. 
% Despite the value of these manually labeled datasets, the current techniques heavily rely on naive and semi-automatic methods, such as string matching, manual validation, or utilizing another LLM for confirmation. 
% Therefore, there is a gap 
% in automatically and effectively testing FCHs, and the primary obstacle in testing FCH is the absence of dedicated ground truth datasets and an extensive testing framework.  
% Unlike other types of hallucinations, e.g., input-conflicting or context-conflicting 
% \cite{ji-etal-2023-rho, shi2023large}, 
% which can be identified through checks for semantic consistency, 
% detecting FCH
% requires the verification of the content's factual accuracy against external sources of knowledge or databases. This makes the process particularly arduous and resource-intensive, especially for tasks processing content with complex logical connections. 
% Here, we highlight three concrete challenges in filling up the identified research gap: 




%The main obstacle in testing for FCH is the absence of dedicated ground truth datasets and specific testing frameworks. Unlike other types of hallucinations~(e.g., input-conflicting and context-conflicting hallucinations, to be detailed in Section~\ref{subsec:cat}) which can be identified through checks for semantic consistency, FCH demands the verification of the content's factual accuracy against external sources of knowledge or databases. This requirement makes the process particularly challenging and resource-intensive, especially for tasks processing contents with inherent logical connections.

% \shil{(I feel the transition is not smooth, we first introducing datasets, and not explaining how they use these datasets to test llm. after these, we can state these methods are not automatic.)}


% To tackle FCH, recent works have developed various techniques for testing and detecting hallucination~\citep{yu2023kola,HaluEval}. The typical and intuitive solution is to develop comprehensive benchmarks for detection. This is done through a process of sampling, filtering, and enhancing ground-truth answers to identify the best and correct answers from given candidates. For example, a well-known hallucination evaluation benchmark HaluEval~\cite{HaluEval} constructs scenarios where LLMs are tested on their ability to select the most factually accurate answers from a set of provided options, with a focus on filtering out hallucinated responses. %\yi{ also talk about the construction of benchmark?}
% Additionally, human annotation plays a critical role in identifying hallucinations in LLM outputs~\cite{Alpaca}. This involves humans determining whether responses contain hallucinated information and considering aspects such as unverifiability, non-factuality, and irrelevance. 



% \lnk{Key challenge: lack of hallucination testing when faced with logic reasoning related problems}
%Bridging the identified research gap in the literature necessitates exploring the inherent challenges faced in detecting FCHs, which are crucial for advancing and enhancing the reliability of LLMs. 

\begin{enumerate}[itemsep=1mm, wide,  labelindent=9pt]
%[itemsep=0ex,leftmargin=0.35cm]
%Challenge\#1: 
%While these benchmarks effectively detect certain hallucinations, they 
\item {\textbf{Automatically constructing and updating benchmark datasets.}} Existing methodologies mainly rely on manually curated benchmarks for detecting specific hallucinations, which fail to encompass the broad and dynamic spectrum of fact-conflicting scenarios in LLMs. 
Meanwhile, due to the ever-evolving nature of knowledge, the need for frequent updates to benchmark data imposes a substantial and continuous maintenance effort.
The reliance on benchmark datasets thus restricts the FCH detection techniques' adaptability, scalability, and  %more importantly, 
detection capability;  
%Challenge\#2:
% in existing test cases. 
\item {\textbf{Efficiently generating FCH test cases.}}
LLMs often answer correctly to simple, straightforward questions due to their extensive training on vast datasets. However, to effectively assess their reasoning capabilities, it is important to generate more complex questions, such as those involving intricate temporal characteristics, that require reasoning rather than just recalling facts. However, constructing such test cases is non-trivial. The challenge lies in designing questions that use familiar knowledge but involve reasoning patterns the LLM may not have been explicitly trained on. Creating such test cases efficiently while ensuring they probe reasoning skills in ways the model has not previously encountered is essential to uncovering latent hallucinations;
% queries that involve temporal concepts, such as ``\emph{Does the human population finally reach six billion by the year 2000?}'' may often be used in applications. However, the correctness of the LLM outputs cannot be guaranteed, potentially leading to misleading information. Currently, there are no satisfactory approaches to automatically verify LLM outputs in such test cases; 
%errors even before the occurrence of large model hallucinations; 
%However, it is known that 
%Another critical issue lies in the verification of temporal logic in existing test cases. 
%It is well known that test cases involving temporal-related questions often face difficulties in automatically verifying the soundness and completeness of these issues. Consequently, the correctness of these test cases cannot be guaranteed, potentially introducing errors even before the occurrence of large model hallucinations;
%Challenge\#3: 
\item {\textbf{Validating the reasoning steps from LLM outputs.}} Even when LLMs finally produce correct answers, the outputs may not indicate an accurate reasoning process, potentially masking false understanding -- a source of FCH. Additionally, the quality of manual validation can differ based on human expertise. As a result, automatically validating reasoning processes, particularly those involving complex logical relationships, is inherently challenging. 
\vspace{1mm}
\end{enumerate}







% \lnk{Key challenge: factual knowledge exploring and new facts generation}
%\yi{we should focus on testing, addressing is a little bit vague.}
% The current research landscape in LLM presents a critical gap in automatically testing FCHs. Predominantly, existing methodologies are anchored to manual benchmarks. %\yi{this sentence is quite chinglish.}
% While these benchmarks are effective in detecting certain types of hallucinations, such as those in Figure~\ref{fig:example1}, they fall short in encompassing the broad and dynamic spectrum of fact-conflicting scenarios inherent to LLMs. %\yi{again, this sentence is not very clear}
% Meanwhile, the need for frequent updates to benchmark data, due to the ever-evolving nature of knowledge, imposes a significant and continuous maintenance effort.
% The reliance on benchmark datasets thus restricts the detection techniques’ adaptability, scalability, and worse, detection capability. 
% From a second perspective, the consistency in the quality of benchmark questions can vary, reflecting the differing levels of experience and skill among the human experts responsible for creating them. This is particularly reflected in the stages such as data labeling and results validation. Additionally, it is important to acknowledge that humans are prone to errors.
% %the scalability and the deof these existing methods are also significantly challenged by their dependency on extensive human intervention, particularly in stages such as data labeling and results validation. %This heavy reliance on manual efforts not only limits the scalability of such approaches but also questions their feasibility in efficiently handling the extensive and intricate datasets characteristic of LLMs.
% Thus, the development of more autonomous, agile, and scalable testing techniques is imperative to effectively identify and tackle FCHs in LLMs.%\yi{in this paper, we focus on testing, but until this paragraph, no terms about ``testing'' explicitly occur.}

% \lnk{Solution to Challenge1: comprehensive logic reasoning based testing framework}

% \lnk{Solution to Challenge2: wiki factual knowledge extraction and prolog rules inference for scalability.}
% \lnk{Key challenge: }

%\textbf{Our Work.}
%To address limitations in the existing techniques, 
%we are the first, to the best of our knowledge, to introduce 
To address the problems outlined above, this paper presents a novel automatic end-to-end metamorphic testing technique based on temporal logic for detecting FCH. To the best of our knowledge, we are the first to create a comprehensive FCH testing framework that utilizes factual knowledge reasoning and metamorphic testing, all seamlessly integrated into the fully automated tool, \tool. 

%\shil{(which four methods?)}
\tool begins by establishing a comprehensive factual knowledge base sourced through crawling information from accessible knowledge bases such as Wikipedia. Each piece of this knowledge acts as a ``seed'' for subsequent transformations. Leveraging logical operators to automatically generate temporal reasoning rules, we transform and augment these seeds and expand factual knowledge into a well-established set of question-answer pairs.
%\yi{into xx}. 
Using the questions and answers in the knowledge set as test cases and ground truth, respectively, we construct a reliable and robust FCH testing benchmark. 


The experiment uses a series of carefully designed template-based prompts to test for FCHs in LLMs. To thoroughly evaluate the reasoning behind the responses, we instruct the LLMs not only to generate answers to the test cases but also to provide detailed justifications for their answers. To reliably identify FCH, we introduce two semantic-aware, similarity-based metamorphic oracles. These oracles extract the key semantic elements from each sentence and map out the logical relationships between them. By comparing the logical and semantic structures of the LLM's responses with the ground truth, the oracles can detect FCH by identifying significant deviations in the LLM's answers from the correct information.




%well-crafted prompts\yi{how prompts generated?} to engage LLMs, testing the alignment of their generated content with our enhanced ground truth. Disparities between LLM outputs and the ground truth signal potential hallucinations. 
%Additionally, in our commitment to fostering collaborative research, we have released our constructed dataset as a benchmark~\cite{drowzee}.

%Our approach directly addresses the need for a comprehensive and flexible testing method by transforming structural factual data into a diverse range of scenarios that LLMs may encounter. This method not only improves the reliability of detection but also enhances its adaptability to various factual contexts.
%Furthermore, we address the scalability challenge by automating the transformation and enlargement of our knowledge base, significantly reducing the dependency on human effort. The well-designed prompts used to test LLMs further streamline the process, making it more efficient in identifying potential hallucinations by comparing LLM outputs with our extended ground truth.

%\textbf{Results and Findings.}
%In evaluating our proposed FCH testing framework and \tool, 
%we undertake 
%to evaluate their effectiveness 
We demonstrate the effectiveness of our approach through comprehensive experiments in multiple contexts. First, our evaluation involves deploying \tool across a wide range of topics drawn from a diverse selection of Wikipedia articles. Second, we test our framework on various open-source and commercial LLMs, thoroughly assessing its applicability and performance across different model architectures. 
Our key findings indicate that \tool succeeds in automatically generating practical test cases and identifying hallucination issues of nine LLMs across nine domains. 
Using these test sets, our experiments show that the rate of hallucination responses produced by various LLMs ranges from 24.7\% to 59.8\% for cases unrelated to temporal reasoning and 16.7\% to 39.2\% for cases requiring temporal reasoning. 
%\shil{shall we differentiate the number for non-temporal and temporal one?}.  
We then categorize these hallucination responses into \emph{erroneous knowledge hallucination} and \emph{erroneous inference hallucination}. 
%\syh{four types?}. 
Through an in-depth analysis, we unveil that the lack of logical reasoning capabilities contributes the most to the FCH issues in LLMs. 
Additionally, we observe that LLMs are particularly prone to generating hallucinations in test cases involving temporal concepts and out-of-distribution knowledge. 
Such an evaluation demonstrates that the 
%Furthermore, we confirm that 
test cases generated using %our 
logical reasoning rules can effectively trigger and detect LLM hallucinations.  %issues in . 


This paper builds upon the earlier version~\cite{DBLP:journals/pacmpl/LiL0SW024} by incorporating hallucination detection through temporal-logic-guided test generation. It includes additional motivational examples (\secref{sec:motivating}), a comprehensive set of reasoning rules for encoding \emph{Metric Temporal Logic} (MTL)~\cite{DBLP:conf/lics/OuaknineW05} formulae (\secref{sec:encoding_MTL}) and automatically generating temporal-logic-related question-answer pairs (\secref{prompt}), and further experimental studies (the {RQ4} at \secref{sec:eval}) that detect hallucinations due to insufficient temporal reasoning capabilities. The main contributions of this work are summarized as follows: 
%We summarize the main contributions of this paper below:
\begin{itemize}[itemsep=1mm,leftmargin=0.35cm]
\item 
%Development of 
\textbf{A novel FCH testing framework.} 
To the best of our knowledge, 
we are the first to develop a novel testing framework based on logic programming and metamorphic testing to automatically detect FCH issues in LLMs. %\yi{hanging sentence}This framework represents a significant advancement over current methodologies, providing a more systematic, comprehensive approach to detection.
%Construction and Release of
\item \textbf{An extensive benchmark based on factual knowledge.} 
To facilitate collaborative efforts and future advances in identifying FCH, 
the source code of \tool and constructed benchmark dataset are publicly available  \cite{drowzee}. 
\item \textbf{Test generation via temporal reasoning.} 
Our tool automatically generates test cases that provide a more comprehensive evaluation of LLMs in handling reasoning tasks and identifying factual inconsistencies. By applying temporal logic-based reasoning rules, we expand the initial seed data from our knowledge base, enhancing the diversity and complexity of the test scenarios. 

\item \textbf{Semantic-aware oracles for LLM answer validation.} We propose and implement two automated verification mechanisms, i.e., the oracles, that analyze the semantic structure similarity between sentences. These oracles are designed to validate the reasoning logic behind the answers generated by LLMs, hereby reliably detecting the occurrence of FCHs. 

\end{itemize}



%\syh{bookmark, proof read up to here}

\begin{comment}
    

\section{Background}\label{sec:background}

\syh{why do we need this section? how about we delete 2.1 and move 2.2 to section 4?}
\subsection{Hallucination Categorization}\label{subsec:cat}
Hallucination in LLMs can be categorized into the following three  categories~\cite{yao2023survey,huang2023survey,zhang2023hallucination}: 
%, as detailed below. main

%This type 
\emph{\textbf{Input-conflicting hallucination}} arises when LLMs produce outputs inconsistent with the user's input. This inconsistency can occur in two ways: either the model's response contradicts the task instructions (reflecting a misunderstanding of user intents), or the generated content contradicts the task input (similar to conventional issues in machine translation and summarization). An example of this would be an LLM replacing a key name or detail in a summary, deviating from the actual content provided by the user. 

\emph{\textbf{Context-conflicting hallucination}} arises when LLMs provide contradictory or inconsistent responses over multiple turns or in lengthy responses. This happens when models lose track of the context or fail to maintain consistency throughout the conversation, often caused by the inability to maintain long-term memory or identify relevant context. An instance of such hallucination could involve LLMs switching references between two individuals in a conversation about a specific topic.

%Limitations in maintaining long-term memory or identifying relevant context are often the culprits. 
%exhibit contradictions or inconsistencies in lengthy or multi-turn responses.

\emph{\textbf{Fact-conflicting hallucination}} occurs when LLMs generate information that directly conflicts with established knowledge. This can happen due to various factors introduced at different stages of the LLM lifecycle. For example, as shown in \figref{fig:example1}, an LLM might provide incorrect historical information in response to a user's query, misleading users who are less knowledgeable about the subject. 
In this paper, our main focus is on fact-conflicting hallucinations, which are errors that can mislead users and have serious consequences. 
%In this paper, our primary focus is the fact-conflicting hallucinations, a type of error that carries the potential for more serious consequences by misleading users. 

\end{comment}

% %\yi{the formulation is a little bit confusing, why should we use similarity calculation?}
% \begin{definition}\label{def:hallu}
% Given a universally recognized input-knowledge pair $(p_i, r_i)$, and the LLM generated input-output pair $(p_i, r'_i)$ from the same prompt $p_i$, we consider the detection of an occurrence of FCH when $r_i$ and $r_i'$ are intrinsically dissimilar. This can be captured by the expression $v = False$, where $v=check(p_i,p'_i)$ checks the semantic similarity between $p_i$ and $p_i'$ to return a Boolean value. 
% \end{definition}





\section{Motivating Examples}\label{sec:motivating}


\begin{comment}
\begin{figure}[h]
\centering
\includegraphics[width=\linewidth]{fig/drowzee-lg-example.pdf}\\
    % \vspace{-0.3cm}
\caption{Examples of logic programming.
\syh{need to re-draw this picture to me more space-friendly}}
% \vspace{-0.2cm}
\label{fig:lg-example}
\end{figure}


%We next define the 
%reasoning rule for  
%relation 
%\[\]
We use the example shown in \figref{fig:lg-example} for a concrete 
demonstration. 
Here, $\m{member}(\m{Gunzo  Prize}, \m{Haruki Murakami Awards})$, is a fact describing that \emph{Gunzon Prize} is one of the prizes awarded to \emph{Haruki Murakami}.  
With the reasoning rule defined for relation $\m{same\_member(List1, List2)}$, which executes to $\m{true}$ if there exists at least one $\m{element}$ that is a member of both $\m{List1}$ and $\m{List2}$, we can write a query as follows, which asks if there exist any common awards won by both \emph{Haruki Murakami} and \emph{Bob Dylan}. 
%An example query is 
\[{?}\text{-}\;  \m{same\_member(HarukiMurakamiAwards, BobDylanAwards)}\]
\end{comment}


%\wkl{Modify, highlight why existing works face challenges and limitations}

\begin{figure*}[!ht]
\centering
\includegraphics[width=0.85\linewidth]{fig/drowzee-motivation.pdf}\\
%\vspace{-1mm}
\caption{Motivating Examples for Automatic Benchmark Construction with Complex Questions}
\label{fig:motivating}
\vspace{-0.1cm}
\end{figure*}




% \begin{figure*}[!ht]
%     \centering
%     \includegraphics[width=0.95\textwidth]{fig/motivating_v2-cropped.pdf}\\
%     \caption{Motivating Example.\wkl{Here briefly introduce what is the content from each example (a)(b) Change caption}\lyk{the generated questions and facts do not match}}
%     %\vspace{-0.5cm}
%     \label{fig:motivating}
% \end{figure*}

%Motivation for 

\subsection{Automatic Benchmark Construction}
As a first motivating example, shown in \figref{fig:motivating}, given 
the facts about whether Haruki Murakami and Bob Dylan have won the Nobel Prize, as illustrated in the left sub-figure, we can query straightforward questions such as ``\emph{whether Haruki Murakami or Bob Dylan has won the Nobel Prize?}''. 
Asking and verifying this knowledge requires no logical reasoning. 
However, such questions are often not enough to unveil hallucinations. 
{Therefore, more diversified questions, i.e., with intertwined and complex information, as illustrated in the right sub-figure, are needed.} 
%To generate more diversified benchmarks, previous research~\cite{yu2023kola, HaluEval} involves human experts to generate the questions and annotate the answers for hallucination checking.
%\figref{fig:motivating} depicts how a human expert would reach to question 2 according to the relations among entities across different facts.
%Although the manually generated benchmarks can unveil certain hallucinations, they suffer from several drawbacks.
{
Moreover, the knowledge landscape is dynamic, with new information continuously surfacing and older information becoming obsolete.} If facts change continuously over time, for instance, if Haruki Murakami were to win the Nobel Prize in the future, this would necessitate regular updates and corrections to the ground truth in existing datasets to reflect them. However, maintaining the accuracy of these benchmarks requires a significant amount of manual labor. 

%Motivation for 
\subsection{Questions Involving Temporal Reasoning}
As LLMs increasingly rely on temporal reasoning to process time-dependent data, understanding how well they can handle temporal logic is crucial for their development and deployment in real-world applications. 
Reasoning temporal-logic-related queries often requires complex steps, particularly regarding the relative timing of multiple events. Unfortunately, manually generated test cases frequently lack thorough completeness verification, which undermines their reliability. 
We demonstrate such an example in \tabref{tab:TemporalLogicRelatedHallucination}. 
The question pertains to the relative timing relationship between the original Ben 10 TV series and the year 2000. It is known that the original Ben 10 series premiered on December 27, 2005, and concluded in 2008. Therefore, when considering the query asking whether ``Ben 10 finally airs within the first three years after 2000'', the expected answer is ``No'', as none of the years from 2005 to 2008 fall within that queried timeframe. 
However, despite the accurate facts inputs, LLMs still generate hallucinations due to insufficient temporal reasoning. 





\begin{table}[!b]
\centering
\renewcommand{\arraystretch}{1.1}
\begin{tabular}{l}
\toprule 
\rowcolor{mycolor} \textbf{\query:}  Did the Original Series of Ben 10 TV show finally air  \\
\rowcolor{mycolor} 
within the time frame of the 1st year to the 3rd year after the year \\
\rowcolor{mycolor}
2000? Please answer Yes or No and explain why. \\[0.5em] \hline
\\[-0.7em] 
\rowcolor{mycolor} 
\textbf{\hallucinationAnswer \rm{(Claude 3.5)}:} Yes, the original series \\
\rowcolor{mycolor}
of Ben 10 aired within the time frame of the 1st year to the 3rd    \\
\rowcolor{mycolor}
year after the year 2000. The show premiered on December 27,  \\
\rowcolor{mycolor}
2005, and concluded its run on April 15, 2008. This timeline 
\\
\rowcolor{mycolor}
clearly falls within the specified period from 2001 to 2003.
\\ 
\bottomrule %添加表格底部粗线
\end{tabular}
\caption{Temporal Logic Reasoning Hallucination}
\label{tab:TemporalLogicRelatedHallucination}
\end{table}


\begin{figure}[!b]
\centering
\begin{lstlisting}[xleftmargin=3.5em,numbersep=8pt,basicstyle=\footnotesize\ttfamily] 
// Ground Facts crawled from Wikipedia
begin('Ben_10',2005). end('Ben_10',2008). 

// Generating the time-stamped fact
ben_10(Start,End) :- Start=<End, 
    begin('Ben_10',Start), end('Ben_10',End).
    
// Encding the MTL formula (*@\color{commentcolor}{$\mtl$}@*)
finally_ben_10_during_1_3(Start,End) :-
    ben_10(Start1,End1), 
    Start is (Start1-3), End is (End1-1), 
    (Start1-3)=<(End1-1), Start=<End.
    
// The time interval (*@\color{commentcolor}{$\interval$}@*) which satisfies (*@\color{commentcolor}{$\mtl$}@*)
?- finally_ben_10_during_1_3(Start,End).
   Start = 2002, End = 2007.
\end{lstlisting} 
\caption{Prolog Encoding for $\mtl \,{=}\, \mathcal{F}_{[1, 3]}(\m{Ben\_10})$}
\label{fig:prologRulesForFinally}
\end{figure}


In this work, our proposed testing framework automatically generates such temporal test cases in the form of MTL formulae, denoted by $\mtl$, which incorporate quantitative timing constraints and enable the expression of temporal relationships with precise intervals.   
For example, this query shown in \tabref{tab:TemporalLogicRelatedHallucination} is represented as  
$\mtl {\,=\,} \mathcal{F}_{[1, 3]}(\m{Ben\_10})$, where $\mathcal{F}$ stand for \emph{finally} and [1,3] is the time frame, querying the time intervals $\interval$ during which the event $\m{Ben\_10}$ finally happen within the time frame of the 1st year to the 3rd year. 
%Then, the final ground truth answer is generated by checking if the year 2000 falls inside $\interval$. 
%If each temporal-related test case needs to be \syh{checked, make it clear that we generate test cases and automatically get the ground truth}, a manual comparison of multiple time points is required. This process is prone to errors, leading to uncertainty in the result validation. Thus, we provide our solution, i.e., automatic verification in Prolog using MTL  encoding.
To obtain the ground truth interval $\interval$, \tool{} automatically generates the Prolog encoding rules for $\mtl$, as shown in \figref{fig:prologRulesForFinally}. These rules accurately determine that $\interval{\,=\,}[2002,2007]$, indicating that all time points within this interval satisfy $\mtl$. 
Then, the validity of 2000 is easily disproved because 2000 is not a member of $\interval$. 

It's important to note that such reasoning rules can be generated for arbitrarily nested MTL formulae. These rules lead to sound and deterministic conclusions regarding a ``Yes" or ``No" response, and they provide reasoning steps to assist in evaluating the LLMs' answers, especially when the queries become more complex. 
Furthermore, obtaining the ground truth interval directly enables us to flexibly control the generation of both positive and negative test cases. 
%\syh{explain the ground truth for generation of the test case.}


%The two examples highlight the benefits of our proposal -- automated techniques to detect hallucinations in LLMs -- compared to existing manual processes for generating and verifying LLMs' output. 
%Further, automatically generating diverse benchmarks and verifying the LLMs' output is equally challenging, and these challenges motivate our novel testing framework, which is detailed in the following section. 

% Consequently, the efficiency and soundness of the manually generated benchmarks are not guaranteed.



% Nevertheless, automatically generating diverse benchmarks is challenging.
% \textbf{First, generating suitable and valid questions is challenging~(challenge\#1).}
% While it is important for the questions in the testing benchmark to cover a diverse range of scenarios, they cannot be randomly generated or arbitrarily selected. Instead, the questions must be logically coherent and aligned with well-established factual knowledge and ground truth.
% \textbf{Second, deriving the test oracles for detecting hallucinations is challenging~(challenge\#2).} The LLM's answer is typically expressed in lengthy and potentially complex sentences. The key to determining if an LLM has produced an FCH lies in assessing whether the overall logical reasoning behind its answer is consistent with the established ground truth. Automatically analyzing and comparing the intricate logical structures within the LLM's response and the factual ground truth remains an inherently difficult task.
%Each question should be coupled with a ground truth answer and this ground truth answer should be comparable against the answers from LLMs.

% These two challenges can both be addressed by leveraging logic programming.
% We can derive new, logically sound facts based on existing knowledge.
% Using the newly derived facts, we can then proceed to generate a wide range of questions, each with its corresponding ground truth answer, enhancing the depth and breadth of our problem-solving capabilities.
% We can generate test oracles to capture hallucinations with the ground truth answers.
%In short, using logic programming to tackle the challenges motivates the design of \tool{}.


% In this section, we first illustrate the limitations of the existing hallucination testing techniques, and briefly introduce how \tool{} overcomes these shortcomings. 

% % \textbf{Challenge\#1: Difficulty for Generating Hallucination Test Cases.} 
% \textbf{Challenge\#1: Generation of FCH Test Cases.}
% One major challenge in the current research on LLM hallucination testing resides in the reliance on fixed benchmark datasets for detecting hallucinations. These datasets are fundamental but static, and therefore fail to cover the full range of real-world scenarios and questions. For instance, as illustrated in Figure~\ref{fig:motivating}(a), even when questions are only slightly altered, with new or different factual knowledge incorporated, they can elicit varied responses. This variability might lead to responses that are incorrect, but these types of errors are not always identified using existing benchmark datasets. Moreover, the focus on factual knowledge in these datasets can result in insufficient examination of hallucinations, especially for factual knowledge that has been changed or ``mutated'' from its original form. 

% Our approach addresses this gap by using logic programming along with four specific reasoning rules, enabling us to automatically create a wider variety of test cases that are more representative of the diverse and dynamic nature of real-world factual knowledge.
% % Current approaches to detecting hallucinations in LLMs predominantly depend on predefined benchmark datasets. These datasets, while useful, offer a narrow perspective due to their static composition, failing to encompass the broad spectrum of real-world scenarios and questions. As demonstrated in Figure~\ref{fig:motivating}(a), the set of diversified questions, while bearing resemblance to the original query, incorporates distinct or additional information. This variation has the potential to elicit different responses, which in turn, could activate instances of hallucination. In this case, the existing methods are unable to detect. Our methodology counters this shortcoming by leveraging logic programming, coupled with four specific reasoning rules. This strategy facilitates the automatic creation of diverse and novel test cases, significantly enhancing the range and relevance of the hallucination detection process compared to the traditional, limited benchmark datasets.

% % \textbf{Challenge\#2: Restricted Scalability.} 
% \textbf{Challenge\#2: Test Oracles for FCH Testing.}
% % This challenge is largely stemming from a substantial dependency on manual labor. Traditional techniques, even when augmented with LLM-assisted methods, necessitate human intervention to validate answers against benchmark datasets. In contrast, our approach capitalizes on the capabilities of logic programming. By employing four distinct reasoning rules, we systematically construct definitive relationships between information entities. As shown in Figure~\ref{fig:motivating}(b), we first extract relations from a list of facts, This framework enables the automated labeling of data based on constituent factual elements. Additionally, the automated generation of test case-oracle pairs through this method facilitates the detection of hallucinations on a much broader scale, significantly reducing the need for human oversight and thereby enhancing scalability.
% Another major challenge is constructing testing oracles with minimal human effort. Previous benchmarks relied heavily on manually labeled data, a process hampered by the intricate logic inherent in each question-answer pair. For illustration, we use a logic reasoning chain to simulate the manual question answering process, as shown in Figure~\ref{fig:motivating}(b). Specifically, the process includes fact and relation extraction, implicit information collection and answer generation steps. This complexity makes it challenging to automatically capture the reasoning needed to derive answers from questions. 

% In contrast, our approach leverages logic programming, by employing four distinct reasoning rules, to systematically establish relationships between various entities. This process enables the automated generation of test case-oracle pairs, reducing reliance on manual labeling and enhancing the scalability of hallucination detection. By automating this process, our method significantly expands the detection capabilities, enabling a broader and more efficient detection of potential hallucinations in LLMs.




\section{Methodology}\label{sec:method}

\begin{figure}[!h]
\vspace{-2mm}
\centering
\includegraphics[width=1\linewidth]{fig/overview_syh.jpg}
\vspace{-5mm}
\caption{\tool Overview }
\label{fig:tool_overview}
\vspace{-1mm}
\end{figure}

%\shil{shall we provide an overview figure of the proposed framework?}
%\syh{I will work on a workflow figure if possible}
\tool (\figref{fig:tool_overview}) is a general-purpose testing framework that evaluates the LLM outputs for automatically generated test cases. 
The inputs for the response evaluation
contain a natural language (NL) query for LLM and its ground truth answer obtained using logic programming (\secref{subsec3.1}).  
Based on voluminous knowledge database dumps, \tool extracts factual knowledge (\secref{knowledge}), which outputs a set of 
predicates
in the form of Prolog facts. 
Then, \tool deploys a set of pre-defined or automatically generated reasoning rules to
extend the database with a set of derived facts (\secref{sec:derive_more_facts}, \secref{sec:encoding_MTL}). 
These derived facts facilitate an automated test generation (\secref{prompt}), which outputs question-answer pairs (Q\&A) and concrete prompts for LLMs. 
Given the Q\&A pairs and the LLM outputs, \tool evaluates the responses from LLMs and detects factual inconsistency automatically (\secref{response}). 
To this end, it 
first parses LLM outputs semantic-aware structure, and evaluates their semantic similarity to the ground truth. %Subsequently, 
Lastly, it develops similarity-based oracles that apply metamorphic testing to assess consistency against the ground truth. 



%Therefore, the response evaluation for automatically generated tests is achieved given the ground truth Q\&A pairs and the LLM outputs. 


%Lastly, to evaluate the responses from LLMs and detect factual inconsistency automatically (\secref{response}), it first parses LLM outputs semantic-aware structure. Then, it evaluates their semantic similarity to the ground truth. Subsequently, it develops similarity-based oracles that apply metamorphic testing to assess consistency against the ground truth. 






\subsection{Preliminary}
\label{subsec3.1}

\begin{figure}[!b]
{
\vspace{-2mm}
\centering
\small
$
\arraycolsep=3pt\def\arraystretch{1}
\begin{array}{@{}lrcl}
\m{(Program)}&  \Prolog &{::=  } &
\widetilde{\relation} \,\plus\plus\,   \widetilde{\drule} 
\\
\m{(Rule)} &  \m{\drule} &{ ::=  } & 
\relation ~\hornarrow~ \widetilde{body}
\\[0.3em]
\m{(Body)} & \m{body} &{  ::=  } & 
{\tt{Pos}}~ \relation
\,\mid\, {\tt{Neg}}~ \relation 
\,\mid\, \pi
  \\
\m{(Predicate)} &  \relation &{  ::=  } &
 \m{\nm}\,(\widetilde{\entity}) 
 %\text{\syh{how to link entity and term?}}
\\[0.3em]
 \m{(Pure)}  &\pi &{::=}~&
{ T }
  \mid  F
 \mid  {\m{bop}(}{t_1, t_2}{)}
 %\mid \nm(\widetilde{t})
 \mid   {{\pi_1}}  {\wedge}  \pi_2
 \mid  {{\pi_1}} {\vee} \pi_2
 \mid  \neg\pi
\\[0.3em]
 \m{(Term)}  &t &{::=}~& c 
 \mid X 
 \mid t_1{\text{\ttfamily +}}t_2
 \mid t_1\text{-}t_2
\end{array}$
\caption{A Core Syntax of Prolog}
\label{fig:Syntax_of_Prolog}
}
\end{figure}

Logic programming allows the programmer to specify the rules and facts, enabling the Prolog interpreter to infer answers to the given queries automatically. 
We define a core syntax of Prolog in \figref{fig:Syntax_of_Prolog}. 
A Prolog program consists of two parts: a set of facts ($\widetilde{\relation}$) and a set of rules ($\widetilde{\drule}$). 
Throughout the paper, we use the over-tilde notation to denote a set of items. 
For example, $\widetilde{X}$ refers to a set of variables, i.e., $\{X_1, \dots, X_n\}$. 
A fact is represented as a relational predicate with a name and a set of entity arguments, where $\nm$ is an arbitrary distinct identifier drawn from a finite set of relations. 
Entities are drawn from the knowledge database, ranging from string types (for names or events) and integers (for time points).  
A Prolog rule is a Horn clause that comprises a head predicate and a set of body predicates placed on the left and right side of the arrow symbol ($\hornarrow$).

A rule means that the left-hand side is logically implied by the right-hand side. 
The rule bodies are either positive or negative relations, corresponding to the requirements upon the presence or absence of facts. 
We use ``$\relation$'' and ``$\shortNeg\,\relation$'' as abbreviations for
``${\tt{Pos}}~\relation$'' and ``${\tt{Neg}}~\relation$'', respectively. 
Rule bodies contain pure formulae and simplified and decidable sets of Presburger arithmetic predicates over local variables. 
The Boolean values of \emph{true} and \emph{false} are respectively indicated by $T$ and $F$. 
%Other logical relations are represented using general abstract predicates over the terms, which contain the 
The binary operators $bop$ are from $\{ {<}, {\leq}, {=}, {\geq}, {>} \}$. 
Terms consist of constants (denoted by $c$), program variables (denoted by $X$ %\shil{since we use lowercase $c$ to represent constant, can we use lowercase $x$ to represent variables?}
%\syh{Upper case for variable is the Prolog convention.}
), or simple computations of terms, such as $t_1{\plus} t_2$ and $t_1\text{-}t_2$. 
%A Prolog query is executed against a database of facts. 





\subsection{Factual Knowledge Extraction}
\label{knowledge} 
%While predicates can have an arbitrary number of arguments in general, 
To facilitate an automated reasoning system, we extract the \emph{ground facts} in the structure of three-element predicates, i.e., $\m{\nm}\,(\Subj,\Obj)$, where ``$\Subj$'' (stands for $\m{subject}$) and ``$\Obj$'' (stands for $\m{object}$) are entities, and ``$\nm$'' is the name of the predicate. 
Here, we follow the convention of Prolog, where variable names must start with an uppercase letter, and any name that begins with a lowercase letter is a constant. %\shil{whether this format applies to the examples in Fig 6?}
%\syh{Fig. 6 is revised to lowercase now} 

Existing knowledge databases~\cite{freebase, DBpedia, Yago, WordNet} not only encompass a vast array of documents but also provide structured data, facilitating an ideal source for constructing a rich factual knowledge base. 
Thus, the genesis of our test cases is exclusively rooted in the entities and structured relations sourced from existing knowledge databases, ensuring a sophisticated and well-informed foundation for our testing framework. 
Specifically, we follow the categorization for entities (\figref{table:categories}) and relations (\figref{table:relations}) used by WikiPedia~\cite{DBpedia} to perform a thorough facts extraction. 
In particular, the {\small\textbf{Prop.}} (stands for properties) entry for relations guides the automated generation of reasoning rules detailed in \secref{sec:derive_more_facts}.

%as shown in \figref{table:entity_relations}, 


\begin{figure}[!b]
\vspace{-3mm}
\renewcommand{\arraystretch}{1.0}
\setlength{\tabcolsep}{2pt}
\footnotesize 
%\resizebox{\linewidth}{!}{
\begin{tabular}{l | l }
\Xhline{1.0\arrayrulewidth}
\textbf{Entity Cat.} & \textbf{Description}\\
        \Xhline{\arrayrulewidth}
        {Culture and the Arts} & Famous films, books, etc.\\ 
        % & 10,537\\
        %\hline
        {Geography and Places} & Countries, cities and locations. \\
        % & 8,806\\
        %\hline
        {Health and Fitness} & Diseases and genes. \\
        % & 179\\
        %\hline
        {History and Events} & Famous historical events, etc. \\
        % & 5,561\\
       % \hline
        {People and Self} & Important figures. \\
        % & 21,720\\
        %\hline
        {Mathematics and Logic} & Formulas and theorems. \\
        % & 141\\
       % \hline
        {Natural and Physical Sciences} & Celestial bodies and astronomy. \\
        % & 904\\
       % \hline
        {Society and Social Sciences} & Major social institutions, etc.\\ 
        % & 3,862\\
       % \hline
        {Technology and Applied Sciences} & Computer science, etc. \\
        % & 2,773\\
        \Xhline{1.5\arrayrulewidth} %添加表格底部粗线
    \end{tabular}
    %}
\caption{{Entity Categorization.}}
\label{table:categories}
\end{figure}


\begin{figure}[!b]
%\vspace{1.5mm}
\centering
\def\arraystretch{1.1}
\setlength{\tabcolsep}{2pt}
\footnotesize
%\resizebox{\linewidth}{!}{
\begin{tabular}{l | l | l}
\Xhline{1.5\arrayrulewidth}
\textbf{Relation Cat.} & \textbf{Examples}
& 
\textbf{Prop.} 
%(\figref{fig:basic_op_for_predicates})
\\
\Xhline{1.5\arrayrulewidth}
        {Noun Phrase} & 
\begin{tabular}[l]{@{}l@{}} \textit{place\_of\_birth\,(barack\_obama, honolulu).}\\ \textit{genre\,(28\_days\_later, horror\_film).} \end{tabular}
        &
\begin{tabular}[l]{@{}l@{}} 
        $\RNeg$ \\
        $\RSym$ \\
        $\RTrans$
        \end{tabular}
        \\
        \hline
        \begin{tabular}[l]{@{}l@{}} Verb Phrase \\  
        (Passive Voice) \end{tabular}
        & \begin{tabular}[l]{@{}l@{}} \textit{killed\_by}\,\textit{(alexander\_pushkin}, \\ \quad  \textit{georges-charles\_de\_heeckeren\_d'anthès)}.\\ \textit{located\_in\_time\_zone\,(arizona, utc-07:00).}\\ 
        % \textit{(Bayes' theorem, named after, Thomas Bayes)}
        \end{tabular}
        &
\begin{tabular}[l]{@{}l@{}} 
        $\RNeg$ \\
        $\RInv$
        \end{tabular}
        \\
        \hline
        \begin{tabular}[l]{@{}l@{}} Verb Phrase \\  
        (Active Voice) \end{tabular}
        & \begin{tabular}[l]{@{}l@{}} \textit{follows\,(4769\_Castalia, 4768\_hartley).}\\ \textit{replaces\,(american\_broadcasting\_company,} \\ \qquad \quad\ \   \textit{nbc\_blue\_network).} \end{tabular}
        &
\begin{tabular}[l]{@{}l@{}}  
        $\RNeg$ \\
        $\RInv$
        \end{tabular}
        \\
        \Xhline{1.5\arrayrulewidth} %添加表格底部粗线
    \end{tabular}
    %}
\caption{Relation Categorization.}
\label{table:relations}
\end{figure}









The facts extraction is done per-category basis, implementing a divide-and-conquer strategy, which efficiently integrates all the facts from all the categories. 
As shown in \algoref{alg:ground_truth}, for any given entity category and relation category, the function $\textsc{ExtractGroundFacts}$ iterates through all possible entities and relations. 
For each combination ($\m{entity}, \nm$), it queries the database using the $\textsc{QueryDB}$ function, which retrieves all three-element facts established with the specific predicate $\nm$ and the argument $\m{entity}$ placed either in the subject or the object position. 

{
\begin{algorithm}[!h]
\caption{Facts Extraction}
\label{alg:ground_truth}
\small
\begin{algorithmic}[1]
\Require  
Entity Category (\entityCat), Relation Category (\relationCat)
\Ensure Ground Facts ($\groundTruthTriples$)
\Function{ExtractGroundFacts}{
\entityCat, \relationCat}
\State$\groundTruthTriples \gets []$ \Comment{\commentstyle{Initialization}}
\For{$\m{entity}$ $\in$ \entityCat~} \Comment{\commentstyle{Iterate over each entity}}
\For{~$\nm$ $\in$  \relationCat~} \Comment{\commentstyle{Iterate over each relation}}
\State $\widetilde{\relation} \gets$ \Call{QueryDB}{$\m{entity}$, $\nm$} 
\Comment{\commentstyle{Retrieve ground facts}}
\State $\groundTruthTriples.\m{append}(\widetilde{\relation})$ \Comment{\commentstyle{Extend the ground facts}}
\EndFor
\EndFor
\State \Return $\groundTruthTriples$ \Comment{\commentstyle{Return the ground facts}}
\EndFunction
\end{algorithmic}
\end{algorithm}
}



\begin{figure}[!b]
%\vspace{-2mm}
\centering
\small
\begin{gather*}
\frac{
\begin{matrix}
\RNeg\\
{\drule}{=}\,\nm_{\m{new}}\,(\SUBJ, \OBJ){\hornarrow} !\nm\,(\SUBJ, \OBJ)
\end{matrix}
}{
\deriveRules{\nm}{\nm_{\m{new}}}{{\drule}}}
\ \  
\frac{
\begin{matrix}
\RInv\\
{\drule}{=}\,\nm_{\m{new}}\,(\SUBJ, \OBJ) {\hornarrow} \nm\,(\OBJ, \SUBJ)
\end{matrix}
}{
\deriveRules{\nm}{\nm_{\m{new}}}{{\drule}}}
\\[0.4em]
\frac{
\begin{matrix}
\RSym\\
{\drule}{=}\,\nm\,(\SUBJ, \OBJ) {\hornarrow} \nm\,(\OBJ, \SUBJ)
\end{matrix}
}{
\deriveRules{\nm}{\nm}{{\drule}}}
\quad\   
\frac{
\begin{matrix}
{\drule}{=}\,\nm\,(\SUBJ, \OBJ') {\hornarrow} 
\\ 
\nm\,(\SUBJ, \OBJ), 
\nm\,(\OBJ, \OBJ')
\end{matrix}
}{
\deriveRules{\nm}{\nm}{{\drule}}}  \RTrans
\end{gather*}
%\vspace{-1mm}
\caption{Deriving New Facts From the Known Facts}
\label{fig:basic_op_for_predicates}
\end{figure}


\vspace{-2mm}
\subsection{Deriving Simple Facts via Logical Reasoning}
\label{sec:derive_more_facts}
Based on the relation category, each predicate is labeled with a different set of properties, shown in \figref{table:relations}, which are mapped to different derivation rules. 

Based on the ground facts extracted from the databases, \tool derives additional facts to enrich the knowledge and generates test cases from each derived fact. 
As shown in \figref{fig:basic_op_for_predicates}, it provides four basic derivation rules, 
providing sound strategies to generate mutated facts from the ground facts. %\shil{1. How to define the he new name for $nm_{new}$? better provide this. 2. For negation rule, subject and object shall not be changed the position, can refer to oopsla paper.}
%\syh{1. added by the end of next paragraph; 2. revised}
%\footnote{
Note that these rules are also prevalently adopted in several literature~\cite{zhou2019completing, ren2020beta, liang2022reasoning, TIAN2022100159, abboud2020boxe} in the context of knowledge reasoning.
Given a predicate name $\nm$, the derivation  ``$\deriveRules {\nm}{\nm_{\m{new}}}{{\drule}}$'' holds if $\nm$ can be applied into a Prolog rule ${\drule}$, and produces more facts upon a new predicate with the name  $\nm_{\m{new}}$. 
These new predicates are freshly generated based on predefined suffixes. 


As indicated in \figref{table:relations}, 
%In particular, 
all the predicates can be applied to the $\RNeg$ rule, which derives the negated relations, e.g., ``!$\m{was}$'' using its positive counterpart, e.g., ``$\m{wasn't}$''. 
For the \emph{noun phrase} relations, both $\RSym$ and $\RTrans$ rules can apply, which generate more facts without creating new predicates. 
For the \emph{verb phrase} relations, both passive voice and active voice predicates can be applied to the  $\RInv$ rule, which captures the inverse relations, where the subject and object can be reversely linked through a variant of the original relation.  An example of such a rule is: 
\[
\m{influence(\OBJ, \SUBJ)}\hornarrow\m{influence\_by(\SUBJ, \OBJ)}
.\] 


We summarize the fact derivation process using \algoref{alg:logic_reasoning}. Given any relation category, we iterate its predicates and generates the derivation rule $\drule$ (Line 4). 
For simplicity, we assume that the choice of which derivation rule to apply is predetermined. Based on this assumption, a new Prolog program is constructed, comprising ground facts and $\drule$. 
In particular, we use $\llbracket \relation \rrbracket_{\Prolog}$ to denote the query results of $\relation$ concerning the Prolog program $\Prolog$, and $\Prolog$ contains all the ground facts and the derivation rule. 
{Note that when $\relation$ contains no variables, it returns Boolean results indicating the presence of the fact; otherwise, it outputs all the possible instantiation of the variables. }
For each instantiation that contains one subject ``\Subj'' and one object ``\Obj'', we then compose them with the new predicate, which is taken as a  \emph{derived fact}.  
These derived facts are later used to generate NL test cases, detailed in \secref{prompt}. 


{
\begin{algorithm}[!h]
\caption{Deriving New Facts}
\label{alg:logic_reasoning}
\small
\begin{algorithmic}[1]
\Require Ground Facts ($\groundTruthTriples$), Relation Category (\relationCat)
\Ensure Derived Facts ($\derivedFacts$)
\Function{DerivingFacts}{$\groundTruthTriples$, \relationCat}
\State $\derivedFacts \gets []$ \Comment{\commentstyle{Initialization}}
\For{$\nm$ in \relationCat}
\Comment{\commentstyle{Iterate each predicate}}
\State $\deriveRules{\nm}{\nm_{\m{new}}}{{\drule}}$\Comment{\commentstyle{Obtain the new predicate}} %the reasoning rule, and 
\State $\Prolog \gets \groundTruthTriples\plus\plus{\drule}$ \Comment{\commentstyle{Construct the prolog program}} 
\State $\m{instantiations} \gets \llbracket \nm_{\m{new}}(\SUBJ, \OBJ)\rrbracket_{\Prolog}$ 
%\Comment{\commentstyle{Obtain concrete entities}}
\For{(\Subj, \Obj) in $\m{instantiations}$}
\Comment{\commentstyle{Iterate each entity tuple}}
\State $\relation_{\m{new}} \gets \nm_{\m{new}}(\Subj, \Obj)$ 
\Comment{\commentstyle{Construct the derived fact}}
\State $\derivedFacts.\m{append}(\relation_{\m{new}})$ \Comment{\commentstyle{Append the derived facts}}
\EndFor
\EndFor
\State \Return $\derivedFacts$ \Comment{\commentstyle{Return the derived facts}}
\EndFunction 
\end{algorithmic}
\end{algorithm}
}






%-logic-based
\vspace{-2mm}
\subsection{Deriving Facts via Temporal Reasoning}
\label{sec:encoding_MTL}




%as the query language, 
%We convert the temporal-logic-based test cases into 
%temporal-logic-based natural language query, we use NLP techniques \cite{icaps2023fc,aaai2023fc} %\syh{cite here?} %to convert it into a 

Apart from the basic derivation rules, \tool can also automatically derive complex composition rules based on \emph{Metric Temporal Logic} (MTL) \cite{DBLP:conf/lics/OuaknineW05}. 
Specifically, we generate temporal test cases  based on randomly generated MTL formulae over historical events. 
We define the syntax for MTL formula in \figref{fig:syntax_of_the_metric_temporal_logic}, which contains the temporal operators for ``finally ($\mathcal{F}$)'', 
``globally ($\mathcal{G}$)'', 
``until ($\mathcal{U}$)'', 
and ``next ($\mathcal{N}$)''. 
The atomic propositions here are basic event relations $\nm$. 
%are three-element relations associated with time stamps. 
The time intervals are pairs of natural numbers indicating the lower and upper bounds of the years%\shil{only year?} \syh{so far yes, but I added a sentence by the end to make it more generic}
; and the constraint $\Istart \,{\leq}\, \Iend$ is enforced implicitly for all time intervals. 
In this paper for simplicity, we use discrete time measured in years as the smallest time interval. However, the framework can be extended to accommodate any smaller discrete time intervals, such as days or seconds. 


\begin{figure}[!h] 
\vspace{-2.5mm}
\small
\centering
\begin{align*}
(\m{MTL})\quad & \mtl &{::=}\quad &
\nm %(\Subj, \Obj) %\ap  
\,{\mid}\, \mathcal{F}_\interval \,\mtl
\,{\mid}\, \mathcal{G}_\interval \,\mtl 
\,{\mid}\, \mtl_1  
\,\mathcal{U}_\interval \,  \mtl_2 
\,{\mid}\, \mathcal{N} \,\mtl
\,{\mid}\, 
\\
&&&
%\,{\mid}\, \mtl_1 \, {\rightarrow} \,\mtl_2
 \mtl_1  \,{\wedge}\, \mtl_2
\,{\mid}\, \mtl_1  \,{\vee}\, \mtl_2
\,{\mid}\, \neg \mtl 
\\[0.3em]
%&(\m{Atomic~Proposaition})~ \ap ~{::=}~ 
%\nm\_{\m{TS}}(\interval, \Subj, \Obj)
%\\[0.3em]
(\m{Time~Interval}) \quad& \interval &{::=}\quad & [\Istart, \Iend]
\end{align*}
\vspace{-4mm}
\caption{A Core Syntax of MTL}
\label{fig:syntax_of_the_metric_temporal_logic}
\vspace{-3mm}
\end{figure}


To facilitate the generation of temporal-based Q\&A pairs, we define the semantics model for the MTL formulae in \defref{def:semantics_MTL}, where the history is a set of facts. 
Here, we use $\history$ as a set of historical relations, 
e.g., ``$\nm\_{\m{TS}}(\interval, \Subj, \Obj)$'', which are the time-stamped version relations of the three-element relations ``$\nm(\Subj, \Obj)$'', derived by one of the following rules: \\[-0.5em]

\noindent 
{\small $\ \   
\nm\_{\m{TS}}(\interval, \Subj, \Obj) \hornarrow \nm(\Subj, \Obj), \m{start}(\Obj, n_1), \m{end}(\Obj, n_2), \interval{=}[n_1, n_2]. 
$}\\[-0.5em]

\noindent  {\small $\ \  \nm\_{\m{TS}}(\interval, \Subj, \Obj) \hornarrow \nm(\Subj, \Obj), \m{start}(\Subj, n_1), \m{end}(\Subj, n_2), \interval{=}[n_1, n_2].$}
\\

\noindent which construct the event intervals using the time stamps associated with the object or the subject, respectively. 
The ``$\m{start}$'' and ``$\m{end}$'' predicates are originally generated from the knowledge database and mark the starting and ending points of the duration of the object (or subject) event. 
For simplicity, we use ``$\nm\_{\m{TS}}(\interval)$'' to abbreviate ``$\nm\_{\m{TS}}(\interval, \Subj, \Obj)$'' when $\Subj$ and $\Obj$ are unambiguously unique from the context. 
We also use $\llbracket \nm\_{\m{TS}}(\interval) %, \Subj, \Obj
\rrbracket_{\history}$ to denote the validity of querying the presence of a fact $\nm\_{\m{TS}}(\interval)$ 
against the historical facts $\history$, which stores all the time-stamped three-element predicates. 


\begin{definition}[A Point-based Semantics for MTL]
\label{def:semantics_MTL}
Given a set of (historical) facts $\history$, recording all the events that happened in history, an MTL formula $\mtl$, and a concrete time point  $\timepoint$, the satisfaction relation $(\history, \timepoint) \models \mtl$  (read at the time point \timepoint, the history $\history$ satisfies $\mtl$) is recursively defined as follows: 

{
\small
\begin{align*}
%%%%%%%%%%%%%%%%%%%%%%%%%%%%%%
%%%%%%%%%%% AP  R   %%%%%%%%%%
%%%%%%%%%%%%%%%%%%%%%%%%%%%%%%
(\history, \timepoint) &\models 
\nm &\m{iff}&~ 
\m{\exists\,\interval}.~ 
\llbracket \nm\_{\m{TS}}(\interval) \text{$\rrbracket_{\history}$}{=}\m{true}
~\m{and}~
\timepoint\,{\in}\,\interval
\\[0.1em]
%, \Subj, \Obj
%%%%%%%%%%%%%%%%%%%%%%%%%%%%%%
%%%%%%%%%%% Finally %%%%%%%%%%
%%%%%%%%%%%%%%%%%%%%%%%%%%%%%%
(\history, \timepoint) &\models \mathcal{F}_\interval \,\mtl & 
\m{iff}&~ 
\m{\exists\,\distance}.~\distance\,{\in}\,I  ~ \m{and}
~ (\history, \timepoint\plus\distance)\models\mtl
\\[0.1em]
%%%%%%%%%%%%%%%%%%%%%%%%%%%%%%
%%%%%%%%%%% Globally %%%%%%%%%%
%%%%%%%%%%%%%%%%%%%%%%%%%%%%%%
(\history, \timepoint) &\models \mathcal{G}_\interval\,\mtl & 
\m{iff}&~ 
\m{\forall\,\distance}.~\distance\,{\in}\,I  ~ \m{and}
~ (\history, \timepoint\plus\distance)\models\mtl
\\[0.1em]
%%%%%%%%%%%%%%%%%%%%%%%%%%%%%%
%%%%%%%%%%% Next %%%%%%%%%%
%%%%%%%%%%%%%%%%%%%%%%%%%%%%%%
(\history, \timepoint) &\models \mathcal{N}\,\mtl & 
\m{iff}&~ 
(\history, \timepoint\plus 1)\models\mtl
\\[0.1em]
%%%%%%%%%%%%%%%%%%%%%%%%%%%%%%
%%%%%%%%%%% negation %%%%%%%%%%
%%%%%%%%%%%%%%%%%%%%%%%%%%%%%%
(\history, \timepoint) &\models\neg \mtl & \m{iff}&~
(\history, \timepoint)\not\models\mtl
\\[0.1em]
%%%%%%%%%%%%%%%%%%%%%%%%%%%%%%
%%%%%%%%%%% Unitl %%%%%%%%%%
%%%%%%%%%%%%%%%%%%%%%%%%%%%%%%
(\history, \timepoint) &\models \mtl_1 \, \mathcal{U}_\interval \,\mtl_2  & \m{iff}&~  \m{\exists\,\distance}.~ \distance\,{\in}\,\interval  ~ \m{and}~ (\history, \timepoint\plus\distance)\models\mtl_2 ~ \m{and}
\\[0.1em] 
&&& ~ 
\m{\forall}\, 
k~\m{with} ~\timepoint{<}k{<}(\timepoint\plus\distance), 
(\history, k)\models \mtl_1
\\[0.1em]
%%%%%%%%%%%%%%%%%%%%%%%%%%%%%%
%%%%%%%%%%% conjunction %%%%%%%%%%
%%%%%%%%%%%%%%%%%%%%%%%%%%%%%%
(\history, \timepoint) &\models\mtl_1 \, {\wedge} \,\mtl_2 & \m{iff}&~ (\history, \timepoint)\models\mtl_1 ~\m{and}~ (\history, \timepoint)\models\mtl_2
\\[0.1em]
%%%%%%%%%%%%%%%%%%%%%%%%%%%%%%
%%%%%%%%%%% disjunction %%%%%%%%%%
%%%%%%%%%%%%%%%%%%%%%%%%%%%%%%
(\history, \timepoint) &\models\mtl_1 \, {\vee} \,\mtl_2 & \m{iff}&~ (\history, \timepoint)\models\mtl_1 ~\m{or}~ (\history, \timepoint)\models\mtl_2 
\end{align*}}
\end{definition}
\vspace{2mm}



We randomly generate temporal test cases based on the rich set of historical events and the syntax templates defined in \figref{fig:syntax_of_the_metric_temporal_logic}. 
Each temporal question consists of a concrete MTL formula and a concrete time point, i.e., $(\phi, \timepoint)$. 
For example, the query ``\emph{At 1800, does Victorian era finally come within 40 years?}'' is represented as $(\mathcal{F}_{[0, 40]} \m{victorian\_era}, 1800)$. 
Next, we show how to obtain the expected answer by automatically generating Prolog reasoning rules. 

Given a query $\mtl$, the relation ``$\encoding{\mtl}{\nm}{\widetilde{\drule}}$'' holds if $\mtl$ can be translated into a set of Prolog rules, i.e., $\widetilde{\drule}$. 
Querying ``$\nm(\interval)$'' 
%with the set of Prolog rules 
$\widetilde{\drule}$, against the known database facts yields a set of instantiation of the interval $\interval$. 
The validity of $\mtl$ at any given time point $\timepoint$ is then indicated by the existence of a concrete interval  $\interval$ such that $\timepoint\,{\in}\,\interval$. 
We define the full set of encoding rules for MTL operators in \figref{fig:encoding_rules_mtl}. 

These encoding rules deploy several auxiliary predicates: the 
``$\m{findall}(\interval, \nm)$'' relation indicates that $\interval$ is a union of all the time intervals which satisfy $\nm$; 
the  ``$\m{compl}(\interval, \interval_1)$'' relation indicates that time intervals $\interval$ and $\interval_1$ complement each other, and their union encompasses the entire set of time points; the union and intersection operations, denoted by $\cup$ and $\cap$, are applied to two sets of time intervals. 

\begin{figure}[!h]
% \begin{minipage}[b]{1\linewidth}
\vspace{-2mm}
\vspace{0mm}
\begin{lstlisting}[xleftmargin=6em,numbersep=5pt,basicstyle=\footnotesize\ttfamily]
//nm1 = charles_dickens
charles_dickens_TS([1812, 1870]).
//nm2 = victorian_era
victorian_era_TS([1837, 1901]).
\end{lstlisting} 
\vspace{-1mm}
\caption{Database $\history_s$ Containing Two Time-stamped Events}
\label{fig:Prolog_encoding_Example}
\vspace{-2mm}
\end{figure}


Next, we illustrate the encoding rules for each MTL operator using a few examples. 
To facilitate the illustration, we use a small database $\history_s$ defined in \figref{fig:Prolog_encoding_Example}, which contains two facts: ``\emph{The author Charles Dickens was born in 1812 and he lived until 1870, which spanned a significant portion of the Victorian era}'' and ``\emph{The Victorian era started from 1837 until Queen Victoria died in 1901}'': 

\begin{figure}[!b]
\vspace{-3mm}
\centering\small
\begin{gather*} 
%%%%%%%%%%%%%%%%%%%%%%%%%%%%%%
%%%%%%%%%%% AP R %%%%%%%%%%%%%
%%%%%%%%%%%%%%%%%%%%%%%%%%%%%%
\frac{
\begin{matrix}
\widetilde{\drule} = [\nmNEW(\interval) \hornarrow \nm\_{\m{TS}}(\interval).]
\end{matrix}
}{\encoding {\nm}{\nmNEW}{\widetilde{\drule}}}\ [\trans\text{-}\m{AP}]
\\[0.6em]
%%%%%%%%%%%%%%%%%%%%%%%%%%%%%%
%%%%%%%%%% Finally %%%%%%%%%%%
%%%%%%%%%%%%%%%%%%%%%%%%%%%%%%
\frac{
\begin{matrix}
\widetilde{\drule} {=} 
[\nmNEW([\interval^\prime_\m{start}\text{-}\interval_{\m{end}}, \interval^\prime_\m{end}\text{-}\interval_{\m{start}}]) \hornarrow \nm(\interval^\prime).]
\end{matrix}
}{\encoding {\mathcal{F}_\interval\,\mtl}{\nmNEW}{\widetilde{\drule} }}\ [\trans\text{-}\m{Finally}]
\\[0.6em]
%%%%%%%%%%%%%%%%%%%%%%%%%%%%%%
%%%%%%%%% Globally %%%%%%%%%%%
%%%%%%%%%%%%%%%%%%%%%%%%%%%%%%
\frac{
\begin{matrix}
\widetilde{\drule} {=} 
[\nmNEW([\interval^\prime_\m{start}\text{-}\interval_{\m{start}}, \interval^\prime_\m{end}\text{-}\interval_{\m{end}}]) \hornarrow \nm(\interval^\prime).]
\end{matrix}
}{\encoding {\mathcal{G}_\interval\,\mtl}{\nmNEW}{\widetilde{\drule} }}\ [\trans\text{-}\m{Globally}]
\\[0.6em]
%%%%%%%%%%%%%%%%%%%%%%%%%%%%%%
%%%%%%%%% Next %%%%%%%%%%%
%%%%%%%%%%%%%%%%%%%%%%%%%%%%%%
\frac{
\begin{matrix}
\widetilde{\drule} {=} 
[\nmNEW([\interval^\prime_\m{start}\text{-}1, \interval^\prime_\m{end}\text{-}1]) \hornarrow \nm(\interval^\prime).]
\end{matrix}
}{\encoding {\mathcal{N}\,\mtl}{\nmNEW}{\widetilde{\drule} }}\ [\trans\text{-}\m{Next}]
\\[0.6em]
%%%%%%%%%%%%%%%%%%%%%%%%%%%%%%
%%%%%%%%% Until %%%%%%%%%%%
%%%%%%%%%%%%%%%%%%%%%%%%%%%%%%
\frac{
\begin{matrix}
\encoding{\mtl_1}{\nm_1}{\widetilde{\drule}_1}
\qquad 
\encoding{\mtl_2}{\nm_2}{\widetilde{\drule}_2}
\\[0.2em]
\widetilde{\drule}_3{=} [\m{helper1}([\interval^\prime_{\m{start}}\plus\interval_{\m{start}}, \interval^\prime_{\m{end}}\plus1]) \hornarrow 
\nm_1(\interval^\prime).]
\\[0.2em]
\widetilde{\drule}_4{=} [\m{helper2}(\interval_1\,{\cap}\,\interval_2) \hornarrow 
\m{helper1}(\interval_1), 
\nm_2(\interval_2).] 
\\[0.2em]
\encoding {\mathcal{F}_\interval\,(\m{helper2})}{\nm_f}{\widetilde{\drule}_5 }
\\[0.2em]
\widetilde{\drule}_6{=} [\nmNEW(\interval_1\cap \interval_2) \hornarrow 
\nm_1(\interval_1), 
\nm_
f(\interval_2). ] 
\end{matrix}
}{\encoding{\mtl_1\,\mathcal{U}_\interval\,\mtl_2}{\nmNEW}{\widetilde{\drule}_1\cup \widetilde{\drule}_2\cup
\widetilde{\drule}_3\cup
\widetilde{\drule}_4\cup
\widetilde{\drule}_5\cup
\widetilde{\drule}_6}}\ [\trans\text{-}\m{Until}]
%\shil{
%~2. ~what's ~the~ definition ~\interval ~in~ \mathcal{F}? }
%\\ \shil{~3. ~what's ~the ~meaning ~of ~;?}\text{\syh{to~construct~list~from~single~rules}}
\\[0.6em]
%%%%%%%%%%%%%%%%%%%%%%%%%%%%%%
%%%%%%%%% Negation %%%%%%%%%%%
%%%%%%%%%%%%%%%%%%%%%%%%%%%%%%
\frac{
\begin{matrix}
\encoding{\mtl}{\nm}{\widetilde{\drule}_1}
\\ 
\widetilde{\drule}{=}[\nmNEW(\interval) \hornarrow
\m{findall}(\interval_1, \nm), \m{compl}(\interval_1, \interval).]
\end{matrix}
}{
\encoding{\neg\mtl}{\nmNEW}{
\widetilde{\drule}_1\,{\cup}\,\widetilde{\drule}}
}\ [\trans\text{-}\m{Neg}]
\\[0.6em]
%%%%%%%%%%%%%%%%%%%%%%%%%%%%%%
%%%%%%%%% Conjunction %%%%%%%%%%%
%%%%%%%%%%%%%%%%%%%%%%%%%%%%%%
\frac{
\begin{matrix}
[\trans\text{-}\m{Conj}]\\
\encoding{\mtl_1}{\nm_1}{\widetilde{\drule}_1}
\qquad 
\encoding{\mtl_2}{\nm_1}{\widetilde{\drule}_2}
\\
\widetilde{\drule}{=}[\nmNEW(\interval_1\,{\cap}\,\interval_2) \hornarrow
\m{findall}(\interval_1, \nm_1), \m{findall}(\interval_2, \nm_2)]
\end{matrix}
}{
\encoding{\mtl_1{\wedge}\mtl_2}{\nmNEW}{ \widetilde{\drule}_1\,{\cup}\,\widetilde{\drule}_2\,{\cup}\,\widetilde{\drule}}
}
\\[0.6em]
%%%%%%%%%%%%%%%%%%%%%%%%%%%%%%
%%%%%%%%% Disjunction %%%%%%%%%%%
%%%%%%%%%%%%%%%%%%%%%%%%%%%%%%
\frac{
\begin{matrix}
[\trans\text{-}\m{Disj}]\\
\encoding{\mtl_1}{\nm_1}{\widetilde{\drule}_1}
\qquad 
\encoding{\mtl_2}{\nm_1}{\widetilde{\drule}_2}
\\
\widetilde{\drule}{=}[\nmNEW(\interval_1\,{\cup}\,\interval_2) \hornarrow
\m{findall}(\interval_1, \nm_1), \m{findall}(\interval_2, \nm_2)]
\end{matrix}
}{
\encoding{\mtl_1{\vee}\mtl_2}{\nmNEW}{ \widetilde{\drule}_1\,{\cup}\,\widetilde{\drule}_2\,{\cup}\,\widetilde{\drule}}
}
\end{gather*}
\caption{Encoding MTL Formula $\mtl$ using Prolog Rules}
\label{fig:encoding_rules_mtl}
\end{figure}




\begin{enumerate}[itemsep=0.7em,leftmargin=!,wide]
\item 
When 
$\mtl\,{=}\,\m{charles\_dickens}$ and $\timepoint\,{=}\,1800$: \\ 
According to the encoding rule $[\trans\text{-}\m{AP}]$, the generated Prolog rule is: $\m{\nm1(\interval)}\hornarrow\,\m{
charles\_dickens\_TS(\interval)}$.
Now, querying ``$\nm1(\interval)$'' against $\history_s$ yields $\interval\,{=}\,[1812, 1870]$. Since $1800\,{\not\in}\,\interval$, the expected result of this query is false.

Similarly, when 
$\mtl\,{=}\,\m{victorian\_era}$ and $\timepoint\,{=}\,1900$: \\ 
According to the encoding rule $[\trans\text{-}\m{AP}]$, the generated Prolog rule is: $\m{\nm2(\interval)}\hornarrow\,\m{
victorian\_era\_TS(\interval)}$.
Now, querying ``$\nm2(\interval)$'' against $\history_s$ yields $\interval\,{=}\,[1837, 1901]$. Since $1900\,{\in}\,\interval$, the expected result of this query is true. 

\item When $\mtl\,{=}\,\mathcal{F}_{[0, 40]}\,\m{victorian\_era}$ and $\timepoint\,{=}\,1800$: \\
According to the encoding rule $[\trans\text{-}\m{Finally}]$, the generated Prolog rule is: 
$\m{finally\_\nm2([n_1\text{-}40, n_2\text{-}0])}\hornarrow \m{
\nm2([n_1, n_2])}.$
Now, querying ``$\m{finally\_\nm2}(\interval)$'' against $\history_s$ yields $\interval\,{=}\,[1797, 1901]$. Since $1800\,{\in}\,\interval$, the expected result is true. 
Indeed, all the time points in $\interval$ satisfy the property: ``\emph{within 40 years, finally Victorian era came/still exist}''. 

\item When $\mtl\,{=}\,\mathcal{G}_{[30, 50]}\,\m{victorian\_era}$ and $\timepoint\,{=}\,1800$: \\
According to the rule $[\trans\text{-}\m{Globally}]$, the generated Prolog rule is: $\m{globally\_\nm2([n_1\text{-}30, n_2\text{-}50])}\hornarrow \m{
\nm2([n_1, n_2])}.$
Now, querying ``$globally\_\nm2(\interval)$'' against $\history_s$ yields $\interval\,{=}\,[1807, 1851]$. Since $1800\,{\not\in}\,\interval$, the expected result is false. 
Indeed, only all the time points in $\interval$ satisfy that ``\emph{Victorian era is globally true throughout the 30th to the 50th years in the future}''. 

\item When $\mtl\,{=}\,\mathcal{N}\,\m{victorian\_era}$ and $\timepoint\,{=}\,1836$: \\
According to the rule $[\trans\text{-}\m{Next}]$, the generated Prolog rule is: $\m{next\_\nm2([n_1\text{-}1, n_2\text{-}1])}\hornarrow \m{
\nm2([n_1, n_2])}.$
Now, querying ``$\m{next\_\nm2}(\interval)$'' against $\history_s$ yields $\interval\,{=}\,[1836, 1900]$. Since  $1836\,{\in}\,\interval$, the expected result is true. 
Indeed, all the time points in $\interval$ satisfy that ``\emph{next year Victorian era came/still exist}''. 

\item When $\mtl\,{=}\,\m{charles\_dickens}
~\mathcal{U}_{[10, 20]}\,\m{victorian\_era}$ and $\timepoint{=}1800$: This query aims to determine the time interval $\interval$ that encompasses all time points $\timepoint'$ for which there exists a future year ($\timepoint'\plus\distance$) when the Victorian era had begun; and during the time from $\timepoint'$ to $\timepoint'\plus\distance$, Charles Dickens must have been born and remained alive throughout. Lastly, check if $1800\,{\in}\,\interval$.

In $[\trans\text{-}\m{Until}]$, 
$\m{helper1}$ computes all the possible  $\timepoint'\plus d$  
%where \\  $d\,{\in}\,[10, 20]$, 
such that $(\history, k)\models \m{charles\_dickens} ~\m{forall}~ 
k~\m{with} ~\timepoint'{<}k{<}(\timepoint'\plus\distance)$. 
Next $\m{helper2}$ computes the overlapping  intervals of $\m{helper1}$ and the intervals that also satisfy $(\history, \timepoint'\plus\distance)\,{\models}\,\m{victorian\_era}$. 
Then $\nm_f$ computes the interval of $\timepoint'$ which finally reach $\m{helper2}$ within 10 to 20 years. 
Lastly, the final answer of the interval of $\timepoint'$~is the intersection of $\nm_f$ and $\nm_1$. 
Therefore, given the concrete query $(\phi, \timepoint)$, from $[\trans\text{-}\m{Until}]$, 
the generated rules are shown in \figref{fig:until10-20-encoding}. 




\begin{figure}[!h]
\vspace{-3mm}
{
\begin{align*}
&\m{helper1([n_1\plus10, n_2\plus1])}\hornarrow \m{\nm1([n_1, n_2])}.
& // [1822, 1871]
\\
&\m{helper2(\interval_1\cap\interval_2)}\hornarrow\m{helper1(\interval_1)}, \m{\nm2(\interval_2)}. 
& // [1837, 1871]
\\
& \m{\nm_f([n_1\text{-}20, n_2\text{-}10])}\hornarrow \m{
helper2([n_1, n_2])}.
& // [1817, 1861]
\end{align*}
\vspace{-4mm}
\[\m{charles}\_\m{until}\_\m{victorian\_era}(\interval_1\cap\interval_2) \hornarrow \m{\nm1(\interval_1)}, \m{\nm_f(\interval_2)}.\]}
\caption{Prolog Rules Generated for an "Until" Query}
\label{fig:until10-20-encoding}
\vspace{-1mm}
\end{figure}

Querying ``$\m{charles}\_\m{until}\_\m{victorian\_era}$'' against $\history_s$ yields $\interval\,{=}\,[1817, 1861]$. Since $1800\,{\not\in}\,\interval$, the expected result is false. 
Indeed, only all the time points in $\interval$ satisfy $\phi$ under the semantic definition of \emph{Until}, cf.  \defref{def:semantics_MTL}. For example when $\timepoint'{=}1817$, there exists $\distance{=}20$ such that $\phi$ holds; and when $\timepoint'{=}1861$ there exists $\distance{=}10$ such that $\phi$ holds. 

Note that, in this encoding, the interval of ``\emph{Until}'' operators does not include $[0, 0]$, as $\mtl_1  
\,\mathcal{U}_{[0, 0]} \,  \mtl_2$ essentially equals $\mtl_2$. Therefore, when the interval compasses $[0, 0]$, we use the following rule to decompose the encoding: (Note that when $\interval'\,{=}\,\interval{\setminus}[0, 0]$, it means $\interval'\cup[0, 0]\,{=}\,\interval$)
\begin{align*}
\frac{
[0, 0] \subseteq \interval 
\qquad 
\interval^\prime \,{=}\, \interval{\setminus}[0, 0]
}{
\mtl_1  
\,\mathcal{U}_\interval \,  \mtl_2 
\equiv  (\mtl_1  
\,\mathcal{U}_{\interval^\prime} \,  \mtl_2)  \vee \mtl_2 
} \ [\trans\text{-}\m{Until}\text{-}0]
\end{align*}

\vspace{2mm}
\item When $\mtl\,{=}\,\neg\,\m{victorian\_era}$ and $\timepoint\,{=}\,1800$: \\
By $[\trans\text{-}\m{Neg}]$, the generated Prolog rule is: 
$\m{neg\_\nm2(\interval)}\hornarrow$ 
$\m{findall}(\interval_1, \nm1), \m{compl}(\interval_1, \interval).$ 
Querying ``$\m{neg\_\nm2}(\interval)$'' against $\history_s$ yields $\interval\,{=}\,[1, 1836] \cup [1902, 2024].$
Here, we take all the after-century years to be the full set. 
Since $1800\,{\in}\,\interval$, the expected result is true. 
Indeed, all the time points in $\interval$ satisfy that ``\emph{Victorian era has not come/already passed}''.  


\item When $\mtl\,{=}\,\m{charles\_dickens}\,{\wedge}\,\m{victorian\_era}$ and $\timepoint{=}1900$: 
By $[\trans\text{-}\m{Conj}]$, the generated Prolog rule is: 
$\m{\nm1\_and\_\nm2(\interval_1{\cap}\interval_2)}\hornarrow 
\m{findall}(\interval_1, \nm1), \m{findall}(\interval_2, \nm2)$. 
Now, querying ``$\m{\nm1\_and\_\nm2}(\interval)$'' against $\history_s$ yields $\interval\,{=}\,[1837, 1870]$. Since $1900\,{\not\in}\,\interval$, the expected result is false. 
Indeed, only the time points in $\interval$ satisfy that ``\emph{Victorian era exists and Charles Dickens is alive}''. 


\item When $\mtl\,{=}\,\m{charles\_dickens}\,{\vee}\,\m{victorian\_era}$~and
$\timepoint{=}1900$: 
By $[\trans\text{-}\m{Disj}]$, the generated Prolog rule is as follows: 
$\m{\nm1\_or\_\nm2(\interval_1{\cup}\interval_2)}\hornarrow 
\m{findall}(\interval_1, \nm1), \m{findall}(\interval_2, \nm2)$. 
Now, querying ``$\m{\nm1\_or\_\nm2}(\interval)$'' against $\history_s$ yields $\interval\,{=}\,[1812, 1901]$. Since $1900\,{\in}\,\interval$, the expected result is true. 
Indeed, all the time points in $\interval$ satisfy that ``\emph{Victorian era exists or Charles Dickens is alive}''.  
%\shil{check implementation?}
\end{enumerate}
\vspace{3mm}






{\emph{\textbf{Remark.}}} 
While discrete-time MTL is commonly employed for model-checking timed verification~ \cite{DBLP:phd/us/Henzinger91}, utilizing Prolog to encode MTL for reasoning about the temporal relationships among events and detecting LLM hallucination is novel. 
These encoding rules can recursively accommodate the entire range of MTL formulas, including those with any level of nesting. 
We provide a formal definition for the correctness of these encoding rules in \theoref{ThemSoundAndComplete} and demonstrate that they are both sound and complete.



\begin{restatable}[Correctness of the encoding rules]{thm}{ThemSoundAndComplete}
\label{ThemSoundAndComplete}
~\\
Given any $\history$, 
$\mtl$, and 
$\encoding {\mtl}{\nm}{\widetilde{\drule}}$, let $\Prolog{=}\history \plus\plus \widetilde{\drule}$, we define,  \\
(1) Soundness: \\
$\forall\, \interval$.  
$\llbracket \nm(\interval) \rrbracket_{ \Prolog} {=} \m{true}$, then 
$\forall\, \timepoint\,{\in}\, \interval$, we have 
$(\history, \timepoint) \models \mtl$;  \\
(2) Completeness: \\ 
$\forall \,\timepoint\,$. $(\history, \timepoint) \models \mtl$, then $\exists\, \interval$. $\llbracket \nm(\interval) \rrbracket_{\Prolog} {=} \m{true}$  and $\timepoint\,{\in}\,\interval$. 
\end{restatable}

\begin{proof}
By structural induction over $\phi$. 
%a case analysis of the encoding rules.
The detailed proofs are given in the Appendix. %\appref{app:correctness}.  
\end{proof}




\begin{table*}[!t]
% \def\arraystretch{1.0}
\setlength{\tabcolsep}{1pt}
\centering
% \footnotesize
\caption{Relation-Template Mapping Patterns.}
 % \lnk{need to refine}}
\label{table:template}
\footnotesize
%\resizebox{\linewidth}{!}{
\begin{tabular}{l l}
\toprule 
\textbf{Relation} & \textbf{Template Examples}  \\
    \midrule
{Noun Phrase} & \begin{tabular}[l]{@{}l@{}} - Is it true that 
$\langle \m{Subject}\rangle$ and 
$\langle\m{Object}\rangle$ share 
$\langle\m{Relation}\rangle$? 
\\ - $\langle\m{Subject}\rangle$ and $\langle\m{Object}\rangle$ have/made/shared totally different $\langle\m{Relation}\rangle$. Please judge the truthfulness of this statement.
%\\Please judge the truthfulness of this statement. 
    \end{tabular}  \\
    \midrule
    \begin{tabular}[l]{@{}l@{}} Verb Phrase \\ (Passive Voice) \end{tabular} & \begin{tabular}[l]{@{}l@{}} - Is it true that $\langle \m{Subject}\rangle$ is/was/are/were $\langle \m{Relation}\rangle$ $\langle\m{Object}\rangle$? \\ - It is impossible for $\langle \m{Subject}\rangle$ to be $\langle\m{Relation}\rangle$ $\langle\m{Object}\rangle$. Am I right?%\\ Other formats... 
    \end{tabular}  \\
    \midrule
\begin{tabular}[l]{@{}l@{}} Verb Phrase \\ (Active Voice) \end{tabular}
 & \begin{tabular}[l]{@{}l@{}} - Is it true that 
 $\langle \m{Subject}\rangle$
 $\langle\m{Relation}\rangle$
 $\langle\m{Object}\rangle$?  \\ - $\langle \m{Subject}\rangle$ $\langle\m{Relation}\rangle$ $\langle\m{Object}\rangle$. 
 %\\ Please judge the truthfulness of this statement.
 %\\ Other formats... 
 \end{tabular}  \\

    \bottomrule %添加表格底部粗线
\end{tabular}
%}
\end{table*}

\begin{table*}[!t]
\setlength{\tabcolsep}{3pt}
\centering
% \footnotesize
\caption{Temporal-Template Mapping Patterns (implicitly querying upon year $\iyear$).}
\label{table:temporal_template}
\footnotesize
\begin{tabular}{l  l }
\toprule 
\textbf{MTL Formulae} & \textbf{Template Examples}  \\
\midrule 
\mtltoNL($\nm$) &  Did $\langle\nm\rangle$ happen at year $\langle \iyear \rangle$? 
\\  \midrule
\mtltoNL($\mathcal{F}_\interval \,\mtl$) & Did ``Event'' finally happen within the time frame of $\langle \interval \rangle$ after the year $\langle \iyear \rangle$, where ``Event'' is defined as $\langle \mtltoNL(\mtl)\rangle$? 
\\ 
\midrule 
\mtltoNL($\mathcal{G}_\interval \,\mtl$) &  Did ``Event'' globally happen within the time frame of $\langle \interval \rangle$ after the year $\langle \iyear \rangle$, where ``Event'' is defined as $\langle \mtltoNL(\mtl)\rangle$?
\\ 
\midrule 
\mtltoNL($\mathcal{N} \,\mtl$) & Did ``Event'' happen in the next year of $\langle \iyear \rangle$, where ``Event'' is defined as $\langle \mtltoNL(\mtl)\rangle$? 
\\ 
\midrule 
\mtltoNL($\mtl_1 \, \mathcal{U}_\interval \,\mtl_2$) &  
\begin{tabular}[l]{@{}l@{}} 
Did ``Event$_1$'' happen continuously until ``Event$_2$'' started, during the period $\langle \interval \rangle$ after the year $\langle \iyear \rangle$, \\ 
where ``Event$_1$'' is $\langle \mtltoNL(\mtl_1) \rangle$ and ``Event$_2$'' is $\langle \mtltoNL(\mtl_2) \rangle$?
\end{tabular}
\\ 
\midrule 
\mtltoNL($\mtl_1  \,{\wedge}\,  \mtl_2$) &  Did both ``Event$_1$'' and  ``Event$_2$'' happen at year $\langle \iyear \rangle$, where ``Event$_1$'' is $\langle \mtltoNL(\mtl_1) \rangle$ and ``Event$_2$'' is $\langle \mtltoNL(\mtl_2) \rangle$? 
\\ 
\midrule 
\mtltoNL($\mtl_1  \,{\vee}\,  \mtl_2$) &  Did either ``Event$_1$'' or ``Event$_2$''  happen at year $\langle \iyear \rangle$, where ``Event$_1$'' is $\langle \mtltoNL(\mtl_1) \rangle$ and ``Event$_2$'' is $\langle \mtltoNL(\mtl_2) \rangle$? 
\\ 
\midrule 
\mtltoNL($\neg \mtl $) &  Did ``Event'' not happen at year $\langle \iyear \rangle$, where ``Event'' is $\langle \mtltoNL(\mtl) \rangle$? 
\\ 
\bottomrule %添加表格底部粗线
\end{tabular}
\end{table*}




\subsection{Benchmark Construction}
\label{prompt}
%From the derived facts, 
\tool constructs question-answer~(Q\&A) pairs and prompts to facilitate the testing for FCH. 
To address the challenge of the high human efforts required in the test oracle generation, we design an automated approach based on mapping relations between various entities to problem templates, largely reducing reliance on manual efforts. 

\textbf{\emph{Question Generation.}}
%\wkl{rewritten, check}
%\syh{checked, ok}
To facilitate efficient and systematic generation of test cases and prompts, we have adopted a method that leverages entity relationships and mappings of temporal operators to predefined Q\&A templates. 

When constructing relation-based Q\&A templates (without temporal operators), one key aspect lies in aligning various types of relations with the corresponding question templates from the mutated triples, i.e., the predicate type in the triple. Different relation types possess unique characteristics and expressive requirements, leading to various predefined templates. 
As listed in \tabref{table:template}, we map the relation types to question templates based on speech and the grammatical tense of the predicate to guarantee comprehensive coverage. 

When constructing temporal-logic-related queries, we define a mapping pattern for each temporal operator, as outlined in \tabref{table:temporal_template}. For any query expressed as ``$\mtl$''  with any concrete year $\iyear$ in query, the $\mtltoNL(\mtl)$ function converts the MTL formula $\mtl$ into a natural language query. In this context, $\mtltoNL(\mtl')$ is called recursively to generate the natural language description for the CTL subformula $\mtl'$. 



In both mapping patterns, we enhance the construction of the naturally formatted questions by leveraging an LLM to reformulate the structure and grammar of the Q\&A pairs. 

\textbf{\emph{Answer Generation.}}
We note that the answer to the corresponding question is readily attainable from the factual knowledge and the Prolog reasoning rules, defined in both \figref{fig:basic_op_for_predicates} and \figref{fig:encoding_rules_mtl}. 
Primarily, it is easy to determine whether the answer is \emph{true} or \emph{false} based on the mutated triples and the ground-truth time intervals using temporal reasoning. 
Meanwhile, mutated templates with positive and negative semantics via the usage of synonyms or antonyms, which greatly enhance the diversity of the questions, can be treated similarly as the negation rule defined in \figref{fig:basic_op_for_predicates}. 
Specifically, if the answer to a question with original semantics is Yes/No, then for a question with mutated opposite semantics, the corresponding answer would %naturally 
be the opposite, i.e., No/Yes. For example, after obtaining the original Q\&A pair \textit{- ``Is it true that Crohn's disease and Huntington's disease could share similar symptoms and signs? - Yes.''}, we can use antonyms to mutate it into \textit{- ``Is it true that Crohn's disease and Huntington's disease have different symptoms and signs? - No.''}
%\syh{what does this mean?}







% In total, we have defined 60 templates according to the pre-defined rules.
% , from which we have generated 194,850 Q\&A pairs. To prevent the dataset scope from becoming overly extensive, we conduct an initial screening based on categories of reasoning rules, ultimately yielding a total of 14,228 question pairs.
\begin{comment}   
\begin{table*}[!t]
    % \def\arraystretch{1.0}
    \setlength{\tabcolsep}{1ex}
	\centering
        \Large
	% \footnotesize
	\caption{Relation-Template Mapping Pattern.}
 % \lnk{need to refine}}
        \label{table:template}
	\resizebox{\linewidth}{!}{
	\begin{tabular}{l l l}
    \toprule 
    \textbf{Relation} & \textbf{Template} &\textbf{Example} \\
    \midrule
    \textbf{Noun Phrase} & \begin{tabular}[l]{@{}l@{}} - Is it true that \textit{Subject} and \textit{Object} share \textit{Relation}? \\ - \textit{Subject} and \textit{Object} have/made/shared totally different [Relation]. \\Please judge the truthfulness of this statement. \end{tabular} & \begin{tabular}[l]{@{}l@{}} New Triple: \textit{[Crohn's disease, similar\_symptoms\_and\_signs, Huntington's disease]}\\ Template: - Is it true that \textit{Crohn's disease} and \textit{Huntington's disease} share \textit{similar symptoms and signs}? -Yes. \\- Does \textit{Crohn's disease} and \textit{Huntington's disease} have similarities on {symptoms and signs}? - Yes. \end{tabular} \\
    \midrule
    \textbf{Verb Phrase in Passive Voice} & \begin{tabular}[l]{@{}l@{}} - Is it true that \textit{Subject} is/was/are/were \textit{Relation} \textit{Object}? \\ - It is impossible for \textit{Subject} to be \textit{Relation} \textit{Object}. Am I right?\\ Other formats... \end{tabular} & \begin{tabular}[l]{@{}l@{}} New Triple: \textit{[Kuratowski's theorem, not\_proved\_by, Kurt Gödel]}\\ Template: - Is it true that Kuratowski's theorem was proved by Kurt Gödel? - No.\\- Someone else other than Kurt Gödel proved Kuratowski's theorem, am I right? - Yes. \end{tabular} \\
    \midrule
    \textbf{Verb Phrase in Active Voice} & \begin{tabular}[l]{@{}l@{}} - Is it true that \textit{Subject Relation Object}?  \\ - \textit{Subject} \textit{Relation} \textit{Object}. Please judge the truthfulness of this statement.\\ Other formats... \end{tabular} & \begin{tabular}[l]{@{}l@{}} New Triple: \textit{[Baby Don't Lie, appeared\_before, Spark the Fire]}\\ Template: - Is it true that \textit{Baby Don't Lie} appeared before \textit{Spark the Fire}? - Yes.\\- \textit{Baby Don't Lie} never appeared before \textit{Spark the Fire}. \\ Please judge the truthfulness of this statement. -No.   \end{tabular} \\
    % \midrule
    % \textbf{Custom-Designed} & \begin{tabular}[l]{@{}l@{}l@{}l@{}l@{}} Given the \textit{SubjectList}, is it true that \textit{SubjectSelected} is the {Relation} among them?\\ Other formats... \end{tabular} & \begin{tabular}[l]{@{}l@{}l@{}l@{}l@{}} New Triple: [\textit{Roman Holiday, Hindenburg disaster newsreel footage, Lassie Come Home, The White Parade}, \textit{descending\_duration\_order}, \\ \textit{Roman Holiday, Lassie Come Home, The White Parade}]\\ Template: - Given the list [Roman Holiday, Hindenburg disaster newsreel footage, Lassie Come Home, The White Parade],\\ is it true that Roman Holiday have the longest duration among them? - Yes. \end{tabular} \\
    \bottomrule %添加表格底部粗线
\end{tabular}}
\end{table*}
\end{comment}

\textbf{\emph{Prompt Construction.}}
% \wkl{here maybe give a sample prompt screenshot as a figure to illustrate?}
As illustrated in \tabref{table:prompt}, before initiating any interaction with LLMs, we predefine specific instructions and prompts, requesting the model to utilize its inherent knowledge and inferential capabilities to deliver explicit (Yes/No/I don't know) judgments on our queries. Additionally, we instruct the model to present its reasoning process in a template following the judgment. The main goal is to ensure that LLMs give easy-to-understand responses using standardized prompts and instructions. 
This approach also helps the model to effectively exercise its reasoning abilities based on the given instructions and examples.


\begin{table*}[!t]
%\vspace{-0.3cm}
    % \def\arraystretch{1.0}
    \setlength{\tabcolsep}{1ex}
	\centering
    %\large 
	\small
	\caption{Prompt Template. %\shil{Shall we restrict the answer for relation query begnin with Yes/No/I don't know?}
    }
        \label{table:prompt}
        \vspace{-0.1cm}
	%\resizebox{\linewidth}{!}{
	\begin{tabular}{l}
    \toprule 
    \rowcolor{mycolor}
    \textbf{\instruction:} Answer the question with your knowledge and reasoning power.\\
    \midrule
    \rowcolor{mycolor} \textbf{\query (Relation):}  Given the $\langle \textit{question} \rangle$: \textit{question}, please provide an answer with your knowledge and reasoning power.\\ 
    \rowcolor{mycolor} Think of it step by step with a human-like reasoning process. After giving the answer, list the knowledge used in your\\ 
    \rowcolor{mycolor} reasoning process in the form of declarative sentences and point by point. The answer must contain `Yes', `No' or `I \\
    \rowcolor{mycolor} don't know' at the beginning. \\
    \midrule
    \rowcolor{mycolor} \textbf{\query (Temporal):}  Given the question: 
    $\langle \textit{question} \rangle$, please provide an answer with your knowledge and reasoning power \\
    \rowcolor{mycolor}  upon metric temporal logic. Think it step by step with a human-like reasoning process. After giving the answer, list the \\
    \rowcolor{mycolor} evidence from your temporal reasoning  in the form of declarative sentences and point by point. The answer must contain   \\
\rowcolor{mycolor} `Yes', `No' or `I don't know' at the beginning.\\
    \bottomrule %添加表格底部粗线
\end{tabular}
\end{table*}




\vspace{-1mm}
\subsection{Response Evaluation}\label{response}
%
%\begin{lstlisting}
% \begin{align*}
%     <program>::=&<decoder>|<query>|<model>|<condition>|<distribution>\\&|<query*>|<parser> \\
%     <query*>::=&statement(transform(s,R,o))\\
%     <parser>::=&extract(statement)\\
%     <transform>::=&[Neg]|[Sym]|[Inverse]|[Trans]|[Comp]
% \end{align*}
% \begingroup\vspace*{-1cm}
% \captionof{figure}{Syntax of Extended LMQL.}\label{sec:syntax}
% \vspace*{\baselineskip}\endgroup
%\end{lstlisting}
%
% \begin{lstlisting}[language=Python, caption=LMQL Program Grammar]
% <decoder> ::= argmax | beam(n=<int>) | sample(n=<int>)
% <query> ::= (<python_statement>)+
% <cond> ::= <cond> and <cond> | <cond> or <cond> | not <cond> | <cond_term>
% <cond_term> ::= <python_expression>
% <cond_op> ::= < | > | = | in
% <dist> ::= <var> over <python_expression>
% \end{lstlisting}
% \begin{grammar}
% <LMQL Program> ::= <decoder> <query>

% <decoder> ::= `argmax' | `beam(n=\textit{int})' | `sample(n=\textit{int})'

% <query> ::= (<python\_statement>)+

% <cond> ::= <cond> `and' <cond> | <cond> `or' <cond> | `not' <cond> | <cond\_term>

% <cond\_term> ::= <python\_expression>

% <cond\_op> ::= `<` | `>' | `=' | `in'

% <dist> ::= <var> `over' <python\_expression>
% \end{grammar}

% To facilitate the automated the query and answer validation process, we extend the previously proposed LMQL~(language Model Query Language)~\cite{Beurer-Kellner-2023} designed for LLM programming. LMQL utilizes SQL-like elements and a imperative syntax for scripting, as shown in Figure~\ref{sec:syntax}. More specifically, LMQL defines the interactive process with an LLM as a python-like $<program>$, including a $<decoder>$ to the decoding procedure employed by the LMQL runtime when solving the
% query, a $<query>$ to model the interaction with the model, a $<model>$ to denote the LLM to interact with, a $<condition>$ to place constraints on the variables in the program, a $<distribution>$ to represent the probability for output predictions from the LLM. language is augmented with additional constructs to facilitate the interaction and generation capabilities of LLMs.

% Firstly, we introduce the <query*> element as an extension to the existing <query> element. The <query*> block models the interaction with the LLM, serving as the prompt that is fed into the model. These query strings allow for the use of specially escaped subfields, similar to Python's f-strings. These subfields, denoted by "[varname]", represent phrases that will be generated by the LLM, also known as holes.

% Furthermore, we introduce a new element named <parser>. The <parser> element is responsible for extracting triples (subject-predicate-object) from the LLM-generated answers for further semantic comparison. The <parser> is capable of processing one or multiple sentences, extracting triples from each sentence sequentially and recording the derived results for further use. The <parser> element plays a crucial role in the answer validation process by breaking down the LLM's generated responses into structured triples. These triples can then be compared against a knowledge base or a set of predefined rules to validate the semantic correctness and coherence of the generated content.


The objective of this module is to enhance the detection of FCH in LLM outputs, specifically focusing on identifying the discrepancies between LLM responses and the verified ground truths. Recognizing the inherent limitation in directly accepting ``Yes'' or ``No'' answers from LLMs, our approach underscores the automated detection of factual consistency during the reasoning process presented by LLMs. 
% This analysis is vital for accurately determining the factual consistency of LLM responses, thereby addressing the primary challenge in identifying FCH within LLM outputs. 
% To achieve automated detection of factual consistency, our methodology first incorporates a parsing step that leverages advanced NLP techniques. This step is designed to extract essential semantic elements from each sentence within LLM outputs, assembling these elements into a coherent, semantic-aware structure. 
% The foundational premise of our approach is predicated on evaluating the semantic similarity between these constructed structures, aiming to discern the degree of consistency in their underlying semantics.
% Subsequently, we propose the development of a set of similarity-based testing oracles. These oracles are instrumental in applying metamorphic testing principles, enabling us to systematically assess the consistency or inconsistency between LLM responses and the established ground truth. 
Our approach is structured around several critical steps, as listed in \algoref{alg:eval} and detailed below:
% \shil{(SL: to cite Algorithm 3 and when explanation, referring to line number? For lines 11 to 15, use ``or" to include three conditions?)}
% The key target of this module is to detect the FCH by identifying the inconsistency between answers from the LLMs and the groundtruth in Q\&A pairs. However, as we unable to directly trust the yes or no answers from LLM directly, we need to parse the reasoning process carefully before reaching the verdict on the correctness with accuracy, which is the aforementioned challenges for detecting FCH in LLMs.  
% %This module outlines our approach for detecting hallucinations in the responses of the target LLMs. A key insight is the premise that any response contradicting the answer of the factual Q\&A pairs we provide is inherently regarded as an occurrence of FCH.
% To facilitate the automatic detection, we first design a parsing step to utilize an NLP-based approach to extract the critical semantic component from each sentence from LLM outputs, and assemble them into a semantic-aware structure. The key insight is to examine the similarity between these structures to determine the consistency in their semantics. Then, by designing a set of similarity-based testing oracles, we are able to utilize metamorphic testing to determine the (in)consistency between LLM answers and groundtruth.  This method primarily comprises the following steps:
%During the aforementioned prompt design process, we have prepared question prompts and their corresponding answers, thereby establishing ground truth Q\&A pairs. Therefore, we can automatically compare the LLM output with the expected ground truth answer to detect the discrepancies. Moreover, we utilize an NLP-based approach to compare the semantics in the reasoning process to identify inconsistencies to assist in understanding the cause of FCH. %our similarity between the responses from LLMs with these ground truth Q\&A pairs to verify the consistency and logical soundness of the LLMs' responses. 
% Detailed conclusions are discussed in the following section.


%In summary, we propose a method based on semantic parsing and metamorphic relations to verify whether LLM responses contain FCHs. This method primarily comprises the following steps:

%labelwidth=!,
\begin{enumerate}[wide,  labelindent=9pt]
%Step 1. 
\item \textbf{Preliminary Screening.} Given the LLM response $\llmResponse$, we first eliminate scenarios when the LLM declines to provide an answer, indicated by the ``answer'' field of LLM's responses. 
Most of these responses arise because the LLM lacks the relevant knowledge for the reasoning process. As these responses adhere to the LLM's principle of honesty, we categorize them as correct and normal responses.
% (as described in Algorithm~\ref{alg:eval} Line 7-8). 

%Step 2. 
\item \textbf{Response Parsing and Semantic Structure Construction.} Provided with the remaining suspicious responses from $\llmResponse$ and ground truth facts $\groundTruthTriples$, we use \textsc{ExtractTriple} function to extract triples that follow the same structure as the fact defined in the \secref{subsec3.1}. For each LLM response, the extracted triples ($\widetilde{\m{Trpl}}$) are based on the statements contained in the \textit{reasoning process} part of the LLM's response, which is further utilized to construct a response semantic structure $\semantic_{\m{resp}}$ using the \textsc{BuildGraph} function. In this structure, the $\widetilde{\entity}$ are depicted as \emph{nodes} ($\m{N}$), and the relational predicate ($\nm$) between them are illustrated as \emph{edges} ($\m{E}$). Concurrently using the same approach, we construct another semantic structure $\semantic_{\m{ground}}$ using $\groundTruthTriples$.

%Step 3. 
\item \textbf{Similarity-based Metamorphic Testing and Oracles.} 
We apply metamorphic relations to identify hallucination answers from LLMs, i.e., comparing the similarity between semantic structures generated by LLMs and the ground truth counterparts. Note that we provide four classifications: correct responses ($\m{CO}$), hallucinations caused by error inference ($\m{EI}$), hallucinations caused by erroneous knowledge ($\m{EK}$), and hallucinations containing both issues ($\m{OL}$). 
Specifically, the oracles for metamorphic testing can be divided into the following types:
 
% We then apply metamorphic relations to detect and evaluate potential errors in LLM responses, based on the relationships between inputs and outputs, rather than relying on traditional labeled data. 
% In our context, metamorphic relations specifically refer to comparing the similarity between semantic structures generated by LLMs and the ground truth counterparts, to identify and classify hallucination answers from LLMs.
% (as mentioned in Algorithm~\ref{alg:eval} Line 12-18). 
\end{enumerate}

\begin{comment}

tree = Leaf() | Node ()

resp{
    answer = bool 
    steps = tree ??
}

ground_truth{
    answer = bool 
    reasoning = tree
}


evaluation (resp, ground_truth)
    if resp.answer = refusal then CO 
    else 
        s_e = SE(resp, R_derived)
        s_n = SN(resp, R_derived)
        if s_e < threadhold_e then EI 
        else if s_n < threadhold_n then EK 
        else CO 


    
\end{comment}


% \lnk{check this algo}
% \begin{algorithm}[!h]
% \caption{Response Evaluation}
% \label{alg:eval}
% \small
% \begin{algorithmic}[1]
% \Require LLM Response ($\llmResponse$), Ground Facts ($\groundTruthTriples$), Threshold ($\theta_{\m{e}}, \theta_{\m{n}}$)
% \Ensure Evaluation Result ($\eval$)
% \Function{EvaluateResponse}{$\llmResponse$, $\groundTruthTriples$, $\theta_{\m{e}}$, $\theta_{\m{n}}$}
%     % \State $hallu\_ei, hallu\_ek, hallu\_both \gets$ [] \Comment{\commentstyle{Initialization}}
%     % \State $\eval, \eval_{\m{ei}}, \eval_{\m{ek}}, \eval_{\m{co}} \gets$ [] \Comment{\commentstyle{Initialization}}
%     % \State $refuse\_to\_answer \gets$ \Call{FindRefuseToAnswer}{$\llmResponse$} \Comment{\commentstyle{Find `refuse to answer' responses}}
%     % \State $suspicious\_resps \gets$ \Call{FilterSuspiciousRes}{$\llmResponse$, $GT\_Answer$} \Comment{\commentstyle{Filter suspicious responses}}
%     % \For{$\m{resp}$ in $\llmResponse$} \Comment{\commentstyle{Iterate each response}}
%     \If{$\llmResponse.answer = \m{refusal}$}
%         \State
%         $\eval$ $\in$ $CO$ \Comment{\commentstyle{Preliminary Screening}}
%     % \State $\eval_{\m{co}}$.append($\m{resp_\m{refusal}}$) \Comment{\commentstyle{Preliminary Screening}}
%     \Else
%         \State $\deriveKG{\m{resp}}{\groundTruthTriples}{\semantic_{\m{resp}}}{\semantic_{\m{ground}}}$
%         %\Comment{\commentstyle{Extract Semantic Structure}}
%         \State $\m{s}_{\m{e}} \gets$ $\similarity_\m{e}${$(\semantic_{\m{resp}}$, $\semantic_{\m{ground}})$} \Comment{\commentstyle{Calculate edge similarity}}
%         \State $\m{s}_{\m{n}} \gets$ $\similarity_\m{n}${$(\semantic_{\m{resp}}$, $\semantic_{\m{ground}})$} \Comment{\commentstyle{Calculate node similarity}}
%         % \If{$edge\_sim < \theta\_e$ and $node\_sim < \theta\_n$}
%             % \State $\eval$.append($response$) \Comment{\commentstyle{Append mixed hallucination}}
%         \If{$\m{s}_{\m{e}} < \theta_{\m{e}}$}
%             \State 
%             $\eval$ $\in$ $EI$  \Comment{\commentstyle{Append error inference hallucination}}
%         \ElsIf{$\m{s}_{\m{n}} < \theta_{\m{n}}$}
%             \State 
%             $\eval$ $\in$ $EK$  \Comment{\commentstyle{Append error knowledge hallucination}}
%         \Else
%             \State
%             $\eval$ $\in$ $CO$  
%             \Comment{\commentstyle{Append correct response}}
%         \EndIf
%     \EndIf
%     % \EndFor
%     % \State $\eval$.extend($\eval_{\m{ei}}, \eval_{\m{ek}}, \eval_{\m{co}}$) \Comment{\commentstyle{Merge the result}}
%     % \State $evaluation\_result \gets$ \Call{GenerateResult}{$hallu\_both$, $hallu\_ei$, $hallu\_ek$}
%     % \shil{(we use three lists \_both, \_ei, and \_ek)} $contradictory\_answers$} 
%     % \Comment{Generate evaluation result}
%     \State \Return $\eval$ \Comment{\commentstyle{Return the evaluation result}}
% \EndFunction
% \end{algorithmic}
% \end{algorithm}
\begin{algorithm}[!b]
\caption{Response Evaluation}
\label{alg:eval}
\small
\begin{algorithmic}[1]
\Require LLM Response ($\llmResponse$), Ground Facts ($\groundTruthTriples$), Threshold ($\theta_{\m{e}}, \theta_{\m{n}}$)
\Ensure Evaluation Result Category~($CO, EK, EI, OL$)
\Function{EvaluateResponse}{$\llmResponse$, $\groundTruthTriples$, $\theta_{\m{e}}$, $\theta_{\m{n}}$}
    % \State $hallu\_ei, hallu\_ek, hallu\_both \gets$ [] \Comment{\commentstyle{Initialization}}
    % \State $\eval, \eval_{\m{ei}}, \eval_{\m{ek}}, \eval_{\m{co}} \gets$ [] \Comment{\commentstyle{Initialization}}
    % \State $refuse\_to\_answer \gets$ \Call{FindRefuseToAnswer}{$\llmResponse$} \Comment{\commentstyle{Find `refuse to answer' responses}}
    % \State $suspicious\_resps \gets$ \Call{FilterSuspiciousRes}{$\llmResponse$, $GT\_Answer$} \Comment{\commentstyle{Filter suspicious responses}}
    % \For{$\m{resp}$ in $\llmResponse$} \Comment{\commentstyle{Iterate each response}}
    \State $CO, EK, EI, OL \gets []$ \Comment{\commentstyle{Initialization}}
    \If{$\llmResponse.answer = refusal$}
        \State
        $CO.\m{append}(\llmResponse)$ \Comment{\commentstyle{Preliminary Screening}}
    % \State $\eval_{\m{co}}$.append($\m{resp_\m{refusal}}$) \Comment{\commentstyle{Preliminary Screening}}
    \Else
        \State $\widetilde{\m{Trpl}} \gets$ \Call{ExtractTriple}{$\m{Resp.reasoning}$} 
        % \Comment{\commentstyle{Extract useful triples}}
        \State $\semantic_{\m{resp}}, \semantic_{\m{ground}} \gets$ \Call{BuildGraph}{$\widetilde{\m{Trpl}}, \groundTruthTriples$} 
        % \Comment{\commentstyle{Build semantic structure}}
        % \State $\deriveKG{\m{Resp}}{\rall}{\semantic_{\m{resp}}}{\semantic_{\m{ground}}}$\lnk{More specific} \Comment{\commentstyle{Extract Semantic Structure}}
        \State $\m{s}_{\m{e}} \gets$ $\similarity_\m{e}${$(\semantic_{\m{resp}}$, $\semantic_{\m{ground}})$} \Comment{\commentstyle{Calculate edge similarity}}
        \State $\m{s}_{\m{n}} \gets$ $\similarity_\m{n}${$(\semantic_{\m{resp}}$, $\semantic_{\m{ground}})$} \Comment{\commentstyle{Calculate node similarity}}
        % \If{$edge\_sim < \theta\_e$ and $node\_sim < \theta\_n$}
            % \State $\eval$.append($response$) \Comment{\commentstyle{Append mixed hallucination}}
        \If {$s_e < \theta_{e}$ and $s_n < \theta_{n}$}
            \State
            $OL.\m{append}(\llmResponse)$  \Comment{\commentstyle{Append  overlapped cases}}
        \ElsIf{$\m{s}_{\m{e}} < \theta_{\m{e}}$}
            \State 
            $EI.\m{append}(\llmResponse)$  \Comment{\commentstyle{Append error inference}}
        \ElsIf{$\m{s}_{\m{n}} < \theta_{\m{n}}$}
            \State 
            $EK.\m{append}(\llmResponse)$  \Comment{\commentstyle{Append error knowledge}}
        \Else
            \State
            $CO.\m{append}(\llmResponse)$
            \Comment{\commentstyle{Append correct response}}
        \EndIf
    \EndIf
    % \EndFor
    % \State $\eval$.extend($\eval_{\m{ei}}, \eval_{\m{ek}}, \eval_{\m{co}}$) \Comment{\commentstyle{Merge the result}}
    % \State $evaluation\_result \gets$ \Call{GenerateResult}{$hallu\_both$, $hallu\_ei$, $hallu\_ek$}
    % \shil{(we use three lists \_both, \_ei, and \_ek)} $contradictory\_answers$} 
    % \Comment{Generate evaluation result}
    \State \Return $CO, EK, EI, OL$ 
    % \Comment{\commentstyle{Return the result}}
\EndFunction
\end{algorithmic}
\end{algorithm}

\textbf{Edge Vector Metamorphic Oracle ($MO_E$)}: This oracle is based on the similarity of edge vectors between $\semantic_{\m{resp}}$ and $\semantic_{\m{ground}}$. If the vector similarity ($\m{s}_{\m{e}}$) between the edges in the $\semantic_{\m{resp}}$ and those in $\semantic_{\m{ground}}$ falls below a predetermined threshold $\theta_{\m{e}}$, it indicates that the LLM's answer significantly diverges from the ground truth. This suggests the presence of an FCH, and vice versa. %Conversely, the LLM's answer is considered to {align with} the ground truth. % indicates the correct answer. Otherwise, it detects an occurrence of FCH.
More specifically, we utilize \emph{Jaccard Similarity}~\cite{J_S} to calculate the similarity score between edge vectors extracted from $\semantic_{\m{resp}}$ and  $\semantic_{\m{ground}}$. 
$$
\similarity_{\m{e}}(\semantic_{\m{resp}}, \semantic_{\m{ground}}) = \frac{|\widetilde{E}_{\m{resp}} \cap \widetilde{E}_{\m{ground}}|}{|\widetilde{E}_{\m{resp}} \cup \widetilde{E}_{\m{ground}}|}, $$check if $  \similarity_{\m{e}}(\semantic_{\m{resp}}, \semantic_{\m{ground}})  < \theta_{\m{e}} \ 
$~
where $\widetilde{E}_{\m{resp}}$ and $\widetilde{E}_{\m{ground}}$ denote the set of edges extracted from $\semantic_{\m{resp}}$ and $\semantic_{\m{ground}}$. 
% , and \( \theta_E \) is a predefined threshold~(to be detailed in Section~\ref{sec:ex_setup}). 
% Intuitively, the similarity score is calculated as the proportion of identical edges shared between the two sets against the total number of unique edges in both sets. If the similarity score is smaller than the threshold, then an FCH is detected. Note that when determining the joint and union of sets $E_{LLM}$ and $E_{GT}$, we consider two edges as identical if their corresponding relations are identical or represented by synonymous words, and vice versa.  

% Define a function \( \text{Sim}_E(KG_{LLM}, KG_{GT}) \) to calculate the similarity of edge vectors between the knowledge graph generated by the language model, \( KG_{LLM} \), and the ground truth knowledge graph, \( KG_{GT} \).
% $$\text{Sim}_E(KG_{LLM}, KG_{GT}) = \frac{|E_{\text{LLM}} \cap E_{\text{GT}}|}{|E_{\text{LLM}} \cup E_{\text{GT}}|}$$

% If \( \text{Sim}_E(KG_{LLM}, KG_{GT}) < \theta_E \), where \( \theta_E \) is a predefined threshold, then an error inference hallucination is identified.

   
\textbf{Node Vector Metamorphic Oracle ($MO_N$)}: This relation examines the similarity of node vectors between $\semantic_{\m{resp}}$ and $\semantic_{\m{ground}}$. 
Defined in a similar manner as $MO_E$, if the node similarity between the nodes ($\m{s}_{\m{n}}$) in the $\semantic_{\m{resp}}$ and those in $\semantic_{\m{ground}}$ falls below a predetermined threshold $\theta_{\m{n}}$, it indicates that the LLM's answer significantly diverges from the ground truth, and vice versa.
$MO_N$ can be captured by the Jaccard Similarity, defined as follows:
%When the similarity between the nodes in the $KG_{LLM}$ and those in $KG_{GT}$ is below a predetermined threshold, this metamorphic relation exposes an error knowledge hallucination.
%Define a function \( \text{Sim}_N(KG_{LLM}, KG_{GT}) \) to measure the similarity of node vectors between \( KG_{LLM} \) and \( KG_{GT} \).

$$\similarity_{\m{n}}(\semantic_{\m{resp}}, \semantic_{\m{ground}}) = \frac{|N_{\m{resp}} \cap N_{\m{ground}}|}{|N_{\m{resp}} \cup N_{\m{ground}}|}, $$check if $
\similarity_{\m{n}}(\semantic_{\m{resp}}, \semantic_{\m{ground}})  < \theta_{\m{n}}  
$~
where $N_{\m{resp}}$ and $N_{\m{ground}}$ denotes the set of nodes extracted from $\semantic_{\m{resp}}$ and $\semantic_{\m{ground}}$.
% , and \( \theta_N \) is a predefined threshold~(to be detailed in Section~\ref{sec:ex_setup}). 
% Intuitively, the similarity score is calculated as the proportion of identical edges/nodes shared between the two sets against the total number of unique edges/nodes in both sets. If the similarity score is smaller than the threshold, then an FCH is detected. 
Note that when determining the joint and union of the edges/nodes sets, we consider two edges/nodes as identical if their corresponding entities are identical or synonymous, and vice versa.
%If \( \text{Sim}_N(KG_{LLM}, KG_{GT}) < \theta_N \), where \( \theta_N \) is a predetermined threshold, then an error knowledge hallucination is recognized.


% \textbf{Answer Consistency Metamorphic Oracle ($MO_C$)}: This relation is distinct in that it focuses on the consistency or inconsistency of the model's final answer with the ground truth, regardless of whether the node or edge vector similarities meet the thresholds. This relation helps identify situations where, despite a seemingly reasonable reasoning process, the outcome contradicts the facts (or vice versa), indicating contradictory answers.

% Consider the final answer \( Ans_{LLM} \) provided by LLMs and the ground truth answer \( Ans_{GT} \).
% If the similarity between \( KG_{LLM} \) and \( KG_{GT} \) is above or below the threshold but there exists a contradiction or consistency between \( Ans_{LLM} \) and \( Ans_{GT} \), this scenario is considered as a contradictory answer.
% $$ \text{If similarity between } KG_{LLM} \text{ and } KG_{GT} \text{ is above or below threshold and } A_{LLM} \neq A_{GT} \text{, then identify a contradictory answer.} $$


% \subsection{Feedback Loop}
% Based on the evaluation results from the preceding section, this module is employed to select test case types that trigger higher levels of FCHs and to mutate them for more hallucination answers, thereby enhancing the ability of the testing process to expose LLM FCHs.


% \section{Implementation}
% 

% To implement factual knowledge extractor, we use Wikipedia and Wikidata as sources to extract entities and structured information as base factual knowledge. 
% As one of the most widely used knowledge bases, wikipedia is renowned for its diverse range of entities and its extensive content coverage across multiple domains. Supplementing Wikipedia, Wikidata provides structured data, making it an ideal resource for the construction and enrichment of factual knowledge. 
% After downloading the latest Wikipedia dump, we employ wikiextractor\cite{Wikiextractor2015} to extract relevant text from Wiki pages. 
% In parallel, we invoke Wikidata's SPARQL\cite{sparql} query module for the extraction of triples. 
% The factual knowledge extractor extracts entities and structured information sourced from Wikipedia as base factual knowledge. Wikipedia, one of the most widely used knowledge bases, is renowned for its diverse range of entities and its extensive content covering multiple domains (such as science, art, history, geography, and technology). Wikidata is a supplement of Wikipedia and it provides structured data, resulting in an ideal resource for the construction and enrichment of factual knowledge.
% After extracting base factual knowledge, we employ wikiextractor\cite{Wikiextractor2015} to extract relevant text from Wikipedia pages, and in parallel, extracting triples by invoking Wikidata's SPARQL\cite{sparql} query module.

% For logic reasoning processor, we implement it with SWI-Prolog, an open-source advanced logical programming interpreter. Note that to prevent errors due to excessive stacked strings, when a large number of facts need to be inserted into Prolog, we employ a sampling method and extract a subset of entities to form a query, ensuring that the logical processor can operate normally.


\section{Evaluation}\label{sec:eval}
% We implement \tool with 4,071 lines of Python code and 826 lines of Prolog code. 
Our evaluation targets the following research questions:

\begin{itemize}[wide]
\item \textbf{RQ1} (Effectiveness): How effective is \tool for in generating test cases and identifying LLM FCH issues?

\item \textbf{RQ2} (Hallucination Categorization and Analysis): What are the categorizations of LLM FCH issues? 

\item \textbf{RQ3} (Ablation Study): Whether the four types of logic reasoning rules %(\figref{fig:basic_op_for_predicates}) 
can identify LLM FCH issues independently? 

\item \textbf{RQ4} (FCH Caused by Temporal Reasoning): 
How effective is \tool in detecting temporal-logic-related hallucinations; what are their categorizations; 
and which temporal operators trigger the most/least hallucinations? 
\end{itemize}


%$\bullet$ \textbf{RQ4 (Comparison with Existing Works): What superiority does \tool possess over existing works?} This RQ qualitatively compares \tool with existing FCH detection approaches and logical reasoning benchmarks.

% $\bullet$ \textbf{RQ4 (Feedback Effectiveness): Can feedback segment effectively trigger FCH of LLMs when faced with a specific category of questions?} This RQ discovers which type of questions could better trigger FCHs of LLMs in various domains and demonstrate the effectiveness of the feedback procedure.
% $\bullet$ \textbf{RQ4 (Real-world Application):}
% Exploring Model Editing for Enhancement (Double-confirm the usefulness of generated knowledge)
% This RQ attempts to mitigate LLM hallucination through model editing techniques.

\subsection{Experimental Setup}\label{sec:ex_setup}
\noindent \textbf{Knowledge Extraction.} 
We use Wikipedia and Wikidata as sources to extract entities and structured information as base factual knowledge. After downloading the latest Wikipedia dump, we employ wikiextractor~\cite{Wikiextractor2015} to extract relevant text from Wiki pages. In parallel, we invoke Wikidata's SPARQL~\cite{sparql} query module to extract facts. 
Through data processing involving simplification and filtration, we amass a collection of basic factual knowledge encompassing 54,483 entities and 1,647,206 facts. %\shil{Any new entities or facts?}


\noindent \textbf{Logic Reasoning Processor.} 
For the logic reasoning module, we apply SWI-Prolog~\cite{wielemaker2012swi}, an open-source advanced logical programming interpreter. To effectively prevent errors due to excessive stacked strings, and ensure the proper operation of the logical processor when inserting a large number of facts into Prolog, we employ a sampling method and extract a subset of entities to form a query. 
%\syh{sampling and subset?}

\noindent \textbf{Models Under Test.} 
To guarantee a reliable evaluation for the RQs, we evaluate nine state-of-the-art large language models with \tool. Considering the diverse nature of LLMs, we select two distinct categories: the first category comprises API-accessible models with closed-source architecture including ChatGPT (gpt-3.5-turbo-0613), GPT-4, and GPT-4o~\cite{OpenAI2023GPT4TR}, and the second category consists of open-source LLMs with deployability, including Llama2-7B-chat-hf, Llama2-70B-chat-hf~\cite{touvron2023llama}, Mistral-7B-Instruct-v0.2~\cite{jiang2023mistral}, Mixtral-8x7B-Instruct-v0.1~\cite{jiang2024mixtral}, Llama3.1-8B-Instruct, and Llama3.1-70B-Instruct~\cite{llama32024tr}.  
% \syh{is here up to date?}

\noindent \textbf{Response Validation Threshold $\theta$.} 
%To validate responses from LLMs as described in \secref{response}, 
We apply StanfordOpenIE~\cite{angeli-etal-2015-leveraging} for knowledge triple extraction from LLM responses and then use SentenceBERT~\cite{reimers2019sentence} to calculate the vector similarity of nodes and edges from the constructed semantic structures. We also utilize GPT-4o to extract triples for some complex responses that StanfordOpenIE cannot handle effectively. Here, we set the threshold to 0.8, considering knowledge triples as semantically equivalent if they exceed this threshold and vice versa. To determine the threshold value, we sample 30 test cases and corresponding LLM responses from each of the nine knowledge domains listed in \figref{table:categories}. Through this analysis, we find that by setting the threshold values for both $\theta_\m{e}$ and $\theta_\m{n}$ at 0.8, with the given 270 test cases that are correctly classified, we can estimate the true positives among all test cases through \textit{Laplace's approach in the Sunrise problem}~\cite{laplace1951philosophical}, resulting in 99.6\% when detecting non-equivalent LLM answers as FCHs. In other words, all instances where an LLM's answer has a semantic similarity score below 0.8 compared to the ground truth were correctly identified as FCH cases. 

\noindent \textbf{Running Environment.} 
Our experiments are conducted on a server running Ubuntu 22.04 with two 64-core AMD EPYC 7713, 512 GB RAM, and two NVIDIA A100 PCIe 80GB GPUs. Our experiments consume a total of 240 GPU hours. %\shil{whether the GPU hours cover new experiments?}

% \subsection{Implementation}


\subsection{RQ1: Effectiveness}
% \yi{motivation, method, result}
% The efficacy of \tool as delineated in Section~\ref{method} is intrinsically linked to its effectiveness for identifying FCH issues. Our evaluation on \tool on comprehensive performances of diverse LLMs and the domain-specific effectiveness categorized for each LLM.
To reveal the effectiveness of \tool, we evaluate the statistics of test cases generated by \tool and then evaluate the capabilities of identifying LLM FCH issues with the generated test cases. 
To further assess the effectiveness of test cases for uncovering FCH issues in specific knowledge domains, we evaluate the performances of LLMs on test cases across various knowledge domains.

\head{Effectiveness on Generating (Non-Temporal) Q\&A Test Cases.} We apply \tool to generate a Q\&A test benchmark, amounting to a comprehensive total of 7,200 test cases, designed to provide a broad and detailed evaluation of LLM FCH issues across specific knowledge domains. 
% Note that \tool is a scalable framework and can generate Q\&A pairs according to the specific entity and its relevant properties.
% Table~\ref{table:statitics} lists the domain-specific test case statistics.
% \begin{table}[!ht]
%     % \def\arraystretch{1.0}
%     \setlength{\tabcolsep}{1ex}
% 	\centering
%     \large 
% 	% \footnotesize
% 	\caption{Dataset Statistics.}
%         \label{table:statitics}
% 	\resizebox{\linewidth}{!}{
% 	\begin{tabular}{c c c c c c c c c c c}
%     \toprule 
%     \textbf{Type} & \textbf{Culture} & \textbf{Geo.} & \textbf{History} & \textbf{People} & \textbf{Society} & \textbf{Tech.} & \textbf{Math} & \textbf{Health} & \textbf{Nature}\\
%     \midrule
%     \textbf{Count} & 2,265 & 1,892 & 1,354 & 4,084 & 2,184 & 1,677 & 194 & 334 & 224 \\
%     \bottomrule %添加表格底部粗线
% \end{tabular}}
% \end{table}

\begin{figure}[!t]
\hspace{-4mm}
\centering
\begin{subfigure}{}%{0.49\linewidth}
\centering
\includegraphics[width=1\linewidth]{fig/rq1_overall_hallucination_rate.pdf}\\
\vspace{-0.2cm}
\caption{Overall Hallucination Rate of Various LLMs}
\vspace{0.3cm}
\label{fig:overall}
\end{subfigure}
    % \hfill
    %\hspace{-0.5em}
    %\vspace{-0.3cm}
\begin{subfigure}{}%{0.51\linewidth}
\centering
\includegraphics[width=0.9\linewidth]{fig/rq1_each_hallucination_rate.pdf}
\caption{Hallucination Rate Heatmap of Specific Domain}
\label{fig:rq1.2}
\end{subfigure}
\vspace{-3mm}
\end{figure}


\head{Effectiveness across LLMs.} We instruct LLMs under test utilizing Q\&A pairs derived from \tool, and automatically label both hallucination and normal responses. The results are presented in \figref{fig:overall}, illustrating the proportion of hallucination responses versus normal responses from LLMs under test.

Among all models, GPT-4 exhibits the best performance, demonstrating the lowest proportion of hallucinatory responses in the test cases generated by \tool, at only 24.7\%, while GPT-4o and ChatGPT fall slightly behind with 
28.1\% and 42.1\%. Open-source LLMs including Llama2-7B-chat-hf, Llama3.1-8B-Instruct, and Mistral-7B-Instruct-v0.2 with fewer parameters perform worse, but their counterparts with larger parameters (i.e., Llama2-70B-chat-hf, Llama3.1-70B-Instruct, and Mixtral-8x7B-Instruct-v0.1) achieve higher normal response rates surpassing ChatGPT on \tool. %\shil{shall we add explanations for Llama3 series?}
This indicates that the test cases generated by \tool successfully trigger hallucination responses across various LLMs when confronted with questions requiring logical reasoning capabilities.
% Among all models, GPT-4 exhibits optimal performance, with the lowest proportion of hallucination responses at only 24.7\% on test cases generated by \tool. ChatGPT, though slightly less effective, follows closely behind. In contrast, Llama2-7B-chat-hf demonstrated subpar performance in handling inferential questions, with the hallucination rate reaching 56.9\%. It is noteworthy that Mistral-7B-Instruct, having undergone meticulous instruction-based fine-tuning, exhibits performance surpassing that of ChatGPT on \tool.
% Remarkably, it also outperformed LLMs with larger parameters such as Llama2-13B-chat and Vicuna-13B-v1.5. We can infer that instruction tuning and alignment with human feedback can aid in mitigating hallucinations, as exemplified by the enhanced reasoning capabilities displayed by Mistral-7B-Instruct, a model fine-tuned on publicly available instruction datasets. Furthermore, a discernible performance gap persists between open-source and commercial models in addressing reasoning problems, highlighting an urgent need for more effective fine-tuning strategies and methods to reduce hallucinations.


\head{Effectiveness on Specific Domain Knowledge for Each LLM.}
To further explore the effectiveness of \tool in identifying FCH issues spanning various domains of LLMs, we compare hallucination response across nine specific domain knowledge. 
\figref{fig:rq1.2} presents the generated heatmaps of the confusion matrices for the specific knowledge field hallucination response rate based on the testing results.
It can be clearly observed that different models exhibit varying strengths and weaknesses across distinct knowledge domains. 
% Overall, the performance in the domain of geographical knowledge stands out as the most proficient, with the overall probability of generating hallucination responses being approximately 39\% on average. 
% Taking GPT-4 as an example, it is evident that it performs commendably in the domains of geography, society, and technology. This suggests that the GPT-4 model has developed a relatively comprehensive knowledge framework in these areas, demonstrating strong generalization capabilities in reasoning and understanding entity relationships. 
% An interesting finding is that, within the domains of natural and physical sciences, where LLMs generally exhibit weaker performance, the formulation of generated questions necessitates an extensive understanding of astrophysical entities and their interrelationships. It is plausible that this realm of knowledge is underrepresented in the training datasets of contemporary LLMs, thereby requiring improved pre-training. Such a disparity in knowledge representation is likely a significant factor in the observed underperformance of LLMs in these specific domains.
An interesting finding is that, within the domains of natural sciences and mathematics, LLMs generally exhibit weaker performance. This is potentially because there are many astrophysical or mathematical entities and their interrelationships in generated test cases by \tool. To answer such questions, the LLM needs an extensive understanding of astrophysical knowledge and mathematical theory. Thus, we infer that this realm of knowledge is not well-covered in the training datasets of LLMs under test, thereby resulting in high hallucination rates. Such a disparity in knowledge is likely a significant factor in the observed underperformance of LLMs in these specific domains.

% \tool proficiently identified a significant propensity (ranging from 30\% to 60\%) for contemporary LLMs to generate hallucination responses when confronted with logical reasoning challenges. The test outcomes varied across different categories of questions, underscoring the ability of \tool to effectively discern and evaluate the FCH issues inherent in large language models.
% \lnk{too long}


\begin{tcolorbox}[title=ANSWER to RQ1, boxrule=0.8pt,boxsep=1.5pt,left=2pt,right=2pt,top=2pt,bottom=1pt]

Our evaluation using \tool reveals that existing LLMs have a notable tendency to produce FCH when faced with logical reasoning challenges, with hallucination rates ranging from 24.7\% to 59.8\%. The results varied across knowledge domains, highlighting that LLMs are more prone to FCH when answering questions that require highly specialized, domain-specific knowledge. 

\end{tcolorbox} 

\subsection{RQ2: FCH Categorization and Analysis}
\subsubsection{FCH Categorization}
We categorize the hallucination responses in more detail and focus primarily on the two types: error knowledge response, error inference response, and contradictory response. Note that we consider refusal to respond, such as `I don't know' due to the lack of relevant knowledge to be adhering to LLMs' honesty and truthfulness principles. Therefore, we categorize refusal to respond as a normal response. 
% Examples of the hallucination we found are illustrated in Figure~\ref{}. 
% For a fairer categorization, we apply GPT-4 for automatic assistance and manually check 100 responses that are randomly selected from hallucination-related test cases and draw our conclusions ultimately.
To ensure fair and unbiased categorization, 100 hallucination-related responses were randomly selected and independently categorized by three co-authors, who then discussed the results to reach a consensus categorization.
% To determine the appropriate threshold setting, authors independently label the selected test cases. Subsequently, after reaching a consensus, we compare them with the answers from the oracle and conclude to set the threshold to 0.8 ultimately.

\head{Error Knowledge Response.} Originated from LLMs utilizing erroneous or contextually inappropriate knowledge during the reasoning process.

\head{Error Inference Response.} The most frequently occurring type is attributed to the lack of reasoning power and flawed reasoning thoughts of LLMs.
% This category of hallucination represents the most frequently occurring type. This is attributed to the lack of reasoning power as well as the flawed reasoning thoughts of existing LLMs.

% \head{Contradictory Response.} Wrong reasoning process but along with a correct conclusion.

% \head{Meaningless Response.} 
% Meaningless and chaotic answers containing affirmative and negative responses.
% \lnk{need to rename}
% This type of hallucination contains both affirmative and negative responses, rendering the answer meaningless and chaotic.
% This type of r

% Furthermore, we observe that on Vicuna-13b-v1.3, a model fine-tuned on Llama, the likelihood of generating nonsensical responses to user queries has significantly increased. These meaningless replies include repeating the question instead of providing an answer, and producing responses entirely unrelated to the query such as writing useless code or counter-questioning the user.

\subsubsection{Hallucination Measurement} 
Here, we provide the distribution of the hallucination categorization results, as demonstrated in \figref{fig:rq2}. There is partial overlap between these two types of hallucinations because incorrect reasoning processes may also involve erroneous knowledge. Among these issues, there are several contradictory answers primarily arising from inconsistency between incorrect reasoning processes and correct answers; thus, it exists in these two types of errors. Error inference hallucination presents the most, totalling half of the results on average. 
%\shil{To check after adding GPT-40, is this statement still correct?} 
This indicates that the primary cause of FCH issues in logical reasoning is the insufficient reasoning capability of LLMs.
Besides, error knowledge adopted by LLMs during the logical reasoning process leads to 42.0\% %\shil{to check?} 
FCH issues on average. The overlaps account for 7.9\%-21.1\% at the hallucination ratio, which indicates there are entities where LLMs have learned entirely erroneous relevant information, necessitating the employment of certain measures for correction.
% As for the meaningless response, this category was primarily identified in the results of Llama2-7B-Chat and Vicuna-13B-v1.5, which indirectly evidences the relatively weaker reasoning capabilities of these two models. 
% Meaningless responses are extensively detected in the responses of the Llama2-7B-Chat, encompassing a significant number of 2,293 test cases. 
% For 2,293 test cases categorized as meaningful responses from Llama2-7B-Chat, we hypothesize that Llama2-7B-Chat tends to offer a neutral response aimed at satisfying all sides, yet this approach conflicts with the user's demand for a binary Yes/No answer, ultimately leading to a meaningless response.

\begin{figure}
\centering
\includegraphics[width=\linewidth]{fig/rq2.pdf}\\
    \vspace{-0.1cm}
    \caption{FCH Categorization.}
    \vspace{-0.3cm}
    \label{fig:rq2}
\end{figure}

\subsubsection{Case Study}
The preceding analysis broadly summarizes the distribution of categories for logical reasoning-related FCH. According to our investigation, error inference response and error knowledge response are the most prevalent two types.
% Now we delve into some more specific examples and provide three cases for inspiring insights.

% \head{Temporal Related Hallucination.}
\head{Error Inference Hallucination.}
One of the most common types of logical reasoning leading to error inference hallucination is temporal attribute reasoning, proven to be a category of reasoning task that performs poorly on LLMs~\cite{qiu2023large}. %Experiments on time-related reasoning tasks are comprehensively conducted and unsatisfactory performance of LLMs are observed.  
% Among the 678 time-related test cases that lead to FCH in GPT-3.5-turbo, there are 276 related to time duration, 313 related to chronological orders, and the rest 89 related to typical timings such as festivals. 
As illustrated in \figref{fig:case1}, error inference with correct knowledge leads to a hallucination response from Mistral-7B-v0.2. As knowledge provided by the LLM reasoning process, it is clear that the answer should be `Yes' as the 1874 Canadian federal election applies to the jurisdiction of Canada. However, it appears that the LLM has become ensnared by its limitations.
% Given that time duration is the most complicated scenario, we hereby provide an example. As illustrated in Figure~\ref{fig:case1}, time duration-related Q\&A pairs suffer from error inference hallucination. LLMs could obtain the correct factual knowledge, i.e. William Armstrong, 1st Baron Armstrong (1810-1900), and Richard Leveridge (1670-1758). The expected answer should be Yes, while LLM appears to be confused with the lifespan calculation.
% The most puzzling aspect is that despite listing valid and correct knowledge internally, the LLM still employs a divergent line of reasoning compared to a commonly accepted human logic to determine the outcome. 
A possible explanation for this phenomenon is that the LLM does not utilize its reasoning abilities but rather relies on unreliable intuition to respond when faced with a question lacking detailed instructions. This insight inspires us to explore methods for effectively enhancing the reasoning capabilities of LLMs through a single interaction, guiding these models toward uncovering answers in a way that mirrors human reasoning processes.

\begin{figure}
\centering
\includegraphics[width=\linewidth]{fig/drowzee-case-1.pdf}\\%\vspace{-0.3cm}
\caption{Error Inference Hallucination from Mistral-7B-v0.2}
%\vspace{-0.2cm}
\label{fig:case1}
\end{figure}

\begin{tcolorbox}
\vspace{-0.15cm}
\textbf{Finding 1.} LLMs exhibit weaker performance in sensitivity to temporal information, as well as in their ability to discern sequential logic, which may result in error inference hallucination.
\vspace{-0.15cm}
\end{tcolorbox}

% \head{OOD Contextual Misleading Hallucination.}
\head{Error Knowledge Hallucination.}
\figref{fig:case2} demonstrates a classic example of LLM hallucination caused by using error knowledge for logical reasoning. General Dmitry Karbyshev (1880-1945) was a Russian Imperial Army soldier who served in several wars during World War I (1914-1918) and II (1939-1945), and Louis Bernacchi (1876-1942) was an Australian physicist and astronomer who served in the Royal Naval Volunteer Reserve during World War I. Thus, the ground truth answer to this question should be `Yes'. 
However, when testing with Llama2-7B-chat-hf, an %inspiring 
observation is that when LLMs encounter unfamiliar knowledge, they do not adhere to the honesty principle; instead, they fabricate knowledge and its sources. We subsequently employ an RAG-based scheme to reintroduce relevant knowledge, leading to the restoration of normal responses.
\begin{figure}
    \centering
    \small
    \includegraphics[width=\linewidth]{fig/drowzee-case-2.pdf}\\%\vspace{-0.3cm}
    \caption{Error Knowledge Hallucination from Llama2-7B-chat-hf.}
\vspace{-0.3cm}
\label{fig:case2}
%\vspace{-0.1cm}
\end{figure}

We further conduct an out-of-distribution (OOD) knowledge experiment to figure out the cause of error knowledge hallucination. OOD is another factor that could cause FCH issues~\cite{zhang2023hallucination}. We design contextual reasoning utilizing recent sporting events and natural disasters from Wikipedia since June 2023, which is considered unutilized information in LLMs' training data based on their up-to-date introductions. We construct a series of test cases containing contextual descriptions of recent events using \tool, observing whether LLMs can be guided to respond to OOD knowledge and trigger FCH. 
\figref{fig:case2.2} is a typical case of OOD contexts leading to error knowledge hallucination. In the initial test of GPT-3.5-turbo, we provide information on several wildfires that happened from July 2023 to December 2023, and we confirm that this information is not in the LLM's training data. The LLM subsequently indicates that it has acquired this knowledge through this interactive process. However, a turning point emerges when we use test cases designed by \tool in the second test. Despite our questions based on preliminary factual knowledge provided, the LLM still confidently responds with a wrong answer.
\begin{figure}
    \centering
    \small
    \includegraphics[width=\linewidth]{fig/drowzee-case-3.pdf}\\ %\vspace{-0.3cm}
    \caption{OOD-attributed Error Knowledge Example from GPT-3.5-turbo.}
\vspace{-0.5cm}
    \label{fig:case2.2}
\end{figure}

We analyze several potential causes for this situation. One possibility is that LLMs store incorrect knowledge in the first turn because what we provided was merely a list of events, rather than a list of events in their order of occurrence. In short, the normal reasoning process involves defining the earliest occurring events only after knowing the times of all events. However, the LLM opts to judge based on the order we provide event knowledge, which is contrary to facts. Another potential is that when LLMs encounter OOD knowledge if they do not strictly adhere to the principle of honesty by stating \textit{I do not know...}, they tend to complete and analyze the response based on error knowledge in their existing knowledge bases. Nevertheless, such responses are likely to induce hallucinations.

\begin{tcolorbox}
% \vspace{-0.15cm}
\textbf{Finding 2.} LLMs readily make erroneous assessments of misleading and unfamiliar knowledge and lead to error knowledge hallucination due to their assumptions.
% \vspace{-0.15cm}
\end{tcolorbox}


% \head{Refusal Normal Response.}
% During our analysis, we encountered an intriguing phenomenon, i.e. refusal normal response generated from GPT-4. It is important to note that our prompts include instructions mandating a definitive Yes/No response. While most models adhere to this directive for generating responses, GPT-4 occasionally deviates, offering a principled text \textit{As an AI, I don't know......} when it genuinely could not answer a question, as Figure~\ref{fig:case3} illustrated. We surmise that GPT-4, being the most advanced LLM in terms of performance and reasoning capabilities, retains a degree of critical thinking ability. It produces a reasoned response when confronted with demands exceeding its capabilities, in line with the principles of honesty preserved during its pre-training and fine-tuning. This aspect is something that most open-source models have yet to achieve. This insight inspires us to adhere to principles of honesty in the design and training of open-source LLMs. Additionally, it highlights the need to equip these models with robust critical thinking and logical reasoning capabilities to ensure more authentic and credible responses.
% \begin{figure}
%     \centering
%     \includegraphics[width=\linewidth]{fig/case3-cropped.pdf}\\
%     % \vspace{-0.3cm}
%     \caption{GPT-4 Refusal Normal Response Example.}
%     % \vspace{-0.3cm}
%     \label{fig:case3}
% \end{figure}
% \begin{tcolorbox}
% % \vspace{-0.15cm}
% \textbf{Finding 3.} More advanced LLMs demonstrate a stronger orientation towards honesty, critical thinking, and principled responses.
% % \vspace{-0.15cm}
% \end{tcolorbox}


\begin{tcolorbox}[title=ANSWER to RQ2, boxrule=0.8pt,boxsep=1.5pt,left=2pt,right=2pt,top=2pt,bottom=1pt]
The detected FCH can be categorized into two types and the lack of reasoning capabilities poses a broader threat than the use of incorrect knowledge or inadequate inference strategies. 
\end{tcolorbox} 

\begin{figure}[!b]
\centering\includegraphics[width=0.8\linewidth]{fig/rq3-2.pdf}\\
\caption{Generation Rules that Trigger the Most Hallucination Responses on diverse LLMs across domains. The Number on Each Cell (the Unit: \%) Represents the Triggered FCH Ratio of the Corresponding Rule type.}
%\vspace{-0.3cm}
    \label{fig:rq3}
\end{figure}

\subsection{RQ3: Ablation Study}
We conduct an ablation study to investigate the capacity of each inference rule so that they can be distinctly used to uncover anomalies.
The four types of rules illustrated in \figref{fig:basic_op_for_predicates} are separately applied to generate Q\&A pairs. The symmetric reasoning rule is primarily utilized within the composite reasoning rule and does not introduce new knowledge on its own. Therefore, we did not include the symmetric reasoning rule as a separate condition in our ablation study.
% and the statistics are manifested as follows: 5,679 for transitive rules, 3,829 for inverse rules, 3,881 test cases for negation rules, and 839 for composition rules that integrate multiple inference rules.
For better visualization and understanding, we present the distribution of hallucination-related responses discovered with diverse rule-generated questions by \tool in Figure~\ref{fig:rq3}. The figure illustrates which type of rule can trigger the most hallucination responses for different LLMs and different domains of knowledge. It is distinctly evident that following the successful generation of various test cases using the four rules and their combinations, a substantial number of hallucinations are elicited across nine LLMs, with the composite rule yielding the highest amount of hallucinations. Following closely behind are the test cases generated using transitive rules, which have triggered a significant rate of FCHs in both the health and society domains.

From the comparison between four inference rules, we can conclude that all four inference rules demonstrate effectiveness when generating FCH test cases and inducing hallucination performances for LLM interaction.



\begin{tcolorbox}[title=ANSWER to RQ3, boxrule=0.8pt,boxsep=1.5pt,left=2pt,right=2pt,top=2pt,bottom=2pt]
The experimental results showcase the independence of four inference rules in eliciting FCHs, and the composite rules can trigger the most FCHs across various domains, which has proved to be a sound approach to generating test cases. 
\end{tcolorbox} 



\subsection{RQ4: FCH Caused by Temporal Reasoning} 

This section presents the evaluation results based on temporal-property-related test cases. 

%\head{Effectiveness on Generating (Non-Temporal) Q\&A Test Cases.} We apply \tool to generate a Q\&A test benchmark, amounting to a comprehensive total of 7,200 test cases, designed to provide a broad and detailed evaluation of LLM FCH issues across specific knowledge domains. 

\begin{figure}[!b]
\centering
\includegraphics[width=1\linewidth]{fig/rq4-1.pdf}
\caption{LLM Hallucination Rate with Temporal Test Cases}
\label{fig:rq4-Effectiveness}
\end{figure}


\head{Effectiveness on Generating Temporal Q\&A Test Cases.} 
\tool generates 1,800 temporal test cases, and as illustrated in \figref{fig:rq4-Effectiveness}, there is a noticeable reduction in the hallucination rate for each LLM when compared to the results presented in \figref{fig:overall}, which suggests that LLMs possess some degree of temporal reasoning capabilities. However, it's important to note that these LLMs still exhibit temporal hallucinations, albeit to varying extents.
The result largely aligns with the previous discovery that the newer generation of LLMs performs better than the older version, as seen in Llama3-* and Llama2-*. However, fine-tuning LLM with more parameters does not improve reasoning ability, as seen in Llama2-7B and Llama2-70B. 



\head{Temporal Hallucinations Categorizations.} 
As illustrated in \figref{fig:rq4-Categorization}, inference errors constitute a higher proportion of hallucinations than knowledge errors, which is consistent with the previous finding of \figref{fig:rq2}. 

\begin{figure}[!t]
\centering
\includegraphics[width=1\linewidth]{fig/rq4-2.pdf}
\caption{RQ4: FCH Categorization.}
\label{fig:rq4-Categorization}
\end{figure}



% \begin{table*}[ht]
% \centering
% \begin{tabular}{l|c|c|c|c|c|c|c|c}
% \hline
% \textbf{Model} & \textbf{Next} & \textbf{Conj} & \textbf{Disj} & \textbf{Finally} & \textbf{Globally} & \textbf{Until} & \textbf{Neg} & \textbf{Mixed} \\ 
% \hline
% GPT-3.5-turbo  & 107 & 70  & 91  & 111 & 42  & 51 & 0 & 43 \\
% GPT-4          & 56  & 22  & 39  & 62  & 25  & 64  & 3  & 30 \\
% GPT-4o         & 64  & 51  & 87  & 26  & 12  & 61  & 0  & 30 \\
% Llama2-7b      & 136 & 134 & 132 & 104 & 35  & 84  & 4  & 46 \\
% Llama2-70b     & 132 & 125 & 122 & 126 & 40  & 74  & 22 & 52 \\
% Llama3.1-8b      & 75  & 81  & 95  & 64  & 16  & 78  & 6  & 52 \\
% Llama3.1-70b     & 59  & 54  & 141 & 57  & 19  & 73  & 0  & 41 \\
% Mistral7b      & 107 & 19  & 44  & 110 & 41  & 65  & 0  & 45 \\
% Mistral8x7b    & 63  & 82  & 106 & 65  & 22  & 70  & 2  & 42 \\
% \hline
% \end{tabular}
% \caption{Hallucination Case Counts among Different Temporal Operators}
% \label{tab:counts}
% \end{table*}

\begin{figure}[ht]  % figure* 用于双栏环境中的跨栏图像
    \centering
    \includegraphics[width=0.75\linewidth]{fig/ablation_heatmap.pdf}  % 图片宽度设置为 \textwidth
    \caption{Hallucination Rates wrt Temporal Operators}  % 图注
    \label{fig:heatmap}  % 图像引用标签
\end{figure}



%\head{Which temporal operators trigger the most/least hallucinations}
\head{Ablation Study upon  Temporal Operators.} 
Figure \figref{fig:heatmap} illustrates the results of an ablation study examining various types of temporal test cases. We categorize these temporal test cases based on their outermost layer operator. Most of the test cases are single-layer temporal operators, except for conjunction and disjunction, which may include nested temporal formulas.
We record the likelihood of each type of temporal test case triggering hallucinations. For instance, when testing GPT-4, it is observed that 20\% of the test cases related to the "Next" operator successfully trigger hallucinations. 
Notably, the "Neg" operator triggers the fewest hallucinations, whereas operators like "Finally," "Globally," and "Until" lead to the highest occurrence of hallucinations.
Overall, these findings indicate that a single layer of temporal operators is sufficiently effective in detecting LLM hallucinations related to the temporal reasoning capability. 

%evaluating the reasoning capabilities of LLMs.
%, compared to the \syh{mixed or nested} test cases. 





% \subsection{RQ4: Comparison with Existing Works}
% We qualitatively compare \tool with the state-of-the-art FCH evaluation approaches and existing natural language reasoning benchmarks to demonstrate the superiority.

% As illustrated in Table~\ref{table:comparison1}, we enumerate the characteristics of the sota FCH evaluation approaches. To assess FCH in LLM responses, existing approaches uniformly opt for a wiki-related knowledge base as the foundation for constructing ground truth facts. Their main distinction from \tool lies in the manner of task construction and the metrics employed to measure hallucinations.

% \textbf{Task Construction Methods.} Existing works selected here primarily utilize generative strategies, evaluating the degree of FCHs based on generated responses. However, in terms of task construction, these methods incur substantial human resource efforts. Apart from the KoLA-KM, KA~\cite{yu2023kola}, which is essentially a collection of existing Q\&A datasets, both TruthfulQA~\cite{lin-etal-2022-truthfulqa} and HaluEval~\cite{HaluEval} rely on human annotations to construct Q\&A pairs. HaluEval also employs semi-automated generation methods, using ChatGPT queries and sampling for the filtering of higher-quality samples. \tool, on the other hand, utilizes Prolog-assisted automatic inference to derive new knowledge triples and generate templates for new questions, achieving maximum automation of construction while ensuring the complexity of the questions.

% \textbf{Response Evaluation Metrics.} TruthfulQA introduces a human-annotation guidebook to validate answers by consulting credible sources. Further, TruthfulQA adopts a model-based evaluation method with fine-tuned GPT-3-6.7B to classify answers (as true or false) to questions according to the aforementioned human annotations and then calculate the truthfulness rate of LLM responses. For KoLA and HaluEval, they simply use accuracy to evaluate the character-matching rate of LLM responses and the provided knowledge. Thus, \tool considers the structural similarity of LLM responses with original knowledge triples and the reasoning process, offering superiority over those simple evaluation metrics.

% \begin{table*}[!ht]
%     % \def\arraystretch{1.0}
%     \setlength{\tabcolsep}{1ex}
% 	\centering
% 	% \footnotesize
% 	\caption{Comparison with SOTA FCH Evaluation Approaches.}
%  % \lnk{need to refine}}
%         \label{table:comparison1}
% 	\resizebox{\linewidth}{!}{
% 	\begin{tabular}{c c c c c}
%     \toprule 
%     \textbf{Dataset} &\textbf{Fact Source} & \textbf{Construction Method} &\textbf{Test Oracle} \\
%     \midrule
%     TruthfulQA & Wikipedia pages \& websites & Human annotations & Truthfulness Rate\\
%     \midrule
%     KoLA-KM, KA & Wikidata5M \& websites & Existing datasets consolidation & Standardized score (F1)\\ 
%     \midrule
%     HaluEval-QA & Wikipedia & Human annotations \& ChatGPT query & Accuracy\\ 
%     \midrule
%     \tool{}-Dataset  & Wikidata triples & Prolog-aided reasoning \& template-based generation & Structural Similarity\\
%     \bottomrule
%     \end{tabular}}
% \end{table*}

% As listed in Table~\ref{table:comparison2}, we provide several benchmarks aided for natural language reasoning. Existing reasoning benchmarks lean more towards logical predicate-formatted inputs and outputs, lacking natural language-formatted questions, thus limiting their suitability for testing with LLMs.

% \begin{table*}[!ht]
%     % \def\arraystretch{1.0}
%     \setlength{\tabcolsep}{1ex}
% 	\centering
%     \small
% 	% \footnotesize
% 	\caption{Comparison with Natural Language Reasoning Benchmarks.}
%  % \lnk{need to refine}}
%         \label{table:comparison2}
% 	\resizebox{\linewidth}{!}{
% 	\begin{tabular}{c c c c c c}
%     \toprule 
%     \textbf{Benchmark} &\textbf{Size} &\textbf{Reasoning Type} & \textbf{Data Source} & \textbf{Task}& \textbf{Automation}\\
%     \midrule
%     FOLIO & 1.4k & First-order logic reasoning & Expert-written & Theorem Proving & \ding{55}\\
%     \midrule
%     DEER & 1.2k & Inductive reasoning & Wikipedia & Rule Generation & \ding{55}\\
%     \midrule
%     % HotpotQA & 112k & Multi-hop reasoning & Wikipedia & Question Answering & \checkmark \\
%     % \midrule
%     \tool & Scalable & Deductive reasoning & Wikidata & Question Answering & \checkmark
%     \\
%     \bottomrule
%     \end{tabular}}
% \end{table*}

% \begin{tcolorbox}[title=ANSWER to RQ4, boxrule=0.8pt,boxsep=1.5pt,left=2pt,right=2pt,top=2pt,bottom=2pt]
% Compared with existing works, \tool exhibits better scalability, less human labor, and more reasonable evaluation metrics.
% \end{tcolorbox} 
\section{Limitations and future directions}
In this work, we introduce a theoretical framework for understanding the mechanism of weak-to-strong (W2S) generalization in the variance-dominated regime where both the student and teacher have sufficient capacities for the downstream task. Leveraging the low intrinsic dimensionality of finetuning (FT), we characterize model capacities from three perspectives: FT approximation errors for ``accuracy'', intrinsic dimensions for ``complexity'', and student-teacher correlation for ``alignment''. Our analysis shows that W2S generalization is driven by variance reduction in the discrepancy between the weak teacher and strong student features. 
This generalization analysis is followed by a case study on the relative W2S performance in terms of performance gap recovery (PGR) and outperforming ratio (OPR). We show that while larger sample sizes imply better W2S generalization in an absolute sense, the relative W2S performance can degenerate as the sample size increases.
Our results provide theoretical insights into the choice of weak teachers and sample sizes in W2S pipelines. 

An interesting implication of our analysis is that the mechanism of W2S may differ as the balance between variance and bias shifts. In the variance-dominated regime studied in this work, W2S can benefit from a lower intrinsic dimension of the strong student due to the resulting variance reduction in the subspace of discrepancy from the weak teacher. In contrast, in the bias-dominated regime, the lower approximation error of the strong student is generally brought by the larger ``capacity'' of the strong model corresponding to a higher intrinsic dimension~\citep{ildiz2024high,wu2024provable}. 
This calls for future studies on unified views and transitions between the two regimes, which will provide a more comprehensive understanding of W2S.
Toward this goal, a limitation of our analysis is the quantification for the advantage of W2S in bias (see \Cref{fn:bias_strong}), which could be a promising next step.


\section{Related works}

\paragraph{LLM alignment.}
Pretrained LLMs demonstrate remarkable capabilities across a broad spectrum of tasks \citep{brown2020language}.
Their performance at downstream tasks, such as conversational modeling, is significantly enhanced through alignment with human preferences \citep{ouyang2022training, bai2022training}. 
RLHF \citep{christiano2017deep} has emerged as a foundational framework for this alignment, typically involving learning a reward function via a preference model, often using the Bradley-Terry model \citep{bradley1952rank}, and tuning the LLM using reinforcement learning (RL) to optimize this reward. 
Despite its success, RLHF's practical implementation is notoriously complex, requiring multiple LLMs, careful hyperparameter tuning, and navigating challenging optimization landscapes.

Recent research has focused on simplifying this process. A line of works studies the direct alignment algorithms \citep{zhao2023slic, rafailov2024direct, azar2024general}, which directly optimize the LLM in a supervised manner without first constructing a separate reward model. In particular, the representative DPO \citep{rafailov2024direct} attracts significant attention in both academia and industry. After these, SimPO \citep{meng2024simpo} simplifies DPO by using length regularization in place of a reference model. 
% However, these approaches are primarily offline and can suffer from performance degradation when facing distribution shifts during deployment, yielding poorer generalization.
% Iterative DPO \citep{xiong2024iterative} attempts to address this by enabling iterative optimization for improved alignment.

% These challenges can be potentially tackled by connecting LLM alignment with IR.
Although LLMs are adopted for IR \citep{tay2022transformer}, there is a lack of study to improve direct LLM alignment with IR principles.
% While significant progress has been made in LLM alignment, the connection with IR remains largely unexplored.
% However, LiPO does not explore the potential of online optimization methods inspired by IR. 
This paper fills this gap by establishing a systematic link between LLM alignment and IR methodologies, and introducing a novel iterative LLM alignment approach that leverages insights from retriever optimization to advance the state of the art.
The most related work is LiPO \citep{liu2024lipo}, which applies learning-to-rank objectives.
% However, LiPO focuses on offline settings and it is unclear how to obtain list-wise data for it.
However, LiPO relies on off-the-shelf listwise preference data, which is hard to satisfy in practice.

\paragraph{Language models for information retrieval.}
Language models (LMs) have become integral to modern IR systems \citep{zhu2023large}, particularly after the advent of pretrained models like BERT \citep{kenton2019bert}.  
A typical IR pipeline employs retrievers and rerankers, often based on dual-encoder and cross-encoder architectures, respectively \citep{humeau2019poly}. 
Dense Passage Retrieval (DPR) \citep{karpukhin2020dense} pioneered the concept of dense retrieval, laying the groundwork for subsequent research. 
Building on DPR, studies have emphasized the importance of hard negatives in training \citep{zhan2021optimizing, qu2020rocketqa} and the benefits of online retriever optimization \citep{xiong2020approximate}.

In the realm of reranking, \citep{nogueira2019passage} were among the first to leverage pretrained language models for improved passage ranking. 
This was followed by MonoT5 \citep{nogueira2020document}, which scaled rerankers using large encoder-decoder transformer architectures, and RankT5 \citep{zhuang2023rankt5}, which introduced pairwise and listwise ranking objectives. 
Recent work has also highlighted the importance of candidate list preprocessing before reranking \citep{meng2024ranked}.

Despite the pervasive use of LMs in IR, the interplay between LLM alignment and IR paradigms remains largely unexplored. 
This work aims to bridge this gap, establishing a strong connection between LLM alignment and IR, and leveraging insights from both fields to advance our understanding of LLM alignment from an IR perspective.

\section{Conclusion}
This paper introduces \textbf{ReLearn}, a novel unlearning framework via positive optimization that balances forgetting, retention, and linguistic capabilities. 
Our key contributions encompass a practical unlearning paradigm, comprehensive metrics (KFR, KRR, LS), and a mechanistic analysis comparing reverse and positive optimization. 
%As underscored, unlearning should not only erase knowledge but also relearn knowledge for constructive outputs.

\label{sec:bibtex}
\section*{Limitations}
While ReLearn shows promising performance, several limitations remain.
(1) Computational Overhead: Data synthesis may hinder scalability.
(2) Metric Sensitivity: Our metrics still have limited sensitivity to subtle knowledge nuances.
(3) Theoretical Grounding: Understanding the dynamics of knowledge restructuring requires deeper theoretical investigation, which we plan to explore in the future work.

%\pagebreak

%\usepackage[round]{natbib}
% \bibliographystyle{ACM-Reference-Format}
\bibliographystyle{IEEEtran}
%\citestyle{apacite}
% \printbibliography
\balance
\normalem
\bibliography{8.ref}

%\newpage 
%\appendix


%% !TEX root = ../main.tex

\section{Empirical Studies}

\subsection{Sample Size and Regularization Strength}
\label{appendix:sec:empirical:sample_size_and_regularization_strength}
We reproduce figure \Cref{fig:simulation_n_and_alpha} from the main text in
\Cref{fig:simulation_n_and_alpha_copy}, which shows the impact of the sample size $n$
and the regularization strength $\alpha$ on the performance of GES and XGES variants. We
complete it with \Cref{fig:simulation_n_and_alpha_edges} to show the corresponding
number of predicted edges. It indeed reveals that increasing $n$ causes GES and its
variants to over-insert.

\begin{figure}[h]
    \centering
    \begin{subfigure}{0.45\linewidth}
        \centering
        \includegraphics[width=\linewidth]{fig/simulation_n_and_alpha.pdf}
        \caption{Evolution of the performance of GES and XGES variants with the number of
        samples $n$.}
        \label{fig:simulation_n_and_alpha_copy}
    \end{subfigure}
    \hfill
    \begin{subfigure}{0.45\linewidth}
        \centering
        \includegraphics[width=\linewidth]{fig/simulation_n_and_alpha_edges.pdf}
        \caption{Evolution of the number of predicted edges with the number of samples $n$.}
        \label{fig:simulation_n_and_alpha_edges}
    \end{subfigure}
    \caption{Evolution of the performance of GES and XGES variants with (left) the
    number of samples $n$, and (right) the regularization strength $\alpha$. Increasing
    $n$ hurts GES and its variants (as it reduces the BIC regularization and
    over-insertion is exacerbated) while it helps XGES. Increasing the BIC
    regularization with $\alpha$ helps GES but without letting it catch up with XGES.
    Missing points indicate the method did not run.}
\end{figure}

\begin{figure}[h]
    \centering
    \begin{subfigure}{0.49\linewidth}
        \centering
        \includegraphics[width=\linewidth]{fig/simulation_main_recall.pdf}
        \caption{Recall of GES and XGES on simulated data. }
    \end{subfigure}
    \hfill
    \begin{subfigure}{0.49\linewidth}
        \centering
        \includegraphics[width=\linewidth]{fig/simulation_main_precision.pdf}
        \caption{Precision of GES and XGES on simulated data.}
    \end{subfigure}
    \begin{subfigure}{0.49\linewidth}
        \centering
        \includegraphics[width=\linewidth]{fig/simulation_main_f1.pdf}
        \caption{F1 of GES and XGES on simulated data.}
    \end{subfigure}
    \caption{Performance of GES and XGES on simulated data measured with precision, recall and F1. With $n=10000, \alpha=2$ and 30 seeds (same data as \cref{fig:simulation_main}). XGES outperforms other methods in the three metrics.}
    \label{fig:simulation_main_precision_recall}
\end{figure}

\begin{figure}
    \centering
    \includegraphics[width=0.7\linewidth]{fig/simulation_main_score_diff.pdf}
    \caption{Difference in score between the true graph/MEC $G^*$ and the graph/MEC returned by GES and XGES on simulated data. XGES returns graphs with higher scores than GES.}
    \label{fig:simulation_main_score_diff}
\end{figure}


\subsection{Complementary metrics: precision, recall, F1 score, BIC score.}
\label{appendix:sec:empirical:f1_score}
We complement \Cref{fig:simulation_main} from the main text with
\Cref{fig:simulation_main_precision_recall} to show the F1 score of GES and XGES for the same simulated data, as well as the precision/recall breakdown. We also report the BIC score of the graphs returned by GES and XGES in \Cref{fig:simulation_main_score_diff}. 
We find similar conclusions as in the main text.

The methods return a MEC $\hat M$ to predict the true MEC $M^*$. The F1 score, precision
and recall are defined for binary classification problems. For each ordered pair of
nodes $(i,j)$ we say $M$ contains $(i,j)$ if $(i,j)$ is directed in $M$ from $i$ to $j$
or if $(i,j)$ is undirected in $M$ $(i,j)$ (but not if $(j,i)$ is directed in $M$ from $j$ to $i$). Then, the binary classification problem is to predict for each $(i,j)$ if
$M^*$ contains $(i,j)$ or not, predicted by whether $\hat M$ contains $(i,j)$ or not.

\clearpage
\newpage
\subsection{More variables and edge densities}
\label{appendix:sec:empirical:more_variables_and_edge_densities}
We show the performances of the methods on more combinations of $d$ and $\rho$ in
\Cref{fig:simulation_d_and_rho}. We fixed $n=10000, \alpha=2$.
We find similar trends as in \Cref{fig:simulation_main} and \Cref{fig:simulation_correlation_speed}.

For edge density $\rho=1$, we find no significant difference between GES and XGES. This
is expected as the true graph is really sparse and the methods are less likely to
encounter local optima.
\begin{figure}[h]
    \centering
    \includegraphics[width=0.98\linewidth]{fig/simulation_shd_wide_d.pdf}
    \caption{Performance of GES and XGES on more combinations of $d$ and $\rho$. We fixed
    $n=10000, \alpha=2$ and we report error bars and averages over 5 seeds. XGES consistently outperforms GES and its variants. The dashed lines indicate the number of true edges.}
    \label{fig:simulation_d_and_rho}
\end{figure}


\subsection{Number of calls to the scoring function}
\label{appendix:sec:empirical:number_of_calls}
We show the number of calls to the scoring function for the different methods in
\Cref{fig:simulation_number_of_calls}. We find that even though XGES repeatedly applies
XGES-0, it only makes around one order of magnitude more BIC score evaluations. fGES was
not included because we could not extract the number of calls from the Tetrad
implementation. 

\begin{figure}[h]
    \centering
    \includegraphics[width=0.4\linewidth]{fig/n_bic_eval.pdf}
    \caption{Number of calls to the scoring function for the different methods. We fixed
    $n=10000, \alpha=2$, and we report error bars and averages over 5 seeds.}
    \label{fig:simulation_number_of_calls}
\end{figure}

\subsection{The OPS variant}
\label{appendix:sec:empirical:ops}
\citet{chickering2002optimal} evaluated GES against a variant called OPS that also
    considered insertions and deletions simultaneously. But OPS did not prioritize
    deletions over insertions, resulting in limited changes to GES in their experiments.
    We show the performance of OPS in \Cref{fig:simulation_ops}. We
    corroborate the results of \citet{chickering2002optimal} that OPS has performances
   similar to GES.

 Mathematically, deleting an edge can only increase the BIC score by at most $\frac{\alpha}{2} \log n$ which is usually smaller than the increase from inserting an edge.
 Hence even if OPS considers deletions and insertions ``together'', we conjecture that most deletions are only considered at the end of the search because insertions have higher scores and are applied first. OPS then encounters the same local maxima as GES. 
 This highlights the importance of prioritizing deletions over insertions with XGES.\looseness=-1

 \begin{figure}
    \centering
    \includegraphics[width=0.6\linewidth]{fig/simulation_main_shd_with_ops.pdf}
    \caption{Same experiment as \Cref{fig:simulation_main} but with the addition of the OPS variant. OPS performs very similarly to GES, suggesting that XGES-0's heuristic favoring deletions over insertions in XGES-0 is important. }
    \label{fig:simulation_ops}
 \end{figure}


\subsection{The double descent of GES with the sample size}
\label{appendix:sec:empirical:double_descent}
We show the performance of GES and XGES on a small graph with $d=15$ and $\rho=2$ for
sample sizes up to $n=10^8$ in \Cref{fig:simulation_double_descent}. XGES monotonically and quickly improves to an SHD of 0. We find that GES
improves from $n=10^2$ to $n=10^4$ but worsens around $n=10^4$. It eventually improves
again after $10^6$. The error bands are bootstrapped 95\% confidence intervals over 30 
seeds. We chose the graph to be small so that we could observe the second descent of
GES with sample sizes up to $10^8$. However, we doubt that such large sample sizes
are practical. We further believe that the issue worsens with larger graphs. (\textit{Note: We reimplemented GES and GES-r to scale to $n=10^8$, we verified that we obtained the same results as the original implementations for $n$ up to $10^6$. See parameter \texttt{-b} in the XGES code.})

When the sample size increases, the strength of the BIC regularization relative to the likelihood decreases in $\frac{\log(n)}{n}$. We hypothesize that GES worsens because
that decrease might lead to over-inserting, and the encounter of local optima. 


\begin{figure}[h]
    \centering
    \includegraphics[width=0.7\linewidth]{fig/simulation_double_descent.pdf}
    \caption{Performance of GES and XGES on a small graph with $d=15$ and $\rho=2$ for
    sample sizes up to $n=10^8$. We find that GES worsens around $n=10^4$ but improves
    after $n=10^6$. It exhibits a double descent behavior with the sample size.}
    \label{fig:simulation_double_descent}
\end{figure}


\subsection{The impact of the simulation}
\label{appendix:sec:empirical:impact_of_simulation}

We describe the simulation procedure and show the impact of changing it. 

\subsubsection{The simulation procedure}
\label{appendix:sec:simulated_data}

In the experiments, we use the following procedure to obtain simulated data.

\begin{enumerate}
    \item Select $d$ and $\rho$.
    \item Draw a DAG $G^*$ as follows:
        \begin{itemize}[leftmargin=*]
            \item $G^* \sim \text{Erdos-Renyi}(\#nodes=d, p=\frac{2\rho}{d-1})$.
            \item Orient each edge $(i,j)$ from $\min(i,j)$ to $\max(i,j)$, that forms a
            DAG.
            \item Draw a permutation $\sigma$ of $[d]$ and relabel each node $i \mapsto
            \sigma(i)$. 
        \end{itemize}
    \item Choose a distribution $p^*$ as follows for each $i
    \in [d]$:
    \begin{itemize}[leftmargin=*]
        \item Sample weights $W_i$ away from $0$ to ensure faithfulness.
        \begin{enumerate}
            \item In most simulations, we draw $W_i \in [1,3]^{\left|\Pa^i_{G^*}\right|}$ from
            $\mathcal{U}([1,3])$.
            \item In \Cref{fig:simulation_main_negative}, we change the simulation to sample weights $W_i \in
            [-3,-1] \cup [1,3]$, by additionaly drawing a Bernoulli variable $b_i \sim
            \mathcal{B}(0.5)$ and setting $W_i \leftarrow -W_i$ if $b_i=1$.
        \end{enumerate}
        
        
        \item $L_1$-normalize $W_i$ as $W_i \leftarrow W_i / \sum_j |W_{ij}|$.
        \item Draw the scale of the Gaussian $\varepsilon_i \sim \mathcal{U}(0, \varepsilon_{\text{max}})$. 
        \begin{enumerate}
            \item By default in most plot $\varepsilon_{\text{max}}=0.5$. 
            \item We vary it in \Cref{fig:simulation_epsilon}.
        \end{enumerate}
        
        
        \item Define $p^*(x_i \mid x_{\Pa^i_{G^*}}) = \mathcal{N}(W_i^\top
        x_{\Pa^i_{G^*}}, \varepsilon_i^2)$.
    \end{itemize}
    \item Draw $n$ i.i.d. samples from $p^*$.
\end{enumerate}


\subsubsection{Varying the scale of the Gaussian noise}

We show the impact of varying the scale of the Gaussian noise $\varepsilon_{\text{max}}$
in \Cref{fig:simulation_epsilon}. As expected, GES and XGES are almost not impacted by
the scale of the noise. This is expected, as the methods explicitly model the noise variable.

\begin{figure}[h]
    \centering
    \includegraphics[width=0.9\linewidth]{fig/simulation_noise.pdf}
    \caption{Performance of GES and XGES on different scales of the Gaussian noise. We
    fixed $n=10000, \alpha=2$.}
    \label{fig:simulation_epsilon}
\end{figure}


\clearpage
\subsubsection{Different sampling of weights}
We find similar conclusions when we change the sampling of weights from $W_i \in [1,3]$
to $W_i \in [-3,-1] \cup [1,3]$ . We show the results in
\Cref{fig:simulation_main_negative}.

\begin{figure}[h]
    \centering
    \includegraphics[width=0.5\linewidth]{fig/simulation_main_negative.pdf}
    \caption{Performance of GES and XGES similar to \Cref{fig:simulation_main} but with
    a different sampling of weights $W$. $n=10000, \alpha=2$. }
    \label{fig:simulation_main_negative}
\end{figure}


% The Tetrad implementation, called fGES, considers a single Insert(X,Y,T S) per (X,Y) as
% the best insert over all T (in term of score, not validity). Yet, the best T might lead
% to an invalid insert, and so all X->Y insert (potentially with another T), will be
% ignored. 



\clearpage
\section{Theoretical Details}
We detail the theoretical guarantees of GES and XGES. We first recall the problem only
in terms of MECs and not CPDAGs. We review the theoretical guarantees of GES and prove
those of XGES. We then show how the MEC formulation can be translated into a CPDAG
formulation for Deletes and Reversals.

\parhead{Notations and Vocabulary.}
\begin{itemize}[noitemsep]
    \item $d$: number of variables.
    \item $G$: a DAG over $d$ variables.
    \item $P$: a PDAG over $d$ variables.
    \item $M$: a MEC over $d$ variables.
    \item $\mathcal{M}$: space of all MEC over $d$ variables.
    \item $S$: a locally consistent score that is score equivalent. 
    \item $S(G)$ or $S(M)$ or $S(P)$: the score of a DAG $G$, a MEC $M$, or a CPDAG $P$.
    We hide the dependence in $\data$ for simplicity.\looseness=-1
    \item $G + (x \rightarrow y)$: the DAG obtained by adding the edge $x \rightarrow y$ 
    to $G$. It is undefined if $x \rightarrow y$ is already in $G$ or if the resulting
    graph is not a DAG.
    \item $G - (x \rightarrow y)$: the DAG obtained by removing the edge $x \rightarrow
    y$. It is undefined if $x \rightarrow y$ is not in $G$.
    % use the counter-clockwise arrow for reversals
    \item  $G \circlearrowleft (x \rightarrow y)$: the DAG obtained by reversing the
    edge $x \rightarrow y$. It is undefined if $x \rightarrow y$ is not in $G$ or if the
    resulting graph is not a DAG.
    \item An edge is \textit{compelled} in a MEC $M$ if it is always directed in the
    same direction in all DAGs in $M$. By definition, the compelled edges of $M$ are
    exactly the directed edges of $M$'s canonical PDAG. 
    \item An edge is \textit{reversible} in a MEC $M$ if it is directed in one direction
    in some DAGs in $M$ and the other direction in other DAGs in $M$. By definition, the
    reversible edges of $M$ are exactly the undirected edges of $M$'s canonical PDAG.
\end{itemize}


\subsection{Navigating the Space of MECs}
GES explores the space of MECs by iteratively going from one MEC to the other, each time
selecting one with a higher score. GES defines a set of possible candidates that can be
reached from a given MEC. This defines the search space of GES
\citep{chickering2002optimal}.


For a MEC $M$, define:
\begin{itemize}
    \item Insertions: $\mathcal{I}(M) = \{ M' \in \mathcal{M} \mid \exists G \in M, \exists
    (x,y) \in [d]^2, G + (x \rightarrow y) \in M'\}$.
    \item Deletions: $\mathcal{D}(M) = \{M' \in \mathcal{M} \mid \exists G \in M, \exists (x,y)
    \in [d]^2, G - (x \rightarrow y) \in M'\}$.
    \item Reversals : $\mathcal{R}_c(M) = \{ M' \in \mathcal{M} \mid \exists G \in M, \exists (x,y)
    \in [d]^2, (x,y) \text{ is compelled in }P, G \circlearrowleft (x \rightarrow y) \in M'\}$.
    \item Reversals: $\mathcal{R}_r(M) = \{ M' \in \mathcal{M} \mid \exists G \in M, \exists (x,y)
    \in [d]^2, (x,y) \text{ is reversible in }P, G \circlearrowleft (x \rightarrow y) \in M'\}$.
\end{itemize}

The original GES algorithm first navigates through MECs only using $\mathcal{I}$, and
then only using $\mathcal{D}$. GIES proposes to add a last step and navigate through MECs using only
$\mathcal{R}_c \cup \mathcal{R}_r$ (after $\mathcal{I}$ and $\mathcal{D}$)
\citep{hauser2012characterization}.

In contrast, XGES navigates through MECs using simultaneously $\mathcal{I}$,
$\mathcal{D}$ and $\mathcal{R}_c$. The XGES heuristics favor using $\mathcal{D}$, then
$\mathcal{R}_c$, and finally $\mathcal{I}$ (see \Cref{alg:xges-0}).

\begin{remark}
    XGES could also use $\mathcal{R}_r$, we leave this for future work.
\end{remark}

\subsection{Theoretical Guarantees of GES}
\label{appendix:sec:theoretical_guarantees_ges}

We reformulate the results in \citep{chickering2002optimal}. Given a distribution $p^*$
faithful to a graph $G^*$ with MEC $M^*$. GES is \textit{correct} if it returns $M^*$,
the MEC of $G^*$. It is the MEC formed by all the DAGs faithful to $p^*$. 

\begin{theorem}[\protect{\citep[Lemma 9]{chickering2002optimal}}] \label{thm:chickering9}
    If no candidate MECs
    in $\mathcal{I}(M)$ can increase the score $S$, then $M$ has a special structure:
    all the independencies in $M$ are also independencies in $p^*$. We write it:
    $$\max_{M' \in \mathcal{I}(M)} S(M') \leq S(M) \Rightarrow \texttt{P1}(M; p^*),$$ where
    $\texttt{P1}(M; p^*)$ is the proposition that all the independencies in $M$ are also
    independencies in $p^*$.
\end{theorem}
Note that $p^*$ might have more independencies than $M$, i.e. $p^*$ is not necessarily
faithful to $M$. This is because $M$ might have superfluous edges. For example, the MEC
of the complete DAGs satisfies $\texttt{P1}(M; p^*)$. 


\begin{theorem}[\protect{\citep[Lemma 10]{chickering2002optimal}}]
    \label{thm:chickering10a} If all the independencies in $M$ are also independencies
    in $p^*$, then the same is true for all the MECs in $\mathcal{D}(M)$ that have a
    higher score than $M$. We write it:
    $$\texttt{P1}(M; p^*) \Rightarrow (\forall M' \in \mathcal{D}(M), S(M') \geq S(M)
    \Rightarrow \texttt{P1}(M'; p^*)).$$
\end{theorem}


\begin{theorem}[\protect{\citep[Lemma 10]{chickering2002optimal}}] \label{thm:chickering10b}
    If all the
    independencies in $M$ are also independencies in $p^*$, and if no candidate MECs in
    $\mathcal{D}(M)$ can increase the score $S$, then $M = M^*$. We write
    it:
    $$\texttt{P1}(M; p^*) \wedge \left[\left(\max_{M' \in \mathcal{D}(M)} S(M')\right) \leq S(M)\right]
    \Rightarrow \texttt{P2}(M; p^*),$$ where $\texttt{P2}(M; p^*)$ is the proposition
    that $M = M^*$ or equivalently that $p^*$ is faithful to all the DAGs in $M$
\end{theorem}

With \Cref{thm:chickering9,thm:chickering10a,thm:chickering10b}, it follows that GES is
correct: it will reach a MEC $M$ at the end of phase 1 such that $\mathtt{P1}(M; p^*)$
is true. From there, all MECs visited in phase 2 will also have $\mathtt{P1}(M; p^*)$
true, and GES will stop on a MEC $M$ such that $\mathtt{P2}(M; p^*)$ is true, i.e. $M =
M^*$.


\subsection{Theoretical Guarantees of XGES-0 and XGES}
\subsubsection{Theoretical Guarantees of XGES-0}
\label{appendix:sec:theoretical_guarantees_xges0}
We now prove that XGES-0 is correct. 

\thXgesZero*
\begin{proof}We prove that XGES-0 terminates and is correct.
    \begin{itemize}
        \item \textbf{Termination.} The algorithm terminates because the search space is finite
        and the score is non-decreasing at each step.
        \item \textbf{Correctness.} Let $\hat M$ be the MEC returned by XGES-0. Since XGES-0
        terminated, it means that no candidate MECs in $\mathcal{I}(\hat M)$ can increase the
        score $S$. By \Cref{thm:chickering9}, $\texttt{P1}(\hat M; p^*)$ is true. It also
        means that no candidate MECs in $\mathcal{D}(\hat M)$ can increase the score $S$. By
        \Cref{thm:chickering10b}, $\texttt{P2}(\hat M; p^*)$ is true. Hence, XGES-0 is
        correct.
    \end{itemize}
\end{proof}

\clearpage
\subsubsection{Theoretical Guarantees of XGES}
\label{appendix:sec:theoretical_guarantees_xges}
We provide the pseudocode of XGES in \Cref{alg:xges}.

\begin{algorithm}[h]
    \LinesNumbered
    \DontPrintSemicolon
    \KwData{Data $\data \in \mathbb{R}^{n\times d}$, score function $S$}
    \KwDefine{$\delta_{\data,M}(O) = S(\text{Apply}(O,M) ; \data) - S(M; \data)$}  
    \KwResult{MEC of $G^*$}
    $M \leftarrow $ XGES-0$(X, S)$\;
    $\mathcal{D} \leftarrow$ all deletes valid for $M$\;
    \While{$|\mathcal{D}| > 0$}{
        $D^* \leftarrow \argmax_{D \in \mathcal{D}} \{\delta_{\data,M}(D)\}$\;
        % $\delta = s_{X,M}(D^*)$\;
        $M' \leftarrow$ Apply($D^*, M$)\;  
        $\widetilde{\mathcal{I}} \leftarrow$ all insertions, from any MEC to any MEC, that reinsert the edge deleted by
        $D^*$ \;
        
        $M' \leftarrow$ XGES-0\textsuperscript{*}$(X, S, M', \widetilde{\mathcal{I}})$\;  
        
        \tcc{XGES-0\textsuperscript{*} is a modified version of XGES from
        \Cref{alg:xges-0} that accepts an initial graph $M'$, and a set of forbidden
        insertions $\widetilde{\mathcal{I}}$}  
        \uIf{$S(M'; \data) > S(M; \data)$}{ $M \leftarrow M'$\; $\mathcal{D} \leftarrow$ all
        deletes valid for $M$\; }  
        \Else{ $\mathcal{D} \leftarrow \mathcal{D} \backslash \{D^*\}$\; } }  
        \Return $M$\;
    \caption{XGES}
    \label{alg:xges}
\end{algorithm}

\thXges*

\begin{proof}

    With a locally consistent score, XGES-0 is correct, so $M$ in line 1 is already the 
    MEC $M^*$ of $G^*$. Such a $M^*$ has the maximum score, so lines 9 and 10 are never
    executed. Hence, XGES is correct.

    For completeness, we provide proof of XGES's termination even when the score is not
    locally consistent. 

    \textbf{Termination in practice.} Every time XGES replaces $M$ by $M'$, the score of
        $M$ strictly increases. Since the search space is finite, XGES cannot infinitely
        replace $M$ by $M'$. But then, the other possibility is to remove $D^*$ from
        $\mathcal{D}$, which is a finite set. Hence, XGES terminates.

\end{proof}





\clearpage
\subsection{Navigating the Space of MECs with CPDAGs}
\label{appendix:sec:navigating_space_cpags}
We review how to translate the MEC formulation into a CPDAG formulation for the
practical implementations of GES and XGES. We recall that each MEC $M$ can be uniquely
represented by a CPDAG $P$.

\subsubsection{Original Parametrization}
\label{appendix:sec:ges_parametrization}
In \Cref{sec:manipulating_mecs} we reviewed that given a MEC $M$ represented by the CPDAG $P$,
each $M' \in \mathcal{I}(M)$ can be uniquely associated to an operator Insert$(x,y,T)$,
where $T$ is a set of nodes, such that applying Insert$(x,y,T)$ to $P$ yields a PDAG
$P'$ that represents $M'$ (up to completing it into a CPDAG). The operator
Insert$(x,y,T)$ modifies $P$ by adding the edge $x \rightarrow y$ and orienting all
edges $t - y$ as $t \rightarrow y$ for all $t \in T$. Not all Insert$(x,y,T)$ operators
can be applied to $P$ and yield a PDAG $P'$ that represents a MEC $M' \in
\mathcal{I}(M)$. All the Insert$(x,y,T)$ operators that correspond to a MEC $M' \in
\mathcal{I}(M)$ are called valid operators. There is a bijection between
$\mathcal{I}(M)$ and the set of valid Insert$(x,y,T)$, and there exist conditions that
can be checked on $P$ to determine whether Insert$(x,y,T)$ is valid or not. 

The same holds for $\mathcal{D}(M)$, $\mathcal{R}_c(M)$ and $\mathcal{R}_r(M)$. We
summarize the operators, their validity conditions, their score and their actions on a CPDAG in \Cref{tab:ges_operators_original},
adapted from \citet{chickering2002optimal} for insertion and deletion, and from
\citet{hauser2012characterization} for reversals.

\begin{table}[h!]
    \resizebox{\linewidth}{!}{
    \begin{tabular}{ C{1.3cm}C{5.5cm}C{5.8cm}C{5.5cm} }
        \toprule
        Operator & Insert$(x,y,T)$ & Delete$(x,y, H)$ & Reversal$(x,y,T)$ \\
        \midrule
        Conditions & 
        \begin{itemize}
            \item $x \not\in \Ad(y)$
            \item $T \subset \Ne(y) \backslash \Ad(x)$
            \item $[\Ne(y) \cap \Ad(x)] \cup T$ is a clique
            \item All semi-directed paths from $y$ to $x$ are blocked by $[\Ne(y) \cap
            \Ad(x)] \cup T$  
        \end{itemize} &
        \begin{itemize}
            \item $x \in \Ch(x) \cup \Ne(x)$
            \item $H \subset \Ne(y) \cap \Ad(x)$
            \item $[\Ne(y) \cap \Ad(x)] \backslash H$ is a clique
        \end{itemize} &
        \begin{itemize}
            \item $y \in \Pa(x)$
            \item $T \subset \Ne(y) \backslash \Ad(x)$
            \item $[\Ne(y) \cap \Ad(x)] \cup T$ is a clique
            \item All semi-directed paths from $y$ to $x$ other than ($y,x$) are blocked
            by $[\Ne(y) \cap \Ad(x)] \cup T\cup\Ne(x)$  
        \end{itemize}\\
        
        Score Increase &  
        $\squeezespaces{0}s(y, [\Ne(y) \cap \Ad(x)] \cup T \cup \Pa(y) \cup \{x\})$ $ -
        s(y, [\Ne(y) \cap \Ad(x)] \cup T \cup \Pa(y))$  
        & $s(y, [\Ne(y) \cap \Ad(x)] \backslash H \cup \Pa(y) \backslash \{x\}) - s(y,
        [\Ne(y) \cap \Ad(x)] \backslash H\cup \Pa(y)\cup\{x\})$  
        & $\squeezespaces{0}s(y, [\Ne(y) \cap \Ad(x)] \cup T \cup \Pa(y) \cup \{x\})$ $ -
        s(y, [\Ne(y) \cap \Ad(x)] \cup T \cup \Pa(y))$ 
        $+s(x, \Pa(x) \backslash \{y\}) - s(x, \Pa(x))$\\
        \\
        Actions & \begin{itemize}
            \item Add $x \rightarrow y$ to $P$.
            \item For all $t \in T$, orient $t - y$ as $t \rightarrow y$.
        \end{itemize} & \begin{itemize}
            \item Remove $x \rightarrow y$ (or $x - y$) from $P$.
            \item Orient all edges $y - h$ as $h \rightarrow y$ for $h \in H$.
            \item Orient all edges $x - h$ as $h \rightarrow x$ for $h \in H$.
        \end{itemize} & \begin{itemize}
            \item Reverse $x \rightarrow y$ into $x \leftarrow y$.
            \item For all $t \in T$, orient $t - y$ as $t \rightarrow y$.
        \end{itemize} 
        \\
        \bottomrule
    \end{tabular}
    } \caption{Summary of the operators and their conditions as described in
    \citep{chickering2002optimal}. The conditions for Insert and Delete are described in
    \citet[Definition 12, Definition 13, Theorem 15, Theorem 17, Table
    1]{chickering2002optimal}, and the score increase in \citet[Corollary 16, Corollary
    18]{chickering2002optimal}. The conditions for Reversal are described in
    \citet[Proposition 34]{hauser2012characterization}, and the score increase in
    \citet[Corollary 36]{hauser2012characterization}.}
    \label{tab:ges_operators_original}
\end{table}

\clearpage
\subsubsection{XGES Parametrization}
\label{appendix:sec:xges_parametrization}
We proposed a slightly different parametrization of the operators with adapted
conditions. We recall from \Cref{sec:updating_score} that with the original
parametrization, the scores of the operators depend on which PDAG they are applied
to. With the goal of an efficient implementation that caches the score of valid
operators to avoid recomputation, it is important and convenient for each operator to
have a unique score, agnostic of the PDAG it is applied to.

% To achieve this, we changed the parametrization of the operators to include the set of
% nodes used in the score computation as part of the operator, and add a validity
% condition about this set. 

We described the new parametrization for the Insert operator in \Cref{sec:updating_score}.
We describe the Delete and Reverse operator in \Cref{tab:ges_operators} hereafter.
For the Delete operator, we also replace the set $H$ by the set $C$, its complement in
$\Ne(y) \cap \Ad(x)$. 


\begin{table*}[h!]
    \centering
    \resizebox{0.95\linewidth}{!}{
\begin{tabular}{ C{0.33\linewidth} C{0.32\linewidth} C{0.33\linewidth} }
\toprule
Insert($x, y, T$; $E$) & Delete($x, y, C; E$) &  Reverse($x, y, T, E, F$)\\
\midrule
\vspace{-0.4cm}
   \begin{itemize}
       \item[\textbf{I1}:] $y \not\in \Ad(x)$
       \item[\textbf{I2}:] $T \subset \Ne(y) \backslash \Ad(x)$
       \item[\textbf{I3}:] $(\Ne(y) \cap \Ad(x)) \cup T$ is a clique
       \item[\textbf{I4}:] All semi-directed paths from $y$ to $x$ have a node in $(\Ne(y) \cap
       \Ad(x)) \cup T$
       \item[\textbf{I5}:] $E = (\Ne(y) \cap \Ad(x)) \cup T \cup \Pa(y)$ 
   \end{itemize}
   & 
\vspace{-0.4cm}
   \begin{itemize}
       \item[\textbf{D1}:] $y \in \Ch(x) \cup \Ne(x)$
       \item[\textbf{D2}:] $C \subset \Ne(y) \cap \Ad(x)$
       \item[\textbf{D3}:] $C$ is a clique
       \item[\textbf{D4}:] $E = C \cup \Pa(y)$
   \end{itemize}
   & \vspace{-0.4cm}\begin{itemize}
        \item[\textbf{R1}:] $y \in \Pa(x)$ 
        \item[\textbf{R2}:] $T \subset \Ne(y) \backslash \Ad(x)$ 
        \item[\textbf{R3}:] $(\Ne(y) \cap \Ad(x)) \cup T$ is a clique 
        \item[\textbf{R4}:] All semi-directed paths from $y$ to $x$ not using edge $y \rightarrow x$
       have a node in $(\Ne(y) \cap \Ad(x)) \cup T \cup\Ne(x)$ 
       \item[\textbf{R5}:] $E = (\Ne(y) \cap \Ad(x)) \cup T \cup \Pa(y)$ 
       \item[\textbf{R6}:] $F = \Pa(x)$
    \end{itemize} \\%[-0.3cm]
    % \midrule
      $\delta = s(y, E \cup \{x\}) - s(y, E )$
    & $\delta=s(y, E \cup \{x\}) - s(y, E \backslash \{x\})$
    & $\delta= s(y, E \cup \{x\}) - s(y, E ) + s(x, F \backslash \{y\}) - s(x, F)$\\
\bottomrule
\end{tabular}}  
\caption{Parametrization of operators by XGES with their validity conditions and score. 
This parametrization renders the score of each operator invariant to the CPDAG it is 
applied to.}
\label{tab:ges_operators}
\end{table*}

\clearpage

\subsection{Efficient Algorithmic Formulation}
\label{appendix:sec:efficient_algorithmic_formulation}

We study how each validity condition described in \Cref{tab:ges_operators} can become 
true after each type of edge update in a PDAG. We only need to consider single-edge updates because 
of \Cref{thm:update_validity}.

\parhead{Edge Updates} There are seven types of edge updates: one of $a\quad b$, $a -
b$, or $a \rightarrow b$ becomes another one (6 = 3*2); and the reversal $a \rightarrow
b$ into $a \leftarrow b$ (which happens only after applying a reversal operator).

For each of these edge updates, we study how they affect the validity conditions of each operator. We 
summarize the results in \Cref{tab:operator_updates} and provide the detailed proofs in the following sections.
We further aggregate the results from \Cref{tab:operator_updates} into \Cref{tab:operator_updates2}.



\begin{table*}[h]
    \centering
    % resize box
    \resizebox{\textwidth}{!}{
    \begin{tabular}{ C{1cm}C{2.5cm}C{2.5cm}C{2.5cm}C{2.5cm}C{2.5cm}C{2.5cm}C{2.5cm} }
    \toprule
     & $a \quad b$ & $a \quad b$ & $a - b$ & $a - b$ & $a \rightarrow b$ & $a
     \rightarrow b$ & $a \rightarrow b$ \\
     & $a - b$ & $a \rightarrow b$ & $a \quad b$ & $a \rightarrow b$ & $a \quad b$ & $a
    - b$ & $a \leftarrow b$ \\
    \midrule
    \textbf{I1} & $\varnothing$ & $\varnothing$ & $\{a, b \} = \{x,y\}$  & $\varnothing$
    & $\{a, b \} = \{x,y\} $ & $\varnothing$ & $\varnothing$ \\[4mm]
    \textbf{I2} & $\curlystack{\bar{a} = y \\ \bar{b} \not\in \Ad(x)}$ & $\varnothing$ &
     $\curlystack{\bar{a}=x \\ \bar{b} \in \Ne(y)} $ & $\varnothing$ &
     $\curlystack{\bar{a}=x \\ \bar{b} \in \Ne(y)} $ & $\curlystack{\bar{a} = y \\
     \bar{b} \not\in \Ad(x)}$  & $\varnothing$\\[4mm]
    \textbf{I3}\textsuperscript{I2} & $\{a, b\} \subset \Ne(y)$ & $\{a, b\} \subset \Ne(y)$ &
    $\curlystack{\bar{a} = y \\ \bar{b} \in \Ad(x)}$ or $\curlystack{\bar{a} = x \\
    \bar{b} \in \Ne(y)}$ & $\curlystack{\bar{a} = y \\ \bar{b} \in \Ad(x) }$ &
    $\curlystack{\bar{a} = x \\ \bar{b} \in \Ne(y) }$ & $\varnothing$ &
    $\varnothing$\\[4mm]
    \textbf{I4} & $\curlystack{\bar{a} = y \\ \bar{b} \in \Ad(x)}$ or
    $\curlystack{\bar{a} = x \\ \bar{b} \in \Ne(y)}$ & $\curlystack{\bar{a} = x \\
    \bar{b} \in \Ne(y)}$ & \texttt{SD}($x,y; \bar{a},\bar{b}$) & \texttt{SD}($x,y; b,a$)
    & \texttt{SD}($x,y; a,b$) & $\curlystack{\bar{a} = y \\ \bar{b} \in \Ad(x)}$ &
    \texttt{SD}($x,y; a,b$) \\[4mm]
    \textbf{I5} & $\curlystack{\bar{a} = y \\ \bar{b} \in \Ad(x)}$ or
    $\curlystack{\bar{a} = x \\ \bar{b} \in \Ne(y)}$ & $b = y$ or $\curlystack{\bar{a} =
    x \\ \bar{b} \in \Ne(y)}$ & $\curlystack{\bar{a} = y \\ \bar{b} \in \Ad(x)}$ or
    $\curlystack{\bar{a} = x \\ \bar{b} \in \Ne(y)}$ & $b = y$ or $\curlystack{\bar{a} =
    y \\ \bar{b} \in \Ad(x)}$ & $b = y$ or $\curlystack{\bar{a} = x \\ \bar{b} \in
    \Ne(y)}$ & $b = y$ or $\curlystack{\bar{a} = y \\ \bar{b} \in \Ad(x)}$ & $\bar{a}=y$
    \\
    \midrule
    \textbf{D1} & $\{a, b \} = \{x,y\} $ & $(a,b) = (x,y)$ & $\varnothing$ &
    $\varnothing$ & $\varnothing$ & $(a,b) = (y,x)$ & $(a,b) = (y,x)$\\
    \textbf{D2} & $\curlystack{\bar{a}=y \\ \bar{b} \in \Ad(x)}$ or
    $\curlystack{\bar{a}=x \\ \bar{b} \in \Ne(y)}$ & $\curlystack{\bar{a}=x \\ \bar{b}
    \in \Ne(y)}$ & $\varnothing$ & $\varnothing$ & $\varnothing$ &
    $\curlystack{\bar{a}=y \\ \bar{b} \in \Ad(x)}$ & $\varnothing$\\
    \textbf{D3}\textsuperscript{D2} & $\{a,b\} \subset \Ne^u(y) \cap \Ad^u(x)$ & $\{a,b\} \subset \Ne^u(y)
    \cap \Ad^u(x)$ & $\varnothing$ & $\varnothing$ & $\varnothing$ & $\varnothing$ &
    $\varnothing$\\
    \textbf{D4} & $\varnothing$ & $b=y$ & $\varnothing$ & $b=y$ & $b=y$ & $b=y$ &
    $\bar{a} = y$\\
    \midrule
    \textbf{R1} & $\varnothing$ & $(a,b) = (y,x)$ & $\varnothing$ & $(a,b) = (y,x) $ &
    $\varnothing$ & $\varnothing$ & $(a,b) = (x,y)$\\
    \textbf{R2} & See \textbf{I2} & See \textbf{I2} & See \textbf{I2} & See \textbf{I2} &
    See \textbf{I2} & See \textbf{I2} & See \textbf{I2}\\
    \textbf{R3}\textsuperscript{R2} & See \textbf{I3} & See \textbf{I3} & See \textbf{I3} &
    See \textbf{I3} & See \textbf{I3} & See \textbf{I3} & See \textbf{I3}\\
    \textbf{R4} & See \textbf{I4} or $\bar{a}=x$ & See \textbf{I4} & See \textbf{I4} 
    & See \textbf{I4} & See \textbf{I4} & See \textbf{I4} or $x=\bar{a}$ & See \textbf{I4} \\
    \textbf{R5} & See \textbf{I5} & See \textbf{I5} & See \textbf{I5} & See \textbf{I5}
    & See \textbf{I5} & See \textbf{I5} & See \textbf{I5} \\
    \textbf{R6} & $\varnothing$ & $b=x$ & $\varnothing$ & $b=x$ & $b=x$ & $b=x$ &
    $\bar{a} = x$\\
    \bottomrule
    \end{tabular}
    } \caption{For each type of edge update involving an edge $(a,b)$, we list necessary
    conditions for each validity conditions of operators Insert$(x,y,T,E)$,
    Delete$(x,y,C,E)$, and Reverse$(x,y,T,E,F)$ to become valid. The notation 
    Condition($\bar a, \bar b$) is a shorthand for $\text{Condition}(a,b) \vee \text{Condition}(b,a)$.    
    The notation \texttt{SD}($x,y; a,b$) is a shorthand for the necessary condition:
    $(a,b)$, in that order, is on a semi-directed path from $y$ to $x$. All operators
    $\Pa, \Ne, \Ad$ and \texttt{SD}($x,y; a,b$) are computed with respect to the PDAG
    before the edge update. All operators $\Pa^u, \Ne^u, \Ad^u$ are computed with
    respect to the PDAG after the edge update.}
    \label{tab:operator_updates}
\end{table*}




\Cref{tab:operator_updates2} rewrites \Cref{tab:operator_updates} with the statements to
be conditions centered around $x$ and $y$, and aggregate all the necessary conditions
together. Whenever a single edge $(a,b)$ is updated, only the Insert operators satisfying
the condition \textbf{I}-any can become valid and need to be checked. The same holds for
Delete with \textbf{D}-any, and Reverse with \textbf{R}-any.
\begin{table*}[h!]
    \centering
    % resize box
    \resizebox{\textwidth}{!}{
    \begin{tabular}{ cC{4.2cm}C{3.5cm}C{4.2cm}C{2.2cm}C{2.5cm}C{2.4cm}C{2.2cm} }
    \toprule
     & $a \quad b$ & $a \quad b$ & $a - b$ & $a - b$ & $a \rightarrow b$ & $a
     \rightarrow b$ & $a \rightarrow b$ \\
     & $a - b$ & $a \rightarrow b$ & $a \quad b$ & $a \rightarrow b$ & $a \quad b$ & $a
    - b$ & $a \leftarrow b$ \\
    \midrule
    \textbf{I1} & $\varnothing$ & $\varnothing$ & $\{x,y\} = \{a, b \}$  & $\varnothing$
    & $\{x,y\} = \{a, b \} $ & $\varnothing$ & $\varnothing$ \\[2mm]
    \textbf{I2} & $y = \bar{a}$ & $\varnothing$ & $\curlystack{x= \bar{a} \\ y \in
     \Ne(\bar{b})} $ & $\varnothing$ & $\curlystack{x= \bar{a} \\ y \in \Ne(\bar{b})} $
     & $y = \bar{a}$  & $\varnothing$\\[5mm]
     \textbf{I3}\textsuperscript{I2} & $y \in \Ne(a)\cap \Ne(b)$ & $y \in \Ne(a)\cap \Ne(b)$ &
     $\curlystack{y = \bar{a} \\ x \in \Ad(\bar{b})}$ or $\curlystack{x = \bar{a} \\ y
     \in \Ne(\bar{b})}$ & $\curlystack{y = \bar{a} \\ x \in \Ad(\bar{b}) }$ &
     $\curlystack{x = \bar{a} \\ y \in \Ne(\bar{b}) }$ & $\varnothing$ &
     $\varnothing$\\[2mm]
    \textbf{I4} & $\curlystack{y = \bar{a} \\ x \in \Ad(\bar{b})}$ or $\curlystack{x =
    \bar{a} \\ y \in \Ad(\bar{b})}$ & $\curlystack{x = \bar{a} \\
    y \in \Ne(\bar{b})}$ & \texttt{SD}($x,y; \bar{a},\bar{b}$) & \texttt{SD}($x,y; b,a$)
    & \texttt{SD}($x,y; a,b$) & $\curlystack{y = \bar{a} \\ x \in \Ad(\bar{b})}$ &
    \texttt{SD}($x,y; a,b$) \\[5mm]
    \textbf{I5} & $\curlystack{y = \bar{a} \\ x \in \Ad(\bar{b})}$ or $\curlystack{x =
    \bar{a} \\ y \in \Ne(\bar{b})}$ & $y = b$ or $\curlystack{x = \bar{a} \\ y \in
    \Ne(\bar{b})}$ & $\curlystack{y = \bar{a} \\ x \in \Ad(\bar{b})}$ or $\curlystack{x
    = \bar{a} \\ y \in \Ne(\bar{b})}$ & $y = b$ or $\curlystack{y = \bar{a} \\ x \in
    \Ad(\bar{b})}$ & $y = b$ or $\curlystack{x = \bar{a} \\ y \in \Ne(\bar{b})}$ & $y =
    b$ or $\curlystack{y = \bar{a} \\ x \in \Ad(\bar{b})}$ & $y=\bar{a}$ \\
    \midrule
    \textbf{I}-any & 
    \begin{itemize}[leftmargin=*]
        \item $y \in \{a,b\}$ 
        \item $y \in \Ne(a)\cap \Ne(b)$
        \item $x = a$ and $y \in \Ne(b)$
        \item $x = b$ and $y \in \Ne(a)$
    \end{itemize}
    & \begin{itemize}[leftmargin=*]
        \item $y = b$
        \item $y \in \Ne(a)\cap \Ne(b)$
        \item $x = a$ and $y \in \Ne(b)$
        \item $x = b$ and $y \in \Ne(a)$
    \end{itemize}
    & \begin{itemize}[leftmargin=*]
        \item $x = a$ and $y \in \Ne(b) \cup \{b\}$
        \item $x = b$ and $y \in \Ne(a) \cup \{a\}$
        \item $y = a$ and $x \in \Ad(b)$
        \item $y = b$ and $x \in \Ad(a)$
        \item \texttt{SD}($x,y; a,b$)
        \item \texttt{SD}($x,y; b,a$)
    \end{itemize}
    & \begin{itemize}[leftmargin=*]
        \item $y = a$ and $x \in \Ad(b)$
        \item $y = b$
        \item \texttt{SD}($x,y; b,a$)
    \end{itemize}
    & \begin{itemize}[leftmargin=*]
        \item $y=b$
        \item $x = a$ and $y \in \Ne(b) \cup \{b\}$
        \item $x = b$ and $y \in \Ne(a) \cup \{a\}$
        \item \texttt{SD}($x,y; a,b$)
    \end{itemize}
    & \begin{itemize}[leftmargin=*]
        \item $y \in \{a,b\}$
    \end{itemize}
    & \begin{itemize}[leftmargin=*]
        \item $y \in \{a,b\}$
        \item \texttt{SD}($x,y; a,b$)
    \end{itemize}
    \\
    \midrule
    \textbf{D1} & $\{x,y\} = \{a, b \} $ & $(x,y) = (a,b)$ & $\varnothing$ &
    $\varnothing$ & $\varnothing$ & $(x,y) = (b,a )$ & $(x,y) = (b,a)$\\
    \textbf{D2} & $\curlystack{y = \bar{a} \\ x \in \Ad(\bar{b})}$ or
    $\curlystack{x = \bar{a} \\ y \in \Ne(\bar{b})}$ & $\curlystack{x = \bar{a} \\ 
    y \in \Ne(\bar{b})}$ & $\varnothing$ & $\varnothing$ & $\varnothing$ &
    $\curlystack{y = \bar{a} \\ x \in \Ad(\bar{b})}$ & $\varnothing$\\
    \textbf{D3}\textsuperscript{D2} & $\curlystack{x \in \Ad^u(a) \cap \Ad^u(b) \\ y
        \in \Ne^u(a) \cap \Ne^u(b)}$ & $\curlystack{ x \in \Ad^u(a) \cap \Ad^u(b) \\ y
        \in \Ne^u(a) \cap \Ne^u(b)}$ & $\varnothing$ & $\varnothing$ & $\varnothing$ &
        $\varnothing$ & $\varnothing$\\
    \textbf{D4} & $\varnothing$ & $y=b$ & $\varnothing$ & $y=b$ & $y=b$ & $y=b$ &
    $y\in\{a,b\}$\\
    \midrule
    \textbf{D}-any &
    \begin{itemize}[leftmargin=*]
        % \item $y = a$ and $x \in \Ad(b)$
        % \item $y = b$ and $x \in \Ad(a)$
        % \item $x = a$ and $y \in \Ne(b) \cup \{b\}$
        % \item $x = b$ and $y \in \Ne(a) \cup \{a\}$
        \item $y \in \{a,b\}$
        \item $x \in \{a,b\}$
        \item $x \in \Ad(a) \cap \Ad(b)$ and $y \in \Ne(a) \cap \Ne(b)$
    \end{itemize}
    & \begin{itemize}[leftmargin=*]
        % \item $x = a$ and $y \in \Ne(b)$
        % \item $x = b$ and $y \in \Ne(a)$
        \item $y = b$
        \item $x \in \{a,b\}$
        \item $x \in \Ad(a) \cap \Ad(b)$ and $y \in \Ne(a) \cap \Ne(b)$
    \end{itemize}
    & $\varnothing$
    & \begin{itemize}[leftmargin=*]
        \item $y = b$
    \end{itemize}
    & \begin{itemize}[leftmargin=*]
        \item $y = b$
    \end{itemize}
    & \begin{itemize}[leftmargin=*]
        \item $y \in \{a, b\}$ % and $x \in \Ad(b) \cup \{b\}$
        % \item $y = b$
    \end{itemize}
    & \begin{itemize}[leftmargin=*]
        \item $y \in \{a,b\}$
    \end{itemize}\\
    \midrule
    \textbf{R}-any & 
    \begin{itemize}[leftmargin=*]
        \item $y \in \{a,b\}$ 
        \item $y \in \Ne(a)\cap \Ne(b)$
        \item $x \in \{a,b\}$
    \end{itemize}
    & \begin{itemize}[leftmargin=*]
        \item $y = b$
        \item $y \in \Ne(a)\cap \Ne(b)$
        \item $x = a$ and $y \in \Ne(b)$
        \item $x = b$
    \end{itemize}
    & \begin{itemize}[leftmargin=*]
        \item $x = a$ and $y \in \Ne(b) \cup \{b\}$
        \item $x = b$ and $y \in \Ne(a) \cup \{a\}$
        \item $y = a$ and $x \in \Ad(b)$
        \item $y = b$ and $x \in \Ad(a)$
        \item \texttt{SD}($x,y; a,b$)
        \item \texttt{SD}($x,y; b,a$)
    \end{itemize}
    & \begin{itemize}[leftmargin=*]
        \item $y = a$ and $x \in \Ad(b)$
        \item $y = b$
        \item \texttt{SD}($x,y; b,a$)
        \item $x = b$
    \end{itemize}
    & \begin{itemize}[leftmargin=*]
        \item $y=b$
        \item $x = a$ and $y \in \Ne(b) \cup \{b\}$
        \item $x = b$
        \item \texttt{SD}($x,y; a,b$)
    \end{itemize}
    & \begin{itemize}[leftmargin=*]
        \item $y \in \{a,b\}$
        \item $x=b$
    \end{itemize}
    & \begin{itemize}[leftmargin=*]
        \item $y \in \{a,b\}$
        \item $x \in \{a,b\}$
        \item \texttt{SD}($x,y; a,b$)
    \end{itemize}
    \\
    \bottomrule
    \end{tabular}
    } 
    \caption{For each type of edge update involving an edge $(a,b)$, we list necessary
    conditions for each validity conditions of operators Insert$(x,y,T,E)$,
    Delete$(x,y,C,E)$, and Reverse$(x,y,T,E,F)$ to become valid.    
    The notation \texttt{SD}($x,y; a,b$) is a shorthand for the necessary condition:
    $(a,b)$, in that order, is on a semi-directed path from $y$ to $x$. All operators
    $\Pa, \Ne, \Ad$ and \texttt{SD}($x,y; a,b$) are computed with respect to the PDAG
    before the edge update. The rows \textbf{I}-any, \textbf{D}-any, and \textbf{R}-any
    aggregate the necessary conditions for each validity condition and express them in 
    a disjunctive form: at least one of the conditions must be true for the operator to
    become valid.}
    \label{tab:operator_updates2}
\end{table*}


\newtheorem{opup}{Operator Update}
\newtheorem{lemma}{Lemma}


\subsubsection{I1}


\begin{opup}[Updates on I1]
    Assume $\mathbf{I1}(x,y, T; E)$ is false and becomes true after an update involving
    $(a,b)$. Then,
    \begin{itemize}
        \item the update cannot be $U1: (a \quad b) \rightsquigarrow (a - b)$,
        \item the update cannot be $U2: (a \quad b) \rightsquigarrow (a \rightarrow b)$,
        \item if the update is $U3: (a - b) \rightsquigarrow (a \quad b)$ then $ \{a,b\}
        =\{x,y\}$,
        \item the update cannot be $U4: (a - b) \rightsquigarrow (a \rightarrow b)$,
        \item if the update is $U5: (a \rightarrow b) \rightsquigarrow (a \quad b)$ then
        $\{a,b\} =\{x,y\}$,
        \item the update cannot be $U6: (a \rightarrow b) \rightsquigarrow (a - b)$,
        \item the update cannot be $U7: (a \rightarrow b) \rightsquigarrow (a \leftarrow
        b)$.
    \end{itemize}
\end{opup}

\begin{proof}
    We recall that $\mathbf{I1}(x,y, T; E)$ is $y \not\in \Ad(x)$. So the assumptions
    are $y \in \Ad(x)$ and $y \not\in \Ad^u(x)$. But U1, U2, U4, U6, and U7 do not
    remove any elements from any $\Ad(x')$ set. So none of them can render
    $\mathbf{I1}(x,y, T; E)$ true. 

    U3 and U5 can only remove elements from $\Ad(a)$ or $\Ad(b)$, and do so only by
    removing $b$ or $a$, respectively. So $\{x,y\} = \{a,b\}$.
\end{proof}

\subsubsection{I2}
We start with a general lemma for I2.

\begin{lemma}[I2 to become true]\label{lemma:I2} Assume $\mathbf{I2}(x,y, T; E)$ is
    false and becomes true after an edge update. Then (i) $\Ne^u(y)$ gained an element,
    or (ii) $\Ad^u(x)$ lost an element that was in $\Ne(y)$.
\end{lemma}
\begin{proof}
    We recall that $\mathbf{I2}(x,y, T; E)$ is $T \subset \Ne(y) \backslash \Ad(x)$. If
     $\mathbf{I2}(x, y, T; E)$ changes from false to true then there exists $t \in T$
     that was not in $\Ne(y) \backslash \Ad(x)$ and is now in $\Ne^u(y) \backslash
     \Ad^u(x)$, which writes $$(t\not \in \Ne(y) \vee t \in \Ad(x)) \wedge t \in
     \Ne^u(y) \wedge t \not\in \Ad^u(x).$$
     \begin{itemize}
        \item If $t \in \Ne(y)$ then we must have $t \in \Ad(x)$ and $t \not\in
        \Ad^u(x)$. So $\Ad^u(x)$ lost $t$, which was in $\Ne(y) \cap \Ad(x)$.
        \item If $t \not\in \Ne(y)$, then $\Ne^u(y)$ gained $t$. 
     \end{itemize}
    In conclusion, either $\Ne^u(y)$ gained an element, or $\Ad^u(x)$ lost an element
     that was is in $\Ne(y)$. %(With more work and assuming \textbf{I1} holds after the
    %  update, we can also show that the element gained by $\Ne^u(y)$ is not in $\Ad(x)$).

\end{proof}


\begin{opup}[Updates on I2]
    Assume $\mathbf{I2}(x,y, T; E)$ is false and becomes true after an update involving
    $(a,b)$. Then,
    \begin{itemize}
        \item if the update is $U1: (a \quad b) \rightsquigarrow (a - b)$ then $y \in
        \{a, b\}$,
        \item the update cannot be $U2: (a \quad b) \rightsquigarrow (a \rightarrow b)$,
        \item if the update is $U3: (a - b) \rightsquigarrow (a \quad b)$ then
            $\curlystack{ a=x \\ b \in \Ne(y)}$ or $\curlystack{b=x \\ a \in \Ne(y)}$,
        \item the update cannot be $U4: (a - b) \rightsquigarrow (a \rightarrow b)$,
        \item if the update is $U5: (a \rightarrow b) \rightsquigarrow (a \quad b)$ then
            $\curlystack{ a=x \\ b \in \Ne(y)}$ or $\curlystack{b=x \\ a \in \Ne(y)}$,
        \item if the update is $U6: (a \rightarrow b) \rightsquigarrow (a - b)$ then $y
        \in \{a, b\}$,
        \item the update cannot be $U7: (a \rightarrow b) \rightsquigarrow (a \leftarrow
        b)$.
    \end{itemize}
\end{opup}

\begin{proof}
    According to \Cref{lemma:I2}, either $\Ne^u(y)$ gained an element or $\Ad^u(x)$ lost
    an element (that was in $\Ne(y)$).

    We now study the necessary conditions for each update, if it was applied and made
    $\mathbf{I2}(x,y, T; E)$ become true.
    \begin{itemize}
        \item The updates U1 and U6 can only add elements to $\Ne^u(a)$ or $\Ne^u(b)$
        and not remove any element to any $\Ad^u(x')$. So $\Ne^u(y)$ gained an element
        and $y \in \{a, b\}$.
        \item The updates U2, U4, and U7 do not add any elements to any $\Ne^u(y')$, and
        do not remove any elements to any $\Ad^u(x')$. So none of them can make
        $\mathbf{I2}(x,y, T; E)$ become true.
        \item The updates U3 and U5 can only remove elements from $\Ad^u(a)$ or
        $\Ad^u(b)$ and not add any element to any $\Ne^u(y')$. So $\Ad^u(x)$ lost an
        element and $x \in \{a, b\}$. If $x = a$ (resp. $x = b$) then the lost element
        must be $b$ (resp. $a$) and so $b \in \Ne(y)$ (resp. $a \in \Ne(y)$).
    \end{itemize}
\end{proof}

\subsubsection{I3}
We start with a general lemma for I3.
\begin{lemma}[I3 to become true]\label{lemma:I3} Assume $\mathbf{I3}(x,y, T; E)$ is
    false and becomes true after an edge update about $(a,b)$. Further, assume that
    $\mathbf{I2}(x,y, T; E)$ is true after the update (regardless of its status before
    the update). Then either (i) $\{a,b\} \subset \Ne(y)$ and the update rendered $a,b$
    adjacent, or (ii) $\Ne^u(y)$ lost an element that was in $\Ne(y) \cap \Ad(x)$, or
    (iii) $\Ad^u(x)$ lost an element that was in $\Ne(y) \cap \Ad(x)$.
\end{lemma}
\begin{proof}
    We recall that $\mathbf{I3}(x,y, T; E)$ is $[\Ne(y) \cap \Ad(x)] \cup T$ is a
    clique, and that $\mathbf{I2}(x,y, T; E)$ is $T \subset \Ne(y) \backslash \Ad(x)$.
    So the assumptions are $[\Ne(y) \cap \Ad(x)] \cup T$ is not a clique (in the
    pre-update PDAG) meanwhile $[\Ne^u(y) \cap \Ad^u(x)] \cup T$ is a clique (in the
    post-update PDAG), and $T \subset \Ne^u(y) \backslash \Ad^u(x)$.

    Since $[\Ne(y) \cap \Ad(x)] \cup T$ is not a clique, it must contain two nodes $c,d$
    that are not connected in the pre-update PDAG. 

    We distinguish two cases: 
    \begin{itemize}
        \item If $\{c,d\} \subset [\Ne^u(y) \cap \Ad^u(x)] \cup T$, then the update must
        have connected $c$ and $d$. So $c$ and $d$ are $a$ and $b$. Also since $T
        \subset \Ne^u(y) \backslash \Ad^u(x)$, then $\{a,b\} \subset [\Ne^u(y) \cap
        \Ad^u(x)] \cup T \subset \Ne^u(y)$. Finally, an update can only change one edge
        at a time, so $\Ne^u(y) = \Ne(y)$ (since $a$ and $b$ are not $y$ as $y$ cannot
        be a neighbor of itself). Hence, $\{a,b\} \subset \Ne(y)$ and $a$ and $b$ became
        adjacent.
        
        \item Else, $c$ or $d$ has been removed from $[\Ne^u(y) \cap \Ad^u(x)] \cup T$
        during the update. Without loss of generality, assume $c$ was removed. Since $T$
        does not change, then $c$ was removed from $[\Ne(y) \cap \Ad(x)]$. So $\Ne^u(y)$
        or $\Ad^u(x)$ lost an element that was in $\Ne(y) \cap \Ad(x)$.
    \end{itemize}
    
    Hence, [$\{a,b\} \subset \Ne(y)$ and $a$ and $b$ became adjacent], or $\Ne(y)$ lost
     an element that was in $\Ne(y) \cap \Ad(x)$ or $\Ad(x)$ lost an element that was in
     $\Ne(y) \cap \Ad(x)$.
\end{proof}


\begin{opup}[Updates on I3]
    Assume $\mathbf{I3}(x,y, T; E)$ is false and becomes true after an update involving
    $(a,b)$. Further, assume that
    $\mathbf{I2}(x,y, T; E)$ is true after the update. Then,
    \begin{itemize}
        \item if the update is $U1: (a \quad b) \rightsquigarrow (a - b)$ then $\{a,b\}
        \subset \Ne(y)$,
        \item if the update is $U2: (a \quad b) \rightsquigarrow (a \rightarrow b)$ then
        $\{a,b\} \subset \Ne(y)$,
        \item if the update is $U3: (a - b) \rightsquigarrow (a \quad b)$ then
        $\curlystack{ a \in  \{x,y\} \\ 
            b \in \Ne(y) \cap \Ad(x) }$ or $\curlystack{ b \in  \{x,y\} \\ 
            a \in \Ne(y) \cap \Ad(x) }$,
        \item if the update is $U4: (a - b) \rightsquigarrow (a \rightarrow b)$ then
        $\curlystack{ a =y \\ 
            b \in \Ne(y) \cap \Ad(x) }$ or $\curlystack{ b = y \\ 
            a \in \Ne(y) \cap \Ad(x) }$,
        \item if the update is $U5: (a \rightarrow b) \rightsquigarrow (a \quad b)$ then
        $\curlystack{ a = x \\ 
            b \in \Ne(y) \cap \Ad(x) }$ or $\curlystack{ b = x \\ 
            a \in \Ne(y) \cap \Ad(x) }$,
        \item if the update is $U6: (a \rightarrow b) \rightsquigarrow (a - b)$ then it
        is impossible,
        \item if the update is $U7: (a \rightarrow b) \rightsquigarrow (a \leftarrow b)$
        then it is impossible.
    \end{itemize}
    
\end{opup}

\begin{proof}
    According to \Cref{lemma:I3}, either $\{a,b\} \subset \Ne(y)$ and the update
    rendered $a,b$ adjacent, or $\Ne^u(y)$ lost an element that was in $\Ne(y) \cap
    \Ad(x)$, or $\Ad^u(x)$ lost an element that was in $\Ne(y) \cap \Ad(x)$.

    We now study the necessary conditions for each update, if it was applied and made
    $\mathbf{I3}(x,y, T; E)$ become true.
    \begin{itemize}
        \item The updates U1 and U2 can only add elements to sets like $\Ne^u(y)$ or
        $\Ad^u(x)$, so the only possibility is that $\{a,b\} \subset \Ne(y)$.
        \item The update U3 does not render any edge adjacent. So by \Cref{lemma:I3},
        $\Ne(y)$ or $\Ad(x)$ lost an element $c$ that was in $\Ne(y) \cap \Ad(x)$. 
        \begin{itemize}
            \item If it is $\Ne(y)$ that lost $c$, then we have $\{a,b\} = \{y,c\}$.
            Without loss of generality, $a = y$ and $b = c$, so $b \in \Ne(y) \cap
            \Ad(x)$. 
            \item If it is $\Ad(x)$ that lost $c$, then we have $\{a,b\} = \{x,c\}$.
            Without loss of generality, $a = x$ and $b = c$, so $b \in \Ne(y) \cap
            \Ad(x)$.
        \end{itemize} 
            So gathering all cases and with generality:
            $$ \curlystack{ a \in  \{x,y\} \\ 
                b \in \Ne(y) \cap \Ad(x) } \text{ or } \curlystack{ b \in  \{x,y\} \\ 
                a \in \Ne(y) \cap \Ad(x) }$$
        \item The update U4 does not render any edge adjacent, does not remove any
        element from any $\Ad(x')$, but removes elements from $\Ne^u(a)$ or $\Ne^u(b)$
        (resp.$b$ or $a$). So by \Cref{lemma:I3}, $a=y$ (resp. $b=y$) and $b \in \Ne(y)
        \cap \Ad(x)$ (resp. $a \in \Ne(y) \cap \Ad(x)$).
        \item The update U5 does not render any edge adjacent, does not remove any
        element from any $\Ne(y')$, but removes elements from $\Ad^u(a)$ or $\Ad^u(b)$
        (resp.$b$ or $a$). So by \Cref{lemma:I3}, $a=x$ (resp. $b=x$) and $b \in \Ne(y)
        \cap \Ad(x)$ (resp. $a \in \Ne(y) \cap \Ad(x)$).
        \item The updates U6 and U7 cannot remove any element from any $\Ne(y')$ or
        $\Ad(x')$, and do not make $a$ and $b$ adjacent (they were already adjacent). So
        by \Cref{lemma:I3}, they cannot make $\mathbf{I3}(x,y, T; E)$ become true.
\end{itemize}
\end{proof}

\subsubsection{I4}
We start with a general lemma for I4.
\begin{lemma}[I4 to become true]\label{lemma:I4} Assume $\mathbf{I4}(x,y, T; E)$ is
    false and becomes true after an edge update about $(a,b)$. If the update does not
    reverse a directed edge, does not direct an undirected edge, and does not delete an
    edge, then the update must have added an element to $[\Ne(y) \cap \Ad(x)]$.
    
    Otherwise, the update invalidated an edge on a semi-directed path from $y$ to $x$
    (where invalidated means that the edge $(a,b)$ cannot be traversed from $a$ to $b$
    anymore with the semi-directed rules: either $a$ and $b$ are not adjacent anymore,
    or the edge is now $a \leftarrow b$.).
\end{lemma}

\begin{proof}
    We recall that $\mathbf{I4}(x,y, T; E)$ is: all semi-directed paths from $y$ to $x$
    have a node in $[\Ne(y) \cap \Ad(x)] \cup T$. If the condition does not hold before
    the update, then there exists a semi-directed path from $y$ to $x$ with no node in
    $[\Ne(y) \cap \Ad(x)] \cup T$. We distinguish two cases:

    If the update does not remove or reverse any edge, then the semi-directed path is
    still there after the update. For the condition to become true, the update must have
    added an element to $[\Ne(y) \cap \Ad(x)] \cup T$ (one element that is on the
    semi-directed path).

    Since $T$ does not change, the update must have added an element to $[\Ne(y) \cap
    \Ad(x)]$.

    Otherwise, the semi-directed path from $y$ to $x$ is not a semi-directed path
    anymore. So the update invalidated an edge on it: either $a$ and $b$ are not
    adjacent anymore, or the edge is now $a \leftarrow b$.
\end{proof}

\begin{opup}[Updates on I4]
    \label{opup:updates_on_I4}
    Assume $\mathbf{I4}(x,y, T; E)$ is false and becomes true after an update involving
    $(a,b)$. Then,
    \begin{itemize}
        \item if the update is $U1: (a \quad b) \rightsquigarrow (a - b)$ then
        $\curlystack{b=y \\ a \in \Ad(x)}$ or $\curlystack{b=x \\ a \in \Ne(y)}$ or
        $\curlystack{a=y \\ b \in \Ad(x)}$ or $\curlystack{a=x \\ b \in \Ne(y)}$,
        \item if the update is $U2: (a \quad b) \rightsquigarrow (a \rightarrow b)$ then
        $\curlystack{b=x \\ a \in \Ne(y)}$ or $\curlystack{a=x \\ b \in \Ne(y)}$.
        % \item if the update is $U3: (a - b) \rightsquigarrow (a \quad b)$ or $U4: (a -
        % b) \rightsquigarrow (a \rightarrow b)$ or $U5: (a \rightarrow b)
        % \rightsquigarrow (a \quad b)$ or $U7: (a \rightarrow b) \rightsquigarrow (a
        % \leftarrow b)$         
        %  then $a - b$ (or $a \rightarrow b$) is in on a semi-directed path from $y$ to
        % $x$ . $\curlystack{b=x \\ a \in \Ne(y)}$ or $\curlystack{a=x \\ b \in
        % \Ne(y)}$,
        \item if the update is $U3: (a - b) \rightsquigarrow (a \quad b)$ then either
        $(a,b)$ or $(b,a)$ was on a semi-directed path from $y$ to $x$.

        \item if the update is $U4: (a - b) \rightsquigarrow (a \rightarrow b)$ then
        $(b,a)$ was on a semi-directed path from $y$ to $x$.
        \item if the update is $U5: (a \rightarrow b) \rightsquigarrow (a \quad b)$ then
        $(a,b)$ was on a semi-directed path from $y$ to $x$.
        \item if the update is $U6: (a \rightarrow b) \rightsquigarrow (a - b)$ then
        $\curlystack{b=y \\ a \in \Ad(x)}$ or $\curlystack{a=y \\ b \in \Ad(x)}$.
        \item if the update is $U7: (a \rightarrow b) \rightsquigarrow (a \leftarrow b)$
        then $(a,b)$ was on a semi-directed path from $y$ to $x$.
    \end{itemize}
\end{opup}

\begin{proof}
    Assume $\mathbf{I4}(x,y, T; E)$ is false and becomes true after an update involving
    $(a,b)$. U1, U2, and U6 do not reverse any directed edge, do not direct any
    undirected edge, and do not delete any edge. So by \Cref{lemma:I4}, these updates
    must have added an element to $[\Ne(y) \cap \Ad(x)]$. Without loss of generality for
    now, assume $a$ was added. Notice that $a$ cannot have been added to both $\Ne(y)$
    and $\Ad(x)$ (otherwise $y=b$ and $x=b$, yet $x\neq y$), so $a$ was already in one
    of them before the update. 
    \begin{itemize}
        \item If the update is U1 then $a$ can have been added to $\Ne(y)$ or $\Ad(x)$,
        and already present in the other one. Hence we have, in full generality,
        $\curlystack{b=y \\ a \in \Ad(x)}$ or $\curlystack{b=x \\ a \in \Ne(y)}$ or
        $\curlystack{a=y \\ b \in \Ad(x)}$ or $\curlystack{a=x \\ b \in \Ne(y)}$.
        \item If the update is U2 then $a$ can only have been added to $\Ad(x)$, and so
        already present in $\Ne(y)$. Hence we have, in full generality, $\curlystack{b=x
        \\ a \in \Ne(y)}$ or $\curlystack{a=x \\ b \in \Ne(y)}$.
        \item If the update is U6 then $a$ can only have been added to $\Ne(y)$, and so
        already present in $\Ad(x)$. Hence we have, in full generality, $\curlystack{b=y
        \\ a \in \Ad(x)}$ or $\curlystack{a=y \\ b \in \Ad(x)}$.
    \end{itemize}

    U3, U4, U5, and U7 cannot add any element to $[\Ne(y) \cap \Ad(x)]$. So by
    \Cref{lemma:I4}, these updates must have invalidated an edge on a semi-directed path
    from $y$ to $x$. 
    \begin{itemize}
        \item If the update is U3 then either $(a,b)$ or $(b,a)$ was on a semi-directed
        path from $y$ to $x$.
        \item If the update is U4 then $(b,a)$ was on a semi-directed path from $y$ to
        $x$.
        \item If the update is U5 or U7, then $(a,b)$ was on a semi-directed path from
        $y$ to $x$.        
    \end{itemize}
\end{proof}

So far, all the necessary conditions for the updates were efficient to test, e.g. finding
all the insert with $y \in \{a, b\}$.

With $\mathbf{I4}$ however, we have the condition that $(a,b)$ or $(b,a)$ was on a
semi-directed path from $y$ to $x$. This can be inefficient to test. 
A speed-up can be obtained by proceeding as follows
\begin{itemize}
    \item Instead of ensuring that $\mathcal{C}$ always contains all valid Insert
    operators (which means all operators that with $\mathbf{I1}$, $\mathbf{I2}$,
    $\mathbf{I3}$, $\mathbf{I4}$, and $\mathbf{I5}$ are true), we can ensure that
    $\mathcal{C}$ always contains all Insert operator for which $\mathbf{I1}$,
    $\mathbf{I2}$, $\mathbf{I3}$, and $\mathbf{I5}$ are true. 
    \item Recall that at each step, XGES involves 4 substeps described in
    \Cref{sec:efficient_algorithmic_formulation}. Substep 3 verifies that the operator
    is valid. If we notice an operator that is invalid because of $\mathbf{I4}$, then we
    can put it aside. Additionally, we save the path from $y$ to $x$ that was rendering 
    the operator invalid. 
    \item The new necessary condition for the operator to be valid is that an edge on
    the saved path gets removed (or blocked). Whenever we remove or block an edge from
    the path, we can re-verify the operator.
\end{itemize}

\subsubsection{I5}
We start with a general lemma for I5.

\begin{lemma}[I5 to become true]\label{lemma:I5} Assume $\mathbf{I5}(x,y, T; E)$ is
    false and becomes true after an edge update about $(a,b)$. Then (i) $\Pa(y)$
    changed, or (ii) $[\Ne(y) \cap \Ad(x)]$ changed.

    Condition (ii) can be further broken down into: (ii.a) $\Ne^u(y)$ lost an element
    that is in $\Ad(x)$, or (ii.b) $\Ne^u(y)$ gained an element that is in $\Ad(x)$, or
    (ii.c) $\Ne^u(y)$ gained an element that is in $\Ad(x)$, or (ii.d) $\Ad^u(x)$ gained
    an element that is in $\Ne(y)$.
\end{lemma}

\begin{proof}
    Assume $\mathbf{I5}(x,y, T; E)$ is false and becomes true after an edge update about
    $(a,b)$. Since $E$ and $T$ do not change, we must have $[\Ne^u(y) \cap \Ad^u(x)]
    \cup \Pa^u(y) \neq [\Ne(y) \cap \Ad(x)] \cup \Pa(y)$. Either $\Pa(y)$ changed or
    $[\Ne(y) \cap \Ad(x)]$ changed.

    Assume $[\Ne(y) \cap \Ad(x)]$ changed. If it lost an element $c \in \Ne(y) \cap
    \Ad(x)$, then $c$ was removed from $\Ne^u(y)$ or $\Ad^u(x)$. If it gained an element
    $c \not\in \Ne(y) \cap \Ad(x)$, then $c$ was added to $\Ne^u(y)$ or $\Ad^u(x)$.
    Since $c \in \Ne^u(y) \cap \Ad^u(x)$ and only one of $\Ne^u(y)$ and $\Ad^u(x)$ can
    change at a time (see proof \Cref{opup:updates_on_I4}), then either $c$ was added to
    $\Ne^u(y)$ and $c \in \Ad(x)$, or $c$ was added to $\Ad^u(x)$ and $c \in \Ne(y)$.
\end{proof}

\begin{opup}[Updates on I5]
    Assume $\mathbf{I5}(x,y, T; E)$ is false and becomes true after an update involving
    $(a,b)$. Then,
    \begin{itemize}
        \item if the update is $U1: (a \quad b) \rightsquigarrow (a - b)$ then
        $\curlystack{b=y \\ a \in \Ad(x)}$ or $\curlystack{b=x \\ a \in \Ne(y)}$ or
        $\curlystack{a=y \\ b \in \Ad(x)}$ or $\curlystack{a=x \\ b \in \Ad(y)}$,
        \item if the update is $U2: (a \quad b) \rightsquigarrow (a \rightarrow b)$ then
         $y = b$ or $\curlystack{b=x \\ a \in \Ne(y)}$ or $\curlystack{a=x \\ b \in
         \Ne(y)}$,
        \item if the update is $U3: (a - b) \rightsquigarrow (a \quad b)$ then
        $\curlystack{b=y \\ a \in \Ad(x)}$ or $\curlystack{b=x \\ a \in \Ne(y)}$ or
        $\curlystack{a=y \\ b \in \Ad(x)}$ or $\curlystack{a=x \\ b \in \Ad(y)}$,
        \item if the update is $U4: (a - b) \rightsquigarrow (a \rightarrow b)$ then
        $y=b$ or $\curlystack{a=y \\ b \in \Ad(x)}$ or $\curlystack{b = y \\ a \in
        \Ad(x)}$.
        \item if the update is $U5: (a \rightarrow b) \rightsquigarrow (a \quad b)$ then
        $y=b$ or $\curlystack{a=x \\ b \in \Ne(y)}$ or $\curlystack{b = x \\ a \in
        \Ne(y)}$.
        \item if the update is $U6: (a \rightarrow b) \rightsquigarrow (a - b)$ then
        $y=b$ or $\curlystack{a=y \\ b \in \Ad(x)}$ or $\curlystack{b = y \\ a \in
        \Ad(x)}$.
        \item if the update is $U7: (a \rightarrow b) \rightsquigarrow (a \leftarrow b)$
        then $y\in\{a,b\}$.
    \end{itemize}
\end{opup}

\begin{proof}
    According to \Cref{lemma:I5}, either $\Pa(y)$ changed, or $[\Ne(y) \cap \Ad(x)]$
    changed.

    We now study the necessary conditions for each update, if it was applied and made
    $\mathbf{I5}(x,y, T; E)$ become true.
    \begin{itemize}
        \item U1 does not change $\Pa(y)$, and cannot remove any element from any
        $\Ne(y')$ or $\Ad(x')$. So by \Cref{lemma:I5}, (ii.c) or (ii.d) happened. Hence:
        $\curlystack{b=y \\ a \in \Ad(x)}$ or $\curlystack{b=x \\ a \in \Ne(y)}$ or
        $\curlystack{a=y \\ b \in \Ad(x)}$ or $\curlystack{a=x \\ b \in \Ne(y)}$.
        \item U2 changes $\Pa(y)$ if $y = b$. It can also add an element to $\Ad^u(x)$
        so: $y=b$ or $\curlystack{b=x \\ a \in \Ne(y)}$ or $\curlystack{a=x \\ b \in
        \Ne(y)}$.
        \item U3 does not change $\Pa(y)$, and does not add any element to any $\Ne(y')$
        or $\Ad(x')$.\\
        So by \Cref{lemma:I5}, (ii.a) or (ii.b) happened. Hence: $\curlystack{a=x \\ b
        \in \Ne(y) }$ or $\curlystack{a=y \\ b \in \Ad(x)}$ or $\curlystack{b=x \\ a \in
        \Ne(y) }$ or $\curlystack{b=y \\ a \in \Ad(x)}$.
        \item U4 changes $\Pa(y)$ if $y = b$. It also removes an element from $\Ne^u(a)$
        and $\Ne^u(b)$.\\
        So: $y=b$ or $\curlystack{a=y \\ b \in \Ad(x)}$ or $\curlystack{b = y \\ a \in
        \Ad(x)}$.
        \item U5 changes $\Pa(y)$ if $y = b$. It also removes an element from $\Ad^u(a)$
        and $\Ad^u(b)$.\\
        So: $y=b$ or $\curlystack{a=x \\ b \in \Ne(y)}$ or $\curlystack{b = x \\ a \in
        \Ne(y)}$.
        \item U6 changes $\Pa(y)$ if $y = b$. It also adds an element to $\Ne^u(a)$ and
        $\Ne^u(b)$.\\
        So: $y=b$ or $\curlystack{a=y \\ b \in \Ad(x)}$ or $\curlystack{b = y \\ a \in
        \Ad(x)}$.
        \item U7 changes $\Pa(y)$ if $y \in \{a,b\}$. It does not change any $\Ne(y')$
        or $\Ad(x')$. So $y \in \{a,b\}$.
    \end{itemize}
\end{proof}

\subsubsection{D1}

\begin{opup}[Updates on D1]
    Assume $\mathbf{D1}(x,y, T; E)$ is false and becomes true after an update involving
    $(a,b)$. Then, 
    \begin{itemize}
        \item if the update is $U1: (a \quad b) \rightsquigarrow (a - b)$ then $\{a,b\}
        = \{x,y\}$,
        \item if the update is $U2: (a \quad b) \rightsquigarrow (a \rightarrow b)$ then
        $(a,b) = (x,y)$,
        \item the update cannot be $U3: (a - b) \rightsquigarrow (a \quad b)$,
        \item the update cannot be $U4: (a - b) \rightsquigarrow (a \rightarrow b)$,
        \item the update cannot be $U5: (a \rightarrow b) \rightsquigarrow (a \quad b)$,
        \item if the update is $U6: (a \rightarrow b) \rightsquigarrow (a - b)$ then
        $(a,b) = (y,x)$,
        \item if the update is $U7: (a \rightarrow b) \rightsquigarrow (a \leftarrow b)$
        then $(a,b) = (y,x)$.
    \end{itemize}
\end{opup}

\begin{proof}
    We recall that $\mathbf{D1}(x,y, C; E)$ is: $y \in \Ch(x) \cup \Ne(x)$. Assume 
    $\mathbf{D1}(x,y, C; E)$ is false and becomes true after an update involving $(a,b)$.
    \begin{itemize}
        \item U3 and U5 cannot add any element to $y$'s children or neighbors. So they
        cannot make $\mathbf{D1}(x,y, C; E)$ become true.
        \item U4 does not add any elements to any $\Ch^u(x') \cup \Ne^u(x')$, so it cannot
        make $\mathbf{D1}(x,y, C; E)$ become true.
        \item U1 only adds $a$ to $\Ne^u(b)\cup \Ch^u(b)$ and $b$ to $\Ne^u(a)\cup \Ch^u(a)$. So $\{a,b\} = \{x,y\}$.
        \item U2 only adds $b$ to $\Ch^u(a)\cup \Ne^u(a)$ so $(a,b) = (x,y)$.
        \item U6 only adds $a$ to $\Ch^u(b) \cup \Ne^u(b)$ so $(a,b) = (y,x)$.
        \item U7 only adds $a$ to $\Ch^u(b)\cup \Ne^u(b)$ so $(a,b) = (y,x)$. 
    \end{itemize}
\end{proof}

\subsubsection{D2}
We start with a general lemma for D2.

\begin{lemma}[D2 to become true]\label{lemma:D2} Assume $\mathbf{D2}(x,y, C; E)$ is
    false and becomes true after an edge update about $(a,b)$. Then $\Ne^u(y)$ gained an
    element that was in $\Ad(x)$, or $\Ad^u(x)$ gained an element that was in $\Ne(y)$.
\end{lemma}

\begin{proof}
    We recall that $\mathbf{D2}(x,y, C; E)$ is: $C \subset \Ne(y) \cap \Ad(x)$. Assume
    $\mathbf{D2}(x,y, C; E)$ is false and becomes true after an edge update about
    $(a,b)$. Since $C$ does not change, we there exists $c\in C$ such that $c \not\in
    \Ne(y) \cap \Ad(x)$ and $c \in \Ne^u(y) \cap \Ad^u(x)$. Then we conclude with a similar 
    reasoning as in \Cref{lemma:I5}.
\end{proof}

\begin{opup}[Updates on D2]
    Assume $\mathbf{D2}(x,y, C; E)$ is false and becomes true after an update involving
    $(a,b)$. Then, 
    \begin{itemize}
        \item if the update is $U1: (a \quad b) \rightsquigarrow (a - b)$ then 
        $\curlystack{b=y \\ a \in \Ad(x)}$ or $\curlystack{b=x \\ a \in \Ne(y)}$ or
        $\curlystack{a=y \\ b \in \Ad(x)}$ or $\curlystack{a=x \\ b \in \Ne(y)}$,
        \item if the update is $U2: (a \quad b) \rightsquigarrow (a \rightarrow b)$ then
        $\curlystack{b=x \\ a \in \Ne(y)}$ or $\curlystack{a=x \\ b \in \Ne(y)}$,
        \item the update cannot be $U3: (a - b) \rightsquigarrow (a \quad b)$,
        \item the update cannot be $U4: (a - b) \rightsquigarrow (a \rightarrow b)$,
        \item the update cannot be $U5: (a \rightarrow b) \rightsquigarrow (a \quad b)$,
        \item if the update is $U6: (a \rightarrow b) \rightsquigarrow (a - b)$ then
        $\curlystack{a=y \\ b \in \Ad(x)}$ or $\curlystack{b=y \\ a \in \Ad(x)}$,
        \item the update cannot be $U7: (a \rightarrow b) \rightsquigarrow (a \leftarrow b)$.
    \end{itemize}
\end{opup}

\begin{proof}
    According to \Cref{lemma:D2}, either $\Ne^u(y)$ gained an element that was in
    $\Ad(x)$, or $\Ad^u(x)$ gained an element that was in $\Ne(y)$. We now study the
    necessary conditions for each update, if it was applied and made $\mathbf{D2}(x,y, C;
    E)$ become true.
    \begin{itemize}
        \item U3, U4, U5 and U7 do not add any element to $\Ne^u(y)$ or $\Ad^u(x)$. So they
        cannot make $\mathbf{D2}(x,y, C; E)$ become true.
        \item U1's only modifications to sets $\Ne(y')$ and $\Ad(x')$ are to add $a$ to $\Ne^u(b)$, add $a$ to $\Ad^u(b)$, and 
        same exchanging $a$ and $b$. So by \Cref{lemma:D2}, $\curlystack{b=y \\ a \in \Ad(x)}$ or
        $\curlystack{b=x \\ a \in \Ne(y)}$ or $\curlystack{a=y \\ b \in \Ad(x)}$ or $\curlystack{a=x \\ b \in \Ne(y)}$.
        \item U2 's only modifications to sets $\Ne(y')$ and $\Ad(x')$ are to add $b$ to
        and $\Ad^u(a)$ and same exchanging $a$ and $b$. So by \Cref{lemma:D2} 
        $\curlystack{a=x \\ b \in \Ne(y)}$ or $\curlystack{b = x \\ a \in \Ne(y)}$.
        \item U6's only modifications to sets $\Ne(y')$ and $\Ad(x')$ are to add $a$ to
        $\Ne^u(b)$ and same exchanging $a$ and $b$. So by \Cref{lemma:D2} 
        $\curlystack{b=y \\ a \in \Ad(x)}$ or $\curlystack{a=y \\ b \in \Ad(x)}$. 
    \end{itemize}
\end{proof}

\subsubsection{D3}

\begin{opup}[Updates on D3]
    Assume $\mathbf{D3}(x,y, C; E)$ is false and becomes true after an edge update involving
    $(a,b)$. Also assume that $\mathbf{D2}(x,y, C; E)$ holds true after the edge update, then
    \begin{itemize}
        \item if the update is $U1: (a \quad b) \rightsquigarrow (a - b)$ then $\{a,b\}
        \subset \Ne^u(y) \cap \Ad^u(x)$,
        \item if the update is $U2: (a \quad b) \rightsquigarrow (a \rightarrow b)$ then
        $\{a,b\} \subset \Ne^u(y) \cap \Ad^u(x)$,
        \item the update cannot be $U3: (a - b) \rightsquigarrow (a \quad b)$,
        \item the update cannot be $U4: (a - b) \rightsquigarrow (a \rightarrow b)$,
        \item the update cannot be $U5: (a \rightarrow b) \rightsquigarrow (a \quad b)$,
        \item the update cannot be $U6: (a \rightarrow b) \rightsquigarrow (a - b)$,
        \item the update cannot be $U7: (a \rightarrow b) \rightsquigarrow (a \leftarrow b)$.
    \end{itemize}
\end{opup}

\begin{proof}
    We recall that $\mathbf{D3}(x,y, C; E)$ is: $C$ is a clique. Assume 
    $\mathbf{D3}(x,y, C; E)$ is false and becomes true after an update involving $(a,b)$.
    Similarly to
\Cref{lemma:I3}, since $C$ is not changed by an edge update, the only way for 
$\mathbf{D3}(x,y, C; E)$ to become true is for the edge update to connect two nodes in $C$ that
were not adjacent before. Only U1 and U2 render two nodes adjacent, namely $a$ and $b$,
so they are the only updates that can make $\mathbf{D3}(x,y, C; E)$ become true.

For U1 and U2, we have: $\{a,b\} \subset C$. If we further assume that $\mathbf{D2}(x,y, C; E)$ holds true after the edge update, then
$\{a,b\} \subset C \subset \Ne^u(y) \cap \Ad^u(x)$.
\end{proof}

\subsubsection{D4}
\begin{opup}[Updates on D4]
    Assume $\mathbf{D4}(x,y, C; E)$ is false and becomes true after an edge update involving
    $(a,b)$. Then, 
    \begin{itemize}
        \item the update cannot be $U1: (a \quad b) \rightsquigarrow (a - b)$,
        \item if the update is $U2: (a \quad b) \rightsquigarrow (a \rightarrow b)$ then
        $y = b$,
        \item the update cannot be $U3: (a - b) \rightsquigarrow (a \quad b)$,
        \item if the update is $U4: (a - b) \rightsquigarrow (a \rightarrow b)$, then
        $y=b$,
        \item if the update is $U5: (a \rightarrow b) \rightsquigarrow (a \quad b)$, then $y=b$,
        \item if the update is $U6: (a \rightarrow b) \rightsquigarrow (a - b)$ then $y=b$,
        \item if the update is $U7: (a \rightarrow b) \rightsquigarrow (a \leftarrow b)$ then
        $y\in\{a,b\}$.
    \end{itemize}

    \begin{proof}
        Recall that $\mathbf{D4}(x,y, C; E)$ is: $E = C \cup \Pa(y)$. Since $C$ and $E$
        do not change, the only way for $\mathbf{D4}(x,y, C; E)$ to become true is for
        the edge update to change $\Pa(y)$. The only updates that can change $\Pa(y)$
        are U2, U4, U5, U6, and U7, when $y$ is $b$, or U7 when $y$ is $a$ or $b$.
    \end{proof}
    
\end{opup}

\subsubsection{R1 to R6}
The reverse operators are very similar to the insert operators, and the necessary conditions
can be adapted. We add them to \Cref{tab:operator_updates}.

\subsection{Issue with Fast GES }
\label{appendix:subsec:fast_ges}
We found an issue with the Fast GES algorithm that may explain its degraded performance
compared to the GES algorithm. 

During its efficient update of the operators, fGES computes the score of all possible
operators Insert($x,y,T$) for a pair of nodes $x$ and $y$, but only saves the Insert
with the highest score. However, this insert might not be a valid operator (for example,
it might not satisfy the I3 constraint). Meanwhile, another Insert($x,y,T'$) for the
same pair of nodes $x$ and $y$ might be valid and have the highest score of all valid
operators. Such an operator would be missed by fGES.






\end{document}
% \endinput
%%
%% End of file `sample-authordraft.tex'.
