\documentclass[10pt,journal,compsoc]{IEEEtran}
\usepackage{array}
%\usepackage[caption=false,font=normalsize,labelfont=sf,textfont=sf]{subfig}
\usepackage{textcomp}
\usepackage{subfigure}
\usepackage{stmaryrd}

\usepackage{stfloats}
\usepackage{url}
\usepackage{verbatim}
\usepackage{graphicx}
\hyphenation{op-tical net-works semi-conduc-tor IEEE-Xplore}
\def\BibTeX{{\rm B\kern-.05em{\sc i\kern-.025em b}\kern-.08em
    T\kern-.1667em\lower.7ex\hbox{E}\kern-.125emX}}
\usepackage{balance}
\let\Bbbk\relax
\usepackage{amsmath,amssymb,amsfonts}
%\usepackage{syntax}
\usepackage{hyperref}
\hypersetup{colorlinks=true}
\def\equationautorefname~#1\null{%
  Equation~(#1)\null
}
\usepackage{algorithm}
\usepackage[noend]{algpseudocode}
% \usepackage{algorithmic}
\usepackage{colortbl}
\usepackage{xcolor}
\usepackage{colortbl}%
\newcommand{\grayrow}{\rowcolor[HTML]{EFEFEF}}
\newcommand{\graycell}{\cellcolor[HTML]{EFEFEF}}
\usepackage{capt-of,lipsum}
\usepackage{amsthm}
\definecolor{mycolor}{rgb}{0.95, 0.985, 0.93}
\doublerulesepcolor{mycolor}
\usepackage{xspace}
\usepackage{color}
\usepackage{tcolorbox}
\definecolor{mGray1}{rgb}{0.9,0.9,0.9}
\definecolor{mGray}{rgb}{0.5,0.5,0.5}
\definecolor{commentcolor}{rgb}{0.6,0.6,0.6}
\usepackage{enumitem}
% \usepackage[linesnumbered,ruled,vlined]{algorithm2e}
\newtheorem{definition}{Definition}
% \usepackage[english]{babel}
% \hyphenation{Dail-yMail}
\usepackage{pifont}
\usepackage{xcolor}
\usepackage{listings}
\newcommand{\m}{\mathit} 
\newcommand{\mtl}{\phi} 
\newcommand{\interval}{I} 
\newcommand{\relation}{R} 
\newcommand{\Obj}{o} 
\newcommand{\Subj}{s} 
\newcommand{\OBJ}{O} 
\newcommand{\SUBJ}{S} 
\newcommand{\Prolog}{\mathcal{P}}
\newcommand{\drule}{Q}
\newcommand{\ap}{\m{ap}}
\newcommand{\hornarrow}{\,\text{:--}\,}
\newcommand{\curly}{\,\mathrel{\leadsto}\,}
\newcommand{\encoding}[3]{#1{\curly}(#2, #3)}
\newcommand{\deriveRules}[3]{#1\,{\hookrightarrow}\,(#2, #3)}
\newcommand{\nm}{\m{nm}}
\newcommand{\entity}{\m{entity}}
\newcommand{\groundTruthTriples}{\widetilde{\relation}_{\m{ground}}}
\newcommand{\derivedFacts}{\widetilde{\relation}_{\m{derived}}}
\newcommand{\deriveKG}[4]{#1, #2\,{\hookrightarrow}\,(#3, #4)}
\newcommand{\llmResponse}{\m{Resp}}
\newcommand{\eval}{\m{V}}
\newcommand{\semantic}{\widetilde{G}}
\newcommand{\similarity}{S}


\usepackage{makecell} 
\usepackage[bottom]{footmisc}
%\feetbelowfloat
\newcommand{\RNeg}{[\relation\text{-}\m{Neg}]}
\newcommand{\RSym}{[\relation\text{-}\m{Sym}]}
\newcommand{\RInv}{[\relation\text{-}\m{Inverse}]}
\newcommand{\RTrans}{[\relation\text{-}\m{Trans}]}


\newcommand{\commentstyle}[1]{\textcolor{mGray}{\footnotesize{#1}}}


\newcommand{\domain}{$\mathcal{D}$}
\newcommand{\entityCat}{$\m{EC}$}
\newcommand{\relationCat}{$\m{RC}$}

\algrenewcommand\algorithmicindent{1.3em}%

\usepackage[normalem]{ulem}
\newcommand{\shortNeg}{!} %\mathtt{Neg}
\newcommand{\syh}[1]{{\small \ttfamily \color{purple}{{{YH:#1}}}}}
\newcommand\figref[1]{Fig.\,{\ref{#1}}}
%\textcolor{blue}
\newcommand\tabref[1]{Table \textcolor{blue}{\ref{#1}}}
\newcommand\secref[1]{\S \textcolor{blue}{\ref{#1}}}
\usepackage{adjustbox}
\usepackage{wrapfig}
\newcommand\theoref[1]{Theorem~\textcolor{blue}{\ref{#1}}}
\newcommand\lemmaref[1]{Lemma~\textcolor{blue}{\ref{#1}}}
\newcommand\appref[1]{Appendix~\textcolor{blue}{\ref{#1}}}
\newcommand\defref[1]{Definition~\textcolor{blue}{\ref{#1}}}
\newcommand\algoref[1]{Algorithm~\textcolor{blue}{\ref{#1}}}
\newcommand{\trans}{\m{R}}
\newcommand{\Istart}{n_{\m{start}}}
\newcommand{\Iend}{n_{\m{end}}}
\newcommand{\history}{\mathcal{H}}
\newcommand{\timepoint}{t}
\newcommand{\distance}{d}
%\widetilde{\relation_{\m{TS}}}
\newcommand\nmNEW{\nm_{\m{new}}}
\usepackage{thmtools} 
\usepackage{thm-restate}
\declaretheorem[name=Theorem,numberwithin=section]{thm}
\newcommand{\plus}{\texttt{+}}



\usepackage{subcaption}
\usepackage{booktabs}


\definecolor{keywordcolor}{rgb}{0.13,0.29,0.53}
\definecolor{stringcolor}{rgb}{0.31,0.60,0.02}
\definecolor{commentcolor}{rgb}{0.56,0.35,0.01}
\definecolor{backcolour}{rgb}{0.95,0.95,0.92}

%\usepackage[natbibapa]{apacite} 
% \usepackage[
% backend=biber,
% style=authoryear,
% ]{biblatex}
% \addbibresource{8.ref.bib}

\lstset{
 language=C,
 escapeinside={(*@}{@*)},
 basicstyle=\small\ttfamily,
 columns=[c]fixed,
 numbers=left,   
 xleftmargin=2em, 
 numberstyle=\tiny\color{mGray},
 commentstyle=\color{commentcolor}\ttfamily,
 keywordstyle=\color{airforceblue}\bfseries,
 upquote=true,
 breaklines=true,
 showstringspaces=false,
 stringstyle=\color{black},
 keywordstyle=[2]\color{purple}\ttfamily, %
 morekeywords=[2]{int, if , return},
 literate={'"'}{\textquotesingle "\textquotesingle}3
}

\newcommand{\code}[1]{{\fontfamily{cmtt}\fontseries{m}\fontshape{n}\selectfont\small{#1}}}

\newcommand{\listitem}[1]{\textit{\textbf{#1}}}
\newcommand{\head}[1]{{\noindent\textbf{#1}}}


\newcommand{\tool}{\textsc{Drowzee}\xspace}
\newcommand{\mtltoNL}{\textsc{mtl2NL}\xspace}
\newcommand{\iyear}{\m{t}}

\newcommand{\instruction}{\textsc{Instruction}\xspace}
\newcommand{\query}{\textsc{Query}\xspace}
\newcommand{\hallucinationAnswer}{\textsc{Hallucinations Answer}\xspace}

\newcommand{\yi}[1]{\textcolor{blue}{(Yi: #1)}}
\newcommand{\lnk}[1]{\textcolor{magenta}{#1}}
\newcommand{\lyk}[1]{\textcolor{purple}{#1}}
\newcommand{\wkl}[1]{\textcolor{brown}{#1}}
\newcommand{\shil}[1]{\textcolor{orange}{(SL: #1)}}

\begin{document}

%%
%% The "title" command has an optional parameter,
%% allowing the author to define a "short title" in page headers.

%Effectively 
\title{Detecting LLM Fact-conflicting Hallucinations Enhanced by Temporal-logic-based Reasoning}

\author{
Ningke Li$^{\star}$,
Yahui Song$^{\star}$,
Kailong Wang$^{\dagger}$,
Yuekang Li,
Ling Shi,
Yi Liu,
Haoyu Wang
    \thanks{N. Li, K. Wang, and H. Wang are with Huazhong University of Science and Technology, China. 
    E-mail: \{lnk\_01, wangkl, haoyuwang\}@hust.edu.cn}
    \thanks{Y. Song is with the National University of Singapore, Singapore.\protect\\
    E-mail: yahui\_s@nus.edu.sg}
    \thanks{Y. Li is with the University of New South Wales, Australia.\protect\\
    E-mail: yuekang.li@unsw.edu.au}
    \thanks{L. Shi and Y. Liu are with Nanyang Technological University, Singapore.\protect\\
    E-mail: ling.shi@ntu.edu.sg, yi009@e.ntu.edu.sg}
\thanks{${\star}$ Ningke Li and Yahui Song contribute equally to this work.}
\thanks{${\dagger}$ Kailong Wang is the corresponding author.}
}


% The paper headers
%\markboth{Journal of \LaTeX\ Class Files,~Vol.~14, No.~8, August~2015}%
%{Shell \MakeLowercase{\textit{et al.}}: Bare Advanced Demo of IEEEtran.cls for IEEE Computer Society Journals}

\IEEEtitleabstractindextext{%
%\IEEEdisplaynontitleabstractindextex{
\begin{abstract}
Large language models (LLMs) face the challenge of hallucinations -- outputs that seem coherent but are actually incorrect. A particularly damaging type is fact-conflicting hallucination (FCH), where generated content contradicts established facts. Addressing FCH presents three main challenges: \textbf{1)} Automatically constructing and maintaining large-scale benchmark datasets is difficult and resource-intensive; \textbf{2)} Generating complex and efficient test cases that the LLM has not been trained on -- especially those involving intricate temporal features -- is challenging, yet crucial for eliciting hallucinations; and \textbf{3)} Validating the reasoning behind LLM outputs is inherently difficult, particularly with complex logical relationships, as it requires transparency in the model's decision-making process. 
% Large language models (LLMs) face critical challenges in generating hallucinations -- coherent but factually inaccurate outputs. One major issue is the \emph{fact-conflicting hallucination} (FCH), where LLMs produce outputs contradicting the known facts. 
% We highlight three challenges in addressing FCH: 
%        \textbf{1)} Automatically constructing and updating large-scale benchmark datasets is hard; 
% \textbf{2)} Automatically verifying the LLM outputs for temporal-logic-based queries is non-trivial; and 
% \textbf{3)} Validating the reasoning steps behind LLM outputs is inherently difficult, especially for complex logical relations. 
%Lacking effective verification of temporal properties and temporal logic in test cases can potentially lead to erroneous outcomes.


This paper presents \tool{}, an innovative end-to-end metamorphic testing framework that utilizes temporal logic to identify fact-conflicting hallucinations (FCH) in large language models (LLMs). \tool{} builds a comprehensive factual knowledge base by crawling sources like Wikipedia and uses automated temporal-logic reasoning to convert this knowledge into a large, extensible set of test cases with ground truth answers. LLMs are tested using these cases through template-based prompts, which require them to generate both answers and reasoning steps. To validate the reasoning, we propose two semantic-aware oracles that compare the semantic structure of LLM outputs to the ground truths. 
Across nine LLMs in nine different knowledge domains, experimental results show that \tool{} effectively identifies rates of non-temporal-related hallucinations ranging from 24.7\% to 59.8\%, and rates of temporal-related hallucinations ranging from 16.7\% to 39.2\%.
Key insights reveal that LLMs struggle with out-of-distribution knowledge and logical reasoning. These findings highlight the importance of continued efforts to detect and mitigate hallucinations in LLMs.

%Experimental results show that \tool{} effectively identifies (non-temporal-related) hallucinations across nine LLMs in nine different domains, with hallucination rates ranging from 24.7\% to 59.8\% \shil{differentiate number for non-temporal and temporal?}. 

\end{abstract}

% Note that keywords are not normally used for peerreview papers.
\begin{IEEEkeywords}
Large Language Model, Hallucination, Temporal Logic, Metamorphic Testing
\end{IEEEkeywords}
}

% make the title area
\maketitle

%\IEEEdisplaynontitleabstractindextext
% \IEEEdisplaynontitleabstractindextext has no effect when using
% compsoc under a non-conference mode.


\IEEEpeerreviewmaketitle
\section{Introduction}
\label{sec::intro}

Embodied Question Answering (EQA) \cite{das2018embodied} represents a challenging task at the intersection of natural language processing, computer vision, and robotics, where an embodied agent (e.g., a UAV) must actively explore its environment to answer questions posed in natural language. While most existing research has concentrated on indoor EQA tasks \cite{gao2023room, pena2023visual}, such as exploring and answering questions within confined spaces like homes or offices \cite{liu2024aligning}, relatively little attention has been dedicated to EQA tasks in  open-ended city space. Nevertheless, extending EQA to city space is crucial for numerous real-world applications, including autonomous systems \cite{kalinowska2023embodied}, urban region profiling \cite{yan2024urbanclip}, and city planning \cite{gao2024embodiedcity}. 
% 1. 环境复杂性   
%    - 地标重复性问题(如区分相似建筑)  
%    - 动态干扰因素(交通流、行人)  
% 2. 行动复杂性  
%    - 长程导航路径规划  
%    - 移动控制、角度等  
% 3. 感知复杂性  
%    - 复合空间关系推理("A楼东侧商铺西边的车辆")  
%    - 时序依赖的观察结果整合

EQA tasks in city space (referred to as CityEQA) introduce a unique set of challenges that fundamentally differ from those encountered in indoor environments. Compared to indoor EQA, CityEQA faces three main challenges: 

1) \textbf{Environmental complexity with ambiguous objects}: 
Urban environments are inherently more complex,  featuring a diverse range of objects and structures, many of which are visually similar and difficult to distinguish without detailed semantic information (e.g., buildings, roads, and vehicles). This complexity makes it challenging to construct task instructions and specify the desired information accurately, as shown in Figure \ref{fig:example}. 

2) \textbf{Action complexity in cross-scale space}: 
The vast geographical scale of city space compels agents to adopt larger movement amplitudes to enhance exploration efficiency. However, it might risk overlooking detailed information within the scene. Therefore, agents require cross-scale action adjustment capabilities to effectively balance long-distance path planning with fine-grained movement and angular control.

3) \textbf{Perception complexity with observation dynamics}: 
% Rich semantic information in urban settings leads to varying observations depending on distance and orientation, which can impact the accuracy of answer generation. 
Observations can vary greatly depending on distance, orientation, and perspective. For example, an object may look completely different up close than it does from afar or from different angles. These differences pose challenges for consistency and can affect the accuracy of answer generation, as embodied agents must adapt to the dynamic and complex nature of urban environments.


\begin{table}
\centering
\caption{CityEQA-EC vs existing benchmarks.}
\label{table:dataset}
\renewcommand\arraystretch{1.2}
\resizebox{\linewidth}{!}{
\begin{tabular}{cccccc}
             & Place  & Open Vocab & Active & Platform  & Reference \\ \hline
EQA-v1      & Indoor & \textcolor{red}{\ding{55}}          & \textcolor{green}{\ding{51}}      & House3D      & \cite{das2018embodied}  \\
IQUAD        & Indoor & \textcolor{red}{\ding{55}}          & \textcolor{green}{\ding{51}}      & AI2-THOR     & \cite{gordon2018iqa} \\
MP3D-EQA     & Indoor & \textcolor{red}{\ding{55}}          & \textcolor{green}{\ding{51}}      & Matterport3D & \cite{wijmans2019embodied} \\
MT-EQA       & Indoor & \textcolor{red}{\ding{55}}          & \textcolor{green}{\ding{51}}      & House3D      & \cite{yu2019multi} \\
ScanQA       & Indoor & \textcolor{red}{\ding{55}}          & \textcolor{red}{\ding{55}}      & -            & \cite{azuma2022scanqa} \\
SQA3D        & Indoor & \textcolor{red}{\ding{55}}          & \textcolor{red}{\ding{55}}      & -            & \cite{masqa3d} \\
K-EQA        & Indoor & \textcolor{green}{\ding{51}}          & \textcolor{green}{\ding{51}}      & AI2-THOR     & \cite{tan2023knowledge} \\
OpenEQA      & Indoor & \textcolor{green}{\ding{51}}          & \textcolor{green}{\ding{51}}      & ScanNet/HM3D & \cite{majumdar2024openeqa} \\
 \hline
CityEQA-EC   & City (Outdoor)  & \textcolor{green}{\ding{51}}          & \textcolor{green}{\ding{51}}      & EmbodiedCity & - \\ \hline
\end{tabular}}
\end{table}

\begin{figure*}[!htb]
\centering
    \includegraphics[width=0.78\linewidth]{figures/example.pdf}
% \vspace{-0.2cm}
\caption{The typical workflow of the PMA to address City EQA tasks. There are two cars in this area, thus a valid question must contain landmarks and spatial relationships to specify a car. Given the task, PMA will sequentially complete multiple sub-tasks to find the answer.}
% \vspace{-0.2cm}
\label{fig:example}
\end{figure*}

As an initial step toward CityEQA, we developed \textbf{CityEQA-EC}, a benchmark dataset to evaluate embodied agents' performance on CityEQA tasks. The distinctions between this dataset and other EQA benchmarks are summarized in Table \ref{table:dataset}. CityEQA-EC comprises six task types characterized by open-vocabulary questions. These tasks utilize urban landmarks and spatial relationships to delineate the expected answer, adhering to human conventions while addressing object ambiguity. This design introduces significant complexity, turning CityEQA into long-horizon tasks that require embodied agents to identify and use landmarks, explore urban environments effectively, and refine observation to generate high-quality answers.

To address CityEQA tasks, we introduce the \textbf{Planner-Manager-Actor (PMA)}, a novel baseline agent powered by large models, designed to emulate human-like rationale for solving long-horizon tasks in urban environments, as illustrated in Figure \ref{fig:example}. PMA employs a hierarchical framework to generate actions and derive answers. The Planner module parses tasks and creates plans consisting of three sub-task types: navigation, exploration, and collection. The Manager oversees the execution of these plans while maintaining a global object-centric cognitive map \cite{deng2024opengraph}. This 2D grid-based representation enables precise object identification (retrieval) and efficient management of long-term landmark information. The Actor generates specific actions based on the Manager's instructions through its components: Navigator, Explorer, and Collector. Notably, the Collector integrates a Multi-Modal Large Language Model (MM-LLM) as its Vision-Language-Action (VLA) module to refine observations and generate high-quality answers.
PMA's performance is assessed against four baselines, including humans. 
Results show that humans perform best in CityEQA, while PMA achieves 60.73\% of human accuracy in answering questions, highlighting both the challenge and validity of the proposed benchmarks. 

% The Frontier-Based Exploration (FBE) Agent, widely used in indoor EQA tasks, performs worse than even a blind LLM. This underscores the importance of PMA's hierarchical framework and its use of landmarks and spatial relationships for tackling CityEQA tasks.

In summary, this paper makes the following significant contributions:
\vspace{-8pt}
\begin{itemize}[leftmargin=*]
    \item To the best of our knowledge, we present the first open-ended embodied question answering benchmark for city space, namely CityEQA-EC.
    \vspace{-7pt}
    \item We propose a novel baseline model, PMA, which is capable of solving long-horizon tasks for CityEQA tasks with a human-like rationale.
     \vspace{-7pt}
    \item Experimental results demonstrate that our approach outperforms existing baselines in tackling the CityEQA task. However, the gap with human performance highlights opportunities for future research to improve visual thinking and reasoning in embodied intelligence for city spaces.
\end{itemize}




\section{Background}
Integrating LLMs into real-world systems demands a robust and interconnected technical stack, driving the creation of a diverse ecosystem of tools and frameworks to support their lifecycle. This section provides an overview of the LLM lifecycle and technical stack, highlighting their complexity and the associated security challenges.

% \subsection{LLM Lifecycle and Tech Stack}
\noindent \textbf{LLM Lifecycle and Tech Stack.} The lifecycle of LLMs involves multiple interconnected stages, each supported by specialized tools and frameworks, forming a complex and comprehensive technical stack. These stages include data collection and preprocessing, model training, optimization, deployment, and post-deployment monitoring. Each stage is crucial for the integration of LLMs into real-world systems, and together they ensure the models are effective and scalable. However, as the stack is highly interconnected, vulnerabilities introduced at any stage—whether during data handling, model training, or deployment—can compromise the overall integrity and performance of the LLM system. \autoref{fig:stack} illustrates the general architecture of the LLM tech stack, which consists of three primary layers:
\begin{figure}[t]
    \centering
    \includegraphics[width=0.95\linewidth]{Figures/stack.pdf}
    \caption{LLM Lifecycle and Tech Stack.}
    \label{fig:stack}
\end{figure}

\noindent \textbf{[A] Data Layer.} The data layer serves as the foundation of the LLM lifecycle, responsible for managing the collection, transformation, storage, and retrieval of large datasets. This layer handles the initial steps of the data pipeline, beginning with transforming raw data into vector representations using embedding models. Tools like SentenceTransformers~\cite{sentence-transformers} are employed to create high-quality embeddings that convert textual data into vector formats suitable for downstream processes. The embedded data is then indexed and stored in systems that facilitate efficient and scalable retrieval, such as vector databases like FAISS~\cite{faiss} and Qdrant~\cite{qdrant}, which facilitates rapid access to relevant data for tasks such as Retrieval-Augmented Generation (RAG) or LLM caching. 
    
\noindent \textbf{[B] Model Layer.} The model layer is essential for the core development, optimization, and deployment of LLMs, providing the necessary tools and frameworks to enhance model performance. Frameworks like Hugging Face’s Transformers~\cite{transformers} facilitate the implementation and fine-tuning of pre-trained models. Supporting techniques such as model quantization and model merging help optimize the model’s size and computational efficiency. LLM operations (LLMOps), such as lunary~\cite{lunary}, are also integrated into this layer, enabling continuous monitoring and refinement of the model's performance throughout its lifecycle, from the initial development phase to deployment.
Once the model is prepared, it is served and utilized through model serving and inference processes. Frameworks such as Triton Inference Server~\cite{triton-inference-server} or Ollama~\cite{ollama} provide the necessary infrastructure to deploy models into production environments, enabling real-time predictions via API endpoints. The inference process then utilizes these models to generate outputs for various tasks, such as text generation or question answering, based on user queries or system requests.
    
\noindent \textbf{[C] Application Layer.} The application layer is responsible for connecting trained LLMs to real-world systems and users, enabling seamless integration and deployment. This layer focuses primarily on orchestration frameworks that automate workflows and manage the interactions between different components. Orchestration tools like LangChain~\cite{langchain} and AutoGPT~\cite{AutoGPT} enable autonomous decision-making and process automation by chaining LLM calls together.
Supporting tools are essential for extending the LLM’s capabilities. For example, LiteLLM~\cite{litellm} acts as an LLM gateway, serving as a proxy that provides a unified interface for calling multiple models in a consistent format. GPTCache~\cite{gptcache} provides caching services to optimize performance and reduce latency, ensuring faster responses during inference. Tools like Haystack~\cite{haystack} support retrieval-augmented generation (RAG), enhancing the LLM's ability to respond to complex queries by retrieving relevant information from external data sources. Additionally, function-calling frameworks like Composio~\cite{composio} can be integrated to enhance agent capabilities, allowing for dynamic interactions with external APIs and systems.
As many LLM systems interact directly with users, front-end frameworks are also a critical part of this layer. Platforms like Anything-LLM~\cite{anythingllm} and LocalAI~\cite{localai} provide interfaces for users to interact with LLMs, enabling easy access to LLM functionalities through user-friendly interfaces. 

% \subsection{LLM Infrastructure Vulnerabilities}
% As the adoption of LLMs continues to grow, the complexity and interconnected nature of their supporting infrastructure have significantly expanded the potential attack surface for adversaries. The LLM ecosystem consists of various components, including third-party libraries, deployment platforms, and orchestration frameworks, all of which introduce unique vulnerabilities. These vulnerabilities, if exploited, can undermine the security of the entire system—affecting aspects such as data integrity, model performance, and user privacy.

% Each layer of the LLM tech stack introduces its own set of challenges. For example, in the Data Layer, vulnerabilities related to data preprocessing, such as improper data sanitization or malicious data injection, can lead to poisoned training data or biased embeddings that degrade model performance or generate biased outputs. Moreover, vulnerabilities in data indexing and retrieval processes, especially in vector databases like FAISS~\cite{faiss} and Qdrant~\cite{qdrant}, can expose sensitive information or enable unauthorized access to stored data~\cite{vecdbvul}, impacting the overall integrity of the LLM system.

% In the Model Layer, vulnerabilities often arise from the tools and techniques used for model development and optimization. Issues such as insecure model merging, unsafe handling of training data, or flaws in quantization processes could lead to compromised models, performance degradation, or adversarial attacks. LLM operations (LLMOps) integrated into this layer—such as tools for monitoring and adjusting model performance—may also introduce risks if improperly configured or managed, enabling potential misuse or unintentional bias during the deployment phase.

% Finally, in the Application Layer, security risks often stem from the orchestration and deployment frameworks that facilitate communication between the LLM and external systems. Vulnerabilities in API endpoints, poor authentication mechanisms, or improper access control can expose the system to attacks such as unauthorized access, denial-of-service attacks, or even the injection of malicious inputs. Frameworks like LangChain~\cite{langchain}, AutoGPT~\cite{AutoGPT}, and LiteLLM~\cite{litellm}, while providing powerful capabilities for automation and model integration, can be vulnerable if not properly secured. Additionally, front-end frameworks like Anything-LLM~\cite{anythingllm} and LocalAI~\cite{localai} that facilitate user interactions also pose risks if they fail to implement adequate input validation and authentication.

% The increasingly complex LLM ecosystem highlights the need for comprehensive security practices at every stage of development and deployment. Vulnerabilities introduced at any layer can cascade and result in substantial security breaches, affecting the performance, reliability, and safety of LLM systems. 





\section{Motivating Examples}\label{sec:motivating}


\begin{comment}
\begin{figure}[h]
\centering
\includegraphics[width=\linewidth]{fig/drowzee-lg-example.pdf}\\
    % \vspace{-0.3cm}
\caption{Examples of logic programming.
\syh{need to re-draw this picture to me more space-friendly}}
% \vspace{-0.2cm}
\label{fig:lg-example}
\end{figure}


%We next define the 
%reasoning rule for  
%relation 
%\[\]
We use the example shown in \figref{fig:lg-example} for a concrete 
demonstration. 
Here, $\m{member}(\m{Gunzo  Prize}, \m{Haruki Murakami Awards})$, is a fact describing that \emph{Gunzon Prize} is one of the prizes awarded to \emph{Haruki Murakami}.  
With the reasoning rule defined for relation $\m{same\_member(List1, List2)}$, which executes to $\m{true}$ if there exists at least one $\m{element}$ that is a member of both $\m{List1}$ and $\m{List2}$, we can write a query as follows, which asks if there exist any common awards won by both \emph{Haruki Murakami} and \emph{Bob Dylan}. 
%An example query is 
\[{?}\text{-}\;  \m{same\_member(HarukiMurakamiAwards, BobDylanAwards)}\]
\end{comment}


%\wkl{Modify, highlight why existing works face challenges and limitations}

\begin{figure*}[!ht]
\centering
\includegraphics[width=0.85\linewidth]{fig/drowzee-motivation.pdf}\\
%\vspace{-1mm}
\caption{Motivating Examples for Automatic Benchmark Construction with Complex Questions}
\label{fig:motivating}
\vspace{-0.1cm}
\end{figure*}




% \begin{figure*}[!ht]
%     \centering
%     \includegraphics[width=0.95\textwidth]{fig/motivating_v2-cropped.pdf}\\
%     \caption{Motivating Example.\wkl{Here briefly introduce what is the content from each example (a)(b) Change caption}\lyk{the generated questions and facts do not match}}
%     %\vspace{-0.5cm}
%     \label{fig:motivating}
% \end{figure*}

%Motivation for 

\subsection{Automatic Benchmark Construction}
As a first motivating example, shown in \figref{fig:motivating}, given 
the facts about whether Haruki Murakami and Bob Dylan have won the Nobel Prize, as illustrated in the left sub-figure, we can query straightforward questions such as ``\emph{whether Haruki Murakami or Bob Dylan has won the Nobel Prize?}''. 
Asking and verifying this knowledge requires no logical reasoning. 
However, such questions are often not enough to unveil hallucinations. 
{Therefore, more diversified questions, i.e., with intertwined and complex information, as illustrated in the right sub-figure, are needed.} 
%To generate more diversified benchmarks, previous research~\cite{yu2023kola, HaluEval} involves human experts to generate the questions and annotate the answers for hallucination checking.
%\figref{fig:motivating} depicts how a human expert would reach to question 2 according to the relations among entities across different facts.
%Although the manually generated benchmarks can unveil certain hallucinations, they suffer from several drawbacks.
{
Moreover, the knowledge landscape is dynamic, with new information continuously surfacing and older information becoming obsolete.} If facts change continuously over time, for instance, if Haruki Murakami were to win the Nobel Prize in the future, this would necessitate regular updates and corrections to the ground truth in existing datasets to reflect them. However, maintaining the accuracy of these benchmarks requires a significant amount of manual labor. 

%Motivation for 
\subsection{Questions Involving Temporal Reasoning}
As LLMs increasingly rely on temporal reasoning to process time-dependent data, understanding how well they can handle temporal logic is crucial for their development and deployment in real-world applications. 
Reasoning temporal-logic-related queries often requires complex steps, particularly regarding the relative timing of multiple events. Unfortunately, manually generated test cases frequently lack thorough completeness verification, which undermines their reliability. 
We demonstrate such an example in \tabref{tab:TemporalLogicRelatedHallucination}. 
The question pertains to the relative timing relationship between the original Ben 10 TV series and the year 2000. It is known that the original Ben 10 series premiered on December 27, 2005, and concluded in 2008. Therefore, when considering the query asking whether ``Ben 10 finally airs within the first three years after 2000'', the expected answer is ``No'', as none of the years from 2005 to 2008 fall within that queried timeframe. 
However, despite the accurate facts inputs, LLMs still generate hallucinations due to insufficient temporal reasoning. 





\begin{table}[!b]
\centering
\renewcommand{\arraystretch}{1.1}
\begin{tabular}{l}
\toprule 
\rowcolor{mycolor} \textbf{\query:}  Did the Original Series of Ben 10 TV show finally air  \\
\rowcolor{mycolor} 
within the time frame of the 1st year to the 3rd year after the year \\
\rowcolor{mycolor}
2000? Please answer Yes or No and explain why. \\[0.5em] \hline
\\[-0.7em] 
\rowcolor{mycolor} 
\textbf{\hallucinationAnswer \rm{(Claude 3.5)}:} Yes, the original series \\
\rowcolor{mycolor}
of Ben 10 aired within the time frame of the 1st year to the 3rd    \\
\rowcolor{mycolor}
year after the year 2000. The show premiered on December 27,  \\
\rowcolor{mycolor}
2005, and concluded its run on April 15, 2008. This timeline 
\\
\rowcolor{mycolor}
clearly falls within the specified period from 2001 to 2003.
\\ 
\bottomrule %添加表格底部粗线
\end{tabular}
\caption{Temporal Logic Reasoning Hallucination}
\label{tab:TemporalLogicRelatedHallucination}
\end{table}


\begin{figure}[!b]
\centering
\begin{lstlisting}[xleftmargin=3.5em,numbersep=8pt,basicstyle=\footnotesize\ttfamily] 
// Ground Facts crawled from Wikipedia
begin('Ben_10',2005). end('Ben_10',2008). 

// Generating the time-stamped fact
ben_10(Start,End) :- Start=<End, 
    begin('Ben_10',Start), end('Ben_10',End).
    
// Encding the MTL formula (*@\color{commentcolor}{$\mtl$}@*)
finally_ben_10_during_1_3(Start,End) :-
    ben_10(Start1,End1), 
    Start is (Start1-3), End is (End1-1), 
    (Start1-3)=<(End1-1), Start=<End.
    
// The time interval (*@\color{commentcolor}{$\interval$}@*) which satisfies (*@\color{commentcolor}{$\mtl$}@*)
?- finally_ben_10_during_1_3(Start,End).
   Start = 2002, End = 2007.
\end{lstlisting} 
\caption{Prolog Encoding for $\mtl \,{=}\, \mathcal{F}_{[1, 3]}(\m{Ben\_10})$}
\label{fig:prologRulesForFinally}
\end{figure}


In this work, our proposed testing framework automatically generates such temporal test cases in the form of MTL formulae, denoted by $\mtl$, which incorporate quantitative timing constraints and enable the expression of temporal relationships with precise intervals.   
For example, this query shown in \tabref{tab:TemporalLogicRelatedHallucination} is represented as  
$\mtl {\,=\,} \mathcal{F}_{[1, 3]}(\m{Ben\_10})$, where $\mathcal{F}$ stand for \emph{finally} and [1,3] is the time frame, querying the time intervals $\interval$ during which the event $\m{Ben\_10}$ finally happen within the time frame of the 1st year to the 3rd year. 
%Then, the final ground truth answer is generated by checking if the year 2000 falls inside $\interval$. 
%If each temporal-related test case needs to be \syh{checked, make it clear that we generate test cases and automatically get the ground truth}, a manual comparison of multiple time points is required. This process is prone to errors, leading to uncertainty in the result validation. Thus, we provide our solution, i.e., automatic verification in Prolog using MTL  encoding.
To obtain the ground truth interval $\interval$, \tool{} automatically generates the Prolog encoding rules for $\mtl$, as shown in \figref{fig:prologRulesForFinally}. These rules accurately determine that $\interval{\,=\,}[2002,2007]$, indicating that all time points within this interval satisfy $\mtl$. 
Then, the validity of 2000 is easily disproved because 2000 is not a member of $\interval$. 

It's important to note that such reasoning rules can be generated for arbitrarily nested MTL formulae. These rules lead to sound and deterministic conclusions regarding a ``Yes" or ``No" response, and they provide reasoning steps to assist in evaluating the LLMs' answers, especially when the queries become more complex. 
Furthermore, obtaining the ground truth interval directly enables us to flexibly control the generation of both positive and negative test cases. 
%\syh{explain the ground truth for generation of the test case.}


%The two examples highlight the benefits of our proposal -- automated techniques to detect hallucinations in LLMs -- compared to existing manual processes for generating and verifying LLMs' output. 
%Further, automatically generating diverse benchmarks and verifying the LLMs' output is equally challenging, and these challenges motivate our novel testing framework, which is detailed in the following section. 

% Consequently, the efficiency and soundness of the manually generated benchmarks are not guaranteed.



% Nevertheless, automatically generating diverse benchmarks is challenging.
% \textbf{First, generating suitable and valid questions is challenging~(challenge\#1).}
% While it is important for the questions in the testing benchmark to cover a diverse range of scenarios, they cannot be randomly generated or arbitrarily selected. Instead, the questions must be logically coherent and aligned with well-established factual knowledge and ground truth.
% \textbf{Second, deriving the test oracles for detecting hallucinations is challenging~(challenge\#2).} The LLM's answer is typically expressed in lengthy and potentially complex sentences. The key to determining if an LLM has produced an FCH lies in assessing whether the overall logical reasoning behind its answer is consistent with the established ground truth. Automatically analyzing and comparing the intricate logical structures within the LLM's response and the factual ground truth remains an inherently difficult task.
%Each question should be coupled with a ground truth answer and this ground truth answer should be comparable against the answers from LLMs.

% These two challenges can both be addressed by leveraging logic programming.
% We can derive new, logically sound facts based on existing knowledge.
% Using the newly derived facts, we can then proceed to generate a wide range of questions, each with its corresponding ground truth answer, enhancing the depth and breadth of our problem-solving capabilities.
% We can generate test oracles to capture hallucinations with the ground truth answers.
%In short, using logic programming to tackle the challenges motivates the design of \tool{}.


% In this section, we first illustrate the limitations of the existing hallucination testing techniques, and briefly introduce how \tool{} overcomes these shortcomings. 

% % \textbf{Challenge\#1: Difficulty for Generating Hallucination Test Cases.} 
% \textbf{Challenge\#1: Generation of FCH Test Cases.}
% One major challenge in the current research on LLM hallucination testing resides in the reliance on fixed benchmark datasets for detecting hallucinations. These datasets are fundamental but static, and therefore fail to cover the full range of real-world scenarios and questions. For instance, as illustrated in Figure~\ref{fig:motivating}(a), even when questions are only slightly altered, with new or different factual knowledge incorporated, they can elicit varied responses. This variability might lead to responses that are incorrect, but these types of errors are not always identified using existing benchmark datasets. Moreover, the focus on factual knowledge in these datasets can result in insufficient examination of hallucinations, especially for factual knowledge that has been changed or ``mutated'' from its original form. 

% Our approach addresses this gap by using logic programming along with four specific reasoning rules, enabling us to automatically create a wider variety of test cases that are more representative of the diverse and dynamic nature of real-world factual knowledge.
% % Current approaches to detecting hallucinations in LLMs predominantly depend on predefined benchmark datasets. These datasets, while useful, offer a narrow perspective due to their static composition, failing to encompass the broad spectrum of real-world scenarios and questions. As demonstrated in Figure~\ref{fig:motivating}(a), the set of diversified questions, while bearing resemblance to the original query, incorporates distinct or additional information. This variation has the potential to elicit different responses, which in turn, could activate instances of hallucination. In this case, the existing methods are unable to detect. Our methodology counters this shortcoming by leveraging logic programming, coupled with four specific reasoning rules. This strategy facilitates the automatic creation of diverse and novel test cases, significantly enhancing the range and relevance of the hallucination detection process compared to the traditional, limited benchmark datasets.

% % \textbf{Challenge\#2: Restricted Scalability.} 
% \textbf{Challenge\#2: Test Oracles for FCH Testing.}
% % This challenge is largely stemming from a substantial dependency on manual labor. Traditional techniques, even when augmented with LLM-assisted methods, necessitate human intervention to validate answers against benchmark datasets. In contrast, our approach capitalizes on the capabilities of logic programming. By employing four distinct reasoning rules, we systematically construct definitive relationships between information entities. As shown in Figure~\ref{fig:motivating}(b), we first extract relations from a list of facts, This framework enables the automated labeling of data based on constituent factual elements. Additionally, the automated generation of test case-oracle pairs through this method facilitates the detection of hallucinations on a much broader scale, significantly reducing the need for human oversight and thereby enhancing scalability.
% Another major challenge is constructing testing oracles with minimal human effort. Previous benchmarks relied heavily on manually labeled data, a process hampered by the intricate logic inherent in each question-answer pair. For illustration, we use a logic reasoning chain to simulate the manual question answering process, as shown in Figure~\ref{fig:motivating}(b). Specifically, the process includes fact and relation extraction, implicit information collection and answer generation steps. This complexity makes it challenging to automatically capture the reasoning needed to derive answers from questions. 

% In contrast, our approach leverages logic programming, by employing four distinct reasoning rules, to systematically establish relationships between various entities. This process enables the automated generation of test case-oracle pairs, reducing reliance on manual labeling and enhancing the scalability of hallucination detection. By automating this process, our method significantly expands the detection capabilities, enabling a broader and more efficient detection of potential hallucinations in LLMs.

\section{Methodology}

\method consists of three key components.
(1) A hierarchical linguistic structure with supporting corpora for linguistic mechanism analysis;
(2) Linguistic feature analysis for interpreting SAE extracted features; and
(3) Linguistic feature intervention for causal analysis and LLM steering.


\begin{figure*}[tp]
    \centering
    \includegraphics[width=0.97\textwidth]{figure/methology.pdf}
    \vspace{-0.01in}
    \caption{
    The overall framework of \method.
    We propose a large-model linguistic mechanism framework encompassing six dimensions and select classical features from these dimensions for experimentation. 
    The experimental workflow is as follows: 
    (1) Construct minimal contrast and counterfactual datasets; 
    (2) Extract features and evaluate their relevance by analyzing the activation values of base vectors on the datasets; 
    (3) Intervene in the model output by modifying activation values and assess causality using an LLM as a judge.
    }\label{fig:method}
\end{figure*}


\subsection{Linguistic Structure}

\paragraph{Hierarchical Linguistic Structure.}
To systematically interpret the language capabilities of large models, we adopt a six level structure based on theoretical linguistics~\cite{fromkin2017introduction}: phonetics, phonology, morphology, syntax, semantics, and pragmatics.
The structure follows a logical progression from the external, physical realization of sound to the internal, contextual understanding of meaning. 
Each linguistic capability contains several concrete linguistic features, \textit{e.g.,} semantics level includes metaphor, simile, \textit{etc}.
We provide the exact definition for each linguistic capability in Appendix~\ref{app:ling}

% Our structure provides a comprehensive and modular way to explain how large language models achieve different levels of language ability. 
% By finding linguistic features at different levels in the SAE latent space of large models, we can more accurately reveal how these models represent and process natural language, thereby revealing the underlying mechanism of the models' language ability.
% This mechanism can also bring linguists a clearer understanding of how language knowledge is organized.

\paragraph{Dataset Construction.}
The sparse feature activation distribution of SAE is closely related to the conditions under which their corresponding linguistic features hold in linguistic knowledge.
To find the linguistic features and evaluate its dominance, we propose a method to construct the dataset and analyze feature activation frequencies.

For each linguistic feature, we first construct a set of sentences that significantly align with the desired feature. 
The feature activation representing this linguistic feature in SAE’s hidden space will be significantly activated on these sentences. 
However, this is not enough to accurately identify them, as there are some background noise vectors that are activated on all sentences in the dataset and interfere with our judgment. 
We need to include a control group without the feature in the constructed sentences. 

We introduce two types of control groups: minimal pairs and counterfactual sentences. Minimal pairs are constructed by changing only the part of a sentence that corresponds to a particular linguistic feature, while keeping all other parts unchanged. However, this approach often results in syntactically incorrect sentences.

To overcome this limitation, we also construct fully grammatically correct control groups, called counterfactual sentences, which differ from the original sentence only in terms of its linguistic features. Detailed dataset construction procedures are provided in Appendix~\ref{app:data_construction}.

\subsection{Feature Analysis}
We propose a causal probability approach to evaluate the relationship between extracted linguistic features and their activation on sentences containing those features. 

For a given feature \(x\), we define two key probabilities. The \emph{Probability of Necessity} (PN) quantifies how necessary the feature is for the activation of a corresponding base vector, while the \emph{Probability of Sufficiency} (PS) measures the likelihood that introducing the feature triggers activation. These probabilities are then combined into a \emph{Feature Representation Confidence} (FRC) score, which assesses both the representational capacity of the SAE latent space and the discriminative ability of the feature to identify the corresponding linguistic phenomenon. 

During feature analysis, we calculate the FRS on both the minimal contrast dataset and the counterfactual dataset, then average the results. This average more accurately reflects the ability of the base vectors to represent the linguistic features. Detailed definitions and calculation methods are provided in Appendix~\ref{app:frc}.



\subsection{Feature Intervention}
When we modify the values of SAE’s activation during forward propagation, we expect that such targeted interventions will influence the model’s behavior. 
However, our experiments show that altering only a small subset of features may not significantly impact the output—likely because linguistic phenomena are represented by multiple features across various layers. 
To assess the true impact of these interventions, we use a large language model as a judge. For each linguistic feature, we conduct both ablation and enhancement experiments. 
In the ablation experiment, we set the target feature’s activation to $0$, and in the enhancement experiment, we set it to $10$. 
In both cases, we also perform baseline experiments by randomly selecting 25 base vectors from the same layer.

For brevity, we denote the interventions as follows: let \(I_{abl}^{T}\) denote the targeted ablation intervention, \(I_{abl}^{B}\) the baseline ablation intervention, \(I_{enh}^{T}\) the targeted enhancement intervention, and \(I_{enh}^{B}\) the baseline enhancement intervention.

Let \(P_{abl}^{T}\) and \(P_{abl}^{B}\) denote the success probabilities (\textit{i.e.,} the probability that the intended change in the linguistic phenomenon is observed) for the targeted and baseline ablation experiments. The normalized ablation effect is then defined as
\[
\begin{aligned}
E_{abl} &= P_{abl}^{T} - P_{abl}^{B} \\
        &= \frac{P(Y=0 \mid I_{abl}^{T}) - P(Y=0 \mid I_{abl}^{B})}{P(Y=0 \mid I_{abl}^{T})}.
\end{aligned}
\]
Similarly, let \(P_{enh}^{T}\) and \(P_{enh}^{B}\) be the success probabilities for the targeted and baseline enhancement experiments, with \(Y=1\) indicating the presence of the phenomenon. The normalized enhancement effect is given by
\[
\begin{aligned}
E_{enh} &= P_{enh}^{T} - P_{enh}^{B} \\
        &= \frac{P(Y=1 \mid I_{enh}^{T}) - P(Y=1 \mid I_{enh}^{B})}{1 - P(Y=1 \mid I_{enh}^{B})}.
\end{aligned}
\]

Finally, we define the Feature Intervention Confidence (FIC) score as the harmonic mean of the normalized ablation and enhancement effects:
\[
\text{FIC} = \frac{2\, E_{abl}\, E_{enh}}{E_{abl} + E_{enh}}.
\]
When calculating FIC, if one or both of the $E$ values are negative, we incorporate a penalty coefficient $w$ to reflect the weakened or lost causality in such cases. 
This FIC score provides a balanced measure of how effectively targeted interventions, as opposed to random ones, influence the model’s output with respect to specific linguistic features.
The details for FIC are shown in Appendix~\ref{app:fic}.
% The detailed computation can be found in Appendix~\ref{app:fic}.
We now report on the results of our experimental evaluation. 
We use the GQA dataset~\cite{hudson2019gqa}, allowing us to build on the results of \cite{eiter2022neuro}, which uses the CLEVR~\cite{johnson2017clevr} synthetic data.  
Note that we use ground truth ASP representations of the images and queries.  
We examine our practical heuristic in four different ways.  
First, we examine the accuracy improvements when employing FAST-DAP. 
Second, we examine its data efficiency (e.g., how many examples in \textbf{EX} are required to provide useful results).  
Third, we examine the sensitivity of the support threshold for elements of $\Pi_D$. 
Finally, we examine running time.  
We created our implementation in Python~3.11.7 and use the Clingo solver for the ASP engine.
Experiments were run on an Apple~M2 machine with a~10-core CPU, and 32GB of RAM. 
All computations were carried out using only the CPU (the system’s GPU was not used).
We now present the results of each experiment.




\medskip
\noindent\textbf{Accuracy.}  We assess our approach's accuracy against the baseline (no $\Pi^D$), evaluating improvements with and without fallback rules (FBR and No FBR), both utilizing FAST-DAP. For the baseline (no FAST-DAP), the ASP solver either provides an answer or returns ``empty'' if it cannot deduce one. 
On our test set (disjoint from the examples), the baseline accuracy across all question types was~$59.98\%$ without domain information. Incorporating domain information learned from the training set significantly boosted accuracy to~$80.62\%$ without fallback rules, and $81.01\%$ with them.
To gain deeper insights, we analyze specific question types, a subset of which is presented in Table~\ref{tab:acc-indv}.
Some types, such as verification questions, show minimal dependence on domain categorization, while others rely more heavily on it. Additionally, certain questions require translating specific concepts into general terms (FAST-DAP, lines~\ref{a:gtsbeg}-\ref{a:gtsend}), like generalizing ``banana'' to ``fruit'' or ``juice'' to ``drink.'' In Table~\ref{tab:acc-indv}, all non-choice queries require such generalization.







\begin{table}[thb!]
    \setlength{\tabcolsep}{12pt}
    \centering
    \begin{tabular}{ m{3cm} m{2.5cm} m{2.5cm} m{2.5cm} }
        \hline
        \textbf{Question Type} & \textbf{Baseline} & \textbf{FBR (Ours)} & \textbf{No FBR (Ours)} \\
        \hline
        choose\_activity   & 69.02 & 95.11 & 94.84 \\
        choose\_color   & 89.80 & 93.48 & 93.21 \\
        choose\_older   & 0 & 97.24 & 97.24 \\
        choose\_rel   & 73.88 & 85.48 & 81.72 \\
        choose\_vposition   & 96.27 & 94.98 & 94.93 \\
        \hline
        and  & 94.25 & 91.93 & 91.83 \\
        verify\_age   & 86.89 & 97.54 & 97.54 \\
        verify\_color   & 95.71 & 96.58 & 96.44 \\
        verify\_location   & 49.28 & 94.5 & 94.5 \\
        query   & 36.07 & 72.83 & 72.20 \\
        \hline
    \end{tabular}
    \caption{Evaluation of answering questions. The ``Baseline'' column shows accuracy (in percentage) without learned domains, ``FBR'' shows accuracy with learned domains and fallback rules, and ``No FBR'' shows accuracy with domain atoms but without using fallback rules.}
    \label{tab:acc-indv}
\end{table}











\medskip
\noindent\textbf{Data Efficiency.} In this second experiment, we aimed to find the optimal sample size for learning domains. We randomly divided the data as follows: 20\% for training, 10\% for validation, and the remaining 70\% for testing. Instead of using the entire training set at once, we divided it into~11 progressively larger subsets as follows:
the first subset served as a baseline model with no samples, the second subset contained 10\% of the training data, the third subset included the first 10\% plus an additional 10\%, making up 20\% of the training data, and this pattern continued until the 11th subset, which encompassed all the training data. 
Each training subset was used independently to learn the domains, and these learned domains were then used to predict the answers in the test set. Figure~\ref{fig:trainsubsets} illustrates the results, showing accuracy across the training data for two scenarios: the black line represents the learned domain without fallback rules, while the red line includes fallback rules. 
As depicted in Figure~\ref{fig:trainsubsets}, using just 10\% of the training set (equivalent to 2\% of the entire dataset) achieves a respectable accuracy of 78.93\%. 
With 20\% of the training data (4\% of the entire dataset), accuracy exceeds 80\%. 
This suggests that a small amount of data can effectively learn domains, with only slight accuracy gains from adding more data.

 


\begin{figure}[t]
    \centering
    \begin{subfigure}[t]{0.45\textwidth}
        \centering
        \includegraphics[width=\textwidth]{images/trainsubsetAccuracy.png}
        \caption{Accuracy on the test set leveraging learned domains from different training subsets. 
        }
        \label{fig:trainsubsets}
    \end{subfigure}
    \hfill
    \begin{subfigure}[t]{0.45\textwidth}
        \centering
        \includegraphics[width=\textwidth]{images/time.png}
        \caption{Execution time of our algorithm for different sample sizes, run in parallel with identical settings.
        }
        \label{fig:time}
    \end{subfigure}
    \caption{Accuracy and running time on different training subsets.}
    \label{fig:acctime}
\end{figure}




\medskip
\noindent\textbf{Threshold Sensitivity.}  FAST-DAP refines the learned domain set by removing domains whose support falls below a specified threshold. This approach helps regularize the outcome since the domains were derived from the application of possibly noisy data and rules. The threshold is a hyper-parameter determined from the validation set. 
We used the $10^{th}$ to $70^{th}$ percentile support values as potential thresholds. For each, we removed domains with lower support, assessed validation accuracy, and selected the threshold with the highest accuracy. Domains below this final threshold were then removed.
Table \ref{tab:hs} illustrates the accuracy achieved at different thresholds. Based on this data, we selected a threshold of $59.5$, and domains with support below this value were excluded to form the final set of domains.






\begin{table}[t]
    \begin{subtable}[t]{0.45\textwidth}
        \centering
        \begin{tabular}{ m{1.5cm} m{1.5cm} m{1.5cm} }
            \hline
            \textbf{Percentile} & \textbf{Threshold} & \textbf{Accuracy} \\
            \hline
            10   & 12.3 & 79.44 \\
            20   & 20.6 & 79.79\\
            30   & 30.9 & 79.89\\
            40   & 46.4 & 80.10\\
            \hline
        \end{tabular}
    \end{subtable}
    \begin{subtable}[t]{0.45\textwidth}
        \centering
        \begin{tabular}{ m{1.5cm} m{1.5cm} m{1.5cm} }
            \hline
            \textbf{Percentile} & \textbf{Threshold} & \textbf{Accuracy} \\
            \hline
            50   & \textbf{59.5} & \textbf{80.54}\\
            60   & 90.8 & 80.02\\
            70   & 121.2 & 79.75\\
                 &      &      \\
            \hline
        \end{tabular}
    \end{subtable}
    \caption{Accuracy results on the validation set after removing domains with support below a threshold.
    }
    \label{tab:hs}
\end{table}



\medskip
\noindent\textbf{Running Time.}  
The running time of our algorithm is primarily influenced by the performance of the ASP solver Clingo, and is directly proportional to the number of atoms it processes. 
Figure~\ref{fig:time} illustrates that the running time grows consistently from the base case with no training samples to the scenario where all training samples are used. 
Incorporating more training samples to learn domains substantially boosts the number of learned new domain atoms, thereby requiring Clingo to process more atoms during deduction. This necessity is the main factor driving the increase in running time. However, note that this increase is bounded by a constant factor related to the domain's size.





\section{Discussion}
\label{sec:discussion}


\begin{figure*}[t!]
\centering
\includegraphics[width=0.85\textwidth]{images/discussion_eggshell_user+app_combined.pdf}
\caption{The top shows the status quo of handoff between the user and \bma. The bottom illustrates our proposed simplified interaction.}
\Description{The top figure illustrates the current state of interactions between the user and Be My AI, where the user submits a photo and Be My AI returns a response in the first time of interaction. Subsequently, the user poses a follow-up question to Be My AI and Be My AI returns a more detailed description in the second time of interaction. The bottom figure is our proposed simplified interaction, where Be My AI learns from previous interactions and returns a detailed description during the first interaction, without user asking follow-up question.}
\label{fig.discussion_eggshell_user+app}
\end{figure*}




In this section, we examine the current state of handoff between users, \bma, and remote sighted assistants, and propose new paradigms to address the challenges identified in our findings. 
Next, we explore how multi-agent systems, both human-human and human-AI interactions, assist visually impaired users, and envision the transition toward AI-AI collaborations for tasks requiring specialized knowledge. Finally, we discuss the potential advantages of real-time video processing in the next generation of AI-powered VQA systems.






\subsection{Handoff Between Users, \bma, and Remote Sighted Assistants}
\label{handoff}

In this study, we illustrated the advantages of the latest LMM-based VQA system in (i) enhancing spatial awareness through detailed scene descriptions of objects' colors, sizes, shapes, and spatial orientations
(Section~\ref{physical_environments}), (ii) enriching users' understanding in social and stylistic contexts by detailing emotional states of people or animals, their body language, ambiance
(Section~\ref{social_stylistic}), and identity recognition (Section~\ref{identity}), 
and (iii) facilitating navigation by interpreting signages (Section~\ref{realtime_feedback}). 


Informed by our findings, despite these various benefits, there are challenges that the system alone cannot overcome.  
\bma{} still requires human intervention, either from the blind user or the remote sighted assistant (RSA), to guide or validate its outputs.  
% 
Users seek confirmation from RSAs or depend on their own spatial memory to overcome AI hallucinations, where \sbma{} inaccurately adds non-existent details to scenes. Users also rely on auditory cues and spatial memory to locate dropped objects and direct the system toward the intended search areas (Section~\ref{physical_environments}).
% 
Moreover, users actively prompt the system to understand their specific objectives, such as checking for eggshells in a frying pan or adjusting appliance dials (Section~\ref{agentic_interaction}). 
% 
There are also instances where users require assistance from RSAs when the system fails to provide adequate support to fulfill users' objectives, such as identifying the centerpiece of puzzles or adjusting the camera angle (Section~\ref{consistency}).
% 
Human assistance or users' O\&M skills are necessary to receive real-time feedback for safe and smooth navigation (Section~\ref{realtime_feedback}). 



Furthermore, our findings revealed that the system might produce inaccurate or controversial interpretations. Users express skepticism towards \sbma's subjective interpretations of animals' emotions and fashion suggestions (Section~\ref{social_stylistic}), and have encountered inaccuracies in \sbma's identification of people's gender and age (Section~\ref{identity}). These instances underline potential areas where human judgment is necessary to corroborate or correct the system's descriptions.





Next, we discuss the handoff~\cite{mulligan2020concept} between users, \bma, and RSAs to mitigate the aforementioned challenges. 




% \begin{figure}[]
% \centering
% \includegraphics[width=0.95\textwidth]{images/discussion_eggshell_user+app.png}
% \caption{Status quo of handoff between the user and \bma.}
% \label{fig.discussion_eggshell_user+app}
% \end{figure}


% \begin{figure}[]
% \centering
% \includegraphics[width=0.95\textwidth]{images/discussion_eggshell_user+app-single-interaction.png}
% \caption{Status quo of handoff between the user and \bma.}
% \label{fig.discussion_eggshell_user+app}
% \end{figure}






%%%%%%%%%%%%%
\subsubsection{Status Quo of Interactions Between Users and \bma}


Through BMA's ``ask more'' function, users are able to request additional details about the images that were not covered in initial descriptions. 
% 
This functionality facilitates a shift in interaction dynamics between users and \bma, even if the system may not accurately understand or answer users' questions in the first attempt. 
In these interactions, users are not merely passive recipients of AI-generated outputs, they actively guide the AI tool with specific prompts to better align AI's responses with their objectives. 


Our findings reported one instance where the AI tool fails to grasp the user's intent to check the presence for eggshells in the beginning (Figure~\ref{fig.discussion_eggshell_user+app} top). First, the user submits an image of eggs in the pan. Respond to the image, the system describes the quantity and object (``Inside the frying pan, there are three eggs'') and states of the yolks and whites (``whites separated'', ``yolk has broken'', ``mixing with the egg white''). Next, the user clarifies her inquiry by asking, ``are there any shells in my eggs?''
This prompts the system to understand the user's goal, reevaluate the image, subsequently confirming the presence (``Yes, there is a small piece of eggshell in the frying pan'') and location of an eggshell to help her remove it (``near the bottom left of the broken egg yolk''). 


This interaction exemplifies the status quo of handoff, where the user and \bma{} engage in a back-and-forth dialogue to refine the descriptions based on the ``ask more'' function and the user's precise prompts. 
% 
\rev{While this iterative process allows the system to eventually understand the users' intent without RSAs' intervention, it places cognitive burden on users who must carefully craft and iteratively refine their prompts. The cognitive load increases as users mentally track what information they've already received, analyze gaps between their needs and the system's responses, and develop increasingly specific queries. 
}






\rev{To reduce users' cognitive load, we propose enabling the system to adopt a mechanism that combines multi-source data input with long-term and short-term memory capabilities~\cite{zhong2024memorybank}. With explicit user consent, future versions of LMM-based VQA systems could integrate data from users' mobile devices (e.g., location information, time data) alongside historical interaction data within the system (e.g., contexts and follow-up questions) to recognize user preferences and common inquiries, and infer their needs, thereby generating responses more effectively in similar contexts. Long-term memory serves as a repository for capturing generalized user preferences, behavior patterns, and aggregated insights across multiple users. This long-term memory is particularly effective for improving system intelligence by identifying common user needs and optimizing general responses~\cite{priyadarshini2023human}. Meanwhile, short-term memory can focus on task-specific optimization within a single interaction session. It retains context from the immediate conversation, such as recent user inputs and system responses, to enhance relevance and coherence in real time. Short-term memory operates dynamically, clearing retained data once the session ends or the task is completed, thereby ensuring privacy and preventing unnecessary data retention.}


% However, this interaction could be further simplified by training \bma{} to learn from previous dialogues. By recognizing user preferences and common inquiries, such as identifying unusual elements in the scenario (eggshells in cooking eggs), we envision that the system could proactively address user concerns more efficiently and thus reduce the need for multiple clarifying prompts (Figure~\ref{fig.discussion_eggshell_user+app} bottom). 
\rev{For example, when identifying unusual elements during cooking (e.g., eggshells in cooking eggs), \bma{} could utilize the user's immediate input while referencing short-term memory from the current session or recent similar interactions. Additionally, by leveraging long-term memory, the system can learn from the user's past questions and query patterns to better match their habits and preferences, i.e., user's typical needs. Furthermore, multi-source data input, such as time or location information, can assist the system in inferring the user's current context, for instance, recognizing that the user is preparing a specific meal at a particular time or place, which allows the system to provide more relevant and context-aware assistance. This approach enables \bma{} to proactively anticipate user intent and deliver targeted responses, reducing the need for multiple clarifying prompts (Figure~\ref{fig.discussion_eggshell_user+app} bottom).} 

\rev{Cognitive Load Theory (CLT) suggests that well-designed interactions can significantly reduce users' extraneous load while enhancing the effective management of germane load~\cite{sweller1988cognitive, chandler1991cognitive}. Following the principles of CLT, we recommend using the above design to enable LMM-based VQA systems to minimize unnecessary clarifying prompts, thereby reducing users' cognitive load. 

}






\begin{figure*}[h!]
\centering
\includegraphics[width=0.8\textwidth]{images/discussion_eggshell_user+app+rsa.png}
\caption{Handoff between the user, \bma, and RSA for identity interpretations.}
\Description{The user submits a photo of people to Be My AI. Be My AI recognizes the requirement for identity interpretations and direct the photo to a remote sighted assistant. The remote sighted assistant returns a description of people's identities to Be My AI. Be My AI learns from the assistant's responses.}
\label{fig.discussion_eggshell_user+app+rsa}
\end{figure*}



%%%%%%%%%%%%%

\subsubsection{AI Deferral Learning for Identity Interpretations} 
\label{sec:deferral_learning}
% 


Our findings elucidated \sbma's capabilities and limitations in interpreting identity attributes. Although the system can describe aspects like gender, age, appearance, and ethnicity, it may make errors due to its reliance on stereotypical indicators and its inability to interpret non-visible details (Section~\ref{identity}).
% 
However, Stangl et al.'s work~\cite{stangl2020person} pointed out that PVI \rev{seek identity interpretations from AI assistants} across various contents, including \rev{browsing} social networking sites where our participants reported using \bma.


% human's ability 
This reveals a tension between PVI's interests in knowing about \rev{identity} attributes and the AI's challenges in providing reliable information\rev{~\cite{hanley2021computer}}. The conflict arises because attributes such as age and gender are not purely perceptual and cannot be accurately identified by visual cues alone. \rev{However}, RSAs \rev{can draw on contextual clues, past interactions, and cultural knowledge to make more nuanced observations about these human traits.} These social strategies are not typically accessible to AI systems. 
% \rev{This suggests a need for human-AI collaboration: certain tasks are best handled by AI assistants, others by human assistants, and still others through coordinated effort between both~\cite{gonzalez2024investigating}.}


% To mitigate these issues, we consider the potential benefits of AI \rev{deferral} learning that involves handoff between the user, \bma, and RSA. The process is illustrated in Figure~\ref{fig.discussion_eggshell_user+app+rsa}.

% To facilitate effective handoff between AI and human assistants,
To mitigate these issues, \rev{we propose adopting a deferral learning architecture~\cite{mozannar2020consistent, raghu2019algorithmic}, where an AI model learns when to defer decisions to humans and when to make decisions by itself. As detailed by Han et al.~\cite{han2024uncovering} and illustrated in Figure \ref{fig.discussion_eggshell_user+app+rsa}, this architecture creates a three-stage information flow:


\begin{itemize}
    \item \textbf{Stage 1}: It begins when users submit image-based queries to \bma. At this stage, the system uses a detection mechanism to identify sensitive contents, focusing particularly on those involving human physical traits. Current large-language models have already incorporated such mechanisms~\cite{perez2022red, bai2022training}; however, they still struggle to interpret human identity with consistent accuracy~\cite{hanley2021computer}.    
    
    % 
    \item \textbf{Stage 2}: Rather than declining sensitive requests outright, \bma{} redirects these queries to RSAs. This maintains the system's helpfulness while ensuring accurate responses. 
    % 
    \item \textbf{Stage 3}: RSAs provide descriptions by leveraging contextual understanding, such as analyzing the users' current environment and cultural background.
    % Additionally, RSAs employ a variety of social strategies by integrating contextual information, historical interactions, and cultural insights to deduce user attributes and preferences, ensuring that responses are not only relevant but also culturally and contextually appropriate. 
    % RSAs provide detailed descriptions by leveraging contextual understanding and social awareness.    
    %\textcolor{red}{RSAs can employ a variety of social strategies to deduce these attributes by leveraging context, prior interactions, and cultural insights.} 
\end{itemize}




In contrast to prior work that addresses stereotypical identity interpretation through purely computational approaches~\cite{wang2019balanced,wang2020towards,ramaswamy2021fair}, our proposed AI deferral learning takes a hybrid human-AI approach that combines AI capabilities with human expertise. 
% 
While previous AI-only solutions have made progress in reducing bias, they still struggle with identity interpretation~\cite{hanley2021computer}.
The challenges arise not only from technical issues but also from the ontological and epistemological limitations of social categories (e.g., the inherent instability of identity categories), as well as from social context and salience (e.g., describing a photograph of the Kennedy assassination merely as ``ten people, car'').
%
Our system leverages RSAs who have got more experience and probably more success in identifying people's identity through their human perception abilities and real-world experience. 
RSAs can interpret subtle contextual cues, understand cultural nuances, and adapt to diverse presentation styles that may challenge AI systems. 
% For example, RSAs can understand cultural markers of identity, and perceive age and ethnicity across different cultural contexts. 
Through the AI deferral learning architecture, AI assistants can learn continuously from human assistants' responses and improve its ability to handle similar situations. The three-way interactions between users, AI assistants, and RSAs generate rich contextual data that can enhance the AI system's identity detection mechanisms.} 

% Such process could also be adapted for requests involving the interpretation of social contexts \rev{like assessing animals' emotions (Section~\ref{social_stylistic})}. It can facilitate \bma{} continuously improve through observation and learning from human expertise. 

% When equipped with memory capabilities and given user consent, the AI assistant can build a knowledge base of individuals whom users frequently encounter. Moreover, the three-way interactions between users, AI assistants, and RSAs generate rich contextual data that can enhance the AI system's identity detection mechanisms.



% compared to prior works attempted to address the issues of stereotypical identity interpretation, how does the proposed AI referral learning could perform better?
% \textcolor{red}{[add prior work on identity interpretation (e.g., CV)]}

% \textcolor{red}{referral, when handoff to human,  challenges of ai referral learning is how to get the dataset. here bma get the dataset for training from rsa.}


% \rev{
% In contrast to prior work that addresses stereotypical identity interpretation through purely computational approaches~\cite{wang2019balanced,wang2020towards,ramaswamy2021fair}, our proposed AI referral learning takes a hybrid human-AI approach that combines AI capabilities with human expertise. 
% % 
% While previous AI-only solutions have made progress in reducing bias, they still struggle with identity interpretation~\cite{hanley2021computer}.
% The challenges arise not only from technical issues but also from the ontological and epistemological limitations of social categories (e.g., the inherent instability of identity categories), as well as from social context and salience (e.g., describing a photograph of the Kennedy assassination merely as ``ten people, car'').
% %
% Our system leverages RSAs who have got more experience and probably more success in identifying people's identity through their human perception abilities and real-world experience. 
% RSAs can interpret subtle contextual cues, understand cultural nuances, and adapt to diverse presentation styles that may challenge AI systems. For example, RSAs can understand cultural markers of identity, and perceive age and ethnicity across different cultural contexts. 
% Through the AI referral learning framework, \bma{} can learn from RSAs' nuanced interpretations in these scenarios, gradually improving its ability to handle similar situations. The system logs RSAs' responses and uses them as training examples, allowing \bma{} to develop more context aware in identity recognition. 
% }
% % 
% Such referral learning processes could also be adapted for requests involving the interpretation of social contexts. It can facilitate \bma{} continuously improve through observation and learning from human expertise. 


\begin{figure*}[t!]
\centering
\includegraphics[width=0.8\textwidth]{images/discussion_eggshell_user+app+rsa-check.png}
\caption{Handoff between the user, \bma, and RSA for fact-checking.}
\Description{The user submit a photo to Be My AI, Be My AI returns a description with possible AI hallucination. Then, the user directs the photo to a remote sighted assistant, who returns a corrected description to the user.}
\label{fig.discussion_eggshell_user+app+rsa-check}
\end{figure*}


%%%%%%%%%%%%%
\subsubsection{Fact-Checking for AI Hallucination Problem}
% ai hallucinations - informed by our findings, cannot be addressed, ask humans 



Our findings highlighted that the AI-generated detailed descriptions helped users understand their physical surroundings. However, there were instances where AI systems hallucinated, i.e. incorrectly added non-existent details to the descriptions, which led to confusion (Section~\ref{physical_environments}). 
% 
\rev{
In fact, hallucinations is a known problem for large language models upon which \bma{} is built~\cite{gonzalez2024investigating}. Current approaches to address this problem include Chain-of-thought (CoT) prompting~\cite{wei2022chain}, self-consistency~\cite{wang2022self}, and retrieval-augmented generation (RAG)~\cite{lewis2020retrieval}.
% 

In CoT prompting~\cite{wei2022chain}, users ask an AI model to show its reasoning steps, like solving a math problem step by step rather than simply giving the final answer. It is similar to the ``think aloud'' protocol in HCI. 
% 
Self-consistency~\cite{wang2022self} is an extension of CoT prompting. Instead of generating just one chain of thought, the model is asked to generate multiple different reasoning paths for the same task. Each reasoning path might arrive at a different answer. The model then takes a ``majority vote'' among these different answers to determine the final response.
% 
In RAG~\cite{lewis2020retrieval}, AI models are provided with relevant information retrieved from a vector storage as ``context'' to reduce factual errors in their responses.
}

\rev{
In light of these techniques, users adopt various strategies that mirror CoT prompting, self-consistency, and RAG. We outline some potential strategies below.
% with a visual (Figure~\ref{fig.discussion_eggshell_user+app+rsa-check}).
\begin{itemize}
    \item \textbf{Part-Whole Prompting}: This strategy parallels the Chain of Thought (CoT) prompting. A user sends an image to \bma{} and requests an initial overall description, followed by a systematic breakdown that justifies this description. For example, users might first ask for a description of the image as a ``whole'', then request to divide the image into smaller ``parts'', like a $3 \times 3$ grid, and describe each grid individually. If the descriptions of the individual parts align coherently, it increases the likelihood that the overall description is accurate. This approach would require processing more information; however, it will provide users with greater confidence in the AI's response, as it enables them to verify consistency between the whole and its constituent parts.  
    % 

    \item \textbf{Prompting from Multiple Perspectives}: This strategy resembles the self-consistency technique. A user sends an image to \bma{} and requests multiple descriptions from different perspectives. For example, users might ask for one description that focuses on the background and another that emphasizes foreground objects. Users can also request descriptions from the viewpoint of objects within the image (e.g., ``How would a person sitting on a chair see this scene?'' and ``How would a person sitting on the floor see this scene?''). While gathering descriptions from multiple perspectives may increase the likelihood of hallucination, it can also help identify common elements that appear consistently across different viewpoints, potentially indicating true features of the image.    
     % 
    \item \textbf{Prompting with Human Knowledge}: This strategy resembles the RAG approach. A user sends an image to \bma{}, provides their current understanding of the image and its context, and requests a description that complements their knowledge. For example, in Figure~\ref{eggshells}, users can specify that someone took the picture in a kitchen environment and that it should show a frying pan containing eggs. Users possess this knowledge through their familiarity with physical environments, self-exploration, spatial memory, and touch~\cite{gonzalez2024investigating}. The user-provided knowledge will help the AI model ground its response in an accurate context~\cite{liu2024coquest}.     
    % 
    \item \textbf{Pairing with Remote Human Assistants}: While the previous three strategies rely on multiple prompting and response aggregation to identify facts, this approach leverages the traditional remote sighted assistance framework.
    This strategy (shown in Figure~\ref{fig.discussion_eggshell_user+app+rsa-check}) differs from the deferral learning framework (Section~\ref{sec:deferral_learning}) in that users forward the AI responses to human assistants, rather than the AI assistant deferring to humans for the response.
    In this strategy, a user first sends an image to \bma{} to receive a description. When users suspect inaccuracies through triangulation~\cite{gonzalez2024investigating}, such as descriptions that conflict with their spatial memory or common sense (e.g., implausible objects like a palm tree in a cold region), they can request a RSA to fact-check the description. The RSA then verifies the description and sends corrected information back to the user. This verification process is likely easier and faster for a RSA than composing a description from scratch, as the RSA's work involves checking rather than creating content.
\end{itemize}
}

\rev{
In summary, AI hallucination presents both challenges and opportunities. By addressing these issues, future work will strengthen the way users, AI models, and human assistants interact with each other.
}















%%%%%%%%%%%%%
%%%%%%%%%%%%%
\subsection{Towards Multi-Agent Systems for Assisting Visually Impaired Users}


%%%%%%%%%%%%
This section examines the transition from human-human interactions to human-AI and AI-AI systems in supporting PVI. We explore how these multi-agent systems, which involve the collaborative efforts of multiple agents (AI or human), are designed to adaptively meet the diverse needs of PVI. 


Lee et al.~\cite{lee2020emerging} identified four contexts in which a professional human-assisted VQA system (Aira) offer support to PVI. The type of information required by PVI is incremental in these contexts. 
First, \textit{scene description} and \textit{object identification} acquire information about ``what is it.''
Second, \textit{navigation} requires description about PVI's surroundings and obstacles (``what is it'') and directional information (``where is it'' and ``how to get to the destination'').
Third, \textit{task performance} like putting on lipstick, cooking, and teaching a class. This context requires description (``what is it'') and domain knowledge on ``how to do it.'' 
% directional information (``where is the intended object'' and ``how to get it''), as well as
Forth, \textit{social engagement} like helping PVI in public spaces or interacting with other people. This needs description (``what is it''), directional information to navigate in social space, and discreet communication (PVI prefer not to disclose their use of VQA systems).


Our study reported how participants used \bma{} for tasks like matching outfits and assessing makeup, fitting under the category of \textit{task performance}. Some participants raised concerns about the accuracy of \sbma's interpretations and suggestions, indicating their preference for human subjectivity in this context. 
% 
Contrasting with this, Lee et al.'s work~\cite{lee2020emerging} highlighted that remote sighted assistants (RSAs), even those professionals RSAs from Aira, sometimes lack the specialized information or domain knowledge required in task performance, thereby they need to collaborate with other RSAs to find solutions. 


Furthering this investigation, Xie et al.~\cite{xie2023two} paired two RSAs to assist one visually impaired user in synchronous sessions, validating the need for RSAs to complement each other's description in task performance like aiding the user in applying makeup and matching outfits. They also explored the challenges in this human-human collaboration, revealing collaboration breakdowns between two opinionated RSAs. 
% 
To address these issues, they proposed a collaboration modality in which one ``silent'' RSA supports the other RSA by researching but not directly communicating. This approach suggested that two RSAs in this multi-agent system should not deliver information simultaneously but have a clear division of labor, designating who takes the lead, to avoid overwhelming PVI with information. 


% AI-RSA handoff (section 5.1) to AI-AI (GPT + expert AI)
Transitioning from human-human to human-AI collaboration, the handoff between the user, \bma{} and RSA (Section~\ref{handoff}) opens up new opportunities for multi-agent systems. 
% 
Our proposed modality of human-AI collaboration integrates the scalable, on-demand capabilities of AI-based visual assistance with the contextual understanding and adaptability of RSAs. 
This multi-agent system involves the AI system recognizing its own limitations and seamlessly handing off tasks to a RSA when appropriate. This collaboration aligns with prior work~\cite{xie2023two}, where AI (\bma) and human (RSA) maintain a clear division of labor, minimizing cumbersome back-and-forth and reducing potential confusion for PVI. 



Looking ahead, we envision the potential for AI-AI collaboration as part of the future multi-agent systems to assist PVI, especially for task performance. 
A domain-specific AI expert can be trained to handle more specialized tasks such as matching outfits, performing mathematical computations, or answering chemistry-related questions. \bma, as the core AI system, can provide general visual descriptions (``what is it'') and delegate more specialized tasks requiring domain knowledge to the domain-specific AI expert. 
% 
This approach is in line with the human-AI collaboration (Section~\ref{handoff}) by ensuring effective handoffs when necessary. By leveraging AI agents with more specialized capabilities, this multi-agent system can better adapt to PVI's needs. 



However, similar to concerns around human-human and human-AI interactions, these AI-AI collaborations must be carefully designed with clear protocols and handoff points for transitioning tasks between AI agents. It is important to make these transitions as seamless and transparent as possible to PVI, thereby avoiding any complexity or confusion. 




%%%%%%%%%%%
%%%%%%%%%%%
\subsection{Towards Real-Time Video Processing in LMM-based VQA Systems}


One of the most significant advantages of \bma{} and other LMM-based assistive tools is their ability to provide contextually relevant and personalized assistance to users. By leveraging machine learning and natural language understanding, these systems can understand and respond to a wide range of user queries. This level of contextual awareness represents a significant advancement over pre-LMM-based assistive technologies, which often fail to adapt to the diverse needs and preferences of individual users.


However, our findings also identified several challenges and limitations associated with the reliance on static images by current LMM-based assistive tools. 
Participants in our study reported frustration with the need to take multiple pictures to capture the desired information, a process they found time-consuming and cognitively demanding (Section~\ref{realtime_feedback}). This iterative process hinders efficiency and also poses safety risks, as participants struggled with taking images while navigating around obstacles.


% \rev{[real-time video process can also facilitate the transition from conveying ``what'' to ``how'' questions]}
To mitigate these issues, integrating real-time video processing capabilities into future LMM-based VQA systems could offer significant benefits. 
% 
Our findings suggest that the dynamic nature of video serves as a foundation for subsequent guidance (Section~\ref{realtime_feedback}), which is currently provided by human assistants through video-based remote sighted assistance.
% 
Shifting to real-time video processing would allow LMM-based VQA systems to transition from identifying objects (answering ``what is it'') to offering practical advice (addressing ``how to do it''), such as how to adjust the camera angle or how to navigate to a destination.
% 
By continuously analyzing the user's surroundings through real-time video feeds, these systems can dynamically interpret changes and provide immediate feedback, thus eliminating the need for static image captures. This capability would enhance the user experience by offering seamless navigation aid in real time. 


The feasibility of real-time video processing is supported by existing technologies demonstrated in commercial products and research prototypes. For instance, systems that utilize sophisticated algorithms for real-time object segmentation in video streams~\cite{wang2021swiftnet} have shown significant potential in other domains. Building on these techniques for video analysis could significantly extend the capabilities of future LMM-based VQA systems. 


Transitioning from static image analysis to real-time video processing can alleviate the burden of iteratively taking pictures and adjusting angles experienced by users. It can also enhance the utility and safety of LMM-based VQA systems, particularly during navigation. 
% 
This progression, driven by ongoing advancements in machine learning and computer vision, is essential for the development of more adaptive and responsive assistive technologies that align with the dynamic nature of real-world environments.




% Our findings highlight a limitation in the capabilities of \bma{} when it comes to providing actionable, goal-oriented guidance to PVI (Section~\ref{appliances}). While \bma{} is good at conveying ``what'' information with most of time accurately describing the visual content of a scene or object, such as identifying a thermostat on the wall. 
% 
% It often struggles with providing ``how'' information, guiding users on the specific actions required to interact with or operate elements in their environment, such as how to adjust the theromstat. This ``what'' and ``how'' divide poses a major challenge to the effectiveness and usability of \bma{}, as PVI rely on it not only for understanding their surroundings but also for completing tasks and achieving their goals.



% To address this limitation and design more goal-oriented AI-powered assistant prosthetics, we propose the following key design implications. 
% % 
% Future AI-powered assistive technologies should be designed with a focus on action-oriented reasoning and task-specific guidance. Although this could be achieved through further user inquiries~\cite{truhn2023large}, integrating knowledge bases~\cite{zhu2014reasoning} and event/behavior reasoning engines~\cite{chen2008using} to enable contextual inference of actions and intentions, and associating visual elements' feedback with reasoning, would greatly reduce the cognitive burden on PVI and enhance the user experience. By leveraging this knowledge, assistive technologies can provide more relevant and actionable guidance to PVI, helping them effectively navigate and interact with their environment to complete desired tasks. Our findings emphasize the importance of human-centered design principles, particularly in the design of assistive technologies, which should be reinforced through a goal-oriented technical roadmap that adapts to users' needs, preferences, and external environments~\cite{fischer2001user,amershi2014power}. By emphasizing action goal-oriented reasoning~\cite{huffman1993goal,letier2002agent}, future AI-powered prosthetics will be optimized, further benefiting PVI.






% reference about real-time image processing:
% By continuously analyzing the user's surroundings and providing relevant information without the need for explicit image capture, these systems could offer a more seamless and efficient user experience. The integration of \textcolor{red}{real-time image processing} into AI-powered VQA systems aligns with the growing availability of commercial products and research prototypes that leverage advanced object detection and text recognition technologies. For example, Microsoft's Seeing AI \cite{SeeingAI2020} and various currency recognition systems \cite{liu2008camera, parlouar2009assistive, paisios2012exchanging} demonstrate the feasibility and potential impact of real-time image processing in assistive technology. By building upon these existing approaches and incorporating state-of-the-art deep learning techniques for object detection \cite{girshick2014rich, girshick2015fast, ren2016faster, krizhevsky2017imagenet} and text recognition \cite{ma2018arbitrary, he2017deep, zhou2017east, yao2016scene, liu2017deep, lyu2018multi}, future AI-powered VQA systems could provide even more robust and reliable assistance to users.







% \subsection{Subjectivity Interpretations}
% % positive, identity
% know gender, race, humans have problem in identifying it, humans also guess but not make public mistake. not purely perceptual. discussion: neutral, conflict. tension between prior work and this one. only objective info and pvi will use social skills. intent in physical social gathering - limitations of prior work, digital interactions
% % https://dl.acm.org/doi/10.1145/3313831.3376404



% % negative, social cues
% [Discussion:] ai, human agency, undermine pvi as people, they can understand the social meanings. compare to human agents, subjectivity comes from ai no verification 








% \subsection{Design Implications for AI-Powered Assistant Prosthetics} %Albert
% % Design Implications: How AI-Powered Visual Question Answering Should Look Like
% As demonstrated by our research, \bma{} has shown significant potential in empowering PVI by providing a more intuitive, user-friendly, and context-aware assistive experience. However, to further enhance the usability and effectiveness of AI-powered VQA systems like \bma{}, several design implications should be considered.
% %lmm: large multi model 
% %
% %Suggest features that could improve \bma's usability, like real-time image processing to reduce the need for multiple pictures.
% %
% %Recommend design changes that might help \bma become more goal-oriented and contextually aware, such as advanced machine learning models trained on diverse environments.



% \subsubsection{Towards Goal-Oriented AI-Powered Assistant Prosthetics}

% %bma good at what information (``what'' is about the visual content) but not ``how'' information
% %
% %(Section~\ref{appliances}) recognize thermostat (what), but not adjust dial (how)

% Our findings highlight a limitation in the capabilities of \bma{} when it comes to providing actionable, goal-oriented guidance to PVI (Section~\ref{appliances}). While \bma{} is good at conveying ``what'' information with most of time accurately describing the visual content of a scene or object, such as identifying a thermostat on the wall. 
% % 
% It often struggles with providing ``how'' information, guiding users on the specific actions required to interact with or operate elements in their environment, such as how to adjust the theromstat. This ``what'' and ``how'' divide poses a major challenge to the effectiveness and usability of \bma{}, as PVI rely on it not only for understanding their surroundings but also for completing tasks and achieving their goals.

% To address this limitation and design more goal-oriented AI-powered assistant prosthetics, we propose the following key design implications. 
% % 
% Future AI-powered assistive technologies should be designed with a focus on action-oriented reasoning and task-specific guidance. Although this could be achieved through further user inquiries~\cite{truhn2023large}, integrating knowledge bases~\cite{zhu2014reasoning} and event/behavior reasoning engines~\cite{chen2008using} to enable contextual inference of actions and intentions, and associating visual elements' feedback with reasoning, would greatly reduce the cognitive burden on PVI and enhance the user experience. By leveraging this knowledge, assistive technologies can provide more relevant and actionable guidance to PVI, helping them effectively navigate and interact with their environment to complete desired tasks. Our findings emphasize the importance of human-centered design principles, particularly in the design of assistive technologies, which should be reinforced through a goal-oriented technical roadmap that adapts to users' needs, preferences, and external environments~\cite{fischer2001user,amershi2014power}. By emphasizing action goal-oriented reasoning~\cite{huffman1993goal,letier2002agent}, future AI-powered prosthetics will be optimized, further benefiting PVI.
 


% \subsubsection{Towards Real-Time Processing AI-Powered Assistant}
% One of the most significant advantages of \bma{} and other LMM-based assistive tools is their ability to provide contextually relevant and personalized assistance to users. By leveraging the power of machine learning and the ability for understanding of natural language, these systems can understand and respond to a wide range of user queries. This level of contextual awareness represents a significant advancement over traditional assistive technologies, which often struggle to adapt to the diverse needs and preferences of individual users.

% However, our findings also identify several challenges and limitations of current LMM-based assistive tools, particularly in terms of their reliance on user-generated images. Participants in our study reported frustration with the need to take multiple pictures to capture the desired information, which can be time-consuming and cognitively demanding (Section~\ref{navigation}). To address this issue, we envision the integration of real-time image processing capabilities into future AI-powered VQA systems. By continuously analyzing the user's surroundings and providing relevant information without the need for explicit image capture, these systems could offer a more seamless and efficient user experience. The integration of real-time image processing into AI-powered VQA systems aligns with the growing availability of commercial products and research prototypes that leverage advanced object detection and text recognition technologies. For example, Microsoft's Seeing AI \cite{SeeingAI2020} and various currency recognition systems \cite{liu2008camera, parlouar2009assistive, paisios2012exchanging} demonstrate the feasibility and potential impact of real-time image processing in assistive technology. By building upon these existing approaches and incorporating state-of-the-art deep learning techniques for object detection \cite{girshick2014rich, girshick2015fast, ren2016faster, krizhevsky2017imagenet} and text recognition \cite{ma2018arbitrary, he2017deep, zhou2017east, yao2016scene, liu2017deep, lyu2018multi}, future AI-powered VQA systems could provide even more robust and reliable assistance to PVI.



% \subsubsection{Towards Reliable AI-Powered Assistant Prosthetics}



% Our findings highlight the significant potential of AI-Powered assistive technologies like \bma{} in enhancing the perception and understanding of surroundings for PVI. However, our study also reveals a notable drawback of AI-powered assistant prosthetics, namely AI hallucinations. These errors, where the artifact inaccurately identifies objects that aren't present, can lead to confusion and mistrust among users. Participants in our study reported instances where \bma{} erroneously added non-existent details to scenes (Section~\ref{scene}). 
% It can be argued that the presence of AI hallucinations poses a major challenge to the reliability and robustness of AI-powered assistive prosthetics. If users cannot trust the information provided by these systems, their effectiveness as cognitive extensions is severely compromised. This issue is particularly critical for PVI, who rely on these technologies to navigate and make sense of their environment.

% To address the problem of AI hallucinations, our participants applied various strategies, such as consulting human assistants for verification or relying on their own knowledge to identify inaccuracies. While these strategies showcase PVI's adaptability and problem-solving skills, they also highlight the need for more reliable and robust AI-powered assistive technologies.

% One potential approach to mitigating AI hallucinations is to incorporate uncertainty estimation and communication into the design of these technologies. By quantifying and conveying the confidence level of predictions, AI-powered assistive systems can help users assess the reliability of the information provided. This approach has been explored in the context of other AI-based systems, such as medical diagnosis~\cite{leibig2017leveraging,begoli2019need} and autonomous vehicles~\cite{michelmore2018evaluating}.

% Another strategy is to develop AI-powered assistive technologies that can learn and adapt to user feedback over time. By allowing PVI or human assistants to correct errors and provide input to improve system performance, which is not only able to continuously improve system's reliability and robustness and also help aware users about possible AI hallucinations. This approach aligns with the principles of interactive machine learning, emphasizing the importance of human-in-the-loop learning for AI systems~\cite{amershi2014power,retzlaff2024human}.

% In addition to these technical solutions, involving PVI in the design and evaluation of AI-powered assistive technologies is crucial. By engaging users as co-designers and co-evaluators, researchers and designers can gain a better understanding of the challenges and requirements of PVI, leading to the development of more reliable and robust systems. This participatory design approach has been widely advocated in the assistive technology domain~\cite{frauenberger2015pursuit,lee2004trust,zhang2023redefining}.




% \subsubsection{Envisioning Multimodal AI-Powered Assistant Prosthetics} 

% %integrate multimodal input, participants can use gestures, touch, haptic feedback to interact with \bma. 

% %llm

% Our research demonstrates the superior ability and performance of \bma{} compared to previous AI-powered systems when handling various tasks. In tasks such as object recognition, processing complex information, and interpreting graphical elements, \bma{} not only completes the tasks but also provides extended explanations and further task collaboration capabilities. 
% This undoubtedly greatly expands the interaction scenarios and user experience of \bma{}. 

% However, as mentioned earlier, the current \bma{} faces challenges, including higher interaction costs due to multiple photo captures and AI hallucinations.
% % previously mentioned, we found that the current \bma{} still has some challenges and limitations in terms of interaction experience, such as the additional interaction costs caused by the need for multiple photo captures and AI hallucinations.
% % 
% Apart from integrating additional advanced technologies to expand \bma{}'s capabilities, we also suggest adopting multimodal interaction. This approach, a key design insight from our research, underscores the value of diverse interactions and feedback in AI-powered assistive tools.
% % apart from integrating other advanced technologies to expand \bma{}'s capabilities, we suggest introducing the concept of multimodal interaction, which is another key design implication derived from our research, emphasizing the importance of multimodal interaction and feedback in artificial intelligence assistive tools. 
% % 
% Although \bma{} primarily relies on voice-based input and output, integrating multimodal interaction methods, such as touch, gestures, and haptic feedback, will greatly enhance the usability and accessibility of LMM-based assistive functions. This perspective of combining multimodal interaction has been widely studied in some previous literature \cite{turk2014multimodal, reeves2004guidelines}.
% % oviat2017handbook
% For example, touch-based gestures can enable users to navigate the system interface more effectively, while haptic feedback can provide additional spatial and contextual information to supplement voice output. These multimodal interactions also allow \bma{} to adapt and personalize to better meet the diverse needs of PVI. Future research should further explore how to design and implement these multimodal interaction technologies and assess their long-term impact on PVI user experience and quality of life.

% %Our research demonstrates the superior ability and performance of \bma{} compared to previous systems and tools when handling various tasks. For instance, in tasks such as object recognition, processing complex information, and interpreting graphical elements, \bma{} not only accomplishes the tasks but also provides \textcolor{red}{multimodal input,} extended explanations, and further task collaboration capabilities. This undoubtedly greatly expands the interaction scenarios and user experience of \bma. Traditional applications and tools, such as those using conventional OCR technology and rapid reading applications like ``Seeing AI,'' are considered effective. \textcolor{red}{[cite:] However, users find it challenging to process a complete task flow using such tools, as their functions are designed for single tasks.} In other words, traditional applications and tools can only complete a specific step in a task flow and cannot connect the context of the entire task. Leveraging the LMM's ability to understand multimodal data, interact using natural language, and be context-sensitive, \bma{} can comprehend more practical needs and provide personalized assistance to users based on their requirements. Furthermore, \bma{} demonstrates its potential to replace previous assistive methods and technologies. \textcolor{red}{This is particularly evident in a series of scenarios where \bma{} substitutes RSA services, such as when participants use \bma{} for reading and control tasks. These tasks, which are considered replaceable by \bma{}, are common needs of PVI in RSA services. }

% %Another key design implication that emerged from our research is the importance of multimodal interaction and feedback in AI-powered assistive tools. While \bma{} primarily relies on voice-based input and output, \textcolor{red}{participants expressed interest in exploring other interaction modalities, such as touch, gestures, and haptic feedback.=> integrate other interaction modalities, such as touch, gestures, and haptic feedback, into \bma} The incorporation of multimodal interaction techniques, which have been extensively studied in the HCI literature \cite{turk2014multimodal, reeves2004guidelines, oviatt2017handbook}, could greatly enhance the usability and accessibility of LMM-based assistive tools. For example, touch-based gestures could enable users to navigate the system's interface more efficiently, while haptic feedback could provide additional spatial and contextual information to supplement voice output.

%\subsubsection{future ai assistant prosthetic}
%from 5.4, about design process? 
%
%integrate new tech: generative ai 
%next iteration




% \subsubsection{Opportunities in Human-AI Collaboration}
% % Evolution of Human-AI Collaboration
% % Future AI Assistant Prosthetic - From Temporal Dimension to Consider Design Implications
% Again, while our findings demonstrate the significant potential of AI-powered assistive technologies in enhancing PVI's capabilities across various contexts, participants also highlighted some limitations of these technologies. However, we argue that these limitations may be part of an adjustment period as PVI learn to interact with and adapt to AI-powered assistive technologies.


% This observation aligns with the findings of~\citet{10.1145/3563657.3595977}, who described how users, when initially using AI-powered assistive technologies, may encounter responses that do not match their expectations. Despite this, most participants were able to quickly learn and adapt to the system, reaching a point where they could collaborate with \bma{} to complete tasks and perceive it as an effective technology.

% This contradictory state may give rise to a form of Human-AI ``confrontation.'' However, this ``confrontation'' appears to diminish as the system undergoes iterative upgrades and PVI update their understanding of the system's capabilities. From a phenomenological perspective, Ihde's theory~\cite{ihde1991instrumental} suggests that the science of tools evolves alongside people's cognition, implying that the relationship between PVI and AI-powered assistive technologies is not static but rather a dynamic process of mutual adaptation and growth.


% As PVI become more familiar with the capabilities and limitations of AI-powered assistive technologies, they develop strategies to leverage these technologies effectively and compensate for the shortcomings. This process of adaptation and learning is a critical aspect of the distributed cognition framework, which emphasizes the role of artifacts and the environment in shaping cognitive processes \cite{hollan2000distributed}.


% Moreover, the iterative nature of AI-powered assistive technologies, with ongoing updates and improvements, allows for a continuous refinement of the Human-AI interaction. As these technologies become more sophisticated and attuned to the needs and preferences of PVI, the initial ``confrontation'' may give way to a more seamless and synergistic collaboration between PVI and AI-powered assistive technologies.

% This perspective highlights the importance of considering the temporal dimension of design implications in the context of AI-powered assistive technologies. Rather than viewing the limitations of these technologies as static barriers, we should recognize the potential for PVI to adapt and develop new cognitive strategies in response to these limitations, and for the technologies themselves to evolve in response to user feedback and needs.


%%%%%%%%%%
% \subsubsection{Ethical Considerations and User Involvement in the Design of AI-Powered Assistive Technologies}
% As we continue to explore the potential of AI-powered assistive technologies for PVI, it is crucial to involve the target users in the design and evaluation process \cite{10.1145/3025453.3025899}. Our research with \bma{} underscores the importance of actively engaging with PVI throughout the design process to ensure that these systems are tailored to their specific needs, preferences, and contexts of use. By involving users as co-designers and co-evaluators, we can create assistive tools that are not only technologically advanced but also truly empowering and inclusive.

% Moreover, the development of AI-powered assistive technologies should be guided by a commitment to ethical and responsible innovation~\cite{floridi2018ai4people,jobin2019global}. As LMM-based systems become increasingly sophisticated and integrated into users' daily lives, it is essential to consider issues of privacy, security, and fairness. Researchers and designers should ensure these technologies are transparent, accountable, and aligned with the values and goals of the communities they serve~\cite{10.1145/3351095.3372873}.

% In conclusion, our research with \bma{} highlights the immense potential of LMM-based assistive tools for PVI while also revealing key design implications for future AI-powered VQA systems. By integrating real-time image processing, supporting multimodal interaction and feedback, and prioritizing user-centered design principles, we can create assistive technologies that are more usable, effective, and empowering. As the field of AI-powered assistive technology continues to evolve, it is essential for HCI researchers and practitioners to collaborate with PVI's communities to ensure that these innovations are developed in an inclusive, ethical, and responsible manner.



%\subsection{\textcolor{red}{Most assistive technologies die after their introduction}}
%
%\textcolor{red}{[combine with design implications, with sub-headings]}

%Assistive technologies for PVI have undergone a rapid evolution in recent years~\cite{10.1145/3597638.3608412}, with the introduction of LLM-based applications like \bma{} marking a significant shift in the landscape. \textcolor{red}{Traditional assistive technologies, such as VQA}~\cite{bigham2010vizwiz_nearly,BeMyEyes2020}, screen readers and braille displays~\cite{muhsin2024review,alves2009assistive}, have long been the primary methods for blind individuals to access the content and navigate the world. However, assistive technology is undergoing a transition from human-driven to AI-driven~\cite{xu2023transitioning, 10.1145/3234695.3239330}, and the emergence of a series of technology-dominated assistive services has challenged the dominance of these traditional technologies, offering a more intuitive, user-friendly, and versatile experience for blind users. Our research further reveals that \bma{} has elevated AI-led assistive technology to new heights, which inevitably leads one to imagine the possibility of a future where assistive technology is fully AI-dominated.
%
%Despite the initial promise and potential of many traditional assistive technologies, a significant number of them gradually lost their appeal after the emergence of applications like \bma{}, and may eventually die out after users fully adopt \bma{}. This phenomenon can be attributed to several factors, including the lack of comprehensive functionality, high complexity of operation, social difficulties and limited usage scenarios~\cite{gori2016devices,manjari2020survey,tapu2020wearable}. These technologies failed to meet the evolving needs and expectations of blind users in one or more aspects, while \bma{} seems to offer a promising solution that integrates and expands the capabilities of various traditional assistive technologies.
%
%
%While \bma{} is not perfect at present, the introduction of \bma{} represents a significant milestone in the development of assistive technologies for blind individuals. These applications offer new possibilities and empower PVI in unprecedented ways. Building on this foundation, it is crucial to approach their development and adoption from a user-centered perspective, ensuring that they are accessible, inclusive, and responsive to the diverse needs of the blind community~\cite{10.1145/3025453.3025895,federici2012assistive}.
%%%%%%%%%%%%%%%%%%%%%%%%%%%%%%%%%%%%%%%%
%this paragraph can be removed.
%The rise and fall of traditional assistive technologies, coupled with the emergence of LLM-based applications \bma{}, offer valuable lessons for the future development and adoption of assistive tools for PVI. First and foremost, user-centered design and continuous adaptation must be at the heart of assistive technology development to ensure that these tools truly meet the evolving needs and expectations of blind users. Second, the success of LLM-based applications highlights the importance of leveraging advanced technologies, such as natural language processing and machine learning, to create more intuitive and versatile assistive tools. Finally, the challenges and concerns surrounding the adoption of LLM-based applications underscore the need for ongoing research, collaboration, and dialogue among researchers, developers, and blind users to ensure that these technologies are developed and deployed in an accessible, inclusive, and responsible manner.


%The introduction of \bma{} represents a significant milestone in the evolution of assistive technologies for blind people. While these applications offer new possibilities and empower blind individuals in unprecedented ways, it is crucial to approach their development and adoption with a user-centered perspective, ensuring that they are accessible, inclusive, and responsive to the diverse needs of the blind community. As the field of assistive technology continues to evolve, researchers and developers must remain committed to creating tools that truly enhance the lives of blind individuals and promote their full participation in the digital world. %The decline of traditional assistive technologies in the face of LLM-based applications serves as a reminder of the importance of continuous innovation, adaptation, and user-centered design in the quest to create a more accessible and inclusive world for all.





\section{Related works}

\paragraph{LLM alignment.}
Pretrained LLMs demonstrate remarkable capabilities across a broad spectrum of tasks \citep{brown2020language}.
Their performance at downstream tasks, such as conversational modeling, is significantly enhanced through alignment with human preferences \citep{ouyang2022training, bai2022training}. 
RLHF \citep{christiano2017deep} has emerged as a foundational framework for this alignment, typically involving learning a reward function via a preference model, often using the Bradley-Terry model \citep{bradley1952rank}, and tuning the LLM using reinforcement learning (RL) to optimize this reward. 
Despite its success, RLHF's practical implementation is notoriously complex, requiring multiple LLMs, careful hyperparameter tuning, and navigating challenging optimization landscapes.

Recent research has focused on simplifying this process. A line of works studies the direct alignment algorithms \citep{zhao2023slic, rafailov2024direct, azar2024general}, which directly optimize the LLM in a supervised manner without first constructing a separate reward model. In particular, the representative DPO \citep{rafailov2024direct} attracts significant attention in both academia and industry. After these, SimPO \citep{meng2024simpo} simplifies DPO by using length regularization in place of a reference model. 
% However, these approaches are primarily offline and can suffer from performance degradation when facing distribution shifts during deployment, yielding poorer generalization.
% Iterative DPO \citep{xiong2024iterative} attempts to address this by enabling iterative optimization for improved alignment.

% These challenges can be potentially tackled by connecting LLM alignment with IR.
Although LLMs are adopted for IR \citep{tay2022transformer}, there is a lack of study to improve direct LLM alignment with IR principles.
% While significant progress has been made in LLM alignment, the connection with IR remains largely unexplored.
% However, LiPO does not explore the potential of online optimization methods inspired by IR. 
This paper fills this gap by establishing a systematic link between LLM alignment and IR methodologies, and introducing a novel iterative LLM alignment approach that leverages insights from retriever optimization to advance the state of the art.
The most related work is LiPO \citep{liu2024lipo}, which applies learning-to-rank objectives.
% However, LiPO focuses on offline settings and it is unclear how to obtain list-wise data for it.
However, LiPO relies on off-the-shelf listwise preference data, which is hard to satisfy in practice.

\paragraph{Language models for information retrieval.}
Language models (LMs) have become integral to modern IR systems \citep{zhu2023large}, particularly after the advent of pretrained models like BERT \citep{kenton2019bert}.  
A typical IR pipeline employs retrievers and rerankers, often based on dual-encoder and cross-encoder architectures, respectively \citep{humeau2019poly}. 
Dense Passage Retrieval (DPR) \citep{karpukhin2020dense} pioneered the concept of dense retrieval, laying the groundwork for subsequent research. 
Building on DPR, studies have emphasized the importance of hard negatives in training \citep{zhan2021optimizing, qu2020rocketqa} and the benefits of online retriever optimization \citep{xiong2020approximate}.

In the realm of reranking, \citep{nogueira2019passage} were among the first to leverage pretrained language models for improved passage ranking. 
This was followed by MonoT5 \citep{nogueira2020document}, which scaled rerankers using large encoder-decoder transformer architectures, and RankT5 \citep{zhuang2023rankt5}, which introduced pairwise and listwise ranking objectives. 
Recent work has also highlighted the importance of candidate list preprocessing before reranking \citep{meng2024ranked}.

Despite the pervasive use of LMs in IR, the interplay between LLM alignment and IR paradigms remains largely unexplored. 
This work aims to bridge this gap, establishing a strong connection between LLM alignment and IR, and leveraging insights from both fields to advance our understanding of LLM alignment from an IR perspective.

\section{Conclusion}
%In conclusion, this paper presents a significant leap forward in addressing the challenge of FCHs in LLMs. By developing a systematic framework based on logic programming and integrating it into \tool, we have effectively tackled the limitations of current detection methodologies. Our approach, grounded in transforming a comprehensive factual knowledge base sourced from Wikipedia through advanced logic reasoning methods, has demonstrated superior performance in detecting factual inaccuracies across various LLMs. The automation of this process marks a notable advancement in scalability, reducing reliance on manual intervention. Furthermore, the release of our enriched dataset as a benchmark contributes to the broader research community, paving the way for future innovations in hallucination detection. This work not only enhances the reliability and usability of LLMs but also sets a new standard for research in this critical area of language processing technology. 
We target the critical challenge of FCH in LLM, where they generate outputs contradicting established facts. We developed a novel automated testing framework that combines logic programming and metamorphic testing to systematically detect FCH issues in LLMs. Our novel approach constructs a comprehensive factual knowledge base by crawling sources like Wikipedia, then applies innovative logic reasoning rules to transform this knowledge into a large set of test cases with ground truth answers. 
These reasoning rules are either predefined relations or automatically generated from randomly sampled temporal formulae. 
LLMs are evaluated on these test cases through template prompts, with two semantic-aware oracles analyzing the similarity between the logical/semantic structures of the LLM outputs and ground truth to validate reasoning and pinpoint FCHs. 

Across diverse subjects and LLM architectures, our framework automatically generated over 9,000 useful test cases, uncovering hallucination rates as high as 59.8\% and identifying lack of logical reasoning as a key contributor to FCH issues. This work pioneers automated FCH testing capabilities, providing a comprehensive benchmark, data augmentation techniques, and answer validation methods. The implications are far-reaching --- enhancing LLM reliability and trustworthiness for high-stakes applications by exposing critical weaknesses while advancing systematic evaluation methodologies.


%\pagebreak

%\usepackage[round]{natbib}
% \bibliographystyle{ACM-Reference-Format}
\bibliographystyle{IEEEtran}
%\citestyle{apacite}
% \printbibliography
\balance
\normalem
\bibliography{8.ref}

%\newpage 
%\appendix


%
\subsection{Correctness of the encoding rules}
\label{app:correctness}

\ThemSoundAndComplete*
\begin{proof} By a case analysis of the encoding rules and semantic definitions: 

\begin{enumerate}[itemsep=1.5em,leftmargin=!]
\vspace{1em}
\item Atomic Proposition: 
{
\small 
\begin{align*}
\frac{
\begin{matrix}
\widetilde{\drule} = [\nmNEW(\interval) \hornarrow \nm\_{\m{TS}}(\interval).]
\end{matrix}
}{\encoding {\nm}{\nmNEW}{\widetilde{\drule}}}\ [\trans\text{-}\m{AP}]
\end{align*}
\vspace{-1mm}
\begin{align*}
(\history, \timepoint) &\models 
\nm &\m{iff}&~ 
\m{\exists\,\interval}.~ 
\llbracket \nm\_{\m{TS}}(\interval) \text{$\rrbracket_{\history}$}{=}\m{true}
~\m{and}~
\timepoint\,{\in}\,\interval
\\[0.1em]
%, \Subj, \Obj
\end{align*}
\vspace{-8mm}
}

In   
$[\trans\text{-}\m{AP}]$, with $\nm$, 
$\forall \interval.~\llbracket \nmNEW(I) \rrbracket_{\Prolog} {=} \m{true}$, 
its premise indicates that 
$\llbracket  \nm_{\m{TS}}(\interval)\rrbracket_{\history} {=} \m{true}$. 
Next, from the semantic definition, we have 
$\forall  \timepoint\,{\in}\,\interval, (\history, \timepoint) {\models} \nm$; thus, the rule is sound. 
\\
From the semantic definition, $\forall   (\history, \timepoint) {\models} \nm$, it indicates that $\exists\interval.~\llbracket  \nm_{\m{TS}}(\interval)\rrbracket_{\history} {=} \m{true} ~\m{and}~
\timepoint\,{\in}\,\interval$.  
Next, from $[\trans\text{-}\m{AP}]$, we obtain 
$\llbracket \nmNEW(I) \rrbracket_{\Prolog} {=} \m{true}$; thus, the rule is complete. 



\item  Finally:
{
\small 
\begin{align*}
\frac{
\begin{matrix}
\widetilde{\drule} {=} 
[\nmNEW([\interval^\prime_\m{start}\text{-}\interval_{\m{end}}, \interval^\prime_\m{end}\text{-}\interval_{\m{start}}]) \hornarrow \nm(\interval^\prime).]
\end{matrix}
}{\encoding {\mathcal{F}_\interval\,\mtl}{\nmNEW}{\widetilde{\drule} }}\ [\trans\text{-}\m{Finally}]
\end{align*}
\vspace{-1mm}
\begin{align*}
(\history, \timepoint) &\models \mathcal{F}_\interval \,\mtl & 
\m{iff}&~ 
\m{\exists\,\distance}.~\distance\,{\in}\,I  ~ \m{and}
~ (\history, \timepoint\plus\distance)\models\mtl
\\[0.1em]
\end{align*}
\vspace{-8mm}
}



In $[\trans\text{-}\m{Finally}]$  with 
$\mathcal{F}_{[\Istart, \Iend]}\,\mtl$, \\
$\forall \interval^{\prime\prime}.~\llbracket \nmNEW(\interval^{\prime\prime}) \rrbracket_{\Prolog} {=} \m{true}$, 
its premise indicates that \\
$\llbracket  \nm([\interval^{\prime\prime}_\m{start}\plus\Iend, \interval^{\prime\prime}_\m{end}\plus\Istart])\rrbracket_{\history} {=} \m{true}$, 
which means that $\forall \timepoint \,{\in}\,\interval^{\prime\prime}$, there exists $\distance\,{\in}\,[\Istart, \Iend]$ such that $(\history, \timepoint\plus\distance) {\models} \mtl$
\\
Next, from the semantic definition, we have \\
$\forall  \timepoint\,{\in}\,\interval^{\prime\prime}, 
(\history, \timepoint) {\models} \mathcal{F}_{[\Istart, \Iend]}\,\mtl$; thus the rule is sound. \\
From the semantic definition, $\forall   
(\history, \timepoint) {\models} \mathcal{F}_{[\Istart, \Iend]}\,\mtl$; it indicates that 
$\exists \distance\,{\in}\,[\Istart, \Iend] ~\m{and}~ (\history, \timepoint\plus\distance)\models\mtl$, 
which means that 
$\exists \interval^\prime.~ \timepoint\plus\distance\,{\in}\, \interval^\prime ~\m{and} ~\llbracket \nm(\interval^\prime) \rrbracket_{\Prolog} {=} \m{true}$
\\
Next, from $[\trans\text{-}\m{Finally}]$, 
we obtain \\
$\llbracket \nmNEW([\interval^\prime_\m{start}\text{-}\Iend, \interval^\prime_\m{end}\text{-}\Istart]) \rrbracket_{\Prolog} {=} \m{true}$,  and thus \\
$\timepoint\,{\in}\,[\interval^\prime_\m{start}\text{-}\Iend, \interval^\prime_\m{end}\text{-}\Istart]$; 
thus, the rule is complete. 

\item  Globally:
{
\small 
\begin{align*}
\frac{
\begin{matrix}
\widetilde{\drule} {=} 
[\nmNEW([\interval^\prime_\m{start}\text{-}\interval_{\m{start}}, \interval^\prime_\m{end}\text{-}\interval_{\m{end}}]) \hornarrow \nm(\interval^\prime).]
\end{matrix}
}{\encoding {\mathcal{G}_\interval\,\mtl}{\nmNEW}{\widetilde{\drule} }}\ [\trans\text{-}\m{Globally}]
\end{align*}
\vspace{-1mm}
\begin{align*}
(\history, \timepoint) &\models \mathcal{G}_\interval\,\mtl & 
\m{iff}&~ 
\m{\forall\,\distance}.~\distance\,{\in}\,I  ~ \m{and}
~ (\history, \timepoint\plus\distance)\models\mtl
\\[0.1em]
\end{align*}
\vspace{-8mm}
}

In $[\trans\text{-}\m{Globally}]$  with 
$\mathcal{G}_{[\Istart, \Iend]}\,\mtl$, \\
$\forall \interval^{\prime\prime}.~\llbracket \nmNEW(\interval^{\prime\prime}) \rrbracket_{\Prolog} {=} \m{true}$, 
its premise indicates that \\
$\llbracket  \nm([\interval^{\prime\prime}_\m{start}\plus\Istart, \interval^{\prime\prime}_\m{end}\plus\Iend])\rrbracket_{\history} {=} \m{true}$, which means that $\forall \timepoint \,{\in}\,\interval^{\prime\prime}$, for all $\distance\,{\in}\,[\Istart, \Iend]$ such that $(\history, \timepoint\plus\distance) {\models} \mtl$. \\
Next, from the semantic definition, we have \\
$\forall  \timepoint\,{\in}\,\interval^{\prime\prime}, 
(\history, \timepoint) {\models} \mathcal{G}_{[\Istart, \Iend]}\,\mtl$; thus the rule is sound. \\
From the semantic definition, $\forall   
(\history, \timepoint) {\models} \mathcal{G}_{[\Istart, \Iend]}\,\mtl$; it indicates that $\forall\distance\,{\in}\,[\Istart, \Iend] ~\m{and}~ (\history, \timepoint\plus\distance)\models\mtl$, 
which means that 
$\exists \interval^\prime.~ \timepoint\plus\distance\,{\in}\, \interval^\prime ~\m{and} ~\llbracket \nm(\interval^\prime) \rrbracket_{\Prolog} {=} \m{true}$
\\
Next, from $[\trans\text{-}\m{Globally}]$, 
we obtain \\ 
$\llbracket \nmNEW([\interval^\prime_\m{start}\text{-}\Istart, \interval^\prime_\m{end}\text{-}\Iend]) \rrbracket_{\Prolog} {=} \m{true}$,  and thus \\  $\timepoint\,{\in}\,[\interval^\prime_\m{start}\text{-}\Istart, \interval^\prime_\m{end}\text{-}\Iend]$; 
thus, the rule is complete. 



\item  Next:
{
\small 
\begin{align*}
\frac{
\begin{matrix}
\widetilde{\drule} {=} 
[\nmNEW([\interval^\prime_\m{start}\text{-}1, \interval^\prime_\m{end}\text{-}1]) \hornarrow \nm(\interval^\prime).]
\end{matrix}
}{\encoding {\mathcal{N}\,\mtl}{\nmNEW}{\widetilde{\drule} }}\ [\trans\text{-}\m{Next}]
\end{align*}
\vspace{-1mm}
\begin{align*}
(\history, \timepoint) &\models \mathcal{N}\,\mtl & 
\m{iff}&~ 
(\history, \timepoint\plus 1)\models\mtl
\\[0.1em]
\end{align*}
\vspace{-8mm}
}


In $[\trans\text{-}\m{Next}]$  with 
$\mathcal{N}\,\mtl$, \\
$\forall \interval^{\prime\prime}.~\llbracket \nmNEW(\interval^{\prime\prime}) \rrbracket_{\Prolog} {=} \m{true}$, 
its premise indicates that \\
$\llbracket  \nm([\interval^{\prime\prime}_\m{start}\plus 1, \interval^{\prime\prime}_\m{end}\plus 1])\rrbracket_{\history} {=} \m{true}$. \\
Next, from the semantic definition, we have \\
$\forall  \timepoint\,{\in}\,\interval^{\prime\prime}, 
(\history, \timepoint) {\models} \mathcal{N}\,\mtl$; thus the rule is sound. \\
From the semantic definition, $\forall   
(\history, \timepoint) {\models} \mathcal{N}\,\mtl$, \\ 
it indicates that $ (\history, \timepoint\plus 1)\models\mtl$, which means that \\ 
$\exists \interval^\prime.~ \timepoint\plus1\,{\in}\, \interval^\prime ~\m{and} ~\llbracket \nm(\interval^\prime) \rrbracket_{\Prolog} {=} \m{true}$.\\
Next, from $[\trans\text{-}\m{Next}]$, 
we obtain  \\
$\llbracket \nmNEW([\interval^\prime_\m{start}\text{-}1, \interval^\prime_\m{end}\text{-}1]) \rrbracket_{\Prolog} {=} \m{true}$, and thus \\  $\timepoint\,{\in}\,[\interval^\prime_\m{start}\text{-}1, \interval^\prime_\m{end}\text{-}1]$; 
thus, the rule is complete. 


\item  Until: 
{
\small 
\begin{align*}
\frac{
\begin{matrix}
\encoding{\mtl_1}{\nm_1}{\widetilde{\drule}_1}
\qquad 
\encoding{\mtl_2}{\nm_2}{\widetilde{\drule}_2}
\\[0.2em]
\widetilde{\drule}_3{=} [\m{helper1}([\interval^\prime_{\m{start}}\plus\interval_{\m{start}}, \interval^\prime_{\m{end}}\plus1]) \hornarrow 
\nm_1(\interval^\prime).]
\\[0.2em]
\widetilde{\drule}_4{=} [\m{helper2}(\interval_1\,{\cap}\,\interval_2) \hornarrow 
\m{helper1}(\interval_1), 
\nm_2(\interval_2).] 
\\[0.2em]
\encoding {\mathcal{F}_\interval\,(\m{helper2})}{\nm_f}{\widetilde{\drule}_5 }
\\[0.2em]
\widetilde{\drule}_6{=} [\nmNEW(\interval_1\cap \interval_2) \hornarrow 
\nm_1(\interval_1), 
\nm_
f(\interval_2). ] 
\end{matrix}
}{\encoding{\mtl_1\,\mathcal{U}_\interval\,\mtl_2}{\nmNEW}{\widetilde{\drule}_1\cup \widetilde{\drule}_2\cup
\widetilde{\drule}_3\cup
\widetilde{\drule}_4\cup
\widetilde{\drule}_5\cup
\widetilde{\drule}_6}}\ [\trans\text{-}\m{Until}]
%\shil{
%~2. ~what's ~the~ definition ~\interval ~in~ \mathcal{F}? }
%\\ \shil{~3. ~what's ~the ~meaning ~of ~;?}\text{\syh{to~construct~list~from~single~rules}}
\end{align*}
%\vspace{-1mm}
\begin{align*}
(\history, \timepoint) &\models \mtl_1 \, \mathcal{U}_\interval \,\mtl_2  & \m{iff}&~  \m{\exists\,\distance}.~ \distance\,{\in}\,\interval  ~ \m{and}~ (\history, \timepoint\plus\distance)\models\mtl_2 ~ \m{and}
\\[0.1em] 
&&& ~ 
\m{\forall}\, 
k~\m{with} ~\timepoint{<}k{<}(\timepoint\plus\distance), 
(\history, k)\models \mtl_1
\\[0.1em]
\end{align*}
\vspace{-8mm}
}



In $[\trans\text{-}\m{Until}]$  with 
$\mtl_1\,\mathcal{U}_{[\Istart, \Iend]}\,\mtl_2$, \\
$\forall \interval^{\prime}.~\llbracket \nmNEW(\interval^{\prime}) \rrbracket_{\Prolog} {=} \m{true}$, 
its premise indicates that \\
$\forall \timepoint\,{\in}\,\interval^{\prime}, 
\m{exists~\distance}.~ \distance\,{\in}\, {[\Istart, \Iend]}  ~ \m{and}~ $\\ 
$
(\history, k)\models \mtl_1 ~\m{forall}~ 
k~\m{with} ~\timepoint{<}k{<}(\timepoint\plus\distance)
~ \m{and}~(\history, \timepoint\plus\distance)\models\mtl_2 $, \\
guaranteed by the helper functions $\m{helper1}$, $\m{helper2}$, and $\nm_f$.  Next, from the semantic definition, we have \\
$(\history, \timepoint) {\models} \mtl_1\,\mathcal{U}_{[\Istart, \Iend]}\,\mtl_2$; thus the rule is sound. \\
From the semantic definition, $\forall   
(\history, \timepoint) {\models} \mtl_1\,\mathcal{U}_{[\Istart, \Iend]}\,\mtl_2$, \\ it indicates that $ \m{exists~\distance}.~ \distance\,{\in}\,{[\Istart, \Iend]}  ~ \m{and}~ (\history, \timepoint\plus\distance)\models\mtl_2 ~ \m{and}~(\history, k)\models \mtl_1 ~\m{forall}~ 
k~\m{with} ~\timepoint{<}k{<}(\timepoint\plus\distance)$. \\ 
Next, from $[\trans\text{-}\m{Until}]$, 
$\m{helper1}$ produces the superset of the possible values of $\timepoint{\plus}\distance$ which satisfy the first constrain, then $\m{helper2}$ produces the exact set of the possible values of $\timepoint{\plus}\distance$ which also satisfy the second constrain. 
Lastly, $\nmNEW$ produces the exact set of the possible values of $\timepoint$; thus, the rule is complete. 

\item $\mtl_1  
\,\mathcal{U}_{[0, 0]} \,  \mtl_2$ $\equiv$ $\mtl_2$: 
{
\small 
\begin{align*}
(\history, \timepoint) &\models \mtl_1 \, \mathcal{U}_\interval \,\mtl_2  & \m{iff}&~  \m{\exists\,\distance}.~ \distance\,{\in}\,\interval  ~ \m{and}~ (\history, \timepoint\plus\distance)\models\mtl_2 ~ \m{and}
\\[0.1em] 
&&& ~ 
\m{\forall}\, 
k~\m{with} ~\timepoint{<}k{<}(\timepoint\plus\distance), 
(\history, k)\models \mtl_1
\\[0.1em]
\end{align*}
\vspace{-8mm}
}

By instantiating the above semantic rule for Until operators with $\interval{=}[0, 0]$, we obtain the following semantic rule: 

{
\small 
\begin{align*}
(\history, \timepoint) &\models \mtl_1 \, \mathcal{U}_{[0, 0]} \,\mtl_2  & \m{iff}&~  \m{exists~\distance}.~ \distance{=}0  ~ \m{and}~ (\history, \timepoint\plus 0)\models\mtl_2 ~ \m{and}
\\[0.1em] 
&&& ~ 
(\history, k)\models \mtl_1 ~\m{forall}~ 
k~\m{with} ~\timepoint{<}k{<}(\timepoint\plus 0)
\end{align*}}

Which is essentially: 

{
\small 
\begin{align*}
(\history, \timepoint) &\models \mtl_1 \, \mathcal{U}_{[0, 0]} \,\mtl_2  & \m{iff}&~   (\history, \timepoint)\models\mtl_2 
\end{align*}}

Thus, the conclusion $\mtl_1  
\,\mathcal{U}_{[0, 0]} \,  \mtl_2$ $\equiv$ $\mtl_2$ is sound and complete. 

\item  Negation:
{
\small 
\begin{align*}
\frac{
\begin{matrix}
\encoding{\mtl}{\nm}{\widetilde{\drule}_1}
\\ 
\widetilde{\drule}{=}[\nmNEW(\interval) \hornarrow
\m{findall}(\interval_1, \nm), \m{compl}(\interval_1, \interval).]
\end{matrix}
}{
\encoding{\neg\mtl}{\nmNEW}{
\widetilde{\drule}_1\,{\cup}\,\widetilde{\drule}}
}\ [\trans\text{-}\m{Neg}]
\end{align*}
\vspace{-1mm}
\begin{align*}
(\history, \timepoint) &\models\neg \mtl & \m{iff}&~
(\history, \timepoint)\not\models\mtl
\\[0.1em]
\end{align*}
\vspace{-8mm}
}

In $[\trans\text{-}\m{Neg}]$  with 
$\neg\,\mtl$, \\
$\forall \interval.~\llbracket \nmNEW(\interval) \rrbracket_{\Prolog} {=} \m{true}$, 
its premise indicates that \\ 
$\forall \interval_1.~ \nm(\interval_1), \m{and} ~ \interval \cap \interval_1 \,{=}\, \emptyset$. 
\\
Next, from the semantic definition, we have \\
$\forall  \timepoint\,{\in}\,\interval, 
(\history, \timepoint) {\models} \neg\,\mtl$; thus the rule is sound. \\
From the semantic definition, $\forall   
(\history, \timepoint) {\models} \neg\,\mtl$, it indicates that $ (\history, \timepoint){\not\models}\mtl$, 
which means that 
$\forall \interval'. ~ \timepoint\,{\not\in}\,\interval'$ and $\llbracket \nm(\interval^\prime) \rrbracket_{\Prolog} {=} \m{true}$. 
\\
Next, from $[\trans\text{-}\m{Neg}]$, 
we obtain  \\
$\exists\interval.~ \timepoint\,{\in}\,\interval, \interval \cap (\m{findall}(\interval_1, \nm)) {=} \emptyset,$ and $ \llbracket \nmNEW(\interval) \rrbracket_{\Prolog} {=} \m{true}$; 
thus, the rule is complete. 

\item Disjunction:
{
\small 
\begin{align*}
\frac{
\begin{matrix}
[\trans\text{-}\m{Disj}]\\
\encoding{\mtl_1}{\nm_1}{\widetilde{\drule}_1}
\qquad 
\encoding{\mtl_2}{\nm_1}{\widetilde{\drule}_2}
\\
\widetilde{\drule}{=}[\nmNEW(\interval_1\,{\cup}\,\interval_2) \hornarrow
\m{findall}(\interval_1, \nm_1), \m{findall}(\interval_2, \nm_2)]
\end{matrix}
}{
\encoding{\mtl_1{\vee}\mtl_2}{\nmNEW}{ \widetilde{\drule}_1\,{\cup}\,\widetilde{\drule}_2\,{\cup}\,\widetilde{\drule}}
}
\end{align*}
%\vspace{-1mm}
\begin{align*}
(\history, \timepoint) &\models\mtl_1 \, {\vee} \,\mtl_2 & \m{iff}&~ (\history, \timepoint)\models\mtl_1 ~\m{or}~ (\history, \timepoint)\models\mtl_2 
\end{align*}
\vspace{2mm}
}

In $[\trans\text{-}\m{Disj}]$  with 
$\mtl_1{\vee}\mtl_2$, \\
$\forall \interval.~\llbracket \nmNEW(\interval) \rrbracket_{\Prolog} {=} \m{true}$, 
its premise indicates that \\ 
$\forall \interval_1. \forall \interval_2.~ \nm_1(\interval_1), \nm_2(\interval_2)~ \m{and} ~  \interval {=} \interval_1 \cup \interval_2$. 
Next, from the semantic definition, we have \\
$\forall  \timepoint\,{\in}\,\interval, 
(\history, \timepoint) {\models} \mtl_1 \vee \mtl_2$; thus the rule is sound. \\
From the semantic definition, $\forall   
(\history, \timepoint) {\models} \mtl_1 \vee \mtl_2$, it indicates that $(\history, \timepoint)\models\mtl_1$ or $(\history, \timepoint)\models\mtl_2$.\\
Next, from $[\trans\text{-}\m{Disj}]$, 
$\exists \interval.~\llbracket \nm_1(\interval) \rrbracket_{\Prolog} {=} \m{true}$ or $\llbracket \nm_2(\interval) \rrbracket_{\Prolog} {=} \m{true}$
we obtain  \\
$\llbracket \nmNEW(\interval) \rrbracket_{\Prolog} {=} \m{true}$,  and $\timepoint\,{\in}\,\interval$; 
thus, the rule is complete. 

\item Conjunction:
{
\small 
\begin{align*}
\frac{
\begin{matrix}
[\trans\text{-}\m{Conj}]\\
\encoding{\mtl_1}{\nm_1}{\widetilde{\drule}_1}
\qquad 
\encoding{\mtl_2}{\nm_1}{\widetilde{\drule}_2}
\\
\widetilde{\drule}{=}[\nmNEW(\interval_1\,{\cap}\,\interval_2) \hornarrow
\m{findall}(\interval_1, \nm_1), \m{findall}(\interval_2, \nm_2)]
\end{matrix}
}{
\encoding{\mtl_1{\wedge}\mtl_2}{\nmNEW}{ \widetilde{\drule}_1\,{\cup}\,\widetilde{\drule}_2\,{\cup}\,\widetilde{\drule}}
}
\end{align*}
%\vspace{-1mm}
\begin{align*}
(\history, \timepoint) &\models\mtl_1 \, {\wedge} \,\mtl_2 & \m{iff}&~ (\history, \timepoint)\models\mtl_1 ~\m{and}~ (\history, \timepoint)\models\mtl_2
\\[0.1em]
\end{align*}
\vspace{-2mm}
}

In $[\trans\text{-}\m{Conj}]$  with 
$\mtl_1{\wedge}\mtl_2$, \\
$\forall \interval.~\llbracket \nmNEW(\interval) \rrbracket_{\Prolog} {=} \m{true}$, 
its premise indicates that \\ 
$\forall \interval_1. \forall \interval_2.~ \nm_1(\interval_1), \nm_2(\interval_2)~ \m{and} ~  \interval {=} \interval_1 \cap \interval_2$. 
Next, from the semantic definition, we have \\
$\forall  \timepoint\,{\in}\,\interval, 
(\history, \timepoint) {\models} \mtl_1 \wedge \mtl_2$; thus the rule is sound. \\
From the semantic definition, $\forall   
(\history, \timepoint) {\models} \mtl_1 \wedge \mtl_2$, it indicates that $(\history, \timepoint)\models\mtl_1$ and $(\history, \timepoint)\models\mtl_2$.\\
Next, from $[\trans\text{-}\m{Conj}]$, 
$\exists \interval.~\llbracket \nm_1(\interval) \rrbracket_{\Prolog} {=} \m{true}$ and $\llbracket \nm_2(\interval) \rrbracket_{\Prolog} {=} \m{true}$
we obtain  \\
$\llbracket \nmNEW(\interval) \rrbracket_{\Prolog} {=} \m{true}$,  and $\timepoint\,{\in}\,\interval$; 
thus, the rule is complete. 
\end{enumerate}
\vspace{3mm}

All the encoding rules are sound and complete. 

\end{proof}



\end{document}
% \endinput
%%
%% End of file `sample-authordraft.tex'.
