\section{Conclusion}
\label{sec:conclusion}

In this work, we propose PGMR, a novel SPARQL query generation framework, which incorporates a non-parametric memory module to address the issue of hallucinations in LLM-generated queries.
In this framework, LLMs are responsible for generating intermediate SPARQL queries where KG elements like URIs are represented in natural language and enclosed in special tokens, instead of directly producing full SPARQL queries.
A retriever with a non-parametric memory module is utilized to seamlessly integrate the retrieved URIs into the intermediate query, resulting in the final SPARQL query.
Despite its simplicity, PGMR consistently demonstrates strong performance in SPARQL query generation across a range of datasets, data distribution shifts, and LLMs in both finetuned and few-shot settings.
Remarkably, PGMR leads to a considerable decrease in URI hallucinations, achieving a near-complete elimination of this problem across various instances.



% In this study, we introduce PGMR, a novel SPARQL query generation framework featuring a modular design where LLMs are tasked with generating intermediate queries in SPARQL syntax with natural language forms of KG elements like URIs enclosed in special tokens, instead of directly generating SPARQL queries.
% A retriever with a non-parametric memory module is employed to efficiently incorporate retrieved URIs into the intermediate query to form the final SPARQL query.
% Despite its relative simplicity, PGMR  exhibits consistently high SPARQL query generation performance across challenging datasets, data distributional shifts and LLMs.
% Notably, PGMR achieves significant reductions in URI hallucinations, nearly eliminating this issue altogether.
