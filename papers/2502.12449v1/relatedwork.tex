\section{Related work}
\IEEEpubidadjcol
The skyline detection algorithms can be grouped by two main categories. The first group methods are mainly designed based on feature descriptor and machine learning technologies. These methods are proposed for a long time, so they are regarded as traditional methods. For example, Cornall \textit{et al.}\cite{iet:/content/journals/10.1049/el_20060547}, based on the analysis of the sky RGB channels, employed blue as distinctive feature to extract the boundary of sky and non-sky region. Chiu \textit{et al.} \cite{Chiu} extracted the skyline curve under the brightness and contrasts information. Lie \textit{et al.} \cite{LIE2005221} obtained the edge map from the given image. Subsequently, they constructed the multi-stage map from the edge map for skyline detection. Dong \textit{et al.} \cite{Dong2010} calculated the local maximum grayscale complexity and multiplied it by a parameter which is less than one to obtain a threshold for image binarization. They identified regions with relatively high complexity, such as the sea-to-sky convergence line, and used HT to find the skyline.. Fefilatyev \textit{et al.} \cite{Fefilatyev2006} took the texture and color feature as the input of support vector machine (SVM) to judge pixels whether belong to sky region or not. Ahmad \textit{et al.} \cite{Ahmad2015} proposed a skyline detection algorithm using machine learning and Dynamic Programming (DP) to extract the skyline from the classification map. In this approach, each pixel is assigned a classification score as a likelihood of the pixel belonging to the horizon line, and representing the classification map as a multi-stage graph. Using DP, the horizon line can be extracted by finding the path that maximizes the sum of classification scores. Ahmad \textit{et al.} \cite{Ahmad2021} proposed a bunch of learned filters based on the local structure tensors, and applied these filters to the patches around them for generation of skyline predict score. Generally speaking, the traditional methods require manual design of image feature, and a few labeled data for the training of classifiers or filters. Because of  unreliability of feature design and small amount of model parameters, the robustness of traditional methods are usually unsatisfactory for variety of lightness and weather. 
\par The recent second group methods are mainly based on deep learning technologies, which employs trainable deep neural networks for skyline extraction. For example, Verbickas and Whitehead \cite{Verbickas2014} applied convolutional neural networks (CNNs) to the detection of the horizon, training them to detect the sky and ground regions and the horizon line in flight videos. Poriz \textit{et al.} \cite{Porzi2016} proposed a sky segmentation method based on improved U-Net architecture\cite{Ronneberger2015}, which introduced intermediate levels of supervision to support the learning process. Frajberg \textit{et al.} \cite{Frajberg2017}  presented the results of training a CNN for extracting mountain skylines. Yang \textit{et al.} \cite{Yang2021} detected the sea-sky-line based on the improved YOLOv5 algorithm, replacing MobileNet\cite{2020MobileNets} with SCPDarknet \cite{Bochkovskiy2020} to serve as the backbone. Li \textit{et al.} \cite{Li2024} employed the heatmap as output of deep neural network to extract the skyline. Ahmad \textit{et al.} \cite{Ahmad2017} took a comparison between four sky segment approaches including Edge-less horizontal detection (DCSI) \cite{Ahmad2015Edgeless}, Automatic Labeling Environment (ALE) \cite{Saurer2016}, Fully Convolutional Network (FCN) \cite{Long_2015_CVPR} and SegNet \cite{Badrinarayanan2017}. In this comparing work, the validation test indicated the performance of FCN is best for sky segmentation of mountainous imagery. The deep learning methods are mainly designed based on deep neural networks, which usually contain a large amount of parameters. Their robustness is better than those traditional methods. Simultaneously, the more labeled data is indispensable for the training of deep neural networks.