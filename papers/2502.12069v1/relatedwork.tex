\section{Related Work}
\label{sec: related work}

\subsection{Distributed Consensus:}
\textcolor{black}{Most consensus protocols were initially proposed in reliable communication networks \cite{10.1145/357172.357176,doi:10.1137/0212045,feldman1997optimal,10.1145/42282.42283,184040,lamport2001paxos,castro1999practical,yin2019hotstuff}. They follow classical synchronous or partial synchronous communication model, where there is a known or unknown upper bound on message transmission delays between nodes and all messages are guaranteed to be delivered within this time bound.} As for unreliable link communications such as link loss, some work \cite{doi:10.1137/S009753970443999X, 6312888,1238089} prove the properties of agreement, termination, and validity in designing protocols. \cite{10.1145/3447851.3458739} also models the partial network partition of RAFT and analyzes possible outage caused by possible link loss. However, the analysis often treats links in a deterministic manner, i.e., either faulty or non-faulty. There is few probabilistic analysis of link and consensus reliability. 
In \cite{doi:10.1137/S009753970443999X}, assumption coverage has been proposed as a measure of whether the probability of meeting the required conditions for protocols converges to 1, which is a meaningful measure for assessing the quality of protocols. Nonetheless, this work provides only a vague evaluation of consensus reliability boundaries as the number of nodes approaches infinity and assumes identical link failure rates for different links. 

\subsection{Availability of Quorum System: } Some work analyze the availability of quorum system \cite{Peleg1995TheAO, doi:10.1137/S0097539797325235}. This characterizes the likelihood that a quorum system will be able to deliver service, taking into account the failure probability of various elements within the system. While many consensus algorithms can be represented as quorum-based systems, quorum system availability may not be used to assess the consensus networks reliability, as it usually does not account for communication link failures within the network.



\subsection{Wireless Consensus Applications: }Some recent works have proposed many applications of wireless distributed consensus and analyzed the system performance \cite{9003220, AsheralievaAlia2020RCFf, SeoHyowoon2018CDDS, KimHyesung2020BOFL, LiuYinqiu2019mALB, WangWenbo2019ASoC, ZhangLei2021HMCR, 10041971}. 
\cite{9003220, 10041971, luo2023symbiotic} focus on consensus in critical mission scenarios, e.g., autonomous driving systems wherein vehicles and pedestrians go through intersections, IoT where terminals (drones, sensors, and actuators) act based on coordinated behaviors. \cite{10000789} implements RAFT based on embedded system to make sensing data and action orders consistent in the Internet of Vehicles. 
\cite{XuHao2020RBWB, SeoHyowoon2021CaCC} 
has analyzed the impact of wireless link reliability on consensus reliability and latency, but their models target the geographic distribution of nodes for the specific protocol.
\cite{9184829,9990047, Li} has investigated RAFT consensus reliability in a wireless network to show critical decision-making with different link reliability in IoT. However, the proposed result was only for RAFT.