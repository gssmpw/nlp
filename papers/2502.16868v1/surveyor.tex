\subsection{Online Surveyor}
\label{sec:surveyor}

%The \surveyor~ contains components of \explorer~ and \generator.


\stitle{Exploration}.  The \explorer~ component is designed to give users an intuitive way to interact with the graph database while minimizing the learning curve. %Traditional graph exploration typically relies on query languages, which can require great amount of effort to master. In addition, encountering “supernodes” with exceedingly large numbers of connections can overwhelm users and disrupt the analysis flow.

\begin{figure}
    \centering
    \includegraphics[width=1.0\linewidth]{figures/search.pdf}
    \caption{The Search component of \sys.}
    \label{fig:search_component}
    \vspace*{-2em}
\end{figure}

Traditional graph exploration typically relies on query languages, which can require extra effort to master. We address this by embedding graph queries within interactive UI components. As shown in \reffig{search_component}, the Search module in the \explorer~ helps users pinpoint their initial papers for exploration. Three key interactions are highlighted: ``E1'' searches all nodes containing the ``year'' attribute with a single click; ``E2'' displays a histogram of nodes by ``year'' providing a statistical overview; and ``E3'' filters and retrieves nodes for a specific year (e.g., 2023) by clicking the corresponding histogram bar. These user actions are seamlessly translated into Cypher queries and executed on the underlying graph database.

Furthermore, encountering “supernodes” with exceedingly large numbers of connections can often overwhelm users and disrupt the analysis flow. To address this, we introduce a \statfilter~module that intervenes before displaying all the neighbors. This module can present neighbors either as a histogram, allowing users to quickly overview and multi-select by groups, or as a table, where they can sort by specific attributes and choose the top-k results for further exploration. In \refsec{scenario}, we provide examples showing how this approach streamlines the exploration process.


\stitle{Generation.} Once users finish selecting papers in the \explorer, they can employ the \generator~to convert this explored data into structured reports. By leveraging the natural language understanding and summarization capabilities of LLMs, the \generator~ turns the network of interconnected papers on the canvas into a mind map and, ultimately, a well-formatted narrative report. This process involves three main steps: (1) \textbf{Interpreting User Intentions}: Users describe their desired report in natural language, from which LLM infers which attributes and dimensions of the paper are needed. For instance, if a user asks for a related work section focusing on the paper’s challenges, the LLM may determine that the ``title'' and ``abstract'' attributes and the ``challenges'' dimension are required. Users can review and refine these selections before proceeding. As the dimensions are pre-extracted during the offline \scrapper~ phase, the \generator~ can quickly retrieve them on demand.
(2) \textbf{Generating Mind Maps}: Like a human expert, we prompt the LLM to organize the selected papers into a mind map based on the dimensions mentioned by the users, providing a high-level blueprint for the final report. To accommodate context-size limitations, we adopt an iterative approach that feeds the LLM subsets of the data at a time, gradually constructing the mind map for users to review. % without overwhelming the model.
%The mind map will be shown to the users for checking the correctness and completeness.
(3) \textbf{Writing Reports}: With the mind map in place, the LLM finalizes the literature survey by generating a cohesive report, which can then be downloaded in various formats (e.g., PDF or TeX) to support academic writing.
