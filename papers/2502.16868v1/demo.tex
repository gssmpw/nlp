%%
%% The first command in your LaTeX source must be the \documentclass
%% command.
%%
%% For submission and review of your manuscript please change the
%% command to \documentclass[manuscript, screen, review]{acmart}.
%%
%% When submitting camera ready or to TAPS, please change the command
%% to \documentclass[sigconf]{acmart} or whichever template is required
%% for your publication.
%%
%%
\documentclass[sigconf,9pt,nonacm]{acmart}

\usepackage{xspace}
\usepackage{enumitem}
\usepackage{listings}
\usepackage{tikz}

\newcommand*\circled[1]{\tikz[baseline=(char.base)]{
            \node[shape=circle,draw,inner sep=1pt] (char) {#1};}}

\definecolor{eclipseBlue}{RGB}{42,0.0,255}
\definecolor{eclipseGreen}{RGB}{63,127,95}
\definecolor{eclipsePurple}{RGB}{127,0,85}

\lstset{
  basicstyle=\lst@ifdisplaystyle\fontsize{7}{9.2}\linespread{0.5}\ttfamily
  \fi\ttfamily, % Global Code Style
  extendedchars=true, % Allows 256 instead of 128 ASCII characters
  tabsize=2, % number of spaces indented when discovering a tab
  columns=fullflexible, % fixed to make all characters equal width (but adds too much whitespace!)
%  keepspaces=true, % does not ignore spaces to fit width, convert tabs to spaces
  showstringspaces=false, % lets spaces in strings appear as real spaces
%  breaklines=true, % wrap lines if they don't fit
%  numbers=left, % show line numbers at the left
%  numberstyle=\tiny\ttfamily, % style of the line numbers
  comment=[l]{\#},
  commentstyle=\color{eclipseGreen}, % style of comments
  keywordstyle=\color{eclipsePurple}, % style of keywords
  % stringstyle=\color{eclipseBlue}, % style of strings
  escapeinside={(*}{*)}, % escape symbols
  aboveskip=5pt,
  belowskip=2pt,
  morestring=[b]',
  morestring=[b]",
  frame=tb
}


%%
%% \BibTeX command to typeset BibTeX logo in the docs
\AtBeginDocument{%
  \providecommand\BibTeX{{%
    Bib\TeX}}}

%% Rights management information.  This information is sent to you
%% when you complete the rights form.  These commands have SAMPLE
%% values in them; it is your responsibility as an author to replace
%% the commands and values with those provided to you when you
%% complete the rights form.
%\setcopyright{acmcopyright}
%\copyrightyear{2018}
%\acmYear{2018}
%\acmDOI{XXXXXXX.XXXXXXX}

%% These commands are for a PROCEEDINGS abstract or paper.
\acmConference[Sigmod '25]{International Conference on Management of Data}{June 22--27,
  2025}{Berlin, Germany}
%%
%%  Uncomment \acmBooktitle if the title of the proceedings is different
%%  from ``Proceedings of ...''!
%%
%%\acmBooktitle{Woodstock '18: ACM Symposium on Neural Gaze Detection,
%%  June 03--05, 2018, Woodstock, NY}
%\acmPrice{15.00}
%\acmISBN{978-1-4503-XXXX-X/18/06}


%%
%% Submission ID.
%% Use this when submitting an article to a sponsored event. You'll
%% receive a unique submission ID from the organizers
%% of the event, and this ID should be used as the parameter to this command.
%%\acmSubmissionID{123-A56-BU3}

%%
%% For managing citations, it is recommended to use bibliography
%% files in BibTeX format.
%%
%% You can then either use BibTeX with the ACM-Reference-Format style,
%% or BibLaTeX with the acmnumeric or acmauthoryear sytles, that include
%% support for advanced citation of software artefact from the
%% biblatex-software package, also separately available on CTAN.
%%
%% Look at the sample-*-biblatex.tex files for templates showcasing
%% the biblatex styles.
%%

%%
%% The majority of ACM publications use numbered citations and
%% references.  The command \citestyle{authoryear} switches to the
%% "author year" style.
%%
%% If you are preparing content for an event
%% sponsored by ACM SIGGRAPH, you must use the "author year" style of
%% citations and references.
%% Uncommenting
%% the next command will enable that style.
%%\citestyle{acmauthoryear}

\setlist[itemize]{leftmargin=1.2em, labelsep=0.5em}

%%
%% end of the preamble, start of the body of the document source.
\begin{document}

%%% REVIEW
\newcommand{\tocite}{{\color{red}CITE} }
\newcommand{\toref}{{\color{red}REF} }

%%% LOGO
\newcommand{\usc}{\raisebox{-1pt}{\includegraphics[height=0.8em]{figures/usc_logo.png}}}
\newcommand{\vuam}{\raisebox{-1pt}{\includegraphics[height=0.8em]{figures/vu_logo.png}}}

%%% SIGNS and SYMBOLS
\newcommand{\grad}{\texttt{grad-CROP}}
\newcommand{\att}{\texttt{att-CROP}}
\newcommand{\seg}{\texttt{seg}}
\newcommand{\clip}{\texttt{clip-CROP}}
\newcommand{\sam}{\texttt{sam-CROP}}
\newcommand{\yolo}{\texttt{yolo-CROP}}
\newcommand{\hc}{\texttt{human-CROP}}
\newcommand{\zsvqa}{\texttt{ZSVQA}}
\newcommand{\vic}{\textbf{ViCrop}}
\newcommand{\xmark}{\text{\ding{55}}}
\newcommand{\cmark}{\text{\ding{51}}}
\newcommand{\success}{\texttt{\color{green} \cmark}}
\newcommand{\failure}{\texttt{\color{red} \xmark}}
\newcommand{\rel}{\texttt{rel-att}}
\newcommand{\gra}{\texttt{grad-att}}
\newcommand{\pgra}{\texttt{pure-grad}}
\newcommand{\relh}{\texttt{rel-att$^h$}}
\newcommand{\grah}{\texttt{grad-att$^h$}}
\newcommand{\pgrah}{\texttt{pure-grad$^h$}}


%%% Text Abb.
\makeatletter
\DeclareRobustCommand\onedot{\futurelet\@let@token\@onedot}
\def\@onedot{\ifx\@let@token.\else.\null\fi\xspace}

\def\aka{\emph{a.k.a}\onedot} \def\Eg{\emph{E.g}\onedot}
\def\eg{\emph{e.g}\onedot} \def\Eg{\emph{E.g}\onedot}
\def\ie{\emph{i.e}\onedot} \def\Ie{\emph{I.e}\onedot}
\def\cf{\emph{c.f}\onedot} \def\Cf{\emph{C.f}\onedot}
\def\etc{\emph{etc}\onedot} \def\vs{\emph{vs}\onedot}
\def\wrt{w.r.t\onedot} \def\dof{d.o.f\onedot}
\def\etal{\emph{et al}\onedot}
\makeatletter



\definecolor{myred}{HTML}{FF8577}
\definecolor{mygreen}{HTML}{0FA958}
\definecolor{myblue}{HTML}{1982C4}
\definecolor{codegreen}{rgb}{0,0.5,0}
\definecolor{codegray}{rgb}{0.5,0.5,0.5}
\definecolor{codepurple}{rgb}{0.07,0,0.53}
\definecolor{codered}{RGB}{189,41,0}
\definecolor{codecomment}{RGB}{153,153,153}
\definecolor{backcolour}{rgb}{0.96,0.96,0.96}
\definecolor{royalblue}{rgb}{0.0, 0.14, 0.4}
\definecolor{egyptianblue}{rgb}{0.06, 0.2, 0.65}
\definecolor{royalazure}{rgb}{0.0, 0.22, 0.66}
\definecolor{portlandorange}{rgb}{1.0, 0.35, 0.21}
\definecolor{sienna}{RGB}{183,105,68}
\definecolor{saddlebrown}{RGB}{139,69,19}
\definecolor{mediumbrown}{RGB}{83,41,11}
\definecolor{darkbrown}{RGB}{58,28,7}
\hypersetup{
    colorlinks=true,
    linkcolor=sienna,
    urlcolor=royalblue,
    citecolor=royalblue,
}

%%
%% The "title" command has an optional parameter,
%% allowing the author to define a "short title" to be used in page headers.
\title{Graphy'our Data: Towards End-to-End Modeling, Exploring and Generating Report from Raw Data}

\author{Longbin Lai, Changwei Luo, Yunkai Lou, Mingchen Ju$^{\ddag}$, Zhengyi Yang$^{\ddag}$}
% \authornote{Both authors contributed equally to this research.}
\email{{longbin.lailb, pomelo.lcw, louyunkai.lyk}@alibaba-inc.com}
\email{{mingchen.ju@student., zhengyy.yang@}unsw.edu.au}
\affiliation{%
  %\institution{Alibaba Group}\country{China} \\%
  %\institution{$^{\ddag}$University of New South Wales}\country{Australia}
  \institution{Alibaba Group, China; $^{\ddag}$University of New South Wales, Australia}
  \country{}
}
%%
%% The abstract is a short summary of the work to be presented in the
%% article.
\begin{abstract}
  Large Language Models (LLMs) have recently demonstrated remarkable performance in tasks such as
  Retrieval-Augmented Generation (RAG) and autonomous AI agent workflows. Yet, when faced with large
  sets of unstructured documents requiring progressive exploration, analysis, and synthesis, such as
  conducting literature survey,  existing approaches often fall short. We address this
  challenge -- termed Progressive Document Investigation -- by introducing \sys, an end-to-end platform
  that automates data modeling, exploration and high-quality report generation in a user-friendly manner.
  \sys\ comprises an offline \scrapper that transforms raw documents into a structured
  graph of \fact\ and \dimension\ nodes, and an online \surveyor that enables iterative exploration and
  LLM-driven report generation. We showcase a pre-scrapped graph of over 50,000 papers -- complete
  with their references -- demonstrating how \sys\ facilitates the literature-survey scenario.
  The demonstration video can be found at \url{https://youtu.be/uM4nzkAdGlM}.
  %with focused exploration,
  %analysis, and automatic generation of literature reviews. All code and datasets are open-sourced to
  %encourage further research, adaptation, and community collaboration.
\end{abstract}


%%
%% This command processes the author and affiliation and title
%% information and builds the first part of the formatted document.
\maketitle

\section{Introduction}
\label{sec:intro}

\begin{figure*}[tb]
    \centering
    \includegraphics[width=0.848\linewidth]{figs/circuitnn.pdf} 
    \caption{Illustration of differentiable CircuitNN. CircuitNN is designed based on differentiable NAND gates. After DAS is guided by PI and PO pairs of the truth table, CircuitNN can get the precise circuit architecture logic equivalent to the truth table.}
    \label{fig:circuitnn}
\end{figure*}

% 1. Describe the importance of logic synthesis
% 2. Existing Problems
% (a) Neural Architecture Search: Unstable, Predefined Setting, etc.
% (b) Circuit Generation: Probabilistic Model, Logic Equivalence

With the rapid advancement of technology, the scale of integrated circuits (ICs) has expanded exponentially. 
This expansion has introduced significant challenges in chip manufacturing, particularly concerning power and area metrics.
A primary objective in IC design is achieving the same circuit function with fewer transistors, thereby reducing power usage and area occupancy.

Logic synthesis~\cite{hachtel2005logicsynth}, a critical step in electronic design automation (EDA), transforms behavioral-level circuit designs into optimized gate-level circuits, ultimately yielding the final IC layout. 
The primary goal of logic synthesis is to identify the physical implementation with the fewest gates for a given circuit function. 
This task constitutes a challenging NP-hard combinatorial optimization problem. 
Current logic synthesis tools~\cite{brayton2010abc, wolf2013yosys} rely on human-designed heuristics, often leading to sub-optimal outcomes.

Differentiable architecture search (DAS) techniques~\cite{liu2018darts, chu2020darts} offer novel perspectives on addressing challenges in this problem.
Circuit functions can be represented through truth tables, which map binary inputs to their corresponding outputs. 
Truth tables provide a precise representation of input-output relationships, ensuring the design of functionally equivalent circuits.
Inspired by this, researchers~\cite{deepmind2024ai4sys, wang2024tnet} have begun exploring the application of DAS to synthesize circuits directly from truth tables.
Specifically, \citet{deepmind2024ai4sys} proposed CircuitNN, a framework that learns differentiable connection structures with logic gates, enabling the automatic generation of logic circuits from truth tables.
This approach significantly reduces the complexity of traditional circuit generation. 
Building on this, \citet{wang2024tnet} introduced T-Net, a triangle-shaped variant of CircuitNN, incorporating regularization techniques to enhance the efficiency of DAS.

Despite these advancements, several challenges remain. 
The computational complexity of DAS grows quadratically with the number of gates, posing scalability issues.
Although triangle-shaped architecture~\cite{wang2024tnet} partially mitigates this problem, redundancy persists. 
%Additionally, DAS is susceptible to converging to local optima, limiting the ability to search architectures that satisfy the given truth tables~\cite{liu2018darts}. 
%Furthermore, hyperparameters (network depth and layer width) require extensive searches, introducing complexity and prolonging the synthesis process. 
Additionally, DAS is susceptible to converging to local optima~\cite{liu2018darts} and hyperparameters (network depth and layer width) require extensive searches. 
The challenges arise from the vast search space in DAS. 
% Even with predefined settings for CircuitNN, finding a configuration that meets the truth table requires extensive trial and error during the DAS process. 
Intuitively, limiting the search space through predefined parameters (network depth, gates per layer, and connection probabilities) can significantly reduce the complexity.

Recent advances~\cite{openai2023gpt4, abramson2024alphafold3, esser2024sd3, li2024mar} in conditional generative models have demonstrated remarkable performance across language, vision, and graph generation tasks. 
Motivated by these developments, we propose a novel approach to circuit generation that generates preliminary circuit structures to guide DAS in generating refined circuits matching specified truth tables. 
Firstly, we introduce CircuitVQ, a tokenizer with a discrete codebook for circuit tokenization. 
Built upon our Circuit AutoEncoder framework~\cite{hou2022graphmae,li2023maskgae,wu2025mgvga}, CircuitVQ is trained through a circuit reconstruction task. 
Specifically, the CircuitVQ encoder encodes input circuits into discrete tokens using a learnable codebook, while the decoder reconstructs the circuit adjacency matrix based on these tokens.
Subsequently, the CircuitVQ encoder serves as a circuit tokenizer for CircuitAR pretraining, which employs a masked autoregressive modeling paradigm~\cite{chang2022maskgit, li2023mage}. 
In this process, the discrete codes function as supervision signals. 
After training, CircuitAR can generate discrete tokens progressively, which can be decoded into initial circuit structures by the decoder of the CircuitVQ. 
These prior insights can guide DAS in producing refined circuits that match the target truth tables precisely.

Our key contributions can be summarized as follows:
\begin{itemize}
\item We introduce CircuitVQ, a circuit tokenizer that facilitates graph autoregressive modeling for circuit generation, based on our Circuit AutoEncoder framework;
\item Develop CircuitAR, a model trained using masked autoregressive modeling, which generates initial circuit structures conditioned on given truth tables;
\item Propose a refinement framework that integrates differentiable architecture search to produce functionally equivalent circuits guided by target truth tables;
\item Comprehensive experiments demonstrating the scalability and capability emergence of our CircuitAR and the superior performance of the proposed circuit generation approach.
\end{itemize}

% Motivation
% (a) Diffusion (Vision, Graph), Autoregressive (Language, Vision)
% (b) Circuit Generation for Predefined Setting
% (c) Neural Architecture Search for Strict Logic Equivalence

% Contribution
% (a) Circuit Tokenizer (new transformer arch, training strategy)
% (b) CircuitAR (train and gen strategies, post-ar strategy)
% (c) Extensive Evaluation including BitD (Bit Distance) for Scalability

\begin{figure}[t!]
    \newdimen\base
    \base=0.5cm
    \newdimen\xsep
    \xsep=0.3cm

    \tikzstyle{common} = [font=\scriptsize]
    \tikzstyle{split2} = [rectangle split, rectangle split parts=2, draw=black, inner sep=0pt, outer sep=0pt, minimum width=\base, minimum height=\base, inner ysep=0.5\base, rounded corners=2pt]
    
    \centering
    \begin{tikzpicture}
    
    \node[common, inner sep=0pt] (input) at (0,0) {$\nabla_{W_i}$};
    \node[common, split2, anchor=west, fill=lyyblue!40!white] (decomposein) at ([xshift=\xsep]input.east) {\rotatebox{-90}{$u_i$} \nodepart{two} \rotatebox{-90}{$\delta_{i+1}$}};
    \node[common, circle, draw=black, inner sep=1pt, anchor=west, fill=orange!30!white] (norm) at ([xshift=\xsep]decomposein.east) {norm};
    \node[common, split2, anchor=west, fill=lyyblue!40!white] (normout) at ([xshift=\xsep]norm.east) {\rotatebox{-90}{$\bar{u}_i$} \nodepart{two} \rotatebox{-90}{$\bar{\delta}_{i+1}$}};
    \node[common, trapezium, trapezium angle=30, rotate=-90, draw=black, anchor=south, fill=ugreen!30!white, minimum height=\base, minimum width=3\base, inner sep=0pt, outer sep=0pt, trapezium stretches, text width=2.5\base, align=center] (down) at ([xshift=\xsep]normout.east) {Down};
    \node[common, rectangle, draw=black, anchor=west, fill=red!30!white, minimum height=2.5\base] (hidden) at ([xshift=\xsep]down.north) {\rotatebox{-90}{Dropout}};
    \node[common, trapezium, trapezium angle=30, rotate=90, draw=black, anchor=north, fill=ugreen!30!white, minimum height=\base, minimum width=3\base, inner sep=0pt, outer sep=0pt, trapezium stretches, text width=2.5\base, align=center] (up) at ([xshift=\xsep]hidden.east) {Up};
    \node[common, anchor=west, inner sep=0pt] (residual) at ([xshift=0.7\xsep]up.south) {$\bigoplus$};
    \node[common, split2, anchor=west, fill=lyyblue!40!white] (decomposeout) at ([xshift=0.7\xsep]residual.east) {\rotatebox{-90}{$\tilde{u}_i$} \nodepart{two} \rotatebox{-90}{$\tilde{\delta}_{i+1}$}};
    \node[common, inner sep=0pt, anchor=west] (out) at ([xshift=\xsep]decomposeout.east) {$\tilde{\nabla}_{W_i}$};

    \draw[-latex] (input) to (decomposein);
    \draw[-latex] (decomposein) to (norm);
    \draw[-latex] (norm) to (normout);
    \draw[-latex] (normout) to (down);
    \draw[-latex] (down) to (hidden);
    \draw[-latex] (hidden) to (up);
    \draw[-latex] (up) to (residual);
    \draw[-latex] (residual) to (decomposeout);
    \draw[-latex] (decomposeout) to (out);

    \draw[-latex] (normout.north) |- ($(normout.north) + (0,0.5\base)$) -| (residual.north);
    
    \end{tikzpicture}
    \caption{The model architecture of \method{}.}
    \label{fig:arch}
\end{figure}
\subsection{Online Surveyor}
\label{sec:surveyor}

%The \surveyor~ contains components of \explorer~ and \generator.


\stitle{Exploration}.  The \explorer~ component is designed to give users an intuitive way to interact with the graph database while minimizing the learning curve. %Traditional graph exploration typically relies on query languages, which can require great amount of effort to master. In addition, encountering “supernodes” with exceedingly large numbers of connections can overwhelm users and disrupt the analysis flow.

\begin{figure}
    \centering
    \includegraphics[width=1.0\linewidth]{figures/search.pdf}
    \caption{The Search component of \sys.}
    \label{fig:search_component}
    \vspace*{-2em}
\end{figure}

Traditional graph exploration typically relies on query languages, which can require extra effort to master. We address this by embedding graph queries within interactive UI components. As shown in \reffig{search_component}, the Search module in the \explorer~ helps users pinpoint their initial papers for exploration. Three key interactions are highlighted: ``E1'' searches all nodes containing the ``year'' attribute with a single click; ``E2'' displays a histogram of nodes by ``year'' providing a statistical overview; and ``E3'' filters and retrieves nodes for a specific year (e.g., 2023) by clicking the corresponding histogram bar. These user actions are seamlessly translated into Cypher queries and executed on the underlying graph database.

Furthermore, encountering “supernodes” with exceedingly large numbers of connections can often overwhelm users and disrupt the analysis flow. To address this, we introduce a \statfilter~module that intervenes before displaying all the neighbors. This module can present neighbors either as a histogram, allowing users to quickly overview and multi-select by groups, or as a table, where they can sort by specific attributes and choose the top-k results for further exploration. In \refsec{scenario}, we provide examples showing how this approach streamlines the exploration process.


\stitle{Generation.} Once users finish selecting papers in the \explorer, they can employ the \generator~to convert this explored data into structured reports. By leveraging the natural language understanding and summarization capabilities of LLMs, the \generator~ turns the network of interconnected papers on the canvas into a mind map and, ultimately, a well-formatted narrative report. This process involves three main steps: (1) \textbf{Interpreting User Intentions}: Users describe their desired report in natural language, from which LLM infers which attributes and dimensions of the paper are needed. For instance, if a user asks for a related work section focusing on the paper’s challenges, the LLM may determine that the ``title'' and ``abstract'' attributes and the ``challenges'' dimension are required. Users can review and refine these selections before proceeding. As the dimensions are pre-extracted during the offline \scrapper~ phase, the \generator~ can quickly retrieve them on demand.
(2) \textbf{Generating Mind Maps}: Like a human expert, we prompt the LLM to organize the selected papers into a mind map based on the dimensions mentioned by the users, providing a high-level blueprint for the final report. To accommodate context-size limitations, we adopt an iterative approach that feeds the LLM subsets of the data at a time, gradually constructing the mind map for users to review. % without overwhelming the model.
%The mind map will be shown to the users for checking the correctness and completeness.
(3) \textbf{Writing Reports}: With the mind map in place, the LLM finalizes the literature survey by generating a cohesive report, which can then be downloaded in various formats (e.g., PDF or TeX) to support academic writing.


\begin{figure*}[t]
  \centering
  \includegraphics[width=0.78\linewidth,height=8cm]{figures/scenario.pdf}
  \caption{The demonstration scenario of literature survey of \sys.}
  \label{fig:scenario}
  \vspace*{-1em}
\end{figure*}

\section{Demonstrating Literature Survey}
\label{sec:scenario}
We demonstrate how \sys~ applies to literature surveys, with emphasis on the online \surveyor.
We will showcase the functionalities of \scrapper using a video as it is time-consuming. %For the offline \scrapper, we allow attendees to define a workflow following our guidance, and upload their own papers for extracting the specified data and fetching references.

%\stitle{Offline Scrapper}.

The online \surveyor, shown in \reffig{scenario}, allows the demo attendees to explore a pre-extracted paper network containing over 50,000 papers and 160,000 references. We first look into \reffig{scenario}(a) that is the interface of \explorer~  featuring three primary canvases, metaphorically referred to as ``Past'', ``Present'', and ``Future''. Here, ``Past'' displays already explored papers, and ``Present'' shows the currently active papers for reviewing in detail, while ``Future'' highlights the immediate neighbors (i.e., references) of the active papers.
For exploring the papers, the attendee \circled{1} searches for seed papers whose titles contain “Llama3” using the \search~ module; \circled{2} then selects “The Llama 3 Herd of Models” and moves it to the ``Present'' canvas to review its details. Next, \circled{3} the attendee explores the selected paper’s references by pre-querying its neighbors. As described in \refsec{surveyor}, these neighbors are not immediately added to the canvas to avoid overwhelming the user; instead, \circled{4} the \statfilter~ module presents a histogram or table view, allowing attendees to focus on aggregated groups or order the data and finally, \circled{5} decide from the top-k papers for further exploration. By doing so, these papers are added to the ``Present''canvas, while the previously active papers move to the ``Past'' canvas. By iteratively following this workflow, attendees  can explore as many papers as needed, before proceeding to the \generator~task.

In \reffig{scenario}(b), \circled{6} attendees click to input instructions for the report, e.g., ``Please write me a related work, focusing on their challenge''. Based on this input, an LLM (QWen-Plus~\cite{tongyi} for this demo) identifies the relevant attributes and \dimension~ nodes needed for the report, which are \circled{7} displayed for user verification and possible modification. In the example, the LLM highlights the ``Challenge'' node as well as the ``title'' and ``abstract'' attributes from the selected papers. \circled{8} These data are then passed to the LLM to produce a mind map, effectively categorizing the papers according to the identified ``Challenge''.  \circled{9} Attendees can review the mind map, and \circled{10} proceed to final report generation. The final report is built from the mind map and the user’s instructions, culminating in a point-by-point narrative. Once completed, \circled{11} attendees can download the report in PDF or TeX format, complete with citations.

\section{Extension to Financial Scenarios}
\label{sec:others}
We briefly discuss applying \sys\ to two financial scenarios.
 %leveraging the same concepts of \fact\ and \dimension\ nodes.

\stitle{Company Relationship Analysis}. In this scenario, each company is treated as a \fact~ node,
and the data extracted by \inspector, such as revenues, main business areas and shareholder holdings extracted from financial reports, are represented as \dimension~ nodes.
The \navigator~ component establishes inter-company relationships by leveraging the financial or supply-chain dependencies mentioned in the reports. The generated graph can be used to identify comparable competitors, uncover hidden relationships, or assess contagion effects.
\eat{
Several compelling use cases are enabled:
\begin{itemize}
\item Identifying Comparable Competitors: By clustering or filtering companies with similar profiles (e.g., market segment, growth patterns),
investors can discover promising but less-publicized alternatives to popular stocks. This approach helps avoid the inflated valuations often
found in trending companies and can mitigate investment risk.
%\item Uncovering Hidden Relationships: With company \fact~ nodes linked by shared financial patterns, \sys~ can detect correlations
%such as highly synchronized profitability cycles or overlapping shareholder structures. Analysts can thus reveal nuanced market
%relationships or latent financial risks that may not be apparent through conventional methods.
\item Assessing Contagion Effects: In the event of a major negative incident (e.g., a significant default or bankruptcy), \sys~
highlights the interconnections that might amplify its impact. For instance, after a large real estate firm files for bankruptcy,
the tool can identify which companies share financial dependencies, supply chain relationships, or shareholder overlap—and therefore
might be exposed to heightened risk.
\end{itemize}
}

\stitle{Financial News Analysis}. In this scenario, each news article serves as a \fact~ node, while pertinent details, such as described events and stock performance indicators, can act as \dimension~ nodes. The \navigator~ builds connections among these \fact~ nodes by identifying shared symbols or overlapping financial metrics. This allows analysts to track the evolution of news stories, assess their market impact, or predict future trends based on historical patterns.

\eat{
\sys~ yields several valuable use cases:
\begin{itemize}
	\item News Clustering and Trend Analysis: Since a single news article may not provide sufficient insight on its own, \sys~ can group related articles to reveal broader patterns. Analysts can thus identify recurring themes or correlated market shifts across multiple sources, improving the reliability of forecasts and decisions.
  \item Assessing the Market Impact of Individual News Items: By referencing historically similar news events and their corresponding market responses, \sys~ enables users to evaluate how sudden developments might affect future market movements. This allows for timely, data-driven judgment on whether a new piece of news is likely to have a major or minor market impact.
\end{itemize}
}



%%
%% The next two lines define the bibliography style to be used, and
%% the bibliography file.
\bibliographystyle{ACM-Reference-Format}
\bibliography{demo}


\end{document}
\endinput
%%
%% End of file `sample-sigconf.tex'.
