
\begin{figure*}[t]
  \centering
  \includegraphics[width=0.78\linewidth,height=8cm]{figures/scenario.pdf}
  \caption{The demonstration scenario of literature survey of \sys.}
  \label{fig:scenario}
  \vspace*{-1em}
\end{figure*}

\section{Demonstrating Literature Survey}
\label{sec:scenario}
We demonstrate how \sys~ applies to literature surveys, with emphasis on the online \surveyor.
We will showcase the functionalities of \scrapper using a video as it is time-consuming. %For the offline \scrapper, we allow attendees to define a workflow following our guidance, and upload their own papers for extracting the specified data and fetching references.

%\stitle{Offline Scrapper}.

The online \surveyor, shown in \reffig{scenario}, allows the demo attendees to explore a pre-extracted paper network containing over 50,000 papers and 160,000 references. We first look into \reffig{scenario}(a) that is the interface of \explorer~  featuring three primary canvases, metaphorically referred to as ``Past'', ``Present'', and ``Future''. Here, ``Past'' displays already explored papers, and ``Present'' shows the currently active papers for reviewing in detail, while ``Future'' highlights the immediate neighbors (i.e., references) of the active papers.
For exploring the papers, the attendee \circled{1} searches for seed papers whose titles contain “Llama3” using the \search~ module; \circled{2} then selects “The Llama 3 Herd of Models” and moves it to the ``Present'' canvas to review its details. Next, \circled{3} the attendee explores the selected paper’s references by pre-querying its neighbors. As described in \refsec{surveyor}, these neighbors are not immediately added to the canvas to avoid overwhelming the user; instead, \circled{4} the \statfilter~ module presents a histogram or table view, allowing attendees to focus on aggregated groups or order the data and finally, \circled{5} decide from the top-k papers for further exploration. By doing so, these papers are added to the ``Present''canvas, while the previously active papers move to the ``Past'' canvas. By iteratively following this workflow, attendees  can explore as many papers as needed, before proceeding to the \generator~task.

In \reffig{scenario}(b), \circled{6} attendees click to input instructions for the report, e.g., ``Please write me a related work, focusing on their challenge''. Based on this input, an LLM (QWen-Plus~\cite{tongyi} for this demo) identifies the relevant attributes and \dimension~ nodes needed for the report, which are \circled{7} displayed for user verification and possible modification. In the example, the LLM highlights the ``Challenge'' node as well as the ``title'' and ``abstract'' attributes from the selected papers. \circled{8} These data are then passed to the LLM to produce a mind map, effectively categorizing the papers according to the identified ``Challenge''.  \circled{9} Attendees can review the mind map, and \circled{10} proceed to final report generation. The final report is built from the mind map and the user’s instructions, culminating in a point-by-point narrative. Once completed, \circled{11} attendees can download the report in PDF or TeX format, complete with citations.

\section{Extension to Financial Scenarios}
\label{sec:others}
We briefly discuss applying \sys\ to two financial scenarios.
 %leveraging the same concepts of \fact\ and \dimension\ nodes.

\stitle{Company Relationship Analysis}. In this scenario, each company is treated as a \fact~ node,
and the data extracted by \inspector, such as revenues, main business areas and shareholder holdings extracted from financial reports, are represented as \dimension~ nodes.
The \navigator~ component establishes inter-company relationships by leveraging the financial or supply-chain dependencies mentioned in the reports. The generated graph can be used to identify comparable competitors, uncover hidden relationships, or assess contagion effects.
\eat{
Several compelling use cases are enabled:
\begin{itemize}
\item Identifying Comparable Competitors: By clustering or filtering companies with similar profiles (e.g., market segment, growth patterns),
investors can discover promising but less-publicized alternatives to popular stocks. This approach helps avoid the inflated valuations often
found in trending companies and can mitigate investment risk.
%\item Uncovering Hidden Relationships: With company \fact~ nodes linked by shared financial patterns, \sys~ can detect correlations
%such as highly synchronized profitability cycles or overlapping shareholder structures. Analysts can thus reveal nuanced market
%relationships or latent financial risks that may not be apparent through conventional methods.
\item Assessing Contagion Effects: In the event of a major negative incident (e.g., a significant default or bankruptcy), \sys~
highlights the interconnections that might amplify its impact. For instance, after a large real estate firm files for bankruptcy,
the tool can identify which companies share financial dependencies, supply chain relationships, or shareholder overlap—and therefore
might be exposed to heightened risk.
\end{itemize}
}

\stitle{Financial News Analysis}. In this scenario, each news article serves as a \fact~ node, while pertinent details, such as described events and stock performance indicators, can act as \dimension~ nodes. The \navigator~ builds connections among these \fact~ nodes by identifying shared symbols or overlapping financial metrics. This allows analysts to track the evolution of news stories, assess their market impact, or predict future trends based on historical patterns.

\eat{
\sys~ yields several valuable use cases:
\begin{itemize}
	\item News Clustering and Trend Analysis: Since a single news article may not provide sufficient insight on its own, \sys~ can group related articles to reveal broader patterns. Analysts can thus identify recurring themes or correlated market shifts across multiple sources, improving the reliability of forecasts and decisions.
  \item Assessing the Market Impact of Individual News Items: By referencing historically similar news events and their corresponding market responses, \sys~ enables users to evaluate how sudden developments might affect future market movements. This allows for timely, data-driven judgment on whether a new piece of news is likely to have a major or minor market impact.
\end{itemize}
}
