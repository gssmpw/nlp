\section{Related Studies}
At present, the a significant amount of studies on eHMI focus on interactions between pedestrians and autonomous vehicles (AVs)~\citep{chang2017eyes, Clercq2019eHMI,  loew2022go, lee2022learning, dey2021communicating}. 
Only a limited number of studies have focused on interactions between small types of smart mobility vehicles, such as APMVs, with pedestrians in shared spaces using eHMIs~\citep{zhang2024shared,liehr2024you,liu2024_APMV_eHMI}.
These studies have drawn conclusions about the influence of eHMI usage on pedestrians' subjective evaluations and behaviors by comparing different experimental conditions, such as various eHMIs and driving behavior scenarios.
For example, clear information from the eHMI could help pedestrians understand the vehicle's intention~\citep{liu2021importance,dey2021communicating,liehr2024you}, enhanced sense of safety~\citep{faas2020external,liu2021importance,Clercq2019eHMI}, allows pedestrians to trust the AV~\citep{faas2020external, liu2021importance,dey2021communicating}, reduce decision-making time and hesitation~\citep{chang2017eyes,liu2021importance} and improve the crossing initial time~(CIT)~\citep{loew2022go,lee2022learning}.



However, during interactions between pedestrians and AVs or APMVs equipped with eHMIs, the causal relationships, \ie the causal paths and the causal effects,  and causal process among various psychological states of pedestrians and their walking behaviors, remain insufficiently validated or explored.
Furthermore, the examination of interactions involving pedestrians and APMVs is even less prevalent in existing research.


A preliminary work of this study has been published in \citep{liu2024causal}.
In this preliminary work, we invited 18 participants as pedestrians to interact with the APMV. 
Through causal discovery, we found some direct causal relationships between some factors of subjective evaluations and walking behaviors.
However, due to the small number of participants, some direct causal relationships of subjective factors, such as causal effects related to trust and perceived danger, are not statistically stable and have low reproducibility via bootstrap.