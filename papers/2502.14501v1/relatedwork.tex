\section{Related Work}
\label{sec:relatedwork}

\paragraph{Surveys of Computational Argumentation}

In the last 20 years, the field of CA has witnessed a constant development driven by the potential for real world applications, but also by the increasingly interdisciplinary shape that the field has assumed. The number of  surveys on CA is a clear sign of this progress: ranging from foundational work that set up or updated the conceptual coordinates for the field \cite{peldszus2013argument,stede_argumentation_2019,lawrence2020survey,lauscher2022survey}, to surveys with a data-driven focus \cite{cabrio2018survey,SchaeferStede+2021+45+58}, to specific advances in NLP, such as generation \citep{wang2023survey}. 

Specifically for AQ, \citet{wachsmuth-etal-2017-computational} introduced a first holistic systematization of the field according to AQ dimensions. \citet{wachsmuth-etal-2024-argument} update the survey, taking into account the challenges and potentials for the employment of large language models (LLMs) in AQ assessment. While focused on applications of CA for social good as a whole, the survey by \citet{vecchi-etal-2021-towards} puts a strong interdisciplinary focus on AQ and its interface with deliberation quality.

\textit{No survey so far has targeted a systematic categorization of datasets}.%
\footnote{Dataset repositories for CA do exist, i.e., \href{https://github.com/acidrobin/arglu-repo}{\textit{ARGLU}} and the \href{https://webis.de/data.html\#other-corpora}{Webis database}, but are by no means comprehensive and lack our conceptual categorization and focus on annotators.}
This is the gap we fill:%
\footnote{While this paper was under review, a survey on AQ was published by \citet{ivanova-etal-2024-lets}, highlighting the timeliness of this topic. The authors examined the state of AQ research in general, whereas we focus on its future transition into a task where human label variation plays a significant role. Due to a different search strategy, our survey covers a broader range of datasets (103 compared to 32), and offers a more in-depth analysis (with 32 manually annotated meta-categories, compared to 10). Crucially, we adopt a timely interdisciplinary taxonomy that integrates argument and deliberation quality, which \citet{ivanova-etal-2024-lets} also hint at for future research.} 
we survey {datasets}, focusing on the {consolidation of covered AQ categories into an overarching taxonomy}, and who the {annotators} are.


\paragraph{Dimensions of Argument Quality}
\label{subsec:AQ}

\citet{wachsmuth-etal-2017-computational}'s taxonomy is the most commonly adopted one for AQ assessment. Rooted in argumentation theory, it emphasizes three aspects:

\textit{Logical cogency} and its subcategories promote a valid reasoning process at the level of individual arguments. An argument is considered cogent if its premises are rationally worthy of being believed to be true (\textit{local acceptability}), its premises contribute to the acceptance or rejection of its conclusion (\textit{local relevance}), and if they provide enough support to make the conclusion rational (\textit{local sufficiency}).

\textit{Rhetorical effectiveness} and its subcategories mirror the persuasive power of an author's argument towards a target audience. Characteristics are a clear style (\textit{clarity}), maintaining a tone appropriate to the issue (\textit{appropriateness}), presenting components of the argument in a proper order (\textit{arrangement}), establishing the author's credibility (\textit{credibility}), and evoking emotions that make the audience more receptive (\textit{emotional appeal}).

\textit{Dialectical reasonableness} and its subcategories evaluate the contribution to resolving differences of opinions on a discussion level. Argumentation is deemed reasonable if the consideration and presentation of the arguments put forward for the issue are acceptable to the target audience (\textit{global acceptability}), contribute to the issue's resolutions (\textit{global relevance}), and adequately rebut the contestable counterarguments (\textit{global sufficiency}).

\citet{vecchi-etal-2021-towards} proposed to include \textit{deliberative norms} as a further aspect of AQ. This dimension incorporates democratic values into the dialectical view, adherence to which is particularly relevant to political arguments, but also applies to broader contexts like online communication. While the authors resorted specifically to the \emph{Discourse Quality Index} \cite{steenbergen2003dqi}, communication science has come up with various instruments to empirically measure deliberation quality (e.g., \citealt{stromer-galley2011deliberation, black2011deliberation, graham2003deliberation}). 

The exact criteria of (good) deliberation and consequently the instruments for measuring it are matter of controversial discussion \cite{dellicarpini2003deliberation}. \citet{friess2015deliberation} identified seven dimensions that are prevalent across various frameworks: Deliberative discourse should be an exchange grounded in \textit{rationality}. The exchange should take place through listening, understanding and actively responding to each other's opinions in a substantive way (\textit{interactivity}). Furthermore, deliberation should foster \textit{equality} by equipping all sides with the same opportunity to participate in the discussion and \textit{civility} for a respectful interaction. Arguments should be oriented towards the \textit{common good} of the community, and \textit{constructive} in finding a consensus decision for the issue of discussion. The last dimension relates to the use of \textit{alternative forms of communication} (e.g., storytelling).


\paragraph{Perspectivism and Argument Quality}

AQ assessment is a prime example of a subjective task: beyond logical well-formedness (and even there), the question of good arguments is bound to be answered in conflicting ways by annotators with different features (e.g., socio-demographics, life experiences, personality, and values) \cite{lukin-etal-2017-argument,durmus-cardie-2019-corpus,el-baff-etal-2020-analyzing}. This makes AQ an ideal perspectivist topic. 

Yet, \textit{perspectivist AQ assessment is only at its beginning, also because of the need for suitable data}. As datasets will be reviewed in the remainder of the paper, we focus here on the few works that have specifically targeted the {modeling of annotator perspectives} in AQ, i.e., by integrating label variation in the machine learning workflow. The first explicit step was taken by \citet{romberg-2022-perspective}, who predicted the subjectivity of the annotation as an indicator for trustworthiness of majority vote models. What is more, \citet{heinisch-etal-2023-architectural} compared approaches for modeling annotator-specific behavior.