\section{Introduction}
\label{sec:introduction}

The question of ``what makes an argument good'' is at the core of computational argumentation (CA), the area of natural language processing (NLP) dealing with the mining, assessment, and generation of arguments \cite{stede_argumentation_2019, lauscher2022survey}. While rooted in well-established theories, argument quality (AQ) still exhibits a high degree of subjectivity in perception. This degree may vary across quality aspects; for example, evaluating an argument's benefit in agreement-seeking discussions is considered less subjective than assessing its effectiveness \cite{wachsmuth-etal-2017-computational}.\,\,

The CA community is aware of the variance in annotators' perception \cite{stab-gurevych-2014-annotating, teruel-etal-2018-increasing, hautli-janisz-etal-2022-disagreement}. Not least, this is documented by the generally moderate inter-annotator agreement in AQ annotations --- a widely accepted condition that authors commonly attribute to the subjective nature of the task (e.g., \citealt{wachsmuth-etal-2017-computational, Gretz_Friedman_Cohen-Karlik_Toledo_Lahav_Aharonov_Slonim_2020, ng-etal-2020-creating, ziegenbein-etal-2023-modeling}).%
\footnote{Certainly, not all disagreement is due to subjectivity. We refer the reader to the Limitations section for a discussion.}
\citet{wachsmuth-werner-2020-intrinsic} explicitly question whether an aggregated ground truth is suitable to model AQ.

Meanwhile, the NLP community has started to undergo a fundamental change in the way it deals with subjective tasks. While aggregated ground truth and an according model alignment were long standard, more recent work calls for this course to be reconsidered \cite{basile2020end,plank-2022-problem,cabitza_toward_2023,Frenda2024}: Rather than eradicating any existence of annotator disagreement, the \emph{perspectivist turn} embraces the co-existence of perspectives \cite{uma-etal-2021-semeval, davani-etal-2022-dealing, leonardelli-etal-2023-semeval}. This transformation implies the acceptance of variations in data annotation (through non-aggregated datasets) as well as the consideration of heterogeneity in modeling and evaluation \cite{uma-2021-survey, basile-etal-2021-need, plank-2022-problem}.

We postulate that the perspectivist turn in NLP lends itself as a natural solution to face the issue of subjectivity in modeling AQ. Not only does it have a better shot at promoting diversity and fairness in AQ assessment, such as allowing for valid but minority voices \cite{noble2012minority, prabhakaran-etal-2021-releasing}. It is also likely to be more robust in modeling perceptions of AQ across (changing) societies (e.g., today’s minority groups may become tomorrow’s majority) and target audiences.

Yet, the perspectivist turn so far had only minimal impact on AQ. Presumably, one reason for the limited modeling of perspectives in AQ is the lack of datasets designed for this purpose. Preference has been given to aggregated annotations, whereas individual labeling decisions were often not communicated (e.g., \citealt{persing-ng-2013-modeling, park-cardie-2018-corpus, toledo-etal-2019-automatic, ijcai2022p575}). While a solution may be new datasets, the annotation of argumentation phenomena is highly complex and costly. We therefore deem it essential to first \textit{gain an overview of the options that existing datasets already offer for developing perspectivist models}. This is the goal of the paper at hand. 

We provide a systematic literature review of 103 AQ {datasets and their properties}.% 
\footnote{The resulting database can be accessed publicly here: \href{https://github.com/juliaromberg/perspectivist-turn-aq}{https://github.com/juliaromberg/perspectivist-turn-aq}}
Crucially, to support the perspectivist turn, our collection includes meta-information about annotators and the availability of non-aggregated annotations. While only 24 datasets come with the latter, 14 of them seem relevant to the perspectivist turn. In a pilot study, we conduct a statistical analysis of the disagreement patterns in four of them. We conclude by highlighting the opportunities of available datasets and discuss challenges, for example a lack of transparency and socio-demographic diversity.

\paragraph{Contributions}
(1) We release an extensive database with 32 types of meta-information about 103 AQ datasets. (2) We review the multitude of annotated AQ categories %, as described in the papers, guidelines when available, or looking into examples 
(\textit{what is annotated}) and consolidate them in an overarching taxonomy to foster comparability and interoperability.
(3) We perform a comprehensive meta-analysis of annotators (\textit{who annotates}) across the datasets, uncovering a lack of transparency and socio-demographic diversity, promoting bias in AQ datasets and models.
(4) We deep-dive into the 24 datasets with non-aggregated labels both qualitatively and quantitatively, and discuss their potential for a perspectivist turn in AQ.
