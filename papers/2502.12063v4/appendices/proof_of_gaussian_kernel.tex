\section{\pcref{cor:gaussian_mmd}}\label{proof:gaussian_mmd}
\newcommand{\bconstant}{b} %

%
\cref{cor:gaussian_mmd} follows immediately from the following explicit, non-asymptotic bound. 

\begin{corollary}[\tbf{Detailed Gaussian MMD of \kh}]\label{cor:gaussian_mmd_detailed}
If $\xin \subset \ball^d (R)$ for $R>0$, then $\kh(\delta)$ with $\kernel=\textsc{Gauss}(\eta)$, $n=\nin \geq (2e)^d$, and $\bconstant \defeq \frac{1}{2}$ 
%
delivers
%
\begin{talign}\label{eq:gaussian_mmd_ball_detailed}
\mmd_{\mkernel}^2(\pin,\qout)  
    &\leq 
\frac{1}{\nout^2} \log(\frac{4\nout}{\delta}) \brackets{ e^2 \max\braces{\brackets{ \frac{2e}{d} \log\parenth{ \nin \nout \bconstant}}^d, (\frac{R^2\eta e^3 4}{d})^{d}} + e\log(\frac{1}{\delta'})} + \frac{1}{\nout\bconstant}(\frac{1}{\nout}-\frac{1}{\nin})
%
%
\end{talign}
with probability at least $1-\delta/2-\delta'$.
\end{corollary}
\begin{proof}
Consider the approximate rank parameter 
\begin{talign}\label{eq:b}
    r^\star &\defeq \max\braces{\brackets{ \frac{2e}{d} \log\parenth{ \nin \nout \bconstant}}^d, (R^2\eta e^3 4 /d)^{d}}.
\end{talign}
The assumption $\nin \geq (2e)^d$ and the fact that 
$b \geq 1/(2^d \nout)$ 
ensure that $\log(\nin\nout b) \geq d + \log(\nout b/2^d) \geq d$ and therefore that $r^\star\geq (2e)^d$.  
%
Hence, by \citet[Thm.~3]{altschuler2019massivelyscalablesinkhorndistances}, 
the $(r^\star+1)$-th eigenvalue of $\K$ satisfies
\begin{talign}
    \lambda_{r^\star+1} &\leq \nin \exp\braces{-\frac{d}{2e} \max\braces{\frac{2e}{d}\log(\nin\nout b),(R^2 \eta e^3 4 /d)}\log\parenth{\frac{d \max\braces{\frac{2e}{d}\log(\nin\nout b),(R^2 \eta e^3 4 /d)}}{4 e^2 \eta R^2}}} \\
    &\leq \nin \exp\braces{ - \log(\nin\nout b) \log(e) }
    \leq \nin \parenth{ \frac{1}{\nin \nout \bconstant}} = \frac{1}{\nout \bconstant}.
\end{talign}
Since $\maxnorm{\K} =1$ and $\khd \in \ksubg$ with $\subg$ defined in \cref{khd-sub-gaussian}, the result now follows  from  
\cref{thm:subg_low_rank_gen_kernel}.
%
%
%
%
%
%
%
%
%
%
%
%
%
%
%
%
%
%
%

%
%
%
%
%
%
%
%
%
%
%
%
%
%
%
%
%
%
%
%
%
%

%
%
%
%
%
%
%
%
%
%
%
%
%
%
%
\end{proof}

%
\section{\pcref{cor:gaussian_mmd_manifold}}
\label{proof:gaussian_mmd_manifold}

\begin{assumption}[$d^\star$-manifold with $Q$-smooth atlas {\citep[Assum.~1]{altschuler2019massivelyscalablesinkhorndistances}}]\label{assum:manifold}
Let $\Omega \subset \R^d$ be a smooth compact manifold without boundary of dimension $d^\star < d$. Let $(\mbf \Psi_j,U_j)_{j\in [T]}$ for $T\in \naturals$ be an atlas for $\Omega$, where $(U_j)_j$ are open sets covering $\Omega$ and $\mbf \Psi_j: U_j \mapsto \ball^{d^\star}(r_j)$ are smooth maps with smooth inverses, mapping $U_j$ bijectively to $\ball^{d^\star}(r_j)$. Assume that there exists $Q>0$ such that $\sup_{u\in \ball^{d^\star}(r_j)} \norm{D^{\boldsymbol \alpha} \mbf \Psi_j\inv (u)} \leq Q^{\abss{\boldsymbol \alpha}}$ for all $\boldsymbol \alpha\in \naturals^{d^\star}$ and $j\in [T]$, where $\abss{\boldsymbol \alpha} \defeq \sum_{j=1}^{d^\star} \alpha_j$ and $D^{\boldsymbol \alpha} = \frac{\partial^{\abss{\boldsymbol \alpha}}}{\partial u_1^{\alpha_1}\ldots \partial u_{d^\star}^{\alpha_{d^\star}}}$ for $\boldsymbol \alpha\in \naturals^{d^\star}$.
\end{assumption}

%
\cref{cor:gaussian_mmd_manifold} follows immediately from the following more detailed result.

\begin{corollary}[\tbf{Detailed Intrinsic Gaussian MMD of \kh}]\label{cor:gaussian_mmd_manifold_detailed}
Suppose $\xin$ lies on a manifold $ \Omega \subset \ball^d$ satisfying \cref{assum:manifold}. 
Then $\khd$ with $\kernel=\textsc{Gauss}(\eta)$ and $n=\nin$ delivers
%
%
\begin{talign}\label{eq:gaussian_mmd_manifold_detailed}
\mmd_{\mkernel}^2(\pin,\qout)
    \leq 
\frac{1}{\nout^2} \log(\frac{4\nout}{\delta})\big({\frac{e^2}{ c^{5 d^\star/2}} \log^{\frac{5 d^\star}{2}} \parenth{ \nin \nout }+e\log(\frac{1}{\delta'})}\big) + \frac{1}{\nout}(\frac{1}{\nout}-\frac{1}{\nin})
%
%
\end{talign}
with probability at least $1-
\frac{\delta}{2}-\delta'$ for $c$ independent of $\xin$.
%
\end{corollary}
\begin{proof}
%
%
%
%
%
%
%
%
%
%
%
%
%
\citet[Thm.~4]{altschuler2019massivelyscalablesinkhorndistances} showed that the $(r+1)$-th eigenvalue of $\K$ satisfies \cref{eq:gsn_lambda_manifold} 
%
%
%
for a constant $c$ independent of $\xset=\xin$. 
Since $\maxnorm{\K} =1$ and $\khd \in \ksubg$ with $\subg$ defined in \cref{khd-sub-gaussian}, the result now follows  from  
\cref{thm:subg_low_rank_gen_kernel} with 
$r 
    = 
%
\parenth{  \log\parenth{\nin \nout}/c}^{5 d^\star / 2}.$
\end{proof}
%
%
%
%
%
%
%
%
%