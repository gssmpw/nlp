%
%
\newcommand{\dcomb}{d'}
\newcommand{\Rcomb}{R'}
\newcommand{\Kdeep}{\K_{\textup{deep}}}

%
%
%
%
%
%
%
%

%
%
%
%
%
%
%
%
%
%
%
%
%
%
%
%
%

%

%
%
%
%
%
%
%
%
%
%
%
%
%
%
%
%
%
%
%
%
%
%
%
%
%
%
%
%
%
%
%
%
%
%
%
%
%
%
%
%
%
%
%

%
\section{\pcref{cor:ctt_power_deep_kernel}}\label{proof:ctt_power_deep_kernel}
Define the radius
\begin{talign}\label{eq:R-R-phi}
    \Rcomb \defeq \max_{\y\in \ys\cup\xs} \statictwonorm{(\phi(\y) , \y)},
\end{talign}
the augmented vectors $\ys' \defeq \braces{(\phi(\y),\y)}_{\y\in \ys}$, and the augmented kernel
\begin{talign}
q'((\phi(\x),\x),(\phi(\y),\y))
    \defeq
\kappa(\phi(\x),\phi(\y))
q(\x,\y)
    =
\exp(-\eta\twonorm{(\phi(\x),\x)-(\phi(\y),\y)}^2).
\end{talign}
Since the deep kernel \cref{eq:deep_kernel} takes the form
\begin{talign}
\kdeep(\x,\y) 
    & = (1-\epsilon) q'((\phi(\x),\x),(\phi(\y),\y)) + \epsilon q(\x,\y) 
\end{talign}
we also have
\begin{talign}
\Kdeep \defeq \kdeep(\ys,\ys) = (1-\eps)\bQ' + \eps \bQ
    \qtext{for}
\bQ' \defeq q'(\ys',\ys')
    \qtext{and}
\bQ \defeq q(\ys,\ys).
\end{talign}
Hence, by Weyl's inequality \citep[Thm.~4.3.1]{horn1985matrix} and the Gaussian kernel matrix eigenvalue bound \cref{eq:gsn_lambda_ball},
\begin{talign}
\lambda_{2r+1}(\Kdeep) 
    \leq 
(1-\eps)\lambda_{r+1}(\bQ')+\eps\lambda_{r+1}(\bQ) 
    \leq
n e^{-\frac{d'}{2e} r^{1/d'} \log\parenth{\frac{d' r^{1/d'}}{4 e^2 \eta \Rcomb^2}}}
    \qtext{for} 
(2e)^{d'} \leq r < n.
\end{talign}
Parallel reasoning and the assumption $m\leq n$ yield the same bound for $\lambda_{2r+1}(\kdeep(\xs,\xs))$ and $(2e)^{d'}\leq r < m$.
Now consider the approximate rank parameter 
\begin{talign}\label{eq:approx-rank-deepctt}
    r^\star &\defeq \max\braces{\brackets{ \frac{2e}{d'} \log\parenth{ n \nout \bconstant}}^{d'}, (\Rcomb^2\eta e^3 4 /d')^{d'}}
    \qtext{for}
    b\defeq \half.
\end{talign}
Then, for $n \geq (2e)^{d'}$, we have, 
%
%
exactly as in \cref{proof:gaussian_mmd}, 
\begin{talign}
    \lambda_{2r^\star+1}(\Kdeep) + \lambda_{2r^\star+1}(\kdeep(\xs,\xs))
    &\leq \frac{2}{\nout \bconstant}
\end{talign}
and therefore
\begin{talign}
\errorhat 
    = 
O\left(\sqrt{\log(\frac{n}{s})} \log(\frac{n}{\wtil\beta})\max\braces{\brackets{ \frac{2e}{d'} \log\parenth{ n \nout \bconstant}}^{d'/2}, (\Rcomb^2\eta e^3 4 /d')^{d'/2}}\right).
\end{talign}
Our final step is to bound the quantile of the sole remaining data-dependent term, $\Rcomb$.  Since the inputs are $c$-sub-Gaussian \cref{eq:subexp-dist}, Lem.~1 of \citet{dwivedi2024kernel} with $\psi^{-1}(r) = \frac{\sqrt{\log r}}{\sqrt c}$ implies that the $1-\frac{\wtil \beta}{20 \sblock_n}$ quantile of $\Rcomb$ is $O\big(\sqrt{\log(\frac{n}{\wtil \beta})}\big)$, yielding the result.

%
%
%
%
%
%
%
%
%
%
%
%
%
%

%


%
%
%
%
%
%
%
%
%
%
%
%
%
%
%

%
%
%
%
%
%
%
%
%
%

%
\section{\pcref{cor:ctt_power_deep_kernel_manifold}}
\label{proof:ctt_power_deep_kernel_manifold}

Our reasoning is identical to that in \cref{proof:ctt_power_deep_kernel} with the manifold Gaussian kernel matrix eigenvalue bound \cref{eq:gsn_lambda_manifold} now substituted for the Euclidean ball bound \cref{eq:gsn_lambda_ball} and the approximate rank setting $r^\star=(\log(n\nout)/c)^{5d^\star/2}$ substituted for \cref{eq:approx-rank-deepctt}.

%
%
%
%
%
%
%
%

%
%
%
%
%
%
%


%
%

%
%

%
%

%
%
%
%
%
%
%
%
%
%
%
%
%
%
%
%
%
%
%
%
%

%
%

%
%
%
%
%
%
%
%
%
%
%
%
%
%
%
%
%
%
%
%
%
%
%
%