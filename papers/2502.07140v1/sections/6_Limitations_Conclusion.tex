% \section{Limitations}
% \label{sec:limitations}
% \va{This is more future work rather than limitations. Maybe the depenedency on SMPL input can be one? Also, it cannot handle contact.}
% Since video sequences contain relatively stable backgrounds and stable changes with time flow, extending our proposed methods applicable to video sequences might improve the potential rendering and reconstruction quality.% Our work is based on calibrated cameras, it will be promising to estimate camera locations and parameters.  

\section{Conclusion}
\label{sec:conclusion}
We presented an approach for novel view synthesis of multiple humans from a sparse set of input views. To achieve this, we proposed geometric regularizations that improve geometry training by leveraging a pre-computed SMPL model, along with a patch-based ray consistency loss and a saturation loss that help with novel-view renderings in the sparse-view setting. Our experiments showed state-of-the-art performance for multiple human geometry and appearance reconstruction on real multi-human dataset (CMU Panoptic~\cite{Simon_2017_CVPR,Joo_2017_TPAMI}) and on synthetic data (MultiHuman-Dataset \cite{zheng2021deepmulticap}). Our method still has several limitations. For instance, our method does not model close human interactions, as this is a much more challenging case. Addressing this is an interesting direction for future work.

%Our method still has several limitations. First, we rely on SMPL fittings which might not always be accurate, particularly for scenes with a very large number of humans. A possible solution is to improve the SMPL reconstructions while training the geometry and appearance networks. Second, our method does not model close human interactions, as this is a much challenging case. Addressing this is an interesting direction for future work.




%In comparison to baselines \cite{yariv2021volume,wang2021neus}, we proposed a more efficient optimization method and demonstrated state-of-the-art performance on rendered novel views. In addition, our method enables multiple applications on multi-human datasets during rendering without retraining, including rotation, translation, scaling and removing. All editings doesn't require extra depth, masks or segmentation \etc supervision. %Based on our methods, there are more interesting applications that could be explored in the future. Moreover, since video sequences contain more information than one frame, it is meaningful to extend our methods to videos. The potential rendering and reconstruction quality would be even higher.  \\

