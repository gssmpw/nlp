\documentclass[10pt,twocolumn,letterpaper]{article}

%\usepackage[review]{cvpr}      % To produce the REVIEW version
\usepackage[preprint]{cvpr}
\usepackage{times}
\usepackage{epsfig}
\usepackage{graphicx}
\usepackage{amsmath}
\usepackage{amssymb}
%\usepackage{subfigure}
\usepackage{booktabs}
\usepackage{tabularx}
\usepackage{color}
\usepackage{xcolor,array}
% Include other packages here, before hyperref.

% If you comment hyperref and then uncomment it, you should delete
% egpaper.aux before re-running latex.  (Or just hit 'q' on the first latex
% run, let it finish, and you should be clear).
\definecolor{cvprblue}{rgb}{0.21,0.49,0.74}
\usepackage[pagebackref=true,breaklinks=true,letterpaper=true,colorlinks,citecolor=cvprblue,bookmarks=false]{hyperref}
\def\paperID{349} % *** Enter the Paper ID here
\def\confName{3DV\xspace}
\def\confYear{2024\xspace}
% \iccvfinalcopy % *** Uncomment this line for the final submission

% Pages are numbered in submission mode, and unnumbered in camera-ready
%\ificcvfinal\pagestyle{empty}\fi

\definecolor{mark1}{rgb}{0.2,0.2,0.8}
\definecolor{textcolor}{rgb}{0.42, 0.35, 0.8}
\definecolor{textcolor1}{rgb}{0.45,.42,.75}

\newcommand{\adnane}[1]{\textcolor{blue}{#1}}

\newcommand{\q}[1]{\textcolor{red}{#1}}

\newcommand{\CG}{\mathcal{G}\xspace}
\newcommand{\CV}{\mathcal{V}\xspace}
\newcommand{\CE}{\mathcal{E}\xspace}
\newcommand{\CA}{\mathcal{A}\xspace}
\newcommand{\CF}{\mathcal{F}\xspace}
\newcommand{\CR}{\mathcal{R}\xspace}
\newcommand{\CB}{\mathcal{B}\xspace}
\newcommand{\CX}{\mathcal{X}\xspace}
\newcommand{\CK}{\mathcal{K}\xspace}
\newcommand{\CM}{\mathcal{M}\xspace}
\newcommand{\CC}{\mathcal{C}\xspace}
\newcommand{\CL}{\mathcal{L}\xspace}
\newcommand{\CI}{\mathcal{I}\xspace}
\newcommand{\CQ}{\mathcal{Q}\xspace}
\newcommand{\CO}{\mathcal{O}\xspace}
\newcommand{\CP}{\mathcal{P}\xspace}
\newcommand{\CS}{\mathcal{S}\xspace}
\newcommand{\CT}{\mathcal{T}\xspace}
\newcommand{\CJ}{\mathcal{J}\xspace}
\usepackage[para]{footmisc}
\usepackage{subfig}
% \usepackage{subcaption}
% \usepackage{array}
% \usepackage{colortbl}



\begin{document}

%%%%%%%%% TITLE
%\title{Multi-Human Implicit Surface and Volume Rendering from Sparse Views using Geometry Constraints}
%\title{Multi-Human Implicit Surface and Volume Rendering from Sparse Views using Geometry Constraints}

\title{Few-Shot Multi-Human Neural Rendering Using Geometry Constraints}
%\title{Volume Rendering of Multi-Human from Sparse Images Using Geometry Constraints}

\author{Qian li$^{1}$, Victoria Fern\`andez Abrevaya$^{2}$, Franck Multon$^{1}$, Adnane Boukhayma$^{1}$\\
$^{1}$Inria, University Rennes, IRISA, CNRS, France\\
$^{2}$Max Planck Institute for Intelligent Systems, Germany
}

% \author{First Author\\
% Institution1\\
% Institution1 address\\
% {\tt\small firstauthor@i1.org}
% % For a paper whose authors are all at the same institution,
% % omit the following lines up until the closing ``}''.
% % Additional authors and addresses can be added with ``\and'',
% % just like the second author.
% % To save space, use either the email address or home page, not both
% \and
% Second Author\\
% Institution2\\
% First line of institution2 address\\
% {\tt\small secondauthor@i2.org}
% }

\maketitle
% Remove page # from the first page of camera-ready.
%\ificcvfinal\thispagestyle{empty}\fi


\begin{abstract}
% We present a learning-based method for multiple human rendering from sparse sets of multi-view images. Most current works are focused on single human settings that deliver accurate geometry and appearance using implicit neural representations. However, extending these methods for estimating multiple humans from sparse images remains challenging due to additional occlusion and clutter of multiple humans and the limited number of input views. We propose a neural implicit reconstruction method that addresses the inherent challenges. First, we propose to use geometry constraints by exploiting pre-computed meshes using a human body model (SMPL). Specifically, we regularize the signed distances using the SMPL mesh and leverage bounding boxes for improved rendering. Second, we propose a patch-based ray regularization to minimize rendering inconsistencies and a saturation regularization for robust optimization in variable illumination. Extensive experiments on both real-world and synthetic datasets demonstrate the benefits of our approach and show state-of-the-art performance against existing neural reconstruction methods. 

We present a method for recovering the shape and radiance of a scene consisting of multiple people given solely a few images. 
Multi-human scenes are complex due to additional occlusion and clutter. For single-human settings, existing approaches using implicit neural representations have achieved impressive results that deliver accurate geometry and appearance. 
However, it remains challenging to extend these methods for estimating multiple humans from sparse views. 
We propose a neural implicit reconstruction method that addresses the inherent challenges of this task through the following contributions: First, we propose to use geometry constraints by exploiting pre-computed meshes using a human body model (SMPL). Specifically, we regularize the signed distances using the SMPL mesh and leverage bounding boxes for improved rendering. Second, we propose a ray regularization scheme to minimize rendering inconsistencies, and a saturation regularization for robust optimization in variable illumination.  Extensive experiments on both real and synthetic datasets demonstrate the benefits of our approach and show state-of-the-art performance against existing neural reconstruction methods. 

% Finally, we demonstrate how our framework can be used for editing applications. 
\end{abstract}

% We present a neural learning method for multiple human reconstruction from a sparse set of camera views. Recent neural surface reconstruction methods could generate both 3D geometry and appearance, while usually fail to reconstruct consistent surfaces due to lacking of explicit multi-view geometry constraints, especially for complex scene(\eg multiple human). {\bf Previous works demonstrated that using geometry priors for simple objects significantly enhance the quality of the surfaces. In this work, we propose to extend this idea to using human body model (\eg SMPL) priors, represented by signed distance functions (SDF).} However, since human SMPL usually lacks sufficient details (\eg hair, cloth), we estimate SDF together with its' uncertainty for explicit smooth geometry regularization. In addition, we propose a patch based ray regularization to minimizes potential reconstruction inconsistency and saturation regularization for robust optimization in variable illuminations.  Our evaluations on both real human dataset (CMU Panoptic Dataset\cite{Simon_2017_CVPR,Joo_2017_TPAMI}) and synthetic data (THUman2.0 Dataset and MultiHuman-Dataset \cite{tao2021function4d,zheng2021deepmulticap} demonstrate state-of-the-art performance quantitatively and qualitatively. Extensive experiments show our methods enable a variety editing applications in 3D space without additional assistance of depth, masks, or segmentation.

%Existing multi-person reconstruction is often based on human models, which could reconstruct complex geometry but lose rendering quality due to lacking hair and clothing detail. Recently, volumetric rendering (\eg NeRF\cite{mildenhall2020nerf}) demonstrate promising rendering quality on novel views synthesis from dense input views, while reconstructing fidelity surfaces remains a challenge. Moreover, surface presentation methods \cite{yariv2021volume,wang2021neus} explore surface reconstruction during volume rendering, but it's still hard to handle occlusions and complex geometry (\eg multi-human) from sparse input views. Our methods take the advantages of the human model(\eg SMPL), recent volumetric rendering and scene representation methods. Specifically,we define our multi-human SMPL's surfaces as a zero-level set of a signed distance function (SDF) and train a neural SDF representation. We show how to incorporate this representation in point sampling, neural rendering and reconstruction, and propose a joint optimization for non-rigid shapes (\egclothed humans). Finally,

% We present a learning-based method for reconstructing multiple humans from a sparse set of camera views. Current surface reconstruction methods can generate geometry and appearance simultaneously, but suffer to reconstruct 3D consistent surfaces due to a lack of explicit multi-view geometric constraints, especially for complex scenes (e.g. multiple humans). Previous works show that the use of geometric priors for simple objects (or single scenes) significantly improves the fidelity of reconstructed surfaces. Leveraging that, we propose the use of utilizing human body model (e.g. SMPL) priors, represented by signed distance functions (SDF). However, human SMPL often lacks sufficient details (e.g. hair, cloth), and thus, we estimate the SDF along with its uncertainty to obtain smooth geometry regularization. In addition, we propose a patch-based method for ray regularization to potentially minimize inconsistencies in the reconstruction and to enable saturation regularization to increase the robustness during the optimization of variable illuminations. Our evaluations on both real human datasets (CMU Panoptic Dataset [32, 63]) and synthetic data (THUman2.0 Dataset and MultiHuman Dataset [83, 94] demonstrate state-of-the-art performance quantitatively and qualitatively. Extensive experiments show that our method enables a variety of editing applications in 3D space without additional assistance of depth, masks, or segmentation.


% \textcolor{mark1}{1. #' reviewers pointed that there were no specific design choices to make the method work for the multi-human case, i.e., after defining bounding boxes for each human, the reconstruction was performed independently and individually per-person.'
% Additionally, the reviewer requested a comparison against single person methods. \\
% Different from STNeRF, our sampling and learning not performed on each box individually,  rendering performed on per bounding box. The optimization is perform on the hole scene, not individually person. (Note that the sig-asia paper using sparsity loss to regularize the occupancy
% values) Since we render the hole scene, the occupancy of points on the ray will be close to 1 after converge. also, maybe the  normalized weights in equa11 helps. . goes to sig-asia- Figure 5. \\
% 2. #'The author’s rebuttal pointed that these methods would fail, because the segmentation masks required by single-person methods would not be reliable in multi-person scenarios. This argument, however did not convince ZCeA, as one could compute the segmentation masks within the SMPL bounding box projected onto the image. '\\
% 3. BY ZCeA 'Unless the authors give concrete examples where static reconstruction of humans based on multi-view is needed, I have problems to follow this argumentation. '\\multiple person VS single person\\ 
% Static Scene VS video\\no existing methods to do static multi-human rendering?? More challenge than video. The meaning of static scene reconstruction rather than video.\\ 
% 4. #'The quantitative evaluation discussion provided simply says that our method is better than prior work (same for the InfoNerf comparisons) but does not provide some insights to the reader as to why this is the case. For example I'd expect this section to clearly outline the key components for the proposed method that make it outperform this prior works.'\\
% GeoNeus draw a fig(see-fig-3 in GeoNeus paper) explaining when ray reach surface the weights looks like, it might also helps explain multi-person works for us. Do we need do the same?\\
% 5. #What makes the proposed method outperform prior works even when one person is visible in the scene? How would the ablation study look like (Table 4) if all components were removed and then each one was added. From Table 4 currently it's a little hard to evaluate which contribution has the highest impact as when each module is removed the other 2 are still integrated.\\
% 6. NeuS adpoted L1 loss regularizing rgb rendering for more stable training, we followed this. That might be anther reason for blur results.\\ 
% }
%%%%%%%%% BODY TEXT-----------------------------------------------
\section{Introduction}
\label{sec:intro}

Human reconstruction from single images \cite{choutas2020monocular, kanazawa2018end,liu2022recent}, multiple images \cite{guo2019relightables, collet2015high}, RGB videos \cite{alldieck2018detailed,Kocabas20} or RGB-D data \cite{yu2017bodyfusion,yu2018doublefusion} has received a lot of attention, much less explored is the task of \emph{multiple} human scenario, which is essential for scene understanding, behavior modeling, collaborative augmented reality, and sports analysis.  
%
The multi-human setting introduces additional challenges, as there is now a higher level of occlusion and clutter 
which hinders matching and reconstruction. 
Although in principle one could approach this by first detecting and then independently processing each person, 
simultaneous reconstruction of multiple humans can help to globally reason about occlusion at the level of the scene~\cite{jiang2020coherent, sun2022putting}, 
%which has been shown to produce better results~\cite{jiang2020coherent, sun2022putting}, 
and can potentially recover coherent 3D spatial relations among the people.

Several recent works have attempted to recover multiple humans from a single view \cite{choi2022learning, sun2022putting, sun2021monocular, zanfir2018deep, zanfir2018monocular, fieraru2020three, jiang2020coherent, zhang2021body, ugrinovic2021body,mustafa2021multi}. However, the majority of these are based on regressing the parameters of a human body model --typically SMPL \cite{loper2015smpl}--
which provides coarse reconstructions that %cannot handle 
lack hair, clothing, and geometric details. 
Multi-view settings can help resolve some of the occlusions as well as depth ambiguities, but require a dense array of RGB cameras to achieve a detailed reconstruction \cite{collet2015high, joo2015panoptic,vlasic2009dynamic}.
% which is not easily accessible. % available. 
A more convenient capture system %that could in principle still deliver optimal results 
is the \emph{sparse} multi-view setting, where only a handful of cameras is required.
%, where the number of cameras is limited to 2-15. 
However, due to the decreased number of views and increased level of occlusion, 
existing methods require segmentation masks and a pre-scanned template mesh \cite{liu2011markerless, wu2013set}, rely on a coarse body model \cite{zhang2021lightweight, huang2021dynamic}, or require temporal information \cite{zheng2021deepmulticap, huang2021dynamic}.

A parallel line of work simultaneously tackles the novel-view-synthesis and geometry-reconstruction problems by combining neural coordinate-based representations, \eg implicit signed distance functions (SDFs) \cite{park2019deepsdf}, with differentiable rendering \cite{yariv2021volume,wang2021neus,yariv2020multiview,mildenhall2020nerf}. 
This approach has the advantage of producing, along with geometry, renderings from novel viewpoints that can capture complex surface/light interactions, increasing the scope of applications. 
NeRF~\cite{mildenhall2020nerf}, for example, uses volumetric rendering to produce impressive images under novel views, albeit at the cost of sub-optimal geometries due to the unconstrained volumetric representation. 
SDF-based methods \cite{yariv2021volume,wang2021neus,yariv2020multiview}, while delivering images of slightly lower quality, have been shown to produce 3D surfaces that are competitive with classical approaches. 
For humans, this has been leveraged to obtain geometry and appearance from monocular video \cite{jiang2022selfrecon,chen2021animatable}, RGB-D video \cite{dong2022pina}, and sparse multi-view video \cite{wang2022arah, liu2021neural, peng2021neural, zheng2021deepmulticap, kwon2021neural, peng2021animatable, xu2021h, weng2022humannerf}. 
%% NOTE: \cite{xu2021h} also has experiments for static sparse MV image
However, none of these works, with the exception of \cite{zheng2021deepmulticap,zhang2021editable}, were designed to handle the increased geometric complexity and occlusion of the multi-human case. 
%%%%%%%%%%%%%%%%%%%%%%%%%%%%%%%%%%%%%%%%%%%%%not the only work, STnerf, SIgsia2022 and deep multicap
Current works \cite{zheng2021deepmulticap,zhang2021editable} address the multi-human setting, but both require a set of videos, which effectively becomes a dense array of views as long as deformations are modeled correctly.
%DeepMultiCap \cite{zheng2021deepmulticap} is the only work that addresss the multi-human setting, but the method requires segmentation and was focused on reconstruction from videos, which effectively becomes a dense array of views as long as deformations are modeled correctly.

In this paper, we address the problem of multiple 3D human surfaces and volume rendering from sparse static multi-view images. Our key insight is that human-specific geometric constraints can be leveraged to tackle the challenging sparse-view setting.

Specifically, we first obtain a SMPL body model from the input data and use it to initialize the implicit SDF network, where we define the surface of a multi-human scene as the zero-level set of the SDF. 
Then, the geometry network is optimized with multi-view images by leveraging surface and volume rendering ~\cite{wang2021neus} along with uncertainty estimation methods \cite{deng2022depth,roessle2022dense}, where the SMPL meshes are treated as noisy estimations.  
% [to keep the method general for novel scenes, we do not rely on features pre-trained on a large dataset]
To achieve higher rendering quality from sparse training views, we additionally propose a patch-based regularization loss that guarantees consistency across different rays and a saturation regularization that ensures consistency for variable image illuminations within the same scene.% \va{I feel this is a problem that was not pointed out before, and it would be nice to elaborate better}.

We evaluate our method quantitatively and qualitatively on real multi-human (CMU Panoptic~\cite{Simon_2017_CVPR,Joo_2017_TPAMI}) and synthetic (MultiHuman~\cite{zheng2021deepmulticap}) datasets. We demonstrate results on 5,10,15 and 20 training views, where we achieve state-of-the-art performance in terms of surface reconstruction and novel view quality. 

%In summary, our contributions are: 
%\begin{itemize}
%\item We propose the first neural implicit surface and volume rendering for multiple humans from sparse static images; 
%\item We propose a novel geometric initialization and regularization based on human SMPL, which allows for multi-human rendering simultaneously; %To address the problem of occlusion, we propose the use of SMPL for geometric regularization; 
%\item We propose a patch-based ray consistency regularization and an image saturation regularization that ensures illumination consistency across views;
%\item Our method achieves state-of-the-art performance and the code will be public online. 
%\end{itemize}

%%%%%%%%%%%%%%%%%%%%%%
%% MOTIVATION
% Motivations for multi-human from the multi-human, single-view domain:
%       - A lot of occlusions, if only a crop of the person is used then SOTA methods might fail (\cite{choi2022learning})
%       - Simultanesouly considering multiple person has been empirically shown to work better \cte{all the works on mult-human SMPL}
%       - The reconstructions can be made coherent in 3D space, i.e. with correct relative depth among the people. Plausible 3D spatial relations.
% The problem with SMPL based methods is that they cannot reconstruct clothes, hair, etc

% Applications in behavior analysis, automatic video analysis of sport events, or collaborative augmented reality applications
% human-computer interaction, human behavioral modeling, assisted therapy, monitoring sports performances, protection and security, special effects, modeling and indexing archival footage, or self-driving cars. (this is for monocular)

% We take one step beyond SMPL and use implicit representations for more accurate reconstructions

% "The straightforward solution consists in regarding different people as independent instances and estimating the body shapes and poses one by one using a single-person approach. This strategy, however, may result in inconsistent spatial arrangements and erroneous poses of the reconstructed people." (re-phrase)

% *** Why we cannot do cut one person then reconstruct it --> there should be experiments, else be careful with claims

%%%%%%%%%%%%%%%%%%%%%%
%% STORY
%   - Reconstructing geometry from a scene with multiple humans is hard because of occlusions
%   - Multiple views can help disambiguate, but it's stil hard.
%       => most are single human. For multi-human one can detect and reconstruct, but this can fail - and matching detections among views might not be straight-forward (?) Also, computational time scales (linearly) with the number of people
%       => there are a few multi-human, but ?
%       => people resort to SMPL to make the problem more tractable
%           *** classic matching methods: problem with multi-humans??
%   - SMPL doesn't have clothes or hair so geometry sucks
%   - We consider in particular the scenario where only a sparse number of cameras is available; dense rigs are not easily accessible. Dense => expensive and sophisticated hardware setup and low run-time efficiency (re-phrase)
%   - To address this, we present a method for multi-human 3D reconstruction from multiple views based on neural rendering. 
%   - To handle the occlussions and sparse views, we first fit a SMPL model. Following the recent line of works that make use of geometric quantities to improve reconstruction, we propose to to initialize the reconstruction by building an SDF from the acquired SMPL models.
%   - 

%%%%%%%%%%%%%%%%%%%%%%
% Multi-view capture with dense views and/or temporal data demands a significant cost in time and money, with increased complexity of device requirements, synchronisation, data storage and transfer times. Consequently, 3D reconstruction from a sparse set of images has become an increasingly popular problem~\cite{niemeyer2022regnerf, kim2022infonerf,long2022sparseneus,dong2021shape}, more so after the recent progress in deep neural rendering and reconstruction approaches~\cite{mildenhall2020nerf, park2019deepsdf}. % However, it remains a difficult problem due to the sparsity of 3D cues, requiring carefully designed regularization losses~\cite{kim2022infonerf,niemeyer2022regnerf} or knowledge transfer from pre-trained networks~\cite{yu2021pixelnerf}. This cue scarcity problem becomes even worse when reconstructing cluttered scenes, such as those containing a multitude of people., due to increased occlusions.
% % Multi-view capture with dense views and/or temporal data entails significant capture device requirements, synchronisation, data storage and transfer challenges. Hence, being able to capture scenes from a mere set of static sparse images has become a more than ever popular problem \cite{niemeyer2022regnerf, kim2022infonerf,long2022sparseneus,dong2021shape} especially in the wake of deep learning. 
% Yet, the scarcity of 3D cues in such setups requires carefully designed regularization or knowledge transfer. This scarcity is exacerbated further with cluttered scenes, such as scenes containing a multitude of people.   

% 3D reconstruction of humans from single images \cite{choutas2020monocular, kanazawa2018end,liu2022recent}, multiple images \cite{guo2019relightables, collet2015high}, RGB videos \cite{alldieck2018detailed,Kocabas20} or RGB-D data \cite{yu2017bodyfusion,yu2018doublefusion} has received a lot of attention, much less explored is the task of \emph{multiple} human reconstruction, which is essential for scene understanding, behavior modeling, collaborative augmented reality, and sports analysis.  
% %
% The multi-human setting introduces additional challenges, as there is now a higher level of occlusion and clutter 
% which hinders matching and reconstruction. 
% Although in principle one could approach this by first detecting and then independently processing each person, 
% simultaneous reconstruction of multiple humans can help to globally reason about occlusion at the level of the scene~\cite{jiang2020coherent, sun2022putting}, 
% %which has been shown to produce better results~\cite{jiang2020coherent, sun2022putting}, 
% and can potentially recover coherent 3D spatial relations among the people. %, taking a step towards better scene understanding.
% % simultaneous reconstruction of multiple humans has several advantages. 
% % First, this strategy  allows to further recover coherent 3D spatial relations among the people, taking a step towards better scene understanding.
% % Second, the presence of multiple people can imply a high level of occlusion. While some works explicitly handle this case for single humans \cite{khirodkar2022occluded, zhang2020object}, globally reasoning about occlusion at the level of the scene has been shown to produce better results \cite{jiang2020coherent, sun2022putting}.  
% % Finally, the detect-and-reconstruct approach increases (linearly) in computational time with each new person. 

% % While 
% Several recent works have attempted to recover multiple humans from a single view \cite{choi2022learning, sun2022putting, sun2021monocular, zanfir2018deep, zanfir2018monocular, fieraru2020three, jiang2020coherent, zhang2021body, ugrinovic2021body,mustafa2021multi}. However, the majority of these are based on regressing the parameters of a human body model --typically SMPL \cite{loper2015smpl}--
% which provides coarse reconstructions that %cannot handle 
% lack hair, clothing, and geometric details. 
% Multi-view settings can help resolve some of the occlusions as well as depth ambiguities, but require a dense array of RGB cameras to achieve a detailed reconstruction \cite{collet2015high, joo2015panoptic,vlasic2009dynamic} 
% % which is not easily accessible. % available. 
% A more convenient capture system %that could in principle still deliver optimal results 
% is the \emph{sparse} multi-view setting, where only a handful of cameras is required.
% %, where the number of cameras is limited to 2-15. 
% However, due to the decreased number of views and increased level of occlusion, 
% existing methods require segmentation masks and a pre-scanned template mesh \cite{liu2011markerless, wu2013set}, rely on a coarse body model \cite{zhang2021lightweight, huang2021dynamic}, or require temporal information \cite{zheng2021deepmulticap, huang2021dynamic}.

% %%%\todo{pifu should go somewhere here... pifu is single human though. Goes into the recent line of work: pre-train on a large dataset and then at test time use sparse views: \cite{zheng2021deepmulicap}}


% % --> send the message that MV with SMPL is kind of solved, but we need to take the next step. 
% %but as shown by several works \cite{...}, this is not sufficient to solve the problem. In fact, % in effect, in practice, at bottom
% % "The first sparse for multi-humans" -> \cite{zhang2021lightweight}, 

% % In parallel, there has been development in neural 3D geometry reconstruction by using SDFs \cite{..} or volume-rendered SDFs \cite{...}. These are great but require many views. For the sparse view case, infonerf, regnerf, dietnerf. They show improved results, but as demonstrated here these are not sufficient for the multi-human reconstruction problem.




% %%%%\todo{improving geometry to improve rendering?}

% % Inspired by the success of NeRF~\cite{mildenhall2020nerf}, 
% A parallel line of work simultaneously tackles the novel-view-synthesis and geometry-reconstruction problems by combining neural coordinate-based representations, \eg implicit signed distance functions (SDFs) \cite{park2019deepsdf}, with differentiable rendering \cite{yariv2021volume,wang2021neus,yariv2020multiview,mildenhall2020nerf}. 
% This approach has the advantage of producing, along with geometry, renderings from novel viewpoints that can capture complex surface/light interactions, increasing the scope of applications. 
% NeRF~\cite{mildenhall2020nerf}, for example, uses volumetric rendering to produce impressive images under novel views, albeit at the cost of sub-optimal geometries due to the unconstrained volumetric representation. 
% SDF-based methods \cite{yariv2021volume,wang2021neus,yariv2020multiview}, while delivering images of slightly lower quality, have been shown to produce 3D surfaces that are competitive with classical approaches. 
% For humans, this has been leveraged to obtain geometry and appearance from monocular video \cite{jiang2022selfrecon,chen2021animatable}, RGB-D video \cite{dong2022pina}, and sparse multi-view video \cite{wang2022arah, liu2021neural, peng2021neural, zheng2021deepmulticap, kwon2021neural, peng2021animatable, xu2021h, weng2022humannerf}. 
% %% NOTE: \cite{xu2021h} also has experiments for static sparse MV images (single human)
% %% Monocular video: also "SelfNeRF: Fast Training NeRF for Human from Monocular Self-rotating Video". But this isn't published yet
% %
% However, 
% none of these works, with the exception of \cite{zheng2021deepmulticap}, were designed to handle the increased geometric complexity and occlusion of the multi-human case. DeepMultiCap \cite{zheng2021deepmulticap} is the only %multi-view neural reconstruction 
% work that addresss the multi-human setting, but the method requires a set of videos, which effectively becomes a dense array of views as long as deformations are modeled correctly.

% In this paper we address, for the first time, 
% the problem of reconstructing multiple 3D humans from a \emph{static} and \emph{sparse} set of cameras using neural implicit surfaces. 
% Our key insight is that human-specific geometric constraints can be leveraged to tackle the challenging sparse-view setting.
% %by first fitting  a SMPL body model to the input, building on top of the works 
% % building on the set of works that can faithfully disambiguate pose and shape for multiple people, but only deliver coarse reconstructions, \eg~\cite{huang2017towards,zhang2021lightweight}. %\va{This is a bit risky, since the way we get smpl is a bit shady (in practice, from dense MV cameras)}. 
% Specifically, we first obtain a SMPL body model from the input data, and use this to train a geometry-only implicit SDF network, where we define the surface of a multi-human scene as the zero-level set of the SDF. 
% In a second step, the geometry network is fine-tuned using multi-view images by leveraging the recently-proposed NeuS~\cite{wang2021neus} along with uncertainty-based rendering \cite{deng2022depth,roessle2022dense}, where the SMPL meshes are treated as noisy estimations.  
% % [to keep the method general for novel scenes, we do not rely on features pre-trained on a large dataset]
% To achieve higher rendering quality from sparse training views, we additionally propose a patch-based regularization loss 
% that guarantees consistency across different rays, and a saturation regularization that ensure consistency for variable image illuminations within a same scene.% \va{I feel this is a problem that was not pointed out before, and it would be nice to elaborate better}.

% We evaluate our method quantitatively and qualitatively on real multi-human (CMU Panoptic~\cite{Simon_2017_CVPR,Joo_2017_TPAMI}) and synthetic (MultiHuman~\cite{zheng2021deepmulticap}) datasets. We demonstrate results on 5,10,15 and 20 training views, where we achieve state-of-the-art performance in terms of surface reconstruction and image quality. %Another advantage is that the proposed geometry initialization enables more efficient learning.

% In summary, our contributions are: 
% % 1. We demonstrate an approach to use multiple human bodies as geometric regularization. We propose to use SDF along with estimated uncertainty as explicit geometry constraints, allowing the  learning of details. \\
% (1) We propose the first neural 
% implicit geometry and appearance reconstruction method for multiple humans using a sparse set of static views; 
% (2) To address the problem of occlusion, we propose the use SMPL for geometric regularization; 
% (3) To handle sparse views under occlusion, we propose a patch-based ray consistency regularization, and an image saturation regularization that ensures illumination consistency across views.

%  Code and models will be made available.
% 3.  \todo{I will re-phrase this after finishing the method. This needs to come before in the introduction, if it stays} We combine box rendering with editing, which enables editing multi-humans in 3D space during inference without using masks and segmentation, while preserving the detail and quality of rendered novel multi-human views.   \\



%%%%%%%%%%%%%%%%%%%%%%
% --------------------------------


% Multiple human reconstruction has been a popular and significant topic in computer vision and computer graphics, which enables various applications, such as Virtual Reality (VR) and Augmented Reality (AR), films, teleconferences, and so on. Researches for human reconstruction mainly consist of model-based and model-free, both of them has achieved tremendous progress in recent years. 

% Traditional model-based approaches utilize parametric human bodies (\eg human SMPL \cite{loper2015smpl}) or truncated signed distance fields (TSDFs) for human geometry reconstruction \cite{loper2015smpl,bhatnagar2020combining,liu2021neural,yu2017bodyfusion,yu2018doublefusion,peng2021neural}. %Though human SMPL could provide solid geometry notes, 
% While relying on the human model(\eg SMPL skinning weights of a naked body) may risk losing hair and clothing details. Model-free methods learns from a set of multi-view images or videos with loosing detail constraints \cite{peng2021neural,dong2022pina,peng2021neural,dong2022pina, liu2021neural}, allowing more realistic reconstructed details. With the recent progress of implicit neural representations\cite{sitzmann2019deepvoxels,xu2022point,wiles2020synsin, mildenhall2020nerf,zhang2020NeRF++}, many model-free methods utilize NeRF to achieve photo-realistic rendering. However, NeRF\cite{mildenhall2020nerf} based methods do not sufficiently constrain the 3D geometry, making them hard to reconstruct high-fidelity surfaces or accurate geometry. Further, most existing researches focus on single person reconstruction from videos \cite{peng2021neural,dong2022pina} or images\cite{peng2021neural,dong2022pina, liu2021neural}. For multiple human reconstructions from a single frame, the task becomes even harder to recover geometry and appearance with sufficient details due to multiple human scenes usually containing more complex geometry and occlusions. In this paper, we address those challenges by taking advantage of recent neural implicit surface reconstruction methods\cite{yariv2021volume,wang2021neus,yariv2020multiview} and human bodies models(\eg SMPL\cite{loper2015smpl}), reconstructing multi-human 3D geometry and appearance simultaneously. 

% Recent neural implicit surface reconstruction methods\cite{yariv2021volume,wang2021neus,yariv2020multiview} propose to use a signed distance function (SDF) to present the surface and combine SDF-based density function with volumetric rendering. They could learn an implicit SDF representation and reconstruct geometry and appearance simultaneously. However, the optimization of those methods still depends on direct color field and lack explicit geometry constraints, making them hard to reconstruct consistent geometry and appearance for unseen regions(\eg sparse input view) and occlusion areas (\eg multiple human scene). To alleviate those limitations, we propose to use a multiple human bodies models(SMPL) as geometry initialization, as well as to provide explicit geometry constraints.

% %\cite{yariv2021volume} uses volume density function as Laplace’s cumulative distribution function (CDF) for geometry representation, 
% %Neus\cite{wang2021neus} and Volsdf\cite{yariv2021volume} are among scene representation methodsuse signed distance functions (SDF) for surface representation and introduce the SDF-induced density function to enable the volume rendering to

% %More recently, with the rapidly progress of implicit neural representations\cite{sitzmann2019deepvoxels,xu2022point,wiles2020synsin, mildenhall2020nerf,zhang2020NeRF++}, many researches combine differentiable neural rendering with human bodies model \cite{liu2021neural,peng2021neural,dong2022pina}. Among those,  \cite{liu2021neural} combines human SMPL with implicit learning, and uses it as a proxy to unwarp the surrounding 3D space into a canonical pose, \cite{peng2021neural} could render detailed novel views (\eg with hair,cloths) of the human body from a set sparse input video. Thanks to neural radiance fields (NeRF\cite{mildenhall2020nerf}), those approaches could achieve photo-realistic rendering. However, NeRF\cite{mildenhall2020nerf} does not sufficiently constrain the 3D geometry, making those methods hard to reconstruct high-fidelity surfaces or accurate geometry.

% %Recent scene representation methods combines surface representation with neural volume rendering \cite{yariv2021volume,wang2021neus,yariv2020multiview}. % In order to achieve high quality surface reconstruction while retaining rendering quality, 
% %Among those, \cite{oechsle2021unisurf} proposes a unified framework to reconstruct solid objects from 2D image inputs, 

% Specifically, inspired by Neus\cite{wang2021neus} and Volsdf \cite{yariv2021volume}, we define the surface of multiple human SMPL as a zero-level set of a signed distance function (SDF)  and use it to train a neural SDF presentation as a geometry prior. However, relying on SDF sampled from SMPL directly may risk losing hair and clothing details. Thus, we learn SDF together with uncertainty and combine them together for explicit geometry regularization, which enables the learning of multiple person's clothing and hair details. In addition,
% to achieve higher rendering quality from sparse training views, we propose a patch-based regularization to guarantee consistency across different rays and saturation regularization to ensure image illumination consistency. 
% %retaining multi-view consistency remains a challenge for NeuS-related methods, especially for complex thin structures and large smooth regions. To tackle this challenge, we propose a patch-based ray regularization for photo-metric consistency and saturation regularization for image illumination consistency. 

% %Assuming this SDF presentation storing the multi-human geometry information, we incorporate this by re-initializing the follow-up neural network with the learned weights, utilizing the SDF value for the weights of hierarchical sampling \cite{mildenhall2020nerf} and volume rendering. Inspired by \cite{ortiz2022isdf, dong2022pina}, instead of using the SDF value sampled from SMPL to supervise the neural network learning directly(which consumes lots of time and lacks details), we propose a point-based SDF regularization allowing for the learning of multi-human clothing and hair details. Moreover, we present an approach for human editing during rendering, without additional training or information(\eg depth, mask or segmentation).   \\
% We evaluate our method on both real multi-human datasets (CMU Panoptic Dataset\cite{Simon_2017_CVPR,Joo_2017_TPAMI}) and synthetic data (THUman2.0 Dataset
% and MultiHuman-Dataset \cite{tao2021function4d,zheng2021deepmulticap}) both qualitatively and quantitatively. Specifically,  we demonstrate testing results on 10,15,20 training views,respectively, and achieve state-of-the-art performance on both real datasets and synthetic datasets. %Another advantage is that the proposed geometry initialization enables more efficient learning. 
% In summary, our contributions include:\\
% 1. We demonstrate an approach to use multiple human bodies as geometric initialization. We propose to use SDF along with estimated uncertainty as explicit geometry constraints, allowing the  learning of details. \\
% 2. We propose a patch-based KL regularization to ensure consistency across different rays and image saturation regularization for illumination consistency. \\
% 3. We combine box rendering with editing, which enables editing multi-humans in 3D space during inference without using masks and segmentation, while preserving the detail and quality of rendered novel multi-human views.   \\

\section{Related Work}
\label{sec:relat}

% There is vast amount of works on reconstructing 3D humans from single images \cite{liu2022recent}, monocular video \cite{yuan2022glamr, Kocabas20, alldieck2018video}, RGB-D data \cite{yu2017bodyfusion,yu2018doublefusion,burov2021dynamic} and multi-view data. Here, we focus on methods that take multi-view setups as input, either for single or multiple humans, as well as recent neural surface reconstruction works. 
% \todo{Maybe unify these two again into Human reconstruction, and put initial paragraph there}


\paragraph{Single-Human Reconstruction.} 
There is a vast amount of work on reconstructing 3D humans from single images \cite{bogo2016keep, choutas2020monocular, kanazawa2018end,muller2021self,liu2022recent}, monocular video \cite{yuan2022glamr, Kocabas20, alldieck2018video}, RGB-D data \cite{yu2017bodyfusion,yu2018doublefusion,burov2021dynamic} and multi-view data \cite{starck2007surface,collet2015high,guo2019relightables,huang2018deep}. We concentrate here on the multi-view setting. %, which is the setting of this work. 
% One of the first systems to perform human surface capture was proposed by Starck and Hilton \cite{starck2007surface}, which used a combination of visual hull and stereo reconstruction techniques. 
High-end multi-view capture systems can achieve reconstructions of outstanding quality \cite{leroy2018shape, dou2016fusion4d, guo2019relightables, collet2015high,vlasic2009dynamic,Joo_2017_TPAMI}, but require a complex studio setup that is expensive to build and not easily accessible.
% These settings are expensive to build and not readily accessible. 
To alleviate this, numerous works have been proposed that use instead a sparse set of RGB cameras (\eg between 2 and 15), 
%typically between 2 and 15 \va{I made this up}. 
%To compensate for the lack of views, classic works resorted to tracking a pre-scanned template~\cite{gall2009motion, vlasic2008articulated, carranza2003free, de2008performance} \todo{check all refs}%, relying on temporal information \cite{...} 
where the lack of views and presence of wide baselines is compensated by tracking a pre-scanned template \cite{gall2009motion, vlasic2008articulated, carranza2003free, de2008performance, wu2012full},
using a parametric body model  \cite{huang2017towards, balan2007detailed}, %joo2018total -> a lot of cameras
or more recently, by the use of deep learning \cite{huang2018deep,liang2019shape, wang2022arah, liu2021neural, peng2021neural,  kwon2021neural, peng2021animatable, xu2021h, weng2022humannerf}. 
% % MISSING, maybe:
% \cite{orts2016holoportation} -> 8 cameras, real-time. Uses depth cameras
% gall2009motion -> 4 to 8 cameras
% \cite{wu2012full} -> also sparse, recovers illumnation but in a passive way. tracks a template.
%
%
%
% To achieve better details, a recent trend has been to use deep implicit representations \cite{wang2022arah, peng2021neural,kwon2021neural}. Most of these methods require either a dense set of camera views, or a sparse set of RGB videos. To address the sparse view setting for a single frame, PiFU \cite{saito2019pifu} and follow-up work \cite{saito2020pifuhd} employ a set of pixel-aligned features pre-trained on a large dataset, which might not easily generalize to out-of-distribution scenes. 
% %% Not nerf but deep learning: \cite{liang2019shape}
% %% Nerf: \cite{kwon2021neural, liu2021neural, peng2021animatable, weng2022humannerf,xu2021h}
% %% PiFU-like works : "limited by the training data" | "line of works that condition on pixel-aligned features to recover an implicit representation". They require ground-truth data and thus cannot generalize to arbitrary scenes. Furthermore, they do not recover appearance and cannot do novel view synthesis. From here connect to pixel nerf, and why we do not compare. 
% %%
% \cite{saito2019pifu, saito2020pifuhd, alldieck2022photorealistic, he2020geo, huang2020arch, he2021arch}
%
%
% %%%%%%%%%%%
% %% "DoubleField: Bridging the Neural Surface and Radiance Fields for High-fidelity Human Reconstruction and Rendering" (CVPR'22)
% %% -> this is very relevant, we should've compared
% %%%%%%%%%%%
%
%
% ----
% \paragraph{Multi-view Multi-Human Reconstruction.} 
% %%% GOOD REVIEW IN Dynamic Multi-Person Mesh Recovery From Uncalibrated Multi-View Camera"
% None of these works were designed to handle the increased geometric complexity and occlusion of the multi-human case. 
% Compared with tremendous progress of single human reconstruction, there is a limited number of multiple human reconstruction methods.
\paragraph{Multi-Human Reconstruction.} 
In contrast, there has been a limited number of works that address the problem of \emph{multiple} human reconstruction. 
This is a difficult task since the presence of several people increases the geometric complexity of the scene, introduces occlusions,  and amplifies ambiguities such that commonly used features like color, edges, or key points cannot be correctly assigned. 

For single images and video, the problem has been mainly tackled by regressing the parameters of the SMPL \cite{loper2015smpl} body model~\cite{zhang2020perceiving, choi2022learning, sun2022putting, sun2021monocular, zanfir2018deep, dong2021shape, zanfir2018monocular, fieraru2020three, jiang2020coherent, zhang2021body, ugrinovic2021body, guler2019holopose}. Although this can work robustly with as little as one view, the reconstructions are very coarse and cannot explain hair, clothing, and fine geometric details. 
The only exception is the work of Mustafa \etal~\cite{mustafa2021multi}, which performs model-free reconstruction of multiple humans by combining an explicit voxel-based representation with an implicit function refinement. 
However, % I comment just becasue if not sure, we could save space , the output geometries are still rather coarse \va{maybe remove this, so are ours...}, and 
the method requires training on a large synthetic dataset of multiple people which hinders generalization. Our work, on the other hand, performs 3D reconstructions, produces renderings of novel views, and can generalize to arbitrary multi-human scenes.

Multi-view capture setups can help resolve depth ambiguities and some of the occlusions. 
Classic methods for estimating multiple humans rely heavily on segmentation masks and template mesh tracking \cite{liu2011markerless, liu2013markerless, wu2013set}. We avoid the use of segmentation masks by adopting volumetric rendering for implicit surfaces \cite{wang2021neus}. 
More recently, deep learning-based approaches were proposed, but they either require temporal information~\cite{zheng2021deepmulticap,huang2021dynamic,zhang2021lightweight,shuai2022novel}, pre-training on a large dataset~\cite{zheng2021deepmulticap} which cannot work on general scenes, or a coarse body model \cite{zhang2021lightweight, huang2021dynamic, shuai2022novel} which lacks geometric detail. 
% Furthermore, with the exception of XX, the focus here is on 3D shape estimation, ignoring the quality of rendered novel viewpoints \todo{I think this should go away}. 
Here, we focus on the multi-human setting on static scenes and propose a method that recovers accurate reconstructions and at the same time produces renderings of novel viewpoints. 
% by recovering an implicit representation from multi-view images.
%
%
%%% V: I removed the segmentation argument because we still need SMPL as input, so in practice you might need a segmentation mask just to get SMPL
%%% V: mustafa20224d is temporal and I think they don't do _simultaneous_ reconstruction of all humans, just segmentation. To be on the safe side I'm gonna ignore this paper for now, since it's anyway from 2022 and we could've missed it. 
%
%
%
% ----
% \todo{maybe start neural with this:}
% When the goal is to generate free-viewpoint video, image-based-rendering has also been considered \cite{wu2020multi}, which side-steps 3D reconstruction and produces novel views by reasoning on the available images. A recent succesful family of works, nerf an dimplicit, which are discussed next. 
% Another classic one in IBR (I think): "High-quality video view interpolation using a layered representation"
%
%
\paragraph{Neural Surface Reconstruction and Novel-View Synthesis.} 
For generating free-viewpoint video, image-based rendering has been considered as an alternative or complement to 3D reconstruction~\cite{carranza2003free,wu2020multi,liu2021neural,kwon2021neural, liu2021neural, peng2021animatable, weng2022humannerf,xu2021h}. 
%which side-steps 3D reconstruction and produces novel views by reasoning on the available images. A recent succesful family of works, nerf an dimplicit, which are discussed next. 
% Another classic one in IBR (I think): "High-quality video view interpolation using a layered representation"
% Impressive results have been obtained recently by with methods that combine 3D reconstruction and novel-view symthesis, using deep implicit representations. 
When geometry proxies are available, neural rendering \cite{aliev2020neural,thies2019deferred,jena2022neural} can produce competitive novel view synthesis. Recently, NeRF\cite{mildenhall2020nerf} demonstrated impressive rendering results by representing a 3D scene as a neural radiance field, trained only with calibrated multi-view images through the use of volume rendering. However, due to the unconstrained volumetric representation and self-supervised training on RGB values, reconstructed geometries tend to be too noisy to be useful for 3D applications. 
% Utilizing NeRF for human reconstruction could achieve a realistic appearance rendering, but hard to reconstruct accurate geometry due to NeRF purely reling on RGB values and lacking geometry constraints. 
To recover more accurate 3D geometry along with appearance, DVR~\cite{niemeyer2020differentiable}, IDR~\cite{yariv2020multiview}, and NLR~\cite{kellnhofer2021neural} propose to learn an implicit representation directly from multi-view images but require accurate object masks to work. To avoid the need for segmentation masks, recent works propose to combine implicit representations with volume rendering \cite{oechsle2021unisurf, yariv2021volume, wang2021neus}. 
%To avoid the use of mask segmentations, VolSDF \cite{yariv2021volume} proposed to combine an implicit SDF representation with volume rendering, transforming SDF values into volume densities by using the cumulative distribution function of the Laplace distribution. 
%%%by modeling the volume density as a function of geometry which is defined as Laplace’s cumulative distribution function (CDF).
% NeuS\cite{wang2021neus} uses a SDF to represent the surface and develop a new volume rendering method to train a neural SDF representation. Though those methods could reconstruct 3D geometry and appearance, it remains a challenge for them to reconstruct geometry consistent surface, especially for scenes containing complex geometry or sparse input views. \\
These methods show remarkable reconstruction results but struggle when the number of input views is low. 
Implicit neural representations from sparse input can be obtained by using pre-trained pixel-aligned features or 3D feature volumes for input images ~\cite{saito2019pifu, saito2020pifuhd, alldieck2022photorealistic, he2020geo, huang2020arch, he2021arch, yu2021pixelnerf, li2023learning, jena2024geotransfer} or point clouds ~\citep{boulch2022poco,williams2022neural,huang2023neural,peng2020convolutional,chibane2020implicit,lionar2021dynamic, ouasfi2023mixing, peng2021shape, ouasfi2022few, ouasfi2024robustifying}, 
but this requires ground-truth geometry and is limited by the training data, struggling to generalize to new scenes. Sparse variants that do not require generalizable features were proposed in the image input \eg \cite{niemeyer2022regnerf, kim2022infonerf, long2022sparseneus, younes2025sparsecraft, li2023regularizing} and point cloud input case \eg \cite{NeuralTPS,sparseocc,nap,ouasfi2024robustneuralreconstructionsparse,williams2021neural}. 
InfoNeRF~\cite{kim2022infonerf} regularizes sparse views by adding an entropy constraint on the density of the rays,  RegNeRF~\cite{niemeyer2022regnerf} uses a patch-based regularizer over generated depth maps, and SparseNeuS~\cite{long2022sparseneus} uses a multi-scale approach along with learned features that are fine-tuned on each scene. 
%% DietNeRF: since it needs CLIP (=> "pretraining") I don't know where to place it, and I'm ignoring it for now.
%%% PiFU-like works : "limited by the training data" | "line of works that condition on pixel-aligned features to recover an implicit representation". They require ground-truth data and thus cannot generalize to arbitrary scenes. Furthermore, they do not recover appearance and cannot do novel view synthesis. From here connect to pixel nerf, and why we do not compare. 
%%%
% \cite{saito2019pifu, saito2020pifuhd, alldieck2022photorealistic, he2020geo, huang2020arch, he2021arch}
Our approach builds on NeuS \cite{wang2021neus}, and tackles the sparse view challenge by adding human-specific geometric priors and novel regularizations. 




%%%% This is more for the introduction, not the related work. 
%Specifically, we use multiple human SMPL to train a neural SDF presentation as a geometry prior to optimize the learning. Moreover, %inspired by \cite{deng2022depth,roessle2022dense}, 
%We propose to learn SDF along with uncertainty and propose a explicit geometry constrains to ensure the geometry consistency while encouraging learning details around surfaces. Furthermore, noticing that the reconstruction inconsistency might happen due to insufficient viewpoints or invariant illuminations, we propose a patch-based regularization to ensure consistency across different rays and saturation regularization for illuminations across different views. \todo{pifu line}


% Sparse nerf: regnerf\cite{niemeyer2022regnerf}, infonerf\cite{kim2022infonerf}
% Sparse Neus:\cite{long2022sparseneus} %ECCV 2022

% \paragraph{Concurrent Works.} 
% \va{I don't think this should go here, it's risky. I suggest we add this only if reviewers ask}
% Monosdf\cite{yu2022monosdf}, GeoNeus\cite{fu2022geo}, Neuris\cite{wang2022neuris} VOXURF\cite{wu2022voxurf} % 4 nips 2022

% -------------------------

% \textbf{Human reconstruction.} 
% % single human, multihuman, yiyi
% Previous work related to human reconstruction could be classified into single-human reconstruction and multiple-human reconstructions according to the human numbers in the scene. Single human reconstruction from images or video includes traditional methods, such as depth sensors \cite{theobalt2005image, dou2016fusion4d}, depth array of cameras\cite{debevec2000acquiring, guo2019relightables}, rendering pipeline optimization \cite{wu2020multi}, template-based methods\cite{gall2009motion, carranza2003free, de2008performance}, and learning based methods\cite{burov2021dynamic,loper2015smpl,loper2015smpl,bhatnagar2020combining,  liu2021neural,yu2017bodyfusion,yu2018doublefusion,peng2021neural}. Compared with tremendous progress of single human reconstruction, there is a limited number of multiple human reconstruction methods. Among those, \cite{zanfir2018monocular,zanfir2018deep} are earliest works to utilize human SMPL for multiple human reconstructions, \cite{mustafa2021multi} exploits 3D pose to learn body part scores, localize multiple persons simultaneously and estimate 3d pose and shape
% \cite{mustafa20224d} combines human model with implicit reconstruction for multiple human shape reconstruction. Those works usually rely on segmentation and focus on 3D shape estimation, which cost expensive in computing and ignores the quality of rendered novel viewpoints. In this paper, we focus on multiple human reconstructions on static scenes, learning multiple human geometry and appearances with loose clothing and hair details simultaneously and enabling editing during rendering without the extra requirement of segmentation.

% \textbf{Neural surface reconstruction.} 
% Recently, NeRF\cite{mildenhall2020nerf} demonstrated impressive rendering results by representing a 3D scene as a neural radiance field, with only calibrated color images, camera poses and intrinsic as input. Utilizing NeRF for human reconstruction could achieve a realistic appearance rendering, but hard to reconstruct accurate geometry due to NeRF purely reling
% on RGB values and lacking geometry constraints. In order to reconstruct accurate 3D geometry along with appearance, Volsdf \cite{yariv2021volume} propose to improve geometry representation and reconstruction in neural volume rendering by defining the volume density function as Laplace’s cumulative distribution function(CDF) and modeling it as a function of geometry. NeuS\cite{wang2021neus} uses a signed distance function(SDF) to represent the surface and develop a new volume rendering method to train a neural SDF representation. Though those methods could reconstruct 3D geometry and appearance, it remains a challenge for them to reconstruct geometry consistent surface, especially for scenes containing complex geometry or sparse input views. \\
% Our approach take advantage of existing implicit surface reconstruction (\eg NeuS), and tackle its challenges by adding learned geometry priors and explicit geometry constraints. Specifically, we use multiple human SMPL to train a neural SDF presentation as a geometry prior to optimize the learning. Moreover, %inspired by \cite{deng2022depth,roessle2022dense}, 
% we propose to learn SDF along with uncertainty and propose a explicit geometry constrains to ensure the geometry consistency while encouraging learning details around surfaces. Furthermore, noticing that the reconstruction inconsistency might happen due to insufficient viewpoints or invariant illuminations, we propose a patch-based regularization to ensure consistency across different rays and saturation regularization for illuminations across different views. \todo{pifu line}

% Sparse nerf: regnerf\cite{niemeyer2022regnerf}, infonerf\cite{kim2022infonerf}
% Sparse Neus:\cite{long2022sparseneus} %ECCV 2022
% \textbf{Concurrent Works.} 
% Monosdf\cite{yu2022monosdf}, GeoNeus\cite{fu2022geo}, Neuris\cite{wang2022neuris} VOXURF\cite{wu2022voxurf} % 4 nips 2022




%%%%%%%%%%%%%%%%%%%%%%%%%%%%%%%%%%%%%%%

%%---  Multi-human reconstruction from single image: ---
% Note first that there are numerous works on multi-human 3D pose estimation, but we focus on those that recover 3D meshes (typically shape and pose parameters of a SMPL model) since this is closer to our task.

% \cite{choi2022learning}
% \cite{sun2022putting, sun2021monocular}
% \cite{zanfir2018deep} -> predict 3D pose only and defer the shape reconstruction step to a later optimization step
% \cite{zanfir2018monocular}
% \cite{dong2021shape} -> the final goal is pose, but they use SMPL shape to help with the problem
% \cite{fieraru2020three}
% \cite{jiang2020coherent}
% \cite{zhang2021body}
% \cite{ugrinovic2021body} -> addresses body scale and depth ambiguity


%% --- Multi-human reconstruction from multiple views ---

% + Classic: they track a pre-scanned template of an actor:
% \cite{liu2011markerless} -> only for two people
% \cite{wu2013set}
% This one doesn't, but the input is video:
% \cite{li2018shape}
% The multi-person case introduces ambiguities "where commonly-used features such as color, edges or keypoints cannot be individually assigned

% + Deep learning:
% \cite{huang2021dynamic} -> uses motion model, SMPL. Uncalibrated cameras, video.
% \cite{zhang2021lightweight} -> they say they are the first light-weight and robust that uses only sparse cameras (6 cameras in the experiments). They are also total capture: hand face and body - but still cannot handle clothes and hair.
% icnludes hands and faces but it's still a coarse mesh from a parametric model. 
% \cite{zheng2021deepmulticap}  -> implicit, but video

% Note how many rely on motion, but here we assume only a single view


%% --- Multi-human novel view synthesis (2D) ---
% \cite{lakhal2019view}
% \cite{zhang2021editable} -> Closely related, they do nerf on multiple humans with a layered representation. The input is a video. 

%% -- Single human, MV neural:
% \cite{liu2021neural}
% \cite{weng2022humannerf} -> Humannerf, from video
% \cite{shao2022doublefield}
% \cite{shao2022diffustereo} -> follow up, they build on doublefield


%% Multi-view, RGB-D
% \cite{yu2021function4d}


% MV 3D pose, just in case we ever use this as input:
% \cite{zhang2021direct}




%% MV story:
% - there are high-end MV capture systems requiring a large number of cameras (volumetric)
% \cite{leroy2018shape, collet2015high, orts2016holoportation, joo2015panoptic}
% and more light-weight settings was achieved using a pre-scanned template
% \cite{de2008performance}
% - recently with deep learning, more possible:
% \cite{huang2018deep} -> volumetric representation
% Pifu, pifu hd shows MV results also. This line of work requires pre-training on a large dataset. 
% \cite{saito2019pifu, saito2020pifuhd}
% These are all for single humans
% For multiple humans,









%\section{When Physics-Informed Neural Networks Meet State Space Models}

%\subsection{Continuous-Discrete Mismatch}

%\subsection{Sampling Continuous Dynamics in Sequence}

%PDE to infinite-dimensional ODE

%\subsection{Modeling Sequence with State Space Models}

%infinite-dimensional ODE to SSMs

%\subsection{Linear Time Variant}





%\clearpage

\vspace{-1mm}

\section{Combating Failure Mode with State-Space Model and Sub-sequential Alignment}
\label{sec:ssmsub}


    To address the problems in Section~\ref{sec:fail}, we propose (1) a discrete state-space-based encoder that models the sequences of individual collection points in continuous dynamics, to match with \textit{Continuous-Discrete Mismatch}, and propagates the information from the initial condition to subsequent times (Section~\ref{sec:ssm}).  and (2) a sub-sequence contrastive alignment mechanism that aligns different outputs of the same collection point in different sub-sequences, to form an agreement that eliminates simplicity bias (Section~\ref{sec:subseq}).
    
\vspace{-2mm}


\subsection{Continuous Time Propagation of Initial Condition Information with State Space Model}
\vspace{-1mm}
\label{sec:ssm}
As we discussed in Section~\ref{sec:fail}, the \textit{Continuous-Discrete Mismatch} of PINNs raises the intrinsic difficulty of modeling, since the time dependency in a dynamic PDE system is not captured spontaneously by discrete sampling. 
    We argue that such a dynamic time dependency can be modeled by SSM. 
To this end, we first consider the PDE as a spatially infinite-dimensional ODE to simplify the problem. We view the solution $u_\theta(x,t)$ in a function space that, if we let:
\begin{equation}
    U(t) := u_\theta(\cdot,t),
\end{equation}
be a function $x \to u_\theta(x,t)$, by $M$-point spatial sampling:
\begin{equation}
    U_i(t) := u(x_i,t),
\end{equation}
\begin{equation}
    \mathbf {u}(t) = \left[U_1(t),U_2(t),\cdots,U_M(t) \right]^\top .
    \label{}
\end{equation}

\textbf{Sequential Modeling Continuity with SSM.}
In continuous time, we now model the function $\mathbf {u}(t)$ to the dynamic system described by SSM as in Eq.~\ref{equ:hiddenssm} and~\ref{equ:outputssm}. Here we let $\mathbf x(t) = \text{Embed}(x,t)$, where $\text{Embed}(\cdot)$ is the Spatio-Temporal Embedding in Fig~\ref{fig:main}. After temporal discretization $\mathbf u_k=\mathbf u(k\Delta t),\mathbf h_k=\mathbf h(k\Delta t)$ and $\mathbf x_k=\mathbf x(k\Delta t)$, we get: 
\begin{equation}
    \mathbf u_k=C\bar A^k \mathbf h_0 + C\sum_{i=0}^k\bar A^{k-i}\bar B\mathbf x_i.
    \label{equ:ssmu}
\end{equation}
Reversibly, by the inverse of the discretization rule defined by Eq.~\ref{equ:disc1},~\ref{equ:disc2}, we can restore this temporal dependency to continuous time. This kind of restoration can help achieve PINN's generalization to any moment in $[0, T]$. 

\textbf{Pattern Propagation by Joint Optimization.}
Combine Eq.~\ref{equ:lossequ} with~\ref{equ:ssmu}, in a sequence start with $t=0$, the sum of loss of collection points at time $k\Delta t$, would be: 
\begin{align}
  &\sum_{i=1}^M \mathcal L_\mathcal F(u(x_i,k\Delta t)) =  \frac{1}{M}\mathcal \|\mathcal F( \mathbf 1_M\cdot \mathbf{u}_k)\Arrowvert^2\nonumber\\&=\frac{1}{M}\|\mathcal F\left(\mathbf1_M\cdot(C\bar A^k \mathbf h_0 + C\sum_{i=0}^k\bar A^{k-i}\bar B\mathbf x_i)\right)\Arrowvert^2,
  \label{equ:timeloss}
\end{align}
where $1_M=[1,1,\cdots,1] \in \mathbb R^M$. In Eq.~\ref{equ:timeloss}, we notice that the $\mathbf h_0$ should satisfy both the initial condition and the equation by jointly optimizing the losses:
\begin{align}\label{equ:loss0equ}
   \mathcal L_\mathcal F(\mathbf u_0) =  \frac{1}{M}\|\mathcal F\left(\mathbf1_M\cdot(C \mathbf h_0 )\right)\Arrowvert^2;\\
   \mathcal L_\mathcal I(\mathbf u_0) =  \frac{1}{M}\|\mathcal I\left(\mathbf1_M\cdot(C \mathbf h_0 )\right)\Arrowvert^2 .
   \label{equ:loss0init}
\end{align}
Thereby, for each collection point, the numerical value of its solution should be jointly optimized by Eq.~\ref{equ:timeloss},~\ref{equ:loss0equ}, and~\ref{equ:loss0init}, thus receiving the pattern defined by the initial conditions.


\textbf{Uniformed Derivatives Scale.} Another benefit that can be got from SSM is, by parameterizing differential state matrix $A$ in Eq.\ref{equ:hiddenssm} with HiPPO matrix~\cite{gu2020hippo} which contains the derivative information,  we can align the derivatives of the system with respect to time on a uniform scale. This uniform scale will help to reduce the problem of ruggedness on the loss landscape due to gradient vanishing or exploding.

\textbf{Time-Varying SSM.} In practice, we use the time-varying Selective SSM~\cite{gu2023mamba}, instead of the function defined by Eq.~\ref{equ:ssmu} being the SSM on a linear time-invariant system. The time-varying SSM has two advantages, one is that such input-dependent models typically have stronger representational capabilities~\cite{xu2024infinite}, while the other is that it will make diverse predictions that help to eliminate simplicity bias in the model, as we will discuss in section~\ref{sec:subseq}. This time-variance will make $(\bar A,\bar B, C)$ time-dependent, and therefore, Eq.~\ref{equ:ssmu} and \ref{equ:timeloss} need minor adjustments. These adjustments won't impact the initial condition propagation, and we will discuss them in Appendix~\ref{apx:LTI}.





\subsection{Eliminating Simplicity Bias of Models with Sub-Sequence Contrastive Alignment}
\label{sec:subseq}


\begin{figure}[t!]
    \centering
    \includegraphics[width=\linewidth]{_fig/fig4}
    \vspace{-5mm}
    \caption{Comparison of Sequence Granularity}
    \label{fig4}
    \vspace{-6mm}
  %  \vspace{-1mm}
\end{figure}





Although SSM can make the information about the initial conditions propagate in time coordinates, it still cannot mitigate the simplicity bias of neural networks. 
    The model is still prone to falling into an over-smoothed local optimum. 
        There are two key points to address this over-smoothness caused by simplicity bias: (1) appropriate sequence granularity to guarantee a smooth optimization process. (2) Mitigating the effect of simplicity bias through the diversity of model prediction paradigms~\cite{pagliardiniagree}. 
        
        \textbf{Sequence Granularity.} A proper sequence granularity ensures smooth propagation of the initial conditions while making the model easier to optimize. As shown in Fig.~\ref{fig4}, there are three ways to define sequence, which are pseudo sequence~\cite{zhao2024pinnsformer}, long sequence~\cite{nguyen2024sequence}, and the proposed sub-sequence.
        We propose to use a sub-sequence with medium granularity overlapping. The sub-sequential modeling can avoid: (1) the difficulty of crossing the loss barrier that makes the model trapping in the over-smooth local optimum, which is caused by the huge inertia of long sequence; (2) the difficulty of broadcasting information globally on the time coordinate, that caused by construct on small neighborhoods of a collection point in pseudo sequence. Sub-sequence takes only the first output in the sequence as the output value of the current collection point. Its successors' values will pass information crossing the time coordinate through subsequences alignment and form diverse predictions to eliminate simplicity bias.

\textbf{Contrastive Alignment for Information Propagation.} As shown in Fig.~\ref{fig4}, we construct a sub-sequence for each collection point together with its finite successors, which form overlapping collection points. By aligning the predictions of these collection points with a contrastive loss, each collection point becomes a soft relay of the pattern. Thus, it forms the propagation of patterns in the whole time domain.%$[0,T]$.
%In this way, we can realize a filtered propagation of pattern across sub-sequences by recursive alignment, 


\textbf{Eliminating the Simplicity Bias.} Previous work~\cite{teney2022evading,pagliardiniagree} has pointed out that the agreement obtained from diverse predictions is the key to eliminating the effects of simplicity bias. We argue that this agreement from diverse predictions is naturally obtained in the sub-sequence alignment. This is because the fact that,
    since the SSM we constructed in section~\ref{sec:ssm} is time-varying and a collection point will be at different time coordinates in different sub-sequences, the predictions for this collection point are naturally diverse. And we force these diverse predictions to arrive at a consensus by contrastive alignment.


      %  If we model a PDE with an overly long sequence, when trapped in an oversmoothed local optimum, decreasing the loss of a point at one moment may increase the loss of all other points on the sequence, which causes the model to have a huge inertia when optimizing.
%The large inertia caused by long sequences makes it difficult for the model to cross the loss barrier. At the other extreme, a sequence that is too small may make it difficult to propagate the time dependency. The pseudo sequence can only reflects temporal information on small neighborhoods of a single collection point, instead of broadcasting the information globally on time coordinate.


\section{PINNMamba}

In conjunction with the high-level ideas described in Section~\ref{sec:ssmsub}, in this section, we present PINNMamba, a novel physics-informed learning framework that effectively combats the failure modes in the PINNs.


\begin{figure*}[t!]
    \centering
    \includegraphics[width=\textwidth]{_fig/conv}
    \vspace{-8mm}
    \caption{The ground truth solution, prediction (top), and absolute error (bottom) on convection equations.}
    \label{fig:conv}
    \vspace{-5mm}
  %  \vspace{-1mm}
\end{figure*}



\textbf{Sub-Sequential I/O.} As shown in Fig.~\ref{fig:main}, PINNMamba first samples the grid of collection points over the entire spatio-temporal domain bounded by the PDE. We assume that the grid picks $M$ spatial coordinates and $N$ temporal coordinates, and denote the temporal sampling interval as $\Delta t = T/(N-1)$. For a collection point $(x,t)$, we construct a sequence $X(x,t)$ with its $k-1$ temporal successors:  
\begin{equation}
    X(x,t) = \{(x,t),(x,t+\Delta t),\cdots,(x,t+(k-1)\Delta t)\}.
\end{equation}
PINNMamba takes such $M\times N$ sequences as the input of models. 
    For each sequence $X(x,t)$, PINNMamba computes a sub-sequence prediction $\{\bar u_\theta^t (x,t),\bar u_\theta^t (x,t+\Delta t),\cdots,\bar u_\theta^t (x,t+(k-1)\Delta t)\}$ corresponding to every collection point in the sequence, where $\bar u_\theta^t (x,t+i\Delta t)$ denote the tentative prediction of collection point $(x,t+i\Delta t)$ in a sequence start with time $t$. The $\bar u_\theta^t (x,t)$ will be taken as the output of collection point $(x,t)$ and the rest of the sequence will be used to construct the sub-sequence contrastive alignment loss we will discuss later in Section~\ref{sec:subseq}. The residual losses of the model w.r.t the sub-sequence will be:
    
\vspace{-5mm}

    \small{
    \begin{equation}
    \mathcal L_{\mathcal F}^\text{seq}(u_\theta)= \frac{1}{k|\chi|}\sum_{(x_i,t_i)\in \chi}\sum_{j=0}^{k-1}\|\mathcal F(u_\theta^{t_i}(x_i,t_i+j\Delta t)\|^2;
        %\mathcal L_{\mathcal F}(u_\theta)= \frac{1}{k|\chi|}\sum_{i=1}^{|\chi|}\sum_{j=0}^{k-1}\|\mathcal F(u_\theta^{t_i}(x_i,t_i+k\Delta t)\|^2;
    \label{equ:lossequseq}
\end{equation}
\begin{equation}
    \mathcal L_{\mathcal I}^\text{seq}(u_\theta)= \frac{1}{k|\chi_0|}\sum_{(x_i,t_i)\in \chi_0}\sum_{j=0}^{k-1}\|\mathcal I(u_\theta^{t_i}(x_i,t_i+j\Delta t)\|^2;
    \label{equ:lossinitseq}
\end{equation}
\begin{equation}
    \mathcal L_{\mathcal B}^\text{seq}(u_\theta)= \frac{1}{k|\partial\chi|}\sum_{(x_i,t_i)\in \partial\chi}\sum_{j=0}^{k-1}\|\mathcal B(u_\theta^{t_i}(x_i,t_i+j\Delta t)\|^2.
    \label{equ:lossboundseq}
\end{equation}
}
%As shown in Fig.~\ref{fig:main}, PINNMamba 
\normalsize
\vspace{-5mm}
%\textbf{Sub-Sequence of Collection Points.}

\textbf{Model Architecture.} As shown in Fig.~\ref{fig:main}, PINNMamba employs an encoder-only architecture, which encodes fixed-size input sub-sequence into a sub-sequence prediction with the same length. First, for each token in the sequence, an MLP-based Spatio-Temporal Embedding layer first embeds the $(x,t)$ coordinates into high-dimensional representation. The embeddings will be sent to a Mamba-based encoder, which consists of several PINNMamba blocks. 

The PINNMamba block employed here consists of two branches: (1) the first is a stack of a linear projection layer, a 1d-convolution layer, a Wavelet activation~\cite{zhao2024pinnsformer}, and an SSM layer with parallel scan~\cite{gu2023mamba}; (2) the second is a stake of a linear projection layer and a Wavelet activation. The two branches are then connected with an element-wise multiplication, followed by another linear projection and residual connection. With input $X^l$,
the PINNMamba block can be formulated as: 
\begin{equation}
    X_1^l = \text{SSM}(\sigma(\text{Conv}(W_aX^l)));
\end{equation}
\vspace{-5mm}
\begin{equation}
    X_2^l = \sigma(W_bX^l);
\end{equation}
\vspace{-5mm}
\begin{equation}
    X^{l+1} = X^l+W_c(X_1^l\otimes X_2^l),
\end{equation}
where $\sigma(x)=\omega_1\sin(x)+\omega_2\cos(x)$ is Wavelet activation function~\cite{zhao2024pinnsformer}, in which $\omega_1,\omega_2$ are learnable. $\otimes$ denotes an element-wise multiplication. 

\textbf{Sub-Sequence Contrastive Alignment.} PINNMamba predicts the same collection multiple times in different subsequences. For example, the collection point $(x,k+\Delta t)$ appears on sequences from $X(x,t+\Delta t)$ to $X(x,t+k\Delta t)$. We align the predictions on these subsequences to make the information defined by the initial conditions propagate over time. To do this, for each subsequence, we design a contrastive loss with the last subsequence for alignment:
\begin{align}
        \mathcal L_\text{alig}(u_\theta)= \frac{1}{(k-1)|\chi|}&\sum_{(x_i,t_i)\in \chi} \sum_{j=1}^{k-1} \Big[u_\theta^{t_i}(x_i,t_i+j\Delta t)\nonumber\\&-u_\theta^{t_i+\Delta t}(x_i,t_i+j\Delta t)\Big]^2.
\end{align}
\normalsize

Thus, the empirical loss for PINNMamba is defined as:
\small
\begin{equation}
     \mathcal L(u_\theta)=\lambda_{\mathcal F}\mathcal L_{\mathcal F}^\text{seq}(u_\theta)+\lambda_{\mathcal I}\mathcal L_{\mathcal I}^\text{seq}(u_\theta)+\lambda_{\mathcal B}\mathcal L_{\mathcal B}^\text{seq}(u_\theta)+\lambda_\text{alig}\mathcal L_\text{alig}(u_\theta).
\end{equation}
\vspace{-8mm}
    

\normalsize
\section{Results}\label{sec:results}
This section highlights the benefits of GraNNite optimization techniques, compares performance between Intel\textregistered\ Core\texttrademark\ Ultra Series 1 \& 2 NPUs, and demonstrates the superior energy efficiency of NPUs over CPUs and GPUs for GNN execution.
Since GraNNite is the first hardware-aware framework tailored for optimizing GNN deployment on COTS SOTA NPUs, no existing works exist for direct comparison.
% This section demonstrates how the various GraNNite optimization techniques enhance performance across different GNN models, highlighting significant improvements when compared to traditional CPU and GPU executions on Intel NPUs.
% Version #3

\textbf{Benefits of GraNNite Optimizations:} Fig.~\ref{plot:gnn_progression} illustrates the performance progression of GNN models on the Intel\textregistered\ Core\texttrademark\ Ultra Series 2 NPU, highlighting the impact of various optimizations proposed by GraNNite. Each optimization builds upon the preceding set unless otherwise specified. For example, the performance of QuantGr in GCN reflects a model in which GrAd, NodePad, GraphSplit, and QuantGr are cumulatively applied. However, in SAGE-max, EffOp and GrAx3 target the same model, and their performance gains are not cumulative.
For GCN, the initial optimization, StaGr combined with GraphSplit, achieves a $1.51\times$ speedup over the baseline by efficiently partitioning workloads between the CPU and NPU. Adding GrAd and NodePad introduces support for time-varying graphs and enhances parallelism but reduces performance to $1.4\times$ due to CPU preprocessing overhead and an increased node count on the NPU. GraSp further boosts throughput by $1.1\times$. The most significant improvement, $2.7\times$, is achieved by combining GrAd, NodePad, GraphSplit, and QuantGr, leveraging low-precision arithmetic to minimize computational overhead.
For GAT, EffOp alone provides a $3\times$ speedup, while incorporating approximations (GrAx2) boosts performance to $7.6\times$ with negligible impact on model quality. Similarly, for SAGE-max, EffOp yields a $2\times$ speedup, which increases to $3.2\times$ with GrAx3, again with no quality degradation.
We note that the effects of SymG and CacheG could not be demonstrated as they require modifications to the (proprietary) NPU compiler.
%, which is not open source.

\begin{figure}[t!]
\begin{center}
\includegraphics[width=\columnwidth]{Plots/MTL_vs_LNL_GCN.pdf}% This is a *.eps file
\end{center}
\caption{Performance of GCN on different Intel\textregistered\ NPUs: Intel\textregistered\ Core\texttrademark\ Ultra Series 2 and Intel\textregistered\ Core\texttrademark\ Ultra Series 1.}\label{plot:mtl_vs_lnl}
\end{figure}

\begin{figure}[t!]
\begin{center}
\includegraphics[width=\columnwidth]{Plots/CPU_GPU_NPU.pdf}% This is a *.eps file
\end{center}
\caption{Performance of GNN models on different devices of an Intel\textregistered\ AI PC: NPU outperforms CPU and GPU by a large margin.}\label{plot:cpu_gpu_npu}
\end{figure}

\textbf{Performance Comparison on Intel\textregistered\ Core\texttrademark\ Ultra Series 1 vs. Intel\textregistered\ Core\texttrademark\ Ultra Series 2 NPUs:} Fig.~\ref{plot:mtl_vs_lnl} compares GCN performance across GraNNite optimizations on Intel\textregistered\ Core\texttrademark\ Ultra Series 1 and Intel\textregistered\ Core\texttrademark\ Ultra Series 2 NPUs. Series 2 consistently outperforms series 1 due to its higher tile count (4 vs. 2). For the most optimized configuration (GrAd + NodePad + QuantGr), Intel\textregistered\ Core\texttrademark\ Ultra Series 2 delivers $1.7\times$ and $1.6\times$ higher throughput than Intel\textregistered\ Core\texttrademark\ Ultra Series 1 for the Cora and Citeseer datasets, respectively. This advantage arises from the higher number of MAC units in Series 2, enabling greater data parallelism. However, the observed gains fall short of the theoretical $2\times$ maximum due to limited parallelism inherent in the GCN.  

\textbf{Performance and Energy Efficiency of CPU, GPU, and NPU with GraNNite Optimizations:} Fig.~\ref{plot:cpu_gpu_npu} compares the performance of CPU, GPU, and NPU across three GNN layers: GraphConv (GCN), GraphAttn (GAT), and SAGE (GraphSAGE). For GCN, the NPU achieves a $2.9\times$ speedup over the GPU and $3.3\times$ over the CPU. For GAT layers, the NPU provides $2.3\times$ and $3.8\times$ improvements over the GPU and CPU, respectively. Similarly, for GraphSAGE with mean aggregation, the NPU achieves $6.7\times$ and $10.8\times$ speedups over the GPU and CPU. These results highlight the computational efficiency of NPUs and the effectiveness of GraNNite optimizations in delivering high-performance GNN execution without hardware modifications.  
Fig.~\ref{plot:energy_gcn} demonstrates the energy efficiency of NPUs compared to CPUs and GPUs for GNN execution. For the Cora dataset, the NPU is $4.1\times$ and $8.5\times$ more energy-efficient than the most efficient GPU and CPU implementations, respectively. Similarly, for the Citeseer dataset, the NPU achieves $4.4\times$ and $8.6\times$ greater energy efficiency.


% Version #2
% Fig.~\ref{plot:gnn_progression} shows the performance progression of GNNs on the Intel Lunar Lake NPU, highlighting significant improvements from a series of targeted optimizations proposed by GraNNite. It is to be noted that the optimizations are progressively added unless they are . For example, the performance for QuantGr in GCN is shown for a model with GrAd, NodePad, GraphSplit and QuantGr applied to the GNN model, not just the QuantGr. But for SAGE-max, EffOp and GrAx3 target the same model section, therefore, the performance gains shown in the plot are not cumulative. For GCN, the first optimization (StaGr + GraphSplit), enhances model execution by efficiently distributing the workload between the CPU and NPU, achieving a $1.51\times$ performance boost over the baseline. Adding GrAd and NodePad allows handling time-varying graphs and ensures efficient parallelism, though it slightly reduces performance as compared to (StaGr + GraphSplit) by $1.4\times$ due to the additional pre-processing overhead on CPU and increased number of nodes on the NPU. The most substantial improvement comes from combining GrAd, NodePad, GraphSplit, and QuantGr, which uses low-precision arithmetic to reduce computational load, resulting in a $2.7\times$ performance gain.
% For GAT, EffOp yields a $3\times$ performance boost. When we incorporate approximation, the improvement jumps to $7.6\times$, with almost no degradation in quality.
% For SAGE-max, EffOp yields a $2\times$ performance boost. When we incorporate approximation (GrAx3), the improvement jumps to $3.2\times$, with no degradation in quality.
% Fig.~\ref{plot:mtl_vs_lnl} compares GCN performance across different GraNNite optimization techniques on NPUs of two Intel AI PCs, meteor lake and lunar lake. We observe that Lunar Lake consistently delivers higher performance as it has higher number of tiles (4) as compared to meteor lake (1). For the most optimized version (GrAd + NodePad + QuantGr), lunar lake archives $1.7\times$ ($1.6\times$) higher throughput than meteor lake for Cora (Citeseer) dataset. The presence of higher number of MAC units in lunar lake enables higher data parallelism leading to better performance. Although the performance gain is not equal to the theoretical maximum (4X) due to the limited data parallelism in the GCN model.
% Fig.~\ref{plot:cpu_gpu_npu} compares the performance of CPU (blue), GPU (orange), and NPU (green) across three GNN layer types: GraphConv (GCN), GraphAttn (GAT), and SAGE (GraphSAGE). For GCN, the NPU achieves a remarkable $17.3\times$ speedup over the GPU and $4.6\times$ over the CPU, showcasing its efficiency in handling these workloads. Similarly, the NPU demonstrates $2.3\times$ and $3.8\times$ improvements over GPU and CPU, respectively, for GAT layers, and achieves $6.7\times$ and $10.8\times$ speedups for GraphSAGE with mean aggregation. These results underscore the NPU's computational advantages and the effectiveness of GraNNite's optimizations, enabling high-performance GNN execution on existing hardware without modifications.
% Fig.~\ref{plot:energy_gcn} demonstrates the need for mapping the GNN models on NPU for energy efficiency. We observe that NPU is $4.1\times$ ($4.4\times$) energy efficient than the most energy efficient GPU implementation for Cora (Citeseer) dataset. Similarly, NPU is $8.5\times$ ($8.6\times$) energy efficient than the most energy efficient CPU implementation for Cora (Citeseer) dataset. 
% It is to be noted that, we could not demonstrate the impact of SymG and CacheG as those would require changes in the NPU compiler which is not made open source.

% Version #1
% Fig.~\ref{plot:gnn_progression}(a) shows the performance progression of Graph Convolutional Networks (GCN) on the Intel Lunar Lake NPU, highlighting significant improvements from a series of targeted optimizations. Here, the unoptimized implementation serves as the reference baseline.
% The first optimization, Optimized Graph Partitioning (OGP), enhances data locality by efficiently distributing the workload between the CPU and NPU, achieving a $1.85\times$ performance boost over the baseline. Adding Node Padding (NP) allows handling time-varying graphs and ensures efficient parallelism, though it slightly reduces performance by $1.1\times$ due to the additional processing overhead on the CPU. The most substantial improvement comes from combining OGP, NP, and Quantization, which uses low-precision arithmetic to reduce computational load, resulting in a $2.7\times$ performance gain.

% Fig.~\ref{plot:gnn_progression}(b) demonstrates the performance improvements of Graph Attention Network (GAT) implementations on an Intel NPU, achieving a $7.6\times$ speedup over the baseline.
% The first optimization replaces the ``Select" operation with element-wise multiplication, yielding a $3\times$ performance boost by simplifying the computation. Next, the element-wise multiplication is offloaded to the DPU, providing an additional $3.5\times$ performance gain by focusing computation on the DPU. Finally, eliminating the broadcast addition operation, which causes memory overhead, results in a substantial performance improvement, reaching the $7.6\times$ speedup.

% Fig.~\ref{plot:gnn_progression}(c) showcases the performance gains of a SAGE model with the max aggregation scheme, achieving up to $3.2\times$ speedup over the baseline.
% The first optimization replaces the complex ``Select" operation with a more efficient element-wise multiplication, boosting performance to $2\times$ the baseline. The second optimization swaps the ``ReduceMax" operation for ``MaxPool1D," aligning better with hardware architecture and providing an additional performance increase, reaching the final $3.2\times$ speedup.

% Fig.~\ref{plot:cpu_gpu_npu} compares the performance of CPU (blue), GPU (orange), and NPU (green) across three GNN layer types: GraphConv (GCN), GraphAttn (GAT), and SAGE (GraphSAGE). For GCN, the NPU achieves a remarkable $17.3\times$ speedup over the GPU and $4.6\times$ over the CPU, showcasing its efficiency in handling these workloads. Similarly, the NPU demonstrates $2.3\times$ and $3.8\times$ improvements over GPU and CPU, respectively, for GAT layers, and achieves $6.7\times$ and $10.8\times$ speedups for GraphSAGE with mean aggregation. These results underscore the NPU's computational advantages and the effectiveness of GraNNite's optimizations, enabling high-performance GNN execution on existing hardware without modifications.

% Version #0
% These optimizations demonstrate how integrating algorithmic improvements, memory management, and hardware-friendly approaches unlocks the full performance potential of GCNs on NPUs.

% Fig.~\ref{plot:gcn_progression} illustrates the performance progression of Graph Convolutional Network (GCN) implementations on an Intel Lunar Lake NPU, demonstrating significant enhancements achieved through a series of targeted optimizations. The baseline unoptimized implementation is set as the reference point, representing the lowest performance.
% The first optimization, Optimized Graph Partitioning (OGP), focuses on improving data locality by effectively distributing the workload between the CPU and NPU for a static input graph. This optimization results in a notable performance boost of approximately 1.85X over the baseline.
% Next, the addition of Node Padding (NP) to the OGP approach enables the model to handle time-varying input graphs. This ensures efficient parallelism across compute units, although it slightly reduces performance by about 1.1X compared to OGP alone. This decrease is attributed to the extra processing time required for the normalization matrix on the CPU.
% The most significant performance improvement is observed with the combination of OGP, NP, and Quantization. By employing low-precision arithmetic, this approach reduces the overall computational workload, leading to a remarkable 2.7X enhancement over the initial implementation.
% The consistent increase in performance across these optimization stages underscores the value of integrating algorithmic optimizations like OGP with memory management techniques (NP) and hardware-friendly approaches (quantization). This cumulative application of optimizations highlights that while each individual optimization is beneficial, their combined effect is essential for unlocking the full performance potential of GCNs on NPUs.


% \begin{figure}[t!]
% \begin{center}
% \includegraphics[width=\columnwidth]{Plots/GCN_progression.png}% This is a *.eps file
% \end{center}
% \caption{Progressive performance improvement of GCN through different optimizations}\label{plot:gcn_progression}
% \end{figure}


% These optimizations highlight the importance of reducing unnecessary memory operations and offloading tasks to specialized cores, significantly improving inference latency and efficiency for GAT models on NPUs in resource-constrained environments.


% Fig.~\ref{plot:gat_progression} showcases the performance improvements of Graph Attention Network (GAT) implementations on an Intel NPU, illustrating how a series of optimizations culminate in a substantial 7.6X speedup over the baseline implementation. The baseline serves as the starting point and represents the lowest performance due to the computational inefficiencies inherent in certain operations typically used in GAT models.
% The first optimization involves replacing the "Select" operation with element-wise multiplication, which is a simpler and more parallelizable operation. This initial change yields an impressive improvement of approximately 3X over the baseline performance, highlighting the benefits of simplifying computational tasks.
% In the second stage of optimization, the element-wise multiplication operation is further refined; instead of performing the multiplication operation alongside other computations, it is exclusively executed on the DPU. This focused approach results in a cumulative performance boost of around 3.5X relative to the original implementation, indicating that optimizing where and how computations are performed is critical for enhancing performance.
% The final optimization addresses the broadcast addition operation, which often incurs significant memory overhead by duplicating data across tensors. By eliminating this redundant operation, the GAT implementation experiences a substantial performance enhancement, achieving a maximum of 7.6X speedup over the baseline. 
% This progressive enhancement illustrates the crucial role of reducing unnecessary memory operations and leveraging specialized processing cores for performance-critical tasks. The results emphasize that architectural-aware optimizations—such as offloading specific workloads from the DSP to DPU cores and eliminating redundant operations through approximations—can lead to significant improvements in inference latency for GAT models on NPUs. Such strategies not only optimize computational efficiency but also facilitate faster and more effective execution of GNNs in resource-constrained environments.


% \begin{figure}[t!]
% \begin{center}
% \includegraphics[width=\columnwidth]{Plots/GAT_progression.png}% This is a *.eps file
% \end{center}
% \caption{Progressive performance improvement of GAT through different optimizations}\label{plot:gat_progression}
% \end{figure}


% This progression demonstrates the value of targeted optimizations in reducing computational overhead, enhancing data-parallel processing, and maximizing performance for GNN models on specialized hardware.

% Fig.~\ref{plot:sage_progression} demonstrates the performance improvements of a SAGE model with the max aggregation scheme following a series of targeted optimizations, ultimately achieving a cumulative speedup of up to 3.2X compared to the baseline. The baseline reflects the initial performance prior to any optimizations, serving as a reference for evaluating the impact of each subsequent modification.
% The first optimization involves substituting the "Select" operation—known for its control-flow complexity—with a data-parallel element-wise multiplication. This shift to a computationally more efficient operation delivers a substantial boost, bringing the performance to approximately 2X of the baseline. This optimization illustrates how replacing control-flow-heavy operations with data-parallel alternatives can enhance computational efficiency.
% Building upon this, a second optimization replaces the "ReduceMax" operation with "MaxPool1D," a more streamlined operation that aligns better with the hardware's architecture. This adjustment leads to an additional performance increase, as depicted by the green bar on the right, resulting in a total improvement of 3.2X over the baseline configuration.
% Overall, this progression highlights the impact of carefully selected optimizations in reducing computational overhead, enhancing data-parallel processing, and improving model efficiency. These results underscore the effectiveness of architectural-aware optimizations in maximizing performance for GNN models on specialized hardware.


% \begin{figure}[t!]
% \begin{center}
% \includegraphics[width=\columnwidth]{Plots/SAGE_progression.png}% This is a *.eps file
% \end{center}
% \caption{Progressive performance improvement of SAGE-max through different optimizations}\label{plot:sage_progression}
% \end{figure}



% \subsection{CPU, GPU \& NPU performance per watt for GCN, GAT and GraphSAGE}
% Figure~\ref{plot:power} presents the power consumption breakdown of various components in different operational states of an AI PC, including IDLE and during the execution of GNN models on different devices. The x-axis shows the specific GNN models in use and the devices they are mapped to, allowing for a comparison of power usage across distinct deployment scenarios.
% The first bar on the left represents the system’s IDLE state, where no workload is running on any device. This IDLE power breakdown provides a baseline to compare against the power demands when GNN models are actively running on various devices within the AI PC.
% Moving beyond IDLE, the figure details the power distribution among key system components—IA cores, System Agent, GT, and DRAM—when GNN models are executed, especially highlighting the benefits of NPU deployment. When a model runs on the NPU, the System Agent’s power consumption, shown in blue, increases due to its role in managing the NPU, which draws from the System Agent’s power rail. However, despite this rise in the System Agent’s power draw, the total power usage across all components (including IA cores, GT, and DRAM) remains notably low when models are mapped to the NPU.
% This low cumulative power usage, paired with the NPU’s high processing efficiency (as demonstrated in previous figures), results in excellent performance per watt. Such efficiency makes the NPU highly suitable for applications that demand both high performance and low energy consumption. Specifically, the NPU’s ability to efficiently handle GNN workloads with minimal power draw makes it well-suited for high-performance tasks in power-sensitive settings. In summary, Figure~\ref{plot:power} underscores how the NPU’s balanced approach to speed and power usage makes it a compelling option for deploying GNN models in resource-constrained environments.

% \begin{figure}[t!]
% \begin{center}
% \includegraphics[width=\columnwidth]{Plots/Power.png}% This is a *.eps file
% \end{center}
% \caption{Power consumption of different GNN models on Intel AI PC: NPU takes lower power and compute with a higher speed}\label{plot:power}
% \end{figure}



% Fig.~\ref{plot:cpu_gpu_npu} presents a performance comparison among CPU (blue), GPU (orange), and NPU (green) in executing three types of GNN layers: GraphConv (GCN), GraphAttn (GAT), and SAGE (GraphSAGE). For the GraphConv (GCN), the NPU achieves an impressive 17.3× speedup compared to the GPU and a 4.6× speedup over the CPU. This result highlights the NPU's significant efficiency in managing GCN workloads.
% In the case of the GraphAttn (GAT), the NPU demonstrates a performance improvement of 2.3× over the GPU and 3.8× over the CPU. Likewise, for the SAGE (GraphSAGE) using the mean aggregator scheme, the NPU outperforms the GPU by 6.7× and the CPU by 10.8×. These results clearly indicate the superior computational capabilities of NPUs and efficacy of GraNNite proposed optimizations, particularly when applied to our most optimized GNN layers. The consistent performance advantage of NPUs over traditional architectures like CPUs and GPUs across these benchmarks suggests that existing NPUs can effectively implement GNNs using the proposed optimizations, without necessitating any changes to the underlying hardware.



\begin{figure}[t!]
\begin{center}
\includegraphics[width=\columnwidth]{Plots/Energy_GCN.pdf}% This is a *.eps file
\end{center}
\caption{Normalized Energy Consumption of GCN on Intel\textregistered\ Core\texttrademark\ Ultra Series 2 Devices (CPU, GPU, and NPU), highlighting significant energy savings achieved with GraNNite optimizations.}\label{plot:energy_gcn}
\end{figure}
% \section{Limitations}
% \label{sec:limitations}
% \va{This is more future work rather than limitations. Maybe the depenedency on SMPL input can be one? Also, it cannot handle contact.}
% Since video sequences contain relatively stable backgrounds and stable changes with time flow, extending our proposed methods applicable to video sequences might improve the potential rendering and reconstruction quality.% Our work is based on calibrated cameras, it will be promising to estimate camera locations and parameters.  

\section{Conclusion}
\label{sec:conclusion}
We presented an approach for novel view synthesis of multiple humans from a sparse set of input views. To achieve this, we proposed geometric regularizations that improve geometry training by leveraging a pre-computed SMPL model, along with a patch-based ray consistency loss and a saturation loss that help with novel-view renderings in the sparse-view setting. Our experiments showed state-of-the-art performance for multiple human geometry and appearance reconstruction on real multi-human dataset (CMU Panoptic~\cite{Simon_2017_CVPR,Joo_2017_TPAMI}) and on synthetic data (MultiHuman-Dataset \cite{zheng2021deepmulticap}). Our method still has several limitations. For instance, our method does not model close human interactions, as this is a much more challenging case. Addressing this is an interesting direction for future work.

%Our method still has several limitations. First, we rely on SMPL fittings which might not always be accurate, particularly for scenes with a very large number of humans. A possible solution is to improve the SMPL reconstructions while training the geometry and appearance networks. Second, our method does not model close human interactions, as this is a much challenging case. Addressing this is an interesting direction for future work.




%In comparison to baselines \cite{yariv2021volume,wang2021neus}, we proposed a more efficient optimization method and demonstrated state-of-the-art performance on rendered novel views. In addition, our method enables multiple applications on multi-human datasets during rendering without retraining, including rotation, translation, scaling and removing. All editings doesn't require extra depth, masks or segmentation \etc supervision. %Based on our methods, there are more interesting applications that could be explored in the future. Moreover, since video sequences contain more information than one frame, it is meaningful to extend our methods to videos. The potential rendering and reconstruction quality would be even higher.  \\



%----------------------------------------------------------------------

%\clearpage
{\small
\bibliographystyle{ieee_fullname}
\bibliography{egbib}
}

% ----------------------------------------------
\newpage
\section*{supplementary material}
In this supplementary material, we will provide a theoretical analysis to the proposed memory efficient Transformer adapter (META) in Section~\ref{secS1}, provide a detailed description of the experimental datasets in Section~\ref{secS2}, provide a detailed description of the experimental settings in Section~\ref{secS3},
provide more result comparisons under different pre-trained weights in Section~\ref{secS4},
provide more ablation study results in Section~\ref{secS5}, show class activation map comparisons of instance segmentation before and after adding the Conv branch in Section~\ref{secS6},qualitative visualizations of instance segmentation and semantic segmentation results in Section~\ref{secS7},  as well as the pseudo-code for when the stripe size is set to $2$ in Section~\ref{secS8}. 
% -------------------------------------------
\section{Theoretical Analysis of META}
\label{secS1}
% -------------------------------------------
{\color{red}{\emph{This supplementary is for Section~3 of the main paper.}}} In this section, we will prove that META exhibits superior generalization capability and stronger adaptability compared to existing ViT adapters. 
%
To achieve this goal, we will prove that the proposed memory efficient adapter (MEA) block possesses larger information entropy (IE) than the existing attention-based ViT adapters~\citep{hu2022lora,jie2023fact,chen2022vision,ma2024segment,luo2023forgery,shao2023deepfake}, which provides evidence that the MEA block has more comprehensive feature representations. Then, based on the maximum mean discrepancy (MMD) theory~\citep{cheng2021neural,arbel2019maximum,wang2021rethinking}, larger IE in the ViT adapter framework leads to superior generalization capability and stronger adaptability. The detailed theoretical analysis process is as follows:

\begin{lemma}
% ---------------------------------
In any case of mutual information, the MEA block will gain larger information entropy after fusing $\textbf{X}_{vit}$ and $\textbf{X}_{con}$.
% ---------------------------------
\end{lemma}
% ---------------------------------
\begin{proof}
As introduced in Section~3.2 of the main paper, the proposed MEA block can be viewed as an operation that integrates the ViT features (\ie, the Attn branch and the FFN branch) and the convolution features (\ie, the Conv branch). Therefore, we begin by formalizing the obtained features into the following two basic elements: the ViT features and the convolution features. To formalize the learning setting, we express the ViT features as $\textbf{X}_{vit}$ and the convolution features as $\textbf{X}_{con}$. It is evident that if $\textbf{X}_{vit}$ and $\textbf{X}_{con}$ are extracted from the same image, then $\textbf{X}_{vit}$ and $\textbf{X}_{con}$ are not independently distributed, and there exists some mutual information between them~\citep{zhang2022graph,wu2021cvt,zhang2023cae,peng2021conformer}. Therefore, the IE of the fused feature of $\textbf{X}_{vit}$ and $\textbf{X}_{con}$ within the MEA block can be expressed as:
% ---------------------------------------------------
\begin{equation}
\begin{split}
\label{eqs:1}
\textrm{H}(\textbf{X}_{vit}, \textbf{X}_{con}) = \textrm{H}(\textbf{X}_{vit}) + \textrm{H}(\textbf{X}_{con}) - \textrm{I}(\textbf{X}_{vit}; \textbf{X}_{con}),
\end{split}
\end{equation}
% ---------------------------------------------------
where $\textrm{H}(\cdot)$ is utilized to calculate the IE of the given variate, which can be formulated as:
% ---------------------------------------------------
\begin{equation}
\begin{split}
\label{eqs:2}
\textrm{H}(\textbf{X}_{vit}) = -\sum P(\textbf{x}_{vit}) log(P(\textbf{x}_{vit})),\\
\textrm{H}(\textbf{X}_{con}) = -\sum P(\textbf{x}_{con}) log(P(\textbf{x}_{con})),
\end{split}
\end{equation}
% ---------------------------------------------------
where $P(\textbf{x}_{vit})$ represents the probability of $\textbf{X}_{vit}$ taking on the value of $\textbf{x}_{vit}$. The similar definition of $P(\textbf{x}_{con})$. $\textrm{I}(\cdot;\cdot)$ in Eq.~\eqref{eqs:1} is used to compute the mutual information between $\textbf{X}_{vit}$ and $\textbf{X}_{con}$, which can be expressed as:
% ---------------------------------------------------
\begin{equation}
\begin{split}
\label{eqs:3}
\textrm{I}(\textbf{X}_{vit}; \textbf{X}_{con}) = \sum\sum \textrm{P}(\textbf{X}_{vit}, \textbf{X}_{con}) \textrm{log}(\textrm{P}(\textbf{X}_{vit}, \textbf{X}_{con}) (\textrm{P}(\textbf{X}_{vit}), \textrm{P}(\textbf{X}_{con}))),
\end{split}
\end{equation}
% ---------------------------------------------------
where $\textrm{P}(\textbf{X}_{vit}, \textbf{X}_{con})$ is their joint probability distribution. 
%\textrm{P}(\textbf{X}_{vit})$ and $\textrm{P}(\textbf{X}_{con})$ are the marginal probability distributions of $\textbf{X}_{vit}$ and $\textbf{X}_{con}$, respectively. 
Since $\textrm{I}(\textbf{X}_{vit}; \textbf{X}_{con})$ is always non-negative, $\textrm{H}(\textbf{X}_{vit}, \textbf{X}_{con})$ may still be greater than $\textrm{H}(\textbf{X}_{vit})$ or $\textrm{H}(\textbf{X}_{con})$~\citep{paninski2003estimation,gabrie2018entropy}. This suggests that the IE of the features extracted by MEA is always greater than the feature representation extracted by either of them separately.

Specifically, if $\textrm{I}(\textbf{X}_{vit}; \textbf{X}_{con})$ is small, the IE gain after fusion may still be significant, which is beneficial for improving the generalization capability and adaptability of the block. However, when $\textrm{I}(\textbf{X}_{vit}; \textbf{X}_{con})$ is large, the IE gain after fusion may be reduced. This means that $\textrm{I}(\textbf{X}_{vit}; \textbf{X}_{con})$ may affect the IE improvement of the fused model. Next, we will discuss the impact of $\textbf{X}_{vit}$ and $\textbf{X}_{con}$ on improving the IE of the adapter based on the size of $\textrm{I}(\textbf{X}_{vit}; \textbf{X}_{con})$, which can be divided into the following three cases:

\begin{itemize}
% --------------------------
\item {{Small} $\textrm{I}(\textbf{X}_{vit}; \textbf{X}_{con})$.} This is an ideal state. When the dependency between $\textbf{X}_{vit}$ and $\textbf{X}_{con}$ is small, it indicates that $\textrm{I}(\textbf{X}_{vit}; \textbf{X}_{con})$ is small, that is, $\textbf{X}_{vit}$ and $\textbf{X}_{con}$ respectively represent different information of the image. In this case, fusing $\textbf{X}_{vit}$ and $\textbf{X}_{con}$ can bring a significant increase in IE, which is beneficial to improving the adapter's generalization capability and adaptability.
% --------------------------
\item {{Medium} $\textrm{I}(\textbf{X}_{vit}; \textbf{X}_{con})$.} When $\textrm{I}(\textbf{X}_{vit}; \textbf{X}_{con})$ is between small and large, it indicates that there is a certain degree of correlation between them. In this case, fusing $\textbf{X}_{vit}$ and $\textbf{X}_{con}$ may still bring some IE gain. The specific improvement effect depends on the degree of correlation between $\textbf{X}_{vit}$ and $\textbf{X}_{con}$ and their complementarity in image representations. Fortunately~\citep{zhang2022graph,zhang2023cae,marouf2024mini,liu2023efficientvit}, a large amount of work has validated that ViT and convolutional layers can extract distinctive information from images. Therefore, in this case, fusing $\textbf{X}_{vit}$ and $\textbf{X}_{con}$ can still bring IE gains.
\item {{Large} \myparagraph{$\textrm{I}(\textbf{X}_{vit}; \textbf{X}_{con})$}.} When $\textrm{I}(\textbf{X}_{vit}; \textbf{X}_{con})$ between $\textbf{X}_{vit}$ and $\textbf{X}_{con}$ is large, it indicates that there is a high correlation between them, \ie, global ViT and local convolution features may represent similar or overlapping information of the image. In this case, the IE gain brought by fusing $\textbf{X}_{vit}$ and $\textbf{X}_{con}$ may decrease because there is a lot of information overlap between them. However, in our case, the probability of such a scenario occurring is almost non-existent, fusing $\textbf{X}_{vit}$ and $\textbf{X}_{con}$ may still improve the performance of the model to some extent, because they may capture the detailed information of the image to varying degrees.
% --------------------------
\end{itemize}
% --------------------------

Based on the aforementioned theoretical analysis, we can conclude that the proposed MEA block has a larger IE than existing ViT adapters (which are primarily based on the attention mechanism) under any scenario. This provides evidence that the MEA block has more comprehensive feature representations. 
% ---------------------------------
\end{proof}
% ---------------------------------
As the MEA block includes a parallel convolutional branch, it can better capture local inductive biases compared to the traditional ViT adapter, which mainly uses self-attention~\citep{hu2022lora,jie2023fact,chen2022vision,ma2024segment,luo2023forgery,shao2023deepfake,mercea2024time}. 
%
Therefore, the MEA block's feature space should be more capable of distinguishing different samples, resulting in a larger MMD value. 
%
Our MEA block's feature space is obtained by combining the attention branch, the feed-forward network branch, and the local convolutional branch, enabling it to capture both local and global inductive biases of the given image. 
%
In contrast, the traditional ViT adapter's feature space is mainly obtained through self-attention and may not be able to capture local features well. Therefore, according to the MMD theory~\citep{cheng2021neural,arbel2019maximum,wang2021rethinking}, we can conclude that if the MEA block's feature space is more discriminative than the traditional ViT adapter's feature space, then the MEA block's feature space is more suitable for adapter feature space and can better improve the model's generalization capability and adaptability.

% -------------------------------------------
\section{Introduction of the Experimental Datasets}
\label{secS2}
% -------------------------------------------
{\color{red}{\emph{This supplementary is for Section~4.1 of the main paper.}}}
In our paper, two representative datasets are used to evaluate the effectiveness and efficiency of our method, including MS-COCO~\citep{caesar2018coco} for ODet and ISeg, and ADE20K~\citep{zhou2017scene} for SSeg. Below are the details of the used datasets:

% -------------------------------
\begin{itemize}
% -------------------------------
\item MS-COCO~\citep{caesar2018coco} is a representative yet challenging dataset for common scene IS and object detection, which consists of $118$k, $5$k and $20$k images for the \emph{training} set, the \emph{val} set and the \emph{test} set, respectively. In our experiments, the model is trained on the \emph{training} set and evaluated on the \emph{val} set.
% -------------------------------
\item ADE20K~\citep{zhou2017scene} is a scene parsing dataset with $20$k images and $150$ object categories. Each image has pixel-level annotations for SS of objects and regions within the scene. The dataset is divided into $20$k, $2$k, and $3$k images for \emph{training}, \emph{val} and \emph{test}, respectively. Our model is trained on the \emph{training} set and evaluated on the \emph{val} set.
% -------------------------------
\end{itemize}
% -------------------------------
For data augmentation, random horizontal flip, brightness jittering and random scaling within the range of $[0.5, 2]$ are used in training as in~\citep{chen2022vision,luo2023forgery,zhang2023cae,mercea2024time}. By default, the inference results are obtained at a single scale, unless explicitly specified otherwise.    


% -------------------------------------------
\section{Introduction of the Experimental Settings}
\label{secS3}
% -------------------------------------------
{\color{red}{\emph{This supplementary is for Section~4.2 of the main paper.}}} Experiments on object detection and instance segmentation are conducted using the open-source MMDetection framework~\citep{chen2019mmdetection}. The training batch size is set to $16$, and AdamW~\citep{loshchilov2017decoupled} is used as the optimizer with the initial learning rate of $1 \times 10^{-4}$ and the weight decay of $0.05$. The layer-wise learning rate decay is used and set to $0.9$, and the drop path rate is set to $0.4$. Following~\citep{xiong2024efficient,wang2021pyramid,chen2022vision,liu2022convnet}, to ensure a fair result comparison, we choose two training schedules, 1$\times$ (\ie, $12$ training epochs) and 3$\times$ (\ie, $36$ training epochs). For the 1$\times$ training schedule, images are resized to the shorter side of 800 pixels, with the longer side not exceeding $1,333$ pixels. In inference, the shorter side of images is consistently set to 800 pixels by default. For the 3$\times$ training schedule, the multi-scale training strategy is also used as in~\citep{chen2022vision}, and the shorter side is resized to $480$ to $800$ pixels, while the longer side remains capped at $1,333$ pixels.

{\color{red}{\emph{This supplementary is for Section~4.3 of the main paper.}}} Experiments on semantic segmentation are conducted using the MMSegmentation framework~\citep{mmseg2020}. The input images are cropped to a fix size of 512 $\times$ 512 pixels as in~\citep{xiong2024efficient,chen2022vision}. The training batch size is set to $16$, and AdamW~\citep{loshchilov2017decoupled} is used as the optimizer with the initial learning rate of $1 \times 10^{-5}$ and the weight decay of $0.05$. Following~\citep{li2022exploring,liu2021swin}, the layer-wise learning rate decay is set to $0.9$ and the drop path rate is set to $0.4$. We report the experimental results on both single scale training and multi-scale training strategies. 
% -------------------------------
\begin{table}[t]
\centering
\small
\renewcommand\arraystretch{1.2}
\setlength{\tabcolsep}{6pt}{
\begin{tabular}{r|r|ccl}
\hline \hline 
Methods & Pre-Trained & Params.$\downarrow$ & AP$^\textrm{m}$ $\uparrow$ \\
\hline 
Swin-B~\citep{liu2021swin} & ImageNet-1k~\citep{deng2009imagenet} & 107.1 &  43.3 \\
ViT-Adapter-B~\citep{chen2022vision} & ImageNet-1k~\citep{deng2009imagenet} & 120.2 & 43.6 \\
\cellcolor[gray]{.95}\textbf{META-B$_{{\textrm{(Ours)}}}$} & \cellcolor[gray]{.95}ImageNet-1k~\citep{deng2009imagenet} & \cellcolor[gray]{.95}115.3 & \cellcolor[gray]{.95}44.3$_{\color{red}{+0.7}}$ \\
\cdashline{1-4}[0.8pt/2pt]
Swin-B~\citep{liu2021swin} & ImageNet-22k~\citep{steiner2021train} & 107.1 & 44.3\\
ViT-Adapter-B~\citep{chen2022vision} & ImageNet-22k~\citep{steiner2021train} & 120.2 & 44.6 \\
\cellcolor[gray]{.95}\textbf{META-B$_{{\textrm{(Ours)}}}$} & \cellcolor[gray]{.95}ImageNet-22k~\citep{steiner2021train} & \cellcolor[gray]{.95}115.3  & \cellcolor[gray]{.95}45.2$_{\color{red}{+0.6}}$ \\
\cdashline{1-4}[0.8pt/2pt]
Swin-B~\citep{liu2021swin} & Multi-Modal~\citep{zhu2022uni} & 107.1 &   -- \\
ViT-Adapter-B~\citep{chen2022vision} & Multi-Modal~\citep{zhu2022uni} & 120.2  & 45.3 \\
\cellcolor[gray]{.95}\textbf{META-B$_{{\textrm{(Ours)}}}$} & \cellcolor[gray]{.95}Multi-Modal~\citep{zhu2022uni} & \cellcolor[gray]{.95}115.3  & \cellcolor[gray]{.95}45.9$_{\color{red}{+0.6}}$ \\
\hline \hline 
\end{tabular}
\caption{Result comparisons on Params. (\textbf{M}) and AP (\%) under different pre-trained weights with Mask R-CNN ($3 \times$ +MS schedule)~\citep{he2017mask} as the baseline model on the \emph{val} set of MS-COCO~\citep{caesar2018coco}. ``--'' denotes there is no such a result in its paper.}
\label{tab3}}
\end{table}
% -------------------------------

% -------------------------------------------
\section{Result Comparisons under Different Weights}
\label{secS4}
% -------------------------------------------
{\color{red}{\emph{This supplementary is for Section~4.2 of the main paper.}}} In this section, we present the experimental results of META on object detection and instance segmentation with different pre-trained weights and compare them with other state-of-the-art methods including SwinViT~\citep{liu2021swin} and ViT-Adapter~\citep{chen2022vision} as in~\citep{chen2022vision}. 
Mask R-CNN~\citep{he2017mask} is used as the baseline, and ViT-B~\citep{li2022exploring} is used as the backbone. The 3$\times$ training schedule with MS training strategy is used. The obtained experimental results are given in Table~\ref{tab3}.
%
From this table, we can observe that our method is applicable to different pre-trained weights (\ie, ImageNet-1k~\citep{deng2009imagenet}, ImageNet-22k~\citep{steiner2021train}, and Multi-Modal~\citep{zhu2022uni}), and achieves more accurate AP with fewer model parameters compared to ViT-Adapter~\citep{chen2022vision}, across different pre-trained weights.  

% -------------------------------------------
\section{More Ablation Study Results}
\label{secS5}
% -------------------------------------------
{\color{red}{\emph{This supplementary is for Section~4.4 of the main paper.}}} In our main paper, we present the experimental results of deploying adapters with Attn branch and FFN branch as components on ViT-B~\citep{li2022exploring}. It is noteworthy that the layer normalization operation has been shared between the Attn branch and the FFN branch to reduce the memory access costs associated with the normalization operations. In this section, we demonstrate a result comparison between the experimental results of using shared layer normalization operation and those of not using it in the traditional setting (\ie, the non-shared normalization). The obtained experimental results are shown in Table~\ref{tab:s1}. It can be observed that sharing layer normalization does not significantly improve the performance in terms of AP. However, compared to FPS, FLOPs, MC, our approach can achieve satisfactory performance gains.
% --------------------------
\begin{table*}[t]
\centering
\renewcommand\arraystretch{1.2}
\setlength{\tabcolsep}{1pt}{
\begin{tabular}{r|ccccc|ccccc}
\hline \hline 
Settings & ViT-B & Attn & FFN & Conv & Cascade & AP$^\textrm{m}$ $\uparrow$ & FPS$\uparrow$ & Params.$\downarrow$ & FLOPs$\downarrow$ & MC$\downarrow$ \\
\hline 
Baseline model & \cmark & \xmark & \xmark & \xmark & \xmark & 41.3 & 11.5 & 113.6\textbf{M} & 719\textbf{G} & NA\\
\cdashline{1-11}[0.8pt/2pt]
\cellcolor[gray]{.95}Shared normalization & \cmark & \cmark & \cmark & \xmark & \xmark & \cellcolor[gray]{.95}43.4 & \cellcolor[gray]{.95}11.3 & \cellcolor[gray]{.95}114.4\textbf{M} & \cellcolor[gray]{.95}719\textbf{G} & \cellcolor[gray]{.95}7.5\textbf{GB}\\
Non-shared normalization & \cmark & \cmark & \cmark & \xmark & \xmark & 43.2 & 10.5 & 114.4\textbf{M} & 737\textbf{G} & 8.8\textbf{GB}\\
\hline \hline 
\end{tabular}
\caption{Ablation study results on shared layer normalization.}
\label{tab:s1}}
\end{table*}
% --------------------------

% -------------------------------
{\color{red}{\emph{This supplementary is for Section~4.4 of the main paper.}}} META is proposed as a simple and fast ViT adapter by minimizing inefficient memory access operations. In this section, we compare META with other efficient attention methods and advanced adapter methods~\citep{marouf2024mini,xia2022vision,sung2022vl}. All methods are used with their default settings and the same settings as the injector and extractor in ViT-adapter~\citep{chen2022vision}. Following the same setup as in~\citep{chen2022vision}, the attention mechanism is utilized as the ViT-adapter layer. Therefore, during the experimental comparisons, we replace the attention mechanism in the ViT-adapter with alternative attention mechanisms to ensure a fair comparison. 
The obtained experimental results are given in Table~\ref{tab6}. We can observe that compared to these methods, META achieves new state-of-the-art performance in both accuracy and efficiency. We ultimately achieve an AP of $44.3\%$ with $115.3$\textbf{M} parameters, $720$\textbf{G} FLOPs, $17.4$ FPS, and 8.1 \textbf{GB} MC. 
% -------------------------------
\begin{table}[t]
\centering
\footnotesize
\renewcommand\arraystretch{1.2}
\setlength{\tabcolsep}{5pt}{
\begin{tabular}{r|ccccc}
\hline \hline 
Methods & AP$\uparrow$ & FPS$\uparrow$ & Params. (\textbf{M})$\downarrow$ & FLOPs (\textbf{G})$\downarrow$  & Momory (\textbf{GB})$\downarrow$ \\
\hline 
WindowAtt~\citep{liu2021swin} & 41.2 & 11.6 & 145.0 & 982 & 18.5 \\
PaleAttention~\citep{wu2022pale} & 42.8 & 14.4 & 155.2 & 1,029 & 16.7\\
Attention~\citep{vaswani2017attention} & 43.1 & 5.2 & 188.4 & 1,250 & 18.3 \\
CSWindow~\citep{dong2022cswin}& 43.1 & 13.7 & 144.6 & 990 & 12.9\\
SimplingAtte~\citep{he2023simplifying} & 43.3 & 12.2 & 126.3 & 994 & 17.1\\
DeformableAtt~\citep{xia2022vision} & 43.7 & 13.5 & 166.0 & 988 & 15.2 \\
\cdashline{1-6}[0.8pt/2pt]
MiniAdapters~\citep{marouf2024mini} & 41.9 & 15.0 & 131.8 & 995 & 12.2 \\
VL-Adapter~\citep{sung2022vl} & 42.7 & 14.5 & 167.2 & 993  & 14.0\\
\cellcolor[gray]{.95}\textbf{META-B$_{{\textrm{(Ours)}}}$} & \cellcolor[gray]{.95}44.3 & \cellcolor[gray]{.95}17.4 & \cellcolor[gray]{.95}115.3 & \cellcolor[gray]{.95}720 & \cellcolor[gray]{.95}8.1\\
\hline \hline 
\end{tabular}
\caption{Result comparisons with different adapters.}
\label{tab6}}
\end{table}
% -------------------------------

% -------------------------------------------
\section{Visualizations under the Conv branch}
\label{secS6}
% -------------------------------------------
{\color{red}{\emph{This supplementary is for Section~3.2 of the main paper.}}} In this section, to observe if the adapter has learned local inductive biases through the Conv branch, we visualize the model's class activation maps. The obtained visualizations are given in Figure~\ref{figs1}. From this figure, it can be observed that after adding the Conv branch, the model focuses more on the specific object area (\eg,`` the dog'' and ``the person'') rather than the surrounding area that may extend beyond the object itself, as was the case before adding the Conv branch. This indicates that our method effectively learns local inductive biases after incorporating the Conv branch.
% -------------------------------------------
% This file was created by matlab2tikz.
%
%The latest updates can be retrieved from
%  http://www.mathworks.com/matlabcentral/fileexchange/22022-matlab2tikz-matlab2tikz
%where you can also make suggestions and rate matlab2tikz.
%
\definecolor{mycolor1}{rgb}{0.21569,0.54902,0.72157}%
\definecolor{mycolor2}{rgb}{0.80784,0.16863,0.12157}%
%
\begin{tikzpicture}

\begin{axis}[%
width=0.898in,
height=1.5in,%3.603in,
at={(0.766in,0.486in)},
scale only axis,
xmin=0,
xmax=10,
ymin=0,
ymax=0.8,
xlabel= \phantom{$z$},
ylabel=$p(g_{z^*}|Y)$,
ylabel near ticks,
title={Linearization-based\\ approach},
title style={align=left}, 
axis background/.style={fill=white},
axis x line*=bottom,
axis y line*=left,
legend style={legend cell align=left, align=left, draw=white!15!black}
]
\addplot[ybar interval, fill=mycolor1, fill opacity=0.4, draw=mycolor1, area legend] table[row sep=crcr] {%
x	y\\
3.36	0.0144927536231884\\
3.429	0.0289855072463768\\
3.498	0.0869565217391305\\
3.567	0.217391304347825\\
3.636	0.391304347826087\\
3.705	0.565217391304348\\
3.774	0.449275362318841\\
3.843	0.405797101449276\\
3.912	0.666666666666667\\
3.981	0.420289855072464\\
4.05	0.478260869565218\\
4.119	0.289855072463768\\
4.188	0.289855072463768\\
4.257	0.347826086956522\\
4.326	0.246376811594203\\
4.395	0.304347826086953\\
4.464	0.20289855072464\\
4.533	0.217391304347823\\
4.602	0.246376811594203\\
4.671	0.289855072463768\\
4.74	0.246376811594203\\
4.809	0.188405797101449\\
4.878	0.231884057971015\\
4.947	0.27536231884058\\
5.016	0.391304347826087\\
5.085	0.246376811594203\\
5.154	0.27536231884058\\
5.223	0.217391304347826\\
5.292	0.347826086956518\\
5.361	0.231884057971018\\
5.43	0.2463768115942\\
5.499	0.260869565217395\\
5.568	0.275362318840576\\
5.637	0.289855072463772\\
5.706	0.304347826086953\\
5.775	0.289855072463772\\
5.844	0.362318840579706\\
5.913	0.289855072463768\\
5.982	0.463768115942029\\
6.051	0.420289855072464\\
6.12	0.492753623188406\\
6.189	0.463768115942029\\
6.258	0.463768115942029\\
6.327	0.420289855072459\\
6.396	0.246376811594206\\
6.465	0.202898550724635\\
6.534	0.0869565217391316\\
6.603	0.0579710144927529\\
6.672	0.0289855072463772\\
6.741	0.0144927536231882\\
6.81	0.0144927536231882\\
};
%\addlegendentry{ground truth}

\addplot [color=mycolor2, line width=2.0pt]
  table[row sep=crcr]{%
0	0.00495647934021539\\
0.01	0.00503120737369003\\
0.02	0.00510691052511148\\
0.03	0.00518359894024572\\
0.04	0.00526128282899922\\
0.05	0.00533997246516116\\
0.06	0.00541967818613227\\
0.07	0.00550041039264009\\
0.08	0.00558217954844049\\
0.09	0.00566499618000535\\
0.1	0.00574887087619621\\
0.11	0.00583381428792371\\
0.12	0.00591983712779276\\
0.13	0.00600695016973323\\
0.14	0.006095164248616\\
0.15	0.00618449025985429\\
0.16	0.00627493915898998\\
0.17	0.00636652196126503\\
0.18	0.0064592497411776\\
0.19	0.0065531336320228\\
0.2	0.00664818482541805\\
0.21	0.0067444145708128\\
0.22	0.0068418341749824\\
0.23	0.00694045500150619\\
0.24	0.00704028847022945\\
0.25	0.00714134605670927\\
0.26	0.00724363929164408\\
0.27	0.00734717976028668\\
0.28	0.00745197910184082\\
0.29	0.00755804900884096\\
0.3	0.00766540122651533\\
0.31	0.00777404755213196\\
0.32	0.00788399983432761\\
0.33	0.00799526997241961\\
0.34	0.00810786991570027\\
0.35	0.00822181166271396\\
0.36	0.00833710726051655\\
0.37	0.00845376880391716\\
0.38	0.00857180843470236\\
0.39	0.00869123834084212\\
0.4	0.00881207075567811\\
0.41	0.00893431795709363\\
0.42	0.00905799226666551\\
0.43	0.00918310604879767\\
0.44	0.00930967170983618\\
0.45	0.00943770169716588\\
0.46	0.00956720849828854\\
0.47	0.00969820463988202\\
0.48	0.00983070268684103\\
0.49	0.00996471524129867\\
0.5	0.0101002549416293\\
0.51	0.0102373344614323\\
0.52	0.0103759665084967\\
0.53	0.0105161638237465\\
0.54	0.0106579391801673\\
0.55	0.0108013053817126\\
0.56	0.010946275262192\\
0.57	0.0110928616841387\\
0.58	0.0112410775376588\\
0.59	0.0113909357392594\\
0.6	0.0115424492306589\\
0.61	0.0116956309775763\\
0.62	0.0118504939685016\\
0.63	0.0120070512134457\\
0.64	0.0121653157426715\\
0.65	0.012325300605404\\
0.66	0.012487018868522\\
0.67	0.0126504836152284\\
0.68	0.0128157079437017\\
0.69	0.0129827049657272\\
0.7	0.0131514878053083\\
0.71	0.0133220695972577\\
0.72	0.0134944634857689\\
0.73	0.0136686826229678\\
0.74	0.0138447401674437\\
0.75	0.0140226492827617\\
0.76	0.0142024231359536\\
0.77	0.0143840748959895\\
0.78	0.0145676177322305\\
0.79	0.0147530648128596\\
0.8	0.0149404293032943\\
0.81	0.0151297243645786\\
0.82	0.0153209631517556\\
0.83	0.0155141588122201\\
0.84	0.0157093244840515\\
0.85	0.0159064732943274\\
0.86	0.0161056183574168\\
0.87	0.016306772773255\\
0.88	0.0165099496255975\\
0.89	0.0167151619802561\\
0.9	0.0169224228833143\\
0.91	0.017131745359324\\
0.92	0.0173431424094836\\
0.93	0.0175566270097958\\
0.94	0.0177722121092072\\
0.95	0.0179899106277294\\
0.96	0.0182097354545399\\
0.97	0.018431699446066\\
0.98	0.0186558154240489\\
0.99	0.0188820961735898\\
1	0.0191105544411779\\
1.01	0.0193412029326999\\
1.02	0.0195740543114314\\
1.03	0.0198091211960107\\
1.04	0.0200464161583947\\
1.05	0.0202859517217969\\
1.06	0.0205277403586084\\
1.07	0.0207717944883015\\
1.08	0.0210181264753159\\
1.09	0.0212667486269282\\
1.1	0.0215176731911045\\
1.11	0.0217709123543368\\
1.12	0.0220264782394623\\
1.13	0.0222843829034672\\
1.14	0.0225446383352741\\
1.15	0.0228072564535138\\
1.16	0.0230722491042814\\
1.17	0.0233396280588773\\
1.18	0.0236094050115328\\
1.19	0.0238815915771207\\
1.2	0.0241561992888519\\
1.21	0.0244332395959568\\
1.22	0.0247127238613533\\
1.23	0.0249946633592999\\
1.24	0.0252790692730363\\
1.25	0.0255659526924097\\
1.26	0.0258553246114888\\
1.27	0.026147195926164\\
1.28	0.0264415774317359\\
1.29	0.0267384798204914\\
1.3	0.0270379136792673\\
1.31	0.0273398894870028\\
1.32	0.0276444176122807\\
1.33	0.0279515083108568\\
1.34	0.0282611717231799\\
1.35	0.0285734178718998\\
1.36	0.0288882566593667\\
1.37	0.0292056978651199\\
1.38	0.0295257511433677\\
1.39	0.0298484260204579\\
1.4	0.0301737318923396\\
1.41	0.0305016780220172\\
1.42	0.0308322735369956\\
1.43	0.0311655274267189\\
1.44	0.0315014485400004\\
1.45	0.0318400455824475\\
1.46	0.0321813271138789\\
1.47	0.032525301545736\\
1.48	0.0328719771384891\\
1.49	0.0332213619990379\\
1.5	0.0335734640781073\\
1.51	0.0339282911676388\\
1.52	0.034285850898178\\
1.53	0.0346461507362581\\
1.54	0.0350091979817809\\
1.55	0.0353749997653947\\
1.56	0.0357435630458696\\
1.57	0.0361148946074719\\
1.58	0.0364890010573359\\
1.59	0.0368658888228363\\
1.6	0.0372455641489584\\
1.61	0.0376280330956701\\
1.62	0.0380133015352924\\
1.63	0.0384013751498731\\
1.64	0.0387922594285599\\
1.65	0.0391859596649766\\
1.66	0.0395824809546015\\
1.67	0.0399818281921483\\
1.68	0.040384006068951\\
1.69	0.0407890190703521\\
1.7	0.0411968714730958\\
1.71	0.0416075673427256\\
1.72	0.0420211105309874\\
1.73	0.0424375046732393\\
1.74	0.0428567531858665\\
1.75	0.0432788592637046\\
1.76	0.0437038258774694\\
1.77	0.0441316557711956\\
1.78	0.0445623514596833\\
1.79	0.0449959152259539\\
1.8	0.0454323491187165\\
1.81	0.0458716549498434\\
1.82	0.0463138342918566\\
1.83	0.0467588884754263\\
1.84	0.047206818586881\\
1.85	0.0476576254657295\\
1.86	0.0481113097021966\\
1.87	0.0485678716347722\\
1.88	0.0490273113477746\\
1.89	0.0494896286689283\\
1.9	0.0499548231669573\\
1.91	0.0504228941491942\\
1.92	0.0508938406592059\\
1.93	0.0513676614744362\\
1.94	0.0518443551038658\\
1.95	0.0523239197856913\\
1.96	0.0528063534850216\\
1.97	0.0532916538915952\\
1.98	0.0537798184175165\\
1.99	0.0542708441950129\\
2	0.0547647280742127\\
2.01	0.0552614666209458\\
2.02	0.0557610561145651\\
2.03	0.0562634925457924\\
2.04	0.0567687716145866\\
2.05	0.0572768887280368\\
2.06	0.0577878389982796\\
2.07	0.0583016172404419\\
2.08	0.0588182179706094\\
2.09	0.0593376354038221\\
2.1	0.0598598634520962\\
2.11	0.0603848957224738\\
2.12	0.0609127255151013\\
2.13	0.0614433458213363\\
2.14	0.0619767493218837\\
2.15	0.0625129283849627\\
2.16	0.063051875064503\\
2.17	0.0635935810983738\\
2.18	0.0641380379066434\\
2.19	0.0646852365898719\\
2.2	0.0652351679274361\\
2.21	0.0657878223758891\\
2.22	0.0663431900673527\\
2.23	0.0669012608079454\\
2.24	0.0674620240762456\\
2.25	0.0680254690217902\\
2.26	0.06859158446361\\
2.27	0.0691603588888017\\
2.28	0.0697317804511384\\
2.29	0.0703058369697168\\
2.3	0.070882515927644\\
2.31	0.0714618044707637\\
2.32	0.0720436894064212\\
2.33	0.0726281572022699\\
2.34	0.0732151939851174\\
2.35	0.0738047855398146\\
2.36	0.0743969173081847\\
2.37	0.0749915743879969\\
2.38	0.0755887415319811\\
2.39	0.0761884031468874\\
2.4	0.0767905432925892\\
2.41	0.0773951456812309\\
2.42	0.0780021936764202\\
2.43	0.0786116702924671\\
2.44	0.0792235581936674\\
2.45	0.079837839693634\\
2.46	0.0804544967546748\\
2.47	0.0810735109872175\\
2.48	0.0816948636492834\\
2.49	0.0823185356460092\\
2.5	0.0829445075292169\\
2.51	0.0835727594970344\\
2.52	0.084203271393565\\
2.53	0.0848360227086074\\
2.54	0.0854709925774259\\
2.55	0.0861081597805728\\
2.56	0.0867475027437606\\
2.57	0.0873889995377879\\
2.58	0.0880326278785163\\
2.59	0.0886783651269002\\
2.6	0.0893261882890704\\
2.61	0.0899760740164704\\
2.62	0.0906279986060465\\
2.63	0.0912819380004926\\
2.64	0.0919378677885496\\
2.65	0.0925957632053583\\
2.66	0.0932555991328697\\
2.67	0.0939173501003085\\
2.68	0.0945809902846947\\
2.69	0.0952464935114191\\
2.7	0.0959138332548775\\
2.71	0.0965829826391593\\
2.72	0.0972539144387952\\
2.73	0.0979266010795609\\
2.74	0.0986010146393384\\
2.75	0.0992771268490357\\
2.76	0.0999549090935642\\
2.77	0.100634332412874\\
2.78	0.101315367503047\\
2.79	0.101997984717452\\
2.8	0.102682154067954\\
2.81	0.103367845226185\\
2.82	0.104055027524874\\
2.83	0.104743669959238\\
2.84	0.105433741188425\\
2.85	0.106125209537028\\
2.86	0.106818042996652\\
2.87	0.107512209227537\\
2.88	0.108207675560252\\
2.89	0.108904408997439\\
2.9	0.109602376215622\\
2.91	0.110301543567076\\
2.92	0.111001877081754\\
2.93	0.111703342469276\\
2.94	0.112405905120979\\
2.95	0.113109530112027\\
2.96	0.113814182203577\\
2.97	0.114519825845015\\
2.98	0.115226425176241\\
2.99	0.115933944030024\\
3	0.116642345934409\\
3.01	0.117351594115192\\
3.02	0.118061651498449\\
3.03	0.118772480713129\\
3.04	0.119484044093701\\
3.05	0.120196303682873\\
3.06	0.120909221234355\\
3.07	0.121622758215693\\
3.08	0.122336875811159\\
3.09	0.123051534924699\\
3.1	0.123766696182943\\
3.11	0.124482319938267\\
3.12	0.125198366271925\\
3.13	0.12591479499723\\
3.14	0.126631565662796\\
3.15	0.127348637555838\\
3.16	0.128065969705534\\
3.17	0.128783520886434\\
3.18	0.129501249621939\\
3.19	0.130219114187826\\
3.2	0.130937072615837\\
3.21	0.131655082697321\\
3.22	0.132373101986931\\
3.23	0.133091087806378\\
3.24	0.133808997248241\\
3.25	0.13452678717983\\
3.26	0.135244414247106\\
3.27	0.135961834878651\\
3.28	0.136679005289694\\
3.29	0.137395881486191\\
3.3	0.138112419268956\\
3.31	0.138828574237848\\
3.32	0.139544301796001\\
3.33	0.140259557154114\\
3.34	0.14097429533479\\
3.35	0.141688471176923\\
3.36	0.142402039340136\\
3.37	0.143114954309265\\
3.38	0.143827170398901\\
3.39	0.144538641757969\\
3.4	0.145249322374359\\
3.41	0.145959166079606\\
3.42	0.146668126553618\\
3.43	0.147376157329438\\
3.44	0.148083211798065\\
3.45	0.148789243213316\\
3.46	0.149494204696722\\
3.47	0.150198049242482\\
3.48	0.150900729722448\\
3.49	0.151602198891158\\
3.5	0.152302409390905\\
3.51	0.153001313756856\\
3.52	0.153698864422194\\
3.53	0.154395013723319\\
3.54	0.155089713905069\\
3.55	0.155782917125992\\
3.56	0.156474575463644\\
3.57	0.157164640919932\\
3.58	0.157853065426486\\
3.59	0.158539800850068\\
3.6	0.15922479899801\\
3.61	0.159908011623694\\
3.62	0.160589390432051\\
3.63	0.161268887085104\\
3.64	0.16194645320753\\
3.65	0.16262204039226\\
3.66	0.163295600206099\\
3.67	0.163967084195381\\
3.68	0.164636443891645\\
3.69	0.165303630817341\\
3.7	0.165968596491557\\
3.71	0.166631292435772\\
3.72	0.167291670179631\\
3.73	0.167949681266742\\
3.74	0.168605277260496\\
3.75	0.169258409749901\\
3.76	0.169909030355445\\
3.77	0.170557090734967\\
3.78	0.171202542589549\\
3.79	0.171845337669428\\
3.8	0.17248542777992\\
3.81	0.173122764787353\\
3.82	0.173757300625023\\
3.83	0.174388987299154\\
3.84	0.175017776894874\\
3.85	0.1756436215822\\
3.86	0.176266473622029\\
3.87	0.176886285372142\\
3.88	0.177503009293208\\
3.89	0.178116597954805\\
3.9	0.178727004041434\\
3.91	0.179334180358544\\
3.92	0.179938079838555\\
3.93	0.180538655546892\\
3.94	0.181135860688006\\
3.95	0.181729648611403\\
3.96	0.182319972817676\\
3.97	0.182906786964519\\
3.98	0.183490044872755\\
3.99	0.184069700532347\\
4	0.184645708108407\\
4.01	0.185218021947199\\
4.02	0.185786596582135\\
4.03	0.186351386739756\\
4.04	0.186912347345709\\
4.05	0.187469433530713\\
4.06	0.188022600636505\\
4.07	0.188571804221778\\
4.08	0.189117000068111\\
4.09	0.189658144185865\\
4.1	0.190195192820082\\
4.11	0.190728102456354\\
4.12	0.191256829826677\\
4.13	0.191781331915283\\
4.14	0.192301565964453\\
4.15	0.192817489480306\\
4.16	0.193329060238567\\
4.17	0.193836236290307\\
4.18	0.194338975967659\\
4.19	0.19483723788951\\
4.2	0.195330980967159\\
4.21	0.195820164409952\\
4.22	0.196304747730887\\
4.23	0.196784690752181\\
4.24	0.197259953610816\\
4.25	0.197730496764038\\
4.26	0.198196280994839\\
4.27	0.198657267417387\\
4.28	0.19911341748243\\
4.29	0.199564692982658\\
4.3	0.200011056058033\\
4.31	0.200452469201067\\
4.32	0.200888895262076\\
4.33	0.201320297454377\\
4.34	0.201746639359454\\
4.35	0.202167884932071\\
4.36	0.202583998505353\\
4.37	0.202994944795806\\
4.38	0.203400688908302\\
4.39	0.203801196341013\\
4.4	0.204196432990295\\
4.41	0.204586365155525\\
4.42	0.204970959543887\\
4.43	0.205350183275104\\
4.44	0.205724003886121\\
4.45	0.206092389335735\\
4.46	0.206455308009166\\
4.47	0.206812728722581\\
4.48	0.207164620727553\\
4.49	0.207510953715471\\
4.5	0.207851697821886\\
4.51	0.208186823630804\\
4.52	0.208516302178914\\
4.53	0.208840104959761\\
4.54	0.209158203927852\\
4.55	0.209470571502708\\
4.56	0.209777180572843\\
4.57	0.210078004499687\\
4.58	0.210373017121445\\
4.59	0.210662192756884\\
4.6	0.21094550620906\\
4.61	0.211222932768978\\
4.62	0.211494448219179\\
4.63	0.211760028837266\\
4.64	0.212019651399358\\
4.65	0.212273293183472\\
4.66	0.212520931972839\\
4.67	0.212762546059146\\
4.68	0.212998114245707\\
4.69	0.213227615850562\\
4.7	0.213451030709505\\
4.71	0.213668339179034\\
4.72	0.21387952213923\\
4.73	0.214084560996561\\
4.74	0.214283437686611\\
4.75	0.214476134676732\\
4.76	0.214662634968617\\
4.77	0.214842922100803\\
4.78	0.215016980151093\\
4.79	0.215184793738895\\
4.8	0.21534634802749\\
4.81	0.21550162872622\\
4.82	0.215650622092589\\
4.83	0.215793314934298\\
4.84	0.215929694611184\\
4.85	0.216059749037091\\
4.86	0.216183466681654\\
4.87	0.216300836572001\\
4.88	0.21641184829438\\
4.89	0.21651649199569\\
4.9	0.216614758384949\\
4.91	0.21670663873466\\
4.92	0.216792124882109\\
4.93	0.21687120923057\\
4.94	0.216943884750432\\
4.95	0.21701014498024\\
4.96	0.217069984027651\\
4.97	0.21712339657031\\
4.98	0.217170377856636\\
4.99	0.217210923706529\\
5	0.21724503051199\\
5.01	0.217272695237652\\
5.02	0.217293915421236\\
5.03	0.217308689173911\\
5.04	0.217317015180578\\
5.05	0.217318892700066\\
5.06	0.217314321565234\\
5.07	0.217303302183007\\
5.08	0.217285835534308\\
5.09	0.217261923173913\\
5.1	0.217231567230225\\
5.11	0.217194770404953\\
5.12	0.217151535972715\\
5.13	0.217101867780549\\
5.14	0.217045770247346\\
5.15	0.216983248363191\\
5.16	0.216914307688628\\
5.17	0.21683895435383\\
5.18	0.216757195057696\\
5.19	0.216669037066854\\
5.2	0.216574488214588\\
5.21	0.216473556899676\\
5.22	0.216366252085147\\
5.23	0.216252583296955\\
5.24	0.216132560622568\\
5.25	0.216006194709479\\
5.26	0.215873496763628\\
5.27	0.215734478547749\\
5.28	0.21558915237963\\
5.29	0.215437531130296\\
5.3	0.215279628222107\\
5.31	0.215115457626779\\
5.32	0.214945033863324\\
5.33	0.21476837199591\\
5.34	0.214585487631638\\
5.35	0.21439639691825\\
5.36	0.21420111654175\\
5.37	0.213999663723949\\
5.38	0.213792056219935\\
5.39	0.213578312315465\\
5.4	0.213358450824282\\
5.41	0.213132491085352\\
5.42	0.212900452960033\\
5.43	0.212662356829162\\
5.44	0.212418223590075\\
5.45	0.212168074653546\\
5.46	0.211911931940665\\
5.47	0.211649817879627\\
5.48	0.211381755402469\\
5.49	0.21110776794172\\
5.5	0.210827879426989\\
5.51	0.210542114281486\\
5.52	0.210250497418463\\
5.53	0.209953054237604\\
5.54	0.209649810621332\\
5.55	0.209340792931057\\
5.56	0.209026028003361\\
5.57	0.208705543146111\\
5.58	0.208379366134512\\
5.59	0.2080475252071\\
5.6	0.207710049061664\\
5.61	0.207366966851112\\
5.62	0.207018308179279\\
5.63	0.206664103096667\\
5.64	0.206304382096132\\
5.65	0.205939176108512\\
5.66	0.205568516498194\\
5.67	0.20519243505863\\
5.68	0.204810964007791\\
5.69	0.204424135983574\\
5.7	0.204031984039149\\
5.71	0.203634541638252\\
5.72	0.203231842650435\\
5.73	0.202823921346256\\
5.74	0.202410812392424\\
5.75	0.201992550846891\\
5.76	0.201569172153899\\
5.77	0.201140712138983\\
5.78	0.200707207003916\\
5.79	0.200268693321625\\
5.8	0.199825208031051\\
5.81	0.199376788431969\\
5.82	0.198923472179769\\
5.83	0.198465297280192\\
5.84	0.198002302084028\\
5.85	0.197534525281773\\
5.86	0.197062005898254\\
5.87	0.196584783287207\\
5.88	0.196102897125829\\
5.89	0.195616387409288\\
5.9	0.195125294445206\\
5.91	0.1946296588481\\
5.92	0.194129521533804\\
5.93	0.193624923713846\\
5.94	0.193115906889808\\
5.95	0.192602512847652\\
5.96	0.192084783652017\\
5.97	0.191562761640493\\
5.98	0.19103648941787\\
5.99	0.19050600985036\\
6	0.189971366059799\\
6.01	0.189432601417827\\
6.02	0.188889759540044\\
6.03	0.188342884280149\\
6.04	0.187792019724062\\
6.05	0.187237210184025\\
6.06	0.186678500192687\\
6.07	0.186115934497179\\
6.08	0.185549558053167\\
6.09	0.184979416018895\\
6.1	0.184405553749221\\
6.11	0.183828016789635\\
6.12	0.183246850870269\\
6.13	0.182662101899903\\
6.14	0.182073815959955\\
6.15	0.181482039298473\\
6.16	0.180886818324118\\
6.17	0.18028819960014\\
6.18	0.179686229838354\\
6.19	0.179080955893116\\
6.2	0.17847242475529\\
6.21	0.177860683546225\\
6.22	0.177245779511726\\
6.23	0.176627760016027\\
6.24	0.176006672535775\\
6.25	0.175382564654008\\
6.26	0.174755484054144\\
6.27	0.174125478513977\\
6.28	0.173492595899676\\
6.29	0.1728568841598\\
6.3	0.172218391319313\\
6.31	0.171577165473615\\
6.32	0.170933254782588\\
6.33	0.170286707464643\\
6.34	0.169637571790795\\
6.35	0.168985896078742\\
6.36	0.168331728686962\\
6.37	0.167675118008831\\
6.38	0.167016112466753\\
6.39	0.166354760506308\\
6.4	0.165691110590427\\
6.41	0.16502521119358\\
6.42	0.164357110795988\\
6.43	0.163686857877857\\
6.44	0.163014500913636\\
6.45	0.162340088366297\\
6.46	0.161663668681645\\
6.47	0.160985290282647\\
6.48	0.160305001563796\\
6.49	0.159622850885492\\
6.5	0.158938886568466\\
6.51	0.15825315688822\\
6.52	0.157565710069504\\
6.53	0.156876594280826\\
6.54	0.156185857628989\\
6.55	0.155493548153663\\
6.56	0.154799713821993\\
6.57	0.154104402523241\\
6.58	0.153407662063457\\
6.59	0.152709540160195\\
6.6	0.152010084437263\\
6.61	0.151309342419508\\
6.62	0.15060736152764\\
6.63	0.1499041890731\\
6.64	0.149199872252962\\
6.65	0.148494458144878\\
6.66	0.147787993702068\\
6.67	0.147080525748341\\
6.68	0.146372100973175\\
6.69	0.145662765926825\\
6.7	0.144952567015486\\
6.71	0.144241550496492\\
6.72	0.14352976247357\\
6.73	0.14281724889213\\
6.74	0.142104055534609\\
6.75	0.141390228015859\\
6.76	0.140675811778582\\
6.77	0.139960852088818\\
6.78	0.139245394031472\\
6.79	0.138529482505904\\
6.8	0.137813162221557\\
6.81	0.137096477693644\\
6.82	0.136379473238879\\
6.83	0.135662192971265\\
6.84	0.134944680797929\\
6.85	0.134226980415018\\
6.86	0.133509135303637\\
6.87	0.132791188725845\\
6.88	0.132073183720711\\
6.89	0.131355163100412\\
6.9	0.130637169446397\\
6.91	0.129919245105597\\
6.92	0.129201432186696\\
6.93	0.128483772556459\\
6.94	0.127766307836106\\
6.95	0.127049079397757\\
6.96	0.126332128360921\\
6.97	0.12561549558905\\
6.98	0.124899221686146\\
6.99	0.124183346993427\\
7	0.123467911586051\\
7.01	0.122752955269899\\
7.02	0.122038517578414\\
7.03	0.1213246377695\\
7.04	0.12061135482248\\
7.05	0.119898707435113\\
7.06	0.119186734020668\\
7.07	0.118475472705061\\
7.08	0.117764961324047\\
7.09	0.117055237420477\\
7.1	0.116346338241611\\
7.11	0.11563830073649\\
7.12	0.114931161553372\\
7.13	0.114224957037223\\
7.14	0.113519723227275\\
7.15	0.112815495854637\\
7.16	0.112112310339969\\
7.17	0.111410201791219\\
7.18	0.110709205001419\\
7.19	0.110009354446535\\
7.2	0.109310684283388\\
7.21	0.108613228347628\\
7.22	0.107917020151773\\
7.23	0.107222092883299\\
7.24	0.106528479402805\\
7.25	0.105836212242227\\
7.26	0.105145323603112\\
7.27	0.104455845354962\\
7.28	0.103767809033623\\
7.29	0.103081245839751\\
7.3	0.102396186637321\\
7.31	0.101712661952208\\
7.32	0.101030701970822\\
7.33	0.100350336538802\\
7.34	0.0996715951597692\\
7.35	0.0989945069941437\\
7.36	0.0983191008580132\\
7.37	0.0976454052220644\\
7.38	0.0969734482105724\\
7.39	0.0963032576004468\\
7.4	0.0956348608203369\\
7.41	0.0949682849497942\\
7.42	0.0943035567184924\\
7.43	0.093640702505504\\
7.44	0.0929797483386348\\
7.45	0.0923207198938145\\
7.46	0.0916636424945427\\
7.47	0.0910085411113931\\
7.48	0.0903554403615708\\
7.49	0.0897043645085274\\
7.5	0.0890553374616292\\
7.51	0.0884083827758814\\
7.52	0.0877635236517063\\
7.53	0.0871207829347749\\
7.54	0.0864801831158936\\
7.55	0.0858417463309428\\
7.56	0.0852054943608689\\
7.57	0.0845714486317291\\
7.58	0.083939630214788\\
7.59	0.0833100598266659\\
7.6	0.0826827578295388\\
7.61	0.0820577442313888\\
7.62	0.0814350386863051\\
7.63	0.0808146604948357\\
7.64	0.0801966286043874\\
7.65	0.0795809616096762\\
7.66	0.0789676777532255\\
7.67	0.0783567949259133\\
7.68	0.0777483306675663\\
7.69	0.0771423021676022\\
7.7	0.0765387262657182\\
7.71	0.0759376194526259\\
7.72	0.0753389978708321\\
7.73	0.0747428773154653\\
7.74	0.0741492732351464\\
7.75	0.0735582007329042\\
7.76	0.0729696745671344\\
7.77	0.0723837091526021\\
7.78	0.071800318561487\\
7.79	0.0712195165244711\\
7.8	0.0706413164318677\\
7.81	0.0700657313347914\\
7.82	0.0694927739463699\\
7.83	0.0689224566429943\\
7.84	0.0683547914656103\\
7.85	0.0677897901210473\\
7.86	0.067227463983387\\
7.87	0.0666678240953688\\
7.88	0.0661108811698337\\
7.89	0.0655566455912037\\
7.9	0.0650051274169981\\
7.91	0.0644563363793857\\
7.92	0.0639102818867709\\
7.93	0.0633669730254157\\
7.94	0.0628264185610945\\
7.95	0.0622886269407826\\
7.96	0.0617536062943775\\
7.97	0.0612213644364519\\
7.98	0.060691908868039\\
7.99	0.0601652467784482\\
8	0.0596413850471111\\
8.01	0.0591203302454577\\
8.02	0.0586020886388212\\
8.03	0.0580866661883726\\
8.04	0.0575740685530815\\
8.05	0.0570643010917059\\
8.06	0.0565573688648083\\
8.07	0.0560532766367973\\
8.08	0.0555520288779961\\
8.09	0.0550536297667352\\
8.1	0.0545580831914697\\
8.11	0.0540653927529206\\
8.12	0.0535755617662392\\
8.13	0.0530885932631942\\
8.14	0.0526044899943808\\
8.15	0.052123254431452\\
8.16	0.0516448887693693\\
8.17	0.0511693949286748\\
8.18	0.0506967745577832\\
8.19	0.050227029035292\\
8.2	0.0497601594723113\\
8.21	0.0492961667148101\\
8.22	0.0488350513459819\\
8.23	0.0483768136886253\\
8.24	0.0479214538075419\\
8.25	0.0474689715119493\\
8.26	0.0470193663579096\\
8.27	0.0465726376507721\\
8.28	0.0461287844476299\\
8.29	0.0456878055597898\\
8.3	0.0452496995552554\\
8.31	0.0448144647612222\\
8.32	0.0443820992665838\\
8.33	0.0439526009244504\\
8.34	0.0435259673546765\\
8.35	0.0431021959463999\\
8.36	0.0426812838605893\\
8.37	0.042263228032601\\
8.38	0.041848025174744\\
8.39	0.0414356717788534\\
8.4	0.0410261641188703\\
8.41	0.0406194982534291\\
8.42	0.0402156700284508\\
8.43	0.0398146750797419\\
8.44	0.039416508835599\\
8.45	0.0390211665194178\\
8.46	0.0386286431523058\\
8.47	0.0382389335556999\\
8.48	0.0378520323539862\\
8.49	0.0374679339771224\\
8.5	0.0370866326632633\\
8.51	0.0367081224613873\\
8.52	0.0363323972339243\\
8.53	0.0359594506593843\\
8.54	0.0355892762349867\\
8.55	0.0352218672792885\\
8.56	0.034857216934813\\
8.57	0.0344953181706767\\
8.58	0.0341361637852141\\
8.59	0.0337797464086021\\
8.6	0.0334260585054804\\
8.61	0.0330750923775696\\
8.62	0.0327268401662861\\
8.63	0.0323812938553532\\
8.64	0.0320384452734075\\
8.65	0.0316982860966018\\
8.66	0.0313608078512014\\
8.67	0.0310260019161767\\
8.68	0.030693859525789\\
8.69	0.0303643717721696\\
8.7	0.0300375296078941\\
8.71	0.0297133238485476\\
8.72	0.0293917451752839\\
8.73	0.0290727841373765\\
8.74	0.0287564311547617\\
8.75	0.0284426765205727\\
8.76	0.0281315104036655\\
8.77	0.0278229228511355\\
8.78	0.027516903790824\\
8.79	0.0272134430338157\\
8.8	0.0269125302769252\\
8.81	0.0266141551051738\\
8.82	0.0263183069942544\\
8.83	0.0260249753129863\\
8.84	0.0257341493257578\\
8.85	0.0254458181949572\\
8.86	0.0251599709833921\\
8.87	0.0248765966566958\\
8.88	0.0245956840857209\\
8.89	0.0243172220489209\\
8.9	0.0240411992347177\\
8.91	0.0237676042438558\\
8.92	0.0234964255917429\\
8.93	0.0232276517107765\\
8.94	0.0229612709526558\\
8.95	0.0226972715906801\\
8.96	0.0224356418220311\\
8.97	0.0221763697700415\\
8.98	0.0219194434864471\\
8.99	0.0216648509536248\\
9	0.021412580086814\\
9.01	0.0211626187363224\\
9.02	0.0209149546897159\\
9.03	0.0206695756739917\\
9.04	0.0204264693577359\\
9.05	0.0201856233532632\\
9.06	0.0199470252187409\\
9.07	0.0197106624602952\\
9.08	0.0194765225341003\\
9.09	0.0192445928484506\\
9.1	0.0190148607658151\\
9.11	0.0187873136048738\\
9.12	0.018561938642537\\
9.13	0.0183387231159461\\
9.14	0.0181176542244563\\
9.15	0.0178987191316015\\
9.16	0.01768190496704\\
9.17	0.0174671988284827\\
9.18	0.0172545877836021\\
9.19	0.0170440588719226\\
9.2	0.0168355991066919\\
9.21	0.0166291954767339\\
9.22	0.0164248349482823\\
9.23	0.0162225044667949\\
9.24	0.0160221909587488\\
9.25	0.0158238813334166\\
9.26	0.015627562484623\\
9.27	0.0154332212924815\\
9.28	0.015240844625113\\
9.29	0.0150504193403427\\
9.3	0.0148619322873796\\
9.31	0.0146753703084749\\
9.32	0.0144907202405607\\
9.33	0.0143079689168702\\
9.34	0.0141271031685362\\
9.35	0.0139481098261715\\
9.36	0.0137709757214283\\
9.37	0.0135956876885382\\
9.38	0.0134222325658323\\
9.39	0.0132505971972412\\
9.4	0.013080768433775\\
9.41	0.0129127331349834\\
9.42	0.0127464781703963\\
9.43	0.0125819904209438\\
9.44	0.0124192567803563\\
9.45	0.0122582641565455\\
9.46	0.0120989994729642\\
9.47	0.0119414496699472\\
9.48	0.011785601706032\\
9.49	0.0116314425592593\\
9.5	0.0114789592284545\\
9.51	0.0113281387344882\\
9.52	0.0111789681215181\\
9.53	0.0110314344582108\\
9.54	0.0108855248389431\\
9.55	0.0107412263849851\\
9.56	0.0105985262456626\\
9.57	0.0104574115995006\\
9.58	0.0103178696553468\\
9.59	0.010179887653476\\
9.6	0.0100434528666753\\
9.61	0.00990855260130987\\
9.62	0.00977517419836908\\
9.63	0.00964330503449418\\
9.64	0.00951293252298657\\
9.65	0.00938404411479698\\
9.66	0.00925662729949601\\
9.67	0.00913066960622556\\
9.68	0.00900615860463183\\
9.69	0.00888308190577935\\
9.7	0.00876142716304676\\
9.71	0.00864118207300384\\
9.72	0.00852233437627048\\
9.73	0.00840487185835706\\
9.74	0.00828878235048692\\
9.75	0.0081740537304007\\
9.76	0.00806067392314265\\
9.77	0.00794863090182918\\
9.78	0.0078379126883997\\
9.79	0.00772850735434974\\
9.8	0.00762040302144666\\
9.81	0.00751358786242797\\
9.82	0.00740805010168226\\
9.83	0.00730377801591319\\
9.84	0.00720075993478624\\
9.85	0.00709898424155877\\
9.86	0.00699843937369316\\
9.87	0.00689911382345334\\
9.88	0.00680099613848483\\
9.89	0.00670407492237845\\
9.9	0.00660833883521762\\
9.91	0.00651377659410972\\
9.92	0.00642037697370137\\
9.93	0.00632812880667791\\
9.94	0.00623702098424713\\
9.95	0.00614704245660751\\
9.96	0.00605818223340095\\
9.97	0.00597042938415035\\
9.98	0.00588377303868186\\
9.99	0.00579820238753227\\
10	0.00571370668234159\\
};
%\addlegendentry{linearization}

\addplot [color=mycolor2, line width=2.0pt, forget plot]
  table[row sep=crcr]{%
5.04791147756762	0\\
5.04791147756762	0.6\\
};
\addplot [color=mycolor1, dashed, line width=2.0pt, forget plot]
  table[row sep=crcr]{%
5.0284309151552	0\\
5.0284309151552	0.6\\
};
\end{axis}

\begin{axis}[%
width=0.898in,
height=1.5in,%3.603in,
at={(1.981in,0.486in)},
scale only axis,
xmin=0,
xmax=10,
ymin=0,
ymax=0.8,
axis background/.style={fill=white},
title={Exact moment \\ matching},
title style={align=left}, 
axis x line*=bottom,
axis y line*=left,
legend style={legend cell align=left, align=left, draw=white!15!black}
]
\addplot[ybar interval, fill=mycolor1, fill opacity=0.4, draw=mycolor1, area legend] table[row sep=crcr] {%
x	y\\
3.36	0.0144927536231884\\
3.429	0.0289855072463768\\
3.498	0.0869565217391305\\
3.567	0.217391304347825\\
3.636	0.391304347826087\\
3.705	0.565217391304348\\
3.774	0.449275362318841\\
3.843	0.405797101449276\\
3.912	0.666666666666667\\
3.981	0.420289855072464\\
4.05	0.478260869565218\\
4.119	0.289855072463768\\
4.188	0.289855072463768\\
4.257	0.347826086956522\\
4.326	0.246376811594203\\
4.395	0.304347826086953\\
4.464	0.20289855072464\\
4.533	0.217391304347823\\
4.602	0.246376811594203\\
4.671	0.289855072463768\\
4.74	0.246376811594203\\
4.809	0.188405797101449\\
4.878	0.231884057971015\\
4.947	0.27536231884058\\
5.016	0.391304347826087\\
5.085	0.246376811594203\\
5.154	0.27536231884058\\
5.223	0.217391304347826\\
5.292	0.347826086956518\\
5.361	0.231884057971018\\
5.43	0.2463768115942\\
5.499	0.260869565217395\\
5.568	0.275362318840576\\
5.637	0.289855072463772\\
5.706	0.304347826086953\\
5.775	0.289855072463772\\
5.844	0.362318840579706\\
5.913	0.289855072463768\\
5.982	0.463768115942029\\
6.051	0.420289855072464\\
6.12	0.492753623188406\\
6.189	0.463768115942029\\
6.258	0.463768115942029\\
6.327	0.420289855072459\\
6.396	0.246376811594206\\
6.465	0.202898550724635\\
6.534	0.0869565217391316\\
6.603	0.0579710144927529\\
6.672	0.0289855072463772\\
6.741	0.0144927536231882\\
6.81	0.0144927536231882\\
};
%\addlegendentry{ground truth}

\addplot [color=mycolor2, line width=2.0pt]
  table[row sep=crcr]{%
0	1.57796213037878e-07\\
0.01	1.6732895884297e-07\\
0.02	1.7741695560253e-07\\
0.03	1.88091260969029e-07\\
0.04	1.99384592349323e-07\\
0.05	2.11331411084634e-07\\
0.06	2.239680106454e-07\\
0.07	2.37332609018677e-07\\
0.08	2.51465445472897e-07\\
0.09	2.66408881892191e-07\\
0.1	2.82207508880122e-07\\
0.11	2.98908256840605e-07\\
0.12	3.16560512251946e-07\\
0.13	3.35216239358466e-07\\
0.14	3.54930107512825e-07\\
0.15	3.75759624411323e-07\\
0.16	3.97765275473713e-07\\
0.17	4.21010669628791e-07\\
0.18	4.45562691776965e-07\\
0.19	4.71491662211309e-07\\
0.2	4.98871503289274e-07\\
0.21	5.27779913658211e-07\\
0.22	5.58298550349138e-07\\
0.23	5.90513219064949e-07\\
0.24	6.24514073001227e-07\\
0.25	6.6039582055038e-07\\
0.26	6.982579422525e-07\\
0.27	7.38204917369618e-07\\
0.28	7.80346460473661e-07\\
0.29	8.24797768452321e-07\\
0.3	8.71679778351659e-07\\
0.31	9.21119436488881e-07\\
0.32	9.73249979284244e-07\\
0.33	1.02821122627662e-06\\
0.34	1.08614988580351e-06\\
0.35	1.14721987384284e-06\\
0.36	1.2115826465312e-06\\
0.37	1.27940754689035e-06\\
0.38	1.35087216631238e-06\\
0.39	1.42616272137205e-06\\
0.4	1.50547444655403e-06\\
0.41	1.58901200350241e-06\\
0.42	1.6769899074197e-06\\
0.43	1.76963297126329e-06\\
0.44	1.86717676840831e-06\\
0.45	1.96986811446753e-06\\
0.46	2.07796556898125e-06\\
0.47	2.19173995771237e-06\\
0.48	2.31147491630579e-06\\
0.49	2.4374674560944e-06\\
0.5	2.57002855285888e-06\\
0.51	2.70948375937301e-06\\
0.52	2.8561738425921e-06\\
0.53	3.01045544636769e-06\\
0.54	3.172701780599e-06\\
0.55	3.34330333775836e-06\\
0.56	3.52266863775558e-06\\
0.57	3.71122500213506e-06\\
0.58	3.90941935862813e-06\\
0.59	4.11771907711229e-06\\
0.6	4.33661283806007e-06\\
0.61	4.56661153458997e-06\\
0.62	4.80824920926364e-06\\
0.63	5.06208402680582e-06\\
0.64	5.32869928395452e-06\\
0.65	5.6087044576838e-06\\
0.66	5.90273629307341e-06\\
0.67	6.21145993213505e-06\\
0.68	6.53557008493801e-06\\
0.69	6.87579224441405e-06\\
0.7	7.23288394625544e-06\\
0.71	7.60763607535751e-06\\
0.72	8.00087422029293e-06\\
0.73	8.41346007734252e-06\\
0.74	8.84629290564536e-06\\
0.75	9.30031103506883e-06\\
0.76	9.77649342843766e-06\\
0.77	1.02758612998007e-05\\
0.78	1.07994797904525e-05\\
0.79	1.13484597044685e-05\\
0.8	1.19239593055503e-05\\
0.81	1.25271861770208e-05\\
0.82	1.31593991468468e-05\\
0.83	1.38219102796108e-05\\
0.84	1.45160869373932e-05\\
0.85	1.5243353911568e-05\\
0.86	1.60051956275572e-05\\
0.87	1.68031584246318e-05\\
0.88	1.7638852912887e-05\\
0.89	1.85139564095642e-05\\
0.9	1.9430215456933e-05\\
0.91	2.03894484239879e-05\\
0.92	2.13935481942593e-05\\
0.93	2.24444849420767e-05\\
0.94	2.35443089996663e-05\\
0.95	2.46951538175072e-05\\
0.96	2.58992390204102e-05\\
0.97	2.71588735618243e-05\\
0.98	2.8476458978919e-05\\
0.99	2.98544927510274e-05\\
1	3.12955717640784e-05\\
1.01	3.28023958836825e-05\\
1.02	3.43777716395763e-05\\
1.03	3.60246160241672e-05\\
1.04	3.77459604079575e-05\\
1.05	3.95449545746657e-05\\
1.06	4.14248708788915e-05\\
1.07	4.33891085292137e-05\\
1.08	4.54411979996362e-05\\
1.09	4.75848055723336e-05\\
1.1	4.98237380146774e-05\\
1.11	5.21619473935505e-05\\
1.12	5.4603536029991e-05\\
1.13	5.71527615972265e-05\\
1.14	5.98140423651886e-05\\
1.15	6.25919625946185e-05\\
1.16	6.54912780838939e-05\\
1.17	6.85169218717338e-05\\
1.18	7.16740100989391e-05\\
1.19	7.49678480323591e-05\\
1.2	7.84039362542751e-05\\
1.21	8.19879770204012e-05\\
1.22	8.5725880789721e-05\\
1.23	8.96237729293646e-05\\
1.24	9.36880005977504e-05\\
1.25	9.79251398092017e-05\\
1.26	0.00010234200268325\\
1.27	0.000106945644881829\\
1.28	0.000111743373237537\\
1.29	0.000116742753576166\\
1.3	0.000121951618736632\\
1.31	0.000127378076791452\\
1.32	0.000133030519470876\\
1.33	0.000138917630793744\\
1.34	0.000145048395908117\\
1.35	0.000151432110144684\\
1.36	0.000158078388285897\\
1.37	0.000164997174053755\\
1.38	0.000172198749819085\\
1.39	0.000179693746535117\\
1.4	0.000187493153898097\\
1.41	0.000195608330737591\\
1.42	0.000204051015639069\\
1.43	0.000212833337801297\\
1.44	0.000221967828130926\\
1.45	0.000231467430576641\\
1.46	0.000241345513705077\\
1.47	0.000251615882520645\\
1.48	0.000262292790531263\\
1.49	0.000273390952061908\\
1.5	0.000284925554817743\\
1.51	0.000296912272698468\\
1.52	0.000309367278865386\\
1.53	0.000322307259062539\\
1.54	0.000335749425193115\\
1.55	0.000349711529152155\\
1.56	0.000364211876916429\\
1.57	0.000379269342892173\\
1.58	0.000394903384521187\\
1.59	0.0004111340571456\\
1.6	0.000427982029131418\\
1.61	0.000445468597250736\\
1.62	0.000463615702322315\\
1.63	0.000482445945109952\\
1.64	0.00050198260247786\\
1.65	0.000522249643802037\\
1.66	0.000543271747636321\\
1.67	0.000565074318631597\\
1.68	0.000587683504706312\\
1.69	0.000611126214466206\\
1.7	0.000635430134870859\\
1.71	0.000660623749144346\\
1.72	0.000686736354927006\\
1.73	0.00071379808266499\\
1.74	0.000741839914233901\\
1.75	0.00077089370179259\\
1.76	0.000800992186862692\\
1.77	0.000832169019629243\\
1.78	0.00086445877845729\\
1.79	0.000897896989619027\\
1.8	0.000932520147225665\\
1.81	0.000968365733357731\\
1.82	0.00100547223838721\\
1.83	0.00104387918148447\\
1.84	0.00108362713130246\\
1.85	0.00112475772683027\\
1.86	0.00116731369840771\\
1.87	0.00121133888889208\\
1.88	0.00125687827496791\\
1.89	0.0013039779885898\\
1.9	0.0013526853385483\\
1.91	0.00140304883214798\\
1.92	0.00145511819698661\\
1.93	0.0015089444028236\\
1.94	0.00156457968352566\\
1.95	0.00162207755907679\\
1.96	0.00168149285763936\\
1.97	0.00174288173765267\\
1.98	0.00180630170995434\\
1.99	0.00187181165990996\\
2	0.00193947186953536\\
2.01	0.00200934403959567\\
2.02	0.00208149131166451\\
2.03	0.00215597829012625\\
2.04	0.00223287106410366\\
2.05	0.00231223722929271\\
2.06	0.00239414590968566\\
2.07	0.002478667779163\\
2.08	0.00256587508293435\\
2.09	0.00265584165880763\\
2.1	0.0027486429582654\\
2.11	0.00284435606732658\\
2.12	0.00294305972717127\\
2.13	0.00304483435450568\\
2.14	0.00314976206164373\\
2.15	0.00325792667628116\\
2.16	0.00336941376093756\\
2.17	0.00348431063204102\\
2.18	0.00360270637862959\\
2.19	0.0037246918806432\\
2.2	0.00385035982677906\\
2.21	0.003979804731883\\
2.22	0.00411312295384885\\
2.23	0.00425041270999705\\
2.24	0.00439177409290354\\
2.25	0.00453730908564923\\
2.26	0.00468712157645985\\
2.27	0.00484131737270557\\
2.28	0.00500000421422928\\
2.29	0.00516329178597171\\
2.3	0.00533129172986171\\
2.31	0.00550411765593867\\
2.32	0.00568188515267452\\
2.33	0.00586471179646173\\
2.34	0.00605271716023359\\
2.35	0.00624602282118281\\
2.36	0.00644475236754353\\
2.37	0.00664903140440251\\
2.38	0.00685898755850369\\
2.39	0.00707475048201145\\
2.4	0.00729645185519606\\
2.41	0.00752422538800619\\
2.42	0.0077582068204918\\
2.43	0.00799853392204129\\
2.44	0.00824534648939642\\
2.45	0.00849878634340852\\
2.46	0.00875899732449895\\
2.47	0.00902612528678757\\
2.48	0.00930031809085181\\
2.49	0.00958172559508029\\
2.5	0.00987049964558341\\
2.51	0.010166794064625\\
2.52	0.0104707646375381\\
2.53	0.0107825690980882\\
2.54	0.0111023671122482\\
2.55	0.0114303202603494\\
2.56	0.0117665920175712\\
2.57	0.0121113477327361\\
2.58	0.0124647546053737\\
2.59	0.0128269816610191\\
2.6	0.0131981997247123\\
2.61	0.0135785813926636\\
2.62	0.0139683010020526\\
2.63	0.0143675345989288\\
2.64	0.0147764599041794\\
2.65	0.0151952562775363\\
2.66	0.0156241046795887\\
2.67	0.0160631876317733\\
2.68	0.0165126891743121\\
2.69	0.0169727948220707\\
2.7	0.0174436915183084\\
2.71	0.0179255675862951\\
2.72	0.0184186126787698\\
2.73	0.0189230177252151\\
2.74	0.0194389748769265\\
2.75	0.0199666774498538\\
2.76	0.0205063198651942\\
2.77	0.0210580975877172\\
2.78	0.0216222070618048\\
2.79	0.0221988456451888\\
2.8	0.0227882115403718\\
2.81	0.023390503723716\\
2.82	0.0240059218721913\\
2.83	0.0246346662877682\\
2.84	0.02527693781945\\
2.85	0.0259329377829358\\
2.86	0.0266028678779091\\
2.87	0.0272869301029493\\
2.88	0.0279853266680635\\
2.89	0.0286982599048398\\
2.9	0.0294259321742241\\
2.91	0.0301685457719252\\
2.92	0.0309263028314531\\
2.93	0.031699405224802\\
2.94	0.0324880544607849\\
2.95	0.0332924515810362\\
2.96	0.0341127970536951\\
2.97	0.0349492906647884\\
2.98	0.0358021314073319\\
2.99	0.0366715173681734\\
3	0.0375576456125996\\
3.01	0.0384607120667382\\
3.02	0.0393809113977787\\
3.03	0.0403184368920498\\
3.04	0.041273480330984\\
3.05	0.0422462318650073\\
3.06	0.0432368798853951\\
3.07	0.0442456108941351\\
3.08	0.045272609371842\\
3.09	0.0463180576437729\\
3.1	0.0473821357439927\\
3.11	0.0484650212777422\\
3.12	0.0495668892820659\\
3.13	0.0506879120847558\\
3.14	0.0518282591616751\\
3.15	0.052988096992522\\
3.16	0.0541675889151041\\
3.17	0.0553668949781894\\
3.18	0.0565861717930083\\
3.19	0.057825572383479\\
3.2	0.0590852460352369\\
3.21	0.0603653381435441\\
3.22	0.0616659900601659\\
3.23	0.0629873389392968\\
3.24	0.0643295175826253\\
3.25	0.0656926542836283\\
3.26	0.0670768726711883\\
3.27	0.0684822915526288\\
3.28	0.0699090247562672\\
3.29	0.0713571809735846\\
3.3	0.0728268636011179\\
3.31	0.0743181705821772\\
3.32	0.0758311942484995\\
3.33	0.0773660211619461\\
3.34	0.0789227319563588\\
3.35	0.0805014011796868\\
3.36	0.0821020971365045\\
3.37	0.0837248817310358\\
3.38	0.0853698103108073\\
3.39	0.0870369315110538\\
3.4	0.0887262870999983\\
3.41	0.0904379118251351\\
3.42	0.0921718332606424\\
3.43	0.093928071656055\\
3.44	0.0957066397863258\\
3.45	0.0975075428034124\\
3.46	0.0993307780895168\\
3.47	0.10117633511212\\
3.48	0.103044195280937\\
3.49	0.104934331806946\\
3.5	0.106846709563605\\
3.51	0.10878128495042\\
3.52	0.110738005758985\\
3.53	0.112716811041641\\
3.54	0.114717630982898\\
3.55	0.116740386773744\\
3.56	0.118784990489004\\
3.57	0.120851344967873\\
3.58	0.122939343697765\\
3.59	0.125048870701625\\
3.6	0.127179800428835\\
3.61	0.129331997649863\\
3.62	0.131505317354778\\
3.63	0.133699604655791\\
3.64	0.135914694693932\\
3.65	0.138150412550017\\
3.66	0.14040657316004\\
3.67	0.142682981235104\\
3.68	0.144979431186045\\
3.69	0.147295707052865\\
3.7	0.149631582439104\\
3.71	0.151986820451277\\
3.72	0.154361173643503\\
3.73	0.156754383967443\\
3.74	0.159166182727665\\
3.75	0.161596290542557\\
3.76	0.164044417310898\\
3.77	0.166510262184199\\
3.78	0.168993513544918\\
3.79	0.17149384899066\\
3.8	0.174010935324459\\
3.81	0.176544428551235\\
3.82	0.179093973880531\\
3.83	0.181659205735608\\
3.84	0.184239747768995\\
3.85	0.186835212884572\\
3.86	0.189445203266253\\
3.87	0.192069310413369\\
3.88	0.194707115182791\\
3.89	0.197358187837878\\
3.9	0.200022088104301\\
3.91	0.2026983652328\\
3.92	0.20538655806893\\
3.93	0.208086195129822\\
3.94	0.210796794688034\\
3.95	0.213517864862485\\
3.96	0.216248903716537\\
3.97	0.218989399363226\\
3.98	0.221738830077677\\
3.99	0.224496664416695\\
4	0.227262361345569\\
4.01	0.230035370372056\\
4.02	0.232815131687571\\
4.03	0.235601076315548\\
4.04	0.238392626266977\\
4.05	0.241189194703077\\
4.06	0.243990186105084\\
4.07	0.24679499645112\\
4.08	0.249603013400101\\
4.09	0.252413616482629\\
4.1	0.255226177298835\\
4.11	0.258040059723083\\
4.12	0.260854620115493\\
4.13	0.263669207540208\\
4.14	0.266483163990311\\
4.15	0.269295824619319\\
4.16	0.272106517979162\\
4.17	0.274914566264549\\
4.18	0.277719285563608\\
4.19	0.280519986114707\\
4.2	0.283315972569318\\
4.21	0.286106544260821\\
4.22	0.288890995479109\\
4.23	0.291668615750864\\
4.24	0.294438690125349\\
4.25	0.297200499465598\\
4.26	0.299953320744823\\
4.27	0.302696427347894\\
4.28	0.305429089377727\\
4.29	0.308150573966407\\
4.3	0.310860145590876\\
4.31	0.313557066393003\\
4.32	0.316240596503845\\
4.33	0.318909994371925\\
4.34	0.321564517095314\\
4.35	0.324203420757327\\
4.36	0.326825960765618\\
4.37	0.32943139219449\\
4.38	0.332018970130165\\
4.39	0.334587950018842\\
4.4	0.337137588017281\\
4.41	0.339667141345718\\
4.42	0.342175868642858\\
4.43	0.344663030322738\\
4.44	0.347127888933195\\
4.45	0.349569709515727\\
4.46	0.351987759966485\\
4.47	0.354381311398166\\
4.48	0.356749638502546\\
4.49	0.359092019913418\\
4.5	0.361407738569672\\
4.51	0.363696082078269\\
4.52	0.365956343076854\\
4.53	0.36818781959575\\
4.54	0.37038981541908\\
4.55	0.372561640444749\\
4.56	0.374702611043043\\
4.57	0.376812050413566\\
4.58	0.378889288940274\\
4.59	0.380933664544337\\
4.6	0.382944523034563\\
4.61	0.384921218455149\\
4.62	0.386863113430472\\
4.63	0.38876957950668\\
4.64	0.390639997489835\\
4.65	0.392473757780324\\
4.66	0.394270260703325\\
4.67	0.396028916835047\\
4.68	0.397749147324504\\
4.69	0.399430384210594\\
4.7	0.401072070734212\\
4.71	0.402673661645181\\
4.72	0.404234623503753\\
4.73	0.405754434976448\\
4.74	0.407232587126003\\
4.75	0.408668583695207\\
4.76	0.410061941384398\\
4.77	0.411412190122402\\
4.78	0.412718873330714\\
4.79	0.413981548180693\\
4.8	0.415199785843588\\
4.81	0.416373171733188\\
4.82	0.417501305740899\\
4.83	0.418583802463076\\
4.84	0.419620291420413\\
4.85	0.420610417269232\\
4.86	0.421553840004482\\
4.87	0.422450235154311\\
4.88	0.423299293966035\\
4.89	0.424100723583361\\
4.9	0.424854247214724\\
4.91	0.425559604292603\\
4.92	0.426216550623681\\
4.93	0.426824858529737\\
4.94	0.42738431697915\\
4.95	0.427894731708904\\
4.96	0.428355925337014\\
4.97	0.42876773746526\\
4.98	0.429130024772156\\
4.99	0.429442661096082\\
5	0.429705537508501\\
5.01	0.429918562377215\\
5.02	0.430081661419596\\
5.03	0.430194777745755\\
5.04	0.430257871891611\\
5.05	0.430270921841844\\
5.06	0.430233923042688\\
5.07	0.43014688840459\\
5.08	0.43000984829469\\
5.09	0.429822850519175\\
5.1	0.429585960295475\\
5.11	0.429299260214361\\
5.12	0.428962850191953\\
5.13	0.42857684741169\\
5.14	0.428141386256298\\
5.15	0.427656618229819\\
5.16	0.427122711869768\\
5.17	0.426539852649476\\
5.18	0.425908242870709\\
5.19	0.425228101546648\\
5.2	0.424499664275325\\
5.21	0.423723183103612\\
5.22	0.422898926381879\\
5.23	0.422027178609439\\
5.24	0.421108240270897\\
5.25	0.420142427663547\\
5.26	0.419130072715939\\
5.27	0.418071522797781\\
5.28	0.416967140521312\\
5.29	0.415817303534313\\
5.3	0.414622404304924\\
5.31	0.413382849898431\\
5.32	0.412099061746202\\
5.33	0.410771475406965\\
5.34	0.409400540320603\\
5.35	0.407986719554673\\
5.36	0.406530489543842\\
5.37	0.405032339822449\\
5.38	0.403492772750393\\
5.39	0.401912303232586\\
5.4	0.400291458432158\\
5.41	0.398630777477662\\
5.42	0.396930811164499\\
5.43	0.395192121650785\\
5.44	0.393415282147916\\
5.45	0.391600876606045\\
5.46	0.389749499394726\\
5.47	0.387861754978974\\
5.48	0.38593825759097\\
5.49	0.383979630897678\\
5.5	0.381986507664611\\
5.51	0.379959529416016\\
5.52	0.377899346091709\\
5.53	0.375806615700843\\
5.54	0.373682003972844\\
5.55	0.371526184005785\\
5.56	0.369339835912456\\
5.57	0.367123646464386\\
5.58	0.364878308734077\\
5.59	0.362604521735712\\
5.6	0.360302990064595\\
5.61	0.35797442353558\\
5.62	0.355619536820747\\
5.63	0.353239049086587\\
5.64	0.350833683630945\\
5.65	0.348404167519972\\
5.66	0.345951231225358\\
5.67	0.343475608262067\\
5.68	0.340978034826848\\
5.69	0.338459249437751\\
5.7	0.335919992574906\\
5.71	0.333361006322791\\
5.72	0.330783034014235\\
5.73	0.328186819876394\\
5.74	0.325573108678915\\
5.75	0.322942645384542\\
5.76	0.320296174802359\\
5.77	0.317634441243908\\
5.78	0.314958188182398\\
5.79	0.312268157915207\\
5.8	0.309565091229886\\
5.81	0.306849727073878\\
5.82	0.304122802228139\\
5.83	0.301385050984853\\
5.84	0.298637204829441\\
5.85	0.295879992127034\\
5.86	0.293114137813595\\
5.87	0.290340363091857\\
5.88	0.287559385132253\\
5.89	0.28477191677898\\
5.9	0.281978666261385\\
5.91	0.279180336910783\\
5.92	0.276377626882882\\
5.93	0.273571228885938\\
5.94	0.270761829914778\\
5.95	0.267950110990813\\
5.96	0.265136746908157\\
5.97	0.26232240598598\\
5.98	0.259507749827182\\
5.99	0.256693433083513\\
6	0.253880103227204\\
6.01	0.25106840032923\\
6.02	0.248258956844258\\
6.03	0.24545239740238\\
6.04	0.242649338607682\\
6.05	0.23985038884372\\
6.06	0.237056148085968\\
6.07	0.234267207721267\\
6.08	0.231484150374343\\
6.09	0.228707549741418\\
6.1	0.225937970430952\\
6.11	0.223175967811532\\
6.12	0.220422087866947\\
6.13	0.217676867058438\\
6.14	0.214940832194157\\
6.15	0.212214500305811\\
6.16	0.209498378532504\\
6.17	0.206792964011768\\
6.18	0.20409874377775\\
6.19	0.201416194666557\\
6.2	0.198745783228713\\
6.21	0.196087965648708\\
6.22	0.193443187671595\\
6.23	0.1908118845366\\
6.24	0.188194480917686\\
6.25	0.185591390871025\\
6.26	0.183003017789326\\
6.27	0.180429754362946\\
6.28	0.17787198254772\\
6.29	0.175330073539448\\
6.3	0.172804387754944\\
6.31	0.170295274819593\\
6.32	0.1678030735613\\
6.33	0.165328112010778\\
6.34	0.162870707408052\\
6.35	0.160431166215103\\
6.36	0.158009784134541\\
6.37	0.155606846134221\\
6.38	0.153222626477669\\
6.39	0.150857388760237\\
6.4	0.148511385950859\\
6.41	0.146184860439295\\
6.42	0.14387804408875\\
6.43	0.141591158293747\\
6.44	0.139324414043124\\
6.45	0.137078011988046\\
6.46	0.134852142514887\\
6.47	0.132646985822869\\
6.48	0.130462712006318\\
6.49	0.128299481141408\\
6.5	0.126157443377264\\
6.51	0.124036739031276\\
6.52	0.121937498688509\\
6.53	0.11985984330505\\
6.54	0.117803884315168\\
6.55	0.115769723742151\\
6.56	0.113757454312658\\
6.57	0.111767159574483\\
6.58	0.109798914017555\\
6.59	0.107852783198056\\
6.6	0.105928823865511\\
6.61	0.104027084092708\\
6.62	0.102147603408304\\
6.63	0.100290412931992\\
6.64	0.0984555355120719\\
6.65	0.096642985865296\\
6.66	0.0948527707188535\\
6.67	0.093084888954349\\
6.68	0.0913393317536445\\
6.69	0.0896160827464269\\
6.7	0.0879151181593685\\
6.71	0.0862364069667458\\
6.72	0.0845799110423867\\
6.73	0.0829455853128137\\
6.74	0.0813333779114575\\
6.75	0.0797432303338104\\
6.76	0.0781750775933965\\
6.77	0.0766288483784348\\
6.78	0.0751044652090717\\
6.79	0.0736018445950657\\
6.8	0.072120897193804\\
6.81	0.0706615279685363\\
6.82	0.0692236363467128\\
6.83	0.0678071163783132\\
6.84	0.066411856894059\\
6.85	0.0650377416634018\\
6.86	0.063684649552183\\
6.87	0.0623524546798617\\
6.88	0.0610410265762144\\
6.89	0.059750230337404\\
6.9	0.0584799267813295\\
6.91	0.0572299726021594\\
6.92	0.0560002205239619\\
6.93	0.0547905194533451\\
6.94	0.0536007146310212\\
6.95	0.0524306477822145\\
6.96	0.0512801572658352\\
6.97	0.0501490782223411\\
6.98	0.049037242720213\\
6.99	0.0479444799009781\\
7	0.0468706161227059\\
7.01	0.0458154751019185\\
7.02	0.0447788780538475\\
7.03	0.0437606438309811\\
7.04	0.0427605890598424\\
7.05	0.0417785282759456\\
7.06	0.0408142740568774\\
7.07	0.0398676371534559\\
7.08	0.0389384266189196\\
7.09	0.0380264499361035\\
7.1	0.0371315131425608\\
7.11	0.0362534209535923\\
7.12	0.0353919768831473\\
7.13	0.0345469833625628\\
7.14	0.0337182418571098\\
7.15	0.0329055529803198\\
7.16	0.0321087166060638\\
7.17	0.031327531978362\\
7.18	0.0305617978189021\\
7.19	0.0298113124322489\\
7.2	0.0290758738087266\\
7.21	0.0283552797249624\\
7.22	0.027649327842078\\
7.23	0.0269578158015187\\
7.24	0.0262805413185151\\
7.25	0.0256173022731688\\
7.26	0.0249678967991615\\
7.27	0.024332123370084\\
7.28	0.0237097808833873\\
7.29	0.023100668741957\\
7.3	0.0225045869333163\\
7.31	0.0219213361064622\\
7.32	0.0213507176463456\\
7.33	0.0207925337460005\\
7.34	0.0202465874763387\\
7.35	0.0197126828536178\\
7.36	0.0191906249046004\\
7.37	0.0186802197294185\\
7.38	0.0181812745621606\\
7.39	0.0176935978292\\
7.4	0.0172169992052859\\
7.41	0.0167512896674144\\
7.42	0.0162962815465073\\
7.43	0.0158517885769169\\
7.44	0.0154176259437855\\
7.45	0.014993610328283\\
7.46	0.0145795599507499\\
7.47	0.0141752946117745\\
7.48	0.0137806357312301\\
7.49	0.0133954063853056\\
7.5	0.0130194313415551\\
7.51	0.0126525370920022\\
7.52	0.0122945518843264\\
7.53	0.0119453057511672\\
7.54	0.0116046305375768\\
7.55	0.0112723599266554\\
7.56	0.0109483294634039\\
7.57	0.0106323765768272\\
7.58	0.0103243406003238\\
7.59	0.0100240627903965\\
7.6	0.00973138634372004\\
7.61	0.00944615641260181\\
7.62	0.00916822011887066\\
7.63	0.00889742656623163\\
7.64	0.00863362685112175\\
7.65	0.00837667407210451\\
7.66	0.00812642333783901\\
7.67	0.00788273177366106\\
7.68	0.00764545852681277\\
7.69	0.00741446477035737\\
7.7	0.00718961370581621\\
7.71	0.00697077056456436\\
7.72	0.00675780260802152\\
7.73	0.00655057912667446\\
7.74	0.0063489714379676\\
7.75	0.0061528528830971\\
7.76	0.00596209882274507\\
7.77	0.00577658663178883\\
7.78	0.00559619569302082\\
7.79	0.00542080738991404\\
7.8	0.00525030509846783\\
7.81	0.00508457417816796\\
7.82	0.00492350196209547\\
7.83	0.00476697774621747\\
7.84	0.00461489277789316\\
7.85	0.004467140243628\\
7.86	0.00432361525610802\\
7.87	0.00418421484054641\\
7.88	0.00404883792037373\\
7.89	0.00391738530230245\\
7.9	0.00378975966079661\\
7.91	0.00366586552197609\\
7.92	0.00354560924698525\\
7.93	0.00342889901485462\\
7.94	0.00331564480488402\\
7.95	0.00320575837857484\\
7.96	0.00309915326113891\\
7.97	0.0029957447226103\\
7.98	0.00289544975858655\\
7.99	0.0027981870706246\\
8	0.0027038770463165\\
8.01	0.00261244173906925\\
8.02	0.00252380484761261\\
8.03	0.00243789169525811\\
8.04	0.00235462920893187\\
8.05	0.00227394589800328\\
8.06	0.00219577183293114\\
8.07	0.00212003862374783\\
8.08	0.00204667939840225\\
8.09	0.00197562878098082\\
8.1	0.00190682286982598\\
8.11	0.0018401992155706\\
8.12	0.00177569679910627\\
8.13	0.00171325600950294\\
8.14	0.00165281862189675\\
8.15	0.00159432777536226\\
8.16	0.00153772795078486\\
8.17	0.00148296494874843\\
8.18	0.00142998586745307\\
8.19	0.00137873908067667\\
8.2	0.00132917421579415\\
8.21	0.00128124213186715\\
8.22	0.00123489489781688\\
8.23	0.00119008577069181\\
8.24	0.00114676917404195\\
8.25	0.00110490067641056\\
8.26	0.00106443696995373\\
8.27	0.0010253358491979\\
8.28	0.000987556189944738\\
8.29	0.0009510579283326\\
8.3	0.000915802040062941\\
8.31	0.000881750519800031\\
8.32	0.00084886636075151\\
8.33	0.000817113534437219\\
8.34	0.000786456970653031\\
8.35	0.000756862537636212\\
8.36	0.000728297022438306\\
8.37	0.000700728111511244\\
8.38	0.000674124371511893\\
8.39	0.00064845523032998\\
8.4	0.000623690958343928\\
8.41	0.000599802649908723\\
8.42	0.000576762205079736\\
8.43	0.000554542311575931\\
8.44	0.000533116426985664\\
8.45	0.000512458761217927\\
8.46	0.000492544259201582\\
8.47	0.000473348583834874\\
8.48	0.000454848099187134\\
8.49	0.000437019853954429\\
8.5	0.000419841565170547\\
8.51	0.000403291602174477\\
8.52	0.000387348970835326\\
8.53	0.000371993298035319\\
8.54	0.000357204816411329\\
8.55	0.000342964349355168\\
8.56	0.000329253296272633\\
8.57	0.000316053618101072\\
8.58	0.000303347823085131\\
8.59	0.000291118952810022\\
8.6	0.00027935056849159\\
8.61	0.000268026737522192\\
8.62	0.000257132020271303\\
8.63	0.000246651457139548\\
8.64	0.000236570555864774\\
8.65	0.000226875279078558\\
8.66	0.000217552032111468\\
8.67	0.00020858765104525\\
8.68	0.000199969391009976\\
8.69	0.000191684914724091\\
8.7	0.000183722281275182\\
8.71	0.000176069935139188\\
8.72	0.000168716695435684\\
8.73	0.000161651745416754\\
8.74	0.000154864622186933\\
8.75	0.000148345206651577\\
8.76	0.000142083713690956\\
8.77	0.000136070682557319\\
8.78	0.000130296967492101\\
8.79	0.00012475372856038\\
8.8	0.000119432422699661\\
8.81	0.000114324794980009\\
8.82	0.000109422870072488\\
8.83	0.000104718943922871\\
8.84	0.00010020557562752\\
8.85	9.58755795083083e-05\\
8.86	9.17220173834464e-05\\
8.87	8.77381910310472e-05\\
8.88	8.39176348422412e-05\\
8.89	8.0254108660648e-05\\
8.9	7.67415908050039e-05\\
8.91	7.33742712717205e-05\\
8.92	7.01465451141713e-05\\
8.93	6.70530059954835e-05\\
8.94	6.40884399116234e-05\\
8.95	6.12478190815682e-05\\
8.96	5.85262960013651e-05\\
8.97	5.59191976588905e-05\\
8.98	5.3422019906128e-05\\
8.99	5.10304219858156e-05\\
9	4.87402212093111e-05\\
9.01	4.65473877825589e-05\\
9.02	4.4448039777055e-05\\
9.03	4.24384382427343e-05\\
9.04	4.05149824597284e-05\\
9.05	3.86742053259695e-05\\
9.06	3.69127688776458e-05\\
9.07	3.52274599395366e-05\\
9.08	3.36151859023003e-05\\
9.09	3.20729706238066e-05\\
9.1	3.05979504516463e-05\\
9.11	2.91873703639862e-05\\
9.12	2.78385802259674e-05\\
9.13	2.65490311588883e-05\\
9.14	2.53162720194479e-05\\
9.15	2.413794598636e-05\\
9.16	2.30117872516976e-05\\
9.17	2.19356178143529e-05\\
9.18	2.09073443730507e-05\\
9.19	1.99249553163848e-05\\
9.2	1.89865178073933e-05\\
9.21	1.80901749602264e-05\\
9.22	1.7234143106505e-05\\
9.23	1.64167091490052e-05\\
9.24	1.5636228000354e-05\\
9.25	1.48911201044538e-05\\
9.26	1.41798690384032e-05\\
9.27	1.35010191927194e-05\\
9.28	1.28531735277123e-05\\
9.29	1.22349914038993e-05\\
9.3	1.16451864843962e-05\\
9.31	1.10825247072582e-05\\
9.32	1.05458223257856e-05\\
9.33	1.00339440148562e-05\\
9.34	9.54580104138018e-06\\
9.35	9.08034949702025e-06\\
9.36	8.63658859135671e-06\\
9.37	8.21355900371943e-06\\
9.38	7.81034129194826e-06\\
9.39	7.4260543563832e-06\\
9.4	7.05985395742368e-06\\
9.41	6.71093128503779e-06\\
9.42	6.37851157863778e-06\\
9.43	6.06185279577802e-06\\
9.44	5.76024432816823e-06\\
9.45	5.47300576353238e-06\\
9.46	5.1994856918794e-06\\
9.47	4.93906055478872e-06\\
9.48	4.69113353634792e-06\\
9.49	4.45513349441633e-06\\
9.5	4.23051393092132e-06\\
9.51	4.01675199992853e-06\\
9.52	3.81334755226055e-06\\
9.53	3.61982221547129e-06\\
9.54	3.43571850801515e-06\\
9.55	3.26059898648207e-06\\
9.56	3.0940454248004e-06\\
9.57	2.9356580243395e-06\\
9.58	2.78505465387479e-06\\
9.59	2.64187011840634e-06\\
9.6	2.50575545585127e-06\\
9.61	2.37637726065802e-06\\
9.62	2.25341703341841e-06\\
9.63	2.13657055558007e-06\\
9.64	2.0255472883885e-06\\
9.65	1.92006979521328e-06\\
9.66	1.81987318643892e-06\\
9.67	1.72470458612492e-06\\
9.68	1.63432261966389e-06\\
9.69	1.54849692169029e-06\\
9.7	1.46700766351513e-06\\
9.71	1.38964509938443e-06\\
9.72	1.31620913088152e-06\\
9.73	1.24650888881384e-06\\
9.74	1.18036233194682e-06\\
9.75	1.11759586196676e-06\\
9.76	1.05804395407504e-06\\
9.77	1.00154880263508e-06\\
9.78	9.47959981312304e-07\\
9.79	8.9713411716567e-07\\
9.8	8.48934578167194e-07\\
9.81	8.0323117364319e-07\\
9.82	7.59899867147783e-07\\
9.83	7.18822501295873e-07\\
9.84	6.79886534098404e-07\\
9.85	6.42984786358577e-07\\
9.86	6.0801519970248e-07\\
9.87	5.74880604832494e-07\\
9.88	5.43488499605858e-07\\
9.89	5.1375083655478e-07\\
9.9	4.8558381947767e-07\\
9.91	4.58907708744353e-07\\
9.92	4.33646634970517e-07\\
9.93	4.09728420729044e-07\\
9.94	3.87084409977666e-07\\
9.95	3.65649304893996e-07\\
9.96	3.45361009820035e-07\\
9.97	3.26160482029178e-07\\
9.98	3.07991589039127e-07\\
9.99	2.90800972204353e-07\\
10	2.74537916331524e-07\\
};
%\addlegendentry{moment matching}

\addplot [color=mycolor2, line width=2.0pt, forget plot]
  table[row sep=crcr]{%
5.04760739417037	0\\
5.04760739417037	0.6\\
};
\addplot [color=mycolor1, dashed, line width=2.0pt, forget plot]
  table[row sep=crcr]{%
5.0284309151552	0\\
5.0284309151552	0.6\\
};
\end{axis}

\begin{axis}[%
width=0.898in,
height=1.5in,%3.603in,
at={(3.196in,0.486in)},
scale only axis,
xmin=0,
xmax=10,
ymin=0,
ymax=0.8,
title={Sigma-point \\ propagation},
title style={align=left}, 
axis background/.style={fill=white},
axis x line*=bottom,
axis y line*=left,
legend style={legend cell align=right, align=right, draw=white!15!black, cells={align=right}, font=\tiny}
]
\addplot[ybar interval, fill=mycolor1, fill opacity=0.4, draw=mycolor1, area legend] table[row sep=crcr] {%
x	y\\
3.36	0.0144927536231884\\
3.429	0.0289855072463768\\
3.498	0.0869565217391305\\
3.567	0.217391304347825\\
3.636	0.391304347826087\\
3.705	0.565217391304348\\
3.774	0.449275362318841\\
3.843	0.405797101449276\\
3.912	0.666666666666667\\
3.981	0.420289855072464\\
4.05	0.478260869565218\\
4.119	0.289855072463768\\
4.188	0.289855072463768\\
4.257	0.347826086956522\\
4.326	0.246376811594203\\
4.395	0.304347826086953\\
4.464	0.20289855072464\\
4.533	0.217391304347823\\
4.602	0.246376811594203\\
4.671	0.289855072463768\\
4.74	0.246376811594203\\
4.809	0.188405797101449\\
4.878	0.231884057971015\\
4.947	0.27536231884058\\
5.016	0.391304347826087\\
5.085	0.246376811594203\\
5.154	0.27536231884058\\
5.223	0.217391304347826\\
5.292	0.347826086956518\\
5.361	0.231884057971018\\
5.43	0.2463768115942\\
5.499	0.260869565217395\\
5.568	0.275362318840576\\
5.637	0.289855072463772\\
5.706	0.304347826086953\\
5.775	0.289855072463772\\
5.844	0.362318840579706\\
5.913	0.289855072463768\\
5.982	0.463768115942029\\
6.051	0.420289855072464\\
6.12	0.492753623188406\\
6.189	0.463768115942029\\
6.258	0.463768115942029\\
6.327	0.420289855072459\\
6.396	0.246376811594206\\
6.465	0.202898550724635\\
6.534	0.0869565217391316\\
6.603	0.0579710144927529\\
6.672	0.0289855072463772\\
6.741	0.0144927536231882\\
6.81	0.0144927536231882\\
};
%\addlegendentry{Numerical \\ approx.}

\addplot [color=mycolor2, line width=2.0pt]
  table[row sep=crcr]{%
0	8.87764380319432e-05\\
0.01	9.1700822136384e-05\\
0.02	9.47154517333371e-05\\
0.03	9.78228998119414e-05\\
0.04	0.000101025805675457\\
0.05	0.000104326876427423\\
0.06	0.000107728888484391\\
0.07	0.000111234689115493\\
0.08	0.000114847198009143\\
0.09	0.000118569408867125\\
0.1	0.000122404391026334\\
0.11	0.00012635529110845\\
0.12	0.000130425334697778\\
0.13	0.000134617828047533\\
0.14	0.000138936159814801\\
0.15	0.000143383802824432\\
0.16	0.000147964315862095\\
0.17	0.000152681345496739\\
0.18	0.000157538627932674\\
0.19	0.000162539990891505\\
0.2	0.000167689355524124\\
0.21	0.000172990738352978\\
0.22	0.000178448253244802\\
0.23	0.000184066113414019\\
0.24	0.000189848633456982\\
0.25	0.000195800231417249\\
0.26	0.000201925430882044\\
0.27	0.000208228863110072\\
0.28	0.000214715269190831\\
0.29	0.000221389502235576\\
0.3	0.000228256529600039\\
0.31	0.000235321435139054\\
0.32	0.000242589421493169\\
0.33	0.000250065812407359\\
0.34	0.000257756055081929\\
0.35	0.000265665722555657\\
0.36	0.000273800516121267\\
0.37	0.000282166267773259\\
0.38	0.000290768942688146\\
0.39	0.000299614641737107\\
0.4	0.000308709604031073\\
0.41	0.000318060209498228\\
0.42	0.000327672981493914\\
0.43	0.0003375545894429\\
0.44	0.00034771185151395\\
0.45	0.000358151737326627\\
0.46	0.000368881370690256\\
0.47	0.000379908032374928\\
0.48	0.00039123916291442\\
0.49	0.000402882365440907\\
0.5	0.000414845408551292\\
0.51	0.000427136229204983\\
0.52	0.000439762935652922\\
0.53	0.000452733810397627\\
0.54	0.000466057313184048\\
0.55	0.000479742084020933\\
0.56	0.000493796946232453\\
0.57	0.000508230909539774\\
0.58	0.000523053173172236\\
0.59	0.00053827312900782\\
0.6	0.000553900364742502\\
0.61	0.000569944667088102\\
0.62	0.000586416024998223\\
0.63	0.000603324632921818\\
0.64	0.000620680894083919\\
0.65	0.000638495423793025\\
0.66	0.000656779052774621\\
0.67	0.000675542830530304\\
0.68	0.000694798028721899\\
0.69	0.000714556144579973\\
0.7	0.000734828904336119\\
0.71	0.000755628266678342\\
0.72	0.000776966426228842\\
0.73	0.000798855817043487\\
0.74	0.000821309116132199\\
0.75	0.000844339246999491\\
0.76	0.000867959383204324\\
0.77	0.000892182951938454\\
0.78	0.000917023637622338\\
0.79	0.000942495385517763\\
0.8	0.000968612405356183\\
0.81	0.000995389174981824\\
0.82	0.00102284044400852\\
0.83	0.00105098123748923\\
0.84	0.00107982685959722\\
0.85	0.00110939289731764\\
0.86	0.00113969522414855\\
0.87	0.00117075000380997\\
0.88	0.00120257369395996\\
0.89	0.00123518304991627\\
0.9	0.00126859512838244\\
0.91	0.00130282729117678\\
0.92	0.00133789720896316\\
0.93	0.00137382286498187\\
0.94	0.00141062255877942\\
0.95	0.00144831490993543\\
0.96	0.0014869188617855\\
0.97	0.0015264536851381\\
0.98	0.00156693898198413\\
0.99	0.00160839468919738\\
1	0.00165084108222428\\
1.01	0.00169429877876111\\
1.02	0.00173878874241707\\
1.03	0.00178433228636116\\
1.04	0.00183095107695133\\
1.05	0.00187866713734369\\
1.06	0.00192750285108013\\
1.07	0.00197748096565218\\
1.08	0.00202862459603924\\
1.09	0.00208095722821903\\
1.1	0.00213450272264824\\
1.11	0.00218928531771122\\
1.12	0.0022453296331345\\
1.13	0.00230266067336505\\
1.14	0.00236130383090986\\
1.15	0.00242128488963473\\
1.16	0.0024826300280197\\
1.17	0.00254536582236912\\
1.18	0.00260951924997343\\
1.19	0.0026751176922208\\
1.2	0.00274218893765555\\
1.21	0.00281076118498133\\
1.22	0.00288086304600608\\
1.23	0.00295252354852655\\
1.24	0.00302577213914932\\
1.25	0.00310063868604605\\
1.26	0.00317715348163996\\
1.27	0.00325534724522087\\
1.28	0.0033352511254861\\
1.29	0.00341689670300425\\
1.3	0.00350031599259913\\
1.31	0.00358554144565092\\
1.32	0.00367260595231155\\
1.33	0.0037615428436315\\
1.34	0.00385238589359484\\
1.35	0.00394516932105972\\
1.36	0.00403992779160099\\
1.37	0.00413669641925212\\
1.38	0.00423551076814321\\
1.39	0.00433640685403186\\
1.4	0.0044394211457239\\
1.41	0.00454459056638071\\
1.42	0.00465195249470999\\
1.43	0.00476154476603654\\
1.44	0.00487340567325009\\
1.45	0.00498757396762667\\
1.46	0.00510408885952039\\
1.47	0.00522299001892209\\
1.48	0.00534431757588181\\
1.49	0.00546811212079153\\
1.5	0.00559441470452481\\
1.51	0.00572326683843017\\
1.52	0.00585471049417453\\
1.53	0.00598878810343347\\
1.54	0.00612554255742495\\
1.55	0.00626501720628298\\
1.56	0.00640725585826777\\
1.57	0.00655230277880913\\
1.58	0.00670020268937946\\
1.59	0.00685100076619316\\
1.6	0.00700474263872876\\
1.61	0.0071614743880705\\
1.62	0.00732124254506602\\
1.63	0.00748409408829659\\
1.64	0.00765007644185648\\
1.65	0.00781923747293826\\
1.66	0.00799162548922045\\
1.67	0.00816728923605427\\
1.68	0.00834627789344614\\
1.69	0.00852864107283249\\
1.7	0.00871442881364377\\
1.71	0.00890369157965414\\
1.72	0.00909648025511389\\
1.73	0.00929284614066099\\
1.74	0.00949284094900887\\
1.75	0.00969651680040709\\
1.76	0.00990392621787196\\
1.77	0.0101151221221837\\
1.78	0.0103301578266475\\
1.79	0.0105490870316149\\
1.8	0.0107719638187632\\
1.81	0.0109988426451295\\
1.82	0.0112297783368967\\
1.83	0.0114648260829282\\
1.84	0.0117040414280498\\
1.85	0.0119474802660738\\
1.86	0.0121951988325657\\
1.87	0.0124472536973477\\
1.88	0.0127037017567393\\
1.89	0.0129646002255305\\
1.9	0.0132300066286863\\
1.91	0.0134999787927796\\
1.92	0.0137745748371512\\
1.93	0.0140538531647934\\
1.94	0.0143378724529564\\
1.95	0.0146266916434741\\
1.96	0.0149203699328093\\
1.97	0.0152189667618148\\
1.98	0.0155225418052091\\
1.99	0.0158311549607656\\
2	0.0161448663382137\\
2.01	0.0164637362478493\\
2.02	0.0167878251888549\\
2.03	0.0171171938373269\\
2.04	0.0174519030340098\\
2.05	0.0177920137717355\\
2.06	0.0181375871825679\\
2.07	0.0184886845246504\\
2.08	0.0188453671687578\\
2.09	0.0192076965845497\\
2.1	0.0195757343265274\\
2.11	0.0199495420196917\\
2.12	0.0203291813449033\\
2.13	0.020714714023945\\
2.14	0.0211062018042852\\
2.15	0.0215037064435448\\
2.16	0.0219072896936656\\
2.17	0.0223170132847825\\
2.18	0.0227329389087996\\
2.19	0.0231551282026703\\
2.2	0.0235836427313844\\
2.21	0.0240185439706604\\
2.22	0.0244598932893471\\
2.23	0.0249077519315344\\
2.24	0.0253621809983751\\
2.25	0.0258232414296198\\
2.26	0.0262909939848663\\
2.27	0.0267654992245267\\
2.28	0.0272468174905128\\
2.29	0.0277350088866436\\
2.3	0.0282301332587772\\
2.31	0.0287322501746691\\
2.32	0.029241418903561\\
2.33	0.0297576983955025\\
2.34	0.0302811472604087\\
2.35	0.0308118237468586\\
2.36	0.0313497857206356\\
2.37	0.0318950906430169\\
2.38	0.0324477955488128\\
2.39	0.0330079570241622\\
2.4	0.0335756311840884\\
2.41	0.0341508736498181\\
2.42	0.0347337395258714\\
2.43	0.0353242833769239\\
2.44	0.0359225592044501\\
2.45	0.0365286204231498\\
2.46	0.0371425198371657\\
2.47	0.0377643096160964\\
2.48	0.0383940412708106\\
2.49	0.0390317656290701\\
2.5	0.0396775328109659\\
2.51	0.040331392204175\\
2.52	0.0409933924390448\\
2.53	0.0416635813635106\\
2.54	0.0423420060178543\\
2.55	0.043028712609312\\
2.56	0.0437237464865357\\
2.57	0.0444271521139196\\
2.58	0.0451389730457957\\
2.59	0.0458592519005098\\
2.6	0.0465880303343831\\
2.61	0.0473253490155696\\
2.62	0.0480712475978175\\
2.63	0.048825764694142\\
2.64	0.0495889378504206\\
2.65	0.0503608035189175\\
2.66	0.0511413970317481\\
2.67	0.0519307525742921\\
2.68	0.0527289031585651\\
2.69	0.0535358805965574\\
2.7	0.0543517154735524\\
2.71	0.0551764371214307\\
2.72	0.0560100735919743\\
2.73	0.0568526516301781\\
2.74	0.0577041966475806\\
2.75	0.0585647326956243\\
2.76	0.0594342824390564\\
2.77	0.0603128671293805\\
2.78	0.0612005065783718\\
2.79	0.0620972191316639\\
2.8	0.0630030216424231\\
2.81	0.0639179294451165\\
2.82	0.0648419563293903\\
2.83	0.0657751145140666\\
2.84	0.0667174146212718\\
2.85	0.0676688656507095\\
2.86	0.0686294749540878\\
2.87	0.069599248209715\\
2.88	0.0705781893972751\\
2.89	0.0715663007727959\\
2.9	0.0725635828438214\\
2.91	0.0735700343448021\\
2.92	0.0745856522127145\\
2.93	0.0756104315629238\\
2.94	0.0766443656653014\\
2.95	0.0776874459206114\\
2.96	0.0787396618371763\\
2.97	0.0798010010078386\\
2.98	0.0808714490872275\\
2.99	0.0819509897693467\\
3	0.0830396047654938\\
3.01	0.0841372737825264\\
3.02	0.0852439745014865\\
3.03	0.0863596825565973\\
3.04	0.0874843715146442\\
3.05	0.088618012854755\\
3.06	0.0897605759485901\\
3.07	0.0909120280409574\\
3.08	0.0920723342308645\\
3.09	0.0932414574530208\\
3.1	0.0944193584598023\\
3.11	0.0956059958036935\\
3.12	0.0968013258202172\\
3.13	0.0980053026113663\\
3.14	0.0992178780295499\\
3.15	0.100439001662067\\
3.16	0.101668620816118\\
3.17	0.102906680504373\\
3.18	0.104153123431098\\
3.19	0.105407889978861\\
3.2	0.10667091819583\\
3.21	0.10794214378367\\
3.22	0.109221500086047\\
3.23	0.110508918077769\\
3.24	0.111804326354552\\
3.25	0.113107651123439\\
3.26	0.114418816193881\\
3.27	0.115737742969483\\
3.28	0.117064350440432\\
3.29	0.118398555176626\\
3.3	0.11974027132149\\
3.31	0.121089410586524\\
3.32	0.122445882246559\\
3.33	0.123809593135755\\
3.34	0.125180447644343\\
3.35	0.126558347716114\\
3.36	0.127943192846677\\
3.37	0.129334880082487\\
3.38	0.13073330402065\\
3.39	0.13213835680952\\
3.4	0.133549928150096\\
3.41	0.134967905298216\\
3.42	0.136392173067578\\
3.43	0.137822613833567\\
3.44	0.139259107537919\\
3.45	0.140701531694221\\
3.46	0.142149761394247\\
3.47	0.143603669315142\\
3.48	0.145063125727469\\
3.49	0.146527998504106\\
3.5	0.14799815313001\\
3.51	0.149473452712855\\
3.52	0.15095375799454\\
3.53	0.152438927363579\\
3.54	0.153928816868373\\
3.55	0.155423280231368\\
3.56	0.156922168864102\\
3.57	0.158425331883151\\
3.58	0.159932616126957\\
3.59	0.161443866173569\\
3.6	0.162958924359271\\
3.61	0.164477630798113\\
3.62	0.16599982340235\\
3.63	0.167525337903775\\
3.64	0.169054007875955\\
3.65	0.170585664757372\\
3.66	0.172120137875463\\
3.67	0.17365725447156\\
3.68	0.175196839726729\\
3.69	0.176738716788501\\
3.7	0.17828270679851\\
3.71	0.179828628921003\\
3.72	0.181376300372256\\
3.73	0.182925536450863\\
3.74	0.184476150568913\\
3.75	0.186027954284037\\
3.76	0.187580757332328\\
3.77	0.18913436766213\\
3.78	0.190688591468681\\
3.79	0.192243233229616\\
3.8	0.193798095741313\\
3.81	0.195352980156075\\
3.82	0.196907686020157\\
3.83	0.198462011312601\\
3.84	0.200015752484904\\
3.85	0.201568704501475\\
3.86	0.203120660880904\\
3.87	0.204671413738017\\
3.88	0.206220753826697\\
3.89	0.20776847058349\\
3.9	0.209314352171954\\
3.91	0.210858185527757\\
3.92	0.212399756404511\\
3.93	0.21393884942032\\
3.94	0.215475248105045\\
3.95	0.217008734948253\\
3.96	0.21853909144786\\
3.97	0.220066098159426\\
3.98	0.22158953474612\\
3.99	0.223109180029311\\
4	0.224624812039785\\
4.01	0.226136208069575\\
4.02	0.227643144724377\\
4.03	0.229145397976542\\
4.04	0.230642743218632\\
4.05	0.232134955317513\\
4.06	0.233621808668973\\
4.07	0.235103077252852\\
4.08	0.236578534688654\\
4.09	0.238047954291633\\
4.1	0.239511109129332\\
4.11	0.240967772078555\\
4.12	0.242417715882746\\
4.13	0.243860713209771\\
4.14	0.245296536710069\\
4.15	0.246724959075156\\
4.16	0.248145753096463\\
4.17	0.24955869172449\\
4.18	0.250963548128245\\
4.19	0.252360095754955\\
4.2	0.253748108390029\\
4.21	0.255127360217246\\
4.22	0.256497625879142\\
4.23	0.25785868053759\\
4.24	0.259210299934527\\
4.25	0.260552260452831\\
4.26	0.261884339177307\\
4.27	0.263206313955762\\
4.28	0.264517963460149\\
4.29	0.265819067247764\\
4.3	0.267109405822455\\
4.31	0.268388760695835\\
4.32	0.269656914448466\\
4.33	0.270913650790995\\
4.34	0.27215875462522\\
4.35	0.273392012105055\\
4.36	0.274613210697374\\
4.37	0.275822139242714\\
4.38	0.2770185880158\\
4.39	0.278202348785892\\
4.4	0.279373214876894\\
4.41	0.280530981227242\\
4.42	0.281675444449509\\
4.43	0.282806402889732\\
4.44	0.283923656686419\\
4.45	0.285027007829217\\
4.46	0.286116260217221\\
4.47	0.287191219716896\\
4.48	0.28825169421959\\
4.49	0.289297493698606\\
4.5	0.290328430265828\\
4.51	0.291344318227856\\
4.52	0.292344974141641\\
4.53	0.293330216869595\\
4.54	0.294299867634142\\
4.55	0.2952537500717\\
4.56	0.29619169028607\\
4.57	0.297113516901198\\
4.58	0.298019061113301\\
4.59	0.298908156742324\\
4.6	0.29978064028272\\
4.61	0.300636350953513\\
4.62	0.301475130747636\\
4.63	0.302296824480521\\
4.64	0.303101279837916\\
4.65	0.303888347422908\\
4.66	0.30465788080214\\
4.67	0.30540973655119\\
4.68	0.306143774299102\\
4.69	0.306859856772047\\
4.7	0.307557849836088\\
4.71	0.308237622539045\\
4.72	0.308899047151428\\
4.73	0.309541999206423\\
4.74	0.310166357538923\\
4.75	0.310772004323572\\
4.76	0.311358825111819\\
4.77	0.311926708867954\\
4.78	0.312475548004121\\
4.79	0.313005238414284\\
4.8	0.313515679507133\\
4.81	0.314006774237922\\
4.82	0.314478429139213\\
4.83	0.314930554350523\\
4.84	0.315363063646858\\
4.85	0.315775874466112\\
4.86	0.31616890793534\\
4.87	0.316542088895868\\
4.88	0.316895345927247\\
4.89	0.317228611370034\\
4.9	0.317541821347388\\
4.91	0.317834915785474\\
4.92	0.318107838432669\\
4.93	0.318360536877553\\
4.94	0.318592962565682\\
4.95	0.318805070815141\\
4.96	0.318996820830853\\
4.97	0.319168175717664\\
4.98	0.319319102492166\\
4.99	0.319449572093287\\
5	0.319559559391609\\
5.01	0.319649043197439\\
5.02	0.319718006267613\\
5.03	0.319766435311032\\
5.04	0.31979432099293\\
5.05	0.319801657937876\\
5.06	0.319788444731501\\
5.07	0.319754683920947\\
5.08	0.319700382014057\\
5.09	0.319625549477275\\
5.1	0.319530200732294\\
5.11	0.319414354151413\\
5.12	0.319278032051646\\
5.13	0.319121260687552\\
5.14	0.31894407024281\\
5.15	0.318746494820531\\
5.16	0.318528572432322\\
5.17	0.318290344986095\\
5.18	0.318031858272638\\
5.19	0.317753161950944\\
5.2	0.317454309532312\\
5.21	0.317135358363221\\
5.22	0.316796369606996\\
5.23	0.316437408224253\\
5.24	0.316058542952159\\
5.25	0.315659846282491\\
5.26	0.315241394438521\\
5.27	0.31480326735073\\
5.28	0.314345548631364\\
5.29	0.313868325547838\\
5.3	0.313371688995012\\
5.31	0.312855733466336\\
5.32	0.312320557023887\\
5.33	0.311766261267315\\
5.34	0.311192951301694\\
5.35	0.310600735704312\\
5.36	0.309989726490407\\
5.37	0.309360039077857\\
5.38	0.30871179225085\\
5.39	0.308045108122549\\
5.4	0.307360112096763\\
5.41	0.306656932828639\\
5.42	0.305935702184406\\
5.43	0.305196555200171\\
5.44	0.304439630039798\\
5.45	0.30366506795188\\
5.46	0.302873013225835\\
5.47	0.302063613147121\\
5.48	0.301237017951625\\
5.49	0.300393380779205\\
5.5	0.299532857626443\\
5.51	0.298655607298603\\
5.52	0.297761791360826\\
5.53	0.296851574088582\\
5.54	0.295925122417399\\
5.55	0.294982605891887\\
5.56	0.294024196614091\\
5.57	0.293050069191175\\
5.58	0.292060400682484\\
5.59	0.291055370545979\\
5.6	0.290035160584095\\
5.61	0.288999954889018\\
5.62	0.287949939787436\\
5.63	0.286885303784746\\
5.64	0.285806237508785\\
5.65	0.284712933653071\\
5.66	0.283605586919601\\
5.67	0.28248439396122\\
5.68	0.281349553323585\\
5.69	0.280201265386755\\
5.7	0.279039732306417\\
5.71	0.27786515795479\\
5.72	0.276677747861214\\
5.73	0.275477709152461\\
5.74	0.274265250492784\\
5.75	0.273040582023736\\
5.76	0.271803915303767\\
5.77	0.270555463247651\\
5.78	0.269295440065736\\
5.79	0.268024061203064\\
5.8	0.266741543278376\\
5.81	0.26544810402302\\
5.82	0.264143962219805\\
5.83	0.2628293376418\\
5.84	0.261504450991124\\
5.85	0.260169523837733\\
5.86	0.258824778558242\\
5.87	0.257470438274798\\
5.88	0.25610672679402\\
5.89	0.254733868546051\\
5.9	0.253352088523716\\
5.91	0.251961612221839\\
5.92	0.250562665576713\\
5.93	0.249155474905767\\
5.94	0.247740266847442\\
5.95	0.246317268301297\\
5.96	0.244886706368375\\
5.97	0.24344880829184\\
5.98	0.242003801397917\\
5.99	0.24055191303714\\
6	0.239093370525952\\
6.01	0.237628401088657\\
6.02	0.23615723179975\\
6.03	0.234680089526651\\
6.04	0.233197200872859\\
6.05	0.231708792121536\\
6.06	0.230215089179558\\
6.07	0.228716317522038\\
6.08	0.22721270213734\\
6.09	0.22570446747261\\
6.1	0.224191837379833\\
6.11	0.22267503506244\\
6.12	0.221154283022476\\
6.13	0.21962980300835\\
6.14	0.218101815963184\\
6.15	0.216570541973764\\
6.16	0.215036200220133\\
6.17	0.213499008925817\\
6.18	0.21195918530871\\
6.19	0.210416945532633\\
6.2	0.208872504659575\\
6.21	0.207326076602634\\
6.22	0.205777874079673\\
6.23	0.204228108567692\\
6.24	0.202676990257946\\
6.25	0.201124728011806\\
6.26	0.199571529317377\\
6.27	0.198017600246892\\
6.28	0.196463145414885\\
6.29	0.194908367937152\\
6.3	0.193353469390517\\
6.31	0.191798649773403\\
6.32	0.190244107467226\\
6.33	0.188690039198609\\
6.34	0.187136640002441\\
6.35	0.185584103185767\\
6.36	0.184032620292533\\
6.37	0.182482381069184\\
6.38	0.180933573431129\\
6.39	0.179386383430064\\
6.4	0.177840995222173\\
6.41	0.17629759103721\\
6.42	0.174756351148454\\
6.43	0.173217453843555\\
6.44	0.17168107539627\\
6.45	0.170147390039088\\
6.46	0.16861656993675\\
6.47	0.167088785160669\\
6.48	0.165564203664248\\
6.49	0.164042991259093\\
6.5	0.162525311592137\\
6.51	0.161011326123657\\
6.52	0.159501194106197\\
6.53	0.157995072564391\\
6.54	0.156493116275693\\
6.55	0.154995477751992\\
6.56	0.153502307222147\\
6.57	0.1520137526154\\
6.58	0.15052995954569\\
6.59	0.149051071296865\\
6.6	0.147577228808773\\
6.61	0.146108570664244\\
6.62	0.144645233076952\\
6.63	0.143187349880159\\
6.64	0.141735052516324\\
6.65	0.14028847002759\\
6.66	0.138847729047126\\
6.67	0.137412953791335\\
6.68	0.135984266052907\\
6.69	0.134561785194725\\
6.7	0.133145628144603\\
6.71	0.131735909390866\\
6.72	0.130332740978749\\
6.73	0.128936232507617\\
6.74	0.127546491129\\
6.75	0.126163621545422\\
6.76	0.124787726010036\\
6.77	0.123418904327037\\
6.78	0.122057253852859\\
6.79	0.120702869498137\\
6.8	0.119355843730431\\
6.81	0.1180162665777\\
6.82	0.116684225632517\\
6.83	0.115359806057013\\
6.84	0.114043090588543\\
6.85	0.112734159546067\\
6.86	0.111433090837222\\
6.87	0.11013995996609\\
6.88	0.108854840041642\\
6.89	0.10757780178685\\
6.9	0.106308913548457\\
6.91	0.105048241307385\\
6.92	0.103795848689784\\
6.93	0.102551796978697\\
6.94	0.10131614512634\\
6.95	0.100088949766973\\
6.96	0.0988702652303605\\
6.97	0.0976601435558059\\
6.98	0.0964586345067445\\
6.99	0.0952657855858868\\
7	0.0940816420508967\\
7.01	0.0929062469305942\\
7.02	0.091739641041668\\
7.03	0.0905818630058867\\
7.04	0.0894329492677946\\
7.05	0.088292934112881\\
7.06	0.087161849686207\\
7.07	0.0860397260114811\\
7.08	0.0849265910105658\\
7.09	0.0838224705234065\\
7.1	0.0827273883283668\\
7.11	0.081641366162959\\
7.12	0.0805644237449543\\
7.13	0.0794965787938631\\
7.14	0.0784378470527686\\
7.15	0.0773882423105042\\
7.16	0.0763477764241583\\
7.17	0.0753164593418964\\
7.18	0.0742942991260859\\
7.19	0.0732813019767105\\
7.2	0.0722774722550628\\
7.21	0.0712828125077007\\
7.22	0.0702973234906556\\
7.23	0.0693210041938795\\
7.24	0.0683538518659187\\
7.25	0.0673958620388005\\
7.26	0.0664470285531221\\
7.27	0.0655073435833272\\
7.28	0.0645767976631603\\
7.29	0.063655379711284\\
7.3	0.0627430770570496\\
7.31	0.0618398754664067\\
7.32	0.0609457591679419\\
7.33	0.0600607108790333\\
7.34	0.0591847118321099\\
7.35	0.0583177418010041\\
7.36	0.0574597791273863\\
7.37	0.0566108007472685\\
7.38	0.0557707822175694\\
7.39	0.0549396977427251\\
7.4	0.0541175202013394\\
7.41	0.0533042211728588\\
7.42	0.0524997709642644\\
7.43	0.0517041386367693\\
7.44	0.0509172920325106\\
7.45	0.0501391978012268\\
7.46	0.0493698214269105\\
7.47	0.0486091272544258\\
7.48	0.0478570785160813\\
7.49	0.04711363735815\\
7.5	0.0463787648673247\\
7.51	0.0456524210971024\\
7.52	0.0449345650940864\\
7.53	0.0442251549241985\\
7.54	0.0435241476987926\\
7.55	0.0428314996006617\\
7.56	0.0421471659099276\\
7.57	0.0414711010298096\\
7.58	0.0408032585122601\\
7.59	0.0401435910834611\\
7.6	0.0394920506691755\\
7.61	0.0388485884199429\\
7.62	0.0382131547361147\\
7.63	0.0375856992927223\\
7.64	0.0369661710641687\\
7.65	0.0363545183487397\\
7.66	0.0357506887929274\\
7.67	0.0351546294155594\\
7.68	0.0345662866317275\\
7.69	0.0339856062765119\\
7.7	0.0334125336284925\\
7.71	0.0328470134330438\\
7.72	0.0322889899254083\\
7.73	0.0317384068535412\\
7.74	0.0311952075007242\\
7.75	0.0306593347079419\\
7.76	0.0301307308960168\\
7.77	0.0296093380874989\\
7.78	0.029095097928305\\
7.79	0.0285879517091046\\
7.8	0.0280878403864484\\
7.81	0.0275947046036344\\
7.82	0.0271084847113114\\
7.83	0.0266291207878126\\
7.84	0.0261565526592195\\
7.85	0.0256907199191516\\
7.86	0.02523156194828\\
7.87	0.0247790179335609\\
7.88	0.0243330268871889\\
7.89	0.0238935276652653\\
7.9	0.0234604589861821\\
7.91	0.0230337594487173\\
7.92	0.022613367549841\\
7.93	0.0221992217022314\\
7.94	0.0217912602514977\\
7.95	0.0213894214931097\\
7.96	0.0209936436890328\\
7.97	0.0206038650840679\\
7.98	0.0202200239218939\\
7.99	0.0198420584608144\\
8	0.0194699069892059\\
8.01	0.019103507840669\\
8.02	0.0187427994088807\\
8.03	0.0183877201621492\\
8.04	0.0180382086576705\\
8.05	0.0176942035554866\\
8.06	0.0173556436321467\\
8.07	0.0170224677940708\\
8.08	0.0166946150906171\\
8.09	0.0163720247268531\\
8.1	0.0160546360760318\\
8.11	0.0157423886917736\\
8.12	0.0154352223199552\\
8.13	0.0151330769103064\\
8.14	0.0148358926277161\\
8.15	0.0145436098632491\\
8.16	0.0142561692448743\\
8.17	0.0139735116479077\\
8.18	0.0136955782051693\\
8.19	0.0134223103168583\\
8.2	0.0131536496601461\\
8.21	0.0128895381984906\\
8.22	0.0126299181906737\\
8.23	0.0123747321995629\\
8.24	0.0121239231006\\
8.25	0.0118774340900205\\
8.26	0.0116352086928026\\
8.27	0.0113971907703519\\
8.28	0.0111633245279221\\
8.29	0.0109335545217749\\
8.3	0.0107078256660808\\
8.31	0.0104860832395661\\
8.32	0.0102682728919047\\
8.33	0.0100543406498618\\
8.34	0.00984423292318888\\
8.35	0.00963789651027523\\
8.36	0.00943527860355735\\
8.37	0.00923632679469026\\
8.38	0.00904098907948317\\
8.39	0.00884921386260283\\
8.4	0.00866094996204767\\
8.41	0.00847614661339551\\
8.42	0.00829475347382863\\
8.43	0.00811672062593896\\
8.44	0.00794199858131661\\
8.45	0.00777053828392544\\
8.46	0.00760229111326832\\
8.47	0.00743720888734606\\
8.48	0.00727524386541275\\
8.49	0.0071163487505313\\
8.5	0.00696047669193228\\
8.51	0.00680758128717966\\
8.52	0.00665761658414656\\
8.53	0.00651053708280472\\
8.54	0.00636629773683088\\
8.55	0.00622485395503358\\
8.56	0.00608616160260388\\
8.57	0.00595017700219317\\
8.58	0.00581685693482179\\
8.59	0.00568615864062177\\
8.6	0.00555803981941703\\
8.61	0.00543245863114457\\
8.62	0.00530937369612007\\
8.63	0.00518874409515118\\
8.64	0.00507052936950197\\
8.65	0.00495468952071215\\
8.66	0.00484118501027384\\
8.67	0.00472997675916999\\
8.68	0.00462102614727719\\
8.69	0.00451429501263655\\
8.7	0.00440974565059588\\
8.71	0.00430734081282629\\
8.72	0.00420704370621683\\
8.73	0.00410881799165\\
8.74	0.00401262778266171\\
8.75	0.00391843764398876\\
8.76	0.00382621259000694\\
8.77	0.00373591808306305\\
8.78	0.00364752003170384\\
8.79	0.0035609847888051\\
8.8	0.00347627914960381\\
8.81	0.00339337034963658\\
8.82	0.00331222606258721\\
8.83	0.00323281439804643\\
8.84	0.00315510389918684\\
8.85	0.00307906354035585\\
8.86	0.00300466272458949\\
8.87	0.00293187128105012\\
8.88	0.0028606594623906\\
8.89	0.00279099794204787\\
8.9	0.00272285781146872\\
8.91	0.00265621057727016\\
8.92	0.00259102815833753\\
8.93	0.00252728288286249\\
8.94	0.00246494748532389\\
8.95	0.0024039951034138\\
8.96	0.00234439927491129\\
8.97	0.00228613393450666\\
8.98	0.00222917341057807\\
8.99	0.0021734924219235\\
9	0.00211906607445008\\
9.01	0.00206586985782318\\
9.02	0.00201387964207761\\
9.03	0.00196307167419305\\
9.04	0.00191342257463601\\
9.05	0.00186490933387047\\
9.06	0.00181750930883927\\
9.07	0.0017712002194183\\
9.08	0.00172596014484571\\
9.09	0.00168176752012786\\
9.1	0.00163860113242423\\
9.11	0.00159644011741304\\
9.12	0.00155526395563949\\
9.13	0.00151505246884854\\
9.14	0.00147578581630389\\
9.15	0.00143744449109497\\
9.16	0.00140000931643379\\
9.17	0.00136346144194302\\
9.18	0.00132778233993727\\
9.19	0.00129295380169897\\
9.2	0.00125895793375037\\
9.21	0.00122577715412341\\
9.22	0.00119339418862865\\
9.23	0.0011617920671249\\
9.24	0.00113095411979088\\
9.25	0.00110086397340032\\
9.26	0.00107150554760171\\
9.27	0.00104286305120415\\
9.28	0.00101492097847045\\
9.29	0.000987664105418677\\
9.3	0.000961077486133441\\
9.31	0.000935146449087961\\
9.32	0.000909856593478026\\
9.33	0.000885193785569004\\
9.34	0.000861144155056849\\
9.35	0.000837694091444159\\
9.36	0.000814830240432241\\
9.37	0.000792539500330117\\
9.38	0.000770809018481389\\
9.39	0.000749626187709831\\
9.4	0.000728978642784507\\
9.41	0.000708854256905303\\
9.42	0.000689241138209577\\
9.43	0.000670127626300705\\
9.44	0.000651502288799236\\
9.45	0.000633353917917341\\
9.46	0.000615671527057199\\
9.47	0.000598444347433983\\
9.48	0.000581661824723986\\
9.49	0.000565313615738524\\
9.5	0.000549389585124119\\
9.51	0.000533879802089483\\
9.52	0.000518774537159811\\
9.53	0.000504064258958826\\
9.54	0.000489739631019037\\
9.55	0.000475791508620601\\
9.56	0.000462210935659209\\
9.57	0.000448989141543317\\
9.58	0.000436117538121123\\
9.59	0.000423587716637555\\
9.6	0.000411391444721603\\
9.61	0.000399520663404253\\
9.62	0.000387967484167287\\
9.63	0.000376724186023164\\
9.64	0.000365783212626227\\
9.65	0.00035513716941539\\
9.66	0.000344778820788498\\
9.67	0.00033470108730853\\
9.68	0.000324897042941733\\
9.69	0.000315359912327858\\
9.7	0.000306083068082558\\
9.71	0.000297060028132048\\
9.72	0.000288284453080093\\
9.73	0.00027975014360736\\
9.74	0.000271451037903185\\
9.75	0.000263381209129764\\
9.76	0.000255534862918768\\
9.77	0.000247906334900376\\
9.78	0.000240490088264704\\
9.79	0.000233280711355582\\
9.8	0.000226272915296628\\
9.81	0.000219461531649576\\
9.82	0.000212841510104734\\
9.83	0.000206407916203538\\
9.84	0.000200155929093061\\
9.85	0.000194080839312381\\
9.86	0.000188178046610692\\
9.87	0.000182443057797005\\
9.88	0.000176871484621314\\
9.89	0.000171459041687076\\
9.9	0.000166201544394823\\
9.91	0.000161094906916755\\
9.92	0.000156135140202134\\
9.93	0.000151318350013282\\
9.94	0.000146640734991985\\
9.95	0.000142098584756127\\
9.96	0.000137688278026307\\
9.97	0.000133406280782254\\
9.98	0.000129249144448798\\
9.99	0.000125213504111178\\
10	0.000121296076759446\\
};
%\addlegendentry{sigma points}

\addplot [color=mycolor2, line width=2.0pt, forget plot]
  table[row sep=crcr]{%
5.04857024213078	0\\
5.04857024213078	0.6\\
};
\addplot [color=mycolor1, dashed, line width=2.0pt, forget plot]
  table[row sep=crcr]{%
5.0284309151552	0\\
5.0284309151552	0.6\\
};
\end{axis}
\end{tikzpicture}%
% -------------------------------------------

% -------------------------------------------
\section{Qualitative Visualization results}
\label{secS7}
% -------------------------------------------
{\color{red}{\emph{This supplementary is for Section~4.2 and~4.3 of the main paper.}}} In this section, we show qualitative results on both instance segmentation and semantic segmentation. To demonstrate the superiority of our method, we present visualization results of ablation studies on instance segmentation, as well as comparisons with state-of-the-art methods on both instance segmentation and semantic segmentation. 
% 
The obtained visualization results are shown in Figure~\ref{figs2}. From the results, it can be observed that compared to other methods, our method can achieve more accurate object masks that better fit the actual boundaries of the objects themselves.
% -------------------------------------------
% \begin{figure}[t]
%     \centering
%     \includegraphics[width=\linewidth]{figure/images/concept_affordance_prompt_v2.pdf}
%     \caption{Results of two different levels of prompts.}
%     \label{fig:prompt}
% \end{figure}
% -------------------------------------------

as well as the pseudo-code for when the stripe size is set to $2$ in Section~\ref{secS8}. 
% -------------------------------------------
\section{Pseudo-code fo stripe size = $2$}
\label{secS8}
% -------------------------------------------
In this code snippet, stripe size is set to 2, and relevant features are directly obtained using the gather function instead of reshaping them with img2windows. This operation can reduce unnecessary reshaping operations and improves the efficiency of the code.
% -------------------------------------------
\begin{python}
function cross_shaped_window_attention(x, num_heads, window_size):
    # x: given feature
    # num_heads: head number
    # window_size: window size

    # Get dimensions
    (batch_size, seq_length, d_model) = shape(x)

    # Split into multiple heads
    Q, K, V = split_heads(x, num_heads)

    # Initialize attention output
    attention_output = zeros(batch_size, seq_length, d_model)

    # Initialize previous head's output for cascaded attention
    previous_Q = zeros(batch_size, seq_length, d_model)
    previous_K = zeros(batch_size, seq_length, d_model)
    previous_V = zeros(batch_size, seq_length, d_model)

    # Calculate attention for each head
    for head in range(num_heads):
        for position in range(seq_length):
            # Get cross-shaped window indices
            window_indices = get_cross_shaped_window_indices(position, window_size)

            # Gather Q, K, V for the current window
            Q_window = gather(Q[head], window_indices)
            K_window = gather(K[head], window_indices)
            V_window = gather(V[head], window_indices)

            # Incorporate previous head's output for cascaded attention
            if head > 0:
                Q_window += previous_Q
                K_window += previous_K
                V_window += previous_V

            # Calculate attention scores
            attention_scores = softmax(Q_window * K_window^T / sqrt(d_k))

            # Compute the attention output for the current position
            attention_output[position] = attention_scores * V_window

        # Update previous head's output for the next head
        previous_Q = Q_window
        previous_K = K_window
        previous_V = V_window

    # Final linear transformation
    attention_output = linear_transform(attention_output)
    return attention_output

function feed_forward_network(x):
    # Feed Forward Network
    x = ReLU(linear(x))
    x = linear(x)
    return x
\end{python}

\begin{python}
def get_cross_shaped_window_indices(position, window_size, seq_length):
    # Initialize the list of indices
    indices = []

    # Add the current position
    indices.append(position)

    # Add vertical neighbors (up and down)
    for offset in range(-window_size, window_size + 1):
        if position + offset >= 0 and position + offset < seq_length:
            indices.append(position + offset)

    # Remove duplicates and sort the indices
    indices = list(set(indices))
    indices.sort()

    return indices
\end{python}
% ----------------------------------------------


\end{document}