\documentclass[10pt,twocolumn,letterpaper]{article}

%\usepackage[review]{cvpr}      % To produce the REVIEW version
\usepackage[preprint]{cvpr}
\usepackage{times}
\usepackage{epsfig}
\usepackage{graphicx}
\usepackage{amsmath}
\usepackage{amssymb}
%\usepackage{subfigure}
\usepackage{booktabs}
\usepackage{tabularx}
\usepackage{color}
\usepackage{xcolor,array}
% Include other packages here, before hyperref.

% If you comment hyperref and then uncomment it, you should delete
% egpaper.aux before re-running latex.  (Or just hit 'q' on the first latex
% run, let it finish, and you should be clear).
\definecolor{cvprblue}{rgb}{0.21,0.49,0.74}
\usepackage[pagebackref=true,breaklinks=true,letterpaper=true,colorlinks,citecolor=cvprblue,bookmarks=false]{hyperref}
\def\paperID{349} % *** Enter the Paper ID here
\def\confName{3DV\xspace}
\def\confYear{2024\xspace}
% \iccvfinalcopy % *** Uncomment this line for the final submission

% Pages are numbered in submission mode, and unnumbered in camera-ready
%\ificcvfinal\pagestyle{empty}\fi

\definecolor{mark1}{rgb}{0.2,0.2,0.8}
\definecolor{textcolor}{rgb}{0.42, 0.35, 0.8}
\definecolor{textcolor1}{rgb}{0.45,.42,.75}

\newcommand{\adnane}[1]{\textcolor{blue}{#1}}

\newcommand{\q}[1]{\textcolor{red}{#1}}

%
% --- inline annotations
%
\newcommand{\red}[1]{{\color{red}#1}}
\newcommand{\todo}[1]{{\color{red}#1}}
\newcommand{\TODO}[1]{\textbf{\color{red}[TODO: #1]}}
% --- disable by uncommenting  
% \renewcommand{\TODO}[1]{}
% \renewcommand{\todo}[1]{#1}



\newcommand{\VLM}{LVLM\xspace} 
\newcommand{\ours}{PeKit\xspace}
\newcommand{\yollava}{Yo’LLaVA\xspace}

\newcommand{\thisismy}{This-Is-My-Img\xspace}
\newcommand{\myparagraph}[1]{\noindent\textbf{#1}}
\newcommand{\vdoro}[1]{{\color[rgb]{0.4, 0.18, 0.78} {[V] #1}}}
% --- disable by uncommenting  
% \renewcommand{\TODO}[1]{}
% \renewcommand{\todo}[1]{#1}
\usepackage{slashbox}
% Vectors
\newcommand{\bB}{\mathcal{B}}
\newcommand{\bw}{\mathbf{w}}
\newcommand{\bs}{\mathbf{s}}
\newcommand{\bo}{\mathbf{o}}
\newcommand{\bn}{\mathbf{n}}
\newcommand{\bc}{\mathbf{c}}
\newcommand{\bp}{\mathbf{p}}
\newcommand{\bS}{\mathbf{S}}
\newcommand{\bk}{\mathbf{k}}
\newcommand{\bmu}{\boldsymbol{\mu}}
\newcommand{\bx}{\mathbf{x}}
\newcommand{\bg}{\mathbf{g}}
\newcommand{\be}{\mathbf{e}}
\newcommand{\bX}{\mathbf{X}}
\newcommand{\by}{\mathbf{y}}
\newcommand{\bv}{\mathbf{v}}
\newcommand{\bz}{\mathbf{z}}
\newcommand{\bq}{\mathbf{q}}
\newcommand{\bff}{\mathbf{f}}
\newcommand{\bu}{\mathbf{u}}
\newcommand{\bh}{\mathbf{h}}
\newcommand{\bb}{\mathbf{b}}

\newcommand{\rone}{\textcolor{green}{R1}}
\newcommand{\rtwo}{\textcolor{orange}{R2}}
\newcommand{\rthree}{\textcolor{red}{R3}}
\usepackage{amsmath}
%\usepackage{arydshln}
\DeclareMathOperator{\similarity}{sim}
\DeclareMathOperator{\AvgPool}{AvgPool}

\newcommand{\argmax}{\mathop{\mathrm{argmax}}}     



\begin{document}

%%%%%%%%% TITLE
%\title{Multi-Human Implicit Surface and Volume Rendering from Sparse Views using Geometry Constraints}
%\title{Multi-Human Implicit Surface and Volume Rendering from Sparse Views using Geometry Constraints}

\title{Few-Shot Multi-Human Neural Rendering Using Geometry Constraints}
%\title{Volume Rendering of Multi-Human from Sparse Images Using Geometry Constraints}

\author{Qian li$^{1}$, Victoria Fern\`andez Abrevaya$^{2}$, Franck Multon$^{1}$, Adnane Boukhayma$^{1}$\\
$^{1}$Inria, University Rennes, IRISA, CNRS, France\\
$^{2}$Max Planck Institute for Intelligent Systems, Germany
}

% \author{First Author\\
% Institution1\\
% Institution1 address\\
% {\tt\small firstauthor@i1.org}
% % For a paper whose authors are all at the same institution,
% % omit the following lines up until the closing ``}''.
% % Additional authors and addresses can be added with ``\and'',
% % just like the second author.
% % To save space, use either the email address or home page, not both
% \and
% Second Author\\
% Institution2\\
% First line of institution2 address\\
% {\tt\small secondauthor@i2.org}
% }

\maketitle
% Remove page # from the first page of camera-ready.
%\ificcvfinal\thispagestyle{empty}\fi


\begin{abstract}
% We present a learning-based method for multiple human rendering from sparse sets of multi-view images. Most current works are focused on single human settings that deliver accurate geometry and appearance using implicit neural representations. However, extending these methods for estimating multiple humans from sparse images remains challenging due to additional occlusion and clutter of multiple humans and the limited number of input views. We propose a neural implicit reconstruction method that addresses the inherent challenges. First, we propose to use geometry constraints by exploiting pre-computed meshes using a human body model (SMPL). Specifically, we regularize the signed distances using the SMPL mesh and leverage bounding boxes for improved rendering. Second, we propose a patch-based ray regularization to minimize rendering inconsistencies and a saturation regularization for robust optimization in variable illumination. Extensive experiments on both real-world and synthetic datasets demonstrate the benefits of our approach and show state-of-the-art performance against existing neural reconstruction methods. 

We present a method for recovering the shape and radiance of a scene consisting of multiple people given solely a few images. 
Multi-human scenes are complex due to additional occlusion and clutter. For single-human settings, existing approaches using implicit neural representations have achieved impressive results that deliver accurate geometry and appearance. 
However, it remains challenging to extend these methods for estimating multiple humans from sparse views. 
We propose a neural implicit reconstruction method that addresses the inherent challenges of this task through the following contributions: First, we propose to use geometry constraints by exploiting pre-computed meshes using a human body model (SMPL). Specifically, we regularize the signed distances using the SMPL mesh and leverage bounding boxes for improved rendering. Second, we propose a ray regularization scheme to minimize rendering inconsistencies, and a saturation regularization for robust optimization in variable illumination.  Extensive experiments on both real and synthetic datasets demonstrate the benefits of our approach and show state-of-the-art performance against existing neural reconstruction methods. 

% Finally, we demonstrate how our framework can be used for editing applications. 
\end{abstract}

% We present a neural learning method for multiple human reconstruction from a sparse set of camera views. Recent neural surface reconstruction methods could generate both 3D geometry and appearance, while usually fail to reconstruct consistent surfaces due to lacking of explicit multi-view geometry constraints, especially for complex scene(\eg multiple human). {\bf Previous works demonstrated that using geometry priors for simple objects significantly enhance the quality of the surfaces. In this work, we propose to extend this idea to using human body model (\eg SMPL) priors, represented by signed distance functions (SDF).} However, since human SMPL usually lacks sufficient details (\eg hair, cloth), we estimate SDF together with its' uncertainty for explicit smooth geometry regularization. In addition, we propose a patch based ray regularization to minimizes potential reconstruction inconsistency and saturation regularization for robust optimization in variable illuminations.  Our evaluations on both real human dataset (CMU Panoptic Dataset\cite{Simon_2017_CVPR,Joo_2017_TPAMI}) and synthetic data (THUman2.0 Dataset and MultiHuman-Dataset \cite{tao2021function4d,zheng2021deepmulticap} demonstrate state-of-the-art performance quantitatively and qualitatively. Extensive experiments show our methods enable a variety editing applications in 3D space without additional assistance of depth, masks, or segmentation.

%Existing multi-person reconstruction is often based on human models, which could reconstruct complex geometry but lose rendering quality due to lacking hair and clothing detail. Recently, volumetric rendering (\eg NeRF\cite{mildenhall2020nerf}) demonstrate promising rendering quality on novel views synthesis from dense input views, while reconstructing fidelity surfaces remains a challenge. Moreover, surface presentation methods \cite{yariv2021volume,wang2021neus} explore surface reconstruction during volume rendering, but it's still hard to handle occlusions and complex geometry (\eg multi-human) from sparse input views. Our methods take the advantages of the human model(\eg SMPL), recent volumetric rendering and scene representation methods. Specifically,we define our multi-human SMPL's surfaces as a zero-level set of a signed distance function (SDF) and train a neural SDF representation. We show how to incorporate this representation in point sampling, neural rendering and reconstruction, and propose a joint optimization for non-rigid shapes (\egclothed humans). Finally,

% We present a learning-based method for reconstructing multiple humans from a sparse set of camera views. Current surface reconstruction methods can generate geometry and appearance simultaneously, but suffer to reconstruct 3D consistent surfaces due to a lack of explicit multi-view geometric constraints, especially for complex scenes (e.g. multiple humans). Previous works show that the use of geometric priors for simple objects (or single scenes) significantly improves the fidelity of reconstructed surfaces. Leveraging that, we propose the use of utilizing human body model (e.g. SMPL) priors, represented by signed distance functions (SDF). However, human SMPL often lacks sufficient details (e.g. hair, cloth), and thus, we estimate the SDF along with its uncertainty to obtain smooth geometry regularization. In addition, we propose a patch-based method for ray regularization to potentially minimize inconsistencies in the reconstruction and to enable saturation regularization to increase the robustness during the optimization of variable illuminations. Our evaluations on both real human datasets (CMU Panoptic Dataset [32, 63]) and synthetic data (THUman2.0 Dataset and MultiHuman Dataset [83, 94] demonstrate state-of-the-art performance quantitatively and qualitatively. Extensive experiments show that our method enables a variety of editing applications in 3D space without additional assistance of depth, masks, or segmentation.


% \textcolor{mark1}{1. #' reviewers pointed that there were no specific design choices to make the method work for the multi-human case, i.e., after defining bounding boxes for each human, the reconstruction was performed independently and individually per-person.'
% Additionally, the reviewer requested a comparison against single person methods. \\
% Different from STNeRF, our sampling and learning not performed on each box individually,  rendering performed on per bounding box. The optimization is perform on the hole scene, not individually person. (Note that the sig-asia paper using sparsity loss to regularize the occupancy
% values) Since we render the hole scene, the occupancy of points on the ray will be close to 1 after converge. also, maybe the  normalized weights in equa11 helps. . goes to sig-asia- Figure 5. \\
% 2. #'The author’s rebuttal pointed that these methods would fail, because the segmentation masks required by single-person methods would not be reliable in multi-person scenarios. This argument, however did not convince ZCeA, as one could compute the segmentation masks within the SMPL bounding box projected onto the image. '\\
% 3. BY ZCeA 'Unless the authors give concrete examples where static reconstruction of humans based on multi-view is needed, I have problems to follow this argumentation. '\\multiple person VS single person\\ 
% Static Scene VS video\\no existing methods to do static multi-human rendering?? More challenge than video. The meaning of static scene reconstruction rather than video.\\ 
% 4. #'The quantitative evaluation discussion provided simply says that our method is better than prior work (same for the InfoNerf comparisons) but does not provide some insights to the reader as to why this is the case. For example I'd expect this section to clearly outline the key components for the proposed method that make it outperform this prior works.'\\
% GeoNeus draw a fig(see-fig-3 in GeoNeus paper) explaining when ray reach surface the weights looks like, it might also helps explain multi-person works for us. Do we need do the same?\\
% 5. #What makes the proposed method outperform prior works even when one person is visible in the scene? How would the ablation study look like (Table 4) if all components were removed and then each one was added. From Table 4 currently it's a little hard to evaluate which contribution has the highest impact as when each module is removed the other 2 are still integrated.\\
% 6. NeuS adpoted L1 loss regularizing rgb rendering for more stable training, we followed this. That might be anther reason for blur results.\\ 
% }
%%%%%%%%% BODY TEXT-----------------------------------------------
\section{Introduction}
\label{sec:intro}

Human reconstruction from single images \cite{choutas2020monocular, kanazawa2018end,liu2022recent}, multiple images \cite{guo2019relightables, collet2015high}, RGB videos \cite{alldieck2018detailed,Kocabas20} or RGB-D data \cite{yu2017bodyfusion,yu2018doublefusion} has received a lot of attention, much less explored is the task of \emph{multiple} human scenario, which is essential for scene understanding, behavior modeling, collaborative augmented reality, and sports analysis.  
%
The multi-human setting introduces additional challenges, as there is now a higher level of occlusion and clutter 
which hinders matching and reconstruction. 
Although in principle one could approach this by first detecting and then independently processing each person, 
simultaneous reconstruction of multiple humans can help to globally reason about occlusion at the level of the scene~\cite{jiang2020coherent, sun2022putting}, 
%which has been shown to produce better results~\cite{jiang2020coherent, sun2022putting}, 
and can potentially recover coherent 3D spatial relations among the people.

Several recent works have attempted to recover multiple humans from a single view \cite{choi2022learning, sun2022putting, sun2021monocular, zanfir2018deep, zanfir2018monocular, fieraru2020three, jiang2020coherent, zhang2021body, ugrinovic2021body,mustafa2021multi}. However, the majority of these are based on regressing the parameters of a human body model --typically SMPL \cite{loper2015smpl}--
which provides coarse reconstructions that %cannot handle 
lack hair, clothing, and geometric details. 
Multi-view settings can help resolve some of the occlusions as well as depth ambiguities, but require a dense array of RGB cameras to achieve a detailed reconstruction \cite{collet2015high, joo2015panoptic,vlasic2009dynamic}.
% which is not easily accessible. % available. 
A more convenient capture system %that could in principle still deliver optimal results 
is the \emph{sparse} multi-view setting, where only a handful of cameras is required.
%, where the number of cameras is limited to 2-15. 
However, due to the decreased number of views and increased level of occlusion, 
existing methods require segmentation masks and a pre-scanned template mesh \cite{liu2011markerless, wu2013set}, rely on a coarse body model \cite{zhang2021lightweight, huang2021dynamic}, or require temporal information \cite{zheng2021deepmulticap, huang2021dynamic}.

A parallel line of work simultaneously tackles the novel-view-synthesis and geometry-reconstruction problems by combining neural coordinate-based representations, \eg implicit signed distance functions (SDFs) \cite{park2019deepsdf}, with differentiable rendering \cite{yariv2021volume,wang2021neus,yariv2020multiview,mildenhall2020nerf}. 
This approach has the advantage of producing, along with geometry, renderings from novel viewpoints that can capture complex surface/light interactions, increasing the scope of applications. 
NeRF~\cite{mildenhall2020nerf}, for example, uses volumetric rendering to produce impressive images under novel views, albeit at the cost of sub-optimal geometries due to the unconstrained volumetric representation. 
SDF-based methods \cite{yariv2021volume,wang2021neus,yariv2020multiview}, while delivering images of slightly lower quality, have been shown to produce 3D surfaces that are competitive with classical approaches. 
For humans, this has been leveraged to obtain geometry and appearance from monocular video \cite{jiang2022selfrecon,chen2021animatable}, RGB-D video \cite{dong2022pina}, and sparse multi-view video \cite{wang2022arah, liu2021neural, peng2021neural, zheng2021deepmulticap, kwon2021neural, peng2021animatable, xu2021h, weng2022humannerf}. 
%% NOTE: \cite{xu2021h} also has experiments for static sparse MV image
However, none of these works, with the exception of \cite{zheng2021deepmulticap,zhang2021editable}, were designed to handle the increased geometric complexity and occlusion of the multi-human case. 
%%%%%%%%%%%%%%%%%%%%%%%%%%%%%%%%%%%%%%%%%%%%%not the only work, STnerf, SIgsia2022 and deep multicap
Current works \cite{zheng2021deepmulticap,zhang2021editable} address the multi-human setting, but both require a set of videos, which effectively becomes a dense array of views as long as deformations are modeled correctly.
%DeepMultiCap \cite{zheng2021deepmulticap} is the only work that addresss the multi-human setting, but the method requires segmentation and was focused on reconstruction from videos, which effectively becomes a dense array of views as long as deformations are modeled correctly.

In this paper, we address the problem of multiple 3D human surfaces and volume rendering from sparse static multi-view images. Our key insight is that human-specific geometric constraints can be leveraged to tackle the challenging sparse-view setting.

Specifically, we first obtain a SMPL body model from the input data and use it to initialize the implicit SDF network, where we define the surface of a multi-human scene as the zero-level set of the SDF. 
Then, the geometry network is optimized with multi-view images by leveraging surface and volume rendering ~\cite{wang2021neus} along with uncertainty estimation methods \cite{deng2022depth,roessle2022dense}, where the SMPL meshes are treated as noisy estimations.  
% [to keep the method general for novel scenes, we do not rely on features pre-trained on a large dataset]
To achieve higher rendering quality from sparse training views, we additionally propose a patch-based regularization loss that guarantees consistency across different rays and a saturation regularization that ensures consistency for variable image illuminations within the same scene.% \va{I feel this is a problem that was not pointed out before, and it would be nice to elaborate better}.

We evaluate our method quantitatively and qualitatively on real multi-human (CMU Panoptic~\cite{Simon_2017_CVPR,Joo_2017_TPAMI}) and synthetic (MultiHuman~\cite{zheng2021deepmulticap}) datasets. We demonstrate results on 5,10,15 and 20 training views, where we achieve state-of-the-art performance in terms of surface reconstruction and novel view quality. 

%In summary, our contributions are: 
%\begin{itemize}
%\item We propose the first neural implicit surface and volume rendering for multiple humans from sparse static images; 
%\item We propose a novel geometric initialization and regularization based on human SMPL, which allows for multi-human rendering simultaneously; %To address the problem of occlusion, we propose the use of SMPL for geometric regularization; 
%\item We propose a patch-based ray consistency regularization and an image saturation regularization that ensures illumination consistency across views;
%\item Our method achieves state-of-the-art performance and the code will be public online. 
%\end{itemize}

%%%%%%%%%%%%%%%%%%%%%%
%% MOTIVATION
% Motivations for multi-human from the multi-human, single-view domain:
%       - A lot of occlusions, if only a crop of the person is used then SOTA methods might fail (\cite{choi2022learning})
%       - Simultanesouly considering multiple person has been empirically shown to work better \cte{all the works on mult-human SMPL}
%       - The reconstructions can be made coherent in 3D space, i.e. with correct relative depth among the people. Plausible 3D spatial relations.
% The problem with SMPL based methods is that they cannot reconstruct clothes, hair, etc

% Applications in behavior analysis, automatic video analysis of sport events, or collaborative augmented reality applications
% human-computer interaction, human behavioral modeling, assisted therapy, monitoring sports performances, protection and security, special effects, modeling and indexing archival footage, or self-driving cars. (this is for monocular)

% We take one step beyond SMPL and use implicit representations for more accurate reconstructions

% "The straightforward solution consists in regarding different people as independent instances and estimating the body shapes and poses one by one using a single-person approach. This strategy, however, may result in inconsistent spatial arrangements and erroneous poses of the reconstructed people." (re-phrase)

% *** Why we cannot do cut one person then reconstruct it --> there should be experiments, else be careful with claims

%%%%%%%%%%%%%%%%%%%%%%
%% STORY
%   - Reconstructing geometry from a scene with multiple humans is hard because of occlusions
%   - Multiple views can help disambiguate, but it's stil hard.
%       => most are single human. For multi-human one can detect and reconstruct, but this can fail - and matching detections among views might not be straight-forward (?) Also, computational time scales (linearly) with the number of people
%       => there are a few multi-human, but ?
%       => people resort to SMPL to make the problem more tractable
%           *** classic matching methods: problem with multi-humans??
%   - SMPL doesn't have clothes or hair so geometry sucks
%   - We consider in particular the scenario where only a sparse number of cameras is available; dense rigs are not easily accessible. Dense => expensive and sophisticated hardware setup and low run-time efficiency (re-phrase)
%   - To address this, we present a method for multi-human 3D reconstruction from multiple views based on neural rendering. 
%   - To handle the occlussions and sparse views, we first fit a SMPL model. Following the recent line of works that make use of geometric quantities to improve reconstruction, we propose to to initialize the reconstruction by building an SDF from the acquired SMPL models.
%   - 

%%%%%%%%%%%%%%%%%%%%%%
% Multi-view capture with dense views and/or temporal data demands a significant cost in time and money, with increased complexity of device requirements, synchronisation, data storage and transfer times. Consequently, 3D reconstruction from a sparse set of images has become an increasingly popular problem~\cite{niemeyer2022regnerf, kim2022infonerf,long2022sparseneus,dong2021shape}, more so after the recent progress in deep neural rendering and reconstruction approaches~\cite{mildenhall2020nerf, park2019deepsdf}. % However, it remains a difficult problem due to the sparsity of 3D cues, requiring carefully designed regularization losses~\cite{kim2022infonerf,niemeyer2022regnerf} or knowledge transfer from pre-trained networks~\cite{yu2021pixelnerf}. This cue scarcity problem becomes even worse when reconstructing cluttered scenes, such as those containing a multitude of people., due to increased occlusions.
% % Multi-view capture with dense views and/or temporal data entails significant capture device requirements, synchronisation, data storage and transfer challenges. Hence, being able to capture scenes from a mere set of static sparse images has become a more than ever popular problem \cite{niemeyer2022regnerf, kim2022infonerf,long2022sparseneus,dong2021shape} especially in the wake of deep learning. 
% Yet, the scarcity of 3D cues in such setups requires carefully designed regularization or knowledge transfer. This scarcity is exacerbated further with cluttered scenes, such as scenes containing a multitude of people.   

% 3D reconstruction of humans from single images \cite{choutas2020monocular, kanazawa2018end,liu2022recent}, multiple images \cite{guo2019relightables, collet2015high}, RGB videos \cite{alldieck2018detailed,Kocabas20} or RGB-D data \cite{yu2017bodyfusion,yu2018doublefusion} has received a lot of attention, much less explored is the task of \emph{multiple} human reconstruction, which is essential for scene understanding, behavior modeling, collaborative augmented reality, and sports analysis.  
% %
% The multi-human setting introduces additional challenges, as there is now a higher level of occlusion and clutter 
% which hinders matching and reconstruction. 
% Although in principle one could approach this by first detecting and then independently processing each person, 
% simultaneous reconstruction of multiple humans can help to globally reason about occlusion at the level of the scene~\cite{jiang2020coherent, sun2022putting}, 
% %which has been shown to produce better results~\cite{jiang2020coherent, sun2022putting}, 
% and can potentially recover coherent 3D spatial relations among the people. %, taking a step towards better scene understanding.
% % simultaneous reconstruction of multiple humans has several advantages. 
% % First, this strategy  allows to further recover coherent 3D spatial relations among the people, taking a step towards better scene understanding.
% % Second, the presence of multiple people can imply a high level of occlusion. While some works explicitly handle this case for single humans \cite{khirodkar2022occluded, zhang2020object}, globally reasoning about occlusion at the level of the scene has been shown to produce better results \cite{jiang2020coherent, sun2022putting}.  
% % Finally, the detect-and-reconstruct approach increases (linearly) in computational time with each new person. 

% % While 
% Several recent works have attempted to recover multiple humans from a single view \cite{choi2022learning, sun2022putting, sun2021monocular, zanfir2018deep, zanfir2018monocular, fieraru2020three, jiang2020coherent, zhang2021body, ugrinovic2021body,mustafa2021multi}. However, the majority of these are based on regressing the parameters of a human body model --typically SMPL \cite{loper2015smpl}--
% which provides coarse reconstructions that %cannot handle 
% lack hair, clothing, and geometric details. 
% Multi-view settings can help resolve some of the occlusions as well as depth ambiguities, but require a dense array of RGB cameras to achieve a detailed reconstruction \cite{collet2015high, joo2015panoptic,vlasic2009dynamic} 
% % which is not easily accessible. % available. 
% A more convenient capture system %that could in principle still deliver optimal results 
% is the \emph{sparse} multi-view setting, where only a handful of cameras is required.
% %, where the number of cameras is limited to 2-15. 
% However, due to the decreased number of views and increased level of occlusion, 
% existing methods require segmentation masks and a pre-scanned template mesh \cite{liu2011markerless, wu2013set}, rely on a coarse body model \cite{zhang2021lightweight, huang2021dynamic}, or require temporal information \cite{zheng2021deepmulticap, huang2021dynamic}.

% %%%\todo{pifu should go somewhere here... pifu is single human though. Goes into the recent line of work: pre-train on a large dataset and then at test time use sparse views: \cite{zheng2021deepmulicap}}


% % --> send the message that MV with SMPL is kind of solved, but we need to take the next step. 
% %but as shown by several works \cite{...}, this is not sufficient to solve the problem. In fact, % in effect, in practice, at bottom
% % "The first sparse for multi-humans" -> \cite{zhang2021lightweight}, 

% % In parallel, there has been development in neural 3D geometry reconstruction by using SDFs \cite{..} or volume-rendered SDFs \cite{...}. These are great but require many views. For the sparse view case, infonerf, regnerf, dietnerf. They show improved results, but as demonstrated here these are not sufficient for the multi-human reconstruction problem.




% %%%%\todo{improving geometry to improve rendering?}

% % Inspired by the success of NeRF~\cite{mildenhall2020nerf}, 
% A parallel line of work simultaneously tackles the novel-view-synthesis and geometry-reconstruction problems by combining neural coordinate-based representations, \eg implicit signed distance functions (SDFs) \cite{park2019deepsdf}, with differentiable rendering \cite{yariv2021volume,wang2021neus,yariv2020multiview,mildenhall2020nerf}. 
% This approach has the advantage of producing, along with geometry, renderings from novel viewpoints that can capture complex surface/light interactions, increasing the scope of applications. 
% NeRF~\cite{mildenhall2020nerf}, for example, uses volumetric rendering to produce impressive images under novel views, albeit at the cost of sub-optimal geometries due to the unconstrained volumetric representation. 
% SDF-based methods \cite{yariv2021volume,wang2021neus,yariv2020multiview}, while delivering images of slightly lower quality, have been shown to produce 3D surfaces that are competitive with classical approaches. 
% For humans, this has been leveraged to obtain geometry and appearance from monocular video \cite{jiang2022selfrecon,chen2021animatable}, RGB-D video \cite{dong2022pina}, and sparse multi-view video \cite{wang2022arah, liu2021neural, peng2021neural, zheng2021deepmulticap, kwon2021neural, peng2021animatable, xu2021h, weng2022humannerf}. 
% %% NOTE: \cite{xu2021h} also has experiments for static sparse MV images (single human)
% %% Monocular video: also "SelfNeRF: Fast Training NeRF for Human from Monocular Self-rotating Video". But this isn't published yet
% %
% However, 
% none of these works, with the exception of \cite{zheng2021deepmulticap}, were designed to handle the increased geometric complexity and occlusion of the multi-human case. DeepMultiCap \cite{zheng2021deepmulticap} is the only %multi-view neural reconstruction 
% work that addresss the multi-human setting, but the method requires a set of videos, which effectively becomes a dense array of views as long as deformations are modeled correctly.

% In this paper we address, for the first time, 
% the problem of reconstructing multiple 3D humans from a \emph{static} and \emph{sparse} set of cameras using neural implicit surfaces. 
% Our key insight is that human-specific geometric constraints can be leveraged to tackle the challenging sparse-view setting.
% %by first fitting  a SMPL body model to the input, building on top of the works 
% % building on the set of works that can faithfully disambiguate pose and shape for multiple people, but only deliver coarse reconstructions, \eg~\cite{huang2017towards,zhang2021lightweight}. %\va{This is a bit risky, since the way we get smpl is a bit shady (in practice, from dense MV cameras)}. 
% Specifically, we first obtain a SMPL body model from the input data, and use this to train a geometry-only implicit SDF network, where we define the surface of a multi-human scene as the zero-level set of the SDF. 
% In a second step, the geometry network is fine-tuned using multi-view images by leveraging the recently-proposed NeuS~\cite{wang2021neus} along with uncertainty-based rendering \cite{deng2022depth,roessle2022dense}, where the SMPL meshes are treated as noisy estimations.  
% % [to keep the method general for novel scenes, we do not rely on features pre-trained on a large dataset]
% To achieve higher rendering quality from sparse training views, we additionally propose a patch-based regularization loss 
% that guarantees consistency across different rays, and a saturation regularization that ensure consistency for variable image illuminations within a same scene.% \va{I feel this is a problem that was not pointed out before, and it would be nice to elaborate better}.

% We evaluate our method quantitatively and qualitatively on real multi-human (CMU Panoptic~\cite{Simon_2017_CVPR,Joo_2017_TPAMI}) and synthetic (MultiHuman~\cite{zheng2021deepmulticap}) datasets. We demonstrate results on 5,10,15 and 20 training views, where we achieve state-of-the-art performance in terms of surface reconstruction and image quality. %Another advantage is that the proposed geometry initialization enables more efficient learning.

% In summary, our contributions are: 
% % 1. We demonstrate an approach to use multiple human bodies as geometric regularization. We propose to use SDF along with estimated uncertainty as explicit geometry constraints, allowing the  learning of details. \\
% (1) We propose the first neural 
% implicit geometry and appearance reconstruction method for multiple humans using a sparse set of static views; 
% (2) To address the problem of occlusion, we propose the use SMPL for geometric regularization; 
% (3) To handle sparse views under occlusion, we propose a patch-based ray consistency regularization, and an image saturation regularization that ensures illumination consistency across views.

%  Code and models will be made available.
% 3.  \todo{I will re-phrase this after finishing the method. This needs to come before in the introduction, if it stays} We combine box rendering with editing, which enables editing multi-humans in 3D space during inference without using masks and segmentation, while preserving the detail and quality of rendered novel multi-human views.   \\



%%%%%%%%%%%%%%%%%%%%%%
% --------------------------------


% Multiple human reconstruction has been a popular and significant topic in computer vision and computer graphics, which enables various applications, such as Virtual Reality (VR) and Augmented Reality (AR), films, teleconferences, and so on. Researches for human reconstruction mainly consist of model-based and model-free, both of them has achieved tremendous progress in recent years. 

% Traditional model-based approaches utilize parametric human bodies (\eg human SMPL \cite{loper2015smpl}) or truncated signed distance fields (TSDFs) for human geometry reconstruction \cite{loper2015smpl,bhatnagar2020combining,liu2021neural,yu2017bodyfusion,yu2018doublefusion,peng2021neural}. %Though human SMPL could provide solid geometry notes, 
% While relying on the human model(\eg SMPL skinning weights of a naked body) may risk losing hair and clothing details. Model-free methods learns from a set of multi-view images or videos with loosing detail constraints \cite{peng2021neural,dong2022pina,peng2021neural,dong2022pina, liu2021neural}, allowing more realistic reconstructed details. With the recent progress of implicit neural representations\cite{sitzmann2019deepvoxels,xu2022point,wiles2020synsin, mildenhall2020nerf,zhang2020NeRF++}, many model-free methods utilize NeRF to achieve photo-realistic rendering. However, NeRF\cite{mildenhall2020nerf} based methods do not sufficiently constrain the 3D geometry, making them hard to reconstruct high-fidelity surfaces or accurate geometry. Further, most existing researches focus on single person reconstruction from videos \cite{peng2021neural,dong2022pina} or images\cite{peng2021neural,dong2022pina, liu2021neural}. For multiple human reconstructions from a single frame, the task becomes even harder to recover geometry and appearance with sufficient details due to multiple human scenes usually containing more complex geometry and occlusions. In this paper, we address those challenges by taking advantage of recent neural implicit surface reconstruction methods\cite{yariv2021volume,wang2021neus,yariv2020multiview} and human bodies models(\eg SMPL\cite{loper2015smpl}), reconstructing multi-human 3D geometry and appearance simultaneously. 

% Recent neural implicit surface reconstruction methods\cite{yariv2021volume,wang2021neus,yariv2020multiview} propose to use a signed distance function (SDF) to present the surface and combine SDF-based density function with volumetric rendering. They could learn an implicit SDF representation and reconstruct geometry and appearance simultaneously. However, the optimization of those methods still depends on direct color field and lack explicit geometry constraints, making them hard to reconstruct consistent geometry and appearance for unseen regions(\eg sparse input view) and occlusion areas (\eg multiple human scene). To alleviate those limitations, we propose to use a multiple human bodies models(SMPL) as geometry initialization, as well as to provide explicit geometry constraints.

% %\cite{yariv2021volume} uses volume density function as Laplace’s cumulative distribution function (CDF) for geometry representation, 
% %Neus\cite{wang2021neus} and Volsdf\cite{yariv2021volume} are among scene representation methodsuse signed distance functions (SDF) for surface representation and introduce the SDF-induced density function to enable the volume rendering to

% %More recently, with the rapidly progress of implicit neural representations\cite{sitzmann2019deepvoxels,xu2022point,wiles2020synsin, mildenhall2020nerf,zhang2020NeRF++}, many researches combine differentiable neural rendering with human bodies model \cite{liu2021neural,peng2021neural,dong2022pina}. Among those,  \cite{liu2021neural} combines human SMPL with implicit learning, and uses it as a proxy to unwarp the surrounding 3D space into a canonical pose, \cite{peng2021neural} could render detailed novel views (\eg with hair,cloths) of the human body from a set sparse input video. Thanks to neural radiance fields (NeRF\cite{mildenhall2020nerf}), those approaches could achieve photo-realistic rendering. However, NeRF\cite{mildenhall2020nerf} does not sufficiently constrain the 3D geometry, making those methods hard to reconstruct high-fidelity surfaces or accurate geometry.

% %Recent scene representation methods combines surface representation with neural volume rendering \cite{yariv2021volume,wang2021neus,yariv2020multiview}. % In order to achieve high quality surface reconstruction while retaining rendering quality, 
% %Among those, \cite{oechsle2021unisurf} proposes a unified framework to reconstruct solid objects from 2D image inputs, 

% Specifically, inspired by Neus\cite{wang2021neus} and Volsdf \cite{yariv2021volume}, we define the surface of multiple human SMPL as a zero-level set of a signed distance function (SDF)  and use it to train a neural SDF presentation as a geometry prior. However, relying on SDF sampled from SMPL directly may risk losing hair and clothing details. Thus, we learn SDF together with uncertainty and combine them together for explicit geometry regularization, which enables the learning of multiple person's clothing and hair details. In addition,
% to achieve higher rendering quality from sparse training views, we propose a patch-based regularization to guarantee consistency across different rays and saturation regularization to ensure image illumination consistency. 
% %retaining multi-view consistency remains a challenge for NeuS-related methods, especially for complex thin structures and large smooth regions. To tackle this challenge, we propose a patch-based ray regularization for photo-metric consistency and saturation regularization for image illumination consistency. 

% %Assuming this SDF presentation storing the multi-human geometry information, we incorporate this by re-initializing the follow-up neural network with the learned weights, utilizing the SDF value for the weights of hierarchical sampling \cite{mildenhall2020nerf} and volume rendering. Inspired by \cite{ortiz2022isdf, dong2022pina}, instead of using the SDF value sampled from SMPL to supervise the neural network learning directly(which consumes lots of time and lacks details), we propose a point-based SDF regularization allowing for the learning of multi-human clothing and hair details. Moreover, we present an approach for human editing during rendering, without additional training or information(\eg depth, mask or segmentation).   \\
% We evaluate our method on both real multi-human datasets (CMU Panoptic Dataset\cite{Simon_2017_CVPR,Joo_2017_TPAMI}) and synthetic data (THUman2.0 Dataset
% and MultiHuman-Dataset \cite{tao2021function4d,zheng2021deepmulticap}) both qualitatively and quantitatively. Specifically,  we demonstrate testing results on 10,15,20 training views,respectively, and achieve state-of-the-art performance on both real datasets and synthetic datasets. %Another advantage is that the proposed geometry initialization enables more efficient learning. 
% In summary, our contributions include:\\
% 1. We demonstrate an approach to use multiple human bodies as geometric initialization. We propose to use SDF along with estimated uncertainty as explicit geometry constraints, allowing the  learning of details. \\
% 2. We propose a patch-based KL regularization to ensure consistency across different rays and image saturation regularization for illumination consistency. \\
% 3. We combine box rendering with editing, which enables editing multi-humans in 3D space during inference without using masks and segmentation, while preserving the detail and quality of rendered novel multi-human views.   \\

\section{Related Work}
\label{sec:relat}

% There is vast amount of works on reconstructing 3D humans from single images \cite{liu2022recent}, monocular video \cite{yuan2022glamr, Kocabas20, alldieck2018video}, RGB-D data \cite{yu2017bodyfusion,yu2018doublefusion,burov2021dynamic} and multi-view data. Here, we focus on methods that take multi-view setups as input, either for single or multiple humans, as well as recent neural surface reconstruction works. 
% \todo{Maybe unify these two again into Human reconstruction, and put initial paragraph there}


\paragraph{Single-Human Reconstruction.} 
There is a vast amount of work on reconstructing 3D humans from single images \cite{bogo2016keep, choutas2020monocular, kanazawa2018end,muller2021self,liu2022recent}, monocular video \cite{yuan2022glamr, Kocabas20, alldieck2018video}, RGB-D data \cite{yu2017bodyfusion,yu2018doublefusion,burov2021dynamic} and multi-view data \cite{starck2007surface,collet2015high,guo2019relightables,huang2018deep}. We concentrate here on the multi-view setting. %, which is the setting of this work. 
% One of the first systems to perform human surface capture was proposed by Starck and Hilton \cite{starck2007surface}, which used a combination of visual hull and stereo reconstruction techniques. 
High-end multi-view capture systems can achieve reconstructions of outstanding quality \cite{leroy2018shape, dou2016fusion4d, guo2019relightables, collet2015high,vlasic2009dynamic,Joo_2017_TPAMI}, but require a complex studio setup that is expensive to build and not easily accessible.
% These settings are expensive to build and not readily accessible. 
To alleviate this, numerous works have been proposed that use instead a sparse set of RGB cameras (\eg between 2 and 15), 
%typically between 2 and 15 \va{I made this up}. 
%To compensate for the lack of views, classic works resorted to tracking a pre-scanned template~\cite{gall2009motion, vlasic2008articulated, carranza2003free, de2008performance} \todo{check all refs}%, relying on temporal information \cite{...} 
where the lack of views and presence of wide baselines is compensated by tracking a pre-scanned template \cite{gall2009motion, vlasic2008articulated, carranza2003free, de2008performance, wu2012full},
using a parametric body model  \cite{huang2017towards, balan2007detailed}, %joo2018total -> a lot of cameras
or more recently, by the use of deep learning \cite{huang2018deep,liang2019shape, wang2022arah, liu2021neural, peng2021neural,  kwon2021neural, peng2021animatable, xu2021h, weng2022humannerf}. 
% % MISSING, maybe:
% \cite{orts2016holoportation} -> 8 cameras, real-time. Uses depth cameras
% gall2009motion -> 4 to 8 cameras
% \cite{wu2012full} -> also sparse, recovers illumnation but in a passive way. tracks a template.
%
%
%
% To achieve better details, a recent trend has been to use deep implicit representations \cite{wang2022arah, peng2021neural,kwon2021neural}. Most of these methods require either a dense set of camera views, or a sparse set of RGB videos. To address the sparse view setting for a single frame, PiFU \cite{saito2019pifu} and follow-up work \cite{saito2020pifuhd} employ a set of pixel-aligned features pre-trained on a large dataset, which might not easily generalize to out-of-distribution scenes. 
% %% Not nerf but deep learning: \cite{liang2019shape}
% %% Nerf: \cite{kwon2021neural, liu2021neural, peng2021animatable, weng2022humannerf,xu2021h}
% %% PiFU-like works : "limited by the training data" | "line of works that condition on pixel-aligned features to recover an implicit representation". They require ground-truth data and thus cannot generalize to arbitrary scenes. Furthermore, they do not recover appearance and cannot do novel view synthesis. From here connect to pixel nerf, and why we do not compare. 
% %%
% \cite{saito2019pifu, saito2020pifuhd, alldieck2022photorealistic, he2020geo, huang2020arch, he2021arch}
%
%
% %%%%%%%%%%%
% %% "DoubleField: Bridging the Neural Surface and Radiance Fields for High-fidelity Human Reconstruction and Rendering" (CVPR'22)
% %% -> this is very relevant, we should've compared
% %%%%%%%%%%%
%
%
% ----
% \paragraph{Multi-view Multi-Human Reconstruction.} 
% %%% GOOD REVIEW IN Dynamic Multi-Person Mesh Recovery From Uncalibrated Multi-View Camera"
% None of these works were designed to handle the increased geometric complexity and occlusion of the multi-human case. 
% Compared with tremendous progress of single human reconstruction, there is a limited number of multiple human reconstruction methods.
\paragraph{Multi-Human Reconstruction.} 
In contrast, there has been a limited number of works that address the problem of \emph{multiple} human reconstruction. 
This is a difficult task since the presence of several people increases the geometric complexity of the scene, introduces occlusions,  and amplifies ambiguities such that commonly used features like color, edges, or key points cannot be correctly assigned. 

For single images and video, the problem has been mainly tackled by regressing the parameters of the SMPL \cite{loper2015smpl} body model~\cite{zhang2020perceiving, choi2022learning, sun2022putting, sun2021monocular, zanfir2018deep, dong2021shape, zanfir2018monocular, fieraru2020three, jiang2020coherent, zhang2021body, ugrinovic2021body, guler2019holopose}. Although this can work robustly with as little as one view, the reconstructions are very coarse and cannot explain hair, clothing, and fine geometric details. 
The only exception is the work of Mustafa \etal~\cite{mustafa2021multi}, which performs model-free reconstruction of multiple humans by combining an explicit voxel-based representation with an implicit function refinement. 
However, % I comment just becasue if not sure, we could save space , the output geometries are still rather coarse \va{maybe remove this, so are ours...}, and 
the method requires training on a large synthetic dataset of multiple people which hinders generalization. Our work, on the other hand, performs 3D reconstructions, produces renderings of novel views, and can generalize to arbitrary multi-human scenes.

Multi-view capture setups can help resolve depth ambiguities and some of the occlusions. 
Classic methods for estimating multiple humans rely heavily on segmentation masks and template mesh tracking \cite{liu2011markerless, liu2013markerless, wu2013set}. We avoid the use of segmentation masks by adopting volumetric rendering for implicit surfaces \cite{wang2021neus}. 
More recently, deep learning-based approaches were proposed, but they either require temporal information~\cite{zheng2021deepmulticap,huang2021dynamic,zhang2021lightweight,shuai2022novel}, pre-training on a large dataset~\cite{zheng2021deepmulticap} which cannot work on general scenes, or a coarse body model \cite{zhang2021lightweight, huang2021dynamic, shuai2022novel} which lacks geometric detail. 
% Furthermore, with the exception of XX, the focus here is on 3D shape estimation, ignoring the quality of rendered novel viewpoints \todo{I think this should go away}. 
Here, we focus on the multi-human setting on static scenes and propose a method that recovers accurate reconstructions and at the same time produces renderings of novel viewpoints. 
% by recovering an implicit representation from multi-view images.
%
%
%%% V: I removed the segmentation argument because we still need SMPL as input, so in practice you might need a segmentation mask just to get SMPL
%%% V: mustafa20224d is temporal and I think they don't do _simultaneous_ reconstruction of all humans, just segmentation. To be on the safe side I'm gonna ignore this paper for now, since it's anyway from 2022 and we could've missed it. 
%
%
%
% ----
% \todo{maybe start neural with this:}
% When the goal is to generate free-viewpoint video, image-based-rendering has also been considered \cite{wu2020multi}, which side-steps 3D reconstruction and produces novel views by reasoning on the available images. A recent succesful family of works, nerf an dimplicit, which are discussed next. 
% Another classic one in IBR (I think): "High-quality video view interpolation using a layered representation"
%
%
\paragraph{Neural Surface Reconstruction and Novel-View Synthesis.} 
For generating free-viewpoint video, image-based rendering has been considered as an alternative or complement to 3D reconstruction~\cite{carranza2003free,wu2020multi,liu2021neural,kwon2021neural, liu2021neural, peng2021animatable, weng2022humannerf,xu2021h}. 
%which side-steps 3D reconstruction and produces novel views by reasoning on the available images. A recent succesful family of works, nerf an dimplicit, which are discussed next. 
% Another classic one in IBR (I think): "High-quality video view interpolation using a layered representation"
% Impressive results have been obtained recently by with methods that combine 3D reconstruction and novel-view symthesis, using deep implicit representations. 
When geometry proxies are available, neural rendering \cite{aliev2020neural,thies2019deferred,jena2022neural} can produce competitive novel view synthesis. Recently, NeRF\cite{mildenhall2020nerf} demonstrated impressive rendering results by representing a 3D scene as a neural radiance field, trained only with calibrated multi-view images through the use of volume rendering. However, due to the unconstrained volumetric representation and self-supervised training on RGB values, reconstructed geometries tend to be too noisy to be useful for 3D applications. 
% Utilizing NeRF for human reconstruction could achieve a realistic appearance rendering, but hard to reconstruct accurate geometry due to NeRF purely reling on RGB values and lacking geometry constraints. 
To recover more accurate 3D geometry along with appearance, DVR~\cite{niemeyer2020differentiable}, IDR~\cite{yariv2020multiview}, and NLR~\cite{kellnhofer2021neural} propose to learn an implicit representation directly from multi-view images but require accurate object masks to work. To avoid the need for segmentation masks, recent works propose to combine implicit representations with volume rendering \cite{oechsle2021unisurf, yariv2021volume, wang2021neus}. 
%To avoid the use of mask segmentations, VolSDF \cite{yariv2021volume} proposed to combine an implicit SDF representation with volume rendering, transforming SDF values into volume densities by using the cumulative distribution function of the Laplace distribution. 
%%%by modeling the volume density as a function of geometry which is defined as Laplace’s cumulative distribution function (CDF).
% NeuS\cite{wang2021neus} uses a SDF to represent the surface and develop a new volume rendering method to train a neural SDF representation. Though those methods could reconstruct 3D geometry and appearance, it remains a challenge for them to reconstruct geometry consistent surface, especially for scenes containing complex geometry or sparse input views. \\
These methods show remarkable reconstruction results but struggle when the number of input views is low. 
Implicit neural representations from sparse input can be obtained by using pre-trained pixel-aligned features or 3D feature volumes for input images ~\cite{saito2019pifu, saito2020pifuhd, alldieck2022photorealistic, he2020geo, huang2020arch, he2021arch, yu2021pixelnerf, li2023learning, jena2024geotransfer} or point clouds ~\citep{boulch2022poco,williams2022neural,huang2023neural,peng2020convolutional,chibane2020implicit,lionar2021dynamic, ouasfi2023mixing, peng2021shape, ouasfi2022few, ouasfi2024robustifying}, 
but this requires ground-truth geometry and is limited by the training data, struggling to generalize to new scenes. Sparse variants that do not require generalizable features were proposed in the image input \eg \cite{niemeyer2022regnerf, kim2022infonerf, long2022sparseneus, younes2025sparsecraft, li2023regularizing} and point cloud input case \eg \cite{NeuralTPS,sparseocc,nap,ouasfi2024robustneuralreconstructionsparse,williams2021neural}. 
InfoNeRF~\cite{kim2022infonerf} regularizes sparse views by adding an entropy constraint on the density of the rays,  RegNeRF~\cite{niemeyer2022regnerf} uses a patch-based regularizer over generated depth maps, and SparseNeuS~\cite{long2022sparseneus} uses a multi-scale approach along with learned features that are fine-tuned on each scene. 
%% DietNeRF: since it needs CLIP (=> "pretraining") I don't know where to place it, and I'm ignoring it for now.
%%% PiFU-like works : "limited by the training data" | "line of works that condition on pixel-aligned features to recover an implicit representation". They require ground-truth data and thus cannot generalize to arbitrary scenes. Furthermore, they do not recover appearance and cannot do novel view synthesis. From here connect to pixel nerf, and why we do not compare. 
%%%
% \cite{saito2019pifu, saito2020pifuhd, alldieck2022photorealistic, he2020geo, huang2020arch, he2021arch}
Our approach builds on NeuS \cite{wang2021neus}, and tackles the sparse view challenge by adding human-specific geometric priors and novel regularizations. 




%%%% This is more for the introduction, not the related work. 
%Specifically, we use multiple human SMPL to train a neural SDF presentation as a geometry prior to optimize the learning. Moreover, %inspired by \cite{deng2022depth,roessle2022dense}, 
%We propose to learn SDF along with uncertainty and propose a explicit geometry constrains to ensure the geometry consistency while encouraging learning details around surfaces. Furthermore, noticing that the reconstruction inconsistency might happen due to insufficient viewpoints or invariant illuminations, we propose a patch-based regularization to ensure consistency across different rays and saturation regularization for illuminations across different views. \todo{pifu line}


% Sparse nerf: regnerf\cite{niemeyer2022regnerf}, infonerf\cite{kim2022infonerf}
% Sparse Neus:\cite{long2022sparseneus} %ECCV 2022

% \paragraph{Concurrent Works.} 
% \va{I don't think this should go here, it's risky. I suggest we add this only if reviewers ask}
% Monosdf\cite{yu2022monosdf}, GeoNeus\cite{fu2022geo}, Neuris\cite{wang2022neuris} VOXURF\cite{wu2022voxurf} % 4 nips 2022

% -------------------------

% \textbf{Human reconstruction.} 
% % single human, multihuman, yiyi
% Previous work related to human reconstruction could be classified into single-human reconstruction and multiple-human reconstructions according to the human numbers in the scene. Single human reconstruction from images or video includes traditional methods, such as depth sensors \cite{theobalt2005image, dou2016fusion4d}, depth array of cameras\cite{debevec2000acquiring, guo2019relightables}, rendering pipeline optimization \cite{wu2020multi}, template-based methods\cite{gall2009motion, carranza2003free, de2008performance}, and learning based methods\cite{burov2021dynamic,loper2015smpl,loper2015smpl,bhatnagar2020combining,  liu2021neural,yu2017bodyfusion,yu2018doublefusion,peng2021neural}. Compared with tremendous progress of single human reconstruction, there is a limited number of multiple human reconstruction methods. Among those, \cite{zanfir2018monocular,zanfir2018deep} are earliest works to utilize human SMPL for multiple human reconstructions, \cite{mustafa2021multi} exploits 3D pose to learn body part scores, localize multiple persons simultaneously and estimate 3d pose and shape
% \cite{mustafa20224d} combines human model with implicit reconstruction for multiple human shape reconstruction. Those works usually rely on segmentation and focus on 3D shape estimation, which cost expensive in computing and ignores the quality of rendered novel viewpoints. In this paper, we focus on multiple human reconstructions on static scenes, learning multiple human geometry and appearances with loose clothing and hair details simultaneously and enabling editing during rendering without the extra requirement of segmentation.

% \textbf{Neural surface reconstruction.} 
% Recently, NeRF\cite{mildenhall2020nerf} demonstrated impressive rendering results by representing a 3D scene as a neural radiance field, with only calibrated color images, camera poses and intrinsic as input. Utilizing NeRF for human reconstruction could achieve a realistic appearance rendering, but hard to reconstruct accurate geometry due to NeRF purely reling
% on RGB values and lacking geometry constraints. In order to reconstruct accurate 3D geometry along with appearance, Volsdf \cite{yariv2021volume} propose to improve geometry representation and reconstruction in neural volume rendering by defining the volume density function as Laplace’s cumulative distribution function(CDF) and modeling it as a function of geometry. NeuS\cite{wang2021neus} uses a signed distance function(SDF) to represent the surface and develop a new volume rendering method to train a neural SDF representation. Though those methods could reconstruct 3D geometry and appearance, it remains a challenge for them to reconstruct geometry consistent surface, especially for scenes containing complex geometry or sparse input views. \\
% Our approach take advantage of existing implicit surface reconstruction (\eg NeuS), and tackle its challenges by adding learned geometry priors and explicit geometry constraints. Specifically, we use multiple human SMPL to train a neural SDF presentation as a geometry prior to optimize the learning. Moreover, %inspired by \cite{deng2022depth,roessle2022dense}, 
% we propose to learn SDF along with uncertainty and propose a explicit geometry constrains to ensure the geometry consistency while encouraging learning details around surfaces. Furthermore, noticing that the reconstruction inconsistency might happen due to insufficient viewpoints or invariant illuminations, we propose a patch-based regularization to ensure consistency across different rays and saturation regularization for illuminations across different views. \todo{pifu line}

% Sparse nerf: regnerf\cite{niemeyer2022regnerf}, infonerf\cite{kim2022infonerf}
% Sparse Neus:\cite{long2022sparseneus} %ECCV 2022
% \textbf{Concurrent Works.} 
% Monosdf\cite{yu2022monosdf}, GeoNeus\cite{fu2022geo}, Neuris\cite{wang2022neuris} VOXURF\cite{wu2022voxurf} % 4 nips 2022




%%%%%%%%%%%%%%%%%%%%%%%%%%%%%%%%%%%%%%%

%%---  Multi-human reconstruction from single image: ---
% Note first that there are numerous works on multi-human 3D pose estimation, but we focus on those that recover 3D meshes (typically shape and pose parameters of a SMPL model) since this is closer to our task.

% \cite{choi2022learning}
% \cite{sun2022putting, sun2021monocular}
% \cite{zanfir2018deep} -> predict 3D pose only and defer the shape reconstruction step to a later optimization step
% \cite{zanfir2018monocular}
% \cite{dong2021shape} -> the final goal is pose, but they use SMPL shape to help with the problem
% \cite{fieraru2020three}
% \cite{jiang2020coherent}
% \cite{zhang2021body}
% \cite{ugrinovic2021body} -> addresses body scale and depth ambiguity


%% --- Multi-human reconstruction from multiple views ---

% + Classic: they track a pre-scanned template of an actor:
% \cite{liu2011markerless} -> only for two people
% \cite{wu2013set}
% This one doesn't, but the input is video:
% \cite{li2018shape}
% The multi-person case introduces ambiguities "where commonly-used features such as color, edges or keypoints cannot be individually assigned

% + Deep learning:
% \cite{huang2021dynamic} -> uses motion model, SMPL. Uncalibrated cameras, video.
% \cite{zhang2021lightweight} -> they say they are the first light-weight and robust that uses only sparse cameras (6 cameras in the experiments). They are also total capture: hand face and body - but still cannot handle clothes and hair.
% icnludes hands and faces but it's still a coarse mesh from a parametric model. 
% \cite{zheng2021deepmulticap}  -> implicit, but video

% Note how many rely on motion, but here we assume only a single view


%% --- Multi-human novel view synthesis (2D) ---
% \cite{lakhal2019view}
% \cite{zhang2021editable} -> Closely related, they do nerf on multiple humans with a layered representation. The input is a video. 

%% -- Single human, MV neural:
% \cite{liu2021neural}
% \cite{weng2022humannerf} -> Humannerf, from video
% \cite{shao2022doublefield}
% \cite{shao2022diffustereo} -> follow up, they build on doublefield


%% Multi-view, RGB-D
% \cite{yu2021function4d}


% MV 3D pose, just in case we ever use this as input:
% \cite{zhang2021direct}




%% MV story:
% - there are high-end MV capture systems requiring a large number of cameras (volumetric)
% \cite{leroy2018shape, collet2015high, orts2016holoportation, joo2015panoptic}
% and more light-weight settings was achieved using a pre-scanned template
% \cite{de2008performance}
% - recently with deep learning, more possible:
% \cite{huang2018deep} -> volumetric representation
% Pifu, pifu hd shows MV results also. This line of work requires pre-training on a large dataset. 
% \cite{saito2019pifu, saito2020pifuhd}
% These are all for single humans
% For multiple humans,





\subsection{Background}
In this paper, we mainly focus on the multi-constraint instruction $I_c$. It can be formulated as a seed instruction incorporated with ${n}$ constraints:
\begin{equation}
\label{eq1}
    I_c = I_s \oplus C_1 \oplus ... \oplus C_n,
\end{equation}
where the seed instructions $I_s$ describe a task, e.g., write a story, while these constraints $\sum_{i=1}^n C_i$ limit the output from different aspects, e.g., format, length, content, etc. $\oplus$ stands for the concatenation operation. 
% \footnote{We provide more examples of multi-constraint instruction in Appx.~\ref{appx:con_samp}.} 
% seed + 1 2 3 4


\subsection{Probing Task} \label{method}
% 我们尝试从约束的难度角度出发 量化指令中的不同排布情况 to achieve this, two problems need to be soluted: how to quantify the 
% 我们参考约束的分类建模
\subsubsection{Task Formulation}
To investigate the impact of constraint order, we introduce a probing task. In this task, the LLM is given multi-constraint instructions with constraints arranged in various orders. The LLM's task is to generate a response that follows all constraints. We evaluate the LLM in two practical scenarios: single-round and multi-round inference. The LLM's responses are then evaluated to determine its performance across various constraints. The overall procedure is illustrated in Fig.~\ref{fig:method}. In the following sections, we will provide a detailed explanation.





\subsubsection{Multi-constraint Instruction Synthesis}\label{sec:ins_cons}
To ensure the generalizability of probing data, we construct the initial multi-constraint instructions which include a variety of tasks and diverse constraint combinations. The multi-constraint instruction synthesis can be further divided into two parts: seed sampling and constraint sampling. 

For the seed sampling, we sample data from three source datasets: (1) Natural Instructions V2~\cite{wang2022supernaturalinstructions}. It is an instruction collection covering more than 1600 NLP tasks. We filter those tasks that are too easy and could potentially conflict with complex constraints, e.g., object classification and sentiment tagging. Then, we randomly sample 52 instructions from the remaining tasks. (2) Self-Instruct~\cite{wang2023self}. We only sample 83 instances from their initial 175 seed instructions which are formulated by humans. (3) Open Assistant~\cite{kopf2024openassistant}. Following the strategy of Suri~\cite{li2023self}, we filter out the first turn of the conversation with the highest quality (marked as rank 0 in the dataset) and sample 65 instances from them. Overall, we obtain 200 seed instructions, where the number of instructions is denoted as $n_{seed}$.

As for the constraint sampling, we first categorize the constraints into 8 groups with 25 fine-grained types~\cite{zhou2023instructionfollowing}. For each type of constraint, we employ 8 different expressions to describe it\footnote{More details are shown in Appx.~\ref{appx:cons_tax}}. Then, we sample $n$ constraints from the constraint taxonomy and use the predefined rules to avoid possible conflicts. To ensure diversity, we repeat the sampling process to obtain $n_{cc}$ distinct constraint combinations, deriving $n_{seed}\times n_{cc}$ multi-constraint instructions.






\subsubsection{Constraint Reordering} \label{reorder}
To quantitatively construct instructions with different constraint orders, here are two questions that need to be answered: (1) \textit{How do we distinguish the disparity of different constraints}? (2) After we order the constraints based on their disparity, \textit{how do we quantitatively describe the disparity of constraint orders}?

An appropriate solution for the first question is to categorize the constraints based on their difficulty~\cite{chen2024sifo}. In this paper, we also sort the constraints based on their difficulty. However, different from existing works which designate the difficulty of the constraints based on handcraft rules, we measure the difficulty of a constraint via the overall accuracy of following it in our probing datasets. The formulation is as follows:
\begin{equation}
\label{eq2}
    % \small
    \text{Dff}_{C_x}= \text{Softmax}(1-\text{Acc}_{C_{x}}), 
\end{equation}
\begin{equation}
    \label{eq3}
    \text{Acc}_{C_x} = \frac{1}{N_{x}}\sum_{i=1}^{N_{x}}c_x^i.
\end{equation}
The $C_x$ refers to a specific type of constraint, the $N_{x}$ stands for the total number of instructions corresponding to the constraint $C_{x}$, and the $c_x^i$ is a binary value to reflect whether the constraint $C_{x}$ is followed in the $i^{th}$ instruction. 

To quantitatively describe the disparity of constraint order, we propose a novel metric called the Constraint Difficulty Distribution Index (CDDI) which quantifies a specific constraint order based on its difficulty distribution. Given the difficulty of different types of constraints, we can readily attain the difficulty distribution of the constraints incorporated in the multi-constraint instructions. Specifically, for a multi-constraint instruction, we rank the incorporated constraints based on their difficulty, from the hardest to the easiest. We set this “hard-to-easy” constraint order as an anchor since it depicts an extreme situation, i.e., we designate the $\text{CDDI}=1$ when the constraints fall in this order. Consequently, akin to the Kendall tau distance~\cite{cicirello2020kendall}, we measure the difficulty distribution of a specific constraint order $o$ by comparing it with the ``hard-to-easy'' constraint order $o_{max}$. The formula is shown as:
\begin{equation}
    \label{eq4}
    \text{CDDI}_{o} = \frac{N_{con}-N_{dis}}{N_{pair}} = \frac{2(N_{con}-N_{dis})}{n(n-1)}.
\end{equation}
where $N_{con}$ and $N_{dis}$ represent the number of concordant and discordant distribution pairs of constraints between $o$ and $o_{max}$, respectively. The $N_{pair}$ is the total number of compared constraint pairs. Overall, we select $n_{dd}$ different difficulty distributions, finally comprising $n_{seed}\times n_{cc}\times n_{dd}$ instances.







\subsubsection{Sequential-Sensitive Inference}
Given the multi-constraint instructions with different constraint orders, we evaluate the model's performance in two common scenarios: single-round inference and multi-round inference. In single-round inference, the LLM is directly given the multi-constraint instructions with different constraint distributions. We argue that different constraint distributions could impose different levels of difficulty on the LLM to handle. The multi-round inference introduces a more typical setting: the user will first provide the LLM with the core intention (i.e., the seed instruction in this work), and then iteratively put forward the constraints in order to obtain a final response.

To evaluate the model performance, apart from the constraint following accuracy mentioned in Eq.(\ref{eq3}), we also verify its constraint-level accuracy $Acc_{cons}$ and instruction-level accuracy $Acc_{inst}$. Corresponding formulas are shown below:
\begin{equation}
    \label{eq5}
    \small
    \text{Acc}_{\text{cons}} = \frac{1}{mn}\sum_{i=1}^{m}\sum_{j=1}^{n}c_i^j,     \text{Acc}_{\text{inst}} = \frac{1}{m}\sum_{i=1}^{m}\prod_{j=1}^{n}c_i^j.
\end{equation}
% \begin{equation}
%     \label{eq5_1}
%     \text{Acc}_{\text{inst}} = \frac{1}{m}\sum_{i=1}^{m}\prod_{j=1}^{n}c_i^j.
% \end{equation}
where $m$ and $n$ refer to the number of instructions and constraints in the instruction, respectively. Similar to Eq.(\ref{eq3}), the $c_i^j$ is a binary value which equals 1 when the constraint is followed in the $i^{th}$ instruction. All the evaluation is conducted by leveraging the script introduced in ~\cite{zhou2023instructionfollowing}. We only evaluate the final responses produced by the LLMs.


\begin{figure}[t] 
    \centering
        \includegraphics[width=0.5\textwidth]{statistic.pdf}
    % \captionsetup{font={small}} 
    \caption{The statistic of different types of constraints in the probing data. The 7cons and 9cons stand for the setting when $n$=7 and $n$=9, respectively.}
    \label{fig:statistic}
\end{figure}
% , comparing our results against state-of-the-art image-to-image translation methods
% We evaluate our method through editing experiments conducted on two experiments. In \cref{sec:5.1}, we perform a comparison on image-to-image editing across several datasets. In \cref{sec:5.2}, we extend our evaluation to editable Neural Radiance Fields (NeRF) \cite{mildenhall2021nerf}, demonstrating the efficacy of our approach for 3D image editing and providing a comparative analysis with existing techniques.
% result tables

\section{Results} \label{sec:results}
We evaluate our method through editing experiments conducted on two experiments. In \cref{sec:5.1}, we perform a comparison on image-to-image editing across several datasets. In \cref{sec:5.2}, we extend our evaluation to editable Neural Radiance Fields (NeRF) \cite{mildenhall2021nerf}.

\subsection{Text-guided image editing}
\label{sec:5.1}
\noindent\textbf{Baselines.} To evaluate our method, we conduct comparative experiments against four state-of-the-art image editing models: Prompt-to-Prompt (P2P) \cite{hertzprompt}, Plug-and-Play (PNP) \cite{tumanyan2023plug}, DDS \cite{hertz2023delta}, and CDS \cite{nam2024contrastive}. The implementations of the baselines are carried out by referencing the official source code for each method. More details are provided in \cref{sec:s_implement} of Supplementary Materials.

\noindent\textbf{Qualitative Results.} We present the qualitative results comparing our method with the baselines in \cref{fig:ip2p_qual}. Prompt-to-Prompt (P2P) \cite{hertzprompt} performs image editing after applying DDIM inversion \cite{dhariwal2021diffusion, song2020denoising} to the source image, leading to disregarding the structural components of the source image and following the target prompt excessively. Plug-and-Play (PnP) \cite{tumanyan2023plug} has limitations in object recognition, as seen in the fourth row of Fig.~\ref{fig:ip2p_qual}. The third row of Fig.~\ref{fig:ip2p_qual} demonstrates that DDS \cite{hertz2023delta} and CDS \cite{nam2024contrastive} exhibited limitations, particularly in preserving the structural characteristics of the source image. In contrast, our method successfully edits the image while preserving the structural integrity of the source image.
% exhibit limitations such as failing to maintain the handle length and saddle shape of the bike in the first row and being unable to preserve the structure of the shark in the second row. %Furthermore, as seen in the third and fourth rows, the details in the edited target areas lacked refinement, and in the last row, the color of the source image was not preserved. In contrast, our method successfully edits the image aligning with the target text prompt while preserving the structural integrity of the source image.

\noindent\textbf{Quantitative Results.} 
% We employed two datasets: LAION 5B \cite{schuhmann2022laion} and InstructPix2Pix \cite{brooks2023instructpix2pix}.
% ##ORIGINAL## To measure the identity-preserving performance, we utilize two datasets. First, we collect 250 cat images from the LAION 5B dataset \cite{schuhmann2022laion} based on \cite{nam2024contrastive} for \textit{Cat-to-Others} task. We measure Intersection over Union (IoU) to evaluate how much of the area of the source object has been preserved. Second, we gather 28 images from the InstructPix2Pix (IP2P) dataset \cite{brooks2023instructpix2pix}, which contains the pairs of source and target images and corresponding prompts. We calculate the background Peak-Signal-to-Noise-Ratio (PSNR) to assess how the identity of the source image is preserved after editing. In addition, we use the LPIPS score \cite{zhang2018unreasonable} for each experiment to quantify the similarity between source and target images. The results are presented in \cref{tab:2Dquan}. Our method consistently achieves the lowest LPIPS score across all datasets, indicating that it best preserves the structural semantics of the source images. 
To measure the identity-preserving performance, we utilize two datasets. First, we collect 250 cat images from the LAION 5B dataset \cite{schuhmann2022laion} based on \cite{nam2024contrastive} for \textit{Cat-to-Others} task and measure Intersection over Union (IoU). Second, we gather 28 images from the InstructPix2Pix (IP2P) dataset \cite{brooks2023instructpix2pix}, which contains the pairs of source and target images and corresponding prompts and calculate the background Peak-Signal-to-Noise-Ratio (PSNR). Details of the metrics are provided in Supplementary Materials \cref{sec:s_evalmetric}. In addition, we use the LPIPS score \cite{zhang2018unreasonable} for each experiment to quantify the similarity between source and target images. The results are presented in \cref{tab:2Dquan}. Our method consistently achieves the lowest LPIPS score across all datasets, indicating that it best preserves the structural semantics of the source images. 
% We collect 250 images of cats from the LAION 5B dataset \cite{schuhmann2022laion} based on \cite{nam2024contrastive} for \textit{Cat-to-Others} task and 28 images from the InstructPix2Pix dataset \cite{brooks2023instructpix2pix} following the regulations. To evaluate the images translated by each method, we measure Intersection over Union (IoU) on LAION 5B, which primarily consists of object-focused data. We also measure the background PSNR on InstructPix2Pix to assess the extent to which the source image’s identity is preserved after editing. The results are presented in \cref{tab:2Dquan}. 
% Our method consistently achieves the lowest LPIPS score across all datasets, indicating that it best preserves the structural semantics of the source images. 
\begin{table}[b]
\centering
\resizebox{0.98\columnwidth}{!}{
\small{
\begin{tabular}{c|cc|cc|cc}
\hline
& \multicolumn{2}{c|}{cat2pig} & \multicolumn{2}{c|}{cat2squirrel} & \multicolumn{2}{c}{Ip2p}  \\ 
\hline
\multicolumn{1}{c|}{Metric} & IoU ($\uparrow$) & LPIPS ($\downarrow$) & IoU ($\uparrow$) & LPIPS ($\downarrow$) & PSNR ($\uparrow$) & LPIPS ($\downarrow$) \\ 
\hline
P2P \cite{hertzprompt}& 0.58 & 0.42 & 0.52 & 0.46 & 20.88 & 0.47 \\
PnP \cite{tumanyan2023plug}& 0.55 & 0.52 & 0.53 & 0.52 & 23.81 & 0.39 \\
DDS \cite{hertz2023delta}& 0.69 & 0.28 & 0.65 & 0.30 & 26.02 & 0.24 \\  
CDS \cite{nam2024contrastive}& 0.72 & 0.25 & \textbf{0.71} & 0.26 & 27.35 & 0.21 \\
\hline
\textbf{IDS (Ours)} & \textbf{0.74} & \textbf{0.22} & \textbf{0.71} & \textbf{0.24} & \textbf{29.25} & \textbf{0.19} \\
\hline
\end{tabular}
}
}
\vspace{-5pt}
\caption{\textbf{Quantitative results} for image editing. LPIPS \cite{zhang2018unreasonable} and IoU was measured on LAION 5B \cite{schuhmann2022laion}, while LPIPS and background PSNR was measured on InstructPix2Pix \cite{brooks2023instructpix2pix}.}
\label{tab:2Dquan}
\end{table}




%P2P \cite{hertzprompt}& 0.5798 & 0.4229 & 0.5184 & 0.4605 & 20.88 & 0.4695 \\
%PnP \cite{tumanyan2023plug}& 0.5507 & 0.5191 & ??? & 0.5245 & 23.81 & 0.3882 \\
%DDS \cite{hertz2023delta}& 0.6897 & 0.2838 & 0.6456 & 0.2996 & 26.02 & 0.2398 \\  
%CDS \cite{nam2024contrastive}& 0.7249 & 0.2485 & 0.7054 & 0.2612 & 27.35 & 0.2099 \\

\begin{table}[bh!]
\vspace{-5pt}
\centering
%\scalebox{0.65}
\resizebox{1.0\columnwidth}{!}{
%\small{ %
\begin{tabular}{c|ccc|ccc}
\hline
& \multicolumn{3}{c|}{User Preference Rate (\%)} & \multicolumn{3}{c}{GPT score \cite{peng2024dreambench++}}\\ 
\hline
\multicolumn{1}{c|}{Metric} & Text ($\uparrow$) & Preserving ($\uparrow$) & Quality ($\uparrow$) & Text ($\uparrow$) & Preserving ($\uparrow$) & Quality ($\uparrow$) \\ 
\hline
P2P \cite{hertzprompt}& 11.13 & 4.80 & 8.09 & 5.66 & 5.37 & 5.77 \\
PnP \cite{tumanyan2023plug}& 7.72 & 7.17 & 6.93 & 6.54 & 6.77 & 6.74 \\
DDS \cite{hertz2023delta}& 20.30 & 10.82 & 16.23 & 7.60 & 7.51 & 7.37 \\
CDS \cite{nam2024contrastive}& 17.02 & 16.72 & 17.08 & 8.26 & 8.00 & 8.09 \\ 
\hline
\textbf{IDS (Ours)} & \textbf{43.83} & \textbf{60.49} & \textbf{51.67} & \textbf{8.97} & \textbf{9.00} & \textbf{8.80} \\
\hline
\end{tabular}
}
%}
\vspace{-5pt}
\caption{\textbf{User study and GPT scores}  \cite{peng2024dreambench++} show that our method achieved the highest scores across all questions for image editing.}
\label{tab:Userstudy_GPTscore}
\end{table}
For user evaluation, we present 35 comparison sets for four baselines and our method, gathering responses from 47 participants. Participants are asked to choose the most appropriate image for the following three questions: 1. \textit{Which image best fits the text condition?} 2. \textit{Which image best preserves the structural information of the original image?} 3. \textit{Which image has the best quality for text-based image editing?} 
Additionally, we measure the GPT score using the Dreambench++ \cite{peng2024dreambench++} method, which generates human-aligned assessments for the same questions by refining the scoring into ten distinct levels. As shown in \cref{tab:Userstudy_GPTscore}, our method receives the highest ratings for all questions.
% Furthermore, we ask users to select their favorite image from the baselines in order to gauge their preferences, and we compute the selected ratio in percentage terms.
%While our CLIP score was not significantly higher than other methods, it remained comparable. %Considering the outcomes of both metrics, our model demonstrates an ability to maximally preserve the source image's structure during the editing process while minimally and precisely transforming the regions specified by the target prompt.

% Fig 5.2



%%% [START] NeRF Synthetic data Results 
\begin{figure*}[t] % 2-column
\footnotesize
\centering 
% 1st row
\hspace{-3mm}
\raisebox{0.5in}{\rotatebox{90}{\textbf{Synthetic} \cite{mildenhall2021nerf}}}%
\hspace{3mm}%
\begin{tikzpicture}[x=3.5cm, y=3.5cm, spy using outlines={every spy on node/.append style={thick, draw=red}}]
\node[anchor=south] (FigA) at (0,0) {\includegraphics[trim=0 0 0 0 ,clip,width=1.5in]{Fig./Qual/imgs/3D/ficus/cropped_r_3.png}};
\node[anchor=south, yshift=0mm] at (FigA.north) {\footnotesize Source};
% ->
\draw[->, line width=0.8mm, color=red, shorten >=1pt, shorten <=1pt] ($(FigA.center) + (0.15, -0.18)$) -- ($(FigA.center) + (0, -0.3)$);
\end{tikzpicture}
\hspace{-1mm}
\begin{tikzpicture}[x=3.5cm, y=3.5cm, spy using outlines={every spy on node/.append style={thick, draw=red}}]
\node[anchor=south] (FigD) at (0,0) {\includegraphics[trim=0 0 0 0 ,clip,width=1.5in]{Fig./Qual/imgs/3D/ficus/FPDS_cropped_r_3.png}};
\node[anchor=south, yshift=0mm] at (FigD.north) {\footnotesize \textbf{IDS (Ours)}};
% ->
\draw[->, line width=0.8mm, color=red, shorten >=1pt, shorten <=1pt] ($(FigA.center) + (0.15, -0.18)$) -- ($(FigA.center) + (0, -0.3)$);
\end{tikzpicture}
\hspace{-1mm}
\begin{tikzpicture}[x=3.5cm, y=3.5cm, spy using outlines={every spy on node/.append style={thick, draw=red}}]
\node[anchor=south] (FigC) at (0,0) {\includegraphics[trim=0 0 0 0 ,clip,width=1.5in]{Fig./Qual/imgs/3D/ficus/CDS_cropped_r_3.png}};
\node[anchor=south, yshift=0mm] at (FigC.north) {\footnotesize CDS};
% ->
\draw[->, line width=0.8mm, color=red, shorten >=1pt, shorten <=1pt] ($(FigA.center) + (0.15, -0.18)$) -- ($(FigA.center) + (0, -0.3)$);
\end{tikzpicture}
\hspace{-1mm}
\begin{tikzpicture}[x=3.5cm, y=3.5cm, spy using outlines={every spy on node/.append style={thick, draw=red}}]
\node[anchor=south] (FigB) at (0,0) {\includegraphics[trim=0 0 0 0 ,clip,width=1.5in]{Fig./Qual/imgs/3D/ficus/DDS_cropped_r_3.png}};
\node[anchor=south, yshift=0mm] at (FigB.north) {\footnotesize DDS};
% ->
\draw[->, line width=0.8mm, color=red, shorten >=1pt, shorten <=1pt] ($(FigA.center) + (0.15, -0.18)$) -- ($(FigA.center) + (0, -0.3)$);
\end{tikzpicture}

\vspace{-4pt}

\setulcolor{magenta}
\setul{0.3pt}{2pt}
\centering \textit{``A tree in a brown vase" $\to$ ``A tree in a \ul{blue} vase"} 

\vspace{-2pt}

% 2nd row
\hspace{-3mm}
\raisebox{0.37in}{\rotatebox{90}{\textbf{LLFF} \cite{mildenhall2019local} }}%
\hspace{3mm}%
\begin{tikzpicture}[x=3.5cm, y=3.5cm, spy using outlines={every spy on node/.append style={thick, draw=white}}]
\node[anchor=south] (FigA2) at (0,0) {\includegraphics[trim=0 0 0 0 ,clip,width=1.5in]{Fig./Qual/imgs/3D/autumn/original_image009.jpg}};
\spy [magnification=3, size=0.6in] on ($(FigA2.center) + (0.05, 0.05)$) in node [anchor=south west] at ($(FigA2.south west)$);
\end{tikzpicture}
\hspace{-1mm}
\begin{tikzpicture}[x=3.5cm, y=3.5cm, spy using outlines={every spy on node/.append style={thick, draw=white}}]
\node[anchor=south] (FigD2) at (0,0) {\includegraphics[trim=0 0 0 0 ,clip,width=1.5in]{Fig./Qual/imgs/3D/autumn/FPDS_4032_IMG_3006.jpg}};
\spy [magnification=3, size=0.6in] on ($(FigD2.center) + (0.05, 0.05)$) in node [anchor=south west] at ($(FigD2.south west)$);
\end{tikzpicture}
\hspace{-1mm}
\begin{tikzpicture}[x=3.5cm, y=3.5cm, spy using outlines={every spy on node/.append style={thick, draw=white}}]
\node[anchor=south] (FigC2) at (0,0) {\includegraphics[trim=0 0 0 0 ,clip,width=1.5in]{Fig./Qual/imgs/3D/autumn/CDS_4032_IMG_3006.jpg}};
\spy [magnification=3, size=0.6in] on ($(FigC2.center) + (0.05, 0.05)$) in node [anchor=south west] at ($(FigC2.south west)$);
\end{tikzpicture}
\hspace{-1mm}
\begin{tikzpicture}[x=3.5cm, y=3.5cm, spy using outlines={every spy on node/.append style={thick, draw=white}}]
\node[anchor=south] (FigB2) at (0,0) {\includegraphics[trim=0 0 0 0 ,clip,width=1.5in]{Fig./Qual/imgs/3D/autumn/DDS_4032_IMG_3006.jpg}};
\spy [magnification=3, size=0.6in] on ($(FigB2.center) + (0.05, 0.05)$) in node [anchor=south west] at ($(FigB2.south west)$);
\end{tikzpicture}

% 3rd row
\hspace{-3mm}
\raisebox{0.3in}{\rotatebox{90}{\textbf{Depth Map}}}%
\hspace{3mm}%
\hspace{0mm}
\begin{tikzpicture}[x=3.5cm, y=3.5cm, spy using outlines={every spy on node/.append style={thick, draw=white}}]
\node[anchor=south] (FigA3) at (0,0) {\includegraphics[trim=0 0 0 0 ,clip,width=1.5in]{Fig./Qual/imgs/3D/autumn/depth_map/original_depth_088.jpg}};
\end{tikzpicture}
\hspace{-1mm}
\begin{tikzpicture}[x=3.5cm, y=3.5cm, spy using outlines={every spy on node/.append style={thick, draw=white}}]
\node[anchor=south] (FigD3) at (0,0) {\includegraphics[trim=0 0 0 0 ,clip,width=1.5in]{Fig./Qual/imgs/3D/autumn/depth_map/FPDS_depth_088.jpg}};
\end{tikzpicture}
\hspace{-1mm}
\begin{tikzpicture}[x=3.5cm, y=3.5cm, spy using outlines={every spy on node/.append style={thick, draw=white}}]
\node[anchor=south] (FigC3) at (0,0) {\includegraphics[trim=0 0 0 0 ,clip,width=1.5in]{Fig./Qual/imgs/3D/autumn/depth_map/CDS_depth_088.jpg}};
\end{tikzpicture}
\hspace{-1mm}
\begin{tikzpicture}[x=3.5cm, y=3.5cm, spy using outlines={every spy on node/.append style={thick, draw=white}}]
\node[anchor=south] (FigB3) at (0,0) {\includegraphics[trim=0 0 0 0 ,clip,width=1.5in]{Fig./Qual/imgs/3D/autumn/depth_map/DDS_depth_088.jpg}};
\end{tikzpicture}

\vspace{-1pt}
\centering \textit{``The green leaves" $\to$ ``\ul{Yellow and red} leaves in \ul{autumn}"} 

\vspace{-5pt}
\caption{\textbf{Qualitative results on Synthetic 360$^\circ$ and LLFF datasets.} IDS outperforms the baselines by preserving the structural consistency of the source image and maintaining the integrity of regions that should remain unchanged, while precisely editing only the areas specified by the target prompt. Furthermore, comparisons of the depth map results also highlight the superior consistency of our method over other baseline models.}
\label{fig:ficus_qual}
\end{figure*}
% \vspace{-10pt}
\subsection{Editing NeRF}
We conduct experiments involving 3D rendering of edited images to demonstrate the effectiveness of our method in maintaining structural consistency. This approach is particularly relevant as consistency has an even greater impact on outcomes in 3D environments.

\label{sec:5.2}

\noindent\textbf{Datasets.} We evaluated our method on widely used NeRF datasets: Synthetic NeRF \cite{mildenhall2021nerf} and LLFF \cite{mildenhall2019local}. Since NeRF datasets have no given pairs of source and target prompts, we manually composed image descriptions.
%, such as the source prompt ``A tree in a brown vase" and its corresponding target prompt ``A tree in a blue vase" as shown in \cref{fig:ficus_qual}.

\noindent\textbf{Qualitative Results.} \cref{fig:ficus_qual} illustrates the qualitative results of our method compared with NeRF editing baselines. In the first row, the target prompt specifies a precise part of the image for fine-grained editing. DDS \cite{hertz2023delta} and CDS \cite{nam2024contrastive} fail to differentiate and edit the specific area. At the same time, our method accurately identifies the region indicated by the target prompt in the image and performs detailed editing exclusively on that part. 
The second row demonstrates a scenario in which the target prompt is designed to edit the mood of the image. Our approach adjusts the colors associated with ``autumn" and ``leaves" throughout the image while maintaining consistency in the ``trunk" whereas DDS and CDS also changed the ``trunk". In terms of depth maps, our method generates clean depth maps with minimal noise after image editing, whereas DDS and CDS introduce noticeable noise into the depth maps.

%the overall mood of the image on the LLFF dataset \cite{mildenhall2019local}
 % give an attention solely on following the target prompt during editing, leading to unintended alterations of parts that should remain unchanged.
 % Comparing the NeRF depth maps with baselines, 
% \cref{fig:ficus_qual} illustrates the qualitative results of our method compared with NeRF editing baselines such as DDS \cite{hertz2023delta} and CDS \cite{nam2024contrastive}. In the first row, the target prompt specifies a precise part of the image for fine-grained editing on the Synthetic NeRF dataset \cite{mildenhall2021nerf}. Our method accurately identifies the region indicated by the target prompt in the image and performs detailed editing exclusively on that part. In contrast, DDS and CDS fail to differentiate and edit the specific area; they erroneously edit not only the ``vase" but also the ``soil", resulting in inappropriate edits. The second row demonstrates a scenario in which the target prompt is designed to edit the overall mood of the image on the LLFF dataset \cite{mildenhall2019local}, further highlighting the strengths of our method. Our approach adjusts the colors associated with ``autumn" and ``leaves" throughout the image while maintaining consistency in the ``trunk", which should be preserved from the source image. However, DDS and CDS focus solely on following the target prompt during editing, leading to unintended alterations of parts that should remain unchanged. Additionally, comparing the NeRF depth maps with baselines, our method generates clean outputs with minimal noise after image editing, whereas DDS and CDS introduce noticeable noise into the depth maps. 
% \vspace{-10pt}
% % Table for CLIP score
% \begin{table}[H]
% \centering
% \resizebox{0.9\columnwidth}{!}{
% \begin{tabular}{ccc}
% \toprule
% Metric & CLIP \cite{radford2021learning} score ($\uparrow$) & User Preference Rate ($\uparrow$) \\
% \midrule
% CDS \cite{nam2024contrastive}& $0.1597$ & $22.7$ \\
% DDS \cite{hertz2023delta}& $0.1596$ & $??$ \\
% \textbf{FPDS (ours)} & $\mathbf{0.1626}$ & $\mathbf{??}$ \\
% \bottomrule
% \end{tabular}
% }
% \caption{\textbf{Quantitative results of NeRF editing} comparing our method with other baselines for CLIP score and User Preference Rate on the NeRF LLFF dataset \cite{mildenhall2019local}. Higher CLIP scores and User Preference Rates indicate better performance.}
% \label{tab:Nerfclip}
% \end{table}
\begin{table}[thb!]
\centering
\resizebox{0.95\columnwidth}{!}{
\begin{tabular}{c|c|ccc}
\hline
\multirow{2}{*}{Metric} & \multirow{2}{*}{CLIP \cite{radford2021learning}  ($\uparrow$)} & \multicolumn{3}{c}{User Preference Rate (\%)} \\ 
\cline{3-5}
& & Text ($\uparrow$) & Preserving ($\uparrow$) & Quality ($\uparrow$) \\ 
\hline
DDS \cite{hertz2023delta}& 0.1596 & 36.88 & 28.37 & 32.62 \\
CDS \cite{nam2024contrastive}& 0.1597 & 22.70 & 23.40 & 21.28 \\
\hline
\textbf{IDS (Ours)} & \textbf{0.1626} & \textbf{40.42} & \textbf{48.23} & \textbf{46.10} \\
\hline
\end{tabular}
}
\caption{\textbf{Quantitative results of NeRF editing} with respect to CLIP score and User Preference Rate. IDS demonstrates superior quantitative performance compared to the baselines.}
\label{tab:Nerfclip}
\end{table}


\noindent\textbf{Quantitative Results.} Based on edited images, we performed 3D rendering and subsequently conducted quantitative evaluations provided in \cref{tab:Nerfclip}. To assess whether the edited 3D images are precisely aligned with the target prompts, we measured the CLIP \cite{radford2021learning} scores at 200k iterations of training on the LLFF dataset. We additionally present a user evaluation conducted under the same setup in \cref{sec:5.1}. Consistent with the trends observed in the qualitative results, our method demonstrates superior performance in the quantitative evaluations compared to other baselines.
%To demonstrate the effectiveness of our method in maintaining structural consistency during image editing and correcting errors progressively throughout training, we also conduct experiments involving 3D rendering of edited images. This approach is particularly relevant as consistency has an even greater impact on outcomes in 3D environments.


% \section{Limitations}
% \label{sec:limitations}
% \va{This is more future work rather than limitations. Maybe the depenedency on SMPL input can be one? Also, it cannot handle contact.}
% Since video sequences contain relatively stable backgrounds and stable changes with time flow, extending our proposed methods applicable to video sequences might improve the potential rendering and reconstruction quality.% Our work is based on calibrated cameras, it will be promising to estimate camera locations and parameters.  

\section{Conclusion}
\label{sec:conclusion}
We presented an approach for novel view synthesis of multiple humans from a sparse set of input views. To achieve this, we proposed geometric regularizations that improve geometry training by leveraging a pre-computed SMPL model, along with a patch-based ray consistency loss and a saturation loss that help with novel-view renderings in the sparse-view setting. Our experiments showed state-of-the-art performance for multiple human geometry and appearance reconstruction on real multi-human dataset (CMU Panoptic~\cite{Simon_2017_CVPR,Joo_2017_TPAMI}) and on synthetic data (MultiHuman-Dataset \cite{zheng2021deepmulticap}). Our method still has several limitations. For instance, our method does not model close human interactions, as this is a much more challenging case. Addressing this is an interesting direction for future work.

%Our method still has several limitations. First, we rely on SMPL fittings which might not always be accurate, particularly for scenes with a very large number of humans. A possible solution is to improve the SMPL reconstructions while training the geometry and appearance networks. Second, our method does not model close human interactions, as this is a much challenging case. Addressing this is an interesting direction for future work.




%In comparison to baselines \cite{yariv2021volume,wang2021neus}, we proposed a more efficient optimization method and demonstrated state-of-the-art performance on rendered novel views. In addition, our method enables multiple applications on multi-human datasets during rendering without retraining, including rotation, translation, scaling and removing. All editings doesn't require extra depth, masks or segmentation \etc supervision. %Based on our methods, there are more interesting applications that could be explored in the future. Moreover, since video sequences contain more information than one frame, it is meaningful to extend our methods to videos. The potential rendering and reconstruction quality would be even higher.  \\



%----------------------------------------------------------------------

%\clearpage
{\small
\bibliographystyle{ieee_fullname}
\bibliography{egbib}
}


%\title{Generating 3D \hl{Small} Binding Molecules Using Shape-Conditioned Diffusion Models with Guidance}
%\date{\vspace{-5ex}}

%\author{
%	Ziqi Chen\textsuperscript{\rm 1}, 
%	Bo Peng\textsuperscript{\rm 1}, 
%	Tianhua Zhai\textsuperscript{\rm 2},
%	Xia Ning\textsuperscript{\rm 1,3,4 \Letter}
%}
%\newcommand{\Address}{
%	\textsuperscript{\rm 1}Computer Science and Engineering, The Ohio Sate University, Columbus, OH 43210.
%	\textsuperscript{\rm 2}Perelman School of Medicine, University of Pennsylvania, Philadelphia, PA 19104.
%	\textsuperscript{\rm 3}Translational Data Analytics Institute, The Ohio Sate University, Columbus, OH 43210.
%	\textsuperscript{\rm 4}Biomedical Informatics, The Ohio Sate University, Columbus, OH 43210.
%	\textsuperscript{\Letter}ning.104@osu.edu
%}

%\newcommand\affiliation[1]{%
%	\begingroup
%	\renewcommand\thefootnote{}\footnote{#1}%
%	\addtocounter{footnote}{-1}%
%	\endgroup
%}



\setcounter{secnumdepth}{2} %May be changed to 1 or 2 if section numbers are desired.

\setcounter{section}{0}
\renewcommand{\thesection}{S\arabic{section}}

\setcounter{table}{0}
\renewcommand{\thetable}{S\arabic{table}}

\setcounter{figure}{0}
\renewcommand{\thefigure}{S\arabic{figure}}

\setcounter{algorithm}{0}
\renewcommand{\thealgorithm}{S\arabic{algorithm}}

\setcounter{equation}{0}
\renewcommand{\theequation}{S\arabic{equation}}


\begin{center}
	\begin{minipage}{0.95\linewidth}
		\centering
		\LARGE 
	Generating 3D Binding Molecules Using Shape-Conditioned Diffusion Models with Guidance (Supplementary Information)
	\end{minipage}
\end{center}
\vspace{10pt}

%%%%%%%%%%%%%%%%%%%%%%%%%%%%%%%%%%%%%%%%%%%%%
\section{Parameters for Reproducibility}
\label{supp:experiments:parameters}
%%%%%%%%%%%%%%%%%%%%%%%%%%%%%%%%%%%%%%%%%%%%%

We implemented both \SE and \methoddiff using Python-3.7.16, PyTorch-1.11.0, PyTorch-scatter-2.0.9, Numpy-1.21.5, Scikit-learn-1.0.2.
%
We trained the models using a Tesla V100 GPU with 32GB memory and a CPU with 80GB memory on Red Hat Enterprise 7.7.
%
%We released the code, data, and the trained model at Google Drive~\footnote{\url{https://drive.google.com/drive/folders/146cpjuwenKGTd6Zh4sYBy-Wv6BMfGwe4?usp=sharing}} (will release to the public on github once the manuscript is accepted).

%===================================================================
\subsection{Parameters of \SE}
%===================================================================


In \SE, we tuned the dimension of all the hidden layers including VN-DGCNN layers
(Eq.~\ref{eqn:shape_embed}), MLP layers (Eq.~\ref{eqn:se:decoder}) and
VN-In layer (Eq.~\ref{eqn:se:decoder}), and the dimension $d_p$ of generated shape latent embeddings $\shapehiddenmat$ with the grid-search algorithm in the 
parameter space presented in Table~\ref{tbl:hyper_se}.
%
We determined the optimal hyper-parameters according to the mean squared errors of the predictions of signed distances for 1,000 validation molecules that are selected as described in Section ``Data'' 
in the main manuscript.
%
The optimal dimension of all the hidden layers is 256, and the optimal dimension $d_p$ of shape latent embedding \shapehiddenmat is 128.
%
The optimal number of points $|\pc|$ in the point cloud \pc is 512.
%
We sampled 1,024 query points in $\mathcal{Z}$ for each molecule shape.
%
We constructed graphs from point clouds, which are employed to learn $\shapehiddenmat$ with VN-DGCNN layer (Eq.~\ref{eqn:shape_embed}), using the $k$-nearest neighbors based on Euclidean distance with $k=20$.
%
We set the number of VN-DGCNN layers as 4.
%
We set the number of MLP layers in the decoder (Eq.~\ref{eqn:se:decoder}) as 2.
%
We set the number of VN-In layers as 1.

%
We optimized the \SE model via Adam~\cite{adam} with its parameters (0.950, 0.999), %betas (0.95, 0.999), 
learning rate 0.001, and batch size 16.
%
We evaluated the validation loss every 2,000 training steps.
%
We scheduled to decay the learning rate with a factor of 0.6 and a minimum learning rate of 1e-6 if 
the validation loss does not decrease in 5 consecutive evaluations.
%
The optimal \SE model has 28.3K learnable parameters. 
%
We trained the \SE model %for at most 80 hours 
with $\sim$156,000 training steps.
%
The training took 80 hours with our GPUs.
%
The trained \SE model achieved the minimum validation loss at 152,000 steps.


\begin{table*}[!h]
  \centering
      \caption{{Hyper-Parameter Space for \SE Optimization}}
  \label{tbl:hyper_se}
  \begin{threeparttable}
 \begin{scriptsize}
      \begin{tabular}{
%	@{\hspace{2pt}}l@{\hspace{2pt}}
	@{\hspace{2pt}}l@{\hspace{5pt}} 
	@{\hspace{2pt}}r@{\hspace{2pt}}         
	}
        \toprule
        %Notation &
          Hyper-parameters &  Space\\
        \midrule
        %$t_a$    & 
         %hidden layer dimension         & \{16, 32, 64, 128\} \\
         %atom/node embedding dimension &  \{16, 32, 64, 128\} \\
         %$\latent^{\add}$/$\latent^{\delete}$ dimension        & \{8, 16, 32, 64\} \\
         hidden layer dimension            & \{128, 256\}\\
         dimension $d_p$ of \shapehiddenmat        &  \{64, 128\} \\
         \#points in \pc        & \{512, 1,024\} \\
         \#query points in $\mathcal{Z}$                & 1,024 \\%1024 \\%\bo{\{1024\}}\\
         \#nearest neighbors              & 20          \\
         \#VN-DGCNN layers (Eq~\ref{eqn:shape_embed})               & 4            \\
         \#MLP layers in Eq~\ref{eqn:se:decoder} & 4           \\
        \bottomrule
      \end{tabular}
%  	\begin{tablenotes}[normal,flushleft]
%  		\begin{footnotesize}
%  	
%  	\item In this table, hidden dimension represents the dimension of hidden layers and 
%  	atom/node embeddings; latent dimension represents the dimension of latent embedding \latent.
%  	\par
%  \end{footnotesize}
%  
%\end{tablenotes}
%      \begin{tablenotes}
%      \item 
%      \par
%      \end{tablenotes}
\end{scriptsize}
  \end{threeparttable}
\end{table*}

%
\begin{table*}[!h]
  \centering
      \caption{{Hyper-Parameter Space for \methoddiff Optimization}}
  \label{tbl:hyper_diff}
  \begin{threeparttable}
 \begin{scriptsize}
      \begin{tabular}{
%	@{\hspace{2pt}}l@{\hspace{2pt}}
	@{\hspace{2pt}}l@{\hspace{5pt}} 
	@{\hspace{2pt}}r@{\hspace{2pt}}         
	}
        \toprule
        %Notation &
          Hyper-parameters &  Space\\
        \midrule
        %$t_a$    & 
         %hidden layer dimension         & \{16, 32, 64, 128\} \\
         %atom/node embedding dimension &  \{16, 32, 64, 128\} \\
         %$\latent^{\add}$/$\latent^{\delete}$ dimension        & \{8, 16, 32, 64\} \\
         scalar hidden layer dimension         & 128 \\
         vector hidden layer dimension         & 32 \\
         weight of atom type loss $\xi$ (Eq.~\ref{eqn:loss})  & 100           \\
         threshold of step weight $\delta$ (Eq.~\ref{eqn:diff:obj:pos}) & 10 \\
         \#atom features $K$                   & 15 \\
         \#layers $L$ in \molpred             & 8 \\
         %\# \eqgnn/\invgnn layers     &  8 \\
         %\# heads {$n_h$} in $\text{MHA}^{\mathtt{x}}/\text{MHA}^{\mathtt{v}}$                               & 16 \\
         \#nearest neighbors {$N$}  (Eq.~\ref{eqn:geometric_embedding} and \ref{eqn:attention})            & 8          \\
         {\#diffusion steps $T$}                  & 1,000 \\
        \bottomrule
      \end{tabular}
%  	\begin{tablenotes}[normal,flushleft]
%  		\begin{footnotesize}
%  	
%  	\item In this table, hidden dimension represents the dimension of hidden layers and 
%  	atom/node embeddings; latent dimension represents the dimension of latent embedding \latent.
%  	\par
%  \end{footnotesize}
%  
%\end{tablenotes}
%      \begin{tablenotes}
%      \item 
%      \par
%      \end{tablenotes}
\end{scriptsize}
  \end{threeparttable}

\end{table*}


%===================================================================
\subsection{Parameters of \methoddiff}
%===================================================================

Table~\ref{tbl:hyper_diff} presents the parameters used to train \methoddiff.
%
In \methoddiff, we set the hidden dimensions of all the MLP layers and the scalar hidden layers in GVPs (Eq.~\ref{eqn:pred:gvp} and Eq.~\ref{eqn:mess:gvp}) as 128. %, including all the MLP layers in \methoddiff and the scalar dimension of GVP layers in Eq.~\ref{eqn:pred:gvp} and Eq.~\ref{eqn:mess:gvp}. %, and MLP layer (Eq.~\ref{eqn:diff:graph:atompred}) as 128.
%
We set the dimensions of all the vector hidden layers in GVPs as 32.
%
We set the number of layers $L$ in \molpred as 8.
%and the number of layers in graph neural networks as 8.
%
Both two GVP modules in Eq.~\ref{eqn:pred:gvp} and Eq.~\ref{eqn:mess:gvp} consist of three GVP layers. %, which consisa GVP modset the number of layer of GVP modules %is a multi-head attention layer ($\text{MHA}^{\mathtt{x}}$ or $\text{MHA}^{\mathtt{h}}$) with 16 heads.
% 
We set the number of VN-MLP layers in Eq.~\ref{eqn:shaper} as 1 and the number of MLP layers as 2 for all the involved MLP functions.
%

We constructed graphs from atoms in molecules, which are employed to learn the scalar embeddings and vector embeddings for atoms %predict atom coordinates and features  
(Eq.~\ref{eqn:geometric_embedding} and \ref{eqn:attention}), using the $N$-nearest neighbors based on Euclidean distance with $N=8$. 
%
We used $K=15$ atom features in total, indicating the atom types and its aromaticity.
%
These atom features include 10 non-aromatic atoms (i.e., ``H'', ``C'', ``N'', ``O'', ``F'', ``P'', ``S'', ``Cl'', ``Br'', ``I''), 
and 5 aromatic atoms (i.e., ``C'', ``N'', ``O'', ``P'', ``S'').
%
We set the number of diffusion steps $T$ as 1,000.
%
We set the weight $\xi$ of atom type loss (Eq.~\ref{eqn:loss}) as $100$ to balance the values of atom type loss and atom coordinate loss.
%
We set the threshold $\delta$ (Eq.~\ref{eqn:diff:obj:pos}) as 10.
%
The parameters $\beta_t^{\mathtt{x}}$ and $\beta_t^{\mathtt{v}}$ of variance scheduling in the forward diffusion process of \methoddiff are discussed in 
Supplementary Section~\ref{supp:forward:variance}.
%
%Please note that as in \squid, we did not perform extensive hyperparameter optimization for \methoddiff.
%
Following \squid, we did not perform extensive hyperparameter tunning for \methoddiff given that the used 
hyperparameters have enabled good performance.

%
We optimized the \methoddiff model via Adam~\cite{adam} with its parameters (0.950, 0.999), learning rate 0.001, and batch size 32.
%
We evaluated the validation loss every 2,000 training steps.
%
We scheduled to decay the learning rate with a factor of 0.6 and a minimum learning rate of 1e-5 if 
the validation loss does not decrease in 10 consecutive evaluations.
%
The \methoddiff model has 7.8M learnable parameters. 
%
We trained the \methoddiff model %for at most 60 hours 
with $\sim$770,000 training steps.
%
The training took 70 hours with our GPUs.
%
The trained \methoddiff achieved the minimum validation loss at 758,000 steps.

During inference, %the sampling, 
following Adams and Coley~\cite{adams2023equivariant}, we set the variance $\phi$ 
of atom-centered Gaussians as 0.049, which is used to build a set of points for shape guidance in Section ``\method with Shape Guidance'' 
in the main manuscript.
%
We determined the number of atoms in the generated molecule using the atom number distribution of training molecules that have surface shape sizes similar to the condition molecule.
%
The optimal distance threshold $\gamma$ is 0.2, and the optimal stop step $S$ for shape guidance is 300.
%
With shape guidance, each time we updated the atom position (Eq.~\ref{eqn:shape_guidance}), we randomly sampled the weight $\sigma$ from $[0.2, 0.8]$. %\bo{(XXX)}.
%
Moreover, when using pocket guidance as mentioned in Section ``\method with Pocket Guidance'' in the main manuscript, each time we updated the atom position (Eq.~\ref{eqn:pocket_guidance}), we randomly sampled the weight $\epsilon$ from $[0, 0.5]$. 
%
For each condition molecule, it took around 40 seconds on average to generate 50 molecule candidates with our GPUs.



%%%%%%%%%%%%%%%%%%%%%%%%%%%%%%%%%%%%%%%%%%%%%%
\section{Performance of \decompdiff with Protein Pocket Prior}
\label{supp:app:decompdiff}
%%%%%%%%%%%%%%%%%%%%%%%%%%%%%%%%%%%%%%%%%%%%%%

In this section, we demonstrate that \decompdiff with protein pocket prior, referred to as \decompdiffbeta, shows very limited performance in generating drug-like and synthesizable molecules compared to all the other methods, including \methodwithpguide and \methodwithsandpguide.
%
We evaluate the performance of \decompdiffbeta in terms of binding affinities, drug-likeness, and diversity.
%
We compare \decompdiffbeta with \methodwithpguide and \methodwithsandpguide and report the results in Table~\ref{tbl:comparison_results_decompdiff}.
%
Note that the results of \methodwithpguide and \methodwithsandpguide here are consistent with those in Table~\ref{tbl:overall_results_docking2} in the main manuscript.
%
As shown in Table~\ref{tbl:comparison_results_decompdiff}, while \decompdiffbeta achieves high binding affinities in Vina M and Vina D, it substantially underperforms \methodwithpguide and \methodwithsandpguide in QED and SA.
%
Particularly, \decompdiffbeta shows a QED score of 0.36, while \methodwithpguide substantially outperforms \decompdiffbeta in QED (0.77) with 113.9\% improvement.
%
\decompdiffbeta also substantially underperforms \methodwithpguide in terms of SA scores (0.55 vs 0.76).
%
These results demonstrate the limited capacity of \decompdiffbeta in generating drug-like and synthesizable molecules.
%
As a result, the generated molecules from \decompdiffbeta can have considerably lower utility compared to other methods.
%
Considering these limitations of \decompdiffbeta, we exclude it from the baselines for comparison.

\begin{table*}[!h]
	\centering
		\caption{Comparison on PMG among \methodwithpguide, \methodwithsandpguide and \decompdiffbeta}
	\label{tbl:comparison_results_decompdiff}
\begin{threeparttable}
	\begin{scriptsize}
	\begin{tabular}{
		@{\hspace{2pt}}l@{\hspace{2pt}}
		%
		%@{\hspace{2pt}}l@{\hspace{2pt}}
		%
		@{\hspace{2pt}}r@{\hspace{2pt}}
		@{\hspace{2pt}}r@{\hspace{2pt}}
		%
		@{\hspace{6pt}}r@{\hspace{6pt}}
		%
		@{\hspace{2pt}}r@{\hspace{2pt}}
		@{\hspace{2pt}}r@{\hspace{2pt}}
		%
		@{\hspace{5pt}}r@{\hspace{5pt}}
		%
		@{\hspace{2pt}}r@{\hspace{2pt}}
		@{\hspace{2pt}}r@{\hspace{2pt}}
		%
		@{\hspace{5pt}}r@{\hspace{5pt}}
		%
		@{\hspace{2pt}}r@{\hspace{2pt}}
	         @{\hspace{2pt}}r@{\hspace{2pt}}
		%
		@{\hspace{5pt}}r@{\hspace{5pt}}
		%
		@{\hspace{2pt}}r@{\hspace{2pt}}
		@{\hspace{2pt}}r@{\hspace{2pt}}
		%
		@{\hspace{5pt}}r@{\hspace{5pt}}
		%
		@{\hspace{2pt}}r@{\hspace{2pt}}
		@{\hspace{2pt}}r@{\hspace{2pt}}
		%
		@{\hspace{5pt}}r@{\hspace{5pt}}
		%
		@{\hspace{2pt}}r@{\hspace{2pt}}
		@{\hspace{2pt}}r@{\hspace{2pt}}
		%
		@{\hspace{5pt}}r@{\hspace{5pt}}
		%
		@{\hspace{2pt}}r@{\hspace{2pt}}
		%@{\hspace{2pt}}r@{\hspace{2pt}}
		%@{\hspace{2pt}}r@{\hspace{2pt}}
		}
		\toprule
		\multirow{2}{*}{method} & \multicolumn{2}{c}{Vina S$\downarrow$} & & \multicolumn{2}{c}{Vina M$\downarrow$} & & \multicolumn{2}{c}{Vina D$\downarrow$} & & \multicolumn{2}{c}{{HA\%$\uparrow$}}  & & \multicolumn{2}{c}{QED$\uparrow$} & & \multicolumn{2}{c}{SA$\uparrow$} & & \multicolumn{2}{c}{Div$\uparrow$} & %& \multirow{2}{*}{SR\%$\uparrow$} & 
		& \multirow{2}{*}{time$\downarrow$} \\
	    \cmidrule{2-3}\cmidrule{5-6} \cmidrule{8-9} \cmidrule{11-12} \cmidrule{14-15} \cmidrule{17-18} \cmidrule{20-21}
		& Avg. & Med. &  & Avg. & Med. &  & Avg. & Med. & & Avg. & Med.  & & Avg. & Med.  & & Avg. & Med.  & & Avg. & Med.  & & \\ %& & \\
		%\multirow{2}{*}{method} & \multirow{2}{*}{\#c\%} &  \multirow{2}{*}{\#u\%} &  \multirow{2}{*}{QED} & \multicolumn{3}{c}{$\nmax=50$} & & \multicolumn{2}{c}{$\nmax=1$}\\
		%\cmidrule(r){5-7} \cmidrule(r){8-10} 
		%& & & & \avgshapesim(std) & \avggraphsim(std  &  \diversity(std  & & \avgshapesim(std) & \avggraphsim(std \\
		\midrule
		%Reference                          & -5.32 & -5.66 & & -5.78 & -5.76 & & -6.63 & -6.67 & & - & - & & 0.53 & 0.49 & & 0.77 & 0.77 & & - & - & %& 23.1 & & - \\
		%\midrule
		%\multirow{4}{*}{PM} 
		%& \AR & -5.06 & -4.99 & &  -5.59 & -5.29 & &  -6.16 & -6.05 & &  37.69 & 31.00 & &  0.50 & 0.49 & &  0.66 & 0.65 & & - & - & %& 7.0 & 
		%& 7,789 \\
		%& \pockettwomol   & -4.50 & -4.21 & &  -5.70 & -5.27 & &  -6.43 & -6.25 & &  48.00 & 51.00 & &  0.58 & 0.58 & &  \textbf{0.77} & \textbf{0.78} & &  0.69 & 0.71 &  %& 24.9 & 
		%& 2,544 \\
		%& \targetdiff     & -4.88 & \underline{-5.82} & &  -6.20 & \underline{-6.36} & &  \textbf{-7.37} & \underline{-7.51} & &  57.57 & 58.27 & &  0.50 & 0.51 & &  0.60 & 0.59 & &  0.72 & 0.71 & % & 10.4 & 
		%& 1,252 \\
		 \decompdiffbeta             & -4.72 & -4.86 & & \textbf{-6.84} & \textbf{-6.91} & & \textbf{-8.85} & \textbf{-8.90} & &  {72.16} & {72.16} & &  0.36 & 0.36 & &  0.55 & 0.55 & & 0.59 & 0.59 & & 3,549 \\ 
		%-4.76 & -6.18 & &  \textbf{-6.86} & \textbf{-7.52} & &  \textbf{-8.85} & \textbf{-8.96} & &  \textbf{72.7} & \textbf{89.8} & &  0.36 & 0.34 & &  0.55 & 0.57 & & 0.59 & 0.59 & & 15.4 \\
		%& \decompdiffref  & -4.58 & -4.77 & &  -5.47 & -5.51 & &  -6.43 & -6.56 & &  47.76 & 48.66 & &  0.56 & 0.56 & &  0.70 & 0.69  & &  0.72 & 0.72 &  %& 15.2 & 
		%& 1,859 \\
		%\midrule
		%\multirow{2}{*}{PC}
		\methodwithpguide       &  \underline{-5.53} & \underline{-5.64} & & {-6.37} & -6.33 & &  \underline{-7.19} & \underline{-7.52} & &  \underline{78.75} & \textbf{94.00} & &  \textbf{0.77} & \textbf{0.80} & &  \textbf{0.76} & \textbf{0.76} & & 0.63 & 0.66 & & 462 \\
		\methodwithsandpguide   & \textbf{-5.81} & \textbf{-5.96} & &  \underline{-6.50} & \underline{-6.58} & & -7.16 & {-7.51} & &  \textbf{79.92} & \underline{93.00} & &  \underline{0.76} & \underline{0.79} & &  \underline{0.75} & \underline{0.74} & & 0.64 & 0.66 & & 561\\
		\bottomrule
	\end{tabular}%
	\begin{tablenotes}[normal,flushleft]
		\begin{footnotesize}
	\item 
\!\!Columns represent: {``Vina S'': the binding affinities between the initially generated poses of molecules and the protein pockets; 
		``Vina M'': the binding affinities between the poses after local structure minimization and the protein pockets;
		``Vina D'': the binding affinities between the poses determined by AutoDock Vina~\cite{Eberhardt2021} and the protein targets;
		``QED'': the drug-likeness score;
		``SA'': the synthesizability score;
		``Div'': the diversity among generated molecules;
		``time'': the time cost to generate molecules.}
		
		\par
		\par
		\end{footnotesize}
	\end{tablenotes}
	\end{scriptsize}
\end{threeparttable}
  \vspace{-10pt}    
\end{table*}



%===================================================================
\section{{Additional Experimental Results on SMG}}
\label{supp:app:results}
%===================================================================

%-------------------------------------------------------------------------------------------------------------------------------------
\subsection{Comparison on Shape and Graph Similarity}
\label{supp:app:results:overall_shape}
%-------------------------------------------------------------------------------------------------------------------------------------

%\ziqi{Outline for this section:
%	\begin{itemize}
%		\item \method can consistently generate molecules with novel structures (low graph similarity) and similar shapes (high shape similarity), such that these molecules have comparable binding capacity with the condition molecules, and potentially better properties as will be shown in Table~\ref{tbl:overall_results_quality_10}.
%	\end{itemize}
%}

\begin{table*}[!h]
	\centering
		\caption{Similarity Comparison on SMG}
	\label{tbl:overall_sim}
\begin{threeparttable}
	\begin{scriptsize}
	\begin{tabular}{
		@{\hspace{0pt}}l@{\hspace{8pt}}
		%
		@{\hspace{8pt}}l@{\hspace{8pt}}
		%
		@{\hspace{8pt}}c@{\hspace{8pt}}
		@{\hspace{8pt}}c@{\hspace{8pt}}
		%
	    	@{\hspace{0pt}}c@{\hspace{0pt}}
		%
		@{\hspace{8pt}}c@{\hspace{8pt}}
		@{\hspace{8pt}}c@{\hspace{8pt}}
		%
		%@{\hspace{8pt}}r@{\hspace{8pt}}
		}
		\toprule
		$\delta_g$  & method          & \avgshapesim$\uparrow$(std) & \avggraphsim$\downarrow$(std) & & \maxshapesim$\uparrow$(std) & \maxgraphsim$\downarrow$(std)       \\ %& \#n\%$\uparrow$  \\ 
		\midrule
		%\multirow{5}{0.079\linewidth}%{\hspace{0pt}0.1} & \dataset   & 0.0             & 0.628(0.139)          & 0.567(0.068)          & 0.078(0.010)          &  & 0.588(0.086)          & 0.081(0.013)          & 4.7              \\
		%&  \squid($\lambda$=0.3) & 0.0             & 0.320(0.000)          & 0.420(0.163)          & \textbf{0.056}(0.032) &  & 0.461(0.170)          & \textbf{0.065}(0.033) & 1.4              \\
		%& \squid($\lambda$=1.0) & 0.0             & 0.414(0.177)          & 0.483(0.184)          & \underline{0.064}(0.030)  &  & 0.531(0.182)          & \underline{0.073}(0.029)  & 2.4              \\
		%& \method               & \underline{1.6}     & \textbf{0.857}(0.034) & \underline{0.773}(0.045)  & 0.086(0.011)          &  & \underline{0.791}(0.053)  & 0.087(0.012)          & \underline{5.1}      \\
		%& \methodwithsguide      & \textbf{3.7}    & \underline{0.833}(0.062)  & \textbf{0.812}(0.037) & 0.088(0.009)          &  & \textbf{0.835}(0.047) & 0.089(0.010)          & \textbf{6.2}     \\ 
		%\cmidrule{2-10}
		%& improv\% & - & 36.5 & 43.2 & -53.6 &  & 42.0 & -33.8 & 31.9  \\
		%\midrule
		\multirow{6}{0.059\linewidth}{\hspace{0pt}0.3} & \dataset             & 0.745(0.037)          & \textbf{0.211}(0.026) &  & 0.815(0.039)          & \textbf{0.215}(0.047)      \\ %    & \textbf{100.0}   \\
			& \squid($\lambda$=0.3) & 0.709(0.076)          & 0.237(0.033)          &  & 0.841(0.070)          & 0.253(0.038)        \\ %  & 45.5             \\
		    & \squid($\lambda$=1.0) & 0.695(0.064)          & \underline{0.216}(0.034)  &  & 0.841(0.056)          & 0.231(0.047)        \\ %  & 84.3             \\
			& \method               & \underline{0.770}(0.039)  & 0.217(0.031)          &  & \underline{0.858}(0.038)  & \underline{0.220}(0.046)  \\ %& \underline{87.1}     \\
			& \methodwithsguide     & \textbf{0.823}(0.029) & 0.217(0.032)          &  & \textbf{0.900}(0.028) & 0.223(0.048)  \\ % & 86.0             \\ 
		%\cmidrule{2-7}
		%& improv\% & 10.5 & -2.8 &  & 7.0 & -2.3  \\ % & %-12.9  \\
		\midrule
		\multirow{6}{0.059\linewidth}{\hspace{0pt}0.5} & \dataset & 0.750(0.037)          & \textbf{0.225}(0.037) &  & 0.819(0.039)          & \textbf{0.236}(0.070)          \\ %& \textbf{100.0}   \\
			& \squid($\lambda$=0.3)  & 0.728(0.072)          & 0.301(0.054)          &  & \underline{0.888}(0.061)  & 0.355(0.088)          \\ %& 85.9             \\
			& \squid($\lambda$=1.0)  & 0.699(0.063)          & 0.233(0.043)          &  & 0.850(0.057)          & 0.263(0.080)          \\ %& \underline{99.5}     \\
			& \method               & \underline{0.771}(0.039)  & \underline{0.229}(0.043)  &  & 0.862(0.036)          & \textbf{0.236}(0.065) \\ %& 99.2             \\
			& \methodwithsguide    & \textbf{0.824}(0.029) & \underline{0.229}(0.044)  &  & \textbf{0.903}(0.027) & \underline{0.242}(0.069)  \\ %& 99.0             \\ 
		%\cmidrule{2-7}
		%& improv\% & 9.9 & -1.8 &  & 1.7 & 0.0 \\ %& -0.8  \\
		\midrule
		\multirow{6}{0.059\linewidth}{\hspace{0pt}0.7} 
		& \dataset &  0.750(0.037) & \textbf{0.226}(0.038) & & 0.819(0.039) & \underline{0.240}(0.081) \\ %& \textbf{100.0} \\
		%& \dataset & 12.3            & 0.736(0.076)          & 0.768(0.037)          & \textbf{0.228}(0.042) &  & 0.819(0.039)          & \underline{0.242}(0.085)  & \textbf{100.0}   \\
			& \squid($\lambda$=0.3) &  0.735(0.074)          & 0.328(0.070)          &  & \underline{0.900}(0.062)  & 0.435(0.143)          \\ %& 95.4             \\
			& \squid($\lambda$=1.0) &  0.699(0.064)          & 0.234(0.045)          &  & 0.851(0.057)          & 0.268(0.090)          \\ %& \underline{99.9}     \\
			& \method               &  \underline{0.771}(0.039)  & \underline{0.229}(0.043)  &  & 0.862(0.036)          & \textbf{0.237}(0.066) \\ %& 99.3             \\
			& \methodwithsguide     &  \textbf{0.824}(0.029) & 0.230(0.045)          &  & \textbf{0.903}(0.027) & 0.244(0.074)          \\ %& 99.2             \\ 
		%\cmidrule{2-7}
		%& improv\% & 9.9 & -1.3 &  & 0.3 & 1.3 \\%& -0.7  \\
		\midrule
		\multirow{6}{0.059\linewidth}{\hspace{0pt}1.0} 
		& \dataset & 0.750(0.037)          & \textbf{0.226}(0.038) &  & 0.819(0.039)          & \underline{0.242}(0.085)  \\%& \textbf{100.0}  \\
		& \squid($\lambda$=0.3) & 0.740(0.076)          & 0.349(0.088)          &  & \textbf{0.909}(0.065) & 0.547(0.245)       \\ %   & \textbf{100.0}  \\
		& \squid($\lambda$=1.0) & 0.699(0.064)          & 0.235(0.045)          &  & 0.851(0.057)          & 0.271(0.097)          \\ %& \textbf{100.0}   \\
		& \method               & \underline{0.771}(0.039)  & \underline{0.229}(0.043)  &  & 0.862(0.036)          & \textbf{0.237}(0.066) \\ %& \underline{99.3}  \\
		& \methodwithsguide      & \textbf{0.824}(0.029) & 0.230(0.045)          &  & \underline{0.903}(0.027)  & 0.244(0.076)          \\ %& 99.2            \\
		%\cmidrule{2-7}
		%& improv\% &  9.9               & -1.3              &  & -0.7              & -2.1           \\ %       & -0.7 \\
		\bottomrule
	\end{tabular}%
	\begin{tablenotes}[normal,flushleft]
		\begin{footnotesize}
	\item 
\!\!Columns represent: ``$\delta_g$'': the graph similarity constraint; 
%``\#d\%'': the percentage of molecules that satisfy the graph similarity constraint and are with high \shapesim ($\shapesim>=0.8$);
%``\diversity'': the diversity among the generated molecules;
``\avgshapesim/\avggraphsim'': the average of shape or graph similarities between the condition molecules and generated molecules with $\graphsim<=\delta_g$;
``\maxshapesim'': the maximum of shape similarities between the condition molecules and generated molecules with $\graphsim<=\delta_g$;
``\maxgraphsim'': the graph similarities between the condition molecules and the molecules with the maximum shape similarities and $\graphsim<=\delta_g$;
%``\#n\%'': the percentage of molecules that satisfy the graph similarity constraint ($\graphsim<=\delta_g$).
%
``$\uparrow$'' represents higher values are better, and ``$\downarrow$'' represents lower values are better.
%
 Best values are in \textbf{bold}, and second-best values are \underline{underlined}. 
\par
		\par
		\end{footnotesize}
	\end{tablenotes}
\end{scriptsize}
\end{threeparttable}
  \vspace{-10pt}    
\end{table*}
%\label{tbl:overall_sim}


{We evaluate the shape similarity \shapesim and graph similarity \graphsim of molecules generated from}
%Table~\ref{tbl:overall_sim} presents the comparison of shape-conditioned molecule generation among 
\dataset, \squid, \method and \methodwithsguide under different graph similarity constraints  ($\delta_g$=1.0, 0.7, 0.5, 0.3). 
%
%During the evaluation, for each molecule in the test set, all the methods are employed to generate or identify 50 molecules with similar shapes.
%
We calculate evaluation metrics using all the generated molecules satisfying the graph similarity constraints.
%
Particularly, when $\delta_g$=1.0, we do not filter out any molecules based on the constraints and directly calculate metrics on all the generated molecules.
%
When $\delta_g$=0.7, 0.5 or 0.3, we consider only generated molecules with similarities lower than $\delta_g$.
%
Based on \shapesim and \graphsim as described in Section ``Evaluation Metrics'' in the main manuscript,
we calculate the following metrics using the subset of molecules with \graphsim lower than $\delta_g$, from a set of 50 generated molecules for each test molecule and report the average of  these metrics across all test molecules:
%
(1) \avgshapesim\ measures the average \shapesim across each subset of generated molecules with $\graphsim$ lower than $\delta_g$; %per test molecule, with the overall average calculated across all test molecules; }%the 50 generated molecules for each test molecule, averaged across all test molecules;
(2) \avggraphsim\ calculates the average \graphsim for each set; %, with these means averaged across all test molecules}; %} 50 molecules, %\bo{@Ziqi rephrase}, with results averaged on the test set;\ziqi{with the average computed over the test set; }
(3) \maxshapesim\ determines the maximum \shapesim within each set; %, with these maxima averaged across all test molecules; }%\hl{among 50 molecules}, averaged across all test molecules;
(4) \maxgraphsim\ measures the \graphsim of the molecule with maximum \shapesim in each set. %, averaged across all test molecules; }%\hl{among 50 molecules}, averaged across all test molecules;

%
As shown in Table~\ref{tbl:overall_sim}, \method and \methodwithsguide demonstrate outstanding performance in terms of the average shape similarities (\avgshapesim) and the average graph similarities (\avggraphsim) among generated molecules.
%
%\ziqi{
%Table~\ref{tbl:overall} also shows that \method and \methodwithsguide consistently outperform all the baseline methods in average shape similarities (\avgshapesim) and only slightly underperform 
%the best baseline \dataset in average graph similarities (\avggraphsim).
%}
%
Specifically, when $\delta_g$=0.3, \methodwithsguide achieves a substantial 10.5\% improvement in \avgshapesim\ over the best baseline \dataset. 
%
In terms of \avggraphsim, \methodwithsguide also achieves highly comparable performance with \dataset (0.217 vs 0.211, in \avggraphsim, lower values indicate better performance).
%
%This trend remains consistent across various $\delta_g$ values.
This trend remains consistent when applying various similarity constraints (i.e., $\delta_g$) as shown in Table~\ref{tbl:overall_sim}.


Similarly, \method and \methodwithsguide demonstrate superior performance in terms of the average maximum shape similarity across generated molecules for all test molecules (\maxshapesim), as well as the average graph similarity of the molecules with the maximum shape similarities (\maxgraphsim). %maximum shape similarities of generated molecules (\maxshapesim) and the average graph similarities of molecules with the maximum shape similarities (\maxgraphsim). %\bo{\maxgraphsim is misleading... how about $\text{avgMSim}_\text{g}$}
%
%\bo{
%in terms of the maximum shape similarities (\maxshapesim) and the maximum graph similarities (\maxgraphsim) among all the generated molecules.
%@Ziqi are the metrics maximum values or the average of maximum values?
%}
%
Specifically, at \maxshapesim, Table~\ref{tbl:overall_sim} shows that \methodwithsguide outperforms the best baseline \squid ($\lambda$=0.3) when $\delta_g$=0.3, 0.5, and 0.7, and only underperforms
it by 0.7\% when $\delta$=1.0.
%
We also note that the molecules generated by {\methodwithsguide} with the maximum shape similarities have substantially lower graph similarities ({\maxgraphsim}) compared to those generated by {\squid} ({$\lambda$}=0.3).
%\hl{We also note that the molecules with the maximum shape similarities generated by {\methodwithsguide} are with significantly lower graph similarities ({\maxgraphsim}) than those generated by {\squid} ({$\lambda$}=0.3).}
%
%\bo{@Ziqi please rephrase the language}
%
%\bo{
%@Ziqi the conclusion is not obvious. You may want to remind the meaning of \maxshapesim and \maxgraphsim here, and based on what performance you say this.
%}
%
%\bo{\st{This also underscores the ability of {\methodwithsguide} in generating molecules with similar shapes to condition molecules and novel graph structures.}}
%
As evidenced by these results, \methodwithsguide features strong capacities of generating molecules with similar shapes yet novel graph structures compared to the condition molecule, facilitating the discovery of promising drug candidates.
%

\begin{comment}
\ziqi{replace \#n\% with the percentage of novel molecules that do not exist in the dataset and update the discussion accordingly}
%\ziqi{
Table~\ref{tbl:overall_sim} also presents \bo{\#n\%}, the percentage of molecules generated by each method %\st{(\#n\%)} 
with graph similarities lower than the constraint $\delta_g$. 
%
%\bo{
%Table~\ref{tbl:overall_sim} also presents \#n\%, the percentage of generated molecules with graph similarities lower than the constraint $\delta_g$, of different methods. 
%}
%
As shown in Table~\ref{tbl:overall_sim},  when a restricted constraint (i.e., $\delta_g$=0.3) is applied, \method and \methodwithsguide could still generate a sufficient number of molecules satisfying the constraint.
%
Particularly, when $\delta_g$=0.3, \method outperforms \squid with $\lambda$=0.3 by XXX and \squid with $\lambda$=1.0 by XXX.
% achieve the second and the third in \#n\% and only underperform the best baseline \dataset.
%
This demonstrates the ability of \method in generating molecules with novel structures. 
%
When $\delta_g$=0.5, 0.7 and 1.0, both methods generate over 99.0\% of molecules satisfying the similarity constraint $\delta_g$.
%
%Note that \dataset is guaranteed to identify at least 50 molecules satisfying the $\delta_g$ by searching within a training dataset of diverse molecules.
%
Note that \dataset is a search algorithm that always first identifies the molecules satisfying $\delta_g$ and then selects the top-50 molecules of the highest shape similarities among them. 
%
Due to the diverse molecules in %\hl{the subset} \bo{@Ziqi why do you want to stress subset?} of 
the training set, \dataset can always identify at least 50 molecules under different $\delta_g$ and thus achieve 100\% in \#n\%.
%
%\bo{
%Note that \dataset is a search algorithm that always generate molecules XXX
%@Ziqi
%We need to discuss here. For \dataset, \#n\% in this table does not look aligned with that in Fig 1 if the highlighted defination is correct...
%}
%
%Thus, \dataset achieves 100.0\% in \#n\% under different $\delta_g$.
%
It is also worth noting that when $\delta_g$=1.0, \#n\% reflects the validity among all the generated molecules. 
%
As shown in Table~\ref{tbl:overall_sim}, \method and \methodwithsguide are able to generate 99.3\% and 99.2\% valid molecules.
%
This demonstrates their ability to effectively capture the underlying chemical rules in a purely data-driven manner without relying on any prior knowledge (e.g., fragments) as \squid does.
%
%\bo{
%@Ziqi I feel this metric is redundant with the avg graph similarity when constraint is 1.0. Generally, if the avg similarity is small. You have more mols satisfying the requirement right?
%}
\end{comment}

Table~\ref{tbl:overall_sim} also shows that by incorporating shape guidance, \methodwithsguide
%\bo{
%@Ziqi where does this come from...
%}
substantially outperforms \method in both \avgshapesim and \maxshapesim, while maintaining comparable graph similarities (i.e., \avggraphsim\ and \maxgraphsim).
%
Particularly, when $\delta_g$=0.3, \methodwithsguide 
establishes a considerable improvement of 6.9\% and 4.9\%
%\bo{\st{achieves 6.9\% and 4.9\% improvements}} 
over \method in \avgshapesim and \maxshapesim, respectively. 
%
%\hl{In the meanwhile}, 
%\bo{@Ziqi it is not the right word...}
Meanwhile, \methodwithsguide achieves the same \avggraphsim with \method and only slightly underperforms \method in \maxgraphsim (0.223 vs 0.220).
%\bo{
%XXX also achieves XXX
%}
%it maintains the same \avggraphsim\ with \method and only slightly underperforms \method in \maxgraphsim (0.223 vs 0.220).
%
%Compared with \method, \methodwithsguide consistently generates molecules with higher shape similarities while maintaining comparable graph similarities.
%
%\bo{
%@Ziqi you may want to highlight the utility of "generating molecules with higher shape similarities while maintaining comparable graph similarities" in real drug discovery applications.
%
%
%\bo{
%@Ziqi You did not present the details of method yet...
%}
%
%\methodwithsguide leverages additional shape guidance to push the predicted atoms to the shape of condition molecules \bo{and XXX (@Ziqi boosts the shape similarities XXX)} , as will be discussed in Section ``\method with Shape Guidance'' later.
%
The superior performance of \methodwithsguide suggests that the incorporation of shape guidance effectively boosts the shape similarities of generated molecules without compromising graph similarities.
%
%This capability could be crucial in drug discovery, 
%\bo{@Ziqi it is a strong statement. Need citations here}, 
%as it enables the discovery of drug candidates that are both more potentially effective due to the improved shape similarities and novel induced by low graph similarities.
%as it could enable the identification of candidates with similar binding patterns %with the condition molecule (i.e., high shape similarities) 
%(i.e., high shape similarities) and graph structures distinct from the condition molecules (i.e., low graph similarities).
%\bo{\st{and enjoys novel structures (i.e., low graph similarities) with potentially better properties. } \ziqi{change enjoys}}
%\bo{
%and enjoys potentially better properties (i.e., low graph similarities). \ziqi{this looks weird to me... need to discuss}
%}
%\st{potentially better properties (i.e., low graph similarities).}}

%-------------------------------------------------------------------------------------------------------------------------------------
\subsection{Comparison on Validity and Novelty}
\label{supp:app:results:valid_novel}
%-------------------------------------------------------------------------------------------------------------------------------------

We evaluate the ability of \method and \squid to generate molecules with valid and novel 2D molecular graphs.
%
We calculate the percentages of the valid and novel molecules among all the generated molecules.
%
As shown in Table~\ref{tbl:validity_novelty}, both \method and \methodwithsguide outperform \squid with $\lambda$=0.3 and $\lambda$=1.0 in generating novel molecules.
%
Particularly, almost all valid molecules generated by \method and \methodwithsguide are novel (99.8\% and 99.9\% at \#n\%), while the best baseline \squid with $\lambda$=0.3 achieves 98.4\% in novelty.
%
In terms of the percentage of valid and novel molecules among all the generated ones (\#v\&n\%), \method and \methodwithsguide again outperform \squid with $\lambda$=0.3 and $\lambda$=1.0.
%
We also note that at \#v\%,  \method (99.1\%) and \methodwithsguide (99.2\%) slightly underperform \squid with $\lambda$=0.3 and $\lambda$=1.0 (100.0\%) in generating valid molecules.
%
\squid guarantees the validity of generated molecules by incorporating valence rules into the generation process and ensuring it to avoid fragments that violate these rules.
%
Conversely, \method and \methodwithsguide use a purely data-driven approach to learn the generation of valid molecules.
%
These results suggest that, even without integrating valence rules, \method and \methodwithsguide can still achieve a remarkably high percentage of valid and novel generated molecules.

\begin{table*}
	\centering
		\caption{Comparison on Validity and Novelty between \method and \squid}
	\label{tbl:validity_novelty}
	\begin{scriptsize}
\begin{threeparttable}
%	\setlength\tabcolsep{0pt}
	\begin{tabular}{
		@{\hspace{3pt}}l@{\hspace{10pt}}
		%
		@{\hspace{10pt}}r@{\hspace{10pt}}
		%
		@{\hspace{10pt}}r@{\hspace{10pt}}
		%
		@{\hspace{10pt}}r@{\hspace{3pt}}
		}
		\toprule
		method & \#v\% & \#n\% & \#v\&n\% \\
		\midrule
		\squid ($\lambda$=0.3) & \textbf{100.0} & 96.7 & 96.7 \\
		\squid ($\lambda$=1.0) & \textbf{100.0} & 98.4 & 98.4 \\
		\method & 99.1 & 99.8 & 98.9 \\
		\methodwithsguide & 99.2 & \textbf{99.9} & \textbf{99.1} \\
		\bottomrule
	\end{tabular}%
	%
	\begin{tablenotes}[normal,flushleft]
		\begin{footnotesize}
	\item 
\!\!Columns represent: ``\#v\%'': the percentage of generated molecules that are valid;
		``\#n\%'': the percentage of valid molecules that are novel;
		``\#v\&n\%'': the percentage of generated molecules that are valid and novel.
		Best values are in \textbf{bold}. 
		\par
		\end{footnotesize}
	\end{tablenotes}
\end{threeparttable}
\end{scriptsize}
\end{table*}


%-------------------------------------------------------------------------------------------------------------------------------------
\subsection{Additional Quality Comparison between Desirable Molecules Generated by \method and \squid}
\label{supp:app:results:quality_desirable}
%-------------------------------------------------------------------------------------------------------------------------------------

\begin{table*}[!h]
	\centering
		\caption{Comparison on Quality of Generated Desirable Molecules between \method and \squid ($\delta_g$=0.5)}
	\label{tbl:overall_results_quality_05}
	\begin{scriptsize}
\begin{threeparttable}
	\begin{tabular}{
		@{\hspace{0pt}}l@{\hspace{16pt}}
		@{\hspace{0pt}}l@{\hspace{2pt}}
		%
		@{\hspace{6pt}}c@{\hspace{6pt}}
		%
		%@{\hspace{3pt}}c@{\hspace{3pt}}
		@{\hspace{3pt}}c@{\hspace{3pt}}
		@{\hspace{3pt}}c@{\hspace{3pt}}
		@{\hspace{3pt}}c@{\hspace{3pt}}
		@{\hspace{3pt}}c@{\hspace{3pt}}
		%
		%
		}
		\toprule
		group & metric & 
        %& \dataset 
        & \squid ($\lambda$=0.3) & \squid ($\lambda$=1.0)  &  \method & \methodwithsguide  \\
		%\multirow{2}{*}{method} & \multirow{2}{*}{\#c\%} &  \multirow{2}{*}{\#u\%} &  \multirow{2}{*}{QED} & \multicolumn{3}{c}{$\nmax=50$} & & \multicolumn{2}{c}{$\nmax=1$}\\
		%\cmidrule(r){5-7} \cmidrule(r){8-10} 
		%& & & & \avgshapesim(std) & \avggraphsim(std  &  \diversity(std  & & \avgshapesim(std) & \avggraphsim(std \\
		\midrule
		\multirow{2}{*}{stability}
		& atom stability ($\uparrow$) & 
        %& 0.990 
        & \textbf{0.996} & 0.995 & 0.992 & 0.989     \\
		& mol stability ($\uparrow$) & 
        %& 0.875 
        & \textbf{0.948} & 0.947 & 0.886 & 0.839    \\
		%\midrule
		%\multirow{3}{*}{Drug-likeness} 
		%& QED ($\uparrow$) & 
        %& \textbf{0.805} 
        %& 0.766 & 0.760 & 0.755 & 0.751    \\
	%	& SA ($\uparrow$) & 
        %& \textbf{0.874} 
        %& 0.814 & 0.813 & 0.699 & 0.692    \\
	%	& Lipinski ($\uparrow$) & 
        %& \textbf{4.999} 
        %& 4.979 & 4.980 & 4.967 & 4.975    \\
		\midrule
		\multirow{4}{*}{3D structures} 
		& RMSD ($\downarrow$) & 
        %& \textbf{0.419} 
        & 0.907 & 0.906 & 0.897 & \textbf{0.881}    \\
		& JS. bond lengths ($\downarrow$) & 
        %& \textbf{0.286} 
        & 0.457 & 0.477 & 0.436 & \textbf{0.428}    \\
		& JS. bond angles ($\downarrow$) & 
        %& \textbf{0.078} 
        & 0.269 & 0.289 & \textbf{0.186} & 0.200    \\
		& JS. dihedral angles ($\downarrow$) & 
        %& \textbf{0.151} 
        & 0.199 & 0.209 & \textbf{0.168} & 0.170    \\
		\midrule
		\multirow{5}{*}{2D structures} 
		& JS. \#bonds per atoms ($\downarrow$) & 
        %& 0.325 
        & 0.291 & 0.331 & \textbf{0.176} & 0.181    \\
		& JS. basic bond types ($\downarrow$) & 
        %& \textbf{0.055} 
        & \textbf{0.071} & 0.083 & 0.181 & 0.191    \\
		%& JS. freq. bond types ($\downarrow$) & 
        %& \textbf{0.089} 
        %& 0.123 & 0.130 & 0.245 & 0.254    \\
		%& JS. freq. bond pairs ($\downarrow$) & 
        %& \textbf{0.078} 
        %& 0.085 & 0.089 & 0.209 & 0.221    \\
		%& JS. freq. bond triplets ($\downarrow$) & 
        %& \textbf{0.089} 
        %& 0.097 & 0.114 & 0.211 & 0.223    \\
		%\midrule
		%\multirow{3}{*}{Rings} 
		& JS. \#rings ($\downarrow$) & 
        %& 0.142 
        & 0.280 & 0.330 & \textbf{0.043} & 0.049    \\
		& JS. \#n-sized rings ($\downarrow$) & 
        %& \textbf{0.055} 
        & \textbf{0.077} & 0.091 & 0.099 & 0.112    \\
		& \#Intersecting rings ($\uparrow$) & 
        %& \textbf{6} 
        & \textbf{6} & 5 & 4 & 5    \\
		%\method (+bt)            & 100.0 & 98.0 & 100.0 & 0.742 & 0.772 (0.040) & 0.211 (0.033) & & 0.862 (0.036) & 0.211 (0.033) & 0.743 (0.043) \\
		%\methodwithguide (+bt)    & 99.8 & 98.0 & 100.0 & 0.736 & 0.814 (0.031) & 0.193 (0.042) & & 0.895 (0.029) & 0.193 (0.042) & 0.745 (0.045) \\
		%
		\bottomrule
	\end{tabular}%
	\begin{tablenotes}[normal,flushleft]
		\begin{footnotesize}
	\item 
\!\!Rows represent:  {``atom stability'': the proportion of stable atoms that have the correct valency; 
		``molecule stability'': the proportion of generated molecules with all atoms stable;
		%``QED'': the drug-likeness score;
		%``SA'': the synthesizability score;
		%``Lipinski'': the Lipinski 
		``RMSD'': the root mean square deviation (RMSD) between the generated 3D structures of molecules and their optimal conformations; % identified via energy minimization;
		``JS. bond lengths/bond angles/dihedral angles'': the Jensen-Shannon (JS) divergences of bond lengths, bond angles and dihedral angles;
		``JS. \#bonds per atom/basic bond types/\#rings/\#n-sized rings'': the JS divergences of bond counts per atom, basic bond types, counts of all rings, and counts of n-sized rings;
		%``JS. \#rings/\#n-sized rings'': the JS divergences of the total counts of rings and the counts of n-sized rings;
		``\#Intersecting rings'': the number of rings observed in the top-10 frequent rings of both generated and real molecules. } \par
		\par
		\end{footnotesize}
	\end{tablenotes}
\end{threeparttable}
\end{scriptsize}
\end{table*}

%\label{tbl:overall_quality05}

\begin{table*}[!h]
	\centering
		\caption{Comparison on Quality of Generated Desirable Molecules between \method and \squid ($\delta_g$=0.7)}
	\label{tbl:overall_results_quality_07}
	\begin{scriptsize}
\begin{threeparttable}
	\begin{tabular}{
		@{\hspace{0pt}}l@{\hspace{14pt}}
		@{\hspace{0pt}}l@{\hspace{2pt}}
		%
		@{\hspace{4pt}}c@{\hspace{4pt}}
		%
		%@{\hspace{3pt}}c@{\hspace{3pt}}
		@{\hspace{3pt}}c@{\hspace{3pt}}
		@{\hspace{3pt}}c@{\hspace{3pt}}
		@{\hspace{3pt}}c@{\hspace{3pt}}
		@{\hspace{3pt}}c@{\hspace{3pt}}
		%
		%
		}
		\toprule
		group & metric & 
        %& \dataset 
        & \squid ($\lambda$=0.3) & \squid ($\lambda$=1.0)  &  \method & \methodwithsguide  \\
		%\multirow{2}{*}{method} & \multirow{2}{*}{\#c\%} &  \multirow{2}{*}{\#u\%} &  \multirow{2}{*}{QED} & \multicolumn{3}{c}{$\nmax=50$} & & \multicolumn{2}{c}{$\nmax=1$}\\
		%\cmidrule(r){5-7} \cmidrule(r){8-10} 
		%& & & & \avgshapesim(std) & \avggraphsim(std  &  \diversity(std  & & \avgshapesim(std) & \avggraphsim(std \\
		\midrule
		\multirow{2}{*}{stability} 
		& atom stability ($\uparrow$) & 
        %&  0.990 
        & \textbf{0.995} & 0.995 & 0.992 & 0.988 \\
		& molecule stability ($\uparrow$) & 
        %& 0.876 
        & 0.944 & \textbf{0.947} & 0.885 & 0.839 \\
		\midrule
		%\multirow{3}{*}{Drug-likeness} 
		%& QED ($\uparrow$) & 
        %& \textbf{0.805} 
        %& 0.766 & 0.760 & 0.755 & 0.751    \\
	%	& SA ($\uparrow$) & 
        %& \textbf{0.874} 
        %& 0.814 & 0.813 & 0.699 & 0.692    \\
	%	& Lipinski ($\uparrow$) & 
        %& \textbf{4.999} 
        %& 4.979 & 4.980 & 4.967 & 4.975    \\
	%	\midrule
		\multirow{4}{*}{3D structures} 
		& RMSD ($\downarrow$) & 
        %& \textbf{0.420} 
        & 0.897 & 0.906 & 0.897 & \textbf{0.881}    \\
		& JS. bond lengths ($\downarrow$) & 
        %& \textbf{0.286} 
        & 0.457 & 0.477 & 0.436 & \textbf{0.428}    \\
		& JS. bond angles ($\downarrow$) & 
        %& \textbf{0.078} 
        & 0.269 & 0.289 & \textbf{0.186} & 0.200    \\
		& JS. dihedral angles ($\downarrow$) & 
        %& \textbf{0.151} 
        & 0.199 & 0.209 & \textbf{0.168} & 0.170    \\
		\midrule
		\multirow{5}{*}{2D structures} 
		& JS. \#bonds per atoms ($\downarrow$) & 
        %& 0.325 
        & 0.285 & 0.329 & \textbf{0.176} & 0.181    \\
		& JS. basic bond types ($\downarrow$) & 
        %& \textbf{0.055} 
        & \textbf{0.067} & 0.083 & 0.181 & 0.191    \\
	%	& JS. freq. bond types ($\downarrow$) & 
        %& \textbf{0.089} 
        %& 0.123 & 0.130 & 0.245 & 0.254    \\
	%	& JS. freq. bond pairs ($\downarrow$) & 
        %& \textbf{0.078} 
        %& 0.085 & 0.089 & 0.209 & 0.221    \\
	%	& JS. freq. bond triplets ($\downarrow$) & 
        %& \textbf{0.089} 
        %& 0.097 & 0.114 & 0.211 & 0.223    \\
	%	\midrule
	%	\multirow{3}{*}{Rings} 
		& JS. \#rings ($\downarrow$) & 
        %& 0.143 
        & 0.273 & 0.328 & \textbf{0.043} & 0.049    \\
		& JS. \#n-sized rings ($\downarrow$) & 
        %& \textbf{0.055} 
        & \textbf{0.076} & 0.091 & 0.099 & 0.112    \\
		& \#Intersecting rings ($\uparrow$) & 
        %& \textbf{6} 
        & \textbf{6} & 5 & 4 & 5    \\
		%\method (+bt)            & 100.0 & 98.0 & 100.0 & 0.742 & 0.772 (0.040) & 0.211 (0.033) & & 0.862 (0.036) & 0.211 (0.033) & 0.743 (0.043) \\
		%\methodwithguide (+bt)    & 99.8 & 98.0 & 100.0 & 0.736 & 0.814 (0.031) & 0.193 (0.042) & & 0.895 (0.029) & 0.193 (0.042) & 0.745 (0.045) \\
		%
		\bottomrule
	\end{tabular}%
	\begin{tablenotes}[normal,flushleft]
		\begin{footnotesize}
	\item 
\!\!Rows represent:  {``atom stability'': the proportion of stable atoms that have the correct valency; 
		``molecule stability'': the proportion of generated molecules with all atoms stable;
		%``QED'': the drug-likeness score;
		%``SA'': the synthesizability score;
		%``Lipinski'': the Lipinski 
		``RMSD'': the root mean square deviation (RMSD) between the generated 3D structures of molecules and their optimal conformations; % identified via energy minimization;
		``JS. bond lengths/bond angles/dihedral angles'': the Jensen-Shannon (JS) divergences of bond lengths, bond angles and dihedral angles;
		``JS. \#bonds per atom/basic bond types/\#rings/\#n-sized rings'': the JS divergences of bond counts per atom, basic bond types, counts of all rings, and counts of n-sized rings;
		%``JS. \#rings/\#n-sized rings'': the JS divergences of the total counts of rings and the counts of n-sized rings;
		``\#Intersecting rings'': the number of rings observed in the top-10 frequent rings of both generated and real molecules. } \par
		\par
		\end{footnotesize}
	\end{tablenotes}
\end{threeparttable}
\end{scriptsize}
\end{table*}

%\label{tbl:overall_quality07}

\begin{table*}[!h]
	\centering
		\caption{Comparison on Quality of Generated Desirable Molecules between \method and \squid ($\delta_g$=1.0)}
	\label{tbl:overall_results_quality_10}
	\begin{scriptsize}
\begin{threeparttable}
	\begin{tabular}{
		@{\hspace{0pt}}l@{\hspace{14pt}}
		@{\hspace{0pt}}l@{\hspace{2pt}}
		%
		@{\hspace{4pt}}c@{\hspace{4pt}}
		%
		%@{\hspace{3pt}}c@{\hspace{3pt}}
		@{\hspace{3pt}}c@{\hspace{3pt}}
		@{\hspace{3pt}}c@{\hspace{3pt}}
		@{\hspace{3pt}}c@{\hspace{3pt}}
		@{\hspace{3pt}}c@{\hspace{3pt}}
		%
		%
		}
		\toprule
		group & metric & 
        %& \dataset 
        & \squid ($\lambda$=0.3) & \squid ($\lambda$=1.0)  &  \method & \methodwithsguide \\
		%\multirow{2}{*}{method} & \multirow{2}{*}{\#c\%} &  \multirow{2}{*}{\#u\%} &  \multirow{2}{*}{QED} & \multicolumn{3}{c}{$\nmax=50$} & & \multicolumn{2}{c}{$\nmax=1$}\\
		%\cmidrule(r){5-7} \cmidrule(r){8-10} 
		%& & & & \avgshapesim(std) & \avggraphsim(std  &  \diversity(std  & & \avgshapesim(std) & \avggraphsim(std \\
		\midrule
		\multirow{2}{*}{stability}
		& atom stability ($\uparrow$) & 
        %& 0.990 
        & \textbf{0.995} & \textbf{0.995} & 0.992 & 0.988     \\
		& mol stability ($\uparrow$) & 
        %& 0.876 
        & 0.942 & \textbf{0.947} & 0.885 & 0.839    \\
		\midrule
	%	\multirow{3}{*}{Drug-likeness} 
	%	& QED ($\uparrow$) & 
        %& \textbf{0.805} 
        %& \textbf{0.766} & 0.760 & 0.755 & 0.751    \\
	%	& SA ($\uparrow$) & 
        %& \textbf{0.874} 
        %& \textbf{0.813} & \textbf{0.813} & 0.699 & 0.692    \\
	%	& Lipinski ($\uparrow$) & 
        %& \textbf{4.999} 
        %& 4.979 & \textbf{4.980} & 4.967 & 4.975    \\
	%	\midrule
		\multirow{4}{*}{3D structures} 
		& RMSD ($\downarrow$) & 
        %& \textbf{0.420} 
        & 0.898 & 0.906 & 0.897 & \textbf{0.881}    \\
		& JS. bond lengths ($\downarrow$) & 
        %& \textbf{0.286} 
        & 0.457 & 0.477 & 0.436 & \textbf{0.428}    \\
		& JS. bond angles ($\downarrow$) & 
        %& \textbf{0.078} 
        & 0.269 & 0.289 & \textbf{0.186} & 0.200   \\
		& JS. dihedral angles ($\downarrow$) & 
        %& \textbf{0.151} 
        & 0.199 & 0.209 & \textbf{0.168} & 0.170    \\
		\midrule
		\multirow{5}{*}{2D structures} 
		& JS. \#bonds per atoms ($\downarrow$) & 
        %& 0.325 
        & 0.280 & 0.330 & \textbf{0.176} & 0.181    \\
		& JS. basic bond types ($\downarrow$) & 
        %& \textbf{0.055} 
        & \textbf{0.066} & 0.083 & 0.181 & 0.191   \\
	%	& JS. freq. bond types ($\downarrow$) & 
        %& \textbf{0.089} 
        %& \textbf{0.123} & 0.130 & 0.245 & 0.254    \\
	%	& JS. freq. bond pairs ($\downarrow$) & 
        %& \textbf{0.078} 
        %& \textbf{0.085} & 0.089 & 0.209 & 0.221    \\
	%	& JS. freq. bond triplets ($\downarrow$) & 
        %& \textbf{0.089} 
        %& \textbf{0.097} & 0.114 & 0.211 & 0.223    \\
		%\midrule
		%\multirow{3}{*}{Rings} 
		& JS. \#rings ($\downarrow$) & 
        %& 0.143 
        & 0.269 & 0.328 & \textbf{0.043} & 0.049    \\
		& JS. \#n-sized rings ($\downarrow$) & 
        %& \textbf{0.055} 
        & \textbf{0.075} & 0.091 & 0.099 & 0.112    \\
		& \#Intersecting rings ($\uparrow$) & 
        %& \textbf{6} 
        & \textbf{6} & 5 & 4 & 5    \\
		%\method (+bt)            & 100.0 & 98.0 & 100.0 & 0.742 & 0.772 (0.040) & 0.211 (0.033) & & 0.862 (0.036) & 0.211 (0.033) & 0.743 (0.043) \\
		%\methodwithguide (+bt)    & 99.8 & 98.0 & 100.0 & 0.736 & 0.814 (0.031) & 0.193 (0.042) & & 0.895 (0.029) & 0.193 (0.042) & 0.745 (0.045) \\
		%
		\bottomrule
	\end{tabular}%
	\begin{tablenotes}[normal,flushleft]
		\begin{footnotesize}
	\item 
\!\!Rows represent:  {``atom stability'': the proportion of stable atoms that have the correct valency; 
		``molecule stability'': the proportion of generated molecules with all atoms stable;
		%``QED'': the drug-likeness score;
		%``SA'': the synthesizability score;
		%``Lipinski'': the Lipinski 
		``RMSD'': the root mean square deviation (RMSD) between the generated 3D structures of molecules and their optimal conformations; % identified via energy minimization;
		``JS. bond lengths/bond angles/dihedral angles'': the Jensen-Shannon (JS) divergences of bond lengths, bond angles and dihedral angles;
		``JS. \#bonds per atom/basic bond types/\#rings/\#n-sized rings'': the JS divergences of bond counts per atom, basic bond types, counts of all rings, and counts of n-sized rings;
		%``JS. \#rings/\#n-sized rings'': the JS divergences of the total counts of rings and the counts of n-sized rings;
		``\#Intersecting rings'': the number of rings observed in the top-10 frequent rings of both generated and real molecules. } \par
		\par
		\end{footnotesize}
	\end{tablenotes}
\end{threeparttable}
\end{scriptsize}
\end{table*}

%\label{tbl:overall_quality10}

Similar to Table~\ref{tbl:overall_results_quality_desired} in the main manuscript, we present the performance comparison on the quality of desirable molecules generated by different methods under different graph similarity constraints $\delta_g$=0.5, 0.7 and 1.0, as detailed in Table~\ref{tbl:overall_results_quality_05}, Table~\ref{tbl:overall_results_quality_07}, and Table~\ref{tbl:overall_results_quality_10}, respectively.
%
Overall, these tables show that under varying graph similarity constraints, \method and \methodwithsguide can always generate desirable molecules with comparable quality to baselines in terms of stability, 3D structures, and 2D structures.
%
These results demonstrate the strong effectiveness of \method and \methodwithsguide in generating high-quality desirable molecules with stable and realistic structures in both 2D and 3D.
%
This enables the high utility of \method and \methodwithsguide in discovering promising drug candidates.


\begin{comment}
The results across these tables demonstrate similar observations with those under $\delta_g$=0.3 in Table~\ref{tbl:overall_results_quality_desired}.
%
For stability, when $\delta_g$=0.5, 0.7 or 1.0, \method and \methodwithsguide achieve comparable performance or fall slightly behind \squid ($\lambda$=0.3) and \squid ($\lambda$=1.0) in atom stability and molecule stability.
%
For example, when $\delta_g$=0.5, as shown in Table~\ref{tbl:overall_results_quality_05}, \method achieves similar performance with the best baseline \squid ($\lambda$=0.3) in atom stability (0.992 for \method vs 0.996 for \squid with $\lambda$=0.3).
%
\method underperforms \squid ($\lambda$=0.3) in terms of molecule stability.
%
For 3D structures, \method and \methodwithsguide also consistently generate molecules with more realistic 3D structures compared to \squid.
%
Particularly, \methodwithsguide achieves the best performance in RMSD and JS of bond lengths across $\delta_g$=0.5, 0.7 and 1.0.
%
In JS of dihedral angles, \method achieves the best performance among all the methods.
%
\method and \methodwithsguide underperform \squid in JS of bond angles, primarily because \squid constrains the bond angles in the generated molecules.
%
For 2D structures, \method and \methodwithsguide again achieve the best performance 
\end{comment}

%===================================================================
\section{Additional Experimental Results on PMG}
\label{supp:app:results_PMG}
%===================================================================

%\label{tbl:comparison_results_decompdiff}


%-------------------------------------------------------------------------------------------------------------------------------------
%\subsection{{Additional Comparison for PMG}}
%\label{supp:app:results:docking}
%-------------------------------------------------------------------------------------------------------------------------------------

In this section, we present the results of \methodwithpguide and \methodwithsandpguide when generating 100 molecules. 
%
Please note that both \methodwithpguide and \methodwithsandpguide show remarkable efficiency over the PMG baselines.
%
\methodwithpguide and \methodwithsandpguide generate 100 molecules in 48 and 58 seconds on average, respectively, while the most efficient baseline \targetdiff requires 1,252 seconds.
%
We report the performance of \methodwithpguide and \methodwithsandpguide against state-of-the-art PMG baselines in Table~\ref{tbl:overall_results_docking_100}.


%
According to Table~\ref{tbl:overall_results_docking_100}, \methodwithpguide and \methodwithsandpguide achieve comparable performance with the PMG baselines in generating molecules with high binding affinities.
%
Particularly, in terms of Vina S, \methodwithsandpguide achieves very comparable performance (-4.56 kcal/mol) to the third-best baseline \decompdiff (-4.58 kcal/mol) in average Vina S; it also achieves the third-best performance (-4.82 kcal/mol) among all the methods and slightly underperforms the second-best baseline \AR (-4.99 kcal/mol) in median Vina S
%
\methodwithsandpguide also achieves very close average Vina M (-5.53 kcal/mol) with the third-best baseline \AR (-5.59 kcal/mol) and the third-best performance (-5.47 kcal/mol) in median Vina M.
%
Notably, for Vina D, \methodwithpguide and \methodwithsandpguide achieve the second and third performance among all the methods.
%
In terms of the average percentage of generated molecules with Vina D higher than those of known ligands (i.e., HA), \methodwithpguide (58.52\%) and \methodwithsandpguide (58.28\%) outperform the best baseline \targetdiff (57.57\%).
%
These results signify the high utility of \methodwithpguide and \methodwithsandpguide in generating molecules that effectively bind with protein targets and have better binding affinities than known ligands.

In addition to binding affinities, \methodwithpguide and \methodwithsandpguide also demonstrate similar performance compared to the baselines in metrics related to drug-likeness and diversity.
%
For drug-likeness, both \methodwithpguide and \methodwithsandpguide achieve the best (0.67) and the second-best (0.66) QED scores.
%
They also achieve the third and fourth performance in SA scores.
%
In terms of the diversity among generated molecules,  \methodwithpguide and \methodwithsandpguide slightly underperform the baselines, possibly due to the design that generates molecules with similar shapes to the ligands.
%
These results highlight the strong ability of \methodwithpguide and \methodwithsandpguide in efficiently generating effective binding molecules with favorable drug-likeness and diversity.
%
This ability enables them to potentially serve as promising tools to facilitate effective and efficient drug development.

\begin{table*}[!h]
	\centering
		\caption{Additional Comparison on PMG When All Methods Generate 100 Molecules}
	\label{tbl:overall_results_docking_100}
\begin{threeparttable}
	\begin{scriptsize}
	\begin{tabular}{
		@{\hspace{2pt}}l@{\hspace{2pt}}
		%
		@{\hspace{2pt}}r@{\hspace{2pt}}
		%
		@{\hspace{2pt}}r@{\hspace{2pt}}
		@{\hspace{2pt}}r@{\hspace{2pt}}
		%
		@{\hspace{6pt}}r@{\hspace{6pt}}
		%
		@{\hspace{2pt}}r@{\hspace{2pt}}
		@{\hspace{2pt}}r@{\hspace{2pt}}
		%
		@{\hspace{5pt}}r@{\hspace{5pt}}
		%
		@{\hspace{2pt}}r@{\hspace{2pt}}
		@{\hspace{2pt}}r@{\hspace{2pt}}
		%
		@{\hspace{5pt}}r@{\hspace{5pt}}
		%
		@{\hspace{2pt}}r@{\hspace{2pt}}
	         @{\hspace{2pt}}r@{\hspace{2pt}}
		%
		@{\hspace{5pt}}r@{\hspace{5pt}}
		%
		@{\hspace{2pt}}r@{\hspace{2pt}}
		@{\hspace{2pt}}r@{\hspace{2pt}}
		%
		@{\hspace{5pt}}r@{\hspace{5pt}}
		%
		@{\hspace{2pt}}r@{\hspace{2pt}}
		@{\hspace{2pt}}r@{\hspace{2pt}}
		%
		@{\hspace{5pt}}r@{\hspace{5pt}}
		%
		@{\hspace{2pt}}r@{\hspace{2pt}}
		@{\hspace{2pt}}r@{\hspace{2pt}}
		%
		@{\hspace{5pt}}r@{\hspace{5pt}}
		%
		@{\hspace{2pt}}r@{\hspace{2pt}}
		%@{\hspace{2pt}}r@{\hspace{2pt}}
		%@{\hspace{2pt}}r@{\hspace{2pt}}
		}
		\toprule
		\multirow{2}{*}{method} & \multicolumn{2}{c}{Vina S$\downarrow$} & & \multicolumn{2}{c}{Vina M$\downarrow$} & & \multicolumn{2}{c}{Vina D$\downarrow$} & & \multicolumn{2}{c}{{HA\%$\uparrow$}}  & & \multicolumn{2}{c}{QED$\uparrow$} & & \multicolumn{2}{c}{SA$\uparrow$} & & \multicolumn{2}{c}{Div$\uparrow$} & %& \multirow{2}{*}{SR\%$\uparrow$} & 
		& \multirow{2}{*}{time$\downarrow$} \\
	    \cmidrule{2-3}\cmidrule{5-6} \cmidrule{8-9} \cmidrule{11-12} \cmidrule{14-15} \cmidrule{17-18} \cmidrule{20-21}
		 & Avg. & Med. &  & Avg. & Med. &  & Avg. & Med. & & Avg. & Med.  & & Avg. & Med.  & & Avg. & Med.  & & Avg. & Med.  & & \\ %& & \\
		%\multirow{2}{*}{method} & \multirow{2}{*}{\#c\%} &  \multirow{2}{*}{\#u\%} &  \multirow{2}{*}{QED} & \multicolumn{3}{c}{$\nmax=50$} & & \multicolumn{2}{c}{$\nmax=1$}\\
		%\cmidrule(r){5-7} \cmidrule(r){8-10} 
		%& & & & \avgshapesim(std) & \avggraphsim(std  &  \diversity(std  & & \avgshapesim(std) & \avggraphsim(std \\
		\midrule
		Reference                          & -5.32 & -5.66 & & -5.78 & -5.76 & & -6.63 & -6.67 & & - & - & & 0.53 & 0.49 & & 0.77 & 0.77 & & - & - & %& 23.1 & 
		& - \\
		\midrule
		\AR & \textbf{-5.06} & -4.99 & &  -5.59 & -5.29 & &  -6.16 & -6.05 & &  37.69 & 31.00 & &  0.50 & 0.49 & &  0.66 & 0.65 & & 0.70 & 0.70 & %& 7.0 & 
		& 7,789 \\
		\pockettwomol   & -4.50 & -4.21 & &  -5.70 & -5.27 & &  -6.43 & -6.25 & &  48.00 & 51.00 & &  0.58 & 0.58 & &  \textbf{0.77} & \textbf{0.78} & &  0.69 & 0.71 &  %& 24.9 & 
		& 2,150 \\
		\targetdiff     & -4.88 & \textbf{-5.82} & &  \textbf{-6.20} & \textbf{-6.36} & &  \textbf{-7.37} & \textbf{-7.51} & &  57.57 & 58.27 & &  0.50 & 0.51 & &  0.60 & 0.59 & &  \textbf{0.72} & 0.71 & % & 10.4 & 
		& 1,252 \\
		%& \decompdiffbeta                    & 63.03 & %-4.72 & -4.86 & & \textbf{-6.84} & \textbf{-6.91} & & \textbf{-8.85} & \textbf{-8.90} & &  \textbf{72.16} & \textbf{72.16} & &  0.36 & 0.36 & &  0.55 & 0.55 & & 0.59 & 0.59 & & 14.9 \\ 
		%-4.76 & -6.18 & &  \textbf{-6.86} & \textbf{-7.52} & &  \textbf{-8.85} & \textbf{-8.96} & &  \textbf{72.7} & \textbf{89.8} & &  0.36 & 0.34 & &  0.55 & 0.57 & & 0.59 & 0.59 & & 15.4 \\
		\decompdiffref  & -4.58 & -4.77 & &  -5.47 & -5.51 & &  -6.43 & -6.56 & &  47.76 & 48.66 & &  0.56 & 0.56 & &  0.70 & 0.69  & &  \textbf{0.72} & \textbf{0.72} &  %& 15.2 & 
		& 1,859 \\
		\midrule
		%\method & 14.04 & 9.74 & &  -2.80 & -3.87 & &  -6.32 & -6.41 & &  42.37 & 40.40 & &  0.70 & 0.71 & &  0.73 & 0.72 & & 0.71 & 0.74 & & 42 \\
		%\methodwithsguide & 1.04 & -0.33 & &  -4.23 & -4.39 & &  -6.31 & -6.46 & &  46.18 & 44.00 & &  0.69 & 0.71 & &  0.72 & 0.71 & & 0.70 & 0.73 & 53 \\
		\methodwithpguide      & -4.15 & -4.59 & &  -5.41 & -5.34 & &  -6.49 & -6.74 & &  \textbf{58.52} & 59.00 & &  \textbf{0.67} & \textbf{0.69} & &  0.68 & 0.68 & & 0.67 & 0.70 & %& 28.0 & 
		& 48 \\
		\methodwithsandpguide  & -4.56 & -4.82 & &  -5.53 & -5.47 & &  -6.60 & -6.78 & &  58.28 & \textbf{60.00} & &  0.66 & 0.68 & &  0.67 & 0.66 & & 0.68 & 0.71 &
		& 58 \\
		\bottomrule
	\end{tabular}%
	\begin{tablenotes}[normal,flushleft]
		\begin{footnotesize}
	\item 
\!\!Columns represent: {``Vina S'': the binding affinities between the initially generated poses of molecules and the protein pockets; 
		``Vina M'': the binding affinities between the poses after local structure minimization and the protein pockets;
		``Vina D'': the binding affinities between the poses determined by AutoDock Vina~\cite{Eberhardt2021} and the protein pockets;
		``HA'': the percentage of generated molecules with Vina D higher than those of condition molecules;
		``QED'': the drug-likeness score;
		``SA'': the synthesizability score;
		``Div'': the diversity among generated molecules;
		``time'': the time cost to generate molecules.}
		\par
		\par
		\end{footnotesize}
	\end{tablenotes}
	\end{scriptsize}
\end{threeparttable}
\end{table*}


%\label{tbl:overall_results_docking_100}

%-------------------------------------------------------------------------------------------------------------------------------------
%\subsection{{Comparison of Pocket Guidance}}
%\label{supp:app:results:docking}
%-------------------------------------------------------------------------------------------------------------------------------------


\begin{comment}
%-------------------------------------------------------------------------------------------------------------------------------------
\subsection{\ziqi{Simiarity Comparison for Pocket-based Molecule Generation}}
%-------------------------------------------------------------------------------------------------------------------------------------


\begin{table*}[t!]
	\centering
	\caption{{Overall Comparison on Similarity for Pocket-based Molecule Generation}}
	\label{tbl:docking_results_similarity}
	\begin{small}
		\begin{threeparttable}
			\begin{tabular}{
					@{\hspace{0pt}}l@{\hspace{5pt}}
					%
					@{\hspace{3pt}}l@{\hspace{3pt}}
					%
					@{\hspace{3pt}}r@{\hspace{8pt}}
					@{\hspace{3pt}}c@{\hspace{3pt}}
					%
					@{\hspace{3pt}}c@{\hspace{3pt}}
					@{\hspace{3pt}}c@{\hspace{3pt}}
					%
					@{\hspace{0pt}}c@{\hspace{0pt}}
					%
					@{\hspace{3pt}}c@{\hspace{3pt}}
					@{\hspace{3pt}}c@{\hspace{3pt}}
					%
					@{\hspace{3pt}}r@{\hspace{3pt}}
				}
				\toprule
				$\delta_g$  & method          & \#d\%$\uparrow$ & $\diversity_d$$\uparrow$(std) & \avgshapesim$\uparrow$(std) & \avggraphsim$\downarrow$(std) & & \maxshapesim$\uparrow$(std) & \maxgraphsim$\downarrow$(std)       & \#n\%$\uparrow$  \\ 
				\midrule
				%\multirow{6}{0.059\linewidth}{\hspace{0pt}0.1} 
				%& \AR   & 4.4 & 0.781(0.076) & 0.511(0.197) & \textbf{0.056}(0.020) & & 0.619(0.222) & 0.074(0.024) & 21.4  \\
				%& \pockettwomol & 6.6 & 0.795(0.099) & 0.519(0.216) & 0.063(0.020) & & 0.608(0.236) & 0.076(0.022) & \textbf{24.1}  \\
				%& \targetdiff & 2.0 & 0.872(0.041) & 0.619(0.110) & 0.068(0.018) & & 0.721(0.146) & 0.075(0.023) & 17.7  \\
				%& \decompdiffbeta & 0.0 & - & 0.374(0.138) & 0.059(0.031) & & 0.414(0.141) & \textbf{0.058}(0.032) & 9.8  \\
				%& \decompdiffref & 8.5 & 0.805(0.096) & 0.810(0.070) & 0.076(0.018) & & 0.861(0.085) & 0.076(0.020) & 11.3  \\
				%& \methodwithpguide   &  9.9 & \textbf{0.876}(0.041) & 0.795(0.058) & 0.073(0.015) & & 0.869(0.073) & 0.076(0.020) & 17.7  \\
				%& \methodwithsandpguide & \textbf{11.9} & 0.872(0.036) & \textbf{0.813}(0.052) & 0.075(0.014) & & \textbf{0.874}(0.069) & 0.080(0.014) & 17.0  \\
				%\cmidrule{2-10}
				%& improv\% & 40.4$^*$ & 8.8$^*$ & 0.4 & -30.4$^*$ &  & 1.6 & -30.0$^*$ & -26.3$^*$  \\
				%\midrule
				\multirow{7}{0.059\linewidth}{\hspace{0pt}1.0} 
				& \AR & 14.6 & 0.681(0.163) & 0.644(0.119) & 0.236(0.123) & & 0.780(0.110) & 0.284(0.177) & 95.8  \\
				& \pockettwomol & 18.6 & 0.711(0.152) & 0.654(0.131) &   \textbf{0.217}(0.129) & & 0.778(0.121) &   \textbf{0.243}(0.137) &  \textbf{98.3}  \\
				& \targetdiff & 7.1 &  \textbf{0.785}(0.085) & 0.622(0.083) & 0.238(0.122) & & 0.790(0.102) & 0.274(0.158) & 90.4  \\
				%& \decompdiffbeta & 0.1 & 0.589(0.030) & 0.494(0.124) & 0.263(0.143) & & 0.567(0.143) & 0.275(0.162) & 67.7  \\
				& \decompdiffref & 37.3 & 0.721(0.108) & 0.770(0.087) & 0.282(0.130) & & \textbf{0.878}(0.059) & 0.343(0.174) & 83.7  \\
				& \methodwithpguide   &  27.4 & 0.757(0.134) & 0.747(0.078) & 0.265(0.165) & & 0.841(0.081) & 0.272(0.168) & 98.1  \\
				& \methodwithsandpguide &\textbf{45.2} & 0.724(0.142) &   \textbf{0.789}(0.063) & 0.265(0.162) & & 0.876(0.062) & 0.264(0.159) & 97.8  \\
				\cmidrule{2-10}
				& Improv\%  & 21.2$^*$ & -3.6 & 2.5$^*$ & -21.7$^*$ &  & -0.1 & -8.4$^*$ & -0.2  \\
				\midrule
				\multirow{7}{0.059\linewidth}{\hspace{0pt}0.7} 
				& \AR   & 14.5 & 0.692(0.151) & 0.644(0.119) & 0.233(0.116) & & 0.779(0.110) & 0.266(0.140) & 94.9  \\
				& \pockettwomol & 18.6 & 0.711(0.152) & 0.654(0.131) & \textbf{0.217}(0.129) & & 0.778(0.121) & \textbf{0.243}(0.137) & \textbf{98.2}  \\
				& \targetdiff & 7.1 & \textbf{0.786}(0.084) & 0.622(0.083) & 0.238(0.121) & & 0.790(0.101) & 0.270(0.151) & 90.3  \\
				%& \decompdiffbeta & 0.1 & 0.589(0.030) & 0.494(0.124) & 0.263(0.142) & &0.567(0.143) & 0.273(0.156) & 67.6  \\
				& \decompdiffref & 36.2 & 0.721(0.113) & 0.770(0.086) & 0.273(0.123) & & \textbf{0.876}(0.059) & 0.325(0.139) & 82.3  \\
				& \methodwithpguide   &  27.4 & 0.757(0.134) & 0.746(0.078) & 0.263(0.160) & & 0.841(0.081) & 0.271(0.164) & 96.8  \\
				& \methodwithsandpguide      & \textbf{45.0} & 0.732(0.129) & \textbf{0.789}(0.063) & 0.262(0.157) & & \textbf{0.876}(0.063) & 0.262(0.153) & 96.2  \\
				\cmidrule{2-10}
				& Improv\%  & 24.3$^*$ & -3.6 & 2.5$^*$ & -20.8$^*$ &  & 0.0 & -7.6$^*$ & -1.5  \\
				\midrule
				\multirow{7}{0.059\linewidth}{\hspace{0pt}0.5} 
				& \AR   & 14.1 & 0.687(0.160) & 0.639(0.124) & 0.218(0.097) & & 0.778(0.110) & 0.260(0.130) & 89.8  \\
				& \pockettwomol & 18.5 & 0.711(0.152) & 0.649(0.134) & \textbf{0.209}(0.114) & & 0.777(0.121) & \textbf{0.240}(0.131) & \textbf{93.2}  \\
				& \targetdiff & 7.1 & \textbf{0.786}(0.084) & 0.621(0.083) & 0.230(0.111) & & 0.788(0.105) & 0.254(0.127) & 86.5  \\
				%&\decompdiffbeta & 0.1 & 0.595(0.025) & 0.494(0.124) & 0.254(0.129) & & 0.565(0.142) & 0.259(0.138) & 63.9  \\
				& \decompdiffref & 34.7 & 0.730(0.105) & 0.769(0.086) & 0.261(0.109) & & 0.874(0.080) & 0.301(0.117) & 77.3   \\
				& \methodwithpguide  &  27.2 & 0.765(0.123) & 0.749(0.075) & 0.245(0.135) & & 0.840(0.082) & 0.252(0.137) & 88.6  \\
				& \methodwithsandpguide & \textbf{44.3} & 0.738(0.122) & \textbf{0.791}(0.059) & 0.247(0.132) &  & \textbf{0.875}(0.065) & 0.249(0.130) & 88.8  \\
				\cmidrule{2-10}
				& Improv\%   & 27.8$^*$ & -2.7 & 2.9$^*$ & -17.6$^*$ &  & 0.2 & -3.4 & -4.7$^*$  \\
				\midrule
				\multirow{7}{0.059\linewidth}{\hspace{0pt}0.3} 
				& \AR   & 12.2 & 0.704(0.146) & 0.614(0.146) & 0.164(0.059) & & 0.751(0.138) & 0.206(0.059) & 66.4  \\
				& \pockettwomol & 17.1 & 0.731(0.129) & 0.617(0.163) & \textbf{0.155}(0.056) & & 0.740(0.159) & \textbf{0.190}(0.076) & \textbf{71.0}  \\
				& \targetdiff & 6.2 & \textbf{0.809}(0.061) & 0.619(0.087) & 0.181(0.068) & & 0.768(0.119) & 0.196(0.076) & 61.7  \\				
                %& \decompdiffbeta & 0.0 & - & 0.489(0.124) & 0.195(0.080) & & 0.547(0.139) & 0.203(0.087) & 42.0  \\
				& \decompdiffref & 27.7 & 0.775(0.081) & 0.767(0.086) & 0.202(0.062) & & 0.854(0.093) & 0.216(0.068) & 52.6  \\
				& \methodwithpguide   &  24.4 & 0.805(0.084) & 0.763(0.066) & 0.180(0.074) & & 0.847(0.080) & \textbf{0.190}(0.059) & 61.4  \\
				& \methodwithsandpguide & \textbf{36.3} & 0.789(0.081) & \textbf{0.800}(0.056) & 0.181(0.071) & &\textbf{0.878}(0.067) & \textbf{0.190}(0.078) & 61.8  \\
				\cmidrule{2-10}
				& improv\% & 31.1$^*$ & 3.9$^*$ & 4.3$^*$ & -16.5$^*$ &  & 2.8$^*$ & 0.0 & -12.9$^*$  \\
				\bottomrule
			\end{tabular}%
			\begin{tablenotes}[normal,flushleft]
				\begin{footnotesize}
					\item 
					\!\!Columns represent: \ziqi{``$\delta_g$'': the graph similarity constraint; ``\#n\%'': the percentage of molecules that satisfy the graph similarity constraint ($\graphsim<=\delta_g$);
						``\#d\%'': the percentage of molecules that satisfy the graph similarity constraint and are with high \shapesim ($\shapesim>=0.8$);
						``\avgshapesim/\avggraphsim'': the average of shape or graph similarities between the condition molecules and generated molecules with $\graphsim<=\delta_g$;
						``\maxshapesim'': the maximum of shape similarities between the condition molecules and generated molecules with $\graphsim<=\delta_g$;
						``\maxgraphsim'': the graph similarities between the condition molecules and the molecules with the maximum shape similarities and $\graphsim<=\delta_g$;
						``\diversity'': the diversity among the generated molecules.
						%
						``$\uparrow$'' represents higher values are better, and ``$\downarrow$'' represents lower values are better.
						%
						Best values are in \textbf{bold}, and second-best values are \underline{underlined}. 
					} 
					%\todo{double-check the significance value}
					\par
					\par
				\end{footnotesize}
			\end{tablenotes}
		\end{threeparttable}
	\end{small}
	\vspace{-10pt}    
\end{table*}
%\label{tbl:docking_results_similarity}

\bo{@Ziqi you may want to check my edits for the discussion in Table 1 first.
%
If the pocket if known, do you still care about the shape similarity in real applications?
}

\ziqi{Table~\ref{tbl:docking_results_similarity} presents the overall comparison on similarity-based metrics between \methodwithpguide, \methodwithsandpguide and other baselines under different graph similarity constraints  ($\delta_g$=1.0, 0.7, 0.5, 0.3), similar to Table~\ref{tbl:overall}. 
%
As shown in Table~\ref{tbl:docking_results_similarity}, regarding desirable molecules,  \methodwithsandpguide consistently outperforms all the baseline methods in the likelihood of generating desirable molecules (i.e., $\#d\%$).
%
For example, when $\delta_g$=1.0, at $\#d\%$, \methodwithsandpguide (45.2\%) demonstrates significant improvement of $21.2\%$ compared to the best baseline \decompdiff (37.3\%).
%
In terms of $\diversity_d$, \methodwithpguide and \methodwithsandpguide also achieve the second and the third best performance. 
%
Note that the best baseline \targetdiff in $\diversity_d$ achieves the least percentage of desirable molecules (7.1\%), substantially lower than \methodwithpguide and \methodwithsandpguide.
%
This makes its diversity among desirable molecules incomparable with other methods. 
%
When $\delta_g$=0.7, 0.5, and 0.3, \methodwithsandpguide also establishes a significant improvement of 24.3\%, 27.8\%, and 31.1\% compared to the best baseline method \decompdiff.
%
It is also worth noting that the state-of-the-art baseline \decompdiff underperforms \methodwithpguide and \methodwithsandpguide in binding affinities as shown in Table~\ref{tbl:overall_results_docking}, even though it outperforms \methodwithpguide in \#d\%.
%
\methodwithpguide and \methodwithsandpguide also achieve the second and the third best performance in $\diversity_d$ at $\delta_g$=0.7, 0.5, and 0.3. 
%
The superior performance of \methodwithpguide and \methodwithsandpguide in $\#d\%$ at small $\delta_g$ indicates their strong capacity in generating desirable molecules of novel graph structures, thereby facilitating the discovery of novel drug candidates.
%
}

\ziqi{Apart from the desirable molecules, \methodwithpguide and \methodwithsandpguide also demonstrate outstanding performance in terms of the average shape similarities (\avgshapesim) and the average graph similarities (\avggraphsim).
%
Specifically, when $\delta_g$=1.0, \methodwithsandpguide achieves a significant 2.5\% improvement in \avgshapesim\ over the best baseline \decompdiff. 
%
In terms of \avggraphsim, \methodwithsandpguide also achieves higher performance than the baseline \decompdiff of the highest \avgshapesim (0.265 vs 0.282).
%
Please note that all the baseline methods except \decompdiff achieve substantially lower performance in \avgshapesim than \methodwithpguide and \methodwithsandpguide, even though these methods achieve higher \avggraphsim values.
%
This trend remains consistent when applying various similarity constraints (i.e., $\delta_g$) as shown in Table~\ref{tbl:overall_results_docking}.
}

\ziqi{Similarly, \methodwithpguide and \methodwithsandpguide also achieve superior performance in \maxshapesim and \maxgraphsim.
%
Specifically, when $\delta_g$=1.0, for \maxshapesim, \methodwithsandpguide achieves highly comparable performance in \maxshapesim\ compared to the best baseline \decompdiff (0.876 vs 0.878).
%
We also note that \methodwithsandpguide achieves lower \maxgraphsim\ than the \decompdiff with 23.0\% difference. 
%
When $\delta_g$ gets smaller from 0.7 to 0.3, \methodwithsandpguide maintains a high \maxshapesim value around 0.876, while the best baseline \decompdiff has \maxshapesim decreased from 0.878 to 0.854.
%
This demonstrates the superior ability of \methodwithsandpguide in generating molecules with similar shapes and novel structures.
%
}

\ziqi{
In terms of \#n\%, when $\delta_g$=1.0, the percentage of molecules with \graphsim below $\delta_g$ can be interpreted as the percentage of valid molecules among all the generated molecules. 
%
As shown in Table~\ref{tbl:docking_results_similarity}, \methodwithpguide and \methodwithsandpguide are able to generate 98.1\% and 97.8\% of valid molecules, slightly below the best baseline \pockettwomol (98.3\%). 
%
When $\delta_g$=0.7, 0.5, or 0.3, all the methods, including \methodwithpguide and \methodwithsandpguide, can consistently find a sufficient number of novel molecules that meet the graph similarity constraints.
%
The only exception is \decompdiff, which substantially underperforms all the other methods in \#n\%.
}
\end{comment}

%%%%%%%%%%%%%%%%%%%%%%%%%%%%%%%%%%%%%%%%%%%%%
\section{Properties of Molecules in Case Studies for Targets}
\label{supp:app:results:properties}
%%%%%%%%%%%%%%%%%%%%%%%%%%%%%%%%%%%%%%%%%%%%%

%-------------------------------------------------------------------------------------------------------------------------------------
\subsection{Drug Properties of Generated Molecules}
\label{supp:app:results:properties:drug}
%-------------------------------------------------------------------------------------------------------------------------------------

Table~\ref{tbl:drug_property} presents the drug properties of three generated molecules: NL-001, NL-002, and NL-003.
%
As shown in Table~\ref{tbl:drug_property}, each of these molecules has a favorable profile, making them promising drug candidates. 
%
{As discussed in Section ``Case Studies for Targets'' in the main manuscript, all three molecules have high binding affinities in terms of Vina S, Vina M and Vina D, and favorable QED and SA values.
%
In addition, all of them meet the Lipinski's rule of five criteria~\cite{Lipinski1997}.}
%
In terms of physicochemical properties, all these properties of NL-001, NL-002 and NL-003, including number of rotatable bonds, molecule weight, LogP value, number of hydrogen bond doners and acceptors, and molecule charges, fall within the desired range of drug molecules. 
%
This indicates that these molecules could potentially have good solubility and membrane permeability, essential qualities for effective drug absorption.

These generated molecules also demonstrate promising safety profiles based on the predictions from ICM~\cite{Neves2012}.
%
In terms of drug-induced liver injury prediction scores, all three molecules have low scores (0.188 to 0.376), indicating a minimal risk of hepatotoxicity. 
%
NL-001 and NL-002 fall under `Ambiguous/Less concern' for liver injury, while NL-003 is categorized under 'No concern' for liver injury. 
%
Moreover, all these molecules have low toxicity scores (0.000 to 0.236). 
%
NL-002 and NL-003 do not have any known toxicity-inducing functional groups. 
%
NL-001 and NL-003 are also predicted not to include any known bad groups that lead to inappropriate features.
%
These attributes highlight the potential of NL-001, NL-002, and NL-003 as promising treatments for cancers and Alzheimer’s disease.

%\begin{table*}
	\centering
		\caption{Drug Properties of Generated Molecules}
	\label{tbl:binding_drug_mols}
	\begin{scriptsize}
\begin{threeparttable}
	\begin{tabular}{
		@{\hspace{6pt}}r@{\hspace{6pt}}
		@{\hspace{6pt}}r@{\hspace{6pt}}
		@{\hspace{6pt}}r@{\hspace{6pt}}
		@{\hspace{6pt}}r@{\hspace{6pt}}
		@{\hspace{6pt}}r@{\hspace{6pt}}
		@{\hspace{6pt}}r@{\hspace{6pt}}
		@{\hspace{6pt}}r@{\hspace{6pt}}
		@{\hspace{6pt}}r@{\hspace{6pt}}
		@{\hspace{6pt}}r@{\hspace{6pt}}
		%
		}
		\toprule
Target & Molecule & Vina S & Vina M & Vina D & QED   & SA   & Logp  & Lipinski \\
\midrule
\multirow{3}{*}{CDK6} & NL-001 & -6.817      & -7.251    & -8.319     & 0.834 & 0.72 & 1.313 & 5        \\
& NL-002 & -6.970       & -7.605    & -8.986     & 0.851 & 0.74 & 3.196 & 5        \\
\cmidrule{2-9}
& 4AU & 0.736       & -5.939    & -7.592     & 0.773 & 0.79 & 2.104 & 5        \\
\midrule
\multirow{2}{*}{NEP} & NL-003 & -11.953     & -12.165   & -12.308    & 0.772 & 0.57 & 2.944 & 5        \\
\cmidrule{2-9}
& BIR & -9.399      & -9.505    & -9.561     & 0.463 & 0.73 & 2.677 & 5        \\
		\bottomrule
	\end{tabular}%
	\begin{tablenotes}[normal,flushleft]
		\begin{footnotesize}
	\item Columns represent: {``Target'': the names of protein targets;
		``Molecule'': the names of generated molecules and known ligands;
		``Vina S'': the binding affinities between the initially generated poses of molecules and the protein pockets; 
		``Vina M'': the binding affinities between the poses after local structure minimization and the protein pockets;
		``Vina D'': the binding affinities between the poses determined by AutoDock Vina~\cite{Eberhardt2021} and the protein pockets;
		``HA'': the percentage of generated molecules with Vina D higher than those of condition molecules;
		``QED'': the drug-likeness score;
		``SA'': the synthesizability score;
		``Div'': the diversity among generated molecules;
		``time'': the time cost to generate molecules.}
\!\! \par
		\par
		\end{footnotesize}
	\end{tablenotes}
\end{threeparttable}
\end{scriptsize}
  \vspace{-10pt}    
\end{table*}

%\label{tbl:binding_drug_mols}

\begin{table*}
	\centering
		\caption{Drug Properties of Generated Molecules}
	\label{tbl:drug_property}
	\begin{scriptsize}
\begin{threeparttable}
	\begin{tabular}{
		@{\hspace{0pt}}p{0.23\linewidth}@{\hspace{5pt}}
		%
		@{\hspace{1pt}}r@{\hspace{2pt}}
		@{\hspace{2pt}}r@{\hspace{6pt}}
		@{\hspace{6pt}}r@{\hspace{6pt}}
		%
		}
		\toprule
		Property Name & NL-001 & NL-002 & NL-003 \\
		\midrule
Vina S & -6.817 &  -6.970 & -11.953 \\
Vina M & -7.251 & -7.605 & -12.165 \\
Vina D & -8.319 & -8.986 & -12.308 \\
QED    & 0.834  & 0.851  & 0.772 \\
SA       & 0.72    & 0.74    & 0.57    \\
Lipinski & 5 & 5 & 5 \\
%bbbScore          & 3.386                                                                                        & 4.240                                                                                        & 3.892      \\
%drugLikeness      & -0.081                                                                                       & -0.442                                                                                       & -0.325     \\
%molLogP1          & 1.698                                                                                        & 2.685                                                                                        & 2.382      \\
\#rotatable bonds          & 3                                                                                        & 2                                                                                        & 2      \\
molecule weight         & 267.112                                                                                      & 270.117                                                                                      & 390.206    \\
molecule LogP           & 1.698                                                                                        & 2.685                                                                                        & 2.382     \\
\#hydrogen bond doners           & 1                                                                                        & 1                                                                                        & 2      \\
\#hydrogen bond acceptors           & 5                                                                                       & 3                                                                                        & 5      \\
\#molecule charges   & 1                                                                                        & 0                                                                                        & 0      \\
drug-induced liver injury predScore    & 0.227                                                                                        & 0.376                                                                                        & 0.188      \\
drug-induced liver injury predConcern  & Ambiguous/Less concern                                                                       & Ambiguous/Less concern                                                                       & No concern \\
drug-induced liver injury predLabel    & Warnings/Precautions/Adverse reactions & Warnings/Precautions/Adverse reactions & No match   \\
drug-induced liver injury predSeverity & 2                                                                                        & 3                                                                                        & 2      \\
%molSynth1         & 0.254                                                                                        & 0.220                                                                                        & 0.201      \\
%toxicity class         & 0.480                                                                                        & 0.480                                                                                        & 0.450      \\
toxicity names         & hydrazone                                                                                    &   -                                                                                           &   -         \\
toxicity score         & 0.236                                                                                        & 0.000                                                                                        & 0.000      \\
bad groups         & -                                                                                             & Tetrahydroisoquinoline:   allergies                                                          &   -         \\
%MolCovalent       &                                                                                              &                                                                                              &            \\
%MolProdrug        &                                                                                              &                                                                                              &            \\
		\bottomrule
	\end{tabular}%
	\begin{tablenotes}[normal,flushleft]
		\begin{footnotesize}
	\item ``-'': no results found by algorithms
\!\! \par
		\par
		\end{footnotesize}
	\end{tablenotes}
\end{threeparttable}
\end{scriptsize}
  \vspace{-10pt}    
\end{table*}

%\label{tbl:drug_property}

%-------------------------------------------------------------------------------------------------------------------------------------
\subsection{Comparison on ADMET Profiles between Generated Molecules and Approved Drugs}
\label{supp:app:results:properties:admet}
%-------------------------------------------------------------------------------------------------------------------------------------

\paragraph{Generated Molecules for CDK6}
%
Table~\ref{tbl:admet_cdk6} presents the comparison on ADMET profiles between two generated molecules for CDK6 and the approved CDK6 inhibitors, including Abemaciclib~\cite{Patnaik2016}, Palbociclib~\cite{Lu2015}, and Ribociclib~\cite{Tripathy2017}.
%
As shown in Table~\ref{tbl:admet_cdk6}, the generated molecules, NL-001 and NL-002, exhibit comparable ADMET profiles with those of the approved CDK6 inhibitors. 
%
Importantly, both molecules demonstrate good potential in most crucial properties, including Ames mutagenesis, favorable oral toxicity, carcinogenicity, estrogen receptor binding, high intestinal absorption and favorable oral bioavailability.
%
Although the generated molecules are predicted as positive in hepatotoxicity and mitochondrial toxicity, all the approved drugs are also predicted as positive in these two toxicity.
%
This result suggests that these issues might stem from the limited prediction accuracy rather than being specific to our generated molecules.
%
Notably, NL-001 displays a potentially better plasma protein binding score compared to other molecules, which may improve its distribution within the body. 
%
Overall, these results indicate that NL-001 and NL-002 could be promising candidates for further drug development.


\begin{table*}
	\centering
		\caption{Comparison on ADMET Profiles among Generated Molecules and Approved Drugs Targeting CDK6}
	\label{tbl:admet_cdk6}
	\begin{scriptsize}
\begin{threeparttable}
	\begin{tabular}{
		%@{\hspace{0pt}}p{0.23\linewidth}@{\hspace{5pt}}
		%
		@{\hspace{6pt}}l@{\hspace{5pt}}
		@{\hspace{6pt}}r@{\hspace{6pt}}
		@{\hspace{6pt}}r@{\hspace{6pt}}
		@{\hspace{6pt}}r@{\hspace{6pt}}
		@{\hspace{6pt}}r@{\hspace{6pt}}
		@{\hspace{6pt}}r@{\hspace{6pt}}
		%
		%
		@{\hspace{6pt}}r@{\hspace{6pt}}
		%@{\hspace{6pt}}r@{\hspace{6pt}}
		%
		}
		\toprule
		\multirow{2}{*}{Property name} & \multicolumn{2}{c}{Generated molecules} & & \multicolumn{3}{c}{FDA-approved drugs} \\
		\cmidrule{2-3}\cmidrule{5-7}
		 & NL--001 & NL--002 & & Abemaciclib & Palbociclib & Ribociclib \\
		\midrule
\rowcolor[HTML]{D2EAD9}Ames   mutagenesis                             & --   &  --  & & + &  --  & --  \\
\rowcolor[HTML]{D2EAD9}Acute oral toxicity (c)                           & III & III & &  III          & III          & III         \\
Androgen receptor binding                         & +                          & +            &              & +            & +            & +             \\
Aromatase binding                                 & +                          & +            &              & +            & +            & +            \\
Avian toxicity                                    & --                          & --          &                & --            & --            & --            \\
Blood brain barrier                               & +                          & +            &              & +            & +            & +            \\
BRCP inhibitior                                   & --                          & --          &                & --            & --            & --            \\
Biodegradation                                    & --                          & --          &                & --            & --            & --           \\
BSEP inhibitior            & +                          & +            &              & +            & +            & +        \\
Caco-2                                            & +                          & +            &              & --            & --            & --            \\
\rowcolor[HTML]{D2EAD9}Carcinogenicity (binary)                          & --                          & --             &             & --            & --            & --          \\
\rowcolor[HTML]{D2EAD9}Carcinogenicity (trinary)                         & Non-required               & Non-required   &            & Non-required & Non-required & Non-required  \\
Crustacea aquatic toxicity & --                          & --            &              & --            & --            & --            \\
 CYP1A2 inhibition                                 & +                          & +            &              & --            & --            & +             \\
CYP2C19 inhibition                                & --                          & +            &              & +            & --            & +            \\
CYP2C8 inhibition                                 & --                          & --           &               & +            & +            & +            \\
CYP2C9 inhibition                                 & --                          & --           &               & --            & --            & +             \\
CYP2C9 substrate                                  & --                          & --           &               & --            & --            & --            \\
CYP2D6 inhibition                                 & --                          & --           &               & --            & --            & --            \\
CYP2D6 substrate                                  & --                          & --           &               & --            & --            & --            \\
CYP3A4 inhibition                                 & --                          & +            &              & --            & --            & --            \\
CYP3A4 substrate                                  & +                          & --            &              & +            & +            & +            \\
\rowcolor[HTML]{D2EAD9}CYP inhibitory promiscuity                        & +                          & +                    &      & +            & --            & +            \\
Eye corrosion                                     & --                          & --           &               & --            & --            & --            \\
Eye irritation                                    & --                          & --           &               & --            & --            & --             \\
\rowcolor[HTML]{D8E7FF}Estrogen receptor binding                         & +                          & +                    &      & +            & +            & +            \\
Fish aquatic toxicity                             & --                          & +            &              & +            & --            & --            \\
Glucocorticoid receptor   binding                 & +                          & +             &             & +            & +            & +            \\
Honey bee toxicity                                & --                          & --           &               & --            & --            & --            \\
\rowcolor[HTML]{D2EAD9}Hepatotoxicity                                    & +                          & +            &              & +            & +            & +             \\
Human ether-a-go-go-related gene inhibition     & +                          & +               &           & +            & --            & --           \\
\rowcolor[HTML]{D8E7FF}Human intestinal absorption                       & +                          & +             &             & +            & +            & +    \\
\rowcolor[HTML]{D8E7FF}Human oral bioavailability                        & +                          & +              &            & +            & +            & +     \\
\rowcolor[HTML]{D2EAD9}MATE1 inhibitior                                  & --                          & --              &            & --            & --            & --    \\
\rowcolor[HTML]{D2EAD9}Mitochondrial toxicity                            & +                          & +                &          & +            & +            & +    \\
Micronuclear                                      & +                          & +                          & +            & +            & +           \\
\rowcolor[HTML]{D2EAD9}Nephrotoxicity                                    & --                          & --             &             & --            & --            & --             \\
Acute oral toxicity                               & 2.325                      & 1.874    &     & 1.870        & 3.072        & 3.138        \\
\rowcolor[HTML]{D8E7FF}OATP1B1 inhibitior                                & +                          & +              &            & +            & +            & +             \\
\rowcolor[HTML]{D8E7FF}OATP1B3 inhibitior                                & +                          & +              &            & +            & +            & +             \\
\rowcolor[HTML]{D2EAD9}OATP2B1 inhibitior                                & --                          & --             &             & --            & --            & --             \\
OCT1 inhibitior                                   & --                          & --        &                  & +            & --            & +             \\
OCT2 inhibitior                                   & --                          & --        &                  & --            & --            & +             \\
P-glycoprotein inhibitior                         & --                          & --        &                  & +            & +            & +     \\
P-glycoprotein substrate                          & --                          & --        &                  & +            & +            & +     \\
PPAR gamma                                        & +                          & +          &                & +            & +            & +      \\
\rowcolor[HTML]{D8E7FF}Plasma protein binding                            & 0.359        & 0.745     &    & 0.865        & 0.872        & 0.636       \\
Reproductive toxicity                             & +                          & +          &                & +            & +            & +           \\
Respiratory toxicity                              & +                          & +          &                & +            & +            & +         \\
Skin corrosion                                    & --                          & --        &                  & --            & --            & --           \\
Skin irritation                                   & --                          & --        &                  & --            & --            & --         \\
Skin sensitisation                                & --                          & --        &                  & --            & --            & --          \\
Subcellular localzation                           & Mitochondria               & Mitochondria  &             & Lysosomes    & Mitochondria & Mitochondria \\
Tetrahymena pyriformis                            & 0.398                      & 0.903         &             & 1.033        & 1.958        & 1.606         \\
Thyroid receptor binding                          & +                          & +             &             & +            & +            & +           \\
UGT catelyzed                                     & --                          & --           &               & --            & --            & --           \\
\rowcolor[HTML]{D8E7FF}Water solubility                                  & -3.050                     & -3.078              &       & -3.942       & -3.288       & -2.673     \\
		\bottomrule
	\end{tabular}%
	\begin{tablenotes}[normal,flushleft]
		\begin{footnotesize}
	\item Blue cells highlight crucial properties where a negative outcome (``--'') is desired; for acute oral toxicity (c), a higher category (e.g., ``III'') is desired; and for carcinogenicity (trinary), ``Non-required'' is desired.
	%
	Green cells highlight crucial properties where a positive result (``+'') is desired; for plasma protein binding, a lower value is desired; and for water solubility, values higher than -4 are desired~\cite{logs}.
\!\! \par
		\par
		\end{footnotesize}
	\end{tablenotes}
\end{threeparttable}
\end{scriptsize}
  \vspace{--10pt}    
\end{table*}

%\label{tbl:admet_cdk6}

\paragraph{Generated Molecule for NEP}
%
Table~\ref{tbl:admet_nep} presents the comparison on ADMET profiles between a generated molecule for NEP targeting Alzheimer's disease and three approved drugs, Donepezil, Galantamine, and Rivastigmine, for Alzheimer's disease~\cite{Hansen2008}.
%
Overall, NL-003 exhibits a comparable ADMET profile with the three approved drugs.  
%
Notably, same as other approved drugs, NL-003 is predicted to be able to penetrate the blood brain barrier, a crucial property for Alzheimer's disease.
%  
In addition, it demonstrates a promising safety profile in terms of Ames mutagenesis, favorable oral toxicity, carcinogenicity, estrogen receptor binding, high intestinal absorption, nephrotoxicity and so on.
%
These results suggest that NL-003 could be promising candidates for the drug development of Alzheimer's disease.

\begin{table*}
	\centering
		\caption{Comparison on ADMET Profiles among Generated Molecule Targeting NEP and Approved Drugs for Alzhimer's Disease}
	\label{tbl:admet_nep}
	\begin{scriptsize}
\begin{threeparttable}
	\begin{tabular}{
		@{\hspace{6pt}}l@{\hspace{5pt}}
		%
		@{\hspace{6pt}}r@{\hspace{6pt}}
		@{\hspace{6pt}}r@{\hspace{6pt}}
		@{\hspace{6pt}}r@{\hspace{6pt}}
		@{\hspace{6pt}}r@{\hspace{6pt}}
		@{\hspace{6pt}}r@{\hspace{6pt}}
		%
		%
		%@{\hspace{6pt}}r@{\hspace{6pt}}
		%
		}
		\toprule
		\multirow{2}{*}{Property name} & Generated molecule & & \multicolumn{3}{c}{FDA-approved drugs} \\
\cmidrule{2-2}\cmidrule{4-6}
			& NL--003 & & Donepezil	& Galantamine & Rivastigmine \\
		\midrule
\rowcolor[HTML]{D2EAD9} 
Ames   mutagenesis                            & --                      &              & --                                    & --                                 & --                     \\
\rowcolor[HTML]{D2EAD9}Acute oral toxicity (c)                       & III           &                       & III                                  & III                               & II                      \\
Androgen receptor binding                     & +      &      & +            & --         & --         \\
Aromatase binding                             & --     &       & +            & --         & --        \\
Avian toxicity                                & --     &                               & --                                    & --                                 & --                        \\
\rowcolor[HTML]{D8E7FF} 
Blood brain barrier                           & +      &                              & +                                    & +                                 & +                        \\
BRCP inhibitior                               & --     &       & --            & --         & --         \\
Biodegradation                                & --     &                               & --                                    & --                                 & --                        \\
BSEP inhibitior                               & +      &      & +            & --         & --         \\
Caco-2                                        & +      &      & +            & +         & +         \\
\rowcolor[HTML]{D2EAD9} 
Carcinogenicity (binary)                      & --     &                               & --                                    & --                                 & --                        \\
\rowcolor[HTML]{D2EAD9} 
Carcinogenicity (trinary)                     & Non-required    &                     & Non-required                         & Non-required                      & Non-required             \\
Crustacea aquatic toxicity                    & +               &                     & +                                    & +                                 & --                        \\
CYP1A2 inhibition                             & +               &                     & +                                    & --                                 & --                        \\
CYP2C19 inhibition                            & +               &                     & --                                    & --                                 & --                        \\
CYP2C8 inhibition                             & +               &                     & --                                    & --                                 & --                        \\
CYP2C9 inhibition                             & --              &                      & --                                    & --                                 & --                        \\
CYP2C9 substrate                              & --              &                      & --                                    & --                                 & --                        \\
CYP2D6 inhibition                             & --              &                      & +                                    & --                                 & --                        \\
CYP2D6 substrate                              & --              &                      & +                                    & +                                 & +                        \\
CYP3A4 inhibition                             & --              &                      & --                                    & --                                 & --                        \\
CYP3A4 substrate                              & +               &                     & +                                    & +                                 & --                        \\
\rowcolor[HTML]{D2EAD9} 
CYP inhibitory promiscuity                    & +               &                     & +                                    & --                                 & --                        \\
Eye corrosion                                 & --     &       & --            & --         & --         \\
Eye irritation                                & --     &       & --            & --         & --         \\
Estrogen receptor binding                     & +      &      & +            & --         & --         \\
Fish aquatic toxicity                         & --     &                               & +                                    & +                                 & +                        \\
Glucocorticoid receptor binding             & --      &      & +            & --         & --         \\
Honey bee toxicity                            & --    &                                & --                                    & --                                 & --                        \\
\rowcolor[HTML]{D2EAD9} 
Hepatotoxicity                                & +     &                               & +                                    & --                                 & --                        \\
Human ether-a-go-go-related gene inhibition & +       &     & +            & --         & --         \\
\rowcolor[HTML]{D8E7FF} 
Human intestinal absorption                   & +     &                               & +                                    & +                                 & +                        \\
\rowcolor[HTML]{D8E7FF} 
Human oral bioavailability                    & --    &                                & +                                    & +                                 & +                        \\
\rowcolor[HTML]{D2EAD9} 
MATE1 inhibitior                              & --    &                                & --                                    & --                                 & --                        \\
\rowcolor[HTML]{D2EAD9} 
Mitochondrial toxicity                        & +     &                               & +                                    & +                                 & +                        \\
Micronuclear                                  & +     &       & --            & --         & +         \\
\rowcolor[HTML]{D2EAD9} 
Nephrotoxicity                                & --    &                                & --                                    & --                                 & --                        \\
Acute oral toxicity                           & 2.704  &      & 2.098        & 2.767     & 2.726     \\
\rowcolor[HTML]{D8E7FF} 
OATP1B1 inhibitior                            & +      &                              & +                                    & +                                 & +                        \\
\rowcolor[HTML]{D8E7FF} 
OATP1B3 inhibitior                            & +      &                              & +                                    & +                                 & +                        \\
\rowcolor[HTML]{D2EAD9} 
OATP2B1 inhibitior                            & --     &                               & --                                    & --                                 & --                        \\
OCT1 inhibitior                               & +      &      & +            & --         & --         \\
OCT2 inhibitior                               & --     &       & +            & --         & --         \\
P-glycoprotein inhibitior                     & +      &      & +            & --         & --         \\
\rowcolor[HTML]{D8E7FF} 
P-glycoprotein substrate                      & +      &                              & +                                    & +                                 & --                        \\
PPAR gamma                                    & +      &      & --            & --         & --         \\
\rowcolor[HTML]{D8E7FF} 
Plasma protein binding                        & 0.227   &                             & 0.883                                & 0.230                             & 0.606                    \\
Reproductive toxicity                         & +       &     & +            & +         & +         \\
Respiratory toxicity                          & +       &     & +            & +         & +         \\
Skin corrosion                                & --      &      & --            & --         & --         \\
Skin irritation                               & --      &      & --            & --         & --         \\
Skin sensitisation                            & --      &      & --            & --         & --         \\
Subcellular localzation                       & Mitochondria & &Mitochondria & Lysosomes & Mitochondria  \\
Tetrahymena pyriformis                        & 0.053           &                     & 0.979                                & 0.563                             & 0.702                        \\
Thyroid receptor binding                      & +       &     & +            & +         & --             \\
UGT catelyzed                                 & --      &      & --            & +         & --             \\
\rowcolor[HTML]{D8E7FF} 
Water solubility                              & -3.586   &                            & -2.425                               & -2.530                            & -3.062                       \\
		\bottomrule
	\end{tabular}%
	\begin{tablenotes}[normal,flushleft]
		\begin{footnotesize}
	\item Blue cells highlight crucial properties where a negative outcome (``--'') is desired; for acute oral toxicity (c), a higher category (e.g., ``III'') is desired; and for carcinogenicity (trinary), ``Non-required'' is desired.
	%
	Green cells highlight crucial properties where a positive result (``+'') is desired; for plasma protein binding, a lower value is desired; and for water solubility, values higher than -4 are desired~\cite{logs}.
\!\! \par
		\par
		\end{footnotesize}
	\end{tablenotes}
\end{threeparttable}
\end{scriptsize}
  \vspace{--10pt}    
\end{table*}

%\label{tbl:admet_nep}

\clearpage
%%%%%%%%%%%%%%%%%%%%%%%%%%%%%%%%%%%%%%%%%%%%%
\section{Algorithms}
\label{supp:algorithms}
%%%%%%%%%%%%%%%%%%%%%%%%%%%%%%%%%%%%%%%%%%%%%

Algorithm~\ref{alg:shapemol} describes the molecule generation process of \method.
%
Given a known ligand \molx, \method generates a novel molecule \moly that has a similar shape to \molx and thus potentially similar binding activity.
%
\method can also take the protein pocket \pocket as input and adjust the atoms of generated molecules for optimal fit and improved binding affinities.
%
Specifically, \method first calculates the shape embedding \shapehiddenmat for \molx using the shape encoder \SEE described in Algorithm~\ref{alg:see_shaperep}.
%
Based on \shapehiddenmat, \method then generates a novel molecule with a similar shape to \molx using the diffusion-based generative model \methoddiff as in Algorithm~\ref{alg:diffgen}.
%
During generation, \method can use shape guidance to directly modify the shape of \moly to closely resemble the shape of \molx.
%
When the protein pocket \pocket is available, \method can also use pocket guidance to ensure that \moly is specifically tailored to closely fit within \pocket.

\begin{algorithm}[!h]
    \caption{\method}
    \label{alg:shapemol}
         %\hspace*{\algorithmicindent} 
	\textbf{Required Input: $\molx$} \\
 	%\hspace*{\algorithmicindent} 
	\textbf{Optional Input: $\pocket$} 
    \begin{algorithmic}[1]
        \FullLineComment{calculate a shape embedding with Algorithm~\ref{alg:see_shaperep}}
        \State $\shapehiddenmat$, $\pc$ = $\SEE(\molx)$
        \FullLineComment{generate a molecule conditioned on the shape embedding with Algorithm~\ref{alg:diffgen}}
         \If{\pocket is not available}
        \State $\moly = \diffgenerative(\shapehiddenmat, \molx)$
        \Else
        \State $\moly = \diffgenerative(\shapehiddenmat, \molx, \pocket)$
        \EndIf
        \State \Return \moly
    \end{algorithmic}
\end{algorithm}
%\label{alg:shapemol}

\begin{algorithm}[!h]
    \caption{\SEE for shape embedding calculation}
    \label{alg:see_shaperep}
    \textbf{Required Input: $\molx$}
    \begin{algorithmic}[1]
        %\Require $\molx$
        \FullLineComment{sample a point cloud over the molecule surface shape}
        \State $\pc$ = $\text{samplePointCloud}(\molx)$
        \FullLineComment{encode the point cloud into a latent embedding (Equation~\ref{eqn:shape_embed})}
        \State $\shapehiddenmat = \SEE(\pc)$
        \FullLineComment{move the center of \pc to zero}
        \State $\pc = \pc - \text{center}(\pc)$
        \State \Return \shapehiddenmat, \pc
    \end{algorithmic}
\end{algorithm}
%\label{alg:see_shaperep}

\begin{algorithm}[!h]
    \caption{\diffgenerative for molecule generation}
    \label{alg:diffgen}
    	\textbf{Required Input: $\molx$, \shapehiddenmat} \\
 	%\hspace*{\algorithmicindent} 
	\textbf{Optional Input: $\pocket$} 
    \begin{algorithmic}[1]
        \FullLineComment{sample the number of atoms in the generated molecule}
        \State $n = \text{sampleAtomNum}(\molx)$
        \FullLineComment{sample initial positions and types of $n$ atoms}
        \State $\{\pos_T\}^n = \mathcal{N}(0, I)$
        \State $\{\atomfeat_T\}^n = C(K, \frac{1}{K})$
        \FullLineComment{generate a molecule by denoising $\{(\pos_T, \atomfeat_T)\}^n$ to $\{(\pos_0, \atomfeat_0)\}^n$}
        \For{$t = T$ to $1$}
            \IndentLineComment{predict the molecule without noise using the shape-conditioned molecule prediction module \molpred}{1.5}
            \State $(\tilde{\pos}_{0,t}, \tilde{\atomfeat}_{0,t}) = \molpred(\pos_t, \atomfeat_t, \shapehiddenmat)$
            \If{use shape guidance and $t > s$}
                \State $\tilde{\pos}_{0,t} = \shapeguide(\tilde{\pos}_{0,t}, \molx)$
                %\State $\tilde{\pos}_{0,t} = \pos^*_{0,t}$
            \EndIf
            \IndentLineComment{sample $(\pos_{t-1}, \atomfeat_{t-1})$ from $(\pos_t, \atomfeat_t)$ and $(\tilde{\pos}_{0,t}, \tilde{\atomfeat}_{0,t})$}{1.5}
            \State $\pos_{t-1} = P(\pos_{t-1}|\pos_t, \tilde{\pos}_{o,t})$
            \State $\atomfeat_{t-1} = P(\atomfeat_{t-1}|\atomfeat_t, \tilde{\atomfeat}_{o,t})$
            \If{use pocket guidance and $\pocket$ is available}
                \State $\pos_{t-1} = \pocketguide(\pos_{t-1}, \pocket)$
                %\State $\pos_{t-1} = \pos_{t-1}^*$
            \EndIf  
        \EndFor
        \State \Return $\moly = (\pos_0, \atomfeat_0)$
    \end{algorithmic}
\end{algorithm}
%\label{alg:diffgen}

%\input{algorithms/train_SE}
%\label{alg:train_se}

%\begin{algorithm}[!h]
    \caption{Training Procedure of \methoddiff}
    \label{alg:diffgen}
    \begin{algorithmic}[1]
        \Require $\shapehiddenmat, \molx, \pocket$
        \FullLineComment{sample the number of atoms in the generated molecule}
    \end{algorithmic}
\end{algorithm}
%\label{alg:train_diff}

%---------------------------------------------------------------------------------------------------------------------
\section{{Equivariance and Invariance}}
\label{supp:ei}
%---------------------------------------------------------------------------------------------------------------------

%.................................................................................................
\subsection{Equivariance}
\label{supp:ei:equivariance}
%.................................................................................................

{Equivariance refers to the property of a function $f(\pos)$ %\bo{is it the property of the function or embedding (x)?} 
that any translation and rotation transformation from the special Euclidean group SE(3)~\cite{Atz2021} applied to a geometric object
$\pos\in\mathbb{R}^3$ is mirrored in the output of $f(\pos)$, accordingly.
%
This property ensures $f(\pos)$ to learn a consistent representation of an object's geometric information, regardless of its orientation or location in 3D space.
%
%As a result, it provides $f(\pos)$ better generalization capabilities~\cite{Jonas20a}.
%
Formally, given any translation transformation $\mathbf{t}\in\mathbb{R}^3$ and rotation transformation $\mathbf{R}\in\mathbb{R}^{3\times3}$ ($\mathbf{R}^{\mathsf{T}}\mathbf{R}=\mathbb{I}$), %\xia{change the font types for $^{\mathsf{T}}$ and $\mathbb{I}$ in the entire manuscript}), 
$f(\pos)$ is equivariant with respect to these transformations %$g$ (\bo{where is $g$...})
if it satisfies
\begin{equation}
f(\mathbf{R}\pos+\mathbf{t}) = \mathbf{R}f(\pos) + \mathbf{t}. %\ \text{where}\ \hiddenpos = f(\pos).
\end{equation}
%
%where $\hiddenpos=f(\pos)$ is the output of $\pos$. 
%
In \method, both \SE and \methoddiff are developed to guarantee equivariance in capturing the geometric features of objects regardless of any translation or rotation transformations, as will be detailed in the following sections.
}

%.................................................................................................
\subsection{Invariance}
\label{supp:ei:invariance}
%.................................................................................................

%In contrast to equivariance, 
Invariance refers to the property of a function that its output {$f(\pos)$} remains constant under any translation and rotation transformations of the input $\pos$. %a geometric object's feature $\pos$.
%
This property enables $f(\pos)$ to accurately capture %a geometric object's 
the inherent features (e.g., atom features for 3D molecules) that are invariant of its orientation or position in 3D space.
%
Formally, $f(\pos)$ is invariant under any translation $\mathbf{t}$ and  rotation $\mathbf{R}$ if it satisfies
%
\begin{equation}
f(\mathbf{R}\pos+\mathbf{t}) = f(\pos).
\end{equation}
%
In \method, both \SE and \methoddiff capture the inherent features of objects in an invariant way, regardless of any translation or rotation transformations, as will be detailed in the following sections.

%%%%%%%%%%%%%%%%%%%%%%%%%%%%%%%%%%%%%%%%%%%%%
\section{Point Cloud Construction}
\label{supp:point_clouds}
%%%%%%%%%%%%%%%%%%%%%%%%%%%%%%%%%%%%%%%%%%%%%

In \method, we represented molecular surface shapes using point clouds (\pc).
%
$\pc$
serves as input to \SE, from which we derive shape latent embeddings.
%
To generate $\pc$, %\bo{\st{create this}}, \bo{generate $\pc$}
we initially generated a molecular surface mesh using the algorithm from the Open Drug Discovery Toolkit~\cite{Wjcikowski2015oddt}.
%
Following this, we uniformly sampled points on the mesh surface with probability proportional to the face area, %\xia{how to uniformly?}, ensuring the sampling is done proportionally to the face area, with
using the algorithm from PyTorch3D~\cite{ravi2020pytorch3d}.
%
This point cloud $\pc$ is then centralized by setting the center of its points to zero.
%
%

%%%%%%%%%%%%%%%%%%%%%%%%%%%%%%%%%%%%%%%%%%%%%
\section{Query Point Sampling}
\label{supp:training:shapeemb}
%%%%%%%%%%%%%%%%%%%%%%%%%%%%%%%%%%%%%%%%%%%%%

As described in Section ``Shape Decoder (\SED)'', the signed distances of query points $z_q$ to molecule surface shape $\pc$ are used to optimize \SE.
%
In this section, we present how to sample these points $z_q$ in 3D space.
%
Particularly, we first determined the bounding box around the molecular surface shape, using the maximum and minimum \mbox{($x$, $y$, $z$)-axis} coordinates for points in our point cloud \pc,
denoted as $(x_\text{min}, y_\text{min}, z_\text{min})$ and $(x_\text{max}, y_\text{max}, z_\text{max})$.
%
We extended this box slightly by defining its corners as \mbox{$(x_\text{min}-1, y_\text{min}-1, z_\text{min}-1)$} and \mbox{$(x_\text{max}+1, y_\text{max}+1, z_\text{max}+1)$}.
%
For sampling $|\mathcal{Z}|$ query points, we wanted an even distribution of points inside and outside the molecule surface shape.
%
%\ziqi{Typically, within this bounding box, molecules occupy only a small portion of volume, which makes it more likely to sample
%points outside the molecule surface shape.}
%
When a bounding box is defined around the molecule surface shape, there could be a lot of empty spaces within the box that the molecule does not occupy due to 
its complex and irregular shape.
%
This could lead to that fewer points within the molecule surface shape could be sampled within the box.
%
Therefore, we started by randomly sampling $3k$ points within our bounding box to ensure that there are sufficient points within the surface.
%
We then determined whether each point lies within the molecular surface, using an algorithm from Trimesh~\footnote{https://trimsh.org/} based on the molecule surface mesh.
%
If there are $n_w$ points found within the surface, we selected $n=\min(n_w, k/2)$ points from these points, 
and randomly choose the remaining 
%\bo{what do you mean by remaining? If all the 3k sampled points are inside the surface, you get no points left.} 
$k-n$ points 
from those outside the surface.
%
For each query point, we determined its signed distance to the molecule surface by its closest distance to points in \pc with a sign indicating whether it is inside the surface.

%%%%%%%%%%%%%%%%%%%%%%%%%%%%%%%%%%%%%%%%%%%%%
\section{Forward Diffusion (\diffnoise)}
\label{supp:forward}
%%%%%%%%%%%%%%%%%%%%%%%%%%%%%%%%%%%%%%%%%%%%%

%===================================================================
\subsection{{Forward Process}}
\label{supp:forward:forward}
%===================================================================

Formally, for atom positions, the probability of $\pos_t$ sampled given $\pos_{t-1}$, denoted as $q(\pos_t|\pos_{t-1})$, is defined as follows,
%\xia{revise the representation, should be $\beta^x_t$ -- note the space} as follows,
%
\begin{equation}
q(\pos_t|\pos_{t-1}) = \mathcal{N}(\pos_t|\sqrt{1-\beta^{\mathtt{x}}_t}\pos_{t-1}, \beta^{\mathtt{x}}_t\mathbb{I}), 
\label{eqn:noiseposinter}
\end{equation}
%
%\xia{should be a comma after the equation. you also missed it. }
%\st{in which} 
where %\hl{$\pos_0$ denotes the original atom position;} \xia{no $\pos_0$ in the equation...}
%$\mathbf{I}$ denotes the identity matrix;
$\mathcal{N}(\cdot)$ is a Gaussian distribution of $\pos_t$ with mean $\sqrt{1-\beta_t^{\mathtt{x}}}\pos_{t-1}$ and covariance $\beta_t^{\mathtt{x}}\mathbf{I}$.
%\xia{what is $\mathcal{N}$? what is $q$? you abused $q$. need to be crystal clear... }
%\bo{Should be $\sim$ not $=$ in the equation}
%
Following Hoogeboom \etal~\cite{hoogeboom2021catdiff}, 
%the forward process for the discrete atom feature $\atomfeat_t\in\mathbb{R}^K$ adds 
%categorical noise into $\atomfeat_{t-1}$ according to a variance schedule $\beta_t^v\in (0, 1)$. %as follows, %\hl{$\betav_t\in (0, 1)$} as follows,
%\xia{presentation...check across the entire manuscript... } as follows,
%
%\ziqi{Formally, 
for atom features, the probability of $\atomfeat_t$ across $K$ classes given $\atomfeat_{t-1}$ is defined as follows,
%
\begin{equation}
q(\atomfeat_t|\atomfeat_{t-1}) = \mathcal{C}(\atomfeat_t|(1-\beta^{\mathtt{v}}_t) \atomfeat_{t-1}+\beta^{\mathtt{v}}_t\mathbf{1}/K),
\label{eqn:noisetypeinter}
\end{equation}
%
where %\hl{$\atomfeat_0$ denotes the original atom positions}; 
$\mathcal{C}$ is a categorical distribution of $\atomfeat_t$ derived from the %by 
noising $\atomfeat_{t-1}$ with a uniform noise $\beta^{\mathtt{v}}_t\mathbf{1}/K$ across $K$ classes.
%adding an uniform noise $\beta^v_t$ to $\atomfeat_{t-1}$ across K classes.
%\xia{there is always a comma or period after the equations. Equations are part of a sentence. you always missed it. }
%\xia{what is $\mathcal{C}$? what does $q$ mean? it is abused. }

Since the above distributions form Markov chains, %} \xia{grammar!}, 
the probability of any $\pos_t$ or $\atomfeat_t$ can be derived from $\pos_0$ or $\atomfeat_0$:
%samples $\mol_0$ as follows,
%
\begin{eqnarray}
%\begin{aligned}
& q(\pos_t|\pos_{0}) & = \mathcal{N}(\pos_t|\sqrt{\cumalpha^{\mathtt{x}}_t}\pos_0, (1-\cumalpha^{\mathtt{x}}_t)\mathbb{I}), \label{eqn:noisepos}\\
& q(\atomfeat_t|\atomfeat_0)  & = \mathcal{C}(\atomfeat_t|\cumalpha^{\mathtt{v}}_t\atomfeat_0 + (1-\cumalpha^{\mathtt{v}}_t)\mathbf{1}/K), \label{eqn:noisetype}\\
& \text{where }\cumalpha^{\mathtt{u}}_t & = \prod\nolimits_{\tau=1}^{t}\alpha^{\mathtt{u}}_\tau, \ \alpha^{\mathtt{u}}_\tau=1 - \beta^{\mathtt{u}}_\tau, \ {\mathtt{u}}={\mathtt{x}} \text{ or } {\mathtt{v}}.\;\;\;\label{eqn:noiseschedule}
%\end{aligned}
\label{eqn:pos_prior}
\end{eqnarray}
%\xia{always punctuations after equations!!! also use ``eqnarray" instead of ``equation" + ``aligned" for multiple equations, each
%with a separate reference numbering...}
%\st{in which}, 
%where \ziqi{$\cumalpha^u_t = \prod_{\tau=1}^{t}\alpha^u_\tau$ and $\alpha^u_\tau=1 - \beta^u_\tau$ ($u$=$x$ or $v$)}.
%\xia{no such notations in the above equations; also subscript $s$ is abused with shape};
%$K$ is the number of categories for atom features.
%
%The details about noise schedules $\beta^x_t$ and $\beta^v_t$ are available in Supplementary Section \ref{XXX}. \ziqi{add trend}
%
Note that $\bar{\alpha}^{\mathtt{u}}_t$ ($\mathtt{u}={\mathtt{x}}\text{ or }{\mathtt{v}}$)
%($u$=$x$ or $v$) 
is monotonically decreasing from 1 to 0 over $t=[1,T]$. %\xia{=???}. 
%
As $t\rightarrow 1$, $\cumalpha^{\mathtt{x}}_t$ and $\cumalpha^{\mathtt{v}}_t$ are close to 1, leading to that $\pos_t$ or $\atomfeat_t$ approximates 
%the original data 
$\pos_0$ or $\atomfeat_0$.
%
Conversely, as  $t\rightarrow T$, $\cumalpha^{\mathtt{x}}_t$ and $\cumalpha^{\mathtt{v}}_t$ are close to 0,
leading to that $q(\pos_T|\pos_{0})$ %\st{$\rightarrow \mathcal{N}(\mathbf{0}, \mathbf{I})$} 
resembles  {$\mathcal{N}(\mathbf{0}, \mathbb{I})$} 
and $q(\atomfeat_T|\atomfeat_0)$ %\st{$\rightarrow \mathcal{C}(\mathbf{I}/K)$} 
resembles {$\mathcal{C}(\mathbf{1}/K)$}.

Using Bayes theorem, the ground-truth Normal posterior of atom positions $p(\pos_{t-1}|\pos_t, \pos_0)$ can be calculated in a
closed form~\cite{ho2020ddpm} as below,
%
\begin{eqnarray}
& p(\pos_{t-1}|\pos_t, \pos_0) = \mathcal{N}(\pos_{t-1}|\mu(\pos_t, \pos_0), \tilde{\beta}^\mathtt{x}_t\mathbb{I}), \label{eqn:gt_pos_posterior_1}\\
&\!\!\!\!\!\!\!\!\!\!\!\mu(\pos_t, \pos_0)\!=\!\frac{\sqrt{\bar{\alpha}^{\mathtt{x}}_{t-1}}\beta^{\mathtt{x}}_t}{1-\bar{\alpha}^{\mathtt{x}}_t}\pos_0\!+\!\frac{\sqrt{\alpha^{\mathtt{x}}_t}(1-\bar{\alpha}^{\mathtt{x}}_{t-1})}{1-\bar{\alpha}^{\mathtt{x}}_t}\pos_t, 
\tilde{\beta}^\mathtt{x}_t\!=\!\frac{1-\bar{\alpha}^{\mathtt{x}}_{t-1}}{1-\bar{\alpha}^{\mathtt{x}}_{t}}\beta^{\mathtt{x}}_t.\;\;\;
\end{eqnarray}
%
%\xia{Ziqi, please double check the above two equations!}
Similarly, the ground-truth categorical posterior of atom features $p(\atomfeat_{t-1}|\atomfeat_{t}, \atomfeat_0)$ can be calculated~\cite{hoogeboom2021catdiff} as below,
%
\begin{eqnarray}
& p(\atomfeat_{t-1}|\atomfeat_{t}, \atomfeat_0) = \mathcal{C}(\atomfeat_{t-1}|\mathbf{c}(\atomfeat_t, \atomfeat_0)), \label{eqn:gt_atomfeat_posterior_1}\\
& \mathbf{c}(\atomfeat_t, \atomfeat_0) = \tilde{\mathbf{c}}/{\sum_{k=1}^K \tilde{c}_k}, \label{eqn:gt_atomfeat_posterior_2} \\
& \tilde{\mathbf{c}} = [\alpha^{\mathtt{v}}_t\atomfeat_t + \frac{1 - \alpha^{\mathtt{v}}_t}{K}]\odot[\bar{\alpha}^{\mathtt{v}}_{t-1}\atomfeat_{0}+\frac{1-\bar{\alpha}^{\mathtt{v}}_{t-1}}{K}], 
\label{eqn:gt_atomfeat_posterior_3}
%\label{eqn:atomfeat_posterior}
\end{eqnarray}
%
%\xia{Ziqi: please double check the above equations!}
%
where $\tilde{c}_k$ denotes the likelihood of $k$-th class across $K$ classes in $\tilde{\mathbf{c}}$; 
$\odot$ denotes the element-wise product operation;
$\tilde{\mathbf{c}}$ is calculated using $\atomfeat_t$ and $\atomfeat_{0}$ and normalized into $\mathbf{c}(\atomfeat_t, \atomfeat_0)$ so as to represent
probabilities. %\xia{is this correct? is $\tilde{c}_k$ always greater than 0?}
%\xia{how is it calculated?}.
%\ziqi{the likelihood distribution $\tilde{c}$ is calculated by $p(\atomfeat_t|\atomfeat_{t-1})p(\atomfeat_{t-1}|\atomfeat_0)$, according to 
%Equation~\ref{eqn:noisetypeinter} and \ref{eqn:noisetype}.
%\xia{need to write the key idea of the above calculation...}
%
The proof of the above equations is available in Supplementary Section~\ref{supp:forward:proof}.

%===================================================================
\subsection{Variance Scheduling in \diffnoise}
\label{supp:forward:variance}
%===================================================================

Following Guan \etal~\cite{guan2023targetdiff}, we used a sigmoid $\beta$ schedule for the variance schedule $\beta_t^{\mathtt{x}}$ of atom coordinates as below,

\begin{equation}
\beta_t^{\mathtt{x}} = \text{sigmoid}(w_1(2 t / T - 1)) (w_2 - w_3) + w_3
\end{equation}
in which $w_i$($i$=1,2, or 3) are hyperparameters; $T$ is the maximum step.
%
We set $w_1=6$, $w_2=1.e-7$ and $w_3=0.01$.
%
For atom types, we used a cosine $\beta$ schedule~\cite{nichol2021} for $\beta_t^{\mathtt{v}}$ as below,

\begin{equation}
\begin{aligned}
& \bar{\alpha}_t^{\mathtt{v}} = \frac{f(t)}{f(0)}, f(t) = \cos(\frac{t/T+s}{1+s} \cdot \frac{\pi}{2})^2\\
& \beta_t^{\mathtt{v}} = 1 - \alpha_t^{\mathtt{v}} = 1 - \frac{\bar{\alpha}_t^{\mathtt{v}} }{\bar{\alpha}_{t-1}^{\mathtt{v}} }
\end{aligned}
\end{equation}
in which $s$ is a hyperparameter and set as 0.01.

As shown in Section ``Forward Diffusion Process'', the values of $\beta_t^{\mathtt{x}}$ and $\beta_t^{\mathtt{v}}$ should be 
sufficiently small to ensure the smoothness of forward diffusion process. In the meanwhile, their corresponding $\bar{\alpha}_t$
values should decrease from 1 to 0 over $t=[1,T]$.
%
Figure~\ref{fig:schedule} shows the values of $\beta_t$ and $\bar{\alpha}_t$ for atom coordinates and atom types with our hyperparameters.
%
Please note that the value of $\beta_{t}^{\mathtt{x}}$ is less than 0.1 for 990 out of 1,000 steps. %\bo{\st{, though it increases when $t$ is close to 1,000}}.
%
This guarantees the smoothness of the forward diffusion process.
%\bo{add $\beta_t^{\mathtt{x}}$ and $\beta_t^{\mathtt{v}}$ in the legend of the figure...}
%\bo{$\beta_t^{\mathtt{v}}$ does not look small when $t$ is close to 1000...}

\begin{figure}
	\begin{subfigure}[t]{.45\linewidth}
		\centering
		\includegraphics[width=.7\linewidth]{figures/var_schedule_beta.pdf}
	\end{subfigure}
	%
	\hfill
	\begin{subfigure}[t]{.45\linewidth}
		\centering
		\includegraphics[width=.7\linewidth]{figures/var_schedule_alpha.pdf}
	\end{subfigure}
	\caption{Schedule}
	\label{fig:schedule}
\end{figure}

%===================================================================
\subsection{Derivation of Forward Diffusion Process}
\label{supp:forward:proof}
%===================================================================

In \method, a Gaussian noise and a categorical noise are added to continuous atom position and discrete atom features, respectively.
%
Here, we briefly describe the derivation of posterior equations (i.e., Eq.~\ref{eqn:gt_pos_posterior_1}, and   \ref{eqn:gt_atomfeat_posterior_1}) for atom positions and atom types in our work.
%
We refer readers to Ho \etal~\cite{ho2020ddpm} and Kong \etal~\cite{kong2021diffwave} %\bo{add XXX~\etal here...} \cite{ho2020ddpm,kong2021diffwave} 	
for a detailed description of diffusion process for continuous variables and Hoogeboom \etal~\cite{hoogeboom2021catdiff} for
%\bo{add XXX~\etal here...} \cite{hoogeboom2021catdiff} for
the description of diffusion process for discrete variables.

For continuous atom positions, as shown in Kong \etal~\cite{kong2021diffwave}, according to Bayes theorem, given $q(\pos_t|\pos_{t-1})$ defined in Eq.~\ref{eqn:noiseposinter} and 
$q(\pos_t|\pos_0)$ defined in Eq.~\ref{eqn:noisepos}, the posterior $q(\pos_{t-1}|\pos_{t}, \pos_0)$ is derived as below (superscript $\mathtt{x}$ is omitted for brevity),

\begin{equation}
\begin{aligned}
& q(\pos_{t-1}|\pos_{t}, \pos_0)  = \frac{q(\pos_t|\pos_{t-1}, \pos_0)q(\pos_{t-1}|\pos_0)}{q(\pos_t|\pos_0)} \\
& =  \frac{\mathcal{N}(\pos_t|\sqrt{1-\beta_t}\pos_{t-1}, \beta_{t}\mathbf{I}) \mathcal{N}(\pos_{t-1}|\sqrt{\bar{\alpha}_{t-1}}\pos_{0}, (1-\bar{\alpha}_{t-1})\mathbf{I}) }{ \mathcal{N}(\pos_{t}|\sqrt{\bar{\alpha}_t}\pos_{0}, (1-\bar{\alpha}_t)\mathbf{I})}\\
& =  (2\pi{\beta_t})^{-\frac{3}{2}} (2\pi{(1-\bar{\alpha}_{t-1})})^{-\frac{3}{2}} (2\pi(1-\bar{\alpha}_t))^{\frac{3}{2}} \times \exp( \\
& -\frac{\|\pos_t - \sqrt{\alpha}_t\pos_{t-1}\|^2}{2\beta_t} -\frac{\|\pos_{t-1} - \sqrt{\bar{\alpha}}_{t-1}\pos_{0} \|^2}{2(1-\bar{\alpha}_{t-1})} \\
& + \frac{\|\pos_t - \sqrt{\bar{\alpha}_t}\pos_0\|^2}{2(1-\bar{\alpha}_t)}) \\
& = (2\pi\tilde{\beta}_t)^{-\frac{3}{2}} \exp(-\frac{1}{2\tilde{\beta}_t}\|\pos_{t-1}-\frac{\sqrt{\bar{\alpha}_{t-1}}\beta_t}{1-\bar{\alpha}_t}\pos_0 \\
& - \frac{\sqrt{\alpha_t}(1-\bar{\alpha}_{t-1})}{1-\bar{\alpha}_t}\pos_{t}\|^2) \\
& \text{where}\ \tilde{\beta}_t = \frac{1-\bar{\alpha}_{t-1}}{1-\bar{\alpha}_t}\beta_t.
\end{aligned}
\end{equation}
%\bo{marked part does not look right to me.}
%\bo{How to you derive from the second equation to the third one?}

Therefore, the posterior of atom positions is derived as below,

\begin{equation}
q(\pos_{t-1}|\pos_{t}, \pos_0)\!\!=\!\!\mathcal{N}(\pos_{t-1}|\frac{\sqrt{\bar{\alpha}_{t-1}}\beta_t}{1-\bar{\alpha}_t}\pos_0 + \frac{\sqrt{\alpha_t}(1-\bar{\alpha}_{t-1})}{1-\bar{\alpha}_t}\pos_{t}, \tilde{\beta}_t\mathbf{I}).
\end{equation}

For discrete atom features, as shown in Hoogeboom \etal~\cite{hoogeboom2021catdiff} and Guan \etal~\cite{guan2023targetdiff},
according to Bayes theorem, the posterior $q(\atomfeat_{t-1}|\atomfeat_{t}, \atomfeat_0)$ is derived as below (supperscript $\mathtt{v}$ is omitted for brevity),

\begin{equation}
\begin{aligned}
& q(\atomfeat_{t-1}|\atomfeat_{t}, \atomfeat_0) =  \frac{q(\atomfeat_t|\atomfeat_{t-1}, \atomfeat_0)q(\atomfeat_{t-1}|\atomfeat_0)}{\sum_{\scriptsize{\atomfeat}_{t-1}}q(\atomfeat_t|\atomfeat_{t-1}, \atomfeat_0)q(\atomfeat_{t-1}|\atomfeat_0)} \\
%& = \frac{\mathcal{C}(\atomfeat_t|(1-\beta_t)\atomfeat_{t-1} + \beta_t\frac{\mathbf{1}}{K}) \mathcal{C}(\atomfeat_{t-1}|\bar{\alpha}_{t-1}\atomfeat_0+(1-\bar{\alpha}_{t-1})\frac{\mathbf{1}}{K})} \\
\end{aligned}
\end{equation}

For $q(\atomfeat_t|\atomfeat_{t-1}, \atomfeat_0)$, we have % $\atomfeat_t=\atomfeat_{t-1}$ with probability $1-\beta_t+\beta_t / K$, and $\atomfeat_t \neq \atomfeat_{t-1}$
%with probability $\beta_t / K$.
%
%Therefore, this function can be symmetric, that is, 
%
\begin{equation}
\begin{aligned}
q(\atomfeat_t|\atomfeat_{t-1}, \atomfeat_0) & = \mathcal{C}(\atomfeat_t|(1-\beta_t)\atomfeat_{t-1} + \beta_t/{K})\\
& = \begin{cases}
1-\beta_t+\beta_t/K,\!&\text{when}\ \atomfeat_{t} = \atomfeat_{t-1},\\
\beta_t / K,\! &\text{when}\ \atomfeat_{t} \neq \atomfeat_{t-1},
\end{cases}\\
& = \mathcal{C}(\atomfeat_{t-1}|(1-\beta_t)\atomfeat_{t} + \beta_t/{K}).
\end{aligned}
%\mathcal{C}(\atomfeat_{t-1}|(1-\beta_{t})\atomfeat_{t} + \beta_t\frac{\mathbf{1}}{K}).
\end{equation}
%
Therefore, we have
%\bo{why it can be symmetric}
%
\begin{equation}
\begin{aligned}
& q(\atomfeat_t|\atomfeat_{t-1}, \atomfeat_0)q(\atomfeat_{t-1}|\atomfeat_0) \\
& = \mathcal{C}(\atomfeat_{t-1}|(1-\beta_t)\atomfeat_{t} + \beta_t\frac{\mathbf{1}}{K}) \mathcal{C}(\atomfeat_{t-1}|\bar{\alpha}_{t-1}\atomfeat_0+(1-\bar{\alpha}_{t-1})\frac{\mathbf{1}}{K}) \\
& = [\alpha_t\atomfeat_t + \frac{1 - \alpha_t}{K}]\odot[\bar{\alpha}_{t-1}\atomfeat_{0}+\frac{1-\bar{\alpha}_{t-1}}{K}].
\end{aligned}
\end{equation}
%
%\bo{what is $\tilde{\mathbf{c}}$...}
Therefore, with $q(\atomfeat_t|\atomfeat_{t-1}, \atomfeat_0)q(\atomfeat_{t-1}|\atomfeat_0) = \tilde{\mathbf{c}}$, the posterior is as below,

\begin{equation}
q(\atomfeat_{t-1}|\atomfeat_{t}, \atomfeat_0) = \mathcal{C}(\atomfeat_{t-1}|\mathbf{c}(\atomfeat_t, \atomfeat_0)) = \frac{\tilde{\mathbf{c}}}{\sum_{k}^K\tilde{c}_k}.
\end{equation}

%%%%%%%%%%%%%%%%%%%%%%%%%%%%%%%%%%%%%%%%%%%%%
\section{{Backward Generative Process} (\diffgenerative)}
\label{supp:backward}
%%%%%%%%%%%%%%%%%%%%%%%%%%%%%%%%%%%%%%%%%%%%%

Following Ho \etal~\cite{ho2020ddpm}, with $\tilde{\pos}_{0,t}$, the probability of $\pos_{t-1}$ denoised from $\pos_t$, denoted as $p(\pos_{t-1}|\pos_t)$,
can be estimated %\hl{parameterized} \xia{???} 
by the approximated posterior $p_{\boldsymbol{\Theta}}(\pos_{t-1}|\pos_t, \tilde{\pos}_{0,t})$ as below,
%
\begin{equation}
\begin{aligned}
p(\pos_{t-1}|\pos_t) & \approx p_{\boldsymbol{\Theta}}(\pos_{t-1}|\pos_t, \tilde{\pos}_{0,t}) \\
& = \mathcal{N}(\pos_{t-1}|\mu_{\boldsymbol{\Theta}}(\pos_t, \tilde{\pos}_{0,t}),\tilde{\beta}_t^{\mathtt{x}}\mathbb{I}),
\end{aligned}
\label{eqn:aprox_pos_posterior}
\end{equation}
%
where ${\boldsymbol{\Theta}}$ is the learnable parameter; $\mu_{\boldsymbol{\Theta}}(\pos_t, \tilde{\pos}_{0,t})$ is an estimate %estimation
of $\mu(\pos_t, \pos_{0})$ by replacing $\pos_0$ with its estimate $\tilde{\pos}_{0,t}$ 
in Equation~{\ref{eqn:gt_pos_posterior_1}}.
%
Similarly, with $\tilde{\atomfeat}_{0,t}$, the probability of $\atomfeat_{t-1}$ denoised from $\atomfeat_t$, denoted as $p(\atomfeat_{t-1}|\atomfeat_t)$, 
can be estimated %\hl{parameterized} 
by the approximated posterior $p_{\boldsymbol{\Theta}}(\atomfeat_{t-1}|\atomfeat_t, \tilde{\atomfeat}_{0,t})$ as below,
%
\begin{equation}
\begin{aligned}
p(\atomfeat_{t-1}|\atomfeat_t)\approx p_{\boldsymbol{\Theta}}(\atomfeat_{t-1}|\atomfeat_{t}, \tilde{\atomfeat}_{0,t}) 
=\mathcal{C}(\atomfeat_{t-1}|\mathbf{c}_{\boldsymbol{\Theta}}(\atomfeat_t, \tilde{\atomfeat}_{0,t})),\!\!\!\!
\end{aligned}
\label{eqn:aprox_atomfeat_posterior}
\end{equation}
%
where $\mathbf{c}_{\boldsymbol{\Theta}}(\atomfeat_t, \tilde{\atomfeat}_{0,t})$ is an estimate of $\mathbf{c}(\atomfeat_t, \atomfeat_0)$
by replacing $\atomfeat_0$  
with its estimate $\tilde{\atomfeat}_{0,t}$ in Equation~\ref{eqn:gt_atomfeat_posterior_1}.



%===================================================================
\section{\method Loss Function Derivation}
\label{supp:training:loss}
%===================================================================

In this section, we demonstrate that a step weight $w_t^{\mathtt{x}}$ based on the signal-to-noise ratio $\lambda_t$ should be 
included into the atom position loss (Eq.~\ref{eqn:diff:obj:pos}).
%
In the diffusion process for continuous variables, the optimization problem is defined 
as below~\cite{ho2020ddpm},
%
\begin{equation*}
\begin{aligned}
& \arg\min_{\boldsymbol{\Theta}}KL(q(\pos_{t-1}|\pos_t, \pos_0)|p_{\boldsymbol{\Theta}}(\pos_{t-1}|\pos_t, \tilde{\pos}_{0,t})) \\
& = \arg\min_{\boldsymbol{\Theta}} \frac{\bar{\alpha}_{t-1}(1-\alpha_t)}{2(1-\bar{\alpha}_{t-1})(1-\bar{\alpha}_{t})}\|\tilde{\pos}_{0, t}-\pos_0\|^2 \\
& = \arg\min_{\boldsymbol{\Theta}} \frac{1-\alpha_t}{2(1-\bar{\alpha}_{t-1})\alpha_{t}} \|\tilde{\boldsymbol{\epsilon}}_{0,t}-\boldsymbol{\epsilon}_0\|^2,
\end{aligned}
\end{equation*}
where $\boldsymbol{\epsilon}_0 = \frac{\pos_t - \sqrt{\bar{\alpha}_t}\pos_0}{\sqrt{1-\bar{\alpha}_t}}$ is the ground-truth noise variable sampled from $\mathcal{N}(\mathbf{0}, \mathbf{1})$ and is used to sample $\pos_t$ from $\mathcal{N}(\pos_t|\sqrt{\cumalpha_t}\pos_0, (1-\cumalpha_t)\mathbf{I})$ in Eq.~\ref{eqn:noisetype};
$\tilde{\boldsymbol{\epsilon}}_0 = \frac{\pos_t - \sqrt{\bar{\alpha}_t}\tilde{\pos}_{0, t}}{\sqrt{1-\bar{\alpha}_t}}$ is the predicted noise variable. 

%A simplified training objective is proposed by Ho \etal~\cite{ho2020ddpm} as below,
Ho \etal~\cite{ho2020ddpm} further simplified the above objective as below and
demonstrated that the simplified one can achieve better performance:
%
\begin{equation}
\begin{aligned}
& \arg\min_{\boldsymbol{\Theta}} \|\tilde{\boldsymbol{\epsilon}}_{0,t}-\boldsymbol{\epsilon}_0\|^2 \\
& = \arg\min_{\boldsymbol{\Theta}} \frac{\bar{\alpha}_t}{1-\bar{\alpha}_t}\|\tilde{\pos}_{0,t}-\pos_0\|^2,
\end{aligned}
\label{eqn:supp:losspos}
\end{equation}
%
where $\lambda_t=\frac{\bar{\alpha}_t}{1-\bar{\alpha}_t}$ is the signal-to-noise ratio.
%
While previous work~\cite{guan2023targetdiff} applies uniform step weights across
different steps, we demonstrate that a step weight should be included into the atom position loss according to Eq.~\ref{eqn:supp:losspos}.
%
However, the value of $\lambda_t$ could be very large when $\bar{\alpha}_t$ is close to 1 as $t$ approaches 1.
%
Therefore, we clip the value of $\lambda_t$ with threshold $\delta$ in Eq.~\ref{eqn:diff:obj:pos}.




\end{document}