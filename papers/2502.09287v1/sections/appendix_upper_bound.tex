\section{Upper bound}

Here, we prove the results of Section~\ref{section upper bound} of the paper. We begin by proving the expression of $\mathcal{L}_\text{freq}$, to then justify the parametrization of the optimal $b_s$ in Eq.~\eqref{Param_of_the_as}. We finally compute the asymptotic loss in  Eq.~\eqref{Upper bound as and bs} and Theorem~\ref{convergence to window}.

\subsection{Loss in frequency domain}\label{appendix subsection frequency loss}

Here, we prove the expression of the counterpart of $\mathcal{L}_\text{time}$ in the frequency domain, $\mathcal{L}_\text{freq}$. More explicitly, we give a proof of Eq.~\eqref{Frequential loss copy task}.


\begin{proof}
    Denote $(z_k)$ the discrete-time filter such that 
    \[
    z_k = \sum_{k'=0}^{+\infty}(c_{k'} - d_{k'})\gamma(k-k').
    \]
    Therefore, $(z_k)$ is a convolution between $(c_k-d_k)$ and $\gamma_k$, 
    \[
    \sum_{k, k'=0}^{+\infty}(c_k - d_k)(c_{k'} - d_{k'})\gamma(k-k) = \sum_{k=0}^{+\infty}(c_k - d_k)z_k.
    \]
    According to Parseval's theorem and denoting $C(e^{i\omega}), D(e^{i\omega})$ and $\Gamma(e^{i\omega})$ the respective Fourier transforms, we have: 
    \begin{align*}
        \sum_{k=0}^{+\infty}(c_k - d_k)z_k &= \frac{1}{2\pi}\int_0^{2\pi}Z(\omega)\overline{(C(e^{i\omega}) - D(e^{i\omega}))}d\omega\\
        &= \frac{1}{2\pi}\int_0^{2\pi}\Gamma(e^{i\omega})(C(e^{i\omega}) - D(e^{i\omega}))\overline{(C(e^{i\omega}) - D(e^{i\omega}))}d\omega\\
        &= \frac{1}{2\pi}\int_0^{2\pi}\big\vert C(e^{i\omega}) - D(e^{i\omega})\big\vert^2\Gamma(e^{i\omega})d\omega,
    \end{align*}
    by the convolution property of the DTFT. Finally, $(d_k)$ being the shifted impulse filter, its Fourier transform is $D(e^{i\omega})=e^{-iK\omega}$. The Fourier transform $C(e^{i\omega})$ of $(c_k)$ is given by 
        \[
    C(e^{i\omega}) = \sum_{s=1}^S b_s \sum_{k=-\infty}^\infty \left(a_s e^{-i\omega}\right)^k = \sum_{s=1}^S \frac{b_s}{1 - a_s e^{-i\omega}}.
    \]
\end{proof}

\subsection{Parametrization of the optimal \texorpdfstring{$\boldsymbol{b_s}$}{bs}}

\label{appendix subsection asymptotic bs}

For the sake of conciseness, we sometimes denote $a$ and $b$ for the vectors $(a_s)$ and $(b_s)$ respectively. Before proving Lemma~\ref{Lemma param of bs}, we show three general lemmas on Fourier series and remainder of series. We will use them later in the proof of Lemma~\ref{Lemma param of bs}.

\begin{lemma}\label{lemma eigenvector Toeplitz}
Let \(\alpha \in \mathbb{C}\) and \(S \in \mathbb{N}\). Consider the infinite Toeplitz matrix \(T\) defined by $$T(s, s')~=~\frac{1}{2\alpha - \mathrm{i}(s-s')\pi}.$$ Then, \(\frac{2}{e^{2\alpha} - e^{-2\alpha}}\) is an asymptotic eigenvalue of  \(T\), associated with the eigenvector \(z = ((-1)^s)_{s \in \mathbb{Z}}\).
\end{lemma}

\begin{proof}
We compute the action of \(T\) on \(z\):
\[
(Tz)_s= \sum_{s'} T(s, s')(-1)^{s'} = \sum_{s'} \frac{e^{\mathrm{i}\pi s'}}{2\alpha - \mathrm{i}\pi s + \mathrm{i}\pi s'}.
\]

This expression resembles a Fourier series evaluated at \(\omega = \pi\). For \(k \in \mathbb{Z}\), consider:
\[
\frac{1}{2\pi} \int_0^{2\pi} e^{-\frac{2\alpha}{\pi}(\omega-\pi)} e^{-\mathrm{i}k\omega} \, d\omega = \frac{1}{2} \cdot \frac{e^{2\alpha} - e^{-2\alpha}}{2\alpha + \mathrm{i}k\pi}.
\]

Thus:
\begin{equation}
\label{eq:FS}
\frac{1}{2\alpha + \mathrm{i}k\pi} = \frac{1}{2\pi} \cdot \frac{2}{e^{2\alpha} - e^{-2\alpha}} \int_0^{2\pi} e^{-\frac{2\alpha}{\pi}(\omega-\pi)} e^{-\mathrm{i}k\omega} \, d\omega.
\end{equation}

We recognize this as the Fourier coefficient of the function \(f_\alpha(\omega) = \frac{2}{e^{2\alpha} - e^{-2\alpha}} e^{-\frac{2\alpha}{\pi}(\omega-\pi)}\). Therefore, according to Dirichlet's theorem, for \(s \in \mathbb{Z}\):
\[
f_{2\alpha - \mathrm{i}\pi s}(\pi) = \sum_{s'} \frac{e^{\mathrm{i}\pi s'}}{2\alpha - \mathrm{i}\pi s + \mathrm{i}\pi s'} = \sum_{s'} \frac{(-1)^{s'}}{2\alpha - \mathrm{i}\pi(s-s')}.
\]

Simplifying, we find:
\( \displaystyle
\frac{2}{e^{2\alpha - \mathrm{i}\pi s} - e^{-2\alpha + \mathrm{i}\pi s}} = \sum_{s'} \frac{(-1)^{s'}}{2\alpha - \mathrm{i}\pi(s-s')}.
\)
Finally, we obtain
\( \displaystyle
(Tz)_s= \frac{2}{e^{2\alpha} - e^{-2\alpha}} z_s,
\)
and thus:
\( \displaystyle
Tz = \frac{2}{e^{2\alpha} - e^{-2\alpha}} z,
\)
proving that \(\frac{2}{e^{2\alpha} - e^{-2\alpha}}\) is an eigenvalue associated with \(z = ((-1)^s)_{s \in \mathbb{Z}}\).

\end{proof}

Now, we prove two general results on remainders of alternating series. 

\begin{lemma}\label{appendix lemma 12.1}
We have:
    \begin{equation*}
        R_N = \sum_{n=N}^{+\infty}\frac{(-1)^n}{n} = \frac{(-1)^N}{2N} + \frac{1}{2}\sum_{n=N}^{+\infty}\frac{(-1)^n}{n(n+1)}.
    \end{equation*}
\end{lemma}

\begin{proof}
    On the one side,  by grouping two consecutive terms,
    \begin{align*}
    R_N + R_{N+1} &= \sum_{n=N}^{+\infty}\frac{(-1)^n}{n} + \sum_{n=N+1}^{+\infty}\frac{(-1)^n}{n}  = \sum_{n=N}^{+\infty} \Big\{ \frac{(-1)^n}{n} - \frac{(-1)^n}{n+1} \Big\}= \sum_{n=N}^{+\infty}\frac{(-1)^n}{n(n+1)},
    \end{align*}
and on the other side, 
\[
R_{N+1} = R_N - \frac{(-1)^N}{N}.
\]
Therefore, \( \displaystyle
2R_N = \frac{(-1)^N}{N} + \sum_{n=N}^{+\infty}\frac{(-1)^n}{n(n+1)}.
\)
\end{proof}
    

\begin{lemma}\label{appendix lemma 12.2}
For $\alpha\in\mathbb{C}$ such that Re$(\alpha)>0$,
we have
\begin{equation*}
    \sum_{n=N}^{+\infty} \frac{(-1)^n}{\alpha + i\pi n} = \frac{i}{\pi}\times\frac{(-1)^N}{2N} + o\big(\frac{1}{N}\big).
\end{equation*}
\end{lemma}

\begin{proof}
    We denote $R_N = \sum_{n=N}^{+\infty}\frac{(-1)^n}{\alpha - i\pi n}$ and $S_N = \sum_{n=N}^{+\infty}\frac{(-1)^n}{-i\pi n}$. Consequently, 
    \begin{align*}
        R_N - S_N &= \sum_{n=N}^{+\infty}(-1)^n\big(\frac{1}{\alpha - i\pi n} + \frac{1}{i\pi n}\big) = \alpha\sum_{n=N}^{+\infty}\frac{(-1)^n}{\alpha i\pi n + \pi^2 n^2},
        \end{align*}
        leading to
        \begin{align*} R_N &= S_N + \alpha\sum_{n=N}^{+\infty}\frac{(-1)^n}{\alpha i\pi n + \pi^2n^2}= -\frac{i}{\pi}\sum_{n=N}^{+\infty}\frac{(-1)^n}{n} + \alpha\sum_{n=N}^{+\infty}\frac{(-1)^n}{\alpha i\pi n + \pi^2n^2}\\
        &= -\frac{i}{\pi}\times\frac{(-1)^N}{2N} - \frac{i}{\pi}\sum_{n=N}^{+\infty}\frac{(-1)^n}{n(n+1)} + \frac{\alpha}{\pi}\sum_{n=N}^{+\infty}\frac{(-1)^n}{\alpha in + \pi n^2}, 
    \end{align*}
    where the last line stems from Lemma~\ref{appendix lemma 12.1}. The result then follows by classical results on the remainder of alternating series.\footnote{Alternating series' criterion: Let $(a_n)$ be a positive and decreasing sequence such that $a_n\rightarrow 0$. Then, the series $\sum_n(-1)^na_n$ converges. Denoting $R_n = \sum_{k=n+1}^{+\infty}(-1)^ka_k$, we have $\vert R_n\vert\leq a_{n+1}$.}
\end{proof}


\begin{proof}\textbf{(Lemma~\ref{Lemma param of bs})} \hspace*{.15cm}
We adopt the parametrization in Eq.~\eqref{Param_of_the_as} for the poles \((a_s)\) and aim to determine the optimal parameters \((b_s)\) under this configuration. Since $\mathcal{L}_\text{time}(c, d)$ is convex with respect to \((b_s)\), setting the gradient to zero yields:
\[
b = C^{-1} \bar{a}^K,
\]
where \(C_{ss'} = \frac{1}{1 - a_s \bar{a}_{s'}}\) for $s, s'$ in $\llbracket -T, T\rrbracket$ where we recall that $S = 2T+1$. 
Let's denote the matrix $M$ such that $M(s, s') = \frac{K}{2\alpha - i\pi(s-s')} + \frac{1}{2}$ for $s, s'$ in $\llbracket -T, T\rrbracket$. 

First, we remark that, for $s\in\llbracket-T, T\rrbracket$, 
\[
(a^K)_s= (-1)^se^{-\alpha}\in\rb.
\]
We derive the asymptotic expansion for the optimal $b$, using a coordinate-wise approach. We recall that we place ourselves in the case $1\ll S \ll K$. Let us denote $z=((-1)^s)_{s\in\llbracket -T, T\rrbracket}$. For $s\in\llbracket-T, T\rrbracket$,

\begin{align}
    \!\!(Cz)_s \!\! &= \!\!(Mz)_s + (Cz)_s - (Mz)_s\notag\\
    &=\!\!\sum_{s'=-T}^T\frac{(-1)^{s'}K}{2\alpha - i\pi(s-s')}\notag\\
    &\quad + \sum_{s'=-T}^T\frac{(-1)^{s'}}{2} + \sum_{s'=-T}^T(-1)^{s'}\big[\frac{1}{1-e^{-\frac{2\alpha}{K}}e^{\frac{i\pi(s-s')}{K}}}-\frac{K}{2\alpha - i\pi(s-s')}-\frac{1}{2}\big]\notag\\
    &=\!\!\sum_{s'=-T}^T\frac{(-1)^{s'}K}{2\alpha - i\pi(s-s')}  + \frac{1}{2} + \sum_{s'=-T}^T(-1)^{s'}\big[\frac{2\alpha-i\pi(s-s')-K(1-e^{-\frac{2\alpha}{K}}e^{\frac{i\pi(s-s')}{K}})}{(1-e^{-\frac{2\alpha}{K}}e^{i\pi(s-s')})(2\alpha - i\pi(s-s'))} - \frac{1}{2}\big]\notag\\
    &=\!\!\sum_{s'=-T}^T\frac{(-1)^{s'}K}{2\alpha - i\pi(s-s')}  \!+\! \frac{1}{2}\! \notag\\
    &\quad +\!\frac{1}{K}\!\sum_{s'=-T}^T \! (-1)^{s'}\big[\frac{\frac{4\alpha^3}{3} - \pi^2 \alpha (s-s')^2 + i \left(\pi^3\frac{(s-s')^3}{6} - 2\pi \alpha^2 (s-s') \right) + o(1)}{(2\alpha-i\pi(s-s')+o(1))(2\alpha-i\pi(s-s'))}\big].\label{Cz approx}
\end{align}
But, 
\[
\bigg[\frac{- \pi^2 \alpha (s-s')^2 + i \left(\pi^3\frac{(s-s')^3}{6} - 2\pi \alpha^2 (s-s') \right) + o(1)}{(2\alpha-i\pi(s-s')+o(1))(2\alpha-i\pi(s-s'))}\bigg] \underset{+\infty}{\sim}i\pi\frac{s-s'}{6},
\]
So there exists a sequence $(\epsilon_{s'})$ for $s'\in\mathbb{Z}$ such that $\epsilon_{s'}\rightarrow 0$ and 
\begin{align}
    &\sum_{s'=-T}^T \! (-1)^{s'}\big[\frac{\frac{4\alpha^3}{3} - \pi^2 \alpha (s-s')^2 + i \left(\pi^3\frac{(s-s')^3}{6} - 2\pi \alpha^2 (s-s') \right) + o(1)}{(2\alpha-i\pi(s-s')+o(1))(2\alpha-i\pi(s-s'))} \notag
    \\&=\sum_{s'=-T}^T(-1)^{s'}\frac{\alpha}{3} +\sum_{s'=-T}^T(-1)^{s'}\frac{i\pi(s-s')}{6}(1+\epsilon_{s'})\notag\\
    &\label{eq:equivalent partial sums 0(1)}=(-1)^T[\frac{\alpha}{3}-\frac{i\pi s}{6}] +\sum_{s'=-T}^T(-1)^{s'}\frac{i\pi(s-s')\epsilon_{s'}}{6} = O(1)
\end{align}
Note that we had to keep the constant term $\frac{4\alpha^3}{3}$ which corresponds to $s=s'$.

Plugging this into eq~\eqref{Cz approx}, this leads to
 \begin{align*}
    (Cz)_s &=\sum_{s'=-T}^T\frac{(-1)^{s'}K}{2\alpha - i\pi(s-s')}  + \frac{1}{2} + O(\frac{1}{K}) =(-1)^s\frac{2K}{e^{2\alpha}-e^{-2\alpha}} + O(\frac{K}{T}),
 \end{align*}   
where we used Lemma~\ref{appendix lemma 12.2}.
We can deduce from this coordinate-wise equation that 
\[
Cz = \frac{2K}{e^{2\alpha}-e^{-2\alpha}}z + O(\frac{K}{T}).
\]
We finally use the bounded nature of $C$'s condition number\footnote{This is due to the link between eigenvalues of Toeplitz matrices and the Fourier series of the first row~\cite{gray2006toeplitz}, and the relationship $C_{ss'} = \frac{1}{1-\exp(-2\alpha/K) \exp( i \pi (s-s')/K)}
\sim \frac{K}{2 \alpha - i\pi ( s-s')}$ together with Eq.~\eqref{eq:FS}.} to apply $C^{-1}$ and to show:
\[ \displaystyle
C^{-1}\bar{a}^K \sim \frac{e^{2\alpha} - e^{-2\alpha}}{2K}z.
\]
This is valid when $T\rightarrow+\infty, T/K\rightarrow 0$.
\end{proof}


\begin{figure}[ht]
    \centering
    \includegraphics[width=1\linewidth]{img/time_domain_filter_S2000K10000.pdf}
    \caption{\textit{Left: Real values of filter in Eq.~\eqref{definition new filter} in the time-domain. Right: Positions of the $a_s$ on the unit disk. In this case, $S=100, K=10000$. The $x$-axis on the left images represents the timesteps. The representation in Eq.~\eqref{definition new filter} concentrates the poles $a_s$ on a slice of the unit disk, whose size depends on the ratio $\frac{S}{K}$. The $a_s$ operate in pairs of complex conjugates, ensuring that the final filter remains real. Each $a_s$ approximates a single oscillation of the complex exponential, with the oscillations spaced by a distance of $\frac{\pi}{K}$. Therefore when $K$ increases (for fixed $S$), the slice size decreases, imposing smaller phase shifts to capture long-range dependencies in the data (see \cite{orvieto2023resurrecting}). This parametrization allows to build filters than can look far back in time.}}
    \label{fig:time domain filter}
\end{figure}

\subsection{Upper bound of the loss}

In this section, we will prove Theorem~\ref{thm upper bound} through an asymptotic expansion. We will use general results on the remainder of alternate series in Lemmas~\ref{appendix lemma 12.1} and~\ref{appendix lemma 12.2}. We also start with a lemma very similar to Lemma~\ref{lemma eigenvector Toeplitz}. We use all of them later in the proof of Theorem~\ref{thm upper bound}.



\begin{lemma}\label{appendix lemma fourier series}
    Let $\alpha\in\mathbb{C}$ such that $\textnormal{Re}(\alpha)>0$. Then,
    \( \displaystyle     \sum_{s=-\infty}^{+\infty}\frac{(-1)^s}{2\alpha - i\pi s} = \frac{2}{e^{2\alpha}-e^{-2\alpha}}.
    \)
\end{lemma}

\begin{proof}
We consider $    \sum_{s=-\infty}^{+\infty}\frac{(-1)^s}{2\alpha - i\pi s}$.
    This looks like a Fourier series evaluated in $\omega=\pi$. We will look for a function $f_\alpha$ such that $c_s(f_\alpha) = \frac{1}{2\alpha -i\pi s}$. Denoting $f_\alpha$ such that $f_\alpha(\omega) = \frac{2e^{\frac{-2\alpha}{\pi}(\omega-\pi)}}{e^{2\alpha}-e^{-2\alpha}}$, we have,
\begin{align*}
    c_s(f_\alpha) &=\frac{1}{2\pi}\int_{0}^{2\pi}\frac{2e^{\frac{2\alpha}{\pi}(\omega-\pi)}}{e^{2\alpha}-e^{-2\alpha}}e^{-is\omega}d\omega=\frac{1}{\pi}\frac{1}{(e^{2\alpha}-e^{-2\alpha})(\frac{2\alpha}{\pi}-is)}(e^{2\alpha}-e^{-2\alpha})=\frac{1}{2\alpha - i\pi s}.
\end{align*}

Therefore, using Dirichlet's theorem, $f_\alpha$ is the appropriate function and
\begin{align*}
   &f_{\alpha}(\pi) = \sum_{s} \frac{(-1)^{s}}{2\alpha -i\pi s}  \Leftrightarrow\sum_{s=-\infty}^{+\infty} \frac{(-1)^{s}}{2\alpha -i\pi s} = \frac{2}{e^{2\alpha}-e^{-2\alpha}}. 
\end{align*}
\end{proof}


\begin{proof}\textbf{(Theorem~\ref{thm upper bound})}\label{proof thm 15} \hspace*{.2cm}
    We recall that we have the following asymptotic representations for our filter:

\begin{align*}
    a_s &= e^{-\frac{\alpha}{K}}e^{i\frac{s\pi}{K}}, s\in\llbracket-T, T\rrbracket\\
    b_s &= \frac{e^{-\alpha}(e^{2\alpha}-e^{-2\alpha})}{2K}(-1)^s,s\in\llbracket-T, T\rrbracket,
\end{align*}
and that $\mathcal{L}_\text{time}(c, d)$ can therefore write:
\begin{align*}
    \mathcal{L}_\text{time}(c, d) &= \sum_{s, s'=-T}^T\frac{b_s\bar{b}_{s'}}{1-a_s\bar{a}_{s'}} - 2\sum_{s=-T}^Tb_sa_s^K + 1\\
    &= \frac{e^{-2\alpha}(e^{2\alpha}-e^{-2\alpha})^2}{4K^2}\sum_{s, s'=-T}^T\frac{(-1)^{s+s'}}{1 - e^{-\frac{2\alpha}{K}}e^{(s-s')i\frac{\pi}{K}}} \\
    &\qquad - 2\sum_{s=-T}^T\frac{(-1)^se^{-2\alpha}(e^{2\alpha}-e^{-2\alpha})(-1)^s}{2K} + 1\\
    &=  \frac{e^{-2\alpha}(e^{2\alpha}-e^{-2\alpha})^2}{4K^2}\sum_{s, s'=-T}^T\frac{(-1)^{s+s'}}{1 - e^{-\frac{2\alpha}{K}}e^{(s-s')i\frac{\pi}{K}}} - \frac{e^{-2\alpha}(e^{2\alpha}-e^{-2\alpha})S}{K} + 1.
\end{align*}

First, we prove that 
\begin{equation}\label{approx O(1) loss}
    \sum_{s, s'=-T}^T\frac{(-1)^{s+s'}}{1 - e^{\frac{-2\alpha}{K}}e^{(s-s')i\frac{\pi}{K}}} = \sum_{s=-T}^T(-1)^sK\sum_{s'=-T}^T\frac{e^{i\pi s'}}{2\alpha - i\pi(s-s')} + O(1),
\end{equation}
when $T\rightarrow +\infty, T/K \rightarrow 0$. Let us compute:
\begin{align*}
        &\sum_{s, s'=-T}^T\frac{(-1)^{s+s'}}{1 - e^{\frac{-2\alpha}{K}}e^{(s-s')i\frac{\pi}{K}}} - \sum_{s=-T}^T(-1)^sK\sum_{s'=-T}^T\frac{e^{i\pi s'}}{2\alpha - i\pi(s-s')}\\
        =&\sum_{s, s'=-T}^T(-1)^{s+s'}\big[\frac{1}{1-e^{-\frac{2\alpha}{K}}e^{i\frac{\pi}{K}(s-s')}} - \frac{K}{2\alpha - i\pi(s-s')}\big]\\
        =&\sum_{s, s'=-T}^T(-1)^{s+s'}\big[\frac{2\alpha -i\pi(s-s')-K[1-e^{-\frac{2\alpha}{K}}e^{i\frac{\pi}{K}(s-s')}]}{(1-e^{-\frac{2\alpha}{K}}e^{i\frac{\pi}{K}(s-s')})(2\alpha - i\pi(s-s'))}\big].\\
\end{align*}
Let us finish the computation:
\begin{align*}
        =&\sum_{s, s'=-T}^T(-1)^{s+s'}\frac{2\alpha -i\pi(s-s')-[2\alpha - i\pi(s-s')-\frac{4\alpha^2}{2K}+\frac{\pi^2}{2K}(s-s')^2+\frac{4i\pi\alpha}{K}(s-s') +o(\frac{1}{K})]}{(1-e^{-\frac{2\alpha}{K}}e^{i\frac{\pi}{K}(s-s')})(2\alpha - i\pi(s-s'))}\\
        =&\sum_{s, s'=-T}^T\frac{(-1)^{s+s'}}{K}\times\frac{2\alpha^2 - \frac{\pi^2}{2}(s-s')^2 - 2i\pi\alpha(s-s')+o(1)}{(1-e^{-\frac{2\alpha}{K}}e^{i\frac{\pi}{K}(s-s')})(2\alpha - i\pi(s-s'))}\\
        =&\sum_{n=-2T}^{2T}\frac{(-1)^n}{K}\times\frac{2\alpha^2-\frac{\pi^2n^2}{2}-2i\pi\alpha n+o(1)}{(1-e^{-\frac{2\alpha}{K}}e^{i\frac{\pi}{K}n})(2\alpha - i\pi n)}(2T + 1-\vert n\vert) \text{ with the change of variable $n=s-s'$}\\
        =&\sum_{n=-2T}^{2T}\frac{(-1)^n}{K} (2T+1-\vert n\vert)f(n) \text{ with $f(n) = \frac{2\alpha^2-\frac{\pi^2n^2}{2}-2i\pi\alpha n+o(1)}{(1-e^{-\frac{2\alpha}{K}}e^{i\frac{\pi}{K}n})(2\alpha - i\pi n)}$}.
\end{align*}
But notice that \begin{align*}
    f(n) &= \frac{2\alpha^2-\frac{\pi^2n^2}{2}-2i\pi\alpha n+o(1)}{(1-e^{-\frac{2\alpha}{K}}e^{i\frac{\pi}{K}n})(2\alpha - i\pi n)} = \frac{K(2\alpha^2-\frac{\pi^2n^2}{2}-2i\pi\alpha n+o(1))}{(2\alpha - i\pi n +o(1))(2\alpha - i\pi n)},\\
\end{align*}
therefore
\begin{align*}
\sum_{n=-2T}^{2T}\frac{(-1)^n}{K} (2T+1-\vert n\vert)f(n) &= \sum_{n=-2T}^{2T}(-1)^n (2T+1-\vert n\vert) \frac{2\alpha^2-\frac{\pi^2n^2}{2}-2i\pi\alpha n+o(1)}{(2\alpha - i\pi n +o(1))(2\alpha - i\pi n)}\\
&\sim\sum_{n=-2T}^{2T}(-1)^n(2T+1-\vert n\vert)\frac{-\frac{\pi^2n^2}{2}}{-\pi^2n^2}\sim\frac{1}{2},
\end{align*}
using the same reasoning as for equation ~\eqref{eq:equivalent partial sums 0(1)}.
This shows 
\begin{equation}\label{O(1) intermediate}
    \sum_{n=-2T}^{2T}\frac{(-1)^n}{K}(2T+1-\vert n\vert)f(n) = O(1).
\end{equation}
We can therefore conclude:
\begin{align*}
    &\mathcal{L}_{\text{time}}(c,d) = 1 - \frac{e^{-2\alpha}(e^{2\alpha}-e^{-2\alpha})S}{K} + \frac{e^{-2\alpha}(e^{2\alpha}-e^{-2\alpha})^2}{4K^2}\times\sum_{s=-T}^T(-1)^sK\sum_{s'=-T}^T\frac{e^{i\pi s'}}{2\alpha - i\pi(s-s')} + O(\frac{1}{K^2})\\
    &=1 - \frac{e^{-2\alpha}(e^{2\alpha}-e^{-2\alpha})S}{K} + \frac{e^{-2\alpha}(e^{2\alpha}-e^{-2\alpha})^2}{4K}\big[\sum_{s=-T}^T(-1)^s\big(\sum_{s'=-\infty}^{+\infty}\frac{(-1)^{s'}}{(2\alpha-i\pi s)+i\pi s'} + O(\frac{1}{T})\big)\big] + O(\frac{1}{K^2})\\
    &=1 - \frac{e^{-2\alpha}(e^{2\alpha}-e^{-2\alpha})S}{K} + \frac{e^{-2\alpha}(e^{2\alpha}-e^{-2\alpha})^2}{4K}\big[\sum_{s=-T}^T(-1)^s\frac{2\times(-1)^s}{e^{2\alpha}-e^{-2\alpha}} +O(\frac{1}{T})\big] + O(\frac{1}{K^2})\\
    &= 1 - \frac{e^{-2\alpha}(e^{2\alpha}-e^{-2\alpha})S}{K} + \frac{e^{-2\alpha}(e^{2\alpha}-e^{-2\alpha})S}{2K} + O(\frac{1}{TK})\\
    &=  1 - \frac{e^{-2\alpha}(e^{2\alpha}-e^{-2\alpha})S}{2K} + O(\frac{1}{TK}),
\end{align*}
where we used Lemmas~\ref{appendix lemma 12.2} and~\ref{appendix lemma fourier series}.
\end{proof}

\subsection{Proof of Theorem~\ref{convergence to window}}

In this section, we give the proof for Theorem~\ref{convergence to window} that describes the asymptotic behavior of the transfer function $C(e^{i\omega})$. First, we refer the reader to Lemmas~\ref{appendix lemma 12.1} and~\ref{appendix lemma 12.2} where we derive results on the remainder of some series. Then in Lemmas~\ref{appendix lemma 12.3} and~\ref{appendix lemma 12.4}, we derive an asymptotic new form for the transfer function. We combine all these lemmas to prove Theorem~\ref{convergence to window}.


\begin{lemma}\label{appendix lemma 12.3}
For $\alpha$ real and positive and $\Omega$ real,
\begin{equation*}
    \frac{1}{2}\sum_{s=-T}^T\frac{e^{-\alpha}(e^{2\alpha}-e^{-2\alpha})(-1)^s}{K(1 - e^{-\alpha/K}e^{i\pi(\frac{s}{K}-\frac{\Omega}{K})})} = \frac{1}{2}\sum_{s=-T}^T\frac{e^{-\alpha}(e^{2\alpha}-e^{-2\alpha})(-1)^s}{\alpha - i\pi (s-\Omega)} + O\big(\frac{1}{K}\big),
\end{equation*}
as $T\rightarrow+\infty, T/K\rightarrow 0$.
    
\end{lemma}

\begin{proof}
    \begin{align*}
         &\sum_{s=-T}^T\frac{(-1)^s}{K(1 - e^{-\alpha/K}e^{i\pi(\frac{s}{K}-\frac{\Omega}{K})})} - \sum_{s=-T}^T\frac{(-1)^s}{\alpha - i\pi (s-\Omega)}\\
         &= \sum_{s=-T}^T(-1)^s\big[\frac{1}{K(1-e^{-\alpha/K}e^{i\frac{\pi}{K}(s-\Omega)})} - \frac{1}{\alpha -i\pi (s - \Omega)}\big]\\
         &= \sum_{s=-T}^T(-1)^s\frac{\alpha -i\pi(s-\Omega)-K(1 - e^{-\alpha/K}e^{\frac{i\pi}{K}(s-\Omega)})}{K\big(1-e^{-\alpha/K}e^{i\frac{\pi}{K}(s-\Omega)}\big)\big(\alpha - i\pi (s-\Omega)\big)}\\
         &= \sum_{s=-T}^T(-1)^s\frac{\alpha - i\pi(s-\Omega) - K\big(\frac{\alpha}{K} - \frac{i\pi}{K}(s-\Omega) - \frac{\alpha^2}{2K^2} + \frac{\pi^2}{2K^2}(s-\Omega)^2 + \frac{i\pi\alpha}{K^2}(s-\Omega)+o(\frac{1}{K^2})\big)}{K\big(1-e^{-\alpha/K}e^{i\frac{\pi}{K}(s-\Omega)}\big)\big(\alpha - i\pi (s-\Omega)\big)}\\
         &= \sum_{s=-T}^T(-1)^s\frac{\frac{\alpha^2}{2K} - \frac{\pi^2}{2K}(s-\Omega)^2 - \frac{i\pi\alpha}{K}(s-\Omega) + o(\frac{1}{K})}{K\big(1-e^{-\alpha/K}e^{i\frac{\pi}{K}(s-\Omega)}\big)\big(\alpha - i\pi (s-\Omega)\big)}\\
         &=\frac{1}{K}\sum_{s=-T}^T(-1)^s\frac{\frac{\alpha^2}{2} - \frac{\pi^2}{2}(s-\Omega)^2-i\pi\alpha(s-\Omega)+o(1)}{[\alpha-i\pi(s-\Omega)+o(1)][\alpha-i\pi(s-\Omega)]}=O\big(\frac{1}{K}\big).
    \end{align*}
    We used a similar argument as the one in eq.~\eqref{eq:equivalent partial sums 0(1)} and eq.~\eqref{O(1) intermediate}.
    The constant $\frac{e^{-\alpha}(e^{2\alpha}-e^{-2\alpha})}{2}$ does not impact our computation. 
\end{proof}
    
\begin{lemma}\label{appendix lemma 12.4}
For $\alpha$ real and positive, 
\[
\frac{1}{2}\sum_{s=-T}^T\frac{(-1)^se^{-\alpha}(e^{2\alpha}-e^{-2\alpha})}{\alpha - i\pi(s-\Omega)} \underset{S/K\rightarrow 0}{\underset{S\rightarrow+\infty}{\sim} }
\begin{cases}
    \frac{e^{-\alpha}(e^{2\alpha}-e^{-2\alpha})}{2}\times\frac{i(-1)^{T+1}\times 2\lfloor\Omega\rfloor}{2\pi(\lfloor\Omega\rfloor-T)(\lfloor\Omega\rfloor+T)} & \text{if } \vert\Omega\vert > T, \\
\frac{e^{-\alpha}(e^{2\alpha}-e^{-2\alpha})}{e^{\alpha}e^{i\pi\Omega}-e^{-\alpha}e^{-i\pi\Omega}} & \text{if } \vert\Omega\vert < T.
\end{cases}
\] 
\end{lemma}

\begin{proof}
    We decompose $\Omega = \lfloor\Omega\rfloor + \beta = n + \beta$, and look at two different cases.
    
    \paragraph{First case:}$\vert\Omega\vert < T$. We have:
    
\begin{align*}
        &\frac{1}{2}\sum_{s=-T}^T(-1)^s\frac{e^{-\alpha}(e^{2\alpha}-e^{-2\alpha})}{\alpha - i\pi(s-n-\beta)}\\
        &=\frac{(-1)^n}{2}\sum_{s=-(T+n)}^{T-n}\frac{e^{-\alpha}(e^{2\alpha}-e^{-2\alpha})}{\alpha - i\pi(s-\beta)} =\frac{(-1)^n}{2}\sum_{s=-(T+n)}^{T-n}\frac{e^{-\alpha}(e^{2\alpha}-e^{-2\alpha})}{\tilde{\alpha} - i\pi s} \text{ where }\tilde{\alpha} = \alpha+i\pi\beta\\
        &=\frac{(-1)^n}{2}\sum_{s=-\infty}^{+\infty}\frac{e^{-\alpha}(e^{2\alpha}-e^{-2\alpha})}{\tilde{\alpha}-i\pi s}\\
        &+ \big[\frac{(-1)^n}{2}\sum_{s=-(T+n)}^{T-n}\frac{e^{-\alpha}(e^{2\alpha}-e^{-2\alpha})}{\tilde{\alpha}-i\pi s} -  \frac{(-1)^n}{2}\sum_{s=-\infty}^{+\infty}\frac{e^{-\alpha}(e^{2\alpha}-e^{-2\alpha})}{\tilde{\alpha}-i\pi s}\big]\\
        &=\frac{(-1)^n}{2}\sum_{s=-\infty}^{+\infty}\frac{e^{-\alpha}(e^{2\alpha}-e^{-2\alpha})}{\tilde{\alpha}-i\pi s} -\frac{i}{\pi}\times\frac{(-1)^{T-n+1}}{2(T-n+1)} + \frac{i}{\pi}\times\frac{(-1)^{T+n+1}}{2(T+n+1)} + o\big(\frac{1}{T-n}\big) \text{ (Lemma~\ref{appendix lemma 12.2})}\\
        &= (-1)^n\frac{e^{-\alpha}(e^{2\alpha}-e^{-2\alpha})}{e^{\alpha}e^{i\pi\beta} - e^{-\alpha}e^{-i\pi\beta}} + O\big(\frac{1}{{T}}\big) = \frac{e^{-\alpha}(e^{2\alpha}-e^{-2\alpha})}{e^{\alpha}e^{i\pi\Omega} - e^{-\alpha}e^{-i\pi\Omega}} +  O\big(\frac{1}{{T}}\big).
\end{align*}

    \paragraph{Second case:} $\vert\Omega\vert > T$. We consider again: 
    \begin{align*}
         \frac{(-1)^n}{2}\sum_{s=-(T+n)}^{T-n}\frac{e^{-\alpha}(e^{2\alpha}-e^{-2\alpha})}{\tilde{\alpha} - i\pi s} 
        &=\frac{(-1)^ne^{-\alpha}(e^{2\alpha}-e^{-2\alpha})}{2}\big[\sum_{s=-\infty}^{T-n}\frac{1}{\tilde{\alpha} - i\pi s} - \sum_{s=-\infty}^{-T-n-1}\frac{1}{\tilde{\alpha} - i\pi s}\big]\\
        &=\frac{(-1)^ne^{-\alpha}(e^{2\alpha}-e^{-2\alpha})}{2}\big[\sum_{s=n-T}^{+\infty}\frac{1}{\tilde{\alpha}+i\pi s} - \sum_{s=T+n+1}^{+\infty}\frac{1}{\tilde{\alpha}+i\pi s}
        \big]\\
        &= \frac{e^{-\alpha}(e^{2\alpha}-e^{-2\alpha})}{2}\times \frac{i(-1)^{T+1}\times 2n}{2\pi(n-T)(T+n)} + o\big(\frac{1}{T}\big) \text{\quad (Lemma~\ref{appendix lemma 12.2}) }.
    \end{align*}
\end{proof} 

\begin{proof}\textbf{(Theorem~\ref{convergence to window})} \hspace*{.2cm}
For $\Omega\in\mathbb{R}$, 
\begin{align*}
    C(e^{i\omega}) &= \sum_{s=-T}^T\frac{e^{-\alpha}(e^{2\alpha}-e^{-2\alpha})}{2K}\frac{(-1)^s}{1-e^{-\alpha/K}e^{i\frac{\pi}{K}(s-\Omega)}}\\
    &= \sum_{s=-T}^T\frac{e^{-\alpha}(e^{2\alpha}-e^{-2\alpha})}{2}\times\frac{(-1)^s}{\alpha - i\pi(s-\Omega)}\\
    &+\big[\sum_{s=-T}^T\frac{e^{-\alpha}(e^{2\alpha}-e^{-2\alpha})}{2K}\times\frac{(-1)^s}{1-e^{-\alpha/K}e^{i\frac{\pi}{K}(s-\Omega)}} - \sum_{s=-T}^T\frac{e^{-\alpha}(e^{2\alpha}-e^{-2\alpha})}{2}\frac{(-1)^s}{\alpha - i\pi(s-\Omega)}
    \big].
\end{align*}
We then use Lemma~\ref{appendix lemma 12.3} and Lemma~\ref{appendix lemma 12.4} to conclude.
    
\end{proof}

    
