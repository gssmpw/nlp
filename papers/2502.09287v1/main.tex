
\documentclass[12pt]{article}
\usepackage{fullpage}
\usepackage[round]{natbib}
\bibliographystyle{plainnat}
\usepackage{float}
\usepackage{amsmath}
\usepackage{amssymb}
\usepackage{amsthm}
\newtheorem{theorem}{Theorem}
\newtheorem{lemma}{Lemma}
\newtheorem{proposition}{Proposition}
\newtheorem{definition}{Definition}
\usepackage{amsfonts}
\usepackage{times}
\usepackage{stmaryrd}
\usepackage{hyperref}
\usepackage{mwe}
\usepackage{graphicx} 
\usepackage{wrapfig} 
\usepackage{subcaption}
\usepackage{booktabs}
\usepackage{array}
\renewenvironment{proof}[1][Proof]{\noindent\textbf{#1.} }{\qed}


\usepackage[page]{appendix}
\usepackage{etoolbox}

\renewcommand{\appendixtocname}{Appendix Contents.}


\makeatletter
\renewcommand{\appendices}{%
  \clearpage
  \renewcommand{\thesection}{\Alph{section}}
  \renewcommand{\appendixname}{}
  \renewcommand{\appendixpagename}{}
  \let\tf@toc\tf@app
  \addtocontents{app}{\protect\setcounter{tocdepth}{2}}
  \immediate\write\@auxout{%
    \string\let\string\tf@toc\string\tf@app^^J
  }
}


\newcommand{\listofappendices}{%
  \begingroup
  \renewcommand{\contentsname}{\appendixtocname}
  \let\@oldstarttoc\@starttoc
  \def\@starttoc##1{\@oldstarttoc{app}}
  \tableofcontents
  \endgroup
}
\usepackage[utf8]{inputenc}
\newcommand{\BEAS}{\begin{eqnarray*}}
\newcommand{\EEAS}{\end{eqnarray*}}
\newcommand{\BEA}{\begin{eqnarray}}
\newcommand{\EEA}{\end{eqnarray}}
\newcommand{\BEQ}{\begin{equation}}
\newcommand{\EEQ}{\end{equation}}
\newcommand{\BIT}{\begin{itemize}}
\newcommand{\EIT}{\end{itemize}}
\newcommand{\BNUM}{\begin{enumerate}}
\newcommand{\ENUM}{\end{enumerate}}
\newcommand{\BA}{\begin{array}}
\newcommand{\EA}{\end{array}}
\newcommand{\diag}{\mathop{\textnormal{ diag}}}
\newcommand{\Diag}{\mathop{\textnormal{ Diag}}}
\newcommand{\rb}{\mathbb{{R}}}
\renewcommand{\labelitemiii}{$-$}
\def\defin{\stackrel{\vartriangle}{=}}
\def \tX{ \widetilde{X}}
\def \tY{ \widetilde{Y}}
\def \ds { \displaystyle}
\def \hS{ \widehat{ \Sigma} }
\def \S{  { \Sigma} }
\def \L{  { \Lambda} }
\def \E{{\mathbb E}}
\def \P{{\mathbb P}}
\def \Z{{\mathbb Z}}
\def \T{{\mathbb T}}
\def \F{{\mathcal F}}
\def \C{{\mathbb C}}
\def \M{{\mathcal M}}
\def \H{{\mathcal H}}
\def \U{{\mathcal U}}
\def \X{{\mathcal X}}
\def \Y{{\mathcal Y}}
\def \A{{\mathcal A}}
\def \S{{\mathcal S}}
\def \hQ{{\hat{Q}}}
 \def \hfl{ \hat{f}_\lambda }
 \def \hal{ \hat{\alpha}_\lambda }
  \def \fl{ {f}_\lambda }
\def \supp{ { \textnormal{ Supp }}}
\def \card{ { \rm Card }}
 \def \P{{\mathbb P}}
 



\title{\textbf{An Uncertainty Principle\\ for Linear Recurrent Neural Networks}}
\author{
  \textbf{Alexandre Fran\c{c}ois}\\
  INRIA, Ecole Normale Sup\'erieure, PSL Research University, France\\
  \texttt{alexandre.francois@inria.fr}
  \and
  \textbf{Antonio Orvieto}\\
  MPI for Intelligent Systems, ELLIS Institute T\"ubingen, Germany\\
  \texttt{antonio@tue.ellis.eu}
  \and
  \textbf{Francis Bach}\\
  INRIA, Ecole Normale Sup\'erieure, PSL Research University, France\\
  \texttt{francis.bach@inria.fr}
}
\date{}

\begin{document}
\maketitle

\begin{abstract}
 We consider linear recurrent neural networks, which have become a key building block of sequence modeling due to their ability for stable and effective long-range modeling. In this paper, we aim at characterizing this ability on a simple but core copy task, whose goal is to build a linear filter of order $S$ that approximates the filter that looks $K$ time steps in the past (which we refer to as the shift-$K$ filter), where $K$ is larger than $S$. Using classical signal models and quadratic cost, we fully characterize the problem by providing lower bounds of approximation, as well as explicit filters that achieve this lower bound up to constants. The optimal performance highlights an uncertainty principle: the optimal filter has to average values around the $K$-th time step in the past with a range~(width) that is proportional to $K/S$.

\end{abstract}

\section{Introduction}
\label{sec:introduction}
The business processes of organizations are experiencing ever-increasing complexity due to the large amount of data, high number of users, and high-tech devices involved \cite{martin2021pmopportunitieschallenges, beerepoot2023biggestbpmproblems}. This complexity may cause business processes to deviate from normal control flow due to unforeseen and disruptive anomalies \cite{adams2023proceddsriftdetection}. These control-flow anomalies manifest as unknown, skipped, and wrongly-ordered activities in the traces of event logs monitored from the execution of business processes \cite{ko2023adsystematicreview}. For the sake of clarity, let us consider an illustrative example of such anomalies. Figure \ref{FP_ANOMALIES} shows a so-called event log footprint, which captures the control flow relations of four activities of a hypothetical event log. In particular, this footprint captures the control-flow relations between activities \texttt{a}, \texttt{b}, \texttt{c} and \texttt{d}. These are the causal ($\rightarrow$) relation, concurrent ($\parallel$) relation, and other ($\#$) relations such as exclusivity or non-local dependency \cite{aalst2022pmhandbook}. In addition, on the right are six traces, of which five exhibit skipped, wrongly-ordered and unknown control-flow anomalies. For example, $\langle$\texttt{a b d}$\rangle$ has a skipped activity, which is \texttt{c}. Because of this skipped activity, the control-flow relation \texttt{b}$\,\#\,$\texttt{d} is violated, since \texttt{d} directly follows \texttt{b} in the anomalous trace.
\begin{figure}[!t]
\centering
\includegraphics[width=0.9\columnwidth]{images/FP_ANOMALIES.png}
\caption{An example event log footprint with six traces, of which five exhibit control-flow anomalies.}
\label{FP_ANOMALIES}
\end{figure}

\subsection{Control-flow anomaly detection}
Control-flow anomaly detection techniques aim to characterize the normal control flow from event logs and verify whether these deviations occur in new event logs \cite{ko2023adsystematicreview}. To develop control-flow anomaly detection techniques, \revision{process mining} has seen widespread adoption owing to process discovery and \revision{conformance checking}. On the one hand, process discovery is a set of algorithms that encode control-flow relations as a set of model elements and constraints according to a given modeling formalism \cite{aalst2022pmhandbook}; hereafter, we refer to the Petri net, a widespread modeling formalism. On the other hand, \revision{conformance checking} is an explainable set of algorithms that allows linking any deviations with the reference Petri net and providing the fitness measure, namely a measure of how much the Petri net fits the new event log \cite{aalst2022pmhandbook}. Many control-flow anomaly detection techniques based on \revision{conformance checking} (hereafter, \revision{conformance checking}-based techniques) use the fitness measure to determine whether an event log is anomalous \cite{bezerra2009pmad, bezerra2013adlogspais, myers2018icsadpm, pecchia2020applicationfailuresanalysispm}. 

The scientific literature also includes many \revision{conformance checking}-independent techniques for control-flow anomaly detection that combine specific types of trace encodings with machine/deep learning \cite{ko2023adsystematicreview, tavares2023pmtraceencoding}. Whereas these techniques are very effective, their explainability is challenging due to both the type of trace encoding employed and the machine/deep learning model used \cite{rawal2022trustworthyaiadvances,li2023explainablead}. Hence, in the following, we focus on the shortcomings of \revision{conformance checking}-based techniques to investigate whether it is possible to support the development of competitive control-flow anomaly detection techniques while maintaining the explainable nature of \revision{conformance checking}.
\begin{figure}[!t]
\centering
\includegraphics[width=\columnwidth]{images/HIGH_LEVEL_VIEW.png}
\caption{A high-level view of the proposed framework for combining \revision{process mining}-based feature extraction with dimensionality reduction for control-flow anomaly detection.}
\label{HIGH_LEVEL_VIEW}
\end{figure}

\subsection{Shortcomings of \revision{conformance checking}-based techniques}
Unfortunately, the detection effectiveness of \revision{conformance checking}-based techniques is affected by noisy data and low-quality Petri nets, which may be due to human errors in the modeling process or representational bias of process discovery algorithms \cite{bezerra2013adlogspais, pecchia2020applicationfailuresanalysispm, aalst2016pm}. Specifically, on the one hand, noisy data may introduce infrequent and deceptive control-flow relations that may result in inconsistent fitness measures, whereas, on the other hand, checking event logs against a low-quality Petri net could lead to an unreliable distribution of fitness measures. Nonetheless, such Petri nets can still be used as references to obtain insightful information for \revision{process mining}-based feature extraction, supporting the development of competitive and explainable \revision{conformance checking}-based techniques for control-flow anomaly detection despite the problems above. For example, a few works outline that token-based \revision{conformance checking} can be used for \revision{process mining}-based feature extraction to build tabular data and develop effective \revision{conformance checking}-based techniques for control-flow anomaly detection \cite{singh2022lapmsh, debenedictis2023dtadiiot}. However, to the best of our knowledge, the scientific literature lacks a structured proposal for \revision{process mining}-based feature extraction using the state-of-the-art \revision{conformance checking} variant, namely alignment-based \revision{conformance checking}.

\subsection{Contributions}
We propose a novel \revision{process mining}-based feature extraction approach with alignment-based \revision{conformance checking}. This variant aligns the deviating control flow with a reference Petri net; the resulting alignment can be inspected to extract additional statistics such as the number of times a given activity caused mismatches \cite{aalst2022pmhandbook}. We integrate this approach into a flexible and explainable framework for developing techniques for control-flow anomaly detection. The framework combines \revision{process mining}-based feature extraction and dimensionality reduction to handle high-dimensional feature sets, achieve detection effectiveness, and support explainability. Notably, in addition to our proposed \revision{process mining}-based feature extraction approach, the framework allows employing other approaches, enabling a fair comparison of multiple \revision{conformance checking}-based and \revision{conformance checking}-independent techniques for control-flow anomaly detection. Figure \ref{HIGH_LEVEL_VIEW} shows a high-level view of the framework. Business processes are monitored, and event logs obtained from the database of information systems. Subsequently, \revision{process mining}-based feature extraction is applied to these event logs and tabular data input to dimensionality reduction to identify control-flow anomalies. We apply several \revision{conformance checking}-based and \revision{conformance checking}-independent framework techniques to publicly available datasets, simulated data of a case study from railways, and real-world data of a case study from healthcare. We show that the framework techniques implementing our approach outperform the baseline \revision{conformance checking}-based techniques while maintaining the explainable nature of \revision{conformance checking}.

In summary, the contributions of this paper are as follows.
\begin{itemize}
    \item{
        A novel \revision{process mining}-based feature extraction approach to support the development of competitive and explainable \revision{conformance checking}-based techniques for control-flow anomaly detection.
    }
    \item{
        A flexible and explainable framework for developing techniques for control-flow anomaly detection using \revision{process mining}-based feature extraction and dimensionality reduction.
    }
    \item{
        Application to synthetic and real-world datasets of several \revision{conformance checking}-based and \revision{conformance checking}-independent framework techniques, evaluating their detection effectiveness and explainability.
    }
\end{itemize}

The rest of the paper is organized as follows.
\begin{itemize}
    \item Section \ref{sec:related_work} reviews the existing techniques for control-flow anomaly detection, categorizing them into \revision{conformance checking}-based and \revision{conformance checking}-independent techniques.
    \item Section \ref{sec:abccfe} provides the preliminaries of \revision{process mining} to establish the notation used throughout the paper, and delves into the details of the proposed \revision{process mining}-based feature extraction approach with alignment-based \revision{conformance checking}.
    \item Section \ref{sec:framework} describes the framework for developing \revision{conformance checking}-based and \revision{conformance checking}-independent techniques for control-flow anomaly detection that combine \revision{process mining}-based feature extraction and dimensionality reduction.
    \item Section \ref{sec:evaluation} presents the experiments conducted with multiple framework and baseline techniques using data from publicly available datasets and case studies.
    \item Section \ref{sec:conclusions} draws the conclusions and presents future work.
\end{itemize}
\subsection{Lower bounds on sample complexity}\label{sec:sample_compexity}
We establish a lower bound for generalized linear measurements using standard information-theoretic arguments based on Fano's inequality. While the upper bound in Theorem~\ref{thm:alg_general} is derived for the maximum probability of error over all  $k$-sparse vectors, the lower bound applies even in the weaker setting of the average probability of error, where 
$\bx$ is chosen uniformly at random.
\begin{theorem}[Lower bound for GLMs]\label{thm: lower_bdglm} Consider any  sensing matrix $\vecA$.
For a uniformly chosen $k$-sparse vector $\bx$, an algorithm $\phi$ satisfies $$\bbP\inp{\phi(\vecA, \by) \neq \bx}\leq \delta$$   only if the number of measurements $$m\geq \frac{k\log\inp{\frac{n}{k}}}{I}\inp{1 - \frac{h_2(\delta) + \delta k\log{n}}{k\log{n/k}}}$$ for some $I$ such that $I\geq {I(y_i; \bx|\vecA)}, \, i\in [m]$. In particular, when $y\in \inb{-1, 1}$, we have $\bbE\insq{\inp{g(\vecA_i^T\bx)}^2} \geq I(y_i, \bx|\vecA)$ where the expectation is over the randomness of $\vecA$ and $\bx$.
\end{theorem}
The lower bound can be interpreted in terms of a communication problem, where the input message $\bx$ is encoded to $\vecA\bx$. The decoding function takes in as input the encoding map $\vecA$ and the output vector $\by$ in order to recover $\bx$ with high probability. For optimal recovery, one needs at least $\frac{\text{message entropy}}{\text{capacity}}$ number of measurements (follows from noisy channel coding theorem~\cite{thomas2006elements}). In Theorem~\ref{thm: lower_bdglm}, the entropy of the message set $\log{n \choose k}\approx k\log{n/k}$ and the proxy for capacity is the upper bound on mutual information $I$. We provide a detailed proof of the theorem in  Section~\ref{sec:proofs}.


We first present lower bounds for \bcs\  and \logreg. The lower bound for \bcs\ is given for any sensing matrix $\vecA$ which satisfies the power constraint given by \eqref{eq:power_constraint}, whereas the one for \logreg\ is only for the special case when each entry of the sensing matrix is iid $\cN(0,1)$. Recall that \eqref{eq:power_constraint} holds in this case.  For \bcs\ (and \logreg\ respectively), we can use the upper bound of $\bbE\insq{\inp{g(\vecA_i^T\bx)}^2}$ on the mutual information term. The dependence of $\sigma^2$ (and $1/\beta^2$ respectively) requires careful bounding of this term, which is done in the formal proofs in Appendix~\ref{proof:sec:lower_bd}.


As mentioned earlier, we need at least $k\log\inp{n/k}$ measurements for \bcs and \logreg. This is because the entropy of a randomly chosen $k$-sparse vector is approximately $k\log\inp{n/k}$ and we learn at most one bit with each measurement. However, due to corruption with noise, we learn less than a bit of information about the unknown signal with each measurement. The information gain gets worse as the noise level increases. 
Our lower bounds make this reasoning explicit.  
\begin{corollary}[\bcs\ lower bound]\label{thm: lower_bd_bcs} Suppose, each row $\vecA_i, \, i\in [1:m]$ of the sensing matrix $\vecA$ satisfies the power constraint~\eqref{eq:power_constraint}.
For a uniformly chosen $k$-sparse vector $\bx$, an algorithm $\phi$ satisfies $$\bbP\inp{\phi(\vecA, {\by}) \neq \bx}\leq \delta$$ for the problem of $\bcs$ only if the number of measurements $$m\geq \frac{k+\sigma^2}{2}\log\inp{\frac{n}{k}}\inp{1 - \frac{h_2(\delta) + \delta k\log{n}}{k\log{n/k}}}.$$ 
\end{corollary}

\begin{corollary}[\logreg\ lower bound]\label{thm: lower_bd_log_reg} Consider a Gaussian  sensing matrix $\vecA$ where each entry is chosen iid $N(0,1)$.
For a uniformly chosen $k$-sparse vector $\bx$, an algorithm $\phi$ satisfies $$\bbP\inp{\phi(\vecA, \bw) \neq \bx}\leq \delta$$ for the problem of $\logreg$ only if the number of measurements $$m\geq \frac{1}{2}\inp{k+\frac{1}{\beta^2}}\log\inp{\frac{n}{k}}\inp{1 - \frac{h_2(\delta) + \delta k\log{n}}{k\log{n/k}}}.$$ 
\end{corollary}



Theorem~\ref{thm: lower_bdglm} also implies an information theoretic lower bound for \spl, which is presented below and proved in Appendix~\ref{proof:sec:lower_bd}. Note that the denominator term in the bound $\frac{1}{2}\log\inp{1+\frac{k}{\sigma^2}}$ is the capacity of a Gaussian channel with power constraint $k$ and noise variance $\sigma^2$. 
\begin{corollary}[\spl\ lower bound]\label{thm: spl_lower_bd_1}
Under the average power constraint \eqref{eq:power_constraint} on  $\vecA$, for a uniformly chosen $k$-sparse vector $\bx$, an algorithm $\phi$ satisfies $$\bbP\inp{\phi(\vecA, {\by}) \neq \bx}\leq \delta$$ only if the number of measurements
$$m\geq \frac{k\log\inp{\frac{n}{k}}-\inp{h_2(\delta) + \delta k\log{n}}}{\frac{1}{2}\log\inp{1+\frac{k}{\sigma^2}}}.$$
\end{corollary} 

\subsection{Tighter upper and lower bounds for \spl}\label{sec:tighter_bounds_spl}
We present information theoretic upper and lower bounds for \spl\ in this section. Similar to Section~\ref{sec:alg}, our upper bound is for the maximum probability of error, while the lower bounds hold even for the weaker criterion of average probability of error.

We first present an upper bound based on the maximum likelihood estimator (MLE) where  we  decode to $\hat{\bx}$ if, on output $\by$, 
\begin{align*}
\hat{\bx} = \argmax_{\stackrel{\bx\in \inb{0,1}^n}{\wh{\bx} = k}}\,\, p(\by|{\bx})
\end{align*} where $p(\by|{\bx})$ denotes the probability density function of $\by$ on input $\bx$.
\begin{theorem}[MLE upper bound for \spl]\label{thm:upper_bd_mle} Suppose  entries of the measurement matrix $\vecA$ are i.i.d. $\cN(0,1).$
The MLE  is correct with high probability if 
\begin{align}m\geq \max_{l\in[1:k]}  \frac{nN(l)}{\frac{1}{2}\log\inp{\frac{ l}{2\sigma^2}+1}}\label{eq:upper_bd_mle}
\end{align}where  $N(l):=  \frac{k}{n} h_2\inp{\frac{l}{k}} + (1-\frac{k}{n})h_2\inp{\frac{l}{n-k}}$. 
\end{theorem}
We prove the theorem in Appendix~\ref{proof:MLE}. The main proof idea involves analysing the probability that the output of the MLE is $2l$ Hamming distance away from the unknown signal $\bx$ for different values of $l\in [1:k]$ (assuming $k\leq n/2$). This depends on the number of such vectors (approximately $2^{nN(l)}$) and the probability that the MLE outputs a vector which is $2l$ Hamming distance away from $\bx$. 

Note that when $l = k\inp{1-\frac{k}{n}}$, $nN(l) = nh_2(k/n)\approx k\log{\frac{n}{k}}$ and $\log\inp{\frac{k\inp{1-k/n}}{2\sigma^2}+1}\leq \log\inp{\frac{k}{2\sigma^2}+1}$.
Thus, $m$ is at least $\frac{2k\log{n/k}}{\log\inp{\frac{k}{2\sigma^2}+1}}$ (see the bound for Corollary~\ref{thm: spl_lower_bd_1}). It is not immediately clear if this value of $l= k\inp{1-\frac{k}{n}}$ is the optimizer. However, for large $n$, this appears to be the case numerically as shown in Plot~\ref{plot:1}.

\begin{figure}[t]
\includegraphics[width=7cm]{Unknown2.png}
\centering
\caption{The figure shows the plot of the MLE upper bound \eqref{eq:upper_bd_mle} (given by m1) for different values of $k$. This is displayed in blue color. A plot of $\frac{2nN(l)}{\log\inp{\frac{ l}{2\sigma^2}+1}}$ is also presented for $l = k\inp{1-\frac{k}{n}}$ in orange color, given by m2. A part of the plot is zoomed in to emphasize the closeness between the lines. In these plots,  $\sigma^2$ is set to 1,  $n$ is 50000 and $k$ ranges from 1000 to 25000 $(n/2)$. }\label{plot:1}
\end{figure}


Inspired by the MLE analysis, we derive a lower bound with the same structure as \eqref{eq:upper_bd_mle}. We generate the unknown signal $\bx$ using the following distribution: A vector $\tilde{\bx}$ is chosen uniformly at random from the set of all $k$-sparse vectors. Given $\tilde{\bx}$, the unknown input signal $\bx$ is chosen uniformly from the set of all $k$-sparse vector which are at a Hamming distance $2l$ from $\bx$. 
The lower bound is then obtained by computing upper and lower bounds on $I(\vecA, \by;\bx|\tilde{\bx})$.
We show this lower bound only for random matrices where each entry is chosen iid $\cN(0,1)$.
\begin{theorem}[\spl\ lower bound]\label{thm:lower_bd_spl}
If each entry of $\vecA$ is chosen iid $\cN(0,1)$, then for a uniformly chosen $k$-sparse vector $\bx$, an algorithm $\phi$ satisfies 
\begin{align}
    \bbP\inp{\phi(\vecA, {\by}) \neq \bx}\leq \delta\label{eq:spl_lower_bd_l}
\end{align}  only if the number of measurements $$m\geq \max_l\frac{nN(l) - 2\log{n}- h_2(\delta) - \delta k\log{n}}{\frac{1}{2}\log\inp{1+\frac{l}{\sigma^2}\inp{2-\frac{l}{k}}}} .$$
\end{theorem} The proof of Theorem~\ref{thm:lower_bd_spl} is given in Appendix~\ref{proof:MLE}.

If we choose $l = k\inp{1-\frac{k}{n}}$ in Theorem~\ref{thm:lower_bd_spl}, we recover corollary~\ref{thm: spl_lower_bd_1} for the special case of Gaussian design.
% \begin{corollary}\label{corollary2:lower_bd_spl}
% If  each entry of $\vecA$ is chosen iid $\cN(0,1)$, then for a uniformly chosen $k$-sparse vector $\bx$, an algorithm $\phi$ satisfies 
% $$\bbP\inp{\phi(\vecA, {\by}) \neq \bx}\leq \delta$$
% only if the number of measurements 
% $$m\geq \frac{k\log\inp{\frac{n}{k}} - 2\log{n}- h_2(\delta) - \delta k\log{n}}{\log\inp{1+\frac{k}{\sigma^2}}} .$$
% \end{corollary}

% Corollary~\ref{corollary2:lower_bd_spl} can also be proved directly for any sensing matrix $\vecA$ which satisfies \eqref{eq:power_constraint} (non-necessarily a Gaussian design). 


% \begin{figure}[t]
% \includegraphics[width=8cm]{plot.png}
% \centering
% \caption{The figure shows the plot of the MLE upper bound \eqref{eq:upper_bd_mle} (given by m1) for different values of $n$. This is displayed in blue color. A plot of $\frac{2nN(l)}{\log\inp{\frac{ l}{2\sigma^2}+1}}$ is also presented for $l = k\inp{1-\frac{k}{n}}$ in orange color, given by m2. In these plots,  $\sigma^2$ is set to 1 and $k$ is $0.2n$. }\label{plot:1}
% \end{figure}


\section{Results Beyond Squared Loss} \label{subsec:recover_quantify}

In regression problems under some assumptions, \citet{charikar2024quantifying} proves that the strong model’s error is smaller than the weak model’s, with the gap at least the strong model’s error on the weak labels.
This observation naturally raises the following question:
\textit{Can their proof be extended from squared loss to output distribution divergence?}
In this section, we show how to theoretically bridge the gap between squared loss and KL divergence within the overall proof framework established in~\citet{charikar2024quantifying}.
% Interestingly, our analysis reveals that employing KL divergence as the loss function can potentially lead to a reduction in the reverse KL divergence, and vice versa.
To begin with, we restate an assumption used in previous study.
\begin{assumption}[Convexity Assumption~\citep{charikar2024quantifying}] \label{convex_set}
The strong model learns fine-tuning tasks from a function class $\cF_{s}$, which is a convex set. 
\end{assumption}
\vspace{-5pt}
It requires that, for any $f, g \in \cF_s$, and for any $\lambda \in [0,1]$, there exists $h \in \cF$ such that for all $z \in \R^{d_s}$, $h(z) = \lambda f(z) + (1-\lambda) g(z)$. 
% And we do not assume anything about either $f^\star$ or $f_w$, which need not belong to $\cF$. 
To satisfy the convex set assumption, $\cF_s$ can be the class of all linear functions.
In these cases, $\cF_s$ is a convex set. 
Note that it is validated by practice: a popular way to fine-tune a pre-trained model on task-specific data is by tuning the weights of only the last linear layer of the model~\citep{howard2018universal,kumar2022fine}.

\subsection{Upper Bound (Realizability)} \label{subsub:realize}

Firstly, we consider the case where $\exists f_s \in \cF_s$ such that $F_s = F^\star$ (also called ``Realizability''~\citep{charikar2024quantifying}).
It means we can find a $f_s$ such that $f_s \circ h_s = f^\star \circ h^\star$.
This assumption implicitly indicates the strong power of pre-training. 
It requires that the representation $h_s$ has learned extremely enough information during pre-training, which is reasonable in modern large language models pre-trained on very large corpus~\citep{touvron2023llama,achiam2023gpt}.
The scale and diversity of the corpus ensure that the model is exposed to a broad spectrum of lexical, syntactic, and semantic structures, enhancing its ability to generalize effectively across varied language tasks.

We state our result in the realizable setting, which corresponds to Theorem 1 in~\citet{charikar2024quantifying}.
\begin{theorem}[Proved in \cref{proof_theorem_1-main}]
\label{thm:realizable-main}

Given $F^\star$, $F_w$ and $F_{sw}$ defined above.
Consider $\cF_s$ that satisfies Assumption~\ref{convex_set}. 
Consider WTSG using reverse KL divergence loss:
\begin{align*}
    f_{sw} = \argmin_{f \in \cF_{s}}\; \dist(f \circ h_s, f_w \circ h_w).
\end{align*}
Assume that $\exists f_s \in \cF_s$ such that $F_s = F^\star$.
Then
\begin{align} \label{eqn:realizable-main}
    \dist(F^\star, F_{sw}) \le \dist(F^\star, F_w) - \dist(F_{sw}, F_w).
\end{align}
\end{theorem}

\begin{remark}
    The corresponding theorem and proof in the case of forward KL divergence loss is provided in~\cref{thm:realizable} from~\cref{proof_theorem_1}, under an additional assumption.
\end{remark}

In contrast to the symmetric squared loss studied in prior work~\citep{charikar2024quantifying}, the emergence of the reverse KL divergence is inherently tied to the asymmetric properties of the KL divergence.
Although extending previous work to both forward and reverse KL divergences presents significant technical challenges, our results demonstrate the theoretical guarantees of WTSG in these settings.
In Inequality~\eqref{eqn:realizable-main}, the left-hand side represents the error of the weakly-supervised strong model on the true data. 
On the right-hand side, the first term denotes the true error of the weak model, while the second term captures the disagreement between the strong and weak models, which also serves as the minimization objective in WTSG. 
This inequality indicates that the weakly-supervised strong model improves upon the weak model by at least the magnitude of their disagreement, $\dist(F_{sw}, F_w)$.
To reduce the error of $F_{sw}$, \cref{thm:realizable-main} aligns with~\cref{lemma:upper_lower_inf}, highlighting the importance of selecting an effective weak model and the inherent limitations of the optimization objective in WTSG.


% Notice that the error of weak model and strong model in~\cref{thm:realizable-main} is the reverse version, which fundamentally stems from the asymmetric properties of KL divergence.
% Despite this subtle difference, our empirical results in the experiments demonstrate consistent trends between forward and reverse KL divergence.











\subsection{Upper Bound (Non-Realizability)}

Now we relax the ``realizability'' condition and draw $n$ i.i.d. samples to perform WTSG.
We provide the result in the ``unrealizable'' setting, where the condition $F_s = F^\star$ may not be satisfied for any $f_s \in \cF_s$.
It corresponds to Theorem 2 in~\citet{charikar2024quantifying}.

\begin{theorem}[Proved in~\cref{proof_non-realizable-main}] \label{thm:non-realizable-finite-samples-main}
Given $F^\star$, $F_w$ and $F_{sw}$ defined above.
Consider $\cF_s$ that satisfies~\cref{convex_set}.
Consider weak-to-strong generalization using reverse KL:
\begin{align*}
    & f_{sw} = \argmin_{f \in \cF_{s}}\; \dist(f \circ h_s, f_w \circ h_w),
    \\ & \hat{f}_{sw} = \argmin_{f \in \cF_{s}}\; \hat{d}_{\cP}(f \circ h_s, f_w \circ h_w),
\end{align*}
Denote $\dist(F^\star, F_s) = \eps$. 
With probability at least $1-\delta$ over the draw of $n$ i.i.d. samples, there holds
\begin{multline} 
\dist(F^\star, \hat{F}_{sw}) \le \dist(F^\star, F_w) - \dist(\hat{F}_{sw}, F_w) + \\ \cO(\sqrt{\eps}) +  \cO\left(\sqrt{\frac{\cC_{\cF_s}}{n}}\right) + \cO\left(\sqrt{\frac{\log(1/\delta)}{n}}\right),
\end{multline}
where $\cC_{\cF_s}$ is a constant capturing the complexity of the function class $\cF_s$, and the asymptotic notation is with respect to $\eps \to 0, n \to \infty$.
\end{theorem}

\begin{remark}
    The extension to forward KL divergence loss is provided in~\cref{thm:non-realizable-finite-samples} from~\cref{proof_non-realizable}, under an additional assumption.
\end{remark}


Compared to Inequality~\eqref{eqn:realizable-main}, this bound introduces two another error terms: the first term of $\cO(\sqrt{\eps})$ arises due to the non-realizability assumption, and diminishes as the strong ceiling model $F_s$ becomes more expressive.
The remaining two error terms arise from the strong model $\hat{F}_{sw}$ being trained on a finite weakly-labeled sample. They also asymptotically approach zero as the sample size increases.







\begin{figure*}[t]
  \centering
  \subfigure[Realizable (pre-training).]{
    \includegraphics[width=0.31\textwidth]{images/exp_2/realizable-pretrain.pdf}
  }
  \label{fig3:a}
  \subfigure[Non-realizable (pre-training).]{
    \includegraphics[width=0.31\textwidth]{images/exp_2/unrealizable-pretrain.pdf}
  }
  \subfigure[Non-realizable (perturbation).]{
    \includegraphics[width=0.31\textwidth]{images/exp_2/perturb.pdf}
  }
  \subfigure[Realizable (pre-training).]{
    \includegraphics[width=0.31\textwidth]{images/exp_2/realizable-pretrain_forward.pdf}
  }
  \subfigure[Non-realizable (pre-training).]{
    \includegraphics[width=0.31\textwidth]{images/exp_2/unrealizable-pretrain_forward.pdf}
  }
  \subfigure[Non-realizable (perturbation).]{
    \includegraphics[width=0.31\textwidth]{images/exp_2/perturb_forward.pdf}
  }
  \vspace{-5pt}
  \caption{Experiments on synthetic data using reverse KL divergence loss (\textbf{a-c}) and forward KL divergence loss (\textbf{d-f}). 
  Each point corresponds to a task and the gray dotted line represents $y=x$. 
  $h^{\star}$ is a 16-layer MLP. 
  (\textbf{a,d}) Realizable (pre-training): $h_s=h^\star$, and $h_w$ is a 2-layer MLP obtained by pre-training. (\textbf{b,e}) Non-realizable (pre-training): $h_s$ is an 8-layer MLP, and $h_w$ is a 2-layer MLP. Both $h_s$ and $h_w$ are obtained by pre-training. (\textbf{c,f}) Non-realizable (perturbation): Both $h_s$ and $h_w$ are obtained by directly perturbing the weights in $h^{\star}$:  $h_s=h^{\star}+ \mathcal{N}\left(0,0.01\right)$, and $h_w=h^{\star}+\mathcal{N}\left(0,9\right)$.}
  \label{syn_result:reverse}
  \vspace{-10pt}
\end{figure*}




\subsection{Synthetic Experiments} \label{section:syn_exp}
In this section, we conduct experiments on synthetically generated data to validate the theoretical results in~\cref{subsec:recover_quantify}.
While drawing inspiration from the theoretical framework of~\citet{charikar2024quantifying}, we extend their synthetic experiments by replacing the squared loss used in their work with the output distribution divergence defined in~\cref{def:kl_dist_emp}.


\subsubsection{Experimental Setting}

In our setup, The data distribution $\mathcal{P}$ is chosen as $\mathcal{N}(0, \sigma^2 I)$, with $\sigma=500$ to ensure the data is well-dispersed.
The ground truth representation $h^\star:\R^8 \to \R^{16}$ is implemented as a randomly initialized 16-layer multi-layer perceptron (MLP) with ReLU activations.
Let the weak model and strong model representations $h_w, h_s: \R^8 \to \R^{16}$ be 2-layer and 8-layer MLP with ReLU activations, respectively.
Given $h_w$ and $h_s$ frozen, both the strong and weak models learn from the fine-tuning task class $\cF_s$, which consists of linear functions mapping $\R^{16}\to\R$.
This makes $\cF_s$ a convex set. 

For the ``realizable'' setting, we set $h_s = h^\star$.
For the ``unrealizable'' setting, we adopt the approach of~\citet{charikar2024quantifying} and investigate two methods for generating weak and strong representations: (1) \textbf{Pre-training}: 20 models $f_1^\star,\dots,f_{20}^\star: \R^8 \to \R^{16}$ are randomly sampled as fine-tuning tasks. 2000 data points are independently generated from $\cP$ for these tasks. Accordingly, $h_w$ and $h_s$ are obtained by minimizing the average output distribution divergence between ground truth label ($f_t^\star \circ h^\star$) and model prediction ($f_t^\star \circ h_w$ and $f_t^\star \circ h_s$) over the 20 tasks.
(2) \textbf{Perturbations}: As an alternative, we directly perturb the parameters of $h^\star$ to obtain the weak and strong representations. 
Specifically, we add independent Gaussian noises $\mathcal{N}(0, \sigma_s^2)$ and $\mathcal{N}(0, \sigma_w^2)$ to every parameter in $h^\star$ to generate $h_s$ and $h_w$, respectively.
% We add independent Gaussian noise $\mathcal{N}(0, \sigma_s^2)$ to every parameter in $h^\star$ to generate $h_s$. Similarly, we perturb $h^\star$ with $\mathcal{N}(0, \sigma_w^2)$ to generate $h_w$. 
To ensure the strong representation $h_s$ is closer to $h^\star$ than $h_w$~\citep{charikar2024quantifying}, we set $\sigma_s=0.1$ and $\sigma_w=3$.



\noindent \textbf{Weak Model Finetuning.} 
We freeze the weak model representation $h_w$ and train the weak models on new fine-tuning tasks.
We randomly sample 100 new fine-tuning tasks $f_{21}^\star,\dots,f_{120}^\star: \R^8 \to \R^{16}$, and independently generate another 2000 data points from $\mathcal{P}$. 
For each task $t \in \{ 21, \cdots, 120 \}$, the corresponding weak model is obtained by minimizing the output distribution divergence between ground truth label and weak model prediction.

\noindent \textbf{Weak-to-Strong Supervision.} 
Using the trained weak models, we generate weakly labeled data to supervise the strong model.
Specifically, we first independently generate another 2000 data points from $\mathcal{P}$.
Then for each task $t \in \{ 21, \cdots, 120 \}$, the strong model is obtained by minimizing the output distribution divergence between weak model supervision and strong model prediction.
At this stage, the weak-to-strong training procedure is complete. The detailed introduction of above is in~\cref{appendix:syn_train}.

\noindent \textbf{Evaluation.}
We independently draw an additional 2000 samples from $\cP$ to construct the test set.
They are used to estimate $\dist(F^\star, F_{sw})$, $\dist(F^\star, F_w)$ and $\dist(F_{sw}, F_w)$ for each task $t \in \{ 21, \cdots, 120 \}$.
We estimate these quantities using their empirical counterparts: $\disthat(F^\star, F_w)$, $\disthat(F^\star, F_{sw})$, and $\disthat(F_{sw}, F_w)$.
To validate~\cref{thm:realizable-main}-\ref{thm:non-realizable-finite-samples-main} and visualize the trend clearly, we plot $\disthat(F^\star, F_w)-\disthat(F^\star, F_{sw})$ on the $x$-axis versus $\disthat(F_{sw}, F_w)$ on the $y$-axis. The results are presented in~\cref{syn_result:reverse}(a)-(c).
We also examine forward KL divergence loss. 
To validate~\cref{thm:realizable}-\ref{thm:non-realizable-finite-samples}, which extend~\cref{thm:realizable-main}-\ref{thm:non-realizable-finite-samples-main} to the case of using forward KL divergence loss in WTSG,
we plot $\disthat(F_w, F^\star)-\disthat(F_{sw}, F^\star)$ on the $x$-axis versus $\disthat(F_w, F_{sw})$ on the $y$-axis. 
The results are presented in~\cref{syn_result:reverse}(d)-(f).


\subsubsection{Results and Analysis}

\noindent \textbf{Reverse KL divergence loss.}
Similar to previous results of squared loss~\citep{charikar2024quantifying}, the points in our experiments also cluster around the line $y=x$.
This suggests that 
$\disthat(F^\star, F_w)-\disthat(F^\star, F_{sw}) \approx \disthat(F_{sw}, F_w)$.
It is consistent with our theoretical framework, suggesting that the improvement over the weak teacher can be quantified by the disagreement between strong and weak models.

\noindent \textbf{Forward KL divergence loss.}
The observed trend closely mirrors that of reverse KL. 
The dots are generally around the line $y=x$.
It suggest that the relationship 
$\disthat(F_w, F^\star)-\disthat(F_{sw}, F^\star) \approx \disthat(F_w, F_{sw})$
may also hold, indicating a similar theoretical guarantee for forward KL in WTSG.










\section{Experiments}
\label{sec:experiments}
The experiments are designed to address two key research questions.
First, \textbf{RQ1} evaluates whether the average $L_2$-norm of the counterfactual perturbation vectors ($\overline{||\perturb||}$) decreases as the model overfits the data, thereby providing further empirical validation for our hypothesis.
Second, \textbf{RQ2} evaluates the ability of the proposed counterfactual regularized loss, as defined in (\ref{eq:regularized_loss2}), to mitigate overfitting when compared to existing regularization techniques.

% The experiments are designed to address three key research questions. First, \textbf{RQ1} investigates whether the mean perturbation vector norm decreases as the model overfits the data, aiming to further validate our intuition. Second, \textbf{RQ2} explores whether the mean perturbation vector norm can be effectively leveraged as a regularization term during training, offering insights into its potential role in mitigating overfitting. Finally, \textbf{RQ3} examines whether our counterfactual regularizer enables the model to achieve superior performance compared to existing regularization methods, thus highlighting its practical advantage.

\subsection{Experimental Setup}
\textbf{\textit{Datasets, Models, and Tasks.}}
The experiments are conducted on three datasets: \textit{Water Potability}~\cite{kadiwal2020waterpotability}, \textit{Phomene}~\cite{phomene}, and \textit{CIFAR-10}~\cite{krizhevsky2009learning}. For \textit{Water Potability} and \textit{Phomene}, we randomly select $80\%$ of the samples for the training set, and the remaining $20\%$ for the test set, \textit{CIFAR-10} comes already split. Furthermore, we consider the following models: Logistic Regression, Multi-Layer Perceptron (MLP) with 100 and 30 neurons on each hidden layer, and PreactResNet-18~\cite{he2016cvecvv} as a Convolutional Neural Network (CNN) architecture.
We focus on binary classification tasks and leave the extension to multiclass scenarios for future work. However, for datasets that are inherently multiclass, we transform the problem into a binary classification task by selecting two classes, aligning with our assumption.

\smallskip
\noindent\textbf{\textit{Evaluation Measures.}} To characterize the degree of overfitting, we use the test loss, as it serves as a reliable indicator of the model's generalization capability to unseen data. Additionally, we evaluate the predictive performance of each model using the test accuracy.

\smallskip
\noindent\textbf{\textit{Baselines.}} We compare CF-Reg with the following regularization techniques: L1 (``Lasso''), L2 (``Ridge''), and Dropout.

\smallskip
\noindent\textbf{\textit{Configurations.}}
For each model, we adopt specific configurations as follows.
\begin{itemize}
\item \textit{Logistic Regression:} To induce overfitting in the model, we artificially increase the dimensionality of the data beyond the number of training samples by applying a polynomial feature expansion. This approach ensures that the model has enough capacity to overfit the training data, allowing us to analyze the impact of our counterfactual regularizer. The degree of the polynomial is chosen as the smallest degree that makes the number of features greater than the number of data.
\item \textit{Neural Networks (MLP and CNN):} To take advantage of the closed-form solution for computing the optimal perturbation vector as defined in (\ref{eq:opt-delta}), we use a local linear approximation of the neural network models. Hence, given an instance $\inst_i$, we consider the (optimal) counterfactual not with respect to $\model$ but with respect to:
\begin{equation}
\label{eq:taylor}
    \model^{lin}(\inst) = \model(\inst_i) + \nabla_{\inst}\model(\inst_i)(\inst - \inst_i),
\end{equation}
where $\model^{lin}$ represents the first-order Taylor approximation of $\model$ at $\inst_i$.
Note that this step is unnecessary for Logistic Regression, as it is inherently a linear model.
\end{itemize}

\smallskip
\noindent \textbf{\textit{Implementation Details.}} We run all experiments on a machine equipped with an AMD Ryzen 9 7900 12-Core Processor and an NVIDIA GeForce RTX 4090 GPU. Our implementation is based on the PyTorch Lightning framework. We use stochastic gradient descent as the optimizer with a learning rate of $\eta = 0.001$ and no weight decay. We use a batch size of $128$. The training and test steps are conducted for $6000$ epochs on the \textit{Water Potability} and \textit{Phoneme} datasets, while for the \textit{CIFAR-10} dataset, they are performed for $200$ epochs.
Finally, the contribution $w_i^{\varepsilon}$ of each training point $\inst_i$ is uniformly set as $w_i^{\varepsilon} = 1~\forall i\in \{1,\ldots,m\}$.

The source code implementation for our experiments is available at the following GitHub repository: \url{https://anonymous.4open.science/r/COCE-80B4/README.md} 

\subsection{RQ1: Counterfactual Perturbation vs. Overfitting}
To address \textbf{RQ1}, we analyze the relationship between the test loss and the average $L_2$-norm of the counterfactual perturbation vectors ($\overline{||\perturb||}$) over training epochs.

In particular, Figure~\ref{fig:delta_loss_epochs} depicts the evolution of $\overline{||\perturb||}$ alongside the test loss for an MLP trained \textit{without} regularization on the \textit{Water Potability} dataset. 
\begin{figure}[ht]
    \centering
    \includegraphics[width=0.85\linewidth]{img/delta_loss_epochs.png}
    \caption{The average counterfactual perturbation vector $\overline{||\perturb||}$ (left $y$-axis) and the cross-entropy test loss (right $y$-axis) over training epochs ($x$-axis) for an MLP trained on the \textit{Water Potability} dataset \textit{without} regularization.}
    \label{fig:delta_loss_epochs}
\end{figure}

The plot shows a clear trend as the model starts to overfit the data (evidenced by an increase in test loss). 
Notably, $\overline{||\perturb||}$ begins to decrease, which aligns with the hypothesis that the average distance to the optimal counterfactual example gets smaller as the model's decision boundary becomes increasingly adherent to the training data.

It is worth noting that this trend is heavily influenced by the choice of the counterfactual generator model. In particular, the relationship between $\overline{||\perturb||}$ and the degree of overfitting may become even more pronounced when leveraging more accurate counterfactual generators. However, these models often come at the cost of higher computational complexity, and their exploration is left to future work.

Nonetheless, we expect that $\overline{||\perturb||}$ will eventually stabilize at a plateau, as the average $L_2$-norm of the optimal counterfactual perturbations cannot vanish to zero.

% Additionally, the choice of employing the score-based counterfactual explanation framework to generate counterfactuals was driven to promote computational efficiency.

% Future enhancements to the framework may involve adopting models capable of generating more precise counterfactuals. While such approaches may yield to performance improvements, they are likely to come at the cost of increased computational complexity.


\subsection{RQ2: Counterfactual Regularization Performance}
To answer \textbf{RQ2}, we evaluate the effectiveness of the proposed counterfactual regularization (CF-Reg) by comparing its performance against existing baselines: unregularized training loss (No-Reg), L1 regularization (L1-Reg), L2 regularization (L2-Reg), and Dropout.
Specifically, for each model and dataset combination, Table~\ref{tab:regularization_comparison} presents the mean value and standard deviation of test accuracy achieved by each method across 5 random initialization. 

The table illustrates that our regularization technique consistently delivers better results than existing methods across all evaluated scenarios, except for one case -- i.e., Logistic Regression on the \textit{Phomene} dataset. 
However, this setting exhibits an unusual pattern, as the highest model accuracy is achieved without any regularization. Even in this case, CF-Reg still surpasses other regularization baselines.

From the results above, we derive the following key insights. First, CF-Reg proves to be effective across various model types, ranging from simple linear models (Logistic Regression) to deep architectures like MLPs and CNNs, and across diverse datasets, including both tabular and image data. 
Second, CF-Reg's strong performance on the \textit{Water} dataset with Logistic Regression suggests that its benefits may be more pronounced when applied to simpler models. However, the unexpected outcome on the \textit{Phoneme} dataset calls for further investigation into this phenomenon.


\begin{table*}[h!]
    \centering
    \caption{Mean value and standard deviation of test accuracy across 5 random initializations for different model, dataset, and regularization method. The best results are highlighted in \textbf{bold}.}
    \label{tab:regularization_comparison}
    \begin{tabular}{|c|c|c|c|c|c|c|}
        \hline
        \textbf{Model} & \textbf{Dataset} & \textbf{No-Reg} & \textbf{L1-Reg} & \textbf{L2-Reg} & \textbf{Dropout} & \textbf{CF-Reg (ours)} \\ \hline
        Logistic Regression   & \textit{Water}   & $0.6595 \pm 0.0038$   & $0.6729 \pm 0.0056$   & $0.6756 \pm 0.0046$  & N/A    & $\mathbf{0.6918 \pm 0.0036}$                     \\ \hline
        MLP   & \textit{Water}   & $0.6756 \pm 0.0042$   & $0.6790 \pm 0.0058$   & $0.6790 \pm 0.0023$  & $0.6750 \pm 0.0036$    & $\mathbf{0.6802 \pm 0.0046}$                    \\ \hline
%        MLP   & \textit{Adult}   & $0.8404 \pm 0.0010$   & $\mathbf{0.8495 \pm 0.0007}$   & $0.8489 \pm 0.0014$  & $\mathbf{0.8495 \pm 0.0016}$     & $0.8449 \pm 0.0019$                    \\ \hline
        Logistic Regression   & \textit{Phomene}   & $\mathbf{0.8148 \pm 0.0020}$   & $0.8041 \pm 0.0028$   & $0.7835 \pm 0.0176$  & N/A    & $0.8098 \pm 0.0055$                     \\ \hline
        MLP   & \textit{Phomene}   & $0.8677 \pm 0.0033$   & $0.8374 \pm 0.0080$   & $0.8673 \pm 0.0045$  & $0.8672 \pm 0.0042$     & $\mathbf{0.8718 \pm 0.0040}$                    \\ \hline
        CNN   & \textit{CIFAR-10} & $0.6670 \pm 0.0233$   & $0.6229 \pm 0.0850$   & $0.7348 \pm 0.0365$   & N/A    & $\mathbf{0.7427 \pm 0.0571}$                     \\ \hline
    \end{tabular}
\end{table*}

\begin{table*}[htb!]
    \centering
    \caption{Hyperparameter configurations utilized for the generation of Table \ref{tab:regularization_comparison}. For our regularization the hyperparameters are reported as $\mathbf{\alpha/\beta}$.}
    \label{tab:performance_parameters}
    \begin{tabular}{|c|c|c|c|c|c|c|}
        \hline
        \textbf{Model} & \textbf{Dataset} & \textbf{No-Reg} & \textbf{L1-Reg} & \textbf{L2-Reg} & \textbf{Dropout} & \textbf{CF-Reg (ours)} \\ \hline
        Logistic Regression   & \textit{Water}   & N/A   & $0.0093$   & $0.6927$  & N/A    & $0.3791/1.0355$                     \\ \hline
        MLP   & \textit{Water}   & N/A   & $0.0007$   & $0.0022$  & $0.0002$    & $0.2567/1.9775$                    \\ \hline
        Logistic Regression   &
        \textit{Phomene}   & N/A   & $0.0097$   & $0.7979$  & N/A    & $0.0571/1.8516$                     \\ \hline
        MLP   & \textit{Phomene}   & N/A   & $0.0007$   & $4.24\cdot10^{-5}$  & $0.0015$    & $0.0516/2.2700$                    \\ \hline
       % MLP   & \textit{Adult}   & N/A   & $0.0018$   & $0.0018$  & $0.0601$     & $0.0764/2.2068$                    \\ \hline
        CNN   & \textit{CIFAR-10} & N/A   & $0.0050$   & $0.0864$ & N/A    & $0.3018/
        2.1502$                     \\ \hline
    \end{tabular}
\end{table*}

\begin{table*}[htb!]
    \centering
    \caption{Mean value and standard deviation of training time across 5 different runs. The reported time (in seconds) corresponds to the generation of each entry in Table \ref{tab:regularization_comparison}. Times are }
    \label{tab:times}
    \begin{tabular}{|c|c|c|c|c|c|c|}
        \hline
        \textbf{Model} & \textbf{Dataset} & \textbf{No-Reg} & \textbf{L1-Reg} & \textbf{L2-Reg} & \textbf{Dropout} & \textbf{CF-Reg (ours)} \\ \hline
        Logistic Regression   & \textit{Water}   & $222.98 \pm 1.07$   & $239.94 \pm 2.59$   & $241.60 \pm 1.88$  & N/A    & $251.50 \pm 1.93$                     \\ \hline
        MLP   & \textit{Water}   & $225.71 \pm 3.85$   & $250.13 \pm 4.44$   & $255.78 \pm 2.38$  & $237.83 \pm 3.45$    & $266.48 \pm 3.46$                    \\ \hline
        Logistic Regression   & \textit{Phomene}   & $266.39 \pm 0.82$ & $367.52 \pm 6.85$   & $361.69 \pm 4.04$  & N/A   & $310.48 \pm 0.76$                    \\ \hline
        MLP   &
        \textit{Phomene} & $335.62 \pm 1.77$   & $390.86 \pm 2.11$   & $393.96 \pm 1.95$ & $363.51 \pm 5.07$    & $403.14 \pm 1.92$                     \\ \hline
       % MLP   & \textit{Adult}   & N/A   & $0.0018$   & $0.0018$  & $0.0601$     & $0.0764/2.2068$                    \\ \hline
        CNN   & \textit{CIFAR-10} & $370.09 \pm 0.18$   & $395.71 \pm 0.55$   & $401.38 \pm 0.16$ & N/A    & $1287.8 \pm 0.26$                     \\ \hline
    \end{tabular}
\end{table*}

\subsection{Feasibility of our Method}
A crucial requirement for any regularization technique is that it should impose minimal impact on the overall training process.
In this respect, CF-Reg introduces an overhead that depends on the time required to find the optimal counterfactual example for each training instance. 
As such, the more sophisticated the counterfactual generator model probed during training the higher would be the time required. However, a more advanced counterfactual generator might provide a more effective regularization. We discuss this trade-off in more details in Section~\ref{sec:discussion}.

Table~\ref{tab:times} presents the average training time ($\pm$ standard deviation) for each model and dataset combination listed in Table~\ref{tab:regularization_comparison}.
We can observe that the higher accuracy achieved by CF-Reg using the score-based counterfactual generator comes with only minimal overhead. However, when applied to deep neural networks with many hidden layers, such as \textit{PreactResNet-18}, the forward derivative computation required for the linearization of the network introduces a more noticeable computational cost, explaining the longer training times in the table.

\subsection{Hyperparameter Sensitivity Analysis}
The proposed counterfactual regularization technique relies on two key hyperparameters: $\alpha$ and $\beta$. The former is intrinsic to the loss formulation defined in (\ref{eq:cf-train}), while the latter is closely tied to the choice of the score-based counterfactual explanation method used.

Figure~\ref{fig:test_alpha_beta} illustrates how the test accuracy of an MLP trained on the \textit{Water Potability} dataset changes for different combinations of $\alpha$ and $\beta$.

\begin{figure}[ht]
    \centering
    \includegraphics[width=0.85\linewidth]{img/test_acc_alpha_beta.png}
    \caption{The test accuracy of an MLP trained on the \textit{Water Potability} dataset, evaluated while varying the weight of our counterfactual regularizer ($\alpha$) for different values of $\beta$.}
    \label{fig:test_alpha_beta}
\end{figure}

We observe that, for a fixed $\beta$, increasing the weight of our counterfactual regularizer ($\alpha$) can slightly improve test accuracy until a sudden drop is noticed for $\alpha > 0.1$.
This behavior was expected, as the impact of our penalty, like any regularization term, can be disruptive if not properly controlled.

Moreover, this finding further demonstrates that our regularization method, CF-Reg, is inherently data-driven. Therefore, it requires specific fine-tuning based on the combination of the model and dataset at hand.
\section{Discussion of Assumptions}\label{sec:discussion}
In this paper, we have made several assumptions for the sake of clarity and simplicity. In this section, we discuss the rationale behind these assumptions, the extent to which these assumptions hold in practice, and the consequences for our protocol when these assumptions hold.

\subsection{Assumptions on the Demand}

There are two simplifying assumptions we make about the demand. First, we assume the demand at any time is relatively small compared to the channel capacities. Second, we take the demand to be constant over time. We elaborate upon both these points below.

\paragraph{Small demands} The assumption that demands are small relative to channel capacities is made precise in \eqref{eq:large_capacity_assumption}. This assumption simplifies two major aspects of our protocol. First, it largely removes congestion from consideration. In \eqref{eq:primal_problem}, there is no constraint ensuring that total flow in both directions stays below capacity--this is always met. Consequently, there is no Lagrange multiplier for congestion and no congestion pricing; only imbalance penalties apply. In contrast, protocols in \cite{sivaraman2020high, varma2021throughput, wang2024fence} include congestion fees due to explicit congestion constraints. Second, the bound \eqref{eq:large_capacity_assumption} ensures that as long as channels remain balanced, the network can always meet demand, no matter how the demand is routed. Since channels can rebalance when necessary, they never drop transactions. This allows prices and flows to adjust as per the equations in \eqref{eq:algorithm}, which makes it easier to prove the protocol's convergence guarantees. This also preserves the key property that a channel's price remains proportional to net money flow through it.

In practice, payment channel networks are used most often for micro-payments, for which on-chain transactions are prohibitively expensive; large transactions typically take place directly on the blockchain. For example, according to \cite{river2023lightning}, the average channel capacity is roughly $0.1$ BTC ($5,000$ BTC distributed over $50,000$ channels), while the average transaction amount is less than $0.0004$ BTC ($44.7k$ satoshis). Thus, the small demand assumption is not too unrealistic. Additionally, the occasional large transaction can be treated as a sequence of smaller transactions by breaking it into packets and executing each packet serially (as done by \cite{sivaraman2020high}).
Lastly, a good path discovery process that favors large capacity channels over small capacity ones can help ensure that the bound in \eqref{eq:large_capacity_assumption} holds.

\paragraph{Constant demands} 
In this work, we assume that any transacting pair of nodes have a steady transaction demand between them (see Section \ref{sec:transaction_requests}). Making this assumption is necessary to obtain the kind of guarantees that we have presented in this paper. Unless the demand is steady, it is unreasonable to expect that the flows converge to a steady value. Weaker assumptions on the demand lead to weaker guarantees. For example, with the more general setting of stochastic, but i.i.d. demand between any two nodes, \cite{varma2021throughput} shows that the channel queue lengths are bounded in expectation. If the demand can be arbitrary, then it is very hard to get any meaningful performance guarantees; \cite{wang2024fence} shows that even for a single bidirectional channel, the competitive ratio is infinite. Indeed, because a PCN is a decentralized system and decisions must be made based on local information alone, it is difficult for the network to find the optimal detailed balance flow at every time step with a time-varying demand.  With a steady demand, the network can discover the optimal flows in a reasonably short time, as our work shows.

We view the constant demand assumption as an approximation for a more general demand process that could be piece-wise constant, stochastic, or both (see simulations in Figure \ref{fig:five_nodes_variable_demand}).
We believe it should be possible to merge ideas from our work and \cite{varma2021throughput} to provide guarantees in a setting with random demands with arbitrary means. We leave this for future work. In addition, our work suggests that a reasonable method of handling stochastic demands is to queue the transaction requests \textit{at the source node} itself. This queuing action should be viewed in conjunction with flow-control. Indeed, a temporarily high unidirectional demand would raise prices for the sender, incentivizing the sender to stop sending the transactions. If the sender queues the transactions, they can send them later when prices drop. This form of queuing does not require any overhaul of the basic PCN infrastructure and is therefore simpler to implement than per-channel queues as suggested by \cite{sivaraman2020high} and \cite{varma2021throughput}.

\subsection{The Incentive of Channels}
The actions of the channels as prescribed by the DEBT control protocol can be summarized as follows. Channels adjust their prices in proportion to the net flow through them. They rebalance themselves whenever necessary and execute any transaction request that has been made of them. We discuss both these aspects below.

\paragraph{On Prices}
In this work, the exclusive role of channel prices is to ensure that the flows through each channel remains balanced. In practice, it would be important to include other components in a channel's price/fee as well: a congestion price  and an incentive price. The congestion price, as suggested by \cite{varma2021throughput}, would depend on the total flow of transactions through the channel, and would incentivize nodes to balance the load over different paths. The incentive price, which is commonly used in practice \cite{river2023lightning}, is necessary to provide channels with an incentive to serve as an intermediary for different channels. In practice, we expect both these components to be smaller than the imbalance price. Consequently, we expect the behavior of our protocol to be similar to our theoretical results even with these additional prices.

A key aspect of our protocol is that channel fees are allowed to be negative. Although the original Lightning network whitepaper \cite{poon2016bitcoin} suggests that negative channel prices may be a good solution to promote rebalancing, the idea of negative prices in not very popular in the literature. To our knowledge, the only prior work with this feature is \cite{varma2021throughput}. Indeed, in papers such as \cite{van2021merchant} and \cite{wang2024fence}, the price function is explicitly modified such that the channel price is never negative. The results of our paper show the benefits of negative prices. For one, in steady state, equal flows in both directions ensure that a channel doesn't loose any money (the other price components mentioned above ensure that the channel will only gain money). More importantly, negative prices are important to ensure that the protocol selectively stifles acyclic flows while allowing circulations to flow. Indeed, in the example of Section \ref{sec:flow_control_example}, the flows between nodes $A$ and $C$ are left on only because the large positive price over one channel is canceled by the corresponding negative price over the other channel, leading to a net zero price.

Lastly, observe that in the DEBT control protocol, the price charged by a channel does not depend on its capacity. This is a natural consequence of the price being the Lagrange multiplier for the net-zero flow constraint, which also does not depend on the channel capacity. In contrast, in many other works, the imbalance price is normalized by the channel capacity \cite{ren2018optimal, lin2020funds, wang2024fence}; this is shown to work well in practice. The rationale for such a price structure is explained well in \cite{wang2024fence}, where this fee is derived with the aim of always maintaining some balance (liquidity) at each end of every channel. This is a reasonable aim if a channel is to never rebalance itself; the experiments of the aforementioned papers are conducted in such a regime. In this work, however, we allow the channels to rebalance themselves a few times in order to settle on a detailed balance flow. This is because our focus is on the long-term steady state performance of the protocol. This difference in perspective also shows up in how the price depends on the channel imbalance. \cite{lin2020funds} and \cite{wang2024fence} advocate for strictly convex prices whereas this work and \cite{varma2021throughput} propose linear prices.

\paragraph{On Rebalancing} 
Recall that the DEBT control protocol ensures that the flows in the network converge to a detailed balance flow, which can be sustained perpetually without any rebalancing. However, during the transient phase (before convergence), channels may have to perform on-chain rebalancing a few times. Since rebalancing is an expensive operation, it is worthwhile discussing methods by which channels can reduce the extent of rebalancing. One option for the channels to reduce the extent of rebalancing is to increase their capacity; however, this comes at the cost of locking in more capital. Each channel can decide for itself the optimum amount of capital to lock in. Another option, which we discuss in Section \ref{sec:five_node}, is for channels to increase the rate $\gamma$ at which they adjust prices. 

Ultimately, whether or not it is beneficial for a channel to rebalance depends on the time-horizon under consideration. Our protocol is based on the assumption that the demand remains steady for a long period of time. If this is indeed the case, it would be worthwhile for a channel to rebalance itself as it can make up this cost through the incentive fees gained from the flow of transactions through it in steady state. If a channel chooses not to rebalance itself, however, there is a risk of being trapped in a deadlock, which is suboptimal for not only the nodes but also the channel.

\section{Conclusion}
This work presents DEBT control: a protocol for payment channel networks that uses source routing and flow control based on channel prices. The protocol is derived by posing a network utility maximization problem and analyzing its dual minimization. It is shown that under steady demands, the protocol guides the network to an optimal, sustainable point. Simulations show its robustness to demand variations. The work demonstrates that simple protocols with strong theoretical guarantees are possible for PCNs and we hope it inspires further theoretical research in this direction.

\section*{Acknowledgements}
The authors would like to thank Laurent Baratchart and Sylvain Chevillard for their helpful discussion on rational approximations of the complex exponential. We also thank Sajad Movahedi and Felix Sarnthein for their helpful comments on this manuscript.
This work has received support from the French government, managed by the National Research Agency, under the France 2030 program with the reference "PR[AI]RIE-PSAI" (ANR-23-IACL-0008). 
Antonio Orvieto is supported by the Hector Foundation.


\bibliography{main.bib}

\newpage
\setcounter{section}{0}

\begin{appendices}

In this Appendix, we provide a detailed proof for all our theoretical results. We start in Appendix~\ref{RNN basics} with an equivalence of various representations of linear RNNs, then in Appendix~\ref{review} with a review of fundamentals of signal processing.
\listofappendices

\counterwithin{figure}{section}
\counterwithin{table}{section}


\newpage


\section{Recurrent Neural Networks and Diagonal forms}

\label{RNN basics}

 

Linear recurrent networks such as SSMs, in their simplest form, are causal models acting on a $d$ dimensional input sequence with $L$ elements $U\in\mathbb{R}^{d\times L}$, producing an output sequence $Y\in\mathbb{R}^{d\times L}$ through a filtering process parametrized by variables $A\in\mathbb{R}^{N\times N}$, $B\in\mathbb{R}^{N\times d}, P\in\mathbb{R}^{d\times N}$. Let $U_n\in\mathbb{R}^{d}$ denote the $n$-th timestamp data contained in $U$, a linear RNN processes the inputs as follows~\citep{gu2022parameterization,orvieto2023resurrecting}
\begin{equation}
    X_{n} = A X_{n-1} + B U_n,\qquad Y_{n} = PX_n.
    \label{eq:appendix linear_RNN}
\end{equation}

\begin{proposition}[Linear RNNs and convolution form]
    Let $A\in\mathbb{R}^{S\times S}$ such that $A$ is diagonal, $B\in\mathbb{R}^{S\times 1}, P\in\mathbb{R}^{1\times S}$, and $u = (u_n)_{n\in\mathbb{Z}}$ be a univariate input signal. The output signal $(y_n)_{n\in\mathbb{Z}}$ can write 
    \[
    y_n = \sum_{k=0}^\infty c_ku_{n-k}
    \]
    with $c_k = \sum_{s=1}^Sa_s^kb_s$.
\end{proposition}

\begin{proof}
    We have $A = \begin{pmatrix}
        a_1 & \dots & \\
        & \ddots & \\
        & \dots & a_S
    \end{pmatrix}, B = \begin{pmatrix}
        b_1\\
        \vdots\\
        b_S
    \end{pmatrix}, P = \begin{pmatrix}
        p_1 & \dots & p_S
    \end{pmatrix}$.

\begin{align*}
    X_n &= AX_{n-1} + Bu_n\\
    &= A(AX_{n-2} + Bu_{n-1}) + Bu_n  = \dots = \sum_{k=0}^nA^kBu_{n-k}\\
    &= \sum_{k=0}^n\begin{pmatrix}
        a_1^k & \dots & \\
        & \ddots & \\
        & \dots & a_S^k
    \end{pmatrix}\begin{pmatrix}
        b_1\\\vdots\\b_S
    \end{pmatrix}u_{n-k} = \sum_{k=0}^n\begin{pmatrix}
        a_1^kb_1\\\vdots \\a_S^kb_s
    \end{pmatrix}u_{n-k}.
\end{align*}
Finally,
\begin{align*}
    y_n = \begin{pmatrix}
        p_1 & \dots & p_S
    \end{pmatrix}X_n = \sum_{k=0}^n\begin{pmatrix}
        p_1 & \dots & p_S
    \end{pmatrix}\begin{pmatrix}
        a_1^kb_1\\\vdots\\a_N^kb_N
    \end{pmatrix}u_{n-k} & = \sum_{s=1}^S\sum_{k=0}^np_sa_s^kb_su_{n-k}\\
    &=\sum_{k=0}^nu_{n-k}\sum_{s=1}^Sa_s^kb_sp_s = \sum_{k=0}^nu_{n-k}c_k,
\end{align*}
with $c_k = \sum_{s=1}^Sa_s^kb_sp_s$. In this paper, we consider without loss of generality $\begin{pmatrix}
p_1 \dots p_s
\end{pmatrix} = \begin{pmatrix}
    1 \dots 1
\end{pmatrix}.$
\end{proof}
\section{Some fundamentals of signal processing}\label{section fundamentals} 
\label{review}

In this section, we will recall some fundamentals definitions and results in signal processing. We will only look at discrete-time signals. Throughout this section, we denote $(x_n)_{n\in\mathbb{Z}}$ or $x_n$ a discrete time signal, and $x_k$ the value taken by the signal at time $k$. For example, let us denote $(e_n)$ the impulse signal such that 
\begin{equation}
e_n =
\begin{cases}
    1, n = 0\\
    0, n\neq 0.
\end{cases}  
\label{appendix impulse signal}
\end{equation}
This signal is  useful because the response of a system to a impulse signal gives a lot of insights. In particular it fully describes a linear time-invariant system. For more on signal processing, we refer the reader to \cite{oppenheim1996signals}.

\subsection{Linear Time-invariant systems}

A system is said to be \textit{time-invariant} if its response to a certain input signal does not depend on time. It is said to be \textit{linear} if its output response to a linear combinations of inputs is the same linear combinations of the output responses of the individual inputs. A system is said to be \textit{causal} if the output at a present time depends on the input up the present time only. 

There exist several ways to represent the input-output behavior of LTI system. We will only look at the impulse response representation (convolution). 



\begin{proposition}[Convolution]
    Let $h_n$ be the impulse response of an LTI system $H$ (i.e., the output of system $H$ subject to input $e_n$), and $x_n$ be an input signal. In this case, the output signal of the system $y_n$ writes 
    \begin{equation}
        y_n = \sum_{k=-\infty}^{+\infty}x_kh_{n-k}.
        \label{appendix conv LTI}
    \end{equation}
\end{proposition}


\textit{Causal systems.} The output $y_n$ of a causal system depends only on past or present values of the input. This forces $h_k=0$ for $k<0$ and the convolution sum is rewritten 
\[
y_n = \sum_{k=0}^{+\infty}h_kx_{n-k}.
\]

\textit{Stable systems.} A system is stable if the output is guaranteed to be bounded for every bounded input. 



\subsection{Discrete-Time Fourier Transform}

In this section, we denote $x_n$ a complex-valued discrete-time signal.

\begin{definition}
    The discrete-time Fourier transform of signal $x_n$ is given by
    \[
    X(\omega) = \sum_{n=-\infty}^{+\infty}x_ne^{-i\omega n}.
    \]
    This function takes values in the frequency space.
    The inverse discrete-time Fourier transform is given by 
    \[
    x_n = \frac{1}{2\pi}\int_0^{2\pi}X(\omega)e^{i\omega n}d\omega.
    \]
\end{definition}

The Discrete-Time Fourier transform presents some notable properties that we recall in Table~\ref{table:dtft-properties}.

\begin{table}[h!]
\centering
\renewcommand{\arraystretch}{1.5}
\begin{tabular}{|c|c|}
\hline
\textbf{Property} & \textbf{Relation} \\ \hline
Time Shifting & 
$x_{n-k} \overset{DTFT}{\longleftrightarrow} e^{-i\omega k} X(\omega)$ \\ \hline
Convolution in Time & 
$x_n * y_n \overset{DTFT}{\longleftrightarrow} X(\omega) Y(\omega)$ \\ \hline
Frequency Differentiation & 
$j \frac{d}{d\omega} X(\omega) \overset{DTFT}{\longleftrightarrow} -n x_n$ \\ \hline
Differencing in Time & 
$x_n - x{n-1} \overset{DTFT}{\longleftrightarrow} \left(1 - e^{-i\omega}\right) X(\omega)$ \\ \hline
\end{tabular}
\caption{Properties of the Discrete-Time Fourier Transform (DTFT). For each property, assume $x_n\overset{DTFT}{\longleftrightarrow} X(\omega)$ and $y_n\overset{DTFT}{\longleftrightarrow} Y(\omega)$.}
\label{table:dtft-properties}
\end{table}

We recall Parseval's theorem that establishes a fundamental equivalence between the inner product of two signals in the time domain and their corresponding representation in the frequency domain.

\begin{theorem}[Parseval]
    For two complex-valued discrete-time signals \((x_n)\) and \((y_n)\) with discrete-time Fourier transforms \(X(e^{i\omega})\) and \(Y(e^{i\omega})\), Parseval's theorem yields:
    \begin{equation}
        \sum_{n=-\infty}^{+\infty}x_n\overline{y_n} = \frac{1}{2\pi}\int_0^{2\pi}X(e^{i\omega})\overline{Y(e^{i\omega})}d\omega.
        \label{Parseval thm}
    \end{equation}
In particular, Parseval's theorem yields an energy conservation result:
    $$
        \sum_{n=-\infty}^{+\infty}\vert x_n\vert^2 = \frac{1}{2\pi}\int_0^{2\pi}\vert X(e^{i\omega})\vert^2 d\omega.
    $$
\end{theorem}
The following proposition will be useful in our lower bound proof in Appendix~\ref{appendix subsection white noise loss}.
\begin{proposition}\label{proposition semi parseval}
    Let $w_n$ be a causal discrete-time complex-valued signal with Fourier transform $W(\omega)$. We have the following equality:
    \[
    \sum_{L=0}^{+\infty}L\vert w_l\vert^2 = \frac{i}{2\pi}\int_0^{2\pi}\frac{dW(\omega)}{d\omega}\overline{W}(\omega)d\omega.
    \]
\end{proposition}

\begin{proof}
    By definition of the DTFT, $W(\omega) = \sum_{L=0}^{+\infty}w_Le^{-i\omega L}$. Therefore, 
    \begin{align*}
        \sum_{L=0}^{+\infty}L\vert w_L\vert^2 &= \sum_{L=0}^{+\infty}Lw_L\bar{w}_L = \frac{1}{2\pi}\sum_{L=0}^{+\infty}\sum_{L'=0}^{+\infty}Lw_L\bar{w}_{L'}\int_0^{2\pi}e^{-i\omega(L-L')}d\omega\\
        &= \frac{i}{2\pi}\int_0^{2\pi}\sum_{L=0}^{+\infty}-iL\omega_Le^{-iL\omega}\sum_{L'=0}^{+\infty}\bar{w}_{L'}e^{iL'\omega}d\omega.
    \end{align*}
    Provided that the sequence $(Lw_L)_{L\geq 0}$ is summable, $\frac{dW(\omega)}{d\omega}=\sum_{L=0}^{+\infty}-iLw_Le^{-i\omega L}$, which proves the result.
\end{proof}

\subsection{Fourier series}\label{appendix subsection Fourier series}
We recall basics of Fourier Series. For more about Fourier series and their applications, we refer the reader to \cite{serov2017fourier}.

\begin{definition}[Fourier series]
    Let $f: \mathbb{R}\rightarrow \mathbb{R}$ be a piecewise continuous and $2\pi$-periodic function. The Fourier series of $f$ is the series of functions 
    \[
    S(f) = \sum_{n=-\infty}^{+\infty}
c_n(f)e^{int},  
\]
where $c_n(f)$ are the Fourier coefficients of $f$, such that 
\[
c_n(f) = \frac{1}{2\pi}\int_{-\pi}^\pi f(t)e^{-int}dt.
\]
The partial sums of these series write
\[
S_n(f)(t) = \sum_{k=-n}^nc_k(f)e^{ikt}
\]
\end{definition}

\begin{theorem}[Dirichlet]
    Let $f$ be piecewise $\mathcal{C}^1$ and $2\pi$-periodic. Therefore, for every $x\in\mathbb{R}$, $S_n(f)(x)$ converges to 
    \[
    \frac{f(x+0) + f(x-0)}{2},
    \]
    where $f(x+0)$ (resp. $f(x-0)$) denotes the right-hand (resp. left-hand) limit of $f$ at $x$.
\end{theorem}

\paragraph{Remark:}If the function \( f \) is not \( 2\pi \)-periodic, its graph on the interval \([0, 2\pi]\) can be extended periodically over \(\mathbb{R}\). In this case, Dirichlet's theorem is applicable at potential discontinuities at \( 0 \) and \( 2\pi \).

\subsection{A natural pair for autocorrelation}\label{appendix subsection natural pair}

A natural parametrization is to represent autocorrelation with $\gamma(k) = \rho^{\vert k\vert}$ with $\vert\rho\vert < 1$, as done in the main paper. This models exponentially decreasing autocorrelation between data. The natural associated time-frequency pair to represent is 
\[
(\gamma(k), \Gamma(e^{i\omega})) = (\rho^{\vert k\vert}, \frac{1-\rho^2}{\vert 1-\rho e^{-i\omega}\vert^2})
.\]
Indeed, as $\vert\rho\vert<1$, the sequence \((\rho^{\vert k\vert}e^{ik\omega})_{k\in\mathbb{Z}}\) is summable, \(\gamma\) admits a Fourier transform that we denote~$\Gamma$. For $\omega\in\mathbb{R}$.
\begin{align*}
    \Gamma(e^{i\omega}) &= \sum_{k=-\infty}^{+\infty}\rho^{\vert k\vert}e^{-i\omega k} =\sum_{k=1}^{+\infty}\rho^k e^{i\omega k} + \sum_{k=0}^{+\infty}\rho^ke^{-i\omega k}\\
    &= \frac{1}{1-\rho e^{i\omega k}} -1 + \frac{1}{1-\rho e^{-i\omega k}} =\frac{1-\rho^2}{\vert 1-\rho e^{-i\omega}\vert^2}.
\end{align*}

\begin{figure}[h]
    \centering
    \includegraphics[width=1\linewidth]{img/spectral_power_density.pdf}
    \caption{\textit{The autocorrelation factor $\rho$ determines the width of the spectral power density $\Gamma(e^{i\omega})$. The larger $\rho$, the narrower the spectral power density. This means that increasing $\rho$ in $\mathcal{L}_\text{freq}(c, d)$ narrows the bandwidth over which we evaluate the difference $\vert C(e^{i\omega}) - D(e^{i\omega})\vert^2$, leading to improved performance.}}
    \label{figure spectral power density}
\end{figure}

\section{Omitted Details in \pref{sec: lower_bound}}\label{app:lower_bound}
In this section, we show omitted proofs in \pref{sec: lower_bound}. We first prove \pref{thm:hedge_lower_bound}, which shows that Hedge suffers $\Omega(T^{\frac{1}{3}})$ alternating regret in the expert problem.
\begin{proof}[Proof of \pref{thm:hedge_lower_bound}]
We prove the lower bound by constructing two environments and showing that if the Hedge algorithm achieves $\order(T^{1/3})$ alternating regret for one environment, then it will suffer $\Omega(T^{1/3})$ alternating regret for the other one.

\textbf{Environment 1}: We consider the time horizon to be $3T$ and $3$ actions with the loss vector cycling between the three basis vectors in $\mathbb{R}^3: (1,0,0), (0,1,0), (0,0,1)$ and we have $\min_{i\in[3]} \sum_{t=1}^{3T} \ell_{t,i} = T$. Direct calculation shows that Hedge with learning rate $\eta>0$ predict the $p_t$ sequence as follows: $p_{3t-2}=\rbr{\frac{1}{3},\frac{1}{3},\frac{1}{3}}, p_{3t-1}=\rbr{\frac{e^{-\eta}}{2+e^{-\eta}},\frac{1}{2+e^{-\eta}},\frac{1}{2+e^{-\eta}}}, p_{3t}=\rbr{\frac{e^{-\eta}}{1+2e^{-\eta}},\frac{e^{-\eta}}{1+2e^{-\eta}},\frac{1}{1+2e^{-\eta}}}$, $t\in[T]$. Thus, we can bound the alternating regret as follows:
\begin{align*}
   & \RegAlt = T\rbr{\frac{2}{3}+\frac{1+e^{-\eta}}{2+e^{-\eta}}+\frac{1+e^{-\eta}}{1+2e^{-\eta}}} - 2T= \frac{T(1-e^{-\eta})^2}{3(2+e^{-\eta})(1+2e^{-\eta})}\ge \frac{T(1-e^{-\eta})^2}{27}.
\end{align*}
To proceed, note that when $\eta \geq 1$, the above inequality already means that $\RegAlt=\Omega(T)$. Therefore, we only consider the case when $\eta\leq 1$. Using the fact that $e^{-\eta} \le 1-\eta+\frac{\eta^2}{2}$ for $\eta\geq 0$, we can further lower bound $\frac{T(1-e^{-\eta})^2}{27}$ by $
    \RegAlt \ge \frac{T(\eta-\frac{{\eta}^2}{2}^2)}{27}\ge \frac{\eta^2T}{108},$
where the last inequality is due to $\eta\leq 1$. Therefore, we know that $\RegAlt = \Omega(\eta^2T)$.

\textbf{Environment 2}: We consider the time horizon to be $T$ and $3$ actions with the loss vector $\ell_t$ being $(1,0,0)$ for all rounds. Here, the benchmark $\min_{i\in[3]} \sum_{t=1}^T \ell_{t,i} $ is $0$, and $p_{t,1}=\frac{e^{-\eta T}}{2+e^{-\eta T}}$ for all $t\in[T]$. In this case, if $\eta\leq \frac{2}{T}$, we know that $p_{t,1}\geq \frac{e^{-2}}{2+e^{-2}}$ for all $t\in [T]$ and the algorithm will be suffering $\Omega(T)$ regret. When $1\geq \eta\geq \frac{2}{T}$, we have:
\begin{align*}
    \RegAlt &= \sum_{t=0}^T \frac{2e^{-\eta t}}{2+e^{-\eta t}} - \frac{1}{3} -\frac{e^{-\eta T}}{2+e^{-\eta T}} \ge \sum_{t=1}^{T-1} \frac{2}{1+2e^{\eta t}}\ge \int_{1}^{T} \frac{2}{3e^{\eta t}}dt= \frac{2}{3\eta}\left[-e^{-\eta t}\right]\Big|_{1}^{T}\ge\frac{e^{-1}}{3\eta},
\end{align*}
where the second inequality uses $e^{\eta t}\geq 1$ and the last inequality uses $\eta\leq 1$.
Therefore, we have $\RegAlt = \Omega\Big(\frac{1}{\eta}\Big)$. Combining both environments, we know that Hedge with learning rate $\eta$ suffers a $\Omega(\max\{\frac{1}{\eta},\eta^2 T\})$, leadning to a $\Omega(T^{\frac{1}{3}})$ lower bound.
\end{proof}


Next, we show that \PRM suffers $\Omega(\sqrt{T})$ alternating regret in the adversarial environment.

\begin{proof}[Proof of \pref{thm: PRM+}]
The procedure of \PRM is as follows: Let $\wh{R}_1 = {R}_1 = {r}_{0} = \bm{0}$, where $\bm{0}$ is a zero vector with all components equal to zero,
and for $t\ge 1$, \PRM selects $p_t$ to be $\hat{{R}}_t/\norm{\hat{{R}}_t}_1$ where $\hat{{R}}_t = [{R}_t+{r}_{t-1}]^+$ and update ${R}_{t+1}$ to be $[{R}_t+{r}_t]^+$, where ${r}_t = \langle {p}_t, {\ell}_t\rangle \bm{1}_d - {\ell}_t$.
Here, $\bm{1}_d$ is a vector with all components equal to one.
We use the following loss sequence to show the lower bound: for $k\ge 0$, consider
    \begin{align*}
    {\ell}_{2k}=
    \begin{bmatrix}
        1\\
        0
    \end{bmatrix},\quad
    {\ell}_{2k+1}=
    \begin{bmatrix}
        -0.5\\
        0
    \end{bmatrix}.
\end{align*}
To simplify notation, denote $\alpha_k=R_{2k+1,2}$ for $k\in [\frac{T}{2}-1]$. \pref{lem: PRM+} shows that for $k\ge 5$, $\alpha_k$ follows the following recurrence relation: $\alpha_{k+1}=\alpha_k+\frac{1}{1+\alpha_k}$. We use this recurrence to compute the values of $p_t$ for all rounds $t>10$.


Thus, using \pref{lem: PRM+}, the alternating loss (standard loss + cheating loss) for the rounds $t=2k+1$ and $t=2k+2$ can be calculated as
\begin{align*}
    \inner{{p}_{2k+1},\ell_{2k}+\ell_{2k+1}}+\inner{{p}_{2k+2},\ell_{2k+1}+\ell_{2k+2}}= \frac{1}{2(1+\alpha_k)}.
\end{align*}
Since the action $2$ always has a loss of 0, the benchmark here is 0.
Therefore, the alternating regret is:
\begin{equation}\label{eqn:prm+-regalt}
    \RegAlt = C+\frac{1}{2}\sum_{k=5}^{\frac{T}{2}}\frac{1}{1+\alpha_k},
\end{equation}
where $C$ is a constant bounding the regret for the first $10$ rounds.
To estimate the quantity above, we prove $\alpha_k\le 2\sqrt{k}-1$ by induction. The base case $k=5$ can be verified by direct calculation. Now, let us assume that the claim holds for $k$. Then, for $k+1$, 
\begin{align*}
\alpha_{k+1}&=\alpha_k +\frac{1}{1+\alpha_k}\le 2\sqrt{k}-1+\frac{1}{2\sqrt{k}}\le\frac{4k+1}{2\sqrt{k}}-1\le 2\sqrt{k+1}-1.
\end{align*}
The first inequality comes from the fact that the function $f(x)=x+\frac{1}{1+x}$ is monotonically increasing for $x\ge 0$.
Substituting it in \pref{eqn:prm+-regalt}, we get
\begin{align*}
\RegAlt\ge C+\frac{1}{4}\sum_{k=5}^{\frac{T}{2}} \frac{1}{\sqrt{k}}=\Theta(\sqrt{T}).
\end{align*}
Therefore, $\RegAlt = \Omega(\sqrt{T})$ for the loss sequence proposed.
\end{proof}

\begin{lemma}\label{lem: PRM+}
   Suppose that the loss vector sequence satisfies that $\ell_{2k}=\begin{bmatrix}
    1\\ 0
\end{bmatrix},{\ell}_{2k+1}=\begin{bmatrix}
    -0.5\\ 0
\end{bmatrix}$ for $k\geq 0$. Then, \PRM guarantees that for $k\geq 5$
\begin{align*}
&{p}_{2k+1}=\begin{bmatrix}
    0\\ 1
\end{bmatrix},{p}_{2k+2}=\begin{bmatrix}
    \frac{1}{1+\alpha_k}\\ \frac{\alpha_k}{1+\alpha_k}
\end{bmatrix},\\
&{R}_{2k+2}=\begin{bmatrix}
    0.5\\ \alpha_k
\end{bmatrix},R_{2k+3}=\begin{bmatrix}
    0\\ \alpha_k+\frac{1}{1+\alpha_k}
\end{bmatrix},\\
&\wh{R}_{2k+2}=\begin{bmatrix}
    1\\ \alpha_k
\end{bmatrix},\wh{R}_{2k+3}=\begin{bmatrix}
    0\\ \alpha_k+\frac{2}{1+\alpha_k}
\end{bmatrix},
\end{align*}
where $\alpha_5>2$ is certain constant and $\alpha_{k+1}=\alpha_k+\frac{1}{1+\alpha_k}$ for $k\geq 5$.
\end{lemma}

\begin{proof}
It can be verified that \pref{lem: PRM+} holds true when $k=5$. For $k>5$, we prove by induction. Suppose \pref{lem: PRM+} holds for $k$. Then, for $k+1$, we have,
\begin{align*}
    &{p}_{2k+3} =\frac{\wh{R}_{2k+3}}{\norm{\wh{R}_{2k+3}}_1} =\begin{bmatrix}
        0\\ 1
    \end{bmatrix},  \\
    &R_{2k+4} = [R_{2k+3}+r_{2k+3}]^+ = \begin{bmatrix}
        0.5\\
        \alpha_k+\frac{1}{1+\alpha_k}
    \end{bmatrix} =
    \begin{bmatrix}
        0.5\\
        \alpha_{k+1}
    \end{bmatrix},
    \\
    &r_{2k+3} = \inner{p_{2k+3},\ell_{2k+3}}\mathbf{1}_d - \ell_{2k+3} = \begin{bmatrix}
        0.5\\
        0
    \end{bmatrix},\\
    &\hat{R}_{2k+4} = [R_{2k+4}+r_{2k+3}]^+ = 
    \begin{bmatrix}
        1\\
        \alpha_{k+1} 
    \end{bmatrix}.
\end{align*}
Since $\alpha_k\ge 2$, we have $\alpha_{k+1}\ge 2$ as well. Using this, we can see that
\begin{align*}
    &{p}_{2k+4} =
    \frac{\hat{R}_{2k+4}}{\|\hat{R}_{2k+4}\|_1} = 
    \begin{bmatrix}
         \frac{1}{1+\alpha_{k+1}}\\
         \frac{\alpha_{k+1}}{1+\alpha_{k+1}}
    \end{bmatrix}, \\
    &r_{2k+4} = \inner{p_{2k+4},\ell_{2k+4}}\mathbf{1}_d - \ell_{2k+4} = \begin{bmatrix}
        -\frac{\alpha_{k+1}}{1+\alpha_{k+1}}\\
        \frac{1}{1+\alpha_{k+1}}
    \end{bmatrix}, \\
    &R_{2k+5} = [R_{2k+4}+r_{2k+4}]^+ = 
    \begin{bmatrix}
        0\\
        \alpha_{k+1}+\frac{1}{1+\alpha_{k+1}}
    \end{bmatrix}, \\
    &\hat{R}_{2k+5} = [R_{2k+5}+r_{2k+4}]^+ = 
    \begin{bmatrix}
        0\\
        \alpha_{k+1}+\frac{2}{1+\alpha_{k+1}}
    \end{bmatrix},
\end{align*}
where the third equality uses the fact that $\frac{\alpha_{k+1}}{1+\alpha_{k+1}}\geq \frac{2}{3}$. Thus, the claim is true for $k+1$, and hence it holds for all $k\ge 5$.
\end{proof}
\section{Proof of Theorem~\ref{theorem: upper bound} (Upper Bound of Algorithm~\ref{alg: main})}
\label{sec: appendix upper bound}
We break down the upper bound on the number of arm pulls used by Algorithm~\ref{alg: main} as follows. We bound the number of rounds required for a non-satisfying arm $k \not\in \A_{\epsilon}(\nu)$ to be eliminated in Lemma~\ref{lem: elim suboptimal}. Then in Lemma~\ref{lem: termination}, we bound the number of rounds each non-eliminated arm has gone through when the termination condition of the while-loop is triggered. Combining these lemmas with the number of arm pulls used by $\mathtt{QuantEst}$ for each round index $t \ge 1$ and active arm $k \in \A_t$ as stated in~\eqref{eq: QuantEst arm pulls}
gives us an upper bound on the total number of arm pulls.

We first present a useful lemma that will be used in the proofs of the two subsequent lemmas. 

\begin{lemma}[$\max \mathrm{LCB}$ is non-decreasing]
\label{lem: max LCB increasing}
    Under Event $E$ as defined in Lemma~\ref{lem: good events}, we have 
    \begin{equation}
    \max\limits_{a \in \mathcal{A}_{t}} 
            \mathrm{LCB}_{t}(a) \ge 
    \max\limits_{a \in \mathcal{A}_{\tau}} 
            \mathrm{LCB}_{\tau}(a).
    \end{equation}
    for all rounds $t > \tau  \ge 1$.
\end{lemma}
\begin{proof}
    Let round index $\tau \ge 1$ be arbitrary
    and let $k \in \argmax\limits_{a \in \A_{\tau}} 
            \mathrm{LCB}_{\tau}(a).$
    We have $k \in \A_{\tau+1}$ since 
    $\mathrm{UCB}_{\tau}(k) > \mathrm{LCB}_{\tau}(k) 
    = \max\limits_{a \in \A_{\tau}} 
            \mathrm{LCB}_{\tau}(a)$ by~\eqref{eq:  quantile anytime bound} 
    of the anytime quantile bounds.        
    It then follows that
    \begin{equation}
        \max\limits_{a \in \mathcal{A}_{\tau+1}} 
            \mathrm{LCB}_{\tau+1}(a) 
        \ge \mathrm{LCB}_{\tau+1}(k) 
        \ge \mathrm{LCB}_{\tau}(k) = 
        \max\limits_{a \in \A_{\tau}} 
            \mathrm{LCB}_{\tau}(j),
    \end{equation}    
    where the second inequality follows from~\eqref{eq:  quantile anytime bound} 
    of the anytime quantile bounds. 
    Applying the argument repeatedly yields the claim for all $t > \tau.$
\end{proof}

\begin{lemma}[Elimination of non-satisfying arms]
\label{lem: elim suboptimal}
     Fix an instance $\nu \in \cE$, and suppose Algorithm~\ref{alg: main} is run with input $(\A, \lambda, \epsilon, q, \delta)$ and parameter $c \ge 1$.
    Let $\A_{\epsilon} = \A_{\epsilon}(\nu) $ be as defined in~\eqref{def: performance def} and let the gap $\Delta_{k} = \Delta_{k}(\nu, \lambda, \epsilon, c, q)$ be as defined in Definition~\ref{def: our gap} 
    for each arm $k \in \A$.
    Consider an arm $k \not\in \A_{\epsilon}$.
    Under Event $E$ as defined in Lemma~\ref{lem: good events}, when the round index~$t$
    of Algorithm~\ref{alg: main} satisfies $\Delta^{(t)}  \le \frac{1}{2} \Delta_k$, we have  $k \not\in \A_{t+1}$.
\end{lemma}
\begin{proof}
    If $k \not\in \A_t$, then $k \not\in \A_{t+1}$ trivially. Therefore, we assume for the rest of the proof that $k \in \A_t$, and we will show that
    \begin{equation}
    \label{eq: eliminate condition}
        \mathrm{UCB}_t(k) \le \max\limits_{a \in \mathcal{A}_{t}} \mathrm{LCB}_t(a)
    \end{equation}
    or equivalently
    \begin{equation}
    \label{eq: eliminate condition equivalent}
        \mathrm{UCB}_t(k) < \max\limits_{a \in \mathcal{A}_{t}} \mathrm{LCB}_t(a) + \tilde{\epsilon},
    \end{equation}
    where $\tilde{\epsilon} = \tilde{\epsilon}(\lambda, \epsilon, c)$ is as defined in Lines~\ref{line: number of points} and~\ref{line: tilde epsilon} of Algorithm~\ref{alg: main}.
    Note that these conditions are equivalent because both
    $\mathrm{UCB}_t(k)$ and
    $\max\limits_{a \in \A_t } \mathrm{LCB}_t(a)$ 
    are elements of 
    \begin{equation}
        \left[ 0, 
        \tilde{\epsilon}, 
        2\tilde{\epsilon}, \cdots,
        (n-1) \tilde{\epsilon}, \lambda\right],
    \end{equation}
   which follows from Lines~\ref{line: list of points},~\ref{eq: initiate default conf interval}, and~\ref{ltk def}--\ref{UCB definition} of Algorithm~\ref{alg: main}.

    
    Since $k \not\in \A_{\epsilon}$, when the round index $t$ satisfies~$\Delta^{(t)} \le \frac{1}{2} \Delta_k $ we have
    \begin{equation}
    \label{eq: gap k realized with arm j}
        \mathrm{UCB}_t(k)
        < Q_k \big( q + \Delta^{(t)} \big)  + \tilde{\epsilon} 
        \le Q^+_{j}\big(q - \Delta^{(t)} \big) 
    \end{equation}
    for some arm $j \in \A$  by~\eqref{eq: upper approx quantile anytime bound} of the anytime quantile bounds
    and Definition~\ref{def: our gap}. 
    We now consider two cases: (i) $j \in \A_t$ and (ii) $j \not\in \A_t$.

    
    If $j \in \A_t$, we have
    \begin{equation}
    \label{eq: j in At}
        Q^+_{j}\big(q - \Delta^{(t)}\big) 
        \le \mathrm{LCB}_t(j) + \tilde{\epsilon} 
        \le \max\limits_{a \in \mathcal{A}_{t}} \mathrm{LCB}_t(a) + \tilde{\epsilon} 
    \end{equation}
    by~\eqref{eq: lower approx quantile anytime bound} of the anytime quantile bounds and the assumption that $j \in \A_t$. Combining~\eqref{eq: gap k realized with arm j} and~\eqref{eq: j in At} gives us condition~\eqref{eq: eliminate condition equivalent} as desired.
    
    If $j \not\in \A_t$, then it is eliminated at some round $\tau < t$, i.e., 
    $j \in \A_{\tau}$ but $j \not\in \A_{\tau + 1}$.
    By \eqref{eq: quantile anytime bound} of the anytime quantile bounds, the definition of active arm set (Line~\ref{line: active arm} of Algorithm~\ref{alg: main}) applied to $\A_{\tau + 1}$,
    and the fact that max LCB is non-decreasing (Lemma~\ref{lem: max LCB increasing}), 
    we have 
    \begin{equation}
    \label{eq: j not in At}
    Q_{j}(q) 
    \le
    \mathrm{UCB}_{\tau}(j) 
    \le
            \max\limits_{a \in \mathcal{A}_{\tau}} 
            \mathrm{LCB}_{\tau}(a)     
        \le
        \max\limits_{a \in \mathcal{A}_{t}} 
            \mathrm{LCB}_{t}(a). 
    \end{equation}
    Combining~\eqref{eq: gap k realized with arm j}, the trivial inequality 
    $Q^+_{j}\big(q - \Delta^{(t)}\big) 
        \le Q_{j}(q) $, and~\eqref{eq: j not in At} yields
    condition~\eqref{eq: eliminate condition} as desired.
\end{proof}

\begin{remark}
    \label{rem: elim suboptimal}
    As seen in the analysis for the case  $j \not\in \A_t$ above, the property that $\max \mathrm{LCB}$ is non-decreasing (Lemma~\ref{lem: max LCB increasing}) is crucial in establishing~\eqref{eq: j not in At}. We will see below that the same argument is used again in establishing~\eqref{eq: LCB_t(k) > Fj}. This property of Lemma~\ref{lem: max LCB increasing} itself is a consequence of ensuring $\mathrm{LCB}_t(k)$ is non-decreasing in $t$; see Remark~\ref{rem: LCB non decreasing}.
\end{remark}



\begin{lemma}[While-loop termination]
\label{lem: termination}
     Fix an instance $\nu \in \cE$, and suppose Algorithm~\ref{alg: main} is run with input $(\A, \lambda, \epsilon, q, \delta)$ and parameter $c \ge 1$.
    Let $\A_{\epsilon} = \A_{\epsilon}(\nu) $ be as defined in~\eqref{def: performance def} and let the gap $\Delta_{k} = \Delta_{k}(\nu, \lambda, \epsilon, c, q)$ be as defined in Definition~\ref{def: our gap} 
    for each arm $k \in \A$.
    Under Event $E$, when the round index~$t$
    of Algorithm~\ref{alg: main} satisfies $\Delta^{(t)} \le \frac{1}{2} \max \limits_{a \in \A_{\epsilon}} \Delta_a$, Algorithm~\ref{alg: main} will terminate in round $t+1$.
\end{lemma}

\begin{proof}
     If $\A_{t+1} = \{ k^* \}$, then
      \begin{equation}
          \max\limits_{a \in \A_{t+1} \setminus \{k^*\} }                 \mathrm{UCB}_t(a) - (c+1)\tilde{\epsilon}
      = -\infty \le \mathrm{LCB}_{t}(k^*),
      \end{equation}
     and so the algorithm will terminate and return arm $k^*$ in round $t+1$.
     Therefore, we assume for the rest of the proof that
     there exists another arm $a \ne k^*$ such that $a \in \A_{t+1}$. 

     We first show that the following condition is sufficient to trigger the termination condition of the while-loop (Lines~\ref{line: start while loop}--\ref{line: end while loop}) of Algorithm~\ref{alg: main}: There exists an arm $k \in \A_{t+1}$
     satisfying
    \begin{equation}
    \label{eq: suf cond trigger termination}
          \mathrm{LCB}_t(k)  
          \ge
          \max\limits_{a \in \A_{t+1} \setminus \{k\} }
        Q_{a}\big(q + \Delta^{(t)}\big) -  (c+1)\tilde{\epsilon} .
    \end{equation}
    Using~\eqref{eq: upper approx quantile anytime bound} of the anytime quantile bound, 
    condition~\eqref{eq: suf cond trigger termination} implies that
    \begin{equation}
    \label{eq: termination condition strict equality}
        \mathrm{LCB}_t(k)  
        >
          \max\limits_{a \in \mathcal{A}_{t+1} \setminus \{k\} } \mathrm{UCB}_t(a)
          - (c+2)\tilde{\epsilon},
    \end{equation}
    which is equivalent to the termination condition
    \begin{equation}
          \mathrm{LCB}_t(k)  
            \ge
          \max\limits_{a \in \mathcal{A}_{t+1} \setminus \{k\} } \mathrm{UCB}_t(a) - (c+1)\tilde{\epsilon},
    \end{equation}
    where the equivalence follows from an argument similar to the equivalence between~\eqref{eq: eliminate condition} and 
        \eqref{eq: eliminate condition equivalent}.

    It remains to pick an arm $k \in \A_{t+1}$ satisfying condition~\eqref{eq: suf cond trigger termination}.
    Let arm $j \in \argmax\limits_{a \in \A_{\epsilon}} \Delta_a$ and consider the following two cases: (i) $j \in \A_{t+1}$ and (ii) $j \not\in \A_{t+1}$.

    If $j \in \A_{t+1}$, we pick $k = j$. We also pick~$
    T \in \argmax\limits_{\A_{\epsilon} \subseteq S \subseteq \A}
        \Delta_{k}^{(S)}    
    $ 
    to be the set associated to $\Delta_k$ (see Definition~\ref{def: our gap}).
    Note that every arm that is not in $T$
    is a non-satisfying arm since 
    $\A_{\epsilon} \subseteq T$.
    Furthermore, every non-satisfying arm that is not in $T$, hence every arm that is not in $T$, is eliminated,
    which follows from Lemma~\ref{lem: elim suboptimal} and
    \begin{equation}
        \Delta^{(t)} 
        \le \frac{1}{2} \max \limits_{a \in \A_{\epsilon}} \Delta_a
        = 
        \frac{1}{2} \Delta_k \le \frac{1}{2} \min\limits_{a \not\in T} \Delta_a,
    \end{equation} 
    where the last inequality follows from applying~\eqref{eq: Delta k^S} to $k$ and $T$.
    Therefore, we have
    $\A_{t+1} \subseteq T$.
    It follows that
    \begin{align}
         \mathrm{LCB}_t(k)
          &\ge
          Q^+_k\big(q - \Delta^{(t)}\big) - \tilde{\epsilon} \\
          &\ge
        \max\limits_{a \in T \setminus \{k\} }
        Q_{a}\big(q + \Delta^{(t)}\big) - (c+1)\tilde{\epsilon} \\
        &\ge
      \max\limits_{a \in \A_{t+1} \setminus \{k\} }
        Q_{a}\big(q + \Delta^{(t)}\big) - (c+1)\tilde{\epsilon},
    \end{align}
    where the first inequality follows from~\eqref{eq: lower approx quantile anytime bound} of the anytime quantile bound, the second inequality follows from applying~\eqref{eq: Delta k^S} to $k$ and $T$,
    and the last inequality follows from  $\A_{t+1} \subseteq T$.

    If $j \not\in \A_{t+1}$,
    we pick an arm 
    $k \in \argmax\limits_{a \in \mathcal{A}_{t+1}} 
    \mathrm{LCB}_{t}(a)$ arbitrarily.
    We also pick $T \in \argmax\limits_{\A_{\epsilon} \subseteq S \subseteq \A} \Delta_{k}^{(S)}$ and  
    we have $\A_{t+1} \subseteq T$ as in the case above.
    Furthermore, since $j \not\in \A_{t+1}$,
    we have
    \begin{equation}
    \label{eq: LCB_t(k) > Fj}
        Q^+_j \big(q - \Delta^{(t)}\big)
    \le 
    Q_{j}(q) 
    \le
    \max\limits_{a \in \mathcal{A}_{t+1}} 
    \mathrm{LCB}_{t}(a) 
    = \mathrm{LCB}_{t}(k),
    \end{equation}
    where the second inequality follows from an argument similar to~\eqref{eq: j not in At}.
    It follows that
    \begin{align}
         \mathrm{LCB}_t(k) 
        &\ge Q^+_j \big(q - \Delta^{(t)}\big)  \\
        &\ge
        \max\limits_{a \in T \setminus \{j\} }
        Q_{a}\big(q + \Delta^{(t)}\big)   - c \tilde{\epsilon} \\
        &\ge
        \max\limits_{a \in \A_{t+1} \setminus \{k\} }
        Q_{a}\big(q + \Delta^{(t)}\big) - (c+1) \tilde{\epsilon},
    \end{align}
    where the first inequality follows from~\eqref{eq: LCB_t(k) > Fj}, the second inequality follows from applying~\eqref{eq: Delta k^S} to $j$ and $T$,
    and the last inequality follows from  $\A_{t+1} \subseteq T$.
\end{proof}

\section{Experiments}


In this section, we present a series of experiments designed to validate our theoretical findings in a practical setting. Specifically, we assess whether our conclusions hold when transitioning from an idealized infinite-data framework to real-world scenarios with a limited number of samples. 

Let us first introduce the linear recurrent neural network (RNN) used in our study. It is defined by the following recurrence relations:
\begin{align*}
    h_0 &= 0, \\
    x_{t+1} &= Ax_t + Bu_{t+1}, \\
    y_t &= Cx_t,
\end{align*}
where $x_t \in \mathbb{R}^{d_{\text{hidden}}}$ represents the hidden state, $u_t \in \mathbb{R}$ is the input, and $y_t \in \mathbb{R}$ is the output. The network parameters consist of $A \in \mathbb{C}^{d_{\text{hidden}} \times d_{\text{hidden}}}$, $B \in \mathbb{C}^{d_{\text{hidden}} \times 1}$, and $C \in \mathbb{C}^{d_{\text{hidden}}}$. 

Without loss of generality, we adopt a diagonal representation for the matrix $A$. The choice of its initial eigenvalues depends on the specific experiment: we either use a random initialization or employ the structured initialization given by Eq.~\eqref{Param_of_the_as}. 

In the simple experiments conducted below, the objective is to learn a single filter. Consequently, there is no need to decompose the matrix $A$ into multiple diagonal blocks. The matrix $C$ is initialized as:
\[
C_{\text{init}} = \begin{pmatrix} 1 & \dots & 1 \end{pmatrix} \in \mathbb{R}^{d_{\text{hidden}} \times 1}.
\]
and the entries of $B$ are initialized given by Eq.~\eqref{Param_of_the_bs}.\newline

The synthetic dataset consists of autoregressive sequences $X = (u_1, u_2, \dots, u_N)$ of length $N$, generated as:
\begin{equation}
    u_n = \rho u_{n-1} + \epsilon_k, \quad \epsilon_k \sim \mathcal{N}(0, 1 - \rho^2), \quad u_1 \sim U(0,1).
\end{equation}
The objective is to learn a mapping with linear recurrences $f: X \to Y$, where the target is given by:
\begin{equation}
    Y = u_{t^*}
\end{equation}
This corresponds to learning a shift of $N - t^*$ with finite samples. 


\subsection{Random initialization vs. Shift-K initialization}

In this first set of experiments, we analyze the impact of initializing the complex diagonal entries $a_s$ of the linear RNN using phases that are uniformly distributed over a segment of the unit disk, with a constant radius close to 1, as described in the parametrization in Eq.~\eqref{Param_of_the_as}. Additionally, the parameters $b_s$ are initialized following the parametrization given in Eq.~\eqref{Param_of_the_bs}. We call this initialization the shift-$K$ initialization. We compare this approach to a standard random initialization to evaluate potential benefits in terms of performance and stability.


\begin{table}[ht]
    \centering
    \begin{tabular}{@{}ll@{}}
        \toprule
        & \textbf{Random init.} \hspace{4,2cm} \textbf{Shift-K init.} \\ \midrule
        Batch size & [20, 50, 100] \hspace{4,4cm} [20, 50, 100] \\
        Number Samples & 130000 \hspace{5,2cm} 130000\\
        Sequence length & 1500 \hspace{5,6cm} 1500 \\
        Position of $t^*$ & 200 \hspace{5,8cm} 200 \\
        Hidden neurons & 128 \hspace{5,8cm} 128 \\
        Input / output dimension & 1 \hspace{6,2cm} 1 \\
        Learning rates & [0.01, 0.005, 0.001, 0.0001] \hspace{2,2cm}[0.01, 0.005, 0.001, 0.0001] \\
        Weight decay & $10^{-5}$ \hspace{5,7cm}$10^{-5}$ \\ 
        $\rho$ & \{0, 0.2, 0.4, 0.6, 0.8, 1\} \hspace{2,7cm} \{0, 0.2, 0.4, 0.6, 0.8, 1\} \\ \midrule
        $a_s$ param. & $a_u = e^{-\alpha/K_{\textnormal{init}}}e^{i\epsilon_u\pi}, \varepsilon \sim \mathcal{U}(-1,1)$ \hspace{1cm} $a_u = e^{-\alpha/K_{\textnormal{init}}}e^{iu\frac{\pi}{K_{\textnormal{init}}}}$\\
         $b_s$ param. & $b_u = \frac{e^{-\alpha}(e^{2\alpha}-e^{-2\alpha})}{2K_{\textnormal{init}}}\times(-1)^u$ \hspace{1,8cm} $b_u = \frac{e^{-\alpha}(e^{2\alpha}-e^{-2\alpha})}{2K_{\textnormal{init}}}\times(-1)^u$  \\
        $\alpha$ & 1 \hspace{6,2cm} 1 \\
       $K_{\textnormal{init}}$ & 1300 \hspace{5,6cm} 1300 \\ \midrule
        Number epochs & 60 \hspace{6cm} 60 \\
        \bottomrule
    \end{tabular}
    \caption{{Experimental details for Figure~\ref{fig:xps} (left)}. We use $[\dots]$ to denote hyperparameters that were scanned over with grid search and $\{\dots\}$ to denote the variable of interest for the figure. We chose the same representation for $b_s$ in both cases because we observed small impact of this parameter on the final results.}
    \label{tab:xp-compare_init}
\end{table}

\begin{figure}
    \centering
    \includegraphics[width=1\linewidth]{img/appendix_2_filters.pdf}

    \vspace*{-.2cm}
    
    \caption{\textit{Comparison of Filters Obtained with Different Initialization Methods. Left: Filter obtained using our proposed shift-$K$ initialization method, which exhibits a more structured and interpretable pattern. Right: Filter obtained with random initialization, which appears significantly noisier, indicating less effective memory propagation. }}
    \label{fig:appendix 2 filters}
\end{figure}

\subsection{Robustness of Shift-K initialization}

In this second set of experiments, we investigate the robustness of our initialization scheme with respect to inaccuracies in the choice of $K_{\textnormal{init}}$ when initializing $a_s$ as in Eq.~\eqref{Param_of_the_as}. In practical applications, the actual shift of the sequence is often unknown, making it impossible to initialize with the exact optimal value of~$K$. A robust initialization method should exhibit resilience to such uncertainties, allowing for performance stability within a reasonable range of $K_{\textnormal{init}}$ values.

\begin{table}[ht]
    \centering
    \begin{tabular}{@{}l@{}}
        \toprule
        \textbf{Shift-K init.} \\ \midrule
        Batch size: [20, 50, 100] \\
        Number of Samples: 150000 \\
        Sequence length: 2250 \\
        Position of $t^*$: 250 \\
        Hidden neurons: 128 \\
        Input / output dimension: 1 \\
        Learning rates: [0.01, 0.005, 0.001, 0.0001] \\
        Weight decay: $10^{-5}$ \\ 
        $\rho$: 0.7 \\ \midrule
        $a_s$ param.: $a_u = e^{-\alpha/K_{\textnormal{init}}}e^{iu\frac{\pi}{K_{\textnormal{init}}}}$\\
        $b_s$ param.: $b_u = \frac{e^{-\alpha}(e^{2\alpha}-e^{-2\alpha})}{2K_{\textnormal{init}}}\times(-1)^u$  \\
        $\alpha$: 1 \\
        $K_{\textnormal{init}}$: \{250, 500, 1000, 2000, 4000, 8000, 16000, 32000\} \\ \midrule
        Number of epochs: 60 \\
        \bottomrule
    \end{tabular}
    \caption{{Experimental details for Figure~\ref{fig:xps} (right).}}
    \label{tab:xp-robustness}
\end{table}


\end{appendices}

\end{document}