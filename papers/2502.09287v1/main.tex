
\documentclass[12pt]{article}
\usepackage{fullpage}
\usepackage[round]{natbib}
\bibliographystyle{plainnat}
\usepackage{float}
\usepackage{amsmath}
\usepackage{amssymb}
\usepackage{amsthm}
\newtheorem{theorem}{Theorem}
\newtheorem{lemma}{Lemma}
\newtheorem{proposition}{Proposition}
\newtheorem{definition}{Definition}
\usepackage{amsfonts}
\usepackage{times}
\usepackage{stmaryrd}
\usepackage{hyperref}
\usepackage{mwe}
\usepackage{graphicx} 
\usepackage{wrapfig} 
\usepackage{subcaption}
\usepackage{booktabs}
\usepackage{array}
\renewenvironment{proof}[1][Proof]{\noindent\textbf{#1.} }{\qed}


\usepackage[page]{appendix}
\usepackage{etoolbox}

\renewcommand{\appendixtocname}{Appendix Contents.}


\makeatletter
\renewcommand{\appendices}{%
  \clearpage
  \renewcommand{\thesection}{\Alph{section}}
  \renewcommand{\appendixname}{}
  \renewcommand{\appendixpagename}{}
  \let\tf@toc\tf@app
  \addtocontents{app}{\protect\setcounter{tocdepth}{2}}
  \immediate\write\@auxout{%
    \string\let\string\tf@toc\string\tf@app^^J
  }
}


\newcommand{\listofappendices}{%
  \begingroup
  \renewcommand{\contentsname}{\appendixtocname}
  \let\@oldstarttoc\@starttoc
  \def\@starttoc##1{\@oldstarttoc{app}}
  \tableofcontents
  \endgroup
}
\usepackage[utf8]{inputenc}
\newcommand{\BEAS}{\begin{eqnarray*}}
\newcommand{\EEAS}{\end{eqnarray*}}
\newcommand{\BEA}{\begin{eqnarray}}
\newcommand{\EEA}{\end{eqnarray}}
\newcommand{\BEQ}{\begin{equation}}
\newcommand{\EEQ}{\end{equation}}
\newcommand{\BIT}{\begin{itemize}}
\newcommand{\EIT}{\end{itemize}}
\newcommand{\BNUM}{\begin{enumerate}}
\newcommand{\ENUM}{\end{enumerate}}
\newcommand{\BA}{\begin{array}}
\newcommand{\EA}{\end{array}}
\newcommand{\diag}{\mathop{\textnormal{ diag}}}
\newcommand{\Diag}{\mathop{\textnormal{ Diag}}}
\newcommand{\rb}{\mathbb{{R}}}
\renewcommand{\labelitemiii}{$-$}
\def\defin{\stackrel{\vartriangle}{=}}
\def \tX{ \widetilde{X}}
\def \tY{ \widetilde{Y}}
\def \ds { \displaystyle}
\def \hS{ \widehat{ \Sigma} }
\def \S{  { \Sigma} }
\def \L{  { \Lambda} }
\def \E{{\mathbb E}}
\def \P{{\mathbb P}}
\def \Z{{\mathbb Z}}
\def \T{{\mathbb T}}
\def \F{{\mathcal F}}
\def \C{{\mathbb C}}
\def \M{{\mathcal M}}
\def \H{{\mathcal H}}
\def \U{{\mathcal U}}
\def \X{{\mathcal X}}
\def \Y{{\mathcal Y}}
\def \A{{\mathcal A}}
\def \S{{\mathcal S}}
\def \hQ{{\hat{Q}}}
 \def \hfl{ \hat{f}_\lambda }
 \def \hal{ \hat{\alpha}_\lambda }
  \def \fl{ {f}_\lambda }
\def \supp{ { \textnormal{ Supp }}}
\def \card{ { \rm Card }}
 \def \P{{\mathbb P}}
 



\title{\textbf{An Uncertainty Principle\\ for Linear Recurrent Neural Networks}}
\author{
  \textbf{Alexandre Fran\c{c}ois}\\
  INRIA, Ecole Normale Sup\'erieure, PSL Research University, France\\
  \texttt{alexandre.francois@inria.fr}
  \and
  \textbf{Antonio Orvieto}\\
  MPI for Intelligent Systems, ELLIS Institute T\"ubingen, Germany\\
  \texttt{antonio@tue.ellis.eu}
  \and
  \textbf{Francis Bach}\\
  INRIA, Ecole Normale Sup\'erieure, PSL Research University, France\\
  \texttt{francis.bach@inria.fr}
}
\date{}

\begin{document}
\maketitle

\begin{abstract}
 We consider linear recurrent neural networks, which have become a key building block of sequence modeling due to their ability for stable and effective long-range modeling. In this paper, we aim at characterizing this ability on a simple but core copy task, whose goal is to build a linear filter of order $S$ that approximates the filter that looks $K$ time steps in the past (which we refer to as the shift-$K$ filter), where $K$ is larger than $S$. Using classical signal models and quadratic cost, we fully characterize the problem by providing lower bounds of approximation, as well as explicit filters that achieve this lower bound up to constants. The optimal performance highlights an uncertainty principle: the optimal filter has to average values around the $K$-th time step in the past with a range~(width) that is proportional to $K/S$.

\end{abstract}

\documentclass[../main.tex]{subfiles}
\graphicspath{{../images/}}
\makeatletter
\def\input@path{{../images/}}
\makeatother
\begin{document}
\section{Introduction}
\begin{figure}
\centering
\begin{tikzpicture}
\node[inner sep=0pt] (ws) at (0, 0) {
\includegraphics[height=.4\textwidth, trim={10cm 0 10cm 0},clip]{world_space.png}};
\node[inner sep=0pt] (cs) at (6,0) {\includegraphics[height=.4\textwidth, trim={10cm 1cm 10cm 4cm},clip]{conf_space.png}};
\end{tikzpicture}
\vspace{-5pt}
\label{fig:pbrm_intro}
\caption{\textbf{Left}: Shows world space obstacles as grey spheres. Robots start and goal configuration is colored red and green, respectively. Configurations along the computed path are colored transparent blue. \textbf{Right:} Mapped world space scenario to configuration space. Obstacle region is the grey mesh. Red spheres are collision-free regions computed by the neural SCDF. The optimized shortest path in the convex corridor is the blue curve.}
\vspace{-25pt}
\end{figure}
Motion planning is the problem of finding a collision-free trajectory that connects a given start and goal configuration. The planning takes place in the configuration space of the robot. For single body robots, like mobile robots or drones, the configuration space and the world space are usually the same. This simplifies the planning, since explicit obstacle representations are available which enables geometrical tools like separating hyperplanes, smallest distance to obstacles etc., to be used when designing motion planning algorithms. For multi-body robots like manipulators, the situation is completely different. The world space obstacles are usually mapped to non-convex regions, and to make the problem even harder, the mapping is usually not known. Forming explicit representations of the obstacle region in the configuration space is usually too expensive or intractable. Despite all of this, sampling based planners are used with great success, which mainly is due to their use of implicit representations of the obstacle region. The basic idea is to construct a graph in the configuration space that covers and connects the collision-free region. From this graph, a path can be extracted that connects a given start and goal configuration. The approach is computationally expensive, since the graph is constructed with the smallest geometrical building block available, points, which represents a collision-check. Furthermore, the extracted paths from the graph are non-smooth and jagged due to the stochastic nature of the approach. This adds an additional post-processing step to the process, where the paths are shortcutted and smoothened, before the path can be used for tracking. Clearly a lot of time is invested to form this graph and produce smooth paths. Thus, if the obstacles start to move, then all of this work is done in no use, since all points that make up this graph need to be re-verified, which is simply too time consuming to be done in real time.
\\\\
In this work, we want to address the existing drawbacks of the sampling based planners. Our main contribution is an improved motion planner where each vertex in the graph covers a collision-free region in the form of a sphere instead of a point and where the edges are formed with neighboring intersecting spheres. This representation has the advantage of instead of returning piecewise linear paths, returning a sequence of overlapping spheres, i.e. a convex corridor, that connects a given start and goal configuration, illustrated in Figure \ref{fig:pbrm_intro}. This convex corridor allows us to use convex optimization to produce smooth trajectories, instead of computationally expensive post-processing methods. The representation further allows us to estimate the coverage of the collision-free space, which gives us awareness and feedback in the offline roadmap construction phase. Finally, our representation is simple to adapt to moving obstacles, simply requery for the new radii and recheck for intersections. 
\\\\
The spherical collision-free regions are formed using a signed distance function (SDF), which is a function that returns the smallest distance from an arbitrary point to the boundary of an obstacle. As the name implies, the distance is signed, thus if the point is inside the obstacle it is negative otherwise positive. If the distance is positive, a sphere with radius equal to the distance is guaranteed to cover a collision-free region. Using an SDF in motion planning is not new, but what is novel about our approach is that we express the distance in the configuration space instead of the world space and by doing so allows us to form these convex collision-free regions. We refer to the resulting SDF as a signed configuration distance function (SCDF). Computing an SCDF analytically is non-trivial, our approach is therefore to parameterize the SCDF with a deep neural network and learn the mapping by supervised learning. Our resulting neural SCDF can compute distances for different parameter values of obstacle shapes and we also show how multiple distances can be combined, thus making our approach flexible.
\section{Related work}
Motion planning algorithms can roughly be divided into three families, grid-based, sampling based and optimization based methods. Grid-based methods (GBM) discretize the planning space from which a graph is then compiled. A standard search method is A$^\star$ \citep{a_star}, which is classified as an \textit{informed} search method, since it employs a heuristic function to speed up the search. A$^\star$ guarantees to return an optimal path at the level of discretization used. GBMs usually discretize the planning space by a regular lattice and this limits the GBMs to problems with low dimensionality due to the curse of dimensionality. Thus, GBMs are usually limited to single-body robots where the degrees of freedom (DOF) are low. To overcome the inherent scaling problem with the GBMs, stochastic methods are usually used for multi-body robots. These methods are termed as sampling-based methods (SBM) and core members within this family are the rapidly-exploring random trees (RRT) \citep{rrt} and the probabilistic roadmap (PRM) \citep{prm}. RRT grows a tree from the start configuration and explores the collision-free region in a rapid way until it is able to connect to the goal region. RRT is usually improved by bi-directional planning \citep{rrt_connect}, i.e. an additional tree is grown from the goal configuration and the trees are tested for connection after any tree has been expanded. RRT is a single-query method, thus it searches for a path from scratch each time it is queried. Contrary to this, PRM is a multi-query method, which solves for multiple queries without starting from scratch. PRM does this by creating a roadmap (graph) that covers the collision-free space as an offline step. The graph is then used to solve for multiple queries. PRMs are used in cases where the environment does not change since the extra offline step is too computationally costly and needs to be re-done if the environment is changed. In our work, we address this inherent issue by using a different roadmap representation. Our vertices in the graph cover a collision-free region in the form of spheres and we form the edges by checking for intersecting spheres. If something in the environment changes, we recompute the spheres radii and recheck the intersections, without relying on collision detection. We use a trained neural network to compute the sphere radius, therefore querying for the radius can be done fast, hence our representation enables the PRM for dynamic environments.
\\\\
In the recent decades, optimization based methods (OBM) \citep{chomp, schulman, itomp, stomp} have been introduced as an alternative to SBM for multi-body robots. Like the SBM, the OBMs scale well to higher dimensional problems and produce smoother motion. It is common to use a SDF in the optimization since it is a smooth function, thus enabling gradient-based methods. However, the standard way of expressing the SDF is in world space. The distance therefore needs to be mapped to the configuration space by the forward kinematics. This mapping makes the optimization problem a non-linear program (NLP), which is computationally expensive to solve. Recently, a different approach has been proposed. In \cite{mp_gcs} motion planning is formulated as a convex optimization problem by using the graph of convex sets framework \citep{gcs}. The underlying idea is to decompose the collision-free space into intersecting convex sets from which a convex optimization problem is formulated. In cases where an explicit representation of the obstacles in the configuration space exists, like for single-body robots, creating collision-free convex regions can be done fast \citep{iris}. For multi-body robots, this is non-trivial. Existing work does this successfully \citep{iris_nlp, iris_c} by an optimization based approach, but the methods are still too time consuming to be used in the presence of moving obstacles. Our approach is instead to use deep learning to learn an SDF expressed in the configuration space. With this, we can query for shortest distances to the collision boundary, which allows us to expand spherical regions which are collision-free. Our approach is fast and therefore enables our suggested roadmap planner to be used in dynamic environments.
\\\\
Recent research has focused on learning collision detection \citep{fk_kernel_distance, diffco, graphdistnet} by predicting the signed distance between the robot links and the surrounding obstacles in the world space. The learned SDF is used in trajectory optimization but since the distance is expressed in the world space, the problem becomes an NLP and therefore takes a long time to solve. We take a novel approach and suggest to instead express the signed distance in the configuration space. This allows us to improve the PRM at the same time as it enables convex optimization for trajectory optimization, which runs faster and is more reliable than NLP solvers. In \cite{cspf} a learned signed distance function in the configuration space is proposed similar to our approach. However, their approach is restricted to point cloud representations, while we propose to represent the obstacles as parameterized geometric shapes, e.g. spheres. Furthermore, we also show how to use our learned SCDF to improve an existing roadmap planner.
\section{Problem formulation}
A robot is located in the world space, $\W \subset \R^3 $. The unique location of the robot is given by its configuration $\q \in \C$, where $\C$ is the configuration space. The set of points covered by the robots bodies at a certain configuration is expressed as $\B(\q) \subset \W$. The robot is surrounded by $\NrObst$ obstacles $\O = \bigcup_{i=1}^{\NrObst} \O_i$, where  $\O_i \subset \W$. The representation of the obstacle in the configuration space is the set $\C\O_i = \{\q \in \C \: |\: \B(\q) \cap \O_i \neq \emptyset \}$. The obstacle space is formed as $\Co = \bigcup_{i=1}^{\NrObst} \C \O_i$. The complement is referred to as the free space, $\Cf = \C \setminus \Co$. The path planning problem is a tuple, ($\Cf$, $\qStart$, $\qGoal$), where we want to connect a query pair, consisting of a start, $\qStart$, and goal configuration, $\qGoal$, with a geometric path, $\q(s): [0, 1] \mapsto \Cf$, such that $\q(0)=\qStart$ and $\q(1)=\qGoal$, or report correctly when such a path does not exist.
\end{document}

\subsection{Lower bounds on sample complexity}\label{sec:sample_compexity}
We establish a lower bound for generalized linear measurements using standard information-theoretic arguments based on Fano's inequality. While the upper bound in Theorem~\ref{thm:alg_general} is derived for the maximum probability of error over all  $k$-sparse vectors, the lower bound applies even in the weaker setting of the average probability of error, where 
$\bx$ is chosen uniformly at random.
\begin{theorem}[Lower bound for GLMs]\label{thm: lower_bdglm} Consider any  sensing matrix $\vecA$.
For a uniformly chosen $k$-sparse vector $\bx$, an algorithm $\phi$ satisfies $$\bbP\inp{\phi(\vecA, \by) \neq \bx}\leq \delta$$   only if the number of measurements $$m\geq \frac{k\log\inp{\frac{n}{k}}}{I}\inp{1 - \frac{h_2(\delta) + \delta k\log{n}}{k\log{n/k}}}$$ for some $I$ such that $I\geq {I(y_i; \bx|\vecA)}, \, i\in [m]$. In particular, when $y\in \inb{-1, 1}$, we have $\bbE\insq{\inp{g(\vecA_i^T\bx)}^2} \geq I(y_i, \bx|\vecA)$ where the expectation is over the randomness of $\vecA$ and $\bx$.
\end{theorem}
The lower bound can be interpreted in terms of a communication problem, where the input message $\bx$ is encoded to $\vecA\bx$. The decoding function takes in as input the encoding map $\vecA$ and the output vector $\by$ in order to recover $\bx$ with high probability. For optimal recovery, one needs at least $\frac{\text{message entropy}}{\text{capacity}}$ number of measurements (follows from noisy channel coding theorem~\cite{thomas2006elements}). In Theorem~\ref{thm: lower_bdglm}, the entropy of the message set $\log{n \choose k}\approx k\log{n/k}$ and the proxy for capacity is the upper bound on mutual information $I$. We provide a detailed proof of the theorem in  Section~\ref{sec:proofs}.


We first present lower bounds for \bcs\  and \logreg. The lower bound for \bcs\ is given for any sensing matrix $\vecA$ which satisfies the power constraint given by \eqref{eq:power_constraint}, whereas the one for \logreg\ is only for the special case when each entry of the sensing matrix is iid $\cN(0,1)$. Recall that \eqref{eq:power_constraint} holds in this case.  For \bcs\ (and \logreg\ respectively), we can use the upper bound of $\bbE\insq{\inp{g(\vecA_i^T\bx)}^2}$ on the mutual information term. The dependence of $\sigma^2$ (and $1/\beta^2$ respectively) requires careful bounding of this term, which is done in the formal proofs in Appendix~\ref{proof:sec:lower_bd}.


As mentioned earlier, we need at least $k\log\inp{n/k}$ measurements for \bcs and \logreg. This is because the entropy of a randomly chosen $k$-sparse vector is approximately $k\log\inp{n/k}$ and we learn at most one bit with each measurement. However, due to corruption with noise, we learn less than a bit of information about the unknown signal with each measurement. The information gain gets worse as the noise level increases. 
Our lower bounds make this reasoning explicit.  
\begin{corollary}[\bcs\ lower bound]\label{thm: lower_bd_bcs} Suppose, each row $\vecA_i, \, i\in [1:m]$ of the sensing matrix $\vecA$ satisfies the power constraint~\eqref{eq:power_constraint}.
For a uniformly chosen $k$-sparse vector $\bx$, an algorithm $\phi$ satisfies $$\bbP\inp{\phi(\vecA, {\by}) \neq \bx}\leq \delta$$ for the problem of $\bcs$ only if the number of measurements $$m\geq \frac{k+\sigma^2}{2}\log\inp{\frac{n}{k}}\inp{1 - \frac{h_2(\delta) + \delta k\log{n}}{k\log{n/k}}}.$$ 
\end{corollary}

\begin{corollary}[\logreg\ lower bound]\label{thm: lower_bd_log_reg} Consider a Gaussian  sensing matrix $\vecA$ where each entry is chosen iid $N(0,1)$.
For a uniformly chosen $k$-sparse vector $\bx$, an algorithm $\phi$ satisfies $$\bbP\inp{\phi(\vecA, \bw) \neq \bx}\leq \delta$$ for the problem of $\logreg$ only if the number of measurements $$m\geq \frac{1}{2}\inp{k+\frac{1}{\beta^2}}\log\inp{\frac{n}{k}}\inp{1 - \frac{h_2(\delta) + \delta k\log{n}}{k\log{n/k}}}.$$ 
\end{corollary}



Theorem~\ref{thm: lower_bdglm} also implies an information theoretic lower bound for \spl, which is presented below and proved in Appendix~\ref{proof:sec:lower_bd}. Note that the denominator term in the bound $\frac{1}{2}\log\inp{1+\frac{k}{\sigma^2}}$ is the capacity of a Gaussian channel with power constraint $k$ and noise variance $\sigma^2$. 
\begin{corollary}[\spl\ lower bound]\label{thm: spl_lower_bd_1}
Under the average power constraint \eqref{eq:power_constraint} on  $\vecA$, for a uniformly chosen $k$-sparse vector $\bx$, an algorithm $\phi$ satisfies $$\bbP\inp{\phi(\vecA, {\by}) \neq \bx}\leq \delta$$ only if the number of measurements
$$m\geq \frac{k\log\inp{\frac{n}{k}}-\inp{h_2(\delta) + \delta k\log{n}}}{\frac{1}{2}\log\inp{1+\frac{k}{\sigma^2}}}.$$
\end{corollary} 

\subsection{Tighter upper and lower bounds for \spl}\label{sec:tighter_bounds_spl}
We present information theoretic upper and lower bounds for \spl\ in this section. Similar to Section~\ref{sec:alg}, our upper bound is for the maximum probability of error, while the lower bounds hold even for the weaker criterion of average probability of error.

We first present an upper bound based on the maximum likelihood estimator (MLE) where  we  decode to $\hat{\bx}$ if, on output $\by$, 
\begin{align*}
\hat{\bx} = \argmax_{\stackrel{\bx\in \inb{0,1}^n}{\wh{\bx} = k}}\,\, p(\by|{\bx})
\end{align*} where $p(\by|{\bx})$ denotes the probability density function of $\by$ on input $\bx$.
\begin{theorem}[MLE upper bound for \spl]\label{thm:upper_bd_mle} Suppose  entries of the measurement matrix $\vecA$ are i.i.d. $\cN(0,1).$
The MLE  is correct with high probability if 
\begin{align}m\geq \max_{l\in[1:k]}  \frac{nN(l)}{\frac{1}{2}\log\inp{\frac{ l}{2\sigma^2}+1}}\label{eq:upper_bd_mle}
\end{align}where  $N(l):=  \frac{k}{n} h_2\inp{\frac{l}{k}} + (1-\frac{k}{n})h_2\inp{\frac{l}{n-k}}$. 
\end{theorem}
We prove the theorem in Appendix~\ref{proof:MLE}. The main proof idea involves analysing the probability that the output of the MLE is $2l$ Hamming distance away from the unknown signal $\bx$ for different values of $l\in [1:k]$ (assuming $k\leq n/2$). This depends on the number of such vectors (approximately $2^{nN(l)}$) and the probability that the MLE outputs a vector which is $2l$ Hamming distance away from $\bx$. 

Note that when $l = k\inp{1-\frac{k}{n}}$, $nN(l) = nh_2(k/n)\approx k\log{\frac{n}{k}}$ and $\log\inp{\frac{k\inp{1-k/n}}{2\sigma^2}+1}\leq \log\inp{\frac{k}{2\sigma^2}+1}$.
Thus, $m$ is at least $\frac{2k\log{n/k}}{\log\inp{\frac{k}{2\sigma^2}+1}}$ (see the bound for Corollary~\ref{thm: spl_lower_bd_1}). It is not immediately clear if this value of $l= k\inp{1-\frac{k}{n}}$ is the optimizer. However, for large $n$, this appears to be the case numerically as shown in Plot~\ref{plot:1}.

\begin{figure}[t]
\includegraphics[width=7cm]{Unknown2.png}
\centering
\caption{The figure shows the plot of the MLE upper bound \eqref{eq:upper_bd_mle} (given by m1) for different values of $k$. This is displayed in blue color. A plot of $\frac{2nN(l)}{\log\inp{\frac{ l}{2\sigma^2}+1}}$ is also presented for $l = k\inp{1-\frac{k}{n}}$ in orange color, given by m2. A part of the plot is zoomed in to emphasize the closeness between the lines. In these plots,  $\sigma^2$ is set to 1,  $n$ is 50000 and $k$ ranges from 1000 to 25000 $(n/2)$. }\label{plot:1}
\end{figure}


Inspired by the MLE analysis, we derive a lower bound with the same structure as \eqref{eq:upper_bd_mle}. We generate the unknown signal $\bx$ using the following distribution: A vector $\tilde{\bx}$ is chosen uniformly at random from the set of all $k$-sparse vectors. Given $\tilde{\bx}$, the unknown input signal $\bx$ is chosen uniformly from the set of all $k$-sparse vector which are at a Hamming distance $2l$ from $\bx$. 
The lower bound is then obtained by computing upper and lower bounds on $I(\vecA, \by;\bx|\tilde{\bx})$.
We show this lower bound only for random matrices where each entry is chosen iid $\cN(0,1)$.
\begin{theorem}[\spl\ lower bound]\label{thm:lower_bd_spl}
If each entry of $\vecA$ is chosen iid $\cN(0,1)$, then for a uniformly chosen $k$-sparse vector $\bx$, an algorithm $\phi$ satisfies 
\begin{align}
    \bbP\inp{\phi(\vecA, {\by}) \neq \bx}\leq \delta\label{eq:spl_lower_bd_l}
\end{align}  only if the number of measurements $$m\geq \max_l\frac{nN(l) - 2\log{n}- h_2(\delta) - \delta k\log{n}}{\frac{1}{2}\log\inp{1+\frac{l}{\sigma^2}\inp{2-\frac{l}{k}}}} .$$
\end{theorem} The proof of Theorem~\ref{thm:lower_bd_spl} is given in Appendix~\ref{proof:MLE}.

If we choose $l = k\inp{1-\frac{k}{n}}$ in Theorem~\ref{thm:lower_bd_spl}, we recover corollary~\ref{thm: spl_lower_bd_1} for the special case of Gaussian design.
% \begin{corollary}\label{corollary2:lower_bd_spl}
% If  each entry of $\vecA$ is chosen iid $\cN(0,1)$, then for a uniformly chosen $k$-sparse vector $\bx$, an algorithm $\phi$ satisfies 
% $$\bbP\inp{\phi(\vecA, {\by}) \neq \bx}\leq \delta$$
% only if the number of measurements 
% $$m\geq \frac{k\log\inp{\frac{n}{k}} - 2\log{n}- h_2(\delta) - \delta k\log{n}}{\log\inp{1+\frac{k}{\sigma^2}}} .$$
% \end{corollary}

% Corollary~\ref{corollary2:lower_bd_spl} can also be proved directly for any sensing matrix $\vecA$ which satisfies \eqref{eq:power_constraint} (non-necessarily a Gaussian design). 


% \begin{figure}[t]
% \includegraphics[width=8cm]{plot.png}
% \centering
% \caption{The figure shows the plot of the MLE upper bound \eqref{eq:upper_bd_mle} (given by m1) for different values of $n$. This is displayed in blue color. A plot of $\frac{2nN(l)}{\log\inp{\frac{ l}{2\sigma^2}+1}}$ is also presented for $l = k\inp{1-\frac{k}{n}}$ in orange color, given by m2. In these plots,  $\sigma^2$ is set to 1 and $k$ is $0.2n$. }\label{plot:1}
% \end{figure}


\subsection{The Dual Sampling Matrix}\label{sec:dual_sampling_matrix}
Associated with each triplet $(u; i, j)$, we define the \textit{dual sampling matrix} as follows:
\begin{align}\label{eq:def_B}
    B \in \mathbb{R}^{n_1 \times n_2} : B = e_u(\Tilde{e_i} + \Tilde{e}_j)^T
\end{align}
If we endow the triplets with randomness, $B$ is a random matrix, whose mean is:
\begin{align}\label{eq:dual_matrix_mean}
    \Bar{B} \triangleq \mathbb{E}[{B}] &= \mathbb{E}[e_u(\Tilde{e}_i + \Tilde{e}_j)^T] = \mathbb{E}[e_u]\mathbb{E}[(\Tilde{e}_i + \Tilde{e}_j)^T] = \frac{2}{n_1n_2}11^T
\end{align}
Here, $11^T$ is a matrix of all ones of shape $n_1 \times n_2$.

Let $B_1, \ldots,  B_{\Dataset}$ denote the dual sampling matrices for each of the datapoints, similar to the notation for $A$. Define the empirical mean of the dual sampling matrices, $B_{\Dataset}$, as follows:
\begin{align}
    B_{\Dataset} = \frac{1}{m}\sum_{k = 1}^mB_k
\end{align}
In our analysis, we will use the fact that this empirical mean $B_{\Dataset}$ is close to the statistical mean $\Bar{B}$, in a manner made precise by Lemma \ref{lem:BD_concentration}. In preparation for this concentration result, we two parameters, $L$ and $b$. (The same notation was used to denote related terms for the random matrix $S_{\Dataset}$ in the previous section; however, the correct interpretation should be clear from context.) $L$ is a uniform bound on $\norm{B}_2$. For each triplet $(u; i, j)$, the operator norm of the corresponding dual sampling matrix is $\sqrt{2}$. It follows that $L = \sqrt{2}$. The definition and bound for $v$ is given in the lemma below.
\begin{lemma}\label{lem:B_bounds}
    Let $B$ be the random dual sampling matrix as defined above. Let $b^1 \triangleq \norm{\mathbb{E}[B^TB]}_2$,  $b^2 \triangleq \norm{\mathbb{E}[BB^T]}_2$, and $b = \max\{b^1, b^2\}$. Then $$b \leq \frac{4}{\min\{n_1, n_2\}}.$$
\end{lemma}
\begin{proof}
We know that $\norm{e_u}_2^2 = 1$ and $\norm{\Tilde{e}_i + \Tilde{e}_j}_2^2 = 2$ almost surely. Further,
\begin{align*}
    \mathbb{E}[e_ue_u^T] = \frac{1}{n_1}I_{n_1} , \qquad 
    \mathbb{E}[(\Tilde{e}_i + \Tilde{e}_j)(\Tilde{e}_i + \Tilde{e}_j)^T] = \frac{1}{\binom{n_2}{2}}(11^T + (n_2-2)I_{n_2})
\end{align*}
Using these identities, we get
\begin{align*}
    \mathbb{E}[B^TB] 
    &= \mathbb{E}[(\Tilde{e}_i + \Tilde{e}_j)e_u^Te_u(\Tilde{e}_i + \Tilde{e}_j)^T] \\
    &= \mathbb{E}_{i,j}[(\Tilde{e}_i + \Tilde{e}_j)\mathbb{E}_u[e_u^Te_u]w^T] \\
    &= \mathbb{E}_{i,j}[(\Tilde{e}_i + \Tilde{e}_j)(\Tilde{e}_i + \Tilde{e}_j)^T] \\
    &= \frac{1}{\binom{n_2}{2}}(11^T + (n_2-2)I_{n_2}) \\
    \mathbb{E}[BB^T] 
    &= \mathbb{E}[e_u(\Tilde{e}_i + \Tilde{e}_j)^T(\Tilde{e}_i + \Tilde{e}_j)e_u^T] \\
    &= \mathbb{E}_u[e_u\mathbb{E}_{i,j}[(\Tilde{e}_i + \Tilde{e}_j)^T(\Tilde{e}_i + \Tilde{e}_j)]e_u^T] \\
    &= 2\mathbb{E}_u[e_ue_u^T] \\
    &= \frac{2}{n_1}I_{n_1}
\end{align*}
Computing the operator norms of these matrices is straightforward:
\begin{align*}
    b^1 &= \norm{\mathbb{E}[B^TB]}_2 = \frac{1}{\binom{n_2}{2}} \norm{11^T + (n_2-2)I_{n_2}}_2 \leq \frac{1}{\binom{n_2}{2}} \left(\norm{11^T}_2 + (n_2-2)\norm{I_{n_2}}_2 \right)= \frac{1}{\binom{n_2}{2}} \left(n_2 + (n_2-2)\right)= \frac{4}{n_2} \\
    b^2 &= \norm{\mathbb{E}[BB^T]}_2 = \frac{2}{n_1} \norm{I_{n_1}}_2 = \frac{2}{n_1} \leq  \frac{4}{n_1}
\end{align*}
\begin{align*}
    \therefore b = \max\{b^1, b^2\} \leq \frac{4}{\min\{n_1, n_2\}}
\end{align*}
\end{proof}

\subsection{Algebraic Upper Bounds on $\Dataset(WZ^T)$}
This subsection contains three lemmas that we shall use in the proof of Lemmas \ref{lem:convexity_upperbound} and \ref{lem:smoothness_upperbound}. The first of these three lemmas, Lemma \ref{lem:D_to_dual_matrix}, gives an upper bound on $\Dataset(WZ^T)$ as a quadratic form around the random matrix $B_{\Dataset}$ that we defined earlier in the section.

Before we state the result, we introduce some additional notation. Corresponding to any matrix $Z \in \Real{n \times r}$, define the vector $z \in \Real{n}$ as follows:
\begin{align}\label{eq:def_z}
    z_j = \norm{Z_j}_2 \ \forall \ j \in [n]
\end{align}
It follows from the definition that
\begin{align}\label{eq:z_Z_identities}
    \norm{z}_{1} = \norm{Z}_{F}^2, \ \norm{z}_{\infty} = \norm{Z}_{2, \infty}^2 
\end{align}
Following the convention of splitting the matrix $Z$ into user and item components $Z = (Z_U, Z_V)$, we split the vector $z$ into vectors $z_U \in \Real{n_1}$ and $z_V \in \Real{n_2}$ ($z = (z_U, z_V)$). The norms of these vectors satisfy the following relations:
\begin{align}\label{eq:z_vector_relations}
    \norm{z}_{1} = \norm{z_U}_{1} + \norm{z_V}_{1}, \ \norm{z}_{2}^2 = \norm{z_U}_{2}^2 + \norm{z_V}_{2}^2, \  \norm{z}_{\infty} = \max \{ \norm{z_U}_{\infty}, \norm{z_V}_{\infty}\}
\end{align}

With these notations and identities in place, we proceed to establish the following result.
\begin{lemma}\label{lem:D_to_dual_matrix}
    For any two matrices $W$ and $Z$ in $\Real{n \times r}$,
    \begin{align*}
        \mathcal{D}(WZ^T) \leq &4(w_U^T B_{\Dataset} z_V + z_U^T B_{\Dataset} w_V) 
    \end{align*}
\end{lemma}
\begin{proof}
\begin{align*}
    \mathcal{D}(WZ^T) 
    &= \frac{1}{m} \sum_{k = 1}^m \llangle A_k + A_k^T, WZ^T \rrangle^2 \quad (\text{by } \eqref{eq:def_D_operator})\\
    &= \frac{1}{m} \sum_{(u; i, j) \in \Dataset} (\llangle W_u, Z_i - Z_j \rrangle + \llangle Z_u, W_i - W_j \rrangle)^2 \quad (\text{by \eqref{eq:A_AT_YZT_identity2}})\\
    &\leq \frac{2}{m} \sum_{(u; i, j) \in \Dataset}  \llangle W_u, Z_i - Z_j \rrangle^2 + \llangle Z_u, W_i - W_j \rrangle^2 \quad (\text{by } (a+b)^2 \leq 2(a^2 + b^2))\\
    &\leq \frac{2}{m} \sum_{(u; i, j) \in \Dataset}  \norm{W_u}_2^2\norm{Z_i - Z_j}_2^2 + \norm{Z_u}_2^2\norm{W_i - W_j}_2^2  \quad (\text{by Cauchy-Schwarz inequality})\\
    &\leq \frac{2}{m} \sum_{(u; i, j) \in \Dataset}  \norm{W_u}_2^2(\norm{Z_i}_2 + \norm{Z_j}_2)^2 + \norm{Z_u}_2^2(\norm{W_i}_2 + \norm{W_j}_2)^2  \quad (\text{by triangle inequality})\\
    &\leq \frac{4}{m} \sum_{(u; i, j) \in \Dataset}  \norm{W_u}_2^2(\norm{Z_i}_2^2 + \norm{Z_j}_2^2) + \norm{Z_u}_2^2(\norm{W_i}_2^2 + \norm{W_j}_2^2)  \quad (\text{by } (a+b)^2 \leq 2(a^2 + b^2))\\
    &= \frac{4}{m} \sum_{(u; i, j) \in \Dataset} w_u(z_i + z_j) + \frac{4}{m} \sum_{(u; i, j) \in \Dataset} z_u(w_i + w_j) \\
    &= \frac{4}{m} \sum_{(u; i, j) \in \Dataset} w_U^T\left(e_u(\Tilde{e_i} + \Tilde{e}_j)^T\right)z_V + \frac{4}{m} \sum_{(u; i, j) \in \Dataset} z_U^T\left(e_u(\Tilde{e_i} + \Tilde{e}_j)^T\right)w_V  \\
    &= 4 w_U^T\left(\frac{1}{m} \sum_{(u; i, j) \in \Dataset}e_u(\Tilde{e_i} + \Tilde{e}_j)^T\right)z_V + 4 z_U^T\left(\frac{1}{m} \sum_{(u; i, j) \in \Dataset} e_u(\Tilde{e_i} + \Tilde{e}_j)^T\right)w_V  \\
    &= 4w_U^T B_{\Dataset} z_V + 4z_U^T B_{\Dataset} w_V
\end{align*}
\end{proof}

The next lemma builds upon the previous result to obtain an upper bound in terms of $\norm{B_{\Dataset} - \Bar{B}}_{2}$.
\begin{lemma}\label{lem:deltadelta_intermediate}
    For any $Z \in \Real{n \times r}$,
    \begin{align*}
        \Dataset(ZZ^T) &\leq 2\left( \gamma \norm{Z}_F^2 + 2\norm{B_{\Dataset} - \Bar{B}}_{2} \norm{Z}_{2,\infty}^2 \right)\norm{Z}_F^2
    \end{align*}
\end{lemma}
\begin{proof}
We start by using the relations in \eqref{eq:z_vector_relations} along with the arithmetic mean-geometric mean (AM-GM) inequality to obtain the following bound
\begin{align}\label{eq:am_gm_inequality}
    \norm{z_U}_1\norm{z_V}_1 \leq \left(\frac{\norm{z_U}_1 + \norm{z_V}_1}{2}\right)^2 = \frac{\norm{z}_1^2}{4}, \quad \norm{z_U}_2\norm{z_V}_2 \leq \left(\frac{\norm{z_U}_2 + \norm{z_V}_2}{2}\right)^2 \leq \frac{\norm{z}_2^2}{2} 
\end{align}
Using the bound in \eqref{eq:am_gm_inequality}, we can show the desired result as follows.
\begin{align*}
    \frac{\Dataset(ZZ^T)}{8}
    &\leq z_U^T B_{\Dataset}z_V \quad (\text{by Lemma \ref{lem:D_to_dual_matrix}})\\
    &=  z_U^T\Bar{B}z_V + z_U^T(B_{\Dataset} - \Bar{B})z_V \\
    &\leq z_U^T\Bar{B}z_V + \norm{z_U}_2 \norm{(B_{\Dataset} - \Bar{B})z_V}_2 \quad (\text{by the Cauchy-Schwarz inequality}) \\
    &\leq  z_U^T\Bar{B}z_V + \norm{B_{\Dataset} - \Bar{B}}_{2}\norm{z_U}_2\norm{z_V}_2  \quad (\text{by definition of the operator norm})\\
    &=  \frac{2}{n_1n_2}z_U^T11^Tz_V + \norm{B_{\Dataset} - \Bar{B}}_{2}\norm{z_U}_2\norm{z_V}_2\quad (\text{by } \eqref{eq:dual_matrix_mean})\\
    &\leq  \gamma z_U^T11^Tz_V + \norm{B_{\Dataset} - \Bar{B}}_{2}\norm{z_U}_2\norm{z_V}_2 \quad (2/(n_1n_2) \leq  2/(n_1(n_2 - 1)) = \gamma)\\
    &\leq \gamma\norm{z_U}_1\norm{z_V}_1 + \norm{B_{\Dataset} - \Bar{B}}_{2}\norm{z_U}_2\norm{z_V}_2 \quad (1^Tz \leq \norm{z}_1)\\
    &\leq \frac{1}{4}\left(\gamma\norm{z}_1^2 + 2\norm{B_{\Dataset} - \Bar{B}}_{2}\norm{z}_2^2\right) \quad (\text{by \eqref{eq:am_gm_inequality}})\\
    &\leq \frac{1}{4}\left(\gamma\norm{z}_1^2 + 2\norm{B_{\Dataset} - \Bar{B}}_{2}\norm{z}_\infty \norm{z}_1\right) \quad (\text{by Hölder's inequality})\\
    &= \frac{1}{4} \left(\gamma\norm{Z}_F^2 + 2\norm{B_{\Dataset} - \Bar{B}}_{2}\norm{Z}_{2,\infty}^2\right)\norm{Z}_F^2 \quad (\text{by } \eqref{eq:z_Z_identities}) \\
    \therefore \Dataset(ZZ^T) &\leq 2\left(\gamma\norm{Z}_F^2 + 2\norm{B_{\Dataset} - \Bar{B}}_{2}\norm{Z}_{2,\infty}^2\right)\norm{Z}_F^2
\end{align*}
\end{proof}

The third and final result of this section builds on Lemma \ref{lem:D_to_dual_matrix} in a different way as compared to the previous one. 
Here, we obtain a bound in terms of the $\ell_1$ operator norm of $B_{\Dataset}$. For any matrix $X \in \Real{n_1 \times n_2}$, 
\begin{align}\label{eq:def_operator_onenorm}
    \norm{X}_1 \triangleq \sup_{v: \norm{v}_1 = 1} \norm{Xv}_1
\end{align}
It follows that for any $v \in \Real{n_2}$,
\begin{align}\label{eq:operator_onenorm_consequence}
    \norm{Xv}_1 \leq \norm{X}_1 \norm{v}_1
\end{align}
It can be easily shown that
\begin{align}\label{eq:operator_onenorm_calculation}
    \norm{X}_1 =\max_{j \in [n_2]} \sum_{i \in [n_1]}|x_{ij}|
\end{align}
In addition, we will need Hölder's inequality, which states that for any vectors $a, b$, 
\begin{align}\label{eq:holder_inequality}
    \langle a, b \rangle \leq  \norm{a}_{\infty} \norm{b}_1  \ \Rightarrow \norm{a}_2^2 \leq \norm{a}_{\infty} \norm{a}_1 
\end{align}

Using these inequalities, we get the next result.
\begin{lemma}\label{lem:quadratic_form_to_one_norm}
    For any matrices $W, Z \in \mathbb{R}^{n \times r}$,
    \begin{align*}
        \mathcal{D}(WZ^T) \leq 4(\max\{\norm{B_\Dataset}_1, \norm{B_\Dataset^T}_1\}) \norm{Z}_{2, \infty}^2\norm{W}_F^2,
    \end{align*}
\end{lemma}
\begin{proof}
We start by invoking Lemma \ref{lem:D_to_dual_matrix}, we get:
\begin{align*}
    \mathcal{D}(WZ^T) &\leq 4w_U^T B_{\Dataset} z_V + 4z_U^T B_{\Dataset} w_V \\
    &= 4z_V^T B_{\Dataset}^T w_U + 4z_U^T B_{\Dataset} w_V
\end{align*}
Applying \eqref{eq:z_vector_relations}, \eqref{eq:operator_onenorm_consequence} and \eqref{eq:holder_inequality}, we get:
\begin{align*}
    z_V^T B_{\Dataset}^T w_U &= \langle z_V, B_{\Dataset}^T w_U \rangle \leq \norm{z_V}_{\infty}\norm{B_{\Dataset}^T w_U }_{1} \leq \norm{z_V}_{\infty} \norm{B_{\Dataset}^T}_1 \norm{w_U}_{1} \leq \norm{z}_{\infty} \norm{B_{\Dataset}^T}_1 \norm{w_U}_{1}\\
    z_U^T B_{\Dataset} w_V &= \langle z_U, B_{\Dataset} w_V \rangle \leq \norm{z_U}_{\infty}\norm{B_{\Dataset}^T w_V}_{1} \leq \norm{z_U}_{\infty} \norm{B_{\Dataset}}_1 \norm{w_V}_{1} \leq \norm{z}_{\infty} \norm{B_{\Dataset}}_1 \norm{w_V}_{1}
\end{align*}
Putting the above inequalities together, we get the desired result:
\begin{align*}
    \mathcal{D}(WZ^T) 
    &\leq 4z_V^T B_{\Dataset}^T w_U + 4z_U^T B_{\Dataset} w_V \\
    &\leq 4\norm{z}_{\infty} \norm{B_{\Dataset}^T}_1 \norm{w_U}_{1} + 4\norm{z}_{\infty} \norm{B_{\Dataset}}_1 \norm{w_V}_{1} \\
    &\leq 4\norm{z}_{\infty} (\max\{\norm{B_\Dataset}_1, \norm{B_\Dataset^T}_1\}) (\norm{w_U}_{1} + \norm{w_V}_{1}) \\
    &= 4\norm{z}_{\infty} (\max\{\norm{B_\Dataset}_1, \norm{B_\Dataset^T}_1\}) \norm{w}_{1} \quad (\text{by} \eqref{eq:z_vector_relations}) \\
    &= 4\norm{Z}_{2, \infty}^2 (\max\{\norm{B_\Dataset}_1, \norm{B_\Dataset^T}_1\}) \norm{W}_F^2 \quad (\text{by} \eqref{eq:z_Z_identities})   
\end{align*}
\end{proof}

\subsection{Norm Bounds on the Dual Sampling Matrix}
First, we provide an upper bound on $\norm{B_{\mathcal{D}} - \Bar{B}}_2$. This result will be used in conjunction with Lemma \ref{lem:deltadelta_intermediate} to prove Lemma \ref{lem:convexity_upperbound}.
\begin{lemma}\label{lem:BD_concentration}
    Let $\epsilon \in (0, 1)$ and $\delta \in (0, 1)$ be given. Suppose the number of samples $m$ is at least $(5/\epsilon^2)n\log(n/\delta)$.
    Then, with probability at least $1 - \delta$,
    \begin{align*}
        \norm{B_{\mathcal{D}} - \Bar{B}}_2 \leq \frac{\epsilon}{\min\{n_1, n_2\}}
    \end{align*}
\end{lemma}
\begin{proof}
    The matrix Bernstein inequality (Lemma \ref{lem:matrix_bernstein}) states that 
    \begin{align*}
        P(\norm{\bar{B}_m - \bar{B}}_2 \geq t) \leq n\exp\left(-\frac{mt^2/2}{v + 2Lt/3}\right),
    \end{align*}
    where $v = \max\{\norm{\mathbb{E}[BB^T]}_2, \norm{\mathbb{E}[B^TB]}_2\}$ and $L = \sup_{B} \norm{B}_2$. We have already established that $L = \sqrt{2}$ and $v = 4/(\min\{n_1, n_2\})$ (see Lemma \ref{lem:B_bounds}).  We would like $\norm{\bar{B}_m - \bar{B}}_2$ to be bounded above by $t = \epsilon/(\min\{n_1, n_2\})$ (for some $\epsilon \in (0, 1)$) with probability at least $1-\delta$. Therefore, the number of samples $m$ must satisfy:
    \begin{align*}
        n\exp\left(-\frac{mt^2/2}{v + 2Lt/3}\right) &\leq \delta \\
        \Leftrightarrow \frac{mt^2/2}{v + 2Lt/3} &\geq \log\left(\frac{n}{\delta}\right) 
    \end{align*}
    Plugging in the value $L = \sqrt{2}$ and noting that $v = 4t/\epsilon$, we get
    \begin{align*}
        \frac{mt^2/2}{(4t/\epsilon) + 2\sqrt{2}t/3} &\geq \log\left(\frac{n}{\delta}\right) \\
        \Leftrightarrow \frac{m}{4/\epsilon + 2\sqrt{2}/3} &\geq \frac{2}{t}\log\left(\frac{n}{\delta}\right) \\
        \Leftrightarrow m &\geq \left(\frac{4}{\epsilon} + \frac{2\sqrt{2}}{3} \right)\frac{2\min\{n_1, n_2\}}{\epsilon}\log\left(\frac{n}{\delta}\right)
    \end{align*}
    Finally, note that $4 + 2\sqrt{2}\epsilon/3 \leq 5$ ($\because \epsilon < 1$) and $2\min\{n_1, n_2\} \leq n_1 + n_2 = n$. Therefore, $m \geq (5/\epsilon^2) n\log\left(n/\delta\right)$ is a sufficient condition for the concentration result to hold.
\end{proof}

Next, we move on to proving a high probability bound on $\max\{\norm{B_\Dataset}_1, \norm{B_\Dataset^T}_1\}$. This result will be used in conjunction with Lemma \ref{lem:quadratic_form_to_one_norm} to prove Lemma \ref{lem:smoothness_upperbound}.

For this result, we need to introduce some new notation and some basic inequalities. Define the random matrix $C \in \Real{d_1 \times d_2}$ as follows:
\begin{align}\label{eq:def_C}
    C = \frac{1}{m} \sum_{k = 1}^m e_{i_k}\Tilde{e}_{j_k}^T
\end{align}
where $(i_k)_{k \in [m]}$ are sampled i.i.d. uniformly at random from $[n_1]$ and $(j_k)_{k \in [m]}$ are sampled i.i.d. uniformly at random from $[n_2]$, independent of $(i_k)_{k \in [m]}$. Let $C_i \in \Real{n_2}$ denote the $i\textsuperscript{th}$ row of $C$, but expressed as a column vector. Then
\begin{align}
    C_i = \frac{1}{m} \sum_{k = 1}^m \mathbf{1}_{i_k = i}\Tilde{e}_{j_k}
\end{align}
It follows that 
\begin{align}
    \norm{C_i}_1 = \frac{1}{m} \sum_{k = 1}^m \mathbf{1}_{i_k = i}
\end{align}
Note that $\norm{C_i}_1$ is the empirical mean of $m$ i.i.d. Bernoulli random variables of mean $1/n_1$. Thus, we can bound it from above by the Chernoff bound.
\begin{lemma}[Chernoff bound]\label{lem:chernoff_bound}
    Suppose $x_1, x_2, \ldots, x_m$ are i.i.d. Bernoulli random variables with parameter $p$ and let $\epsilon > 0$ be given. Then:
    \begin{align*}
        P\left(\frac{1}{m}\sum_{k = 1}^m x_k \geq p + \epsilon \right) \leq \exp\left(-\frac{m\epsilon^2}{2p(1-p)}\right)
    \end{align*}
\end{lemma}
Using Lemma \ref{lem:chernoff_bound} with $p = \epsilon = 1/n_1$, we get that for any $i \in [n_1]$, 
\begin{align*}
    P\left(\norm{C_i}_1 \geq \frac{2}{n_1}\right) \leq \exp\left(-\frac{m}{2n_1}\right)
\end{align*}
Using the union bound, it follows that 
\begin{align*}
    P\left(\max_{i \in [n_1]}\norm{C_i}_1 \geq \frac{2}{n_1}\right) \leq n_1\exp\left(-\frac{m}{2n_1}\right)
\end{align*}
Finally, by \eqref{eq:operator_onenorm_calculation}, we know that 
\begin{align*}
    \norm{C^T}_1 = \max_{i \in [n_1]}\norm{C_i}_1
\end{align*}
In conclusion, 
\begin{align}\label{eq:CT_bound}
    P\left(\norm{C^T}_1 \geq \frac{2}{n_1}\right) \leq n_1\exp\left(-\frac{m}{2n_1}\right)
\end{align}
Since $n_1$ and $n_2$ are arbitrary in the above analysis, one can use the same logic to show that
\begin{align}\label{eq:C_bound}
    P\left(\norm{C}_1 \geq \frac{2}{n_2}\right) \leq n_2\exp\left(-\frac{m}{2n_2}\right)
\end{align}

\begin{lemma}\label{lem:bound_on_one_norm}
    Suppose the number of samples $m$ is at least {$2n\log(4n/\delta)$}. Then, with probability at least $1-\delta$,
    \begin{align*}
        \max\{\norm{B_\Dataset}_1, \norm{B_\Dataset^T}_1\} \leq \frac{4}{\min\{n_1, n_2\}}
    \end{align*}
\end{lemma}
\begin{proof}
    Define the following two matrices
    \begin{align*}
        B_{\Dataset}^1 = \frac{1}{m} \sum_{(u; i, j) \in \mathcal{D}} e_{u}\Tilde{e}_{i}^T \, ; \quad B_{\Dataset}^2 = \frac{1}{m} \sum_{(u; i, j) \in \mathcal{D}} e_{u}\Tilde{e}_{j}^T
     \end{align*}
     Both $B_{\Dataset}^1$ and $B_{\Dataset}^1$ are statistically identical to the random matrix $C$ defined in \eqref{eq:def_C}. By \eqref{eq:C_bound}, we have that if $m \geq 2n_2\log(4n_2/\delta)$,
     \begin{align}\label{eq:B_bound_simple}
         P\left(\norm{B_{\Dataset}^1}_1 \geq \frac{2}{n_2}\right) &\leq \frac{\delta}{4} ,  \qquad P\left(\norm{B_{\Dataset}^2}_1 \geq \frac{2}{n_2}\right) \leq \frac{\delta}{4} \\
     \end{align}
     By construction, 
     $B_{\Dataset} = B_{\Dataset}^1 + B_{\Dataset}^2$. By the triangle inequality, we get $\norm{B_{\Dataset}}_1 \leq \norm{B_{\Dataset}^1}_1 + \norm{B_{\Dataset}^2}_1$. Therefore, 
     \begin{align}\label{eq:B_bound_union}
         \norm{B_{\Dataset}}_1 \geq \frac{4}{n_2} \Rightarrow \ \norm{B_{\Dataset}^1}_1 + \norm{B_{\Dataset}^2}_1 \geq \frac{4}{n_2} \ \Rightarrow \ \norm{B_{\Dataset}^1}_1 \geq \frac{2}{n_2} \text{ or } \norm{B_{\Dataset}^2}_1 \geq \frac{2}{n_2}.
     \end{align}
     Put together, we get that if $m \geq 2n_2\log(4n_2/\delta)$,
     \begin{align*}
         P\left(\norm{B_{\Dataset}}_1 \geq \frac{4}{n_2}\right) &\leq P\left(\norm{B_{\Dataset}^1}_1 \geq \frac{2}{n_2} \text{ or } \norm{B_{\Dataset}^2}_1 \geq \frac{2}{n_2}\right) \quad (\text{by \eqref{eq:B_bound_union}})\\
         &\leq \mathbb{P}\left(\norm{B_{\Dataset}^1}_1 \geq \frac{2}{n_2}\right) +  \mathbb{P}\left(\norm{B_{\Dataset}^2}_1 \geq \frac{2}{n_2}\right) \\
         &\leq \frac{\delta}{2}.  \quad (\text{by \eqref{eq:B_bound_simple}})
     \end{align*}     
     By a similar argument, we can show that if $m \geq 2n_1\log(4n_1/\delta)$,
     \begin{align*}
         P\left(\norm{B_{\Dataset}^T}_1 \geq \frac{4}{n_1}\right) \leq \frac{\delta}{2}
     \end{align*}
     Finally, note that 
     \begin{align*}
         \norm{B_{\Dataset}}_1 \leq \frac{4}{n_2} \text{ and } \norm{B_{\Dataset}^T}_1 \leq \frac{4}{n_1} &\Rightarrow 
        \max\{\norm{B_\Dataset}_1, \norm{B_\Dataset^T}_1\} \leq \frac{4}{\min\{n_1, n_2\}} \\
        \therefore \norm{B_\Dataset^T}_1 \geq \frac{4}{\min\{n_1, n_2\}} &\Rightarrow \norm{B_{\Dataset}}_1 \geq \frac{4}{n_2} \text{ or } \norm{B_{\Dataset}^T}_1 \geq \frac{4}{n_1} 
     \end{align*}
     Invoking the union bound once again, we get that if $m \geq 2n\log(4n/\delta)$,
     \begin{align*}
        P\left(\norm{B_\Dataset^T}_1 \geq \frac{4}{\min\{n_1, n_2\}}\right) &\leq P\left(\norm{B_{\Dataset}}_1 \geq \frac{4}{n_2} \text{ or } \norm{B_{\Dataset}^T}_1 \geq \frac{4}{n_1} \right) \\
        &\leq P\left(\norm{B_{\Dataset}}_1 \geq \frac{4}{n_2}\right) + P\left(\norm{B_{\Dataset}^T}_1 \geq \frac{4}{n_1} \right) \\
        &\leq \delta
    \end{align*}
\end{proof}





\subsection{Proof of Lemma \ref{lem:convexity_upperbound}}


\lSCUB*
\begin{proof}
The proof follows from the following facts:
\begin{itemize}
    \item $\Dataset(\Delta\Delta^T) \leq 2\left(\gamma\norm{\Delta}_F^2 + 2\norm{B_{\Dataset} - \Bar{B}}_{2}\norm{\Delta}_{2,\infty}^2\right)\norm{\Delta}_F^2$, by Lemma \ref{lem:deltadelta_intermediate}.
    \item $\norm{\Delta}_F^2 \leq \epsilon \sigma^{*}_r \ \forall \ Z \in \mathcal{B}(\epsilon)$.
    \item $\norm{\Delta}_{2,\infty}^2 \leq 52\mu r \sigma_1^{*}/n \ \forall \ Z \in \overline{\Incoherentset}$. This can be derived as follows. 
    $$\norm{\Delta}_{2,\infty}^2 = \norm{Z - \solset }_{2,\infty}^2 \leq 2\left(\norm{Z}_{2,\infty}^2 + \norm{\solset }_{2,\infty}^2\right) \leq 2\left(\frac{12\mu\norm{Z^*}_F^2}{n} + \frac{\mu\norm{Z^*}_F^2}{n}\right) \leq \frac{52\mu r \sigma_1^*}{n},$$
    where the last step follows from the fact that $\norm{Z^*}_F^2 \leq 2r\sigma^*_1$.
    \item The number of samples is at least $5\left(13 \mu r \kappa/\epsilon\right)^2 n\log\left(n/\delta\right)$ $(845 = 5 \cdot 13^2)$.  By Lemma \ref{lem:BD_concentration}, with probability at least $1-\delta$,
    $$\norm{B_{\Dataset} - \Bar{B}}_{2} \leq \frac{\epsilon}{13\mu r \kappa}\frac{1}{\min\{n_1, n_2\}}$$  
\end{itemize}
Combining these inequalities, we get that with probability at least $1-\delta$, $\forall \ Z \in \mathcal{B} \cap \overline{\Incoherentset}$,
\begin{align*}
    \Dataset(\Delta\Delta^T) &\leq 2\left(\gamma\norm{\Delta}_F^2 + 2\norm{B_{\Dataset} - \Bar{B}}_{2}\norm{\Delta}_{2,\infty}^2\right)\norm{\Delta}_F^2 \\
    &\leq 2\left(\epsilon\gamma\sigma^*_r + 2\frac{\epsilon}{13\mu r \kappa}\frac{1}{ \min\{n_1, n_2\}} \frac{52 \mu r \sigma^*_1}{n} \right)\norm{\Delta}_F^2 \\
    &\leq 10 \epsilon \gamma \sigma^*_r\norm{\Delta}_F^2
\end{align*}
The last step is reasoned as follows:
\begin{align*}
    \frac{2}{n\min\{n_1, n_2\}} = \frac{2}{(n_1 + n_2)\min\{n_1, n_2\}} \leq \frac{2}{\max\{n_1, n_2\}\min\{n_1, n_2\}} = \frac{2}{n_1n_2} \leq \frac{2}{n_1(n_2-1)} = \gamma
\end{align*}
\end{proof}


\subsection{Proof of Lemma \ref{lem:smoothness_upperbound}}
The proof of Lemma \ref{lem:smoothness_upperbound} depends on Lemmas \ref{lem:quadratic_form_to_one_norm} and \ref{lem:bound_on_one_norm}.
\lSUB*
\begin{proof}
By Lemma \ref{lem:quadratic_form_to_one_norm}, we have that for any matrices $W, Z \in \mathbb{R}^{n \times r}$,
    \begin{align*}
        \mathcal{D}(WZ^T) \leq 4(\max\{\norm{B_\Dataset}_1, \norm{B_\Dataset^T}_1\}) \norm{Z}_{2, \infty}^2\norm{W}_F^2,
    \end{align*}
By Lemma \ref{lem:bound_on_one_norm} and the assumption on the number of samples we have made, we get that with probability at least $1-\delta$,
    \begin{align*}
        \max\{\norm{B_\Dataset}_1, \norm{B_\Dataset^T}_1\} \leq \frac{4}{\min\{n_1, n_2\}}
    \end{align*}
Putting these inequalities together, we get that with probability at least $1-\delta$, for any matrices $W, Z \in \mathbb{R}^{n \times r}$,
    \begin{align*}
        \mathcal{D}(WZ^T) \leq \frac{16}{\min\{n_1, n_2\}} \norm{Z}_{2, \infty}^2\norm{W}_F^2,
    \end{align*}
For the first of the bounds we wish to prove, we replace $W$ by $\Delta$ and $Z$ by $\solset$. We know that
\begin{align*}
    &\norm{\solset}_{2, \infty}^2 = \frac{\mu}{n}\norm{Z^*}_F^2 \leq \frac{2\mu r \sigma^*_1}{n} \quad ( \because \norm{Z^*}_F^2 \leq 2r\sigma^*_1) \\
    \Rightarrow &\mathcal{D}(\Delta\solset^T) \leq \frac{16}{\min\{n_1, n_2\}} \frac{2\mu r \sigma^*_1}{n}\norm{\Delta}_F^2 \leq 16\gamma(\mu r \sigma^*_1) \norm{\Delta}_F^2
\end{align*}
Here, as in the proof of Lemma \ref{lem:convexity_upperbound}, we use the fact that $2/(n\min\{n_1, n_2\}) \leq \gamma$. The second and third bounds can be derived in a similar fashion.
For the second bound, we use the bound that we established in the proof of Lemma \ref{lem:convexity_upperbound}.
$$\norm{\Delta}_{2, \infty}^2 \leq \frac{52 \mu r \sigma^*_1}{n}$$
Finally, for the third bound, we use the fact that $Z \in \overline{\Incoherentset}$ (see Lemma \ref{lem:incoherence_of_projection}) to get the bound
$$\norm{\solset}_{2, \infty}^2 \leq 12 \frac{\mu}{n}\norm{Z^*}_F^2 \leq \frac{24 \mu r \sigma^*_1}{n}$$
\end{proof}

\section{Experiments: Planning outperforms Heuristics}
\label{sec:experiment}

We begin our empirical demonstrations by showcasing the effectiveness of our planning framework on both synthetic and real datasets. We focus on the simplest planning algorithm, 1-step lookaheads (Algorithm~\ref{alg:complete}), and show that even basic planning can hold great promise. 
We illustrate our framework using two uncertainty quantification modules---GPs and 
\ensembles/ \ensembleplus. 

Throughout this section, we focus on evaluating the mean squared error of 
a regression model $\model$,  and develop adaptive policies that minimize uncertainty on $g(f)$ defined in~\eqref{eqn:l2-g-f}.
When GPs provide a valid model of uncertainty, 
our experiments show that our planning framework significantly outperforms other baselines. 
We further demonstrate that our conceptual framework extends to deep learning-based uncertainty quantification methods such as  \ensembleplus while highlighting computational challenges that need to be resolved in order to scale our ideas. 
For simplicity, we assume a naive predictor, i.e., $\psi(\cdot) \equiv 0$. However, we emphasize that this problem is just as complex as if we were using a sophisticated model $\psi(.)$. The performance gap between the algorithms 
primarily depends
on the level  of uncertainty in our prior beliefs.

To evaluate the performance of our algorithm, we benchmark it against several baselines. 
%Active learning baselines use an acquisition function $\ac$ to select points that have the highest   function value: $X\opt_t \in \argmax_{X \in \xpoolj{t}} \ac({X})$ at every step $t$. These methods may also need an UQ module, which we simply use the same UQ module as in our algorithm, and it  outputs $V(X)$ that measures the the uncertainty of each point $X \in \xpoolj{t}$.
Our first set of baselines are from active learning~\citep{AggarwalKoGuHaPh14}:
\\ % \noindent\textbf{Active Learning Heuristics:} 
\textbf{(1)} 
\textsf{Uncertainty Sampling (Static):}  In this approach, we query the samples for which the model is least certain about. Specifically, we estimate the variance of the latent output $f(X)$ for each $X \in \xpool$ using the UQ module and select the top-$K$ points with the highest uncertainty. \\
\textbf{(2)} \textsf{Uncertainty Sampling (Sequential):} This is a greedy heuristic that sequentially selects the points with the highest uncertainty within a batch, while updating the posterior beliefs using pseudo labels from the current posterior state. Unlike \textsf{Uncertainty Sampling (Static)}, this method takes into account the information gained from each point within batch, and hence tries to diversify the selected points within a batch. 

 
We also compare our approach to the  \textbf{(3)} \textsf{Random Sampling}, which selects each batch uniformly at random from the pool. Additionally, we compare solving the planning problem using  \textsf{REINFORCE}-based policy gradients with   $\mathsf{Smoothed\text{-}Autodiff}$ policy gradients.\footnote{Our code repository is available at
  \url{https://github.com/namkoong-lab/adaptive-labeling}.}
%Detailed experimental setups are provided in Section \ref{sec:details-experiments}.

%We repeat all experiments with 10 random seeds.




\begin{figure}[t]
\centering
\begin{minipage}[b]{0.49\textwidth}
\centering
\includegraphics[width=\textwidth, height=5cm]{figures/original_scale/Var_of_l_2_loss.pdf}
\caption{(Synthetic data) Variance of mean squared loss evaluated through the posterior belief $\mu_t$ at each horizon $t$. This is the objective that policy gradient methods like \textsf{REINFORCE} and $\ouralgo$ optimizes. 1-step lookaheads are surprisingly effective even in long horizons.}
\label{fig:var-l2-sim}
\end{minipage}
\hfill
\begin{minipage}[b]{0.49\textwidth}
\centering \includegraphics[width=\textwidth, height=5cm]{figures/original_scale/Error_of_estimated_model_l_2_loss.pdf}
\caption{(Synthetic data) Error between MSE calculated based on collected data $\mc{D}^{0:T}$ vs. population oracle MSE over $\mc{D}_{\rm eval} \sim P_X$. Reducing uncertainty over posteriors directly leads to better OOD evaluations. 1-step lookaheads significantly outperform active learning heuristics in small horizons.}
\label{fig:mean-l2-sim}
\end{minipage}
%\caption{Simulated data for GPs}
%\label{fig:both_plots}
\end{figure}

\subsection{Planning with Gaussian processes}
\label{sec:experiment-plan-GP}
We now briefly describe the data generation process for the GP experiments,  deferring a more detailed discussion of the dataset generation to Section~\ref{sec:details-experiments}. 
We use both the synthetic data and the real data to test our methodology.
For the \emph{simulated data},  we construct a setting where the general population is distributed across \emph{51 non-overlapping clusters} while the initial labeled data $\dtrain$ just comes from one cluster. In contrast, both $\dpool \defeq (\xpool,\ypool),\deval \defeq (\xeval,\yeval)$ are generated   from all the clusters. 
We begin with a low-dimensional scenario, generating a one-dimensional regression setting using a GP. %Gaussian Process (GP).
Although the data-generating process is not known to the algorithms,  we assume that the GP hyperparameters are known to all the algorithms
to ensure fair comparisons. This can be viewed as a setting where our prior is well-specified, allowing us to isolate the effects
of different policy optimization approaches
 without any concerns about the misspecified priors. We select $10$ batches, each of size $K=5$ across $T = 10$ time horizons.

To examine the robustness of our method against the distributional assumptions made  in the simulated case, we then move to a real dataset where the correct prior is not known. We simulate selection bias from the eICU dataset~\citep{PollardJoRaCeMaBa18}, which contains real-world patient data with in-hospital mortality outcomes. 
We conduct a $k$-means clustering to generate 51 clusters and then select data from those clusters. We view this to be a credible replication of practice, as severe distribution shifts are common due to selection bias in clinical labels.  To convert the binary mortality labels into a regression setting, we train a  random forest classifier and fit a GP on predicted scores, which serves as the UQ module for all the algorithms. As before, the task is to select 10 batches, each consisting of 5 samples, across 10 time horizons.

 In Figures~\ref{fig:var-l2-sim} and~\ref{fig:mean-l2-sim}, we present results for the simulated data. 
Figure~\ref{fig:var-l2-sim} shows the variance of $\ell_2$ loss, and Figure~\ref{fig:mean-l2-sim} presents the error in the estimated $\ell_2$ loss using $\mu_t$ (relative to true $\ell_2$ loss, that is unknown to the algorithm). 
As we can see from these plots, our method one-step lookahead  gives substantial improvements  over active learning baselines and random sampling. In addition,
compared to the one-step lookahead planning approach using \textsf{REINFORCE}-based policy gradients, 
we observe that $\mathsf{Smoothed\text{-}Autodiff}$-based policy gradients provide significantly more robust performance over all horizons.

In Figures~\ref{fig:var-l2-real}~and~\ref{fig:mean-l2-real}, we observe similar findings on the eICU data. We see that planning policies (\textsf{REINFORCE} and $\mathsf{Smoothed\text{-}Autodiff}$) consistently outperform other heuristics by a large margin.  Active learning baselines perform poorly in these small-horizon batched problems and can sometimes be even worse than the random search baselines.  Overall, our results show the importance of careful planning in adaptive labeling for reliable model evaluation. 

We offer some intuition as to why one-step lookahead planning may outperform other heuristic algorithms. 
 First,  \textsf{Uncertainty sampling (Static)} while myopically selects the
 top-$K$ inputs with the highest uncertainty, it fails to consider 
the overlap in information content among the ``best” instances; see \citep{AggarwalKoGuHaPh14} for more details. 
In other words,  it might acquire points from the same region with high uncertainty while failing to induce diversity among the batch.
Although \textsf{Uncertainty Sampling (Sequential)} somewhat addresses the issue of information overlap, a significant drawback of 
this algorithm
is the disconnect between the objective we aim to optimize and the algorithm. For example, it might sample from a region with high uncertainty but very low density. 

\begin{figure}[t]
\centering
\begin{minipage}[b]{0.48\textwidth}
\centering
\includegraphics[width=\textwidth, height=5cm]{figures/original_scale/Var_of_l_2_loss_real.pdf}
\caption{(Real-world eICU data) Variance of mean squared loss evaluated through the posterior belief $\mu_t$ at each horizon $t$. Even 1-step lookaheads are extremely effective planners, and auto-differentiation-based pathwise policy gradients provide a reliable optimization algorithm based on low-variance gradient estimates.}
\label{fig:var-l2-real}
\end{minipage}
\hfill
\begin{minipage}[b]{0.48\textwidth}
\centering \includegraphics[width=\textwidth, height=5cm]{figures/original_scale/Error_of_estimated_model_l_2_loss_real.pdf}
\caption{(Real-world eICU data) Error between MSE calculated based on collected data $\mc{D}^{0:T}$ vs. population oracle MSE over $\mc{D}_{\rm eval} \sim P_X$. Reducing uncertainty over posteriors directly leads to better OOD evaluations. Our method significantly outperforms active learning-based heuristics, and random sampling.}
\label{fig:mean-l2-real}
\end{minipage}
%\caption{Real data for GPs}
\end{figure}
 
%\vspace{-1.5cm}
% \begin{wrapfigure}{r}{.32\columnwidth}
%   \vspace{-.5cm} 
%   \centering
% \includegraphics[scale=.29]{figures/Var of l2l_2 loss.pdf}
%   \vspace{-0.2cm}
%   \caption{Results of GP}
% \label{fig:var-l2-gp}
%   \vspace{-0.1cm}
% \end{wrapfigure}


% Attempts have been made  in the past to address these  drawbacks heuristically  (see \citep{AggarwalKoGuHaPh14}). We give a unified computational framework while approaching the problem in a more principled manner and solving it more optimally.




\subsection{Planning with  neural network-based uncertainty quantification methods ($\ensembleplus$)}


We now provide a proof-of-concept that shows the generalizability of our conceptual framework  to the deep learning-based UQ modules, specifically focusing on $\ensembleplus$ due to their previously observed superior performance~\citep{OsbandWenAsDwIbLuRo23}. Recall that implementing our framework with deep learning-based UQ modules  requires us to retrain the model across multiple possible random actions $\bm{a}(\theta)$ sampled from the current policy $\pi_\theta$.
This requires significant computational resources, in sharp contrast to the GPs where the posteriors are in closed form and can be readily updated and differentiated. 

Due to the computational constraints, we test $\ensembleplus$ on a toy setting to demonstrate the generalizability of our framework. We consider a setting where the general population consists of four clusters, while the initial labeled data only comes from one cluster. Again we generate data using GPs.  The task is to select a batch of 2 points in one horizon. We detail the $\ensembleplus$ architecture in Section \ref{sec:details-experiments}, and we assume prior uncertainty to be large (depends on the scaling of the prior generating functions). 
The results are summarized in the Table~\ref{tab:UQ_ensemble}.

% \begin{table}[H]
% \vspace{-10pt}
% \caption{Performance under \ensembleplus as UQ module}
%     \centering
%     \begin{tabular}{|m{3cm}|m{2.5cm}|m{2cm}|} 
%     \hline
%       Algorithm   & Variance of $\loss_2$ loss estimate & Error of $\loss_2$ loss estimate  \\ \hline Random Sampling 
%          & $1710.9 \pm 1352.1$ & $8.67\pm6.62$ 
%       \\ \hline \ouralgo & $1.30 \pm 0.68$ & $0.91\pm0.25$ \\ \hline
%     \end{tabular}
%     \label{tab:UQ_ensemble}
%     %\vspace{-10pt}
% \end{table}




\begin{table}[h]
\vspace{-10pt}
\caption{Performance under \ensembleplus as the UQ module}
\centering
\begin{tabular}{|l|l|l|}
\hline
Algorithm   & Variance of $\loss_2$ loss estimate & Error of $\loss_2$ loss estimate  \\
\hline
\textsf{Random sampling} & 7129.8 $\pm$ 1027.0 & 136.2 $\pm$ 8.28 \\ \hline
\textsf{Uncertainty sampling (Static)} & 10852 $\pm$ 0.0 & 162.156 $\pm$ 0.0 \\ \hline
\textsf{Uncertainty sampling (Sequential)} & 8585.5 $\pm$ 898.9 & 144 $\pm$ 6.93 \\ \hline
\textsf{REINFORCE} & 1697.1 $\pm$ 0.0 & 45.27 $\pm$ 0.0 \\ \hline
\ouralgo & 1697.1 $\pm$ 0.0 & 45.27 $\pm$ 0.0 \\ \hline
\end{tabular}
%\caption{Comparison of different algorithms based on variance   and   error in $\ell_2$ loss estimation with Ensemble $+$ as the UQ module. Our results demonstrate that {\ouralgo} and REINFORCE outperformthe other active learning based heuristics, confirming the benefits of our MDP formulation for the adaptive labeling problem, as also demonstrated in Section 4.\\
%\footnotesize{Experimental details: We use Gaussian Processes as our data generating process, GP parameters are the same as in Section D.3.  The task is to select a batch of 2 points along one horizon.The marginal distribution $p_X$ has 4 \textit{non-overlapping} clusters. Initial data comes from one cluster, while pool and evaluation points comes from all the clusters. We have $20$ initial labeled data points, $10$ pool points, and $252$ evaluation points.  Training procedures are similar to the one in Section D.3.} }
\label{tab:UQ_ensemble}
\end{table}



% We faced  issues in scaling up these experiments which will be our focus in the future. 





% \begin{itemize}
%     \item Posteriors should be consistent. Two dimensions: even with less training,  
%     \item the inference should be  fast enough
% \end{itemize}


% Potential research directions for uncertainty quantification

% In this section we consider a simple setting We consider a simpler setting and 


% For synthetic dataset generation, we use ...... For real datasets, we use ...... We compare our methodolgy to several baselines ()    This Section is structured as follows:
% \begin{itemize}
%     \item \textbf{GPs, square loss objective} (Section \ref{}): 
%     %the broad aim of the experiments  in this section is to isolate the performance of our methodology without any concerns for the inefficiencies induced due to a mis-specified prior or imperfect posterior inference. To accomplish this we generate synthetic datasets using GPs (detailed later). We use the well specified prior (GPs - with same hyperparameter setting) as our UQ module.   
%      As GPs provide differentaible posterior inference - any errors induced due to imperfect posterior updates are also isolated. We note that under this setting
%      \item In Section\ref{} we demonstrate why our methodology performs better than other baselines - by devising various synthetic experiments ()
%     \item  \textbf{UQ Benchmarking }(Section \ref{}): Before diving into the experiments using $\ensembleplus$ and ENNs,  we showcase our benchmarking experiments in Section \ref{}. We use real datasets We observe that ENNs perform better
%      \item \textbf{Ensemble $+$}, objective: recall, accuracy
%     \item \textbf{ENN}, objective: recall, accuracy
% \end{itemize}




% In Section {}, we test 
% \subsection{Experimental details}

% \begin{itemize}
%     \item UQ methodologies - GPs, ENNs
%     \item Objectives - Recall,  ATE
%     \item Datasets - ATE-synthetic datasets, Recall-synthetic, real datasets
%     \item Baselines - 
%     \begin{itemize}
%         \item Random sampling
%         \item Active learning - Uncertainty based sampling - In regression setting almost all of the 
%         \item Myopic greedy - Greedy Batch based sampling
%         \item Policy Gradient
%     \end{itemize}
    
% \end{itemize}

% \subsection{Experiments}
%     \begin{itemize}
%     \item GPs with square loss
%     \item Benchmarking ENN
%         \item ENNs with ATE
%         \item ENNs with Recall
%     \end{itemize}

% \subsection{Benefits over other algorithms - intuition and experiments}

%Active learning - Myopic greedy / Don't rely on the objective rather some entropy version.


%%% Local Variables:
%%% mode: latex
%%% TeX-master: "main"
%%% End:

This work identifies signal collapse as a critical bottleneck in one-shot neural network pruning. Performance loss in pruned networks is due to \textbf{signal collapse} in addition to the removal of critical parameters. We propose \textbf{REFLOW} (\textbf{Re}storing \textbf{F}low of \textbf{Low}-variance signals), a simple yet effective method that mitigates signal collapse without computationally expensive weight updates. By focusing on signal preservation, REFLOW highlights the importance of mitigating signal collapse in sparse networks and enables magnitude pruning to match or surpass state-of-the-art one-shot pruning methods such as CHITA, CBS, and WF.

REFLOW consistently achieves state-of-the-art accuracy across diverse architectures, restoring ResNeXt-101 from under 4.1\% to 78.9\% top-1 accuracy at 80\% sparsity on ImageNet. Its lightweight design makes it a practical solution for both research and deployment, delivering high-quality sparse models without the overhead of traditional approaches. These findings challenge the traditional emphasis on weight selection strategies and underscore the critical role of signal propagation for achieving high-quality sparse networks in the context of one-shot pruning.




\section*{Acknowledgements}
The authors would like to thank Laurent Baratchart and Sylvain Chevillard for their helpful discussion on rational approximations of the complex exponential. We also thank Sajad Movahedi and Felix Sarnthein for their helpful comments on this manuscript.
This work has received support from the French government, managed by the National Research Agency, under the France 2030 program with the reference "PR[AI]RIE-PSAI" (ANR-23-IACL-0008). 
Antonio Orvieto is supported by the Hector Foundation.


\bibliography{main.bib}

\newpage
\setcounter{section}{0}

\begin{appendices}

In this Appendix, we provide a detailed proof for all our theoretical results. We start in Appendix~\ref{RNN basics} with an equivalence of various representations of linear RNNs, then in Appendix~\ref{review} with a review of fundamentals of signal processing.
\listofappendices

\counterwithin{figure}{section}
\counterwithin{table}{section}


\newpage


\section{Recurrent Neural Networks and Diagonal forms}

\label{RNN basics}

 

Linear recurrent networks such as SSMs, in their simplest form, are causal models acting on a $d$ dimensional input sequence with $L$ elements $U\in\mathbb{R}^{d\times L}$, producing an output sequence $Y\in\mathbb{R}^{d\times L}$ through a filtering process parametrized by variables $A\in\mathbb{R}^{N\times N}$, $B\in\mathbb{R}^{N\times d}, P\in\mathbb{R}^{d\times N}$. Let $U_n\in\mathbb{R}^{d}$ denote the $n$-th timestamp data contained in $U$, a linear RNN processes the inputs as follows~\citep{gu2022parameterization,orvieto2023resurrecting}
\begin{equation}
    X_{n} = A X_{n-1} + B U_n,\qquad Y_{n} = PX_n.
    \label{eq:appendix linear_RNN}
\end{equation}

\begin{proposition}[Linear RNNs and convolution form]
    Let $A\in\mathbb{R}^{S\times S}$ such that $A$ is diagonal, $B\in\mathbb{R}^{S\times 1}, P\in\mathbb{R}^{1\times S}$, and $u = (u_n)_{n\in\mathbb{Z}}$ be a univariate input signal. The output signal $(y_n)_{n\in\mathbb{Z}}$ can write 
    \[
    y_n = \sum_{k=0}^\infty c_ku_{n-k}
    \]
    with $c_k = \sum_{s=1}^Sa_s^kb_s$.
\end{proposition}

\begin{proof}
    We have $A = \begin{pmatrix}
        a_1 & \dots & \\
        & \ddots & \\
        & \dots & a_S
    \end{pmatrix}, B = \begin{pmatrix}
        b_1\\
        \vdots\\
        b_S
    \end{pmatrix}, P = \begin{pmatrix}
        p_1 & \dots & p_S
    \end{pmatrix}$.

\begin{align*}
    X_n &= AX_{n-1} + Bu_n\\
    &= A(AX_{n-2} + Bu_{n-1}) + Bu_n  = \dots = \sum_{k=0}^nA^kBu_{n-k}\\
    &= \sum_{k=0}^n\begin{pmatrix}
        a_1^k & \dots & \\
        & \ddots & \\
        & \dots & a_S^k
    \end{pmatrix}\begin{pmatrix}
        b_1\\\vdots\\b_S
    \end{pmatrix}u_{n-k} = \sum_{k=0}^n\begin{pmatrix}
        a_1^kb_1\\\vdots \\a_S^kb_s
    \end{pmatrix}u_{n-k}.
\end{align*}
Finally,
\begin{align*}
    y_n = \begin{pmatrix}
        p_1 & \dots & p_S
    \end{pmatrix}X_n = \sum_{k=0}^n\begin{pmatrix}
        p_1 & \dots & p_S
    \end{pmatrix}\begin{pmatrix}
        a_1^kb_1\\\vdots\\a_N^kb_N
    \end{pmatrix}u_{n-k} & = \sum_{s=1}^S\sum_{k=0}^np_sa_s^kb_su_{n-k}\\
    &=\sum_{k=0}^nu_{n-k}\sum_{s=1}^Sa_s^kb_sp_s = \sum_{k=0}^nu_{n-k}c_k,
\end{align*}
with $c_k = \sum_{s=1}^Sa_s^kb_sp_s$. In this paper, we consider without loss of generality $\begin{pmatrix}
p_1 \dots p_s
\end{pmatrix} = \begin{pmatrix}
    1 \dots 1
\end{pmatrix}.$
\end{proof}
\section{Some fundamentals of signal processing}\label{section fundamentals} 
\label{review}

In this section, we will recall some fundamentals definitions and results in signal processing. We will only look at discrete-time signals. Throughout this section, we denote $(x_n)_{n\in\mathbb{Z}}$ or $x_n$ a discrete time signal, and $x_k$ the value taken by the signal at time $k$. For example, let us denote $(e_n)$ the impulse signal such that 
\begin{equation}
e_n =
\begin{cases}
    1, n = 0\\
    0, n\neq 0.
\end{cases}  
\label{appendix impulse signal}
\end{equation}
This signal is  useful because the response of a system to a impulse signal gives a lot of insights. In particular it fully describes a linear time-invariant system. For more on signal processing, we refer the reader to \cite{oppenheim1996signals}.

\subsection{Linear Time-invariant systems}

A system is said to be \textit{time-invariant} if its response to a certain input signal does not depend on time. It is said to be \textit{linear} if its output response to a linear combinations of inputs is the same linear combinations of the output responses of the individual inputs. A system is said to be \textit{causal} if the output at a present time depends on the input up the present time only. 

There exist several ways to represent the input-output behavior of LTI system. We will only look at the impulse response representation (convolution). 



\begin{proposition}[Convolution]
    Let $h_n$ be the impulse response of an LTI system $H$ (i.e., the output of system $H$ subject to input $e_n$), and $x_n$ be an input signal. In this case, the output signal of the system $y_n$ writes 
    \begin{equation}
        y_n = \sum_{k=-\infty}^{+\infty}x_kh_{n-k}.
        \label{appendix conv LTI}
    \end{equation}
\end{proposition}


\textit{Causal systems.} The output $y_n$ of a causal system depends only on past or present values of the input. This forces $h_k=0$ for $k<0$ and the convolution sum is rewritten 
\[
y_n = \sum_{k=0}^{+\infty}h_kx_{n-k}.
\]

\textit{Stable systems.} A system is stable if the output is guaranteed to be bounded for every bounded input. 



\subsection{Discrete-Time Fourier Transform}

In this section, we denote $x_n$ a complex-valued discrete-time signal.

\begin{definition}
    The discrete-time Fourier transform of signal $x_n$ is given by
    \[
    X(\omega) = \sum_{n=-\infty}^{+\infty}x_ne^{-i\omega n}.
    \]
    This function takes values in the frequency space.
    The inverse discrete-time Fourier transform is given by 
    \[
    x_n = \frac{1}{2\pi}\int_0^{2\pi}X(\omega)e^{i\omega n}d\omega.
    \]
\end{definition}

The Discrete-Time Fourier transform presents some notable properties that we recall in Table~\ref{table:dtft-properties}.

\begin{table}[h!]
\centering
\renewcommand{\arraystretch}{1.5}
\begin{tabular}{|c|c|}
\hline
\textbf{Property} & \textbf{Relation} \\ \hline
Time Shifting & 
$x_{n-k} \overset{DTFT}{\longleftrightarrow} e^{-i\omega k} X(\omega)$ \\ \hline
Convolution in Time & 
$x_n * y_n \overset{DTFT}{\longleftrightarrow} X(\omega) Y(\omega)$ \\ \hline
Frequency Differentiation & 
$j \frac{d}{d\omega} X(\omega) \overset{DTFT}{\longleftrightarrow} -n x_n$ \\ \hline
Differencing in Time & 
$x_n - x{n-1} \overset{DTFT}{\longleftrightarrow} \left(1 - e^{-i\omega}\right) X(\omega)$ \\ \hline
\end{tabular}
\caption{Properties of the Discrete-Time Fourier Transform (DTFT). For each property, assume $x_n\overset{DTFT}{\longleftrightarrow} X(\omega)$ and $y_n\overset{DTFT}{\longleftrightarrow} Y(\omega)$.}
\label{table:dtft-properties}
\end{table}

We recall Parseval's theorem that establishes a fundamental equivalence between the inner product of two signals in the time domain and their corresponding representation in the frequency domain.

\begin{theorem}[Parseval]
    For two complex-valued discrete-time signals \((x_n)\) and \((y_n)\) with discrete-time Fourier transforms \(X(e^{i\omega})\) and \(Y(e^{i\omega})\), Parseval's theorem yields:
    \begin{equation}
        \sum_{n=-\infty}^{+\infty}x_n\overline{y_n} = \frac{1}{2\pi}\int_0^{2\pi}X(e^{i\omega})\overline{Y(e^{i\omega})}d\omega.
        \label{Parseval thm}
    \end{equation}
In particular, Parseval's theorem yields an energy conservation result:
    $$
        \sum_{n=-\infty}^{+\infty}\vert x_n\vert^2 = \frac{1}{2\pi}\int_0^{2\pi}\vert X(e^{i\omega})\vert^2 d\omega.
    $$
\end{theorem}
The following proposition will be useful in our lower bound proof in Appendix~\ref{appendix subsection white noise loss}.
\begin{proposition}\label{proposition semi parseval}
    Let $w_n$ be a causal discrete-time complex-valued signal with Fourier transform $W(\omega)$. We have the following equality:
    \[
    \sum_{L=0}^{+\infty}L\vert w_l\vert^2 = \frac{i}{2\pi}\int_0^{2\pi}\frac{dW(\omega)}{d\omega}\overline{W}(\omega)d\omega.
    \]
\end{proposition}

\begin{proof}
    By definition of the DTFT, $W(\omega) = \sum_{L=0}^{+\infty}w_Le^{-i\omega L}$. Therefore, 
    \begin{align*}
        \sum_{L=0}^{+\infty}L\vert w_L\vert^2 &= \sum_{L=0}^{+\infty}Lw_L\bar{w}_L = \frac{1}{2\pi}\sum_{L=0}^{+\infty}\sum_{L'=0}^{+\infty}Lw_L\bar{w}_{L'}\int_0^{2\pi}e^{-i\omega(L-L')}d\omega\\
        &= \frac{i}{2\pi}\int_0^{2\pi}\sum_{L=0}^{+\infty}-iL\omega_Le^{-iL\omega}\sum_{L'=0}^{+\infty}\bar{w}_{L'}e^{iL'\omega}d\omega.
    \end{align*}
    Provided that the sequence $(Lw_L)_{L\geq 0}$ is summable, $\frac{dW(\omega)}{d\omega}=\sum_{L=0}^{+\infty}-iLw_Le^{-i\omega L}$, which proves the result.
\end{proof}

\subsection{Fourier series}\label{appendix subsection Fourier series}
We recall basics of Fourier Series. For more about Fourier series and their applications, we refer the reader to \cite{serov2017fourier}.

\begin{definition}[Fourier series]
    Let $f: \mathbb{R}\rightarrow \mathbb{R}$ be a piecewise continuous and $2\pi$-periodic function. The Fourier series of $f$ is the series of functions 
    \[
    S(f) = \sum_{n=-\infty}^{+\infty}
c_n(f)e^{int},  
\]
where $c_n(f)$ are the Fourier coefficients of $f$, such that 
\[
c_n(f) = \frac{1}{2\pi}\int_{-\pi}^\pi f(t)e^{-int}dt.
\]
The partial sums of these series write
\[
S_n(f)(t) = \sum_{k=-n}^nc_k(f)e^{ikt}
\]
\end{definition}

\begin{theorem}[Dirichlet]
    Let $f$ be piecewise $\mathcal{C}^1$ and $2\pi$-periodic. Therefore, for every $x\in\mathbb{R}$, $S_n(f)(x)$ converges to 
    \[
    \frac{f(x+0) + f(x-0)}{2},
    \]
    where $f(x+0)$ (resp. $f(x-0)$) denotes the right-hand (resp. left-hand) limit of $f$ at $x$.
\end{theorem}

\paragraph{Remark:}If the function \( f \) is not \( 2\pi \)-periodic, its graph on the interval \([0, 2\pi]\) can be extended periodically over \(\mathbb{R}\). In this case, Dirichlet's theorem is applicable at potential discontinuities at \( 0 \) and \( 2\pi \).

\subsection{A natural pair for autocorrelation}\label{appendix subsection natural pair}

A natural parametrization is to represent autocorrelation with $\gamma(k) = \rho^{\vert k\vert}$ with $\vert\rho\vert < 1$, as done in the main paper. This models exponentially decreasing autocorrelation between data. The natural associated time-frequency pair to represent is 
\[
(\gamma(k), \Gamma(e^{i\omega})) = (\rho^{\vert k\vert}, \frac{1-\rho^2}{\vert 1-\rho e^{-i\omega}\vert^2})
.\]
Indeed, as $\vert\rho\vert<1$, the sequence \((\rho^{\vert k\vert}e^{ik\omega})_{k\in\mathbb{Z}}\) is summable, \(\gamma\) admits a Fourier transform that we denote~$\Gamma$. For $\omega\in\mathbb{R}$.
\begin{align*}
    \Gamma(e^{i\omega}) &= \sum_{k=-\infty}^{+\infty}\rho^{\vert k\vert}e^{-i\omega k} =\sum_{k=1}^{+\infty}\rho^k e^{i\omega k} + \sum_{k=0}^{+\infty}\rho^ke^{-i\omega k}\\
    &= \frac{1}{1-\rho e^{i\omega k}} -1 + \frac{1}{1-\rho e^{-i\omega k}} =\frac{1-\rho^2}{\vert 1-\rho e^{-i\omega}\vert^2}.
\end{align*}

\begin{figure}[h]
    \centering
    \includegraphics[width=1\linewidth]{img/spectral_power_density.pdf}
    \caption{\textit{The autocorrelation factor $\rho$ determines the width of the spectral power density $\Gamma(e^{i\omega})$. The larger $\rho$, the narrower the spectral power density. This means that increasing $\rho$ in $\mathcal{L}_\text{freq}(c, d)$ narrows the bandwidth over which we evaluate the difference $\vert C(e^{i\omega}) - D(e^{i\omega})\vert^2$, leading to improved performance.}}
    \label{figure spectral power density}
\end{figure}

\section{Lower bound}


In this section, we provide the proofs of the two lower bounds.

\subsection{White noise case (Theorem~\ref{lower bound white noise})}\label{appendix subsection white noise loss}
We start by the representation of our loss function as a quadratic form.
\begin{proposition} \label{prop white noise loss}
    In the white noise case, the correlation factor $\rho$ is null. The loss $\mathcal{L}_\text{time}(c, d)$ writes 
    \[
    \mathcal{L}_\text{time}(c, d)= 1 + \sum_{k=0}^{+\infty}\vert c_k\vert^2 - 2\textnormal{Re}\big(\sum_{k=0}^{+\infty}c_kd_k\big),
    \]
    where $c_k=\sum_{s=1}^Sa_s^kb_s$. Therefore, the loss writes 
    \[
    \mathcal{L}_\text{time}(c, d) = 1 + \sum_{s, s'}^S\frac{b_s\bar{b}_{s'}}{1-a_s\bar{a}_{s'}} - 2\textnormal{Re}\big(\sum_{s=1}^Sb_sa_s^K\big).
    \]
\end{proposition}

\begin{proof}
    On the one hand, 
    \begin{align*}
        \sum_{k=0}^{+\infty}\vert c_k\vert^2 &= \sum_{k=0}^{+\infty}\big\vert\sum_{s=1}^Sa_s^kb_s\big\vert 
        =\sum_{k=0}^{+\infty}\sum_{s=1}^S\sum_{s'=1}^Sa_s^k\bar{a}_{s'}^kb_sb_{s'} \\
        &= \sum_{s=1}^S\sum_{s'=1}^Sb_sb_{s'}\sum_{k=0}^{+\infty}a_s^k\bar{a}_{s'}^k =\sum_{s=1}^S\sum_{s'=1}^Sb_sb_{s'}\frac{1}{1-a_s\bar{a}_{s'}}.
    \end{align*}
    On the other hand,
    \begin{align*}
        \textnormal{Re}\big(\sum_{k=0}^{+\infty}c_kd_k\big) &= c_Kd_K
        =\sum_{s=1}^Sb_sa_s^K.
    \end{align*}
    Hence the result.
\end{proof}

\begin{proposition}[Performance criterion]
    Minimizing the loss in Proposition \ref{prop white noise loss} boils down to maximizing the following performance criterion
    \[
    F_K = \sum_{s,s'=1}^S\bar{a}_s^K(C^{-1})_{ss'}a_{s'}^K,
    \]
    where $C_{ss'} = \frac{1}{1-a_s\bar{a}_{s'}}$.
\end{proposition}
\begin{proof}
    The loss $\mathcal{L}_\text{time}$ writes 
    \[
    1 + \langle\bar{b}, C\bar{b}\rangle - \langle\bar{b}, a^K\rangle -\langle a^K, \bar{b}\rangle.
    \]
    We thus want to maximize with respect to $a_s$ and $b_s$ the quantity
    \[
    \langle\bar{b}, a^K\rangle +\langle a^K, \bar{b}\rangle - \langle\bar{b}, C\bar{b}\rangle.
    \]
    This is convex and quadratic with respect to $b$, and the minimizer $\bar{b}^*$ is $C^{-1}a^K$, leading to the performance criterion
    \[
    F_K = \langle a^K, C^{-1}a^K\rangle = \sum_{s, s'=1}^S\bar{a}_s^K(C^{-1})_{ss'}a_{s'}^K.
    \]
\end{proof}

We can now move to the proof of Theorem \ref{lower bound white noise}, by first analyzing properties of the matrix $C$.

\paragraph{Linear algebra preview.} We use the similarities with Cauchy matrices and their so-called displacement structure~\citep{yang2003generalized,calvetti1996solution}.

Starting from
$$
C - \Diag( {a}) C \Diag(\bar{a})  = 1_S 1_S^\top,
$$
we get by  post multiplying by $\Diag(\bar{a})^{-1}$,
$$
C\Diag(\bar{a})^{-1}  - \Diag( {a}) C  = 1_S 1_S^\top\Diag(\bar{a})^{-1}
$$
and thus, by pre and post multiplying by $C^{-1}$:
$$
\Diag(\bar{a})^{-1}C^{-1}  - C^{-1}\Diag( {a})    = C^{-1}1_S 1_S^\top\Diag(\bar{a})^{-1}C^{-1},
$$
leading to
$$
\Diag(\bar{a})^{-1}C^{-1}\Diag( {a}) ^{-1}   - C^{-1} = C^{-1}1_S 1_S^\top\Diag(\bar{a})^{-1}C^{-1}\Diag( {a})^{-1}
= u v^*,
$$
with $ u = C^{-1} 1_S$ and $v = \Diag( \bar{a})^{-1}  C^{-1} \Diag(a)^{-1} 1_S$. This leads to a closed form expression for the inverse:
$$
(C^{-1})_{ss'} \big( \frac{1}{ \bar{a}_s a_{s'}}-1 \big) = u_s \bar{v}_{s'}. 
$$
We get
$$
v-u = \big[ \Diag( \bar{a})^{-1}  C^{-1} \Diag(a)^{-1} - C^{-1} \big] 1_S
= u v^\ast 1_S = u 1_S^\top\Diag(\bar{a})^{-1}C^{-1}\Diag( {a})^{-1}1_S,
$$
which leads to $ \ds v  = u ( 1 +1_S^\top\Diag(\bar{a})^{-1}C^{-1}\Diag( {a})^{-1}1_S )$. Moreover we can write
\BEAS
1_S^\top\Diag(\bar{a})^{-1}C^{-1}\Diag( {a})^{-1}1_S 
& = & 1_S^\top ( C^{-1} + u v^\ast) 1_S=  1_S^\top ( C^{-1} + C^{-1} 1_S v^\ast) 1_S\\
& = & 1_S^\top C^{-1} 1_S \cdot (1 + v^\ast 1_S )
\\
& = &  1_S^\top C^{-1} 1_S \cdot (1 + 1_S^\top\Diag(\bar{a})^{-1}C^{-1}\Diag( {a})^{-1}1_S ) , \EEAS
which leads to
$1_S^\top\Diag(\bar{a})^{-1}C^{-1}\Diag( {a})^{-1}1_S   = \frac{ 1_S^\top C^{-1} 1_S }{1- 1_S^\top C^{-1} 1_S }$, and thus
 $$ 1 +1_S^\top\Diag(\bar{a})^{-1}C^{-1}\Diag( {a})^{-1}1_S  = \frac{1}{1-1_S^\top C^{-1} 1_S} =
\frac{1}{1 - u^\top 1_S}.$$
Moreover, we have for any $z \in \mathbb{C}$, if all $a_s$ are distinct:
$$
\sum_{s'=1}^S \frac{ u_{s'}}{1 - z \bar{a}_{s'}} = 1 - \prod_{s'=1}^S \bar{a}_{s'} \prod_{s'=1}^S \frac{ a_{s'} - z}{1 - z \bar{a}_{s'}}
$$
(the two rational functions have the same degrees, the same poles and are equal for $z=a_1,\dots,a_S$),
which leads to for $z=0$,
$$
\sum_{s'=1}^S  { u_{s'}}=1_S^\top C^{-1} 1_S = \sum_{s'=1}^S   u_{s'} = 1 - \prod_{s'=1}^S | {a}_{s'}|^2,
$$
and thus $\ds 1 - 1_S^\top C^{-1} 1 = \prod_{s'=1}^S | {a}_{s'}|^2$.

We have, if $|z|=1$,
$$
\Big|\prod_{s'=1}^S \frac{ a_{s'} - z}{1 - z \bar{a}_{s'}}\Big|
= 1,
$$
which will be used in the bound (such expressions are typically referred to as Blaschke products~\citep{baratchart2016minimax}, and are known to have unit magnitude).


\paragraph{Proof of the lower bound (by upper bounding $F_K$).}
We have, using our linear algebra preview,
$$
F_K = \langle a^K, C^{-1} a^K \rangle
= \sum_{s,s'=1}^S \bar{a}_s^K (C^{-1})_{ss'} a_{s'}^K
= \sum_{s,s'=1}^S (\bar{a}_s  a_{s'})^{K+1} \frac{u_s \bar{v}_{s'} }{1 -  \bar{a}_s a_{s'}}.
$$
We get, using our linear algebra results,
$$
F_K - F_{K+1} = 
\sum_{s,s'=1}^S (\bar{a}_s  a_{s'})^{K+1} (1 -  \bar{a}_s a_{s'})\frac{u_s \bar{v}_{s'} }{1 -  \bar{a}_s a_{s'}}
=  \frac{1}{\prod_{s'=1}^S | {a}_{s'}|^2} \Big| \sum_{s=1}^S \bar{a}_s^{K+1}  u_s \Big|^2.
$$
This leads to 
\BEAS
F_K & = &  \sum_{L=K}^{+\infty}
( F_L - F_{L+1}) = 
\sum_{L=K+1}^{+\infty} \Big| \sum_{s=1}^S \bar{a}_s^L  u_s \Big|^2   \frac{1}{\prod_{s'=1}^S | {a}_{s'}|^2} 
.\EEAS
We have:
\BEAS
\sum_{L=K+1}^{+\infty} \Big| \sum_{s=1}^S \bar{a}_s^L  u_s \Big|^2 
& \leqslant &  \frac{1}{K+1} 
\sum_{L=0}^{+\infty} L \Big| \sum_{s'=1}^S \bar{a}_{s'}^L  u_{s'} \Big|^2 \mbox{ since } 1_{L \geqslant K+1} \leqslant \frac{L}{K+1}.
 \EEAS
 We consider the sequence $\ds w_L =  \sum_{s=1}^S \bar{a}_s^L  u_s$, with Fourier series
 $$
 W(\omega) = \sum_{L=0}^{+\infty} w_L e^{-i \omega L} 
 =  \sum_{s=1}^S \frac {u_s}{1-\bar{a}_s e^{-i \omega }}   =
  1 - \prod_{s'=1}^S \bar{a}_{s'} \prod_{s'=1}^S \frac{ a_{s'} - e^{-i\omega}}{1 -  e^{-i\omega} \bar{a}_{s'}}.
 $$
 We then use Proposition \ref{proposition semi parseval} to write:
 $$
 \sum_{L = 0 }^{+\infty}
 L | w_L|^2 =   \frac{i}{2\pi} \int_0^{2\pi} W'(\omega) \overline{W(\omega)}d\omega
, $$
 leading to
 \BEAS
&&\sum_{L=K+1}^{+\infty} \Big| \sum_{s=1}^S \bar{a}_s^L  u_s \Big|^2\\
& \leqslant &  \frac{1}{K+1} 
  \frac{i}{2\pi   }  \int_0^{2\pi} 
\frac{d}{d\omega} \Big[    - \prod_{s=1}^S \bar{a}_{s} \prod_{s=1}^S \frac{ a_{s} - e^{i\omega}}{1 - e^{i\omega} \bar{a}_{s}}
 \Big]
\overline{ \Big(
 1 - \prod_{s'=1}^S \bar{a}_{s'} \prod_{s'=1}^S \frac{ a_{s'} - e^{i\omega}}{1 - e^{i\omega} \bar{a}_{s'}}
 \Big)}
      d\omega
\\
& = &  \frac{1}{K+1} 
  \frac{i}{2\pi   }  \int_0^{2\pi} 
\frac{d}{d\omega} \Big[      \prod_{s=1}^S \bar{a}_{s} \prod_{s=1}^S \frac{ a_{s} - e^{i\omega}}{1 - e^{i\omega} \bar{a}_{s}}
 \Big]
\overline{ \Big(
    \prod_{s'=1}^S \bar{a}_{s'} \prod_{s'=1}^S \frac{ a_{s'} - e^{i\omega}}{1 - e^{i\omega} \bar{a}_{s'}}
 \Big)}
      d\omega.
\\
      \EEAS
      We now have, by taking derivatives of the product:
     \BEAS
     \frac{d}{d\omega} \Big[       \prod_{s=1}^S \frac{ a_{s} - e^{i\omega}}{1 - e^{i\omega} \bar{a}_{s}}
 \Big] & = & 
  \prod_{s=1}^S \frac{ a_{s} - e^{i\omega}}{1 - e^{i\omega} \bar{a}_{s}}
  \sum_{s=1}^S \frac{1 - e^{i\omega} \bar{a}_{s}}{ a_{s} - e^{i\omega}} 
   \frac{d}{d\omega} \Big[      \frac{ a_{s} - e^{i\omega}}{1 - e^{i\omega} \bar{a}_{s}}\Big]
\\
 & = & 
  \prod_{s=1}^S \frac{ a_{s} - e^{i\omega}}{1 - e^{i\omega} \bar{a}_{s}}
  \sum_{s=1}^S \frac{1 - e^{i\omega} \bar{a}_{s}}{ a_{s} - e^{i\omega}} 
   \frac{d}{d\omega} \Big[     \frac{1}{\bar{a}_s} +   \frac{a_s - \frac{1}{\bar{a}_s}}{1 - e^{i\omega} \bar{a}_{s}}\Big]
\\
 & = & 
  \prod_{s=1}^S \frac{ a_{s} - e^{i\omega}}{1 - e^{i\omega} \bar{a}_{s}}
  \sum_{s=1}^S \frac{1 - e^{i\omega} \bar{a}_{s}}{ a_{s} - e^{i\omega}} 
  \Big[
(1-|a_s|^2) \frac{ -i e^{i \omega}}{ (1 - e^{i\omega} \bar{a}_{s})^2}\Big]
\\
 & = & 
  \prod_{s=1}^S \frac{ a_{s} - e^{i\omega}}{1 - e^{i\omega} \bar{a}_{s}}
  \sum_{s=1}^S (1-|a_s|^2) \frac{-i  }{ |e^{-i\omega}  - \bar{a}_{s}|^2}
. \EEAS 
      This leads to, using the unit magnitude of $\frac{ a_{s} - e^{i\omega}}{1 - e^{i\omega} \bar{a}_{s}}$,
\BEAS
      F_K
      & \leqslant &  \frac{1}{K+1} 
  \frac{1}{2\pi   } 
   \sum_{s=1}^S ( 1- |a_s|^2) 
  \int_0^{2\pi}  \frac{1}{|a_s - e^{i\omega}|^2}
      d\omega =  \frac{S}{K+1},
\EEAS
using an explicit integration $\ds \frac{1}{2\pi   } 
    \int_0^{2\pi}  \frac{1}{|a_s - e^{i\omega}|^2}
      d\omega = \frac{1}{1-|a_s|^2}$.
    
 The approximation error $\mathcal{L}_\text{time}(c, d)$ is  thus 
$
1 - F_K$, 
which leads to the desired result.

\subsection{Autocorrelated case (Theorem~\ref{theorem 
autocorrelated lower bound})}
\label{proof auto}
We follow the same proof technique as for Theorem~\ref{lower bound white noise}, and compute first an explicit expression of the loss, this time, by introducing a new $a_s$, equal to $\rho$, with the introduction of new weights $w_s = b_s a_s / ( a_s - \rho)$ for $s \in \{1,\dots,S\}$, the weight $w_{S+1}$ being determined by the linear constraint.
\begin{lemma}
    In the autocorrelated case ($\rho \neq 0$), $\mathcal{L}_\text{time}(c, d)$ as in Eq.~\eqref{correlated time domain loss} writes 
    \begin{equation}
    1 - 2(1-\rho^2)\textnormal{Re}\big(\sum_{s=1}^{S+1}\frac{w_sa_s^k}{1-a_s\rho}\big) + (1-\rho^2)\sum_{s, s'}^{S+1}\frac{w_s\bar{w}_{s'}}{1-a_s\bar{a}_{s'}},
    \label{appendix constrained autocorrelated loss}
    \end{equation}    
    where $a_{S+1}=\rho$ and the constraint $\sum_{s=1}^{S+1}w_sa_s^{-1}=0$ holds. 
    \end{lemma}
\begin{proof}
   We aim to minimize 
    \[
    \sum_{k, k'}(c_k-d_k)(c_{k'}-d_{k'})\gamma(k-k'),
    \]
    where $\gamma(k-k')=\rho^{\vert k-k'\vert}$.
    Denoting $C(e^{i\omega}), D(e^{i\omega})$ and $\Gamma(e^{i\omega})$ the Fourier transforms of $(c_n), (d_n)$ and $(\gamma_n)$ respectively, Parseval's theorem yields 
    \[
    \sum_{k, k'}(c_k-d_k)(c_{k'}-d_{k'})\gamma(k-k') = \frac{1}{2\pi}\int_{-\pi}^\pi\big\vert C(e^{i\omega}) - D(e^{i\omega})\big\vert^2\Gamma(e^{i\omega})d\omega.
    \]

    We have $D(e^{i\omega})=e^{-iK\omega}$ (Fourier transform of a shifted Dirac at timestep K), and 
    \begin{align*}
        C(e^{i\omega}) &= \sum_{k=0}^{+\infty}\sum_{s=1}^Sb_sa_s^ke^{-i\omega k} = \sum_{s=1}^S\frac{b_s}{1-a_se^{-i\omega}},\\
        \Gamma(e^{i\omega})&=\sum_{k=-\infty}^{+\infty}\gamma(k)e^{-i\omega k} = \frac{1}{1 - \rho e^{-i\omega}}\frac{1-\rho^2}{1 - \rho e^{i\omega}}.
    \end{align*}
    The criterion becomes (with an error of $1$ if $C=0$):
\BEAS
&&\frac{1}{2\pi} \int_0^{2\pi} | D(e^{i\omega}) - C(e^{i\omega})|^2 \Gamma(e^{i\omega}) d\omega\\
& = & 
\frac{ 1-\rho^2}{2\pi} \int_0^{2\pi} \Big| D(e^{i\omega})\frac{1}{1 - \rho e^{-i\omega}} - C(e^{i\omega}) \frac{1}{1 - \rho e^{-i\omega}}\Big|^2  d\omega \\
& = & 1 - 
\frac{1-\rho^2}{2\pi} 2 {\textnormal{ Re}} \Big(\int_0^{2\pi} 
\overline{D(e^{i\omega})\frac{1}{1 - \rho e^{-i\omega}}}
C(e^{i\omega}) \frac{1}{1 - \rho e^{-i\omega}}
\Big) d\omega   \\
& & \hspace*{2cm} + \frac{1-\rho^2}{2\pi}  \int_0^{2\pi} \Big|C(e^{i\omega}) \frac{1}{1 - \rho e^{-i\omega}}\Big|^2  d\omega.
\EEAS 
We have 
$$
\frac{1}{1- a_se^{-i\omega}}\frac{1}{1 - \rho e^{-i\omega}}
= \frac{1}{a_s-\rho} \Big( \frac{a_s}{1 - a_s e^{-i\omega}} - \frac{\rho}{1 - \rho e^{-i\omega}} \Big),
$$
and thus
\BEAS
C(e^{i\omega}) \frac{1}{1 - \rho e^{-i\omega}}  & = &  \sum_{s=1}^{S}  
\frac{b_s}{a_s-\rho} \Big( \frac{a_s}{1 - a_s e^{-i\omega}} - \frac{\rho}{1 - \rho e^{-i\omega}} \Big) \\
 & = & \sum_{s=1}^{S+1} \frac{w_s  }{1- a_se ^{-i\omega}},
\EEAS
with $w_s = b_s a_s / ( a_s - \rho)$, $a_{S+1} = \rho$, and the constraint $\ds \sum_{s=1}^{S+1} w_sa_s^{-1}  = 0$.
The criterion becomes
\BEAS
& & 1 - 
(1-\rho^2)\sum_{s=1}^{S+1} 2 {\textnormal{Re}} \Big( \frac{w_s a_s^K}{1 - a_s \rho}
 \Big)   +(1-\rho^2) \sum_{s,s'=1}^{S+1} \frac{\bar{w}_s w_{s'}  }{1- a_s \bar{a}_s'},
\EEAS 
after straightforward computations.
\end{proof}

\paragraph{Proof of Theorem \ref{theorem autocorrelated lower bound}.}

The minimum with respect to $w$ in Eq.~\eqref{appendix constrained autocorrelated loss} with the constraint is greater than the unconstrained minimizer, equal to
\BEAS
H_K & =  & 1 - (1-\rho^2)\sum_{s,s'=1}^{S+1} 
\frac{ \bar{a}_s^K}{1 - \bar{a}_s \rho}\frac{ a_{s'}^K}{1 - a_{s'} \rho} (C^{-1})_{ss'},
\EEAS
where we recall that $C_{ss'} = \frac{1}{1-a_s\bar{a}_{s'}}$.

Using linear algebra properties from above with $S+1$ zeros and poles, we get
\BEAS
H_K&= & 1 -  (1-\rho^2)\sum_{s,s'=1}^{S+1}
\frac{ 1}{1 - \bar{a}_s \rho}\frac{1}{1 - a_{s'} \rho}  \frac{(\bar{a}_s a_{s'})^{K+1}u_s \bar{v}_{s'}}{1 - \bar{a}_s a_{s'}} 
\\
& = & 1 - (1-\rho^2)\frac{1}{\prod_{s=1}^{S+1} |a_s|^2 }
\sum_{s,s'=1}^{S+1} 
\frac{ 1}{1 - \bar{a}_s \rho}\frac{1}{1 - a_{s'} \rho}  \frac{(\bar{a}_s a_{s'})^{K+1}u_s \bar{u}_{s'}}{1 - \bar{a}_s a_{s'}} ,
\EEAS
where we recall that $u = C^{-1}1_S$ and $v = \text{Diag}(\bar{a})^{-1}C^{-1}\text{Diag}(a)^{-1}1_S$.

We have
\BEAS
H_{K+1} - H_K 
& = & \frac{1-\rho^2}{\prod_{s=1}^{S+1} |a_s|^2 }
\sum_{s,s'=1}^{S+1} 
\frac{ 1}{1 - \bar{a}_s \rho}\frac{1}{1 - a_{s'} \rho}   (\bar{a}_s a_{s'})^{K+1}u_s \bar{u}_{s'} 
\\
& = & \frac{1-\rho^2}{\prod_{s=1}^{S+1} |a_s|^2 }
\Big| \sum_{s=1}^{S+1}
\frac{ 1}{1 - \bar{a}_s \rho}   \bar{a}_s  ^{K+1}u_s  
\Big|^2,
 \EEAS
 leading to
 \BEAS
 H_K & = & \sum_{L=K}^{+\infty} ( H_L - H_{L+1} ) + 1 \\
 & = & 1 - \frac{1-\rho^2}{\prod_{s=1}^{S+1} |a_s|^2 }
 \sum_{L = K}^{+\infty}
 \Big| \sum_{s=1}^{S+1} 
\frac{ 1}{1 - \bar{a}_s \rho}   \bar{a}_s  ^{L} \bar{a}_su_s  
\Big|^2 \\
& \geqslant & 
1 - \frac{1}{K} \frac{1-\rho^2}{\prod_{s=1}^{S+1} |a_s|^2 }
 \sum_{L = 0 }^{+\infty}
 L\Big| \sum_{s=1}^{S+1}
\frac{ 1}{1 - \bar{a}_s \rho}   \bar{a}_s  ^{L} \bar{a}_s u_s  
\Big|^2 ,
 \EEAS
 using $1_{L \geqslant K} \leqslant \frac{L}{K}$.
 
 The sequence $\ds w_L = \sum_{s=1}^{S+1} 
\frac{ 1}{1 - \bar{a}_s \rho}   \bar{a}_s  ^{L} \bar{a}_su_s  $, has Fourier series
\BEAS
W(\omega) & = &  \sum_{L=0}^{+\infty} w_L e^{-i\omega L}
= \sum_{L=0}^{+\infty}   e^{-i\omega L}\sum_{s=1}^{S+1} 
\frac{ 1}{1 - \bar{a}_s \rho}   \bar{a}_s  ^{L}\bar{a}_s u_s \\
& = & \sum_{s=1}^{S+1}
\frac{ 1}{1 - \bar{a}_s \rho}   \frac{\bar{a}_s u_s}{1 - \bar{a}_s e^{-i\omega}} 
= \sum_{s=1}^{S+1} u_s \Big( 
\frac{ 1}{1 - \bar{a}_s \rho}   - \frac{1}{1 - \bar{a}_s e^{-i\omega}} \Big)\frac{1}{  \rho - e^{-i\omega}} \\
& = & 
\frac{1}{  \rho - e^{-i\omega}} \Big( \prod_{s=1}^{S+1} \bar{a}_s\Big) \Big( \prod_{s=1}^{S+1} \frac{a_s - e^{-i\omega}}{1-e^{-i\omega} \bar{a}_s}
-\prod_{s=1}^{S+1} \frac{a_s - \rho }{1- \rho \bar{a}_s}
\Big)
\\
 & = & 
\frac{1}{  \rho - e^{-i\omega}} \Big(\prod_{s=1}^{S+1} \bar{a}_s \Big) \prod_{s=1}^{S+1} \frac{a_s - e^{-i\omega}}{1-e^{-i\omega} \bar{a}_s},
 \EEAS
 because of the link between $u,C$ and rational functions.
 
We have:
\BEAS
 1 - H_K
 & \leqslant & 
 \frac{1-\rho^2}{K} \frac{1}{\prod_{s=1}^{S+1} |a_s|^2 }
 \sum_{L = 0 }^{+\infty}
 L | w_L|^2 \\
 & = & \frac{1-\rho^2}{K} \frac{1}{\prod_{s=1}^{S+1} |a_s|^2 }
\frac{i}{2\pi} \int_0^{2\pi} W'(\omega) \overline{W(\omega)}d\omega
\ \mbox{ using properties of Fourier Series,} \\
 & = & \frac{1-\rho^2}{K}  
\frac{i}{2\pi} \int_0^{2\pi} \frac{d}{d\omega} \Big(
\frac{1}{  \rho - e^{-i\omega}}   \prod_{s=1}^{S+1} \frac{a_s - e^{-i\omega}}{1-e^{-i\omega} \bar{a}_s}
\Big)
\overline{\frac{1}{  \rho - e^{-i\omega}}   \prod_{s=1}^{S+1} \frac{a_s - e^{-i\omega}}{1-e^{-i\omega} \bar{a}_s}}d\omega
\\
 & = & \frac{1-\rho^2}{K}  
\frac{i}{2\pi} \int_0^{2\pi}  
\frac{-i e^{-i\omega}}{  (\rho - e^{-i\omega})^2}   \prod_{s=1}^{S+1} \frac{a_s - e^{-i\omega}}{1-e^{-i\omega} \bar{a}_s}
\overline{\frac{1}{  \rho - e^{-i\omega}}   \prod_{s=1}^{S+1} \frac{a_s - e^{-i\omega}}{1-e^{-i\omega} \bar{a}_s}}d\omega
\\
& & \hspace*{1cm} + 
\frac{1-\rho^2}{K}  
\frac{i}{2\pi} \int_0^{2\pi} 
\frac{1}{  \rho - e^{-i\omega}} \frac{d}{d\omega} \Big(   \prod_{s=1}^{S+1} \frac{a_s - e^{-i\omega}}{1-e^{-i\omega} \bar{a}_s}
\Big)
\overline{\frac{1}{  \rho - e^{-i\omega}}   \prod_{s=1}^{S+1} \frac{a_s - e^{-i\omega}}{1-e^{-i\omega} \bar{a}_s}}d\omega.
\EEAS
Using the following identities,
\BEAS
 \frac{a_s - e^{-i\omega}}{1-e^{-i\omega} \bar{a}_s}
& = & \frac{1}{\bar{a}_s} + \frac{a_s - 1/ \bar{a}_s}{1-e^{-i\omega} \bar{a}_s}, \\
\frac{d}{d\omega} \Big( \frac{a_s - e^{-i\omega}}{1-e^{-i\omega} \bar{a}_s}
\Big) & = & \frac{a_s - 1/ \bar{a}_s}{(1-e^{-i\omega} \bar{a}_s)^2} \bar{a}_s (-i e^{-i\omega})
=  i e^{-i\omega}\frac{1-|a_s|^2 }{(1-e^{-i\omega} \bar{a}_s)^2}, \\
\Big| \frac{a_s - e^{-i\omega}}{1-e^{-i\omega} \bar{a}_s}
\Big|  & = & 1, \EEAS
we get
\BEAS
 1 - H_K & \leqslant & \frac{1-\rho^2}{K}  
\frac{1}{2\pi} \int_0^{2\pi}  
\frac{e^{-i\omega}}{  (\rho - e^{-i\omega})^2 (\rho - e^{i\omega})}     d\omega
\\
& & \hspace*{4cm} +
\frac{1-\rho^2}{K}  
\sum_{s=1}^{S+1}  ( 1- |a_s|^2) 
\frac{1}{2\pi} \int_0^{2\pi} 
\frac{1}{  |\rho - e^{-i\omega}|^2}  
 \frac{1}{|a_s - e^{-i\omega}|^2} d\omega
\\
& = & \frac{1-\rho^2}{K}  
\frac{1}{2\pi} \int_0^{2\pi}  
\frac{e^{-i\omega}}{  (\rho - e^{-i\omega})^2 (\rho - e^{i\omega})}     d\omega
\\
& & \hspace*{-1cm}
+
\frac{(1-\rho^2)^2}{K}  
\frac{1}{2\pi} \int_0^{2\pi} 
\frac{1}{  |\rho - e^{-i\omega}|^4}  d\omega
 +
\frac{1-\rho^2}{K}  
\sum_{s=1}^{S}  ( 1- |a_s|^2) 
\frac{1}{2\pi} \int_0^{2\pi} 
\frac{1}{  |\rho - e^{-i\omega}|^2}  
 \frac{1}{|a_s - e^{-i\omega}|^2} d\omega
\\
& = & - \frac{1}{K} \frac{1}{1-\rho^2}   +
 \frac{1}{K} \frac{1+\rho^2}{1-\rho^2}   + 
\frac{1-\rho^2}{K}  
\sum_{s=1}^{S}  ( 1- |a_s|^2) 
\frac{1}{2\pi} \int_0^{2\pi} 
\frac{1}{  |\rho - e^{-i\omega}|^2}  
 \frac{1}{|a_s - e^{-i\omega}|^2} d\omega  
 \EEAS
 by exact integration. Then, using $\ds\frac{1}{  |\rho - e^{-i\omega}|^2}   \leqslant \frac{1}{(1-\rho)^2}$, 
 and $\ds\frac{1}{2\pi} \int_0^{2\pi} 
  \frac{1}{|a_s - e^{-i\omega}|^2} d\omega  = \frac{1}{1-|a_s|^2}$, we get
  \BEAS
1-H_K
& \leqslant &  
 \frac{1}{K} \frac{\rho^2}{1-\rho^2}   + 
\frac{1+\rho}{1-\rho}\frac{S}{K}  \leqslant  \frac{1}{K} \frac{\rho}{1-\rho}   + 
\frac{2}{1-\rho}\frac{S}{K}
= \frac{1}{K} \frac{1}{1-\rho} ( \rho + 2 S).
 \EEAS
Thus, we get an approximation error greater than
$\displaystyle
\Big( 1 - \frac{1}{K} \frac{3S}{1-\rho} \Big)_+.
$ (since it is always nonnegative).

\section{Proof of Theorem~\ref{theorem: upper bound} (Upper Bound of Algorithm~\ref{alg: main})}
\label{sec: appendix upper bound}
We break down the upper bound on the number of arm pulls used by Algorithm~\ref{alg: main} as follows. We bound the number of rounds required for a non-satisfying arm $k \not\in \A_{\epsilon}(\nu)$ to be eliminated in Lemma~\ref{lem: elim suboptimal}. Then in Lemma~\ref{lem: termination}, we bound the number of rounds each non-eliminated arm has gone through when the termination condition of the while-loop is triggered. Combining these lemmas with the number of arm pulls used by $\mathtt{QuantEst}$ for each round index $t \ge 1$ and active arm $k \in \A_t$ as stated in~\eqref{eq: QuantEst arm pulls}
gives us an upper bound on the total number of arm pulls.

We first present a useful lemma that will be used in the proofs of the two subsequent lemmas. 

\begin{lemma}[$\max \mathrm{LCB}$ is non-decreasing]
\label{lem: max LCB increasing}
    Under Event $E$ as defined in Lemma~\ref{lem: good events}, we have 
    \begin{equation}
    \max\limits_{a \in \mathcal{A}_{t}} 
            \mathrm{LCB}_{t}(a) \ge 
    \max\limits_{a \in \mathcal{A}_{\tau}} 
            \mathrm{LCB}_{\tau}(a).
    \end{equation}
    for all rounds $t > \tau  \ge 1$.
\end{lemma}
\begin{proof}
    Let round index $\tau \ge 1$ be arbitrary
    and let $k \in \argmax\limits_{a \in \A_{\tau}} 
            \mathrm{LCB}_{\tau}(a).$
    We have $k \in \A_{\tau+1}$ since 
    $\mathrm{UCB}_{\tau}(k) > \mathrm{LCB}_{\tau}(k) 
    = \max\limits_{a \in \A_{\tau}} 
            \mathrm{LCB}_{\tau}(a)$ by~\eqref{eq:  quantile anytime bound} 
    of the anytime quantile bounds.        
    It then follows that
    \begin{equation}
        \max\limits_{a \in \mathcal{A}_{\tau+1}} 
            \mathrm{LCB}_{\tau+1}(a) 
        \ge \mathrm{LCB}_{\tau+1}(k) 
        \ge \mathrm{LCB}_{\tau}(k) = 
        \max\limits_{a \in \A_{\tau}} 
            \mathrm{LCB}_{\tau}(j),
    \end{equation}    
    where the second inequality follows from~\eqref{eq:  quantile anytime bound} 
    of the anytime quantile bounds. 
    Applying the argument repeatedly yields the claim for all $t > \tau.$
\end{proof}

\begin{lemma}[Elimination of non-satisfying arms]
\label{lem: elim suboptimal}
     Fix an instance $\nu \in \cE$, and suppose Algorithm~\ref{alg: main} is run with input $(\A, \lambda, \epsilon, q, \delta)$ and parameter $c \ge 1$.
    Let $\A_{\epsilon} = \A_{\epsilon}(\nu) $ be as defined in~\eqref{def: performance def} and let the gap $\Delta_{k} = \Delta_{k}(\nu, \lambda, \epsilon, c, q)$ be as defined in Definition~\ref{def: our gap} 
    for each arm $k \in \A$.
    Consider an arm $k \not\in \A_{\epsilon}$.
    Under Event $E$ as defined in Lemma~\ref{lem: good events}, when the round index~$t$
    of Algorithm~\ref{alg: main} satisfies $\Delta^{(t)}  \le \frac{1}{2} \Delta_k$, we have  $k \not\in \A_{t+1}$.
\end{lemma}
\begin{proof}
    If $k \not\in \A_t$, then $k \not\in \A_{t+1}$ trivially. Therefore, we assume for the rest of the proof that $k \in \A_t$, and we will show that
    \begin{equation}
    \label{eq: eliminate condition}
        \mathrm{UCB}_t(k) \le \max\limits_{a \in \mathcal{A}_{t}} \mathrm{LCB}_t(a)
    \end{equation}
    or equivalently
    \begin{equation}
    \label{eq: eliminate condition equivalent}
        \mathrm{UCB}_t(k) < \max\limits_{a \in \mathcal{A}_{t}} \mathrm{LCB}_t(a) + \tilde{\epsilon},
    \end{equation}
    where $\tilde{\epsilon} = \tilde{\epsilon}(\lambda, \epsilon, c)$ is as defined in Lines~\ref{line: number of points} and~\ref{line: tilde epsilon} of Algorithm~\ref{alg: main}.
    Note that these conditions are equivalent because both
    $\mathrm{UCB}_t(k)$ and
    $\max\limits_{a \in \A_t } \mathrm{LCB}_t(a)$ 
    are elements of 
    \begin{equation}
        \left[ 0, 
        \tilde{\epsilon}, 
        2\tilde{\epsilon}, \cdots,
        (n-1) \tilde{\epsilon}, \lambda\right],
    \end{equation}
   which follows from Lines~\ref{line: list of points},~\ref{eq: initiate default conf interval}, and~\ref{ltk def}--\ref{UCB definition} of Algorithm~\ref{alg: main}.

    
    Since $k \not\in \A_{\epsilon}$, when the round index $t$ satisfies~$\Delta^{(t)} \le \frac{1}{2} \Delta_k $ we have
    \begin{equation}
    \label{eq: gap k realized with arm j}
        \mathrm{UCB}_t(k)
        < Q_k \big( q + \Delta^{(t)} \big)  + \tilde{\epsilon} 
        \le Q^+_{j}\big(q - \Delta^{(t)} \big) 
    \end{equation}
    for some arm $j \in \A$  by~\eqref{eq: upper approx quantile anytime bound} of the anytime quantile bounds
    and Definition~\ref{def: our gap}. 
    We now consider two cases: (i) $j \in \A_t$ and (ii) $j \not\in \A_t$.

    
    If $j \in \A_t$, we have
    \begin{equation}
    \label{eq: j in At}
        Q^+_{j}\big(q - \Delta^{(t)}\big) 
        \le \mathrm{LCB}_t(j) + \tilde{\epsilon} 
        \le \max\limits_{a \in \mathcal{A}_{t}} \mathrm{LCB}_t(a) + \tilde{\epsilon} 
    \end{equation}
    by~\eqref{eq: lower approx quantile anytime bound} of the anytime quantile bounds and the assumption that $j \in \A_t$. Combining~\eqref{eq: gap k realized with arm j} and~\eqref{eq: j in At} gives us condition~\eqref{eq: eliminate condition equivalent} as desired.
    
    If $j \not\in \A_t$, then it is eliminated at some round $\tau < t$, i.e., 
    $j \in \A_{\tau}$ but $j \not\in \A_{\tau + 1}$.
    By \eqref{eq: quantile anytime bound} of the anytime quantile bounds, the definition of active arm set (Line~\ref{line: active arm} of Algorithm~\ref{alg: main}) applied to $\A_{\tau + 1}$,
    and the fact that max LCB is non-decreasing (Lemma~\ref{lem: max LCB increasing}), 
    we have 
    \begin{equation}
    \label{eq: j not in At}
    Q_{j}(q) 
    \le
    \mathrm{UCB}_{\tau}(j) 
    \le
            \max\limits_{a \in \mathcal{A}_{\tau}} 
            \mathrm{LCB}_{\tau}(a)     
        \le
        \max\limits_{a \in \mathcal{A}_{t}} 
            \mathrm{LCB}_{t}(a). 
    \end{equation}
    Combining~\eqref{eq: gap k realized with arm j}, the trivial inequality 
    $Q^+_{j}\big(q - \Delta^{(t)}\big) 
        \le Q_{j}(q) $, and~\eqref{eq: j not in At} yields
    condition~\eqref{eq: eliminate condition} as desired.
\end{proof}

\begin{remark}
    \label{rem: elim suboptimal}
    As seen in the analysis for the case  $j \not\in \A_t$ above, the property that $\max \mathrm{LCB}$ is non-decreasing (Lemma~\ref{lem: max LCB increasing}) is crucial in establishing~\eqref{eq: j not in At}. We will see below that the same argument is used again in establishing~\eqref{eq: LCB_t(k) > Fj}. This property of Lemma~\ref{lem: max LCB increasing} itself is a consequence of ensuring $\mathrm{LCB}_t(k)$ is non-decreasing in $t$; see Remark~\ref{rem: LCB non decreasing}.
\end{remark}



\begin{lemma}[While-loop termination]
\label{lem: termination}
     Fix an instance $\nu \in \cE$, and suppose Algorithm~\ref{alg: main} is run with input $(\A, \lambda, \epsilon, q, \delta)$ and parameter $c \ge 1$.
    Let $\A_{\epsilon} = \A_{\epsilon}(\nu) $ be as defined in~\eqref{def: performance def} and let the gap $\Delta_{k} = \Delta_{k}(\nu, \lambda, \epsilon, c, q)$ be as defined in Definition~\ref{def: our gap} 
    for each arm $k \in \A$.
    Under Event $E$, when the round index~$t$
    of Algorithm~\ref{alg: main} satisfies $\Delta^{(t)} \le \frac{1}{2} \max \limits_{a \in \A_{\epsilon}} \Delta_a$, Algorithm~\ref{alg: main} will terminate in round $t+1$.
\end{lemma}

\begin{proof}
     If $\A_{t+1} = \{ k^* \}$, then
      \begin{equation}
          \max\limits_{a \in \A_{t+1} \setminus \{k^*\} }                 \mathrm{UCB}_t(a) - (c+1)\tilde{\epsilon}
      = -\infty \le \mathrm{LCB}_{t}(k^*),
      \end{equation}
     and so the algorithm will terminate and return arm $k^*$ in round $t+1$.
     Therefore, we assume for the rest of the proof that
     there exists another arm $a \ne k^*$ such that $a \in \A_{t+1}$. 

     We first show that the following condition is sufficient to trigger the termination condition of the while-loop (Lines~\ref{line: start while loop}--\ref{line: end while loop}) of Algorithm~\ref{alg: main}: There exists an arm $k \in \A_{t+1}$
     satisfying
    \begin{equation}
    \label{eq: suf cond trigger termination}
          \mathrm{LCB}_t(k)  
          \ge
          \max\limits_{a \in \A_{t+1} \setminus \{k\} }
        Q_{a}\big(q + \Delta^{(t)}\big) -  (c+1)\tilde{\epsilon} .
    \end{equation}
    Using~\eqref{eq: upper approx quantile anytime bound} of the anytime quantile bound, 
    condition~\eqref{eq: suf cond trigger termination} implies that
    \begin{equation}
    \label{eq: termination condition strict equality}
        \mathrm{LCB}_t(k)  
        >
          \max\limits_{a \in \mathcal{A}_{t+1} \setminus \{k\} } \mathrm{UCB}_t(a)
          - (c+2)\tilde{\epsilon},
    \end{equation}
    which is equivalent to the termination condition
    \begin{equation}
          \mathrm{LCB}_t(k)  
            \ge
          \max\limits_{a \in \mathcal{A}_{t+1} \setminus \{k\} } \mathrm{UCB}_t(a) - (c+1)\tilde{\epsilon},
    \end{equation}
    where the equivalence follows from an argument similar to the equivalence between~\eqref{eq: eliminate condition} and 
        \eqref{eq: eliminate condition equivalent}.

    It remains to pick an arm $k \in \A_{t+1}$ satisfying condition~\eqref{eq: suf cond trigger termination}.
    Let arm $j \in \argmax\limits_{a \in \A_{\epsilon}} \Delta_a$ and consider the following two cases: (i) $j \in \A_{t+1}$ and (ii) $j \not\in \A_{t+1}$.

    If $j \in \A_{t+1}$, we pick $k = j$. We also pick~$
    T \in \argmax\limits_{\A_{\epsilon} \subseteq S \subseteq \A}
        \Delta_{k}^{(S)}    
    $ 
    to be the set associated to $\Delta_k$ (see Definition~\ref{def: our gap}).
    Note that every arm that is not in $T$
    is a non-satisfying arm since 
    $\A_{\epsilon} \subseteq T$.
    Furthermore, every non-satisfying arm that is not in $T$, hence every arm that is not in $T$, is eliminated,
    which follows from Lemma~\ref{lem: elim suboptimal} and
    \begin{equation}
        \Delta^{(t)} 
        \le \frac{1}{2} \max \limits_{a \in \A_{\epsilon}} \Delta_a
        = 
        \frac{1}{2} \Delta_k \le \frac{1}{2} \min\limits_{a \not\in T} \Delta_a,
    \end{equation} 
    where the last inequality follows from applying~\eqref{eq: Delta k^S} to $k$ and $T$.
    Therefore, we have
    $\A_{t+1} \subseteq T$.
    It follows that
    \begin{align}
         \mathrm{LCB}_t(k)
          &\ge
          Q^+_k\big(q - \Delta^{(t)}\big) - \tilde{\epsilon} \\
          &\ge
        \max\limits_{a \in T \setminus \{k\} }
        Q_{a}\big(q + \Delta^{(t)}\big) - (c+1)\tilde{\epsilon} \\
        &\ge
      \max\limits_{a \in \A_{t+1} \setminus \{k\} }
        Q_{a}\big(q + \Delta^{(t)}\big) - (c+1)\tilde{\epsilon},
    \end{align}
    where the first inequality follows from~\eqref{eq: lower approx quantile anytime bound} of the anytime quantile bound, the second inequality follows from applying~\eqref{eq: Delta k^S} to $k$ and $T$,
    and the last inequality follows from  $\A_{t+1} \subseteq T$.

    If $j \not\in \A_{t+1}$,
    we pick an arm 
    $k \in \argmax\limits_{a \in \mathcal{A}_{t+1}} 
    \mathrm{LCB}_{t}(a)$ arbitrarily.
    We also pick $T \in \argmax\limits_{\A_{\epsilon} \subseteq S \subseteq \A} \Delta_{k}^{(S)}$ and  
    we have $\A_{t+1} \subseteq T$ as in the case above.
    Furthermore, since $j \not\in \A_{t+1}$,
    we have
    \begin{equation}
    \label{eq: LCB_t(k) > Fj}
        Q^+_j \big(q - \Delta^{(t)}\big)
    \le 
    Q_{j}(q) 
    \le
    \max\limits_{a \in \mathcal{A}_{t+1}} 
    \mathrm{LCB}_{t}(a) 
    = \mathrm{LCB}_{t}(k),
    \end{equation}
    where the second inequality follows from an argument similar to~\eqref{eq: j not in At}.
    It follows that
    \begin{align}
         \mathrm{LCB}_t(k) 
        &\ge Q^+_j \big(q - \Delta^{(t)}\big)  \\
        &\ge
        \max\limits_{a \in T \setminus \{j\} }
        Q_{a}\big(q + \Delta^{(t)}\big)   - c \tilde{\epsilon} \\
        &\ge
        \max\limits_{a \in \A_{t+1} \setminus \{k\} }
        Q_{a}\big(q + \Delta^{(t)}\big) - (c+1) \tilde{\epsilon},
    \end{align}
    where the first inequality follows from~\eqref{eq: LCB_t(k) > Fj}, the second inequality follows from applying~\eqref{eq: Delta k^S} to $j$ and $T$,
    and the last inequality follows from  $\A_{t+1} \subseteq T$.
\end{proof}

\section{Experiments}


In this section, we present a series of experiments designed to validate our theoretical findings in a practical setting. Specifically, we assess whether our conclusions hold when transitioning from an idealized infinite-data framework to real-world scenarios with a limited number of samples. 

Let us first introduce the linear recurrent neural network (RNN) used in our study. It is defined by the following recurrence relations:
\begin{align*}
    h_0 &= 0, \\
    x_{t+1} &= Ax_t + Bu_{t+1}, \\
    y_t &= Cx_t,
\end{align*}
where $x_t \in \mathbb{R}^{d_{\text{hidden}}}$ represents the hidden state, $u_t \in \mathbb{R}$ is the input, and $y_t \in \mathbb{R}$ is the output. The network parameters consist of $A \in \mathbb{C}^{d_{\text{hidden}} \times d_{\text{hidden}}}$, $B \in \mathbb{C}^{d_{\text{hidden}} \times 1}$, and $C \in \mathbb{C}^{d_{\text{hidden}}}$. 

Without loss of generality, we adopt a diagonal representation for the matrix $A$. The choice of its initial eigenvalues depends on the specific experiment: we either use a random initialization or employ the structured initialization given by Eq.~\eqref{Param_of_the_as}. 

In the simple experiments conducted below, the objective is to learn a single filter. Consequently, there is no need to decompose the matrix $A$ into multiple diagonal blocks. The matrix $C$ is initialized as:
\[
C_{\text{init}} = \begin{pmatrix} 1 & \dots & 1 \end{pmatrix} \in \mathbb{R}^{d_{\text{hidden}} \times 1}.
\]
and the entries of $B$ are initialized given by Eq.~\eqref{Param_of_the_bs}.\newline

The synthetic dataset consists of autoregressive sequences $X = (u_1, u_2, \dots, u_N)$ of length $N$, generated as:
\begin{equation}
    u_n = \rho u_{n-1} + \epsilon_k, \quad \epsilon_k \sim \mathcal{N}(0, 1 - \rho^2), \quad u_1 \sim U(0,1).
\end{equation}
The objective is to learn a mapping with linear recurrences $f: X \to Y$, where the target is given by:
\begin{equation}
    Y = u_{t^*}
\end{equation}
This corresponds to learning a shift of $N - t^*$ with finite samples. 


\subsection{Random initialization vs. Shift-K initialization}

In this first set of experiments, we analyze the impact of initializing the complex diagonal entries $a_s$ of the linear RNN using phases that are uniformly distributed over a segment of the unit disk, with a constant radius close to 1, as described in the parametrization in Eq.~\eqref{Param_of_the_as}. Additionally, the parameters $b_s$ are initialized following the parametrization given in Eq.~\eqref{Param_of_the_bs}. We call this initialization the shift-$K$ initialization. We compare this approach to a standard random initialization to evaluate potential benefits in terms of performance and stability.


\begin{table}[ht]
    \centering
    \begin{tabular}{@{}ll@{}}
        \toprule
        & \textbf{Random init.} \hspace{4,2cm} \textbf{Shift-K init.} \\ \midrule
        Batch size & [20, 50, 100] \hspace{4,4cm} [20, 50, 100] \\
        Number Samples & 130000 \hspace{5,2cm} 130000\\
        Sequence length & 1500 \hspace{5,6cm} 1500 \\
        Position of $t^*$ & 200 \hspace{5,8cm} 200 \\
        Hidden neurons & 128 \hspace{5,8cm} 128 \\
        Input / output dimension & 1 \hspace{6,2cm} 1 \\
        Learning rates & [0.01, 0.005, 0.001, 0.0001] \hspace{2,2cm}[0.01, 0.005, 0.001, 0.0001] \\
        Weight decay & $10^{-5}$ \hspace{5,7cm}$10^{-5}$ \\ 
        $\rho$ & \{0, 0.2, 0.4, 0.6, 0.8, 1\} \hspace{2,7cm} \{0, 0.2, 0.4, 0.6, 0.8, 1\} \\ \midrule
        $a_s$ param. & $a_u = e^{-\alpha/K_{\textnormal{init}}}e^{i\epsilon_u\pi}, \varepsilon \sim \mathcal{U}(-1,1)$ \hspace{1cm} $a_u = e^{-\alpha/K_{\textnormal{init}}}e^{iu\frac{\pi}{K_{\textnormal{init}}}}$\\
         $b_s$ param. & $b_u = \frac{e^{-\alpha}(e^{2\alpha}-e^{-2\alpha})}{2K_{\textnormal{init}}}\times(-1)^u$ \hspace{1,8cm} $b_u = \frac{e^{-\alpha}(e^{2\alpha}-e^{-2\alpha})}{2K_{\textnormal{init}}}\times(-1)^u$  \\
        $\alpha$ & 1 \hspace{6,2cm} 1 \\
       $K_{\textnormal{init}}$ & 1300 \hspace{5,6cm} 1300 \\ \midrule
        Number epochs & 60 \hspace{6cm} 60 \\
        \bottomrule
    \end{tabular}
    \caption{{Experimental details for Figure~\ref{fig:xps} (left)}. We use $[\dots]$ to denote hyperparameters that were scanned over with grid search and $\{\dots\}$ to denote the variable of interest for the figure. We chose the same representation for $b_s$ in both cases because we observed small impact of this parameter on the final results.}
    \label{tab:xp-compare_init}
\end{table}

\begin{figure}
    \centering
    \includegraphics[width=1\linewidth]{img/appendix_2_filters.pdf}

    \vspace*{-.2cm}
    
    \caption{\textit{Comparison of Filters Obtained with Different Initialization Methods. Left: Filter obtained using our proposed shift-$K$ initialization method, which exhibits a more structured and interpretable pattern. Right: Filter obtained with random initialization, which appears significantly noisier, indicating less effective memory propagation. }}
    \label{fig:appendix 2 filters}
\end{figure}

\subsection{Robustness of Shift-K initialization}

In this second set of experiments, we investigate the robustness of our initialization scheme with respect to inaccuracies in the choice of $K_{\textnormal{init}}$ when initializing $a_s$ as in Eq.~\eqref{Param_of_the_as}. In practical applications, the actual shift of the sequence is often unknown, making it impossible to initialize with the exact optimal value of~$K$. A robust initialization method should exhibit resilience to such uncertainties, allowing for performance stability within a reasonable range of $K_{\textnormal{init}}$ values.

\begin{table}[ht]
    \centering
    \begin{tabular}{@{}l@{}}
        \toprule
        \textbf{Shift-K init.} \\ \midrule
        Batch size: [20, 50, 100] \\
        Number of Samples: 150000 \\
        Sequence length: 2250 \\
        Position of $t^*$: 250 \\
        Hidden neurons: 128 \\
        Input / output dimension: 1 \\
        Learning rates: [0.01, 0.005, 0.001, 0.0001] \\
        Weight decay: $10^{-5}$ \\ 
        $\rho$: 0.7 \\ \midrule
        $a_s$ param.: $a_u = e^{-\alpha/K_{\textnormal{init}}}e^{iu\frac{\pi}{K_{\textnormal{init}}}}$\\
        $b_s$ param.: $b_u = \frac{e^{-\alpha}(e^{2\alpha}-e^{-2\alpha})}{2K_{\textnormal{init}}}\times(-1)^u$  \\
        $\alpha$: 1 \\
        $K_{\textnormal{init}}$: \{250, 500, 1000, 2000, 4000, 8000, 16000, 32000\} \\ \midrule
        Number of epochs: 60 \\
        \bottomrule
    \end{tabular}
    \caption{{Experimental details for Figure~\ref{fig:xps} (right).}}
    \label{tab:xp-robustness}
\end{table}


\end{appendices}

\end{document}