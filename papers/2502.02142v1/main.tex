\documentclass[conference]{IEEEtran}

\IEEEoverridecommandlockouts
% The preceding line is only needed to identify funding in the first footnote. If that is unneeded, please comment it out.

% *** PACKAGES ***
\usepackage{cite}
\usepackage{amsmath,amssymb,amsfonts}
\usepackage{algorithmic}
\usepackage{graphicx}
\usepackage{textcomp}
\usepackage{xcolor}

% *** EXTRA PACKAGES ***
\usepackage{mathtools,amsmath}
\usepackage{amsmath,amssymb,amsfonts}
\usepackage{graphicx} % add this line in the preamble
\usepackage{amsmath}  % For advanced math typesetting
\usepackage{graphicx} % For including graphics
\usepackage{textcmds} % For quote style
\usepackage{tikz}
\usepackage{multirow}
\usepackage{subcaption}
\usepackage{array, makecell}%
\newcommand*\circled[1]{\tikz[baseline=(char.base)]{
            \node[shape=circle,fill,inner sep=2pt] (char) {\textcolor{white}{#1}};}}

\def\BibTeX{{\rm B\kern-.05em{\sc i\kern-.025em b}\kern-.08em
    T\kern-.1667em\lower.7ex\hbox{E}\kern-.125emX}}

\title{Towards Efficient LUT-based PIM: A Scalable and Low-Power Approach for Modern Workloads}

\author{\IEEEauthorblockN{Bahareh Khabbazan, Marc Riera, Antonio González }
\IEEEauthorblockA{\textit{dept. of Computer Architecture } \\
\textit{Universitat Polit\`{e}cnica de Catalunya (UPC)}\\
Barcelona, Spain\\
\{bahareh.khabbazan, marc.riera.villanueva, antonio.gonzalez\}@upc.edu}
}

%%%%%%%%%%%---Make Space-----%%%%%%%%%%%%%
% shrink horizontally (letterspace=-25 suggested by julian)
%\usepackage[letterspace=-27,tracking=true,final]{microtype}
%%%%%%%%%%%%%%%%%%%%%%%%%%%%%%%%%%%%
\begin{document}

% Paper title
%\title{Lama: A Lightweight Mechanism towards LUT-based PIM Approach }
\maketitle



\begin{abstract}
Data movement in memory-intensive workloads, such as deep learning, incurs energy costs that are over three orders of magnitude higher than the cost of computation. Since these workloads involve frequent data transfers between memory and processing units, addressing data movement overheads is crucial for improving performance. Processing-using-memory (PuM) offers an effective solution by enabling in-memory computation, thereby minimizing data transfers. In this paper we propose \textbf{Lama}, a LUT-based PuM architecture designed to efficiently execute SIMD operations by supporting independent column accesses within each mat of a DRAM subarray. Lama exploits DRAM's mat-level parallelism and open-page policy to significantly reduce the number of energy-intensive memory activation (ACT) commands, which are the primary source of overhead in most PuM architectures. Unlike prior PuM solutions, Lama supports up to 8-bit operand precision without decomposing computations, while incurring only a 2.47\% area overhead. Our evaluation shows Lama achieves an average performance improvement of 8.5$\times$ over state-of-the-art PuM architectures and a 3.8$\times$ improvement over CPU, along with energy efficiency gains of 6.9$\times/$8$\times$, respectively, for bulk 8-bit multiplication.

We also introduce \textbf{LamaAccel}, an HBM-based PuM accelerator that utilizes Lama to accelerate the inference of attention-based models. LamaAccel employs exponential quantization to optimize product/accumulation in dot-product operations, transforming them into simpler tasks like addition and counting. LamaAccel delivers up to 9.3$\times/$19.2$\times$ reduction in energy and 4.8$\times/$9.8$\times$ speedup over TPU/GPU, along with up to 5.8$\times$ energy reduction and 2.1$\times$ speedup over a state-of-the-art PuM baseline.

% ##Longer version for LamaAccel##
% Additionally, we introduce LamaAccel, an HBM-based PuM accelerator designed to enhance the inference of attention-based models. LamaAccel utilizes Lama's in-memory computing capabilities and utilizes a data mapping approach to allocate encoders/decoders block of LLMs across multiple pseudo-channels of HBM and pipelines the processing of different layers within each block. Further, LamaAccel employs exponential quantization to simplify dot-product accumulation into memory-friendly addition and counting operations. Our evaluation demonstrates that LamaAccel delivers a 6.9$\times/$12.1$\times$ reduction in energy and a 4.4$\times/$7.2$\times$ speedup over TPU/GPU, along with a 3.7$\times$ energy reduction and 1.8$\times$ speedup over a state-of-the-art PuM baseline.
\end{abstract}

% Keywords
%\begin{IEEEkeywords}
%Processing in-memory; Large-language models; SIMD; Domain-specific accelerator.
%\end{IEEEkeywords}

%%%%%% -- PAPER CONTENT STARTS-- %%%%%%%%
\section{Introduction}

Despite the remarkable capabilities of large language models (LLMs)~\cite{DBLP:conf/emnlp/QinZ0CYY23,DBLP:journals/corr/abs-2307-09288}, they often inevitably exhibit hallucinations due to incorrect or outdated knowledge embedded in their parameters~\cite{DBLP:journals/corr/abs-2309-01219, DBLP:journals/corr/abs-2302-12813, DBLP:journals/csur/JiLFYSXIBMF23}.
Given the significant time and expense required to retrain LLMs, there has been growing interest in \emph{model editing} (a.k.a., \emph{knowledge editing})~\cite{DBLP:conf/iclr/SinitsinPPPB20, DBLP:journals/corr/abs-2012-00363, DBLP:conf/acl/DaiDHSCW22, DBLP:conf/icml/MitchellLBMF22, DBLP:conf/nips/MengBAB22, DBLP:conf/iclr/MengSABB23, DBLP:conf/emnlp/YaoWT0LDC023, DBLP:conf/emnlp/ZhongWMPC23, DBLP:conf/icml/MaL0G24, DBLP:journals/corr/abs-2401-04700}, 
which aims to update the knowledge of LLMs cost-effectively.
Some existing methods of model editing achieve this by modifying model parameters, which can be generally divided into two categories~\cite{DBLP:journals/corr/abs-2308-07269, DBLP:conf/emnlp/YaoWT0LDC023}.
Specifically, one type is based on \emph{Meta-Learning}~\cite{DBLP:conf/emnlp/CaoAT21, DBLP:conf/acl/DaiDHSCW22}, while the other is based on \emph{Locate-then-Edit}~\cite{DBLP:conf/acl/DaiDHSCW22, DBLP:conf/nips/MengBAB22, DBLP:conf/iclr/MengSABB23}. This paper primarily focuses on the latter.

\begin{figure}[t]
  \centering
  \includegraphics[width=0.48\textwidth]{figures/demonstration.pdf}
  \vspace{-4mm}
  \caption{(a) Comparison of regular model editing and EAC. EAC compresses the editing information into the dimensions where the editing anchors are located. Here, we utilize the gradients generated during training and the magnitude of the updated knowledge vector to identify anchors. (b) Comparison of general downstream task performance before editing, after regular editing, and after constrained editing by EAC.}
  \vspace{-3mm}
  \label{demo}
\end{figure}

\emph{Sequential} model editing~\cite{DBLP:conf/emnlp/YaoWT0LDC023} can expedite the continual learning of LLMs where a series of consecutive edits are conducted.
This is very important in real-world scenarios because new knowledge continually appears, requiring the model to retain previous knowledge while conducting new edits. 
Some studies have experimentally revealed that in sequential editing, existing methods lead to a decrease in the general abilities of the model across downstream tasks~\cite{DBLP:journals/corr/abs-2401-04700, DBLP:conf/acl/GuptaRA24, DBLP:conf/acl/Yang0MLYC24, DBLP:conf/acl/HuC00024}. 
Besides, \citet{ma2024perturbation} have performed a theoretical analysis to elucidate the bottleneck of the general abilities during sequential editing.
However, previous work has not introduced an effective method that maintains editing performance while preserving general abilities in sequential editing.
This impacts model scalability and presents major challenges for continuous learning in LLMs.

In this paper, a statistical analysis is first conducted to help understand how the model is affected during sequential editing using two popular editing methods, including ROME~\cite{DBLP:conf/nips/MengBAB22} and MEMIT~\cite{DBLP:conf/iclr/MengSABB23}.
Matrix norms, particularly the L1 norm, have been shown to be effective indicators of matrix properties such as sparsity, stability, and conditioning, as evidenced by several theoretical works~\cite{kahan2013tutorial}. In our analysis of matrix norms, we observe significant deviations in the parameter matrix after sequential editing.
Besides, the semantic differences between the facts before and after editing are also visualized, and we find that the differences become larger as the deviation of the parameter matrix after editing increases.
Therefore, we assume that each edit during sequential editing not only updates the editing fact as expected but also unintentionally introduces non-trivial noise that can cause the edited model to deviate from its original semantics space.
Furthermore, the accumulation of non-trivial noise can amplify the negative impact on the general abilities of LLMs.

Inspired by these findings, a framework termed \textbf{E}diting \textbf{A}nchor \textbf{C}ompression (EAC) is proposed to constrain the deviation of the parameter matrix during sequential editing by reducing the norm of the update matrix at each step. 
As shown in Figure~\ref{demo}, EAC first selects a subset of dimension with a high product of gradient and magnitude values, namely editing anchors, that are considered crucial for encoding the new relation through a weighted gradient saliency map.
Retraining is then performed on the dimensions where these important editing anchors are located, effectively compressing the editing information.
By compressing information only in certain dimensions and leaving other dimensions unmodified, the deviation of the parameter matrix after editing is constrained. 
To further regulate changes in the L1 norm of the edited matrix to constrain the deviation, we incorporate a scored elastic net ~\cite{zou2005regularization} into the retraining process, optimizing the previously selected editing anchors.

To validate the effectiveness of the proposed EAC, experiments of applying EAC to \textbf{two popular editing methods} including ROME and MEMIT are conducted.
In addition, \textbf{three LLMs of varying sizes} including GPT2-XL~\cite{radford2019language}, LLaMA-3 (8B)~\cite{llama3} and LLaMA-2 (13B)~\cite{DBLP:journals/corr/abs-2307-09288} and \textbf{four representative tasks} including 
natural language inference~\cite{DBLP:conf/mlcw/DaganGM05}, 
summarization~\cite{gliwa-etal-2019-samsum},
open-domain question-answering~\cite{DBLP:journals/tacl/KwiatkowskiPRCP19},  
and sentiment analysis~\cite{DBLP:conf/emnlp/SocherPWCMNP13} are selected to extensively demonstrate the impact of model editing on the general abilities of LLMs. 
Experimental results demonstrate that in sequential editing, EAC can effectively preserve over 70\% of the general abilities of the model across downstream tasks and better retain the edited knowledge.

In summary, our contributions to this paper are three-fold:
(1) This paper statistically elucidates how deviations in the parameter matrix after editing are responsible for the decreased general abilities of the model across downstream tasks after sequential editing.
(2) A framework termed EAC is proposed, which ultimately aims to constrain the deviation of the parameter matrix after editing by compressing the editing information into editing anchors. 
(3) It is discovered that on models like GPT2-XL and LLaMA-3 (8B), EAC significantly preserves over 70\% of the general abilities across downstream tasks and retains the edited knowledge better.
\section{Background}
\label{sec:background}


\subsection{Code Review Automation}
Code review is a widely adopted practice among software developers where a reviewer examines changes submitted in a pull request \cite{hong2022commentfinder, ben2024improving, siow2020core}. If the pull request is not approved, the reviewer must describe the issues or improvements required, providing constructive feedback and identifying potential issues. This step involves review commment generation, which play a key role in the review process by generating review comments for a given code difference. These comments can be descriptive, offering detailed explanations of the issues, or actionable, suggesting specific solutions to address the problems identified \cite{ben2024improving}.


Various approaches have been explored to automate the code review comments process  \cite{tufano2023automating, tufano2024code, yang2024survey}. 
Early efforts centered on knowledge-based systems, which are designed to detect common issues in code. Although these traditional tools provide some support to programmers, they often fall short in addressing complex scenarios encountered during code reviews \cite{dehaerne2022code}. More recently, with advancements in deep learning, researchers have shifted their focus toward using large-language models to enhance the effectiveness of code issue detection and code review comment generation.

\subsection{Knowledge-based Code Review Comments Automation}

Knowledge-based systems (KBS) are software applications designed to emulate human expertise in specific domains by using a collection of rules, logic, and expert knowledge. KBS often consist of facts, rules, an explanation facility, and knowledge acquisition. In the context of software development, these systems are used to analyze the source code, identifying issues such as coding standard violations, bugs, and inefficiencies~\cite{singh2017evaluating, delaitre2015evaluating, ayewah2008using, habchi2018adopting}. By applying a vast set of predefined rules and best practices, they provide automated feedback and recommendations to developers. Tools such as FindBugs \cite{findBugs}, PMD \cite{pmd}, Checkstyle \cite{checkstyle}, and SonarQube \cite{sonarqube} are prominent examples of knowledge-based systems in code analysis, often referred to as static analyzers. These tools have been utilized since the early 1960s, initially to optimize compiler operations, and have since expanded to include debugging tools and software development frameworks \cite{stefanovic2020static, beller2016analyzing}.



\subsection{LLMs-based Code Review Comments Automation}
As the field of machine learning in software engineering evolves, researchers are increasingly leveraging machine learning (ML) and deep learning (DL) techniques to automate the generation of review comments \cite{li2022automating, tufano2022using, balachandran2013reducing, siow2020core, li2022auger, hong2022commentfinder}. Large language models (LLMs) are large-scale Transformer models, which are distinguished by their large number of parameters and extensive pre-training on diverse datasets.  Recently, LLMs have made substantial progress and have been applied across a broad spectrum of domains. Within the software engineering field, LLMs can be categorized into two main types: unified language models and code-specific models, each serving distinct purposes \cite{lu2023llama}.

Code-specific LLMs, such as CodeGen \cite{nijkamp2022codegen}, StarCoder \cite{li2023starcoder} and CodeLlama \cite{roziere2023code} are optimized to excel in code comprehension, code generation, and other programming-related tasks. These specialized models are increasingly utilized in code review activities to detect potential issues, suggest improvements, and automate review comments \cite{yang2024survey, lu2023llama}. 




\subsection{Retrieval-Augmented Generation}
Retrieval-Augmented Generation (RAG) is a general paradigm that enhances LLMs outputs by including relevant information retrieved from external databases into the input prompt \cite{gao2023retrieval}. Traditional LLMs generate responses based solely on the extensive data used in pre-training, which can result in limitations, especially when it comes to domain-specific, time-sensitive, or highly specialized information. RAG addresses these limitations by dynamically retrieving pertinent external knowledge, expanding the model's informational scope and allowing it to generate responses that are more accurate, up-to-date, and contextually relevant \cite{arslan2024business}. 

To build an effective end-to-end RAG pipeline, the system must first establish a comprehensive knowledge base. It requires a retrieval model that captures the semantic meaning of presented data, ensuring relevant information is retrieved. Finally, a capable LLM integrates this retrieved knowledge to generate accurate and coherent results \cite{ibtasham2024towards}.




\subsection{LLM as a Judge Mechanism}

LLM evaluators, often referred to as LLM-as-a-Judge, have gained significant attention due to their ability to align closely with human evaluators' judgments \cite{zhu2023judgelm, shi2024judging}. Their adaptability and scalability make them highly suitable for handling an increasing volume of evaluative tasks. 

Recent studies have shown that certain LLMs, such as Llama-3 70B and GPT-4 Turbo, exhibit strong alignment with human evaluators, making them promising candidates for automated judgment tasks \cite{thakur2024judging}

To enable such evaluations, a proper benchmarking system should be set up with specific components: \emph{prompt design}, which clearly instructs the LLM to evaluate based on a given metric, such as accuracy, relevance, or coherence; \emph{response presentation}, guiding the LLM to present its verdicts in a structured format; and \emph{scoring}, enabling the LLM to assign a score according to a predefined scale \cite{ibtasham2024towards}. Additionally, this evaluation system can be enriched with the ability to explain reasoning behind verdicts, which is a significant advantage of LLM-based evaluation \cite{zheng2023judging}. The LLM can outline the criteria it used to reach its judgment, offering deeper insights into its decision-making process.





\section{Lama Overview}\label{LamaOverview}
Lama is a lightweight LUT-based mechanism designed to efficiently execute complex arithmetic operations in bulk. Our primary goal is to address the critical inefficiencies of existing LUT-based PuM approaches, particularly those caused by the need for successive ACT commands during bulk operations, as well as the challenges associated with supporting operands larger than 4 bits. Additionally, Lama tackles the performance limitations imposed by DRAM's structural constraints. To overcome these challenges, Lama introduces a novel execution scheme that eliminates the reliance on successive ACT commands for performing bulk arithmetic operations. As shown previously in Figure~\ref{fig:vector_matrix_mult}, Lama clusters operations into batches, which are defined for simpler use as \textit{operand-coalesced batches}, where each batch consists of multiple operations that share a scalar operand. In other words, an operand-coalesced batch is a function $f$ whose inputs are a scalar $a$ and a vector $b$, and the result is another vector that consists of applying the function $f$ to each pair of elements $a$ and $b_{i}$.

To perform LUT-based computing and derive the result of a function $f$ applied to operands $a$ and $b$ (i.e., $f(a,b)$), Lama defines two key operations: \textbf{\textit{LUT activation}} which involves a memory ACT operation that activates a row, with the row index determined by the value of $a$, and \textbf{\textit{LUT retrieval}}, where the actual results for concurrent operations $f(a,b_{i})$ are fetched through one internal column access (ICA). The starting point for the column access is determined by the value of $b_{i}$, which addresses the specific column positions in the activated row. These two operations are executed consecutively to efficiently compute the desired function result.

% Alternative name instead of LUT retrieval: LUT readout
Lama supports LUT retrieval for any given function $f$, enabling arithmetic operations on operands with bit-widths of up to 8-bit. Depending on the specific operation, the result $f(a,b_{i})$ can have a bit-width of up to 16-bit (e.g., multiplication).

Figure~\ref{fig:lut_data_layout} illustrates the data layout for a set of LUTs used to perform bulk operations on operand-coalesced batches. Each LUT is dedicated to one independent function $f(a,b_{i})$. The size of the LUT varies based on the operands and result precision of function $f$. For example, a 4-bit multiplication LUT occupies one mat, while an 8-bit multiplication LUT requires eight mats to fit its data. To enable parallel processing, the LUT is replicated across the entire subarray. The \textit{degree of parallelism}, denoted as $p$, reflects the number of simultaneous operations that can be performed through a LUT retrieval, and is influenced by the LUT data size. In scenarios where the LUT data size is small, such as 4-bit multiplications, $p$ can reach up to 16, i.e., the number of mats available in HBM2. On the other hand, for larger LUT data sizes like 8-bit multiplications, $p$ is reduced to 2 due to the increased data requirements.

If the number of parallel operations ($p$) that can be executed simultaneously is less than the size of the operand-coalesced batch, completing all operations within the batch requires issuing additional memory commands. However, Lama avoids the need for extra LUT activations in this scenario. The reason is that the LUT activation, which is based on the scalar operand, remains valid for the entire operand-coalesced batch. Consequently, Lama reuses the already activated row in the local row buffer by taking advantage of the open-page policy. Only the LUT retrievals for the subsequent sets of $p$ operations are performed, ensuring that only a single ACT command is required for the entire batch of operand-coalesced operations.

\begin{figure}[t!]
\centering
\includegraphics[height=2.7cm, width=8.9cm]{figures/lut_data_layout_small.pdf}
%\vskip -0.15in
\caption{LUT data layout for (a) 4-bit and (b) 8-bit multiplications.}
\label{fig:lut_data_layout}
\vskip -0.15in
\end{figure}

\subsection{Enabling Independent Column Selection in DRAM Mats}
In conventional commodity DRAMs, memory read operations involve a column-address counter/latch that selects the same column position across all mats within a subarray using Column Select Lines (CSLs). Since each mat provides 8-bit data per column access, the column counter can select from 64 different data positions in HBM2. However, to efficiently execute operand-coalesced batches in bulk, independent column access across mats is essential. This capability allows each mat to access different data positions within the same row.

To facilitate this, Lama replicates the column counter/latch within each bank to match the number of mats in a subarray. In HBM2, where each subarray consists of 16 mats, Lama introduces 16 column counters/latches per bank, enabling independent column selection across all mats. This setup is suitable because only one subarray can connect its local data lines to the global sense amplifiers at a time for read or write operations, and thus there is no need to replicate per subarray.

Moreover, to accommodate both conventional memory operations and LUT-based modes, Lama's column counters are designed to receive the column selection signal from the address register or from a temporary buffer via a multiplexer. The added area overhead of these replicated column counters is minimal, and their impact on overall memory capacity and performance is negligible, as will be detailed in Section~\ref{overhead}.

\subsection{Mask Logic}
When performing a LUT retrieval operation, each mat transfers consecutive column positions, but not all of the fetched data from mats may correspond to the desired result $f(a,b_{i})$. This issue arises when the LUT data for the function $f$ spans across multiple mats, reducing the degree of parallelism ($p$) to less than 16. In such cases, the mask logic selects valid results while filtering out irrelevant data from the remaining mats.

The mask logic's functionality is illustrated in Figure~\ref{fig:mask_logic}. The design features a multiplexer with sixteen inputs, each connected to a different mat in the subarray. To filter out irrelevant data and retain the target values, the mask logic operates in a serial manner, where in each iteration, the select port of the multiplexer is determined by a small finite-state machine that uses the most significant bits (MSBs) of the vector element $b_{i}$ and the precision of the operation to choose the correct value.

% A small state machine determines the number of MSB bits required for selection, depending on the precision of the operation.

Because the number of valid results $f(a,b)$ retrieved in a single LUT access depends on the precision and degree of parallelism, the size of the result can vary. To handle this variation, a buffer is placed after the multiplexer, concatenating results until they reach 16 bytes before outputting. This ensures a consistent bitwidth across different precision levels and parallelism degrees.

% The mask logic's functionality is illustrated in Figure~\ref{fig:mask_logic}. It consists of a multiplexer with sixteen inputs, each connected to one of the mats in a subarray, a state machine responsible for generating the selection signal, and a fifo to concatenate the selected values. The state machine determines the correct multiplexer selection based on the precision of the operands and the 3-bit MSB of operand $b$. The selected values are then stored in the shift register, which aligns the data with the read/write bitwidth in each read cycle.

\begin{figure}[t!]
\centering
\includegraphics[height=1.8cm, width=4cm]{figures/mask_logic_new.pdf}
%\vskip -0.15in
\caption{Mask logic.}
\label{fig:mask_logic}
%\vskip -0.15in
\end{figure}

% When $p=16$, all retrieved results are valid, and the mask logic is bypassed. However, when $p<16$, the mask logic ensures that only the relevant results are processed, minimizing unnecessary data transfers. These added components incur minimal area overhead and do not interfere with the memory's normal operations, allowing seamless switching between storage and computation modes.

% The number of cycles is at most 8 while $p=8$ with the latency of internal cycles which is very short compared with cycles is needed to perform READ command operations ($t_{CL} = 16ns$).

%%% START GUIDELINE %%%

% 1) CASE STUDY FOR BULK OPERATIONS: Our scheme can operate multiple operations at one time with one constraint. operations have to have the same first operand and they can have different op2 which is the case while doing mult of an operand of a matrix into all its corresponded values. We kept the architecture of commodity DRAM + repeat col counter per MAT in order to be able to read different places in each MAT + mask logic (serializer). we use index of op1 as index of row addr and index of op2 as col addr + Figure4.

% 2) Discuss on different precision support and changes in the level of parallelism.

%% Methodology and Evaluation %%

%1) The hardware caracteristics of our scheme are summerized in Table 2. HBM timings, energy, num of MAC, SAs, FAW window (like TransPIM), in the related paragraph, mention bandwidth. 16 col counters, temp size

%2) Overhead breakdown of ourscheme and the power of each added module + area of HBM like table 2 in TransPIM. In the related paragraph mention how we scaled the implementation technology to 22nm to match the memory technology. we implement our scheme using verilog, synopsis, etc.

%3) For the first case studY: We evaluate ourscheme in case of performing INT4-8 multiplications and compared it with pLUTo baseline and SIMDRAM. In all cases CPU is the baselineour implementation assumes the parralel operation of 4banks at the same time, so in the pLUTo baseline, the same level of parralelism withh 4 SA is implemented. We assume 1024 multiplication operations

%%% END GUIDELINE %%%

\section{Case study 1: Lama for Bulk Multiplications}
In this section, we detail how the Lama technique handles bulk multiplications for one operand-coalesced batch. To simplify the explanation, we first describe the process for 4-bit bulk multiplications and then discuss the differences when scaling up to 8-bit bulk multiplications.

As outlined in Section~\ref{HBM_organization}, each memory bank in DRAM is organized into subarrays, with each subarray consisting of 16 mats. Lama leverages both bank-level parallelism and mat-level parallelism within each bank, using all 16 mats in a subarray to perform simultaneous arithmetic operations, thereby enabling efficient bulk execution. Figure~\ref{fig:case_study1} illustrates the DRAM bank structure with the modifications of Lama to carry out bulk multiplications. To allow multiple subarrays to keep their rows open simultaneously, the global row buffer must be isolated from the local row buffers within each subarray. To achieve this, Lama employs a technique proposed in \cite{salp}, which involves adding a tri-state buffer per mat after the column-selection logic (CSL). This prevents short circuits in the master data line (MDL) for activated mats aligned in the same vertical position. This method ensures that multiple rows in different subarrays can remain open, while only one \textit{designated} subarray is used to handle column commands at any given time.

% Additionally, the bus arbiter manages data communication through the global I/O bus, coordinating interactions between the mask logic, temporary buffer, and global row-buffer.

\begin{figure}[t!]
\centering
\includegraphics[width=1.0\columnwidth]{figures/case_study1_new.pdf}
%\vskip -0.15in
\caption{Detailed steps to perform bulk 4-bit multiplications using Lama.}
\label{fig:case_study1}
\vskip -0.15in
\end{figure}

Figure~\ref{fig:case_study1} provides an overview of the DRAM bank structure and the execution flow required to implement Lama. Lama involves several key components: the \textbf{source subarray}, which stores the vector operands $b$ for all operand-coalesced batches, with the elements in each vector organized across one or more rows; the \textbf{compute subarray}, which holds the LUT data following the layout shown in Figure~\ref{fig:lut_data_layout} (for 4-bit operations each LUT for $f(a,b_{i})$ occupies a single mat and it is replicated in 16 mats); the \textbf{temporary buffer}, a small buffer used to temporarily store elements of the vector operand $b$ fetched from the source subarray; the \textbf{mask logic}, which filters and processes fetched data as needed; the \textbf{column counters}, previously described in Section \ref{LamaOverview}; and finally, the \textbf{bus arbiter}, responsible for managing data transfers through the global I/O bus and coordinating interactions between the mask logic, temporary buffer, and global row buffer.

In addition, for simplicity, we refer to the process of reading elements of the vector operand $b$ from the source subarray and storing them in the temporary buffer as an \textit{internal memory read} operation. Each internal read consists of two consecutive \textbf{internal column accesses (ICAs)} that fetch the entire 32-byte DRAM atom, corresponding to 8 bits per mat in each ICA. It is important to note that the key distinction between an internal read and a standard memory read is that the internal read does not transfer data to the I/O buffer nor sends it to the host; instead, it stores the data locally in the temporary buffer.

% ...corresponding to 32 different operands $b$ in 8-bit precision. ---> Fixed amount due to padding.

As previously mentioned, Lama decomposes the vector-matrix multiplication into multiple operand-coalesced batches. Each batch corresponds to a scalar-vector operation with a scalar operand $a$ and a vector operand $b$, where the elements of $b$ are stored across one or more rows in the memory. To manage row addressing in the \textit{source subarray}, Lama relies on the \textit{positional index} of the scalar operand $a$ within its original vector before decomposing the vector-matrix multiplication. This positional index is used to issue the ACT command to the corresponding row in the source subarray, which holds the coalesced batch related to that scalar operand. If the vector operand $b$ of the coalesced batch exceeds the storage capacity of a single row, Lama scales the positional index to span multiple rows. In contrast, for the \textit{compute subarray}, the \textit{value} of $a$ is used to ACTIVATE the appropriate LUT row where the function results are stored. Further details on this activation process are provided later in the section.

% For example, in Figure~\ref{fig:vector_matrix_mult}, row 1 in the source subarray holds the operand-coalesced batch {$M_{11}, M_{12}, M_{13}$}, which is associated with the first scalar element $V_{1}$ in vector $V$.

Lama execution flow employs a systematic approach to perform bulk multiplications for one of the operand-coalesced batches residing in the source subarray. Here we explain the process for 4-bit bulk multiplications as an example. First, in the source subarray, a row is activated based on the corresponding positional index of the scalar operand $a$ through an ACT command (\circled{1}), and all elements of vector operand $b$ in the selected row are stored in the local-row buffer of the source subarray, where they will remain for multiple iterations. Then, an internal read command is issued to fetch 32 different elements of $b$ and store them in the temporary buffer (\circled{2}). It is important to note that Lama statically performs zero-padding on elements of $b$ to match the fixed bitwidth of the hardware no matter the precision, that is, 4-bit elements are padded to 8-bit, which corresponds to the reading granularity of each mat. The indexed row in the source subarray is left open after the internal read operation, and remains open until all elements of $b$ in that row have been processed to finish the computations of the operand-coalesced batch.

% The indexed row in the source subarray remains open after the internal read operation, until finishing the computation iteration on all subset of the operand-coalesced batch.

Next, the LUT-based computation begins in the compute subarray by performing a LUT activation based on the \textit{value} of the scalar $a$ as an index (\circled{3}). After the desired row is activated and resides in the local row buffer, the column address counter/latch units initiate the LUT retrieval operation using the first 16 elements of $b$ stored in the temporary buffer ($b_{0}$ to $b_{15}$) as shown in\circled{4}. In the first LUT retrieval, each mat outputs an 8-bit result for the operation $f(a,b_{i})$. These results are directly transferred to the host via the global I/O bus. For the next set of 16 elements of $b$, the column counters perform an additional LUT retrieval, and the results are again transferred to the host.

This iterative process continues until all operations for the entire operand-coalesced batch are completed. Once all computations are finalized, a PRECHARGE command is issued to close the open rows in both the source and compute subarrays. In the event that the elements of $b$ span over several rows, additional ACT and PRECHARGE commands are issued per row, requiring more iterations to complete the batch. Note that with a parallelism degree of $p = 16$, all LUT-retrieved values are valid, leaving the mask logic idle throughout the entire process for 4-bit bulk multiplications.

% The temp buffer size is chosen as 64-byte in order to keep the multiplication results while keeping the remaining $b$ operands that do not perform the LUT retrieval yet.

\subsection{8-bit Multiplications}
Figure~\ref{fig:timeline_multiplication} depicts the timeline of performing 8-bit bulk multiplications within a memory bank using Lama. The process for accessing elements of $b$ from the source subarray remains consistent with the 4-bit operation, where Lama fetches 32 different $b$ elements per internal memory read. However, the LUT-based computation differs in two key aspects when scaling up to 8-bit operations.

\begin{figure*}[t!]
\centering
\includegraphics[width=1.0\textwidth]{figures/timeline_multiplication.pdf}
%\vskip -0.15in
\caption{Timeline of performing multiplications using Lama. Each read operation includes two internal column accesses (ICA). The "M" block represents the latency of the mask logic. LUT retrieval includes one ICA followed by the mask logic. An ACT command is issued only once to access the row indexed by the scalar operand $a$ in both the source and compute subarrays. For subsequent iterations with new elements of $b$, no additional ACT commands are needed (indicated by the red block) due to the open-page policy.}
\label{fig:timeline_multiplication}
\vskip -0.15in
\end{figure*}

First, the LUT data for an 8-bit multiplication spans across eight mats (Figure~\ref{fig:lut_data_layout}b), reducing the parallelism degree to $p=2$. During one LUT retrieval, only two elements of $b$ are selected from the temporary buffer and broadcasted to all 16 column counters, with each set of eight consecutive column counters processing the same element $b$. While this procedure limits the parallelism within each bank, Lama compensates it by leveraging bank-level parallelism, maintaining high throughput even when handling higher operand precision.

% ...meaning each set of 8 column counters receives the same operand $b$. This limits the number of simultaneous operations but ensures the integrity of the computation across higher operand precision.

Second, after performing the LUT retrieval for these two elements of $b$, not all fetched values are valid. Consequently, the mask logic filters out invalid results within each set of eight mats, by using the three MSBs of $b$ elements to select the mat containing the valid result of each set. The valid results are stored in the mask logic buffer until gathering 16 bytes, and then they are outputted. The mask logic operates in two cycles (since $p=2$) to serially select the valid $f(a,b_{i})$ for each element of $b$, and its latency is added to the LUT retrieval operation as illustrated in Figure~\ref{fig:timeline_multiplication} by the "M" block. As demonstrated in the evaluation section~\ref{eval_mult8b}, this added latency hardly impacts performance compared to state-of-the-art PuM techniques.

To complete the fetching of a 16-bit multiplication result for each of the two elements of $b$, two LUT retrieval operations are done. This is because each internal column access (ICA) retrieves 8 bits from each mat, and only two of the 16 mats provide valid data. During the first LUT retrieval, data is accessed according to the two elements of $b$. For the second LUT retrieval, the column counters increment the previously determined addresses by one, to access the subsequent 8 bits in each mat.

This procedure continues until all computations related to the current operand-coalesced batch are completed. However, the ACT command in both the source and compute subarray is only issued once during the initial iteration. For all subsequent iterations, Lama efficiently reuses the already-opened row in the local row buffer of both subarrays, eliminating the need for additional ACT commands. Additional memory commands are only necessary when the $b$ elements are stored in multiple rows of the source subarray.

% For other operand precision, the latency of the mask logic varies based on the parallelism level $p$ associated with that precision. However, when $p=16$ the mask logic is bypassed resulting in no additional latency overhead.

Lama is not limited to multiplication operations; it can execute any arithmetic function $f(a,b)$, including addition, division, and other complex functions. Multiplication is used as an example in this context due to its relevance, and the computational complexity it introduces when handled within memory systems.

\subsection{Column Addressing for Varying Operand Precision}
For all supported operand precisions except 4-bit, the LUT multiplication results are extended to a 16-bit format to be word aligned in memory, requiring two Internal Column Accesses (ICAs) to fetch the complete result. During each ICA, the column counter selects from 64 possible positions (each mat has 64 8-bit columns), using the elements of $b$ to determine the correct address. The 5 least significant bits (LSBs) of $b$ with a zero bit appended to the right,\{$b_{i}[4:0], 0$\}, is used as the 6-bit column address for the first ICA. For the second ICA, the appended bit is changed to one to fetch the next 8 bits from each mat to complete the 16-bit result. For 5-bit precision, all 16 mats provide a valid result (the LUT occupies a single mat and is replicated 16 times in each bank), each mat providing the result for a different $b_{i}$. For precisions higher than 5 bits, the LUT occupies multiple mats, so the remaining most significant bits (MSBs) of $b_{i}$ are used by the mask logic to select the valid results. For instance, for 6-bit precision, the LUT occupies 2 mats, and the most significant bit of $b_{i}$ is used to select which one of each pair of mats provides the desired result. In this case, each bank can process 8 operations in parallel, each corresponding to a different element of $b$.

Table~\ref{t:mult_different_precisions} shows the degree of parallelism per bank ($p$) and the number of bits from $b$ required for column addressing across varying operand precisions. For 4-bit and 5-bit multiplications, the parallelism level $p$ reaches its maximum value, utilizing all 16 mats within a subarray. The mask logic is bypassed in these cases, as all fetched results are valid. In 6-bit multiplication, $p$ is reduced to 8, as the LUT data spans two mats. The 5 LSBs of $b_i$ are used for column selection, while the MSB is used by the mask logic to select one data element out of two consecutive mats. In 7-bit multiplication, the LUT data spans four mats, decreasing $p$ to 4. In this case, the two MSBs of $b_i$ are used by the mask logic to select a valid mat among four consecutive mats. For 8-bit multiplication, the three MSBs select one element among eight consecutive mats.

\begin{table}[t!]
%\scriptsize
\caption{Parallelism Degree ($p$), Column Addressing, and Masking Requirements for Different Operand Bitwidths in Lama MUL.}
\label{t:mult_different_precisions}
\vskip -0.20in
\begin{center}
\resizebox{1.0\columnwidth}{!}{%
    \begin{tabular}{|c|c|c|c|c|}
    \hline
    \textbf{bitwidth}                  & \textbf{p}                 & \textbf{\#LSBs for Column Addressing}        & \textbf{\#MSBs for Masking}          & \textbf{\#ICAs}  \\
    \hline
               4-bit                   &   16                       & 4                                                            &  $\times$                                            & 1                \\
    \hline
               5-bit                   &   16                       & 5                                                            &  $\times$                                            & 2                \\
    \hline
               6-bit                   &   8                        & 5                                                            &  1                                                   & 2                \\
    \hline
               7-bit                   &   4                        & 5                                                            &  2                                                   & 2                \\
    \hline
               8-bit                   &   2                        & 5                                                            &  3                                                   & 2                \\
    \hline
    \end{tabular}%
}
\end{center}
\vskip -0.20in
\end{table}

\subsection{Column Addressing for Varying Precision}
For operand precisions greater than 4 bits, LUT multiplication results are extended to a 16-bit format to be word-aligned in memory, requiring two ICAs to fetch the complete result. During each ICA, the column counter selects from 64 possible positions (each mat has 64 8-bit columns), using the elements of $b$ to determine the correct address. The 5 least significant bits (LSBs) of $b$ with a zero bit appended to the right, \{$b_{i}[4:0], 0$\}, serve as the 6-bit column address for the first ICA. For the second ICA, the appended bit is changed to one to fetch the next 8 bits from each mat, completing the 16-bit result. For 5-bit precision, all 16 mats provide a valid result (the LUT data occupies a single mat and is replicated 16 times in each bank), each mat providing the result for a different $b_{i}$. For precisions higher than 5 bits, the LUT spans multiple mats, so the remaining MSBs of $b_{i}$ are used by the mask logic to select the valid results. For instance, in 6-bit precision, the LUT occupies 2 mats, and the MSB of $b_{i}$ selects which one of each pair of mats provides the desired result. In this case, each bank can process 8 operations in parallel, each corresponding to a different element of $b$.

Table~\ref{t:mult_different_precisions} summarizes the degree of parallelism per bank ($p$) and the number of bits from $b$ required for column addressing across varying operand precisions. For 4-bit and 5-bit multiplications, $p$ reaches its maximum value, utilizing all 16 mats within a subarray, with the mask logic bypassed since all fetched results are valid. In 6-bit multiplication, $p$ is reduced to 8, as the LUT data spans two mats, using the 5 LSBs of $b_i$ for column selection and the MSB by the mask logic to select one data element out of two consecutive mats. In 7-bit multiplication, the LUT data spans four mats, decreasing $p$ to 4, with the two MSBs of $b_i$ are used by the mask logic to select a valid mat among four consecutive mats. For 8-bit multiplication, the three MSBs select one element among eight consecutive mats.

\begin{table}[t!]
%\captionsetup{font=small}
%\scriptsize
%\caption{Parallelism Degree ($p$), Column Addressing, and Masking Requirements for Different Operand Bitwidths in Lama MUL.}
\caption{Parameters for different operand bitwidths in Lama MUL.}
\label{t:mult_different_precisions}
\begin{center}
%\vskip -0.20in
\resizebox{1.0\columnwidth}{!}{%
    \begin{tabular}{|c|c|c|c|c|}
    \hline
    \textbf{bitwidth}                  & \textbf{p}                 & \textbf{\# $b_{i}$ LSBs for Column Addressing}        & \textbf{\# $b_{i}$ MSBs for Masking}          & \textbf{\#ICAs}  \\
    \hline
               4-bit                   &   16                       & 4                                                            &  $\times$                                            & 1                \\
    \hline
               5-bit                   &   16                       & 5                                                            &  $\times$                                            & 2                \\
    \hline
               6-bit                   &   8                        & 5                                                            &  1                                                   & 2                \\
    \hline
               7-bit                   &   4                        & 5                                                            &  2                                                   & 2                \\
    \hline
               8-bit                   &   2                        & 5                                                            &  3                                                   & 2                \\
    \hline
    \end{tabular}%
}
\end{center}
\vskip -0.1in
\end{table}


\subsection{Methodology and Configuration}
This section presents the methodology for evaluating Lama's performance and energy consumption in the case study of bulk multiplications. Table~\ref{t:configuration} outlines the hardware characteristics of Lama's evaluation platform. Our implementation assumes a memory setup based on the standard HBM2 specification~\cite{jedec}. The timing and energy parameters are derived from \cite{fine-grained}.

As shown in Table~\ref{t:configuration}, the evaluation of Lama also considers bank-level parallelism by distributing different coalesced batches across various banks. As shown in Figure~\ref{fig:timeline_multiplication}, performing all computations for a coalesced batch requires only two ACT commands. Consequently, when all banks within a channel are utilized for computation, a total of 32 ACT commands are issued. To accommodate these 32 ACT commands, 8 {\fontfamily{cmss}\selectfont tFAW} windows are needed. To avoid any stalls while all banks are operational, the computation time for each coalesced batch must exceed 8 times the {\fontfamily{cmss}\selectfont tFAW} window duration. For 4-bit precision, which has the shortest computation time, the coalesced batch size must be greater than 128 elements of $b$ to avoid being limited by the {\fontfamily{cmss}\selectfont tFAW} constraint. For other precision, this restriction is less stringent since the computation time per batch is longer.

Furthermore, within each bank, only one subarray is required and enabled to perform arithmetic operations for a given function, including those with 8-bit precision operands. To accommodate various functions, additional subarrays can be allocated for LUT computation. The host-to-HBM bandwidth is assumed to be 256 GB/s, consistent with the specifications in \cite{fine-grained}.

\begin{table}[t!]
%\scriptsize
\caption{Architectural Parameters for Lama.}
\label{t:configuration}
%\vskip -0.20in
\begin{center}
%\vskip -0.22in
\resizebox{1.0\columnwidth}{!}{%
    \begin{tabular}{|>{\centering\arraybackslash}m{3.3cm}|m{7.4cm}|}
    \hline
    \makecell{\textbf{HBM Organization}}  &    channels/die (4-die stack) = 2 (8), pch/channel = 2,\newline  banks/channel = 16  (banks/pch = 8),  banks/group = 4,  \newline subarrays/bank = 64, bank rows = 32k, row buffer size/channel (row buffer size/pch)= 2KB (1KB), mat size = $512\times512$, DQ size = 128-bit/channel            \\
    \hline
    \textbf{HBM Timing (ns)}              &  t\textsubscript{RC} = 45, t\textsubscript{RCD} = 16, t\textsubscript{RAS} = 29, t\textsubscript{CL} = 16, t\textsubscript{RRD} = 2, t\textsubscript{WR} = 16, \newline t\textsubscript{CCD\textsubscript{S}} = 2, t\textsubscript{CCD\textsubscript{L}} = 4, t\textsubscript{FAW} = 12,  \# of activates in t\textsubscript{FAW} = 8  \\
    \hline
    \textbf{HBM Energy (pJ)}              &    e\textsubscript{ACT} = 909, e\textsubscript{Pre-GSA} = 1.51, e\textsubscript{Post-GSA} = 1.17, e\textsubscript{I/O} = 0.80  \\
    \hline
    \textbf{Bank-level Configuration}              &        Clock = 500MHz, Column Counter/Latch = $16(8b)$,  \newline  Mask Logic = $1$, Temporary Buffer = $1(64B)$               \\
    \hline
    \end{tabular}%
}
\end{center}
%\vskip -0.20in
\end{table}

The objective of our evaluation is to demonstrate that Lama significantly reduces the number of memory commands, particularly activation (ACT) commands, compared to previous PuM techniques. This reduction leads to notable energy savings and performance improvements in bulk multiplications, while also providing flexibility to support up to 8-bit operand precision.

\textit{Simulation.} We developed an in-house simulator to model the performance and energy characteristics of Lama and all PuM baselines. The simulator leverages the HBM2 architectural configuration and the timing/energy parameters shown in Table~\ref{t:configuration} to calculate the number of commands, energy consumption, and latency, for bulk multiplications at 4-bit and 8-bit integer precision.

\textit{Baselines.} For comparison, we evaluate a baseline CPU on a real system equipped with an Intel\textsuperscript{\circledR} Xeon W-2245 CPU~\cite{intel}, utilizing AVX-512 Intel\textsuperscript{\circledR} Streaming SIMD Extensions for 8-bit multiplications. We also evaluate prior PuM approaches including pLUTo~\cite{pluto}, as a LUT-based technique, and SIMDRAM~\cite{simdram}, as a charge-sharing-based PuM technique. To ensure a fair comparison, the level of parallelism is consistent across all the baselines and our Lama implementation.

\subsection{Overheads}\label{overhead}
We assume an 8GB HBM with 4 layers (1GB per channel) as the memory configuration for Lama. Lama introduces two major components to the commodity DRAM: $(i)$ column counters and $(ii)$ mask logic. In addition, it includes a temporary buffer. These components were implemented using Verilog HDL and synthesized on the Synopsys Design Compiler with a 28nm technology library. In order to match the rate of column access time $t_{CCD} = 2$ns, the added bank-level components are configured to run at 500 MHz clock frequency. The area and power data obtained from the synthesis are scaled to 22nm to match the memory technology. We take into account the difference in manufacturing process between logic and DRAM similar to previous studies~\cite{technology_difference, transpim, drisa, fulcrum}, where the DRAM process incurs around 50\% additional area overhead for the logic.

Table~\ref{t:overhead} summarizes the area and power consumption of each unit. Each memory bank in Lama is equipped with sixteen column counters, one mask logic unit, and a temporary buffer. Assuming that all banks across the HBM2 channels are equipped with these units, the additional components consume approximately 1.32 mm\textsuperscript{2}, resulting in a 2.47\% area overhead. This is significantly lower than the area overheads reported in \cite{pluto, simdram}, thus preventing any significant loss in DRAM density and memory capacity.

% Furthermore, the power consumption of the added components is only Y\% of the total power budget of HBM2.

\begin{table}[t!]
%\scriptsize
\caption{Summary of the area and power consumption of the added logic in the Lama architecture.}
\label{t:overhead}
%\vskip -0.40in
\begin{center}
\resizebox{1.0\columnwidth}{!}{%
    \begin{tabular}{|c|cc|}
    \hline
    \textbf{Units}     &  \textbf{Area(um\textsuperscript{2} per Bank}) & \textbf{Power (mW) per Bank}        \\
    \hline
    Column Counter/Latch   &     5002.8                            &       1.49             \\
    Mask Logic             &     1628                              &       1.01             \\
    Temporary Buffer       &     3636.6                            &       3.76             \\
    Others                 &     19.73                             &       0.09             \\
    \hline
    \end{tabular}%
\quad
    \begin{tabular}{|c|c|}
    \hline
    \textbf{Lama}         &  \textbf{Area(mm\textsuperscript{2}})       \\
    \hline
    8GB HBM2              &     53.15                                   \\
    Overhead              &     1.32                                    \\
    \hline
    \end{tabular}%
}
\end{center}
%\vskip -0.20in
\end{table}

\subsection{Evaluation}\label{eval_mult8b}
Previous PuM architectures have demonstrated promising results in executing bulk operations. However, these gains come at the cost of frequent ACT commands, which significantly increase energy consumption. Additionally, they face significant challenges in supporting operand sizes greater than 4-bit precision. For higher precision, their performance deteriorates due to the substantial increase in the number of issued commands, particularly for complex arithmetic operations.

To demonstrate Lama's higher performance in bulk multiplications, we compare it against existing proposals using 4-bit and 8-bit operand precisions. Table~\ref{t:cs1_eval_baselines} presents the latency, energy consumption, performance and command count for both precisions. To ensure a fair comparison, the level of parallelism is set to four for all schemes, which is consistent with the parallelism level used in the evaluation section of the pLUTo paper~\cite{pluto}. In Lama, this parallelism is achieved by executing operations across four different banks, where each bank performs operations required for one operand-coalesced batch, with one subarray per bank dedicated to LUT computation. Meanwhile, the other baselines employ subarray-level parallelism, using four subarrays within a single bank.

All techniques perform the same 1024 multiplication operations using 4 scalar operands, meaning that each bank (or subarray) handles computations related to one coalesced batch, with each bank or subarray executing 256 multiplication operations. Four key conclusions can be drawn from the results in Table~\ref{t:cs1_eval_baselines}.

First, Lama significantly reduces the number of ACT commands compared to other schemes. The ACT command count in Lama involves both reading the elements of $b$ from the source subarray and performing the LUT computation in the compute subarray. As precision increases from 4-bit to 8-bit, Lama requires the same ACT command count because row accesses are independent of operand precision. The only increase is in the number of read commands, which have a much lower energy impact compared to ACT commands.

Second, the SIMDRAM~\cite{simdram} baseline is based on the Triple Row Activation (TRA) mechanism in \cite{ambit}, which executes bitwise operations using a sequence of commands for MAJ/NOT operations. As shown in the study, each additional simultaneous row activation increases energy consumption by 22\% over a single-row activation. SIMDRAM also requires multiple execution cycles, and as operand precision increases, the number of cycles grows exponentially, leading to significant latency inefficiency. Lama achieves $13.7\times$ $(13.4\times)$ throughput for 4-bit (8-bit) precision and $5.8\times$ $(5.4\times)$ energy improvement over SIMDRAM.

Third, the pLUTo~\cite{pluto} baseline can perform 256 simultaneous bulk operations per subarray. However, multiple ACT commands are required to search all the LUT rows for the matching results, which increases energy consumption, particularly for 8-bit multiplication, where the number of rows activated rises exponentially. Lama shows $9.6\times$ $(8.3\times)$ energy savings over pLUTo in 4-bit (8-bit) multiplication. In terms of speedup, Lama achieves $3.8\times$ $(3.5\times)$ throughput for 4-bit (8-bit) precision. The decrease in throughput at 8-bit precision is due to the reduced parallelism and the extra cycles for filtering invalid data in the mask logic.

% CPU comparision
Fourth, compared to the CPU baseline for 8-bit multiplication, the LUT-based PuM architectures, Lama and pLUTo, show performance improvements of $3.8\times$ and $1.09\times$, respectively. For energy efficiency, all PuM architectures demonstrate greater energy savings than the CPU, with Lama obtaining the highest energy savings at $8\times$ over the CPU baseline.

%%% Original Version %%%
Overall, Lama effectively handles multiplications of up to 8-bit operands while reducing the number of memory commands required to perform the computations in-memory, resulting in higher performance and a more energy-efficient solution for complex operations.

\begin{table}[t!]
%\scriptsize
\caption{Comparison of Lama for bulk multiplication vs. prior PuM works. All methods execute 1024 multiplications in both 4-bit and 8-bit integers with a parallelism level of 4. Results are calculated based on the HBM2 configuration provided in Table~\ref{t:configuration}.}
\label{t:cs1_eval_baselines}
%\vskip -0.40in
\begin{center}
%\vskip -0.2in
\resizebox{1.0\columnwidth}{!}{%
    \begin{tabular}{|c|c|c|c|c|}
    \hline
    \makecell{Methods}                        &  \textbf{pLUTo~\cite{pluto}}    &   \textbf{SIMDRAM~\cite{simdram}}   &    \textbf{Lama} &    \textbf{CPU}   \\
    \hline
    \multicolumn{4}{|c|}{{\textbf{INT-4 multiplication}}}  \\  %\multicolumn{6}{q}{\textbf{INT-8 multiplication}}
    \hline
    {Latency (ns)}              &              2240               &              7964                 &           583       &    -\\
    \hline
    {Energy (nJ)}               &              247.4             &              151.23                &           25.8      &    -\\
    \hline
    {Performance (GOPs/s)}      &              0.46               &              0.13                 &           1.75      &    -\\
    \hline
    {Num ACT commands}          &              1088               &              310                  &            8        &    -\\
    \hline
    {Num Total commands}        &              2176               &              465                  &           112       &    -\\
    \hline
    \multicolumn{4}{|c|}{{\textbf{INT-8 multiplication}}}  \\
    \hline
    {Latency (ns)}              &              8963              &              34065                &           2534      &    9760.4\\
    \hline
    {Energy (nJ)}               &              989.7             &              646.9               &           118.8     &    7900\\
    \hline
    {Performance (GOPs/s)}      &              0.11               &               0.03                &           0.4      &    0.1\\
    \hline
    {Num ACT commands}          &              4352               &               1326                &            8        &    -\\
    \hline
    {Num Total commands}        &              8704               &               1989                &           592      &    -\\
    \hline
    \end{tabular}%
}
\end{center}
\vskip -0.24in
\end{table}

% Add the CPU comparision

%%% START GUIDELINE %%%

% 1) CASE STUDY - MATRIX MULTIPLICATION: Since doing accumulation using the memory is costly, need more data transport to hos for partial sums, we propose OURSHEME-suffix that applies exponential quantization which makes MMs into the form of the sum of four terms. each term refers to the number of occurances for an specific exponent. We break the method into 3 parts: 1- read weights 2- compute sum of product 3- update the occurance.

% 2) p-p stage: after having an array that shows how many times each exponent occures, the result sent to the host for requantization. after all operations related to the current input finishes, the new input will send to the memory to start a new iteration.

% 3) Clarify why we put 8b for counter -> put a table.

% 4) Execution flow: Put figx that shows encoder layers execution time. We make GCP between num of enc/dec modules, since Decs need to be executed by the num of tokens -> slower -> only one psch will assign to the encoder, Talk about pipelining different executions.

%% Methodology and Evaluation

% 5) For the second case study: in case of performing MMs, we have chose 3 different LLMs with different workloads... we applied exponential quantization, keep accuracy drop less than 1%. Table x, shows the accuracy of models and their baseline in FP32 precision.

% 6) For the second case study: Hardware baselines: TPU parameters, pLUTo: pLUTo 1: scaled version with different precision per each layer. pLUTO2: with all layers fixed at 4b precision.

%%% END GUIDELINE %%%

\section{Case study 2: LamaAccel}
Attention-based models rely heavily on matrix multiplications to process input data, which involve numerous multiply and accumulate operations. These operations demand substantial data movement between memory and compute units. While LUT-based Processing-using-Memory (PuM) techniques can efficiently handle multiplication operations with negligible area overhead and minimal modifications to commodity memory, the accumulation process within memory presents a significant challenge. It requires extensive intra-bank (or subarray) data movement to reorganize data for accumulation, which diminishes the efficiency of PuM operations~\cite{transpim}. Many previous proposals address this challenge by integrating computational units within memory banks to perform accumulation and support hardware acceleration for DNNs~\cite{transpim,fulcrum,sal-pim}.

We propose LamaAccel, an HBM-based accelerator for large language models (LLMs) that efficiently addresses the challenge of handling accumulation in dot-product operations without requiring significant modifications to the memory organization. LamaAccel employs the \textbf{exponential quantization} technique based on the DNA-TEQ approach to simplify the dot-product of activations and weights into the sum of four terms (see Equation~\ref{eqn:conv_extended}). Each term consists in a base $b$ to the power of different exponents ($int_{A_{i}}, int_{W_{i}}$, and $int_{A_{i}} + int_{W_{i}}$), as explained in Section~\ref{Exponential_quant}. The weights and activations are encoded using the format $\{S_{A/W}, int_{A/W}\}$, where $S_{A/W}$ denotes the sign of the real value prior to quantization, and $int_{A/W}$ is a 3- to 7-bit integer. For the rest of this section, when we talk about the precision of a given layer, we are referring to the precision of the exponent. The dot-product operations can be efficiently implemented by counting the frequency of each exponent's occurrence, including the addition of exponents as seen in term 1. This approach transforms the problem of memory-intensive dot-product operations into more memory-friendly addition and counting tasks, which are efficiently handled by our Lama technique.

% The exponent is represented with the same precision as the layer, using the format $(S_{AorW}, int_{AorW})$, where $S_{AorW}$ denotes the sign of the real value prior to quantization, and $int_{AorW}$ is a 3- to 7-bit integer.

We leverage Lama to efficiently perform addition and counting operations directly within memory. After counting all exponents involved in computing an output activation, the result is transferred to the HBM logic die for post-processing, where the data is re-quantized and sent back as a new activation for the next layer. This execution flow offers a key advantage: the number of unique exponents to be counted per output neuron is much smaller than the total number of dot-product partial results that would typically be accumulated, especially in higher-precision layers. For instance, in a 6-bit precision layer, only $2^6$ unique exponents have to be counted, significantly reducing the number of data transfers to the logic die and, hence, reducing the energy overhead.

\subsection{LLM Layer Mapping}
LamaAccel further leverages HBM's pseudo-channels to pipeline the execution of encoder and decoder blocks in large language models (LLMs), enabling simultaneous parallel processing of multiple inferences. Each pseudo-channel is dedicated to executing an individual encoder or decoder block, with each layer following an input-stationary dataflow for efficient execution. For LLMs used in text generation tasks that consist of both encoder and decoder blocks, the throughput of decoder blocks is lower due to the serialized token execution. To maintain a balanced pipeline, LamaAccel allocates a greater number of resources (pseudo-channels) to the decoder blocks, thereby accelerating their execution during inference.

The layers within an encoder or decoder block are processed sequentially, utilizing the resources (i.e., banks) within each pseudo-channel. For fully-connected (FC) layers whose weights are statically known, the weights are pre-stored in the banks. For matrix-multiplication operations inside the self-attention blocks employing the $K$ and $V$ matrices, LamaAccel writes those matrices into the banks as if they were the weight parameters of FC layers, to compute the score matrix and the final attention output respectively.

Within each bank, mat-level parallelism is employed to process computations related to different output neurons with the same activation concurrently. The degree of parallelism ($p$) and thus the number of output neurons processed in parallel varies depending on the precision of each layer and is adjusted dynamically.

\subsection{Data Layout}\label{data_layout_casestudy2}
LamaAccel employs the subarrays of a DRAM bank for three distinct purposes during the execution of a DNN layer. First, a group of subarrays (\textit{source subarrays}) is dedicated to storing the exponentially quantized weights of a layer. The exponent for each weight is pre-computed and stored statically. Each weight is stored in an 8-bit format as $(S_{W}, int_{W})$, with padding in the exponent. This fixed 8-bit format ensures compatibility with all possible precision levels across layers. Figure~\ref{fig:LamaAccel_datalayout}a (shown in purple) illustrates the organization of weights in one source subarray. Since each layer's execution follows an input-stationary dataflow, all the weights corresponding to an input index are stored in the same row of the subarray. The weights in each row are arranged so that in every column access, 16 weights for 16 different output neurons are fetched from all mats. For example, when activating row 0, corresponding to input activation 0, the first column access retrieves weights $W_{0,0}$ to $W_{0,15}$ corresponding to the first 16 output neurons.

\begin{figure}[t!]
\centering
\includegraphics[width=0.99\columnwidth]{figures/LamaAccel_datalayout_new.pdf}
%\vskip -0.15in
\caption{LamaAccel data layout.}
\label{fig:LamaAccel_datalayout}
\vskip -0.15in
\end{figure}

The second group of subarrays (\textit{compute subarrays} shown in Figure~\ref{fig:LamaAccel_datalayout}b in green) is dedicated to storing the LUT data required during the computation of the sum of exponents ($int_{A} + int_{W}$) corresponding to the first term in Equation~\ref{eqn:conv_extended}. All exponent sums are pre-computed and stored as 8-bit padded results. For layers with up to 6-bit exponents, each mat can accommodate one LUT. In each row of a mat, all possible cases for a given input activation exponent ($int_{A}$) are covered, ensuring the maximum degree of parallelism ($p=16$). However, for layers with 7-bit precision exponents, storing all possible values related to a single input activation in the same row requires two mats, reducing the degree of parallelism to $p=8$. This is because a 7-bit exponent can yield $2^7$ possible values for the sum of exponents, which must be stored as 8-bit results. Consequently, a row needs 1024 bits to store these values, requiring two mats to accommodate the data.

The third group of subarrays (\textit{counters subarrays} shown in Figure~\ref{fig:LamaAccel_datalayout}c in blue) is dedicated to tracking the occurrence of exponents for output neurons during the counting process. Each subarray updates the occurrences for $p$ output neurons simultaneously. Each output neuron ($ON_{i}$) requires three separate arrays of counters ($AC_{i}$) to track occurrences of the exponents ($int_{A_{i}}$, $int_{W_{i}}$, and $int_{A_{i}} + int_{W_{i}}$). The number of counters in each array varies depending on the layer's precision. For example, for 4-bit exponents, the array counter for $int_{A}$ contains $2^4$ counters. To determine the appropriate bitwidth for the counters, we analyzed the maximum occurrences of all possible exponents across the layers of our evaluated DNNs. This analysis showed that an 8-bit size per counter is sufficient to track the counts for each term without encountering numerical instability.

% The shorter version (no added details about the counters size):
% For example, for 4-bit exponents, the array counter for $int_{A}$ contains $2^4$ counters. Based on our dataset analysis, an 8-bit size per counter is sufficient to track the count for each term without encountering numerical instability.

% The row address is determined by the group of output neurons currently being processed in parallel. Within each row, the specific counter in $AC_{i}$ to be updated is selected via column access. We aim to keep the three $AC_{i}$ for each output neuron in the same row as much as possible to take advantage of the open-page policy, allowing the row to remain open until all three array counters are updated with the exponent occurrence according to current input exponent.

Each row of the counters subarray contains the array counters for different output neurons, allowing their occurrences to be updated simultaneously. Figure~\ref{fig:LamaAccel_datalayout}c illustrates the organization of the array counters in a subarray. For a layer precision of 3 and 4 bits, all array counters for an output neuron fit within a single mat, maintaining the parallelism degree of $p=16$, meaning that occurrences for the same term across 16 output neurons can be updated simultaneously. However, with 5-bit precision, two subarrays are required to fit all array counters while still preserving $p=16$. In the case of 6-bit precision, the array counters are distributed across two subarrays, providing a parallelism degree of $p=8$. For the highest precision of 7-bit, the parallelism degree decreases further to $p=4$, with the array counter placement following a similar organization to the 6-bit precision case. The total number of subarrays required for a given layer is determined by the number of output neurons and the parallelism level $p$.

\subsection{Execution Flow}\label{ExecutionFlow_cs2}
This section details how LamaAccel computes each term in Equation~\ref{eqn:conv_extended}. The bank structure in LamaAccel is similar to that in \textit{Case Study 1}, with two key enhancements. First, to accommodate the encoded input activation value {$S_{A_{i}}, int_{A_{i}}$} currently being processed for all output neurons, each bank is equipped with an 8-bit activation buffer. Second, each column address counter/latch is enhanced with an XNOR gate and a de-multiplexer, allowing it to increment or decrement the number of occurrences for each exponent based on the XNOR result of the input and weight signs. Additionally, the latch size of the column counters is extended from 5 bits (as used in commodity memories) to 8 bits, allowing it to store the number of occurrences of a given exponent. This extra size of the latch for each counter has been accounted for in the area overhead calculation of the column address counters, as discussed in Section~\ref{overhead}.

% This section details how LamaAccel computes each term in Equation~\ref{eqn:conv_extended}. The bank structure in LamaAccel is similar to that in \textit{Case Study 1}, with one key enhancement: each column address counter/latch is equipped with an XNOR gate and a de-multiplexer. This enhancement supports the more complex operations needed to count the occurrences of each exponent within the terms. The column address counter reads the current occurrence data from the temporary buffer and updates the count (incrementing or decrementing) based on the XNORed signs of the input and weights. The updated count is then written back to the appropriate row in the counters subarray through the de-multiplexer. In addition, each bank includes an 8-bit activation buffer, which stores the encoded input activation for computations involving all output neurons in the bank. The column address counters access this buffer to retrieve the sign $S_{A_{i}}$ and the exponent $int_{A_{i}}$, which are used to select and update the corresponding counter.

LamaAccel’s dataflow follows an \textit{input-stationary} approach, where the input activation remains stationary across the processing elements while computations for different output neurons are performed. In this scheme, for each iteration, a given input activation is broadcast to all banks within a pseudo-channel and stored in the activation latch, with each bank responsible for processing a subset of output neurons. By keeping the input activation constant and distributing the weight-related computations across banks, LamaAccel minimizes the need for data transfers and avoids the replication of weights across banks. This approach contrasts with weight-stationary dataflows, which replicate weights across multiple banks, leading to higher memory overhead. The memory controller in the logic die orchestrates the commands for output neuron computations in each bank based on the current input activation, ensuring efficient parallel execution across all banks in the pseudo-channel.

% In each iteration, the same input activation is broadcast to all banks within the pseudo-channel and stored temporarily in the activation buffer. While this approach involves replicating input movement across banks, it avoids replicating weights across all banks, which would occur in other dataflows such as weight-stationary. LamaAccel’s input-stationary approach significantly reduces memory replication overhead, leading to superior performance.

The execution flow is consistent across terms 1, 2 and 3, so we focus on detailing the process for the first term, which also involves the addition of exponents. The flow for terms 2 and 3 is similar but does not include the addition of exponents. The execution process in LamaAccel is divided into three key steps: (1) acquiring the weights associated with the output neurons being processed, (2) computing the sum of exponents, and (3) updating the number of occurrences of exponents. Each step operates with a different degree of parallelism ($p$), depending on the number of simultaneous operations that can be performed. This flexibility allows for efficient utilization of resources and optimized performance based on the specific requirements of each step.

\textbf{Step 1: Weight Acquisition.} In this stage, the memory controller issues an ACT command to activate the row $\#i$th in the \textit{source subarray}, based on the \textit{positional index} $i$ of the currently processed activation exponent $int_{A_{i}}$. This row contains 1024 encoded weights ($W_{i,0}$ to $W_{i,1023}$) in HBM2, corresponding to 1024 different output neurons for the current input activation. Following the ACT command, an internal column access (ICA), is executed to fetch 16 weights, which are then stored in the temporary buffer. This stage mirrors the execution steps (\circled{1} and \circled{2}) for bulk multiplication, illustrated in Figure~\ref{fig:case_study1}, and supports a degree of $p=16$, facilitating efficient weight retrieval. Once the initial 16 weights are fetched, the corresponding computations for these weights are performed. Subsequent weights can be fetched without issuing a new ACT command, as the row remains open. This allows for retrieving 16 weights with only a single ICA command each time, reducing the overhead and enhancing energy efficiency.

\textbf{Step 2: Computing the Sum of Exponents.} The process begins by performing a LUT activation for the row $\#int_{A_{i}}$th in the compute subarray, based on the \textit{value} of the activation exponent ($int_{A_{i}}$). Following this, the column address counters initiate the LUT retrieval operation using $p$ weight exponents ($int_{W}$) stored in the temporary buffer. Each mat outputs the sum of exponents, which is then stored in the temporary buffer. This continues until the sum for all weight exponents fetched in the first stage is fully computed. For layers with precision lower than 7-bit, LamaAccel supports the full parallelism degree ($p=16$), in which the mask logic is bypassed and all retrieved values are fetched during one ICA. However, in the case of 7-bit precision, $p$ is equal to 8 and so eight weight exponents are broadcast to all column address counters/latches -with each pair of consecutive column counters processing the same weight exponent- while the mask logic filters out invalid results. Overall, two ICAs are required to compute the sums for all stored weight exponents in the temporary buffer for 7-bit precision, and only one ICA for lower precisions. As previously mentioned, each iteration handles computations for a specific input activation across all output neurons in the layer. Since each row in the compute subarray contains all possible sum results for the current value of the activation exponent, the same row remains open during these iteration. This eliminates the need for repeated ACT commands when processing new sets of output neurons, requiring only ICA commands, which improves energy efficiency.

\begin{figure*}[t!]
\centering
\includegraphics[width=1.0\textwidth]{figures/case_study2_stage3.pdf}
%\vskip -0.15in
\caption{Counting the occurrence of exponents.}
\label{fig:step3}
\vskip -0.15in
\end{figure*}

\textbf{Step 3: Counting the occurrence of Exponents.} In this step, the number of exponent occurrences is updated based on the XNOR result of the signs of the weights and input activations, denoted as $S_{W_{i}}S_{A_{i}}$. If the XNOR result is 1, the sign multiplication is negative and the corresponding occurrence count is decremented; otherwise, it is incremented. As shown in Figure~\ref{fig:step3}, the process is as follows: \textit{First}, the memory controller issues the ACT command to activate the row in the counters subarray where the array counter ($AC_{A+W}$) for 16 output neurons is stored. The precision of each layer determines whether the selected array counters fit within one row ($p=16$) or span multiple rows ($p=8,4$). If $p<16$ the counting step is repeated to cover all 16 output neurons processed in the previous steps. The corresponding occurrences are fetched and stored into the temporary buffer through one ICA (\circled{1}). \textit{Second}, the fetched occurrences are loaded into the column address counter/latch for being updated in the next cycle, as shown in \circled{2}. \textit{Third}, the XNOR of $S_{W_{i}} and S_{A_{i}}$ is computed for each occurrence by retrieving the signs from the temporary and activation buffers. Then, the counters are updated based on the XNOR result (\circled{3}). \textit{Forth}, the updated occurrences are written back into the temporary buffer. Thanks to the open-page policy, a new ACT command is not required, as the row is kept open (\circled{4}).

% Flexibility for different precision to skip updating column counters for some precisions.
% Note: This part is a candidate to be remove for sending the paper to ISCA.
As described in previous sections, during a single ICA, each mat in the subarray provides an 8-bit value, selected through the column address counters/latches during the column selection process. During this stage, For layers where $p<16$, not all fetched occurrences from the 16 mats align with the target exponent occurrence counter in the array counter. This mismatch occurs because, depending on the layer's precision, the array counter for a given output neuron may span multiple mats. To address this, LamaAccel ensures that only column address counters with valid fetched occurrences perform the count up/down operation, while counters associated with non-valid occurrences remain in latch mode, preserving the fetched values without modification.

After completing the computations for term 1 of the equation, LamaAccel proceeds to compute terms 2 and 3. If the array counters for these terms reside in the same row as the array counter $A+W$, no new ACT command is required to open a new row in the counters subarray. Otherwise, new ACT commands are issued to access the corresponding array counters for these terms.

Once computations for all three terms for a given set of output neurons are complete, a PRECHARGE command closes the opened row in the counters subarray. This step is crucial for enabling the system to move on to the next set of output neurons, as each row in the counters subarray corresponds to a specific set of neurons. These steps are repeated within the current input activation iteration until all computations for all output neurons are completed. After processing all input activations for a layer, the values in the array counters are transferred to the logic die for post-processing, as detailed in Equation~\ref{eqn:conv_extended}, to produce the input activations for the next layer. Importantly, the next layer's computations can begin in a separate pseudo-channel containing the next layer's weights as soon as the first input activation is ready. This overlapping of post-processing and the next layer's computation helps reduce overall latency.

\begin{table}[t!]
%\scriptsize
\caption{Baseline accuracy vs accuracy after performing exponential quantization for the evaluated LLM models. The average bitwidth is the mean each layer's exponents.}
\label{t:LLMs}
\vskip -0.20in
\begin{center}
\resizebox{1.0\columnwidth}{!}{%
    \begin{tabular}{|c|c|c|c|c|c|}
    \hline
    \textbf{Network}       & \textbf{Task}        & \textbf{Baseline Acc}                  & \textbf{Quantized Acc} & \textbf{Avg bit} & \textbf{max SL}          \\
    \hline
      \multirow{2}{*}{\textbf{BERT-Base}}  &   SQuAD1    &  88.68\% (F1)                  & 88.13\%            &              6.45     & 384                      \\
    
        &   GLUE-SST2                      &  91.70\% (Exact match)                       & 90.82\%            &              3.48     & 128                      \\
    \hline
    \multirow{2}{*}{\textbf{BART-Large}}   &   CNN-DM    &  29.98\% (F1 Rouge L)          & 29.13\%            &              5.71     & 142                      \\
    
       &   MNLI                            &  90.17\%  (F1)                               & 89.34 \%           &              4.88     & 1024                     \\
    \hline
    \textbf{GPT-2-Small}  &   IMDB         &  94.46\%  (F1)                               & 94.16\%            &              6.03     & 1024                     \\
    \hline
    \end{tabular}%
}
\end{center}
\vskip -0.20in
\end{table}

\subsection{Methodology}
The hardware characteristics for LamaAccel are the same as those for the Lama evaluation platform, summarized in Table~\ref{t:configuration}. For LamaAccel we evaluate three widely-used attention models: BERT\cite{bert}, BART\cite{bart}, and GPT-2\cite{gpt-2}, across a range of representative NLP tasks, including text classification (IMDB~\cite{imdb} and MNLI~\cite{mnli}), question answering (SQuAD1.1~\cite{squad}), text summarization (CNN-DM~\cite{cnn-dm,cnn-dm2}), and sentimental analysis (GLUE-SST2~\cite{glue-sst2}). The BERT model corresponds to the base model consisting of 12 encoder blocks, while the BART model is the large version with both encoder and decoder blocks (12 encoders and 12 decoders). For GPT-2, we use the GPT-2 small model, which consists of 12 decoder blocks. All workloads are implemented in PyTorch using the HuggingFace Transformers library~\cite{huggingface}. 

The models are exponentially quantized, with quantization parameters and layer precision determined using the search algorithm described in \cite{DNA-TEQ}, based on Equation~\ref{eqn:conv_extended}. The quantization ensures less than 1\% accuracy loss compared to the baseline models without requiring fine-tuning. Table~\ref{t:LLMs} presents the baseline accuracy in FP32 and the accuracy after quantization. Additionally, for each model, various tasks are evaluated, and the average exponent bitwidth for each task is also reported. 

For performance and energy evaluations, we assume the workloads operate at their maximum sequence length, as indicated in Table~\ref{t:LLMs} under "Max SL". We employ the Lama simulator configured as described in the methodology of Case Study 1. The area overhead is consistent with that reported in Section~\ref{overhead}, with an additional area of $0.01 \ \text{mm}^2$ added to the HBM2 due to the extra components described in Section~\ref{ExecutionFlow_cs2}.

% ...assuming the same bitwidth and parameters obtained from the search algorithm.

For comparison with GPUs, we use as baseline a Nvidia RTX A6000. To measure GPU performance, we focus on the kernel execution time, excluding the data initialization overhead. Energy consumption is measured using Python bindings for {\fontfamily{cmss}\selectfont nvml} API~\cite{NVIDIA}. 

For comparison with a TPU, we extend the ScaleSim~\cite{scale} simulator to model a TPU-like architecture using the specifications from the Google Edge TPU Coral~\cite{tpu-edge}, which has a $64 \times 64$ systolic array, a chip area of $43.29 \ \text{mm}^2$, and a frequency of 480 MHz, similar to the added components in LamaAccel's banks. We assume an 8MB global on-chip SRAM~\cite{tpu-edge-sram} and 1GB of off-chip LPDDR4 memory~\cite{tpu-edge-ddr4}, with all layers set to 8-bit integers.

For comparison with previous PuM accelerators we implement pLUTo~\cite{pluto}, described in Section~\ref{pluto_background}, using the same dataflow and layer mapping as LamaAccel, with subarray-level parallelism set to 16 to match LamaAccel's bank-level parallelism. Given that pLUTo supports only 4-bit precision, we assume that all layers in the LLM workloads are uniformly quantized to 4-bit precision for a fair comparison. Although this uniform quantization does not guarantee that accuracy drops will remain below 1\% in the evaluated workloads, we overlook this limitation to maintain consistency with pLUTo's constraints.

\subsection{Evaluation}
Figure~\ref{fig:case_study2_evaluation_tpu} illustrates the speedup and normalized energy savings of a PuM baseline (pLUTo) and LamaAccel over TPU for three LLM workloads across different NLP tasks. LamaAccel consistently achieves significant speedups, from $3.4\times$ (BERT for SQuAD1) to $4.7\times$ (BERT for SST2), with an average improvement of $4.1\times$. Compared to pLUTo, LamaAccel delivers an average speedup of $1.7\times$ across all workloads. Notably, since pLUTo only supports up to 4-bit operand precision, all workloads in pLUTo are executed at 4-bit precision. In contrast, LamaAccel processes most workloads at higher average precision, yet still outperforms pLUTo in all cases, demonstrating better speedups even with higher precision.

\begin{figure}
    \centering
    \begin{subfigure}{\columnwidth}
        \centering
        \includegraphics[height=2.8cm, width=8.1cm]{figures/TPU_speedup.pdf}
    \end{subfigure}
    %\vskip +0.03in
    \begin{subfigure}{\columnwidth}
        \centering
        \includegraphics[height=2.9cm, width=8.2cm]{figures/TPU_normalized_energy.pdf}
        %\vskip -0.15in
        \label{fig:third}
    \end{subfigure}
    \caption{LamaAccel and pLUTo speedup and energy savings, normalized to TPU.}
    \vskip -0.20in
    \label{fig:case_study2_evaluation_tpu}
\end{figure}

For tasks performed with lower average bitwidth, the speedup achieved is generally higher. In all tasks except for the BART CNN text summarization, each pseudo-channel is assigned to a single encoder or decoder block. In BERT for SST2, with the lowest average bitwidth of 3.4, the speedup is higher than the rest of the workloads. For BART's text summarization on the CNN dataset, decoder blocks generate tokens sequentially, one token at a time, making the decoder blocks a bottleneck during inference. The bottleneck arises because each step depends on generating the previous token, which slows down the process. To address this, we allocate 2 pseudo-channels for the encoder blocks and the remaining resources for faster execution of the decoder blocks, ultimately achieving a speedup of $3.6\times$.

Energy savings in LamaAccel are closely linked to the average precision of each LLM. As the precision of the model layers becomes lower, LamaAccel shows greater energy savings compared to the TPU baseline. BERT on the GLUE-SST2 task, which has the lowest average precision of 3.48 (see Table~\ref{t:LLMs}), achieves the highest energy saving of $9.2\times$. Conversely, BERT on the SQuAD task, with the highest average precision, results in a lower energy saving of $4.4\times$ compared to TPU. LamaAccel reduces the higher precision overhead by minimizing the number of ACT commands and exploiting the open-page policy, allowing for simultaneous operations across multiple output neurons. In the weight acquisition and sum computation stages, it reuses the already activated row, requiring only additional read commands. Since the energy cost of column accesses is much lower than that of new ACT commands, LamaAccel exhibits better energy efficiency than pLUTo. Additionally, during the counting phase, the mapping of array counters is designed to ensure that even with increased precision, the system maintains maximum efficiency in processing output neurons.

In Figure~\ref{fig:case_study2_evaluation_gpu} LamaAccel performance per area is shown relative to the GPU baseline. The NVIDIA RTX A6000 features a die size of $628 \ \text{mm}^2$ on an $8 \ \text{nm}$ process, whereas LamaAccel occupies $53.15 \ \text{mm}^2$ on a $21 \ \text{nm}$ node, equivalent to the area of a single HBM2 stack. For this comparison, LamaAccel's technology node is not scaled to the GPUs, inherently favoring the GPU. Despite this, LamaAccel achieves an average of $7.2\times$ higher performance per area across workloads. In energy efficiency, LamaAccel outperforms the GPU baseline by $6.1\times$ to $19.2\times$, with higher savings in tasks using lower precision. While LamaAccel's inference throughput is lower than that of the GPU, primarily due to the limited resources in a single HBM2 stack compared to a high-end GPU, its scalable architecture allows throughput to scale linearly with additional HBM2 stacks, bringing it closer to the performance of high-end GPUs.

\begin{figure}[t!]
\centering
\includegraphics[height=3.5cm, width=8.5cm]{figures/GPU_energy_speedup.pdf}
%\vskip -0.15in
\caption{Speedup and energy savings of LamaAccel normalized to the GPU. In the case of energy savings, higher is better.}
\label{fig:case_study2_evaluation_gpu}
\vskip -0.15in
\end{figure}

\section{Conclusion }
This paper introduces the Latent Radiance Field (LRF), which to our knowledge, is the first work to construct radiance field representations directly in the 2D latent space for 3D reconstruction. We present a novel framework for incorporating 3D awareness into 2D representation learning, featuring a correspondence-aware autoencoding method and a VAE-Radiance Field (VAE-RF) alignment strategy to bridge the domain gap between the 2D latent space and the natural 3D space, thereby significantly enhancing the visual quality of our LRF.
Future work will focus on incorporating our method with more compact 3D representations, efficient NVS, few-shot NVS in latent space, as well as exploring its application with potential 3D latent diffusion models.

\section{Acknowledgement}\label{acknowledgement}
This work has been supported by the CoCoUnit ERC Advanced Grant of the EU’s Horizon 2020 program (grant No 833057), the Spanish State Research Agency (MCIN/AEI) under grant PID2020-113172RB-I00, the Catalan Agency for University and Research (AGAUR) under grant 2021SGR00383, and the ICREA Academia program.
%%%%%%% -- PAPER CONTENT ENDS -- %%%%%%%%

%%%%%%%%% -- BIB STYLE AND FILE -- %%%%%%%%
\bibliographystyle{IEEEtranS}
\bibliography{references}
%%%%%%%%%%%%%%%%%%%%%%%%%%%%%%%%%%%%

\end{document}
