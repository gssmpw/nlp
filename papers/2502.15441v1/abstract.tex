\begin{abstract}

Declarative specifications have a vital role to play in developing
safe and dependable software systems.  Writing specifications
correctly, however, remains particularly challenging.  This paper
presents a controlled experiment on using large language models (LLMs)
to write declarative formulas in the well-known language Alloy.  Our
use of LLMs is three-fold.  One, we employ LLMs to write complete
Alloy formulas from given natural language descriptions (in English).
Two, we employ LLMs to create alternative but equivalent formulas in
Alloy with respect to given Alloy formulas.  Three, we employ LLMs to
complete sketches of Alloy formulas and populate the holes in the
sketches by synthesizing Alloy expressions and operators so that the
completed formulas accurately represent the desired properties (that
are given in natural language).  We conduct the experimental
evaluation using \NumSubjects{} well-studied subject specifications
and employ two popular LLMs, namely ChatGPT and DeepSeek.  The
experimental results show that the LLMs generally perform well in
synthesizing complete Alloy formulas from input properties given in
natural language or in Alloy, and are able to enumerate multiple
unique solutions.  Moreover, the LLMs are also successful at
completing given sketches of Alloy formulas with respect to natural
language descriptions of desired properties (without requiring test
cases).  We believe LLMs offer a very exciting advance in our ability
to write specifications, and can help make specifications take a
pivotal role in software development and enhance our ability to build
robust software.

\keywords{Alloy \and declarative programming \and specifications \and
  LLMs \and ChatGPT \and DeepSeek \and SAT.}

\end{abstract}

% LocalWords:  LLMs ChatGPT DeepSeek
