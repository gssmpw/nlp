\begin{table*}[!t]
\centering
\begin{tabular}{l|c|c}
\hline
\Intro{Property} & \Intro{OpenAI o3-mini} & \Intro{DeepSeek R1}\\
\hline
Antisymmetric & \cmark & \cmark \\
\hline
Circular & \xmark & \cmark \\
\hline
Connex & \cmark & \cmark \\
\hline
Cycle & \cmark & \cmark \\
\hline
DAG & \cmark & \cmark \\
\hline
Function & \cmark & \xmark \\
\hline
Functional & \cmark & \cmark \\
\hline
Irreflexive & \cmark & \cmark \\
\hline
Reflexive & \cmark & \cmark \\
\hline
Symmetric & \cmark & \cmark \\
\hline
Transitive & \cmark & \cmark \\
\hline
\end{tabular}
\vspace*{2ex}
\caption{Alloy sketching results.  For each property and each LLM, the
  table shows whether the LLM successfully completes the sketching
  task.  For the property \Intro{Circular}, OpenAI o3-mini completed
  the input sketch with an Alloy formula that had a syntax error.
  Likewise, for the property \Intro{Function}, DeepSeek R1 completed
  the input sketch with an Alloy formula that had a syntax error.  For
  both these cases (where the LLM failed in its first attempt), we
  created a new query to inform it of the syntax-error and asked it to
  try again.  In each case the LLM succeeded to correctly complete the
  input sketch on the second attempt.}
\label{tab:sketching-results}
\vspace*{-2ex}
\end{table*}
