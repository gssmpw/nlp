\section{Conclusion}
\label{sec:conclusion}

This paper presented a three-fold use of large language models (LLMs)
in writing declarative formulas in the Alloy language.  One, LLMs were
employed to write complete Alloy formulas from given natural language
descriptions (in English).  Two, LLMs were employed to create
alternative but equivalent formulas in Alloy with respect to the given
Alloy formulas.  Three, LLMs were employed to complete sketches of
Alloy formulas and populate the holes in the sketches by synthesizing
Alloy expressions and operators so that the completed formulas
accurately represent the desired properties (given in natural
language).  An experimental evaluation using \NumSubjects{}
well-studied subject specifications and two popular LLMs, namely
ChatGPT and DeepSeek, were conducted.  A key aspect of the evaluation
was that for each synthesis problem (from English to Alloy and from
Alloy to Alloy), the LLMs were asked to generate multiple equivalent
but non-identical Alloy formulas as solutions, thus providing a deeper
look into how well the LLMs handle the semantic and syntactic
intricacies of the Alloy language.  The experimental results showed
that the LLMs generally performed quite well on synthesizing complete
Alloy formulas from input specifications given in natural language or
in Alloy, and were generally able to enumerate multiple unique
solutions.  Moreover, the LLMs were also successful at completing
given sketches of Alloy formulas.

LLMs hold much promise in enabling us to utilize the power of
specifications in building safe and reliable software systems.  Future
work will look at further utilizing LLMs in writing specifications,
and making LLMs an integral part of the process of writing
specifications -- just like they are today for writing
implementations.

% LocalWords:  LLMs ChatGPT DeepSeek
