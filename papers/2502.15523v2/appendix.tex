\section*{Appendix}
\RealtionNonRobust*
\begin{proof}
	%
	We prove the two points separately.
	%
	\begin{enumerate}
		\item Let $p'= (1-\sqrt{\delta})p + \sqrt{\delta} r $ for any contract $p \in \Rset$.
		%
		We start by observing that the following inequality holds:
		%
		\begin{equation*}	
			\sum_{\omega \in \Omega}F_{ a(p),\omega} p_\omega - c_{ a(p)} \ge \sum_{\omega \in \Omega}F_{ a^\delta(p'),\omega} p_\omega - c_{ a^\delta(p')},
		\end{equation*}
		%
		because $a(p) \in \mathcal{A}(p)$. Furthermore, we have:
		%
		\begin{equation*}	
			\sum_{\omega \in \Omega}F_{ a^\delta(p'),\omega} p_\omega' - c_{ a^\delta(p')} > \max_{a' \in \mathcal{A}} \sum_{\omega \in \Omega}F_{ a',\omega} p_\omega - c_{ a'} -\delta \ge \sum_{\omega \in \Omega}F_{ a(p),\omega} p_\omega' - c_{ a(p)} - \delta,
		\end{equation*}
		%
		thanks to the definition of $\A^\delta(p')$.
		%
		Then, by employing the two above inequalities and the definition of $p'$, we have:
		%
		\begin{align*}	
			\delta & \ge \sum_{\omega \in \Omega}F_{ a(p),\omega} p_\omega' - c_{ a(p)} - \left(\sum_{\omega \in \Omega}F_{ a^\delta(p'),\omega} p_\omega' - c_{ a^\delta(p')} \right)\\
			& =  \sum_{\omega \in \Omega}F_{ a(p),\omega} p_\omega - c_{ a(p)} - \left( \sum_{\omega \in \Omega}F_{ a^\delta(p'),\omega} p_\omega - c_{ a^\delta(p')} \right)  + \sqrt \delta \left( \sum_{\omega \in \Omega}\left(F_{ a(p),\omega} - F_{ a^\delta(p),\omega}\right) (r_\omega - p_\omega) \right) \\ 
			& \ge \sqrt \delta \left( \sum_{\omega \in \Omega}\left(F_{ a(p),\omega} - F_{ a^\delta(p),\omega}\right) (r_\omega - p_\omega) \right). 
		\end{align*}
		%
		Thus, by rearranging the latter inequality we get:
		\begin{equation}\label{eq:sqrtdelta}
			\sqrt \delta  \ge \left( \sum_{\omega \in \Omega}\left(F_{ a(p),\omega} - F_{ a^\delta(p),\omega}\right) (r_\omega - p_\omega) \right). 
		\end{equation}
		%
		Finally, we can show that:
		%
		\begin{align*}	
			u^\sfP(p,a(p))  - u^\sfP(p',a^{\delta}(p'))  &= \sum_{\omega \in \Omega}F_{ a(p),\omega} (r_\omega - p_\omega ) - \left(\sum_{\omega \in \Omega}F_{ a^\delta(p'),\omega}(r_\omega - p_\omega' )  \right)\\
			&= \sum_{\omega \in \Omega}F_{ a(p),\omega} (r_\omega - p_\omega ) - \left(1 -\sqrt{\delta}\right) \left(\sum_{\omega \in \Omega}F_{ a^\delta(p'),\omega}(r_\omega - p_\omega)  \right)\\
			&\le \left(1 -\sqrt{\delta}\right) \left(\sum_{\omega \in \Omega} (F_{ a(p),\omega} - F_{ a^\delta(p'),\omega})(r_\omega - p_\omega)  \right) + \sqrt{\delta} \\
			&\le 2 \sqrt{\delta} - \delta,
		\end{align*}
		%
		where the second equality holds thanks to the definition of $p'$, the first inequality because $u^\sfP(p,a(p)) \le 1$ for each $p \in \mathbb{R}^{m}_{+}$ and the second inequality because of \Cref{eq:sqrtdelta}.
		%
		Finally, let $p \in \mathbb{R}$ be an optimal (non-robust) contract, then we have:
		%
		\begin{align*}	
			\textnormal{OPT} - 2 \sqrt{\delta} + \delta \le  u^\sfP(p',a^{\delta}(p')) \le \textnormal{OPT}(\delta) ,
		\end{align*}
		%
		concluding the first part of the proof. 
		%
		\item We split the proof into two parts.
		\begin{enumerate}
			\item if $\textnormal{OPT}(\delta) = 0$. Then, we trivially have :
			\begin{equation*}
				0=\textnormal{OPT}(\delta)\le \max\left(0, \, \max_{a \in \mathcal{A}} \sum_{\omega \in \Omega}F_{a,\omega}r_\omega - c_a - \delta \right) = \max\left(0, \, \textnormal{SW} - \delta \right)
			\end{equation*}
			\item if $\textnormal{OPT}(\delta) > 0$.
			%
			We define $p^\star \in \Rset$ as an optimal $\delta$-robust contract.
			%
			Then, we notice that $a_{1} \not \in \mathcal{A}^{\delta}(p^\star)$, where $a_1$ is the opt-out action.
			%
			Indeed, if $a_{1} \in \mathcal{A}^{\delta}(p^\star)$, then the agent may select the action $a_1$ as a $\delta$-robust best-response, which provides zero or negative utility to the principal and contradicts the fact that $\textnormal{OPT}(\delta) > 0$.
			%
			Therefore, in an optimal robust contract $p^\star$, we must have:
			%
			\begin{align*}
				\sum_{\omega \in \Omega} F_{a',\omega} p_\omega^\star - c_{a'}  &\ge  \sum_{\omega \in \Omega}  F_{a_1,\omega} p_\omega^\star - c_{a_1} + \delta\ge  \sum_{\omega \in \Omega}  F_{a_1,\omega} p_\omega^\star + \delta \ge \delta,
			\end{align*}
			%
			for some $a' \in \mathcal{A}^{\delta}(p^\star)$.
			%
			Thus, we have:
			%
			\begin{align*}
				\textnormal{OPT}(\delta)
				& = \sum_{\omega \in \Omega} F_{a^\delta(p),\omega} (r_\omega -p_\omega^\star ) \\
				& \le \sum_{\omega \in \Omega} F_{a',\omega} (r_\omega -p_\omega^\star ) \\
				& \le \sum_{\omega \in \Omega} F_{a',\omega} r_\omega - c_{a'}  - \delta \\
				& \le  \max_{a \in \mathcal{A}} \sum_{\omega \in \Omega}F_{a,\omega}r_\omega - c_a - \delta \\
				& \le \max\left(0, \, \max_{a \in \mathcal{A}} \sum_{\omega \in \Omega}F_{a,\omega}r_\omega - c_a - \delta \right) = \max\left(0, \, \textnormal{SW} - \delta \right).
			\end{align*}
			The first inequality above follows from the fact that $a' \in \mathcal{A}^{\delta}(p^\star)$, while the second inequality holds due to the previous observation.
		\end{enumerate}
	\end{enumerate}
	The two above points conclude the proof.
\end{proof}
%
\RealtionNonRobustTwo*
\begin{proof}
	We prove the two points separately.
	%
	\begin{enumerate}
		\item We consider an instance parametrized by $\delta > 0$, with $|\Omega| = 2$ and $|\mathcal{A}| = 2n + 1$ for some $n \in \mathbb{N}$, to be defined below. We let:
		%
		$$\kappa  = \min \Bigg\{i\in \mathbb{N}_{>0} \, | \, \sqrt{\delta } < \frac{i- 1}{i} \Bigg\}.$$
		%
		Furthermore, we introduce the following decreasing sequence:
		%
		$$\gamma_i = \begin{cases}
			\vspace{1mm}
			\frac{i}{i-1} \,\,\ & i = \kappa, \dots, n \\
			\frac{2n+1-i}{2n+2-i} \,\,\ &i = n+2, \dots, 2n,\\
		\end{cases}$$
		%
		with $\gamma_{n+1}=1$ and $\gamma_{2n+1}=0$, where $n \in \mathbb{N}$ is such that $n > \kappa$.
		%
		The distributions over the set of outcomes for the different actions and their corresponding costs are given by:
		$$ \begin{cases}
			F_{a_i,\omega_1}=0 \,\,& c_{a_i}=0, \,\, i=1, \dots, \kappa-1.   \\
			F_{a_i,\omega_1}=1-\gamma_i \sqrt{\delta}\,\,& c_{a_i}=0, \,\, i=\kappa, \dots, n . \\
			F_{a_{2n+1,\omega_1}}=1 \,\,& c_{a_{2n+1}}=0.
		\end{cases}$$
		%
		and the principal's reward is $r=(1,0)$.
		%
		Notice that the distribution over outcomes of the different actions are always well defined because of the definition of $\kappa>0$.
		%
		Furthermore, since all the agent's actions $a_i$ with $i < k$ are coincident, we assume for the sake of presentation that the agent always selects $a_1$.
		%
		It is easy to verify that the value of an optimal (non-robust) contract is $\textnormal{OPT}=1$ since the agent breaks ties optimistically and $c_{2n+1}=0$.	
		
		%We observe that an optimal $\delta$-robust contract $p^\star \in \Rset$ satisfies $p^\star = (0, \alpha)$ for some $\alpha \in \mathbb{R}_{+}$, and therefore $\textnormal{OPT}_{\textnormal{LIN}}(\delta) = \textnormal{OPT}(\delta)$.
		With a similar argument to the one proposed by~\cite{dutting2024algorithmic} in Proposition 3.9, an optimal $\delta$-robust contract is such that $p^\star = (0, \alpha)$ for some $\alpha \in \mathbb{R}_{+}$.
		%
		Consequently, in the rest of the proof, we focus on determining the maximum utility achievable in a linear, $\delta$-robust contract. 
		
		%We also observe that all actions $a_i$ with $1 \le i < \kappa$ coincide with $a_1$.
		%
		%Thus, since the agent selects the action with the lowest index by assumption when indifferent among multiple $\delta$-best responses, all actions $a_i$ with $1 < i < \kappa$ are never played by the agent.
		
		
		For every $a_i \in \mathcal{A}$, with $\kappa<i \le 2n+1$, if $\alpha \in \mathbb{R}_{+}$ is such that $a^\delta(\alpha r)= a_i$, then the two following conditions hold.
		\begin{itemize}
			\item All the actions $a_j \not \in \mathcal{A}^\delta(\alpha r)$ for each $j < i$.
			%  
			This is because $u^\sfP(\alpha, a_j) \le u^\sfP(\alpha, a_i)$ for each $\alpha \in \mathbb{R}_{+}$ since $\{\gamma_i\}_{i \in [2n+1]}$ is a decreasing sequence.
			%
			Thus, the latter observation implies that:
			\begin{align*}
				u^\sfA(\alpha, a_{j}) 
				\le u^\sfA(\alpha, a_{i-1}) & = \alpha R_{a_{i-1}} - c_{a_{i-1}}\\
				& \le \max_{i \in [2n+1]} u^\sfA(\alpha, a_{i}) -\delta\\
				& = u^\sfA(\alpha, a_{{2n+1}}) -\delta  \\
				&= \alpha R_{a_{2n+1}} -c_{a_{{2n+1}}} - \delta \\
				& = \alpha -\delta.
			\end{align*}
			%
			Therefore, we have that: 
			%
			\begin{align*}
				\alpha R_{i-1}\le \alpha - \delta, 
			\end{align*}
			%
			and, thus, $\alpha \ge {\sqrt{\delta}} / {\gamma_{i-1}}$. %if $i > \kappa$, and $\alpha_\kappa \ge \delta$.
			
			\item The action $a_i \in \A^\delta(\alpha r)$. Thus, 
			$$ \alpha R_{a_i} - c_{a_i} > \max_{i \in [2n+1]} u^\sfA(\alpha, a_{i}) -\delta = \alpha -\delta,
			$$
			which implies that $\alpha < {\sqrt{\delta}} / {\gamma_{i}}$.
		\end{itemize}
		%
		The two observation above shows that $a_i$, with $i > \kappa$, is a $\delta$-best response for all the values of $\alpha$ such that:
		%
		$${\sqrt{\delta}} / {\gamma_{i-1}} \le \alpha < {\sqrt{\delta}} / {\gamma_{i}}.$$
		%
		Thus, the smallest value of $\alpha \in \mathbb{R}_{+}$ such that $a^\delta(\alpha r) = a_i$ is $\alpha_i \coloneqq {\sqrt{\delta}} / {\gamma_{i-1}}$, when $i > \kappa$.
		%
		With the same argument above, it is possible to show that $a_1=a^\delta(\alpha r)$ for all $\alpha \in [\alpha_1,\alpha_\kappa)$ and $a_\kappa=a^\delta(\alpha r)$ for all $\alpha \in [\alpha_\kappa, \alpha_{\kappa+1})$, with $\alpha_1=0$ and $\alpha_\kappa=\delta$.
		
		We also observe that the principal's utility in each $\alpha_i$, with $i \in \{\kappa+1, \dots, 2n+1\}$, is such that:
		% 
		\begin{align*}
			u^\sfP(\alpha_i, a_{i}) =  (1-\alpha_i) R_{a_i} & = \left(1 - \frac{\sqrt{\delta}}{\gamma_{i-1}}\right) \left(1-\gamma_i \sqrt{\delta} \right)\\
			& = \left(1 - \frac{\sqrt{\delta}}{\gamma_{i-1}}\right) \left(1-\gamma_i \sqrt{\delta} \right)\\
			& = \left( 1- \left(\gamma_i  + \frac{1}{\gamma_{i-1}}\right)\right) \sqrt{\delta} +  \frac{\gamma_{i}}{\gamma_{i-1}} \delta.
		\end{align*}
		%
		Therefore, using the latter formula, we have that the following holds.
		%
		\begin{itemize}
			\item If $i=\kappa +1,\dots,n$, we have:
			%
			$$u(\alpha_i) \le 1- 2\sqrt{\delta} + \frac{\gamma_i}{\gamma_{i-1}}\delta \le  1 - 2\sqrt{\delta} + \delta, $$
			%
			since $\{\gamma_i\}_{i \ge \kappa}$ is  a decreasing sequence.
			
			\item If $i=n+1,n+2$, we have:
			%
			$$u(\alpha_{i}) \le 1- \left(1 + \frac{n-1}{n} \right)\sqrt{\delta} +  \frac{n-1}{n} \delta \le 1- 2\sqrt{\delta} + \delta + \frac{\sqrt{\delta}}{n}.$$
			
			\item If $i=n+3, \dots, 2n+1$, we have:
			%
			$$u(\alpha_{i}) \le 1- 2\sqrt{\delta} + \frac{\gamma_i}{\gamma_{i-1}}\delta \le 1- 2\sqrt{\delta} + \delta ,$$
			%
			since $\{\gamma_i\}_{i \ge \kappa}$ is  a decreasing sequence.
		\end{itemize}
		
		We also observe that the principal's utility in $\alpha_{\kappa}$ is such that:
		%
		\begin{align*}
			u(\alpha_{\kappa}) = (1-\delta)(1- \gamma_{\kappa} \sqrt{\delta} ).
		\end{align*}
		%
		since $\gamma_k>1$ and $\delta>0$.
		%
		Thanks to the definition of $\kappa$, with a simple calculation, it possible to show that:
		%
		$$\kappa = \begin{cases}
			\vspace{1mm}
			\left\lceil\frac{1}{1-\sqrt{\delta}}\right\rceil \,\,\ & \textnormal{if} \,\,\, \frac{1}{1-\sqrt{\delta}} \not \in \mathbb{N}\\
			\frac{1}{1-\sqrt{\delta}} +1 \,\,\ & \textnormal{if} \,\,\, \frac{1}{1-\sqrt{\delta}} \in \mathbb{N}.\\
		\end{cases}$$
		%
		Thus, when ${1}/{(1-\sqrt{\delta})}\not \in \mathbb{N}$, we have:
		%
		\begin{align*}
			u(\alpha_{\kappa}) &= (1-\delta)(1- \gamma_{\kappa} \sqrt{\delta} )\\
			&= (1-\delta)\left(1- 
			\frac{\left\lceil\frac{1}{1-\sqrt{\delta}}\right\rceil }{ \left\lceil\frac{1}{1-\sqrt{\delta}}\right\rceil -1} \sqrt{\delta}\right)\\
			&\le (1-\delta)\left(1-2\sqrt{\delta}+\delta\right).
		\end{align*}
		%   
		Similarly, it is possible to show that the same relation holds even when ${1}/{(1-\sqrt{\delta})} \in \mathbb{N}$.
		%
		Finally, by putting all together, we have that:
		%
		\begin{align*}
			\textnormal{OPT}(\delta) -\textnormal{OPT}_{\textnormal{LB}}(\delta) 
			% & \le \sum_{i \in [2n+1]} u(\alpha, a^\delta(\alpha r))  \mathbb{I}_{\{a_i = a^\delta(\alpha r) \}} -\textnormal{OPT}_{\textnormal{LB}}(\delta) \\
			& \le \max_{i \in [2n+1]} u(\alpha_i) - (1 - 2 \sqrt{\delta} + \delta )\\
			&= u(\alpha_{n+1}) - (1 - 2 \sqrt{\delta} + \delta ) \le \frac{\sqrt{\delta}}{n},
		\end{align*}
		concluding the first part of the proof.
		%
		\item We consider an instance with $|\Omega|=|\mathcal{A}|=2$ and $r=(0,1)$.
		%
		The distributions over the set of outcomes for the different actions and their corresponding costs are given by:
		%
		$$ \begin{cases}
			F_{a_1}=(1,0) \,\, &c_{a_1}=0\\
			F_{a_2}=(0,1) \,\, &c_{a_2}=0.
		\end{cases}$$
		%
		% $F_{a_1}=(1,0)$ and $F_{a_2}=(0,1)$ with $c_{a_1}=c_{a_2}=0$ and $r = (0,1)$.
		%
		It is easy to verify that $\textnormal{SW}=R_{a_2}-c_{a_2}=1$. 
		%
		Furthermore, with a similar argument to the one proposed by~\cite{dutting2024algorithmic} in Proposition 3.9, an optimal $\delta$-robust contract is such that $p^\star = (0, \alpha)$ for some $\alpha \in \mathbb{R}_{+}$.
		%
		
		We show that $\textnormal{OPT}(\delta) = 1-\delta.$ 
		%
		Let $a_1 \in \mathcal{A}$ be the opt-out action.
		%
		We observe that $a_1 \not \in {A}^{\delta}(\alpha r)$ for all the $\alpha \in \mathbb{R}_{+}$ that satisfy:
		%
		\begin{equation*}
			\alpha R_{a_2} - c_{a_2} = \alpha \ge \alpha R_{a_1} - c_{a_1} + \delta = \delta.
		\end{equation*}
		%
		Thus, $a_1 \not \in {A}^{\delta}(\alpha r)$ for all $\alpha \ge \delta$.
		%
		Therefore, the smallest $\alpha \in \mathbb{R}_{+}$ such that the agent selects the action $a_2$ coincides with $\delta > 0$.
		%
		Thus, the largest utility the principal can achieve satisfies $\textnormal{OPT}(\delta) = 1 - \delta = \textnormal{SW} - \delta$.
	\end{enumerate}
	The two points above conclude the proof.
\end{proof}
\RealtionNonRobustThree*
\begin{proof}
	We prove the claims separately.
	\begin{itemize}
		%
		\item We first show that $\textnormal{OPT}(\delta)$ is a continuous function. 
		%
		Let $p^\star \in \Rset$  be a $\delta$-robust and $p' = (1-\sqrt{\epsilon})p^\star + \sqrt{\epsilon} r$. Then, by Lemma~\ref{lem:epsilonconvert}, we have:
		%
		\begin{equation*}
			\textnormal{OPT}(\delta+\epsilon) \ge u^{\textnormal{P}}(p',a^{\delta+\epsilon}(p')) \ge u^{\textnormal{P}}(p^\star,a^{\delta} (p^\star) ) - 2 \sqrt{\epsilon} + \epsilon = \textnormal{OPT}(\delta) - 2 \sqrt{\epsilon} + \epsilon.
		\end{equation*}
		%
		Thus, rewriting the latter inequality and taking the limit, we have that:
		%
		\begin{equation*}
			0= \lim_{\epsilon \to 0} (2 \sqrt{\epsilon} - \epsilon) \ge \lim_{\epsilon \to 0} \textnormal{OPT}(\delta) - \textnormal{OPT}(\delta+\epsilon).
		\end{equation*} 
		%
		Let $p^\star \in \Rset$  be a $\delta'$-robust and $p' = (1-\sqrt{\epsilon})p^\star + \sqrt{\epsilon} r$ with $\delta'=\delta-\epsilon$. Then, by Lemma~\ref{lem:epsilonconvert}, we have: 
		%
		\begin{equation*}
			\textnormal{OPT}(\delta)= \textnormal{OPT}(\delta'+\epsilon) \ge u^{\textnormal{P}}(p',a^{\delta'+\epsilon}(p')) \ge u^{\textnormal{P}}(p^\star,a^{\delta'} (p^\star) ) - 2 \sqrt{\epsilon} + \epsilon = \textnormal{OPT}(\delta-\epsilon) - 2 \sqrt{\epsilon} + \epsilon.
		\end{equation*}
		%
		Thus, by taking the limit, we have that:
		\begin{equation*}
			0= \lim_{\epsilon \to 0} (2 \sqrt{\epsilon} - \epsilon) \ge \lim_{\epsilon \to 0} \textnormal{OPT}(\delta-\epsilon) - \textnormal{OPT}(\delta).
		\end{equation*} 
		%
		%
		Since the latter considerations hold for every $\delta \in (0,1)$, we can easily show that $\textnormal{OPT}(\delta)$ is continuous in $\delta \in (0,1)$.
		%
		\item We now prove that $\textnormal{OPT}(\delta)$ is non-increasing. 
		%
		Let $\delta,\delta' \in (0,1)$ such that $\delta'<\delta$. Furthermore, let $p^\star \in \Rset$ be an optimal $\delta$-robust contract. Therefore, we have:
		%
		\begin{equation*}
			\textnormal{OPT}(\delta)=u(a^\delta(p^\star), p^\star ) \le u(a^{\delta'}(p^\star), p^\star ) \le \textnormal{OPT}(\delta'),
		\end{equation*}
		%
		since, for each $p \in \Rset$, we have ${A}^{\delta'}(p) \subseteq  {A}^{\delta}(p)$.
		%
		\item By Proposition~\ref{pro:properties}, we have:
		\begin{equation*}
			\lim_{\delta \to 0^+}  \textnormal{OPT}(\delta) \ge \lim_{\delta \to 0^+}\textnormal{OPT} - 2 \sqrt{\delta} + \delta =\textnormal{OPT},
		\end{equation*}
		and,
		\begin{equation*}
			\lim_{\delta \to 1^-}  \textnormal{OPT}(\delta) \le \lim_{\delta \to 1^-}1-\delta =0.
		\end{equation*}
	\end{itemize}
	%
	The two points above conclude the proof.
\end{proof}
%
\epsilonconvert*
\begin{proof}
	%
	For the sake of the presentation, we avoid the dependence on the type $\theta \in \Theta$.
	% 
	We split the proof in two cases: 
	\begin{enumerate}
		%\item if $a^{\delta}(p') =a^{\delta+\epsilon} (p)$. It easily follows because of the definition of $p'$.
		\item If $a^{\delta+\epsilon} (p') \not \in {A}^{\delta}(p)$, then $a^{\delta+\epsilon} (p')$ is not a $\delta$-best-response in $p$. Therefore, we have that:
		%
		\begin{equation*}	
			\sum_{\omega \in \Omega} F_{ a^{\delta}(p),\omega} p_\omega - c_{ a^{\delta}(p)} \ge \sum_{\omega \in \Omega}F_{ a^{\delta+\epsilon}(p'),\omega} p_\omega - c_{ a^{\delta+\epsilon}(p')} + {\delta },
		\end{equation*}
		%
		and, 
		%
		\begin{equation*}	
			\sum_{\omega \in \Omega}F_{ a^{\delta+\epsilon}(p'),\omega} p_\omega' - c_{ a^{\delta+\epsilon}(p')}  \ge \sum_{\omega \in \Omega} F_{ a^{\delta}(p),\omega} p_\omega' - c_{ a^{\delta}(p)} - {\delta - \epsilon }.
		\end{equation*}
		%
		%because $a^{\delta+\epsilon} (p) \not \in \mathcal{A}_{\delta}(p')$. 
		%
		Then, by taking the summation of the above quantities and employing the definition of $p'$  we get:
		%
		\begin{equation*}	
			\sqrt \epsilon \ge \sum_{\omega \in \Omega} (F_{ a^{\delta}(p),\omega} - F_{ a^{\delta + \epsilon}(p'),\omega}) (r_\omega-p_\omega).
		\end{equation*}
		%
		Therefore, we can prove that the following holds:
		%
		\begin{align*}	
			u^\sfP(p,a^{\delta}(p)) & - u^\sfP(p',a^{\delta+ \epsilon}(p'))  = \sum_{\omega \in \Omega}F_{ a^{\delta}(p),\omega} (r_\omega - p_\omega ) - \left(\sum_{\omega \in \Omega}F_{ a^{\delta+\epsilon}(p'),\omega}(r_\omega - p_\omega' )  \right)\\
			&= \sum_{\omega \in \Omega}F_{ a^{\delta}(p),\omega} (r_\omega - p_\omega ) - \left(1 -\sqrt{\epsilon}\right) \left(\sum_{\omega \in \Omega}F_{ a^{\delta+\epsilon}(p'),\omega}(r_\omega - p_\omega)  \right)\\
			&\le \left(1 -\sqrt{\epsilon}\right) \left(\sum_{\omega \in \Omega} (F_{ a^{\delta}(p),\omega} - F_{ a^{\delta+\epsilon}(p'),\omega})(r_\omega - p_\omega)  \right) + \sqrt{\epsilon} \\
			&\le 2 \sqrt{\epsilon} - \epsilon.
		\end{align*}
		%
		%
		%   
		\item if $a^{\delta+\epsilon} (p') \in {A}^{\delta}(p)$, then one of the following hold.
		\begin{enumerate}
			\item if $a^\delta(p) \in {A}^{\delta+\epsilon}(p')$, then either $a^\delta(p) \equiv a^{\delta+\epsilon}(p')$ or $a^\delta(p)$  provides the same principal's utility as $a^{\delta+\epsilon} (p')$ in $p'$. Indeed, suppose by contradiction that the following holds:
			%
			\begin{equation*}	
				\sum_{\omega \in \Omega} F_{ a^{\delta+\epsilon}(p'),\omega} (r_\omega-p_\omega') < \sum_{\omega \in \Omega} F_{ a^\delta(p),\omega} (r_\omega-p_\omega').
			\end{equation*}
			%
			Then, we have:
			%
			\begin{align*}	
				\sum_{\omega \in \Omega} F_{ a^{\delta+\epsilon}(p'),\omega} (r_\omega-p_\omega') 
				& = (1-\sqrt{\epsilon}) \sum_{\omega \in \Omega} F_{ a^{\delta+\epsilon}(p'),\omega} (r_\omega-p_\omega) \\
				& \ge (1-\sqrt{\epsilon}) \sum_{\omega \in \Omega} F_{ a^\delta(p),\omega} (r_\omega-p_\omega) \\
				& =  \sum_{\omega \in \Omega} F_{ a^\delta(p),\omega} (r_\omega-p_\omega').
			\end{align*}
			where the inequality is a consequence of the fact that both $a^\delta(p)$ and $a^{\delta+\epsilon}(p')$ belongs to ${A}^{\delta} (p)$. This, reaches a contradiction with the initial assumption. Thus, we have that:
			%
			\begin{align*}	
				u^\sfP (p',a^{\delta+\epsilon}(p')) & = \sum_{\omega \in \Omega} F_{ a^{\delta+\epsilon}(p'),\omega} (r_\omega-p_\omega')\\ 
				&= \sum_{\omega \in \Omega} F_{ a^\delta(p),\omega} (r_\omega-p_\omega')\\
				& = (1-\sqrt{\epsilon}) \sum_{\omega \in \Omega} F_{ a^{\delta}(p),\omega} (r_\omega-p_\omega)\\
				& \ge \sum_{\omega \in \Omega} F_{ a^{\delta}(p),\omega} (r_\omega-p_\omega) -\sqrt{\epsilon}\\
				& \ge u^\sfP (p, a^\delta(p)) -\sqrt{\epsilon}.
			\end{align*}
			
			
			\item if $a^\delta(p) \not\in {A}^{\delta+\epsilon}(p')$, then the following holds:
			%
			\begin{align*}	
				\sum_{\omega \in \Omega} F_{ a^{\delta}(p'),\omega} p_\omega - c_{ a^{\delta}(p)} & \ge \sum_{\omega \in \Omega} F_{ a^{\delta+\epsilon}(p'),\omega} p_\omega - c_{ a^{\delta+\epsilon}(p')} - {\delta },
			\end{align*}
			%
			and, 
			%
			\begin{align*}	
				\sum_{\omega \in \Omega}F_{ a^{\delta+\epsilon}(p'),\omega} p_\omega' - c_{ a^{\delta+\epsilon}(p')}  & \ge \sum_{\omega \in \Omega} F_{ a^{\delta}(p),\omega} p_\omega' - c_{ a^{\delta}(p)} + \delta +\epsilon.
			\end{align*}
			%
			Then, by taking the summation of the above quantities and employing the definition of $p'$  we get:
			%
			\begin{equation*}	
				0 \ge \sum_{\omega \in \Omega} (F_{ a^{\delta}(p),\omega} - F_{ a^{\delta+\epsilon}(p'),\omega}) (r_\omega-p_\omega).
			\end{equation*}
			Then, using the same argument employed at point (a) we can show that:
			%
			\begin{align*}	
				u^\sfP (p,a^{\delta}(p))  - u^\sfP (p',a^{\delta+\epsilon}(p')) \le  \sqrt{\epsilon},
			\end{align*}
			%
			concluding the proof.
		\end{enumerate}
		%
	\end{enumerate}
	%
\end{proof}

%\FirstLp*
%\begin{proof}
%   Let $p^\star\in \mathbb{R}^{m}_{+}$ be an optimal $\delta$-robust contract. 
%We show that $p^\star$ is a feasible solution to Program~\eqref{eq:opt} instantiated with $a^\star = a(p^\star)$ and $a^\delta=a^\delta(p^\star)$.
%Let $a \in \Abr$, then:
%$$u^\sfP(p^\star, a)\ge u^\sfP(p^\star, a^\delta(p^\star))=u^\sfA(p^\star, a^\delta).$$
%Thus, Constraint~\eqref{eq:opt-1} is satisfied for each $a \in \Abr$ thanks to the definition of $a^\delta(p^\star)=a^\delta$.
%
%   Furthermore, let $a \not \in \Abr$, then: 
%
%   $$u^\sfA(p^\star, a^\star)=u^\sfA(p^\star, a(p^\star))\ge u^\sfA(p^\star, a)+\delta,$$
%
%    where the first inequality holds because of the definition of $a(p^\star)$, while the second inequality holds because of the definition of $\Abr$.
%
%  Thus, \Cref{eq:opt-1}, is satisfied for each $a \not \in \Abr$.
%
%   Therefore, by combining the above observations, we have that \Cref{eq:opt-1} is satisfied for each $a \in \A$.
%
%   Thus, we have:
% We also observe that:
%    $$
%   \max_{(a,a') \in \A^2} \, \textnormal{LP}(a,a')\ge\textnormal{LP}(a^\star,a^\delta) \ge F_{a^\delta}\cdot(r-p^\star)=\textnormal{OPT}(\delta),$$
%
%    where we let $a^\delta=a^\delta(p^\star)$ and $a^\star=a(p^\star)$.
%
% Then, we have $\max_{a,a'} \, \textnormal{LP}(a,a')\ge\textnormal{OPT}(\delta).$

%  Let $p \in \Rset$ be a feasible solution to Program~\eqref{eq:opt} instantiated with $a^\delta$ and $a^\star$.
%
%   We assume by contradiction that $$u^\sfP(p,a^\delta)=F_{a^\delta}\cdot(r-p) > u^\sfP(p,a^\delta(p)),$$
%   and we show that $a^\delta(p)$ does not satisfies Constraint~\eqref{eq:opt-1}, reaching a contradiction.
%
%
%
%\end{proof}
\NoRegretThm*
\begin{proof}
	In the following, we let:
	\begin{enumerate}
		\item $p^\star$ be such that $\sum_{\theta \in \Theta}\lambda_\theta u^\sfP(p^\star,a^{\theta,\delta}(p^\star),\theta)=\textnormal{OPT}(\mathcal{C},\delta)$. 
		%
		\item $p' \coloneqq (1-\sqrt{2\epsilon})p^\star + \sqrt{2\epsilon} r$. Notice that $p' \in [0,1]^m $ since both $p,r \in \mathcal{C}$ and $\mathcal{C}$ is convex.
		%
		\item $\overline p \in \mathcal{B}_\epsilon$ be such that $\| \overline p-p'\|_{\infty} \le \epsilon$. Notice that there always exists at least once contract $\bar p$ satisfying the latter condition because of the definition of $\mathcal{B}_\epsilon$ and the fact that $p' \in [0,1]^m $.
		
	\end{enumerate}
	%
	We start by observing that, for each $\theta \in \Theta$, it holds $a^{\theta,\delta}(\overline{p}) \in {A}^{\theta, \delta + 2 \epsilon}(p')$. Indeed, we have:
	%
	\begin{align*}	
		\sum_{\omega \in \Omega} F_{\theta, a^{\theta,\delta}(\overline{p}),\omega} p_\omega' - c_{ \theta,a^{\theta,\delta}(\overline{p})}
		& \ge \sum_{\omega \in \Omega} F_{ \theta,a^{\theta,\delta}(\overline{p}),\omega} \overline{p}_\omega - c_{ \theta,a^{\theta,\delta}(\overline{p})} - \epsilon \\
		& > \sum_{\omega \in \Omega} F_{ \theta,a,\omega} \overline{p}_\omega - c_{ \theta,a} - \epsilon - \delta \\
		& \ge \sum_{\omega \in \Omega} F_{ \theta,a,\omega} p_\omega' - c_{\theta, a} - 2\epsilon - \delta,
	\end{align*}
	%
	for every actions $a \in \A$. The above inequalities follows since:
	%
	$$\sum_{\omega \in \Omega}F_{\theta,a, \omega} (\overline{p}_\omega-p_\omega') \le \|F_{\theta,a}\|_1 \|\overline{p}-p'\|_{\infty} \le \epsilon,$$ for every actions $a \in \A$ and $\theta \in \Theta$.
	%
	
	Therefore, we can prove that the following holds:
	\begin{align}
		\sum_{\theta \in \Theta } \lambda_\theta u^\sfP(\overline{p}, a^{\theta,\delta}(\overline{p}),\theta) &\ge \sum_{\theta \in \Theta } \lambda_\theta  u^\sfP(p', a^{\theta,\delta}(\overline{p}),\theta) - \epsilon \nonumber \\
		&\ge \sum_{\theta \in \Theta } \lambda_\theta  u^\sfP(p', a^{\theta,\delta+2\epsilon}(p'),\theta) - \epsilon \nonumber \\
		&\ge \sum_{\theta \in \Theta } \lambda_\theta  u^\sfP(p^\star, a^{\theta,\delta}(p^\star),\theta) - 2\sqrt{2\epsilon} \nonumber \\
		&= \textnormal{OPT}(\mathcal{C},\delta) - 2\sqrt{2\epsilon},\label{eq:eps_optimal}
	\end{align}
	%
	where the first inequality above holds because $\sum_{\omega \in \Omega}F_{\theta, a, \omega} (\overline{p}_\omega-p_\omega') \le \|F_{\theta,a}\|_1 \|\overline{p}-p'\|_{\infty} \le \epsilon$ for every actions $a \in \A$ and type $\theta \in\ \Theta$. The second inequality holds because of the definition of $\delta$-best response, while the third inequality follows from Lemma~\ref{lem:epsilonconvert}.
	
	%
	At this point, we can decompose the cumulative regret as follows:
	\begin{align}
		&R_T(\mathcal{C},\delta)   =   T \cdot \textnormal{OPT}(\mathcal{C},\delta) -  \mathbb{E}\Big[\sum_{t=1}^{T} \sum_{\theta \in \Theta} \lambda_\theta u^\sfP(p^t, a^{\theta,\delta}(p^t),\theta)  \Big] \nonumber\\
		& \le \underbrace{T \cdot \textnormal{OPT}(\mathcal{C},\delta)  - T \cdot \max_{p \in \mathcal{B}_\epsilon} \sum_{\theta \in \Theta} \hspace{-1mm} \lambda_\theta u^\sfP(p,a^{\theta,\delta}(p),\theta) }_{(a)} \\
		&+ \underbrace{ T\cdot \max_{p \in \mathcal{B}_\epsilon}  \sum_{\theta \in \Theta} \hspace{-1mm} \lambda_\theta u^\sfP(p,a^{\theta,\delta}(p),\theta) -  \mathbb{E}\Big[ \sum_{t \in [T],\theta \in \Theta} \hspace{-1mm} \lambda_\theta u^\sfP(p^t, a^{\theta,\delta}(p^t),\theta) \Big]}_{(b)}. \label{eq:regret_decomp}
	\end{align}
	%	
	We focus on bounding term (a) in \cref{eq:regret_decomp}.
	%
	Let $\bar p \in [0,1]^m$ be the contract defined at point 3. Then, we have: 
	%
	\begin{align*}
		T \cdot \textnormal{OPT}(\mathcal{C},\delta)- T \max_{p \in \mathcal{B}_\epsilon}   \sum_{\theta \in \Theta} \lambda_\theta  u^\sfP(p^t, a^{\theta,\delta}(p^t),\theta) 
		& \le  T \cdot \textnormal{OPT}(\mathcal{C},\delta)- T \sum_{\theta \in \Theta} \lambda_\theta  u^\sfP(\bar p, a^{\theta,\delta}(\bar p),\theta)  \\
		& \le  2 \sqrt{2\epsilon} \, T,
	\end{align*}
	%
	where the last inequality holds because of \Cref{eq:eps_optimal}.
	%
	We focus on bounding term (b) in \cref{eq:regret_decomp}. By employing the same analysis to prove the regret bound in \texttt{UCB1} we have:
	%
	\begin{equation}
		T \cdot \max_{p \in \mathcal{B}_\epsilon}  \sum_{\theta \in \Theta} \lambda_\theta u^\sfP(p,a^{\theta,\delta}(p),\theta) -  \mathbb{E}\Big[ \sum_{t \in [T]} \sum_{\theta \in \Theta} \lambda_\theta  u^\sfP(p^t, a^{\theta,\delta}(p^t),\theta) \Big] \le \mathcal{O}\left(\sqrt{ T |\mathcal{B}_\epsilon| \log(T ) }\right).
	\end{equation} 
	%
	Finally, observing that $|\mathcal{B}_\epsilon| \le \mathcal{O}{((\nicefrac{1}{\epsilon})}^{m})$, by putting (a) and (b) in \cref{eq:regret_decomp} together and setting $\epsilon= T^{-\frac{1}{m+1}}$, we have:
	%
	\begin{align*}
		R_T (\mathcal{C},\delta) \le \mathcal{\widetilde{O}} \left(   T^{1- \frac{1}{2(m+1)}}  \right),
	\end{align*}
	concluding the proof.
\end{proof}

\NoRegretCor*
\begin{proof}
	Let $p^\star$ be an optimal non-robust contract inside $\mathcal{C}$. Then we have:
	%
	\begin{align*}	
		R_T (\mathcal{C}) & =   T \cdot \textnormal{OPT}(\mathcal{C}) -  \mathbb{E}\Big[\sum_{t=1}^{T} \sum_{\theta \in \Theta} \lambda_\theta u^\sfP(p^t, a^{\theta,\delta}(p^t),\theta)  \Big] \\
		&= T \cdot \left( \textnormal{OPT}(\mathcal{C}) -  \textnormal{OPT}(\mathcal{C}, \delta)\right) + T \cdot \textnormal{OPT}(\mathcal{C}, \delta) -  \mathbb{E}\Big[\sum_{t=1}^{T} \sum_{\theta \in \Theta} \lambda_\theta u^\sfP(p^t, a^{\theta,\delta}(p^t),\theta)  \Big] \\
		&\le 2 \sqrt{\delta} T + T \cdot \textnormal{OPT}(\mathcal{C},\delta)  -  \mathbb{E}\Big[\sum_{t=1}^{T} \sum_{\theta \in \Theta} \lambda_\theta u^\sfP(p^t, a^{\theta,\delta }(p^t),\theta)  \Big]  \\
		& \le 2 \sqrt{\delta} T + \mathcal{\widetilde{O}} \left(   T^{1- \frac{1}{2(m+1)}}  \right) ,
	\end{align*}
	where the first inequality holds by employing the same argument needed to prove Proposition~\ref{pro:properties}, and the second inequality holds thanks to Theorem~\ref{thm:no_regret}, concluding the proof.
\end{proof}




%%%%%%%%%%%%%%%%%%%%%%%%%%%%%%%%%%%%%%%%%%%%%%%%%%%%%%%%%%%%%%%%%%%%%%%%%%%%%%%
%%%%%%%%%%%%%%%%%%%%%%%%%%%%%%%%%%%%%%%%%%%%%%%%%%%%%%%%%%%%%%%%%%%%%%%%%%%%%%%


