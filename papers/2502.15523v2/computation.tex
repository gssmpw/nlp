\section{Computing Optimal Robust Contracts}\label{sec:computational_problem}



Now, we present our main result: a polynomial-time algorithm to compute an optimal $\delta$-robust contract.
%
%\bac{and we show that, differently from Stackeberg games, the problem admits a polynomial time solution. }
%



\subsection{Characterizing an Optimal Contract}

We begin by presenting an optimization problem that characterizes an optimal $\delta$-robust contract. 
Consider an arbitrary optimal $\delta$-robust contract $p^\star$, as well as two arbitrary actions $a^\star \in A(p^\star)$ and $a^\delta \in \arg\min_{a \in A^\delta(p^\star)} u^\sfP(p^\star, a)$.
By fixing $a^\star$ and $a^\delta$, we show that the following optimization problem, over the variable $p \in \mathbb{R}_+^m$, characterizes an optimal $\delta$-robust contract $p^\star$.
% optimal contracts can be characterized as optimal solutions to the following optimization problem, over variable $p$.
\begin{align}
	\label{eq:opt}
	\max_{p\in  \mathbb{R}_+^m}
	\quad  
	u^\sfP(p, a^\delta)
	%F_{a^\delta} \cdot (\rr - p) 
\end{align}
subject to the following \emph{disjunctive} constraints, which must hold for every agent's action $a \in \mathcal{A}$:
\begin{align}
	\hspace{-2mm}\Big( u^\sfA(p, a) \le u^\sfA(p, a^\star) -   \delta  \Big) \vee \Big(u^\sfP(p, a) \ge u^\sfP(p, a^\delta)\Big).\tag{\ref*{eq:opt}a}\label{eq:opt-1}
	% &\Big( u^\sfA(p, a) \le u^\sfA(p, a^\star) - \delta \Big) 
	%\;\bigvee \nonumber\\
	%&\qquad\qquad\qquad
	%\Big( u^\sfP(p, a \ge u^\sfP(p, a^\delta) \Big).
	%\tag{\ref*{eq:opt}a}\label{eq:opt-1}
\end{align}

% \albo{An intuition on the constraint would be helpful.}

Intuitively, \Cref{eq:opt-1} requires that each action $a$ is either \emph{not} a $\delta$-best response (first inequality) or no worse than $a^\delta$ for the principal (second inequality). This ensures that the objective function $u^\sfP(p,a^\delta)$ captures the principal's utility in a $\delta$-robust contract $p$.
More formally, we establish the following lemma (recall that $\Psi(p) = \min_{a \in A^\delta(p)} u^\sfP(p, a)$ denotes the principal's utility in a $\delta$-robust contract $p$).


% \begin{lemma}
	% If $p$ satisfies \Cref{eq:opt-1}, then $u^\sfP(p, a^\delta) \le \Psi(p)$.
	% Moreover, if $p$ is an optimal solution to \Cref{eq:opt}, then $u^\sfP(p, a^\delta) = \Psi(p)$. 
	% \end{lemma}

% \begin{proof}
	
	% \end{proof}

% \begin{lemma}
	% If $p$ is an optimal solution to \Cref{eq:opt}, then $u^\sfP(p, a^\delta) = \Psi(p)$. Additionally, $a^\star \in \BR(p)$ and $a^\delta \in \BR_\delta(p)$. 
	% \end{lemma}

% \begin{proof}
	
	% \end{proof}

\begin{lemma}
	\label{lmm:Psi-opt}
	% Suppose that $p^\star$ is an optimal $\delta$-robust contract; moreover, $a^\star \in \BR(p^\star)$, and $a^\delta \in \arg\min_{a \in \BR_\delta(p^\star)} u^\sfP(p^\star, a)$.
	Every optimal solution $p \in \mathbb{R}_+^m$ to Problem~\eqref{eq:opt} is an optimal $\delta$-robust contract, i.e., $\Psi(p) = \Psi(p^\star)$.
\end{lemma}

\begin{proof}
	First, observe that $p^\star$ is a feasible solution to Problem~\eqref{eq:opt}.
	Indeed, if an action $a$ is {\em not} a $\delta$-best response to $p^\star$, then by definition this means $u^\sfA(p^\star, a) \le u^\sfA(p^\star, a^\star) - \delta$; otherwise, it must be that $u^\sfP(p^\star, a) \ge u^\sfP(p^\star, a^\delta)$ as $a^\delta$ is by definition the worst $\delta$-best response in $p^\star$.
	As a result, \Cref{eq:opt-1} holds for $p^\star$ for every $a \in \mathcal{A}$.
	
	Furthermore, notice that, according to the definition of $a^\delta$:
	\[
	\Psi(p^\star) = \min_{a \in A_\delta(p^\star)} u^\sfP(p^\star, a) = u^\sfP(p^\star, a^\delta).
	\]
	
	In order to complete the proof, it is sufficient to show that $u^\sfP(p, a^\delta) \le \Psi(p)$ for every feasible solution $p\in \mathbb{R}_+^m$ to Problem~\eqref{eq:opt}.
	%to complete the proof.
	Indeed, if the condition above holds, for any arbitrary optimal solution $p'\in \mathbb{R}_+^m$ to Problem~\eqref{eq:opt}, we have:
	\begin{align*}
		\Psi(p^\star)
		= u^\sfP(p^\star, a^\delta) 
		\le u^\sfP(p', a^\delta)
		\le \Psi(p') 
		\le \Psi(p^\star),
	\end{align*}
	where $u^\sfP(p^\star, a^\delta) \le u^\sfP(p', a^\delta)$ since $p^\star$ is a feasible solution to Problem~\eqref{eq:opt} and $p'$ is optimal,
	while $\Psi(p') \le \Psi(p^\star)$ since $p^\star$ is an optimal $\delta$-robust contract by definition.
	%
	Then, it must be the case that $\Psi(p') = \Psi(p^\star)$.
	
	Now, to complete the proof, consider any feasible solution $p\in \mathbb{R}_+^m$ to Problem~\eqref{eq:opt}. We show that $u^\sfP(p, a^\delta) \le \Psi(p)$.
	%to complete the proof.
	%
	Pick an arbitrary best-response action $a' \in A(p)$ and consider any $a \in A^\delta(p)$.
	By definition,
	\begin{align*}
		u^\sfA(p, a) > u^\sfA(p, a') - \delta = \max_{a \in \Actions} u^\sfA(p, a) - \delta 
		\ge u^\sfA(p, a^\star) - \delta.
	\end{align*}
	Hence, for \Cref{eq:opt-1} to hold for action $a$, it must be that $u^\sfP(p, a) \ge u^\sfP(p, a^\delta)$.
	Since the choice of $a$ is arbitrary, this holds for every $a \in A^\delta(p)$.
	Consequently,
	\[
	\Psi(p) = \min_{a \in A^\delta(p)} u^\sfP(p, a) \ge u^\sfP(p, a^\delta).
	\qedhere
	\]
\end{proof}

The above lemma implies that we can effectively ``guess'' $a^\star$ and $a^\delta$, fixing these actions in Problem~\eqref{eq:opt} and solving the optimization problem to obtain $p^\star$ (or possibly a different, but still optimal, $\delta$-robust contract).
Since there are only $O(n^2)$ possible combinations of the values of $a^\star$ and $a^\delta$, the approach is efficient as long as Problem~\eqref{eq:opt} can be solved efficiently.
A correct guess yields a contract $p$ such that $\Psi(p) = \Psi(p^\star)$, whereas $\Psi(p) \le \Psi(p^\star)$ for incorrect guesses.
%
Thus, by comparing the $\Psi$ values, we can identify a correct guess and a corresponding optimal $\delta$-robust contract.

It remains to show how to efficiently solve Problem~\eqref{eq:opt}.

\subsection{Solving Problem~\eqref{eq:opt}}

Problem~\eqref{eq:opt} does \emph{not} directly admit any efficient solution algorithm due to the non-convex constraint in~\Cref{eq:opt-1}.
To deal with this issue, we rewrite \Cref{eq:opt-1} as follows:
\begin{align}
	\Big( F_a \cdot p \le c_a + u^\sfA(p, a^\star) - \delta \Big) \;\vee  \;  \Big( F_a \cdot p \le F_a \cdot r - u^\sfP(p, a^\delta) \Big),
	\label{eq:opt-1-re}
\end{align}
by expanding the utilities as $u^\sfA(p, a) = F_a \cdot p - c_a$ and $u^\sfP(p, a) = F_a \cdot (r - p)$ and rearranging the terms.

Hence, \Cref{eq:opt-1} is satisfied for action $a \in \mathcal{A}$ if and only if $F_a \cdot p$ is smaller than the maximum of the right-hand sides of the two inequalities in~\Cref{eq:opt-1-re}.
The constraint effectively reduces to the second inequality for all $p \in \Rset$ such that 
\begin{align}
	\label{eq:right-sides}
	c_a + u^\sfA(p, a^\star) - \delta \le F_a \cdot r - u^\sfP(p, a^\delta),
\end{align} 
%while it reduces to the second inequality for all other $p\in \Rset$.
while for the other $p\in \Rset$, it reduces to the first inequality.

Consequently, we can partition the contract space based on the satisfiability of \Cref{eq:right-sides}, considering all $a \in \mathcal{A}$.
Within each subspace in the partition, only one inequality in \Cref{eq:opt-1-re} is active for every $a \in \mathcal{A}$. So, effectively, \Cref{eq:opt-1-re} reduces to a linear constraint for each action $a$, and the optimization problem to an LP.
%
It then suffices to solve an LP for every subspace, each generating an optimal contract within its corresponding subspace.
%
Among these contracts, the one providing the highest utility is an optimal solution to Problem~\eqref{eq:opt}.


Now, if the linear inequalities in \Cref{eq:right-sides} (one for each action $a \in \mathcal{A}$) were $n$ arbitrary inequalities, the above partition may consist of exponentially many subspaces, making the approach inefficient. 
Fortunately, the partition is much more well-structured, since the hyperplanes corresponding to the inequalities are {\em parallel}, as the coefficients of $p$ in \Cref{eq:right-sides} are invariant with respect to $a$.
% Surprisingly, in the context of contract design, the partition is well-structured and involves only $O(n)$ subspaces as we demonstrate next.
As a result, they partition the space into only $O(n)$ subspaces.


Next, we show how to exploit the above observation to solve Problem~\eqref{eq:opt}, by solving $O(n)$ suitable subproblems instead.

\subsection{Formulating the Subproblems}

Let us first rearrange \Cref{eq:right-sides} as follows:
\begin{align}
	\label{eq:right-sides-re}
	u^\sfA(p, a^\star) + u^\sfP(p, a^\delta) - \delta \le \nu_a \coloneqq F_a \cdot r - c_a, 
\end{align} 
where $\nu_a$ is exactly the social welfare generated by action $a $ (which is independent of the specific contract adopted).

Then, we can re-order agent's actions $a_1, \dots, a_n$ in such a way that $\nu_{a_1} \le \nu_{a_2} \le \dots \le \nu_{a_n}$. For simplicity, we write $\nu_j = \nu_{a_j}$, and we let $\nu_0 = -\infty$ and $\nu_{n+1} = +\infty$.
% It is then straightforward that for each $j=1,\dots,n+1$, the following is a subspace in the partition:
% \[
% \left\{ 
% p \in \mathbb{R}^{m}_{+} :  
% \nu_{j-1} \le u^\sfA(p, a^\star) + u^\sfP(p, a^\delta) - \delta \le \nu_{j}
% \right\}.
% \]
Then, the following lemma is straightforward.

\begin{lemma}
	\label{lmm:right-sides}
	For every contract $p \in \mathbb{R}^{m}_{+}$, if it holds that $\nu_{j-1} \leq u^\sfA(p, a^\star) + u^\sfP(p, a^\delta) - \delta \leq \nu_{j}$, then:
	%
	% If $u^\sfA(p, a^\star) + u^\sfP(p, a^\delta) - \delta \in [\nu_{j-1}, \nu_{j}]$, then:
	\begin{itemize}
		\item $F_a \cdot p \le F_a \cdot r - u^\sfP(p, a^\delta) \Longleftrightarrow$ \Cref{eq:opt-1-re} holds for all actions $a  \in \{a_\ell \mid \ell \le j - 1\}$; and 
		%
		\item $F_a \cdot p \le c_a + u^\sfA(p, a^\star) - \delta \Longleftrightarrow$ \Cref{eq:opt-1-re} holds for all actions $a  \in \{a_\ell \mid j \le \ell \}$.
	\end{itemize}
\end{lemma}

In other words, the condition in the lemma defines a suitable subspace of contracts $\mathcal{P}_j$ for each $j \in \{1,\dots,n+1\}$. 
The following LP solves for a $p \in \Rset$ that is optimal within $\mathcal{P}_j$. 
\begin{align}
	\label{eq:lp-nu-j}
	\max_{p \in \mathbb{R}_+^m}
	\quad
	u^\sfP(p, a^\delta)
	% F_{a^\delta} \cdot (\rr - p) 
	\tag{\ref*{eq:lp-nu-j-const}}
\end{align}
subject to the following constraints:
\begin{subequations}
	\label{eq:lp-nu-j-const}
	\begin{align}
		%&\hspace{-2mm}
		%p_\omega \ge 0 \quad \text{ for } \omega \in \Omega \\
		&%\hspace{-3mm}
		\nu_{j-1} \le u^\sfA(p, a^\star) + u^\sfP(p, a^\delta) - \delta \le  \nu_j \label{eq:lp-nu-j-1} \\
		&%\hspace{-3mm}
		F_a {\cdot} \, p \le F_a \cdot r - u^\sfP(p, a^\delta)
		\quad \forall a \in \{a_\ell \mid  \ell \le j{-}1\}  \label{eq:lp-nu-j-2} \\
		&%\hspace{-3mm}
		F_a {\cdot} \, p \le c_a + u^\sfA(p, a^\star) - \delta
		\quad\quad \forall a \in \{a_\ell \mid j \le \ell \}. \label{eq:lp-nu-j-3}
	\end{align}
\end{subequations}
Since $\bigcup_{j=1}^{n+1} \mathcal{P}_j = \mathbb{R}^m_+$, solving the LP in Problem~\eqref{eq:lp-nu-j-const} for all $j \in \{1,\dots,n+1\}$ and picking the best among the obtained solutions gives an optimal solution to Problem~\eqref{eq:opt}.


\begin{remark}
	The left-hand side of \Cref{eq:right-sides-re} is roughly (up to a $\delta$ difference) the social welfare of an optimal contract.
	Thus, \Cref{lmm:right-sides} can be interpreted as follows.
	For low-social-welfare actions $a_1,\dots, a_{j-1}$, yielding a sufficiently high utility for the principal automatically provides a low utility for the agent.
	%
	This fulfills the first inequality in \Cref{eq:opt-1-re} (and \Cref{eq:opt-1}).
	Conversely, for high-social-welfare actions $a_j,\dots, a_n$, a sufficiently low utility for the agent automatically gives a high utility for the principal.
	%
	This fulfills the second inequality in \Cref{eq:opt-1-re} (and \Cref{eq:opt-1}).
\end{remark}

\begin{algorithm}[!htp]
	\caption{Compute an optimal $\delta$-robust contract}
	\label{alg:robust_poly}
	\begin{algorithmic}[1]
		% \REQUIRE \bac{Princiapl-agent isnstance and $\delta>0$} \jiarui{this require is somewhat obvious. we can omit?} \bac{Yess, you are right!}
		\State $p^\star \leftarrow \texttt{null}$,\; 
		$\psi^\star \leftarrow -\infty$
		\ForAll{$(a^\star,a^\delta) \in \mathcal{A} \times \mathcal{A}$} \label{ln:inner-for}
		\ForAll{$j=1,\dots,n+1$}
		\State 
		% \begin{align}
			% \label{eq:lp-nu-j}
			% \max
			% \quad
			% u^\sfP(p, a^\delta)
			% % F_{a^\delta} \cdot (\rr - p) 
			% \tag{\ref*{eq:lp-nu-j-const}}
			% \end{align}
		Solve Problem~\eqref{eq:lp-nu-j-const} instantiated with $(a^\star,a^\delta)$ s.t. \Cref{eq:lp-nu-j-1,eq:lp-nu-j-2,eq:lp-nu-j-3},
		% solve $\max_{p \in \mathbb{R}_+^m} u^\sfP(p, a^\delta)$,
		% s.t. 
		% Eqs.~\eqref{eq:lp-nu-j-1}--\eqref{eq:lp-nu-j-3}, 
		and let an optimal solution be $p'$
		
		\If{$\Psi(p') > \psi^\star$}
		\State $p^\star \leftarrow p'$ 
		\State $\psi^\star \leftarrow \Psi(p')$
		\EndIf
		
		% the following constraints:
		% \begin{subequations}
			% \label{eq:lp-nu-j-const}
			% \begin{align}
				% %&\hspace{-2mm}
				% %p_\omega \ge 0 \quad \text{ for } \omega \in \Omega \\
				% &\hspace{-2mm}
				% \nu_{j-1} \le u^\sfA(p, a^\star) + u^\sfP(p, a^\delta) - \delta \le  \nu_j \label{eq:lp-nu-j-1} \\
				% &\hspace{-2mm}
				% F_a \cdot p \le F_a \cdot r - u^\sfP(p, a^\delta)
				% \,\, \forall a \in \{a_{i} \}_{i<j} \label{eq:lp-nu-j-2} \\
				% &\hspace{-2mm}
				% F_a \cdot p \le c_a + u^\sfA(p, a^\star) - \delta
				% \,\, \forall a \in \{a_i\}_{i\ge j} \label{eq:lp-nu-j-3}
				% \end{align}
			% \end{subequations}
		\label{ln:inner-for-end}
		\EndFor
		\EndFor
		\State {\bf return} $p^\star$
	\end{algorithmic}
\end{algorithm}
%
%\bac{ABUSE OF NOTATION IN THE ABOVE LP $a_0$}
% \albo{Maybe we should write that $\{ a_1, a_0 \}$ and $\{ a_n, a_{n+1} \}$ are to be considered empty set?}
% \jiarui{What about writing $a \in \{a_\ell : 1\le \ell \le j-1\}$ in \Cref{eq:lp-nu-j-2,eq:lp-nu-j-3}}
% \albo{Better ;)}

\subsection{Putting All Together}

We summarize our results in this section into \Cref{alg:robust_poly} and the following main theorem.

\begin{theorem}
	\Cref{alg:robust_poly} computes an optimal $\delta$-robust contract in polynomial time.
\end{theorem}

\begin{proof}[Proof Sketch]
	The polynomial runtime of \Cref{alg:robust_poly} is obvious as it enumerates $O(n^3)$ value combinations and solves an LP for each of them.
	To see the correctness of the algorithm, note that the inner for-loop of \Cref{alg:robust_poly} effectively solves Problem~\eqref{eq:opt}, for the pair $(a^\star, a^\delta)$ enumerated in the outer for-loop.
	Now, if $a^\star$ and $a^\delta$ happen to be the agent's exact and $\delta$-best responses, respectively, under some optimal contract, then according to \Cref{lmm:Psi-opt}, the outer loop produces an optimal contract $p'$, with $\Psi(p') \ge \Psi(p)$ for all $p \in \mathbb{R}_+^m$. By comparing the $\Psi$ values, the algorithm identifies and outputs such an optimal contract.
\end{proof}

%\clearpage