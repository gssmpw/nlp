\section{Formative Study}
\label{formative-study}
% To understand viewer behavior and challenges in assimilating collective knowledge in science videos, we conducted semi-structured interviews with 10 frequent consumers of these videos with danmaku feature. Our study validates the benefits of danmaku in science videos, categorizing three user behavior modes and identifying the challenges encountered. 
% % These insights informed the design of \textit{SystemName}, an interactive system to aid viewer comprehension of collective knowledge.
To further investigate the needs and challenges specific to danmaku, we conducted a formative study with consumers of science videos featuring danmaku. Based on our findings and insights from previous works, we derived the design pipeline for CoKnowledge.

\subsection{Participants and Procedure}
With IRB approval, we recruited 10 participants (five female, five male; labeled PF1 to PF10) for the formative study through online ads, social media, and word-of-mouth. All participants were self-identified frequent viewers of online science videos with the danmaku feature (six watched daily, three watched 4 to 6 days a week, and one watched at least once a week).  In semi-structured interviews, participants discussed their viewing strategies, interactions with danmaku, and perceptions of its role in collective knowledge. We also explored challenges in assimilating collective knowledge and solicited suggestions for technical support. Participants were encouraged to provide examples to contextualize their responses. 
% Interviews lasted 30-45 minutes, and participants received \$6 for their time.
Following established protocols \cite{klein2017quality, baker2021avatar, liu2021makes}, interviews were audio-recorded, transcribed, and thematically analyzed by two researchers.
% Two researchers conducted a thematic analysis to inductively code and identify themes. 
We monitored the codebook for saturation after eight interviews \cite{saunders2018saturation}, completing the last two to confirm that no new themes emerged.

\subsection{Summary of Findings}
Our findings validate danmaku's benefits in science videos, identifying three user behavior modes and challenges in consuming collective knowledge.

\begin{table*}[h!]
\centering
\scalebox{0.75}{ 
\begin{tabular}{ >{\centering\arraybackslash}m{1.2cm} >{\centering\arraybackslash}m{5cm} >{\centering\arraybackslash}m{4.8cm} >{\centering\arraybackslash}m{4cm} >{\centering\arraybackslash}m{3.5cm} }
  \hline
   &  & \textit{Overview} & \textit{Focused} & \textit{Exploration} \\
  \hline
  \multirow{5}{*}{Dynamics} & Knowledge need & High-level knowledge abstraction \circled{1}& Concise knowledge \circled{2}& Structured knowledge \circled{3}\\
  & Knowledge scope & Entire video \circled{4}& Current timestamp \circled{5}& Short segments \circled{6}\\
  & Engagement level & Active \circled{7}& Mostly passive \circled{8}& Active \circled{9}\\
  & Interaction behavior & Skimming, locating \circled{10}& Typical viewing behavior \circled{11}& Thorough analysis \circled{12}\\
  & Focal point & Entire interface \circled{13}& Video segment view \circled{14}& Entire interface \circled{15}\\
  \hline
  \multirow{5}{*}{Challenges} & Mutual distractions \circled{16}& \ding{55} & \ding{51} & \ding{55} \\
  & Short display duration \circled{17}& \ding{55} & \ding{51} & \ding{55} \\
  & Danmaku locating difficulties \circled{18}& \ding{51} & \ding{55} & \ding{55} \\
  & \textbf{Low information density} & \ding{51} & \ding{51} & \ding{51} \\
  & \textbf{Obscure information structure} & \ding{51} & \ding{51} & \ding{51} \\
  \hline
\end{tabular}
}
\caption{Behavior patterns and challenges for assimilating collective knowledge in online science videos. The labeled items are addressed in section \ref{ui}.}
\label{formative-study-findings}
\end{table*}

\subsubsection{Advantages of Danmaku}

We inquired about participants' perceptions of the support provided by danmaku in knowledge acquisition while watching science videos. 
% They highlighted numerous advantages of danmaku, including those mentioned in existing work. 
% Most participants (9 out of 10) indicated that danmaku offered a \textbf{sense of co-presence}. PF9 remarked, \textit{``When danmaku is turned off, the interaction feels like a conversation between just me and the uploader. When it’s on, everyone is involved in discussing the topic.''} 
The majority (9 out of 10) highlighted that danmaku fostered a \textbf{sense of co-presence}, with PF9 noting it shifted the experience from an ``isolated interaction with the uploader'' to a more collective discussion.
This co-presence enhanced engagement with the content, thereby boosting motivation and efficiency in knowledge acquisition.
% Furthermore, 8 participants emphasized that danmaku helped \textbf{avoid knowledge bias} by presenting diverse viewpoints. Specifically, PF8 mentioned, \textit{``Sometimes I worry that the uploader's perspective might be biased, but danmaku provides a variety of viewpoints.''} These diverse perspectives offered users additional references, facilitating a comprehensive understanding of the content.
Furthermore, 8 participants emphasized that danmaku helped \textbf{avoid knowledge bias} by presenting diverse viewpoints, as PF8 indicated, offering ``alternative perspectives'' to those of the uploader.
% All participants agreed that danmaku synchronously supplemented knowledge. PF2 stated, \textit{``[Compared to comments,] I can see real-time, immediate supplements or extensions of the video content through danmaku.''} 
All participants agreed that danmaku synchronously supplemented knowledge, with PF2 highlighting its ability to offer ``real-time'' and ``immediate'' extensions of video content.
\textbf{The closer integration} of danmaku with the video enhanced the immediacy and contextual relevance of the knowledge.
These advantages aligned with those discussed in existing literature \cite{he2021beyond, yao2017understanding, chen2015understanding, ma2014analysis, huang2024sharing}, further affirming danmaku as a valuable source for constructing collective knowledge.
% In addition to the advantages mentioned in previous work, participants also identified other benefits of danmaku. For instance, PF5 noted, \textit{``I can find a sense of agreement in the viewpoints expressed in danmaku,''} while PF7 mentioned that the flow of danmaku helps them stay focused and reduces distractions. These advantages confirm danmaku as a valuable source for constructing collective knowledge.



\subsubsection{Viewing Practices and Challenges}

Participants reported various practices when assimilating collective knowledge. We identified three viewing modes—\textbf{Overview}, \textbf{Focused}, and \textbf{Exploration}—and examined their corresponding features (Table \ref{formative-study-findings}).



% \begin{table}[h!]
% \centering
% \begin{tabular}{ >{\centering\arraybackslash}m{3.2cm} >{\centering\arraybackslash}m{4cm} >{\centering\arraybackslash}m{3.2cm} >{\centering\arraybackslash}m{3cm} }
%   \hline
%    & \textit{Overview} & \textit{Focused} & \textit{Exploration} \\
%   \hline
%   Knowledge need & High-level knowledge abstract & Concise knowledge & Structured knowledge \\
%   \hline
%   Knowledge scope & Entire video & Current timestamp & Short segments \\
%   \hline
%   Engagement level & Active & Mostly passive & Active \\
%   \hline
%   Interaction behavior & Skimming, navigating & Typical viewing behavior & Thorough analysis \\
%   \hline
%   Focal point & Entire interface & Video segment view & Entire interface \\
%   \hline 
%   Mutual distractions  & \ding{55} & \ding{51} & \ding{55} \\
%   \hline
%   Short duration & \ding{55} & \ding{51} & \ding{55} \\
%   \hline
%   Challenges in navigating desired danmaku & \ding{51} & \ding{55} & \ding{55} \\
%   \hline
%   \textbf{Low Information Density} & \ding{51} & \ding{51} & \ding{51} \\
%   \hline
%   \textbf{Obscure Information Structure} & \ding{51} & \ding{51} & \ding{51} \\
%   \hline
% \end{tabular}
% \caption{Behavior patterns and challenges for assimilating collective knowledge in online science videos}
% \label{formative-study-findings}
% \end{table}

% \begin{table}[h!]
% \centering
% \scalebox{0.85}{ 
% \begin{tabular}{ >{\centering\arraybackslash}m{1.2cm} >{\centering\arraybackslash}m{4.3cm} >{\centering\arraybackslash}m{4cm} >{\centering\arraybackslash}m{3.3cm} >{\centering\arraybackslash}m{2.8cm} }
%   \hline
%    &  & \textit{Overview} & \textit{Focused} & \textit{Exploration} \\
%   \hline
%   \multirow{5}{*}{Dynamics} & Knowledge need & High-level knowledge abstract \xm{abstraction?} & Concise knowledge & Structured knowledge \\
%   & Knowledge scope & Entire video & Current timestamp & Short segments \\
%   & Engagement level & Active & Mostly passive & Active \\
%   & Interaction behavior & Skimming, lcoating & Typical viewing behavior & Thorough analysis \\
%   & Focal point & Entire interface & Video segment view & Entire interface \\
%   \hline
%   \multirow{5}{*}{Challenges} & Mutual distractions & \ding{55} & \ding{51} & \ding{55} \\
%   & Short display duration & \ding{55} & \ding{51} & \ding{55} \\
%   & Danmaku locating difficulties & \ding{51} & \ding{55} & \ding{55} \\
%   & \textbf{Low information density} & \ding{51} & \ding{51} & \ding{51} \\
%   & \textbf{Obscure information structure} & \ding{51} & \ding{51} & \ding{51} \\
%   \hline
% \end{tabular}
% }
% \caption{Behavior patterns and challenges for assimilating collective knowledge in online science videos}
% \label{formative-study-findings}
% \end{table}




In the \textbf{Overview} mode, many participants (PF1, PF4-7, PF10) actively skimmed through the video to quickly understand or review its main content and locate segments of interest. Such behaviors were often observed before watching the video, but also during (PF6) and after viewing (PF6, PF7, PF10). However, users struggled with effectively \textbf{locating desired danmaku content}. PF1 remarked, \textit{``There is no way to preview danmaku to locate the good ones,''} PF6 added, \textit{``I saw interesting danmaku, but when I wanted to revisit them later, I could not find them.''}

The \textbf{Focused} mode described the situation when users concentrated solely on watching the video. In this mode, users were mostly passive, occasionally engaging in typical viewing behaviors like pausing, replaying, skipping, or adjusting playback speed to shift into other modes. All participants reported being overwhelmed by the volume of danmaku and finding the danmaku and video to be \textbf{mutually distracting}. Danmaku often blocked the video content (PF1, PF2-5, PF8), causing users’ attention to oscillate between the two, significantly hindering the assimilation of collective knowledge. Many participants (7 out of 10) also complained about the \textbf{short display duration} of danmaku, with PF10 noting that comments \textit{``fly by too quickly''} to be fully read.


When participants encountered interesting or complex content in the video or danmaku, they entered the \textbf{Exploration} mode. In this mode, users fixated on a short segment of the video, examining the meaning of danmaku and its relationship with the video (PF6, PF10), and may also seek additional information from external sources to aid their understanding (PF1, PF2, PF10).


We identified two key challenges that fundamentally hinder the assimilation of collective knowledge across all viewing modes: \textbf{low information density} and \textbf{obscure information structure}, which were frequently criticized by participants. All participants agreed that most danmaku is \textit{``meaningless for knowledge acquisition''} and characterized by excessive repetition, often burying valuable comments (PF1-6, PF7, PF9) and causing frustration (PF3). Furthermore, the majority (8 out of 10) mentioned that even informative danmaku was chaotic because the \textit{``information pattern was unorganized''} (PF7), and the relationship between danmaku and video knowledge lacked clarity (PF10), making it challenging to synthesize danmaku into the framework of collective knowledge. 



\subsection{Design Requirements}
\label{drs}

Based on the findings, we formulated the following design requirements to guide the subsequent design process (as illustrated in Fig \ref{fig:design-pipeline}): 

\begin{enumerate}[label=\textbf{DR\arabic*}, leftmargin=3em]
    \item \textbf{Accommodate diverse viewing modes and their corresponding dynamics while mitigating mode-specific challenges.} We have labeled the items in Table \ref{formative-study-findings} and addressed them accordingly in section \ref{ui}.
    
    \item \textbf{Reduce the volume of danmaku by excluding low-quality, knowledge-irrelevant comments and consolidating duplicate ones.} 
    In addition to minimizing the distraction of video-viewing, reducing the number of danmaku can increase information density. This allows users to concentrate more on individual comments, lowering the chances of comments scrolling away before being fully read. Furthermore, it is easier to locate specific danmaku, as they are less likely to be overshadowed by others.
    % In addition to increasing information density, reducing the number of danmaku can minimize the distraction to video-viewing. This allowed users to concentrate more on individual comments, reducing the frequency of instances where danmaku would scroll away before being fully read. Furthermore, it made it easier to locate specific danmaku, as they were less likely to be overshadowed by others.
    
    \item \textbf{Clearly present the diverse themes of danmaku and the structures of collective knowledge.} The various perspectives danmaku offers can enhance the sense of discussion and provide viewers with more sources of information. Further dissecting video knowledge and arranging danmaku around the corresponding knowledge points can highlight their inherent connections and the overall information structure.
    % Further organizing the patterns of danmaku with the video content would clarify the information structure and strengthen the correspondence between them.
\end{enumerate}

% \yh{While the correspondence of DRs and Findings is obvious to me, but I am not sure if it is the case for readers. Can I just simply list the DRs here?}

% \begin{table}[h!]
% \centering
% \begin{tabular}{| m{6cm} | c | c | c |}
%   \hline
%   \rowcolor{lightgray}
%   \textbf{Challenge} & \textbf{Overview} & \textbf{Focused} & \textbf{Exploration} \\
%   \hline
%   Mutual distractions  & \ding{55} & \ding{51} & \ding{55} \\
%   \hline
%   Short duration & \ding{55} & \ding{51} & \ding{55} \\
%   \hline
%   Challenges in navigating desired danmaku & \ding{51} & \ding{55} & \ding{55} \\
%   \hline
%   \textbf{Low Information Density} & \ding{51} & \ding{51} & \ding{55} \\
%   \hline
%   \textbf{Obscure Information Structure} & \ding{51} & \ding{51} & \ding{55} \\
%   \hline
% \end{tabular}
% \caption{Challenges in different modes}
% \end{table}


\begin{figure*}[h]
  \centering
  \includegraphics[width=\linewidth]{images/design-pipeline.jpg}
  \caption{This figure illustrates an overview of our design pipeline: based on the findings derived from the formative study, we established design requirements to address each finding. Subsequently, we developed corresponding features that constituted the components of our design to meet these requirements.}\label{fig:design-pipeline}
  \Description{This figure illustrates an overview of our design pipeline: based on the findings derived from the formative study, we established design requirements to address each finding. Subsequently, we developed corresponding features that constituted the components of our design to meet these requirements. Specifically, let's start with the correspondence of findings to design requirements. Various behavior patterns and mode-specific challenges (Mutual distractions, short display duration, and danmaku locating difficulties) correspond to DR1. Mode-specific challenges and low information density correspond to DR2. Obscure information structure and danmaku advantages correspond to DR3. Then let's go to the correspondence of design requirements to design features. DR1 corresponds to Overview, Focused, and Exploration modes. DR2 corresponds to danmaku filtering and clustering. DR3 corresponds to danmaku classification. And we also conducted a content analysis to assist automatic danmaku filtering and classification. The Overview, Focused, and Exploration modes constitue the user interface of CoKnowledge. The danmaku filtering, classification and clustering constitue the NLP pipeline of CoKnowledge.}
\end{figure*}