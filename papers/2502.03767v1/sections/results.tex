\section{Results}

% All participant ratings mentioned in Section 5.4 were measured using a 7-point Likert scale. For the data items that were measured under both conditions, we first conducted the Shapiro-Wilk test \cite{shapiro1965analysis} to assess their normality. The results indicated that most of the items did not meet the normality assumption. Therefore, we chose to utilize the Wilcoxon Signed Rank test \cite{woolson2005wilcoxon}. 
In this section, we present our evaluation of CoKnowledge, focusing on its usefulness, design \& interaction, and usability. 
According to Shapiro-Wilk test \cite{shapiro1965analysis}, most of the measurement items did not meet the normality assumption. 
Therefore, following \cite{sakai2017probability, sullivan2012using, tomczak2014need}, we applied the Wilcoxon Signed Rank test \cite{woolson2005wilcoxon} and reported the test statistic, p-value, and the corresponding effect sizes. Descriptive statistics were used to summarize participants' responses regarding the perceived helpfulness of CoKnowledge's features.
Our findings are supported by statistical inferences and further enriched by qualitative reflections from the participants.

\subsection{System Usefulness}
\label{sec:usefulness}
We evaluated CoKnowledge's effectiveness in supporting the assimilation of collective knowledge while sufficiently leveraging the capabilities of danmaku. Detailed statistics are presented in Table \ref{Usefulness}. 
% \xm{The table is at the end of the paper. Move it here.}.

\begin{table*}[]
% \includegraphics[width= 0.95\linewidth]{images/tab-1.jpg}
\begin{tabular}{ccccccc}
\hline
Category                                                                           & Item                                & \begin{tabular}[c]{@{}c@{}}Baseline\\ Mean (S.D.)\end{tabular} & \begin{tabular}[c]{@{}c@{}}CoKnowledge\\ Mean (S.D.)\end{tabular} & Z              & p        & Eff. Size   \\ \hline
\multicolumn{1}{c}{}                                                               & Total                               & 4.95(2.76)                             & 7.73(2.67)                               & -4.66    & 0.000*** & 0.82   \\
\multicolumn{1}{c}{}                                                               & Danmaku                             & 1.79(1.79)                             & 3.90(1.99)                               & -4.81   & 0.000*** & 0.86   \\
\multicolumn{1}{c}{}                                                               & Video                               & 3.16(1.91)                             & 3.83(1.54)                               & -2.07  & 0.039*  & 0.30   \\
\multicolumn{1}{c}{}                                                               & Danmaku comprehension               & 1.08(1.29)                             & 2.35(1.43)                               & -3.81    & 0.000*** & 0.55   \\
\multicolumn{1}{c}{}                                                               & Danmaku recall                      & 0.72(1.30)                             & 1.56(1.23)                               & -2.57     & 0.010**  & 0.37   \\
\multicolumn{1}{c}{}                                                               & Video comprehension                 & 1.48(1.28)                             & 1.88(1.09)                               & -1.99      & 0.047*  & 0.29   \\
\multicolumn{1}{c}{\multirow{-7}{*}{Grade}}                                        & Video recall                        & 1.68(1.04)                             & 1.94(0.88)                               & -1.39      & $0.164^{-}$  & 0.20   \\ \hline
                                                                                   & Total                               & 12.06(3.35)                            & 12.92(3.33)                              & -2.01       & 0.044*  & 0.29   \\
                                                                                   & Danmaku                             & 6.54(2.91)                             & 7.38(2.78)                               & -2.24     & 0.025*  & 0.32   \\
                                                                                   & Video                               & 5.52(0.88)                             & 5.54(0.82)                               & -0.10       & $0.918^{-}$  & 0.01   \\
                                                                                   & Danmaku comprehension               & 3.25(1.63)                             & 3.81(1.45)                               & -2.47      & 0.014*  & 0.36   \\
                                                                                   & Danmaku recall                      & 3.29(1.53)                             & 3.56(1.44)                               & -1.20     & $0.228^{-}$  & 0.17   \\
                                                                                   & Video comprehension                 & 2.77(0.47)                             & 2.83(0.38)                               & -0.73     & $0.467^{-}$  & 0.11   \\
\multirow{-7}{*}{\begin{tabular}[c]{@{}c@{}}Confidence\\ (Objective)\end{tabular}} & Video recall                        & 2.75(0.56)                             & 2.71(0.58)                               & 0.41     & $0.685^{-}$  & 0.06   \\ \hline
                                                                                   & Comprehension                       & 3.96(1.63)                             & 4.74(1.60)                               & -1.86     & 0.049*  &  0.38      \\
                                                                                   & Recall                              & 3.83(1.40)                             & 4.63(1.38)                               & -1.98     & 0.047*  & 0.40       \\
                                                                                   & Confidence                          & 3.71(1.55)                             & 4.83(1.20)                               & -2.51   & 0.012** & 0.62   \\
\multirow{-4}{*}{\begin{tabular}[c]{@{}c@{}}Subjective\\ Evaluation\end{tabular}}  & Efficiency                          & 4.13(1.60)                             & 5.13(1.33)                               & -2.26    & 0.024*  & 0.46   \\ \hline
                                                                                   & Co-presence                         & 3.38(1.74)                             & 4.25(1.77)                               & -2.31      & 0.021*  & 0.47   \\
                                                                                   & Avoid knowledge bias                 & 3.92(1.67)                             & 5.00(1.45)                               & -2.69     & 0.007** & 0.55   \\
\multirow{-3}{*}{\begin{tabular}[c]{@{}c@{}}Danmaku\\ Advantage\end{tabular}}      & Close connection with video content & 4.04(1.85)                             & 5.58(1.38)                               & -3.57    & 0.000***& 0.73   \\ \hline
                                                                                   & Mutual distractions $\downarrow$              & 5.13(1.03)                             & 4.42(1.67)                               & 1.96      & $0.051^{+}$  & 0.40   \\
                                                                                   & Short display duration    $\downarrow$                  & 4.71(1.60)                             & 3.92(1.72)                               &  1.67  & $0.096 ^{+}$  & 0.38   \\
                                                                                   & Danmaku locating difficulties    & 3.92(1.64)                             & 4.67(1.44)                               & -1.45     & $0.147^{-}$  & 0.30   \\
                                                                                   & Low information density    $\downarrow$         & 5.83(1.01)                             & 3.63(1.69)                               & 3.82    & 0.000***& 0.78   \\
\multirow{-5}{*}{Challenges}                                                 & Obscure information structure       & 3.83(1.69)                             & 5.29(1.27)                               & -2.89   & 0.004** & 0.59   \\ \hline
\end{tabular}
\caption{The statistical analysis of system usefulness with Baseline and CoKnowledge, where the p-value (-: p > .100, +: .050 < p < .100, *:p < .050, **:p < .010, ***:p < .001) is reported. $\downarrow$  indicates that a lower score is better. By default, higher scores are better.}
\label{Usefulness}
\end{table*}

\subsubsection{Collective Knowledge Assimilation} 
While using CoKnowledge, we observed a significant improvement in the overall scores of in-task quizzes compared to the baseline system with an average of 56.1\% increase. %The average score across the entire test increased by 56.1\% with CoKnowledge. 
A more detailed analysis revealed 
% \xm{present or past tense? Be consistent across the paper.} 
that the quiz scores for both the danmaku segment and the video segment were significantly higher under the system condition than the baseline condition. Specifically, the mean score increased by 117.8\% for the danmaku segment and 21.1\% for the video segment with CoKnowledge. 
Further examination showed a significant difference in danmaku comprehension, danmaku recall, and video comprehension, although not in video recall. The participants also had significantly fewer `unsure' answers in danmaku-related questions. This indicates that participants acquired more knowledge with high confidence from the danmaku. Meanwhile, their mastery of video content was not undermined. 

These findings align with the subjective evaluations, where participants reported significantly higher levels of \textbf{comprehension and recall} of collective knowledge, higher \textbf{efficiency} in knowledge digestion, and stronger \textbf{confidence} in answering the quiz when using CoKnowledge (Table \ref{Usefulness}).
% In the post-study interview, the participants reflected on the perceived advantages of CoKnowledge. 
% \xm{Give the list of participants (PXX, PXX, ...) holding each view.}
In the post-study interview, all participants recognized CoKnowledge's remarkable effectiveness in aiding collective knowledge assimilation.
P14 stated that \textit{``after using CoKnowledge, I feel like my knowledge has significantly expanded and developed.''} 
% \xm{use `` '' instead of ```` for quotes...}
P19 explained the considerable improvement in absorbing knowledge from live comments, \textit{``Initially, I could only capture less than 10\% of the danmaku [normally], but with CoKnowledge, I was able to process most of the danmaku. As for the video, I could remember most of it even with the baseline system, so the improvement with CoKnowledge wasn’t as significant.''}
Still, several participants noted the benefits of understanding the videos brought by our system. For instance, \textit{“The reduction in the quantity of danmaku [after data processing] allowed me to focus on the relationship between the danmaku and the video... [When answering questions in the video segment,] I first recalled some danmaku and then the corresponding video content''} (P15). 

%as it helped ``considerably expand and develop'' their knowledge (P14). 
% During interviews, all participants explicitly recognized the remarkable effectiveness of CoKnowledge in facilitating the assimilation of collective knowledge. For instance, P14 stated that \textit{``after using CoKnowledge, I feel like my knowledge has significantly expanded and developed.``} 
% Moreover, participants self-reported a significantly higher efficiency in knowledge digestion while using CoKnowledge.
% We also observed significant differences in participants' confidence in answering the quiz, both objectively and subjectively.

% A more detailed analysis reveals that the quiz scores for both the danmaku segment and the video segment were significantly higher under the system condition than the baseline condition. Specifically, the mean score for the danmaku segment increased by 117.8\% with CoKnowledge, while the mean score for the video segment rose by 21.1\%. Regarding confidence, a significant difference was observed in the danmaku segment, while no significant difference was found in the video segment.
% Further examination shows a significant difference in danmaku comprehension, danmaku recall, and video comprehension, with no significant difference observed in video recall. This indicates that even though participants acquired more knowledge from the danmaku, their mastery of video content did not decrease; instead, it improved as well. As P15 noted, \textit{“The reduction in the quantity of danmaku (after data processing) allowed me to focus on the relationship between the danmaku and the video... (When answering questions in the video segment,) I first recalled some danmaku and then the corresponding video content.``} Regarding the greater improvement of the danmaku segment, P19 explained, \textit{``Initially, I could only capture less than 10\% of the danmaku, but with CoKnowledge, I was able to process most of the danmaku. As for the video, I could remember most of it even with the baseline system, so the improvement with CoKnowledge wasn’t as significant.``}


In addition to the overall video analysis, we also analyzed quiz scores separately for different videos, with the statistics presented in Table \ref{Video_type}.
% \xm{move the table here}. 
%Briefly, for the hard science videos, all items exhibited significant differences across the two conditions except for the grade related to video comprehension. For the popular science videos, we observed significant differences in the overall test score ($Z = 2.96, p = 0.003$), danmaku segment ($Z = 3.81, p < 0.001$), and danmaku comprehension segment ($Z = 2.84, p = 0.005$), while there were no significant differences in the video segment score, danmaku recall segment, video comprehension segment, or video recall segment.
Overall, the system's effect on enhancing the assimilation of astronomy video content is greater than that of health videos. 
% , which can be attributed to the lower difficulty in understanding the content of popular science videos and their associated danmaku. 
Several participants (P1-3, P12, P13) gave the reason for this phenomenon in the interview, for example, \textit{``the health videos are more relatable to daily life and their danmaku is rarely difficult to understand. Even without the system, my knowledge absorption was quite effective''} (P1).
% As P1 noted: \textit{ ``[In terms of content,] the popular science videos are more relatable to daily life, especially the danmaku, which are rarely difficult to understand. Even without the system, my knowledge absorption was quite effective.``}

\begin{table*}[]
\begin{tabular}{ccccccc}
\hline
Category                    & Item                    & \begin{tabular}[c]{@{}c@{}}Baseline\\ Mean (S.D.)\end{tabular} & \begin{tabular}[c]{@{}c@{}}CoKnowledge\\ Mean (S.D.)\end{tabular} & \multicolumn{1}{c}{Z}  & \multicolumn{1}{c}{p}   & Eff. Size    \\
\hline
                            & Total                   & 3.64(2.70)                                                     & 6.94(2.13)                                                       & -3.65 & 0.000*** & 0.75 \\
                            & Danmaku                 & 1.54(1.93)                                                     & 3.68(1.96)                                                       & -3.05 & 0.002** & 0.62 \\
                            & Video                   & 2.10(1.88)                                                     & 3.26(1.32)                                                       & -2.01 & 0.044* & 0.41 \\
                            & Danmaku comprehension & 0.93(1.45)                                                     & 2.08(1.19)                                                       & -2.74 & 0.006** & 0.56 \\
                            & Danmaku recall          & 0.61(1.15)                                                     & 1.60(1.27)                                                       & -2.07 & 0.038* & 0.42 \\
                            & Video comprehension     & 0.81(1.14)                                                     & 1.35(0.96)                                                       & -1.59 & $0.111^{-}$ & 0.33 \\
\multirow{-7}{*}{Astronomy} & Video recall            & 1.29(1.11)                                                     & 1.92(0.84)                                                       & 2.37 & 0.018* & 0.48 \\
\hline
                            & Total                   & 6.26(2.15)                                                     & 8.51(2.97)                                                       & -2.96 & 0.003** & 0.60 \\
                            & Danmaku                 & 2.04(1.65)                                                     & 4.13(2.04)                                                       & -3.81 & 0.000*** & 0.78 \\
                            & Video                   & 4.22(1.25)                                                     & 4.39(1.56)                                                       & -0.89  & $0.372^{-}$ & 0.18 \\
                            & Danmaku comprehension & 1.22(1.11)                                                     & 2.61(1.61)                                                       & -2.84 & 0.005** & 0.58 \\
                            & Danmaku recall          & 0.82(1.46)                                                     & 1.51(1.22)                                                       & -1.51  & $0.131^{-}$ & 0.31 \\
                            & Video comprehension     & 2.15(1.04)                                                     & 2.42(0.94)                                                       & -1.13 & $0.260^{-}$ & 0.23 \\
\multirow{-7}{*}{Health}   & Video recall            & 2.07(0.82)                                                     & 1.97(0.94)                                                       & 0.16 & $0.876^{-}$ & 0.03 \\
\hline
\end{tabular}
\caption{The statistical analysis of quiz grades for different video pairs with Baseline and CoKnowledge, where the p-value (-: p > .100, +: .050 < p < .100, *:p < .050, **:p < .010, ***:p < .001) is reported. For all items, a higher score indicates better performance.}
\label{Video_type}
\end{table*}

\subsubsection{Danmaku Utilization}
\label{sec:danmaku-util}

This subsection examines whether CoKnowledge can fully harness danmaku's potential  (Table \ref{Usefulness}), as specified in the requirements derived from the \hyperref[formative-study]{Formative Study}. 
% \xm{reference the corresponding subsection}. %as indicated by findings from the formative study. 

Participants reported a significantly higher \textbf{information density} in the danmaku when using CoKnowledge, and all participants highly commended this aspect in the interview. For instance, P16 noted that \textit{``This greatly lowered the barrier to grasping collective knowledge.''} Furthermore, while using CoKnowledge, participants found the \textbf{structure of collective knowledge} to be significantly clearer and the \textbf{connection between the danmaku and the video content} to be significantly tighter. They attributed this clarity to the integration of various features in the interview. To be more specific, P2, P5, P7, and P15 noted that the classification of danmaku made \textit{``the structure of the danmaku (with respect to the video content) less chaotic''} (P2). Additionally, the progress bar directory provided an overall organization of the video content (P15), and the knowledge graph (P9, P13, P14) enhanced \textit{``the clarity of the structure of danmaku and the corresponding video segments''} (P13).

Even though danmaku comments appear in multiple locations in our system, the participants did not find it significantly more \textbf{challenging to locate} them than using the baseline.
We also observed a marginally significant reduction in the reported \textbf{mutual distraction between danmaku and video content} and in the likelihood of missing danmaku due to \textbf{short display durations} under the CoKnowledge condition. P1, P6, P14, and P16 highly praised the setting that allowed longer danmaku to remain on screen for an extended period, despite that some found the coloring of floating comments (for encoding classification results) somewhat distracting (e.g., P10). 
% Consequently, the participants perceived a significantly stronger \textbf{connection between the danmaku and the video content}. Several of them (P5, P15) attributed this improvement to the classification of danmaku.

Furthermore, participants reported a significantly enhanced \textbf{sense of co-presence} with CoKnowledge. Some participants (P4-6, P12, P24) mentioned that the classification of danmaku allowed them to \textit{``selectively focus on those that shared common topics (with them), which strengthened the sense of identification''} (P24).  Additionally, \textit{``the clearer perception of differing stances in danmaku content''} increased their \textit{``sense of interaction''} (P4). As a result, the system condition demonstrated a significantly stronger ability to \textbf{mitigate knowledge bias} compared to the baseline. As P5 remarked, \textit{``I deliberately focused on danmaku with negative interpretations rather than blindly trusting the uploader’s narrative.''} 
However, some participants held different views. P20 remarked, \textit{``The system's processing of information places me in a higher position to observe a large group of people’s conversations from a distance.''} %which creates a greater sense of distance.``} 
P23 further added, \textit{``After the data is processed [and abstracted], it loses the original sense of watching together with everyone.''}


% There was only a marginal difference in the reported mutual distraction between danmaku and video content across the two conditions. P10 noted that \textit{``the colorful danmaku was somewhat distracting.``} Similarly, participants perceived only a marginal reduction in missed danmaku with short display durations. However, some participants (P1, P6, P14, P16) highly praised the setting that allowed longer danmaku to remain on screen for an extended period. No significant difference was observed regarding CoKnowledge's effectiveness in facilitating navigation to the desired danmaku.

% Concerning the two primary obstacles, participants reported a significantly higher information density in the danmaku when using CoKnowledge. During the interviews, all participants highly commended CoKnowledge's handling of this aspect. For instance, P16 noted that \textit{``This significantly lowered the barrier to assimilating collective knowledge.``} Furthermore, while using CoKnowledge, participants found the structure of collective knowledge to be significantly clearer. In the interviews, participants attributed this clarity to the integration of various features. They noted that the classification of danmaku made \textit{``the structure of the danmaku (with respect to the video content) less chaotic``} (P2, P7). Additionally, the progress bar directory provided an overall organization of the video content (P15), and the knowledge graph enhanced \textit{``the clarity of the structure of danmaku and the corresponding video segments``} (P9, P13, P14).


\subsection{Design \& Interaction}
\label{sec:interaction}
We examined participants' interactions with CoKnowledge and their perceptions of its feature helpfulness. Figure \ref{fig:use_pattern} presents an overview of usage patterns across different stages. Descriptive statistics of the ratings are presented in Figure \ref{fig:user-rating}.

\subsubsection{Use Patterns}
\label{sec:use-parttern}

\begin{figure*}[h]
  \centering
  \includegraphics[width= \linewidth]{images/use-pattern.png}
  \caption{Usage patterns of CoKnowledge across the stages of before, during, and after viewing. Mode transition timelines represent the average mode usage of participants within each pattern group. Box plots illustrate the quantitative distribution of mode activations and time spent in each mode, with data point shapes indicating the pattern associated with individual participants.}\label{fig:use_pattern}
  \Description{Usage patterns of CoKnowledge across the stages of before, during, and after viewing. Mode transition timelines represent the average mode usage of participants within each pattern group. Box plots illustrate the quantitative distribution of mode activations and time spent in each mode, with data point shapes indicating the pattern associated with individual participants. The specific patterns are further explained in section 6.2.1}
\end{figure*}

We identified the participants' interaction patterns with the system by reviewing and analyzing system log data and screen recordings, and further compared them with participants' behaviors in a traditional viewing environment.

Before starting the video (Figure \ref{fig:use_pattern} A), only a few participants (P18, P20, P23) utilized features within the overview mode for preview. Two of them (P18, P23) aimed to capture peaks in danmaku activity and points of interest. P20, on the other hand, sought to \textit{``get a general sense of the video and danmaku to have an overall understanding [of story flow].''} Many participants (16 out of 24) appreciated the functionality of overview mode, emphasizing that, unlike traditional viewing methods relying on brief video descriptions or titles for a cursory understanding of the video (P4, P9), CoKnowledge enabled them to preview the evolution of the entire video and extensive danmaku in a more structured manner.

During video viewing (Figure \ref{fig:use_pattern} B), most participants (22 out of 24) preferred to spend the majority of their time in the focused mode, aligning with traditional viewing habits. For instance, P5 noted, \textit{``My previous habit [of watching science videos] led me to focus primarily on the video section.''} A few of them (P4, P5, P8, P15, P20) preferred to simultaneously view the video and the subtitle-danmaku list on the side view to follow the content in real-time. In contrast to the majority, two participants (P16, P22) spent most of their time in the exploration and overview modes, with P22 stating, \textit{``The amount of information acquired in focused mode is relatively limited.''} Although many participants retained certain aspects of traditional viewing habits, we observed considerable changes in their actual behavior. In conventional environments, participants often minimized danmaku distractions by disabling it (P2, P17, P20), shrinking its display area (P1, P4, P6, P7, P11), or increasing transparency (P1, P4, P19), focusing almost entirely on the video content (P3, P5, P13). With CoKnowledge, the manageable danmaku volume eliminated the need for such adjustments, enabling participants to engage more deeply with danmaku knowledge in their innate format. When encountering confusing or intriguing information during viewing, participants also displayed new strategies. Traditionally, they might have paused, rewatched, or searched online. With CoKnowledge, most participants (15 out of 24) switched to the exploration mode, leveraging features like KG, subtitle-danmaku list, AI-assisted explanations, and related-danmaku display to efficiently investigate the content from alternative perspectives with reduced rewatching. Nonetheless, some participants (P2, P11, P12) found mode switching cumbersome due to the automatic adjustment of video view size, which hindered visibility of video details in exploration mode and overview mode (P11, P22).


After completing the video (Figure \ref{fig:use_pattern} C), some participants (10 out of 24) chose to spend time reviewing the video content. Among them, several (P2, P15, P16, P20, P22, P24) opted to use the overview mode for a comprehensive review and then navigated to the desired timestamps before switching to the exploration mode for thorough analysis (P2, P24). In addition to this mode-integrated approach, other participants (P3, P8, P14) relied on memory to locate specific video timestamps before utilizing the exploration mode, while P6 revisited specific content by browsing the subtitle-danmaku list. Without CoKnowledge, nearly all participants (22 out of 24) indicated they would not have revisited the video, or if they did, it would have been limited to monotonous rewatching.

Overall, compared to traditional viewing environment, CoKnowledge offered diverse features to support various stages of interaction, catering to different needs and accommodating a range of user strategies, thereby enabling multi-dimensional exploration of collective knowledge.

\subsubsection{Helpfulness of the features}
The participants' ratings indicated that the three modes were overall successful in fulfilling the corresponding knowledge needs and were effectively integrated to enhance the assimilation of collective knowledge (Figure \ref{fig:user-rating}). 
We detail the user perception of each design feature below. 
%We begin by discussing the data processing methods. 
\begin{figure*}[h]
  \centering
  \includegraphics[width=0.85\linewidth]{images/user-rating.jpg}
  \caption{User ratings on CoKnowledge's features, where \textit{UI} stands for \textit{User Interface}. For all items, a higher rating indicates better performance.}
  \label{fig:user-rating}
  \Description{User ratings on CoKnowledge’s features, where UI stands for User Interface. For all items, a higher rating indicates better performance. All features received high ratings with means over 5.}
\end{figure*}


\textbf{Filtering} emerged as the CoKnowledge's highest-rated feature. Our participants found the volume of danmaku in the baseline condition overwhelming, leading to an inability to retain any of the comments (P3, P24); by contrast, CoKnowledge's filtering of danmaku reduced the quantity to a manageable level (P9, P11, P14). On the one hand, this allowed participants to absorb the knowledge from time-synced comments while watching the video (P1, P7, P8, P15). 
As P15 stressed, \textit{``Danmaku filtering not only enables me to focus more specifically on \textit{knowledge danmaku} but also more efficiently concentrate on the corresponding video content.''}
On the other hand, this feature significantly lowered their cognitive load when interacting with other features (e.g., \textit{Wordstream} and \textit{KG}) (P15 - 17, P20). However, some (P4, P5, P13) noted that if their goal was pure entertainment, they might avoid filtering, as \textit{``the removed danmaku [, though not informative,] brought joy''} (P13).

The key benefits of \textbf{danmaku classification} were discussed in Section \ref{sec:danmaku-util}. This feature enhanced the sense of co-presence (P12, P24), bridged the gap between the danmaku and video content (P5, P15), and helped structure collective knowledge (P2, P7). Additionally, explicitly labeling the theme of each specific comment allowed users to focus on those related to negative interpretations and supplementary knowledge, thereby avoiding knowledge bias (P5, P15). However, using color to encode these classifications was also noted as potentially distracting (P10).

As for \textbf{danmaku clustering}, many participants (10 out of 24) praised it for reducing the need to mentally process repetitive danmaku while maintaining an idea of the volume. 
% \xm{correct?}.
P2 mentioned, \textit{``When I see a message coming in a large quantity, I pay more attention to it.''} P3 used the information for a different reason, arguing that the same comment posted by many users tended to lack valuable information, and clustering could help avoid such comments all at once. %Regardless of the underlying reasons, this method reduced participants' cognitive load. 
In brief, these data processing methods collectively constitute an NLP pipeline, serving as the foundational framework upon which CoKnowledge is built.


%As shown in Table \ref{User-ratings}, participants rated most of the UI components above 5. 
All main UI components received a user rating above 5 (out of 7), as shown in Figure \ref{fig:user-rating}. 
Regarding \textbf{Wordstream}, some participants noted that it helped them \textit{``understand the content at each time segment, and the classification also provides insights into the overall content of danmaku''} (P1, P19). P4 further commented that the peaks in Wordstream allow them to \textit{``identify moments of intense discussion.''} However, some participants (P2, P3, P6) found Wordstream difficult to understand, as certain keywords were challenging to interpret without contextual information.
Hence, participants' preferences for using Wordstream varied. P23 remarked, \textit{``By reviewing Wordstream before watching the video, I can get a general sense of the danmaku feedback, form some hypotheses, and watch the video with expectations, which can then be validated.''} Some other users (P1, P13, P21, P22) were inclined to employ this feature after watching the video to review the flow of collective knowledge, while P3 referred to it during the video to recall previous content.
The \textbf{progress bar directory}, the other component in overview mode, was favored for its precise indexing of specific parts of the video, besides its ability to structure the video content as mentioned in Section \ref{sec:danmaku-util}. 
% Regarding the other component in overview mode, the progress bar directory, beyond its ability to structure the video content as mentioned in Section \ref{sec:danmaku-util}, P5 noted that it could be used to precisely index specific parts of the video.
 
As for \textbf{KG}, some participants felt that it enabled them to quickly and intuitively grasp the content (P9, P13, P14, P22) and provided a clear structure for the collective knowledge of the current video segment (P15). P17 added, \textit{``With KG, I can specifically understand which aspect of the video content each danmaku corresponds to.''} P12 also mentioned a sense of achievement after fully understanding a knowledge graph. However, others found that the knowledge graph contained too much information for a single segment, making it difficult to keep up (P1, P3, P10). Some also complained that \textit{``the links were cluttered [with many danmaku]''} (P2) and found KG hard to understand due to abstract keywords (P3, P4).

%(May consider delete this paragraph to shorten the result section?) Regarding the features in the side view, 
Approximately one-fourth of the participants considered the \textbf{subtitle-danmaku list} to be a substantial source of information. P9 found it \textit{``very convenient and clear... If I miss something from the danmaku or video content, a quick glance allows me to understand it, making it an excellent tool for filling in the gaps.''}
Some participants (P5, P22) queried \textbf{related danmaku} to \textit{``gain a more comprehensive understanding of a particular comment''} (P22).
Several users (P1, P14, P16, P23) sought out \textbf{AI-assisted explanations}, as they recognized its assistance in quickly grasping unfamiliar concepts (P1, P23), especially for \textit{``hard science videos with challenging terminology''} (P6).

% \xm{Things mentioned in this subsection are mostly pros. Anything to improve or missing?}


\subsection{System Usability}
\label{sec:usability}
\begin{figure*}[h]
  \centering
  \includegraphics[width= 0.85\linewidth]{images/usability.jpg}
  \caption{The usability of CoKnowledge and the baseline system. * indicates a significant difference for the item. $\downarrow$  indicates that a lower rating is better. By default, higher ratings are better.}\label{fig:usability}
  \Description{The evaluation of usability of CoKnowledge and the baseline system. The figure shows that despite the additional interactive features, CoKnowledge still maintained comparable usability and task workload levels to the baseline.}
\end{figure*}


Figure \ref{fig:usability} summarizes our assessment of CoKnowledge's usability. %First, we compared the task load between the two conditions. 
Overall, participants reported significantly greater \textbf{success in accomplishing tasks} with CoKnowledge than with the baseline (\textit{Z} = 2.11, \textit{p} = 0.04), while they perceived no statistical difference in other \textbf{task load} items (e.g., mental, physical, temporal demand, and effort). %Although other items did not show significant differences, the mean ratings indicate that the system condition outperformed the baseline condition. 
This suggests that even with the additional features in CoKnowledge, the workload for users was still on par with that of the baseline condition. As discussed in Sections \ref{sec:usefulness} and \ref{sec:interaction}, this is largely achieved by the processing pipeline for danmaku, which considerably reduced the number of comments and effectively structured the collective knowledge.
Regarding \textbf{general usability}, participants were significantly more inclined to use CoKnowledge again (\textit{Z} = 2.29, \textit{p} = 0.02). 
They did not find our system statistically harder to learn or use than the baseline even with the many added features, despite that some users (P16, P22, P23) mentioned that the additional information could be overwhelming.
%For other dimensions, CoKnowledge received a mean rating of 5.71 for \textit{ease to learn}, 5.25 for \textit{ease to use}, and 2.88 for \textit{inconsistency}, indicating good usability on a 7-point Likert scale. 
% However, some participants expressed concerns, such as \textit{``the difficulty in managing so many features``} (P16, P22) and \textit{``the overwhelming amount of information presented by CoKnowledge``} (P23).










% \begin{table}[]
% \begin{tabular}{ccccccc}
% \hline
% Category                                                                           & Item                                & \begin{tabular}[c]{@{}c@{}}Baseline\\ Mean(S.D.)\end{tabular} & \begin{tabular}[c]{@{}c@{}}CoKnowledge\\ Mean(S.D.)\end{tabular} & Z              & p        & Eff. Size   \\ \hline
% \multicolumn{1}{c}{}                                                               & Total                               & 4.95(2.76)                             & 7.73(2.67)                               & 4.66    & 0.000*** & 0.82   \\
% \multicolumn{1}{c}{}                                                               & Danmaku                             & 1.79(1.79)                             & 3.90(1.99)                               & 4.81   & 0.000*** & 0.86   \\
% \multicolumn{1}{c}{}                                                               & Video                               & 3.16(1.91)                             & 3.83(1.54)                               & 2.07  & 0.039*  & 0.30   \\
% \multicolumn{1}{c}{}                                                               & Danmaku Comprehension               & 1.08(1.29)                             & 2.35(1.43)                               & 3.81    & 0.000*** & 0.55   \\
% \multicolumn{1}{c}{}                                                               & Danmaku Recall                      & 0.72(1.30)                             & 1.56(1.23)                               & 2.57     & 0.010**  & 0.37   \\
% \multicolumn{1}{c}{}                                                               & Video Comprehension                 & 1.48(1.28)                             & 1.88(1.09)                               & 1.99      & 0.047*  & 0.29   \\
% \multicolumn{1}{c}{\multirow{-7}{*}{Grade}}                                        & Video Recall                        & 1.68(1.04)                             & 1.94(0.88)                               & 1.39      & $0.164^{+}$  & 0.20   \\ \hline
%                                                                                    & Total                               & 12.06(3.35)                            & 12.92(3.33)                              & 2.01       & 0.044*  & 0.29   \\
%                                                                                    & Danmaku                             & 6.54(2.91)                             & 7.38(2.78)                               & 2.24     & 0.025*  & 0.32   \\
%                                                                                    & Video                               & 5.52(0.88)                             & 5.54(0.82)                               & 0.10       & $0.918^{-}$  & 0.01   \\
%                                                                                    & Danmaku Comprehension               & 3.25(1.63)                             & 3.81(1.45)                               & 2.47      & 0.014*  & 0.36   \\
%                                                                                    & Danmaku Recall                      & 3.29(1.53)                             & 3.56(1.44)                               & 1.20     & $0.228^{-}$  & 0.17   \\
%                                                                                    & Video Comprehension                 & 2.77(0.47)                             & 2.83(0.38)                               & 0.73     & $0.467^{-}$  & 0.11   \\
% \multirow{-7}{*}{\begin{tabular}[c]{@{}c@{}}Confidence\\ (Objective)\end{tabular}} & Video Recall                        & 2.75(0.56)                             & 2.71(0.58)                               & 0.41     & $0.685^{-}$  & 0.06   \\ \hline
%                                                                                    & Comprehension                       & 3.96(1.63)                             & 4.74(1.60)                               & 1.86     & 0.049*  &  0.38      \\
%                                                                                    & Recall                              & 3.83(1.40)                             & 4.63(1.38)                               & 1.98     & 0.047*  & 0.40       \\
%                                                                                    & Confidence                          & 3.71(1.55)                             & 4.83(1.20)                               &2.51   & 0.012** & 0.62   \\
% \multirow{-4}{*}{\begin{tabular}[c]{@{}c@{}}Subjective\\ Evaluation\end{tabular}}  & Efficiency                          & 4.13(1.60)                             & 5.13(1.33)                               & 2.26    & 0.024*  & 0.46   \\ \hline
%                                                                                    & Co-presence                         & 3.38(1.74)                             & 4.25(1.77)                               & 2.31      & 0.021*  & 0.47   \\
%                                                                                    & Avoid Knowledge Gap                 & 3.92(1.67)                             & 5.00(1.45)                               & 2.69     & 0.007** & 0.55   \\
% \multirow{-3}{*}{\begin{tabular}[c]{@{}c@{}}Danmaku\\ Advantage\end{tabular}}      & Close Connection with Video Content & 4.04(1.85)                             & 5.58(1.38)                               & 3.57    & 0.000***& 0.73   \\ \hline
%                                                                                    & Mutual Distractions                 & 5.13(1.03)                             & 4.42(1.67)                               & 1.96      & $0.051^{+}$  & 0.40   \\
%                                                                                    & Short Duration                      & 4.71(1.60)                             & 3.92(1.72)                               &  1.67  & $0.096 ^{+}$  & 0.38   \\
%                                                                                    & Challenges in Navigating Danmaku    & 3.92(1.64)                             & 4.67(1.44)                               & 1.45     & $0.147^{-}$  & 0.30   \\
%                                                                                    & Low Information Density             & 5.83(1.01)                             & 3.63(1.69)                               & 3.82    & 0.000***& 0.78   \\
% \multirow{-5}{*}{Danmaku Drawback}                                                 & Obscure Information Structure       & 3.83(1.69)                             & 5.29(1.27)                               & 2.89   & 0.004** & 0.59   \\ \hline
% \end{tabular}
% \caption{The statistical analysis of system usefulness with Baseline and CoKnowledge, where the p-value (-: p > .100, +: .050 < p < .100, *:p < .050, **:p < .010, ***:p < .001) is reported.}
% \label{Usefulness}
% \end{table}


% \begin{table}[]
% \begin{tabular}{ccccccc}
% \hline
% Category                    & Item                    & \begin{tabular}[c]{@{}c@{}}Baseline\\ Mean (S.D.)\end{tabular} & \begin{tabular}[c]{@{}c@{}}CoKnowledge\\ Mean (S.D.)\end{tabular} & \multicolumn{1}{c}{Z}  & \multicolumn{1}{c}{p}   & Eff. Size    \\
% \hline
%                             & Total                   & 3.64(2.70)                                                     & 6.94(2.13)                                                       & 3.65 & 0.000*** & 0.75 \\
%                             & Danmaku                 & 1.54(1.93)                                                     & 3.68(1.96)                                                       & 3.05 & 0.002** & 0.62 \\
%                             & Video                   & 2.10(1.88)                                                     & 3.26(1.32)                                                       & 2.01 & 0.044* & 0.41 \\
%                             & Danmaku   Comprehension & 0.93(1.45)                                                     & 2.08(1.19)                                                       & 2.74 & 0.006** & 0.56 \\
%                             & Danmaku Recall          & 0.61(1.15)                                                     & 1.60(1.27)                                                       & 2.07 & 0.038* & 0.42 \\
%                             & Video Comprehension     & 0.81(1.14)                                                     & 1.35(0.96)                                                       & 1.59 & $0.111^{-}$ & 0.33 \\
% \multirow{-7}{*}{Technical} & Video Recall            & 1.29(1.11)                                                     & 1.92(0.84)                                                       & 2.37 & 0.018* & 0.48 \\
% \hline
%                             & Total                   & 6.26(2.15)                                                     & 8.51(2.97)                                                       & 2.96 & 0.003** & 0.60 \\
%                             & Danmaku                 & 2.04(1.65)                                                     & 4.13(2.04)                                                       & 3.81 & 0.000*** & 0.78 \\
%                             & Video                   & 4.22(1.25)                                                     & 4.39(1.56)                                                       & 0.89  & $0.372^{-}$ & 0.18 \\
%                             & Danmaku   Comprehension & 1.22(1.11)                                                     & 2.61(1.61)                                                       & 2.84 & 0.005** & 0.58 \\
%                             & Danmaku Recall          & 0.82(1.46)                                                     & 1.51(1.22)                                                       & 1.51  & $0.131^{-}$ & 0.31 \\
%                             & Video Comprehension     & 2.15(1.04)                                                     & 2.42(0.94)                                                       & 1.13 & $0.260^{-}$ & 0.23 \\
% \multirow{-7}{*}{Popular}   & Video Recall            & 2.07(0.82)                                                     & 1.97(0.94)                                                       & 0.16 & $0.876^{-}$ & 0.03 \\
% \hline
% \end{tabular}
% \caption{The statistical analysis of quiz grades for different video pairs with Baseline and CoKnowledge, where the p-value (-: p > .100, +: .050 < p < .100, *:p < .050, **:p < .010, ***:p < .001) is reported.}
% \label{Video_type}
% \end{table}


% \begin{table}[h]
% \centering
% \begin{tabular}{cccc}
% \hline
% Category & Feature & Mean & S.D. \\
% \hline
% \multirow{4}{*}{Mode} 
% & Overview & 5.38 & 1.50 \\
% & Focused & 5.63 & 1.06 \\
% & Exploration & 5.58 & 1.06 \\
% & Integration & 5.63 & 0.97 \\
% \hline
% \multirow{3}{*}{Data processing methods} 
% & Filtering & 5.92 & 1.02 \\
% & Classification & 5.46 & 1.53 \\
% & Clustering & 5.46 & 1.47 \\
% \hline
% \multirow{7}{*}{UI components} 
% & \textit{KG} & 5.54 & 1.25 \\
% & \textit{Wordstream} & 5.38 & 1.50 \\
% & Progress bar directory & 5.46 & 1.29 \\
% & Stream & 4.33 & 1.47 \\
% & List & 5.25 & 1.39 \\
% & Related danmaku& 5.29 & 1.76 \\
% & AI explanation & 5.21 & 1.87 \\
% \hline
% \end{tabular}
% \caption{User ratings on CoKnowledge's features. \textit{UI} stands for \textit{User Interface}}
% \label{User-ratings}
% \end{table}



% \begin{table}[]
% \begin{tabular}{ccccccc}
% \hline
% Category                        & Item            & \multicolumn{1}{c}{\begin{tabular}[c]{@{}c@{}}Baseline\\ Mean (S.D.)\end{tabular}} & \multicolumn{1}{l}{\begin{tabular}[c]{@{}c@{}}CoKnowledge\\ Mean (S.D.)\end{tabular}} & \multicolumn{1}{c}{Z}    & \multicolumn{1}{c}{p}   & Eff. Size \\
% \hline
%                                 & Mental demand   & 4.58 (1.89)  & 4.17 (1.66)  & 0.77  & $0.44^{+}$ & 0.16  \\
%                                 & Physical demand & 3.67 (1.83)  & 3.17 (1.34)  & 0.67  & $0.50 ^{+}$ & 0.14  \\
%                                 & Temporal demand & 3.83 (1.86)  & 2.96 (1.92)  & 1.59  & $0.11 ^{+}$ & 0.32  \\
%                                 & Performance     & 3.46 (1.41)  & 4.21 (1.35)  & 2.11  & 0.04*  & 0.43  \\
%                                 & Effort          & 4.33 (1.52)  & 4.21 (1.14)  & 0.27  & $0.79 ^{+}$ & 0.05  \\
% \multirow{-6}{*}{Task load} & Frustration     & 3.00 (2.11)  & 2.42 (1.50)  & 1.13  & $0.26 ^{+}$ & 0.23  \\\hline
%                                 & Ease to learn   & 5.71 (1.04)  & 5.71 (1.16)  & 0.16  & $0.88 ^{+}$ & 0.03  \\
%                                 & Ease to use     & 5.21 (1.29)  & 5.25 (1.45)  & 0.13  & $0.89  ^{+}$& 0.03  \\
%                                 & Future use      & 4.96 (1.40)  & 5.92 (1.14)  & 2.29  & 0.02*  & 0.47  \\
% \multirow{-4}{*}{SUS}           & Inconsistency   & 3.54 (1.41)  & 2.88 (1.33)  & 1.31  & $0.19 ^{+}$ & 0.27  \\
% \hline
% \end{tabular}
% \caption{The statistical analysis of usability of Baseline and CoKnowledge, where the p-value (-: p > .100, +: .050 < p < .100, *:p < .050, **:p < .010, ***:p < .001) is reported. \textit{SUS} stands for \textit{System Usability Scale}}
% \label{Usability}
% \end{table}
