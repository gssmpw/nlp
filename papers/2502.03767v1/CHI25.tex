%%
%% This is file `sample-acmlarge.tex',
%% generated with the docstrip utility.
%%
%% The original source files were:
%%
%% samples.dtx  (with options: `all,journal,bibtex,acmlarge')
%% 
%% IMPORTANT NOTICE:
%% 
%% For the copyright see the source file.
%% 
%% Any modified versions of this file must be renamed
%% with new filenames distinct from sample-acmlarge.tex.
%% 
%% For distribution of the original source see the terms
%% for copying and modification in the file samples.dtx.
%% 
%% This generated file may be distributed as long as the
%% original source files, as listed above, are part of the
%% same distribution. (The sources need not necessarily be
%% in the same archive or directory.)
%%
%%
%% Commands for TeXCount
%TC:macro \cite [option:text,text]
%TC:macro \citep [option:text,text]
%TC:macro \citet [option:text,text]
%TC:envir table 0 1
%TC:envir table* 0 1
%TC:envir tabular [ignore] word
%TC:envir displaymath 0 word
%TC:envir math 0 word
%TC:envir comment 0 0
%%
%%
%% The first command in your LaTeX source must be the \documentclass
%% command.
%%
%% For submission and review of your manuscript please change the
%% command to \documentclass[manuscript, screen, review]{acmart}.
%%
%% When submitting camera ready or to TAPS, please change the command
%% to \documentclass[sigconf]{acmart} or whichever template is required
%% for your publication.
%%
%%
% \documentclass[manuscript,review,anonymous]{acmart}
\documentclass[sigconf]{acmart}
\let\Bbbk\relax
\usepackage{amsmath, amssymb, amsfonts}
\AtBeginDocument{%
  \providecommand\BibTeX{{%
    Bib\TeX}}}
% \usepackage{array}
% \usepackage{amssymb}
% \citestyle{acmauthoryear}
\usepackage{pifont}
\usepackage[normalem]{ulem}
\usepackage{colortbl}
\usepackage{xcolor}
\usepackage{enumitem}
\usepackage{todonotes}
\usepackage{booktabs} 
\usepackage{multirow}
\usepackage{tabularx}
\usepackage{makecell}
\usepackage{tikz}
\usepackage{graphicx}
\usepackage{subcaption}
\usepackage{tikz}
\newcommand*\circled[1]{\tikz[baseline=(char.base)]{
            \node[shape=circle,draw,inner sep=0.4pt, minimum size=12.8pt] (char) {\vphantom{1g}\normalfont\small\textbf{#1}};}}
% \newcommand{\mxj}[1]{{\color{red} #1}}
% \newcommand{\xm}[1]{{\color{blue} #1}}
% \newcommand{\yh}[1]{{\color{purple} #1}}
% \newcommand{\mm}[1]{{\color{orange} #1}}
%%

%% Rights management information.  This information is sent to you
%% when you complete the rights form.  These commands have SAMPLE
%% values in them; it is your responsibility as an author to replace
%% the commands and values with those provided to you when you
%% complete the rights form.
\copyrightyear{2025}
\acmYear{2025}
\setcopyright{acmlicensed}\acmConference[CHI '25]{CHI Conference on Human Factors in Computing Systems}{April 26-May 1, 2025}{Yokohama, Japan}
\acmBooktitle{CHI Conference on Human Factors in Computing Systems (CHI '25), April 26-May 1, 2025, Yokohama, Japan}
\acmDOI{10.1145/3706598.3713682}
\acmISBN{979-8-4007-1394-1/25/04}



%%
%% These commands are for a JOURNAL article.
% \acmJournal{POMACS}
% \acmVolume{37}
% \acmNumber{4}
% \acmArticle{111}
% \acmMonth{8}


\begin{document}

%%
%% The "title" command has an optional parameter,
%% allowing the author to define a "short title" to be used in page headers.
\title{CoKnowledge: Supporting Assimilation of Time-synced Collective Knowledge in Online Science Videos}

%%
%% The "author" command and its associated commands are used to define
%% the authors and their affiliations.
%% Of note is the shared affiliation of the first two authors, and the
%% "authornote" and "authornotemark" commands
%% used to denote shared contribution to the research.

% \authornote{Both authors contributed equally to this research.}

% \orcid{1234-5678-9012}
% \author{G.K.M. Tobin}
% \authornotemark[1]
% \email{webmaster@marysville-ohio.com}
\author{Yuanhao Zhang}
\email{yzhangiy@connect.ust.hk}
\affiliation{%
  \institution{Hong Kong University of Science and Technology}
  \city{Hong Kong}
  \country{China}
}

\author{Yumeng Wang}
\email{ywanglu@connect.ust.hk}
\affiliation{%
  \institution{Hong Kong University of Science and Technology}
  \city{Hong Kong}
  \country{China}
}


\author{Xiyuan Wang}
\email{wangxy7@shanghaitech.edu.cn}
\affiliation{%
  \institution{ShanghaiTech University}
  \city{Shanghai}
  \country{China}
}

\author{Changyang He}
\email{changyang.he@mpi-sp.org}
\affiliation{%
  \institution{Max Planck Institute for Security and Privacy}
  \city{Bochum}
  \country{Germany}
}

\author{Chenliang Huang}
\email{clhuang77@gmail.com}
\affiliation{%
  \institution{New York University}
  \city{New York}
  \country{United States}
}

\author{Xiaojuan Ma}
\email{mxj@cse.ust.hk}
\affiliation{%
  \institution{Hong Kong University of Science and Technology}
  \city{Hong Kong}
  \country{China}
}
%%
%% By default, the full list of authors will be used in the page
%% headers. Often, this list is too long, and will overlap
%% other information printed in the page headers. This command allows
%% the author to define a more concise list
%% of authors' names for this purpose.
\renewcommand{\shortauthors}{Zhang et al.}





%%
%% The abstract is a short summary of the work to be presented in the
%% article.
\begin{abstract}
  % In online video-sharing platforms, viewers engage with science content through danmaku, a system of scene-aligned, time-synced comments that, together with the video content, constitute “collective knowledge.” However, the inherent nature of danmaku often impedes viewers' ability to effectively assimilate this collective knowledge in science videos. With a formative study, we examined viewers' practices for processing collective knowledge and the specific barriers they encountered. Building on these insights, we designed a processing pipeline to filter, classify, and cluster danmaku, leading to the development of CoKnowledge—a tool incorporating video abstracts, knowledge graphs, and supplementary configurations. Through a within-subject study (N=24), CoKnowledge could significantly enhance participants’ comprehension and recall of collective knowledge compared to a Bilibili-like baseline. Additionally, we analyzed user interaction patterns and perceptions of the design elements. Finally, we present design considerations for developing similar support tools.
  Danmaku, a system of scene-aligned, time-synced, floating comments, can augment video content to create `collective knowledge'. However, its chaotic nature often hinders viewers from effectively assimilating the collective knowledge, especially in knowledge-intensive science videos. With a formative study, we examined viewers' practices for processing collective knowledge and the specific barriers they encountered. Building on these insights, we designed a processing pipeline to filter, classify, and cluster danmaku, leading to the development of CoKnowledge -- a tool incorporating a video abstract, knowledge graphs, and supplementary danmaku features to support viewers' assimilation of collective knowledge in science videos. A within-subject study (N=24) showed that CoKnowledge significantly enhanced participants’ comprehension and recall of collective knowledge compared to a baseline with unprocessed live comments. Based on our analysis of user interaction patterns and feedback on design features, we presented design considerations for developing similar support tools. 

\end{abstract}

%%
%% The code below is generated by the tool at http://dl.acm.org/ccs.cfm.
%% Please copy and paste the code instead of the example below.
%%
\begin{CCSXML}
<ccs2012>
   <concept>
       <concept_id>10003120.10003121.10003129</concept_id>
       <concept_desc>Human-centered computing~Interactive systems and tools</concept_desc>
       <concept_significance>500</concept_significance>
       </concept>
   <concept>
       <concept_id>10003120.10003121.10011748</concept_id>
       <concept_desc>Human-centered computing~Empirical studies in HCI</concept_desc>
       <concept_significance>300</concept_significance>
       </concept>
 </ccs2012>
\end{CCSXML}

\ccsdesc[500]{Human-centered computing~Interactive systems and tools}
\ccsdesc[300]{Human-centered computing~Empirical studies in HCI}

%%
%% Keywords. The author(s) should pick words that accurately describe
%% the work being presented. Separate the keywords with commas.
\keywords{Collective Knowledge, Online Video Platforms, Danmaku, Science Communication}

% \received{20 February 2007}
% \received[revised]{12 March 2009}
% \received[accepted]{5 June 2009}

\begin{teaserfigure}
  \includegraphics[width=\textwidth]{images/teaser.png}
  \caption{Overview of design features in CoKnowledge: A) Overview mode integrated a progress bar directory and Wordstream for video knowledge abstraction. B) Focused mode enlarged the video window with minimal interactive features. C) Exploration mode provided timestamped knowledge graphs for structured analysis. D) Side view, constant across all modes, offered supplementary danmaku features, including related danmaku display, AI-assisted explanations, and a subtitle-danmaku list.}
  \Description{Overview of design features in CoKnowledge: A) Overview mode integrated a progress bar directory and Wordstream for video knowledge abstraction. B) Focused mode enlarged the video window with minimal interactive features. C) Exploration mode provided timestamped knowledge graphs for structured analysis. D) Side view, constant across all modes, offered supplementary danmaku features, including related danmaku display, AI-assisted explanations, and a subtitle-danmaku list. The system defaulted to Overview mode when users just loaded a video. When users began playing the video, CoKnowledge automatically transitioned to Focused mode. When users paused at a specific timestamp, CoKnowledge entered Exploration mode.}
  \label{fig:teaser}
\end{teaserfigure}

% \begin{figure}[h]
%   \centering
%   \includegraphics[width=\linewidth]{images/teaser.png}
%   \caption{User interface of Coknowledge...\xm{the subfigures in each mode are too small to read...}}\label{fig:teaser}
% \end{figure}

%%
%% This command processes the author and affiliation and title
%% information and builds the first part of the formatted document.
\maketitle


\documentclass[../main.tex]{subfiles}
\graphicspath{{../images/}}
\makeatletter
\def\input@path{{../images/}}
\makeatother
\begin{document}
\section{Introduction}
\begin{figure}
\centering
\begin{tikzpicture}
\node[inner sep=0pt] (ws) at (0, 0) {
\includegraphics[height=.4\textwidth, trim={10cm 0 10cm 0},clip]{world_space.png}};
\node[inner sep=0pt] (cs) at (6,0) {\includegraphics[height=.4\textwidth, trim={10cm 1cm 10cm 4cm},clip]{conf_space.png}};
\end{tikzpicture}
\vspace{-5pt}
\label{fig:pbrm_intro}
\caption{\textbf{Left}: Shows world space obstacles as grey spheres. Robots start and goal configuration is colored red and green, respectively. Configurations along the computed path are colored transparent blue. \textbf{Right:} Mapped world space scenario to configuration space. Obstacle region is the grey mesh. Red spheres are collision-free regions computed by the neural SCDF. The optimized shortest path in the convex corridor is the blue curve.}
\vspace{-25pt}
\end{figure}
Motion planning is the problem of finding a collision-free trajectory that connects a given start and goal configuration. The planning takes place in the configuration space of the robot. For single body robots, like mobile robots or drones, the configuration space and the world space are usually the same. This simplifies the planning, since explicit obstacle representations are available which enables geometrical tools like separating hyperplanes, smallest distance to obstacles etc., to be used when designing motion planning algorithms. For multi-body robots like manipulators, the situation is completely different. The world space obstacles are usually mapped to non-convex regions, and to make the problem even harder, the mapping is usually not known. Forming explicit representations of the obstacle region in the configuration space is usually too expensive or intractable. Despite all of this, sampling based planners are used with great success, which mainly is due to their use of implicit representations of the obstacle region. The basic idea is to construct a graph in the configuration space that covers and connects the collision-free region. From this graph, a path can be extracted that connects a given start and goal configuration. The approach is computationally expensive, since the graph is constructed with the smallest geometrical building block available, points, which represents a collision-check. Furthermore, the extracted paths from the graph are non-smooth and jagged due to the stochastic nature of the approach. This adds an additional post-processing step to the process, where the paths are shortcutted and smoothened, before the path can be used for tracking. Clearly a lot of time is invested to form this graph and produce smooth paths. Thus, if the obstacles start to move, then all of this work is done in no use, since all points that make up this graph need to be re-verified, which is simply too time consuming to be done in real time.
\\\\
In this work, we want to address the existing drawbacks of the sampling based planners. Our main contribution is an improved motion planner where each vertex in the graph covers a collision-free region in the form of a sphere instead of a point and where the edges are formed with neighboring intersecting spheres. This representation has the advantage of instead of returning piecewise linear paths, returning a sequence of overlapping spheres, i.e. a convex corridor, that connects a given start and goal configuration, illustrated in Figure \ref{fig:pbrm_intro}. This convex corridor allows us to use convex optimization to produce smooth trajectories, instead of computationally expensive post-processing methods. The representation further allows us to estimate the coverage of the collision-free space, which gives us awareness and feedback in the offline roadmap construction phase. Finally, our representation is simple to adapt to moving obstacles, simply requery for the new radii and recheck for intersections. 
\\\\
The spherical collision-free regions are formed using a signed distance function (SDF), which is a function that returns the smallest distance from an arbitrary point to the boundary of an obstacle. As the name implies, the distance is signed, thus if the point is inside the obstacle it is negative otherwise positive. If the distance is positive, a sphere with radius equal to the distance is guaranteed to cover a collision-free region. Using an SDF in motion planning is not new, but what is novel about our approach is that we express the distance in the configuration space instead of the world space and by doing so allows us to form these convex collision-free regions. We refer to the resulting SDF as a signed configuration distance function (SCDF). Computing an SCDF analytically is non-trivial, our approach is therefore to parameterize the SCDF with a deep neural network and learn the mapping by supervised learning. Our resulting neural SCDF can compute distances for different parameter values of obstacle shapes and we also show how multiple distances can be combined, thus making our approach flexible.
\section{Related work}
Motion planning algorithms can roughly be divided into three families, grid-based, sampling based and optimization based methods. Grid-based methods (GBM) discretize the planning space from which a graph is then compiled. A standard search method is A$^\star$ \citep{a_star}, which is classified as an \textit{informed} search method, since it employs a heuristic function to speed up the search. A$^\star$ guarantees to return an optimal path at the level of discretization used. GBMs usually discretize the planning space by a regular lattice and this limits the GBMs to problems with low dimensionality due to the curse of dimensionality. Thus, GBMs are usually limited to single-body robots where the degrees of freedom (DOF) are low. To overcome the inherent scaling problem with the GBMs, stochastic methods are usually used for multi-body robots. These methods are termed as sampling-based methods (SBM) and core members within this family are the rapidly-exploring random trees (RRT) \citep{rrt} and the probabilistic roadmap (PRM) \citep{prm}. RRT grows a tree from the start configuration and explores the collision-free region in a rapid way until it is able to connect to the goal region. RRT is usually improved by bi-directional planning \citep{rrt_connect}, i.e. an additional tree is grown from the goal configuration and the trees are tested for connection after any tree has been expanded. RRT is a single-query method, thus it searches for a path from scratch each time it is queried. Contrary to this, PRM is a multi-query method, which solves for multiple queries without starting from scratch. PRM does this by creating a roadmap (graph) that covers the collision-free space as an offline step. The graph is then used to solve for multiple queries. PRMs are used in cases where the environment does not change since the extra offline step is too computationally costly and needs to be re-done if the environment is changed. In our work, we address this inherent issue by using a different roadmap representation. Our vertices in the graph cover a collision-free region in the form of spheres and we form the edges by checking for intersecting spheres. If something in the environment changes, we recompute the spheres radii and recheck the intersections, without relying on collision detection. We use a trained neural network to compute the sphere radius, therefore querying for the radius can be done fast, hence our representation enables the PRM for dynamic environments.
\\\\
In the recent decades, optimization based methods (OBM) \citep{chomp, schulman, itomp, stomp} have been introduced as an alternative to SBM for multi-body robots. Like the SBM, the OBMs scale well to higher dimensional problems and produce smoother motion. It is common to use a SDF in the optimization since it is a smooth function, thus enabling gradient-based methods. However, the standard way of expressing the SDF is in world space. The distance therefore needs to be mapped to the configuration space by the forward kinematics. This mapping makes the optimization problem a non-linear program (NLP), which is computationally expensive to solve. Recently, a different approach has been proposed. In \cite{mp_gcs} motion planning is formulated as a convex optimization problem by using the graph of convex sets framework \citep{gcs}. The underlying idea is to decompose the collision-free space into intersecting convex sets from which a convex optimization problem is formulated. In cases where an explicit representation of the obstacles in the configuration space exists, like for single-body robots, creating collision-free convex regions can be done fast \citep{iris}. For multi-body robots, this is non-trivial. Existing work does this successfully \citep{iris_nlp, iris_c} by an optimization based approach, but the methods are still too time consuming to be used in the presence of moving obstacles. Our approach is instead to use deep learning to learn an SDF expressed in the configuration space. With this, we can query for shortest distances to the collision boundary, which allows us to expand spherical regions which are collision-free. Our approach is fast and therefore enables our suggested roadmap planner to be used in dynamic environments.
\\\\
Recent research has focused on learning collision detection \citep{fk_kernel_distance, diffco, graphdistnet} by predicting the signed distance between the robot links and the surrounding obstacles in the world space. The learned SDF is used in trajectory optimization but since the distance is expressed in the world space, the problem becomes an NLP and therefore takes a long time to solve. We take a novel approach and suggest to instead express the signed distance in the configuration space. This allows us to improve the PRM at the same time as it enables convex optimization for trajectory optimization, which runs faster and is more reliable than NLP solvers. In \cite{cspf} a learned signed distance function in the configuration space is proposed similar to our approach. However, their approach is restricted to point cloud representations, while we propose to represent the obstacles as parameterized geometric shapes, e.g. spheres. Furthermore, we also show how to use our learned SCDF to improve an existing roadmap planner.
\section{Problem formulation}
A robot is located in the world space, $\W \subset \R^3 $. The unique location of the robot is given by its configuration $\q \in \C$, where $\C$ is the configuration space. The set of points covered by the robots bodies at a certain configuration is expressed as $\B(\q) \subset \W$. The robot is surrounded by $\NrObst$ obstacles $\O = \bigcup_{i=1}^{\NrObst} \O_i$, where  $\O_i \subset \W$. The representation of the obstacle in the configuration space is the set $\C\O_i = \{\q \in \C \: |\: \B(\q) \cap \O_i \neq \emptyset \}$. The obstacle space is formed as $\Co = \bigcup_{i=1}^{\NrObst} \C \O_i$. The complement is referred to as the free space, $\Cf = \C \setminus \Co$. The path planning problem is a tuple, ($\Cf$, $\qStart$, $\qGoal$), where we want to connect a query pair, consisting of a start, $\qStart$, and goal configuration, $\qGoal$, with a geometric path, $\q(s): [0, 1] \mapsto \Cf$, such that $\q(0)=\qStart$ and $\q(1)=\qGoal$, or report correctly when such a path does not exist.
\end{document}

\section{Related Work}
\label{sec:related-work}
%We now contextualize our work with related literature so that our contributions are highlighted. We cover FMTS, perturbations in time-series, 
% robustness testing of FMs, 
%and rating of AI systems. 

\noindent \textbf{Foundation Models Supporting Time Series} 
The use of FMs for time series forecasting has advanced significantly. 
% \cite{lu2022frozen} first demonstrated that transformers pre-trained on text data (LLMs) can effectively solve sequence modeling tasks in other modalities, paving the way for leveraging language pre-trained transformers for time series analysis. Recent studies have focused on reprogramming LLMs for time series tasks through parameter-efficient fine-tuning and suitable tokenization strategies \cite{zhou2023one, gruver2024large, jin2023time, cao2023tempo, ekambaram2024tiny}. These methods have successfully adapted transformers to the unique challenges of time series forecasting. \cite{zhou2023one} and \cite{jin2023time} further illustrate the versatility and robustness of fine-tuned language pre-trained transformers for diverse time series tasks.
\cite{lu2022frozen} showed that transformers pre-trained on text data can solve sequence modeling tasks in other modalities, enabling their application to time series analysis. Recent studies have reprogrammed LLMs for time series tasks through parameter-efficient fine-tuning and tokenization strategies \cite{zhou2023one, gruver2024large, jin2023time, cao2023tempo, ekambaram2024tiny}. 
% These methods have successfully adapted transformers to the unique challenges of time series forecasting. 
\cite{zhou2023one} and \cite{jin2023time} further illustrate the versatility and robustness of fine-tuned language pre-trained transformers for diverse time series tasks.
% Several models have contributed to the advancement of time series forecasting. \cite{ansari2024chronos} and \cite{woo2024unified} have improved forecasting accuracy and model generalization.  
% % \cite{ansari2024chronos} and \cite{woo2024unified} have pushed the boundaries of forecasting accuracy and model generalization. 
% \cite{rasul2023lag} and \cite{das2023decoder} have explored new tokenization strategies and fine-tuning methods to improve model performance. Additionally, \cite{garza2023timegpt} and \cite{ekambaram2024tiny} have focused on creating lightweight and efficient models for real-time applications. \cite{talukder2024totem} stands out with its unique approach to integrating multiple temporal patterns, enhancing forecasting precision.
% FMs trained from scratch have achieved SOTA on time series tasks. Zero-shot forecasting, exemplified by \cite{gruver2024large}, showcases the ability of these models to make accurate predictions without domain-specific training. \cite{cao2023tempo} and \cite{goswami2024moment} have introduced approaches to enhance the performance and efficiency of time series models, leveraging transformer architectures to capture temporal dependencies more effectively. In our experiments, we select Gemini-V and Phi-3 as the GP models and Chronos and MOMENT as TS models due to their SOTA performance in their respective categories.
Several models have advanced time series forecasting. \cite{ansari2024chronos} and \cite{woo2024unified} have improved forecasting accuracy and model generalization, while
% \cite{ansari2024chronos} and \cite{woo2024unified} have pushed the boundaries of forecasting accuracy and model generalization. 
\cite{rasul2023lag} and \cite{das2023decoder} have explored new tokenization strategies and fine-tuning methods. \cite{garza2023timegpt} and \cite{ekambaram2024tiny} developed lightweight models for real-time applications, and \cite{talukder2024totem} integrated multiple temporal patterns to improve precision. FMs trained from scratch, like \cite{gruver2024large}, achieved SOTA in zero-shot forecasting, with \cite{cao2023tempo} and \cite{goswami2024moment} further improving model performance. 
%In our experiments, we select Gemini-V and Phi-3 as the GP models and Chronos and MOMENT as TS models due to their SOTA performance in their respective categories.
Please see Section~\ref{sec:exp_app} for the FMTS we selected due to their SOTA performance in their respective categories.

%The use of FMs for time series forecasting has seen significant advancements in recent years. \cite{lu2022frozen} first demonstrated that transformers pre-trained on text data (LLMs) can effectively solve sequence modeling tasks in other modalities. This work opened the door to leveraging language pre-trained transformers for time series analysis. Recent studies have built on this foundation, focusing on reprogramming LLMs for time series tasks through parameter-efficient fine-tuning and suitable tokenization strategies \cite{zhou2023one, gruver2024large, jin2023time, cao2023tempo, ekambaram2024tiny}. These methods have proven successful in adapting the powerful capabilities of transformers to the unique challenges of time series forecasting. OneFitsAll \cite{zhou2023one} and Time-LLM \cite{jin2023time} further illustrate how language pre-trained transformers can be fine-tuned for diverse time series tasks, demonstrating their versatility and robustness. 
% \zhen{reason why we didn't include these models in our study, weights not available? or other justification, to prevent that naturally raised question from readers.}\kl{Good point. We need to discuss. I added 2 sentences at the bottom but they are probably not very convincing.}
%Several other models have contributed to the advancement of time series forecasting. Chronos \cite{ansari2024chronos} and Moirai \cite{woo2024unified} have pushed the boundaries of forecasting accuracy and model generalization. Lag-llama \cite{rasul2023lag} and TimesFM \cite{das2023decoder} have explored new tokenization strategies and fine-tuning methods to improve model performance. Additionally, Time-GPT1 \cite{garza2023timegpt} and Tiny-Time Mixers \cite{ekambaram2024tiny} have focused on creating lightweight and efficient models suitable for real-time applications. TOTEM \cite{talukder2024totem} stands out with its unique approach to integrating multiple temporal patterns, further enhancing forecasting precision.
%Aside from reprogramming LLMs for time series, FMs trained from scratch have achieved SOTA on times series tasks. 
%Zero-shot forecasting, exemplified by \cite{gruver2024large}, showcases the ability of these models to make accurate predictions without domain-specific training.  TEMPO \cite{cao2023tempo} and MOMENT \cite{goswami2024moment} have introduced approaches to enhance the performance and efficiency of time series models, leveraging transformer architectures to capture temporal dependencies more effectively.
% \zhen{and these are on various time series tasks including time series forecasting?}
% \zhen{These are models that are specifically trained for time series forecasting, I'd suggest mentioning them first after the LLM reprogramming, and then expanding to the models that are trained across time series tasks instead. The flow of this subsection feels a bit odd as of now.} \kl{Done.}
%In our experiments, we select Gemini-V and Phi-3 as the GP models and Chronos and MOMENT as TS models due to their SOTA performance in their respective categories. 

%\vspace{-0.3em}
\noindent \textbf{Perturbations in Time Series Data} TS data is commonly stored in spreadsheets and databases, which are prone to changes due to acts of omission (e.g., negligence, data-entry errors) or commission (e.g., adversarial attacks, sabotage). Omission errors are most common \cite{spreadsheets-errors-risks-survey}. Tools like Microsoft Excel and Google Sheets are widely used for data collection and analysis, allowing end-user programming \cite{spreadsheets-future-workshop}. However, over 90\% of spreadsheets contain errors due to issues like incorrect formulae, leading to multi-billion dollar losses \cite{spreadsheet-qa-survey}.
%\cite{spreadsheet-qa-survey,spreadsheets-errors-risks-survey}.
Adversarial attacks are also increasing in data stores and AI models for tasks like forecasting.
% \cite{papernot2016transferability} introduced a black-box attack method using a substitute model to generate adversarial examples, demonstrating transferability across tasks. \cite{baluja2017adversarial} focused on white-box attacks using gradient information. 
\cite{karim2019adversarial} adapted these concepts to time series, exploring both black-box and white-box attacks. \cite{oregi2018adversarial} revealed the vulnerability of distance-based classifiers. \cite{rathore2020untargeted} examined various adversarial attacks on time series classifiers. TSFool \cite{li2022tsfool} introduced a multi-objective black-box attack to craft imperceptible adversarial time series to fool RNN classifiers.
%Time series (TS) data is widely stored and manipulated in spreadsheets and databases. These are also the tools which see considerable changes or perturbations due to acts of omission that are unintended (e.g., negligence, data-entry errors) or commission which are deliberate (e.g., adversarial attacks, sabotage). 
%Among these, changes due to omission are most common \cite{spreadsheets-errors-risks-survey}.
%For example, a spreadsheet, implemented in tools like Microsoft Excel and Google Sheets, is a common data collection and analysis environment that also allows end-user programming \cite{spreadsheets-future-workshop}. They are used widely at the workplace and are often a door opener to more advanced scientific tools. But gaining expertise in them needs practice since a large proportion of spreadsheets ($\succ$ 90\%) are known to have errors due to issues like incorrect formulae caused by improper understanding of behavior during routine operations like copy-paste and end-user programming, which have caused losses of multi-billion dollars \cite{spreadsheet-qa-survey,spreadsheets-errors-risks-survey}.
% \zhen{do we need to relate our perturbations to these attacks? otherwise, we must manage the readers' expectations on what types of perturbations we focus on other than adversarial attacks, and motivate it properly}
%\zhen{Play down this a bit, and emphasize and justify why we focus on the type of perturbations we consider in the paper, to mimic operational errors in practice apart from adversarial attacks, citing the 2024 and 1996 papers Biplav added.} 
%Furthermore, adversarial attacks are also increasing both in data stores and in AI models created to solve tasks like forecasting.
%Foundational work by ~\cite{papernot2016transferability} introduced a black-box attack method that involved training a substitute model to generate adversarial examples capable of misleading the target model, demonstrating the transferability property across similar tasks. In contrast, research by ~\cite{baluja2017adversarial} focused on white-box attacks, using gradient information and probabilistic outputs to craft adversarial examples. Researchers~\cite{karim2019adversarial} have adapted these concepts to the time series domain, exploring both black-box and white-box attacks on time series classification models. In addition, ~\cite{oregi2018adversarial} revealed the susceptibility of distance-based time-series classifiers to adversarial examples. ~\cite{rathore2020untargeted} examined untargeted, targeted, and universal adversarial attacks on time series classifiers, demonstrating the effectiveness of these attacks across various datasets. TSFool~\cite{li2022tsfool} introduced a multi-objective black-box attack to craft highly imperceptible adversarial time series to fool RNN classifiers.
%Adversarial attacks on time-series data are initially focused on time-series classification tasks, leveraging concepts adapted from adversarial attacks in other domains.
%explored adversarial sample crafting for time series classification using elastic similarity measures,  %These works collectively underscore the ongoing efforts to understand and mitigate the risks posed by adversarial attacks on time series classification models.
% More recently, research into adversarial attacks on time series forecasting models has revealed distinct challenges and novel attack strategies. One primary challenge is targeted attacks. While targeted adversarial attacks on time series classification aim to misclassify specific instances, achieving similar precision in time series forecasting is more complex due to the sequential nature of the data. Perturbations must be designed to influence specific aspects of the forecast (e.g., directional shifts or amplitude changes) without disrupting the overall temporal dependencies, making precise control more challenging~\cite{govindarajulu2023targeted}. Another challenge is attacks on multivariate forecasting. Adversarial attacks could exploit the inter-dependences between variables. ~\cite{liu2022robust} introduced sparse and indirect cross-time-series attacks in multivariate settings, which are more effective and realistic than direct attacks in univariate cases.
% \zhen{Biplav, could we make a quick comment here as well that we focus more on data error side in practice, other than attacks? and cite the paper that you mentioned on data errors? Otherwise this section of adversarial attacks feel a bit standalone to other sections}
%These challenges underscore the need for ongoing research to develop effective adversarial attack strategies and robust defense mechanisms tailored to the unique characteristics of time series forecasting models.
% -----


\noindent \textbf{Rating AI Systems} Several works have assessed and rated AI systems for trustworthiness from a third-party perspective without access to training data. \cite{srivastava2020rating} proposed a method to rate AI systems for bias, specifically targeting gender bias in machine translators \cite{srivastava2018towards}, and used visualizations to communicate these ratings \cite{bernagozzi2021vega}. They conducted user studies on trust perception through visualizations \cite{vega-userstudy-translatorbias}, but these lacked causal interpretation. \cite{kausik2024rating} introduced a causal analysis approach to rate bias in sentiment analysis systems, extending it to assess their impact when used with translators \cite{kausik2023the}. We extend their method to rate MM-TSFM for robustness against perturbations. Causal analysis offers advantages over statistical analysis by determining accountability, aligning with humanistic values, and quantifying the direct influence of various attributes on forecasting accuracy.


\section{Formative Study}
\label{formative-study}
% To understand viewer behavior and challenges in assimilating collective knowledge in science videos, we conducted semi-structured interviews with 10 frequent consumers of these videos with danmaku feature. Our study validates the benefits of danmaku in science videos, categorizing three user behavior modes and identifying the challenges encountered. 
% % These insights informed the design of \textit{SystemName}, an interactive system to aid viewer comprehension of collective knowledge.
To further investigate the needs and challenges specific to danmaku, we conducted a formative study with consumers of science videos featuring danmaku. Based on our findings and insights from previous works, we derived the design pipeline for CoKnowledge.

\subsection{Participants and Procedure}
With IRB approval, we recruited 10 participants (five female, five male; labeled PF1 to PF10) for the formative study through online ads, social media, and word-of-mouth. All participants were self-identified frequent viewers of online science videos with the danmaku feature (six watched daily, three watched 4 to 6 days a week, and one watched at least once a week).  In semi-structured interviews, participants discussed their viewing strategies, interactions with danmaku, and perceptions of its role in collective knowledge. We also explored challenges in assimilating collective knowledge and solicited suggestions for technical support. Participants were encouraged to provide examples to contextualize their responses. 
% Interviews lasted 30-45 minutes, and participants received \$6 for their time.
Following established protocols \cite{klein2017quality, baker2021avatar, liu2021makes}, interviews were audio-recorded, transcribed, and thematically analyzed by two researchers.
% Two researchers conducted a thematic analysis to inductively code and identify themes. 
We monitored the codebook for saturation after eight interviews \cite{saunders2018saturation}, completing the last two to confirm that no new themes emerged.

\subsection{Summary of Findings}
Our findings validate danmaku's benefits in science videos, identifying three user behavior modes and challenges in consuming collective knowledge.

\begin{table*}[h!]
\centering
\scalebox{0.75}{ 
\begin{tabular}{ >{\centering\arraybackslash}m{1.2cm} >{\centering\arraybackslash}m{5cm} >{\centering\arraybackslash}m{4.8cm} >{\centering\arraybackslash}m{4cm} >{\centering\arraybackslash}m{3.5cm} }
  \hline
   &  & \textit{Overview} & \textit{Focused} & \textit{Exploration} \\
  \hline
  \multirow{5}{*}{Dynamics} & Knowledge need & High-level knowledge abstraction \circled{1}& Concise knowledge \circled{2}& Structured knowledge \circled{3}\\
  & Knowledge scope & Entire video \circled{4}& Current timestamp \circled{5}& Short segments \circled{6}\\
  & Engagement level & Active \circled{7}& Mostly passive \circled{8}& Active \circled{9}\\
  & Interaction behavior & Skimming, locating \circled{10}& Typical viewing behavior \circled{11}& Thorough analysis \circled{12}\\
  & Focal point & Entire interface \circled{13}& Video segment view \circled{14}& Entire interface \circled{15}\\
  \hline
  \multirow{5}{*}{Challenges} & Mutual distractions \circled{16}& \ding{55} & \ding{51} & \ding{55} \\
  & Short display duration \circled{17}& \ding{55} & \ding{51} & \ding{55} \\
  & Danmaku locating difficulties \circled{18}& \ding{51} & \ding{55} & \ding{55} \\
  & \textbf{Low information density} & \ding{51} & \ding{51} & \ding{51} \\
  & \textbf{Obscure information structure} & \ding{51} & \ding{51} & \ding{51} \\
  \hline
\end{tabular}
}
\caption{Behavior patterns and challenges for assimilating collective knowledge in online science videos. The labeled items are addressed in section \ref{ui}.}
\label{formative-study-findings}
\end{table*}

\subsubsection{Advantages of Danmaku}

We inquired about participants' perceptions of the support provided by danmaku in knowledge acquisition while watching science videos. 
% They highlighted numerous advantages of danmaku, including those mentioned in existing work. 
% Most participants (9 out of 10) indicated that danmaku offered a \textbf{sense of co-presence}. PF9 remarked, \textit{``When danmaku is turned off, the interaction feels like a conversation between just me and the uploader. When it’s on, everyone is involved in discussing the topic.''} 
The majority (9 out of 10) highlighted that danmaku fostered a \textbf{sense of co-presence}, with PF9 noting it shifted the experience from an ``isolated interaction with the uploader'' to a more collective discussion.
This co-presence enhanced engagement with the content, thereby boosting motivation and efficiency in knowledge acquisition.
% Furthermore, 8 participants emphasized that danmaku helped \textbf{avoid knowledge bias} by presenting diverse viewpoints. Specifically, PF8 mentioned, \textit{``Sometimes I worry that the uploader's perspective might be biased, but danmaku provides a variety of viewpoints.''} These diverse perspectives offered users additional references, facilitating a comprehensive understanding of the content.
Furthermore, 8 participants emphasized that danmaku helped \textbf{avoid knowledge bias} by presenting diverse viewpoints, as PF8 indicated, offering ``alternative perspectives'' to those of the uploader.
% All participants agreed that danmaku synchronously supplemented knowledge. PF2 stated, \textit{``[Compared to comments,] I can see real-time, immediate supplements or extensions of the video content through danmaku.''} 
All participants agreed that danmaku synchronously supplemented knowledge, with PF2 highlighting its ability to offer ``real-time'' and ``immediate'' extensions of video content.
\textbf{The closer integration} of danmaku with the video enhanced the immediacy and contextual relevance of the knowledge.
These advantages aligned with those discussed in existing literature \cite{he2021beyond, yao2017understanding, chen2015understanding, ma2014analysis, huang2024sharing}, further affirming danmaku as a valuable source for constructing collective knowledge.
% In addition to the advantages mentioned in previous work, participants also identified other benefits of danmaku. For instance, PF5 noted, \textit{``I can find a sense of agreement in the viewpoints expressed in danmaku,''} while PF7 mentioned that the flow of danmaku helps them stay focused and reduces distractions. These advantages confirm danmaku as a valuable source for constructing collective knowledge.



\subsubsection{Viewing Practices and Challenges}

Participants reported various practices when assimilating collective knowledge. We identified three viewing modes—\textbf{Overview}, \textbf{Focused}, and \textbf{Exploration}—and examined their corresponding features (Table \ref{formative-study-findings}).



% \begin{table}[h!]
% \centering
% \begin{tabular}{ >{\centering\arraybackslash}m{3.2cm} >{\centering\arraybackslash}m{4cm} >{\centering\arraybackslash}m{3.2cm} >{\centering\arraybackslash}m{3cm} }
%   \hline
%    & \textit{Overview} & \textit{Focused} & \textit{Exploration} \\
%   \hline
%   Knowledge need & High-level knowledge abstract & Concise knowledge & Structured knowledge \\
%   \hline
%   Knowledge scope & Entire video & Current timestamp & Short segments \\
%   \hline
%   Engagement level & Active & Mostly passive & Active \\
%   \hline
%   Interaction behavior & Skimming, navigating & Typical viewing behavior & Thorough analysis \\
%   \hline
%   Focal point & Entire interface & Video segment view & Entire interface \\
%   \hline 
%   Mutual distractions  & \ding{55} & \ding{51} & \ding{55} \\
%   \hline
%   Short duration & \ding{55} & \ding{51} & \ding{55} \\
%   \hline
%   Challenges in navigating desired danmaku & \ding{51} & \ding{55} & \ding{55} \\
%   \hline
%   \textbf{Low Information Density} & \ding{51} & \ding{51} & \ding{51} \\
%   \hline
%   \textbf{Obscure Information Structure} & \ding{51} & \ding{51} & \ding{51} \\
%   \hline
% \end{tabular}
% \caption{Behavior patterns and challenges for assimilating collective knowledge in online science videos}
% \label{formative-study-findings}
% \end{table}

% \begin{table}[h!]
% \centering
% \scalebox{0.85}{ 
% \begin{tabular}{ >{\centering\arraybackslash}m{1.2cm} >{\centering\arraybackslash}m{4.3cm} >{\centering\arraybackslash}m{4cm} >{\centering\arraybackslash}m{3.3cm} >{\centering\arraybackslash}m{2.8cm} }
%   \hline
%    &  & \textit{Overview} & \textit{Focused} & \textit{Exploration} \\
%   \hline
%   \multirow{5}{*}{Dynamics} & Knowledge need & High-level knowledge abstract \xm{abstraction?} & Concise knowledge & Structured knowledge \\
%   & Knowledge scope & Entire video & Current timestamp & Short segments \\
%   & Engagement level & Active & Mostly passive & Active \\
%   & Interaction behavior & Skimming, lcoating & Typical viewing behavior & Thorough analysis \\
%   & Focal point & Entire interface & Video segment view & Entire interface \\
%   \hline
%   \multirow{5}{*}{Challenges} & Mutual distractions & \ding{55} & \ding{51} & \ding{55} \\
%   & Short display duration & \ding{55} & \ding{51} & \ding{55} \\
%   & Danmaku locating difficulties & \ding{51} & \ding{55} & \ding{55} \\
%   & \textbf{Low information density} & \ding{51} & \ding{51} & \ding{51} \\
%   & \textbf{Obscure information structure} & \ding{51} & \ding{51} & \ding{51} \\
%   \hline
% \end{tabular}
% }
% \caption{Behavior patterns and challenges for assimilating collective knowledge in online science videos}
% \label{formative-study-findings}
% \end{table}




In the \textbf{Overview} mode, many participants (PF1, PF4-7, PF10) actively skimmed through the video to quickly understand or review its main content and locate segments of interest. Such behaviors were often observed before watching the video, but also during (PF6) and after viewing (PF6, PF7, PF10). However, users struggled with effectively \textbf{locating desired danmaku content}. PF1 remarked, \textit{``There is no way to preview danmaku to locate the good ones,''} PF6 added, \textit{``I saw interesting danmaku, but when I wanted to revisit them later, I could not find them.''}

The \textbf{Focused} mode described the situation when users concentrated solely on watching the video. In this mode, users were mostly passive, occasionally engaging in typical viewing behaviors like pausing, replaying, skipping, or adjusting playback speed to shift into other modes. All participants reported being overwhelmed by the volume of danmaku and finding the danmaku and video to be \textbf{mutually distracting}. Danmaku often blocked the video content (PF1, PF2-5, PF8), causing users’ attention to oscillate between the two, significantly hindering the assimilation of collective knowledge. Many participants (7 out of 10) also complained about the \textbf{short display duration} of danmaku, with PF10 noting that comments \textit{``fly by too quickly''} to be fully read.


When participants encountered interesting or complex content in the video or danmaku, they entered the \textbf{Exploration} mode. In this mode, users fixated on a short segment of the video, examining the meaning of danmaku and its relationship with the video (PF6, PF10), and may also seek additional information from external sources to aid their understanding (PF1, PF2, PF10).


We identified two key challenges that fundamentally hinder the assimilation of collective knowledge across all viewing modes: \textbf{low information density} and \textbf{obscure information structure}, which were frequently criticized by participants. All participants agreed that most danmaku is \textit{``meaningless for knowledge acquisition''} and characterized by excessive repetition, often burying valuable comments (PF1-6, PF7, PF9) and causing frustration (PF3). Furthermore, the majority (8 out of 10) mentioned that even informative danmaku was chaotic because the \textit{``information pattern was unorganized''} (PF7), and the relationship between danmaku and video knowledge lacked clarity (PF10), making it challenging to synthesize danmaku into the framework of collective knowledge. 



\subsection{Design Requirements}
\label{drs}

Based on the findings, we formulated the following design requirements to guide the subsequent design process (as illustrated in Fig \ref{fig:design-pipeline}): 

\begin{enumerate}[label=\textbf{DR\arabic*}, leftmargin=3em]
    \item \textbf{Accommodate diverse viewing modes and their corresponding dynamics while mitigating mode-specific challenges.} We have labeled the items in Table \ref{formative-study-findings} and addressed them accordingly in section \ref{ui}.
    
    \item \textbf{Reduce the volume of danmaku by excluding low-quality, knowledge-irrelevant comments and consolidating duplicate ones.} 
    In addition to minimizing the distraction of video-viewing, reducing the number of danmaku can increase information density. This allows users to concentrate more on individual comments, lowering the chances of comments scrolling away before being fully read. Furthermore, it is easier to locate specific danmaku, as they are less likely to be overshadowed by others.
    % In addition to increasing information density, reducing the number of danmaku can minimize the distraction to video-viewing. This allowed users to concentrate more on individual comments, reducing the frequency of instances where danmaku would scroll away before being fully read. Furthermore, it made it easier to locate specific danmaku, as they were less likely to be overshadowed by others.
    
    \item \textbf{Clearly present the diverse themes of danmaku and the structures of collective knowledge.} The various perspectives danmaku offers can enhance the sense of discussion and provide viewers with more sources of information. Further dissecting video knowledge and arranging danmaku around the corresponding knowledge points can highlight their inherent connections and the overall information structure.
    % Further organizing the patterns of danmaku with the video content would clarify the information structure and strengthen the correspondence between them.
\end{enumerate}

% \yh{While the correspondence of DRs and Findings is obvious to me, but I am not sure if it is the case for readers. Can I just simply list the DRs here?}

% \begin{table}[h!]
% \centering
% \begin{tabular}{| m{6cm} | c | c | c |}
%   \hline
%   \rowcolor{lightgray}
%   \textbf{Challenge} & \textbf{Overview} & \textbf{Focused} & \textbf{Exploration} \\
%   \hline
%   Mutual distractions  & \ding{55} & \ding{51} & \ding{55} \\
%   \hline
%   Short duration & \ding{55} & \ding{51} & \ding{55} \\
%   \hline
%   Challenges in navigating desired danmaku & \ding{51} & \ding{55} & \ding{55} \\
%   \hline
%   \textbf{Low Information Density} & \ding{51} & \ding{51} & \ding{55} \\
%   \hline
%   \textbf{Obscure Information Structure} & \ding{51} & \ding{51} & \ding{55} \\
%   \hline
% \end{tabular}
% \caption{Challenges in different modes}
% \end{table}


\begin{figure*}[h]
  \centering
  \includegraphics[width=\linewidth]{images/design-pipeline.jpg}
  \caption{This figure illustrates an overview of our design pipeline: based on the findings derived from the formative study, we established design requirements to address each finding. Subsequently, we developed corresponding features that constituted the components of our design to meet these requirements.}\label{fig:design-pipeline}
  \Description{This figure illustrates an overview of our design pipeline: based on the findings derived from the formative study, we established design requirements to address each finding. Subsequently, we developed corresponding features that constituted the components of our design to meet these requirements. Specifically, let's start with the correspondence of findings to design requirements. Various behavior patterns and mode-specific challenges (Mutual distractions, short display duration, and danmaku locating difficulties) correspond to DR1. Mode-specific challenges and low information density correspond to DR2. Obscure information structure and danmaku advantages correspond to DR3. Then let's go to the correspondence of design requirements to design features. DR1 corresponds to Overview, Focused, and Exploration modes. DR2 corresponds to danmaku filtering and clustering. DR3 corresponds to danmaku classification. And we also conducted a content analysis to assist automatic danmaku filtering and classification. The Overview, Focused, and Exploration modes constitue the user interface of CoKnowledge. The danmaku filtering, classification and clustering constitue the NLP pipeline of CoKnowledge.}
\end{figure*}
\section{Design}\label{sec:design}

%%%%%%%%%%%%%%%%%%%%%%%%%%%%%%


\begin{figure*}[t]
    \centering
    \includegraphics[trim = 15 530 15 15, width=1\textwidth]{Algorithm_drawio.pdf}
    \caption{Overview of KiSS}
    \label{fig:overview}
\end{figure*}


The results we gleaned from the previous section (see Section~\ref{sec:work_anly}) helped in developing our policy: KiSS. The KiSS or \textbf{Keep it Separated Serverless} policy aims to address critical challenges in Function-as-a-Service (FaaS) platforms, particularly in edge computing environments, by achieving the following objectives:

\begin{itemize}
    \item \textbf{Reduced Cold Start Latency:} Prioritizes high-frequency functions to minimize delays in real-time applications.
    \item \textbf{Improved Resource Efficiency:} Optimizes memory and compute usage while avoiding unnecessary overhead from static warm states.
    \item \textbf{Minimized Inter-Function Interference:} Enhances throughput and scalability through modular resource partitioning.
    \item \textbf{Improved Function Service Rate:} Adopts resource-aware policies to reduce dropped requests and maximize system reliability.
\end{itemize}


\subsection{KiSS Policy Overview}

KiSS introduces a modular, data-driven orchestration strategy designed to optimize serverless execution in resource-constrained environments, particularly at the edge. By leveraging our workload analysis (refer Section 2.5), our policy segments functions based on key metrics—memory footprint, invocation frequency, and execution time—to optimize performance across diverse workloads.

The edge computing context introduces unique challenges like limited memory, heterogeneous resources, and dynamic workloads. Generalized cloud strategies often fail to adapt to such constraints. KiSS addresses this gap by analyzing workload characteristics and implementing a resource-efficient, modular strategy that aligns with edge-specific demands.

\subsection{Components of KiSS Policy Design}
Figure~\ref{fig:overview} shows the overall architecture of KiSS. 
The incoming \textit{FaaS traffic} will include both small and large functions. 
The \textit{request handler} accepts the incoming functions and shares the function information to the workload analyzer. 
The \textit{workload analyser} processes the function information to profile the incoming function traffic information and generate data such as invocation frequency, memory footprint etc.
The \textit{KiSS policy} uses this data to estimate where this function will be placed between the two different warm pool partitions.

The \textit{load balancer} implements a partitioning logic where functions are allocated to distinct warm pools using (\textit{invoker 1} and \textit{invoker 2}) based on profiling thresholds:

(i)~Small Functions Pool: Dedicated to high-frequency, low-memory functions to ensure low latency, and (ii)~Large Functions Pool: Allocated for low-frequency, memory-intensive functions, minimizing contention with smaller containers.
Each warm pool operates autonomously achieving Policy Independence.
The \textit{Warm Pool Replacement Policy} for each warm container pool can independently implement different workload-specific strategies to reduce contention and enhance temporal locality.


These factors form the foundation of KiSS’s multi-tiered warm pool framework, allowing it to effectively manage serverless resources and enhance performance in edge computing. By addressing these challenges, KiSS positions itself as a practical and scalable solution for FaaS platforms in environments with diverse and demanding resource constraints.


\subsection{Innovations of KiSS Policy}

One of the most innovative features of KiSS is its multi-level warm pool partitioning, which isolates high- and low-frequency functions into separate pools. This design eliminates inefficiencies inherent in monolithic resource strategies by ensuring that small, frequently invoked functions are always ready to execute, while larger, less frequent functions remain accessible without competing for resources. This adaptability extends to the ability to add more pools as workload patterns evolve, making KiSS a flexible and future-proof solution. Moreover, its modular architecture supports diverse deployment scenarios, from centralized clouds to resource-constrained edge environments. Integration with traffic-aware schedulers ensures that KiSS maintains scalability and responsiveness even under fluctuating workloads.


\subsubsection{Advantages of KiSS}

The advantages of KiSS are particularly pronounced in edge environments. By keeping frequently accessed containers in warm states, it drastically reduces cold start latency, which is critical for real-time applications such as IoT and AI analytics. Static warm pool partitioning, based on workload analysis, optimizes memory usage by eliminating unnecessary overhead, ensuring that resources are used efficiently even in environments with stringent memory constraints. This strategy not only enhances performance but also reduces operational costs by consolidating memory usage and minimizing cold starts. KiSS’s platform-agnostic design further enhances its versatility, enabling seamless deployment across various serverless frameworks.


\section{Evaluation}
We provide three sets of insights into this section, organised as \textit{findings (F*)}. We quantitatively study the effect of the adversarial and counterfactual perturbations on the performance of informal reasoners and autoformalisation methods. Then, we dive deeper into method variants. Finally, 
we analyse the nature of formalisation errors made by the models.

\subsection{Robustness Analysis}
\paragraph{\textbf{\emph{F1: Noise perturbations have a stronger effect on formalisation methods than informal \ac{LLM} reasoners.}}}
Table~\ref{tab:distraction_k4_formalisation} shows that, on average, the accuracy of both direct and \ac{CoT} informal reasoning remains between $73\%$ and $74\%$ in the face of added noise. While the autoformalisation method performs similarly to informal reasoners on the original dataset, its performance decreases between $4\%$ and $11\%$. The accuracy drops especially with logical (L) and tautological (T) distractions, whose logical language formats trick the \ac{LLM} into formalizing the noisy clauses. On the other hand, the linguistically complex and more natural sentences of encyclopedic distractions show a minor effect, suggesting that \acp{LLM} successfully avoids formalizing the more complicated sentences.

\paragraph{\textbf{\emph{F2: All \ac{LLM}-based reasoning methods suffer a drop for counterfactual perturbations.}}} % influence .}}}
Table~\ref{tab:distraction_k4_formalisation} shows that counterfactual statements cause a significant decrease in performance for both the informal reasoners and autoformalisation methods of between $12\%$ and $13\%$ on average. 
Moreover, this observation also holds for all tested models, i.e., none are robust towards counterfactual perturbations across every evaluated dimension. Even the strongest model, GPT 4o-mini, yields a performance of 63-68\%, which is relatively close to the random performance of 50\%. The high impact of counterfactual statements (the single ``not'' inserted) could be due to the inability of \acp{LLM} to overwrite prior knowledge with explicitly stated information or memorization of the answers. We study the error sources further in §\ref{subsec:errors}.  

\noindent \paragraph{\textbf{\emph{F3: Introducing multiple noise sentences has an effect only for logical distractions.}}}
We show the impact of introducing between one and four sentences for the two top-performing autoformalisation models in Figure~\ref{fig:length_distraction}. The figure shows similar trends with and without counterfactual perturbations.
As additional logical distractions are introduced, the model performance consistently decreases. Tautological (T) distractions lead to a decline in accuracy with a single disruptive sentence, yet adding more noise does not worsen the outcome. 
The tautological corpus introduces truth constants for all sentences as a persistent unseen logical construct. Given that this leads only to a decrease for a single occurrence, we can assume that a model can consistently handle the same unseen logical construct. In contrast, the logical corpus increases the chance of adding text, requiring new, previously unseen reasoning constructs for each added sentence. The impact of encyclopedic noise remains negligible, generalising F1 to $k$ sentences. Similarly, counterfactual perturbations remain much more effective for all settings, generalising F2.

\begin{table}[!t]
\small
\setlength{\modelspacing}{2pt}
\setlength{\tabcolsep}{1.7pt} % Default value: 6pt
\setlength{\belowrulesep}{4pt}
\begin{threeparttable}
    \centering
    \begin{tabular}{cc l r rrr @{\quad} rrrr}
\toprule
\multirow{2}{*}{} & \multirow{2}{*}{} & Reasoning & \multirow{2}{*}{O} & \multicolumn{3}{c}{Distraction} & \multicolumn{4}{c}{Counterfactual} \\
 & & Format & & E& L & T & $\text{O}_C$ & $\text{E}_C$& $\text{L}_C$ & $\text{T}_C$\\
\midrule
\multirow{6}{*}{\rotatebox{90}{Gemma-2}} & \multirow{3}{*}{\rotatebox{90}{9b}}
   & Informal (direct) & \textbf{0.78} & \textbf{0.80} & \textbf{0.79} & \textbf{0.77} & 0.58 & 0.52 & 0.50 & 0.59 \\
 & & Informal (CoT) & 0.72 & 0.78 & 0.73 & 0.76 & 0.61 & \textbf{0.57} & \textbf{0.60} & \textbf{0.66} \\
 & & Formal (FOL) & 0.62 & 0.58 & 0.52 & 0.53 & \textbf{0.63} & 0.52 & 0.46 & 0.46 \\[\modelspacing]
\cmidrule{2-11}
 & \multirow{3}{*}{\rotatebox{90}{27b}} 
   & Informal (direct) & 0.71 & 0.69 & \textbf{0.66} & \textbf{0.68} & 0.59 & 0.51 & 0.54 & 0.59 \\
 & & Informal (CoT) & 0.66 & 0.65 & 0.64 & 0.63 & 0.62 & 0.58 & \textbf{0.62} & \textbf{0.64} \\
 & & Formal (FOL) & \textbf{0.74} & \textbf{0.74} & 0.61 & 0.61 & \underline{\textbf{0.72}} & \underline{\textbf{0.67}} & 0.58 & 0.51 \\[\modelspacing]
\midrule
\multirow{6}{*}{\rotatebox{90}{Mistral}} & \multirow{3}{*}{\rotatebox{90}{7B}} 
   & Informal (direct) & 0.77 & \textbf{0.77} & 0.75 & \textbf{0.79} & \textbf{0.63} & \textbf{0.54} & \textbf{0.54} & \textbf{0.66} \\
 & & Informal (CoT) & \textbf{0.79} & 0.75 & \textbf{0.77} & 0.78 & 0.55 & 0.52 & \textbf{0.54} & 0.58 \\
 & & Formal (FOL) & 0.62 & 0.58 & 0.54 & 0.57 & 0.50 & \textbf{0.54} & 0.51 & 0.52 \\[\modelspacing]
\cmidrule{2-11}
 & \multirow{3}{*}{\rotatebox{90}{Small}} 
   & Informal (direct) & \textbf{0.77} & \textbf{0.76} & \textbf{0.76} & \textbf{0.75} & 0.61 & 0.51 & 0.56 & 0.59 \\
 & & Informal (CoT) & 0.72 & 0.72 & 0.72 & 0.71 & \textbf{0.62} & \textbf{0.59} & \textbf{0.62} & \textbf{0.68} \\
 & & Formal (FOL) & 0.68 & 0.59 & 0.53 & 0.64 & 0.54 & 0.55 & 0.49 & 0.51 \\[\modelspacing]
\midrule
\multirow{6}{*}{\rotatebox{90}{Llama-3.1}} & \multirow{3}{*}{\rotatebox{90}{8B}} 
   & Informal (direct) & 0.63 & 0.61 & 0.64 & 0.66 & 0.61 & \textbf{0.62} & 0.59 & 0.61 \\
 & & Informal (CoT) & 0.73 & \textbf{0.73} & \textbf{0.71} & \textbf{0.72} & \textbf{0.62} & 0.59 & \textbf{0.61} & \textbf{0.65} \\
 & & Formal (FOL) & \textbf{0.77} & 0.71 & 0.63 & 0.52 & 0.60 & 0.58 & 0.55 & 0.52 \\[\modelspacing]
\cmidrule{2-11}
 & \multirow{3}{*}{\rotatebox{90}{70B}} 
   & Informal (direct) & 0.77 & 0.74 & 0.74 & 0.73 & 0.62 & 0.53 & 0.56 & 0.64 \\
 & & Informal (CoT) & \textbf{0.78} & \textbf{0.75} & \textbf{0.76} & \textbf{0.76} & 0.64 & 0.61 & \textbf{0.66} & \underline{\textbf{0.73}} \\
 & & Formal (FOL) & 0.74 & 0.73 & 0.71 & 0.71 & \textbf{0.66} & \textbf{0.62} & 0.59 & 0.57 \\[\modelspacing]
 \midrule
\multirow{3}{*}{\rotatebox{90}{GPT}} & \multirow{3}{*}{\rotatebox{90}{4o-mini}} 
   & Informal (direct) & 0.78 & 0.77 & 0.79 & 0.79 & 0.64 & 0.61 & 0.61 & 0.63 \\
 & & Informal (CoT) & 0.80 & 0.80 & \underline{\textbf{0.81}} & \underline{\textbf{0.82}} & \textbf{0.68} & \textbf{0.63} & \underline{\textbf{0.68}} & \textbf{0.64} \\
 & & Formal (FOL) & \underline{\textbf{0.84}} & \underline{\textbf{0.82}} & 0.73 & 0.79 & 0.63 & 0.62 & 0.57 & 0.54 \\[\modelspacing]
 \midrule
\multicolumn{2}{c}{\multirow{3}{*}{\textbf{Avg}}} 
 & Informal (direct) & 0.74 & 0.73 & 0.73 & 0.73 & 0.61 & 0.55 & 0.56 & 0.62 \\
 & & Informal (CoT) & 0.74 & 0.74 & 0.73 & 0.74 & 0.62 & 0.58 & 0.62 & 0.65 \\
  & & Formal (FOL) & 0.72 & 0.68 &	0.61 & 0.62 & 0.61 & 0.59 & 0.54 & 0.52 \\
\bottomrule
\end{tabular}
\caption{Accuracies of informal and autoformalisation-based deductive reasoners. The best overall model per dataset is underlined; the best model version is marked in bold.}
\label{tab:distraction_k4_formalisation}
\end{threeparttable}
\end{table} 

\begin{figure}[!t]
    \centering
    \scriptsize
    \begin{tikzpicture}
        \begin{axis}[name=gpt,
            title={GPT-4o-mini},
            width=0.6\linewidth,
            height=0.6\linewidth,
            xlabel={\# Noise sentences},
            ylabel={Accuracy},
            xmin=-0.1, xmax=4.1,
            ymin=0.5, ymax=0.9,
            xtick={1,2,4},
            ytick={0.55, 0.6, 0.65, 0.75, 0.8, 0.85},
            title style={yshift=-0.6em},
            legend style={at={(1,-0.15)},
	           anchor=north,legend columns=-1},
            x label style={at={(axis description cs:1,-0.05)},anchor=north},
            y label style={at={(axis description cs:-0.15,0.5)},anchor=south},
            ymajorgrids=true,
            grid style=dashed,
        ]
            \addplot[color=blue, mark=square,]
                coordinates {
                (0,0.848076939582825)(1,0.823076903820038)(2,0.826923072338104)(4,0.821153819561005)
                };
            \addplot[color=red, mark=triangle,]
                coordinates {
                (0,0.848076939582825)(1,0.817307710647583)(2,0.801923096179962)(4,0.759615361690521)
                };
            \addplot[color=green, mark=diamond,] 
                coordinates {
                (0,0.848076939582825)(1,0.767307698726654)(2,0.769230782985687)(4,0.803846180438995)
                };
            \addplot[color=blue, mark=square*] 
                coordinates {
                (0,0.627777755260468)(1,0.622222244739533)(2,0.600000023841858)(4,0.633333325386047)
                };
            \addplot[color=red, mark=triangle*,] 
                coordinates {
                (0,0.627777755260468)(1,0.611111104488373)(2,0.611111104488373)(4,0.594444453716278)
                };
            \addplot[color=green, mark=diamond*,] 
                coordinates {
                (0,0.627777755260468)(1,0.572222232818604)(2,0.538888871669769)(4,0.555555582046509)
                };
                \legend{E,L,T,$\text{E}_C$, $\text{L}_C$ , $\text{T}_C$}
        \end{axis}

        \begin{axis}[name=llama, at={($(gpt.east)+(0.1cm,0)$)},anchor=west,
            title={Llama 3.1 70b},
            width=0.6\linewidth,
            height=0.6\linewidth,
            xmin=-0.1,, xmax=4.1,
            ymin=0.5, ymax=0.9,
            xtick={1,2,4},
            ytick={0.55, 0.6, 0.65, 0.75, 0.8, 0.85},
            title style={yshift=-0.6em},
            yticklabel=\empty,
            ymajorgrids=true,
            grid style=dashed,
        ]
            \addplot[color=blue, mark=square,]
                coordinates {
                (0,0.838461518287659)(1,0.817307710647583)(2,0.805769205093384)(4,0.817307710647583)
                };
            \addplot[color=red, mark=triangle,]
                coordinates {
                (0,0.838461518287659)(1,0.819230794906616)(2,0.803846180438995)(4,0.771153867244721)
                };
            \addplot[color=green, mark=diamond,]
                coordinates {
                (0,0.838461518287659)(1,0.803846180438995)(2,0.807692289352417)(4,0.805769205093384)
                };
            \addplot[color=blue, mark=square*]
                coordinates {
                (0,0.627777755260468)(1,0.622222244739533)(2,0.577777802944183)(4,0.594444453716278)
                };
            \addplot[color=red, mark=triangle*,]
                coordinates {
                (0,0.627777755260468)(1,0.583333313465118)(2,0.561111092567444)(4,0.577777802944183)
                };
            \addplot[color=green, mark=diamond*,]
                coordinates {
                (0,0.627777755260468)(1,0.627777755260468)(2,0.566666662693024)(4,0.577777802944183)
                };
        \end{axis}
    \end{tikzpicture}
    \caption{Influence of the number of noisy sentences for FOL.}
    \label{fig:length_distraction}
\end{figure}



\subsection{Impact of Method Design}
\paragraph{\textbf{\emph{F4: \ac{CoT} prompting is most impactful when both noise and counterfactual perturbations are applied.}}}
The accuracies for the individual \acp{LLM} in Table~\ref{tab:distraction_k4_formalisation} show that the impact of \ac{CoT} is negligible for noise-only datasets (first four columns). Meanwhile, the benefit from \ac{CoT} is most pronounced in the datasets that combine noise and counterfactual perturbations.
The better-performing informal prompting strategy for a model remains stable for all types of distractions. Still, the decline in performance due to counterfactuals leads to a less consistent preference for a specific prompting style.

\paragraph{\textbf{\emph{F5: The best-performing grammar differs per model and is unstable across data versions.}}}

The evaluation of different logical forms for formal \ac{LLM}-based reasoning in Table~\ref{tab:distraction_k4_logical_form} shows the preference of some models for specific syntactic formats.
Llama 3.1 70B has a considerable improvement of $12\%$ with TPTP syntax on the original set, while Llama 3.1 8B benefits from the R-FOL syntax. However, all grammars show a declining accuracy trend and increased syntax errors for noise perturbations, where the best grammar loses its advantage over the rest. 
When comparing the grammars on the counterfactual partitions, we observe that TPTP is consistently more robust than the standard first-order logic grammar. Here, GPT 4o-mini shows a reduction from $O$ to $O_C$ of $20\%$ for FOL and only $12\%$ for the TPTP grammar. Since this does not correlate with fewer syntax errors, the formalisation in TPTP prevents semantical errors for counterfactual premises. 
A positive reading of these results, especially the minor differences between FOL and R-FOL, is that autoformalisation \acp{LLM} can adapt to the grammar syntax prescribed in the prompt without further loss in performance.

\begin{table}[!t]
\small
\setlength{\modelspacing}{2pt}
\setlength{\tabcolsep}{1.7pt} % Default value: 6pt
\setlength{\belowrulesep}{4pt}
\begin{threeparttable}
    \centering
    \begin{tabular}{cc l r rrr @{\quad} rrrr}
\toprule
\multirow{2}{*}{} & \multirow{2}{*}{} & Grammar & \multirow{2}{*}{O} & \multicolumn{3}{c}{Distraction} & \multicolumn{4}{c}{Counterfactual} \\
 & & Syntax & & E& L & T & $\text{O}_C$ & $\text{E}_C$& $\text{L}_C$ & $\text{T}_C$\\
\midrule
\multirow{6}{*}{\rotatebox{90}{Llama-3.1}} & \multirow{3}{*}{\rotatebox{90}{8B}} 
   & FOL & 0.77 & \textbf{0.71} & 0.61 & \textbf{0.53} & 0.58 & \textbf{0.55} & 0.52 & \textbf{0.56} \\
 & & R-FOL & \textbf{0.78} & 0.69 & \textbf{0.62} & \textbf{0.53} & 0.58 & \textbf{0.55} & \textbf{0.54} & 0.52 \\
 & & TPTP & 0.73 & 0.67 & 0.55 & 0.51 & \textbf{0.68} & 0.54 & 0.46 & 0.51 \\[\modelspacing]
\cmidrule{2-11}
 & \multirow{3}{*}{\rotatebox{90}{70B}} 
   & FOL & 0.76 & 0.73 & 0.71 & \textbf{0.72} & 0.67 & 0.57 & 0.63 & 0.56 \\
 & & R-FOL & 0.76 & 0.73 & 0.67 & 0.71 & 0.64 & 0.57 & 0.53 & 0.64 \\
 & & TPTP & \underline{\textbf{0.88}} & \underline{\textbf{0.84}} & \underline{\textbf{0.81}} & \textbf{0.72} & \underline{\textbf{0.81}} & \underline{\textbf{0.68}} & \underline{\textbf{0.67}} & \underline{\textbf{0.68}} \\[\modelspacing]
\midrule
\multirow{3}{*}{\rotatebox{90}{GPT}} & \multirow{3}{*}{\rotatebox{90}{4o-mini}} 
   & FOL & \textbf{0.84} & \textbf{0.82} & \textbf{0.72} & \underline{\textbf{0.78}} & 0.64 & \textbf{0.63} & \textbf{0.61} & 0.51 \\
 & & R-FOL & \textbf{0.84} & 0.77 & 0.70 & \underline{\textbf{0.78}} & \textbf{0.72} & 0.56 & 0.54 & \textbf{0.63} \\
 & & TPTP & 0.83 & \textbf{0.82} & 0.71 & 0.71 & 0.69 & \textbf{0.63} & 0.57 & 0.57 \\
\bottomrule
\end{tabular}
\caption{Accuracies of different formalisation grammars for autoformalisation.}
\label{tab:distraction_k4_logical_form}
\end{threeparttable}
\end{table} 

\paragraph{\textbf{\emph{F6: Feedback does not help \acp{LLM} self-correct to mitigate robustness issues.}}}
\autoref{tab:distraction_k4_feedback} shows the results with different error recovery mechanisms. The results indicate that no feedback strategy emerges as a winner in the different datasets. 
All feedback variants reduce syntax errors for noise perturbations, but given the lack of a consistent increase in accuracy, the corrected formalisations are most likely to contain semantic errors still. 
The type of feedback message only has a minor influence on correcting syntax errors, whereas Llama 3.1 70b and GPT 4o-mini correct slightly more syntax errors with specific error messages. This finding aligns with \cite{huang2023large}, who also found that \acp{LLM} cannot consistently self-correct their reasoning after receiving relevant feedback.

\begin{table}[!ht]
\small
\setlength{\modelspacing}{2pt}
\setlength{\tabcolsep}{1.7pt} % Default value: 6pt
\setlength{\belowrulesep}{4pt}
\begin{threeparttable}
    \centering
    \begin{tabular}{cc l r rrr @{\quad} rrrr}
\toprule
\multirow{2}{*}{} & \multirow{2}{*}{} & \multirow{2}{*}{Feedback} & \multirow{2}{*}{O} & \multicolumn{3}{c}{Distraction} & \multicolumn{4}{c}{Counterfactual} \\
 & & & & E& L & T & $\text{O}_C$ & $\text{E}_C$& $\text{L}_C$ & $\text{T}_C$\\
\midrule
\multirow{8}{*}{\rotatebox{90}{Llama-3.1}} & \multirow{4}{*}{\rotatebox{90}{8B}} 
   & No recovery & 0.77 & \textbf{0.72} & 0.62 & 0.53 & 0.59 & 0.58 & 0.56 & \textbf{0.56} \\
 & & Error type & \textbf{0.79} & 0.71 & 0.63 & \textbf{0.56} & \textbf{0.66} & 0.54 & 0.52 & 0.51 \\
 & & Error message & 0.78 & 0.71 & \textbf{0.67} & 0.55 & 0.59 & 0.53 & \underline{\textbf{0.64}} & 0.49 \\
 & & Warning & 0.74 & 0.66 & 0.58 & 0.55 & 0.55 & \textbf{0.60} & 0.49 & 0.49 \\[\modelspacing]
\cmidrule{2-11}
 & \multirow{4}{*}{\rotatebox{90}{70B}} 
   & No recovery & \textbf{0.77} & \textbf{0.72} & \textbf{0.73} & 0.71 & \textbf{0.64} & 0.59 & \textbf{0.61} & 0.56 \\
 & & Error type & 0.72 & 0.70 & 0.72 & \textbf{0.73} & 0.62 & 0.56 & 0.60 & 0.58 \\
 & & Error message & 0.71 & 0.70 & \textbf{0.73} & 0.71 & \textbf{0.64} & 0.59 & 0.54 & \underline{\textbf{0.64}} \\
 & & Warning & 0.69 & \textbf{0.72} & 0.72 & 0.72 & 0.62 & \underline{\textbf{0.65}} & \textbf{0.61} & 0.63 \\[\modelspacing]
\midrule
\multirow{4}{*}{\rotatebox{90}{GPT}} & \multirow{4}{*}{\rotatebox{90}{4o-mini}} 
   & No recovery & \underline{\textbf{0.84}} & \underline{\textbf{0.82}} & 0.73 & 0.79 & 0.64 & \textbf{0.62} & 0.56 & \textbf{0.56} \\
 & & Error type & 0.83 & 0.79 & 0.74 & 0.76 & 0.67 & 0.57 & 0.56 & \textbf{0.56} \\
 & & Error message & \underline{\textbf{0.84}} & 0.78 & \underline{\textbf{0.77}} & \underline{\textbf{0.80}} & 0.62 & 0.59 & 0.56 & \textbf{0.56} \\
 & & Warning & \underline{\textbf{0.84}} & 0.75 & 0.73 & 0.76 & \underline{\textbf{0.70}} & 0.61 & \textbf{0.61} & 0.55 \\
 \bottomrule
\end{tabular}
\caption{Accuracies of error recovery strategies.}
\label{tab:distraction_k4_feedback}
\end{threeparttable}
\end{table} 

\subsection{Error Analysis}
\label{subsec:errors}
\paragraph{\textbf{\emph{F7: Autoformalisation increases syntax errors for noise perturbations.}}}
The low performance for noise perturbations correlates with more syntax errors for all models and distraction categories (cf. execution rates in Table~\ref{tab:appendix_k4_formalisation_exec}). The three worst-performing models (both Mistral models, Gemma-2 9b) generate, at best, for $37\%$  and, at worst, for only $4\%$ of the samples, a valid logical form.
Gemma-2 9b and Llama3.1 8b produce more syntax errors than the larger counterparts, suggesting that larger models are more robust towards noise perturbations. 
The accuracy of syntactically valid samples is higher than the informal reasoning methods for most distractions (Table~\ref{tab:appendix_k4_formalisation_vacc}), motivating informal reasoning as a backup strategy for formal reasoning. The error message feedback reveals two common syntax errors: 1) errors by models with an initial low execution rate exhibit issues with the template structure, including using incorrect keywords or adding conversational phrases;
2) perturbation-related errors, the most common of which is using undefined truth constants as part of tautological distractions. 

\paragraph{\textbf{\emph{F8: Autoformalisation increases semantic errors for counterfactuals.}}}
Unlike the introduced noise, counterfactual perturbations do not lead to more syntax errors. The execution rate in Table~\ref{tab:appendix_k4_formalisation_exec} is stable or improves for counterfactuals. However, we see a drop in accuracy for the counterfactual column $\text{O}_C$ in Table~\ref{tab:distraction_k4_formalisation} and can conclude that the number of logical forms with semantic errors has to increase. This suggests that the introduced negation is not correctly formalised. Looking at the warnings generated by the feedback mechanism, for GPT 4o-mini, $161$ warning messages are generated on the unperturbed data. $54$ of these were fixed with a single iteration. Not considering predicates and individuals as part of the context is the most frequent warning across all models. 
\begin{table}[ht!]
\centering
\caption{\textbf{Super Resolution Performance Results.} Our proposed WGAN EEG Spatial Upsampling method significantly outperforms a baseline of Bicubic Interpolation commonly used in EEG upsampling pipelines.}
\label{tab:results}
\resizebox{0.8\linewidth}{!}{%
\begin{tabular}{@{}cccccc@{}}
\toprule
\multirow{2}{*}{\textbf{Dataset}} & \multirow{2}{*}{\textbf{Scale}} & \multicolumn{2}{c}{\textbf{Bicubic}} & \multicolumn{2}{c}{\textbf{WGAN}} \\ \cmidrule(l){3-6} 
                      &   & \textbf{MSE} & \textbf{MAE} & \textbf{MSE}    & \textbf{MAE}   \\
\toprule
\multirow{2}{*}{Val}  & 2 & 3.71E7       & 3.89E3       & \textbf{2.01E3} & \textbf{24.38} \\
                      & 4 & 7.23E7       & 6.42E3       & \textbf{8.53E3} & \textbf{63.83} \\
\midrule
\multirow{2}{*}{Test} & 2 & 3.75E7       & 3.91E3       & \textbf{2.06E3} & \textbf{24.66} \\
                      & 4 & 7.30E7       & 6.45E3       & \textbf{8.68E3} & \textbf{64.39} \\
\bottomrule
\end{tabular}%
}
\end{table}
This work identifies signal collapse as a critical bottleneck in one-shot neural network pruning. Performance loss in pruned networks is due to \textbf{signal collapse} in addition to the removal of critical parameters. We propose \textbf{REFLOW} (\textbf{Re}storing \textbf{F}low of \textbf{Low}-variance signals), a simple yet effective method that mitigates signal collapse without computationally expensive weight updates. By focusing on signal preservation, REFLOW highlights the importance of mitigating signal collapse in sparse networks and enables magnitude pruning to match or surpass state-of-the-art one-shot pruning methods such as CHITA, CBS, and WF.

REFLOW consistently achieves state-of-the-art accuracy across diverse architectures, restoring ResNeXt-101 from under 4.1\% to 78.9\% top-1 accuracy at 80\% sparsity on ImageNet. Its lightweight design makes it a practical solution for both research and deployment, delivering high-quality sparse models without the overhead of traditional approaches. These findings challenge the traditional emphasis on weight selection strategies and underscore the critical role of signal propagation for achieving high-quality sparse networks in the context of one-shot pruning.



\section*{Conclusion}
This paper aims to enhance our understanding of the computational complexity of computing various Shapley value variants. We found that for various ML models --- including decision trees, regression tree ensembles, weighted automata, and linear regression --- both local and global interventional and baseline SHAP can be computed in polynomial time under HMM modeled distributions. This extends popular algorithms, such as TreeSHAP, beyond their empirical distributional scope. We also establish strict complexity gaps between the various SHAP variants (baseline, interventional, and conditional) and prove the intractability of computing SHAP for tree ensembles and neural networks in simplified scenarios. Overall, we present SHAP as a versatile framework whose complexity depends on four key factors: \begin{inparaenum}[(i)] \item model type, \item SHAP variant, \item distribution modeling approach, \item and local vs. global explanations\end{inparaenum}. We believe this perspective provides deeper insight into the computational complexity of SHAP, paving the way for future work.




%We believe that our framework provides a more intricate understanding of SHAP computation complexity across different models, distributions, and variants, paving the way for further research.

Our work opens promising directions for future research. First, expanding our computational analysis to other SHAP-related metrics, such as asymmetric SHAP~\citep{frye20} and SAGE~\citep{covert2020understanding}, would be valuable. Additionally, we aim to explore more expressive distribution classes and relaxed assumptions beyond those in Section \ref{sec:tractable} while maintaining tractable SHAP computation. Finally, when exact computation is intractable (Section \ref{sec:intractable}), investigating the approximability of SHAP metrics through approximation and parameterized complexity theory~\citep{downey2012parameterized} is an important direction.

%Our work opens several promising avenues for future research on the computational properties of explainable AI methods, with a particular focus on SHAP. First, it would be interesting to broaden the computational analysis conducted in this work to include other popular SHAP-related metrics in the literature, such as asymmetric SHAP \cite{frye20} and SAGE \cite{covert2020understanding}. Also, in the future, we aim to explore more expressive distribution classes and relaxed distributional assumptions—extending beyond those examined in Section \ref{sec:tractable} —that still yield tractable SHAP computation. Finally, when exact computation proves intractable (Section \ref{sec:intractable}), it is worthwhile to theoretically investigate the question of the approximability of computing the SHAP metrics across various configurations, through the lens of approximation and parametrized complexity theory \cite{arora2009computational}.

%This paper aims to deepen our understanding of the computational complexity involved in obtaining different Shapley value variants. We found that for a variety of ML models, including decision trees, tree ensembles for regression, weighted automata, and linear regression models — computing both local and global interventional and baseline SHAP can be done in polynomial time when distributions are modeled by HMMs. This extends the distributional scope of popular algorithms like TreeSHAP, which is limited to empirical distributions. Additionally, we demonstrate a strict complexity gap between SHAP variants, showing that interventional and baseline SHAP can be strictly easier to compute than conditional SHAP. Despite these positive results, we uncovered intractability for various SHAP variants in neural networks and tree ensembles. Finally, we provided generalized complexity relations across SHAP variants. We believe that our framework offers a deeper understanding of the complexity involved in computing SHAP across various variants, models, distributions, as well as in both local and global computations, laying the groundwork for future research.
\section{Acknowledgments}
We would like to thank Kostis Kaffes, Tanvir Ahmed Khan, and Yuhong Zhong for their feedback.
We also thank the CloudLab team for their help in supporting our experiments.
This work was supported by IBM, and NSF awards CNS-2143868 and CNS-2106530.
Tal Zussman was supported by NSF award DGE-2036197.
Ioannis Zarkadas is an Onassis Foundation scholar.




%%
%% The next two lines define the bibliography style to be used, and
%% the bibliography file.
\bibliographystyle{ACM-Reference-Format}
\bibliography{CHI25}




\end{document}
\endinput
%%
%% End of file `sample-acmlarge.tex'.
