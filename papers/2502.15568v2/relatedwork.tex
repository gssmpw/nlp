\section{Related Work}
Research into the algorithmic political biases of LLMs predates the advent of ChatGPT, with early studies proposing methods to mitigate such biases~\cite{aibias, liu2022quantifying}.
%Following the emergence of ChatGPT, researchers have applied these tests—for example, political questionnaires on Dutch and German politics have suggested that ChatGPT would favor left-wing parties \cite{van2023chatgpt, hartmann2023political}.
%Following the emergence of ChatGPT, researchers have applied these tests.\todo{which tests?}
Political questionnaires focused on Dutch and German politics suggest that ChatGPT tends to favor left-wing parties~\cite{van2023chatgpt, hartmann2023political}.
Other studies have shown that ChatGPT treats different demographic groups and politicians unequally~\cite{mcgee2023chat, rozado2023danger, mcgee2023were}.
When subjected to the Political Compass Test---both in its default mode and while simulating US Democrat and Republican personas---ChatGPT's responses displayed a significant alignment with Democratic leaning~\cite{motoki2023more}.
Furthermore, an evaluation using 15 distinct political affiliation tests revealed that 14 of them indicated a progressive bias in its responses~\cite{rozado2023political}.
%When subjected to the political compass test, both as itself and while using US Democrat and Republican personas, ChatGPT's responses showed a significant overlap with Democrat leanings \cite{motoki2023more}, and an evaluation using 15 different political affiliation tests found that 14 indicated a progressive bias \cite{rozado2023political}. %These observations point to an overall progressive and libertarian bias in ChatGPT. 
However, many of these studies are limited by single-test evaluations, which fail to account for the stochastic nature of LLMs.
Ignoring this inherent variability diminishes the reliability and informativeness of these findings.
To address this limitation, \cite{rutinowski2024self} conducted an analysis of ChatGPT utilizing multiple test repetitions and similarly observed a left-leaning political alignment.
%However, most of these studies have been limited by single-test evaluations that neglect the stochastic nature of LLMs. Disregarding this inherent variability significantly reduces the reliability and informativeness of these findings.
%For this reason, \cite{rutinowski2024self} evaluated the political bias of ChatGPT based on multiple test repetitions. They also observed a left-leaning political alignment. 


Similar patterns of political bias have been observed in other LLMs.
For instance, \cite{rettenberger2024assessing} analyzed the political orientation of Mistral7B and various versions of Meta's Llama using Wahl-O-Mat statements from the 2024 European election.
Their analysis, conducted through a single round of questioning, revealed that \enquote{\itshape larger models, such as Llama3-70B, tend to align more closely with left-leaning political parties}.
Political biases also arise in DeepSeek, an LLM recently attracting public attention.
An evaluation of DeepSeek R1 found that it subtly promotes authoritarianism by emphasizing stability.
%Furthermore, the model avoids negative statements about the Chinese government, thereby conforming to governmental censorship measures.
At the same time, DeepSeek R1 provides critical perspectives on issues such as freedom of religion and freedom of the press \citep{gupta2025comparative}.


%---------- Analyses of Stochastic LLMs ----------