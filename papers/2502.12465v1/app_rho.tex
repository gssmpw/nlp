
In this section, we prove \cref{thm:rho-il}, which is an immediate
corollary of \Cref{thm:rho}, a result of \citet{baraud2018rho}) which we prove for completeness below.% 
First, recall the function $\tau: (0,\infty) \to \RR$ defined to be
\begin{align}
  \tau(x) = \frac{\sqrt{1/x} - 1}{\sqrt{1/x} + 1},
\end{align}
and note that $\abs{\tau(x)} \leq 1$ for all $x > 0$.  The utility of
the $\tau$ function can be captured in the following lemma (originally
from \citet{baraud2018rho}), demonstrating that its expectation can be related to Hellinger distances.
\begin{lemma}[see e.g. {\citet[Theorem 97]{lerasle2019lecture}}]\label{lemma:rho-estimator-bounds}
For any set $\MX$ and densities $p,q,\pstar \in \Delta(\MX)$, it holds that
\begin{equation} -4\Dhels{\pstar}{q} + \frac{3}{8} \Dhels{\pstar}{p} \leq \EE_{x \sim \pstar}\left[\tau\left(\frac{p(x)}{q(x)}\right)\right] \leq 4\Dhels{\pstar}{p} - \frac{3}{8} \Dhels{\pstar}{q}
\label{eq:tau-hels}
\end{equation}
and
\begin{equation}
\EE_{x \sim \pstar}\left[\tau^2\left(\frac{p(x)}{q(x)}\right)\right] \leq 3\sqrt{2}\left(\Dhels{p^\st}{p} + \Dhels{p^\st}{q}\right).
\label{eq:tau-sq-hels}
\end{equation}
\end{lemma}
Following \citet{baraud2018rho,lerasle2019lecture} and using \Cref{lemma:rho-estimator-bounds}, we can now prove the following theorem on misspecified distribution learning in Hellinger distance.
\begin{theorem}\label{thm:rho}
Fix a set $\cX$, a family of distributions $\cP \subset \Delta(\cX)$, a distribution $p^\st \in \Delta(\cX)$. Let $n \in \NN$ and $\delta \in (0,1/2)$. Let $x\ind{1},\dots,x\ind{n}$ be $n$ i.i.d. samples from $p^\st$. Then the $\rho$-estimator 
\[\phat := \argmin_{p \in \cP} \sup_{q \in \cP} 
\sum_{i=1}^n \tau\left(\frac{p(x\ind{i})}{q(x\ind{i})}
\right)\]
satisfies, with probability at least $1-\delta$,
\begin{equation} \Dhels{\phat}{p^\st} \lesssim \frac{\log(|\cP|/\delta)}{n} + \min_{p\in\cP} \Dhels{p}{p^\st}.
\label{eq:rho-thm}\end{equation}
\end{theorem}

\begin{proof}[\pfref{thm:rho}]
Note that $\tau$ has range in $[-1,1]$. By Bernstein's inequality and a union bound over $p,q \in \cP$, there is an event $\cE$ that occurs with probability at least $1-\delta$, in which for all $p,q \in \cP$,
\begin{equation}\left| \sum_{i=1}^n \tau\left(\frac{p(x\ind{i})}{q(x\ind{i})}
\right) - n \cdot \EE_{x \sim \pstar}\left[\tau\left(\frac{p(x)}{q(x)}\right)\right]\right| \leq \frac{n}{4} \cdot \EE_{x \sim \pstar}\left[\tau^2\left(\frac{p(x)}{q(x)}\right)\right] + 4\log(4|\cP|/\delta).\label{eq:tau-generalization}\end{equation}
Condition on the event $\cE$ henceforth. Let $\pbar := \argmin_{p \in \cP} \Dhels{p}{\pstar}$. Then
\begin{align}
\frac{3}{8}\Dhels{\pstar}{\phat}
&\leq 4\Dhels{\pstar}{\pbar} + \EE_{x \sim \pstar}\left[\tau\left(\frac{\phat(x)}{\pbar(x)}\right)\right] \\ 
&\leq 4\Dhels{\pstar}{\pbar} + \frac{1}{n} \sum_{i=1}^n\tau\left(\frac{\phat(x\ind{i})}{\pbar(x\ind{i})}\right) + \frac{1}{12\sqrt{2}} \EE_{x \sim \pstar}\left[\tau^2\left(\frac{\phat(x)}{\pbar(x)}\right)\right] + \frac{12\sqrt{2}\log(4|\cP|/\delta)}{n} \\ 
&\leq \frac{17}{4}\Dhels{\pstar}{\pbar} + \frac{1}{4}\Dhels{\pstar}{\phat} + \frac{1}{n}\sum_{i=1}^n\tau\left(\frac{\phat(x\ind{i})}{\pbar(x\ind{i})}\right) + \frac{12\sqrt{2}\log(4|\cP|/\delta)}{n}
\end{align}
where the first inequality is by \cref{eq:tau-hels} of \cref{lemma:rho-estimator-bounds}, the second inequality is by \cref{eq:tau-generalization}, and the third inequality is by \cref{eq:tau-sq-hels} of \cref{lemma:rho-estimator-bounds}. Rearranging, we get
\begin{align}
\frac{1}{8}\Dhels{\pstar}{\phat}
&\leq \frac{17}{4}\Dhels{\pstar}{\pbar} + \frac{1}{n}\sum_{i=1}^n\tau\left(\frac{\phat(x\ind{i})}{\pbar(x\ind{i})}\right) + \frac{12\sqrt{2}\log(4|\cP|/\delta)}{n} \\ 
&\leq \frac{17}{4}\Dhels{\pstar}{\pbar} + \sup_{q \in \cP} \frac{1}{n}\sum_{i=1}^n\tau\left(\frac{\phat(x\ind{i})}{q(x\ind{i})}\right) + \frac{12\sqrt{2}\log(4|\cP|/\delta)}{n} \\ 
&\leq \frac{17}{4}\Dhels{\pstar}{\pbar} + \sup_{q \in \cP} \frac{1}{n}\sum_{i=1}^n\tau\left(\frac{\pbar(x\ind{i})}{q(x\ind{i})}\right) + \frac{12\sqrt{2}\log(4|\cP|/\delta)}{n}.\label{eq:rho-hels-intermediate}
\end{align}
Now for any $q \in \cP$, we have
\begin{align}
\frac{1}{n}\sum_{i=1}^n\tau\left(\frac{\pbar(x\ind{i})}{q(x\ind{i})}\right)
&\leq \EE_{x \sim \pstar}\left[\tau\left(\frac{\pbar(x)}{q(x)}\right)\right] + \frac{1}{12\sqrt{2}} \EE_{x \sim \pstar}\left[\tau^2\left(\frac{\pbar(x)}{q(x)}\right)\right] + \frac{12\sqrt{2}\log(4|\cP|/\delta)}{n} \\ 
&\leq 4\Dhels{\pstar}{\pbar} - \frac{3}{8} \Dhels{\pstar}{q} + \frac{1}{4}\left(\Dhels{\pstar}{\pbar} + \Dhels{\pstar}{q}\right) + \frac{12\sqrt{2}\log(4|\cP|/\delta)}{n} \\ 
&\leq \frac{17}{4}\Dhels{\pstar}{\pbar} + \frac{12\sqrt{2}\log(4|\cP|/\delta)}{n}
\end{align}
where the first inequality is by \cref{eq:tau-generalization} and the second inequality is by \cref{lemma:rho-estimator-bounds}. Substituting into \cref{eq:rho-hels-intermediate}, we get
\[\Dhels{\pstar}{\wh p} \leq 68 \Dhels{\pstar}{\pbar} + \frac{192\sqrt{2}\log(4|\cP|/\delta)}{n}\]
as claimed.
\end{proof}
We can now prove \cref{thm:rho-il} as a corollary of \cref{thm:rho}.
%
\begin{proof}[\pfref{thm:rho-il}]
  Note that for any policies $\pi,\pi'$ and trajectory $o=(s_1,a_1,\dots,s_H,a_H)$,
  \begin{align}
    \frac{\pp^\pi(\obs)}{\pp^{\pi'}(\obs)} = \prod_{h = 1}^H \frac{\bbP_h(s_{h+1} \mid{} a_h, s_h) \pi_h(a_h \mid{} s_h)}{\bbP_h(s_{h+1} \mid{} a_h, s_h) \pi_h'(a_h \mid{} s_h)} = \prod_{h = 1}^H \frac{\pi_h(a_h \mid{} s_h)}{\pi_h'(a_h \mid{} s_h)},
  \end{align}
  and thus, from \cref{eq:rhobc},
  \begin{align}
    \pihat = \argmin_{\pi \in \Pi} \sup_{\pi'}\sum_{i = 1}^n \tau\left( \frac{\pp^\pi(\obs)}{\pp^{\pi'}(\obs)} \right).
  \end{align}
  The result then follows from \cref{thm:rho} by letting $\cP = \left\{ \pp^{\pi} 
  \mid{} \pi \in \Pi \right\}$, $\pstar = \pp^{\pistar}$ and observing that $\phat = \pp^{\pihat}$ by the preceding display.
\end{proof}

%




