\section{Conclusion}
In this work, we introduce relevance saturation bias, where queries with numerous relevant documents are more likely to receive clicks. We refer to a query's probability of receiving clicks as the query-level click propensity. To characterize relevance saturation bias, we propose a new click hypothesis: the click probability of a document is determined by its relevance, observation probability, and query-level click propensity. Based on this hypothesis, we propose the Dual Inverse Propensity Weighting (\m) method, which contains inverse query-level click propensity weighting and inverse position-level propensity weighting to alleviate relevance saturation bias and position bias, respectively. We demonstrate that leveraging \m~can learn an unbiased ranking model and achieve superior performance compared to existing ULTR methods on the Baidu-ULTR dataset. 