\section{Related Work}
\label{sec:rel_work}

The Gossip protocol is also known as the Epidemic Protocol, as these algorithms were first developed to imitate how epidemics spread \cite{10.1145/41840.41841}. One of the first gossip algorithms to be studied was rumor spreading or randomized broadcast \cite{652705be-95a4-3886-b436-08eb167091e1, 892324}, where in the latter the authors give an algorithm for rumor spreading in $O(\log n)$ rounds and $O(n \log \log n)$ messages. 

In \cite{kempe2003gossip}, the authors study a gossip protocol to compute aggregates such as sums and counts in $O(\log n)$ rounds. Furthermore, they also develop algorithms for random sampling and quantile computation (also known as randomized selection), the latter of which can be computed in $O(\log^2 n)$ rounds. \cite{haeupler2018optimal} improve on that, and give an algorithm to compute exact quantiles in $O(\log n)$ rounds and approximate quantiles in $O(\log \log n + \log (1 / \varepsilon))$ rounds. The problem of computing quantiles has also been studied in the centralized setting \cite{BLUM1973448} and the distributed setting \cite{10.1145/1248377.1248401}. Other problems such as computing the mode \cite{10.1145/2933057.2933097} (also known as plurality consensus), have also been studied in the gossip setting. 

Both in \cite{10.1007/978-3-642-14162-1_10} and \cite{892324}, the authors investigate a gossip-based rumor spreading in the presence of adaptive failures, but the adversary here only has the power to crash certain messages, and cannot send altered messages (which is much more powerful). To the best of our knowledge, this is the first time anyone has investigated a gossip model for aggregation in the presence of adversarial nodes.