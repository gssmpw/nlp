%%
%% This is file `sample-sigconf.tex',
%% generated with the docstrip utility.
%%
%% The original source files were:
%%
%% samples.dtx  (with options: `all,proceedings,bibtex,sigconf')
%% 
%% IMPORTANT NOTICE:
%% 
%% For the copyright see the source file.
%% 
%% Any modified versions of this file must be renamed
%% with new filenames distinct from sample-sigconf.tex.
%% 
%% For distribution of the original source see the terms
%% for copying and modification in the file samples.dtx.
%% 
%% This generated file may be distributed as long as the
%% original source files, as listed above, are part of the
%% same distribution. (The sources need not necessarily be
%% in the same archive or directory.)
%%
%%
%% Commands for TeXCount
%TC:macro \cite [option:text,text]
%TC:macro \citep [option:text,text]
%TC:macro \citet [option:text,text]
%TC:envir table 0 1
%TC:envir table* 0 1
%TC:envir tabular [ignore] word
%TC:envir displaymath 0 word
%TC:envir math 0 word
%TC:envir comment 0 0
%%
%%
%% The first command in your LaTeX source must be the \documentclass
%% command.
%%
%% For submission and review of your manuscript please change the
%% command to \documentclass[manuscript, screen, review]{acmart}.
%%
%% When submitting camera ready or to TAPS, please change the command
%% to \documentclass[sigconf]{acmart} or whichever template is required
%% for your publication.
%%
%%
\documentclass[sigconf]{acmart}




\copyrightyear{2025}
\acmYear{2025}
%% \setcopyright{cc}
%% \setcctype{CC-BY}
\setcopyright{rightsretained}
\acmConference[FPGA '25]{Proceedings of the 2025 ACM/SIGDA International Symposium on Field Programmable Gate Arrays}{February 27--March 1, 2025}{Monterey, CA, USA}
\acmBooktitle{Proceedings of the 2025 ACM/SIGDA International Symposium on Field Programmable Gate Arrays (FPGA '25), February 27--March 1, 2025, Monterey, CA, USA}
\acmDOI{10.1145/3706628.3708867}
\acmISBN{979-8-4007-1396-5/25/02}

% The following includes the CC license icon appropriate for your paper.
% Download the image from www.scomminc.com/pp/acmsig/4ACM-CC-by-88x31.eps
% and place within your figs or figures folder

% \makeatletter
% \gdef\@copyrightpermission{
%   \begin{minipage}{0.3\columnwidth}
%    \href{https://creativecommons.org/licenses/by/4.0/}{\includegraphics[width=0.90\textwidth]{4ACM-CC-by-88x31.eps}}
%   \end{minipage}\hfill
%   \begin{minipage}{0.7\columnwidth}
%    \href{https://creativecommons.org/licenses/by/4.0/}{This work is licensed under a Creative Commons Attribution International 4.0 License.}
%   \end{minipage}
%   \vspace{5pt}
% }
% \makeatother


%%
%% Submission ID.
%% Use this when submitting an article to a sponsored event. You'll
%% receive a unique submission ID from the organizers
%% of the event, and this ID should be used as the parameter to this command.
%%\acmSubmissionID{123-A56-BU3}

%%
%% For managing citations, it is recommended to use bibliography
%% files in BibTeX format.
%%
%% You can then either use BibTeX with the ACM-Reference-Format style,
%% or BibLaTeX with the acmnumeric or acmauthoryear sytles, that include
%% support for advanced citation of software artefact from the
%% biblatex-software package, also separately available on CTAN.
%%
%% Look at the sample-*-biblatex.tex files for templates showcasing
%% the biblatex styles.
%%

%%
%% The majority of ACM publications use numbered citations and
%% references.  The command \citestyle{authoryear} switches to the
%% "author year" style.
%%
%% If you are preparing content for an event
%% sponsored by ACM SIGGRAPH, you must use the "author year" style of
%% citations and references.
%% Uncommenting
%% the next command will enable that style.
%%\citestyle{acmauthoryear}

\usepackage{comment}
\usepackage{graphicx}
\usepackage{xcolor}
\usepackage{multirow}
\usepackage{siunitx}
\usepackage{todonotes}
\usepackage{hyperref}
% \usepackage{cite}
\usepackage{amsmath}
\usepackage{amsfonts}
\usepackage{algorithmic}
\usepackage{subfig}
\usepackage{makecell}
\usepackage{svg}
\usepackage{tablefootnote}
\usepackage[flushleft]{threeparttable}
\usepackage[symbol]{footmisc}
\usepackage{fixltx2e}
\usepackage{caption} 
\captionsetup[table]{skip=4pt}
\usepackage{soul}




%These 4 lines solve the problem with the url overflowing in the references for WaveNet. Had to comment hyperref above.
\usepackage{url}
% \usepackage[breaklinks]{hyperref}
% \usepackage{breakurl}


%%
%% end of the preamble, start of the body of the document source.
\begin{document}

%%
%% The "title" command has an optional parameter,
%% allowing the author to define a "short title" to be used in page headers.
\title{Systolic Sparse Tensor Slices: FPGA Building Blocks for Sparse and Dense AI Acceleration}



\author{Endri Taka}
\affiliation{%
  \institution{The University of Texas at Austin}
  \city{Austin}
  \state{TX}
  \country{United States}
}
  \email{endri.taka@utexas.edu}


\author{Ning-Chi Huang}
\affiliation{%
  \institution{National Yang Ming Chiao Tung University}
  \city{Hsinchu}
  \country{Taiwan}
  % \email{nchuang@cs.nctu.edu.tw}
}

\author{Chi-Chih Chang}
\affiliation{%
  \institution{National Yang Ming Chiao Tung University}
  \city{Hsinchu}
  \country{Taiwan}
  % \email{brian1009.en08@nycu.edu.tw}
}

\author{Kai-Chiang Wu}
\affiliation{%
  \institution{National Yang Ming Chiao Tung University}
  \city{Hsinchu}
  \country{Taiwan}
  % \email{kcw@cs.nctu.edu.tw}
}


\author{Aman Arora}
\affiliation{%
  \institution{Arizona State University}
  \city{Tempe}
  \state{AZ}
  \country{United States}
  % \email{aman.kbm@asu.edu}
}

\author{Diana Marculescu}
\affiliation{%
  \institution{The University of Texas at Austin}
  \city{Austin}
  \state{TX}
  \country{United States}
  % \email{dianam@utexas.edu}
}

%%
%% By default, the full list of authors will be used in the page
%% headers. Often, this list is too long, and will overlap
%% other information printed in the page headers. This command allows
%% the author to define a more concise list
%% of authors' names for this purpose.
\renewcommand{\shortauthors}{Endri Taka et al.}

%%
%% The abstract is a short summary of the work to be presented in the
%% article.
\begin{abstract}

FPGA architectures have recently been enhanced to meet the substantial computational demands of modern deep neural networks (DNNs).
To this end, both FPGA vendors and academic researchers have proposed in-fabric blocks that perform efficient tensor computations. 
However, these blocks are primarily optimized for dense computation, while most DNNs exhibit sparsity. 
To address this limitation, we propose incorporating \textit{structured} sparsity support into FPGA architectures.
We architect 2D systolic in-fabric blocks, 
named systolic sparse tensor (SST) slices, that support multiple degrees of sparsity to efficiently accelerate a wide variety of DNNs.
SSTs support dense operation, 2:4 (50\%) and 1:4 (75\%) sparsity, as well as a new 1:3 (66.7\%) sparsity level to further increase flexibility.
When demonstrating on general matrix multiplication (GEMM) accelerators, which are the heart of most current DNN accelerators, our sparse SST-based designs attain up to 5$\times$ higher FPGA frequency and 10.9$\times$ lower area, compared to traditional FPGAs.
Moreover, evaluation of the proposed SSTs on state-of-the-art sparse ViT and CNN models exhibits up to 3.52$\times$ speedup with minimal area increase of up to 13.3\%, compared to dense in-fabric acceleration.




\end{abstract}



\begin{CCSXML}
<ccs2012>
   <concept>
       <concept_id>10010583.10010600.10010628.10010629</concept_id>
       <concept_desc>Hardware~Hardware accelerators</concept_desc>
       <concept_significance>500</concept_significance>
       </concept>
   <concept>
       <concept_id>10010520.10010521.10010528.10010535</concept_id>
       <concept_desc>Computer systems organization~Systolic arrays</concept_desc>
       <concept_significance>500</concept_significance>
       </concept>
 </ccs2012>
\end{CCSXML}

\ccsdesc[500]{Hardware~Hardware accelerators}
\ccsdesc[500]{Computer systems organization~Systolic arrays}



%%
%% Keywords. The author(s) should pick words that accurately describe
%% the work being presented. Separate the keywords with commas.
% \keywords{FPGA, structured sparsity, computer architecture, deep learning}


\keywords{FPGA, structured sparsity, hardware acceleration, matrix multiplication, computer architecture, deep learning, machine learning}



%%
%% This command processes the author and affiliation and title
%% information and builds the first part of the formatted document.
\maketitle



\section{Introduction}
\label{sec:Introduction}
\section{Introduction}

Large language models (LLMs) have achieved remarkable success in automated math problem solving, particularly through code-generation capabilities integrated with proof assistants~\citep{lean,isabelle,POT,autoformalization,MATH}. Although LLMs excel at generating solution steps and correct answers in algebra and calculus~\citep{math_solving}, their unimodal nature limits performance in plane geometry, where solution depends on both diagram and text~\citep{math_solving}. 

Specialized vision-language models (VLMs) have accordingly been developed for plane geometry problem solving (PGPS)~\citep{geoqa,unigeo,intergps,pgps,GOLD,LANS,geox}. Yet, it remains unclear whether these models genuinely leverage diagrams or rely almost exclusively on textual features. This ambiguity arises because existing PGPS datasets typically embed sufficient geometric details within problem statements, potentially making the vision encoder unnecessary~\citep{GOLD}. \cref{fig:pgps_examples} illustrates example questions from GeoQA and PGPS9K, where solutions can be derived without referencing the diagrams.

\begin{figure}
    \centering
    \begin{subfigure}[t]{.49\linewidth}
        \centering
        \includegraphics[width=\linewidth]{latex/figures/images/geoqa_example.pdf}
        \caption{GeoQA}
        \label{fig:geoqa_example}
    \end{subfigure}
    \begin{subfigure}[t]{.48\linewidth}
        \centering
        \includegraphics[width=\linewidth]{latex/figures/images/pgps_example.pdf}
        \caption{PGPS9K}
        \label{fig:pgps9k_example}
    \end{subfigure}
    \caption{
    Examples of diagram-caption pairs and their solution steps written in formal languages from GeoQA and PGPS9k datasets. In the problem description, the visual geometric premises and numerical variables are highlighted in green and red, respectively. A significant difference in the style of the diagram and formal language can be observable. %, along with the differences in formal languages supported by the corresponding datasets.
    \label{fig:pgps_examples}
    }
\end{figure}



We propose a new benchmark created via a synthetic data engine, which systematically evaluates the ability of VLM vision encoders to recognize geometric premises. Our empirical findings reveal that previously suggested self-supervised learning (SSL) approaches, e.g., vector quantized variataional auto-encoder (VQ-VAE)~\citep{unimath} and masked auto-encoder (MAE)~\citep{scagps,geox}, and widely adopted encoders, e.g., OpenCLIP~\citep{clip} and DinoV2~\citep{dinov2}, struggle to detect geometric features such as perpendicularity and degrees. 

To this end, we propose \geoclip{}, a model pre-trained on a large corpus of synthetic diagram–caption pairs. By varying diagram styles (e.g., color, font size, resolution, line width), \geoclip{} learns robust geometric representations and outperforms prior SSL-based methods on our benchmark. Building on \geoclip{}, we introduce a few-shot domain adaptation technique that efficiently transfers the recognition ability to real-world diagrams. We further combine this domain-adapted GeoCLIP with an LLM, forming a domain-agnostic VLM for solving PGPS tasks in MathVerse~\citep{mathverse}. 
%To accommodate diverse diagram styles and solution formats, we unify the solution program languages across multiple PGPS datasets, ensuring comprehensive evaluation. 

In our experiments on MathVerse~\citep{mathverse}, which encompasses diverse plane geometry tasks and diagram styles, our VLM with a domain-adapted \geoclip{} consistently outperforms both task-specific PGPS models and generalist VLMs. 
% In particular, it achieves higher accuracy on tasks requiring geometric-feature recognition, even when critical numerical measurements are moved from text to diagrams. 
Ablation studies confirm the effectiveness of our domain adaptation strategy, showing improvements in optical character recognition (OCR)-based tasks and robust diagram embeddings across different styles. 
% By unifying the solution program languages of existing datasets and incorporating OCR capability, we enable a single VLM, named \geovlm{}, to handle a broad class of plane geometry problems.

% Contributions
We summarize the contributions as follows:
We propose a novel benchmark for systematically assessing how well vision encoders recognize geometric premises in plane geometry diagrams~(\cref{sec:visual_feature}); We introduce \geoclip{}, a vision encoder capable of accurately detecting visual geometric premises~(\cref{sec:geoclip}), and a few-shot domain adaptation technique that efficiently transfers this capability across different diagram styles (\cref{sec:domain_adaptation});
We show that our VLM, incorporating domain-adapted GeoCLIP, surpasses existing specialized PGPS VLMs and generalist VLMs on the MathVerse benchmark~(\cref{sec:experiments}) and effectively interprets diverse diagram styles~(\cref{sec:abl}).

\iffalse
\begin{itemize}
    \item We propose a novel benchmark for systematically assessing how well vision encoders recognize geometric premises, e.g., perpendicularity and angle measures, in plane geometry diagrams.
	\item We introduce \geoclip{}, a vision encoder capable of accurately detecting visual geometric premises, and a few-shot domain adaptation technique that efficiently transfers this capability across different diagram styles.
	\item We show that our final VLM, incorporating GeoCLIP-DA, effectively interprets diverse diagram styles and achieves state-of-the-art performance on the MathVerse benchmark, surpassing existing specialized PGPS models and generalist VLM models.
\end{itemize}
\fi

\iffalse

Large language models (LLMs) have made significant strides in automated math word problem solving. In particular, their code-generation capabilities combined with proof assistants~\citep{lean,isabelle} help minimize computational errors~\citep{POT}, improve solution precision~\citep{autoformalization}, and offer rigorous feedback and evaluation~\citep{MATH}. Although LLMs excel in generating solution steps and correct answers for algebra and calculus~\citep{math_solving}, their uni-modal nature limits performance in domains like plane geometry, where both diagrams and text are vital.

Plane geometry problem solving (PGPS) tasks typically include diagrams and textual descriptions, requiring solvers to interpret premises from both sources. To facilitate automated solutions for these problems, several studies have introduced formal languages tailored for plane geometry to represent solution steps as a program with training datasets composed of diagrams, textual descriptions, and solution programs~\citep{geoqa,unigeo,intergps,pgps}. Building on these datasets, a number of PGPS specialized vision-language models (VLMs) have been developed so far~\citep{GOLD, LANS, geox}.

Most existing VLMs, however, fail to use diagrams when solving geometry problems. Well-known PGPS datasets such as GeoQA~\citep{geoqa}, UniGeo~\citep{unigeo}, and PGPS9K~\citep{pgps}, can be solved without accessing diagrams, as their problem descriptions often contain all geometric information. \cref{fig:pgps_examples} shows an example from GeoQA and PGPS9K datasets, where one can deduce the solution steps without knowing the diagrams. 
As a result, models trained on these datasets rely almost exclusively on textual information, leaving the vision encoder under-utilized~\citep{GOLD}. 
Consequently, the VLMs trained on these datasets cannot solve the plane geometry problem when necessary geometric properties or relations are excluded from the problem statement.

Some studies seek to enhance the recognition of geometric premises from a diagram by directly predicting the premises from the diagram~\citep{GOLD, intergps} or as an auxiliary task for vision encoders~\citep{geoqa,geoqa-plus}. However, these approaches remain highly domain-specific because the labels for training are difficult to obtain, thus limiting generalization across different domains. While self-supervised learning (SSL) methods that depend exclusively on geometric diagrams, e.g., vector quantized variational auto-encoder (VQ-VAE)~\citep{unimath} and masked auto-encoder (MAE)~\citep{scagps,geox}, have also been explored, the effectiveness of the SSL approaches on recognizing geometric features has not been thoroughly investigated.

We introduce a benchmark constructed with a synthetic data engine to evaluate the effectiveness of SSL approaches in recognizing geometric premises from diagrams. Our empirical results with the proposed benchmark show that the vision encoders trained with SSL methods fail to capture visual \geofeat{}s such as perpendicularity between two lines and angle measure.
Furthermore, we find that the pre-trained vision encoders often used in general-purpose VLMs, e.g., OpenCLIP~\citep{clip} and DinoV2~\citep{dinov2}, fail to recognize geometric premises from diagrams.

To improve the vision encoder for PGPS, we propose \geoclip{}, a model trained with a massive amount of diagram-caption pairs.
Since the amount of diagram-caption pairs in existing benchmarks is often limited, we develop a plane diagram generator that can randomly sample plane geometry problems with the help of existing proof assistant~\citep{alphageometry}.
To make \geoclip{} robust against different styles, we vary the visual properties of diagrams, such as color, font size, resolution, and line width.
We show that \geoclip{} performs better than the other SSL approaches and commonly used vision encoders on the newly proposed benchmark.

Another major challenge in PGPS is developing a domain-agnostic VLM capable of handling multiple PGPS benchmarks. As shown in \cref{fig:pgps_examples}, the main difficulties arise from variations in diagram styles. 
To address the issue, we propose a few-shot domain adaptation technique for \geoclip{} which transfers its visual \geofeat{} perception from the synthetic diagrams to the real-world diagrams efficiently. 

We study the efficacy of the domain adapted \geoclip{} on PGPS when equipped with the language model. To be specific, we compare the VLM with the previous PGPS models on MathVerse~\citep{mathverse}, which is designed to evaluate both the PGPS and visual \geofeat{} perception performance on various domains.
While previous PGPS models are inapplicable to certain types of MathVerse problems, we modify the prediction target and unify the solution program languages of the existing PGPS training data to make our VLM applicable to all types of MathVerse problems.
Results on MathVerse demonstrate that our VLM more effectively integrates diagrammatic information and remains robust under conditions of various diagram styles.

\begin{itemize}
    \item We propose a benchmark to measure the visual \geofeat{} recognition performance of different vision encoders.
    % \item \sh{We introduce geometric CLIP (\geoclip{} and train the VLM equipped with \geoclip{} to predict both solution steps and the numerical measurements of the problem.}
    \item We introduce \geoclip{}, a vision encoder which can accurately recognize visual \geofeat{}s and a few-shot domain adaptation technique which can transfer such ability to different domains efficiently. 
    % \item \sh{We develop our final PGPS model, \geovlm{}, by adapting \geoclip{} to different domains and training with unified languages of solution program data.}
    % We develop a domain-agnostic VLM, namely \geovlm{}, by applying a simple yet effective domain adaptation method to \geoclip{} and training on the refined training data.
    \item We demonstrate our VLM equipped with GeoCLIP-DA effectively interprets diverse diagram styles, achieving superior performance on MathVerse compared to the existing PGPS models.
\end{itemize}

\fi 


\section{Related Work}
\label{sec:Related_work}
\section{Related Work}

\noindent\textbf{Diffusion Efficiency Improvements:} 
\citet{das2023image} utilized the shortest path between two Gaussians and \citet{song2020denoising} generalized DDPMs via a class of non-Markovian diffusion processes to reduce the number of diffusion steps. \citet{nichol2021improved} introduced a few simple modifications to improve the log-likelihood. \citet{pandey2022diffusevae, pandey2021vaes} used DDPMs to refine VAE-generated samples. \citet{rombach2022high} performed the diffusion process in the lower dimensional latent space of an autoencoder to achieve high-resolution image synthesis, and \citet{liu2023audioldm} studied using such latent diffusion models for audio. \citet{popov2021grad} explored using a text encoder to extract better representations for continuous-time diffusion-based text-to-speech generation. More recently, \citet{nielsendiffenc} explored using a time-dependent image encoder to parameterize the mean of the diffusion process. Orthogonal to the above, PriorGrad \citep{lee2021priorgrad} and follow-up work \citep{koizumi22_interspeech} studied utilizing informative prior extracted from the conditioner data for improving learning efficiency. \textit{However, they become sub-optimal when the conditioner are degraded versions of the target data, posing challenges in applications like signal restoration tasks.}

\noindent\textbf{Diffusion-Based Signal Restoration:}
Built on top of the diffusion models for audio generation, e.g., \citet{kong2020diffwave,chen2020wavegrad,leng2022binauralgrad}, many SE models have been proposed. The pioneering work of \citet{lu2022conditional} introduced conditional DDPMs to the SE task and demonstrated the potential. Other works \citep{serra2022universal,welker2022speech,richter2023speech,yen2023cold,lemercier2023storm,tai2024dose} have also attempted to improve SE by exploiting diffusion models. In the vision domain, diffusion models have demonstrated impressive performance for IR tasks \citep{li2023diffusion,zhu2023denoising,huang2024wavedm,luo2023refusion,xia2023diffir,fei2023generative,hurault2022gradient,liu20232,chung2024direct,chungdiffusion,zhoudenoising,xiaodreamclean,zheng2024diffusion}. A notable IR work is \cite{ozdenizci2023restoring} that achieved impressive performance on several benchmark datasets for restoring vision in adverse weather conditions. \textit{Despite showing promising results, existing works have not fully exploited prior information about the data as they mostly settle on standard Gaussian priors.} 

\section{Architecture \& Design Overview}
\label{sec:Architecture_Overview}





\subsection{Fine-Grained Structured Sparsity}
\label{subsec:Fine_grained_structured_sparsity}
In this work, we leverage the regular patterns of fine-grained structured sparsity to architect SST slices with low area overhead. 
Fig. \ref{fig:random_sparsity} depicts a 50\% unstructured sparse matrix, where the non-zero data are distributed \textit{randomly}, \emph{i.e.,} there is no specific pattern of their locations.
In contrast, in Fig. \ref{fig:structured_sparsity}, the 2:4 structured sparsity pattern is illustrated, which has the same sparsity level (50\%), but in every group of four \textit{consecutive} elements
% row-wise, 
there are two non-zero values. 
Notice that the location of the two non-zero values can vary significantly within the four-element group, offering \textit{fine-grained} sparsity flexibility.  
These types of constraints enable low area hardware enhancements, allowing for efficient  exploitation of sparsity.


\begin{figure}[t]
\vspace{-0.70cm}
\centering
\subfloat[]
{\includegraphics[width=0.31\linewidth]{03_Architecture_Overview/unstructured_50_percent_sparsity.pdf}
\label{fig:random_sparsity}}
\subfloat[]{\includegraphics[width=0.685\linewidth]{03_Architecture_Overview/structured_2_4_sparsity.pdf}
\label{fig:structured_sparsity}} 

\vspace{-0.35cm}

\caption{50\% unstructured sparse matrix (a) and 2:4 (50\%) structured sparse matrix along with compressed format (b).} 
\label{fig:sparsity_struct_unstruct}
\vspace{-0.50cm}
\end{figure}


A 2:4 sparse matrix can be efficiently stored in  compressed format by saving only the non-zero values.
The location of each non-zero data is encoded using 2-bit indices, as shown in Fig. \ref{fig:structured_sparsity}.
Similar to 2:4, for 1:4 (75\%) sparsity one every four consecutive elements is non-zero, while for 1:3 (66.7\%) sparsity there is one non-zero every three consecutive elements.  
For all aforementioned patterns, 2-bit indices are required to encode the location of each non-zero element.
This is a very efficient compressed format, as we show in Sec. \ref{subsec:AIE_ML_comparison}, where comparison among other formats is performed.










\subsection{Systolic Sparse Tensor Slices Architecture}
\label{subsec:Sparse_Tensor_slices_architecture}

In this section, we present an
% high-level 
overview of the proposed SST slices. 
The core compute unit of the SSTs is a 4$\times$4 systolic array (SA) \cite{Kung_SA_1982}, as depicted in Fig. \ref{fig:ST_architecture}.
A 2D SA consists of homogeneous processing elements (PEs), where each PE performs a multiply--accumulate (MAC) operation and forwards the input operands to the neighboring PEs. 
This architecture allows maximization of data reuse
in GEMM, 
while also delivering high performance due to its regular and highly scalable design.
Hence, SAs have become a prime architecture in many DNN accelerators \cite{TPUV2_v3_2021, TPUv42021, S2TA_HPCA_2022, Vegeta_HPCA_2023, SA_CNN_FPGA_2017, SA_attention_FPGA_TECS_2023, Scale_sim_2020}.
In this work, we show that incorporating SST slices in FPGAs leads to high performance and scalable dense/sparse GEMM accelerators. 




\begin{figure}[tbp]
\vspace{-0.50cm}
\centering
\includegraphics[width=0.89\linewidth]{03_Architecture_Overview/SST_architecture.pdf}

\vspace{-0.50cm}

\caption{Systolic Sparse Tensor slice architecture.}
\label{fig:ST_architecture}

\vspace{-0.60cm}

\end{figure}


\subsubsection{SST Operation} 
We enhance the systolic PEs with sparse features by introducing sparse processing elements (SPEs), while maintaining the properties of the SAs discussed above. 
Our SSTs utilize an output stationary SA, consisting of 16 SPEs.
The accumulations remain stationary in the SPEs, while input operands are propagated to their neighbors every clock cycle.
The $a\_data$ of an input matrix $A$ are propagated and reused across SPEs horizontally, while the $b\_data$ of an input matrix $B$ are propagated and reused vertically (Fig. \ref{fig:ST_architecture}).
The matrix $A$ can be either sparse or dense (typically to map \textit{weights}), while $B$ is dense (typically to map input \textit{activations}).
The architecture of the SPEs is delineated in Sec. \ref{subsec:Sparse_Processing_Element}.


Besides the 4$\times$4 SPE grid, we also implement pipeline registers to delay the input operands for systolic data setup \cite{TPUv1_2017} (arranged in triangular manner in Fig. \ref{fig:ST_architecture}).
Systolic setup is needed at the interface for loading input matrices (typically from on-chip memory), when chaining multiple SSTs to construct larger SA grids (Sec. \ref{subsec:Matrix_multiplication_mapping}).
Multiplexers are used to select either the systolic setup
% registers
or directly the input data, and are configured \textit{statically} (during bitstream loading). 




\begin{figure*}[tbp]
\vspace{-0.40cm}
\centering
\includegraphics[width=1.00\textwidth]{03_Architecture_Overview/SPE_all_sparse_modes.pdf}

\vspace{-0.45cm}

\caption{Sparsity modes in Systolic Sparse Elements of the SST slices (multiplexing logic omitted for clarity).}
\label{fig:SPE_sparse_modes}
\vspace{-0.35cm}
\end{figure*}


The SSTs support both int8 and bfloat16 precisions, while accumulations are realized in 32-bit integer (int32) and IEEE 32-bit floating-point (fp32), respectively, similar to Nvidia GPUs \cite{Nvidia_accelerate_sparse_2021} and Google TPUs \cite{TPUV2_v3_2021}.
When the SA operation is completed, the output matrix $C$ is extracted via the $c\_data$ output ports.
We note that SPEs finish their operation in a \textit{diagonal} fashion cycle after cycle.
In the first cycle, $SPE00$ finishes, in the second cycle both $SPE01$ and $SPE10$ finish, etc. (Fig. \ref{fig:ST_architecture}). 
However, to maintain \textit{regularity}, thus simplifying the downstream logic (typically in CLBs), we extract the output values in a column-wise manner (four values per cycle).
To achieve that, we introduce a six-element buffer (consisting of registers), to store the data before getting extracted.
% column-wise.
This is particularly important in sustaining 100\% SPE utilization (16 MACs per cycle) at the steady-state (matrices processed one after the other).

Since SPEs complete their operation diagonally, the six upper-triangular SPE outputs, shown in Fig. \ref{fig:ST_architecture}, need to be stored in the buffer.
Suppose a cycle $T$ where the outputs of the biggest diagonal, \emph{i.e.,} $SPEs$ $\{03, 12, 21, 30\}$ are generated.
In cycle $T$, the values of the first column, \emph{i.e.,} $SPEs$ $\{00, 10, 20, 30\}$ can be extracted, where $SPEs$ $\{00, 10, 20\}$ are loaded from the buffer, while only $SPE30$ is directly extracted.
In cycle $T + 1$, the values of $SPEs$ $\{03, 12, 21\}$ replace the position of $SPEs$ $\{00, 10, 20\}$ in the buffer (since they have been extracted).
Therefore, the locations of the six-element buffer are being effectively reused, and in cycle $T+1$, the second column can be extracted, \emph{i.e.,} $SPEs$ $\{01, 11, 21, 31\}$.
In a similar fashion, the rest two columns are extracted in cycles $T+2$ and $T+3$, respectively, while the buffer locations are efficiently reused due to replacement. 








\subsubsection{Global Routing Interface \& Dedicated Wires}

As illustrated in Fig. \ref{fig:ST_architecture}, the $a\_data$ and $b\_data$ input ports as well as $c\_data$ output ports are connected to the global FPGA routing resources.
When chaining multiple SSTs, $a\_data$ and $b\_data$ are forwarded to their next in the chain SSTs, horizontally and vertically, respectively.
Horizontally, the data are propagated via the $a\_data\_out$ using the FPGA routing resources.
However, vertically, we utilize \textit{dedicated} wires to propagate the data (via the $b\_ded\_out$ ports) and connect them to the next SST in the \textit{same} FPGA column (via the $b\_ded\_in$ ports).
These vertical dedicated wires provide efficient connections without the usage of the global routing resources, matching the columnar nature of the modern FPGA fabric \cite{FPGA_architecture_2021, FPGA_for_DL_2024}.
This approach significantly reduces routing resources, as opposed to  \cite{TS_Aman_FPGA_2021, Aman_TS_TRETS_2022}, where all inputs/outputs (I/Os) of the in-fabric tensor slices are connected to global routing (see comparison in Sec. \ref{subsec:Dedicated_wires_benefits}).




Besides the I/Os shown in Fig. \ref{fig:ST_architecture}, the SSTs include also several dynamic control signals.
In particular, an input $enable$ signal is used to control the operation of the SSTs.
This signal should be deasserted when the operation of the SST needs to be stalled.
% in the case where the SST needs to stall its operation.
Moreover, an $accumulate$ signal is utilized to control the accumulation in the SPEs.
This signal remains asserted when accumulation is desirable in the SPEs, \emph{e.g.,} during tiling to process larger matrices.
However, for every new matrix operation the $accumulate$ signal should be deasserted for one cycle.
Another input signal is the  $d\_type$, which  dynamically selects the precision, \emph{i.e.,} int8 or bfloat16.
Finally, a 2-bit $sparsity\_level$ signal is employed to dynamically select the sparsity level of the matrix $A$, \emph{i.e.,} dense, 2:4, 1:3, 1:4.
This signal is utilized internally in the SSTs to control the operation of the supported sparsity modes, as discussed in the next section.

Regarding the outputs of the SST, an $accumulate\_out$ signal is used for systolic distribution of the $accumulate$ signal when chaining multiple SSTs.
This systolic distribution makes the control logic significantly simpler, since the $accumulate$ is set only for the first SST in the 2D array layout, while being distributed to the remaining SSTs, as explained in Sec. \ref{subsec:Matrix_multiplication_mapping}.
Another output signal is the $valid\_out$, which is asserted when the output $c\_data$ are extracted column-wise (Fig. \ref{fig:ST_architecture}). 
This signal simplifies the downstream logic, since only $valid\_out$ needs to be checked for valid output data.





\subsection{Sparse Processing Element}
\label{subsec:Sparse_Processing_Element}

The SPE architecture in all supported sparsity modes is illustrated in Fig. \ref{fig:SPE_sparse_modes}.
When the SPE is configured in dense mode (Fig. \ref{fig:SPE_sparse_modes}a), it operates as a regular dense PE.
In particular, one MAC operation is performed every cycle, while the elements of matrices $A$ and $B$ are forwarded to their neighboring SPEs (via pipeline registers) horizontally and vertically, respectively.
The output value of matrix $C$ in each SPE is calculated after $K$ cycles, when considering the $M$$\times$$K$$\times$$N$ matrix dimensions depicted in Fig. \ref{fig:SPE_sparse_modes}a.



In 2:4 mode (50\% sparsity), matrix $A$ is stored in a compressed format of $M$$\times$$K/2$ size for both values and the 2-bit indices, as shown in Fig. \ref{fig:SPE_sparse_modes}b.
In order to achieve speedup over the dense case, we increase the number of ports of each SPE in the vertical dimension, to load four elements of the matrix $B$ in parallel.
Furthermore, the 2-bit index is also loaded in the SPE to select the corresponding $B$ values (via a 4:1 multiplexer), which need to get multiplied with each $A$ value.
However, in 2:4 sparsity, for every four values of the matrix $B$, two MAC operations need to be performed.
Hence, the four $B$ values remain in the SPE registers for two clock cycles (the four green registers in Fig. \ref{fig:SPE_sparse_modes}b are loaded every two cycles).
To correctly synchronize the systolic operation, two pipeline stages are required for both the $A$ values and their indices (blue and yellow registers in Fig. \ref{fig:SPE_sparse_modes}b), which are loaded every cycle. 
In this manner, one MAC operation is performed every cycle, ensuring 100\% utilization.
Moreover, the output value of matrix $C$ in each SPE is calculated in $K/2$ cycles, achieving 2$\times$ acceleration over dense operation.




\begin{table}[t]
\centering
\caption{Summary of supported sparsity levels in SST slices.}

\setlength\tabcolsep{9pt}
\renewcommand{\arraystretch}{1.0}
\resizebox{0.80\linewidth}{!}{
% \vspace{-0.10cm}
\begin{tabular}{c|cc|c|c}
\Xhline{2.5\arrayrulewidth}

\textbf{Sparsity}  &  
\multicolumn{2}{c|}{\textbf{Compres. ratio}} & \textbf{Speedup} & \textbf{SPE} \\

\cline{2-3}

\textbf{level}  & \textbf{int8} & \textbf{bfloat16} 
 & \textbf{over dense} & \textbf{util.} \\

\hline
\hline

\textbf{Dense (0\%)} & 1$\times$ & 1$\times$ & 1$\times$ & 100\% \\ 

\textbf{2:4 (50\%)} & 1.6$\times$ & 1.78$\times$ & 2$\times$ & 100\% \\ 

\textbf{1:3 (66.7\%)} & 2.4$\times$ & 2.67$\times$ & 3$\times$ & 100\% \\ 

\textbf{1:4 (75\%)} & 3.2$\times$ & 3.56$\times$ & 4$\times$ & 100\% \\ 


\Xhline{2.5\arrayrulewidth}

\end{tabular}
}

\label{tb:sparsity_summary_benefits}

\vspace{-0.50cm}

\end{table}



\begin{figure*}[ht]

\vspace{-0.55cm}

\centering
\includegraphics[width=0.81\textwidth]{03_Architecture_Overview/GEMM_2D_design.pdf}

\vspace{-0.35cm}

\caption{2D systolic GEMM design: dense implementation (a) and dynamic configuration of all supported sparsity modes (b).}
\label{fig:GEMM_2D_array_SSTs}

\vspace{-0.45cm}

\end{figure*}


Similar to 2:4, for 1:4 (75\%) sparsity, four $B$ values are loaded in the SPE.
However, in this case the $B$ values are loaded every cycle, while only one pipeline stage is required for the $A$ value and its index, as shown in Fig. \ref{fig:SPE_sparse_modes}c.
The output value needs $K/4$ cycles to be calculated, offering 4$\times$ speedup over the dense case.
To fill the sparsity gap between 50\% and 75\% as explained in Sec. \ref{sec:Introduction}, we leverage the same hardware enhancements for 2:4 and 1:4 sparsity to also support the 1:3 (66.7\%) pattern.
In particular, as illustrated in Fig. \ref{fig:SPE_sparse_modes}d, the 1:3 operation is similar to 1:4, with the main difference being that three $B$ values are loaded every cycle instead of four.
The indices ensure that the fourth $B$ value is never selected, thus no additional hardware is required.
In this case, $K/3$ cycles are needed for every SPE output value, providing 3$\times$ acceleration.
Finally, we note that similar to 1:3, 2:3 sparsity can also be supported by directly utilizing the 2:4 mode, leading to 33.3\% sparsity.
However, a matrix is typically considered sparse when it has sparsity of 50\% or higher \cite{AMD_AIE_ML_kernel_guide}.
Hence, in this work, we do not consider the 2:3 pattern. 







In Table \ref{tb:sparsity_summary_benefits}, we present a summary of the supported sparsity levels.
First, we show the \textit{compression ratio}, \emph{i.e.,} the memory reduction over dense storage due to compressed format, for both int8 and bfloat16 precisions.
For all sparsity levels, bfloat16 offers a higher compression ratio over int8, \emph{e.g.,} 1.78$\times$ \emph{vs.} 1.6$\times$ for 2:4 sparsity. 
This is because 2-bit indices are required for both 8-bit and 16-bit data types, resulting in relatively lower overhead for the compressed representation in bfloat16 compared to int8.
This compressed format substantially reduces both on-chip and off-chip memory requirements, achieving up to 3.56$\times$ reduction (Table \ref{tb:sparsity_summary_benefits}).


Second, we notice that every sparsity level is
% effectively
translated to its corresponding speedup, \emph{e.g.,} 4$\times$ for 1:4 (75\%) sparsity, achieving 100\% SPE utilization in all cases (Table \ref{tb:sparsity_summary_benefits}).
For all sparsity levels, data reuse is maximized since both indices and values are propagated and reused horizontally and vertically, similar to dense operation.






\subsection{GEMM Design Utilizing Multiple SST Slices}
\label{subsec:Matrix_multiplication_mapping}


In this section, we describe parametric GEMM implementations for both dense and sparse configurations, utilizing multiple SST slices.
Both implementations are highly regular and scale effectively on the FPGA fabric, attaining high frequencies as shown in Sec. \ref{subsec:Sparse_GEMM_implementation}.




\subsubsection{Dense Implementation}

Fig. \ref{fig:GEMM_2D_array_SSTs}a depicts a parametric GEMM accelerator comprising a 2D array of SST slices, which are configured in \textit{dense} mode (Sec. \ref{subsec:Sparse_Processing_Element}). 
The 2D array consists of $Y \cdot X$ SST slices, implementing a total SA size of $(Y \cdot 4) \times (X \cdot 4)$.
This size is denoted as the \textit{native} size of the GEMM accelerator. 
On-chip memory buffers are implemented to store the input matrices $A$, $B$ and the output matrix $C$.
The input buffers $A$, $B$ are located in the left and top edges of the 2D array, respectively, and are partitioned into banks, providing sufficient bandwidth to feed the SSTs.
In particular, for buffer $A$, $Y$ banks are required, while for buffer $B$, $X$ banks are needed.
Since each SST slice includes a 4$\times$4 SA, each bank needs to provide a bandwidth of 32-bits per cycle when SSTs are configured for int8 precision, while for bfloat16, 64-bits per cycle are required.
Regarding the output buffer $C$, $X \cdot Y$ banks are needed due to the output stationary architecture of the SSTs, each receiving an output of 128-bits per cycle (four 32-bit values as explained in Sec. \ref{subsec:Sparse_Tensor_slices_architecture}).
 



The data from the buffers $A$, $B$ are propagated in a systolic fashion between the SST slices, as illustrated in Fig. \ref{fig:GEMM_2D_array_SSTs}a.
Horizontally, the $A$ data are propagated via the global routing resources of the FPGA fabric.
Vertically, the $B$ data are inserted via global routing wires in the first SST at each vertical chain ($Y$ SSTs in Fig. \ref{fig:GEMM_2D_array_SSTs}a), while being forwarded to the next SSTs via dedicated wires.
It is important to note here that the SSTs comprising each vertical chain are \textit{physically} contiguous in the FPGA.
Nevertheless, in the horizontal dimension, the $Y$ chains might not follow the \textit{logical} arrangement shown in Fig. \ref{fig:GEMM_2D_array_SSTs}a inside the FPGA fabric, due to the routing flexibility of FPGAs.
This depends on decisions made by the FPGA place and route (PnR) algorithm.
Finally, notice the (static) systolic data setup configuration specifically for the SST slices that interface with the buffers $A$, $B$ (left and top edge of the 2D array).


Control logic, mapped to the CLB resources of the FPGA, is utilized to orchestrate the entire operation of the GEMM design.
We also implement tiling logic (in CLBs) to exploit data reuse in GEMM, as well as to support arbitrary GEMM sizes based on the available on-chip memory resources.
When mapping an arbitrary GEMM of $M^\prime$$\times$$K^\prime$$\times$$N^\prime$ dimensions, $M^\prime$ must be a multiple of $(Y \cdot 4)$, while $N^\prime$ must be a multiple of $(X \cdot 4)$, since the \textit{native} size of the accelerator is $(Y \cdot 4) \times (X \cdot 4)$.
Note that there is no constraint on the reduction $K^\prime$ dimension.
Finally, although not shown in Fig. \ref{fig:GEMM_2D_array_SSTs}a, $accumulate$ signals utilized during tiling (Sec. \ref{subsec:Sparse_Tensor_slices_architecture}) are propagated in a systolic fashion among the 2D array of SSTs, similar to the $A$, $B$ data.
The control logic sets the $accumulate$ signal only for the first SST in both vertical and horizontal dimensions (\emph{i.e.,} the SST fed by buffers $A_{1}$ and $B_{1}$), which significantly simplifies the overall logic.






\subsubsection{Dynamic Sparse Configuration}

Fig. \ref{fig:GEMM_2D_array_SSTs}b illustrates a parametric GEMM design that is \textit{dynamically} configured to support all the sparsity modes in the SSTs, \emph{i.e.,} dense, 2:4, 1:3 and 1:4.
This dynamic configuration is particularly important for layer-wise sparsity exploitation in DNNs, since each layer might require different sparsity level for optimal trade-off between DNN accuracy and speedup (Sec. \ref{subsec:Performance_estimation_DNNs}).
We note that the implementation is similar to the dense design (Fig. \ref{fig:GEMM_2D_array_SSTs}a), with main differences lying in the design of buffers $A$ and $B$, as well as in the vertical and horizontal propagation of the data.
Horizontally, the $A$ data are kept in the buffer banks in compressed format for the sparse modes (2:4, 1:3 and 1:4).
In this case, both non-zero data and indices are loaded in the SSTs and are propagated horizontally (see Sec. \ref{subsec:Sparse_Processing_Element}).
More specifically, for int8, each buffer $A$ bank needs to provide a bandwidth of 40-bits per cycle, due to the additional 8-bits indices for the four vertical SPEs at the interface of each SST.
Similarly, for bfloat16, 72-bits per cycle are required.


Vertically, four $B$ banks are needed to feed the SSTs due to the 4$\times$ increase in ports for 2:4 and 1:4 sparsity acceleration (Sec. \ref{subsec:Sparse_Processing_Element}).
Each $B$ bank provides the same bandwidth as the dense design (Fig. \ref{fig:GEMM_2D_array_SSTs}a), \emph{i.e.,} 32-bits and 64-bits per cycle for int8 and bfloat16, respectively.
Similar to the dense design, the $B$ data are inserted in the first SST at each vertical chain and dedicated wires are used to propagate them vertically (Fig. \ref{fig:GEMM_2D_array_SSTs}b).
These dedicated wires are particularly important for the sparse design, since otherwise 4$\times$ more vertical wires would be required to use global routing compared to the dense design (in the case where \textit{only} non-dedicated wires are employed).




Besides sparse operation, the design in Fig. \ref{fig:GEMM_2D_array_SSTs}b also supports dense computation.
This is because dense computation might still be needed, even if all weights in DNNs are sparse.
For instance, in Transformer-based DNNs \cite{Attention_all_you_need_2017, BERT_2019, ViT_2020}, the QKV (Query, Key, Value) 
% self-attention 
GEMMs do not involve weights, and are typically computed as dense.
For dense computation, only one $B$ bank is sufficient at each vertical chain, \emph{e.g.,} $B_{x,0}$ (Fig. \ref{fig:GEMM_2D_array_SSTs}b).
However, multiplexing logic can be employed to 
% effectively 
utilize the remaining $B_{x,1}$, $B_{x,2}$, $B_{x,3}$ banks.
This is particularly important to ensure efficient utilization of on-chip memory resources, which leads to maximized data reuse and thus optimized energy efficiency \cite{Versal_vs_Stratix_FCCM_2024, MaxEVA_2023}.
Finally, for 1:3 sparsity, only banks $B_{x,0}$, $B_{x,1}$, $B_{x,2}$ are required, leaving bank $B_{x,3}$ unused. 
However, similar to the dense operation, 
% additional 
multiplexing logic can be employed to reuse this bank when it comprises multiple BRAMs, which we do not explore in this work (see Sec. \ref{subsec:Sparse_GEMM_implementation} for implementation details). 



The dynamic configuration among all sparsity levels 
% in Fig. \ref{fig:GEMM_2D_array_SSTs}b
is implemented in the control logic (using CLBs).
However, FPGA accelerators can be designed in a custom fashion depending on the sparsity of each DNN.
% required in DNN.
For instance, a specific DNN might require only 1:3 sparsity and dense computation across all of its layers.
The control and tiling logic for sparsity is similar to the dense design (Fig. \ref{fig:GEMM_2D_array_SSTs}a), showcasing a marginal increase in CLB resources (see Sec. \ref{subsec:Sparse_GEMM_implementation}).






\section{Evaluation}
\label{sec:Evaluation}
\section{Evaluation}

% Our proposed framework was compared with Apollo \cite{b7Apollo1, b7Apollo2}, which demonstrates that it can model analytic operators using data content. Two loss functions were utilized, the root-mean-square deviation error (RMSE), and the mean absolute error (MAE). The selection of these two loss functions is because they fulfil the disadvantages of each other, while RMSE is sensitive to outlier MAE is not and the MAE cannot take into account the direction of the error while the RMSE can achieve it. Speedup was computed to determine how quickly our framework can model the operator $\Phi$. We utilised the \textit{Speedup} and \textit{Amortized Speedup}, which assesses the require time to approximate each operator in comparison to exhaustively executing them on all datasets (more is better). Particularly, the speedup is equalled $\frac{T{^{(i)}_{op}}}{T{^{(i)}_{SimOp} + T_{vec} + T_{sim} + T_{pred}}}$, where $T{^{(i)}_{op}}$ is the execution time for operator $i$, across all the datasets, $T{^{(i)}_{SimOp}}$ is the time needed to model the operator with the datasets selected from the similarity search, $T_{vec}$  is the time needed to compute the vector embedding for each dataset, $T_{sim}$, is the time needed to perform similarity search, and $T_{pred}$ is the time needed to predict on the dataset $D_o$. In addition to the dataset vectorisation, which is done once for each data lake, we calculate amortised speedup. Furthermore, an experimental evaluation of our proposed model for dataset vectorization NumTabData2Vec has been performed to show that our approach can transform a dataset to a vector embedding representation space $z$. For the evaluation experiments, three different NumTabData2Vec were built to project the dataset representation with vector sizes of $100$, $200$, and $300$. Each model has eight transformer layers and is trained parallel using four NVIDIA A10s GPUs, and trained for fifty epochs.
We compared our framework with Apollo \cite{b7Apollo1, b7Apollo2}, which models analytic operators using data content. Two loss functions to measure prediction accuracy are employed: root-mean-square error (RMSE) and mean absolute error (MAE). RMSE is sensitive to outliers, while MAE is not; conversely, RMSE accounts for error direction, which MAE cannot. Speedup metrics are also used to evaluate how efficiently our framework models operator $\Phi$. Specifically, \textit{Speedup} and \textit{Amortized Speedup} measure the time required to approximate each operator versus exhaustively executing them on all datasets. Speedup is defined as $\frac{T{^{(i)}_{op}}}{T{^{(i)}_{SimOp} + T_{vec} + T_{sim} + T_{pred}}}$, where $T{^{(i)}_{op}}$ is the time to execute operator $i$ on all datasets, $T{^{(i)}_{SimOp}}$ is the time to model the operator with datasets from similarity search, $T_{vec}$ is the vector embedding computation time, $T_{sim}$ is the similarity search time, and $T_{pred}$ is the prediction time for $D_o$. Amortized speedup includes dataset vectorization, performed once per data lake for multiple operators (in our case two operators).
We also evaluate our dataset vectorization model, NumTabData2Vec, which projects datasets into vector embedding space $z$. Three versions were built with vector sizes of $100$, $200$, and $300$, each featuring eight transformer layers. The models were trained for 50 epochs on four NVIDIA A10 GPUs in parallel.

\subsection{Evaluation Setup}
Our framework is deployed over an AWS EC2 virtual machine server running with 48 VCPUs of AMD EPYC 7R32 processors at 2.40GHz, and four A10s GPUs with 24GB of memory each, $192GB$ of RAM memory, and $2TB$ of storage, running over Ubuntu 24.4 LTS. Our code is written in Python (v.3.9.1) and PyTorch modules (v.2.4.0). Apollo was deployed in a virtual machine with 8 VCPUs Intel Xeon E5-2630 @ 2.30GHz, $64GB$ of RAM memory, and $250GB$ of storage, running Ubuntu 24.4 LTS like in their experimental evaluation. 

\subsection{Datasets}
\begin{table}[!ht]
    \centering
    \setlength\doublerulesep{0.5pt}
    \caption{Dataset properties for experimental evaluation}
    \label{tab:table-evaluation-datasets}
    \begin{tabular}{||c|c|c|c||}
        \hline
         \makecell{Dataset Name}& \makecell{\# Files} & \makecell{\# Tuples} & \makecell{\# Columns}\\ \hline\hline
         Household Power & & & \\
         Consumption \cite{b21HPCdataset} & $401$ & $2051$ & 7\\
         \hline
         Adult \cite{b22AdultDataset} & $100$ & $228$ & 14\\
         \hline
         Stocks \cite{b23StockMarketDataset} & $508$ & $1959 - 13$ & 7 \\
         \hline
         Weather \cite{b23WeatherDataset} & $49$ & $516$ & 7 \\ \hline
    \end{tabular}

\end{table}

We evaluated our framework using four diverse datasets to represent real-world scenarios. Table \ref{tab:table-evaluation-datasets} summarizes these datasets and their attributes. The vectorization module, NumTabData2Vec, was trained on data separate from the experimental evaluation data, split $60\%$ for training and $40\%$ for testing.
The Household Power Consumption (HPC) dataset \cite{b21HPCdataset} contains electric power usage measurements from a household in Sceaux, France. It includes $401$ datasets, each with $2051$ tuples and seven features recorded at one-minute intervals. The Adult dataset \cite{b22AdultDataset}, commonly used for binary classification, predicts whether an individual earns more or less than $50K$ annually. It comprises $100$ datasets, each with $228$ individuals and various socio-economic features.
The Stock Market dataset \cite{b23StockMarketDataset} includes daily NASDAQ stock prices obtained from Yahoo Finance, with $508$ datasets. Each dataset contains $13$ to $1959$ tuples, each describing seven feature attributes. The Weather dataset \cite{b23WeatherDataset} provides hourly weather measurements from $36$ U.S. cities between $2012$ and $2017$, split into $49$ datasets, each with $516$ tuples and seven features.


\begin{figure}[!t]
     \centering
     \begin{subfigure}[b]{0.24\textwidth}
         \centering
         \includegraphics[width=\textwidth]{Figures/Results/Sim_Search/HPC/HPC_LR_RMSE_Loss_fig.pdf}
         \caption{Linear Regression RMSE error loss}
         \label{fig:HPC-LR-RMSE}
     \end{subfigure}
     \hfill 
     \begin{subfigure}[b]{0.24\textwidth}
         \centering
         \includegraphics[width=\textwidth]{Figures/Results/Sim_Search/HPC/HPC_LR_MAE_Loss_fig.pdf}
         \caption{Linear Regression MAE error loss}
         \label{fig:HPC-LR-MAE}
     \end{subfigure}
        
     \begin{subfigure}[b]{0.24\textwidth}
         \centering
         \includegraphics[width=\textwidth]{Figures/Results/Sim_Search/HPC/HPC_MLP_RMSE_Loss_fig.pdf}
         \caption{MLP for Regression RMSE error loss}
         \label{fig:HPC-MLP-RMSE}
     \end{subfigure}
     \hfill 
     \begin{subfigure}[b]{0.24\textwidth}
         \centering
         \includegraphics[width=\textwidth]{Figures/Results/Sim_Search/HPC/HPC_MLP_MAE_Loss_fig.pdf}
         \caption{MLP for Regression MAE error loss}
         \label{fig:HPC-MLP-MAE}
     \end{subfigure}
        \caption{Household power consumption dataset prediction error loss}
        \label{fig:HPC-EVAL-RES}
\end{figure}

Our framework was evaluated by registering the accuracy of predicting the output of various ML operators over multiple datasets in $D$ without actually executing the operator on them. To evaluate our scheme and its parameters, we use all four datasets, ranging the size of the produced vectors as well as the similarity functions used.
We project all datasets into $k$-dimensional spaces with varying vector dimensions ($100$, $200$, and $300$). For each dataset in Table \ref{tab:table-evaluation-datasets}, we model different operators: For the regression datasets (Household Power Consumption and Stock Market), we model Linear Regression (LR) and Multi-Layer Perceptron (MLP) operators; for the classification datasets (Weather and Adult), we model the Support Vector Machine (SVM) and MLP classifier operators. Each experiment has been executed $10$ times and we report the average of the error loss, as well as the speedup. 

\begin{figure}[!t]
     \centering
     \begin{subfigure}[b]{0.24\textwidth}
         \centering
         \includegraphics[width=\textwidth]{Figures/Results/Sim_Search/Stocks/Stocks_LR_RMSE_Loss_fig.pdf}
         \caption{Linear Regression RMSE error loss}
         \label{fig:Stock-LR-RMSE}
     \end{subfigure}
     \hfill 
     \begin{subfigure}[b]{0.24\textwidth}
         \centering
         \includegraphics[width=\textwidth]{Figures/Results/Sim_Search/Stocks/Stocks_LR_MAE_Loss_fig.pdf}
         \caption{Linear Regression MAE error loss}
         \label{fig:Stock-LR-MAE}
     \end{subfigure}
        
     \begin{subfigure}[b]{0.24\textwidth}
         \centering
         \includegraphics[width=\textwidth]{Figures/Results/Sim_Search/Stocks/Stocks_MLP_RMSE_Loss_fig.pdf}
         \caption{MLP for Regression RMSE error loss}
         \label{fig:Stock-MLP-RMSE}
     \end{subfigure}
     \hfill 
     \begin{subfigure}[b]{0.24\textwidth}
         \centering
         \includegraphics[width=\textwidth]{Figures/Results/Sim_Search/Stocks/Stocks_MLP_MAE_Loss_fig.pdf}
         \caption{MLP for Regression MAE error loss}
         \label{fig:Stock-MLP-MAE}
     \end{subfigure}
        \caption{Stock market dataset prediction error loss}
        \label{fig:Stock-EVAL-RES}
\end{figure}
\subsection{Evaluation Results}



Figures \ref{fig:HPC-EVAL-RES}, \ref{fig:Stock-EVAL-RES}, \ref{fig:Weather-EVAL-RES}, and \ref{fig:Adult-EVAL-RES} present the evaluation results for each method, comparing the performance of different similarity search techniques across various vector embedding representation spaces. The red (with hatches), brown, and blue bars correspond to vector embeddings of size 100, 200, and 300 respectively. In each sub-figure, the y-axis represents the error loss value, while the x-axis displays the similarity search method applied over the vector embeddings. Figures \ref{fig:HPC-EVAL-RES} and \ref{fig:Stock-EVAL-RES} show the results for the Stock market and Household power consumption datasets, where the bottom sub-figure demonstrates the MLP regression model, and the top sub-figure presents the LR model. Figures \ref{fig:Weather-EVAL-RES} and \ref{fig:Adult-EVAL-RES} depict the evaluation results for the Weather and Adult datasets. In these Figures, the top sub-figure shows the SVM with SGD results, while the bottom sub-figure shows the MLP classifier. The left sub-figures in all Figures use the RMSE loss function, whereas the right sub-figures use the MAE loss function. 



Figure \ref{fig:HPC-EVAL-RES}, we show, for the HPC dataset, shows as increase the vector dimension size there is slightly lower prediction error for all the operator modelling. While for different similarity methods did not result in any significant differences in the prediction error loss for all the operator modelling. This suggests that, regardless the similarity selection method, our framework effectively selects the most optimal subset of data to improve model predictions on the unseen input dataset $D_o$. Additionally, we observe higher error loss with a vector size of 100, which can be attributed to the reduced representation capacity of lower-dimensional vectors. This limitation results in fewer ``right" datasets being selected.

For the stock market dataset, Figure \ref{fig:Stock-EVAL-RES} depicts that a vector embedding representation of size $300$ models more accurate operators, with cosine similarity performing best in the similarity search and modelling the most optimal operator. However, due to the inherent volatility in Stock market data from different days, all models in the stock market dataset experiments exhibit high loss values. 

In the weather dataset, the SVM operator results from sub-figures \ref{fig:Weather-SVM-RMSE} and \ref{fig:Weather-SVM-MAE} show that using $300$ vectors in the representation space consistently led to more accurate operator models across all similarity methods. Specifically, cosine similarity in combination with the $300$-dimensional vector embedding reduced the error rate in operator predictions, demonstrating that projecting datasets into this representation space and applying cosine similarity improves the prediction accuracy on the modelled operator. For the MLP classifier from sub-figures \ref{fig:Weather-MLP-RMSE} and \ref{fig:Weather-MLP-MAE}, the results illustrate that using vector embeddings of size $200$ and K-Means clustering produced the most accurate MLP classifier operators.

% Overall, we observe that the error loss was minimized 
% (** what do you mean, minimized? In general, here you should comment on the effect of similarity function, the effect of vector size and the effect of different operators to the accuracy of prediction. E.g., in Household dataset shows little effect in all bars, but in Stock, the cosine seems better and larger size of vectors leads to better performance etc. **)
% in most cases, indicating that our framework effectively selects the most relevant datasets from the data lake $D$, thereby improving data quality and reducing $\Phi$ prediction errors on the target dataset $D_o$. This demonstrates that the datasets are accurately transformed into the vector embedding representation space, allowing for the selection of datasets most similar to $D_o$. 

%Adult


%Weather
\begin{figure}[t!]
     \centering
     \begin{subfigure}[b]{0.24\textwidth}
         \centering
         \includegraphics[width=\textwidth]{Figures/Results/Sim_Search/Weather/Weather_SVM_RMSE_Loss_fig.pdf}
         \caption{SVM with SGD RMSE error loss}
         \label{fig:Weather-SVM-RMSE}
     \end{subfigure}
     \hfill 
     \begin{subfigure}[b]{0.24\textwidth}
         \centering
         \includegraphics[width=\textwidth]{Figures/Results/Sim_Search/Weather/Weather_SVM_MAE_Loss_fig.pdf}
         \caption{SVM with SGD MAE error loss}
         \label{fig:Weather-SVM-MAE}
     \end{subfigure}
        
     \begin{subfigure}[b]{0.24\textwidth}
         \centering
         \includegraphics[width=\textwidth]{Figures/Results/Sim_Search/Weather/Weather_MLP_RMSE_Loss_fig.pdf}
         \caption{MLP RMSE error loss}
         \label{fig:Weather-MLP-RMSE}
     \end{subfigure}
     \hfill 
     \begin{subfigure}[b]{0.24\textwidth}
         \centering
         \includegraphics[width=\textwidth]{Figures/Results/Sim_Search/Weather/Weather_MLP_MAE_Loss_fig.pdf}
         \caption{MLP MAE error loss}
         \label{fig:Weather-MLP-MAE}
     \end{subfigure}
        \caption{Weather dataset prediction error loss}
        \label{fig:Weather-EVAL-RES}
\end{figure}

On the other hand, the Adult dataset shows the lowest error rates, with error loss values consistently below $0.5$ across all vector embedding dimensions and similarity search methods (see Figure \ref{fig:Adult-EVAL-RES}). The Adult dataset, besides exhibiting a high number of rows, also has a higher number of columns, which demonstrates that our framework performs consistently well even with larger datasets.
Additionally, we observe that the lowest prediction error across all datasets occurs when using higher-dimensional vector embeddings. With a trade-off between accuracy and execution time as the difference to generate all data lake available datasets vector embedding representation between $100$ and $300$ size dimension in the vector representation space to be less than $60$ seconds. This confirms that a higher number of vector dimensions leads to more accurate predictions, consistent with findings in previous research \cite{b8Word2Vec}.


\begin{figure}[!t]
     \centering
     \begin{subfigure}[b]{0.24\textwidth}
         \centering
         \includegraphics[width=\textwidth]{Figures/Results/Sim_Search/Adult/Adult_MLP_RMSE_Loss_fig.pdf}
         \caption{SVM with SGD RMSE error loss}
         \label{fig:Adult-LR-RMSE}
     \end{subfigure}
     \hfill 
     \begin{subfigure}[b]{0.24\textwidth}
         \centering
         \includegraphics[width=\textwidth]{Figures/Results/Sim_Search/Adult/Adult_MLP_RMSE_Loss_fig.pdf}
         \caption{SVM with SGD MAE error loss}
         \label{fig:Adult-LR-MAE}
     \end{subfigure}
     
     \begin{subfigure}[b]{0.24\textwidth}
         \centering
         \includegraphics[width=\textwidth]{Figures/Results/Sim_Search/Adult/Adult_MLP_RMSE_Loss_fig.pdf}
         \caption{MLP RMSE error loss}
         \label{fig:Adult-MLP-RMSE}
     \end{subfigure}
     \hfill 
     \begin{subfigure}[b]{0.24\textwidth}
         \centering
         \includegraphics[width=\textwidth]{Figures/Results/Sim_Search/Adult/Adult_MLP_MAE_Loss_fig.pdf}
         \caption{MLP MAE error loss}
         \label{fig:Adult-MLP-MAE}
     \end{subfigure}
        \caption{Adult dataset prediction error loss}
        \label{fig:Adult-EVAL-RES}
\end{figure}





We conducted an experimental evaluation using the Sampling Ratio (SR) approach, similar to Apollo \cite{b7Apollo1}, but employed neural networks built from the vector embeddings of each dataset. The SR approach involves a unified random selection of $l\%$ datasets from the vector representation space, using this subset to construct a neural network for predicting operator outputs. We tested SR values of $0.1$, $0.2$, and $0.4$, as well as vector embedding dimensions of $100$, $200$, and $300$, across all datasets. 
Figure \ref{fig:SR-EVAL-RES} presents the sampling ratio results for the Adult dataset using MLP (sub-figure \ref{fig:Adult-SR-RMSE}) and for the Weather dataset using LR (sub-figure \ref{fig:Weather-SR-SVM-MAE}). In each sub-figure the y-axis represents the RMSE prediction error loss while the x-axis denotes the vector dimension



\begin{figure}[htpb!]
     \centering
     \begin{subfigure}[b]{0.24\textwidth}
         \centering
         \includegraphics[width=\textwidth]{Figures/Results/SR/Adult/Adult_MLP_SR_RMSE_Loss_fig.pdf}
         \caption{Adult Dataset MLP Operator RMSE error loss}
         \label{fig:Adult-SR-RMSE}
     \end{subfigure}
     \hfill 
     \begin{subfigure}[b]{0.24\textwidth}
         \centering
         \includegraphics[width=\textwidth]{Figures/Results/SR/HPC/HPC_LR_SR_RMSE_Loss_fig.pdf}
         \caption{HPC dataset LR Operator RMSE error loss}
         \label{fig:Weather-SR-SVM-MAE}
     \end{subfigure}
     \caption{Sampling Ratio prediction results}
        \label{fig:SR-EVAL-RES}
\end{figure}

Both experiments demonstrate that as the vector embedding dimension increases, coupled with a larger sampling ratio (SR) value, there is a slight decrease in the prediction error loss. This improvement occurs because higher-dimensional vector embeddings provide a more accurate representation of the datasets in k-dimensions, with better dataset selection leading to enhanced prediction accuracy. Comparing the SR approach to our similarity search method for the HPC dataset, the SR approach was approximately $15\%$ less accurate in operator prediction across all vector embedding dimensions. A similar trend was observed in the Weather dataset. However, the Stock dataset exhibited a much larger discrepancy, with the SR approach performing about $70\%$ worse in prediction accuracy across all vector embedding dimensions. Likewise, in the Adult dataset, the SR approach delivered the poorest performance, with nearly $90\%$ lower prediction accuracy compared to the similarity search methods.

\begin{table*}[htbp]
    \centering
        \caption{Evaluation results of our framework exported analytic operator with lowest prediction error in comparison with Apollo}
    \label{tab:table-eval-res}
    % \scalebox{0.8}{
    \setlength\doublerulesep{0.5pt}
    % \begin{adjustbox}{width=\linewidth,center}
    \begin{tabular}{|c|c|c|c|c|c|c|}
    \hline
         \makecell{Dataset\\Name} & Method & Operator & RMSE &  MAE & Speedup  & Amortized Speedup \\
         \hline\hline
         \multirow{7}{*}{\makecell{Household\\Power\\Consumption}}& \makecell{$300$V Cosine} & LR & $\mathbf{6.61}$ & $\mathbf{5.42}$ & $0.0017$ & $\mathbf{1.99}$ \\ \cline{2-7}
                  & \makecell{$300$V SR-$0.2$} & LR & $7.77$ & $6.66$ &  $0.0018$  & $1.42$\\ \cline{2-7} 
        & \makecell{Apollo-SR $0.1$} & LR & $2968.01$ &  $2352.55$ & $\mathbf{0.015}$ & $0.024$ \\ \cline{2-7}
         & \makecell{Apollo-SR $0.2$} & LR & $2811.49$ &  $2229.50$ & $0.015$ & $0.024$ \\ \cline{2-7}\cline{2-7}
         & \makecell{$300$V K-Means} & MLP Regr. & $\mathbf{6.70}$ & $\mathbf{3.38}$ &  $0.9249$  & $\mathbf{1.99}$\\ \cline{2-7}
         & \makecell{Apollo-SR $0.1$} & MLP Regr. & $3322.05$ &  $2606.99$ & $2.38$ & $1.74$ \\ \cline{2-7}
         & \makecell{Apollo-SR $0.2$} & MLP Regr. & $3850.01$ &  $2609.36$ & $\mathbf{2.38}$ & $1.74$\\ \cline{1-7} \cline{1-7} 
         % Stock
         % \multirow{5}{*}{\makecell{Stock}}& \multirow{1}{*}{ \makecell{$100$V Euclidean}} & LR & $229388.93$ & $193066.03$ \\ \cline{2-5}
        \multirow{7}{*}{\makecell{Stock}} &  \makecell{$300$V Cosine} & LR & $306382.28$ & $125335.65$ & $0.00085$ & $\mathbf{1.91}$\\ \cline{2-7}
        & \makecell{$300$V SR-$0.4$} & LR & $21861625.91$ & $5674215.265$ &  $0.00087$  & $0.33$\\ \cline{2-7}
        & \makecell{Apollo-SR $0.1$} & LR & $\mathbf{153665.92}$ &  $\mathbf{118236.48}$ & $\mathbf{0.00093}$ & $0.00096$\\ \cline{2-7}
         & \makecell{Apollo-SR $0.2$} & LR & $166844.95$ &  $133306.68$ & $0.00093$ & $0.00096$\\ \cline{2-7}\cline{2-7}
         &  \makecell{$300$V Cosine} & MLP Regr. & $\mathbf{140236.47}$ & $\mathbf{123571.12}$ & $0.63$ & $\mathbf{1.91}$\\ \cline{2-7}
         & \makecell{Apollo-SR $0.1$} & MLP Regr. &  $175150.82$ &  $145123.09$ & $\mathbf{0.93}$ & $0.96$\\ \cline{2-7}
         & \makecell{Apollo-SR $0.2$} & MLP Regr. & $174390.81$ &  $146338.73$ & $0.93$ & $0.96$\\ \cline{1-7} \cline{1-7}
         % Weather
         \multirow{7}{*}{\makecell{Weather}}& \multirow{1}{*}{ \makecell{$300$V Cosine}} & \makecell{SVM SGD}& $\mathbf{14.13}$ & $\mathbf{7.63}$ & $1.06$ & $\mathbf{22.8}$ \\ \cline{2-7}
               & \makecell{Apollo-SR $0.1$} & SVM & $69.51$ &  $25.52$ & $\mathbf{2.10}$ &  $1.16$\\ \cline{2-7}
                        & \makecell{Apollo-SR $0.2$} & SVM & $68.70$ &  $22.81$ & $2.10$ & $1.16$\\ \cline{2-7} \cline{2-7}
       &  \multirow{1}{*}{ \makecell{$200$V Cosine}}& MLP & $\mathbf{14.29}$ & $\mathbf{4.03}$ & $1.03$  & $\mathbf{22.8}$\\ \cline{2-7}
        &  \multirow{1}{*}{ \makecell{$200$V SR-$0.4$}}& MLP & $15.95$ & $13.31$ & $1.02$  & $1.77$\\ \cline{2-7}
         & \makecell{Apollo-SR $0.1$} & MLP & $69.62$ &  $23.10$ & $\mathbf{1.34}$ & $1.14$ \\ \cline{2-7}
         & \makecell{Apollo-SR $0.2$} & MLP & $673.56$ &  $\mathbf{84.70}$ & $1.32$ & $1.14$\\ \cline{1-7} \cline{1-7}
         
         % Adult
         \multirow{7}{*}{\makecell{Adult}}& \multirow{1}{*}{ \makecell{$300$V Cosine}} & \makecell{SVM SGD}& $\mathbf{0.36}$ & $\mathbf{0.2}$ & $0.37$   & $\mathbf{2.78}$\\ \cline{2-7}
                  & \makecell{Apollo-SR $0.1$} & SVM & $68.32$ &  $22.95$ & $\mathbf{0.75}$ & $0.85$ \\ \cline{2-7}
                 & \makecell{Apollo-SR $0.2$} & SVM & $68.88$ &  $22.88$ & $0.74$ & $0.85$\\ \cline{2-7} \cline{2-7}

         &  \multirow{1}{*}{ \makecell{$300$V K-Means}}& MLP & $\mathbf{0.36}$ & $\mathbf{0.19}$ & $0.30$ & $2.78$ \\ \cline{2-7}
        & \makecell{$300$V SR-$0.2$} & MLP & $6.01$ & $6.00$ &  $0.54$  & $\mathbf{3.54}$\\ \cline{2-7}
         & \makecell{Apollo-SR $0.1$} & MLP & $71.11$ &  $26.51$ & $\mathbf{1.07}$ & $1.31$\\ \cline{2-7}
         & \makecell{Apollo-SR $0.2$} & MLP & $70.16$ &  $25.74$ & $1.05$ & $1.31$\\ \cline{1-7}
         
    \end{tabular}
    % }
\end{table*}

% Table \ref{tab:table-eval-res} illustrates the model operators for each dataset and each loss function, amortized speedup and speedup from our framework in comparison with the same model operators from the Apollo \cite{b7Apollo1, b7Apollo2} framework with SR of $0.1$ and $0.2$. The values $100$V, $200$V, and $300$V in the method column correspond to the dimensions of the vector embedding used for each dataset. The lowest prediction error for each modelled operator in each dataset is highlighted in the method that is used in the similarity search step from our pipeline. Apollo outperforms our framework only on the stock dataset for SR equal with $0.1$ in the LR analytic operator for both RMSE and MAE loss function which performs $50\%$ and $6\%$ better on each loss function equivalent. While our framework for the MLP for Regression outperforms the Apollo modelled operator for $20\%$ and $84\%$ for RMSE and MAE loss functions. However, this difference in the Stock dataset for LR operator modelling is not significant. In the remaining datasets, our framework illustrates that it can outperform Apollo for different values of SR. This makes us confirm that our similarity search using similarity functions selects the most similar datasets $D_r$ from data lake directory $D$, increasing data quality and minimising $\Phi$ prediction errors on the dataset $D_o$. For the Adult dataset, our model operators also perform better, which indicates our method's advantage with an increased number of dataset features (columns). In term of speedup we can see that Apollo outperformed our framework of all modelled operators. In terms of speedup we can see that Apollo outperformed our framework of all modelled operators. This is due to the vectorisation method of our framework which consists of big complexity time. Furthermore, in amortized speedup in most of the amortized speedup in which the vectorization is not counted because it is executed only one time and can be reused our framework surpasses Apollo framework in most of the operators with a big difference with our framework to be between $10\%$ and $60\%$ faster than Apollo. Additionally, most datasets demonstrate better amortized speedup when using the SR approach within our framework. This is because the prediction process relies solely on the vector representation, rather than leveraging all dataset tuples as done in the similarity search method for operator modelling. However, in terms of prediction accuracy, the SR approach does not perform as well as the similarity search method, which achieves superior results.

Table \ref{tab:table-eval-res} compares model operators, loss functions, and speedup metrics for our framework and Apollo at SR values of $0.1$ and $0.2$. Methods $100$V, $200$V, and $300$V denote vector embedding dimensions. The lowest prediction errors align with our pipeline's similarity search method.
Apollo outperforms our framework on the Stock dataset for the LR analytic operator at SR equals with $0.1$ (with $50\%$ and $6\%$ improvements for RMSE and MAE, respectively). However, our framework excels with the MLP regression operator, improving RMSE and MAE by $20\%$ and $17\%$, respectively. The LR operator's performance gap on the Stock dataset is minor.
For other datasets, our framework consistently surpasses Apollo across different SR values. This demonstrates the effectiveness of our similarity search approach, which enhances data quality and reduces $\Phi$ prediction errors by identifying relevant datasets $D_r$ from the data lake directory $D$. The Adult dataset also highlights our framework's advantage with increasing feature dimensions.
Although Apollo achieves better raw speedup due to the higher complexity of our framework's vectorization step, our framework outperforms it in amortized speedup. By excluding the reusable vectorization process, it achieves speed gains of $10\%$ to $60\%$ for most operators.
The SR approach, leveraging vector embedding representations, enhances operator prediction compared to Apollo and achieves greater amortized speedup. However, the similarity search method outperforms both Apollo and the SR approach in prediction accuracy and amortized speedup, establishing its clear superiority across most datasets and operator scenarios.

\subsection{NumTabData2Vec Evaluation Results}

\begin{figure}[!ht]
    \centering
    \includegraphics[width=0.4\textwidth]{Figures/Results/Representation/V200_representation.pdf}
    \caption{Vector representation for each dataset from NumTabData2Vec}
    \label{fig:eval-data-repr}
\end{figure}


\begin{table}[!htp]
    \centering
    \caption{Similarity between vectors of different datasets scenarios}
    \label{tab:vec-rep-sim}
    \setlength\doublerulesep{0.5pt}
    \begin{tabular}{||c|c||}
    \hline
    Model Name & Similarity \\
    \hline\hline
     \makecell{NumTabData2Vec\\$100$ Vector size} & $0.54$\\
     \hline
      \makecell{NumTabData2Vec\\$200$ Vector size}   & $0.18$\\
      \hline
       \makecell{NumTabData2Vec\\$300$ Vector size}  & $0.16$\\ \hline
    \end{tabular}
\end{table}

% Our proposed model, \textit{NumTabData2Vec}, for dataset vectorization is compared between all the available dataset scenarios to determine whether it can effectively distinguish between them based on qualitative differences. The comparison involves selecting $n$ random datasets for each detaset scenario and projecting them into their respective vector embedding representations. Then for each dataset scenario, it gains the average vector embedding representation by the average vector embedding representation of the $n$ random datasets. The vector embedding representation for each dataset scenario depicted in Figure \ref{fig:eval-data-repr} in from the $k$-dimensional space (size of $200$) transformed to the 3d space using the PCA. Figure \ref{fig:eval-data-repr} demonstrates that each dataset occupies a distinct dimension, with non-overlapping or clustering closely together. This indicates that \textit{NumTabData2Vec} can identify the datasets from various situations and does not have a close representation like previous methods achieved it with the same accuracy but on different data types (such as word, and graphs) \cite{b8Word2Vec, b9Graph2Vec} and not in an entire dataset. Table \ref{tab:vec-rep-sim}, further illustrates the average cosine similarity between the vector embeddings of all datasets, demonstrating how dissimilar are the datasets in their vector representation. As the size dimension of the vector embedding representation increases, the model's ability to distinguish across datasets improves as their average similarity decreases. Furthermore, this indicates that larger vector dimension sizes are unneeded since between $100$ and $300$ is sufficient.

Our proposed model, \textit{NumTabData2Vec}, was evaluated to determine its ability to distinguish dataset scenarios based on qualitative differences. For each scenario, $n$ random datasets were selected, and their vector embeddings averaged to represent the scenario. These embeddings, initially in a 200-dimensional space, were projected into 3D using PCA and are shown in Figure \ref{fig:eval-data-repr}. The figure illustrates that each dataset scenario occupies a distinct space, with minimal overlap or clustering. This demonstrates that \textit{NumTabData2Vec} effectively distinguishes datasets, outperforming prior methods like Word2Vec and Graph2Vec \cite{b8Word2Vec, b9Graph2Vec}, which achieved similar accuracy but on different data types (e.g., words, graphs) rather than entire datasets. Table \ref{tab:vec-rep-sim} further highlights the average cosine similarity between dataset embeddings, showing greater dissimilarity as vector dimensions increase. However, results suggest that dimensions between $100$ and $300$ are sufficient for accurate distinction, avoiding the need for larger vector sizes.

\begin{figure}[!ht]
    \centering
    \includegraphics[width=0.4\textwidth]{Figures/Results/Representation/plot_representation_200Vectors.pdf}
    \caption{Synthetic data vector embedding representation}
    \label{fig:eval-sd-data-repr}
\end{figure}

To evaluate \textit{NumTabData2Vec}'s ability to distinguish datasets with varying row and column counts, we generated synthetic numerical tabular datasets of different dimensions and vectorized them. Figure \ref{fig:eval-sd-data-repr} shows datasets with columns ranging from three to thirty and rows from ten to one thousand, projected from a $200$-dimensional space to 2D using PCA. Each bullet caption c and r corresponds to the columns and rows of the dataset, respectively. Datasets with the same number of columns cluster closely in the representation space, and a similar pattern is observed for datasets with the same number of rows. These results indicate that our method effectively distinguishes datasets based on size during vectorization.

\begin{table}[!htp]
    \centering
    \caption{NumTabData2Vec execution time for different dataset dimensions and different vector sizes }

    \begin{adjustbox}{width=\columnwidth,center}
    \label{tab:vec-exec-time}
    \setlength\doublerulesep{0.5pt}
    \begin{tabular}{||c|c|c|c|c||}
    \hline
     \makecell{\# of columns} & \makecell{\# of rows} & \makecell{$50$ Vectors\\Execution time} & \makecell{$100$ Vectors\\Execution time} & \makecell{$200$ Vectors\\Execution time} \\
    \hline\hline
     $3$ & $100$ & $0.0004$ sec & $0.00042$ sec & $0.00051$ sec\\ \hline
     $3$ & $500$ & $0.0004$ sec & $0.00041$ sec & $0.00049$ sec\\ \hline
     $3$ & $1000$ & $0.0004$ sec & $0.00041$ sec & $0.00049$ sec\\ \hline
     $3$ & $1500$ & $0.0004$ sec & $0.00041$ sec & $0.00055$ sec\\ \hline
     $3$ & $1800$ & $0.0004$ sec & $0.00041$ sec & $0.00055$ sec\\ \hline
     \hline
     $10$ & $100$ & $0.0004$ sec & $0.0004$ sec & $0.00057$ sec\\ \hline
     $10$ & $500$ & $0.00039$ sec & $0.0004$ sec & $0.00051$ sec\\ \hline
     $10$ & $1000$ & $0.00041$ sec & $0.00042$ sec & $0.00052$ sec\\ \hline
     $10$ & $1500$ & $0.00041$ sec & $0.00042$ sec & $0.00055$ sec\\ \hline
     $10$ & $1800$ & $0.00041$ sec & $0.00042$ sec & $0.00052$ sec\\ \hline
     \hline
     $20$ & $100$ & $0.0004$ sec & $0.00042$ sec & $0.0005$ sec\\ \hline
     $20$ & $500$ & $0.0004$ sec & $0.00042$ sec & $0.0005$ sec\\ \hline
     $20$ & $1000$ & $0.00042$ sec & $0.00043$ sec & $0.00052$ sec\\ \hline
     $20$ & $1500$ & $0.00043$ sec & $0.00044$ sec & $0.00054$ sec\\ \hline
     $20$ & $1800$ & $0.00044$ sec & $0.00044$ sec & $0.00054$ sec\\ \hline    
     \hline\hline
    \end{tabular}
    \end{adjustbox}
\end{table}

To evaluate how dataset dimensions affect the execution time of \textit{NumTabData2Vec}, we created synthetic datasets with varying numbers of rows ($100$, $500$, $1000$, $1500$, and $1800$) and columns ($3$, $10$, and $20$). These datasets were vectorized into different dimensions, and the execution times were recorded. Table \ref{tab:vec-exec-time} shows that increasing the k-dimension requires approximately $20\%$ more time to generate the vector embeddings. This is expected, as a higher k-dimension involves more hyperparameters, which naturally increases computation time.

Interestingly, varying the number of columns did not significantly impact execution time. However, increasing the number of rows resulted in approximately $5\%$ additional execution time. This is because larger datasets require the extraction of more features, which has a modest impact on the model's execution time.

\begin{figure}[!ht]
    \centering
    \includegraphics[width=0.4\textwidth]{Figures/Results/Representation/plot_representation_noise_data_200Vectors.pdf}
    \caption{HPC Dataset vector embedding representation with addition of Noise}
    \label{fig:eval-nd-data-repr}
\end{figure}

To evaluate \textit{NumTabData2Vec}'s ability to distinguish datasets based on different properties like distribution and order, we introduced Gaussian noise to random $l\%$ of data tuples in an HPC dataset. Figure \ref{fig:eval-nd-data-repr} visualises the original and noise-modified datasets, projected from a 200-dimensional space to 2D using PCA. Each bullet caption g denotes the percentage of Gaussian noise added in the dataset. As noise increases, the representation space shifts further from the original dataset, indicating that \textit{NumTabData2Vec} effectively captures distribution differences. Additionally, since the HPC dataset has an inherent order, the model's sensitivity to noise demonstrates its ability to distinguish datasets based on ordering as well.

\begin{figure}[!ht]
    \centering
    \includegraphics[width=0.4\textwidth]{Figures/Results/Representation/plotrepresentationnoisedata1col200Vectors.pdf}
    \caption{HPC Dataset vector embedding representation with addition of Noise in the first column}
    \label{fig:eval-nd-data-repr-1col}
\end{figure}

To evaluate how fine-grained as distinction can be, we introduced noise into a single column and repeated the previous experiment, with the difference being that noise was added exclusively to the first column. Figure \ref{fig:eval-nd-data-repr-1col} visualizes the dataset's 2D vector space. The amount of Gaussian noise added to the dataset's first column is indicated by g in the bullet caption. The results show that as more noise is introduced to the column, the vector representation moves further away from the original dataset. In contrast to the previous experiment shown in Figure \ref{fig:eval-nd-data-repr}, the noisy dataset's representation stays closest to the original when only a single column is modified. Also in this experiment the dataset points in the 2-dimension are more grouped between them instead the previous experiment. 

%closely grouped compared to the previous experiment.

\section{Conclusion}
\label{sec:Conclusion}
We present RiskHarvester, a risk-based tool to compute a security risk score based on the value of the asset and ease of attack on a database. We calculated the value of asset by identifying the sensitive data categories present in a database from the database keywords. We utilized data flow analysis, SQL, and Object Relational Mapper (ORM) parsing to identify the database keywords. To calculate the ease of attack, we utilized passive network analysis to retrieve the database host information. To evaluate RiskHarvester, we curated RiskBench, a benchmark of 1,791 database secret-asset pairs with sensitive data categories and host information manually retrieved from 188 GitHub repositories. RiskHarvester demonstrates precision of (95\%) and recall (90\%) in detecting database keywords for the value of asset and precision of (96\%) and recall (94\%) in detecting valid hosts for ease of attack. Finally, we conducted an online survey to understand whether developers prioritize secret removal based on security risk score. We found that 86\% of the developers prioritized the secrets for removal with descending security risk scores.

\section*{Acknowledgment}
This work was supported by the National Science
Foundation CCF Grant No. 2107085, the ONR Minerva program, and iMAGiNE -- the Intelligent Machine Engineering Consortium at UT Austin.

%%
%% The next two lines define the bibliography style to be used, and
%% the bibliography file.

\newpage

\bibliographystyle{ACM-Reference-Format}
\bibliography{bibliography}


\end{document}
\endinput
%%
%% End of file `sample-sigconf.tex'.
