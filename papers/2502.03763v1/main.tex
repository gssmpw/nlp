%%
%% This is file `sample-sigconf.tex',
%% generated with the docstrip utility.
%%
%% The original source files were:
%%
%% samples.dtx  (with options: `all,proceedings,bibtex,sigconf')
%% 
%% IMPORTANT NOTICE:
%% 
%% For the copyright see the source file.
%% 
%% Any modified versions of this file must be renamed
%% with new filenames distinct from sample-sigconf.tex.
%% 
%% For distribution of the original source see the terms
%% for copying and modification in the file samples.dtx.
%% 
%% This generated file may be distributed as long as the
%% original source files, as listed above, are part of the
%% same distribution. (The sources need not necessarily be
%% in the same archive or directory.)
%%
%%
%% Commands for TeXCount
%TC:macro \cite [option:text,text]
%TC:macro \citep [option:text,text]
%TC:macro \citet [option:text,text]
%TC:envir table 0 1
%TC:envir table* 0 1
%TC:envir tabular [ignore] word
%TC:envir displaymath 0 word
%TC:envir math 0 word
%TC:envir comment 0 0
%%
%%
%% The first command in your LaTeX source must be the \documentclass
%% command.
%%
%% For submission and review of your manuscript please change the
%% command to \documentclass[manuscript, screen, review]{acmart}.
%%
%% When submitting camera ready or to TAPS, please change the command
%% to \documentclass[sigconf]{acmart} or whichever template is required
%% for your publication.
%%
%%
\documentclass[sigconf]{acmart}




\copyrightyear{2025}
\acmYear{2025}
%% \setcopyright{cc}
%% \setcctype{CC-BY}
\setcopyright{rightsretained}
\acmConference[FPGA '25]{Proceedings of the 2025 ACM/SIGDA International Symposium on Field Programmable Gate Arrays}{February 27--March 1, 2025}{Monterey, CA, USA}
\acmBooktitle{Proceedings of the 2025 ACM/SIGDA International Symposium on Field Programmable Gate Arrays (FPGA '25), February 27--March 1, 2025, Monterey, CA, USA}
\acmDOI{10.1145/3706628.3708867}
\acmISBN{979-8-4007-1396-5/25/02}

% The following includes the CC license icon appropriate for your paper.
% Download the image from www.scomminc.com/pp/acmsig/4ACM-CC-by-88x31.eps
% and place within your figs or figures folder

% \makeatletter
% \gdef\@copyrightpermission{
%   \begin{minipage}{0.3\columnwidth}
%    \href{https://creativecommons.org/licenses/by/4.0/}{\includegraphics[width=0.90\textwidth]{4ACM-CC-by-88x31.eps}}
%   \end{minipage}\hfill
%   \begin{minipage}{0.7\columnwidth}
%    \href{https://creativecommons.org/licenses/by/4.0/}{This work is licensed under a Creative Commons Attribution International 4.0 License.}
%   \end{minipage}
%   \vspace{5pt}
% }
% \makeatother


%%
%% Submission ID.
%% Use this when submitting an article to a sponsored event. You'll
%% receive a unique submission ID from the organizers
%% of the event, and this ID should be used as the parameter to this command.
%%\acmSubmissionID{123-A56-BU3}

%%
%% For managing citations, it is recommended to use bibliography
%% files in BibTeX format.
%%
%% You can then either use BibTeX with the ACM-Reference-Format style,
%% or BibLaTeX with the acmnumeric or acmauthoryear sytles, that include
%% support for advanced citation of software artefact from the
%% biblatex-software package, also separately available on CTAN.
%%
%% Look at the sample-*-biblatex.tex files for templates showcasing
%% the biblatex styles.
%%

%%
%% The majority of ACM publications use numbered citations and
%% references.  The command \citestyle{authoryear} switches to the
%% "author year" style.
%%
%% If you are preparing content for an event
%% sponsored by ACM SIGGRAPH, you must use the "author year" style of
%% citations and references.
%% Uncommenting
%% the next command will enable that style.
%%\citestyle{acmauthoryear}

\usepackage{comment}
\usepackage{graphicx}
\usepackage{xcolor}
\usepackage{multirow}
\usepackage{siunitx}
\usepackage{todonotes}
\usepackage{hyperref}
% \usepackage{cite}
\usepackage{amsmath}
\usepackage{amsfonts}
\usepackage{algorithmic}
\usepackage{subfig}
\usepackage{makecell}
\usepackage{svg}
\usepackage{tablefootnote}
\usepackage[flushleft]{threeparttable}
\usepackage[symbol]{footmisc}
\usepackage{fixltx2e}
\usepackage{caption} 
\captionsetup[table]{skip=4pt}
\usepackage{soul}




%These 4 lines solve the problem with the url overflowing in the references for WaveNet. Had to comment hyperref above.
\usepackage{url}
% \usepackage[breaklinks]{hyperref}
% \usepackage{breakurl}


%%
%% end of the preamble, start of the body of the document source.
\begin{document}

%%
%% The "title" command has an optional parameter,
%% allowing the author to define a "short title" to be used in page headers.
\title{Systolic Sparse Tensor Slices: FPGA Building Blocks for Sparse and Dense AI Acceleration}



\author{Endri Taka}
\affiliation{%
  \institution{The University of Texas at Austin}
  \city{Austin}
  \state{TX}
  \country{United States}
}
  \email{endri.taka@utexas.edu}


\author{Ning-Chi Huang}
\affiliation{%
  \institution{National Yang Ming Chiao Tung University}
  \city{Hsinchu}
  \country{Taiwan}
  % \email{nchuang@cs.nctu.edu.tw}
}

\author{Chi-Chih Chang}
\affiliation{%
  \institution{National Yang Ming Chiao Tung University}
  \city{Hsinchu}
  \country{Taiwan}
  % \email{brian1009.en08@nycu.edu.tw}
}

\author{Kai-Chiang Wu}
\affiliation{%
  \institution{National Yang Ming Chiao Tung University}
  \city{Hsinchu}
  \country{Taiwan}
  % \email{kcw@cs.nctu.edu.tw}
}


\author{Aman Arora}
\affiliation{%
  \institution{Arizona State University}
  \city{Tempe}
  \state{AZ}
  \country{United States}
  % \email{aman.kbm@asu.edu}
}

\author{Diana Marculescu}
\affiliation{%
  \institution{The University of Texas at Austin}
  \city{Austin}
  \state{TX}
  \country{United States}
  % \email{dianam@utexas.edu}
}

%%
%% By default, the full list of authors will be used in the page
%% headers. Often, this list is too long, and will overlap
%% other information printed in the page headers. This command allows
%% the author to define a more concise list
%% of authors' names for this purpose.
\renewcommand{\shortauthors}{Endri Taka et al.}

%%
%% The abstract is a short summary of the work to be presented in the
%% article.
\begin{abstract}

FPGA architectures have recently been enhanced to meet the substantial computational demands of modern deep neural networks (DNNs).
To this end, both FPGA vendors and academic researchers have proposed in-fabric blocks that perform efficient tensor computations. 
However, these blocks are primarily optimized for dense computation, while most DNNs exhibit sparsity. 
To address this limitation, we propose incorporating \textit{structured} sparsity support into FPGA architectures.
We architect 2D systolic in-fabric blocks, 
named systolic sparse tensor (SST) slices, that support multiple degrees of sparsity to efficiently accelerate a wide variety of DNNs.
SSTs support dense operation, 2:4 (50\%) and 1:4 (75\%) sparsity, as well as a new 1:3 (66.7\%) sparsity level to further increase flexibility.
When demonstrating on general matrix multiplication (GEMM) accelerators, which are the heart of most current DNN accelerators, our sparse SST-based designs attain up to 5$\times$ higher FPGA frequency and 10.9$\times$ lower area, compared to traditional FPGAs.
Moreover, evaluation of the proposed SSTs on state-of-the-art sparse ViT and CNN models exhibits up to 3.52$\times$ speedup with minimal area increase of up to 13.3\%, compared to dense in-fabric acceleration.




\end{abstract}



\begin{CCSXML}
<ccs2012>
   <concept>
       <concept_id>10010583.10010600.10010628.10010629</concept_id>
       <concept_desc>Hardware~Hardware accelerators</concept_desc>
       <concept_significance>500</concept_significance>
       </concept>
   <concept>
       <concept_id>10010520.10010521.10010528.10010535</concept_id>
       <concept_desc>Computer systems organization~Systolic arrays</concept_desc>
       <concept_significance>500</concept_significance>
       </concept>
 </ccs2012>
\end{CCSXML}

\ccsdesc[500]{Hardware~Hardware accelerators}
\ccsdesc[500]{Computer systems organization~Systolic arrays}



%%
%% Keywords. The author(s) should pick words that accurately describe
%% the work being presented. Separate the keywords with commas.
% \keywords{FPGA, structured sparsity, computer architecture, deep learning}


\keywords{FPGA, structured sparsity, hardware acceleration, matrix multiplication, computer architecture, deep learning, machine learning}



%%
%% This command processes the author and affiliation and title
%% information and builds the first part of the formatted document.
\maketitle



\section{Introduction}
\label{sec:Introduction}
% 
% 
The widespread integration of communication networks and smart devices in modern control systems has increased the vulnerability of industrial systems to online cyber-attacks, e.g., Industroyer, Blackenergy, etc \citep{osti_1505628}.
% Modern control systems have seen a large push to include communication networks and smart devices to increase performance, made possible by improvements in communication device cost and energy consumption. This trend has been coupled with the usage of open-standard communication protocols among industrial control systems, making them vulnerable to online cyber-attacks such as Industroyer, Blackenergy, etc \citep{osti_1505628}. 
To counter this, methods have been developed to improve security by achieving attack detection, mitigation, and monitoring, among others \citep{sandberg2022secure}. This paper focuses on active attack diagnosis to mitigate stealthy attacks. 
%
%\subsection{Literature review}

Active diagnosis techniques rely on the inclusion of additional moduli to control systems
% inclusion within the control system of additional moduli 
to alter the behavior of the system compared to information known by the attacker. 
For instance, the concept of additive watermarking was introduced in \cite{mo2015physical}, where noise signals of known mean and variance are added at the plant and compensated for it at the controller. 
This compensation, however, is not exact, causing some performance degradation. Thus, trade-offs between performance and detectability  are necessary \citep{zhu2023detection}.
% A later work \citep{zhu2023detection} designs the watermark signal by trading performance for detection. Thus, although additive watermarking serves as a good detection scheme, they endure performance losses even in the nominal case. 

In encrypted control \citep{darup2021encrypted}, the sensor data is encrypted, sent to the controller, and then operated on directly. Encrypted input signals are sent back to the plant for decryption. Although encryption is widespread in IT security, in control systems it presents some concerns, such as the introduction of time delays \citep{stabile2024verifiable}, while it may present inherent weaknesses \citep{alisic2023model}.
% they are not preferred as they introduce time delays \citep{stabile2024verifiable} which can cause instability, and some encryption schemes can be very weak  \citep{alisic2023model}. 

In moving target defense \citep{griffioen2020moving}, the plant is augmented with fictitious dynamics, known to the controller. The plant output is transmitted to the controller along with the fictitious states over a network under attack. 
The additional measurements then aide in the detection of attacks. 
This comes at the cost of higher communication bandwidth needs, which increases rapidly with the dimension of the augmented systems.
% Since the dynamics of the fictitious dynamics are exactly known to the controller, the attack is detected easily. However, when the scale of the system increases, the communication bandwidth used by moving the target defense approach increases rapidly. 

Other recently proposed works include two-way coding \citep{fang2019two}, a weak encryuption technique, and dynamic masking \citep{abdalmoaty2023privacy}, which enhances privacy as well as security, have been shown to be effective against zero-dynamics attacks.
% Two-way coding \citep{fang2019two} and dynamic masking \citep{abdalmoaty2023privacy} are other recently proposed approaches. Two-way coding is another form of weak encryption technique whilst dynamic masking proposes an architecture that enhances both privacy and security. These schemes are shown to be effective against zero dynamics attacks but remain to be studied for other classes of attacks. 
% Recent extensions include \citep{mukherjee2021secure,ramos2024privacy}.
% Some other works which are related are \citep{mukherjee2021secure}, an extension of \cite{fang2019two}. The work \citep{ramos2024privacy} is an extension of moving target defense for multi-agent systems. 
Furthermore, filtering techniques for attack detection are proposed by \cite{murguia2020security,hashemi2022codesign,escudero2023safety}, while not focusing on stealthy attacks.
% The works \citep{murguia2020security,hashemi2022codesign,escudero2023safety} develop filtering techniques to guarantee safety, without being focused on stealthy covert attacks.

Multiplicative watermarking (mWM) has been proposed by the authors as a diagnosis technique \citep{ferrari2020switching}. mWM consists of a pair of filters on each communication channel between the plant and its controller; the scheme is affine to weak encryption, whereby ``encoding'' and ``decoding'' are done by changing signals' dynamic characteristics through inverse pairs of filters. This enables original signals to be recovered exactly, and thus does not lead to performance degradation.
% A multiplicative watermark is an affine to a weak encryption technique, through which the signal is ``encoded'' by a filter, changing its dynamic behavior. The use of inverse pairs means that the original signal can be recovered, through ``decoding'' via an inverse filter. As such, differently to techniques based on additive watermarking, no performance is lost due to the injection of noise, and there are no bandwidth limitations.

%\subsection{Contributions}
One of the critical features of multiplicative watermarking is that to detect stealthy attacks, the mWM filter parameters must be switched over time. In this paper, an algorithm to optimally design the mWM parameters after a switching event is presented, enhancing detection performance, without changing the switching time.
% This is done without changing the switching time, which is taken as given.

\textcolor{black}{
To formalize the filter design problem, we suppose the defender is interested in optimal performance against adversaries injecting covert attacks with matched system parameters \citep{smith2015covert}, including the mWM parameters prior to the switch. This scenario represents a worst case where malicious agents can take full control of the system while remaining undetected.
Thus, the attack strategy is explicitly included within the formulation of the closed-loop system, and the mWM filters are chosen by solving an optimization problem minimizing the attack-energy-constrained output-to-output gain (AEC-OOG) \citep{anand2023risk}, a variation of the output-to-output gain proposed in  \cite{teixeira2015strategic}.
}
The main contributions of this paper are:
% We consider an adversary injecting a covert attack with matched system parameters \citep{smith2015covert}, i.e., an attacker with full knowledge of the control system parameters, including those of the mWM filters before the switch. This scenario is taken as a worst case, as it has been shown that this class of attacks can be made stealthy. To quantitatively define a cost, the output-to-output gain (OOG) \citep{teixeira2015strategic} is leveraged,
% a metric introduced to evaluate the impact of an additive attack in a control system. %Specifically, OOG evaluates the worst-case performance loss that an attacker injecting an undetectable attack can obtain. 
% Here, the maximum performance loss caused by a stealthy adversary with limited energy is taken, the attack-energy-constrained OOG (AEC-OOG) \citep{anand2023risk}. The main contributions of this paper are:
\begin{enumerate}
%[label=\alph*.]
\item The problem of optimally designing the switching mWM filters is formulated as an optimization problem, with the AEC-OOG is taken as the objective;%where the AEC-OOG is taken as the impact metric; 
\item The worst-case scenario of a covert attack with exact knowledge of plant and mWM filter parameters is embedded within the design problem;
% The optimization problem is defined to incorporate the worst-case scenario of a covert attack with exact knowledge of plant and mWM filter parameters;
\item The feasibility of the optimization problem is shown to be dependent only on stability conditions; 
\item A solution scheme is proposed to promote randomization of the mWM filter parameters such that an eavesdropping adversary cannot remain stealthy.
\end{enumerate} 

This builds on the results of \cite{ferrari2020switching}, where the focus was on the design of the switching protocols, rather than the parameters themselves.
Compared to previous work \citep{gallo2021design}, this paper introduces an optimization problem which is always feasible (thanks to the use of AEC-OOG in the objective), while also considering a more sophisticated class of covert attacks, where the presence of watermark is known to the adversary. 
Moreover, this paper poses a different objective than \citep{zhang2023hybrid}; indeed, while \citep{zhang2023hybrid} provided a design strategy to ensure certain privacy properties, in this paper we address the problem of optimal parameter design following a switching event.


%\subsection{Organization}
The rest of the paper is organized as follows. 
After formulating the problem in Section~\ref{sec:PF}, we propose our design algorithm in Section~\ref{sec:main}, and analyze its properties. It is then evaluated through a numerical example in Section~\ref{sec:NE}, and concluding remarks are given Section~\ref{sec:Con}.
% We provide the problem background in Section~\ref{sec:PF}. We formulate the design problem in Section~\ref{sec:main}, together with an analysis of its properties. The proposed algorithm is evaluated through a numerical example in Section \ref{sec:NE}. Concluding remarks are offered in Section \ref{sec:Con}.

\section{Related Work}
\label{sec:Related_work}
\section{Related Work}
In this section, we provide a broad overview of self-supervised learning research that has inspired our work, along with recent trends in image clustering using pre-trained models.


\subsection{Self-Supervised Learning}
Self-supervised learning learns representations from data without explicit labels. The objective is to create a representation space where positive pairs are closer together, while negative pairs are pushed farther apart \cite{geiping2023cookbook}.

SimCLR \cite{chen2020simple} uses data augmentations, such as flipping and colour jittering, to create positive and negative pairs for optimizing objectives. It also introduces a projection head that maps embeddings into a space where contrastive loss is applied. BYOL \cite{grill2020bootstrap} shows that high-quality representations can be learned by simply maximizing agreement between two augmented views of the same input, without requiring negative pairs. Building on these advancements, SimSiam \cite{chen2020exploringsimplesiameserepresentation} eliminates the need for both negative pairs and momentum encoders by introducing a stop-gradient operation, which effectively prevents representational collapse. Inspired by these methods, we adopt similar ideas to develop a simple and effective self-supervised framework for image clustering.

\subsection{Pre-trained Models in Vision} 
Building on advances in self-supervised learning, CLIP \cite{radford2021learning} introduced a paradigm of contrastive pre-training that aligns images with corresponding textual descriptions. This approach enables broad task generalization without task-specific fine-tuning. DINO \cite{caron2021emerging}, which stands for self-distillation with no labels, demonstrates a self-supervised method for optimizing a student network from a teacher network based on vision input data only.

One of the key advantages of pre-trained models like CLIP is their ability to eliminate the need for training models from scratch for downstream tasks, significantly reducing computational costs and time. Instead of training a self-supervised neural network from the ground up, pre-trained models provide high-quality feature representations out of the box, leading to faster experimentation and improved performance on a variety of tasks. The scalability of CLIP has been further validated by openCLIP \cite{Cherti_2023}, which extended CLIP using the larger Vision Transformer models \cite{dosovitskiy2020image}. Similarly, models such as DINO~\cite{9709990} and DINOv2~\cite{oquab2024dinov2learningrobustvisual} are capable of processing visual data and mapping it to high-quality latent representations.

\subsection{Image Clustering via Pre-trained Models}
To address the challenges of scaling to modern image datasets, methods such as NMCE \cite{li2022neural} and MLC \cite{deng2023acp} have integrated deep learning with manifold clustering using the minimum coding rate principle \cite{Arthur_Vassilvitskii_2007}. Building on this idea, CPP \cite{chu2024image} further refines CLIP features and estimates the optimal number of clusters when unknown. TEMI \cite{adaloglou2023exploring} improves clustering by leveraging associations between image features, introducing a variant of pointwise mutual information with instance weighting. Unlike our approach, TEMI utilizes a nearest-neighbors set and an exponential moving average for parameter optimization.

SIC \cite{cai2023semantic} leverages multi-modality by mapping images to a semantic space and generating pseudo-labels based on image-semantic relationships. More recently, TAC~\cite{li2023image} utilizes the textual semantics of WordNet~\cite{miller1995wordnet} to enhance image clustering by selecting and retrieving nouns that best distinguish the images, facilitating collaboration between text and image modalities through mutual cross-modal neighborhood distillation.

Current pre-trained approaches often rely on heavy or complex architectures to ensure consistency, motivating us to develop a simple yet effective pipeline for image clustering. Our method requires only a simple clustering head and basic data augmentations, demonstrating strong competitiveness among recent models.









\section{Architecture \& Design Overview}
\label{sec:Architecture_Overview}





\subsection{Fine-Grained Structured Sparsity}
\label{subsec:Fine_grained_structured_sparsity}
In this work, we leverage the regular patterns of fine-grained structured sparsity to architect SST slices with low area overhead. 
Fig. \ref{fig:random_sparsity} depicts a 50\% unstructured sparse matrix, where the non-zero data are distributed \textit{randomly}, \emph{i.e.,} there is no specific pattern of their locations.
In contrast, in Fig. \ref{fig:structured_sparsity}, the 2:4 structured sparsity pattern is illustrated, which has the same sparsity level (50\%), but in every group of four \textit{consecutive} elements
% row-wise, 
there are two non-zero values. 
Notice that the location of the two non-zero values can vary significantly within the four-element group, offering \textit{fine-grained} sparsity flexibility.  
These types of constraints enable low area hardware enhancements, allowing for efficient  exploitation of sparsity.


\begin{figure}[t]
\vspace{-0.70cm}
\centering
\subfloat[]
{\includegraphics[width=0.31\linewidth]{03_Architecture_Overview/unstructured_50_percent_sparsity.pdf}
\label{fig:random_sparsity}}
\subfloat[]{\includegraphics[width=0.685\linewidth]{03_Architecture_Overview/structured_2_4_sparsity.pdf}
\label{fig:structured_sparsity}} 

\vspace{-0.35cm}

\caption{50\% unstructured sparse matrix (a) and 2:4 (50\%) structured sparse matrix along with compressed format (b).} 
\label{fig:sparsity_struct_unstruct}
\vspace{-0.50cm}
\end{figure}


A 2:4 sparse matrix can be efficiently stored in  compressed format by saving only the non-zero values.
The location of each non-zero data is encoded using 2-bit indices, as shown in Fig. \ref{fig:structured_sparsity}.
Similar to 2:4, for 1:4 (75\%) sparsity one every four consecutive elements is non-zero, while for 1:3 (66.7\%) sparsity there is one non-zero every three consecutive elements.  
For all aforementioned patterns, 2-bit indices are required to encode the location of each non-zero element.
This is a very efficient compressed format, as we show in Sec. \ref{subsec:AIE_ML_comparison}, where comparison among other formats is performed.










\subsection{Systolic Sparse Tensor Slices Architecture}
\label{subsec:Sparse_Tensor_slices_architecture}

In this section, we present an
% high-level 
overview of the proposed SST slices. 
The core compute unit of the SSTs is a 4$\times$4 systolic array (SA) \cite{Kung_SA_1982}, as depicted in Fig. \ref{fig:ST_architecture}.
A 2D SA consists of homogeneous processing elements (PEs), where each PE performs a multiply--accumulate (MAC) operation and forwards the input operands to the neighboring PEs. 
This architecture allows maximization of data reuse
in GEMM, 
while also delivering high performance due to its regular and highly scalable design.
Hence, SAs have become a prime architecture in many DNN accelerators \cite{TPUV2_v3_2021, TPUv42021, S2TA_HPCA_2022, Vegeta_HPCA_2023, SA_CNN_FPGA_2017, SA_attention_FPGA_TECS_2023, Scale_sim_2020}.
In this work, we show that incorporating SST slices in FPGAs leads to high performance and scalable dense/sparse GEMM accelerators. 




\begin{figure}[tbp]
\vspace{-0.50cm}
\centering
\includegraphics[width=0.89\linewidth]{03_Architecture_Overview/SST_architecture.pdf}

\vspace{-0.50cm}

\caption{Systolic Sparse Tensor slice architecture.}
\label{fig:ST_architecture}

\vspace{-0.60cm}

\end{figure}


\subsubsection{SST Operation} 
We enhance the systolic PEs with sparse features by introducing sparse processing elements (SPEs), while maintaining the properties of the SAs discussed above. 
Our SSTs utilize an output stationary SA, consisting of 16 SPEs.
The accumulations remain stationary in the SPEs, while input operands are propagated to their neighbors every clock cycle.
The $a\_data$ of an input matrix $A$ are propagated and reused across SPEs horizontally, while the $b\_data$ of an input matrix $B$ are propagated and reused vertically (Fig. \ref{fig:ST_architecture}).
The matrix $A$ can be either sparse or dense (typically to map \textit{weights}), while $B$ is dense (typically to map input \textit{activations}).
The architecture of the SPEs is delineated in Sec. \ref{subsec:Sparse_Processing_Element}.


Besides the 4$\times$4 SPE grid, we also implement pipeline registers to delay the input operands for systolic data setup \cite{TPUv1_2017} (arranged in triangular manner in Fig. \ref{fig:ST_architecture}).
Systolic setup is needed at the interface for loading input matrices (typically from on-chip memory), when chaining multiple SSTs to construct larger SA grids (Sec. \ref{subsec:Matrix_multiplication_mapping}).
Multiplexers are used to select either the systolic setup
% registers
or directly the input data, and are configured \textit{statically} (during bitstream loading). 




\begin{figure*}[tbp]
\vspace{-0.40cm}
\centering
\includegraphics[width=1.00\textwidth]{03_Architecture_Overview/SPE_all_sparse_modes.pdf}

\vspace{-0.45cm}

\caption{Sparsity modes in Systolic Sparse Elements of the SST slices (multiplexing logic omitted for clarity).}
\label{fig:SPE_sparse_modes}
\vspace{-0.35cm}
\end{figure*}


The SSTs support both int8 and bfloat16 precisions, while accumulations are realized in 32-bit integer (int32) and IEEE 32-bit floating-point (fp32), respectively, similar to Nvidia GPUs \cite{Nvidia_accelerate_sparse_2021} and Google TPUs \cite{TPUV2_v3_2021}.
When the SA operation is completed, the output matrix $C$ is extracted via the $c\_data$ output ports.
We note that SPEs finish their operation in a \textit{diagonal} fashion cycle after cycle.
In the first cycle, $SPE00$ finishes, in the second cycle both $SPE01$ and $SPE10$ finish, etc. (Fig. \ref{fig:ST_architecture}). 
However, to maintain \textit{regularity}, thus simplifying the downstream logic (typically in CLBs), we extract the output values in a column-wise manner (four values per cycle).
To achieve that, we introduce a six-element buffer (consisting of registers), to store the data before getting extracted.
% column-wise.
This is particularly important in sustaining 100\% SPE utilization (16 MACs per cycle) at the steady-state (matrices processed one after the other).

Since SPEs complete their operation diagonally, the six upper-triangular SPE outputs, shown in Fig. \ref{fig:ST_architecture}, need to be stored in the buffer.
Suppose a cycle $T$ where the outputs of the biggest diagonal, \emph{i.e.,} $SPEs$ $\{03, 12, 21, 30\}$ are generated.
In cycle $T$, the values of the first column, \emph{i.e.,} $SPEs$ $\{00, 10, 20, 30\}$ can be extracted, where $SPEs$ $\{00, 10, 20\}$ are loaded from the buffer, while only $SPE30$ is directly extracted.
In cycle $T + 1$, the values of $SPEs$ $\{03, 12, 21\}$ replace the position of $SPEs$ $\{00, 10, 20\}$ in the buffer (since they have been extracted).
Therefore, the locations of the six-element buffer are being effectively reused, and in cycle $T+1$, the second column can be extracted, \emph{i.e.,} $SPEs$ $\{01, 11, 21, 31\}$.
In a similar fashion, the rest two columns are extracted in cycles $T+2$ and $T+3$, respectively, while the buffer locations are efficiently reused due to replacement. 








\subsubsection{Global Routing Interface \& Dedicated Wires}

As illustrated in Fig. \ref{fig:ST_architecture}, the $a\_data$ and $b\_data$ input ports as well as $c\_data$ output ports are connected to the global FPGA routing resources.
When chaining multiple SSTs, $a\_data$ and $b\_data$ are forwarded to their next in the chain SSTs, horizontally and vertically, respectively.
Horizontally, the data are propagated via the $a\_data\_out$ using the FPGA routing resources.
However, vertically, we utilize \textit{dedicated} wires to propagate the data (via the $b\_ded\_out$ ports) and connect them to the next SST in the \textit{same} FPGA column (via the $b\_ded\_in$ ports).
These vertical dedicated wires provide efficient connections without the usage of the global routing resources, matching the columnar nature of the modern FPGA fabric \cite{FPGA_architecture_2021, FPGA_for_DL_2024}.
This approach significantly reduces routing resources, as opposed to  \cite{TS_Aman_FPGA_2021, Aman_TS_TRETS_2022}, where all inputs/outputs (I/Os) of the in-fabric tensor slices are connected to global routing (see comparison in Sec. \ref{subsec:Dedicated_wires_benefits}).




Besides the I/Os shown in Fig. \ref{fig:ST_architecture}, the SSTs include also several dynamic control signals.
In particular, an input $enable$ signal is used to control the operation of the SSTs.
This signal should be deasserted when the operation of the SST needs to be stalled.
% in the case where the SST needs to stall its operation.
Moreover, an $accumulate$ signal is utilized to control the accumulation in the SPEs.
This signal remains asserted when accumulation is desirable in the SPEs, \emph{e.g.,} during tiling to process larger matrices.
However, for every new matrix operation the $accumulate$ signal should be deasserted for one cycle.
Another input signal is the  $d\_type$, which  dynamically selects the precision, \emph{i.e.,} int8 or bfloat16.
Finally, a 2-bit $sparsity\_level$ signal is employed to dynamically select the sparsity level of the matrix $A$, \emph{i.e.,} dense, 2:4, 1:3, 1:4.
This signal is utilized internally in the SSTs to control the operation of the supported sparsity modes, as discussed in the next section.

Regarding the outputs of the SST, an $accumulate\_out$ signal is used for systolic distribution of the $accumulate$ signal when chaining multiple SSTs.
This systolic distribution makes the control logic significantly simpler, since the $accumulate$ is set only for the first SST in the 2D array layout, while being distributed to the remaining SSTs, as explained in Sec. \ref{subsec:Matrix_multiplication_mapping}.
Another output signal is the $valid\_out$, which is asserted when the output $c\_data$ are extracted column-wise (Fig. \ref{fig:ST_architecture}). 
This signal simplifies the downstream logic, since only $valid\_out$ needs to be checked for valid output data.





\subsection{Sparse Processing Element}
\label{subsec:Sparse_Processing_Element}

The SPE architecture in all supported sparsity modes is illustrated in Fig. \ref{fig:SPE_sparse_modes}.
When the SPE is configured in dense mode (Fig. \ref{fig:SPE_sparse_modes}a), it operates as a regular dense PE.
In particular, one MAC operation is performed every cycle, while the elements of matrices $A$ and $B$ are forwarded to their neighboring SPEs (via pipeline registers) horizontally and vertically, respectively.
The output value of matrix $C$ in each SPE is calculated after $K$ cycles, when considering the $M$$\times$$K$$\times$$N$ matrix dimensions depicted in Fig. \ref{fig:SPE_sparse_modes}a.



In 2:4 mode (50\% sparsity), matrix $A$ is stored in a compressed format of $M$$\times$$K/2$ size for both values and the 2-bit indices, as shown in Fig. \ref{fig:SPE_sparse_modes}b.
In order to achieve speedup over the dense case, we increase the number of ports of each SPE in the vertical dimension, to load four elements of the matrix $B$ in parallel.
Furthermore, the 2-bit index is also loaded in the SPE to select the corresponding $B$ values (via a 4:1 multiplexer), which need to get multiplied with each $A$ value.
However, in 2:4 sparsity, for every four values of the matrix $B$, two MAC operations need to be performed.
Hence, the four $B$ values remain in the SPE registers for two clock cycles (the four green registers in Fig. \ref{fig:SPE_sparse_modes}b are loaded every two cycles).
To correctly synchronize the systolic operation, two pipeline stages are required for both the $A$ values and their indices (blue and yellow registers in Fig. \ref{fig:SPE_sparse_modes}b), which are loaded every cycle. 
In this manner, one MAC operation is performed every cycle, ensuring 100\% utilization.
Moreover, the output value of matrix $C$ in each SPE is calculated in $K/2$ cycles, achieving 2$\times$ acceleration over dense operation.




\begin{table}[t]
\centering
\caption{Summary of supported sparsity levels in SST slices.}

\setlength\tabcolsep{9pt}
\renewcommand{\arraystretch}{1.0}
\resizebox{0.80\linewidth}{!}{
% \vspace{-0.10cm}
\begin{tabular}{c|cc|c|c}
\Xhline{2.5\arrayrulewidth}

\textbf{Sparsity}  &  
\multicolumn{2}{c|}{\textbf{Compres. ratio}} & \textbf{Speedup} & \textbf{SPE} \\

\cline{2-3}

\textbf{level}  & \textbf{int8} & \textbf{bfloat16} 
 & \textbf{over dense} & \textbf{util.} \\

\hline
\hline

\textbf{Dense (0\%)} & 1$\times$ & 1$\times$ & 1$\times$ & 100\% \\ 

\textbf{2:4 (50\%)} & 1.6$\times$ & 1.78$\times$ & 2$\times$ & 100\% \\ 

\textbf{1:3 (66.7\%)} & 2.4$\times$ & 2.67$\times$ & 3$\times$ & 100\% \\ 

\textbf{1:4 (75\%)} & 3.2$\times$ & 3.56$\times$ & 4$\times$ & 100\% \\ 


\Xhline{2.5\arrayrulewidth}

\end{tabular}
}

\label{tb:sparsity_summary_benefits}

\vspace{-0.50cm}

\end{table}



\begin{figure*}[ht]

\vspace{-0.55cm}

\centering
\includegraphics[width=0.81\textwidth]{03_Architecture_Overview/GEMM_2D_design.pdf}

\vspace{-0.35cm}

\caption{2D systolic GEMM design: dense implementation (a) and dynamic configuration of all supported sparsity modes (b).}
\label{fig:GEMM_2D_array_SSTs}

\vspace{-0.45cm}

\end{figure*}


Similar to 2:4, for 1:4 (75\%) sparsity, four $B$ values are loaded in the SPE.
However, in this case the $B$ values are loaded every cycle, while only one pipeline stage is required for the $A$ value and its index, as shown in Fig. \ref{fig:SPE_sparse_modes}c.
The output value needs $K/4$ cycles to be calculated, offering 4$\times$ speedup over the dense case.
To fill the sparsity gap between 50\% and 75\% as explained in Sec. \ref{sec:Introduction}, we leverage the same hardware enhancements for 2:4 and 1:4 sparsity to also support the 1:3 (66.7\%) pattern.
In particular, as illustrated in Fig. \ref{fig:SPE_sparse_modes}d, the 1:3 operation is similar to 1:4, with the main difference being that three $B$ values are loaded every cycle instead of four.
The indices ensure that the fourth $B$ value is never selected, thus no additional hardware is required.
In this case, $K/3$ cycles are needed for every SPE output value, providing 3$\times$ acceleration.
Finally, we note that similar to 1:3, 2:3 sparsity can also be supported by directly utilizing the 2:4 mode, leading to 33.3\% sparsity.
However, a matrix is typically considered sparse when it has sparsity of 50\% or higher \cite{AMD_AIE_ML_kernel_guide}.
Hence, in this work, we do not consider the 2:3 pattern. 







In Table \ref{tb:sparsity_summary_benefits}, we present a summary of the supported sparsity levels.
First, we show the \textit{compression ratio}, \emph{i.e.,} the memory reduction over dense storage due to compressed format, for both int8 and bfloat16 precisions.
For all sparsity levels, bfloat16 offers a higher compression ratio over int8, \emph{e.g.,} 1.78$\times$ \emph{vs.} 1.6$\times$ for 2:4 sparsity. 
This is because 2-bit indices are required for both 8-bit and 16-bit data types, resulting in relatively lower overhead for the compressed representation in bfloat16 compared to int8.
This compressed format substantially reduces both on-chip and off-chip memory requirements, achieving up to 3.56$\times$ reduction (Table \ref{tb:sparsity_summary_benefits}).


Second, we notice that every sparsity level is
% effectively
translated to its corresponding speedup, \emph{e.g.,} 4$\times$ for 1:4 (75\%) sparsity, achieving 100\% SPE utilization in all cases (Table \ref{tb:sparsity_summary_benefits}).
For all sparsity levels, data reuse is maximized since both indices and values are propagated and reused horizontally and vertically, similar to dense operation.






\subsection{GEMM Design Utilizing Multiple SST Slices}
\label{subsec:Matrix_multiplication_mapping}


In this section, we describe parametric GEMM implementations for both dense and sparse configurations, utilizing multiple SST slices.
Both implementations are highly regular and scale effectively on the FPGA fabric, attaining high frequencies as shown in Sec. \ref{subsec:Sparse_GEMM_implementation}.




\subsubsection{Dense Implementation}

Fig. \ref{fig:GEMM_2D_array_SSTs}a depicts a parametric GEMM accelerator comprising a 2D array of SST slices, which are configured in \textit{dense} mode (Sec. \ref{subsec:Sparse_Processing_Element}). 
The 2D array consists of $Y \cdot X$ SST slices, implementing a total SA size of $(Y \cdot 4) \times (X \cdot 4)$.
This size is denoted as the \textit{native} size of the GEMM accelerator. 
On-chip memory buffers are implemented to store the input matrices $A$, $B$ and the output matrix $C$.
The input buffers $A$, $B$ are located in the left and top edges of the 2D array, respectively, and are partitioned into banks, providing sufficient bandwidth to feed the SSTs.
In particular, for buffer $A$, $Y$ banks are required, while for buffer $B$, $X$ banks are needed.
Since each SST slice includes a 4$\times$4 SA, each bank needs to provide a bandwidth of 32-bits per cycle when SSTs are configured for int8 precision, while for bfloat16, 64-bits per cycle are required.
Regarding the output buffer $C$, $X \cdot Y$ banks are needed due to the output stationary architecture of the SSTs, each receiving an output of 128-bits per cycle (four 32-bit values as explained in Sec. \ref{subsec:Sparse_Tensor_slices_architecture}).
 



The data from the buffers $A$, $B$ are propagated in a systolic fashion between the SST slices, as illustrated in Fig. \ref{fig:GEMM_2D_array_SSTs}a.
Horizontally, the $A$ data are propagated via the global routing resources of the FPGA fabric.
Vertically, the $B$ data are inserted via global routing wires in the first SST at each vertical chain ($Y$ SSTs in Fig. \ref{fig:GEMM_2D_array_SSTs}a), while being forwarded to the next SSTs via dedicated wires.
It is important to note here that the SSTs comprising each vertical chain are \textit{physically} contiguous in the FPGA.
Nevertheless, in the horizontal dimension, the $Y$ chains might not follow the \textit{logical} arrangement shown in Fig. \ref{fig:GEMM_2D_array_SSTs}a inside the FPGA fabric, due to the routing flexibility of FPGAs.
This depends on decisions made by the FPGA place and route (PnR) algorithm.
Finally, notice the (static) systolic data setup configuration specifically for the SST slices that interface with the buffers $A$, $B$ (left and top edge of the 2D array).


Control logic, mapped to the CLB resources of the FPGA, is utilized to orchestrate the entire operation of the GEMM design.
We also implement tiling logic (in CLBs) to exploit data reuse in GEMM, as well as to support arbitrary GEMM sizes based on the available on-chip memory resources.
When mapping an arbitrary GEMM of $M^\prime$$\times$$K^\prime$$\times$$N^\prime$ dimensions, $M^\prime$ must be a multiple of $(Y \cdot 4)$, while $N^\prime$ must be a multiple of $(X \cdot 4)$, since the \textit{native} size of the accelerator is $(Y \cdot 4) \times (X \cdot 4)$.
Note that there is no constraint on the reduction $K^\prime$ dimension.
Finally, although not shown in Fig. \ref{fig:GEMM_2D_array_SSTs}a, $accumulate$ signals utilized during tiling (Sec. \ref{subsec:Sparse_Tensor_slices_architecture}) are propagated in a systolic fashion among the 2D array of SSTs, similar to the $A$, $B$ data.
The control logic sets the $accumulate$ signal only for the first SST in both vertical and horizontal dimensions (\emph{i.e.,} the SST fed by buffers $A_{1}$ and $B_{1}$), which significantly simplifies the overall logic.






\subsubsection{Dynamic Sparse Configuration}

Fig. \ref{fig:GEMM_2D_array_SSTs}b illustrates a parametric GEMM design that is \textit{dynamically} configured to support all the sparsity modes in the SSTs, \emph{i.e.,} dense, 2:4, 1:3 and 1:4.
This dynamic configuration is particularly important for layer-wise sparsity exploitation in DNNs, since each layer might require different sparsity level for optimal trade-off between DNN accuracy and speedup (Sec. \ref{subsec:Performance_estimation_DNNs}).
We note that the implementation is similar to the dense design (Fig. \ref{fig:GEMM_2D_array_SSTs}a), with main differences lying in the design of buffers $A$ and $B$, as well as in the vertical and horizontal propagation of the data.
Horizontally, the $A$ data are kept in the buffer banks in compressed format for the sparse modes (2:4, 1:3 and 1:4).
In this case, both non-zero data and indices are loaded in the SSTs and are propagated horizontally (see Sec. \ref{subsec:Sparse_Processing_Element}).
More specifically, for int8, each buffer $A$ bank needs to provide a bandwidth of 40-bits per cycle, due to the additional 8-bits indices for the four vertical SPEs at the interface of each SST.
Similarly, for bfloat16, 72-bits per cycle are required.


Vertically, four $B$ banks are needed to feed the SSTs due to the 4$\times$ increase in ports for 2:4 and 1:4 sparsity acceleration (Sec. \ref{subsec:Sparse_Processing_Element}).
Each $B$ bank provides the same bandwidth as the dense design (Fig. \ref{fig:GEMM_2D_array_SSTs}a), \emph{i.e.,} 32-bits and 64-bits per cycle for int8 and bfloat16, respectively.
Similar to the dense design, the $B$ data are inserted in the first SST at each vertical chain and dedicated wires are used to propagate them vertically (Fig. \ref{fig:GEMM_2D_array_SSTs}b).
These dedicated wires are particularly important for the sparse design, since otherwise 4$\times$ more vertical wires would be required to use global routing compared to the dense design (in the case where \textit{only} non-dedicated wires are employed).




Besides sparse operation, the design in Fig. \ref{fig:GEMM_2D_array_SSTs}b also supports dense computation.
This is because dense computation might still be needed, even if all weights in DNNs are sparse.
For instance, in Transformer-based DNNs \cite{Attention_all_you_need_2017, BERT_2019, ViT_2020}, the QKV (Query, Key, Value) 
% self-attention 
GEMMs do not involve weights, and are typically computed as dense.
For dense computation, only one $B$ bank is sufficient at each vertical chain, \emph{e.g.,} $B_{x,0}$ (Fig. \ref{fig:GEMM_2D_array_SSTs}b).
However, multiplexing logic can be employed to 
% effectively 
utilize the remaining $B_{x,1}$, $B_{x,2}$, $B_{x,3}$ banks.
This is particularly important to ensure efficient utilization of on-chip memory resources, which leads to maximized data reuse and thus optimized energy efficiency \cite{Versal_vs_Stratix_FCCM_2024, MaxEVA_2023}.
Finally, for 1:3 sparsity, only banks $B_{x,0}$, $B_{x,1}$, $B_{x,2}$ are required, leaving bank $B_{x,3}$ unused. 
However, similar to the dense operation, 
% additional 
multiplexing logic can be employed to reuse this bank when it comprises multiple BRAMs, which we do not explore in this work (see Sec. \ref{subsec:Sparse_GEMM_implementation} for implementation details). 



The dynamic configuration among all sparsity levels 
% in Fig. \ref{fig:GEMM_2D_array_SSTs}b
is implemented in the control logic (using CLBs).
However, FPGA accelerators can be designed in a custom fashion depending on the sparsity of each DNN.
% required in DNN.
For instance, a specific DNN might require only 1:3 sparsity and dense computation across all of its layers.
The control and tiling logic for sparsity is similar to the dense design (Fig. \ref{fig:GEMM_2D_array_SSTs}a), showcasing a marginal increase in CLB resources (see Sec. \ref{subsec:Sparse_GEMM_implementation}).






\section{Evaluation}
\label{sec:Evaluation}
\section{Evaluation}
% In light of experiments of CacheBlend (\S\ref{eval:1}) and EPIC (\S\ref{eval:2}), we design our experiments (\S\ref{eval:3}).

% \noindent\textbf{LLM Dataset.} 2WikiMQA, MuSiQue, HotpotQA, SAMSum, MultiNews.

% \noindent\textbf{LLM Baselines.} Full KV recompute, Prefix caching, Full KV reuse, CacheBlend, EPIC.
% \subsection{CacheBlend evaluation}\label{eval:1}
% \begin{itemize}
%     \item TTFT-Score Comparison.
%     \item RPS-TTFT Comparison.
%     \item Sensitivity Analysis. (1) chunk number; (2) chunk length; (3) batch size; (4) recompute ratio; (5) storage device (CPU RAM / slower Disk).
% \end{itemize}
% \subsection{EPIC evaluation}\label{eval:2}
% \begin{itemize}
%     \item TTFT-Score Comparison.
%     \item (CCR+RPS)-TTFT/Throughput Comparison.
%     \item Context length-TTFT Comparison.
%     \item Semantic-based / fixed-token-based splitting.
% \end{itemize}
% \subsection{\sys}\label{eval:3}\
% \noindent\textbf{VLM Model.} InternVL 2.5-8B \cite{chen2024internvl}, Qwen2-VL-7B \cite{wang2024qwen2vl}, LLaVA-1.6-vicuna-7B, LLaVA-1.6-Mistral-7B \cite{liu2024llavanext}.

% \noindent\textbf{VLM Dataset.} SparklesDialogueCC, SparklesDialogueVG \cite{huang2024sparkles}, MMDU \cite{liu2024mmdu}.

% \noindent\textbf{VLM Baselines.} CacheBlend, Prefix caching, Full KV reuse, \sys.

% \begin{itemize}
%     \item TTFT-Score Comparison.
%     \item RPS-TTFT/Throughput Comparison.
%     \item Sensitivity Analysis: Image number.
%     \item Why does CacheBlend fail to work when serving MLLM?
% \end{itemize}

In this section, we evaluate \sys~in terms of response time and generation quality. We also investigate whether \sys~is applicable when the number of images is large.
\subsection{Experimental settings}
We select two prevalent MLLMs in the experiments: LLaVA-1.6-vicuna-7B and LLaVA-1.6-mistral-7B \cite{liu2024llavanext}. All experiments are run on a server with 1 NVIDIA H800-80 GB GPU, 20-core Intel(R) Xeon(R) Platinum CPUs, and 100GB DRAM.

Two datasets are used in our evaluation. (1) \textbf{MMDU} \cite{liu2024mmdu}: This dataset aims to evaluate MLLMs' abilities in multi-turn and multi-image conversations. Each conversation stitches together multiple images and sentence-level text (e.g., ``IMAGE\#1, IMAGE\#2. Can you describe these images as detailed as possible?"). (2) \textbf{SparklesEval} \cite{huang2024sparkles}: This is also a dataset for assessing MLLMs' conversational competence across multiple images and conversation turns. Unlike MMDU, SparklesEval integrates multiple images at word level (e.g., ``Can you link the celebration occurring in IMAGE\#1 and the dirt bike race in IMAGE\#2 ?"). As shown in the examples, the prompts of two datasets are open questions. Previous works adopt GPT score to evaluate the quality of MLLMs' responses to the open questions \cite{liu2024mmdu, huang2024sparkles}. GPT score is a GPT-assisted evaluation that uses a powerful judge model (e.g., GPT-4o, Qwen, etc.) to assess the answers. We also employ this metric and their evaluation prompt, as listed in Appendix~\ref{prompt}.
% (3) \textbf{V*Bench} \cite{wu2024v}:  A dataset specifically designed to evaluate
% MLLMs in their ability to process high-resolution images and focus on visual details. Each sample contains a high-resolution image, a question, and four options.
% We select 100 samples from each of the above datasets for testing, each including 1 to 5 images.

% We use the following metrics to measure the performance of algorithms. (1) Time-To-First-Token (TTFT) refers to the time it takes for LLMs, to generate and return the first token after receiving an request. This metric is designed to measure the time spent in the prefill stage, which can be optimized by addressing the PIC problem. (2) GPT score \cite{liu2024mmdu, huang2024sparkles} is a GPT-assited evaluation  that uses a judge model (e.g., GPT-4o, Qwen, etc.) to assess the quality of model-generated responses. We employ this metric to assess the quality of MLLMs' responses to the open questions in MMDU and SparklesEval. We apply the evaluation prompts in MMDU \cite{liu2024mmdu} to guide the judge model for scoring in the range of 10. 

% (3) F1 score is a metric used to evaluate the similarity between MLLMs’ output and the groundtruth answer. We employ this metric to assess the accuracy of the MLLMs' answers to the multiple-choice questions in V*Bench.

We compare \sys-$k$ with three existing CC algorithms: prefix caching, full reuse, and CacheBlend \cite{yao2024cacheblend}. CacheBlend is also a position-independent algorithm designed for RAG system. It recomputes $r$\% of total tokens with largest KV deviation, so we denote it as CacheBlend-$r$. The primary focus of CacheBlend is the KV deviation, while the \sys's selection process involves the identification of tokens that exhibit both high attention scores and significant KV deviation. We implement the four CC algorithms based on vLLM 0.6.4 \cite{kwon2023efficient}.

% (1) Prefix Caching: This algorithm merely stores and reuses the KV chche of the prefix. And the KV cache of non-prefix tokens needs to be computed during prefill. (2) Full Reuse: This algorithm reduces TTFT by fully reusing the entire KV cache regardless of the position of multimodal data. (3) CacheBlend \cite{yao2024cacheblend}: This is a state-of-the-art partial reuse algorithm that achieves a trade-off between TTFT and generation quality by dynamically selecting partial tokens to recompute.
% Additionally, we evaluate various variants of CacheBlend, denoted as CacheBlend-r, where $r$ represents the ratio of tokens recomputed. Similarly, we test different variants of InfoBlend, denoted as InfoBlend-k, where $k$ indicates the number of tokens recomputed at each chunk boundary.

\subsection{Effectiveness of \sys}
Based on vLLM offline inference, we compare the performance of all algorithms. Specifically, we process all requests sequentially and evaluate their generation quality and processing time for prefill. The workflow initiates with the precomputation of the relevant KV cache for images. Subsequently, we send the user's query along with the cache\_ids of the images to the serving system. Prefix caching will process the query with the KV cache of system prompt only. \sys~concatenates the dummy cache and stored cache, and computes the first output token using selective attention mechanism in single step. Full reuse and CacheBlend first compute the KV cache of text, and then produce the first output token with the concatenated KV cache. We record the processing time of the algorithms and finally score for each response.
\begin{figure}[t]
    \centering
    \includegraphics[width=\columnwidth]{figs/legend_result.pdf}
    % \vskip -0.2in
    \includegraphics[width=\columnwidth]{figs/results.pdf}
    \caption{Comparison of TTFT ($\downarrow$ Better) and Score ($\uparrow$ Better) using different models on different datasets. }
    \label{fig:ttft-score}
    % \vskip -0.2in
\end{figure}

\figurename~\ref{fig:ttft-score} presents the experimental results of all algorithms across different models and datasets. The results indicate that \sys~consistently outperforms CacheBlend in terms of both TTFT and score across various configurations. \sys-32 reduces TTFT by up to 54.1\% while maintaining a loss of score within 13.6\% compared to prefix caching. Additionally, it is clear that \sys~exhibits a slight decrease in TTFT compared to full reuse, since \sys~is a single-step process. Overall, compared to other algorithms, \sys~achieves the best trade-off between TTFT and score.

\subsection{Sensitivity analysis}
In order to achieve a more profound comprehension of \sys, a subsequent analysis is necessary to ascertain how the number of images impacts overall performance. We divide the dataset of MMDU into 10 groups in terms of the number of images. We evaluate the TTFT and score of \sys~and baselines on each group. The average value of results are shown in \figurename~\ref{fig:10}. The TTFT of \sys~is consistently shorter than that of prefix caching. When the number of images is 10, \sys~achieves 54.7\% reduction in TTFT. Furthermore, the performance of \sys~remains unaffected by the number of images, exhibiting negligible or no accuracy degradation.
\begin{figure}[t]
    \centering
    \includegraphics[width=0.9\columnwidth]{figs/legend_image_num.pdf}
    \vskip -0.2in
    \subfloat[]{
        \includegraphics[width=0.42\columnwidth]{figs/TTFT_all.pdf}
        \label{fig:10a}
    }
    \subfloat[]{
        \includegraphics[width=0.4\columnwidth]{figs/Score_all.pdf}
        \label{fig:10b}
    }
    \caption{The performance of \sys~as the number of images increases. For clarity, we only present the results of \sys-32. Other variants of \sys~show similar patterns.}
    \label{fig:10}
    % \vskip -0.2in
\end{figure}

% \subsection{Latency and throughput performance of InfoBlend}
% To assess Infoblend's latency and throughput performance, we leverage VLLM's OpenAI-compatible API server to simulate real-world user request patterns. We first select $n$ samples from MMDU and pre-generate KV caches for their contexts. Subsequently, we simulate user request behavior by repeatedly sending the user queries along with the cache\_ids of these 
% $n$ samples at a specified request rate over a period of time. by varying the request rate, We measure the latency and throughput across different experimental conditions.

% In Figure, we present a comparison of latency and throughput between InfoBlend and CacheBlend at varying request rates.  Compared to CacheBlend, InfoBlend achieves up to 80\% reduction in TTFT and 2-3 $\times$ improvement in throughput. This gap increases as the request rate rises.

\section{Conclusion}
\label{sec:Conclusion}
Software development is increasingly conceived as a collaboration activity between developers and AIs. Indeed, IDEs already implement features to enable interactive development, with AI suggesting implementations that are reused by developers.

Although multiple studies show this interaction can be successful, there is still limited understanding of how the models must be configured and used in the context of code generation tasks. This study addresses this gap, systematically investigating the impact of several key parameters, including the repeated submission of a prompt to accommodate for the non-deterministic nature of the models.

Our study reveals several key findings about the usage of ChatGPT. In particular, we discovered how creativity, although up to a limited extent, is useful to increase the range of methods whose code can be generated correctly. A major role is played by parameter top-p, which is commonly underrated, and instead has a major impact on the correctness of the results, with lower values producing better results. Finally, prompts should be submitted multiple times, with $5$ repetitions combined with a temperature of $1.2$ resulting in an effective configuration in our experiments.  

Future work concerns two main research directions. One is about replicating this experiment with other AI assistants, to validate our findings in multiple contexts. The second research direction concerns finding strategies to deal with the need to submit the same prompt multiple times to obtain a useful result, and thus developing approaches able to select or merge multiple responses automatically. 

\section*{Acknowledgment}
This work was supported by the National Science
Foundation CCF Grant No. 2107085, the ONR Minerva program, and iMAGiNE -- the Intelligent Machine Engineering Consortium at UT Austin.

%%
%% The next two lines define the bibliography style to be used, and
%% the bibliography file.

\newpage

\bibliographystyle{ACM-Reference-Format}
\bibliography{bibliography}


\end{document}
\endinput
%%
%% End of file `sample-sigconf.tex'.
