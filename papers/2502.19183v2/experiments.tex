In this section, we numerically validate the performance of \infosedd\ on both synthetic and real-world datasets. Specifically, we evaluate the mutual information and entropy estimators presented in \Cref{alg:infosedd} and \Cref{alg:compute_entropy}, respectively, through the following experiments: (i) benchmarking on high-dimensional distributions where ground truth values are known by construction, and (ii) assessing the accuracy of our method in a \textit{real-world} application by estimating the entropy of spin glass configurations in the Ising model \citep{onsager1944crystal}.



\subsection{Synthetic benchmark}\label{sec:synthetic}


\begin{figure*}[t]
    \centering
    \begin{subfigure}[b]{0.22\textwidth}
        \centering
        \includegraphics[height=18ex]{file/figures/synthetic_0.pdf}
        \caption{0'th token}
        \label{fig:subfig1}
    \end{subfigure}%
    \begin{subfigure}[b]{0.22\textwidth}
        \centering
        \includegraphics[height=18ex]{file/figures/synthetic_20.pdf}
        \caption{20'th token}
        \label{fig:subfig2}
    \end{subfigure}%
    \begin{subfigure}[b]{0.22\textwidth}
        \centering
        \includegraphics[height=18ex]{file/figures/synthetic_50.pdf}
        \caption{50'th token}
        \label{fig:subfig3}
    \end{subfigure}%
    \begin{subfigure}[b]{0.22\textwidth}
        \centering
        \includegraphics[height=18ex]{file/figures/synthetic_200.pdf}
        \caption{200'th token}
        \label{fig:subfig4}
    \end{subfigure}%
    \begin{subfigure}[b]{0.12\textwidth}
        \centering
        \raisebox{3ex}{\includegraphics[height=18ex]{file/figures/synthetic_legend.pdf}}
        % \caption{Legend}
        \label{fig:subfig5}
    \end{subfigure}
    \caption{%\small
        The probability that greedy decoding can attain globally optimal response, with respect to the number of newly generated tokens, and with different starting positions in the generation history.
        The legend is shared across sub-figures.
    }
    \label{fig:synthetic_setting}
    % \vspace{-1em}
\end{figure*}

\section{Sanity Check: Semi-Synthetic Settings}\label{main:synthetic}
The implication of our theoretical analysis is straightforward.
However, it is natural to ask whether the phenomenon actually occurs in real-world LLM decoding scenarios.
To provide clear empirical evidence accompanying our theoretical analysis, in this section, we present semi-synthetic experiments that serves as a sanity check.
In particular, in moderately realistic settings, we show that greedy decoding for text generation with stepwise optimization results in suboptimal responses.
We first outline the semi-synthetic setting, and then present the empirical findings.

% \vspace{-2ex}
\paragraph{Illustrative Approximation}
\looseness=-1
For any modern LLM with a vocabulary size $|\Vcal|$ (typically on the order of $10^4$ to $10^5$), identifying the globally optimal sequence across multiple steps becomes computationally intractable, even for relatively short sequence lengths ($<100$).
To ensure the validity of our claim while providing a clear and accessible illustration, we adopt beam search as an approximation strategy of obtaining globally optimal sequence.
Since we measure the chance that greedy decoding can attain the global optimum with the stepwise optimal response, this approximation serves as an upper bound on achievable performance, indicating the discrepancy between greedy decoding and the true globally optimal response.

% \vspace{-2ex}
\paragraph{Approximating Natural Language Scenarios}
Since the prompt or context of the generation influence model behavior, we align our experimental setting with common human-LLM interactions.
Specifically, we utilize MT-Bench \citep{zheng2023judging} questions as curated prompts, which are designed to evaluate conversational chat models.
These samples serve as an approximation of real-world natural language context distributions, ensuring that our findings are grounded in practical scenarios.


% \vspace{-2ex}
\paragraph{Findings}
For each prompt, together with a certain length of generation history ($0$ means only the prompt is given), we evaluate whether the joint probability of the sequence generated with greedy decoding is greater than or equal to that produced by beam search (the proxy of the global optimum).
This comparison indicates the extent to which greedy decoding deviates from the globally optimal response.
As illustrated in Figure~\ref{fig:subfig1}, greedy decoding consistently results in suboptimal sequences, and the phenomenon can be observed with a small number of newly generated tokens.

In addition, the potential deviation may behave differently across various positions in the generated text.
For instance, when openings of response diverge, it is hard for greedy decoding to achieve optimal results afterwards.
To reduce potential inductive bias resulting from the diversity at early stages of generation, we evaluate generations starting/continuing from various positions throughout generation history, as presented in Figures~\ref{fig:subfig2}--\ref{fig:subfig4}.
We can observe that the deviation persists across different positions, which empirically demonstrate the common existence of sub-optimality in autoregressive decoding for text generation.

% \section{(Semi)-Synthetic Experiments}
% The principles behind our approach are straightforward; however, one might question whether the observed phenomenon persists in real-world LLM decoding scenarios. To provide a clear evidence of the previous theoretical analysis and also address the aforementioned concern, we introduce a (semi)-synthetic experiment designed to demonstrate that, in moderately realistic settings, greedy decoding during text generation with step-wise optimization results in suboptimal response. Before presenting the experimental results, we first outline the semi-synthetic setup.

% \paragraph{Illustrative Approximation}\looseness=-1
% Although there is no doubt on the existence of optimal response, for any modern large language model with a vocabulary size \(|\Vcal|\) ( typically on the order of \(O(10^4)\) to \(O(10^5)\)), identifying the joint optimal sequence across multiple steps becomes computationally intractable, even for relatively short sequence lengths (\(<100\)). To uphold the validity of our claim while offering an accessible illustration, we adopt beam search as an approximation of the globally optimal sequence. Since we measure the probability that greedy decoding achieves optimal response, this approximation serves as an upper bound on achievable performance, highlighting the discrepancy between greedy decoding and the true global optimal response.

% \paragraph{Approximating Natural Language Scenarios}
% One may also curious about the prompt or context in the generation since those factors will influence the generation behaviour as well. It is crucial to demonstrate in realistic scenarios. Specifically, we align our experiment setting with common human interactions with large language models. To achieve this, we utilize MT-Bench questions as curated prompts and contexts, which are designed for evaluating conversational, open-ended chat models. These samples serve as an approximation of real-world natural language context distributions, ensuring that our findings are grounded in practical, human-like use cases.

% \paragraph{Results}
% For each prompt or context, we evaluate whether the joint probability of the sequence generated through greedy decoding is greater than or equal to that of the sequence produced by the approximate global optimum method, i.e., beam search. This comparison highlights the extent to which greedy decoding diverges from the optimal sequence. As demonstrated in Figure \ref{fig: synthetic exp}, greedy decoding consistently results in suboptimal sequences, even when generating a small number ($<10$) of new tokens. Additionally, the behavior varies across positions in the generated text. For example, the beginning of the answer are usually diverge and it can be harder for greedy decoding to achieve optimal. To reduce potential inductive biases arising from greater diversity at the initial stages of generation, we evaluate sequences at various positions throughout the generated text (Figure \ref{fig:subfig1}-\ref{fig:subfig4} capture different starting positions for the evaluation). Our findings indicate that this behavior persists across all evaluated positions, underscoring the importance of the issue. This observation naturally suggests a solution: a reflective decoding strategy using a sliding window mechanism.





\paragraph{Experimental Setup} To evaluate the performance of our mutual information estimator, \infosedd, we design three sets of controlled synthetic experiments, each isolating a key aspect of data complexity. In the first experiment, we fix the mutual information value at $0.5$ and the length of the random vector at $2$, meaning each of the two random variables has one dimension. We then increase the support size using Cantor's mapping (\Cref{sec:isomorphisms}). The second experiment maintains a mutual information value of $0.5$ and binary support for each element of the random vector, but instead increases the length of the representation vector. Finally, in the third experiment, we generate distributions with varying mutual information values while keeping the support binary for each variable. The length of the random vectors is fixed at $10$, and mutual information values are linearly spaced from $0$ to $5$. We compare the results of our proposed methodology against MINE \citep{belghazi2018mine}, a variational neural estimator, MINDE \citep{franzese2023minde}, a generative neural estimator and KSG \citep{kraskov2004estimating}, a \textit{classical} statistical estimator. For a fair comparison, we use \glspl{MLP} based architectures for the neural estimators and scale parameters appropriately when needed (see \Cref{sec:training_details}). When an estimate is not available, we fill the entry in the tables with $-$. We report all results in nats.

\begin{table}[t]
\label{tab:mi_comparison}
\vspace{1em} % Space after title
\begin{center}
\begin{minipage}{0.45\textwidth}
\tiny
   
    \centering
    \caption{Different support dimension $|\support|$ and MI=0.5}
    \begin{tabular}{c|cccc}
        \multicolumn{1}{c}{\bf $|\chi|$} & \multicolumn{1}{c}{\bf MINDE} & \multicolumn{1}{c}{\bf MINE} & \multicolumn{1}{c}{\bf I-SEDD} & \multicolumn{1}{c}{\bf KSG} \\ 
        \hline \\
        2 & 0.51 & 0.50 & 0.50 & 0.49 \\
        4 & 0.04 & 0.46 & 0.53 & 0.49 \\
        16 & 0.02 & 0.00 & 0.51 & 0.49 \\
        64 & 0.00 & -12644 & 0.54 & 0.48 \\
        256 & 0.00 & 0.00 & 0.58 & 0.44 \\
        1024 & 0.00 & - & 0.61 & 0.24 \\
    \end{tabular}
    \label{tab:mi_big_support_table}
\end{minipage}
%\vspace*{-1cm}
\begin{minipage}{0.45\textwidth}
\tiny

    \centering
    \caption{Different random vector lengths and MI=0.5}
    \begin{tabular}{c|cccc}
        \multicolumn{1}{c}{\bf Length} & \multicolumn{1}{c}{\bf MINDE} & \multicolumn{1}{c}{\bf MINE} & \multicolumn{1}{c}{\bf I-SEDD} & \multicolumn{1}{c}{\bf KSG} \\ 
        \hline \\
        2 & 0.55 & 0.50 & 0.50 & 0.51 \\
        8 & 0.39 & 0.51 & 0.55 & 0.51 \\
        32 & 0.27 & 0.73 & 0.52 & 0.07 \\
        128 & 0.35 & 0.43 & 0.48 & 0.00 \\
        512 & 0.38 & 1.22 & 1.60 & 0.00 \\
        2048 & 5.56 & 0.00 & 5.19 & 0.01 \\
    \end{tabular}
    \label{tab:mi_big_vector_table}
\end{minipage}

%\vspace{-2em} % Space before the next table

\begin{minipage}{0.45\textwidth}
\tiny

    \centering
    \caption{Different MI values}
    \begin{tabular}{c|cccc}
        \multicolumn{1}{c}{\bf MI} & \multicolumn{1}{c}{\bf MINDE} & \multicolumn{1}{c}{\bf MINE} & \multicolumn{1}{c}{\bf I-SEDD} & \multicolumn{1}{c}{\bf KSG} \\ 
        \hline \\
        0 & 0.40 & 0.02 & 0.00 & 0.01 \\
        1 & 0.23 & 1.26 & 1.01 & 0.49 \\
        2 & 1.44 & 2.29 & 2.03 & 1.14 \\
        3 & 2.17 & 3.08 & 3.02 & 2.15 \\
        4 & 3.51 & 3.84 & 4.01 & 3.39 \\
        5 & 17.92 & 4.93 & 5.07 & 4.72 \\
    \end{tabular}
    \label{tab:mi_estimators}
\end{minipage}
\hfill
\begin{minipage}{0.45\textwidth}
\includegraphics[scale=0.3]{entropy.png}
\captionof{figure}{Entropy of the Ising model at different temperatures}\label{fig:ising}
\end{minipage}


\end{center}
\vspace{1em} % Space after table
\end{table}


\paragraph{Results and Analysis} The results from these experiments, shown in \Cref{tab:mi_big_support_table,tab:mi_big_vector_table,tab:mi_estimators}, demonstrate that \infosedd\ consistently outperforms competing methods. Notably, our model excels in settings where an inherent property of discrete systems—such as support size, representation length, or mutual information—introduces increased complexity.

Although our competitors perform relatively well in some benchmarks, they fail in at least one experiment. KSG performs well with large support size (\Cref{tab:mi_big_support_table}), but it fails when considering higher dimensions (\Cref{tab:mi_big_vector_table}). MINE and MINDE, on the contrary, excels with higher dimensions but struggle with large support size. MINDE also struggles more with higher mutual information values (\Cref{tab:mi_estimators}). Overall, these experiments motivate the usage of adequate neural estimators when dealing with discrete distributions. 

%Neural information metrics estimators essentially rely on a neural network to decode the relevant information for this task. When the data generating process becomes more convoluted, which in our case means injecting increasingly more complex nuisance in the dataset, information is harder to decode and extracting insights from low density regions of the data distributions becomes more important. Generative modeling approaches offer adequate inductive bias to encourage the exploration of the data distribution, while other neural estimators like MINE restrict themselves to the available training data. As a consequence, MINE is not able to decode the Cantor's pairing function, with a poor performance in related experiments (\Cref{tab:mi_big_support_table}). Additionally, efficiently exploring discrete data distributions remains a daunting task. Score-based generative models like MINDE waste many training steps exploring zero density regions due to their continuous nature, leading to slow convergence to reach the right estimate in experiments with big support (\Cref{tab:mi_big_support_table}) and higher mutual information (\Cref{tab:mi_estimators}). Moreover, training objectives relying on noise prediction use the Euclidean distance to calculate their loss, which is inadequate in discrete settings, especially in the case of non-ordinal data. Info-SEDD, instead, efficiently explores discrete data distribution by exploiting CTMC, leading to a more accurate mutual information estimation. Overall, these experiments motivate the usage of adequate neural estimators when dealing with discrete distributions. 
%\vspace*{-1cm}
\subsection{Spin glasses experiments}
Entropy computation in Ising models enables insights on the thermodynamics properties of the system \citep{cincio2007entropy}, which can be used for scientific discovery in the domain where the Ising model is applied \citep{macy2024ising, schneidman2006weak, sherrington1975solvable}. 

\paragraph{Experimental setup} We consider a simplified Ising model applied to spin glasses \citep{sherrington1975solvable}. We do not include an external field, we set a unitary interaction strength for all the sites interactions, unitary Boltzmann's constant and a $20\times20$ square lattice. The entropy per site of this configuration can be computed in closed form \cite{onsager1944crystal}. We test our model by estimating the entropy per particle at linearly spaced temperatures from $1.0K$ to $4.0K$. We generate our dataset using the Metropolis-Hastings algorithm, with $10000$ samples for each temperature (see \Cref{subsec:app_ising}). We post-process the output of \infosedd\ by dividing the entropy estimate by $400$ to report the entropy per site.

\paragraph{Results and Analysis} Variational estimators cannot estimate large \glspl{KL} divergences reliably with limited samples sizes \citep{mcallester2020formal,song2019understanding}. In this scenario, instead, \infosedd\ accurately estimates large \glspl{KL} divergences (\Cref{fig:ising}), performing particularly well at low temperatures where we need to estimate large \glspl{KL} divergences.