Our work introduces \textsc{INFO-SEDD}, a novel method for estimating information-theoretic quantities in discrete data using \glspl{CTMC}. By leveraging a single parametric model, our approach offers computational and memory efficiency while maintaining accuracy in high-dimensional settings. Our experiments highlight \infosedd's robustness against existing neural estimators and demonstrate its effectiveness on both synthetic benchmarks and real-world applications, such as entropy estimation in the Ising model. These results underscore the importance of using specialized techniques for discrete distributions and open new avenues for scalable and accurate information-theoretic analysis in complex systems.

\section{Acknowledgments}
Alberto Foresti, Giulio Franzese and Pietro Michiardi were partially funded by project MUSECOM2 - AI-enabled MUltimodal SEmantic COMmunications and COMputing, in the Machine Learning-based Communication Systems, towards Wireless AI (WAI), Call 2022, ChistERA.

%Estimating \gls{MI} for high-dimensional discrete random variables is a difficult task, which enables scientific discovery and finds widespread application in machine learning. Our work introduced a novel method, which allows estimating entropy and mutual information with a single training run. Crucially, our method does not require a different training strategy than standard generative modeling. This allows to use pretrained models for information theoretical quantities estimation without additional training, enabling easier scientific investigations in established domains such as Natural Language Processing and protein sequencing. Remarkably, with a single training run, our method allows estimating the mutual information of any subset of components of a physical system. This offers an impressive speed-up in scientific investigation, where other methods would require a different training run for each combination of components.



%Our research opens exciting opportunities fo future exploration. By relying on Levy process theory, this framework could be extended to multiple modalities. Additionally, we offer a straightforward path to estimating information theoretical quantities involving more than two random variables in discrete settings, like O-information, which are gaining traction thanks to their expressive power.
