\section{System Usage}
The typical use case for \ac{RAG} systems involves the summarization of retrieved information to produce concise, generalized content. Numerous studies have significantly advanced the fluency of generated language, the comprehensiveness of the covered information, and the ability to extract core insights. However, there is a notable gap in research concerning the verification of the consistency between the generated output and the original source text.

TrustRAG addresses the ExQA task, with the primary goal of generating content that is both accurate and traceable to the original documents, while clearly indicating the sources of the information. This task scenario emphasizes strict adherence to the constraints of the original reference materials during the generation process, ensuring the credibility of the resulting answer. 
% Example use cases of the TrustRAG include the extraction of definitions or background information on specific concepts from authoritative literature or reports, the citation of relevant legal provisions with proper source attribution, and the summarization of public figures’ statements.

To illustrate the application of the TrustRAG, we present a case study focused on news related to climate change. As depicted in Figure \ref{fig:demo}, the system follows three key steps: (I) constructing a knowledge base and uploading relevant documents, (II) configuring the question-answering application, which involves selecting the appropriate knowledge base, choosing the suitable generation model, and specifying the desired output format, and (III) executing the question-answering task within the generated application instance.

As shown in step (III) of Figure \ref{fig:demo}, TrustRAG presents system outputs in a clear, concise, and structured format:
\begin{itemize}
    \item The answer display begins with a brief summary of a few sentences, providing a high-level response to the user's query. For example, in response to the query "How does climate change affects corals?", concise summary is presented at the top. This summary allows the user to quickly understand the overall output of the system.. 
    \item Key aspects of the response are presented separately, each organized under a bolded subheading to facilitate efficient navigation. In this case, the answer is structured into three main aspects: 1. Rising Ocean Temperatures and Coral Bleaching, 2. Prolonged Heat Stress and Coral Death, 3. Impact on Iconic Coral Reefs.
    \item Beneath each subheading, evidence is directly sourced from the retrieved documents, with the system clearly listing each source, including the title of the article, author(s), publication date, and a link to the original material. 
    \item A right-side panel provides detailed information regarding the system's reasoning process, illustrating how the system interprets the user's query intent and the rationale behind selecting relevant information from the knowledge base.
\end{itemize}



