\documentclass[sigconf]{acmart}
%% NOTE that a single column version is required for 
%% submission and peer review. This can be done by changing
%% the \doucmentclass[...]{acmart} in this template to 
%% \documentclass[manuscript,screen]{acmart}
%% 
%% To ensure 100% compatibility, please check the white list of
%% approved LaTeX packages to be used with the Master Article Template at
%% https://www.acm.org/publications/taps/whitelist-of-latex-packages 
%% before creating your document. The white list page provides 
%% information on how to submit additional LaTeX packages for 
%% review and adoption.
%% Fonts used in the template cannot be substituted; margin 
%% adjustments are not allowed.

%%
% basic
%\usepackage{color,xcolor}
\usepackage{color}
\usepackage{epsfig}
\usepackage{graphicx}
\usepackage{algorithm,algorithmic}
% \usepackage{algpseudocode}
%\usepackage{ulem}

% figure and table
\usepackage{adjustbox}
\usepackage{array}
\usepackage{booktabs}
\usepackage{colortbl}
\usepackage{float,wrapfig}
\usepackage{framed}
\usepackage{hhline}
\usepackage{multirow}
% \usepackage{subcaption} % issues a warning with CVPR/ICCV format
% \usepackage[font=small]{caption}
\usepackage[percent]{overpic}
%\usepackage{tikz} % conflict with ECCV format

% font and character
\usepackage{amsmath,amsfonts,amssymb}
% \let\proof\relax      % for ECCV llncs class
% \let\endproof\relax   % for ECCV llncs class
\usepackage{amsthm} 
\usepackage{bm}
\usepackage{nicefrac}
\usepackage{microtype}
\usepackage{contour}
\usepackage{courier}
%\usepackage{palatino}
%\usepackage{times}

% layout
\usepackage{changepage}
\usepackage{extramarks}
\usepackage{fancyhdr}
\usepackage{lastpage}
\usepackage{setspace}
\usepackage{soul}
\usepackage{xspace}
\usepackage{cuted}
\usepackage{fancybox}
\usepackage{afterpage}
%\usepackage{enumitem} % conflict with IEEE format
%\usepackage{titlesec} % conflict with ECCV format

% ref
% commenting these two out for this submission so it looks the same as RSS example
% \usepackage[breaklinks=true,colorlinks,backref=True]{hyperref}
% \hypersetup{colorlinks,linkcolor={black},citecolor={MSBlue},urlcolor={magenta}}
\usepackage{url}
\usepackage{quoting}
\usepackage{epigraph}

% misc
\usepackage{enumerate}
\usepackage{paralist,tabularx}
\usepackage{comment}
\usepackage{pdfpages}
% \usepackage[draft]{todonotes} % conflict with CVPR/ICCV/ECCV format



% \usepackage{todonotes}
% \usepackage{caption}
% \usepackage{subcaption}

\usepackage{pifont}% http://ctan.org/pkg/pifont

% extra symbols
\usepackage{MnSymbol}


\section{Problem Studied}\label{sec:def}
We first present Fixed-Radius Near Neighbor (FRNN) queries and then formalize Aggregation Queries over Nearest Neighbors (AQNNs) that build on them. We then state our problem.

\subsection{Nearest Neighbor Queries}\label{subsec:FRNN}
We build on generalized Fixed-Radius Near Neighbor (FRNN) queries \cite{FRNNSurvey}. Given a dataset \( D \), a query object \( q \), a radius \( r \), and a distance function \( dist \), a generalized FRNN query retrieves all nearest neighbors of \( q \) within radius \( r \). More formally:
\[
NN_D(q, r) = \{x \in D \mid dist(x, q) \leq r\},
\]
where \(x\) is any data point in \(D\) and \(dist(x, q)\) denotes the distance between them. We use \(|NN_D(q,r)|\) to denote the neighborhood size of \(q\). As shown in Fig. \ref{fig:framework}, given a radius \(r\) and a target patient \(q\), patients in the dotted circle are nearest neighbors, and the neighborhood size is 6.

\subsection{Aggregation Queries over Nearest Neighbors}\label{subsec:AQNN} 
Given an FRNN query object \(q\) in dataset \(D\), a radius \(r\), and an attribute \(\texttt{attr}\), an Aggregation Query over Nearest Neighbors (AQNN) is defined as:
\[ \text{agg}(NN_D(q,r)[\texttt{attr}]) \]
where agg is an aggregation function, such as $\mathtt{AVG}$, $\mathtt{SUM}$, and $\mathtt{PCT}$, and \(NN_D(q,r)[\texttt{attr}]\) denotes the bag of values of attribute \texttt{attr} of all FRNN results of \(q\) within radius \(r\). 
% \end{definition}

An AQNN expresses aggregation operations to capture key insights about the neighborhood of a query object. For example, \(\mathtt{AVG}\) can be used to reflect the average heart rate or systolic blood pressure of patients in the neighborhood, providing a measure of typical health conditions. \(\mathtt{SUM}\) is useful for assessing cumulative effects, such as the total cost of treatments in the neighborhood that instructs public policy in terms of health. Similarly, $\mathtt{PCT}$ can be used to find the proportion of patients in the neighborhood of a patient of interest, relative to the population in the dataset.
%\laks{Why is finding the total \#meds to NNs or the total treatment cost of everyone in the NN interesting?}

% \texttt{MIN} and \texttt{MAX} are not included in the aggregation functions because they only capture extreme values, which may not represent the typical characteristics of the nearest neighbors and are more sensitive to outliers. 
% \laks{AVG is also sensitive to outliers, but we still allow it. isn't the real reason we don't consider MIN/MAX because they are amenable to estimation via sampling?} We choose \texttt{PCT} instead of \texttt{COUNT} in order to provide a normalized measure that remains comparable across different neighborhood sizes. It allows for more consistent interpretation of relative popularity \cite{moore1989introduction}.


Fig. \ref{fig:framework} illustrates an example of an AQNN: ``\textit{Find the average systolic blood pressure of patients similar to an insomnia patient \(q\)}''. The aggregation function is \(\mathtt{AVG}\) and the target attribute of interest is systolic blood pressure. Exact query evaluation requires consulting physicians (or predicting embeddings by an expensive machine learning model) for all 500 patients in \(D\) and calculate \(q\)'s nearest neighbors wrt \(r\) \cite{DBLP:journals/isci/RodriguesGSBA21}. We refer to such highly accurate but computationally expensive models as \textit{oracle models}, denoted as \(O\), including deep learning models trained on domain-specific data or human expert annotations \cite{DBLP:conf/sigmod/LuCKC18}. Using oracle models is very expensive \cite{sze2017efficient, DujianPQA, DBLP:journals/pvldb/KangGBHZ20}. To address that, we seek an approximate solution by \textit{proxy models}, denoted as \(P\), that are at least one order of magnitude cheaper than oracle models. In the example, if consulting physicians for one patient incurs one cost unit, calling a cheap machine learning model instead incurs at most \(0.1\) cost unit. Once the similar patients are identified, their systolic blood pressure values are averaged and returned as  output. The use of a proxy model may reduce the accuracy of the neighborhood prediction and hence, we should judiciously call oracle and proxy models to minimize the error of aggregate results.

Note that the values of the target attribute \texttt{attr} are \textit{not} predicted but are instead known quantities.

\subsection{Problem Statement}
Given an AQNN, our goal is to return an approximate aggregate result by leveraging both oracle and proxy models while reducing error and cost.



\author{Coen van den Elsen}
\email{12744956@uva.nl}
\authornote{Equal contributions.}
\affiliation{%
  \institution{University of Amsterdam}
  \city{Amsterdam}
  \state{Noord-Holland}
  \country{The Netherlands}
}

\author{Francien Barkhof}
\email{12606626@uva.nl}
\authornotemark[1]
\affiliation{%
  \institution{University of Amsterdam}
  \city{Amsterdam}
  \state{Noord-Holland}
  \country{The Netherlands}
}

\author{Thijmen Nijdam}
\email{12994448@uva.nl}
\authornotemark[1]
\affiliation{%
  \institution{University of Amsterdam}
  \city{Amsterdam}
  \state{Noord-Holland}
  \country{The Netherlands}
}


\author{Simon Lupart}
\email{s.c.lupart@uva.nl}
\affiliation{%
  \institution{University of Amsterdam}
  \city{Amsterdam}
  \state{Noord-Holland}
  \country{The Netherlands}
}

\author{Mohammad Aliannejadi}
\email{m.aliannejadi@uva.nl}
\affiliation{%
  \institution{University of Amsterdam}
  \city{Amsterdam}
  \state{Noord-Holland}
  \country{The Netherlands}
}

\renewcommand{\shortauthors}{van den Elsen, Barkhof, Nijdam}

\newcommand{\ourshorttitle}{
Spectral-factorized Positive-definite Curvature Learning for NN Training
}
\newcommand{\ourtitle}{
Spectral-factorized Positive-definite Curvature Learning for NN Training
}
\icmltitlerunning{\ourshorttitle}

\twocolumn[

\icmltitle{\ourtitle}

%
%
%
%

%
%
%
%

%
%
%
%

\begin{icmlauthorlist}
\icmlauthor{Wu Lin}{vector}
\icmlauthor{Felix Dangel}{vector}
\icmlauthor{Runa Eschenhagen}{cambridge}
\icmlauthor{Juhan Bae}{vector,ut}
\icmlauthor{Richard E. Turner}{cambridge}
\icmlauthor{
Roger B. Grosse
}{vector,ut}
\end{icmlauthorlist}

%
\icmlaffiliation{vector}{Vector Institute, Canada}
\icmlaffiliation{cambridge}{Cambridge University, United Kingdom}
\icmlaffiliation{ut}{University of Toronto, Canada}

\icmlcorrespondingauthor{Wu Lin}{yorker.lin@gmail.com \vspace{-0.2cm}}

%
%
%
\icmlkeywords{Natural Gradient Descent, Second Order Method, Optimization, Deep Learning}

\vskip 0.3in
]

%

%
%
%
%
%

\printAffiliationsAndNotice{}  %
%

%
%
%
%


%% Rights management information.  This information is sent to you
%% when you complete the rights form.  These commands have SAMPLE
%% values in them; it is your responsibility as an author to replace
%% the commands and values with those provided to you when you
%% complete the rights form.
\setcopyright{acmlicensed}
\copyrightyear{2018}
\acmYear{2018}
\acmDOI{XXXXXXX.XXXXXXX}

%% These commands are for a PROCEEDINGS abstract or paper.
%\acmConference[Conference acronym 'XX]{Make sure to enter the correct
%  conference title from your rights confirmation emai}{June 03--05,
%  2018}{Woodstock, NY}
%
%  Uncomment \acmBooktitle if th title of the proceedings is different
%  from ``Proceedings of ...''!
%
%\acmBooktitle{Woodstock '18: ACM Symposium on Neural Gaze Detection,
%  June 03--05, 2018, Woodstock, NY} 
\acmISBN{978-1-4503-XXXX-X/18/06}

%%
%% end of the preamble, start of the body of the document source.
\begin{document}

%%
%% The "title" command has an optional parameter,
%% allowing the author to define a "short title" to be used in page headers.
\title{TrustRAG: An Information Assistant with Retrieval Augmented Generation}

%% The abstract is a short summary of the work to be presented in the
%% article.
\begin{abstract}
 \Ac{RAG} has emerged as a crucial technique for enhancing large models with real-time and domain-specific knowledge. While numerous improvements and open-source tools have been proposed to refine the \ac{RAG} framework for accuracy, relatively little attention has been given to improving the trustworthiness of generated results. To address this gap, we introduce TrustRAG, a novel framework that enhances \ac{RAG} from three perspectives: indexing, retrieval, and generation. Specifically, in the indexing stage, we propose a semantic-enhanced chunking strategy that incorporates hierarchical indexing to supplement each chunk with contextual information, ensuring semantic completeness. In the retrieval stage, we introduce a utility-based filtering mechanism to identify high-quality information, supporting answer generation while reducing input length. In the generation stage, we propose fine-grained citation enhancement, which detects opinion-bearing sentences in responses and infers citation relationships at the sentence-level, thereby improving citation accuracy. 
We open-source the TrustRAG framework and provide a demonstration studio designed for excerpt-based question answering tasks \footnote{https://huggingface.co/spaces/golaxy/TrustRAG}. Based on these, we aim to help researchers: 1) systematically enhancing the trustworthiness of \ac{RAG} systems and (2) developing their own \ac{RAG} systems with more reliable outputs.


\end{abstract}

%%
%% The code below is generated by the tool at http://dl.acm.org/ccs.cfm.
%% Please copy and paste the code instead of the example below.
%%
% \begin{CCSXML}
% <ccs2012>
%  <concept>
%   <concept_id>00000000.0000000.0000000</concept_id>
%   <concept_desc>Do Not Use This Code, Generate the Correct Terms for Your Paper</concept_desc>
%   <concept_significance>500</concept_significance>
%  </concept>
%  <concept>
%   <concept_id>00000000.00000000.00000000</concept_id>
%   <concept_desc>Do Not Use This Code, Generate the Correct Terms for Your Paper</concept_desc>
%   <concept_significance>300</concept_significance>
%  </concept>
%  <concept>
%   <concept_id>00000000.00000000.00000000</concept_id>
%   <concept_desc>Do Not Use This Code, Generate the Correct Terms for Your Paper</concept_desc>
%   <concept_significance>100</concept_significance>
%  </concept>
%  <concept>
%   <concept_id>00000000.00000000.00000000</concept_id>
%   <concept_desc>Do Not Use This Code, Generate the Correct Terms for Your Paper</concept_desc>
%   <concept_significance>100</concept_significance>
%  </concept>
% </ccs2012>
% \end{CCSXML}

% \ccsdesc[500]{Do Not Use This Code~Generate the Correct Terms for Your Paper}
% \ccsdesc[300]{Do Not Use This Code~Generate the Correct Terms for Your Paper}
% \ccsdesc{Do Not Use This Code~Generate the Correct Terms for Your Paper}
% \ccsdesc[100]{Do Not Use This Code~Generate the Correct Terms for Your Paper}

%%
%% Keywords. The author(s) should pick words that accurately describe
%% the work being presented. Separate the keywords with commas.
\keywords{Retrieval-augmented Generation, TrustRAG}



%%
%% This command processes the author and affiliation and title
%% information and builds the first part of the formatted document.
\maketitle

\section{Introduction}

Despite the remarkable capabilities of large language models (LLMs)~\cite{DBLP:conf/emnlp/QinZ0CYY23,DBLP:journals/corr/abs-2307-09288}, they often inevitably exhibit hallucinations due to incorrect or outdated knowledge embedded in their parameters~\cite{DBLP:journals/corr/abs-2309-01219, DBLP:journals/corr/abs-2302-12813, DBLP:journals/csur/JiLFYSXIBMF23}.
Given the significant time and expense required to retrain LLMs, there has been growing interest in \emph{model editing} (a.k.a., \emph{knowledge editing})~\cite{DBLP:conf/iclr/SinitsinPPPB20, DBLP:journals/corr/abs-2012-00363, DBLP:conf/acl/DaiDHSCW22, DBLP:conf/icml/MitchellLBMF22, DBLP:conf/nips/MengBAB22, DBLP:conf/iclr/MengSABB23, DBLP:conf/emnlp/YaoWT0LDC023, DBLP:conf/emnlp/ZhongWMPC23, DBLP:conf/icml/MaL0G24, DBLP:journals/corr/abs-2401-04700}, 
which aims to update the knowledge of LLMs cost-effectively.
Some existing methods of model editing achieve this by modifying model parameters, which can be generally divided into two categories~\cite{DBLP:journals/corr/abs-2308-07269, DBLP:conf/emnlp/YaoWT0LDC023}.
Specifically, one type is based on \emph{Meta-Learning}~\cite{DBLP:conf/emnlp/CaoAT21, DBLP:conf/acl/DaiDHSCW22}, while the other is based on \emph{Locate-then-Edit}~\cite{DBLP:conf/acl/DaiDHSCW22, DBLP:conf/nips/MengBAB22, DBLP:conf/iclr/MengSABB23}. This paper primarily focuses on the latter.

\begin{figure}[t]
  \centering
  \includegraphics[width=0.48\textwidth]{figures/demonstration.pdf}
  \vspace{-4mm}
  \caption{(a) Comparison of regular model editing and EAC. EAC compresses the editing information into the dimensions where the editing anchors are located. Here, we utilize the gradients generated during training and the magnitude of the updated knowledge vector to identify anchors. (b) Comparison of general downstream task performance before editing, after regular editing, and after constrained editing by EAC.}
  \vspace{-3mm}
  \label{demo}
\end{figure}

\emph{Sequential} model editing~\cite{DBLP:conf/emnlp/YaoWT0LDC023} can expedite the continual learning of LLMs where a series of consecutive edits are conducted.
This is very important in real-world scenarios because new knowledge continually appears, requiring the model to retain previous knowledge while conducting new edits. 
Some studies have experimentally revealed that in sequential editing, existing methods lead to a decrease in the general abilities of the model across downstream tasks~\cite{DBLP:journals/corr/abs-2401-04700, DBLP:conf/acl/GuptaRA24, DBLP:conf/acl/Yang0MLYC24, DBLP:conf/acl/HuC00024}. 
Besides, \citet{ma2024perturbation} have performed a theoretical analysis to elucidate the bottleneck of the general abilities during sequential editing.
However, previous work has not introduced an effective method that maintains editing performance while preserving general abilities in sequential editing.
This impacts model scalability and presents major challenges for continuous learning in LLMs.

In this paper, a statistical analysis is first conducted to help understand how the model is affected during sequential editing using two popular editing methods, including ROME~\cite{DBLP:conf/nips/MengBAB22} and MEMIT~\cite{DBLP:conf/iclr/MengSABB23}.
Matrix norms, particularly the L1 norm, have been shown to be effective indicators of matrix properties such as sparsity, stability, and conditioning, as evidenced by several theoretical works~\cite{kahan2013tutorial}. In our analysis of matrix norms, we observe significant deviations in the parameter matrix after sequential editing.
Besides, the semantic differences between the facts before and after editing are also visualized, and we find that the differences become larger as the deviation of the parameter matrix after editing increases.
Therefore, we assume that each edit during sequential editing not only updates the editing fact as expected but also unintentionally introduces non-trivial noise that can cause the edited model to deviate from its original semantics space.
Furthermore, the accumulation of non-trivial noise can amplify the negative impact on the general abilities of LLMs.

Inspired by these findings, a framework termed \textbf{E}diting \textbf{A}nchor \textbf{C}ompression (EAC) is proposed to constrain the deviation of the parameter matrix during sequential editing by reducing the norm of the update matrix at each step. 
As shown in Figure~\ref{demo}, EAC first selects a subset of dimension with a high product of gradient and magnitude values, namely editing anchors, that are considered crucial for encoding the new relation through a weighted gradient saliency map.
Retraining is then performed on the dimensions where these important editing anchors are located, effectively compressing the editing information.
By compressing information only in certain dimensions and leaving other dimensions unmodified, the deviation of the parameter matrix after editing is constrained. 
To further regulate changes in the L1 norm of the edited matrix to constrain the deviation, we incorporate a scored elastic net ~\cite{zou2005regularization} into the retraining process, optimizing the previously selected editing anchors.

To validate the effectiveness of the proposed EAC, experiments of applying EAC to \textbf{two popular editing methods} including ROME and MEMIT are conducted.
In addition, \textbf{three LLMs of varying sizes} including GPT2-XL~\cite{radford2019language}, LLaMA-3 (8B)~\cite{llama3} and LLaMA-2 (13B)~\cite{DBLP:journals/corr/abs-2307-09288} and \textbf{four representative tasks} including 
natural language inference~\cite{DBLP:conf/mlcw/DaganGM05}, 
summarization~\cite{gliwa-etal-2019-samsum},
open-domain question-answering~\cite{DBLP:journals/tacl/KwiatkowskiPRCP19},  
and sentiment analysis~\cite{DBLP:conf/emnlp/SocherPWCMNP13} are selected to extensively demonstrate the impact of model editing on the general abilities of LLMs. 
Experimental results demonstrate that in sequential editing, EAC can effectively preserve over 70\% of the general abilities of the model across downstream tasks and better retain the edited knowledge.

In summary, our contributions to this paper are three-fold:
(1) This paper statistically elucidates how deviations in the parameter matrix after editing are responsible for the decreased general abilities of the model across downstream tasks after sequential editing.
(2) A framework termed EAC is proposed, which ultimately aims to constrain the deviation of the parameter matrix after editing by compressing the editing information into editing anchors. 
(3) It is discovered that on models like GPT2-XL and LLaMA-3 (8B), EAC significantly preserves over 70\% of the general abilities across downstream tasks and retains the edited knowledge better.


\section{System Overview}
The architecture of the system is shown in the Figure \ref{fig:architecture}. The system consists of two major components, namely the TrustRAG library and the TrustRAG studio. 

The library functions as the system's back-end, offering a comprehensive set of features for all stages of the \ac{RAG} pipeline. Its capabilities are structured into three modular components: the offline indexing module, the retrieval module, and the generation module.
First, the offline indexing module provide rich parsing functions for different kinds of files (e.g., PDF, Word, Excel, JSON) and converts chunked content into embeddings. 
Second, the retrieval module operates in three stages: query processing, retrieval, and utility assessment. 
Finally, the generation module also follows a three-stage process, comprising basic generation, citation integration, and post-processing.

The studio serves as the system's front-end, offering a user-friendly GUI built on the TrustRAG library. It features two main panels: \textbf{the knowledge manage panel} and \textbf{the conversation manage panel}. In the knowledge manage panel, users can upload their own documents, configure processing options, and select the indexing method. In the conversation manage panel, users can choose search method and the \ac{LLM} for each conversation. Additionally, the studio visualize the intermediate ``thinking'' process of the TrustRAG, including query understanding, document selection, answer reasoning, and sentence citation, to enhance reliability and transparency. 
%The Studio serves as the system's front-end, offering a user-friendly GUI built on the TrustRAG library. It features two main panels: the Knowledge Management Panel and the Conversation Management Panel.








 
 %provides a user-friendly GUI so that ordinary users can lean, practice, and develop neural text matching models easily. It contains three unified API for user accessing
 
%All components in the architecture are stand-alone and interact with each other via HTTP API. For example, any search engine can be added to our system with minimal effort. This design also adds flexibility to these components and permits their use in different applications. For instance, the platform server was also used in the TREC 2017 RTS track evaluation for collecting judgments of tweets from the assessors [6]. The CAL component also has a command line interface through which various simulation experiments can be conducted.

%The platform server is built using Django, a Python web framework. CAL and Search are written in C++. The source code is publicly available.

\section{TrustRAG Library}
TrustRAG is a configurable and modular Retrieval-Augmented Generation (RAG) framework designed for "reliable input and trustworthy output." It consists of key components such as document parsing, text chunking, query optimization, retrieval ranking, content compression, model generation, and answer citation. This section highlights its innovations in semantic enhancement, usefulness enhancement, and citation enhancement.

\begin{figure}[!t]
\centering
\includegraphics[width=0.5\textwidth]{figures/TrustRAG_Library.pdf}
\caption{An overview of the TrustRAG framework.}
\label{fig:trustrag_overview}
\end{figure}

% \lstset{
    basicstyle=\ttfamily\small,
    breaklines=true,
    frame=single,
    caption={The TrustRAG Library},
    label=lst:directory_structure
}

\begin{lstlisting}
TrustRAG/
|--api
|  |--apps
|      |--config
|      |--core
|      |  |--citation
|      |  |--judge
|      |  |--parser
|      |  |--refiner
|      |  |--rerank
|      |  |--rewrite
|      |--handle
|      |  |--exception
|      |  |--response
|--docs
|--examples
|--notebooks
|--resources
|--tests
|--trustrag
|    |--applications
|    |--modules

\end{lstlisting}
% |    |  |--chunks
% |    |  |--citation
% |    |  |--clusters
% |    |  |--document
% |    |  |--engine
% |    |  |--generator
% |    |  |--judger
% |    |  |--postprocess
% |    |  |--processor
% |    |  |--prompt
% |    |  |--refiner
% |    |  |--reranker
% |    |  |--retrieval
% |    |  |--rewriter

\subsection{Semantic-Enhanced Indexing}
Existing text chunking methods, while efficient, often lead to significant semantic loss, particularly when handling long or complex documents~\cite{sarthi2024raptor}. Simple character-based or paragraph-based splitting can disrupt contextual coherence, making it difficult for downstream retrieval and generation tasks to fully utilize the semantic information embedded in the text. 

To address this issue, TrustRAG introduces the \textit{semantic-enhanced chunking} to improve the semantic integrity and coherence for each chunk. Specifically, we firstly take the \ac{LLM} to apply co-reference resolution for each document, which resolves ambiguities caused by pronouns or incomplete references. For instance, when a pronoun like "it" appears in a sentence, the system identifies its antecedent and restores the missing context, thereby enhancing the semantic completeness of the text. This process not only recovers lost semantic information but also provides more accurate contextual support for subsequent generation tasks.
Moreover, we standardize the time fields in the document by converting the relative time references into standardize date formats based on the document publication date. For example, if the document's publication date is ``2025-02-18'', terms like ``yesterday'' and ``last Friday'' will be converted to ``2025-02-17'' and ``2025-02-14'', respectively. 
%To address this issue, TrustRAG introduces \textbf{context decontextualization}, a technique that ensures semantic integrity and coherence in text chunks. By leveraging coreference resolution and semantic completion techniques, TrustRAG resolves ambiguities caused by pronouns or incomplete references. For instance, when a pronoun like "it" appears in a sentence, the system identifies its antecedent and restores the missing context, thereby enhancing the semantic completeness of the text. 
This process not only recovers lost semantic information but also provides more accurate contextual support for subsequent generation tasks. The implementation of this feature can be found in \texttt{trustrag/modules/refiner/decontextualizer.py}.

Furthermore, TrustRAG supports advanced semantic segmentation techniques that dynamically identify semantic boundaries using embedding technologies and large language models (LLMs). Unlike static chunking methods, these techniques allow the system to adaptively split text based on its semantic structure, ensuring higher-quality chunks that preserve contextual coherence. The code is available in \texttt{trustrag/modules/chunks/semantic\_chunk.py}. These innovations improve the quality of text indexing, laying a solid foundation for reliable retrieval and generation.

\subsection{Utility-Enhanced Retrieval}
In conventional RAG systems, the relevance of retrieved documents is often determined solely by vector similarity. However, high similarity does not always translate to usefulness for the generation task. In some cases, even irrelevant documents may inadvertently improve system accuracy, highlighting the need for more intelligent mechanisms to evaluate the utility of retrieved results~\cite{cuconasu2024power}.
TrustRAG addresses this limitation by introducing two key innovations: \textbf{usefulness judgement} and \textbf{fine-grained evidence extraction}.
\begin{itemize}
    \item \textbf{Usefulness Judgement:} TrustRAG employs large language models (LLMs) as discriminators to assess the utility of retrieved documents. Through carefully designed prompts, the system evaluates the relevance of each document to the user's query and the generation task. This evaluation goes beyond surface-level similarity, incorporating deeper contextual understanding to ensure that only the most useful documents are selected. See \texttt{trustrag/modules/judger/llm\_judger.py}.
    
    \item \textbf{Fine-Grained Evidence Extraction:} After identifying useful documents, TrustRAG extracts the most relevant sentences through fine-grained evidence extraction. This process leverages model distillation techniques to reduce computational costs while maintaining high accuracy and relevance. By focusing on the most pertinent information, the system ensures that the generation task receives high-quality inputs. See \texttt{trustrag/modules/refiner/compressor.py}.
\end{itemize}
These enhancements enable TrustRAG to prioritize truly useful information, improving the overall quality and reliability of the retrieval process.

\subsection{Attribution-Enhanced Generation}
The credibility and traceability of generated answers are critical for trustworthiness in RAG systems. Traditional approaches rely heavily on direct reasoning by large models, which can be slow and prone to inaccuracies in citations. Additionally, fine-tuning models to improve citation accuracy may compromise their performance on other tasks, limiting practical applicability.

TrustRAG overcomes these challenges through two key innovations: \textbf{post-generation citation} and \textbf{citation grouping with cross-referencing}.
\begin{itemize}
    \item \textbf{Post-Generation Citation:} Instead of embedding citations during the generation process, TrustRAG matches the generated answers with retrieved reference materials afterward. This approach ensures higher citation accuracy while significantly accelerating the generation process. See \texttt{trustrag/mod
    ules/citation/match\_citation.py}.
    
    \item \textbf{Citation Grouping and Cross-Referencing:} To enhance traceability, TrustRAG organizes citations into logical groups, providing users with clearer reference sources. Furthermore, the system supports cross-referencing, allowing it to establish connections between different citations. This feature not only improves the clarity of references but also strengthens the credibility of the generated answers. See \texttt{trustrag/modules/
    citation/source\_citation.py}.
\end{itemize}
These innovations ensure that TrustRAG delivers both accurate and traceable answers, addressing key limitations of traditional RAG systems.

\begin{figure*}[!t]
\centering
\includegraphics[scale=0.6]{./figures/demo.png}
\caption{Example usage of TrustRAG on Excerpt-based Questions}
\label{fig:demo}
\end{figure*}

\subsection{Additional Modules}
Beyond the three core enhancements, TrustRAG offers a rich set of modular functionalities, each designed to support specific aspects of the RAG pipeline:
\begin{itemize}
    \item \textbf{Document Parsing:} TrustRAG supports efficient parsing of multiple file formats, including PDF, Word, and HTML, with robust cross-language capabilities. This module ensures high parsing success rates and seamless integration with multilingual content. See \texttt{trustrag/modules/document} for implementation details.
    
    \item \textbf{Text Chunking:} From basic to advanced chunking methods, this module ensures semantic consistency and coherence in text segmentation. See \texttt{trustrag/modules/chunks}.
    
    \item \textbf{Query Optimization:} TrustRAG enhances query accuracy and efficiency through techniques such as query expansion, decomposition, disambiguation, and abstraction. These methods help refine user queries, improving the quality of retrieval results. See \texttt{trustrag/mod
    ules/rewriter}.
    
    \item \textbf{Retrieval Ranking:} By employing multi-path retrieval and fusion-based re-ranking, TrustRAG ensures high relevance and reliability in retrieval results. See \texttt{trustrag/modules/
    reranker}.
    
    \item \textbf{Content Compression:} This module extracts high-value information from large volumes of retrieved content using usefulness evaluation and semantic enhancement techniques. See \texttt{trustrag/modules/refiner}.
    
    \item \textbf{Model Generation:} TrustRAG supports flexible invocation of various large language models, providing efficient inference and indexing management capabilities. See \texttt{trustrag/
    modules/generator}.
\end{itemize}

Through these comprehensive innovations, TrustRAG significantly enhances the reliability and trustworthiness of RAG systems. Its modular and configurable design empowers users to tailor the framework to diverse application scenarios, delivering high-quality and trustworthy outputs across a wide range of use cases.



\section{System Usage}
The typical use case for \ac{RAG} systems involves the summarization of retrieved information to produce concise, generalized content. Numerous studies have significantly advanced the fluency of generated language, the comprehensiveness of the covered information, and the ability to extract core insights. However, there is a notable gap in research concerning the verification of the consistency between the generated output and the original source text.

TrustRAG addresses the ExQA task, with the primary goal of generating content that is both accurate and traceable to the original documents, while clearly indicating the sources of the information. This task scenario emphasizes strict adherence to the constraints of the original reference materials during the generation process, ensuring the credibility of the resulting answer. 
% Example use cases of the TrustRAG include the extraction of definitions or background information on specific concepts from authoritative literature or reports, the citation of relevant legal provisions with proper source attribution, and the summarization of public figures’ statements.

To illustrate the application of the TrustRAG, we present a case study focused on news related to climate change. As depicted in Figure \ref{fig:demo}, the system follows three key steps: (I) constructing a knowledge base and uploading relevant documents, (II) configuring the question-answering application, which involves selecting the appropriate knowledge base, choosing the suitable generation model, and specifying the desired output format, and (III) executing the question-answering task within the generated application instance.

As shown in step (III) of Figure \ref{fig:demo}, TrustRAG presents system outputs in a clear, concise, and structured format:
\begin{itemize}
    \item The answer display begins with a brief summary of a few sentences, providing a high-level response to the user's query. For example, in response to the query "How does climate change affects corals?", concise summary is presented at the top. This summary allows the user to quickly understand the overall output of the system.. 
    \item Key aspects of the response are presented separately, each organized under a bolded subheading to facilitate efficient navigation. In this case, the answer is structured into three main aspects: 1. Rising Ocean Temperatures and Coral Bleaching, 2. Prolonged Heat Stress and Coral Death, 3. Impact on Iconic Coral Reefs.
    \item Beneath each subheading, evidence is directly sourced from the retrieved documents, with the system clearly listing each source, including the title of the article, author(s), publication date, and a link to the original material. 
    \item A right-side panel provides detailed information regarding the system's reasoning process, illustrating how the system interprets the user's query intent and the rationale behind selecting relevant information from the knowledge base.
\end{itemize}




\section{Conclusion}
In this demo, we introduce a novel TrustRAG system for risk-aware information seeking scenarios. Users can build their own \ac{RAG} applications with private corpus, study the \ac{RAG} components within the library, and experiment with \ac{RAG} library with customized modules. 
We will showcase the TrustRAG system through the following aspects: (1) we will use a poster to give an overview of the system and briefly show the pipeline of the framework. (2) We will demonstrate how to use the system to create \ac{RAG} application with a private corpus. (3) We will share insights on system's strengths and limitations, along with potential future enhancements.
%We will present our system in the following aspects: (1) We will use a poster to give an overview of system architecture and briefly show the stages of the neural text matching process as well as the system components. (2) We will show the audience how to use the system to complete an example of text matching task, including data set pre-processing, model configuration, train, and test. (3) We will give a brief introduction of the neural text matching models in the system. (4) We will share our thoughts on the strengths and weakness of the system, and further discuss the future work.


%%
%% The next two lines define the bibliography style to be used, and
%% the bibliography file.
\bibliographystyle{ACM-Reference-Format}
\bibliography{sample-base}

\end{document}
\endinput
%%
%% End of file `sample-sigconf.tex'.
