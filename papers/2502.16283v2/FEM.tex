\documentclass[12pt, a4paper]{article}

\usepackage[utf8]{inputenc}
\usepackage{fixltx2e}

% Adjusting page margins
\usepackage{geometry}
\geometry{
 a4paper,
 total={170mm,257mm},
 left=20mm,
 top=20mm,
 }

\usepackage{graphicx}
\usepackage{amsmath}
\DeclareMathOperator{\Tr}{Tr}
\usepackage{fancybox}
\usepackage[bottom]{footmisc}

\usepackage{algorithm2e}

\usepackage[
backend=biber,
style=numeric-comp,
sorting=none,
]{biblatex}

\addbibresource{refs.bib}

\usepackage{hyperref}
\hypersetup{
    colorlinks=true,
    linkcolor=black,
    citecolor=blue,
    urlcolor=blue,
}
\usepackage{xcolor}
\usepackage{sectsty,titling}
\sectionfont{\color{magenta}}
\subsectionfont{\color{magenta}}
\subsubsectionfont{\color{magenta} \normalfont\itshape}

\title{Finite Element Theory for PHIMATS}

\date{February 2025}

\author{Abdelrahman Hussein \\ email \href{mailto:a.h.a.hussein@outlook.com}{a.h.a.hussein@outlook.com}, \href{mailto:abdelrahman.hussein@oulu.fi}{abdelrahman.hussein@oulu.fi}} 

%Set font 
\renewcommand{\familydefault}{\sfdefault}
\definecolor{sub}{HTML}{cde4ff}     % setting sub color to be used
\definecolor{main}{HTML}{5989cf}    % setting main color to be used
\usepackage[many]{tcolorbox}    	% for COLORED BOXES (tikz and xcolor included)

\newcounter{myBoxCounter}[section]
\renewcommand{\themyBoxCounter}{\thesection.\arabic{myBoxCounter}}


\newtcolorbox[auto counter, number within=section]{boxH}[2][]{
    title=Box~\themyBoxCounter: #2,
    colback = sub, 
    colframe = main, 
    boxrule = 0pt, 
    before skip=20pt plus 2pt,
    after skip=20pt plus 2pt,
    #1,
}

% Increment the counter manually
\newcommand{\nextboxnumber}{\stepcounter{myBoxCounter}}

% Numbering equations with sections
\numberwithin{equation}{section}

\usepackage{bm}

\newcommand{\commentedbox}[2]{%
  \mbox{
    \begin{tabular}[t]{@{}c@{}}
    $\boxed{\displaystyle#1}$\\
    #2
    \end{tabular}%
  }%
}

\begin{document}

\maketitle

\noindent\rule{\textwidth}{1pt}

\vspace{1cm}

This document summarizes the main ideas of the finite element method theory and constitutive relations as implemented in the code PHIMATS (\href{https://github.com/ahcomat/PHIMATS.git}{GitHub Repository}). Rather than detailing the derivations or specific models, this document focuses on the key mathematical foundations and numerical strategies used within the implementation. For in-depth theoretical discussions, the reader is encouraged to consult the references. For citing this document, please use: [Abdelrahman Hussein. Finite Element Theory for PHIMATS. 2025. doi: 10.48550/ARXIV.
2502.16283]. Hands-on examples can be found in CaseStudies directory on the GitHub repository.

\vspace{0.75cm}

\noindent \rule{17cm}{1pt}

\tableofcontents

\pagebreak

\section{Isoparametric formulation}

The geometry of a domain can be discretized by building blocks called elements, that are defined by the nodes forming these elements and their coordinates. The unknown variables of interest are calculated at these nodes, i.e. nodal unknowns or \emph{degrees of freedom (DOFs)}. These unknowns can be interpolated at any location within an element using the so called (interpolation) shape functions. Similarly, shape functions can be used to interpolate the geometry of the element. For the \emph{isoparametric formulation}, the same shape functions are used to interpolate both the geometry and the nodal unknowns \cite{Biner2017}. 

The shape functions $N_i(\boldsymbol{\xi})$ are constructed to have the value of 1 for a given node and zero everywhere else. $\boldsymbol{\xi}$ is a non-dimensional coordinate system called \emph{local} or \emph{element} coordinate system, which vary in the range [-1,1]. We will use a 4-node quad element as an example for $N_i(\boldsymbol{\xi})$. For a degree of freedom $f$ with known values at the element nodes $f_1, f_2, f_3, f_4$, the value at any point $\boldsymbol{\xi}$ within the element are given by 

\begin{equation}
    f = \sum_i^n N_i(\boldsymbol{\xi}) f_i = N_1(\boldsymbol{\xi}) f_1 + N_2(\boldsymbol{\xi}) f_2 + N_3(\boldsymbol{\xi}) f_3 + N_4(\boldsymbol{\xi}) f_4
\end{equation}

\noindent Where $n$ is the number of nodes\footnote{$N_i=\mathbf{N}$ will be represented as a row vector. And since all the shape functions $N_i(\boldsymbol{\xi})$ have to be evaluated for a given Gauss point $\boldsymbol{\xi}$, they will be stored for each element as $\texttt{nGauss} \times \texttt{nElNodes}$} per element.

\nextboxnumber
\begin{boxH}[label=ShapeFunctions]{Example of shape functions for a 4-node quadrilateral element}
    \begin{equation*}
        \begin{split}
        N_1 & = 0.25(1 - \xi)(1 - \eta) \\
        N_2 & = 0.25(1 + \xi)(1 - \eta) \\
        N_3 & = 0.25(1 + \xi)(1 + \eta) \\
        N_4 & = 0.25(1 - \xi)(1 + \eta) \\
        \end{split}
    \end{equation*}
\end{boxH}

\noindent Similarly, any point $\boldsymbol{\xi}$ within the element can be mapped to \emph{cartesian} coordinates by 

\begin{equation}
    \begin{split}
    \mathbf{x(\xi)} &= \sum_i^n N_i(\boldsymbol{\xi}) \mathbf{x}_i
    \end{split}
    \label{Eq:CoordsMapping}
\end{equation}

\noindent Where $\mathbf{x}_i$ are the cartesian coordinates of the nodes. 

\subsection{Derivatives}

The cartesian derivatives\footnote{Note this is a matrix of dimensions $\texttt{nDim} \times \texttt{nElNodes}$.} of the variable $f$ at any point $\boldsymbol{\xi}$ defined at the element nodes can be evaluated as

\begin{equation}
    \begin{split}
    \frac{\partial f(\boldsymbol{\xi})}{\partial \mathbf{x}} &= \sum_i^n \frac{\partial N_i(\boldsymbol{\xi})}{\partial \mathbf{x}}f_i^\top \\
    &= \left[
        \begin{array}{rrrr}
        \frac{\partial N_1(\boldsymbol{\xi})}{\partial x} & \dots & \frac{\partial N_n(\boldsymbol{\xi})}{\partial x} \\
        \frac{\partial N_1(\boldsymbol{\xi})}{\partial y} & \dots & \frac{\partial N_n(\boldsymbol{\xi})}{\partial y} \\
        \frac{\partial N_1(\boldsymbol{\xi})}{\partial z} & \dots & \frac{\partial N_n(\boldsymbol{\xi})}{\partial z} \\
        \end{array}
        \right] \left[
        \begin{array}{r}
        f_1 \\
        \vdots \\
        f_n \\
        \end{array}
        \right] \\
    \end{split}
    \label{Eq:IsoparamDeriv}
\end{equation}

\noindent Using the chain rule

\begin{equation}
    \sum_i^n \frac{\partial N_i(\boldsymbol{\xi})}{\partial \mathbf{x}} = \sum_i^n \frac{\partial N_i(\boldsymbol{\xi})}{\partial \boldsymbol{\xi}} \frac{\partial \boldsymbol{\xi}}{\partial \mathbf{x}}
    \label{Eq:ShapeDeriv}
\end{equation}

\noindent $\frac{\partial \mathbf{x}(\boldsymbol{\xi})}{\partial \boldsymbol{\xi}} = \frac{\partial x_i(\boldsymbol{\xi})}{\partial \xi_j}$ is called the jacobian matrix $\mathbf{J}(\boldsymbol{\xi})$ with components

\begin{equation}
    \mathbf{J}(\boldsymbol{\xi}) = \left[\begin{array}{rrr}
        \frac{\partial x}{\partial \xi} & \frac{\partial x}{\partial \eta} & \frac{\partial x}{\partial \zeta} \\
        \frac{\partial y}{\partial \xi} & \frac{\partial y}{\partial \eta} & \frac{\partial y}{\partial \zeta} \\
        \frac{\partial z}{\partial \xi} & \frac{\partial z}{\partial \eta} & \frac{\partial z}{\partial \zeta} \\
        \end{array}\right]
\end{equation}

\noindent Using Eq. (\ref{Eq:CoordsMapping})

\begin{equation}
    \begin{split}
    \mathbf{J}(\boldsymbol{\xi}) &= \sum_i^n \frac{\partial N_i(\boldsymbol{\xi})}{\partial \boldsymbol{\xi}} \mathbf{x_i}^\top \\
    &= \left[
        \begin{array}{rrrr}
        \frac{\partial N_1(\boldsymbol{\xi})}{\partial x} & \dots & \frac{\partial N_n(\boldsymbol{\xi})}{\partial x} \\
        \frac{\partial N_1(\boldsymbol{\xi})}{\partial y} & \dots & \frac{\partial N_n(\boldsymbol{\xi})}{\partial y} \\
        \frac{\partial N_1(\boldsymbol{\xi})}{\partial z} & \dots & \frac{\partial N_n(\boldsymbol{\xi})}{\partial z} \\
        \end{array}
        \right] \left[
        \begin{array}{rrr}
        x_1 & y_1 & z_1 \\
        \vdots \\
        x_n & y_n & z_n \\
        \end{array}\right]
    \end{split}
\end{equation}

\noindent Substituting in Eq. (\ref{Eq:ShapeDeriv})

\begin{equation}
    \sum_i^n\frac{\partial N_i(\boldsymbol{\xi})
    }{\partial \mathbf{x}} = J(\boldsymbol{\xi})^{-1} \sum_i^n\frac{\partial N_i(\boldsymbol{\xi})}{\partial \boldsymbol{\xi}} 
\end{equation}

\noindent It is worth noting that the volume/area mapping can be obtained using 

\begin{equation}
    d\mathbf{x} = \det[\mathbf{J(\boldsymbol{\xi})}] d\boldsymbol{\xi} 
\end{equation}

\subsection{Numerical integration}

\noindent Gauss quadrature is used for integration over the element domain (in local coordinates) of the form

\begin{equation}
    \begin{split}
    \int_{-1}^{+1}\int_{-1}^{+1}\int_{-1}^{+1}f(\boldsymbol{\xi})d\boldsymbol{\xi} = \int_{\Omega_e} f(\boldsymbol{\xi})d\boldsymbol{\xi} &\approx \sum^M_{I=1}\sum^M_{J=1}\sum^M_{K=1} w_Iw_Jw_K f(\boldsymbol{\xi}) \\
    &\approx \sum^M_{I=1}\sum^M_{J=1}\sum^M_{K=1} w_Iw_Jw_K f(\xi_I,\eta_J,\zeta_K) \\
    \end{split}
\end{equation} 

\noindent Where $w_{I,J,K}$ is the weight of the the integration point $\boldsymbol{\xi}$. Therefore, the volume integral over all the element can be taken as the summation of volume integral of all its gauss points.

\section{The weak form}
\label{Sec:WeakForm}

The FEM method for solving a PDE starts with forming a variational principle transforming the \emph{strong form} to \emph{weighted-residual} or a \emph{weak form}. This "weak form" transforms the PDE problem to a minimization problem. This involves the following steps \cite{Biner2017}:

\begin{enumerate}
    \item Multiply the PDE with a \emph{test} function. In the Galerkin FEM, the test function is the "trial" function. 
    \item Integrate ove the whole domain $\Omega$.
    \item Integrate by parts the terms involving spatial gradients using the divergence theorem.
    \item Provide the given boundary conditions.
  \end{enumerate}

\noindent A general case for integration by parts

\begin{equation}
    \int_\Omega \psi (\nabla \cdot \nabla u) d\Omega = -\int_\Omega \nabla \psi \cdot \nabla u d\Omega + \int_{\Gamma} \psi (\nabla u \cdot \mathbf{n}) d\Gamma
\end{equation}

\noindent For example, consider the PDE

\begin{equation}
    \frac{\partial C}{\partial t} + \nabla \cdot D(\mathbf{x}) \nabla C = 0
    \label{Eq:Example}
\end{equation}

\noindent Step(1): Multiply Eq. (\ref{Eq:Example}) by a test function $\psi$

\begin{equation}
    \psi (\frac{\partial C}{\partial t}) + \psi (\nabla \cdot D(\mathbf{x}) \nabla C) = 0
    % \label{Eq:TestFunction}
\end{equation}

\noindent Step(2): Integrate over the domain 

\begin{equation}
    \int_\Omega \psi (\frac{\partial C}{\partial t})d\Omega + \int_\Omega \psi (\nabla \cdot D(\mathbf{x}) \nabla C)d\Omega = 0
    % \label{Eq:TestFunction}
\end{equation}

\noindent Step(3): Integrate by parts the terms involving gradients, i.e. the second term  of the lhs

\begin{equation}
    \int_\Omega \psi (\nabla \cdot D(\mathbf{x}) \nabla C) d\Omega = -\int_\Omega \nabla \psi \cdot D(\mathbf{x}) \nabla C d\Omega + \int_{\Gamma} \psi (D(\mathbf{x}) \nabla C \cdot \mathbf{n}) d\Gamma
    \label{Eq:IntByParts}
\end{equation}

\noindent Step(4): Note that the flux $\mathbf{J} = D(\mathbf{x}) \nabla C$. Some parts of the boundary $\Gamma_1$ will have known $\mathbf{J}_{\Gamma_1}$ (Neumann boundary conditions), while some other parts will have prescribed concentration $C_{\Gamma_2}$ (Dirichlet boundary conditions). Thus, the surface integral of Eq. (\ref{Eq:IntByParts}) becomes 

\begin{equation}
    \int_{\Gamma} \psi (\mathbf{J} \cdot \mathbf{n}) d\Gamma = \int_{\Gamma_1} \psi (\mathbf{J}_{\Gamma_1} \cdot \mathbf{n}) d\Gamma_1 
    % \label{Eq:IntByParts}
\end{equation}

\noindent The weak form of Eq. (\ref{Eq:Example}) becomes

\begin{equation}
    \begin{split}
    \int_\Omega \psi (\frac{\partial C}{\partial t}) -\int_\Omega \nabla \psi \cdot D(\mathbf{x}) \nabla C d\Omega + \int_{\Gamma_1} \psi (\mathbf{J}_{\Gamma_1} \cdot \mathbf{n}) d\Gamma_1 = 0
    \end{split}
    % \label{Eq:TestFunction}
\end{equation}

\section{FEM for heat and mass transfer}

\subsection{Constitutive relations}

The transient heat transfer is expressed as

\begin{equation}
    \rho s \frac{\partial T}{\partial t} - \nabla \cdot (\mathbf{k} \nabla T) - Q
\end{equation}

\noindent Where $\rho$ is the mass density, $c$ is the heat capacity, $Q$ is the heat generated per unit volume per unit time\footnote{source/sink} and $\mathbf{k}$ is the conductivity matrix

\begin{equation}
    \left[
        \begin{array}{ccc}
        k_x & 0 & 0 \\
        0 & k_y & 0 \\
        0 & 0 & k_z \\
        \end{array}\right]
\end{equation}

\subsection{The weak form}

The weak form is 

\begin{equation}
    \int_\Omega \delta T^* \rho s \frac{\partial T}{\partial t} d \Omega + \int_\Omega \nabla \delta T^* \mathbf{k} \nabla T d\Omega - \int_\Omega \delta T^* Q d\Omega - \int_{\Gamma_2} \delta T^* (\mathbf{J}_{\Gamma_2} \cdot \mathbf{n}) d\Gamma_2 = 0
    \label{Eq:HeatWeakForm}
\end{equation}

\subsection{FE discretization}

Using 

\begin{equation}
    T(\boldsymbol{\xi}) = \sum_i^n N_i(\boldsymbol{\xi})T_i \qquad 
    \nabla T(\boldsymbol{\xi}) = \sum_i^n \mathbf{B}_i(\boldsymbol{\xi})T_i
\end{equation}

\noindent Substituting in Eq. (\ref{Eq:HeatWeakForm})\footnote{To simplify the notation, $\sum_i^n N_i(\boldsymbol{\xi})T_i\equiv \mathbf{N}T$ is used with the understanding that $T$ is the element nodal values.},  and eliminating $\delta T^*$

\begin{equation}
    \int_\Omega \mathbf{N}^\top \rho s \mathbf{N} \frac{\partial T}{\partial t} d \Omega + \int_\Omega \mathbf{B}^\top \mathbf{k} \mathbf{B} T d\Omega - \int_\Omega \mathbf{N}^\top Q d\Omega - \int_{\Gamma_2} \mathbf{N}^\top (\mathbf{J}_{\Gamma_2} \cdot \mathbf{n}) d\Gamma_2 = 0
    \label{Eq:HeatDiscritized}
\end{equation}

\noindent Where the capacity matrix $\mathbf{C}$

\begin{equation}
    \mathbf{C} = \int_\Omega \mathbf{N}^\top \rho c \mathbf{N} d\Omega
\end{equation}

\noindent And the conductivity matrix $\mathbf{K}$

\begin{equation}
    \mathbf{K_c} = \int_\Omega \mathbf{B}^\top \mathbf{k} \mathbf{B} d\Omega
\end{equation}

\noindent Eq. (\ref{Eq:HeatDiscritized}) can then be expressed as

\begin{equation}
    \mathbf{C} \dot{\mathbf{T}} + \mathbf{K_c} \mathbf{T} = \mathbf{F}
    \label{Eq:TransientHeat}
\end{equation}

\noindent The implicit time integration of Eq. (\ref{Eq:TransientHeat}) can be done as 

\begin{equation}
    (\mathbf{C} + \Delta t \mathbf{K_c})\mathbf{T}^{t+1} = \Delta t \mathbf{F} + \mathbf{M} \mathbf{T}^t
    \label{Eq:TransientHeat}
\end{equation}

\section{FEM for mesoscale hydrogen transport in metals}

\subsection{Constitutive relations}

The total flux for hydrogen transport, interface $g(\phi)_\mathrm{intf}$ \cite{Hussein2024, Hussein2024b} and phase $\phi_j$ interaction

\begin{equation}
    \mathbf{J} = - \boldsymbol{D} \Big(\nabla c - \frac{\zeta_\mathrm{intf} c }{RT} \nabla g(\phi)_\mathrm{intf} - \frac{\zeta_j c }{RT} \nabla \phi_j \Big)
\end{equation}

\noindent And by applying the mass conservation 

\begin{equation}
    \frac{\partial c}{\partial t} - \nabla \cdot \boldsymbol{D} \Big( \nabla c - \frac{\zeta_\mathrm{intf} c }{RT} \nabla g(\phi)_\mathrm{intf} - \frac{\zeta_j c }{RT} \nabla \phi_j \Big) = 0
    \label{Eq:HydrogenDiffTrap}
\end{equation}

\subsection{The weak form}

Multiply Eq. (\ref{Eq:HydrogenDiffTrap}) by a test function $\eta$ and integrate over the domain 

\begin{equation}
    \int_\Omega \eta (\frac{\partial c}{\partial t}) d\Omega - \int_\Omega \eta \nabla \cdot \mathbf{D} \Big( \nabla c - \frac{\zeta_\mathrm{intf} c }{RT} \nabla g(\phi)_\mathrm{intf} - \frac{\zeta_j c }{RT} \nabla \phi_j \Big) d\Omega = 0
\end{equation}

\noindent Integrate by parts the second term  of the LHS and combining all, the weak form of Eq. (\ref{Eq:HydrogenDiffTrap}) becomes

\begin{equation}
    \int_\Omega \eta \frac{\partial c}{\partial t}  d\Omega + \int_\Omega \nabla \eta \cdot \boldsymbol{D} \left( \nabla c - \frac{\zeta_\mathrm{intf} c }{RT} \nabla g(\phi)_\mathrm{intf} - \frac{\zeta_j c }{RT} \nabla \phi_j  \right)  d\Omega = \int_{\Gamma} \eta \, \mathbf{J}  \cdot \mathbf{n} \, d\Gamma
    \label{Eq:TrapWeakForm}
\end{equation}

\subsection{FE discretization}

Using 

\begin{equation}
    \begin{split}
        c = \mathbf{N \,c} \qquad 
        \nabla c = \mathbf{B} \, \mathbf{c}& \\
        \nabla g(\phi)_\mathrm{intf} = \mathbf{B} \, \boldsymbol{g(\phi)}_\mathrm{intf}& \qquad 
        \nabla \phi_j = \mathbf{B} \, \boldsymbol{\phi_j}
    \end{split}
\end{equation}

\noindent And substituting in \ref{Eq:TrapWeakForm}

\begin{equation}
    \begin{split}
    \int_\Omega \mathbf{N}^\top \boldsymbol{\eta}^\top  \frac{\partial}{\partial t} (\mathbf{N}\,\mathbf{c}) \, d\Omega 
    &+ \int_\Omega \mathbf{B}^\top \boldsymbol{\eta}^\top \mathbf{D} \mathbf{B} \, \mathbf{c} \, d\Omega
    - \int_\Omega \mathbf{B}^\top \boldsymbol{\eta}^\top \mathbf{D} \frac{\zeta_\mathrm{intf}}{RT} \mathbf{B} \, \boldsymbol{g(\phi)}_\mathrm{intf} \, \mathbf{N} \, \mathbf{c} \, d\Omega \\
    &- \int_\Omega \mathbf{B}^\top \boldsymbol{\eta}^\top \mathbf{D} \frac{\zeta_{j}}{RT} \mathbf{B} \, \boldsymbol{\phi_{j}} \, \mathbf{N} \, \mathbf{c} \, d\Omega 
    - \int_{\Gamma} \mathbf{N}^\top \boldsymbol{\eta}^\top (\mathbf{J} \cdot \mathbf{n}) \, d\Gamma = 0
    \end{split}
\end{equation}

\noindent By eliminating $\eta^\top$, we finally get 

\begin{equation}
    \begin{split}
        \int_\Omega \mathbf{N}^\top  \frac{\partial}{\partial t} (\mathbf{N}\,\mathbf{c}) \, d\Omega 
        &+ \int_\Omega \mathbf{B}^\top \mathbf{D} \mathbf{B} \, \mathbf{c} \, d\Omega
        - \int_\Omega \mathbf{B}^\top  \mathbf{D} \frac{\zeta_\mathrm{intf}}{RT} \mathbf{B} \, \boldsymbol{g(\phi)}_\mathrm{intf} \, \mathbf{N} \, \mathbf{c} \, d\Omega \\
        &- \int_\Omega \mathbf{B}^\top \mathbf{D} \frac{\zeta_{j}}{RT} \mathbf{B} \, \boldsymbol{\phi_{j}} \, \mathbf{N} \, \mathbf{c} \, d\Omega 
        - \int_{\Gamma} \mathbf{N}^\top (\mathbf{J} \cdot \mathbf{n}) \, d\Gamma = 0
        \end{split}
    \label{Eq:TrapDiscritized}
\end{equation}

\noindent Where mass matrix $\mathbf{M}$

\begin{equation}
    \mathbf{M} = \int_\Omega \mathbf{N}^\top \mathbf{N} d\Omega
\end{equation}

\noindent And the diffusivity matrix $\mathbf{K_D}$

\begin{equation}
    \mathbf{K_D} = \int_\Omega \mathbf{B}^\top \mathbf{D} \mathbf{B} d\Omega
\end{equation}

\noindent And the interaction matrix $\mathbf{K_I}$

\begin{equation}
    \mathbf{K_I} = \int_\Omega \mathbf{B}^\top \mathbf{D}\frac{\zeta_\mathrm{intf}}{RT} \mathbf{B} \, \boldsymbol{g(\phi)}_\mathrm{intf} \mathbf{N} \, d\Omega + \int_\Omega \mathbf{B}^\top \mathbf{D}\frac{\zeta_{j}}{RT} \mathbf{B} \, \boldsymbol{\phi_{j}} \mathbf{N} \,d\Omega
\end{equation}

\noindent Eq. (\ref{Eq:TrapDiscritized}) can then be expressed as

\begin{equation}
    \mathbf{M} \dot{\mathbf{c}} + [\mathbf{K_D} - \mathbf{K_I}] \mathbf{c} = \mathbf{F}
    \label{Eq:TrapTransient}
\end{equation}

\noindent The implicit time integration of Eq. (\ref{Eq:TrapTransient}) can be done as 

\begin{equation}
    \frac{\mathbf{c}^{t+1}-\mathbf{c}^{t}}{\Delta t} \mathbf{M} + \mathbf{K_D} \mathbf{c}^{t+1} - \mathbf{K_I} \mathbf{c}^{t+1} = \mathbf{F}
\end{equation}

\begin{equation}
    [\mathbf{M} + \Delta t \, \mathbf{K_D} - \Delta t \mathbf{K_I}]\mathbf{c}^{t+1} = \Delta t \mathbf{F} + \mathbf{M} \mathbf{c}^t
    \label{Eq:TrapFEM}
\end{equation}

\noindent Note that Eq. (\ref{Eq:TrapFEM}) is linear and can be solved with linear solvers. 

% \section{Phase-field model for grain growth based on the Allen-Cahn model}

% Each grain is represented by an order parameter $\phi_i$, which takes one within the grain and zero elsewhere. The evolution of $n$ order parameters is described by 

% \begin{equation}
%     \frac{\partial \phi_i}{\partial t} = -L_i \frac{\delta F}{\delta \phi_i} \, , \hspace{0.5 cm} i=1,2, ... ,n
% \end{equation}

% \noindent Where $-L_i$ is the mobility coefficient and $F$ is the total volumetric free energy functional, which is given by 

% \begin{equation}
%     F = \int_V \Big[f(\phi_1, \phi_2, ... , \phi_n) + \sum_i^n \frac{\kappa_i}{2}|\nabla \phi_i|^2 \Big] dV
% \end{equation}

% \noindent By using 

% \begin{equation}
%     \frac{\delta F}{\delta \phi_i} = \frac{\partial f}{\partial \phi_i} - \nabla \cdot \frac{\partial f}{\partial (\nabla \phi_i)}  = \frac{\partial f}{\partial \phi_i} -  \kappa_i \nabla \cdot \nabla \phi_i
% \end{equation}

% \noindent The evolution of $\phi_i$ is  

% \begin{equation}
%     \frac{\partial \phi_i}{\partial t} = -L_i \frac{\partial f}{\partial \phi_i} + L_i \nabla \cdot \nabla \phi_i
%     \label{Eq:EvolutionPhi}
% \end{equation}

% \subsection{The weak form}

% The weak form of Eq. (\ref{Eq:EvolutionPhi}), using the test function $\delta \phi^*$

% \begin{equation}
%     \int_\Omega \delta \phi^* \frac{\phi_i}{\partial t} d\Omega = -L_i \int_\Omega \delta \phi^* \frac{\partial f}{\partial \phi_i} \, d\Omega - L_i \kappa_i \int_\Omega \nabla \delta \phi^* \cdot \nabla \phi_i \, d\Omega
%     \label{Eq:WeakFormPhi}
% \end{equation}

% \subsection{Semi-implicit time integration}

% A semi-implicit time integration scheme \cite{Biner2017} is used\footnote{Note that $\phi_i^{t}$ is the integration point value.}, This avoids the need for updating the Newton-Raphson iteration and reassembling the stiffness matrix. 

% \begin{equation}
%     \int_\Omega \delta \phi^* \phi_i^{t+1} d\Omega + \Delta t L_i \kappa_i \int_\Omega \nabla \delta \phi^* \cdot \nabla \phi_i^{t+1} \, d\Omega = \int_\Omega \delta \phi^* \phi_i^{t} \, d\Omega -\Delta t L_i \int_\Omega \delta \phi^* \frac{\partial f^{t}}{\partial \phi_i} \, d\Omega
%     \label{Eq:PhiSemiImpl}
% \end{equation}

% \subsection{FEM discretization}

% \noindent Using 

% \begin{equation}
%     \begin{split}
%         \phi_i(\boldsymbol{\xi}) = \sum_i^n N_i(\boldsymbol{\xi})\phi_i \qquad 
%     \nabla g(\phi)(\boldsymbol{\xi}) = \sum_i^n \mathbf{B}_i(\boldsymbol{\xi}){\phi}_{\beta i}
%     \end{split}
% \end{equation}

% \begin{equation}
%     \int_\Omega \mathbf{N}^\top \delta \phi^{*\top} \mathbf{N} \phi_i^{t+1} + \Delta t L_i \kappa_i \int_\Omega \mathbf{B}^\top \delta \phi^{*\top} \mathbf{B} \, \phi_i^{t+1} \, d\Omega = \int_\Omega \mathbf{N} \delta \phi \, \phi_i^{t} \, d\Omega -\Delta t L_i \int_\Omega \mathbf{N} \delta \phi \, \frac{\partial f^{t}}{\partial \phi_i} \, d\Omega
% \end{equation}

% \noindent By eliminating $\phi$, we get 

% \begin{equation}
%     \int_\Omega \Big[\mathbf{N}^\top \mathbf{N} \phi_i^{t+1} + \Delta t L_i \kappa_i \mathbf{B}^\top \mathbf{B} \, \phi_i^{t+1} \Big] d\Omega = \int_\Omega \Big[\mathbf{N} \phi_i^{t} \, d\Omega - \Delta t L_i \mathbf{N} \frac{\partial f^{t}}{\partial \phi_i} \Big] d\Omega
% \end{equation}

% \noindent Where the stiffness matrix 

% \begin{equation}
%     \mathbf{K} = \int_\Omega \Big[\mathbf{N}^\top \mathbf{N} + \Delta t L_i \kappa_i \mathbf{B}^\top \mathbf{B} \Big] d\Omega
% \end{equation}

% \noindent And the RHS

% \begin{equation}
%     \mathbf{R} = \int_\Omega \Big[\mathbf{N} \phi_i^{t} \, d\Omega - \Delta t L_i \mathbf{N} \frac{\partial f^{t}}{\partial \phi_i} \Big] d\Omega
% \end{equation}

% \noindent The system of equations is 

% \begin{equation}
%     \mathbf{K} \, \phi^{t+1} = \mathbf{R}
% \end{equation}

% \noindent A staggered solution scheme is used to solve this system of equations \cite{Biner2017}, where one grain is considered at a time.

% \section{Multi-phase field model}

% In the multi-phase field method \cite{Steinbach2009} time evolution of the order parameter vector $\phi_i$ is expressed as \cite{Takaki2009}

% \begin{equation}
%     \frac{\partial \phi_{i}}{\partial t} = -\frac{2}{N} \sum_{j=1}^{N} M^{\phi}_{ij} \left(\sum_{k=1}^{N} \left( \frac{1}{2} \left( a_{ik}^{2} - a_{jk}^{2} \right) \nabla^{2}\phi_{k} + \left( W_{ik} - W_{jk} \right) \phi_{k}  \right) + -\frac{8}{\pi}\sqrt{\phi_{i}\phi_{j}}\Delta G_{ij}\right)
%     \label{Eq:MPF}
% \end{equation}

\section{FEM for small strain isotropic linear elasticity}

\subsection{Constitutive relations}

The stress-strain relations for isotropic linear elasticity are

\begin{equation}
    \begin{split}
    \boldsymbol{\sigma} &= \mathbf{C} \boldsymbol{\varepsilon} \\
    \sigma_{ij} &= C_{ijkl} \varepsilon_{kl}
    \end{split}
\end{equation}

\noindent Using the Lam\'e constants $\lambda = E\nu/(1+\nu)$ and $\mu = G = E/2(1+\nu)$

\begin{equation}
    \boldsymbol{\sigma} = 2G \boldsymbol{\varepsilon} + \lambda \Tr(\boldsymbol{\varepsilon}) \mathbf{I}
\end{equation}

\noindent And in component form

\begin{equation}
    \left[
        \begin{array}{rrr}
        \sigma_{xx} &  \sigma_{xy} & \sigma_{xz} \\
        \sigma_{yx}  &  \sigma_{yy}  &  \sigma_{yz} \\
        \sigma_{zx} &  \sigma_{zy} & \sigma_{zz} \\
        \end{array}
    \right] = 2G \left[
        \begin{array}{rrr}
        \varepsilon_{xx} &  \varepsilon_{xy} & \varepsilon_{xz} \\
        \varepsilon_{yx}  &  \varepsilon_{yy}  &  \varepsilon_{yz} \\
        \varepsilon_{zx} &  \varepsilon_{zy} & \varepsilon_{zz} \\
        \end{array}
    \right] + \lambda(\varepsilon_{xx}+\varepsilon_{yy}+\varepsilon_{zz}) \left[
        \begin{array}{rrr}
        1 &  0 & 0 \\
        0 &  1 & 0 \\
        0 &  0 & 1 \\
        \end{array}
    \right]
\end{equation}

\noindent For numerical convenience, it is more practical to represent the stress-strain relations in \emph{Voigt} notation, where the shear components are stored as \emph{engineering} shears.

\begin{equation}
    \begin{split}
    \left[
        \begin{array}{r}
        \sigma_{xx} \\
        \sigma_{yy} \\
        \sigma_{zz} \\
        \sigma_{xy} \\
        \sigma_{yz} \\
        \sigma_{xz} \\
        \end{array}
    \right] &= \left[
        \begin{array}{cccccc}
        \lambda+2\mu & \lambda & \lambda & 0 & 0 & 0 \\
        \lambda & \lambda+2\mu & \lambda & 0 & 0 & 0 \\
        \lambda & \lambda & \lambda+2\mu & 0 & 0 & 0 \\
        0 & 0 & 0 & \mu & 0 & 0 \\
        0 & 0 & 0 & 0 & \mu & 0 \\
        0 & 0 & 0 & 0 & 0 & \mu \\
        \end{array}
    \right] \left[
        \begin{array}{c}
        \varepsilon_{xx} \\
        \varepsilon_{yy} \\
        \varepsilon_{zz} \\
        2\varepsilon_{xy} \\
        2\varepsilon_{yz} \\
        2\varepsilon_{xz} \\
        \end{array}
    \right] \\
    &= \left[
        \begin{array}{cccccc}
        \lambda+2\mu & \lambda & \lambda & 0 & 0 & 0 \\
        \lambda & \lambda+2\mu & \lambda & 0 & 0 & 0 \\
        \lambda & \lambda & \lambda+2\mu & 0 & 0 & 0 \\
        0 & 0 & 0 & \mu & 0 & 0 \\
        0 & 0 & 0 & 0 & \mu & 0 \\
        0 & 0 & 0 & 0 & 0 & \mu \\
        \end{array}
    \right] \left[
        \begin{array}{c}
        \varepsilon_{xx} \\
        \varepsilon_{yy} \\
        \varepsilon_{zz} \\
        \gamma_{xy} \\
        \gamma_{yz} \\
        \gamma_{xz} \\
        \end{array}
    \right] \\
    \end{split}
\end{equation}

\noindent For plane-strain

\begin{equation}
    \left[
        \begin{array}{r}
        \sigma_{xx} \\
        \sigma_{yy} \\
        \sigma_{xy} \\
        \end{array}
    \right] = \frac{E}{(1+\nu)(1-2\nu)}\left[
        \begin{array}{ccc}
        1-\nu & \nu & 0 \\
        \nu & 1-\nu & 0 \\
        0 & 0 & \frac{1-2\nu}{2} \\
        \end{array}
    \right] \left[
        \begin{array}{c}
        \varepsilon_{xx} \\
        \varepsilon_{yy} \\
        \gamma_{xy} \\
        \end{array}
    \right] \\
\end{equation}

\noindent $\sigma_{zz}\neq 0$\footnote{In PHIMATS, $\sigma_{zz}$ is an output for plasticity models. However, it is not an output for elasticity models, yet, it could be evaluated in post-processing.} and is calculated from

\begin{equation}
    \sigma_{zz} = \nu(\sigma_{xx} + \sigma_{yy})
\end{equation}

\noindent And for plane-stress

\begin{equation}
    \left[
        \begin{array}{r}
        \sigma_{xx} \\
        \sigma_{yy} \\
        \sigma_{xy} \\
        \end{array}
    \right] = \frac{E}{(1-\nu^2)}\left[
        \begin{array}{ccc}
        1 & \nu & 0 \\
        \nu & 1 & 0 \\
        0 & 0 & \frac{1-\nu}{2} \\
        \end{array}
    \right] \left[
        \begin{array}{c}
        \varepsilon_{xx} \\
        \varepsilon_{yy} \\
        \gamma_{xy} \\
        \end{array}
    \right] \\
\end{equation}

\noindent Note that $\varepsilon_{zz}\neq 0$ and is calculated from

\begin{equation}
    \varepsilon_{zz} = -\frac{1}{3E}(\sigma_{xx} + \sigma_{yy})
\end{equation}

\noindent Accordingly, the hydrostatic stress $\sigma_\mathrm{h}$ for plane-strain 

\begin{equation}
    \sigma_{\mathrm{h}} = \frac{1 + \nu}{3} (\sigma_{xx} + \sigma_{yy})
\end{equation}

\noindent And for plane-stress

\begin{equation}
    \sigma_{\mathrm{h}} = \frac{1}{3} (\sigma_{xx} + \sigma_{yy})
\end{equation}


\subsection{The weak form}

The quasi-static balance of linear momentum is 

\begin{equation}
    \begin{split}
    \nabla \cdot \boldsymbol{\sigma} + \mathbf{b} &= 0 \\
    \frac{\partial \sigma_{ij}}{\partial x_j} + b_i &= 0 \\
    \end{split}
    \label{Eq:LinearMomentum}
\end{equation}

\noindent The problem statement for quasi-static linear elasticity is find $u_i$, $\varepsilon_{ij}$ and $\sigma_{ij}$ that satisfies

\begin{enumerate}
    \item The strain-displacement relation $\varepsilon_{ij} = 1/2(\partial u_i/\partial x_j + \partial u_j/\partial x_i)$
    \item The stress-strain relation $\sigma_{ij} = C_{ijkl}\varepsilon_{kl}$
    \item The quasi-static linear balance of momentum $\frac{\partial \sigma_{ij}}{\partial x_j} + b_i = 0$
    \item The displacement and traction boundary conditions $u_i = u_i^{\Gamma_1}$ on $\Gamma_1$ and $\sigma_{ij}n_j = t_j^{\Gamma_2}$ on $\Gamma^2$
\end{enumerate}

\noindent The principle of virtual work is an equivalent integral form of the linear balance of momentum PDE. This makes it more convenient for numerical solution using computers. This essentially involves all the 4-steps mentioned in section \ref{Sec:WeakForm}. We multiply the strong form Eq. (\ref{Eq:LinearMomentum}) by a kinematically admissible virtual displacement field $\delta u_i$ (test function), which means it satisfies $\delta u_i=0$ on $\Gamma^1$. Following Bower \cite{Bower2009}, \emph{This is a complicated way for saying that the small perturbation displacement $\delta u_i$ satisfies the boundary conditions.} The associated \emph{virtual} strain field 

\begin{equation}
    \delta \varepsilon_{ij} = 1/2(\partial u_i/\partial x_j + \partial u_j/\partial x_i)
\end{equation}

\noindent The weak form is then

\begin{equation}
    \int_\Omega \delta \varepsilon_{ji} \sigma_{ij}  d\Omega - \int_\Omega \delta u_j b_i  d\Omega - \int_{\Gamma^2} \delta u_j t_i  d\Gamma^2 = 0
    \label{Eq:LinElasWeakForm}
\end{equation}

\begin{equation}
    \int_\Omega \delta \boldsymbol{\varepsilon}^{\top} \boldsymbol{\sigma} d\Omega - \int_\Omega \delta \mathbf{u}^{\top} \mathbf{b}  d\Omega - \int_{\Gamma^2} \delta \mathbf{u}^{\top} \mathbf{t}  d\Gamma^2 = 0
    \label{Eq:LinElasWeakForm}
\end{equation}

\noindent Will satisfy the strong form for all possible $\delta \mathbf{u}$.
    
\subsection{FE discretization}

The displacements $\mathbf{u}(\boldsymbol{\xi})$ within any point in the element $\boldsymbol{\xi}$ is obtained from the nodal values of the element using the isoparametric formalism

\begin{equation}
    \mathbf{u}(\boldsymbol{\xi}) = \left[
        \begin{array}{c}
        u_{x}(\boldsymbol{\xi}) \\
        u_{y}(\boldsymbol{\xi}) \\
        u_{z}(\boldsymbol{\xi}) \\
        \end{array}
    \right] = \sum_i^n N_i(\boldsymbol{\xi})u_i
    \label{Eq:IsoparamDisp}
\end{equation}

% \noindent Again, using a 4-node quadrilateral element as an example for a 2D case, Eq. (\ref{Eq:IsoparamDisp}) becomes

% \begin{equation}
%     \left[
%         \begin{array}{c}
%         u_{x}(\boldsymbol{\xi}) \\
%         u_{y}(\boldsymbol{\xi}) \\
%         \end{array}
%     \right] = [N_1(\boldsymbol{\xi}) \ N_2(\boldsymbol{\xi}) \ N_3(\boldsymbol{\xi}) \ N_4(\boldsymbol{\xi})] \left[
%         \begin{array}{cc}
%         u1_{x} & u1_{y} \\
%         u2_{x} & u2_{y} \\
%         u3_{x} & u3_{y} \\
%         u4_{x} & u4_{y} \\
%         \end{array}
%     \right]
%     % \label{Eq:IsoparamDisp}
% \end{equation}

\noindent The strains can be obtained from the derivatives of the nodal displacements. For simplicity, we use a 2D case. The small-strain displacement in Voigt notation becomes

\begin{equation}
    \boldsymbol{\varepsilon} = \left[
        \begin{array}{c}
        \varepsilon_{xx} \\
        \varepsilon_{yy} \\
        \gamma_{xy} \\
        \end{array}
        \right] = \left[\begin{array}{c}
        \frac{\partial u_x}{\partial x} \\
        \frac{\partial u_y}{\partial y} \\
        \frac{\partial u_x}{\partial y} + \frac{\partial u_y}{\partial x} \\
        \end{array}\right]
\end{equation}

\noindent Using the displacement values from Eq. (\ref{Eq:IsoparamDisp}), the strains could be evaluated using the cartesian derivatives of the shape functions similar to Eq. (\ref{Eq:IsoparamDeriv}). However, due to the special case of the Voigt representation of the strain tensor, the cartesian derivatives of the shape functions are collected in a matrix called the $\mathbf{B}$ matrix\footnote{Note that $\mathbf{B}$ is of size $\texttt{nStres} \times \texttt{nElDispDofs}$, where $\texttt{nStre}$ is the number of stress components and $\texttt{nElDispDofs}$ is the number of displacement degrees of freedom per element. This will have to be evaluated for every gauss point $\boldsymbol{\xi}$, i.e. stored in size \texttt{nGaus}.}, which has a special form as will be described later. Therefore 

\begin{equation}
    \boldsymbol{\varepsilon}(\boldsymbol{\xi}) = \sum_i^n \mathbf{B}_i(\boldsymbol{\xi}) u_i
\end{equation}

\nextboxnumber
\begin{boxH}[label=Bmatrix]{Example for calculating strain in 4-node quad element}
    \begin{equation*}
        \begin{split}
            \left[
                \begin{array}{c}
                \varepsilon_{xx} \\
                \varepsilon_{yy} \\
                \gamma_{xy} \\
                \end{array}
            \right] = \left[\begin{array}{cccccccc}
                \frac{\partial N_1}{\partial x} & 0 & \frac{\partial N_2}{\partial x} & 0 & \frac{\partial N_3}{\partial x} & 0 & \frac{\partial N_4}{\partial x} & 0 \\
                0 & \frac{\partial N_1}{\partial y} & 0 & \frac{\partial N_2}{\partial y} & 0 & \frac{\partial N_3}{\partial y} & 0 & \frac{\partial N_4}{\partial y} \\
                \frac{\partial N_1}{\partial y} & \frac{\partial N_1}{\partial x} & \frac{\partial N_2}{\partial y} & \frac{\partial N_2}{\partial x} & \frac{\partial N_3}{\partial y} & \frac{\partial N_3}{\partial x} & \frac{\partial N_4}{\partial y} & \frac{\partial N_4}{\partial x} \\
                \end{array}\right]  \left[\begin{array}{c}
                    u1_{x} \\
                    u1_{y} \\
                    u2_{x} \\
                    u2_{y} \\
                    u3_{x} \\
                    u3_{y} \\
                    u4_{x} \\
                    u4_{y} \\
                    \end{array}
                    \right]
        \end{split}
    \end{equation*}
\end{boxH}

\noindent Using

\begin{equation}
    \delta\mathbf{u} = \sum_i^n N_i(\boldsymbol{\xi})\delta \mathbf{u}
    % \label{Eq:IsoparamDisp}
\end{equation}

\noindent And 

\begin{equation}
    \delta \boldsymbol{\varepsilon} = \sum_i^n \mathbf{B}_i(\boldsymbol{\xi}) \delta\mathbf{u}
\end{equation}

\noindent The weak form in Eq. (\ref{Eq:LinElasWeakForm}) can be written as

\begin{equation}
    \int_\Omega \mathbf{B}^\top \delta \mathbf{u}^{\top} \boldsymbol{\sigma}  d\Omega - \int_\Omega \mathbf{N}^\top \delta \mathbf{u}^{\top} \mathbf{b} d\Omega - \int_{\Gamma^2} \mathbf{N}^\top \mathbf{u}^{\top} \mathbf{t} d\Gamma^2 = 0
    % \label{Eq:LinElasWeakForm}
\end{equation}

\noindent Using the stress-strain relations and eliminating the arbitrary $\delta \mathbf{u}$ 

\begin{equation}
    \int_\Omega \mathbf{B}^\top \boldsymbol{\sigma} d\Omega - \int_\Omega \mathbf{N}^\top \mathbf{b} d\Omega - \int_{\Gamma^2} \mathbf{N}^\top \mathbf{t} d\Gamma^2 = 0
    % \label{Eq:LinElasWeakForm}
\end{equation}

\begin{equation}
    \int_\Omega \mathbf{B}^\top \mathbf{D} \mathbf{B} \mathbf{u} d\Omega - \int_\Omega \mathbf{N}^\top \mathbf{b} d\Omega - \int_{\Gamma^2} \mathbf{N}^\top \mathbf{t} d\Gamma^2 = 0
    % \label{Eq:LinElasWeakForm}
\end{equation}

\noindent Ignoring body forces

\begin{equation}
    \int_\Omega \mathbf{B}^\top \mathbf{D} \mathbf{B} \mathbf{u} d\Omega - \int_{\Gamma^2} \mathbf{N}^\top \mathbf{t} d\Gamma^2 = 0
\end{equation}

\noindent Or

\begin{equation}
    \mathbf{f}_\mathrm{int} - \mathbf{f}_\mathrm{ext} = 0
    \label{Eq: StressEquilibrium}
\end{equation}

\noindent The stiffness matrix $\mathbf{K}$

\begin{equation}
    \mathbf{K} = \int_\Omega \mathbf{B}^\top \mathbf{D} \mathbf{B} d\Omega
\end{equation}

\noindent Since the domain is discretized in \texttt{nElem} elements and introducing the element stiffness matrix $\mathbf{k}^{(e)}$, the \emph{global} stiffness matrix becomes

\begin{equation}
    \mathbf{K} = \sum^\texttt{nElem} \mathbf{k}^{(e)}
\end{equation}

\noindent The element stiffness matrix can be defined as 

\begin{equation}
    \mathbf{k}^{(e)} = \int_{\Omega_{(e)}}{\Big[\sum_i^n \mathbf{B}_i(\boldsymbol{\xi})\Big]}^\top \mathbf{D} \sum_i^n \mathbf{B}_i(\boldsymbol{\xi}) d\Omega_{(e)}
\end{equation}

\noindent Where the integral can be evaluated using the gauss quadrature. The reaction force can then be calculated from 

\begin{equation}
    \mathbf{f}_\mathrm{reaction} = -\mathbf{f}_\mathrm{int} = -\int_\Omega \mathbf{B}^\top \boldsymbol{\sigma} d\Omega 
\end{equation}

% % \section{Summary of kinematics and stress/strain measures}

% % \subsection{Deformation gradient}

% % The \emph{reference} or \emph{undeformed} configuration is expressed in the \emph{material} coordinates $\mathbf{X}=(X,Y,Z)$. This description of the solid is called the \emph{Lagrangian} description. The \emph{deformed} configuration is expressed in the \emph{spatial} coordinate system $\mathbf{x}=(x,y,z)$. This is called the \emph{Eulerian} description. We have $\mathbf{x}=\mathbf{x(X)}$. The deformation of an infinitesimal material segment is described by the deformation gradient 

% % \begin{equation}
% %     \mathbf{F} = \frac{\partial \mathbf{x}}{\partial \mathbf{X}} = \mathbf{I} + \frac{\partial \mathbf{u}}{\partial \mathbf{X}}
% % \end{equation}

% % \noindent Or in component form

% % \begin{equation}
% %     F_{ij} = \frac{\partial x_i}{\partial X_j} = \delta_{ij} + \frac{\partial u_i}{\partial X_j}
% % \end{equation}

% % \noindent The jacobian of the deformation gradient is defined as

% % \begin{equation}
% %     J = \det(\mathbf{F}) = \frac{dV}{dV_0}
% % \end{equation}

% % \noindent And form the conservation of mass 

% % \begin{equation}
% %     \rho_o dV_0 = \rho dV
% % \end{equation}

% % \noindent We get 

% % \begin{equation}
% %     \det(\mathbf{F}) = \frac{\rho}{\rho_0}
% % \end{equation}

% % \subsection{Lagrange strain tensor}

% % The Lagrange strain tensor describes the deformation, i.e. change in length and angle of material fiber, when large deformations are expected in the reference configuration. It is defined as

% % \begin{equation}
% %     \boldsymbol{\gamma} = \frac{1}{2}(\mathbf{F}^\top \cdot \mathbf{F} - \mathbf{I}) \ \ \mathrm{or} \ \ \gamma_{ij} = \frac{1}{2}(F_{ki}F_{kj} - \delta_{ij})
% % \end{equation}

% % \noindent Or in terms of displacement  

% % \begin{equation}
% %     \gamma_{ij} = \frac{1}{2}\Big(\frac{\partial u_i}{\partial X_j} + \frac{\partial u_j}{\partial X_i} \Big) + \frac{1}{2}\frac{\partial u_k}{\partial X_i} \frac{\partial u_k}{\partial X_j}
% %     \label{Eq: disp1}
% % \end{equation}

% % \noindent In incremental form

% % \begin{equation}
% %     \boldsymbol{\gamma}^{t+\Delta t} - \boldsymbol{\gamma}^{\Delta t} = \Delta \boldsymbol{\gamma} = \frac{1}{2}(\mathbf{F}^\top \cdot \nabla_\mathbf{X}(\Delta \mathbf{u}) + \nabla_\mathbf{X}(\Delta \mathbf{u})^\top \cdot \mathbf{F}) + \frac{1}{2} \nabla_\mathbf{X}(\Delta \mathbf{u})^\top \cdot \nabla_\mathbf{X}(\Delta \mathbf{u})
% %     \label{Eq: disp2}
% % \end{equation}

% % \subsection{Second Piola-Kirchhoff stress tensor}

% % For engineering applications, the \emph{true} or \emph{Cauchy} stress tensor is usually used, which is expressed in the current \emph{unknown} configuration. On the other side, for large displacements, the Second Piola-Kirchhoff stress tensor $\boldsymbol{\tau}$ is used and is expressed in some previous configuration.  

% % \begin{equation}
% %     \boldsymbol{\tau} = J \mathbf{F}^{-1} \cdot \boldsymbol{\sigma} \cdot \mathbf{F}^{-\top} \ \ \mathrm{or} \ \ \tau_{ij} = J F_{ik}^{-1} \sigma_{kl} \cdot F_{jl}^{-1}
% % \end{equation}

% % \noindent And

% % \begin{equation}
% %     \boldsymbol{\sigma} = \frac{1}{J} \mathbf{F} \cdot \boldsymbol{\tau} \cdot \mathbf{F}^\top  \ \ \mathrm{or} \ \ \sigma_{ij} = J^{-1} F_{ik} \tau_{kl} F_{jl}
% % \end{equation}

% % \subsection{Infinitesimal strain}

% % The linearization of strain displacement relation is 

% % \begin{equation}
% %     \begin{split}
% %     \boldsymbol{\varepsilon} &= \frac{1}{2}\Big(\nabla \mathbf{u}^\top+ \nabla \mathbf{u}) \\
% %     \varepsilon_{ij} &= \frac{1}{2} \Big( u_{i,j} + u_{j,i}) \\
% %     &= \left[
% %         \begin{array}{ccc}
% %         \frac{\partial u_x}{\partial x} &  \frac{1}{2}(\frac{\partial u_x}{\partial y} + \frac{\partial u_y}{\partial x}) & \frac{1}{2}(\frac{\partial u_x}{\partial z} + \frac{\partial u_z}{\partial x}) \\
% %         \frac{1}{2}(\frac{\partial u_y}{\partial x} + \frac{\partial u_x}{\partial y})  &  \frac{\partial u_y}{\partial y}  &  \frac{1}{2}(\frac{\partial u_y}{\partial z} + \frac{\partial u_z}{\partial y}) \\
% %         \frac{1}{2}(\frac{\partial u_z}{\partial x} + \frac{\partial u_x}{\partial z}) &  \frac{1}{2}(\frac{\partial u_z}{\partial y} + \frac{\partial u_y}{\partial z}) & \frac{\partial u_z}{\partial z} \\
% %         \end{array}
% %     \right]
% %     \end{split}
% %     \label{Eq:SmallStrain}
% % \end{equation}

% % \section{FEM for large strain linear elasticity}

% % \subsection{Principle of virtual work (weak form)}

% % The variation of internal energy, or internal virtual work in the current configuration expressed in terms of $\delta \boldsymbol{\varepsilon}$ and $\boldsymbol{\sigma}$

% % \begin{equation}
% %     \delta W = \int_{\Omega} \boldsymbol{\varepsilon}^\top \boldsymbol{\sigma} d\Omega
% % \end{equation}

% % \noindent In a previous configuration 

% % \begin{equation}
% %     \delta W = \int_{\Omega_0} \boldsymbol{\gamma}^\top \boldsymbol{\tau} d\Omega_0
% % \end{equation}

% % \noindent Note that $\Omega_0$ is the reference volume. In incremental form $ \delta \boldsymbol{\gamma}^{t+\Delta t} = \delta \Delta \boldsymbol{\gamma}$ as  the variation of a constant $\delta \boldsymbol{\gamma}^t$ is zero. The incremental form of the principle of virtual work in the reference configuration, ignoring the body forces then reads

% % \begin{equation}
% %     \int_{\Omega_0} \delta \Delta \boldsymbol{\gamma}^\top \Delta \mathbf{D} \Delta \boldsymbol{\gamma} d\Omega_0 + \int_{\Omega_0} \delta \Delta \boldsymbol{\gamma}^\top \boldsymbol{\tau}^t d\Omega_0 = \int_{\Gamma_0} \delta \mathbf{u}^\top \mathbf{t}_0 d\Gamma_0
% %     \label{Eq: VW1}
% % \end{equation}

% % \noindent Where $\Delta \boldsymbol{\tau} = \mathbf{D} \Delta \boldsymbol{\gamma}$ is used for linear elasticity. Since $\Delta \boldsymbol{\gamma}$ has a linear and quadratic contributions from displacement as in Eq(\ref{Eq: disp1} and \ref{Eq: disp2}), the Green-Lagrange strain increment can be decomposed into linear and non-linear parts as 

% % \begin{equation}
% %     \Delta \boldsymbol{\gamma} = \Delta \mathbf{e} + \Delta \boldsymbol{\eta}
% % \end{equation}

% % \noindent Substituting in Eq(\ref{Eq: VW1})

% % \begin{equation}
% %     \int_{\Omega_0} \delta \Delta \mathbf{e}^\top \mathbf{D} \Delta \mathbf{e} d\Omega_0 + \int_{\Omega_0} \delta \Delta \boldsymbol{\eta}^\top \boldsymbol{\tau}^t d\Omega_0 = \int_{\Gamma_0} \delta \mathbf{u}^\top \mathbf{t}_0 d\Gamma_0 - \int_{\Omega_0} \delta \Delta \mathbf{e}^\top \boldsymbol{\tau}^t d\Omega_0
% %     \label{Eq: VW1}
% % \end{equation}

% % \noindent Where the higher order terms in displacement were deleted. This is because the LHS should reduce to a tangential stiffness matrix, which is the slope of the load-displacement curve at a given point, multiplied by the unknown vector $\mathbf{u}$. Only integrals that are linear in $\mathbf{u}$ can be kept as the tangential stiffness matrix is not function of $\mathbf{u}$. The use of iterative solver removed effects on the accuracy of the solution. Additionally, the term $\int_{\Omega_0} \delta \Delta \mathbf{e}^\top \boldsymbol{\tau}^t d\Omega_0$ is not a function of $\mathbf{u}$ and does not contribute to the tangential stiffness matrix, and thus, is moved to the RHS as a part of the internal force vector, thus 

% % \begin{equation}
% %     \Delta \mathbf{f}_\mathrm{int} = \mathbf{f}^{t+\Delta t}_\mathrm{ext} - \mathbf{f}^t_\mathrm{int}
% % \end{equation}

% % \section{$J_2$ Plasticity -- Isotropic hardening}

% % \subsection{Constitutive relations}

% % \subsubsection{Von Mises stress}

% % The hydrostatic stress is defined as 

% % \begin{equation}
% %     \sigma_\mathrm{h} = \sigma_{ii} = \Tr(\boldsymbol{\sigma}) =  (\sigma_{xx}+\sigma_{yy}+\sigma_{zz})
% % \end{equation}

% % \noindent The deviatoric stress is defines as

% % \begin{equation}
% %     \begin{split}
% %         \boldsymbol{\sigma}' &= \sigma_{ij}-\sigma_{ii}\delta_{ij} = \boldsymbol{\sigma} - \Tr(\boldsymbol{\sigma})\mathbf{I} \\
% %         &= \left[
% %         \begin{array}{ccc}
% %         \sigma_{xx}' &  \sigma_{xy} & \sigma_{xz} \\
% %         \sigma_{yx}  &  \sigma_{yy}'  &  \sigma_{yz} \\
% %         \sigma_{zx} &  \sigma_{zy} & \sigma_{zz}' \\
% %         \end{array}
% %     \right] = \left[
% %         \begin{array}{ccc}
% %         \sigma_{xx}-\sigma_\mathrm{h} &  \sigma_{xy} & \sigma_{xz} \\
% %         \sigma_{yx}  &  \sigma_{yy}-\sigma_\mathrm{h}  &  \sigma_{yz} \\
% %         \sigma_{zx} &  \sigma_{zy} & \sigma_{zz}-\sigma_\mathrm{h} \\
% %         \end{array}
% %     \right]
% % \end{split}
% % \end{equation}

% % \noindent The von Mises, or, \emph{equivalent} stress is written as 

% % \begin{equation}
% %     \begin{split}
% %     \sigma_\mathrm{eq} &= \sqrt{\frac{1}{2}\Big[(\sigma_{xx}-\sigma_{xx})^2+(\sigma_{yy}-\sigma_{zz})^2+(\sigma_{xx}-\sigma_{zz})^2\Big]+3(\sigma_{xy}^2+\sigma_{yz}^2+\sigma_{xz}^2)} \\
% %     &= \sqrt{\sigma_{xx}^2+\sigma_{yy}^2+\sigma_{zz}^2-\sigma_{xx}\sigma_{yy}-\sigma_{yy}\sigma_{zz}-\sigma_{zz}\sigma_{xx}+3(\sigma_{xy}^2+\sigma_{yz}^2+\sigma_{xz}^2)} \\
% %     &= \sqrt{\frac{3}{2}\sigma_{ij}\sigma_{ij}-\frac{1}{2}\sigma_{kk}^2} \\
% %     &= \sqrt{\frac{3}{2}\sigma_{ij}'\sigma_{ij}'} \\
% %     \end{split}
% % \end{equation}

% \subsubsection{Plastic incompressibility}

% \noindent The total strain is the sum of the elastic and plastic strains as 

% \begin{equation}
%     \begin{split}    
%     \boldsymbol{\varepsilon} &= \boldsymbol{\varepsilon}^\mathrm{e} + \boldsymbol{\varepsilon}^\mathrm{p} \\
%     \varepsilon_{ij} &= \varepsilon_{ij}^\mathrm{e} + \varepsilon_{ij}^\mathrm{p}
%     \end{split}
% \end{equation}

% \noindent Since there is no volume change during plastic deformation, the consequence for the plastic strain rate

% \begin{equation}
%     \dot{\varepsilon}_{xx}^\mathrm{p}+\dot{\varepsilon}_{yy}^\mathrm{p}+\dot{\varepsilon}_{zz}^\mathrm{p} = 0
% \end{equation}

% \noindent The equivalent plastic strain rate is defined as

% \begin{equation}
%     \dot{p} =\sqrt{\frac{2}{3}\dot{\boldsymbol{\varepsilon}}^\mathrm{p}:\dot{\boldsymbol{\varepsilon}}^\mathrm{p}} =  \sqrt{\frac{2}{3}\dot{\varepsilon}^\mathrm{p}_{ij}\dot{\varepsilon}^\mathrm{p}_{ij}}
% \end{equation}

% \noindent It should be noted that the plastic strain rates are deviatoric by construction due to the incompressibility condition. The accumulated plastic strain is defined as 

% \begin{equation}
%     p = \int_0^t = \dot{p} \, dt
% \end{equation}

% \subsubsection{Yield criterion}

% The von Mises yield function $f(\boldsymbol{\sigma}, \varepsilon^\mathrm{eq})$ defines a yield surface $f(\boldsymbol{\sigma}, \varepsilon^\mathrm{eq})=0$

% \begin{equation}
%     \begin{split}
%     f &= \sigma_\mathrm{eq} - \sigma_\mathrm{y} - R(p) < 0: \quad \text{Elastic deformation} \\
%     f &= \sigma_\mathrm{eq} - \sigma_\mathrm{y} - R(p) = 0: \quad \text{Plastic deformation} \\
%     \end{split}
% \end{equation}

% \noindent Where $R(p)$ is the hardening function. This simplest case is linear hardening $R(p) = hp$. It could also define a non-linear power law hardening $R(p) = hp^n$

% \subsubsection{Flow rule}

% The flow rule determines the direction of plastic flow \cite{Dunne2005, Doghri2000}. This is accomplished through the \emph{normality condition}. It is states that, in the principal stress space, the increment of plastic strain tensor $d\boldsymbol{\varepsilon}^\mathrm{p}$ is normal to the tangent of the yield surface at the loading point $\frac{\partial f}{\partial \boldsymbol{\sigma}}$, thus 

% \begin{equation}
%     \dot{\boldsymbol{\varepsilon}}^\mathrm{p} = \dot{\lambda} \frac{\partial f}{\partial \boldsymbol{\sigma}} \qquad d\boldsymbol{\varepsilon}^\mathrm{p} = d\lambda \frac{\partial f}{\partial \boldsymbol{\sigma}}  
% \end{equation}

% \noindent Where $\dot{\lambda}$ is the plastic multiplier, which determines the magnitude of plastic strain rate. For a von Mises material, the plastic multiplier is the increment of equivalent plastic strain 

% \begin{equation}
%     \dot{p} = \dot{\lambda} \qquad dp = d\lambda
% \end{equation}

% \noindent Thus, for a von Mises material, the normality condition can be stated as 

% \begin{equation}
%     d\boldsymbol{\varepsilon}^\mathrm{p} = d\lambda \frac{\partial f}{\partial \boldsymbol{\sigma}} = \frac{3}{2} \, dp \frac{\boldsymbol{\sigma'}}{\sigma_\mathrm{eq}}
% \end{equation}

% \noindent With the tensor normal to the yield surface

% \begin{equation}
%     \boldsymbol{N} = \frac{3}{2} \frac{\boldsymbol{\sigma'}}{\sigma_\mathrm{eq}} \quad N_{ij} = \frac{3}{2} \frac{\sigma'_{ij}}{\sigma_\mathrm{eq}}
% \end{equation}

% \subsubsection{Consistency condition}

% The consistency condition states that during plastic deformation, the load point, or the solution, should stay on the yield surface. This condition enables determining the plastic multiplier, or for a von Mises plasticity, the equivalent plastic strain. Mathematically, it could be stated as   

% \begin{equation}
%     f = \sigma_\mathrm{eq} - \sigma_\mathrm{y} - R(p) = 0
% \end{equation}

% \subsection{Return mapping algorithm}

% The total elastic strain at increment $(n+1)$ is written as 

% \begin{equation}
%     \boldsymbol{\varepsilon}^\mathrm{e}_{(n+1)} = (\boldsymbol{\varepsilon}_{(n)} + \Delta \boldsymbol{\varepsilon}) - (\boldsymbol{\varepsilon}^\mathrm{p}_{(n)} + \Delta \boldsymbol{\varepsilon}^\mathrm{p})
% \end{equation}

% \noindent The stress is then

% \begin{equation}
%     \boldsymbol{\sigma}_{(n+1)} = 2G (\boldsymbol{\varepsilon}_{(n)} + \Delta \boldsymbol{\varepsilon} - \boldsymbol{\varepsilon}^\mathrm{p}_{(n)}-\Delta \boldsymbol{\varepsilon}^\mathrm{p}) + \frac{1}{3}\lambda \Tr(\boldsymbol{\varepsilon}_{(n)} + \Delta \boldsymbol{\varepsilon} - \boldsymbol{\varepsilon}^\mathrm{p}_{(n)}-\Delta \boldsymbol{\varepsilon}^\mathrm{p}) \mathbf{I}
% \end{equation}

% \noindent And since $\Tr(\Delta \boldsymbol{\varepsilon}^\mathrm{p}) = 0$

% \begin{equation}
%     \boldsymbol{\sigma}_{(n+1)} = \underbrace{
%         2G (\boldsymbol{\varepsilon}_{(n)} + \Delta \boldsymbol{\varepsilon}- \boldsymbol{\varepsilon}^\mathrm{p}_{(n)}) + \frac{1}{3}\lambda \Tr(\boldsymbol{\varepsilon}_{(n)} + \Delta \boldsymbol{\varepsilon}- \boldsymbol{\varepsilon}^\mathrm{p}_{(n)}) \mathbf{I}}_{\substack{\text{elastic predictor} \\ \text{(trial stress $\boldsymbol{\sigma}^\mathrm{tr}$)}}}
%         \underbrace{
%         -2G \Delta \boldsymbol{\varepsilon}^\mathrm{p}}_{\text{plastic corrector}}
%         \label{Eq: Predictor}
% \end{equation}

% \noindent Eq. (\ref{Eq: Predictor}) could be written as 

% \begin{equation}
%     \begin{split}    
%     \boldsymbol{\sigma}_{(n+1)} = \boldsymbol{\sigma}^\mathrm{tr} - 2G \Delta \boldsymbol{\varepsilon}^\mathrm{p} 
%     &= \boldsymbol{\sigma}^\mathrm{tr} - 2G \Delta p \boldsymbol{N} \\
%     &= \boldsymbol{\sigma}^\mathrm{tr} - 2G \Delta p \frac{3}{2}\frac{{\boldsymbol{\sigma}'}^\mathrm{tr}}{\sigma_\mathrm{eq}}
%     \end{split}
% \end{equation}

% \noindent Since only the deviatoric part of the stress needs to be computed. Omitting the subscript $_{(n+1)}$ for simplicity

% \begin{equation}
%     \begin{split}
%     \boldsymbol{\sigma}' &= {\boldsymbol{\sigma}'}^\mathrm{tr} - 2G \Delta p \boldsymbol{N} \\
%     \frac{2}{3} {\sigma_\mathrm{eq}}\boldsymbol{N} &= \frac{2}{3}{\sigma_\mathrm{eq}}^\mathrm{tr}\boldsymbol{N} - 2G \Delta p \boldsymbol{N} \\
%     {\sigma_\mathrm{eq}} &= \sigma_\mathrm{eq}^\mathrm{tr} - 3G \Delta p
%     \end{split}
% \end{equation}

% \noindent Using the yield function

% \begin{equation}
%     f = \sigma_\mathrm{eq}^\mathrm{tr} - 3G \Delta p \underbrace{- R(p) - \sigma_\mathrm{y}}_{\text{$\sigma_\mathrm{eq}$}} = 0
% \end{equation}

% \noindent Therefore, the problem reduces to solving a non-linear scalar equation for the unknown $\Delta p$. Using Taylor's expansion 

% \begin{equation}
%     f + \frac{df}{d\Delta p} d\Delta p + \cdots = 0 , \, \text{with} \, \frac{df}{d\Delta p}= - 3G - \frac{dR(p)}{dp} 
% \end{equation}

% \noindent We can write the increment of the equivalent plastic strain as 

% \begin{equation}
%     d\Delta p = \frac{\sigma_\mathrm{eq}^\mathrm{tr} - 3G \Delta p - R(p) - \sigma_\mathrm{y}}{3G + \frac{dR(p)}{dp}}
% \end{equation}

% \noindent Now, we can solve for the iterations $(k)$

% \begin{equation}
%     \begin{split}
%     d\Delta p &= \frac{\sigma_\mathrm{eq}^\mathrm{tr} - 3G \Delta p^{(k)} - R(p^{(k)}) - \sigma_\mathrm{y}}{3G + \frac{dR(p^{(k)})}{dp}}  \\ 
%     \Delta p^{(k+1)} & = \Delta p^{(k)} + d\Delta p \\  
%     p^{(k+1)} &= p^{(k)} + \Delta p^{(k+1)} \\
%     f^{(k+1)} &= \sigma_\mathrm{eq}^\mathrm{tr} - 3G \Delta p - R(p) - \sigma_\mathrm{y}
%     \end{split}
% \end{equation}

% \noindent Once a convergence criterion $|f^{(k+1)}\le\mathrm{tol}|$ is achieved, the solution at $(n+1)$ can be evaluated as

% \begin{equation}
%     \begin{split}
%     % \sigma_\mathrm{eq} &= \sigma_\mathrm{eq}^\mathrm{tr} - 3G \Delta p \\
%     % \boldsymbol{\sigma}' &= \frac{2}{3}{\sigma_\mathrm{eq}}\mathbf{n}  \\
%     \boldsymbol{\varepsilon}^\mathrm{p} &= \frac{3}{2} \, \Delta p \boldsymbol{N} = \frac{3}{2} \, \Delta p \frac{{\boldsymbol{\sigma}'}^\mathrm{tr}}{\sigma_\mathrm{eq}}  \\
%     \boldsymbol{\varepsilon}^\mathrm{e}&= \boldsymbol{\varepsilon} - \boldsymbol{\varepsilon}^\mathrm{p} \\
%     \boldsymbol{\sigma} &= 2G \boldsymbol{\varepsilon}^\mathrm{e} + \lambda \Tr(\boldsymbol{\varepsilon}^\mathrm{e}) \mathbf{I} 
%     \end{split}
% \end{equation}

% \subsection{Tangential stiffness matrix}

% The non-linear material behavior will result in a non-linear system of equations Eq. (\ref{Eq: StressEquilibrium}) which can be solved using the Newton-Raphson method. This requires the linearization of the stress-strain relation

% \begin{equation}
%     \delta \boldsymbol{\sigma} = \frac{\partial \boldsymbol{\sigma}}{\partial \boldsymbol{\varepsilon}} \delta \boldsymbol{\varepsilon} = \mathbf{D} \delta \boldsymbol{\varepsilon}
% \end{equation}

% \noindent Where $\mathbf{D}$ is the tangential\footnote{Sometimes called material Jacobean} stiffness matrix. For isotropic hardening plasticity, $\mathbf{D}$\footnote{Note that the matrix form is in Voigt notation, which has slightly different terms.} can be calculated from 

% \begin{equation}
%     \mathbf{D} = \mathbf{C^\mathrm{e}} - \frac{H_\mathrm{mod}}{H_\mathrm{mod}+3G} \boldsymbol{N} \otimes \boldsymbol{N}
%     \label{Eq: IsoHardTangStiffness}
% \end{equation}

% \noindent Or in index notation 

% \begin{equation}
%     C_{ijkl}^{\mathrm{ep}} = C_{ijkl}^{\mathrm{el}} - \frac{H_{\mathrm{mod}}}{H_{\mathrm{mod}} + 3G} N_{ij} N_{kl}
% \end{equation}

% \section{The Newton-Raphson method for non-linear systems}

% The Newton-Raphson method is used for solving nonlinear equations. Its implementation is based on the evaluation of the nonlinear function and its derivative. The Taylor expansion of $f(x + \Delta x)$ around $x$ is 

% \begin{equation}
%     f(x + \Delta x) = f(x) + f' \Delta x + \cdots
% \end{equation}

% \noindent Where $f'=df(x)/dx$. The root of the function could be reached by iterating the scheme:

% \begin{equation}
%     x_{n+1} = x_n - \frac{f(x_n)}{f'(x_n)}
% \end{equation}

% \begin{figure}[hbt!]
%     \includegraphics[width=8cm]{Newton_iteration.pdf}
%     \centering
%     \caption{Geometric interpretation of Newton-Raphson method. $x_{n+1}$ is a better estimate than the initial guess $x_{n}$.} 
%     \label{Fig: 1DPF}
% \end{figure}

% \subsection{Incremental iterative analysis}

% The balance of momentum equation of load increment can be written as 

% \begin{equation}
%     \begin{split}
%     \mathbf{f}^{(n+1)}_\mathrm{ext} - \mathbf{f}^{(n)}_\mathrm{int} - \Delta \mathbf{f}_\mathrm{int} &= 0 \\
%     \mathbf{f}^{(n+1)}_\mathrm{ext} - \mathbf{f}^{(n)}_\mathrm{int} - \mathbf{K} \Delta \mathbf{u} &= 0 \\
%     \mathbf{f}^{(n+1)}_\mathrm{ext} - \mathbf{f}^{(n)}_\mathrm{int} - \mathbf{R} &= 0 \\
%     \end{split}
% \end{equation}

% \noindent The system becomes non-linear due to the non-linear material behavior, which is solved using the Newton-Raphson method. The process for each loading increment, omitting the loading increment, is summarized in Algorithm (\ref{Alg: NRSmallPlast})

% \begin{algorithm}[H]
%     \caption{Newton-Raphson Algorithm for Small Strain Plasticity.}
%     \textbf{Initialize:} $j \gets 0$, $\mathbf{u}^{(0)}$ \;
%     \While{$\| \mathbf{R}^{(j)} \| > \text{Tolerance}$ \textbf{and} $j < \text{Max Iterations}$}{
%         \textbf{1.} Calculate the total strain: 
%         \[
%         \boldsymbol{\varepsilon}^{(j)} = \sum \mathbf{B} \mathbf{u}^{(j)} 
%         \]
        
%         \textbf{2.} Update state variables using radial return mapping:
%         \[
%         \boldsymbol{\varepsilon}^{\mathrm{p}(j)}, \, \boldsymbol{\varepsilon}^{\mathrm{e}(j)}, \, \boldsymbol{\sigma}^{(j)}, \, \mathbf{D}^{\mathrm{ep}}
%         \]

%         \textbf{3.} Calculate the internal force: 
%         \[
%         \mathbf{f}_\mathrm{int}^{(j)} = \int_\Omega \mathbf{B}^\top \boldsymbol{\sigma}^{(j)} \, d\Omega
%         \]

%         \textbf{4.} Calculate the residual:
%         \[
%         \mathbf{R}^{(j)} = \mathbf{f}_\mathrm{ext} - \mathbf{f}_\mathrm{int}^{(j)}
%         \]

%         \textbf{5.} Assemble the tangent stiffness matrix:
%         \[
%         \mathbf{K}^{(j)} = \int_\Omega \mathbf{B}^\top \mathbf{D}^{\mathrm{ep}} \mathbf{B} \, d\Omega
%         \]

%         \textbf{6.} Solve the linear system:
%         \[
%         \Delta \mathbf{u}^{(j+1)} = \mathbf{K}^{(j)^{-1}} \mathbf{R}^{(j)}
%         \]

%         \textbf{7.} Update displacement:
%         \[
%         \mathbf{u}^{(j+1)} = \mathbf{u}^{(j)} + \Delta \mathbf{u}^{(j+1)}
%         \]

%         \textbf{8.} Check Convergence:
%         \If{$\| \mathbf{R}^{(j+1)} \| \leq \text{Tolerance}$}{
%             \textbf{Break}
%         }
%     }
%     \textbf{Return:} Converged displacement $\mathbf{u}^{(j+1)}$ \;
%     \label{Alg: NRSmallPlast}
% \end{algorithm}

% \begin{algorithm}[H]
%     \caption{Double-Obstacle Potential for Three-Phase Interfaces.}
%     \textbf{Initialize:} $g\Phi_{ii} \gets 0$, $g\Phi_{ij} \gets 0$, $g\Phi_{ik} \gets 0$, $g\Phi_{jj} \gets 0$, $g\Phi_{jk} \gets 0$, $g\Phi_{kk} \gets 0$ \;
%     Define sets of grains: $\text{choices}_i$, $\text{choices}_j$, $\text{choices}_k$ \;
    
%     \For{$i \in \{1, \dots, N\}$}{
%         \For{$j \in \{1, \dots, i-1\}$}{

%             \textbf{1.} Compute the interaction term:
%             \[
%             \text{term} \gets \phi_i \cdot \phi_j
%             \]
            
%             \textbf{2.} Update $g\Phi$ contributions based on grain categories:
%             \begin{itemize}
%                 \item If $i, j \in \text{choices}_i$:
%                 \[
%                 g\Phi_{ii} \mathrel{+}= \text{term}
%                 \]
%                 \item Else if $i, j \in \text{choices}_j$:
%                 \[
%                 g\Phi_{jj} \mathrel{+}= \text{term}
%                 \]
%                 \item Else if $i, j \in \text{choices}_k$:
%                 \[
%                 g\Phi_{kk} \mathrel{+}= \text{term}
%                 \]
%                 \item Else if $i \in \text{choices}_i$ and $j \in \text{choices}_j$ (or vice versa):
%                 \[
%                 g\Phi_{ij} \mathrel{+}= \text{term}
%                 \]
%                 \item Else if $i \in \text{choices}_i$ and $j \in \text{choices}_k$ (or vice versa):
%                 \[
%                 g\Phi_{ik} \mathrel{+}= \text{term}
%                 \]
%                 \item Else if $i \in \text{choices}_j$ and $j \in \text{choices}_k$ (or vice versa):
%                 \[
%                 g\Phi_{jk} \mathrel{+}= \text{term}
%                 \]
%             \end{itemize}
%         }
%     }
%     \textbf{Return:} Updated $g\Phi$ terms: $g\Phi_{ii}, g\Phi_{ij}, g\Phi_{ik}, g\Phi_{jj}, g\Phi_{jk}, g\Phi_{kk}$ \;
%     \label{Alg: DoubleObstacle}
% \end{algorithm}

  



% % \appendix
% % \setcounter{figure}{0}
% % \setcounter{equation}{0}
% % \counterwithin{figure}{section}
% % \counterwithin{equation}{section}

% % \section{Atomic fractions and volumes}

% % Consider a system of $N_c$ atoms consisting of $N_i$ atoms of type $i$ occupying a 
% % total volume of $V^{tot}$. The concentration $c_i$ (number density) of component $i$
% % per unit volume 

% % \begin{equation}
% %     c_i = \frac{N_i}{V^{tot}} \ \ \ \ \ \ \   \sum_{i=1}^{N_c} c_i = c^{tot}
% %     \label{conc}
% % \end{equation}

\printbibliography

\end{document} 