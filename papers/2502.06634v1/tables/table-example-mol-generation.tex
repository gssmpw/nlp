\begin{table*}[!ht]
\centering
\resizebox{.8\linewidth}{!}{
\begin{tabular}{>{\arraybackslash}p{10cm} p{3cm} p{3cm} p{3cm}} 
    \toprule
    \textbf{Caption} & \textbf{Ground Truth} & \textbf{\oldmodel-Small} & \textbf{\newmodel-Small} \\ 
    \midrule
    \textbf{(1): }
    The molecule is an \textcolor{violet}{indolylmethylglucosinolate} that is the conjugate base of 4-methoxyglucobrassicin, obtained by deprotonation of the sulfo group. It is a conjugate base of a 4-methoxyglucobrassicin. & \raisebox{-\totalheight}{\includegraphics[width=3cm,height=3cm]{pics/gt1.png}} & \raisebox{-\totalheight}{\includegraphics[width=3cm,height=2.5cm]{pics/old1.png}} & \raisebox{-\totalheight}{\includegraphics[width=3cm,height=3cm]{pics/new1.png}} \\ 
    \midrule
    \textbf{(2): }
    The molecule is an \textcolor{violet}{N-carbamoylamino acid} that is aspartic acid with one of its \textcolor{violet}{amino hydrogens} \textcolor{oceanboatblue}{replaced} by a \textcolor{violet}{carbamoyl} group. It has a role as a Saccharomyces cerevisiae metabolite, an Escherichia coli metabolite and a human metabolite. It is a N-carbamoyl-amino acid, an aspartic acid derivative and a C4-dicarboxylic acid. It is a conjugate acid of a N-carbamoylaspartate(2-). & \raisebox{-\totalheight}{\includegraphics[width=3cm,height=2.cm]{pics/gt2.png}} & \raisebox{-\totalheight}{\includegraphics[width=3cm,height=1.5cm]{pics/old2.png}} & \raisebox{-\totalheight}{\includegraphics[width=3cm,height=2.cm]{pics/new2.png}} \\ 
    % \midrule
    % \textbf{(3): }
    % The molecule is the conjugate base of Photinus luciferin; major species at pH 7.3. It is a conjugate base of a Photinus luciferin. It is an enantiomer of an ent-Photinus luciferin(1-). & \raisebox{-\totalheight}{\includegraphics[width=3cm,height=3cm]{pics/gt3.png}} & \raisebox{-\totalheight}{\includegraphics[width=3cm,height=3cm]{pics/old3.png}} & \raisebox{-\totalheight}{\includegraphics[width=3cm,height=3cm]{pics/new3.png}} \\
    \bottomrule 
\end{tabular}
}
\caption{
Example of molecules generated with the same input descriptions. 
Generated SMILES strings are converted to molecule graphs for better visualisation.
}
\vspace{-3mm}
\label{table:example_generated_molecule}
\end{table*}
