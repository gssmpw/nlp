\begin{table*}[!ht]
\centering
\resizebox{.8\linewidth}{!}{
\begin{tabular}{>{\arraybackslash}p{.5cm} p{3.2cm} p{6.cm} p{6.cm} p{6.cm}} 
    \toprule
    \textbf{ID} & \textbf{Molecule} & \textbf{Ground Truth} & \textbf{\oldmodel-Small} & \textbf{\newmodel-Small} \\ 
    % \midrule 
    % \textbf{1} & \raisebox{-\totalheight}{\includegraphics[width=3.5cm,height=3.5cm]{pics/sample1.png}} & The molecule is an indolylmethylglucosinolate that is the conjugate base of 4-methoxyglucobrassicin, obtained by deprotonation of the sulfo group. It is a conjugate base of a 4-methoxyglucobrassicin. & The molecule is an indolylmethylglucosinolate that is the conjugate base of glucobrassicin, obtained by deprotonation of the sulfo group. It is a conjugate base of a glucobrassicin. & The molecule is an indolylmethylglucosinolate that is the conjugate base of 4-methoxyglucobrassicin, obtained by deprotonation of the sulfo group; major species at pH 7.3. It is a conjugate base of a 4-methoxyglucobrassicin. \\ 
    \midrule
    \textbf{1} & \raisebox{-\totalheight}{\includegraphics[width=3.2cm,height=3.2cm]{pics/sample2.png}} & The molecule is an N-carbamoylamino acid that is aspartic acid with one of its amino hydrogens replaced by a carbamoyl group. It has a role as a Saccharomyces cerevisiae metabolite, \textcolor{violet}{an Escherichia coli metabolite and a human metabolite}. It is a N-carbamoyl-amino acid, an aspartic acid derivative and a C4-dicarboxylic acid. It is a conjugate acid of a N-carbamoylaspartate(2-). & The molecule is a member of the class of ureas that is urea in which one of the amino hydrogens is replaced by a carbamoyl group. It has a role as a metabolite. It is a N-acyl-amino acid and a member of ureas. & The molecule is a member of the class of ureas that is malonic acid in which one of the hydrogens attached to the nitrogen is substituted by a carbamoyl group. It has a role as \textcolor{violet}{an Escherichia coli metabolite and a mouse metabolite}. It is a member of ureas and a member of ureas. It derives from a malonic acid. It is a conjugate acid of a N-carbamoylglycinate. \\ 
    \midrule
    \textbf{2} & \raisebox{-\totalheight}{\includegraphics[width=3.2cm,height=3.2cm]{pics/sample3.png}} & The molecule is the conjugate base of Photinus luciferin; major species at pH 7.3. It is a conjugate base of a Photinus luciferin. It is an enantiomer of an ent-Photinus luciferin(1-). & The molecule is a monocarboxylic acid anion that is the conjugate base of ent-Photinus luciferin, obtained by deprotonation of the carboxy group; major species at pH 7.3. It is a conjugate base of an ent-Photinus luciferin. It is an enantiomer of a Photinus luciferin(1-). & The molecule is a monocarboxylic acid anion that is the conjugate base of Photinus luciferin, obtained by deprotonation of the carboxy group; major species at pH 7.3. It is a conjugate base of a Photinus luciferin. It is an enantiomer of a Photinus luciferin(1-). \\ 
    \bottomrule    
\end{tabular}
}
\caption{
Example of captions generated with the same input SMILES strings. 
Input SMILES strings are converted to molecule graphs for better visualisation.
}
\label{table:example_generated_caption}
\vspace{-3mm}
\end{table*}
