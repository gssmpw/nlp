% In addition, a cover letter needs to be written with the
% following:
% \begin{enumerate}
%  \item A 100 word or less summary indicating on scientific grounds
% why the paper should be considered for a wide-ranging journal like
% \textsl{Nature} instead of a more narrowly focussed journal.
%  \item A 100 word or less summary aimed at a non-scientific audience,
% written at the level of a national newspaper.  It may be used for
% \textsl{Nature}'s press release or other general publicity.
%  \item The cover letter should state clearly what is included as the
% submission, including number of figures, supporting manuscripts
% and any Supplementary Information (specifying number of items and
% format).
%  \item The cover letter should also state the number of
% words of text in the paper; the number of figures and parts of
% figures (for example, 4 figures, comprising 16 separate panels in
% total); a rough estimate of the desired final size of figures in
% terms of number of pages; and a full current postal address,
% telephone and fax numbers, and current e-mail address.
% \end{enumerate}

% See \textsl{Nature}'s website
% (\texttt{http://www.nature.com/nature/submit/gta/index.html}) for
% complete submission guidelines.


%%%%%%%%%%%%%%%%%%%%%%%%%%%%%%%%%%%%%%%%%
% Long Lined Cover Letter
% LaTeX Template
% Version 1.0 (1/6/13)
%
% This template has been downloaded from:
% http://www.LaTeXTemplates.com
%
% Original author:
% Matthew J. Miller
% http://www.matthewjmiller.net/howtos/customized-cover-letter-scripts/
%
% License:
% CC BY-NC-SA 3.0 (http://creativecommons.org/licenses/by-nc-sa/3.0/)
%
%%%%%%%%%%%%%%%%%%%%%%%%%%%%%%%%%%%%%%%%%

%----------------------------------------------------------------------------------------
%	PACKAGES AND OTHER DOCUMENT CONFIGURATIONS
%----------------------------------------------------------------------------------------

\documentclass[10pt,stdletter,dateno,sigleft]{newlfm} % Extra options: 'sigleft' for a left-aligned signature, 'stdletternofrom' to remove the from address, 'letterpaper' for US letter paper - consult the newlfm class manual for more options

%\usepackage{charter} % Use the Charter font for the document text

\newsavebox{\Luiuc}\sbox{\Luiuc}{\parbox[b]{1.75in}{\vspace{0.3in}
\includegraphics[width=1.0\linewidth]{imgs/mit_logo.jpg}}} % Company/institution logo at the top left of the page
\makeletterhead{Uiuc}{\Lheader{\usebox{\Luiuc}}}

\newlfmP{sigsize=30pt} % Slightly decrease the height of the signature field
\newlfmP{addrfromphone} % Print a phone number under the sender's address
\newlfmP{addrfromemail} % Print an email address under the sender's address
\PhrPhone{Phone} % Customize the "Telephone" text
\PhrFax{Fax}
\PhrEmail{Email} % Customize the "E-mail" text

\lthUiuc % Print the company/institution logo

%----------------------------------------------------------------------------------------
%	YOUR NAME AND CONTACT INFORMATION
%----------------------------------------------------------------------------------------

\namefrom{\includegraphics[scale=0.5]{prof_signature.jpg} \\
Prof. Cynthia Breazeal} % Name


\addrfrom{
\today\\[12pt] % Date
Massachusetts Institute of Technology \\
% Dept. Such and Such \\
E14-474 75 Amherst Street Cambridge, MA, 02139 
}

\phonefrom{(+1) 617-452-5601} % Phone number



\emailfrom{cynthiab@media.mit.edu} % Email address
%----------------------------------------------------------------------------------------
%	ADDRESSEE AND GREETING/CLOSING
%----------------------------------------------------------------------------------------

\greetto{Dear Sir or Madam Editor,} % Greeting text

\closeline{Sincerely yours,} % Closing text


%\nameto{Editor, Nature Magazine} % Addressee of the letter above the to address

%\addrto{
%Recruitment Officer \\ % To address
%The Corporation \\
%123 Pleasant Lane \\
%City, State 12345
%}

%----------------------------------------------------------------------------------------

\begin{document}
\begin{newlfm}

%----------------------------------------------------------------------------------------
%	LETTER CONTENT
%----------------------------------------------------------------------------------------

% http://www.nature.com/nature/authors/submissions/subs/#a6
%Submissions should be accompanied by a brief covering letter from the corresponding author including full postal address, telephone number and e-mail address. This letter should contain two (100-word or shorter) summaries: a concise paragraph to the editor indicating the scientific grounds why the paper should be considered for a topical, interdisciplinary journal rather than for a single-discipline or archival journal; and a separate, 100-word summary of the paper's appeal to a popular (non-scientific) audience.
%
%The cover letter should state clearly what is included as the submission, including number of words in the text and number of display items (figures, tables, boxes) in the print version of the paper; number of additional words in the text (full Methods and Extended Data legends) and number of Extended Data figures and tables for the online-only version; any Supplementary Information (specifying number of items and format); number of supporting manuscripts.


\noindent We are submitting our manuscript, ``Medical Hallucinations in Foundation Models and Their Impact on Healthcare'', for consideration for publication in \textit{Nature Medicine}.  This work addresses a critical and timely challenge concerning the reliability of AI in medicine, which we believe is exceptionally well-suited for your diverse readership.

\medskip

\noindent Our paper introduces the concept of \textit{medical hallucination}, defining it as the generation of inaccurate or misleading medical information by Foundation Models, particularly Large Language Models (LLMs), that could adversely affect clinical decision-making and patient outcomes. We present a structured taxonomy of these medical hallucinations based on their characteristics and underlying causes, and show how these error types closely mirror cognitive biases in human clinical practice. Through rigorous experiments using state-of-the-art LLM models such as o1, Gemini-2.0-Flash, and Meditron-7B with medical hallucination benchmarks, along with insights from a multi-national clinician survey (n=70), we demonstrate the scale of the problem. We find that  different inference techniques, such as Chain of Thought (CoT), Retrieval-Augmented Generation (RAG), and the incorporation of Internet search, influence hallucination rates. Our study reveals the practical and ethical implications, revealing the potential for patient harm if LLMs are deployed without robust safeguards. 

\medskip

\noindent Our manuscript is approximately XXX words in length (excluding references). It contains 3 figures consisting of 8 separate panels, and we have included Supplementary Information with XXX Extended Data figures. The estimated total length of our report is XXX pages in \textit{Nature Medicine}, with the figures occupying approximately XXX pages.  All authors approve of this submission.

\medskip

\noindent We have no related manuscripts under consideration or in press elsewhere, nor have we had any prior discussions with a \textit{Nature Medicine} editor about this work.

\noindent \textbf{Summary to the Editor:}  Our work is of broad scientific interest because it provides a much-needed framework for addressing a rapidly growing challenge related to the use of AI in healthcare, one with broad implications across AI research and all medical specialities. The increasing use of AI in healthcare demands careful scrutiny of its potential drawbacks. This study goes beyond simply identifying the issue of \textit{medical hallucinations}, to providing practical solutions and guidance for the medical and AI communities.  As such, this work is best situated for an interdisciplinary journal such as Nature Medicine.

\medskip
\noindent \textbf{Summary to the Public:}  AI is becoming common in healthcare, but we show that AI models can generate convincing medical misinformation, referred to as \textit{medical hallucinations}, which could endanger patients.  Our study categorizes these errors, examines how they arise, and demonstrates that even advanced AI models can make dangerous mistakes. We emphasize the importance of careful control, regulation, and verification of AI systems in medicine to ensure patient safety. This framework will support the responsible application of AI in healthcare.

\medskip

\noindent Please do not hesitate to contact us with any questions or comments. We look forward to hearing from you.

%----------------------------------------------------------------------------------------

\end{newlfm}
\end{document}