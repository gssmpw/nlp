

\newcommand{\case}[2]{{%
%\leavevmode\color{blue}%
\textit{``#2''}}}

\newcommand{\caseTable}[2]{{%
%\leavevmode\color{blue}%
\textit{#2}}}

\newcommand{\tosdr}[1]{{%
%\leavevmode\color{blue}%
ToS;DR}}

\newcommand{\qOne}[1]{What is your age?}
\newcommand{\qTwo}[1]{What is your occupation?}
\newcommand{\qThree}[1]{What is your Nationality?}
\newcommand{\qFour}[1]{Q1. How motivated are you to read privacy policies before using an online service to make sure your information is safe?
(1 means `Not motivated at all' and 10 means `Extremely motivated')}
\newcommand{\qFive}[1]{Q2. How likely are you to invest time in understanding privacy policies to avoid potential privacy issues?
(1 means ‘Very unlikely’ and 10 means ‘Very likely’)}
\newcommand{\qSix}[1]{Q3. Assume you are reviewing a privacy policy, do you focus more on specific sections or a general overview?
(1 means ‘Focus much more on specific sections’ and 10 means ‘Focus much more on general overview’)}
\newcommand{\qSeven}[1]{Q4. How confident are you in your ability to manage privacy settings on social media platforms?
(1 means `Not confident at all’ and 10 means ‘Extremely confident’)}
\newcommand{\qEight}[1]{Q5. How comfortable are you with using technologies that protect your privacy, such as virtual private networks (VPNs), or inspect your privacy hygiene, such as Ghostery?
(1 means ‘Not comfortable at all’ and 10 means ‘Very comfortable’)}
\newcommand{\qNine}[1]{Q6. How important is it for you to feel in control of your personal information online?
(1 means ‘Not important at all’ and 10 means ‘Extremely important’)}
\newcommand{\qTen}[1]{Q7. How proactive are you in updating your privacy settings regularly?
(1 means ‘Never proactive’ and 10 means ‘Always proactive’)}
\newcommand{\qEleven}[1]{Q8. How comfortable are you with sharing personal information with online services you use regularly?
(1 means ‘Not comfortable at all’ and 10 means ‘Very comfortable’)}
\newcommand{\qTwelve}[1]{Q9. Despite the risks, how often do you accept terms and conditions before reading them?
(1 means ‘Never’ and 10 means ‘Always’)}
\newcommand{\qThirteen}[1]{Q10. How likely are you to take action (e.g., change settings, stop using a service) if you find a privacy policy concerning?
(1 means ‘Very unlikely’ and 10 means ‘Very likely’)}
\newcommand{\qFourteen}[1]{Q11. Suppose educational resources or tools were available to help you better understand privacy policies, how much would you be interested to learn more about such resources?
(1 means ‘Not interested at all’ and 10 means ‘Extremely interested’)}
\newcommand{\qFifteen}[1]{Have you ever had any of the following negative experiences due to not understanding or not reading a privacy policy?
(E.g., Account getting hacked, Data used elsewhere, Account discontinued, no refund, etc.)\\
Account Hacked\\
Data Breach\\
Received recommendations without opting-in\\
Received personalized ads without opting-in\\
Failed to obtain a refund\\
Observed unexpected charges\\
Account suspended/terminated\\
Misunderstood your rights or responsibilities\\
Data Misused in some other way\\
Other (Please specify)}


\newcommand{\user}[3]{{%     JED: escape characters at the end of lines in commands prevents stray spaces from popping up        
%\leavevmode\color{cyan}%            % Use  this line to add color
#2#3-#1}}                           

% switches the text color for a block of text, showing reviewers and co-authors that it has been edited since they last viewed it
\newcommand{\revised}[1]{{%
%\leavevmode\color{blue}%             % Use  this line to add color
#1}}

% replaces text with boxes. Comment this out and fix the compile errors to unredact
\newcommand{\REDACT}[1]{$\Box REDACTED \Box$} 

% replaces text with boxes. Comment this out and fix the compile errors to unredact
\newcommand{\redactCollege}[1]{[a U.S. University]}  

% a printer that gives you a \cite{} analog with different styling, which is useful to avoid repetition when the venue uses the (author, year) format and you want to refer to the authors as the subject of a sentence.
\newcommand{\briefCite}[1]{\citeauthor{#1} (\citeyear{#1})}

%%%%%%%% QUOTATION STYLING COMMANDS %%%%%%%%
% this quote is intended for use when quoting at length and indenting the whole block, and positions the arguments appropriately
% note that the first argument is the PID, second and third are IGNORED, and fourth is the quote.
\newcommand{\quotateInset}[4]{%         
\vspace{-3pt}%
\begin{quote}%
     \leftskip-10pt%\parindent% this sets the length of the left margin to the length of an indent
     \rightskip-15pt%\parindent% same deal on the right
%\leavevmode\color{brown}% Use  this line to add color
\textbf{#1}: \emph{``#4''}\end{quote}%
\vspace{3pt}}

% an alternate quote command, feel free to mess with these
\newcommand{\quotate}[4]{{%
     \leftmargin1.5\parindent% this sets the length of the left margin to the length of an indent
     \rightmargin1\parindent% same deal on the right
%\leavevmode\color{teal}% Use  this line to add color
#1: \emph{``#4''}}}

%%%%%%%%%%%% Counters the commands will need, then attach behaviors to the right commands
\newcounter{boldifyCounter}
\newcounter{fixmeSectionCounter}
\newcounter{fixmeTotalCounter}
\makeatletter
\@addtoreset{fixmeSectionCounter}{section}
\@addtoreset{fixmeSectionCounter}{subsection}
\@addtoreset{boldifyCounter}{section}
\@addtoreset{boldifyCounter}{subsection}
\makeatother

%%%%%%%%%%%%% Drafting commands %%%%%%%%%%%%%%%%%%%%%%
% this command is hidden text, intended to be toggled from the main by defining a \boldifyON command
\newcommand{\boldify}[1]{}% boldify takes 1 argument and ignores it when boldifies are OFF
\ifdefined\boldifyON
	\renewcommand{\boldify}[1]{% ... and it does more when they are on
        \par\noindent%  suppress the indent for this new paragraph
		\stepcounter{boldifyCounter}% we are handling a boldify now, so we need to increment
		\textbf{{\color{green}**}% get the nice green stars we love
		~\arabic{section}.\arabic{subsection}.\arabic{boldifyCounter}% grab the current section numbers and format them as arabic
		: #1} % and finally the boldify itself
	}
\fi % end the boldify if. Next line tells the word counter to IGNORE all text in this command (it is hidden!)
%TC:macro \boldify 1

% this command generates a list of all the FIXME in the document
\newcommand{\reportOnFIXME}{%
    \newcount\iterCounter
    \iterCounter=1
    \newcount\endCounter
    \endCounter=\totvalue{fixmeTotalCounter}
    \advance \endCounter +1
    There are 
    {\color{red}\total{fixmeTotalCounter}} 
    FIX\_ME\\
    links:
    %\begin{enumerate}[nosep]
    \loop
        %\item 
        \hyperlink{fixTag\the\iterCounter}{\#\the\iterCounter}
        \advance \iterCounter +1
    \ifnum \iterCounter < \endCounter
    \repeat
    %\end{enumerate}
}

% This command is meant to be an eyecatching way to communicate with self or colleagues
\newcommand{\FIXME}[1]{} % FIXME takes 1 argument and ignores it when FIXME is OFF
\ifdefined\fixmeON
	\renewcommand{\FIXME}[1]{\par\noindent% ... and it does more when they are on
		\stepcounter{fixmeSectionCounter}\stepcounter{fixmeTotalCounter}% we are handling a FIXME, need to increment
		{\color{red}\fbox{\color{black}% get that nice red framebox that we love, and switch the text back to black
			\parbox{.965\linewidth}{% put the text inside an invisible box of this size
				\textbf{\hypertarget{fixTag\thefixmeTotalCounter}{FIXME}	\arabic{section}.\arabic{subsection}.% get the section numbers and format as arabic
        		\arabic{fixmeSectionCounter} (\color{red}% switch to red to show the total counter
        		\#\arabic{fixmeTotalCounter}):} #1}}% and then finally the FIXME text itself
        }
	}
\fi % end the FIXME if. Next line tells the word counter to IGNORE all text in this command (it is hidden!)
%TC:macro \FIXME 1

% this is basically the same as FIXME, but without the counters
\newcommand{\FIXED}[1]{}
\ifdefined\fixmeON
	\renewcommand{\FIXED}[1]{\par\noindent%
		{\color{black}\fbox{\color{black}%
			\parbox{.99\columnwidth}{%
				\color{blue}#1}}%
        }
	}
\fi % end the FIXED if. Next line tells the word counter to IGNORE all text in this command (it is hidden!)
%TC:macro \FIXED 1

% this command is hidden text, intended to be toggled from the main by defining a \draftStatusOn command
\newcommand{\draftStatus}[1]{}% draftStatus takes 2 argument and ignores it when draft statuses are OFF
\ifdefined\draftStatusON
	\renewcommand{\draftStatus}[1]{% ... and it does more when they are on
        \hfill **#1
	}
\fi % end the draft status if. Next line tells the word counter to IGNORE all text in this command (it is hidden!)
%TC:macro \draftStatus 1


