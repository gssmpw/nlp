\section*{Results RQ4 - How did demographic traits affect understandability and severity?}
%\label{secRQ4} %Commented since section is defunct

\FIXME{JED@SSN, JED@SNS: For R2-4, we basically swap out this section for a comparative one. Be cognizant of R2-5 and provide signposting and summarization of argumentation structure at the beginning and R3-7's concerns about muddled results.
\\ SSN @JED, @SNS: So far this section expands on the descriptive stats of the demo traits. We haven't touched on how it affects U \& S.
Either we can rename this section and create a new one with U \& S, or just include the box plots here and expand. Please advise\\
JED: the current hypothesis testing section is intended to mostly REPLACE this section. It seems we added some citations, so we should put those in the aforementioned new results section, another results section, or the new discussion. Am open to hybrid approach, but I think it might be hard to implement}




The survey, conducted through the Prolific platform, got responses from 500 participants\footnote{We had demographic information from the one participant whose survey response was unrecoverable.} from the United States, who came from a very diverse demographic distribution.
Based on the information we collected, we observed the following:

\paragraph{Age} 
Based on the age data, the mean age of the participants is 35.65 years, and half of the participants are below the age of 33.
The standard deviation of 11.44 indicates a considerable heterogeneity in the age distribution among the subjects.
The estimated standard error of 0.51 indicates a little margin of error in calculating the population mean, therefore indicating that the sample offers an accurate estimation of the average age of the participants.
In general, the data indicates a wide range of ages, characterised by a central trend around the early to mid-30s.
This distribution aligns with prior findings by authors in this article~\cite{lythreatis2022digital}, who noted that younger users often engage more actively with digital platforms but may undervalue privacy risks, while older users tend to exhibit higher privacy concerns due to greater awareness of potential data misuse.

\paragraph{Gender}
The gender distribution in the dataset indicates that women constitute the majority of participants, representing 52.2\% of the total sample. 
Men constitute 46.8\% of the participants, with 1\% opting not to reveal their gender. These results reflect prior work by Hoy and Milne~\cite{hoy2010gender}, who observed that women are generally more privacy-conscious and hesitant to share personal information online, while men often prioritize functionality over privacy.
%The reasonably even distribution implies a diversified composition.
%However, a small subset of individuals who choose not to disclose their gender shows that the process of collecting data is following principles of inclusion.
%In future analysis, the almost equal distribution of males and females might give significant insights into any gender-based disparities.


\paragraph{Ethnicity}
The slight majority of the participants we sampled identified as White, accounting for 57.6\%.
A total of 13.4\% of the participants identified as Black, 12.2\% identified as Asian, and 10.4\% identified as mixed race.
About 5\% reported belonging to other ethnic groups, while 1\% selected ``N/A'' and 0.4\% opted not to declare their ethnicity. These observations are consistent with Rainie and Perrin~\cite{auxier2019americans}, who documented that ethnic minorities, including Black and Hispanic populations, often express heightened privacy concerns due to systemic trust issues and prior experiences of data misuse.



\paragraph{Employment Status}
A plurality of the participants (36.6\%, 183 individuals) were engaged in full-time employment, while 17.4\% (87 individuals) were working part-time.
Approximately 14.6\% of the respondents reported being unemployed, but an equivalent amount of 14.6\% marked their status as "N/A" (which might include students, retired persons, or those who elect not to reveal their employment status). In addition, 11.2\% of the surveyed population are not currently employed, and 5.6\% are classified as "Other," which may indicate other job circumstances. The observed distribution exhibits a wide range of job statuses, with full-time workers being the most prevalent category, but also substantial representation from individuals who are not now engaged in paid employment. This aligns with Park~\cite{park2014employee}, who found that employed individuals are more likely to manage their privacy settings due to professional risks, while unemployed or part-time workers may demonstrate less engagement with privacy settings.

\paragraph{Student Status}
The bulk of the participants were not students (67.2\%, 336 individuals), while 19.6\% (98 individuals) self-identified as students.
Furthermore, 13.2\% (66 individuals) replied ``N/A.'' Livingstone and Helsper~\cite{livingstone2007gradations} similarly noted that students are often highly active online but may trade privacy for access to services without fully considering the long-term consequences of their decisions.

\begin{table}[htb]
    \centering
    \footnotesize
    \begin{tabular}{@{}p{.465\textwidth}|l|l|l|p{.32\textwidth}@{}} 
     \textbf{Questions}
     & \textbf{Mean} 
     & \textbf{Med.}
     & \textbf{SD}
     & \textbf{Interpretation} 
     \\\hline

     \qFour{}
     & 4.08  & 3 & 2.92 &
     Most Participants are not highly motivated to read privacy policies, but there is significant variation, suggesting some individuals are very motivated while others are not.  
     \\\hline

      \qFive{}
      & 4.12	& 3	& 2.88 &
      Participants are generally unlikely to spend time understanding privacy policies, but again, responses varied widely.
     \\\hline
     
     \qSix{}
      & 6.39	& 7	& 3.00 &
      Participants tend to focus more on the general overview rather than specific sections, but the responses vary greatly.
     \\\hline
     
     \qSeven{}
      & 6.63	& 7	& 2.51 & Participants feel moderately confident in managing privacy settings, with relatively less variation in responses.
     \\\hline
     
    \qEight{}
     & 5.87	& 6	& 3.01 &
     Participants feel moderately comfortable using privacy tools, but some are much more comfortable than others.
    \\\hline

    \qNine{}
     & 7.59	& 8	& 2.33 &
     Control over personal information is very important to most participants, with lower variability in responses. 
    \\\hline
    
    \qTen{}
     & 5.39	& 5	& 2.70 &
     Participants are moderately proactive about updating their privacy settings, but responses are somewhat varied.
    \\\hline
    
    \qEleven{}
     & 4.44	& 4	& 2.46 & 
     Participants are generally uncomfortable sharing personal information online, but comfort levels vary across individuals.
    \\\hline
    
    \qTwelve{}
     & 7.06	& 8	& 2.58 &
     Many participants accept terms and conditions without reading them, and this practice is somewhat common across the group.
    \\\hline

    \qThirteen{}
     & 6.84	& 7	& 2.67 &
     Participants are fairly likely to act (e.g., change settings) when concerned about a privacy policy, though responses vary.
\\\hline

     \qFourteen{}
     & 6.48	& 7	& 2.67 &
     Participants show moderate interest in learning about tools to better understand privacy policies, but interest levels vary.

     \\\hline
    \end{tabular}
    \caption{Attitude questions participants saw evaluating privacy concerns with their mean, median and their standard deviation based on the scores each participant provided and the interpretation column to analyze the mean and standard deviation.}
    \label{table: demoStats}
\end{table}

\paragraph{Demographic Questions outlining participant attitudes, based on GenderMag Facets, as applied to privacy attitudes}
Table~\ref{table: demoStats} showcases the responses about the participants' attitudes toward privacy policies and privacy-related behaviors.
It includes 11 questions, each addressing different aspects of privacy management, ranging from motivation to read privacy policies to comfort with sharing personal information.
For each question, the table provides the mean, median, and standard deviation (SD) of responses, along with an interpretation of the results.

The analysis shows that while participants realize the importance of privacy and are somewhat proactive about privacy settings, many are not motivated to read or understand privacy policies.
For example, the mean response for motivation to read policies (Q1) is 4.08, indicating that participants are generally not inclined to invest time in reading them, though individual responses vary widely. This aligns with findings by Acquisti et al.~\cite{acquisti2015privacy}, who showed that the length and complexity of privacy policies discourage users from reading them.
Similarly, participants are moderately confident in their ability to manage privacy settings (Q4) and feel that control over personal information is important (Q6), with relatively lower variability in responses for these items. 
These findings are consistent with Trepte et al.~\cite{trepte2011privacy}, who found that while users value control over their data, their actual knowledge of privacy settings often falls short.

The table also highlights a general discomfort with sharing personal information online (Q8) and a tendency to accept terms and conditions without reading them (Q9), corroborating prior work on user behaviors regarding privacy management.
The authors in this research~\cite{gross2005information} similarly noted that most users are hesitant to share sensitive data but frequently bypass terms and conditions due to convenience.

\paragraph{Negative experiences}
Figure~\ref{figNegExp} highlights the various categories of adverse experiences encountered by participants, indicating that the most commonly reported concern was the receipt of advertisements, impacting 274 individuals.
This is followed by data breaches reported by 223 participants.
Additional concerns, including unforeseen charges, data exploitation, and account breaches, impacted a considerable number of participants.
Account suspensions were the least prevalent adverse experience, reported by only 70 participants. These results align with findings by Martin and Murphy~\cite{martin2017role}, who noted that targeted advertisements and data breaches are among the most visible and frustrating outcomes of inadequate data privacy practices.
Figure~\ref{figNegMult} depicts the allocation of participants according to the frequency of negative experiences they faced, indicating that most participants encountered either one or two negative incidents.
The proportion of people indicating three or more adverse events consistently declines, with a mere 1\% reporting eight or nine occurrences.
Notably, 4\% of individuals reported no adverse events.
This aligns with findings by Solove~\cite{solove2012introduction}, who documented that privacy breaches are increasingly common but often concentrated among a subset of highly affected users.
\input{figure/3-negativeExp}
\input{figure/4-negativeMult}




