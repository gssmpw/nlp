\section{Introduction
}\label{secIntroduction}


By failing to read privacy policies, users risk a range of potential hazards.
For instance, people may unintentionally authorize data sharing with third parties, which might result in unwanted marketing or worse.
Milne et al.~\cite{milne2004consumers} revealed that a substantial number of users lack awareness of the collection and utilization of their personal information.
Insufficient knowledge about this matter can lead to users relinquishing authority over their personal data, resulting in a variety of consequences.
One example is \textit{Feldman v.\ Google, Inc.}~\cite{2007feldman}, which centers on Google AdWords and unexpected costs incurred by the plaintiff due to click fraud.
This case highlights the difficulties consumers have when they fail to carefully read (and/or comprehend) such agreements.
Another example is \textit{Fteja v.\ Facebook, Inc.}~\cite{2012fteja}, which refers to an allegedly unjustified account deactivation.
The case highlights what happens when individuals accept agreements from digital platforms without fully understanding them, especially their rights and access to social media sites.

\boldify{Interventions for this problem are plentiful, but generally insufficient}

There have been continuous efforts to streamline privacy policies, with varied levels of success.
One solution is \textit{layered policies}, which offer a concise overview alongside links to more comprehensive information.
However, prior work~\cite{mcdonald2009comparative} showed that although layered policies have benefits, users must still exert effort to acquire and comprehend the complete information.
An alternative approach entails symbolizing essential privacy standards with icons~\cite{kelley2010standardizing}.
These visual aids facilitate users' rapid comprehension of the fundamental aspects of a policy; however, they may oversimplify legal terminology.
The practice of simplifying legal documents on websites such as Pinterest~\cite{pinterest},
500px~\cite{500px}, and 
Prolific~\cite{prolific}
stands as proof of the advancement of user-centric design on digital platforms.

\boldify{One major complication is this paradox between reported and actual behavior, in terms of privacy preferences}

The discrepancy between the user's interest in knowing about data collection and their behavior is called the Privacy Paradox~\cite{barnes2006privacy}.
According to prior work~\cite{acquisti2015privacy}, users express concerns about their privacy, but their behavior often contradicts these concerns.
Those authors attribute the contradiction to the complicated and lengthy nature of privacy policies and highlight the necessity for the development of privacy policy presentations that are more accessible to 
\revised{%
people without specific privacy training.
} % end revised
This paradox is apparent in other studies (e.g., \cite{norberg2007privacy}) which demonstrate that despite users asserting the importance of their privacy, they frequently reveal sensitive information in exchange for small incentives or conveniences.
The inconsistency between statements and behavior is an obstacle to resolving privacy concerns in the digital era.

\boldify{And so, we wind up in a mess, with human-centered approaches being a possible way forward}

Unfortunately, existing solutions frequently fail to understand or address the underlying causes of the Privacy Paradox.
Regulations still largely depend on legalese and are not accessible to
\revised{%
individuals with limited knowledge about privacy.
} % end revised
Prior work~\cite{reidenberg2015disagreeable} found that even simpler policies are difficult to comprehend with low legal literacy.
This highlights the necessity for implementing more human-centered methods in the development of privacy policies.
\revised{%
One approach driven by human expertise is Terms of Service; Didn't Read (\tosdr{})~\cite{tosdr}.
Over time, \tosdr{} moderators have curated a taxonomy of privacy concepts and the community has submitted fragments of privacy policies that exhibit a particular concept, which the moderators can then approve or deny.
This amounts to a data labeling process, and though the labels are sparse, their sum total is a valuable indicator of how the policy manages the power balance between user and service provider.
The \tosdr{} team grades each website from A (best) to E (worst) based on their policy documents, such as privacy policy, terms of service, data policy, cookie policy, etc.
The detection and scoring of the individual cases found in a document gives its overall grade.
We chose this particular taxonomy since it is expert-derived and has maintained a long-lived crowdsourcing effort, suggesting a mix of high quality and accessibility to broad audiences.

\boldify{set the scope of what we are doing just before RQs}

We seek to understand how people perceive the taxonomy from \tosdr{} in two main respects:
The first metric is how well the participants report being able to understand the concept underlying individual taxa, which we call \textit{understandability}.
The second metric is the extent to which the concept favors a particular party, either the service provider or user, which we call \textit{severity}.
To collect data on these metrics, 
} % end revised
we conducted an online survey to answer the following research questions:
\begin{enumerate}
\item[\textbf{RQ1}] What does the \textit{understandability} data reveal about participants and privacy concepts?
\item[\textbf{RQ2}] What does the \textit{severity} data reveal about participants and privacy concepts?
\revised{%
\item[\textbf{RQ3}] How do participants' severity and understandability ratings for privacy concepts relate to each other?
\item[\textbf{RQ4}] How did participants' demographic traits and attitudes affect their understandability and severity ratings of privacy concepts?
} % end revised
\end{enumerate}

\boldify{Then, the obligatory contribs list, since often need the structure to know what we did}

By answering and discussing these RQs, we make the following contributions:
(1)~Knowledge about the usefulness of case labels as explanatory ingredients, based on understandability data.
\revised{%
(2)~Knowledge about perceptions of severity of the concepts the case labels represent.
While the information listed in these two contributions is useful in its own right, the severity data can also provide a user-perception-based scoring system.
(3)~With the previous three contributions, we offer a basis for future RQs and hypotheses, particularly in targeting educational interventions.
(4)~Alternative candidate text for the cases with low understandability, based on participants rewritten the cases.
} % end revised
