1AC review (reviewer 4)

 Expertise

   Knowledgeable

 Originality (Round 1)

   High originality

 Significance (Round 1)

   Medium significance

 Research Quality (Round 1)

   Low research quality

 Recommendation (Round 1)

   I recommend Revise and Resubmit

 1AC: The Meta-Review

   In this submission, “Signed, Sealed,... Confused: Exploring the Understandability
               and Severity of Policy Documents”, the authors use survey methods to
               examine how well people understand terms of service documents. Reviews
               generally praised the originality and potential significance of this
               study. However, there were several concerns as well, which would need
               to be addressed before that significance can be realized. Thus, I am
               recommending Revise and Resubmit.

   I’ll summarize the suggestions here, but the authors are encouraged to read each
               review in full:

   1) The limitations of the Prolific sample should be taken more seriously,
               particularly with regard to claims about “laypersons” (2AC)
   2) The framing of the question “Which party does this case tend to favor?”
               appeared to be confusing, and may require additional explanation (2AC)
   3) All three reviewers had concerns about presentation clarity and organization,
               including the introduction (R1), literature review (R1, R3), results
               (2AC), and discussion (R3). The reviewers all had helpful suggestions
               for how to improve the structure of this submission, which I suggest
               the authors consider.
   4) Potentially related to the above, the contribution of this work to the HCI
               community needs to be clarified, and perhaps a more critical
               consideration of the fact that ToS documents are, generally speaking,
               largely designed to favor the service provider would help frame the
               results (R1)
   5) There were several concerns about the analysis, including the sample size, lack
               of inferential statistics, and questions/claims that appear out-of-
               scope given the methods used (R1).

   Overall, the reviewers generally saw potential in this work, but I would encourage
               the others to take these suggestions (particularly those related to
               writing, framing, and methods) seriously; my impression is they are
               all critical in evaluating the contribution of the work.

----------------------------------------------------------------

2AC review (reviewer 2)

 Expertise

   Expert

 Originality (Round 1)

   High originality

 Significance (Round 1)

   High significance

 Research Quality (Round 1)

   High research quality

 Contribution Compared to Length (Round 1)

   The paper length was commensurate with its contribution.

 Figure Descriptions

   The figure descriptions are adequate and follow the accessibility guidelines.

 Recommendation (Round 1)

   I recommend Revise and Resubmit

 Review (Round 1)

   Summary: This paper describes the results of an online survey aimed at assessing
               how Internet users understand the presentation of content from Terms
               of Service documents.

   Originality and Prior Work: The current draft makes a strong case that the survey
               instrument designed for the study can provide a novel perspective on
               the “Privacy Paradox” and past findings on the privacy decision-
               making, the economics of digital privacy, etc. Overall, I think there
               is a strong novel contribution here. The coverage of related work here
               is quite reasonable. While there is a body of work that might also be
               relevant to discuss (one example below), I think the current amount of
               coverage and specificity is already reasonable.

   Goldfarb, A. and Que, V.F., 2023. The economics of digital privacy. Annual Review
               of Economics, 15(1), pp.267-286.

   Significance: I think this paper stands to have high potential impact. The topic
               is generally relevant to a large portion of the HCI field, which
               increasingly involves online platforms in some capacity (arguably the
               growth of AI-related research in HCI will make this work, and follow
               up work, even more important).

   Research quality: The presented research decisions are all reasonably well
               justified. There are a few opportunities for improvement that may
               impact how seriously readers take the study.
   * The Prolific-only sample is likely to skew results somewhat towards a more-
               online population. I’m not sure the current results really reflect
               “laypersons” as claimed, but nonetheless are still very important.
               While not a major issue, highlighting the importance of follow up work
               on a more general audience could be useful as a minor edit. This
               weakness is mentioned (8.2) but I think could be taken more seriously
               in future drafts.
   * I was a little confused at first about the “What party does this case tend to
               favor?” framing; this just didn’t necessarily seem like an intuitive
               way people might think about the Terms of Service for the platforms
               discussed in motivating paragraphs. I did look at the supplement
               materials (data) which was helpful, but still was left thinking the
               paper could benefit from more explanation of this framing and what
               this implies for user-platform interactions in general.
   * The understandability results were all very clear throughout.

   Presentation clarity: In terms of writing quality and results tables, the paper
               was clear throughout. The results might benefit from visualization
               rather than mean and standard deviations of ordinal responses. The
               results section could benefit from a bit more signposting (while a
               reader who has read the first 3 sections already has a pretty decent
               picture of the survey itself in their mind and thus how the results
               might be presented, a quick summary of Section 4 and 5’s organization)
               could be beneficial.

   In summary, this paper is strong throughout. My main concerns are that the sample
               doesn’t quite match some of the claims in the Introduction and
               Discussion about speaking to layperson perceptions (acknowledged in
               limitations) and some of the framing of these survey questions could
               be confusing. That said, most of the results are explained well and
               the implications are relatively clear.

   I look forward to seeing future iterations of this paper.

----------------------------------------------------------------

reviewer review (reviewer 1)

 Expertise

   Knowledgeable

 Originality (Round 1)

   Medium originality

 Significance (Round 1)

   Low significance

 Research Quality (Round 1)

   Low research quality

 Contribution Compared to Length (Round 1)

   The paper length was commensurate with its contribution.

 Figure Descriptions



 Recommendation (Round 1)

   I can go with either Reject or Revise and Resubmit

 Review (Round 1)

   Summary:
   This paper presents a user study of the comprehensibility of elements from terms
               of service. The project is based on the “Terms of Service; Didn’t
               Read” (ToS;DR) project, which digests terms of service documents into
               sets of standardized elements and assigns ratings from A to E based on
               their content. The authors had  499 crowdworks each to review a random
               set of 5 of the TOS;DR policy elements (termed “cases”), asking them
               to provide likert scale ratings of understandability and the severity
               of perceived favoritism (termed “severity”) towards the service
               provider or user. The authors use descriptive statistics to analyze
               the understandability and severity of different policy elements.

   Comments:
   I thank the authors for their submission to CHI 2025. I appreciate the hard work
               that clearly went into conducting this study, and recognize that it
               presents an original idea. Nevertheless, I cannot recommend accepting
               this paper in its current form due to a general lack of structure
               which inhibits its clarity,  an unclear contribution, and the need for
               better justification of the methods. Please see my detailed comments
               on each below:

   **General lack of structure**
   I think it is necessary to discuss this issue first, as it is pervasive and feeds
               into the other major issues I have. Most sentences are well written,
               but the paper lacks an overall structure which makes it difficult to
               comprehend for the reader. For example, the introduction provides some
               useful background information, but lacks an outline of the paper or
               other signposting that is typical in an academic paper (e.g.,
               description of methods, explanation of the results, list of sections
               and their purpose, etc.). The research questions are abruptly
               introduced with expected contributions, with no other details provided
               about the study to come.

   Similarly, the related work begins to discuss background information without a
               clear indication of why the authors are discussing this particular
               research. For example, it is unclear to me why there is a social media
               subsection. There are also no summarizing statements about how a
               particular research area relates to the present study or why the
               present study is unique from the other studies in this area (e.g.,
               “Our study builds upon social media research by Y” or “No prior study
               has done X”). As a result, the related work section reads like a
               largely self-contained annotated bibliography.

   This lack of scaffolding is evident beyond these two sections, but I will leave it
               here for brevity's sake. These types of writing issues could be
               addressed in a revision, although substantial revision is necessary.

   **Lack of clear contribution**
   A more fundamental issue I have with the study is that I do not really understand
               the motivation for conducting it or what contribution it provides. The
               main takeaways seem to be that the cases from TOS;DR are 1) generally
               considered comprehensible by users (with some exceptions) and 2)
               favorable to services. Beyond, maybe, helping to improve TOS;DR, the
               first finding seems to have limited utility to the broader HCI
               community. Perhaps there is some conclusion to be drawn about what
               elements of terms of service are particularly confusing, but most
               users are not referring to ToS;DR (or really any website policy
               document). It is therefore unclear how this insight is useful. This
               second finding seems almost tautological, as terms of services largely
               exist to curtail users’ rights.

   This issue may be solved by fixing the problems with writing. In particular, any
               future revision should clarify why it is important to investigate the
               understandability of ToS;DR cases, beyond the fact that it is a
               standard database. The authors should also make it clear what these
               results mean and what implications they have for the issues raised in
               the introduction, like the privacy paradox and unexpected elements in
               privacy policies.  Adding a discussion section could be a good place
               for the analysis of the broader implications.

   **Insufficient  methods**
   The final major issue I have with the study is that I believe the methods are not
               well justified and, at times, insufficient to answer the research
               questions as currently stated.

   First, I think there is an issue with sample size. 499 seems like a sufficient
               sample size to draw macro conclusions (i.e, that the set of cases is
               generally understandable) but the between case comparisons seem more
               dubious.  It is, of course, impossible to show all participants each
               of the 243 case definitions, but this means that only a small number
               of participants saw each condition. 499/(243/5) = ~10 each. I am glad
               the authors raise this issue in the he threats to validity section.
               This design may be acceptable, but I think justification is necessary
               (i.e., citation to relevant literature with similar design).

   The study also relies on interpreting descriptive statistics to draw its
               conclusions. No statistical testing was performed to validate that the
               differences between cases in understandability or severity were
               significant. Statistical testing is not always necessary nor
               appropriate, but the descriptive analysis seems, at points, highly
               subjective and arbitrary. No defined procedure is explained in methods
               nor is a reference cited to justify this approach.  A particular area
               of concern for me is section 6. The authors initially find no
               correlation between understandability and severity, but nevertheless
               go onto crop the data in different ways to try to draw conclusions
               (i.e., looking at the most and least understandable/severe alone).
               Ultimately, I’m left with really no takeaways from this section.

   Research question 4 (“How did demographic traits affect understandability and
               severity?”) also seems to be unanswerable with the paper’s descriptive
               approach. It is valuable to contextualize the other results by
               describing the characteristics of the participants, but no analysis is
               discussed to determine whether the demographics are associated with
               understandability or severity. The authors do not even hypothesize
               about how the demographics may effect the results. This research
               question is, therefore, left unaddressed.

   Ultimately, any revision of the paper should better justify the methodological
               approach, especially with respect to the sampling approach and the
               reliance on seemingly non-systematic descriptive statistics. The
               procedure and choices to crop the data in certain ways should be
               justified.

   **Misc other issues**
   What follows is a list of other issues that I noticed during my reading, which I
               think should be addressed in any revision.

   1: The research questions use terminology before it has been defined in the paper
               like  “understandability data” or “severity data.” In general, the
               narrow phrasing of the research questions is not very useful.
   2: Throughout related work, papers are summarized or quoted without reference to
               what the research did or the context of the results. This makes it
               difficult for a reader to evaluate the synthesis. E.g., “Slightly over
               half of participants (51%) felt that information about their general
               searching activities,... impacts their feelings about tracking.” [26]
               => What context is this quote from? What is the information referred
               to about general searching activities? “According to Earp et al. [15],
               people are more likely to read and understand privacy regulations when
               they believe that their data is in grave danger.” => How did they draw
               that conclusion? What is the context?
   3: Line 216: “To determine the set of concepts the participants would see, as well
               as the text to represent each concept, we appeal to Terms of Service;
               Didn’t Read (ToSDR) [38].” At this point in the paper, ToSDR is not
               well introduced. I am left with the questions: What is ToSDR? What
               does it contain? Why is it a reasonable source for this purpose? What
               are cases? Are they like parts of a policy?
   4: Footnotes are not an appropriate citation (E.g., footnote 5 and 6). If linking
               to a live website or service, a footnote makes sense.  See:
               https://nam10.safelinks.protection.outlook.com/?url=https%3A%2F%2Fwww.acm.org%2Fpublications%2Fauthors%2Freference-formatting&data=05%7C02%7Cjxd6067%40psu.edu%7C139527e5081d41ab9ca808dcfdeec6ca%7C7cf48d453ddb4389a9c1c115526eb52e%7C0%7C0%7C638664449203068419%7CUnknown%7CTWFpbGZsb3d8eyJWIjoiMC4wLjAwMDAiLCJQIjoiV2luMzIiLCJBTiI6Ik1haWwiLCJXVCI6Mn0%3D%7C0%7C%7C%7C&sdata=Kq%2BI2YscHLnCoMT5nk7b41pJbvrxqPx7vvMtcsMWl%2Bo%3D&reserved=0
   5: Line 302: “However, there is no significant correlation (Pearson’s correlation
               t-test, t(2498) = 6.99, p = 3.45 e-12) between understandability and
               severity.” 1. There is no such thing as “Pearson’s correlation
               t-test,”  2. The p-value appears very small and significant under most
               normal interpretations. Later in the paper, the opposite is stated:
               “we see that higher understandability scores tend to result in a rise
               in the mean severity score.” (line 618). Which is true? Or are the
               supposed to mean different things?
   7: Page 7: table is in the footer.
   8: The question that is the title of 4.2 is not addressed in the subsection. Was
               any analysis performed to actually tell if participant written
               explanations are better?
   9: line 748: Why is the 500th participant included in demographic calculations if
               their other data is not used? They are functionally not part of the
               study.
   10: The privacy paradox is discussed in the introduction, but never returned to.
               Why is this relevant?
   11: There is no complete listing of the survey available to reviewers. This makes
               it challenging to evaluate the methods. It is generally best practice
               to include the survey instrument in the appendix.
   12: The entire subsection 6.1 is confusingly written. Its unclear which responses
               are being discussed when. What do the numbers 110 and 35 reports
               correspond to? I can infer somewhat, but it’s not clear.
   13: Gender subsection of section 7 suggests that only binary gender options were
               presented. This is against best practice. See https://nam10.safelinks.protection.outlook.com/?url=https%3A%2F%2Fwww.morgan-%2F&data=05%7C02%7Cjxd6067%40psu.edu%7C139527e5081d41ab9ca808dcfdeec6ca%7C7cf48d453ddb4389a9c1c115526eb52e%7C0%7C0%7C638664449203088867%7CUnknown%7CTWFpbGZsb3d8eyJWIjoiMC4wLjAwMDAiLCJQIjoiV2luMzIiLCJBTiI6Ik1haWwiLCJXVCI6Mn0%3D%7C0%7C%7C%7C&sdata=xBp%2FMwXRYkf%2FQ2hpCVIqnSTVHoH7O51Oe7P05jN39JE%3D&reserved=0
               klaus.com/gender-guidelines.html
   14: The questions adapted from GenderMag do not seem to me to be appropriately
               called demographics. They seem more like privacy attitude/behavior
               questions.
   15: section  8.2 on external validity should discuss the characteristics of
               Prolific crowdworkers. See Tang et al. https://nam10.safelinks.protection.outlook.com/?url=https%3A%2F%2Fwww.usenix.org%2Fconferen&data=05%7C02%7Cjxd6067%40psu.edu%7C139527e5081d41ab9ca808dcfdeec6ca%7C7cf48d453ddb4389a9c1c115526eb52e%7C0%7C0%7C638664449203109173%7CUnknown%7CTWFpbGZsb3d8eyJWIjoiMC4wLjAwMDAiLCJQIjoiV2luMzIiLCJBTiI6Ik1haWwiLCJXVCI6Mn0%3D%7C0%7C%7C%7C&sdata=DD6uUBpBwdKjY8CV4vfJwnPV%2BKTO%2B9wZdqe7B9kdYQw%3D&reserved=0
               ce/soups2022/presentation/tanghttps://nam10.safelinks.protection.outlook.com/?url=https%3A%2F%2Fwww.usenix.org%2Fconference%2Fsoups2&data=05%7C02%7Cjxd6067%40psu.edu%7C139527e5081d41ab9ca808dcfdeec6ca%7C7cf48d453ddb4389a9c1c115526eb52e%7C0%7C0%7C638664449203128203%7CUnknown%7CTWFpbGZsb3d8eyJWIjoiMC4wLjAwMDAiLCJQIjoiV2luMzIiLCJBTiI6Ik1haWwiLCJXVCI6Mn0%3D%7C0%7C%7C%7C&sdata=g1ut9Nd7jG3NgDohw2AxoxnTs6BngWDLmvHkihZ62Uo%3D&reserved=0
               022/presentation/tang
   16: how was the sample size determined? Was power analysis performed?
   17: The title suggests that the authors found that the policies were confusing,
               but they were largely understandable.

----------------------------------------------------------------

reviewer review (reviewer 3)

 Expertise

   Expert

 Originality (Round 1)

   Medium originality

 Significance (Round 1)

   Medium significance

 Research Quality (Round 1)

   Medium research quality

 Contribution Compared to Length (Round 1)

   The paper length was commensurate with its contribution.

 Figure Descriptions

   The figure descriptions are adequate and follow the accessibility guidelines.

 Recommendation (Round 1)

   I recommend Revise and Resubmit

 Review (Round 1)

   Thank you for your submission, I enjoyed reviewing your paper. You take up
               important, timely research questions on the understandability of
               privacy policies. I found your research questions to be original, and
               your methods sound and rigorous. It is great, for example, to have
               empirical work that shows which privacy concepts are “understandable”
               and your tables and descriptions show a great amount of detail.

   That being said, I do have a number of concerns (e.g. related works, discussion /
               implications) and presentation clarity that in the current state
               obstruct the impact of this work.

   Overall, I find that this paper could be written more clearly. It could use a few
               rounds of heavy copy editing to ensure a smoother flow.

   Next, I find that our paper is not sufficiently motivated in the introduction, nor
               sufficiently grounded adequately in prior work. For example, you
               present your research questions in the introduction without defining
               the research area — what is “understandability data”? What are the
               privacy concepts that ToSDR encompasses? Why is this your research
               site?

   Related Works:

   While your related works section does an adequate job of citing prior works - I
               find the organization unconvincing and the subsections scattered. It
               would strengthen your paper to rewrite this section with a better
               flow. For example, you start this section with {Websites profit off
               people’s trust by selling, sharing, or analyzing personal information
               [39]. As an example of the effects
   of these behaviors, “Slightly over half of participants (51%) felt that
               information about their general searching activities,...impacts their
               feelings about tracking.” [26] }. This does not read well. Summarizing
               the quote would be a better approach. Further as some more general
               writing tips, please use topic sentences and work on the transition
               flows between paragraphs and new ideas.

   Here are some more concrete suggestions towards a more organized structure for
               your related works section:

   1. Starting the section with an overview of a privacy theory or framework you are
               using and how you define privacy policies. You should ground these in
               related works in the HCI space.
   2. A subsection that starts with outlining the problem spaces concerning privacy
               policies (and then continuing with the other subsections you have).

   You cite the contextual integrity framework, which is fantastic. Along these
               lines, I would urge you to review this robust Zotero library for more
               work: https://nam10.safelinks.protection.outlook.com/?url=https%3A%2F%2Fwww.zotero.org%2Fgroups%2F2228317%2Fprivaci%2Flibrary&data=05%7C02%7Cjxd6067%40psu.edu%7C139527e5081d41ab9ca808dcfdeec6ca%7C7cf48d453ddb4389a9c1c115526eb52e%7C0%7C0%7C638664449203255851%7CUnknown%7CTWFpbGZsb3d8eyJWIjoiMC4wLjAwMDAiLCJQIjoiV2luMzIiLCJBTiI6Ik1haWwiLCJXVCI6Mn0%3D%7C0%7C%7C%7C&sdata=kuZvZm3fJMA%2FAtB4ogZvTzz3wauECQHegKaWaY0kpho%3D&reserved=0 but
               especially Shvartzshnaider et al., "Going against the (appropriate)
               flow: a contextual integrity approach to privacy policy analysis".

   Discussion / Implications

   Your findings and implications sections are quite muddled and it is very
               challenging to follow. I would highly recommend combing through these
               sections and trying to rewrite this for flow and readability.
               Importantly, throughout these sections, please cite prior work from
               your related works section. Here, you should be articulating how prior
               research informs (builds on, supports, or discounts, etc.) your
               empirical contributions -- this is crucial for having a paper with
               strong impact.

   I look forward to reviewing your edits. Good luck!

   Other specific edits:

   Introduction:

   Period missing at the end of second bullet point under “By answering these RQs, we
               make the following contributions”

   Methodology:

   Introduction is needed for ToSDR - what is ToSDR? Why ToSDR? What is the object or
               unit of analysis when scraping ToSDR?

   Period missing at the end of the last sentence in section 3.1 (Demographic
               information)

   Findings / Implications / Discussion:

   Period missing at the end of the last sentence in section 4.3

----------------------------------------------------------------

