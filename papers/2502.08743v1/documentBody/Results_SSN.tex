\section{Results - LOOKING FOR A HOME. We need to mine this and locate within the RQ framework provided}

\subsection{Correlation Matrix of Demographic Data}

\begin{table*}[htb]
    \centering
    \begin{tabular}{@{}lc|cc|cc@{}}
     \textbf{Question} & \textbf{Age} &
     \textbf{Gender} &\textbf{Ethnicity} &
     \textbf{Student Status} & \textbf{Employment}
     \\\hline
     \ 1 &-0.05 &	-0.04 &	0.01 &	-0.07	& 0.06
     \\\hline

     \ 2 & -0.05	 &-0.06	 & 0.05	& -0.06 & 	0.08 
     \\\hline
     
     \ 3 &  -0.01	& 0.04	& 0.01	& -0.01 &	0.07
     \\\hline
     
     \ 4 & -0.09 &	-0.04 &	0.08 &	-0.02 &	0.13 
     \\\hline
     
    \ 5 & 0.04 &	-0.08 &	0.02 &	0.00 &	-0.01
    \\\hline

    \ 6 & -0.06	& -0.05&	-0.01&	-0.09&	-0.02
    \\\hline

    \ 7 & -0.05&	-0.02&	0.00&	0.00&	0.08
    \\\hline
    
    \ 8 & 0.00	&0.05 &	0.00 &	0.09 &	0.00
\\\hline
    \ 9 & -0.01	& -0.05	& -0.01	& -0.01 &	0.01
    \\\hline
     \ 10 & -0.04	&-0.10 &	0.02&	-0.01 &	0.07
    \\\hline 
    \ 11 & -0.13	&-0.01	&0.02&	-0.05&	0.02
    \\\hline
    \end{tabular}
    \caption{Correlation Matrix of Demographic Data}
    \label{table:caseF1}
\end{table*}

Younger participants typically have higher levels of privacy awareness and proactivity, displaying stronger motivation to read privacy regulations, greater confidence in understanding and controlling privacy settings, and a greater interest in educational materials compared to older participants.
Women exhibit a moderately higher level of caution compared to men, especially when it comes to their motivation to read privacy policies, their investment of time in comprehending them, and their willingness to take action if a privacy policy raises concerns.
The influence of ethnicity on privacy views is minimal or insignificant, as most correlations indicate little to no discernible variations among different ethnic groups.
Employed persons exhibit greater prudence and proactivity towards privacy, displaying higher motivation to adhere to policies, greater confidence in controlling settings, and a higher likelihood of taking action when privacy issues emerge.
Students typically exhibit lower levels of motivation to peruse privacy regulations, perceive less significance in managing personal information, yet feel more at ease divulging personal data to internet businesses.
The level of proactivity shown by employed and younger individuals in routinely updating their privacy settings is somewhat higher, which indicates their higher level of involvement with data protection policies.
Males and older participants exhibit more ease in utilizing privacy protection protocols like VPNs, but females seem to have lower familiarity with such technology.
The younger participants exhibit a notably greater inclination to acquire further knowledge regarding privacy resources, which indicates their aspiration to have a deeper understanding of the potential hazards and safeguards accessible to them.
Although males exhibit a slightly higher level of comfort in sharing personal data, the influence of gender on sharing activities is negligible, since comfort levels generally remain similar between genders.
Differences in status between students and non-students have minimal impact on the level of proactivity in upgrading privacy settings.
Neither age nor gender significantly affects whether persons accept terms and conditions without reading them.
The employment status systematically has a favorable impact on privacy behaviors, most likely attributed to heightened exposure to data privacy policies in the professional environment.
The influence of ethnicity on attitudes towards privacy is minimal, as most correlations indicate very little to no substantial variations in behavior among different ethnic groups.
In general, younger persons exhibit a greater inclination towards proactivity and caution about privacy issues (e.g., Q4, Q7), express a higher level of interest in acquiring knowledge about privacy (Q11), and are more inclined to engage in taking action (Q10).
Women exhibit a slightly higher level of prudence and worry in several areas, including a greater inclination to peruse privacy policies (Q1), dedicate effort to comprehending them (Q2), and respond promptly when they are a cause for concern (Q10).
Although not explicitly acknowledged in the statistics, it is usually observed that more education tends to result in increased awareness and proactive conduct, as indicated by historical patterns.
\subsection{Understandability and Severity Matrix}

\begin{table*}[htb]
    \centering
    \begin{tabular}{@{}lc|cc|cc@{}}
     \textbf{Understandability Score} & \textbf{Count} &
     \textbf{Average Severity} 
     \\\hline
     \ 1 &124 &	4.02 
     \\\hline

     \ 2 & 97	 &5.38	  
     \\\hline
     
     \ 3 &  84	& 5.23	
     \\\hline
     
     \ 4 & 96 &	4.96  
     \\\hline
     
    \ 5 & 165 &	5.76
    \\\hline

    \ 6 & 157	& 5.72
    \\\hline

    \ 7 & 296 &	5.86
    \\\hline
    
    \ 8 & 350	&6.02
\\\hline
    \ 9 & 336	& 6.10
    \\\hline
     \ 10 & 790	&6.33
    \\\hline 
   
    \end{tabular}
    \caption{Understandability and Severity Matrix}
    \label{table:caseF1}
\end{table*}

The statistical analysis reveals a direct correlation between comprehensibility and seriousness. For lower scores of understandability (1 to 4), the average severity varies between 4.02 and 4.96, indicating that less comprehensible material may result in reduced perceived severity. As the understandability score grows, the average severity levels consistently rise peaking at 6.33 for the greatest understandability score of 10. The observed pattern indicates that more explicit information enables participants to more effectively understand the gravity of problems, resulting in increased assessments of severity. In summary, when the comprehensibility of material improves, the perceived seriousness of problems also rises.




