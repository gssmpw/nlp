\section{LEFTOVERS}

%%%%%%%%%%%%%%%%%%%%%%%%%%%%%%%%%%%%%%%%%%%%%%%%%%%%%%%%%%%%%%%%%%%%%%%%%
\subsection{Which privacy concepts received low severity ratings?}
\label{weakestSection}

Since the overall severity score was moderate to high, about 16\% of the responses had a severity score of less than 3 favoring either party, suggesting that participants viewed these case as largely unimportant.
In terms of the number of cases, 92 cases had a severity score of 3 or less favoring either party.
The following cases were perceived to be the least comprehensible (with an understandability score of 1) and least severe (with a severity score of 1 or -1):
\begin{itemize}
    \item \case{9}{An onion site accessible over Tor is provided}
    \item \case{15}{Content is published under a free license instead of a bilateral one}
    \item \case{44}{Minors must have the authorization of their legal guardians to use the service}
    \item \case{71}{The copyright license that you grant this service is limited to the parties that make up the service's broader platform.}
    \item \case{80}{The service claims to be CCPA compliant for California users}
    \item \case{82}{The service claims to be LGPD compliant for Brazilian users}
    \item \case{179}{You aren't allowed to remove or edit user-generated content}
    \item \case{192}{You can request access, correction and/or deletion of your data}
    \item \case{229}{Your personal data is aggregated into statistics}
    \item \case{146}{Tracking cookies refused will not limit your ability to use the service}
    \item \case{155}{very broad}
    \item \case{193}{You can scrape the site, as long as it doesn't impact the server too much}
\end{itemize}




%%%%%%%%%%%%%%%%%%%%%%%%%%%%%%%%%%%%%%%%%%%%%%%%%%%%%%%%%%%%%%%%%%%%%%%%%
\subsection{How understandable are the least severe case?}
\label{sec:leastSevere}

Out of the 200 least severe responses (i.e., 1 and -1), only 51 responses were in favor of the Service Provider, leaving 149 favoring the User.
Out of those 51 responses that favored the Service Provider, 30 of them had the highest understandability score.

By contrast, in cases with the lowest severity score of 1, a total of 169 responses featured a maximum understandability score of 10.
This suggests that clarity is not necessarily impaired in situations that participants considered less severe.
Of the examples considered, 75 instances were in favor of the User, while 30 were in favor of the Service Provider.

The results indicate that although there is a prevailing pattern of strong comprehensibility being linked to severe consequences, there are individual cases that deviate from this pattern.
Although participants may have difficulty understanding certain cases, especially those that benefit the Service Provider, they may still perceive those cases as serious.

%%%%%%%%%%%%%%%%%%%%%%%%%%%%%%%%%%%%%%%%%%%%%%%%%%%%%%%%%%%%%%%%%%%%%%%%%
\subsection{Summary}

%This analysis sheds light on the complex relationship between understandability and severity in the context of privacy policies.
%While high understandability often correlates with high perceived severity, particularly in Service Provider-favored cases, some participants may perceive cases as severe despite low understanding.
%The party favored also plays a key role in shaping perceptions, with Service Provider-favored cases being more likely to receive extreme severity ratings.
These insights highlight the importance of improving the clarity of privacy policies, particularly in cases where the stakes are high for users.







\subsection{How does this work help us in understanding/resolving the Privacy Paradox?}

\FIXME{R1-19: 10: The privacy paradox is discussed in the introduction, but never returned to. Why is this relevant?\\
JED@SNS: If you do not have a solid answer to R1-19 that is doable QUICKLY, we should make that paradox less primary to the motivation, since I think I have a decent argument I can develop QUICKLY for the other subsections here that should address R1-6.}









\subsection{what will exactly happen when the users are made aware of these policies?}

\FIXME{This needs to move to discussion since we are speculating a bit}

When users become more aware of privacy policies through the use of a tool that simplifies and clarifies them, several significant outcomes become possible.
Several enhancements or modifications may take place if businesses, legislators, and regulators learn about user interaction with privacy rules through this kind of study:

\begin{itemize}[nosep]
\item \textbf{Increased Awareness of Data Rights}: Users will become more knowledgeable about their rights regarding data privacy.
This includes understanding what data is being collected, how their data is being used, and what control they have.
Companies might simplify their privacy policies to make them more user-friendly.
Clear, concise, and transparent policies could become a competitive advantage.

\item \textbf{Empowered Decision-Making}: With a clear understanding of privacy terms, users can make more informed decisions about which services to use and what terms they are comfortable agreeing to allow.
This empowerment could lead to more selective consent, where users opt in or out based on their privacy preferences.

\item \textbf{Demand for Better Privacy Practices}: As public knowledge grows, there could be increased pressure on institutions (i.e., businesses or governments) to adopt or require more user-friendly privacy practices.
This could lead to more transparent data handling and potentially even influence legislative changes.
Companies might seek to collaborate more with privacy experts, legal advisors, and technology providers to improve their privacy policies and practices.

\item \textbf{Behavioral Changes in Digital Spaces}: With a better grasp of privacy implications, users might alter their behavior online, such as being more cautious about sharing personal information or using privacy-enhancing tools and settings.
The awareness of such an analysis in the think tank of the company can drive innovation, leading companies to develop new technologies and methods for protecting user privacy.
Increased awareness can also ensure better compliance with existing and emerging data protection regulations, thereby avoiding legal repercussions.

\item \textbf{Trust in Digital Services}:
By demonstrating a commitment to transparency and user privacy, companies can build greater trust with their customers, which is crucial for brand reputation and customer loyalty.
If companies respond positively by simplifying their privacy policies and respecting user preferences, this could lead to increased trust in these services.
Companies might reevaluate their data collection and processing practices, leading to more ethical and responsible use of user data.
Conversely, if users become aware of overly invasive or unfair practices, it could lead to a backlash against certain companies, prompting them to revise their policies.

\item \textbf{Rise in Demand for Privacy-Focused Services}: Awareness could drive demand for services and products that prioritize user privacy, potentially shifting market trends.
Companies may shift towards more user-centric approaches in handling personal data, including giving users more control over what data is collected and how it is used.

\item \textbf{More Informed and Far-Reaching Public Discourse on Privacy}: Awareness can stimulate public discourse on digital privacy, leading to a more informed public debate and potentially influencing policy decisions at the governmental level.

\end{itemize}












To resolve the privacy paradox, a comprehensive strategy is needed that includes boosting user education, streamlining privacy policies, and enhancing openness in data practices.


%Although there are exceptions, these results indicate that a low understandability score typically correlates with extremely negative perceived severity.
%A subset of participants acknowledge the potential importance of a case, especially when it favors the Service Provider, even if they have difficulty comprehending it.



%These advancements highlight that technology has the capacity to improve understanding and promote user involvement with rules, establishing a standard for clear and transparent digital communications.
%This automated tool will be in the form of a browser extension when upon opening any website's privacy policy or ToS document gives a summarized annotated version of the document for the user to read in a faster and efficient way.
%JED: commented out the above line. Why? we arent trying to propose tools here, just to use them as a motivation on why we need this data that we just collected.
%Addressing this issue through technological innovation not only enhances user empowerment and data literacy but also fosters a more ethically responsible digital environment.
%JED: commented out a line a bit above and also directly above. Why? not adding much. we made the point succinctly already

%An accurate analysis of the data and identification of trends or patterns relating to specific user groups can be achieved by comprehending the demographic profile of our participants.
%This demographic data will enable us to evaluate whether specific demographics have greater difficulty in comprehending privacy policies or perceive them as more stringent.
%JED: commented out hte above. Why? the first sentence says the same thing as the bottom commented line, but more succinctly. The top commented line just doesn't add much

%Through the integration of these inquiries, our objective is to reveal potential disparities among diverse user groups in terms of how consumers perceive and engage with privacy regulations.
%This information will be vital in developing privacy messages that are both inclusive and effective.
%JED: commented out the above. Why? not adding much. Just the facts, ma'am

%This also suggests that some participants considered the case to favor users while some considered it to favor the service providers. % Commented out because we make this point elsewhwere in a stronger way

%Recall that we gave participants access to both the succinct case label, as well as a lengthier paragraph giving more detail, also from \tosdr{}. %Commented out for brevity and it doesn;t really matter that much here anyway

\FIXED{
\begin{enumerate}
\item Identify Key barriers to policy understanding
    \subitem This survey systematically analyzes and identifies particular legal and policy concepts that consistently provide difficulties for users in understanding.
    \item Quantitative Insights into User Perception of Bias in Policy Favoritism
    \subitem  The finding, that two-thirds of the participants perceive a substantial tilt towards the Service Providers, adds to the debate on fairness and equality in policy development by presenting evidence of an imbalance in policy concepts that might impact user trust and consent.
    \item Legal Policy Reform Implications
    \subitem Through the identification of precise areas where user understanding is deficient, this research provides a basis for legislative reform initiatives that seek to streamline policy structure in order to promote more openness and justice in digital services.
    \item By measuring severity and understandability we seek to encompass both affective and cognitive reactions to privacy concepts. Our objective is to identify concepts participants (do not) value, as well as areas that suffer from lack of clarity.  The rewritten cases give the potential alternate text that we can use in place of ToSDR case labels that receive a low understandability rating.
    
\end{enumerate}}