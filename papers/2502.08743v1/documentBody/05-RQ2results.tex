\section{Results RQ2 - What does the \textit{severity} data reveal about participants and privacy concepts?
\draftStatus{one fix remains, revision highlights are in}
}

\revised{%
Next, we follow a similar approach as in the last section, but turn our focus to the \textit{severity} data.
As mentioned in Section~\ref{secAnalysis}, we will at times refer to severity and other times the absolute value of severity, which we will denote $|S|$.

\boldify{Here we want to make two points: 1. cases favor companies; 2. we see little consensus on severity}

Based on the 2495 responses by participants, they seemed to \textbf{perceive cases as broadly in favor of Service Providers}.
Only one-third of the responses indicated that the cases were in favor of the clients/participants, while over half (53\%) of the responses indicate that the cases favor the Service Providers and about 14\% responses were neutral.
Participants typically evaluated cases as having a moderate level of severity, with a mean severity of -2.23 and a median of -2.00.
However, \textbf{there was little overall consensus in severity}, as indicated by a standard deviation of 6.07.
This aligns with prior research by Bansal et al.~\cite{bansal2010impact}, which highlights that users tend to view privacy policies as skewed in favor of organizations, particularly when transparency is limited or trust is compromised. 

% \FIXME{JED@SNS and self: I propose we relocate the cites and chop the next paragraph. I am struggling to write a boldify for it because it isn't adding much. Why i find this so confusing is (1) that the extrema analysis isnt very thoroughgoing, e.g., providing what the 8 and 16 values tell us (skew right?); (2) extrema in the presence of the absolute value makes the reading hard AND the believability low.}
% Considering the minimal value of all the responses (i.e., -1 for favoring Service Providers and 1 for favoring Users, or $|S|=1$), about 8\% of responses assigned $|S|=1$, denoting the least severe cases, and about 16\% assigned $|S|=10$, denoting the maximum severity.
% Furthermore, 45\% of responses rated $|S|=5$ or above.
% These findings reinforce the need for privacy policy designs that not only simplify information but also explicitly highlight the severity of outcomes for users.
% Similar to insights from Isaak and Hanna~\cite{isaak2018user}, transparency regarding the consequences of privacy practices may play a critical role in shaping user perceptions and encouraging informed decisions.

%%%%%%%%%%%%%%%%%%%%%%%%%%%%%%%%%%%%%%%%%%%%%%%%%%%%%%%%%%%%%%%%%%%%%%%%%
\subsection{Which privacy concepts received high severity ratings?}
\label{sec:highSeverity}

\begin{table}[b]
	\centering
    \footnotesize
    %\framebox[\linewidth]{For Measuring!}
	\begin{tabular}{@{}l | l | l@{}}   
    \textbf{Case Label} &
    \textbf{S mean} &
    \textbf{S SD}\\\hline\hline

\caseTable{138}{This service reserves the right to disclose your personal information without notifying you} &
-9.75	& 0.50\\\hline

\caseTable{218}{Your browsing history can be viewed by the service} &
-9.71	& 0.76\\\hline

\caseTable{64}{Service fines users for Terms of Service violations} &
-9.57	& 0.79\\\hline

\caseTable{65}{Some personal data may be kept for business interests or legal obligations} &
-9.22	& 0.83\\\hline

\caseTable{156}{Voice data is collected and shared with third-parties} &
-9.08	& 1.16\\\hline

\caseTable{68}{Terms may be changed any time at their discretion, without notice to you} &
-9.00	& 1.95\\\hline

\caseTable{139}{This service retains rights to your content even after you stop using your account} &
-8.90	& 1.20\\\hline


...& ... &...\\\hline

\caseTable{193}{You can scrape the site, as long as it doesn't impact the server too much} &
5.22	& 3.67\\\hline

\caseTable{238}{Your personal data will not be used for an automated decision-making} &
5.25	& 3.22\\\hline

\caseTable{121}{This service does not collect, use, or share location data} &
5.42	& 3.32\\\hline

\caseTable{186}{You can delete your content from this service} &
5.13	& 3.52\\\hline

\caseTable{203}{You maintain ownership of your content} &
5.82	& 4.39\\\hline

\caseTable{230}{Your personal data is not shared with third parties} &
6.18	& 5.38\\\hline

\caseTable{95}{The service is open-source} &
7.14	& 2.97\\\hline

	\end{tabular}
	
    \normalsize
	\caption{
    Descriptive statistics for the top and bottom ranked cases, based on mean \textit{severity} (S).
    Recall that a positive severity score indicates that participants perceived it to favor the User, while a negative severity score favors the Service Provider.
    }
	\label{tableConcSeverity}
\end{table}


\boldify{Two major points here, the polarizing trend, plus the consensus on severity of terms strongly favoring the service provider}

The severity ratings showed a polarizing trend with about 61\% of the responses having an absolute value of 1, 2, 3, 8, 9, or 10 indicating least severe or most severe.
However, there are 6 cases with $|S| \geq 9$, as found in Table~\ref{tableConcSeverity}. 
For cases that strongly favor the Service Provider, there was strong consensus on severity, as indicated by the low standard deviation.

\boldify{What kinds of cases were in the "terms strongly favoring the service provider" bucket}

The cases that strongly favor the Service Provider feature concepts that one might expect, such as accessing private data, events occurring without notice, and sharing data with third parties.
On the other hand, the cases that strongly favor the User were subject to much less consensus, as exhibited by the relatively high standard deviation.
Important concepts visible here include the ``right to be forgotten'' (\case{186}{You can delete your content from this service}), as well as a variety of prohibitions the Service Provider imposes upon themselves (i.e., not sharing, not collecting, not a decision basis). With a relatively high SD of the cases favoring the User, there is a possibility that some participants would have categorized the case favoring the User based on the keyword `You' or `Your' implying the user-friendly condition.
However, previous studies and current surveys have also shown examples where participants have suffered data breaches and data misuse despite some services having their terms favoring the Users~\cite{thomas2017data}.
} % end revised

%%%%%%%%%%%%%%%%%%%%%%%%%%%%%%%%%%%%%%%%%%%%%%%%%%%%%%%%%%%%%%%%%%%%%%%%%
\subsection{Which privacy concepts exhibited low consensus about severity?}

\begin{table}
	\centering
    \footnotesize
    %\framebox[\linewidth]{For Measuring!}
	\begin{tabular}{@{}l | l | l@{}}   
    \textbf{Case Label} &
    \textbf{S mean} &
    \textbf{S SD}\\\hline\hline
    
\caseTable{155}{very broad} &
0.10	& 0.32\\\hline

\caseTable{169}{You are free to choose the type of copyright license that you want to use over your content} &
0.20	& 0.45\\\hline

\caseTable{138}{This service reserves the right to disclose your personal information without notifying you} &
-9.75	& 0.50\\\hline

\caseTable{218}{Your browsing history can be viewed by the service} &
-9.71	& 0.76\\\hline

\caseTable{64}{Service fines users for Terms of Service violations} &
-9.57	& 0.79\\\hline

\caseTable{65}{Some personal data may be kept for business interests or legal obligations} &
-9.22	& 0.83\\\hline

\caseTable{239}{Your personal information is used for many different purposes} &
-8.00	& 1.00\\\hline

...& ... &...\\\hline

\caseTable{96}{The service is provided 'as is' and to be used at your sole risk} &
-4.44	& 7.00\\\hline

\caseTable{233}{Your personal data is used for limited purposes} &
-1.10	& 7.13\\\hline

\caseTable{93}{The service is not transparent regarding government requests or inquiries that may involve your data} &
-4.54	& 7.17\\\hline

\caseTable{175}{You are responsible for any risks, damages, or losses that may incur by downloading materials} &
-3.00	& 7.21\\\hline

\caseTable{232}{Your personal data is used for advertising} &
-4.25	& 7.23\\\hline

\caseTable{41}{Logs are kept for an undefined period of time} &
-2.83	& 7.41\\\hline

\caseTable{103}{The service will resist legal requests for your information where reasonably possible} &
0.00    & 7.54\\\hline

	\end{tabular}
	
    \normalsize
	\caption{
    Descriptive statistics for the top and bottom ranked cases, based on standard deviation of severity (S)
    }
	\label{tabledispSeverity}
\end{table}


Cases that have a high standard deviation of severity scores indicate a lack of consensus about the impact of that concept.
Recall that the data transformation described in Section~\ref{secAnalysis} will have the effect of increasing standard deviation beyond what we might expect for understandability, due to the increased range.

The absolute values of severity ratings exhibit a mean of 5.91 and a median of 6.00, suggesting that participants often assigned a moderate level of seriousness to privacy concerns on the scale.
However, the standard deviation of 3.04 indicates a substantial amount of variation, indicating that individuals held inconsistent views on the severity of the cases.
Table~\ref{tabledispSeverity} shows the cases with the greatest and least consensus.

Notably, the top half of Table~\ref{tabledispSeverity} shows that cases with strong consensus seemed to have two flavors: those strongly favoring the Service Provider and those favoring neither party.
Interestingly, participants do not seem to be accurately assessing the impact of \case{155}{very broad}.
We have discussed this case before, back in Section~\ref{secUnderstandabilityHiLo}, where we described how it strongly favors the Service Provider.
To some extent, such misperception is unsurprising since that case received the lowest understandability score.
Section~\ref{secRQ3} returns to the relationship between severity and understandability.

Turning to the lower half of Table~\ref{tabledispSeverity}, we see that many of the cases with high standard deviation have means closer to the middle of the [-10, 10] range.
This indicates that although certain participants interpreted a case as being in favor of one party, others regarded it as favoring the \textit{other} party.

%%%%%%%%%%%%%%%%%%%%%%%%%%%%%%%%%%%%%%%%%%%%%%%%%%%%%%%%%%%%%%%%%%%%%%%%%
\subsection{Implications}

First, the \textbf{severity ratings reflect strong power dynamics inherent in privacy regulations}, as many cases appear to favor the protection of the Service Provider over the User.
The polarized ratings of severity clearly indicate that firms tend to put safeguarding their own interests over ensuring consumer privacy.
This is unsurprising in some ways since the Service Provider is the one creating the document in the first place.

Second, \textbf{many cases had very low consensus for severity ratings}.
Certain participants saw these cases as extremely severe (favoring the Service Provider), while others considered them to be less influential, or even favoring the User.
We attribute the limited agreement on severity to two primary contributing factors.
The first is \textit{contrasting viewpoints on the importance of privacy}.
Various participants have different views on the importance of different categories of data and protective techniques.
Although some individuals may not perceive certain privacy regulations as detrimental, others consider them to be grave infringements.
The second is \textit{deficiencies in complexity and comprehensibility}.
Certain privacy terminology may need specialized expertise to appreciate its consequences.
Should participants have limited technical knowledge, they may struggle to precisely assess the seriousness, resulting in increased inconsistency in evaluations.
Thus, we turn to the interactions between severity and understandability next.
