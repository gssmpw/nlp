\section{Results RQ3 - How do severity and understandability relate to each other?
\draftStatus{one fix remains, revision highlights are in}
}
\label{secRQ3}

\revised{%
Here, we examine the relationship between severity and understandability from multiple perspectives, including via correlation and analyzing the extremities of the range from our data collection.
For each question, we cropped the data to meet the specific condition for that question.

\begin{figure}
  \centering
  \includegraphics[width=.85\textwidth]{assets/Scatterplot_UA_vs_S}
  \caption{
    Scatterplot of Average Severity (AVG\_S) vs Average Understandability (AVG\_UA).
  }
  \Description{The scatter plot is a reflection of the varied responses for the average values of understandability and severity. As understandability increases, the severity shows greater variability, indicating that participants respond with a wider range of severity for cases they find more understandable.}
  \label{fig:scatterplot-ua-vs-s}
  \FIXED{JED: List of demands for this figure:
  1. export this figure as a pdf, this is a bitmapped representation and is thus bad for zooming
  2. Remove the chart title so that it fits nicer on the page
  3. Relabel things from UA to U, since that is the notation we use elsewhere in the paper.
  4. Clarify in the caption what a dot is here (I think it is a case)
  }
\end{figure}

Analysis of participant ratings about case severity suggests that \textbf{increased understandability scores weakly correlate with elevated mean severity scores}.
Figure~\ref{fig:scatterplot-ua-vs-s} shows the scatter plot of these data, which demonstrate a weak positive correlation (Pearson, 0.154) between participants' perception of understandability and severity.
In addition, the data illustrate a ``funnel shape'', with very few samples with extreme severity ratings paired with low understanding.


\begin{figure}
    \centering
    \includegraphics[width=0.8\linewidth]{assets/new/Und_vs_Avg_Sev.png}
    \caption{Relationship between average Understandability scores and average Severity scores.
    To read the figure, when the understandability score was 1, the mean severity score was 4, and when the severity score was 1, the mean understandability score was about 8, and so on.
    Note that as the understandability score increases from 1 to 10, the average severity score correspondingly rises from 4 to 6.3.
    %(favoring the service provider to favoring the user). % commented since I don't see why this is a direction shift as described.
    }
    \Description{The figure is of a line chart, with two data series, one for understandability, the other severity. Each categorical X-axis position is given by a rating for one of the data series. On the numerical Y-axis is the average value of the OTHER data type for all cases receiving the X-axis rating. In the figure, we broadly see a trend for understandability that starts high, goes down, then back up. Severity starts low and goes up before plateauing.}
    \label{fig:relationship}
    \FIXED{JED@Anyone: List of demands for this figure:
    1. There is a spelling error in the X-axis label (comparative)
    2. Put the legend in the chart area so that we can make the image smaller on the page
    3. specify the Y axis to be from 2 or 3 to 9 or 10 so that we zoom in on the relevant portion a bit more.
    4. Widen the chart area, so it fills the page a bit better by having a wider aspect ratio.
    5. Remove the title

    Finally,
    6. Export this figure as a pdf, this is a bitmapped representation and is thus bad for zooming.
    7. Clarify and extend what I wrote in the caption since I am guessing this is on the $|S|$ data, as opposed to $S$, otherwise the ranges would be different. Without being sure, it is hard for me to finish the last part of the caption (what the reader is supposed to see in the figure)
    }
\end{figure}

Figure~\ref{fig:relationship} shows the trend of average understandability and severity scores with respect to  each other.
As the severity scores go up, the mean understandability scores also go up, a pattern reflecting the observations from prior work.
In one case, Reidenberg et al.~\cite{reidenberg2015disagreeable}, who found that users tend to engage more critically with privacy cases when the content is simplified and the implications are easier to grasp.
Our results also align with Kumar et al.~\cite{kumar2020strengthening}, who observed that when individuals comprehend privacy concepts, they often express concern about the potential harms, particularly when the implications are framed as directly affecting their rights or interests.

In Section~\ref{sec:highSeverity} we noted that severity is polarized towards values of 1 and 10 favoring both parties. 
Here we note that the mean understandability also follows the polarity, with the extreme cases being the most understandable.
Bravo et al.~\cite{bravo2010bridging} reported similar polarization in user reactions to annotated privacy policies, where simplified summaries often elicited strong opinions even with omission of detail.
Our findings indicate that participants are able to better assess severity once the case is more understandable.



%%%%%%%%%%%%%%%%%%%%%%%%%%%%%%%%%%%%%%%%%%%%%%%%%%%%%%%%%%%%%%%%%%%%%%%%%
\subsection{How severe are the minimally understandable cases?}

In order to determine the ``minimally understandable'' cases, we selected the responses with an understandability score of `1' (110/2495).
When examining the these cases, the data unveils a more intricate structure.

Out of the minimally understandable responses, a total of 35 responses assigned a severity rating of -5 or lower to the case indicating that they considered the cases to be severe.
All of these cases were in favor of the Service Provider. %JED: I am confused by this part, since the - sign above indicates as much, or is the "ALL" here referring to the 110, not the 35
This indicates that despite challenges in understanding the information, some individuals still perceived the implications of the case as severe for the user.
Sundar et al.~\cite{sundar2013unlocking} noted similar patterns, where users relied on heuristic decision-making to assess the implications of privacy scenarios, even when comprehension was low.
Further, a total of 7 responses considered the cases to be most severe and favoring the Service Provider.
One such example of a case is 
\case{71}{The copyright license that you grant this service is limited to the parties that make up the service's broader platform.}
These findings align with Isaak and Hanna~\cite{isaak2018user}, who observed that users can intuitively recognize harm in privacy practices, even when they lack full understanding.

%%%%%%%%%%%%%%%%%%%%%%%%%%%%%%%%%%%%%%%%%%%%%%%%%%%%%%%%%%%%%%%%%%%%%%%%%
\subsection{How severe are the maximally understandable cases?}

We then examined the most understandable instances, in which participants reported `10' understandability (814/2495).
Of these, 424 instances were in favor of the Service Provider, out of which 200 also received the most extreme severity level of -10.
Conversely, only 284 responses (of the 814 maximally understandable cases) indicated a perception of being in favor of the User, with 59 cases classified as the most severe (severity = 10).

Our findings indicate that individuals who have a complete understanding of a case are more likely to gauge it as extremely severe, especially when the Service Provider gains advantages from the case.
This observation aligns with the work of Kelley et al.~\cite{kelley2010standardizing}, which showed that clarity in privacy policies helps users recognize severe consequences tied to specific practices.

%%%%%%%%%%%%%%%%%%%%%%%%%%%%%%%%%%%%%%%%%%%%%%%%%%%%%%%%%%%%%%%%%%%%%%%%%
\subsection{How understandable are the most extremely severe cases?}

Furthermore, we investigated the understandability of the most extremely severe cases, particularly concentrating on situations when participants assigned the most extreme severity $|S|=10$ (421/2495).
Of the 421 responses participants classified with extreme severity, 344 of those were in favor of the Service Provider, and out of the 344, 200 had the highest understandability score of 10.
These observations suggest that participants well understood a substantial proportion of extremely severe cases.

Notably, 8 responses were simultaneously most extremely severe and minimally understandable.
Among these cases, 7 favored the Service Provider.
These findings suggest that some persons may see a situation as highly serious, even without full understanding, particularly when the outcome is unfavorable to the User.
This reflects Bravo et al.~\cite{bravo2010bridging} and Taddicken \cite{taddicken2014privacy}, who noted that even in cases of limited comprehension, the perceived implications of privacy breaches can significantly influence user perceptions.

} % end revised

%%%%%%%%%%%%%%%%%%%%%%%%%%%%%%%%%%%%%%%%%%%%%%%%%%%%%%%%%%%%%%%%%%%%%%%%%
\subsection{Implications}

The results obtained from this study have implications for privacy policy formulation and user education.
The substantial association between high understandability and high perceived severity, particularly in cases that benefit the Service Provider, implies that \textbf{users are more inclined to recognize the seriousness of a situation when they possess a stronger understanding of the details}.
However, the existence of severe cases that have low understandability highlights the urgent requirement to improve the understanding of intricate policy terminology, especially those that unfairly advantage Service Providers.
One potential consequence of this lack of understanding is that users may underestimate the potential dangers associated with specific privacy regulations or practices, which in turn highlights the need for focused educational efforts.
Ultimately, more explicit policies not only advantage the user but also establish confidence and promote openness, therefore reducing legal conflicts and enhancing customer well-being.
