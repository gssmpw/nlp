\section{Results RQ1 - What does the \textit{understandability} data reveal about participants and privacy concepts?
\draftStatus{one fix remains, revision highlights are in}
}

\boldify{Plot the overall course of the argumentation (here the comma-separated list is roughly the section heading list)}

\revised{%
In this section we report results based on the understandability data, interpreting the grand mean, case mean, case variances, and qualitative results.

\boldify{Overall, the ratings were high, suggesting understandability.
Also, the results were not outlier-driven, because we see a majority of SOMETHING being high scores.}

Overall, we find the descriptive statistics of understandability scores aggregated across all cases and participants indicate that \textbf{\textit{participants reported that the concepts represented by the overall case label set were understandable}}, with moderate variation (Mean=7.42, Median=8, SD=2.71).
Further, a large portion (>72\%) of responses rated the cases as highly understandable with a rating of 7 or more, which indicates that \textbf{most participants found the content to be understandable}.
These findings align with prior research indicating that simplified and annotated versions of privacy policies enhance user comprehension~\cite{kelley2010standardizing, reidenberg2015disagreeable}.  

\boldify{HOWEVER, we do see some low scores, which means some mixtures of case+individuals did not work}

However, despite the high overall ratings, some cases remained unclear for some individuals.
About 5\% of cases received at least one response of 1, indicating very low understandability, while 31.66\% cases got the highest score of 10 from at least one participant.
Additionally, 83.93\% of the responses rated the understandability $\geq5$, indicating that the vast majority of participants found the content to be reasonably understandable.
Other researchers have observed similar patterns, like Bravo et al.~\cite{bravo2010bridging}, who found that even simplified policies can sometimes fail to achieve universal clarity.
Despite this limitation, our findings are consistent with evidence that structured, plain-language approaches are effective in bridging the comprehension gap~\cite{milne2004consumers, schaub2015design}.

%%%%%%%%%%%%%%%%%%%%%%%%%%%%%%%%%%%%%%%%%%%%%%%%%%%%%%%%%%%%%%%%%%%%%%%%%
\subsection{Which privacy concepts received low understandability ratings?}
\label{secUnderstandabilityHiLo}

\begin{table}[t]
	\centering
    \footnotesize
    %\framebox[\linewidth]{For Measuring!}
	\begin{tabular}{@{}l | l | l@{}}   
    \textbf{Case Label} &
    \textbf{U mean} &
    \textbf{U SD}\\\hline\hline
    
\caseTable{89}{The service has a no refund policy} &
9.67	& 0.82\\\hline

\caseTable{154}{Users who have been permanently banned from this service are not allowed to re-register under a new account}&
9.67	& 0.50\\\hline

\caseTable{175}{You are responsible for any risks, damages, or losses that may incur by downloading materials} &
9.60	& 0.55\\\hline

\caseTable{131}{This service is only available for use individually and non-commercially} &
9.57	& 1.13\\\hline

\caseTable{191}{You can retrieve an archive of your data} &
9.56	& 1.01\\\hline

\caseTable{27}{If you are the target of a copyright claim, your content may be removed}&
9.50	& 0.55\\\hline

\caseTable{218}{Your browsing history can be viewed by the service} &
9.43	& 1.13\\\hline

...& ... &...\\\hline

\caseTable{15}{Content is published under a free license instead of a bilateral one	} &
3.50	& 2.28\\\hline

\caseTable{19}{Defend, indemnify, hold harmless; survives termination} &
3.31    & 2.72\\\hline
\caseTable{80}{The service claims to be CCPA compliant for California users} &
3.00	& 2.83\\\hline
\caseTable{194}{You cannot delete your account of this service} &
2.82	& 2.68\\\hline

\caseTable{81}{The service claims to be GDPR compliant for European users} &
2.67	& 2.78\\\hline

\caseTable{76}{The policy refers to documents that are missing or unfindable} &
2.63	& 0.74\\\hline

\caseTable{155}{very broad} &
1.70	& 1.34\\\hline

	\end{tabular}
	
    \normalsize
	\caption{Descriptive statistcs for the top and bottom ranked cases, based on mean \textit{understandability} (U).
    Please refer to our supplemental materials for descriptive statistics for all cases.
    }
	\label{tableConcUnderstand}
\end{table}


\boldify{Next we zoom into cases with low understandability, which matter because they are opportunities}

Cases with low understandability ratings indicate opportunities to clarify the case or educate the public.
} % end revised
Table~\ref{tableConcUnderstand} shows the text and ratings of the privacy concepts that participants found most (and least) understandable.
Since the overall understandability was high, only about 11\% of the responses had a low understandability score of less than 3.
In terms of number of cases, only 5 cases had an understandability score of 3 or less.
Fortunately, the corresponding \textit{severity} scores of these cases was between -3.3 to 2.72, suggesting these cases are not very serious.
Our results show that a portion of the participants had difficulty understanding certain privacy concepts, despite the fact that the average score of 7.52 and 85.01\% of scores being 5 or higher indicates that the majority of respondents found the content understandable.

\boldify{Let's look at a case where clarification is due: word salad}

We found some aspects of this data unsurprising, such as \case{155}{very broad} obtaining a very low understandability score, suggesting clarification is warranted for this case.
Note that this case label actually implies that users \textit{provide platforms broad rights} to use, alter, and distribute their material without any additional limitation.
This provision curtails users' autonomy in determining the handling of personal data.

\boldify{And one more interesting one: missing documents/broken links}

Contrastingly, we were surprised that \case{76}{The policy refers to documents that are missing or unfindable} received such a low understandability score.
As with the cases found at the top of Table~\ref{tableConcUnderstand}, the terminology is fairly simple, but participants found this case unclear.
Further, they had a strong consensus about this perception, as evidenced by the low standard deviation.
Apparently, this particular case should be fairly common, as recent work found that \textit{``privacy policies URLs are only available in 34\% of websites''}~\cite{srinath2023lost}.

%%%%%%%%%%%%%%%%%%%%%%%%%%%%%%%%%%%%%%%%%%%%%%%%%%%%%%%%%%%%%%%%%%%%%%%%%
\subsection{Which privacy concepts exhibited low consensus about understandability?}

\begin{table}
	\centering
    \footnotesize
    %\framebox[\linewidth]{For Measuring!}
	\begin{tabular}{@{}l | l | l@{}}   
    \textbf{Case Label} &
    \textbf{U mean} &
    \textbf{U SD}\\\hline\hline
    
\caseTable{147}{Tracking pixels are used in service-to-user communication}&
3.75	& 3.92\\\hline

\caseTable{4}{Accessibility to this service is guaranteed at 99\% or more}&
5.40	& 3.91\\\hline

\caseTable{197}{You cannot opt out of promotional communications}&
4.42	& 3.90\\\hline

\caseTable{44}{Minors must have the authorization of their legal guardians to use the service}&
6.25	& 3.86\\\hline

\caseTable{6}{Alternative accounts are not allowed}&
5.67	& 3.82\\\hline

\caseTable{217}{Your browser's Do Not Track (DNT) headers are respected}&
5.56	& 3.81\\\hline

\caseTable{71}{The copyright license that you grant this service is limited to the parties that make up the service's broader platform}&
5.45	& 3.78\\\hline

...& ... &...\\\hline

\caseTable{43}{Many third parties are involved in operating the service}&
7.00	& 0.76\\\hline

\caseTable{25}{First-party cookies are used}&
9.17	& 0.75\\\hline

\caseTable{76}{The policy refers to documents that are missing or unfindable}&
2.63	& 0.74\\\hline

\caseTable{79}{The service can intervene in user disputes}&
6.00	& 0.63\\\hline

\caseTable{175}{You are responsible for any risks, damages, or losses that may incur by downloading materials}&
9.60	& 0.55\\\hline

\caseTable{27}{If you are the target of a copyright claim, your content may be removed}&
9.50	& 0.55\\\hline

\caseTable{154}{Users who have been permanently banned from this service are not allowed to re-register under a new account}&
9.67	& 0.50\\\hline

	\end{tabular}
	
    \normalsize
	\caption{
    Descriptive statistics for the top and bottom ranked cases, based on standard deviation of understandability (U)
    }
	\label{tableDispUnderstand}
\end{table}


\boldify{Now we switch lenses to examine consensus as a way to triangulate with central tendency}

Large standard deviations indicate variation in the participants' ability to understand particular ideas.
This implies that some participants may have found some elements of the privacy material more difficult to understand than others.
Although participants perceived most cases to be moderately understandable, 45 cases had a standard deviation of 3 or more.
Table~\ref{tableDispUnderstand} illustrates the cases that had the most and least consensus about understandability.

\boldify{Variation among participants can explain some of this, but also could indicate opportunities to clarify/educate}

Notably, several of the cases found in the top portion of Table~\ref{tableDispUnderstand} involve technical content (e.g., tracking pixels, do not track headers) or copyright issues.
As a result, we might expect natural variations in the understandability of these cases due to variations in participants' technical and legal knowledge.
On the other hand, the bottom portion of the table tends to refer to more accessible topics (e.g., responsibility, content removal, number of parties involved).

\revised{%
%%%%%%%%%%%%%%%%%%%%%%%%%%%%%%%%%%%%%%%%%%%%%%%%%%%%%%%%%%%%%%%%%%%%%%%%%
\subsection{How did participants rewrite case labels for low understandability concepts?}

\boldify{Give a little context about the overall data density of the exercise}

Out of 243 unique cases, participants submitted rewrites for 215 cases (88\%).
Out of those 215 unique cases, 72 cases had only 1 rewrite and 63 cases had 2 rewrites.
As the number of rewrites increases, the number of cases decreases, with only 4 unique cases receiving 6 rewrites.
Table~\ref{tableRewwriteStats} illustrates the case labels of the cases that got the most rewrites with the least consensus defined by the standard deviation score of understandability of those cases being 2 or more.
For these cases, participants perceived a larger severity range, as evidenced by standard deviation scores of up to 7.



\boldify{JED is not totally sure, but it seems to me that the idea is that both of these participants generated similar text, which is essentially an expansion of the list}

Participants rephrased this case:
\case{192}{You can request access, correction, and/or deletion of your data} by essentially expanding upon the three-item list provided.
} % end revised
\quotateInset{P146}{ignore}{ignore}{You are allowed to ask the service provider to access your data that it has collected, to correct the data, or to have it deleted from their system altogether. }
\quotateInset{P37}{ignore}{ignore}{You can see your personal data that has been collected. If there are any errors, you are able to make corrections. You can also have your data deleted completely.}

\begin{table}
	\centering
    \footnotesize
    %\framebox[\linewidth]{For Measuring!}
	\begin{tabular}{@{}l |l|l|l@{}}   
    \textbf{Case \#} & \textbf{rewrite \#} &
    \textbf{U SD} &
    \textbf{Severity SD}\\\hline\hline

\caseTable{192}{You can request access, correction and/or deletion of your data} & 6 &
3.71	& 2.30	\\\hline

\caseTable{153}{Usernames can be rejected or changed for any reason} & 6	& 3.52	& 5.13
\\\hline

\caseTable{10}{Any liability on behalf of the service is only limited to the fees you paid as a user} & 5	& 3.08	& 3.16
\\\hline

\caseTable{2}{A license is kept on user-generated content even after you close your account} & 5	& 2.92	& 5.36
\\\hline

\caseTable{36}{Instead of asking directly, this Service will assume your consent merely from your usage.} & 5	& 2.77	& 7.01
\\\hline

\caseTable{94}{The service is only available in some countries approved by its government} & 6	& 2.16	& 2.97
\\\hline

\caseTable{125}{This service gives your personal data to third parties involved in its operation} & 5	& 2.12	& 1.67
\\\hline

\caseTable{201}{You have the right to leave this service at any time} & 6	& 2.00	& 3.78
\\\hline


	\end{tabular}
	
    \normalsize
	\caption{ Cases with maximum rewrites having low understandability (U) consensus, alongside  
    }
	\label{tableRewwriteStats}
\end{table}



\revised{%
To take a second example, participants rewrote this case:
\case{153}{Usernames can be rejected or changed for any reason} in a similar way, expanding upon the provided label and removing passive voice.
} % end revised
\quotateInset{P473}{ignore}{ignore}{The company can, without notifying the user in advance, change or delete screen usernames at any point for any reason.}
\quotateInset{P29}{ignore}{ignore}{The service provider can change your username without having to notify you.}

Despite the case label not incorporating the element of notification, both of these participants drew it out of the more detailed description, which suggests that clarification is important to them.


%%%%%%%%%%%%%%%%%%%%%%%%%%%%%%%%%%%%%%%%%%%%%%%%%%%%%%%%%%%%%%%%%%%%%%%%%
\subsection{Implications}

\boldify{\#1 We can explain with this taxonomy without doing further work}

First, the overall understandability of the cases is fairly high, \textbf{suggesting that \tosdr{} cases are suitable for use as explanation artifacts with little to no modification}.
That said, there is clearly still some room to improve the clarity of the text.
This is evident because ratings were not uniformly high and because we were able to find some rewrites that we thought were better.

\boldify{\#2 By following two sources, we can identify things to spur education and clarification (i.e., we might still want to do some revisions by accepting some of the work)}

Second, we have identified concepts that may be appropriate for educational interventions.
Specifically, the cases with low overall understandability indicate that either the label is confusing or that participants are under-informed about that concept.
Further, the cases with low consensus about the level of understandability indicate that \textit{certain segments} of the sample are under-informed about that concept (we will return to this topic in Section~\ref{secRQ4}).
Last, identifying concepts where expert opinion diverges from participants' perception of the party a concept favors and how much (severity) could prove a valuable guide for educational intervention.
As such, we turn to severity data next.
