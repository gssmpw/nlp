\section{Discussion
\draftStatus{barring audits, R+R DONE}
}

\subsection{What value does a user-centered weighting for an expert taxonomy provide?}

\boldify{Illuminates participants' values, and so has value in its own right, as in Grgic-Hlaca}

Illuminating participants' abilities, values, preferences, and so on has value in its own right.
To take a single example, Grgić-Hlača et al.~\cite{GrgicHlaca2018fairness} performed a similar crowdworker survey to collect attitudes about the fairness of using certain features in recidivism prediction.
Such information can support a variety of tasks, such as bug triage informed by human data or feature engineering.

\boldify{Explainability, via weight the presence of these concepts to "grade" a whole policy. Currently, \tosdr{} essentially uses 3 weights , if I recall?)}

Another application for our data is in explainability.
In Section~\ref{secIntroduction} we mentioned how \tosdr{} assigns letter grades to services based on user reports of cases presence in policy documents.
That process currently uses 3 weights: (-10, 0, 10), which is part of the inspiration for our two-part question for severity because the current weights are essentially a sign function.
By adding nuance to the weighting schema, it is possible to improve user experience by both increasing information quality and customization.
Our work can help users focus on concepts that other people have found important (e.g., output can be in order by severity ratings).

\subsection{What are some future research hypotheses or educational interventions that this work illuminates?}

\boldify{inform educational interventions (understandability low, severity differs from expert opinion, low consensus about severity).}

Ultimately, each case that satisfies any of three conditions offer research and intervention opportunities:
Condition 1 - the concept has low understandability;
Condition 2 - the concept has low consensus about severity;
Condition 3 - the severity differs from expert opinion.
While we leave evaluating Condition 3 to future work, our data supports targeting educational interventions and research efforts around concepts meeting Conditions 1 and/or 2.
While our full data is available in our supplemental materials, here we offer two examples of interesting questions to pursue further.

\subsubsection{Why is Open Source such a strong indicator of privacy protection?}

As Table~\ref{tableConcSeverity} shows, the case most favoring Users, by a \textit{wide} margin, is:
\case{95}{The service is open-source}.
We found this interesting because having open-source code does not \textit{inherently} prevent careless or malicious behavior within that source.
However, it does indicate that the Service Provider holds certain values, namely transparency, which apparently participants felt were more favorable than even prohibitions on 3rd party sharing or ownership of their content.

\subsubsection{To what extent do people bound or protected by laws like CCPA/GDPR actually know the rules specified therein?}

\boldify{Let's look at a less clear-cut, but important case: legal compliance}

One surprising result from Table~\ref{tableConcUnderstand} is that the cases following the form:
\case{80, 81, 82}{The service claims to be [LAW] compliant for [REGION] users}
received very low understandability scores.
We offer three hypotheses for this observation:
The first hypothesis is that participants do not know what the GDPR/CCPA \textit{is} because \textit{``acronyms constitute a private language''}~\cite{CISEcareer} and they do not speak it.
The second is that they recognize the acronym, but do not know the contents of the law because it is not relevant to them.
Specifically, our participants were US-based, and therefore not bound by GDPR.
Meanwhile, based on rough population counts, we would only expect about 12\% of our participants to be California-based, and so most of our participants are unlikely to be bound by CCPA \textit{either}.
The last hypothesis is that they recognize the acronym, but do not know the law's contents, \textit{despite} its applicability to their lives.