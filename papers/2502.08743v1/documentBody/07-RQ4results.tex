\section{Results RQ4 - How did demographic traits affect understandability and severity?
\draftStatus{JED about to work here}}
\label{secRQ4}

The survey, conducted through the Prolific platform, got responses from 499 participants\footnote{We had demographic information from the one participant whose survey response was unrecoverable.} from the United States, who came from a very diverse demographic distribution.
Based on the information we collected, we observed the following:

\paragraph{Age} 
Based on the age data, the mean age of the participants is 35.65 years, and half of the participants are below the age of 33.
The standard deviation of 11.44 indicates a considerable heterogeneity in the age distribution among the subjects.
The estimated standard error of 0.51 indicates a little margin of error in calculating the population mean, therefore indicating that the sample offers an accurate estimation of the average age of the participants.
In general, the data indicates a wide range of ages, characterized by a central trend around the mid-30s.

\paragraph{Gender}
The gender distribution in the dataset indicates that women constitute the majority of participants, representing 52.2\% of the total sample. 
Men constitute 46.8\% of the participants, with 1\% opting not to reveal their gender.

\paragraph{Ethnicity}
The slight majority of the participants we sampled identified as White, accounting for 57.6\%.
A total of 13.4\% of the participants identified as Black, 12.2\% identified as Asian, and 10.4\% identified as mixed race.
About 5\% reported belonging to other ethnic groups, while 1\% selected ``N/A'' and 0.4\% opted not to declare their ethnicity.

\paragraph{Employment Status}
A plurality of the participants (36.6\%, 183 individuals) were engaged in full-time employment, while 17.4\% (87 individuals) were working part-time.
Approximately 14.6\% of the respondents reported being unemployed, but an equivalent amount of 14.6\% marked their status as ``N/A'' (which might include students, retired persons, or those who elect not to reveal their employment status).
In addition, 11.2\% of the surveyed population are not currently employed, and 5.6\% are classified as ``Other,'' which may indicate other job circumstances.
The observed distribution exhibits a wide range of job statuses, with full-time workers being the most prevalent category, but also substantial representation from individuals who are not now engaged in paid employment.

\paragraph{Student Status}
The bulk of the participants were not students (67.2\%, 336 individuals), while 19.6\% (98 individuals) self-identified as students.
Furthermore, 13.2\% (66 individuals) replied ``N/A.''


In order to determine how different demographic traits affect understandability and severity, we tested the following  hypotheses:
\begin{itemize}
\item H1: Employed individuals report higher understandability scores than unemployed individuals
\item H2: Older participants record a higher understandability score than young adults.
\item H3: Avg severity scores differ significantly across occupations
\item H4: There is no significant difference in the perceived severity across genders
\end{itemize}



%%%%%%%%%%%%%%%%%%%%%%%%%%%%%%%%%%%%%%%%%%%%%%%%%%%%%%%%%%%%%%%%%%%%%%%%%
\subsection{H1: Employed individuals report higher understandability scores than unemployed individuals}

In order to test this hypothesis, we categorized participants into two groups:
(1)~\textit{Employed} participants (389 out of 499), who reported their profession or field of work; and
(2)~\textit{Unemployed} participants (110 out of 499), who indicated that they do not work, are students, or selected ``N/A.''
This grouping allowed us to compare understandability scores based on employment status.

We found that \textit{Employed} individuals reported significantly higher understandability scores than \textit{Unemployed} individuals (two-sample t-test, t(161) = -2.93, p = 0.004).
The p-value indicates a statistically significant difference between the groups, rejecting the null hypothesis that employment status has no impact on understandability scores.
The effect size, measured using Cohen's d, was 0.45, which corresponds to a moderate practical significance.
This suggests that while the difference is not only statistically significant, it also has meaningful implications in practical settings.

These findings indicate that employment status plays a role in influencing understandability scores, with \textit{Employed} individuals reporting a better ability to comprehend the material.
This difference may stem from factors such as exposure to workplace communication, problem-solving skills, or familiarity with legal and technical language in professional settings.
Future research could further explore the underlying reasons for this disparity by examining specific aspects of employment, such as job type, education level, or cognitive load, that may contribute to differences in understandability.
Additionally, interventions to enhance understandability for \textit{Unemployed} individuals could help bridge this gap and improve inclusivity in policy and document design.


\subsection{H2: Older participants record a higher understandability score than young adults}

This study hypothesized that older participants would exhibit higher understandability scores compared to younger adults.
Descriptive statistics revealed that participants in the 25--34 age range reported the highest average understandability score (Mean = 7.58, SD = 1.72), while participants aged 65--74 had lower average scores (Mean = 6.17, SD = 2.43).
To test H2, we analyzed understandability scores across different age groups with one-way ANOVA.
We found significant differences among the age groups in average understandability scores (one-way ANOVA, F(6, N=7) = 3.72, p = 0.00125).

These findings suggest that understandability does vary with age, but the trend is not consistent for older participants. These findings suggest that while understandability decreases in older age groups, severity perceptions remain stable across the lifespan.
These results align with prior findings by authors in this article~\cite{lythreatis2022digital}, who noted that younger users often engage more actively with digital platforms but may undervalue privacy risks, while older users tend to exhibit higher privacy concerns due to greater awareness of potential data misuse.
Additional post-hoc tests and a larger sample size should explore this relationship further, especially given the small sample sizes in some age categories (e.g., 75+).

\begin{figure}
    \centering
    \includegraphics[width=0.8\linewidth]{assets/new/avg_under_sev_age.png}
    \caption{Box Plot for Understandability and Severity scores for different age groups}
    \Description{The figure shows 7 bar charts of each data type broken down by age group. Broadly speaking, the understandability distributions seem to rise with age before peaking and declining. For severity, it is less stable, but generally rises with age.}
    \label{fig:ageBoxplot}
\end{figure}

Figure~\ref{fig:ageBoxplot} highlights the distribution of average understandability and severity scores across age groups.
Understandability scores are relatively consistent among participants aged 18 to 64, with medians around 7--8 and minimal variability, while participants aged 75 and above show a sharp decline in scores.
In contrast, severity scores demonstrate consistent medians (6--7) and narrow interquartile ranges across all age groups, suggesting stability in perceptions of severity.
However, outliers are evident in both metrics, particularly among younger (18--24) and older (75+) participants, indicating individual deviations. 

\subsection{H3: Avg severity scores differ significantly across occupations}

To test this hypothesis, we analyzed average severity scores from participants in various occupational groups via a one-way ANOVA, which examines differences in mean scores across multiple categories.

Descriptive statistics indicated that severity scores showed slight variations among occupations, with mean scores ranging from 5.24 (Arts and Media) to 6.50 (Legal).
Some occupations, such as Legal and Community and Social Services, exhibited relatively high average severity scores with low variability, suggesting more consistent perceptions within these groups.
In contrast, occupations like Arts and Media and Farming, Fishing, and Forestry showed greater variability in scores, indicating a broader range of severity perceptions among participants within these fields.

Despite these observed differences, we found no statistically significant difference in average severity scores across occupational groups (one-way ANOVA, F(29, N=30) = 0.86, p = 0.685).
This finding suggests that occupational background does not have a significant impact on how participants perceive severity.
This can also be a result of a low number of participants leading to a low sample size per occupation.
The results imply that severity perceptions may be influenced more by individual or contextual factors rather than by occupational roles.

\subsection{H4: There is no significant difference in the perceived severity across genders}

To test this hypothesis, we compared average severity scores for all genders via a one-tailed t-test.
Descriptive statistics revealed that men had a slightly higher mean severity score (M = 7.45, SD = 1.87) compared to women (M = 7.37, SD = 1.75).
Participants who preferred not to disclose their gender reported the highest average severity score (M = 8.32, SD = 0.94).
However, the small sample size for the ``Prefer not to state'' was insufficient for statistical testing.

The one-tailed t-test, designed to test whether men report significantly higher severity scores than women, resulted in a t-statistic of 0.49 and a p-value of 0.312.
As the p-value exceeded the conventional threshold for statistical significance (p > 0.05), the analysis failed to provide sufficient evidence to support the hypothesis that men perceive cases as more severe than women.
These findings suggest that, on average, there is no significant difference across these two genders in their perception of severity scores.

While the mean scores for men and women appear similar, the slight variation may be attributed to individual differences, noise in the data, or other confounding variables.
The lack of statistical significance indicates that gender may not be a critical factor influencing the perception of severity in this context.
This result aligns with prior research suggesting that perceptions of severity are often shaped by factors other than gender, such as cultural background, professional experiences, or risk tolerance.
However, these results offer a slight contrast with the prior work by Hoy and Milne~\cite{hoy2010gender}, who observed that women are generally more privacy-conscious and hesitant to share personal information online, while men often prioritize functionality over privacy.

\begin{table}[htb]
    \centering
    \footnotesize
    \begin{tabular}{@{}p{.465\textwidth}|l|l|l|p{.32\textwidth}@{}} 
     \textbf{Questions}
     & \textbf{Mean} 
     & \textbf{Med.}
     & \textbf{SD}
     & \textbf{Interpretation} 
     \\\hline

     \qFour{}
     & 4.08  & 3 & 2.92 &
     Most Participants are not highly motivated to read privacy policies, but there is significant variation, suggesting some individuals are very motivated while others are not.  
     \\\hline

      \qFive{}
      & 4.12	& 3	& 2.88 &
      Participants are generally unlikely to spend time understanding privacy policies, but again, responses varied widely.
     \\\hline
     
     \qSix{}
      & 6.39	& 7	& 3.00 &
      Participants tend to focus more on the general overview rather than specific sections, but the responses vary greatly.
     \\\hline
     
     \qSeven{}
      & 6.63	& 7	& 2.51 & Participants feel moderately confident in managing privacy settings, with relatively less variation in responses.
     \\\hline
     
    \qEight{}
     & 5.87	& 6	& 3.01 &
     Participants feel moderately comfortable using privacy tools, but some are much more comfortable than others.
    \\\hline

    \qNine{}
     & 7.59	& 8	& 2.33 &
     Control over personal information is very important to most participants, with lower variability in responses. 
    \\\hline
    
    \qTen{}
     & 5.39	& 5	& 2.70 &
     Participants are moderately proactive about updating their privacy settings, but responses are somewhat varied.
    \\\hline
    
    \qEleven{}
     & 4.44	& 4	& 2.46 & 
     Participants are generally uncomfortable sharing personal information online, but comfort levels vary across individuals.
    \\\hline
    
    \qTwelve{}
     & 7.06	& 8	& 2.58 &
     Many participants accept terms and conditions without reading them, and this practice is somewhat common across the group.
    \\\hline

    \qThirteen{}
     & 6.84	& 7	& 2.67 &
     Participants are fairly likely to act (e.g., change settings) when concerned about a privacy policy, though responses vary.
\\\hline

     \qFourteen{}
     & 6.48	& 7	& 2.67 &
     Participants show moderate interest in learning about tools to better understand privacy policies, but interest levels vary.

     \\\hline
    \end{tabular}
    \caption{Attitude questions participants saw evaluating privacy concerns with their mean, median and their standard deviation based on the scores each participant provided and the interpretation column to analyze the mean and standard deviation.}
    \label{table: demoStats}
\end{table}

\paragraph{Demographic Questions outlining participant attitudes, based on GenderMag Facets, as applied to privacy attitudes}
Table~\ref{table: demoStats} showcases the responses about the participants' attitudes toward privacy policies and privacy-related behaviors.
It includes 11 questions, each addressing different aspects of privacy management, ranging from motivation to read privacy policies to comfort with sharing personal information.
For each question, the table provides the mean, median, and standard deviation (SD) of responses, along with an interpretation of the results.

The analysis shows that while participants realize the importance of privacy and are somewhat proactive about privacy settings, many are not motivated to read or understand privacy policies.
For example, the mean response for motivation to read policies (Q1) is 4.08, indicating that participants are generally not inclined to invest time in reading, though individual responses vary widely.
This aligns with findings by Acquisti et al.~\cite{acquisti2015privacy}, who showed that the length and complexity of privacy policies discourage users from reading them.
Similarly, participants are moderately confident in their ability to manage privacy settings (Q4) and feel that control over personal information is important (Q6), with relatively lower variability in responses for these items. 
These findings are consistent with Trepte et al.~\cite{trepte2011privacy}, who found that while users value control over their data, their actual knowledge of privacy settings often falls short.

The table also highlights a general discomfort with sharing personal information online (Q8) and a tendency to accept terms and conditions without reading them (Q9), corroborating prior work on user behaviors regarding privacy management.
The authors in this research~\cite{gross2005information} similarly noted that most users are hesitant to share sensitive data but frequently bypass terms and conditions due to convenience.

\paragraph{Negative experiences}
Figure~\ref{figNegExp} highlights the various categories of adverse experiences encountered by participants, indicating that the most commonly reported concern was the receipt of advertisements, impacting 274 individuals.
This is followed by data breaches reported by 223 participants.
Additional concerns, including unforeseen charges, data exploitation, and account breaches, impacted a considerable number of participants.
Account suspensions were the least prevalent adverse experience, reported by only 70 participants.
These results align with findings by Martin and Murphy~\cite{martin2017role}, who noted that targeted advertisements and data breaches are among the most visible and frustrating outcomes of inadequate data privacy practices.
Figure~\ref{figNegMult} depicts the allocation of participants according to the frequency of negative experiences they faced, indicating that most participants encountered either one or two negative incidents.
The proportion of people indicating three or more adverse events consistently declines, with a mere 1\% reporting eight or nine occurrences.
Notably, 4\% of individuals reported no adverse events.
This aligns with findings by Solove~\cite{solove2012introduction}, who documented that privacy breaches are increasingly common but often concentrated among a subset of highly affected users.

\begin{figure}[b]
    \centering
    %\framebox[\linewidth]{For Measuring!}
    \includegraphics[width=.9\columnwidth]{assets/no_participants_neg.png}
    
    \caption{No. of participants facing negative experiences}
    \Description{This bar chart shows the trend of how many participants face which negative experiences. As an example, 274 participants have received ads.}
    \label{figNegExp}
\end{figure}
\begin{figure}
    \centering
    %\framebox[\linewidth]{For Measuring!}
    \includegraphics[width=.8\columnwidth]
    {assets/neg_experiences.png}
    
    \caption{Percentage of Participants facing multiple negative experiences}
    \Description{This bar chart shows the trend of how many participants (\%) face a certain number of negative experiences. 4\% of participants have not faced any negative experiences but 28\% faced 1 negative experience.}
    \label{figNegMult}
\end{figure}






