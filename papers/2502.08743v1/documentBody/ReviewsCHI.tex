\section*{1AC review (reviewer 4)}

Expertise

Knowledgeable

Originality (Round 1)

High originality

Significance (Round 1)

Medium significance

Research Quality (Round 1)

Low research quality

Recommendation (Round 1)

\textbf{I recommend Revise and Resubmit}

1AC: The Meta-Review

In this submission, "Signed, Sealed,... Confused: Exploring the Understandability
and Severity of Policy Documents", the authors use survey methods to
examine how well people understand terms of service documents. Reviews
generally praised the originality and potential significance of this
study. However, there were several concerns as well, which would need
to be addressed before that significance can be realized. Thus, I am
recommending Revise and Resubmit.

I'll summarize the suggestions here, but the authors are encouraged to read each
review in full:

1) The limitations of the Prolific sample should be taken more seriously,
particularly with regard to claims about "laypersons" (2AC)
2) The framing of the question "Which party does this case tend to favor?"
appeared to be confusing, and may require additional explanation (2AC)
3) All three reviewers had concerns about presentation clarity and organization,
including the introduction (R1), literature review (R1, R3), results
(2AC), and discussion (R3). The reviewers all had helpful suggestions
for how to improve the structure of this submission, which I suggest
the authors consider.
4) Potentially related to the above, the contribution of this work to the HCI
community needs to be clarified, and perhaps a more critical
consideration of the fact that ToS documents are, generally speaking,
largely designed to favor the service provider would help frame the
results (R1)
5) There were several concerns about the analysis, including the sample size, lack
of inferential statistics, and questions/claims that appear out-of-
scope given the methods used (R1).

Overall, the reviewers generally saw potential in this work, but I would encourage
the others to take these suggestions (particularly those related to
writing, framing, and methods) seriously; my impression is they are
all critical in evaluating the contribution of the work.

----------------------------------------------------------------

\section*{2AC review (reviewer 2)}

Expertise

Expert

Originality (Round 1)

High originality

Significance (Round 1)

High significance

Research Quality (Round 1)

High research quality

Contribution Compared to Length (Round 1)

The paper length was commensurate with its contribution.

Figure Descriptions

The figure descriptions are adequate and follow the accessibility guidelines.

Recommendation (Round 1)

\textbf{I recommend Revise and Resubmit}

Review (Round 1)

Summary: This paper describes the results of an online survey aimed at assessing
how Internet users understand the presentation of content from Terms
of Service documents.

Originality and Prior Work: The current draft makes a strong case that the survey
instrument designed for the study can provide a novel perspective on
the "Privacy Paradox" and past findings on the privacy decision-
making, the economics of digital privacy, etc. Overall, I think there
is a strong novel contribution here. The coverage of related work here
is quite reasonable. 
\hl{R2-1: While there is a body of work that might also be
relevant to discuss (one example below), I think the current amount of
coverage and specificity is already reasonable.}
\FIXED{Add the citation
\\SNS: Citation added in main.bib. Will add in the text as appropriate\\
JED: located in section 2, though it may move to 1.}

Goldfarb, A. and Que, V.F., 2023. The economics of digital privacy. Annual Review
of Economics, 15(1), pp.267-286.

Significance: I think this paper stands to have high potential impact. The topic
is generally relevant to a large portion of the HCI field, which
increasingly involves online platforms in some capacity (arguably the
growth of AI-related research in HCI will make this work, and follow
up work, even more important).
\FIXED{Consider stealing some language from the above for our conclusion\\
JED: located in conclusion}


Research quality: The presented research decisions are all reasonably well
justified. There are a few opportunities for improvement that may
impact how seriously readers take the study.
\hl{R2-2: * The Prolific-only sample is likely to skew results somewhat towards a more-
online population. I'm not sure the current results really reflect
"laypersons" as claimed, but nonetheless are still very important.
While not a major issue, highlighting the importance of follow up work
on a more general audience could be useful as a minor edit. This
weakness is mentioned (8.2) but I think could be taken more seriously
in future drafts.
}
\FIXED{Writing issue, weaken the "laypersons" claim, move up the limitation. Not really sure what it would mean to take this more seriously.
\\SNS: Seems more like demanding a distinction between experts vs others (laypersons)\\
JED: placed in the introduction, at first usage}
\hl{R2-3: * I was a little confused at first about the "What party does this case tend to
favor?" framing; this just didn't necessarily seem like an intuitive
way people might think about the Terms of Service for the platforms
discussed in motivating paragraphs. I did look at the supplement
materials (data) which was helpful, but still was left thinking the
paper could benefit from more explanation of this framing and what
this implies for user-platform interactions in general.
}
\FIXED{We need to hit this cleanly, in that we need to clarify what this means while still staying faithful to what participants saw\\
JED: located in Section 3}

* The understandability results were all very clear throughout.

Presentation clarity: In terms of writing quality and results tables, the paper
was clear throughout. \hl{R2-4: The results might benefit from visualization
rather than mean and standard deviations of ordinal responses.}
\FIXED{Add boxplots and maybe statistical testing
\\SNS: Sameer will be working on this\\
JED: located in Section 7, which we will rewrite fully}
\hl{R2-5: The
results section could benefit from a bit more signposting (while a
reader who has read the first 3 sections already has a pretty decent
picture of the survey itself in their mind and thus how the results
might be presented, a quick summary of Section 4 and 5's organization)
could be beneficial.}
\FIXED{Basically just follow the advice in the last part of this WHILE being wary of oversummarization\\
JED: located in all results sections}

In summary, this paper is strong throughout. 
\hl{R2-2 (again): My main concerns are that the sample
doesn't quite match some of the claims in the Introduction and
Discussion about speaking to layperson perceptions (acknowledged in
limitations)}
and some of the framing of these survey questions could
be confusing. That said, most of the results are explained well and
the implications are relatively clear.

I look forward to seeing future iterations of this paper.

----------------------------------------------------------------

\section*{reviewer review (reviewer 1)}

Expertise

Knowledgeable

Originality (Round 1)

Medium originality

Significance (Round 1)

Low significance

Research Quality (Round 1)

Low research quality

Contribution Compared to Length (Round 1)

The paper length was commensurate with its contribution.

Figure Descriptions



Recommendation (Round 1)

\textbf{I can go with either Reject or Revise and Resubmit}

Review (Round 1)

Summary:
This paper presents a user study of the comprehensibility of elements from terms
of service. The project is based on the "Terms of Service; Didn't
Read" (ToS;DR) project, which digests terms of service documents into
sets of standardized elements and assigns ratings from A to E based on
their content. The authors had  499 crowdworks each to review a random
set of 5 of the TOS;DR policy elements (termed "cases"), asking them
to provide likert scale ratings of understandability and the severity
of perceived favoritism (termed "severity") towards the service
provider or user. The authors use descriptive statistics to analyze
the understandability and severity of different policy elements.

Comments:
I thank the authors for their submission to CHI 2025. I appreciate the hard work
that clearly went into conducting this study, and recognize that it
presents an original idea. Nevertheless, I cannot recommend accepting
this paper in its current form due to a general lack of structure
which inhibits its clarity,  an unclear contribution, and the need for
better justification of the methods. Please see my detailed comments
on each below:

**General lack of structure**
I think it is necessary to discuss this issue first, as it is pervasive and feeds
into the other major issues I have. 
\hl{R1-1: Most sentences are well written,
but the paper lacks an overall structure which makes it difficult to
comprehend for the reader.}
\FIXED{best fixed via boldify/outline kind of exercises. Added TODO}
For example, the 
\hl{R1-2: introduction provides some
useful background information, but lacks an outline of the paper or
other signposting that is typical in an academic paper (e.g.,
description of methods, explanation of the results, list of sections
and their purpose, etc.)
}
\FIXED{Not going to list sections and their purpose, but we can do the rest of this. Located in introduction}
The research questions are abruptly
introduced with expected contributions, with no other details provided
about the study to come.

\hl{R1-3: Similarly, the related work begins to discuss background information without a
clear indication of why the authors are discussing this particular
research. For example, it is unclear to me why there is a social media
subsection. There are also no summarizing statements about how a
particular research area relates to the present study or why the
present study is unique from the other studies in this area (e.g.,
"Our study builds upon social media research by Y" or "No prior study
has done X"). As a result, the related work section reads like a
largely self-contained annotated bibliography.}
\FIXED{We can lightly position ourselves with respect to this work. I generally don't like justification for the chosen method appearing in the related work, but this reviewer seems to want it, so we can move in that direction (but I would recommend against pulling TOO MUCH of that type of content into related work).\\
Positioned in background}

This lack of scaffolding is evident beyond these two sections, but I will leave it
here for brevity's sake. These types of writing issues could be
addressed in a revision, although substantial revision is necessary.

**Lack of clear contribution**
\hl{R1-4: A more fundamental issue I have with the study is that I do not really understand
the motivation for conducting it or what contribution it provides. The
main takeaways seem to be that the cases from TOS;DR are 1) generally
considered comprehensible by users (with some exceptions) and 2)
favorable to services. Beyond, maybe, helping to improve TOS;DR, the
first finding seems to have limited utility to the broader HCI
community. Perhaps there is some conclusion to be drawn about what
elements of terms of service are particularly confusing, but most
users are not referring to ToS;DR (or really any website policy
document). It is therefore unclear how this insight is useful. This
second finding seems almost tautological, as terms of services largely
exist to curtail users' rights.}
\FIXED{They are correct in terms of 1+2, though I would add a third thing, which will come from the comparative stats.
The main fix here is that we need to clarify to them how folks can use 1+2. Tn particular, for \#1, one method to improve understanding of these docs could be discovering instances of these critical concepts (e.g., with classification) and since we know they are understandable. For \#2, this then lets us perform triage, since now we know which concepts user care more/less about.\\
Located in discussion for now (Fixes should appear in a mix of places)}

This issue may be solved by fixing the problems with writing. 
\hl{R1-5: In particular, any
future revision should clarify why it is important to investigate the
understandability of ToS;DR cases, beyond the fact that it is a
standard database.}
\FIXED{Clarify as described, in introduction}
\hl{R1-6: The authors should also make it clear what these
results mean and what implications they have for the issues raised in
the introduction, like the privacy paradox and unexpected elements in
privacy policies.
Adding a discussion section could be a good place
for the analysis of the broader implications.}
\FIXED{We can add a discussion section as requested}

**Insufficient  methods**
The final major issue I have with the study is that I believe the methods are not
well justified and, at times, insufficient to answer the research
questions as currently stated.

\hl{R1-7: First, I think there is an issue with sample size. 499 seems like a sufficient
sample size to draw macro conclusions (i.e, that the set of cases is
generally understandable) but the between case comparisons seem more
dubious.  It is, of course, impossible to show all participants each
of the 243 case definitions, but this means that only a small number
of participants saw each condition. 499/(243/5) = ~10 each. I am glad
the authors raise this issue in the he threats to validity section.
This design may be acceptable, but I think justification is necessary
(i.e., citation to relevant literature with similar design).}
\FIXED{ Located in methodology.
This one is a little thorny... Possible we just ignore, possible we expand that threats. I do not know of similar designs and would not really want to, e.g., triple the sample size so that we get CLT in between-case comparison (that doesn't seem necessary to me since we aren't drawing statistical conclusions of that kind, trying to light pathways for more targeted study of particular concepts)}

\hl{R1-8: The study also relies on interpreting descriptive statistics to draw its
conclusions. No statistical testing was performed to validate that the
differences between cases in understandability or severity were
significant. Statistical testing is not always necessary nor
appropriate, but the descriptive analysis seems, at points, highly
subjective and arbitrary. No defined procedure is explained in methods
nor is a reference cited to justify this approach.  A particular area
of concern for me is section 6. The authors initially find no
correlation between understandability and severity, but nevertheless
go onto crop the data in different ways to try to draw conclusions
(i.e., looking at the most and least understandable/severe alone).
Ultimately, I'm left with really no takeaways from this section.}
\FIXED{Located in the New RQ4 section\\
And here we get yelled at for the expected reason. Replace that section with a better one relying on comparative
\\SNS: Sameer will be working on improving the stats altogether.
\\
SSN: Will resume working on the stats after addressing some comments and come up with a concrete response.}

\hl{R1-8: (mostly duplicated) Research question 4 ("How did demographic traits affect understandability and
severity?") also seems to be unanswerable with the paper's descriptive
approach. It is valuable to contextualize the other results by
describing the characteristics of the participants, but no analysis is
discussed to determine whether the demographics are associated with
understandability or severity. The authors do not even hypothesize
about how the demographics may effect the results. This research
question is, therefore, left unaddressed.}
\FIXED{Same treatment. Very similar to previous point, but slightly more targeted
\\ 
SSN: New hypotheses have been added to tackle this comment.}

Ultimately, any revision of the paper should better justify the methodological
approach, especially with respect to the sampling approach and the
reliance on seemingly non-systematic descriptive statistics. The
procedure and \hl{R1-9: choices to crop the data in certain ways should be
justified.}
\FIXED{JED@SSN: I think this goes away as a knock on when swapping in comparative stats, but seems related to their criticism of section 6.1}

**Misc other issues**
What follows is a list of other issues that I noticed during my reading, which I
think should be addressed in any revision.

\hl{R1-10: 1: The research questions use terminology before it has been defined in the paper
like  "understandability data" or "severity data."}
\FIXED{define these terms, I guess? seems fairly obvious, but we can probably add clarity
\\SNS: Will clarify to ensure there is no confusion\\
located in introduction}
\hl{R1-11: In general, the
narrow phrasing of the research questions is not very useful.}
\FIXED{See if we can broaden., located with RQs}
\hl{R1-12: 2: Throughout related work, papers are summarized or quoted without reference to
what the research did or the context of the results. This makes it
difficult for a reader to evaluate the synthesis. E.g., "Slightly over
half of participants (51\%) felt that information about their general
searching activities,... impacts their feelings about tracking." [26]
=> What context is this quote from? What is the information referred
to about general searching activities? "According to Earp et al. [15],
people are more likely to read and understand privacy regulations when
they believe that their data is in grave danger." => How did they draw
that conclusion? What is the context?}
\FIXED{This is fairly typical, I am not sure what they really want here, since doing this consistently costs a LOT of words (and will make things more boring). Might ignore... still thinking. located in background}
\hl{R1-13: 3: Line 216: "To determine the set of concepts the participants would see, as well
as the text to represent each concept, we appeal to Terms of Service;
Didn't Read (ToSDR) [38]." At this point in the paper, ToSDR is not
well introduced. I am left with the questions: What is ToSDR? What
does it contain? Why is it a reasonable source for this purpose? What
are cases? Are they like parts of a policy?}
\FIXED{Answer these questions just in time in the intro}
\hl{R1-14: 4: Footnotes are not an appropriate citation (E.g., footnote 5 and 6). If linking
to a live website or service, a footnote makes sense.  See:
https://nam10.safelinks.protection.outlook.com/?url=https%3A%2F%2Fwww.acm.org%2Fpublications%2Fauthors%2Freference-formatting&data=05%7C02%7Cjxd6067%40psu.edu%7C139527e5081d41ab9ca808dcfdeec6ca%7C7cf48d453ddb4389a9c1c115526eb52e%7C0%7C0%7C638664449203068419%7CUnknown%7CTWFpbGZsb3d8eyJWIjoiMC4wLjAwMDAiLCJQIjoiV2luMzIiLCJBTiI6Ik1haWwiLCJXVCI6Mn0%3D%7C0%7C%7C%7C&sdata=Kq%2BI2YscHLnCoMT5nk7b41pJbvrxqPx7vvMtcsMWl%2Bo%3D&reserved=0
}
\FIXED{Fine, fix them, passing over this for now since I need to look at all our footnotes. TODO added}
\hl{R1-15: 5: Line 302: "However, there is no significant correlation (Pearson's correlation
t-test, t(2498) = 6.99, p = 3.45 e-12) between understandability and
severity." 1. There is no such thing as "Pearson's correlation
t-test,"  2. The p-value appears very small and significant under most
normal interpretations. Later in the paper, the opposite is stated:
"we see that higher understandability scores tend to result in a rise
in the mean severity score." (line 618). Which is true? Or are the
supposed to mean different things?}
\FIXED{JED@SSN: Can you assist with chasing this out?. TODO added}
\hl{R1-16: 7: Page 7: table is in the footer.}
\FIXED{doesnt seem possible, but we should check our layout for this when submitting revision. Audit added}
\hl{R1-17: 8: The question that is the title of 4.2 is not addressed in the subsection. Was
any analysis performed to actually tell if participant written
explanations are better?}
\FIXED{No, we need to set expectations differently here. Located in that section}
\hl{R1-18: 9: line 748: Why is the 500th participant included in demographic calculations if
their other data is not used? They are functionally not part of the
study.}
\FIXED{JED@SNS: I need help disentangling this. I recall we had trouble identifying WHICH responses didn't belong when faced with a few sums that didnt agree\\
TODO added}
\hl{R1-19: 10: The privacy paradox is discussed in the introduction, but never returned to.
Why is this relevant?}
\FIXED{Add a discussion, as requested}
\hl{R1-20: 11: There is no complete listing of the survey available to reviewers. This makes
it challenging to evaluate the methods. It is generally best practice
to include the survey instrument in the appendix.}
\FIXED{Provide the word document and redacted link. TODO added}
\hl{R1-21: 12: The entire subsection 6.1 is confusingly written. Its unclear which responses
are being discussed when. What do the numbers 110 and 35 reports
correspond to? I can infer somewhat, but it's not clear.}
\FIXED{I also didn't find these parts very compelling. It could be we chop (or condense) it to make room for a deeper comparative stats treatment in a roughly space-neutral way\\
JED@SNS, JED@SSN, what do you think?
\\ 
SSN: In order to clarify this, we can rephrase it better (for lack of a better term, dumb it down)
Basically start with what is meant by least understandable, and how many responses suggest the same. Then from there, since all 110 responses suggest that the case(s) were least understandable,  35 responses suggest that they were very severe (< -5). The explanation seems to make sense. Then, 7/110 responses had the highest severity rating. 
If needed, we can do the same thing for 6.1 - 6.4\\
Located in section 6/RQ3}
\hl{R1-22: 13: Gender subsection of section 7 suggests that only binary gender options were
presented. This is against best practice. See https://nam10.safelinks.protection.outlook.com/?url=https%3A%2F%2Fwww.morgan-%2F&data=05%7C02%7Cjxd6067%40psu.edu%7C139527e5081d41ab9ca808dcfdeec6ca%7C7cf48d453ddb4389a9c1c115526eb52e%7C0%7C0%7C638664449203088867%7CUnknown%7CTWFpbGZsb3d8eyJWIjoiMC4wLjAwMDAiLCJQIjoiV2luMzIiLCJBTiI6Ik1haWwiLCJXVCI6Mn0%3D%7C0%7C%7C%7C&sdata=xBp%2FMwXRYkf%2FQ2hpCVIqnSTVHoH7O51Oe7P05jN39JE%3D&reserved=0
klaus.com/gender-guidelines.html}
\FIXED{We complied with this, but did not state as much. Another reason to show the instrument\\ Located in methodology}
\hl{R1-23: 14: The questions adapted from GenderMag do not seem to me to be appropriately
called demographics. They seem more like privacy attitude/behavior
questions.}
\FIXED{Fine, whatever. Audit added}
\hl{R1-24: 15: section  8.2 on external validity should discuss the characteristics of
Prolific crowdworkers. See Tang et al. https://nam10.safelinks.protection.outlook.com/?url=https%3A%2F%2Fwww.usenix.org%2Fconferen&data=05%7C02%7Cjxd6067%40psu.edu%7C139527e5081d41ab9ca808dcfdeec6ca%7C7cf48d453ddb4389a9c1c115526eb52e%7C0%7C0%7C638664449203109173%7CUnknown%7CTWFpbGZsb3d8eyJWIjoiMC4wLjAwMDAiLCJQIjoiV2luMzIiLCJBTiI6Ik1haWwiLCJXVCI6Mn0%3D%7C0%7C%7C%7C&sdata=DD6uUBpBwdKjY8CV4vfJwnPV%2BKTO%2B9wZdqe7B9kdYQw%3D&reserved=0
ce/soups2022/presentation/tanghttps://nam10.safelinks.protection.outlook.com/?url=https%3A%2F%2Fwww.usenix.org%2Fconference%2Fsoups2&data=05%7C02%7Cjxd6067%40psu.edu%7C139527e5081d41ab9ca808dcfdeec6ca%7C7cf48d453ddb4389a9c1c115526eb52e%7C0%7C0%7C638664449203128203%7CUnknown%7CTWFpbGZsb3d8eyJWIjoiMC4wLjAwMDAiLCJQIjoiV2luMzIiLCJBTiI6Ik1haWwiLCJXVCI6Mn0%3D%7C0%7C%7C%7C&sdata=g1ut9Nd7jG3NgDohw2AxoxnTs6BngWDLmvHkihZ62Uo%3D&reserved=0
022/presentation/tang}
\FIXED{Add the citation, located in threats}
\hl{R1-25: 16: how was the sample size determined? Was power analysis performed?}
\FIXED{No, power analysis is a waste of time. If i recall, we wanted around 10 responses to each case, since that would give a reasonable estimate of mean+SD. We were not intending to draw comparative statistics across cases, which we could barebones at 15 responses to each case. However, that would increase the cost by 50\%\\
JED: located in section 3}
\hl{R1-26: 17: The title suggests that the authors found that the policies were confusing,
but they were largely understandable.}
\FIXED{Good point, we can sharpen the conclusion with this. Basically say that we found participants could understand the individual concepts, while prior work has consistently shown inability to understand whole policies. Therefore, extracting concepts via classification shows promise.\\
JED: located in conclusion}

----------------------------------------------------------------

\section*{reviewer review (reviewer 3)}

Expertise

Expert

Originality (Round 1)

Medium originality

Significance (Round 1)

Medium significance

Research Quality (Round 1)

Medium research quality

Contribution Compared to Length (Round 1)

The paper length was commensurate with its contribution.

Figure Descriptions

The figure descriptions are adequate and follow the accessibility guidelines.

Recommendation (Round 1)

\textbf{I recommend Revise and Resubmit}

Review (Round 1)

Thank you for your submission, I enjoyed reviewing your paper. You take up
important, timely research questions on the understandability of
privacy policies. I found your research questions to be original, and
your methods sound and rigorous. It is great, for example, to have
empirical work that shows which privacy concepts are "understandable"
and your tables and descriptions show a great amount of detail.

That being said, I do have a number of concerns (e.g. related works, discussion /
implications) and presentation clarity that in the current state
obstruct the impact of this work.

Overall, 
\hl{R3-1: I find that this paper could be written more clearly. It could use a few
rounds of heavy copy editing to ensure a smoother flow.}
\FIXED{Spend some time doing copy editing and such. Boldifies constructed retroactively might help too\\
JED: Audit added}

Next, I find that our 
\hl{R3-2: paper is not sufficiently motivated in the introduction}
\FIXED{Audit this, boldifies will probably help}
, nor
\hl{R3-3: sufficiently grounded adequately in prior work}
\FIXED{Add whatever citations they provide. If they don't give guidance, probably ignore}
. For example, you
present your research questions in the introduction without defining
the research area — 
\hl{R3-2 (again): what is "understandability data"? What are the
privacy concepts that ToSDR encompasses? Why is this your research
site?}
\FIXED{JED: located in introduction}

Related Works:

While your related works section does an adequate job of citing prior works - 
\hl{R3-4: I
find the organization unconvincing and the subsections scattered. It
would strengthen your paper to rewrite this section with a better
flow.
For example, you start this section with:

Websites profit off
people's trust by selling, sharing, or analyzing personal information
[39]. As an example of the effects
of these behaviors, "Slightly over half of participants (51\%) felt that
information about their general searching activities,...impacts their
feelings about tracking." [26]. 

This does not read well. Summarizing
the quote would be a better approach.
}
\FIXED{Not sure, would need to read it\\
JED: located in background}
\hl{R3-5: Further as some more general
writing tips, please use topic sentences and work on the transition
flows between paragraphs and new ideas.}
\FIXED{Audit created, improve as able}

Here are some more concrete suggestions towards a more organized structure for
your related works section:

\hl{R3-6: 1. Starting the section with an overview of a privacy theory or framework you are
using and how you define privacy policies. You should ground these in
related works in the HCI space.}
\FIXED{Not fully sure what we do here either, since I do not know much about privacy frameworks\\
JED: located in background}
\hl{2. A subsection that starts with outlining the problem spaces concerning privacy
policies (and then continuing with the other subsections you have).}
\FIXED{This seems somewhat reasonable, but I am not sure I can adequately articulate this problem space\\
JED: located in background}

You cite the contextual integrity framework, which is fantastic. Along these
lines, I would urge you to review this robust Zotero library for more
work: \url{https://nam10.safelinks.protection.outlook.com/?url=https%3A%2F%2Fwww.zotero.org%2Fgroups%2F2228317%2Fprivaci%2Flibrary&data=05%7C02%7Cjxd6067%40psu.edu%7C139527e5081d41ab9ca808dcfdeec6ca%7C7cf48d453ddb4389a9c1c115526eb52e%7C0%7C0%7C638664449203255851%7CUnknown%7CTWFpbGZsb3d8eyJWIjoiMC4wLjAwMDAiLCJQIjoiV2luMzIiLCJBTiI6Ik1haWwiLCJXVCI6Mn0%3D%7C0%7C%7C%7C&sdata=kuZvZm3fJMA%2FAtB4ogZvTzz3wauECQHegKaWaY0kpho%3D&reserved=0} but
especially \hl{R3-6 (cont'd: Shvartzshnaider et al., "Going against the (appropriate)
flow: a contextual integrity approach to privacy policy analysis".}
\FIXED{Check out these resources\\
JED: located in background}

Discussion / Implications

\hl{R3-7: Your findings and implications sections are quite muddled and it is very
challenging to follow. I would highly recommend combing through these
sections and trying to rewrite this for flow and readability.}
\FIXED{Same treatment as above\\
JED: located in al results sections}
\hl{R3-8: Importantly, throughout these sections, please cite prior work from
your related works section. Here, you should be articulating how prior
research informs (builds on, supports, or discounts, etc.) your
empirical contributions -- this is crucial for having a paper with
strong impact.}
\FIXED{Need to rely on Shikha to foreground these concepts\\
JED: located in all results sections}

I look forward to reviewing your edits. Good luck!

\FIXED{I have adjusted for the following copy edit}
Other specific edits:

Introduction:

Period missing at the end of second bullet point under "By answering these RQs, we
make the following contributions"

Methodology:

Introduction is needed for ToSDR - what is ToSDR? Why ToSDR? What is the object or
unit of analysis when scraping ToSDR?

Period missing at the end of the last sentence in section 3.1 (Demographic
information)

Findings / Implications / Discussion:

Period missing at the end of the last sentence in section 4.3

----------------------------------------------------------------

