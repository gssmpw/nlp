%%%%%%%%%%%%%%%%%%%%%%%%%%%%%%%%%%%%%%%%%%%%%%%%%%%%%%%%%%%%%%%%%%%
%%%%%%%%%%%%%%%%%%%%%%%%%%%%%%%%%%%%%%%%%%%%%%%%%%%%%%%%%%%%%%%%%%%
%%%%%%
%%%%%%     Details and proofs for the model checking section
%%%%%%
%%%%%%%%%%%%%%%%%%%%%%%%%%%%%%%%%%%%%%%%%%%%%%%%%%%%%%%%%%%%%%%%%%%
%%%%%%%%%%%%%%%%%%%%%%%%%%%%%%%%%%%%%%%%%%%%%%%%%%%%%%%%%%%%%%%%%%%

\section{Further details and proofs for Section \ref{sec:model-checking}}


\label{sec:app-mc}

\subsection{Transition system $\cT_a^R$}

As explained in Section \ref{sec:model-checking}, the treatment
of the incremental and public-history semantics requires the
switch from the transition systems $\cT_a$ over $P_a$ to transition systems
$\cT_a^R$ for some superset $R$ of $P_a$.
The behavior of $\cT_a^R$ is the same as $\cT_a$, except that $\cT_a^R$
additionally makes nondeterministic guesses for the truth values
of the atomic propositions in $R \setminus P_a$.

\begin{definition}
\label{def:T-a-R}
{\rm
Let  $a \in \Ag$ and $R \in 2^P$ such that $P_a \subseteq R$.
Then,
$\cT_a^R = (S_a^R,\to_a^R,\Init_a^R,R,L_a^{R})$ is defined as follows.
\begin{itemize}
\item
  The state space $S_a^R$  
consists of the pairs $(s,O)$
with $s \in S_a$ and $O \subseteq R$ such that
$L_a(s) = O \cap P_a$.  
\item
  The set of initial states is $\Init_a^R = \{ (\init_a, O) :
   L_a(\init_a) = O \cap P_a \}$.  
\item
 The transition relation is defined as follows.
  If $(s,O), (s',O') \in S_a^R$ then
$(s,O) \to_a^R (s',O')$ if and only if $s \to_a s'$.
\item
  The labeling function is given by
$L_a^{R}(s,O)= O$.
\end{itemize}
  }
\end{definition}

Obviously, we then have
$\Traces^P(\cT_a^{R},(s,O)) \ = \ 
  \bigl\{ f \in \Traces^P(\cT_a,s) : \first(f) \cap R =O \bigr\}
$
for each state $s\in S_a$ and each $O \subseteq R$ with
$O \cap P_a=L_a(s)$. 
%
Furthermore, for each history $h \in (2^P)^+$:
$\Reach(\cT_a^R,h) \ = \ 
  \bigl\{ (s,O) : s \in \Reach(\cT_a,h), O = \last(h) \cap R \bigr\}$.


%%%%%%%%%%%%%%%%%%%%%%%%%%%%%%%%%%%%%%%%%%%%%%%%%%%%%%%%%%%%%%%%%%%%%%%%%%%%
%%%%%
%%%%%    LTL model checking
%%%%%
%%%%%%%%%%%%%%%%%%%%%%%%%%%%%%%%%%%%%%%%%%%%%%%%%%%%%%%%%%%%%%%%%%%%%%%%%%%%

\subsection{LTL model checking subroutines}

\label{sec:LTL-MC}

For both the basis and the step of induction, we rely on (a mild variant of)
standard
LTL model checking techniques \cite{VarWolp-LICS86,VarWolp-InfComp94}
to evaluate the arguments $\varphi$ 
of standpoint subformulas $\StMod{a}{\varphi}$.
We state this in a uniform manner in Lemma \ref{lemma:MC-LTL-subformulas}
for arbitrary transition systems $\cT$ over a set of atomic propositions
$R$ and LTL formulas $\varphi$ over $\cP$ where $R$ is a subset of $\cP$.
%
Then, $(\cT,R,\cP)= (\cT_a^{R},R,P)$ in the basis of induction and
$(\cT,R,\cP) = (\cT_{\chi},P_{\chi},P\cup \{p_1,\ldots,p_k\})$ 
in the step of induction where the $p_i$'s are the fresh atomic
propositions that are introduced for the maximal standpoint
subformulas of $\varphi$.



\begin{lemma}
 \label{lemma:MC-LTL-subformulas}
Let $\cP$ be a set of atomic propositions and $R \subseteq \cP$.
Given a transition system $\cT = (S,\to,\Init,R,L)$ and
an LTL formula $\varphi$ over $\cP$,
the set 
\begin{center}
  $\Sat_{\cT}(\exists \varphi)
  \ = \ 
  \bigl\{ s \in S : 
           \exists f \in \Traces^{\cP}(\cT,s) \text{ s.t. } f\LTLmodels \varphi
  \bigr\}$
\end{center}
is computable in time exponential in the
length of $\varphi$ and polynomial in the size of $\cT$. 
Furthermore, the problem to decide whether
$\Init \cap \Sat_{\cT}(\exists \varphi) \not= \varnothing$ is in $\PSPACE$.
\end{lemma}



\begin{proof}
We apply standard techniques to 
build a nondeterministic B\"uchi automaton (NBA) 
$\cA = \cA_{\psi} = (Y,\delta_{\cA},\init_{\cA},F)$ 
for $\psi$ over the alphabet $2^{\cP}$. That is, the accepted language
$\cL(\cA)$ of $\cA$ are the infinite words $f \in 2^{\cP}$ such that $f \LTLmodels \psi$.
We now consider the product 
$\cT \bowtie \cA = (S',\to'\nolinebreak,\Init',Y,L')$ 
defined as follows:
\begin{itemize}
\item
  The state space $S'$ is $S \times Y$.
\item
  The transition relation ${\to}' \, \subseteq \, (S \times Y)^2$ is defined 
  as follows.
  If $(s,y), (s',y')\in S \times Y$ then
  $(s,y) \to' (s',y')$ iff 
        there exists $H \subseteq \cP$
         such that 
         $s \to s'$, $H \cap R=L(s)$ and $y'\in \delta_{\cA}(y,H)$.
\item
  The labels of the states are mostly irrelevant, we only 
  need to distinguish states
  $(s,y)$ where $y \in F$ from those where $y \notin F$.
  Formally, we can deal with the state space of $\cA$ as atomic propositions
  and
  the labeling function $L' : S' \to Y$ given by
  $L'(s,y) = \{y\}$.
\item
  The set of initial states is
  $\Init' = \bigcup_{s\in S} \Init_{\cT \bowtie \cA}(s)$
  where for $s \in S$,
  $
   \Init_{\cT\bowtie \cA}(s) \ = \ 
   \bigl\{ \ (s,y) \ : \ \exists H \subseteq \cP \text{ s.t. }
             H \cap R = L(s), \
             y \in \delta_{\cA}(\init_{\cA},H) \ 
   \bigr\}$.
\end{itemize}
We then apply standard graph algorithms (backward search from the SCCs)
to determine the set 
$\Sat_{\cT \bowtie \cA}(\exists \Box \Diamond F)$ 
consisting of all states
$(s,y)$ in $\Init'$
that have a path visiting
$S \times F$ infinitely often.
Then, $\Sat_{\cT}(\exists \varphi)$ is the set
of all states $s$ in $\cT$
with
$\Init_{\cT\bowtie \cA}(s) \cap \Sat_{\cT \bowtie \cA}(\exists \Box \Diamond F) \neq \varnothing$.
Thus, $\Sat_{\cT}(\exists \varphi)$ 
is computable with known techniques 
in time exponential in the length of $\varphi$ and polynomial in the
size of $\cT$.
The polynomial space bound for the decision problem 
can be obtained with the same techniques
as for standard automata-based
LTL model checking \cite{VarWolp-LICS86,VarWolp-InfComp94}.
\end{proof}




%%%%%%%%%%%%%%%%%%%%%%%%%%%%%%%%%%%%%%%%%%%%%%%%%%%%%%%%%%%%%%%%%%%%%%%%%%%
%%%%%
%%%%%    soundness of the default history-DFA
%%%%%
%%%%%%%%%%%%%%%%%%%%%%%%%%%%%%%%%%%%%%%%%%%%%%%%%%%%%%%%%%%%%%%%%%%%%%%%%%%

\subsection{Soundness of the default history-DFA}

\label{sec:soundness-history-DFA}

\begin{lemma}
 \label{lemma:DFA-construction}
 \label{lemma:properties-default-history-DFA}
Let $\cD = \pow(\cT,\Obsset,U)$ be as
in Definition \ref{def:default-history-DFA}.
Then: 
\begin{enumerate}
\item [(a)]
  For $h \in (2^P)^+$,
  $\delta_{\cD}(\init_{\cD},h)$ 
  equals the set of all states $s$ in $\cT$ 
  such that
  there exist an initial, finite path $\pi$ in $\cT$
  to $s$ and a word $h' \in (2^P)^+$ with
  $\trace(\pi)=\proj{h'}{R}$ and
  $\proj{h}{\Obsset} = \proj{h'}{\Obsset}$.

\item [(b)]
  $\cD$ is $\Obsset$-deterministic, i.e.,
  if $h_1,h_2 \in (2^P)^+$ with 
  $\proj{h_1}{\Obsset}=\proj{h_2}{\Obsset}$ then
  $\delta_{\cD}(\init_{\cD},h_1)=\delta_{\cD}(\init_{\cD},h_2)$.

\item [(c)]
  The language $\cL(\cD)$ of $\cD$
  equals the set of all words $h \in (2^P)^+$ such that 
  there exist $h'\in (2^P)^+$ 
  with $\proj{h}{\Obsset}=\proj{h'}{\Obsset}$ and
  $U \cap \Reach(\cT,h') \not= \varnothing$.
\end{enumerate}
\end{lemma}


\begin{proof}
Statement (a) can be shown
by induction on the length of $h$.
Statement (b) is a direct consequence of (a).

To prove statement (c),
let $\Hist_{\cT}(U)$ denote the set of all histories $h \in (2^P)^+$ such that 
  there exist $h'\in (2^P)^+$ 
  with $\proj{h}{\Obsset}=\proj{h'}{\Obsset}$ and
  $U \cap \Reach(\cT,h') \not= \varnothing$.
The task is to show $\Hist_{\cT}(U)=\cL(\cD)$.

Suppose first $h\in \Hist_{\cT}(U)$. Let $x = \delta_{\cD}(\init_{\cD},h)$.
We have to show that $x \in F_{\cD}$.
As $h \in \Hist_{\cT}(U)$ there exist $h'\in (2^P)^+$ 
with $\proj{h}{\Obsset}=\proj{h'}{\Obsset}$
and a state $s \in U \cap \Reach(\cT,h')$. 
%
By definition of $\Reach(\cT,h')$, there exists an initial, finite path
$\pi$ in $\cT$ such that $\last(\pi)=s$ and
$\trace(\pi)=\proj{h'}{R}$.
By statement (a), this implies $s\in x$. 
By definition of $F_{\cD}$ and because $s \in U$, we obtain $x\in F_{\cD}$.

Vice versa, suppose that $h\in \cL(\cD)$. Thus,
$\delta_{\cD}(\init_{\cD},h) \in F_{\cD}$.
By the definition of $F_{\cD}$, there is a state
$s \in \delta_{\cD}(\init_{\cD},h) \cap U$.
Statement (a) implies the existence of
an initial, finite path $\pi$ in $\cT$ 
and a history $h' \in (2^P)^+$ with  $\last(\pi)=s$,
$\trace(\pi)= \proj{h'}{R}$ and
$\proj{h}{\Obsset} = \proj{h'}{\Obsset}$.
But then $s \in \Reach(\cT,h')$.
This yields $h \in \Hist_{\cT}(U)$.
\end{proof}


%%%%%%%%%%%%%%%%%%%%%%%%%%%%%%%%%%%%%%%%%%%%%%%%%%%%%%%%%%%%%%%%%%%%%%



\begin{lemma}
\label{lemma:properties-T-chi}
 Let $\chi = \StMod{a}{\varphi}$, let
 $\cT_{\chi}$, $\cT_{\chi}'$ be as in the
 step of induction of the model checking procedure and let $R = R_{\chi}$. Then:
 \begin{enumerate}
 \item [(a)] 
    If $h, h' \in (2^{\cP})^*$ with $\proj{h}{P}=\proj{h'}{P}$ then
    $\Reach(\cT_{\chi},h)=\Reach(\cT_{\chi},h') = \Reach(\cT_{\chi}',\proj{h}{P})$.
 \item [(b)]
    If $h \in (2^P)^+$ then
    $\Reach(\cT_{\chi}',h)$ equals the set of all states $((s,O),x_1,\ldots,x_k)$
    in $\cT_{\chi}'$ (or $\cT_{\chi}$)
    such that 
    $s \in \Reach(\cT_a,h)$, $O =\last(h) \cap R$,
    and there exists $h'\in (2^P)^+$
    with $\obs_{\chi}(h)=\obs_{\chi}(h')$ and $x_i=\delta_i(\init_i,h')$
    for all $i\in \{1,\ldots ,k\}$.
 \item [(c)]
    If $z = ((s,O),x_1,\ldots,x_k)$ is a state in $\cT_{\chi}'$ then
     $\Traces(\cT_{\chi}',z) =  \Traces(\cT_a^R,(s,O))$.
 \end{enumerate}
\end{lemma}

\begin{proof}
  Part (a) is obvious. Part (b) can be shown by induction on $|h|$.
  Part (c) follows from the fact that the projection of each path $\pi_{\chi}$
  in $\cT_{\chi}'$ to the $\cT_a^P$-component is a path in $\cT_a^R$
  with the same trace.
  Vice versa, every path $\pi$ in $\cT_a^P$ can be lifted to a path 
  $\pi_{\chi}$ in
  $\cT_{\chi}'$ such that the projection of $\pi_{\chi}$ to the
  $\cT_a^R$-component equals $\pi$. 
\end{proof}



%%%%%%%%%%%%%%%%%%%%%%%%%%%%%%%%%%%%%%%%%%%%%%%%%%%%%%%%%%%%%%%%%%%%%%%%%



\begin{lemma}
 \label{lemma:soundness-default-history-DFA}
  Let $\chi = \StMod{a}{\varphi}$ be a subformula of $\phi$ and
  let $\cD_{\chi}$ 
  be the default history-DFA
  (Definition \ref{def:default-history-DFA}) constructed for $\chi$.
  That is,
  $\cD_{\chi}=\pow(\cT_a^P,\Obsset_{\chi},\Sat_{\cT_a^P}(\exists \varphi))$
  in the basis of induction and
  $\cD_{\chi}=\pow(\cT_{\chi}',\Obsset_{\chi},\Sat_{\cT_{\chi}}(\exists \varphi'))$
  in the step of induction. 
  Then,
  $\cD_{\chi}$ is a history-DFA for $\chi$, i.e.,
  $\Obsset_{\chi}$-deterministic and
  $\cL(\cD_{\chi}) \ = \ 
   \bigl\{ h \in (2^P)^+ : (*,h) \InnerModels{\chi} \chi \bigr\}$.
\end{lemma}

\begin{proof}
By part (b) of Lemma \ref{lemma:properties-default-history-DFA}, 
$\cD_{\chi}$ is $\Obsset_{\chi}$-deterministic.

We now prove by an induction on the nesting depth of standpoint modalities 
that for all standpoint subformulas $\chi$ of $\phi$ it holds that
\begin{center}
  $\cL(\cD_{\chi}) \ = \ 
   \bigl\{ h \in (2^P)^+ : (*,h) \InnerModels{\chi} \chi \bigr\}$
\end{center}
where $\cL(\cD_{\chi})$ is the accepted language of $\cD_{\chi}$.
In what follows, we simply write $\cD$ instead of $\cD_{\chi}$
and denote the components of $\cD$ by
$(\cX_{\cD},\delta_{\cD},\init_{\cD},F_{\cD})$.
Furthermore, we write $R$ instead of $R_{\chi}$.

{\it Basis of induction:}
We consider a subformula
 $\chi = \StMod{a}{\varphi}$ where $\varphi$
is an LTL formula over $P$. 

If $(*,h) \InnerModels{\chi} \chi$ 
then there exists a history $h'\in (2^P)^+$,
a state $s \in \Reach(\cT_a,h')$ 
and a trace $f'\in \Traces^P(\cT_a,s)$ such that
$\first(f')=\last(h')$,
$\obs_{\chi}(h)=\obs_{\chi}(h')$ and $f' \LTLmodels \varphi$.
Let $O=\first(f') \cap R$. Then,  $(s,O)$ is a state in $\cT_a^R$ with
$f'\in \Traces^P(\cT_a^R,(s,O))$ and therefore
$(s,O)\in \Sat_{\cT_a^R}(\exists \varphi)$.
Furthermore, as $O=\last(h') \cap R$ and $s \in \Reach(\cT_a,h')$ 
we have $(s,O)\in \Reach(\cT_a^R,h')$.
Statement (c) of Lemma \ref{lemma:properties-default-history-DFA}
yields $h \in \cL(\cD)$.

Suppose now that $h \in \cL(\cD)$. 
Then, $\delta_{\cD}(\init_{\cD},h) \in F_{\cD}$.
By definition of $F_{\cD}$, 
$\delta_{\cD}(\init_{\cD},h) \cap \Sat_{\cT_a^R}(\exists \varphi)$ is nonempty.
Pick a state 
$(s,O) \in \delta_{\cD}(\init_{\cD},h) \cap \Sat_{\cT_a^R}(\exists \varphi)$.
Then, there exists a word $f' \in \Traces^P(\cT_a^R,(s,O))$
with $f' \LTLmodels \varphi$ and a history $h' \in (2^P)^+$ such that
$(s,O)\in \Reach(\cT_a^R,h')$ and $\obs_{\chi}(h)=\obs_{\chi}(h')$.
But then, $\first(f') \cap R = \last(h') \cap R$ 
(as both agree with $L_a^R(s,O)= O$)
and $s \in \Reach(\cT_a,h')$.

Recall that $R = R_{\chi}=P_a \cup \Obsset_{\chi}$.
If $\first(f')$ and $\last(h')$ are different then we replace the last symbol
of $h'$ with $\first(h')$. 
In this way, we obtain a history $\tilde{h}$ such that
$(f',\tilde{h})$ is a future-history pair with
$\obs_{\chi}(\tilde{h})=\obs_{\chi}(h')=\obs_{\chi}(h)$,
$s\in \Reach(\cT_a,h)=\Reach(\cT_a,\tilde{h})$,
$f'\in \Traces^P(\cT_a^R,(s,O)) \subseteq \Traces^P(\cT_a,s)$ 
and $(f',\tilde{h})\InnerModels{\chi} \varphi$ (as $f' \LTLmodels \varphi$).
%
This yields $(*,h) \InnerModels{\chi} \chi$. 


{\it Step of induction:}
%
Let $\chi = \StMod{a}{\varphi}$ and let
$\chi_1,\ldots,\chi_k$ be the maximal standpoint subformulas of $\varphi$
and $\cD_i = (X_i,\delta_i,\init_i,F_i)$ the history-DFA for the $\chi_i$'s, which exist by induction hypothesis.
Thus:
\begin{center}
 $\cL(\cD_i)=\{h \in (2^P)^+ : (*,h) \InnerModels{\chi_i} \chi_i\}.$
\end{center}
%
Moreover, $\InnerModels{\varphi}$ equals
$\InnerModels{\chi_i}$ for $i=1,\ldots,k$.
Thus,
$\cL(\cD_i)=\{h \in (2^P)^+ : (*,h) \InnerModels{\varphi} \chi_i\}$.


Recall that $R_{\chi}= P_a \cup \Obsset_{\chi} \cup \Obsset_1 \cup \ldots \cup \Obsset_k$ where $\Obsset_i = \Obsset_{\chi_i}$, $i=1,\ldots,k$.
In what follows, we will write $R$ instead of $R_{\chi}$.


As $\cD_i$ is a history-DFA for $\chi_i$, $\cD_i$ is $\Obsset_i$-deterministic,
which means that $\proj{h_1}{\Obsset_i}=\proj{h_2}{\Obsset_i}$ implies 
$\delta_{i}(\init_i,h_1)=\delta_i(\init_i,h_2)$.
%
As $\Obsset_i$ is a subset of $R$, 
the $\cD_i$'s are $R$-deterministic. That is:
\begin{center}
  Whenever $h_1, h_2 \in (2^P)^+$ with $\proj{h_1}{R}=\proj{h_2}{R}$ then
  $\delta_{i}(\init_i,h_1)=\delta_i(\init_i,h_2)$ for $i=1,\ldots,k$.
\end{center}


For every subformula $\psi'$ of 
    $\varphi' = \varphi[\chi_1/p_1,\ldots,\chi_k/p_k]$,
    let $\psi'[p_1/\chi_1,\ldots,p_k/\chi_k]$ 
    be the corresponding subformula
    of $\varphi$. 
  We then have $\varphi'[p_1/\chi_1,\ldots,p_k/\chi_k] = \varphi$
     and $p_i[p_1/\chi_1,\ldots,p_k/\chi_k] = \chi_i$.
Note that $\varphi'$ and its subformulas
  are LTL formulas over $\cP_{\chi}=P\cup \{p_1,\ldots,p_k\}$. 
  In what follows, we simply write $\cP$ rather than $\cP_{\chi}$.
We first show the following statement:

{\it Claim 1:}
    For each subformula $\psi'$ of $\varphi'$, 
    each history $h' \in (2^P)^+$,
    state $z = ((s,O),x_1,\ldots,x_k)$ in $\cT_{\chi}$ 
    such that $x_j = \delta_j(\init_j,h')$ for all $j\in \{1,\ldots,k\}$
    and  $f \in \Traces^{\cP}(\cT_{\chi},z)$ 
    with $\first(f)\cap P =\last(h')$:
    \begin{center}
       $f \LTLmodels \psi'$ \ iff \
       $(\proj{f}{P},h') \InnerModels{\varphi} 
            \psi'[p_1/\chi_1,\ldots,p_k/\chi_k]$ 
    \end{center}
  where $\InnerModels{\varphi}$ and $\InnerModels{\chi}$ 
  agree for the step, public-history and
  pure observation-based semantics, while 
  $\InnerModels{\varphi}$ is $\decrmodels{Q \cap P_a}$ if $\InnerModels{\chi}$ 
  is $\decrmodels{Q}$ 
  and
  $\InnerModels{\varphi}$ is $\incrmodels{Q \cup P_a}$ if $\InnerModels{\chi}$ 
  is $\incrmodels{Q}$.

The proof of Claim 1
is  by structural induction on the syntactic structure
  of $\psi'$. 
  Let us concentrate here on the most interesting case where
  $\psi'=p_i$ for some $i\in \{1,\ldots,k\}$,
  in which case $\psi'[p_1/\chi_1,\ldots,p_k/\chi_k] = \chi_i$.
  Let $h'$, $z = (s,x_1,\ldots,x_k)$ and $f$ be as in Claim 1.
  Then:
  \begin{center}
    $f \LTLmodels p_i$ \ iff \ $p_i\in \first(f)$ \ iff \ $p_i \in L_{\chi}(z)$
    \ iff \ $x_i \in F_i$
  \end{center}
  As we assume here the soundness of the history-DFA $\cD_i$ for $\chi_i$
  (induction hypothsis), 
  we have:  
  \begin{center}
    $x_i \in F_i$
    \ iff \ $h'\in \cL(\cD_i)$ \ iff \ $(*,h') \InnerModels{\chi_i} \chi_i$
  \end{center} 
  Putting things together we obtain:
  \begin{center}
    $f \LTLmodels \psi'=p_i$ \ iff \ $x_i \in F_i$ \ iff \ 
    $(*,h') \InnerModels{\varphi} \chi_i = \psi'[p_1/\psi_1,\ldots,p_k/\psi_k]$
   \end{center}
As $(\proj{f}{P},h') \InnerModels{\varphi} \chi_i$ iff
$(*,h') \InnerModels{\varphi} \chi_i$, 
this yields the claim for the case $\psi'=p_i$
for some $i\in \{1,\ldots,k\}$.
The other case $\psi'=p\in P$ of the basis of induction 
and the step of induction are obvious.

{\it Claim 2:} For each history $h \in (2^P)^+$:
$(*,h)\InnerModels{\chi} \chi$ iff $h \in \cL(\cD_{\chi})$.


By part (a) of Lemma \ref{lemma:properties-default-history-DFA},
$\delta_{\cD}(\init_{\cD},h)$ is the union of the sets
$\Reach(\cT_{\chi}',h')$ where $h'$ ranges over all histories 
$h'\in (2^P)^+$ with $\obs_{\chi}(h)=\obs_{\chi}(h')$ (i.e.,
$\proj{h}{\Obsset_{\chi}}= \proj{h'}{\Obsset_{\chi}}$).


{\it First part of the proof of Claim 2.}
Suppose first that $(*,h)\InnerModels{\chi} \chi$.
Then, there exists a word $h'\in (2^P)^+$, a state $s\in \Reach(\cT_a,h')$
and a trace $f'\in \Traces^P(\cT_a,s)$ such that 
$\obs_{\chi}(h)=\obs_{\chi}(h')$,
$\first(f')=\last(h')$ 
and $(f',h') \InnerModels{\varphi} \varphi$.

Let $O = \first(f') \cap R$. Then, $O=\last(h') \cap R$, 
$(s,O)\in \Reach(\cT_a^R,h')$ and $f'\in \Traces^P(\cT_a^R, (s,O))$.
Furthermore, let
$x_i=\delta_i(\init_i,h')$ for $i=1,\ldots,k$.
Then, $z = ((s,O),x_1,\ldots,x_k)$ is a state in $\cT_{\chi}$ and 
$\cT_{\chi}' = \proj{\cT_{\chi}}{R}$
with $z \in \Reach(\cT_{\chi}',h')$.

As $\obs_{\chi}(h)=\obs_{\chi}(h')$,
part (a) of Lemma \ref{lemma:properties-default-history-DFA} 
yields $z \in \delta_{\cD}(\init_{\cD},h)$.

We pick a path $\pi \in \Paths(\cT_a^R,(s,O))$ 
with $\trace(\pi) = \proj{f'}{R}$.
There is a path $\pi_{\chi}$ in $\cT_{\chi}$ from $z$ where
the projection of $\pi$ to the $\cT_a^R$-components agrees with $\pi$.

Then, $\trace(\pi_{\chi})$ 
is an infinite string over the alphabet
$2^{P_{\chi}}$ (recall that $P_{\chi} = R \cup \{p_1,\ldots,p_k\}$) with 
$\proj{\trace(\pi_{\chi})}{R}=\trace(\pi)=\proj{f'}{R}$.

On the other hand, $f' \in (2^P)^{\omega}$.
As $R = P_{\chi}\cap P$ and $f'$ and $\trace(\pi_{\chi})$ agree elementwise
on the truth values of the propositions in $R$,
we can ``merge'' $f'$ and $\trace(\pi_{\chi})$ to obtain a word 
$f \in \Traces^{\cP}(\cT_{\chi},z)$ such that
$\proj{f}{P}=f'$ and $\proj{f}{P_{\chi}}=\trace(\pi_{\chi})$.

Claim 1 applied to $f$ and $\psi'=\varphi'$ yields:
\begin{center}
  $f \LTLmodels \varphi'$ \ iff \ 
  $(f',h') \InnerModels{\varphi} \varphi$
\end{center}
As $(f',h') \InnerModels{\varphi} \varphi$ (by assumption), we
obtain $f \LTLmodels \varphi'$. As $f \in \Traces(\cT_{\chi},z)$, we get
$z \in \Sat_{\cT_{\chi}}(\exists \varphi')$.
Thus, $z \in \delta_{\cD}(\init_{\cD},h) \cap \Sat_{\cT_{\chi}}(\exists \varphi')$.
This yields
$\delta_{\cD}(\init_{\cD},h) \in F_{\cD}$.
Therefore, $h \in \cL(\cD)$.


{\it Second part of the proof of Claim 2.}
Let us assume now that $h \in \cL(\cD)$.
Statement (c) of Lemma \ref{lemma:properties-default-history-DFA} 
yields that there exists a history $h' \in (2^P)^+$ with
$\obs_{\chi}(h)=\obs_{\chi}(h')$ and 
$\Sat_{\cT_{\chi}}(\exists \varphi') \cap \Reach(\cT_{\chi}',h') \not= \varnothing$.

Pick an element 
$z \in \Sat_{\cT_{\chi}}(\exists \varphi') \cap \Reach(\cT_{\chi}',h')$
and a trace $f \in \Traces^{\cP}(\cT_{\chi},z)$ 
with $f \LTLmodels \varphi'$. (As before, $\cP=P \cup \{p_1,\ldots,p_k\}$.)

Recall that $\cT_{\chi}$ is
a transition system over $P_{\chi} = R \cup \{p_1,\ldots,p_k\}$.
The transition system $\cT_{\chi}'=\proj{\cT_{\chi}}{P}$ 
agrees with $\cT_{\chi}$, the only difference being that the labelings
are restricted to $P$. That is, the labeling function $L_{\chi}'$
of $\cT_{\chi}'$ is given by $L_{\chi}'(z)=L_{\chi}(z)\cap P$.

State $z$ has the form $((s,O),x_1,\ldots,x_k)$ where $(s,O)$ is a
state in $\cT_a^{R}$, 
i.e., $s$ is a state of $\cT_a$ and $O \subseteq R$ with 
$L_a(s) = O \cap P_a$.
%
As $z \in \Reach(\cT_{\chi}',h')$ we have:
\begin{itemize}
\item
   $x_i = \delta_i(\init_i,h')$ for $i=1,\ldots,k$
\item
   $\last(h') \cap R = L_{\chi}'(z)=O$
\end{itemize}
Using the fact that $f \in \Traces^{\cP}(\cT_{\chi},z)$ we get
$\first(f) \cap P_{\chi} = L_{\chi}(z)$. 
%
As $R \subseteq P_{\chi}$ we obtain:
\begin{center}
  $\first(f)\cap R \ = \ L_{\chi}(z) \cap R \ = \ L_{\chi}'(z)  \ = \ O 
  \ = \ \last(h') \cap R$
\end{center}
For the case where $\first(f) \cap P \not= \last(h)$, 
we proceed as in the basis
of induction. 
That is, we consider the history $\tilde{h}$ that arises from $h$ by
replacing the last symbol in $h$ with $\first(f) \cap P$.
Then, $\proj{h'}{R}=\proj{\tilde{h}}{R}$. This implies:
\begin{itemize}
\item
  $\obs_{\chi}(h)= \obs_{\chi}(h') = \obs_{\chi}(\tilde{h})$ 
  as $\Obsset_{\chi}\subseteq R$
\item
  $x_i =  \delta_i(\init_i,h')  =  \delta_i(\init_i,\tilde{h})$,
  $i=1,\ldots,k$ as the $\cD_i$'s are $R$-deterministic 
\item
  $z \in \Reach(\cT_{\chi}',h')=\Reach(\cT_{\chi}',\tilde{h})$ 
  and therefore $s \in \Reach(\cT_a,\tilde{h})$ 
\end{itemize}
Moreover, $\last(\tilde{h})=\first(f) \cap P$ (by definition of $\tilde{h}$).
Thus, $(\proj{f}{P},\tilde{h})$ is a future-history pair and:
\begin{center}
  $\proj{f}{P} \in 
  \Traces^P(\cT_{\chi}',z) \ = \ \Traces^P(\cT_a^R,(s,O)) 
  \ \subseteq \ \Traces^P(\cT_a,s)$
\end{center}
By Claim 1 and as $f \LTLmodels \varphi'$, we obtain
$(\proj{f}{P}, \tilde{h}) \InnerModels{\varphi} \varphi$.
 
Putting things together, with $f'=\proj{f}{P}$ we have:
$\obs_{\chi}(h)=\obs_{\chi}(\tilde{h})$, 
$f' \in \Traces^P(\cT_a,s)$, $s \in \Reach(\cT_a,\tilde{h})$ 
and $(f', \tilde{h}) \InnerModels{\varphi} \varphi$.
%
But this shows that $(*,h) \InnerModels{\chi} \StMod{a}{\varphi}=\chi$.
\end{proof}

%%%%%%%%%%%%%%%%%%%%%%%%%%%%%%%%%%%%%%%%%%%%%%%%%%%%%%%%%%%%%%%%%%%%%




