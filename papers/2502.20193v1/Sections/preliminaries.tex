
%%%%%%%%%%%%%%%%%%%%%%%%%%%%%%%%%%%%%%%%%%%%%%%%%%%%%%%%%%%%%%%%%%%
%%%%%%%%%%%%%%%%%%%%%%%%%%%%%%%%%%%%%%%%%%%%%%%%%%%%%%%%%%%%%%%%%%%
%%%%
%%%%     Preliminaries: LTL, TS, etc
%%%%
%%%%%%%%%%%%%%%%%%%%%%%%%%%%%%%%%%%%%%%%%%%%%%%%%%%%%%%%%%%%%%%%%%%
%%%%%%%%%%%%%%%%%%%%%%%%%%%%%%%%%%%%%%%%%%%%%%%%%%%%%%%%%%%%%%%%%%%


\section{Preliminaries}

\label{sec:prelim}

Throughout the paper, we assume some familiarity with
linear temporal logic interpreted over transition systems
and automata-based model checking, see e.g.~\cite{CGKPV18,BK08}.

\paragraph*{Notations for strings.}
%
Given an alphabet $\Sigma$, 
we write $\Sigma^*$ for the set of finite strings over $\Sigma$,
$\Sigma^{\omega}$ for the set of infinite strings over $\Sigma$
and $\Sigma^{\infty}$ for $\Sigma^* \cup \Sigma^{\omega}$.
As usual, $\Sigma^+ = \Sigma^* \setminus \{\varepsilon\}$ where
$\varepsilon$ denotes the empty string.
Given a (in)finite string 
$\varsigma = H_0 \, H_1 \ldots H_n$ or
$\varsigma = H_0 \, H_1 \, \ldots$ over $\Sigma$,
let $\first(\varsigma)=H_0$. 
If $\varsigma = H_0 \ldots H_n$ is finite then 
$\last(\varsigma)=H_n$.
Given $i, j \in \Nat$ we write $\varsigma[i \ldots j]$ 
for the substring $H_i \ldots H_j$ if $i \leqslant j$
(and assuming $j \leqslant n$ if $\varsigma$ is a finite string of length $n$)
and $\varsigma[i \ldots j]=\varepsilon$ if $i > j$.
If $i=j$ then $\varsigma[i \ldots i] = \varsigma[i] = H_i$.
So, $\prefix{\varsigma}{j}$ denotes the prefix $H_0 \ldots H_j$. If
 $\varsigma$ is infinite then $\suffix{\varsigma}{j} = H_j \, H_{j+1} \, H_{j+2} \ldots$.


If $\Sigma = 2^P$ is the powerset of $P$ and $R \subseteq P$ then the projection function 
$\proj{}{R} : (2^P)^{\infty} \to (2^R)^{\infty}$ 
is obtained by applying the projection 
$2^P \to 2^R$, $H \mapsto H \cap R$, 
elementwise, i.e., if $\varsigma = H_0 \, H_1 \, H_2 \ldots$ then
$\proj{\varsigma}{R} = 
   (H_0 \cap R) \, (H_1 \cap R) \, (H_2 \cap R) \ldots$.

%%%%%%%%%%%%%%%%%%%

\paragraph*{Transition systems.}
%
A transition system is a tuple $\cT = (S,\to,\Init,R,L)$ where
$S$ is a finite state space, ${\to} \, \subseteq \, S \times S$ a total transition
relation (where totality means that every state $s$ has at least one outgoing transition $s \to s'$), $\Init \subseteq S$ the set of initial states, $R$ a finite set of atomic propositions and $L : S \to 2^R$ the labeling function.
If $\Init$ is a singeleton, say $\Init = \{\init\}$, we simply write
$\cT = (S,\to,\init,R,L)$.

A path in $\cT$ is a (in)finite string $\pi = s_0 \, s_1 \ldots s_n \in S^+$ or $\pi = s_0 \, s_1 \, s_2 \ldots \in S^{\omega}$ such that $s_i \to s_{i+1}$ for all $i$. $\pi$ is initial if $\first(\pi)\in \Init$. 
The trace of $\pi$ is
$\trace(\pi) = L(s_0) \, L(s_1) \ldots \in (2^R)^+ \cup (2^R)^{\omega}$.
%
If $s \in S$ then 
$\Paths(\cT,s)$ denotes the set of infinite paths in $\cT$ starting in $s$ and
$\Traces(\cT,s)= \{\trace(\pi) : \pi \in \Paths(\cT,s) \}$.
%
%
If $P$ is a superset of $R$ then
\begin{center}
  $\Traces^P(\cT,s) =
     \bigl\{  \rho \in  \bigl(2^{P}\bigr)^{\omega} : 
         \proj{\rho}{R} \in \Traces(\cT,s)  \bigr\}$.
\end{center}
Thus, $\Traces(\cT,s)\subseteq (2^R)^{\omega}$, while
$\Traces^P(\cT,s)\subseteq (2^P)^{\omega}$. 
%
Moreover,
$\Paths(\cT) = \bigcup_{s\in \Init} \Paths(\cT,s)$.
$\Traces(\cT)$ and $\Traces^P(\cT)$ have the analogous meaning.
%
If $h \in (2^P)^+$ then $\Reach(\cT,h)$ denotes the
set of states $s$ in $\cT$ that are reachable from $\Init$ via a path $\pi$ 
with $\trace(\pi)=\proj{h}{R}$. 



%%%%%%%%%%%

\paragraph*{Linear temporal logic (LTL).}

The syntax of LTL over $P$ is given by (where $p \in P$): 
\begin{center}
 $\begin{array}{lclcr}
  \varphi & ::= & 
  \true \ \ \big| \ \ p \ \ \big| \ \ \neg \varphi \ \ \big| \ \ \varphi_1 \wedge \varphi_2
              \ \ \big| \ \neXt \varphi \ \ \big| \ \ \varphi_1 \Until \varphi_2 
 \end{array}$
\end{center}
Other Boolean connectives are derived as usual, e.g.,
$\varphi_1 \vee \varphi_2 = \neg (\neg \varphi_1 \wedge \neg \varphi_2)$.
The modalities $\Diamond$ (eventually) and $\Box$ (always) are defined by
$\Diamond \varphi = \true \Until \varphi$ and
$\Box \varphi = \neg \Diamond \neg \varphi$.
%
The standard LTL semantics is formalized by a satisfaction relation
$\LTLmodels$ where formulas are interpreted over infinite traces 
(i.e., elements of $(2^P)^{\omega}$), see e.g. \cite{CGKPV18,BK08}.
%
An equivalent semantics can be provided
using a satisfaction relation $\models$
that interprets
formulas over trace-position pairs 
$(\rho,n) \in (2^P)^{\omega} \times \Nat$ such that
$(\rho,n) \models \varphi$ iff $\suffix{\rho}{n} \LTLmodels \varphi$.

We use here another equivalent formalisation of the semantics of LTL 
(and later its extension $\StpLTL$) that
interpretes formulas over \emph{future-history pairs} 
$(f,h) \in (2^P)^{\omega}\times (2^P)^+$ with $\last(h)=\first(f)$, see the upper part of Figure \ref{fig:semantics}
where $f[1\ldots 0]=\varepsilon$.
Then, $f \LTLmodels \varphi$ iff $(f, \first(f)) \models \varphi$.
%
\begin{figure*}[ht]
\begin{center}
  \begin{tabular}{l}
   \begin{tabular}{ll}
   \begin{tabular}{lcl}
     $(f,h) \models p$ & iff & $p \in f[0]$ 
     \\[1ex]

     $(f,h) \models \varphi_1 \wedge \varphi_2$ & iff &
     $(f,h) \models \varphi_1$ and $(f,h)\models \varphi_2$ 
     \\[1ex]
    \end{tabular}
    & \hspace*{0.5cm}
    \begin{tabular}{lcl}
     $(f,h) \models \neg \varphi$ & iff &
     $(f,h) \not\models \varphi$
     \\[1ex]

     $(f,h) \models \neXt \varphi$ & iff &
     $(\suffix{f}{1},h f[1]) \models \varphi$
     \\[1ex]
    \end{tabular}
   \end{tabular}
    \\
 
   \begin{tabular}{l}
    \begin{tabular}{lcl}
     $(f,h) \models \varphi_1 \Until \varphi_2$ & iff &
     there exists $\ell \in \Nat$ such that
     $(\suffix{f}{\ell}, h f[1\ldots \ell]) \models \varphi_2$
     and
     $(\suffix{f}{j}, h f[1\ldots j]) \models \varphi_1$
     for all $j < \ell$
     \\[1ex]

     $(f,h) \models \StMod{a}{\varphi}$ & iff &
     there exists $h' \in (2^{P})^+$, 
     $t\in \Reach(\cT_a,h')$ and
     $f'\in \Traces^P(\cT_a,t)$ such that
     $\last(h')=\first(f')$, \\
     & &
     $\obs_a(h)=\obs_a(h')$ and $(f',h') \models \varphi$ 
   \end{tabular}
   \end{tabular}
  \end{tabular}
\end{center}
\vspace{-10pt}
\caption{Satisfaction relation $\models$ for $\StpLTL$
    over future-history pairs $(f,h)\in (2^P)^{\omega}\times (2^P)^+$
    with $\last(h)=\first(f)$.
    \vspace{-10pt}}
\label{fig:semantics}
\end{figure*}
%
For interpreting LTL formulas, the history is irrelevant: $f \LTLmodels \varphi$ iff $(f, \first(f)) \models \varphi$
iff $(f,h) \models \varphi$ for some $h$ with
$\last(h)=\first(f)$
iff $(f,h) \models \varphi$ for all $h$ with
$\last(h)=\first(f)$.

If $\cT=(S,\to,\Init,R,L)$ is a transition system with $R \subseteq P$ 
and $\varphi$ an LTL formula over $P$ then
$\cT \LTLmodels \varphi $ iff $f \LTLmodels \varphi$ 
for each $f \in \Traces^P(\cT)$.
%
$\Sat_{\cT}(\exists \varphi)$ denotes the set of states $s\in S$ where
$\{ f \in \Traces^P(\cT,s) : f \LTLmodels \varphi\} \not= \varnothing$.

