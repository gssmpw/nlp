

%%%%%%%%%%%%%%%%%%%%%%%%%%%%%%%%%%%%%%%%%%%%%%%%%%%%%%%%%%%%%%%%%%%
%%%%%%
%%%%%%     Proofs for the complexity section
%%%%%%
%%%%%%%%%%%%%%%%%%%%%%%%%%%%%%%%%%%%%%%%%%%%%%%%%%%%%%%%%%%%%%%%%%%

\section{Proofs for Section \ref{sec:complexity}}

\label{sec:app-complexity}

%%%%%%%%%%%%%%%%%%%%%%%%%%%%%%%%%%%%%%%%%%%%%%%%%%%%%%%%%%%%%%%%%%%

\subsection{Structure of history-DFAs and time bounds}

We start by providing the proof for 
Lemma \ref{lemma:single-exp-run-time-step-public-decr},
which consists of four parts:
step semantics (Lemma \ref{lemma:step-semantics-history-DFA-single-exp}),
public-history semantics
(Lemma \ref{lemma:public-semantics-history-DFA-single-exp}),
decremental semantics
(Lemma \ref{lemma:decr-semantics-history-DFA-single-exp})
and, finally, the pure observation-based semantics under the
additional assumption that the $P_a$'s are pairwise disjoint
(Lemma \ref{lemma:pobs-disjoint-semantics-history-DFA-single-exp}).

For the step, public-history and decremental semantics
we will make use of Remark \ref{remark:exploit-obs-determinism}.
Recall from Remark \ref{T-a-R} that $R_{\chi}=P_a$ and 
$\cT_a^{R_{\chi}}=\cT_a$
for the step,
pure observation-based and decremental semantics.


If $\varphi$ is an $\StpLTL$ formula, let $\Ag(\varphi)$
denote the set of agents $b\in \Ag$ such that $\varphi$ has a subformula
of the form $\StMod{b}{\psi}$.

We start with the step semantics.
Here, we have $\Obsset_{\chi} = \varnothing$ for all $\chi$.
With the notations used in Remark \ref{remark:exploit-obs-determinism},
the history-DFA $\cD_1,\ldots,\cD_k$ are
$\Obsset_{\chi}$-deterministic. Thus, we can think of
$\cD_{\chi}$ to be defined as a product 
$\pow(\cT_a,\varnothing) \bowtie \cD_1 \bowtie \ldots \bowtie \cD_k$.
But now, each of the $\cD_i$'s also has this shape.
In particular, if \( \chi_i = \StMod{a}{\psi_i} \), then the projection of the first coordinate of the states and the transitions between them in the reachable fragment of \( \cD_i \) matches exactly those of \( \pow(\cT_a, \varnothing) \).
Thus, one can incorporate the information on final states
 and drop $\pow(\cT_a,\varnothing)$ from the product.
The same applies to the history-DFAs for nested subformulas 
$\StMod{a}{\psi'}$ of the $\chi_i$'s.
As a consequence, the default history-DFA $\cD_{\chi}$ 
can be redefined for the step
semantics such that
the state space of $\cD_{\chi}$ is contained in
$\prod_{b\in \Ag(\chi)} 2^{S_b}$.
In particular, 
$\cD_{\chi}$ has the shape $\prod_{a\in \Ag(\chi)} \pow(\cT_a,\varnothing)^{k_a}$
for some $k_a > 0$, and one can merge redundant components if $k_a \geqslant 2$.

\begin{lemma}
\label{lemma:step-semantics-history-DFA-single-exp}
  Under the step semantics, for each standpoint formula
  $\chi$ there is a history-DFA where, besides an additional initial state, the state space is contained in
  $\prod_{a\in \Ag(\chi)} 2^{S_a}$.
\end{lemma}

\begin{proof}
For $a \in \Ag$, $H \subseteq P$ and $s \in S_a$, let 
$\delta_a(s,H)$ denote the set of states $s' \in S_a$ such that
$s \to_a s'$ and $L_a(s')=H \cap P_a$.
For $U \subseteq S_a$, $\delta_a(U,H)=\bigcup_{s \in U} \delta_a(s,H)$.
This definition can be extended inductively to histories by
$\delta_a(U,\varnothing)= U$, 
$\delta_a(U,Hh) = \delta_a( \delta_a(U, H), h)$ for $H \in 2^P$ and
$h \in (2^P)^+$.

Let $\Delta_a : 2^{S_a} \to 2^{S_a}$ be given by
$\Delta_a(U)=\bigcup_{H \subset P} \delta_a(U,H)$. That is, $\Delta_a(U)$
is the set of states $s' \in S_a$ such that $s \to_a s'$
for some state $s \in U$.

For $A \subseteq \Ag$, 
let $D_A = \prod_{a \in A} 2^{S_a}$. 
We define
$\Delta_A : D_A \to D_A$ as follows:
$\Delta_A( (U_a)_{a\in A}) = (\Delta_a(U_a))_{a \in A}$.

Furthermore, let $\delta_A^{\tinystep} : D_A \times 2^P \to D_A$ be given by
$$
  \delta_A^{\tinystep}( (U_a)_{a\in A}, H) \ = \ 
  \Delta_A( (U_a)_{a\in A}) 
$$
for each $H \subseteq P$.
Thus, $\delta_A^{\tinystep}( (U_a)_{a\in A}, H) = 
       \delta_A^{\tinystep}( (U_a)_{a\in A}, H')$
for all $H, H' \subseteq P$.

We now show that for each standpoint formula $\chi = \StMod{a}{\varphi}$
there exists a history-DFA $\cD$ of the form 
\begin{center}
  $\cD = (D_A \cup \{\init_{\cD}\},\delta_A^{\tinystep},\init_{\cD},F_{\chi})$ 
\end{center}
for some $F_{\chi} \subseteq D_A$ where $A = \Ag(\chi)$ and
$\delta_A^{\tinystep}(\init_{\cD},H)= (\{\init_a\})_{a\in A}$ 
for each $H \subseteq P$.  Note that the structure $(D_A \cup \{\init_{\cD}\},\delta^{\tinystep}_A, init_{\cD})$ equals $\prod_{a\in \Ag(\chi)} \pow(\cT_a,\varnothing)$.

Recall from Remark \ref{T-a-R} that $R_{\chi}=P_a$ and that
we may deal with $\cT_a$ rather than $\cT_a^{R_{\chi}}$ in the basis
of induction and in the induction step.

The proof is by induction on the nesting depth of standpoint modalities.
The claim is obvious if $\chi = \StMod{a}{\varphi}$ 
contains no nested standpoint modalities 
(in which case $\varphi$ is an LTL formula)
as then $\Ag(\chi)=\{a\}$ and the default history-DFA
arises by the powerset construction applied to $\cT_a$ and 
$\Obsset = \varnothing$. And indeed, 
$\cD = \pow(\cT_a,\varnothing,U)$
(defined as in Definition \ref{def:default-history-DFA})
is $(D_{\{a\}}\cup \{init_{\{\cD\}}\},\delta_{\{a\}}^{\tinystep}, init_{\{\cD\}}, F)$ for some 
$F \subseteq D_{\{a\}}$.

Suppose now that $\varphi$ has $k$ maximal standpoint subformulas, say
$\chi_1 = \StMod{b_1}{\psi_1},\ldots,\chi_k = \StMod{b_k}{\psi_k}$.
Let $B_i = \Ag(\chi_i)$. 
Then, $\Ag(\varphi) = B_1 \cup \ldots \cup B_k$
and $\Ag(\chi)=\{a\} \cup \Ag(\varphi)$.
%
By induction hypothesis,
there are history-DFA $\cD_1,\ldots,\cD_k$ 
for $\chi_1,\ldots,\chi_k$ where 
$\cD_i = (D_{B_i}\cup \{\init_{B_i}\},\delta_{B_i}^{\tinystep},\init_{B_i},F_i)$.
We now revisit the definition of the transition system $\cT_{\chi}$.
In Section \ref{sec:model-checking}, we defined the state space $Z_{\chi}$ 
as the set of tuples $(s,x_1,\ldots,x_k)$ 
where $s \in S_a$ and $x_i$ is a state
in $\cD_i$.
Thus, 
the reachable states in $\cT_{\chi}$ have the
form $(s,(U_b)_{b \in B_1},\ldots, (U_b)_{b\in B_k})$ 
and contain redundant components
if there are agents $b\in \Ag$ that belong to multiple $B_i$'s.
That is, we can redefine the state space of $\cT_{\chi}$ to be
$S_a\times \prod_{b \in \Ag(\varphi)} 2^{S_b}$, so $\cT_{\chi}$ can be seen as the product 
$\cT_a \bowtie \cD_{\Ag(\varphi)}$
where $\cD_A$ is the structure $(D_A \cup \{\init_A\},\delta_A^{\tinystep},\init_A)$.
%
As every state in $\cD_{\Ag(\varphi)}$ has a single successor that is
reached for every input $H \in 2^P$,
the reachable states in the default history-DFA 
$\cD_{\chi}=
 \pow(\cT_{\chi},\varnothing,P_{\chi},\Sat_{\cT_{\chi}}(\exists \varphi'))$ 
defined as in
Definition \ref{def:default-history-DFA} 
have the form $(U,\{ V \})$ where $U \subseteq S_a$ and 
$V \in \prod_{b\in \Ag(\varphi)} 2^{S_b}$.
Thus, $V$ is a tuple $(V_b)_{b\in \Ag(\varphi)}$
where $V_b \subseteq S_b$.
Moreover, if $a \in \Ag(\varphi)$ then $U = V_a$.
Thus, we can redefine the state space of $\cD_{\chi}$ to be
$D_{\Ag(\chi)}$.
In this way, we obtain a history-DFA for $\chi$ 
that has the form 
$(D_{Ag(\chi)} \cup \{init_{Ag(\chi)}\},\delta_{\Ag( \chi)}^{\tinystep},\init_{\Ag(\chi)},F_{\chi})$.
\end{proof}

%%%%%%%%%%%%%%%%%%%%


The situation is similar for the public-history semantics.
Here, $\Obsset_{\chi} = P$ for all $\chi$. Thus, all history-DFA
are $P$-deterministic. With arguments as for the step semantics, 
the history-DFA $\cD_{\chi}$ can be defined in such a way that
the state space of $\cD_{\chi}$ has the shape 
$\prod_{b\in \Ag(\chi)} \pow(\cT_a^P,P)$.
That is, the non-initial states of $\cD_{\chi}$ are elements
of $\prod_{b\in \Ag(\chi)} (2^{S_b}\times 2^P)$. 
%
Moreover, the reachable fragment
contains only states of the form $(T_b,O_b)_{b\in \Ag(\chi)}$
with $T_b \subseteq S_b$ and $O_b \subseteq P$
where $O_b = O_a$ for all $a,b \in \Ag(\chi)$.
Thus, the state space of $\cD_{\chi}$ can even be reduced to
$(\prod_{b\in \Ag(\chi)} 2^{S_b}) \times 2^P$.
The details are shown in the proof of the following lemma:


\begin{lemma}
\label{lemma:public-semantics-history-DFA-single-exp}
  Under the public-history semantics, for each standpoint formula
  $\chi$ there is a history-DFA where, besides an additional initial state, the state space is contained in
  $(\prod_{a\in \Ag(\chi)} 2^{S_a}) \times 2^P$.
\end{lemma}

\begin{proof}
  The arguments are fairly similar to 
the proof of Lemma \ref{lemma:step-semantics-history-DFA-single-exp}.

For $A \subseteq \Ag$, $a \in \Ag$, let $D_A$ and $\delta_a$ be defined
as in the proof of Lemma \ref{lemma:step-semantics-history-DFA-single-exp}. 
We define a transition function
 $\delta_A : D_A \times 2^P \times 2^P \to D_A \times 2^P$  as follows.
For $H \subseteq P$, $T_a \subseteq S_a$ and $O \subseteq P$, let
\begin{center}
  $\delta_a^{\tinypublic}\bigl( \bigl((T_a)_{a\in A},O\bigr), H \bigr) \ = \ 
      \bigl( (\delta_a(T_a,H))_{a\in A}, H \bigr)$
\end{center}
Furthermore, let 
$I_a^H=\{\init_a\}$ if $L_a(\init_a)= H \cap P_a$ and
$I_a^H=\varnothing$ otherwise.
%
We now show that for each standpoint formula $\chi = \StMod{a}{\varphi}$
there exists a history-DFA $\cD$ of the form 
\begin{center}
  $\cD = 
  (D_A \times 2^P \cup \{\init_{\cD}\},\delta_A^{\tinypublic},\init_{\cD},F_{\chi})$ 
\end{center}
for some $F_{\chi} \subseteq D_A \times 2^P$ where $A = \Ag(\chi)$ and
$\delta_A^{\tinypublic}(\init_{\cD},H)= \bigl((I_a^H)_{a\in A}, H \bigr)$ 
for each $H \subseteq P$.

  Again, we use an induction on the nesting depth of standpoint modalities. The case when $\chi = \StMod{a}{\varphi}$, where $\varphi$ is an LTL formula is obvious. The default history-DFA arises by the powerset construction applied to $\cT_a^P$ and $\Obsset_{\chi} = P$. Recall the corresponding definition of the transition function of the default history-DFA $\cD$
\begin{center}  
    \begin{tabular}{r}
  	$\delta_{\cD}(x,H)$  =          
  	$\bigr\{ s' \in S : \text{ there exists } s \in x \text{ with }$
  	\ \ \\
  	
  	$s \to s'$ \text{ and } $L(s')  = H  \bigr\}.$
  	\\[1ex]
  \end{tabular}   
\end{center}
  Then the reachable fragment of $\cD$ contains only states which are sets of tuples $(s, O)$ s.t. if $(s_1, O_1), (s_2, O_2) \in x $, where $x$ is a state in the default history-DFA $\cD$,
then $O_1 = O_2$. 
 
  
  Suppose now that $\varphi$ has $k$ maximal standpoint subformulas, say
  $\chi_1 = \StMod{b_1}{\psi_1},\ldots,\chi_k = \StMod{b_k}{\psi_k}$. 
  Let $B_i=\Ag(\chi_i)$. Then, $\Ag(\varphi) = B_1 \cup \ldots \cup B_k$
  and $\Ag(\chi)=\{a\} \cup \Ag(\varphi)$. 
%
By induction hypothesis, there are history-DFAs $\cD_i$ for $\chi_i$ such that $\cD_i$ has the shape
$(D_{B_i}\times 2^P \cup\{\init_i\}, \delta_{B_i}^{\tinypublic},\init_i,F_i)$.
%
Let us revisit the definition of the transition system $\cT_{\chi}$.
  In Section \ref{sec:model-checking}, we defined the state space $Z_{\chi}$ 
  as the set of tuples $((s, O), x_1,\ldots,x_k)$ 
  where $(s, O) \in S_a^P$ and $x_i$ is a state
  in $\cD_i$.
  Thus, 
  the reachable states in $\cT_{\chi}$ have the
  form $((s,O),(T_b, O)_{b \in B_1},\ldots, (T_b, O)_{b\in B_k})$ 
  and contain redundant components
  if there are agents $b\in \Ag$ that belong to multiple $B_i$'s.
  These components as well as the duplicated $O$-components can be merged.
  But then, the reachable states in the default history-DFA 
  $\cD_{\chi}=
  \pow(\cT_{\chi},P,\Sat_{\cT_{\chi}}(\exists \varphi'))$ 
  defined as in
  Definition \ref{def:default-history-DFA} 
  have the form $(U,\{ V \})$ where $U = (U_a, R) \in 2^{S_a}\times 2^P$ and 
  $V = (V_b, R)_{b\in  \Ag(\varphi)}$ with $V_b \subseteq S_b$.
  Moreover, if $a \in \Ag(\varphi)$ then $U_a = V_a$. 
  In this case, the redundant component can be dropped again.
\end{proof} 

%%%%%%%%%%%%%%%%%%%%%%%%%%%%%%%%%%%%%%%%%%%%%%%%%%%%%%%%%%%%%%%%%%%%%%%

We now turn to the decremental semantics. 
If $\chi =\StMod{a}{\varphi}$ then
$\Obsset_{\chi_i} \subseteq \Obsset_{\chi}$ 
for all maximal standpoint
subformulas $\chi_i$ of $\varphi$, and hence their history-DFAs $\cD_i$
are $\Obsset_{\chi}$-deterministic.
%
In contrast to the step semantics, we cannot guarantee that
  the state space of the generated history-DFA $\cD_{\chi}$ 
  is contained in $\prod_{a\in \Ag(\chi)} 2^{S_a}$. 
  For example, 
  the reachable states of the history-DFA 
  for $\chi=\StMod{a}{ \neXt \StMod{b}{ \neXt \StMod{a}{p} }}$
  under the decremental semantics 
  consist of three components $(T_1,T_2,T_3)$
  where $T_1,T_3 \subseteq S_a$ and $T_2 \subseteq S_b$. 
  Under the decremental semantics, we cannot merge
  the first and the third component as they rely on different observation
  sets ($P_a$ for the first component and $P_a \cap P_b$ 
  for the third component). Thus, $T_1$ and $T_3$ can be different.
However,
 the inductive model checking approach of Section \ref{sec:model-checking}
corresponds to a bottom-up processing of the syntax tree of $v$.
One can now define history-DFAs $\cD_v$ for the nodes in the syntax
tree that have the shape
$\prod_{w} \pow(\cT_{a_w},\Obsset^{\tinydecr}_w)$
where $w$ ranges 
over all nodes in the syntax subtree of the node representing
$\chi$ such that the formula given by $w$ is a standpoint formula
$\StMod{a_w}{\varphi_w}$.


Let, as before, $\phi$ be the $\StpLTL$ formula to be model checked. For each node $v$ in the syntax tree of $\phi$, we write $\phi_v$ for the subformula represented by $v$ and $\Tree(v)$ for the subtree of node $v$. Thus, $\phi  = \phi_r$ for the root node $r$ of the syntax tree. 
All nodes $v$ where $\phi_v$ is a standpoint formula are called standpoint nodes.  $\StpNodes_v$ denotes the set of standpoint nodes in $\Tree(v)$ excluding $v$.
Thus, if $v$ is a standpoint node then 
$\{\phi_w : w \in \StpNodes(v)\}$ 
is the set of all proper standpoint subformulas of $\phi_v$ 
(excluding $\phi_v)$.
The sets $Q^{\tinydecr}_v$ are defined as in Remark \ref{remark:context-dependency-syntax-tree}. Recall that $\Obsset^{\tinydecr}_v=Q^{\tinydecr}_v \cap P_a$ if $v$ is a standpoint node and $\phi_v =\StMod{a}{\varphi_v}$.

Recall that $\pow(\cT,\Obsset)$ denotes the structure defined 
as in Definition \ref{def:default-history-DFA} 
without the declaration of final states. 


\begin{lemma}
\label{lemma:decr-semantics-history-DFA-single-exp}
  Under the decremental semantics, for each standpoint node $v$
  there is a history-DFA $\cD_v$ of the shape 
  $\prod_{w \in \StpNodes_v} \pow(\cT_{a_w},\Obsset^{\tinydecr}_w)$.
\end{lemma}

\begin{proof}
  The proof is by induction on $|\StpNodes_v|$.

  The basis of induction is obvious as $|\StpNodes_v|=0$ implies that
  $\phi_v$ has no proper standpoint subformulas.
  That is, $\varphi_v$ is an LTL formula and we can simply
  use the default history-DFA $\cD_v$ defined as in
  Definition \ref{def:default-history-DFA}. 
  
  Assume the induction hypothesis holds for the maximal standpoint subformulas \( w_i \) for \( i \in \{1, \dots, k\} \), where the associated history-DFA are given by
  \[
  \cD_{w_i} = \prod_{w \in \StpNodes_{w_i}} \pow(\cT_{a_w}, \Obsset^{\tinydecr}_w).
  \]
  
  Now, consider the history-DFA \(\cD_\nu\), where $\phi_v =\StMod{a}{\varphi_v}$. Since \( \Obsset^{\tinydecr}_{w_i} \subseteq \Obsset^{\tinydecr}_\nu \) for \( i \in \{1, \dots, k\} \), it follows that all \(\cD_{w_i}\) are \(\Obsset^{\tinydecr}_\nu\)-deterministic. So,  $ \pow(\cT_a \bowtie \cD_{w_{1}} \bowtie \ldots  \bowtie \cD_{w_{k}}, \Obsset^{\tinydecr}_\nu) = \pow(\cT_a , \Obsset^{\tinydecr}_v) \bowtie \cD_{w_{1}} \bowtie \ldots  \bowtie \cD_{w_{k}}$, which completes the proof. 
\end{proof}

	The product 
	\[
	\prod_{w \in \StpNodes_v} \pow(\cT_{a_w}, \Obsset^{\tinydecr}_w)
	\] 
	can be simplified by merging redundant components. Specifically, if \(w, r \in \StpNodes_v\) with 
	\((a_w, \Obsset^{\tinydecr}_w) = (a_r, \Obsset^{\tinydecr}_r)\), then the corresponding terms can be merged.

We now consider the pure observation-based semantics under the additional
assumption that $P_a \cap P_b =\varnothing$ for $a,b\in \Ag$ with $a \not= b$.
Then, for each family $(H_a)_{a\in \Ag}$ where $H_a \subseteq P_a$ there exists
$H \subseteq P$ with $H_a = H \cap P_a$ for all $a \in \Ag$.
%
Suppose $\chi=\StMod{a}{\varphi}$ is a standpoint subformula
of $\phi$ as in the step of induction and regard the definition of the
transition system $\cT_{\chi}$ which was defined as a product 
$\cT_{\chi} = \cT_a \bowtie \cD_1 \bowtie \ldots \bowtie \cD_k$
where the $\cD_i$'s are history-DFA for the maximal standpoint subformulas
$\chi_i=\StMod{b_i}{\psi_i}$ of $\varphi$. 
(Recall from Remark \ref{T-a-R} that we assume $\cT_a = \cT_a^R$.)
%
As the $\cD_i$'s are $P_{b_i}$-deterministic, we may think of the $\cD_i$
to be DFA over the alphabet $2^{P_{b_i}}$. 
%
We can now rephrase the transition relation $\to_{\chi}$ of $\cT_{\chi}$ 
as follows. 
Let $B = \{ b_i : i=1,\ldots,k\}$ and
for $b\in \Ag$, $I_b = \{i \in \{1,\ldots,k\} : b_i = b\}$.
Then, there is a transition  
 $(s,x_1,\ldots,x_k) \to_{\chi} (s',x_1',\ldots,x_k')$ in $\cT_{\chi}$
if and only if $s \to_a s'$ in $\cT_a$ and 
$x_i'=\delta_i(x_i,L_a(s'))$ for $i \in I_a$ and
for each $b \in B \setminus \{a\}$ there exists $\tilde{H}_b \subseteq P_b$
such that 
$x_i'= \delta_i(x_i,\tilde{H}_b)$ for $i \in I_b$.
Thus, the local transitions of $\cD_i$ for $i \notin I_a$ do not depend on
the local state $s$ of $\cT_a$. Furthermore, there is only synchronisation
between DFAs $\cD_i, \cD_j$ where $b_i=b_j$.   
%
This observation can be used to show by induction on the nesting depth
of standpoint formulas $\chi=\StMod{a}{\varphi}$ 
that the definition of default history-DFAs $\cD_{\chi}$ can be modified
such that the state space is contained in 
$2^{S_a} \times \prod_{b \in \Ag(\varphi)} 2^{S_b}$.


\begin{lemma}
 \label{lemma:pobs-disjoint-semantics-history-DFA-single-exp}
  Suppose that the $P_a$'s are pairwise disjoint.
  Under the pure observation-based semantics, for each standpoint formula
  $\chi = \StMod{a}{\varphi}$ there is a
  history-DFA $\cD_{\chi}$ where the state space is contained in
  $2^{S_a} \times \prod_{b \in \Ag(\varphi)} 2^{S_b}$ along with an additional initial state.
 
\end{lemma}



\begin{proof}
For $a \in \Ag$ and $B \subseteq \Ag$, 
let $\delta_a : 2^{S_a} \times (2^P)^* \to 2^{S_a}$,
$\Delta_a : 2^{S_a} \to 2^{S_a}$ and $\Delta_B : D_B \to D_B$ 
(where $D_B = \prod_{b\in B} 2^{S_b}$) and
$\delta_B^{\tinystep} : D_B \times 2^P \to D_B$
be defined as in
the proof of Lemma \ref{lemma:step-semantics-history-DFA-single-exp}.


Given a pair $(a,B)\in \Ag \times 2^{\Ag}$, we define
$D'_{a,B}= 2^{S_a} \times D_B$ and
$\delta_{a,B}^{\tinypobs} : D'_{a,B} \times 2^P \to D'_{a,B}$ as follows:

$$
  \delta_{a,B}^{\tinypobs}\bigl( (U, (U_b)_{b \in B}), H \bigr) 
  \ = \ 
  \bigl( \delta_a(U,H), \Delta_B^{\tinystep}((U_b)_{b\in B}) \bigr)
$$
for all $H \subseteq P$, $U \subseteq S_a$ and $U_b \subseteq S_b$ for $b\in B$.
Furthermore, we define $$
\delta_{a,B}^{\tinypobs}( init_{a, B}, H)=(\{\init_a\}, (\{\init_b\})_{b\in B}),
$$ when $L_a(init_a) = H\cap P_a$.

We shall write $\cD_{a,B}$ for the structure
$(D_{a,B},\delta_{(a,B)}^{\tinypobs},\init_{a,B})$, where $D_{a,B} = D'_{a,B}\cup \{init_{a, B}\}$
and $\cD_B$ for  $(D_{B}\cup \{\init_{B}\},\delta_{B}^{\tinystep},\init_{B})$,
where $\delta_{B}^{\tinystep}(\init_B, H) = (\{\init_b\})_{b\in B}$, for each $H\subseteq P$.

We now show by induction on the nesting depth of standpoint formulas
that each standpoint formula
$\chi = \StMod{a}{\varphi}$ has a history-DFA of the form 
$(D_{a,E},\delta_{a,E}^{\tinypobs},\init_{a,E},F_{\chi})$
where 
$E = \Ag(\varphi)\setminus \{a\}$ 
if $\varphi$ has no subformula $\StMod{a}{\psi}$ 
where the alternation depth is strictly smaller than $\ad(\varphi)=\ad(\chi)$
and $E = \Ag(\varphi)$ otherwise.
That is, $a \in E$ if and only if 
if $\varphi$ has a subformula $\StMod{a}{\psi}$ 
with $\ad(\StMod{a}{\psi}) < \ad(\chi)$ 


The remaining steps are now similar to the proof of
Lemma \ref{lemma:step-semantics-history-DFA-single-exp}.
%
The basis of induction is obvious as we can deal with 
the history-DFA $\pow(\cT_a,P_a,\ldots)$ 
defined as in Definition \ref{def:default-history-DFA}.

Suppose now that $\varphi$ has $k$ maximal standpoint subformulas, say
$\chi_1 = \StMod{b_1}{\psi_1},\ldots,\chi_k = \StMod{b_k}{\psi_k}$.
%
By induction hypothesis,
there are history-DFA $\cD_1,\ldots,\cD_k$ 
for $\chi_1,\ldots,\chi_k$ that have the form
$\cD_i = (D_{b_i,C_i},\delta_{b_i,C_i}^{\tinypobs},\init_{b_i,C_i},F_i)$
where $C_i = \Ag(\psi_i)$ or $C_i = \Ag(\psi_i)\setminus \{b_i\}$,
depending on whether $\psi_i$ contains a standpoint subformula for agent $b_i$
with alternation depth strictly smaller than $\ad(\chi_i)$.

We now revisit the definition of the transition system $\cT_{\chi}$.
In Section \ref{sec:model-checking}, we defined the state space $Z_{\chi}$ 
as the set of tuples $(s,x_1,\ldots,x_k)$ 
where $s \in S_a$ and $x_i$ is a state
in $\cD_i$. (Recall from Remark \ref{T-a-R} that we may assume that
$\cT_a^R = \cT_a$.)
Thus, 
the reachable states in $\cT_{\chi}$ have the
form $(s,(b_1,(U_c)_{c \in C_1}),\ldots, (b_k,(U_{c\in C_k})))$ 
and contain redundant components
if there are agents $c\in \Ag$ that belong to multiple $C_i$'s
or if an agent $b$ coincides with multiple $b_i$'s.
Thus, we may can redefine the state space of $\cT_{\chi}$ to be
$S_a\times \prod_{b \in B} 2^{S_b} \times
     \prod_{c \in C} 2^{S_c}$ where $B = \{b_1,\ldots,b_k\}$ and
$C = C_1 \ldots \cup C_k$. Note that 
$B \cap C \not= \varnothing$ as well as $a \in B \cup C$ is possible.

Using the assumption that $P_a \cap P_b = \varnothing$ for $a \not= b$,
the transitions in $\cT_{\chi}$ are given by:
$$
  \bigl( s, (T_b)_{b\in B}, (U_c)_{c\in C} \bigr) \to_{\chi}
  \bigl( s', (T_b')_{b\in B}, (U_c')_{c\in C} \bigr)
$$
if and only if the following conditions hold:
\begin{itemize}
\item 
   $s \to_a s'$
\item 
   $(U_c')_{c\in C} = \Delta_C ((U_c)_{c\in C}$
\item
    for each $b \in B$:
    \begin{itemize}
    \item if $b = a$ then $T_a'=\delta_a(T_a,L_a(s'))$ 
    \item if $b \not= a$ then $T_b' = \Delta_b(T_b)$ 
           (as $P_a \cap P_b =\varnothing$)
    \end{itemize}
\end{itemize}
Note that $\delta_a(T_a,L_a(s')) = \delta_a(T_a,H)$
          for each $H \subseteq P$ with $H \cap P_a = L_a(s')$ by definition
of $\delta_a$.

Thus, if $b \in B \cap C$ and $b \not= a$ 
then $U_b = T_b$ for all reachable states 
$\bigl( s, (T_b)_{b\in B}, (U_c)_{c\in C} \bigr)$
in $\cT_{\chi}$. Hence, we can further merge components of
the states in $\cT_{\chi}$ and redefine $\cT_{\chi}$ 
such that its state space $Z_{\chi}$ is contained
\begin{itemize}
\item 
    in $S_a\times  \prod_{c \in E} 2^{S_c}$ if $a \notin B$
\item 
    in $S_a\times 2^{S_a} \times \prod_{c \in E} 2^{S_c}$ if $a \in B$
\end{itemize}
where $E = (B \setminus \{a\}) \cup C$.
%
Thus, $\cT_{\chi}$ can be redefined as the product 
$\cT_a \bowtie \cD_{a,E}$ if $a \in B$ and
as the product $\cT_a \bowtie \cD_{E}$ if $a \notin B$.

We now consider the powerset construction used to define the
default history-DFA $\cD_{\chi}= \pow(\cT_{\chi}',P_a,\ldots)$.
(Recall that $\cT_{\chi}'$ is essentially the same $\cT_{\chi}$ except
the labelings are restricted to $P$.)
Both $\cD_{a,E}$ and $\cD_E$ ar $P_a$-deterministic.
This yields:
\begin{itemize}
\item
  If $a \in B$ then the reachable states of 
  $\cD_{\chi} = \pow(\cT_a \bowtie \cD_{a,E}, P_a,\ldots)$ have the form
  $(T,\{ (U,(U_c)_{c\in E}\})$ where  $U=T$.
\item
  If $a \notin B$ then the states of 
  $\cD_{\chi} = \pow(\cT_a \bowtie \cD_{E}, P_a,\ldots)$ have the form
  $(T,\{ (U_c)_{c\in E}\})$.
\end{itemize}
%
In both cases, we can redefine the default history-DFA such that
$\cD_{\chi}$ has the form 
$(D_{a,E},\delta_{a,E}^{\tinypobs},\init_{a,E},F_{\chi})$.
\end{proof}

%%%%%%%%%%%%%%%%%%%%%%%%%%%%%%%%%%%%%%%%%%%%%%%%%%%%%%%%%%%%%%%%%

\subsection{Embedding of $\StpLTL$ into LTLK}


  In Lemma \ref{lemma:embdding-pobs-in-LTLK}
  we saw that standpoint LTL under 
  the pure observation-based semantics can be embedded into
  LTLK. 
  We now show  (see Lemma \ref{lemma:step-public-in-LTLKone}
  and Lemma \ref{lemma:embedding-incr-in-LTLK} below) that
  the embedding proposed in Lemma \ref{lemma:embdding-pobs-in-LTLK}
  can be modified for the other four semantics.
  While the proposed embedding for the pure observation-based
  semantics is fairly natural and preserves
  the set of agents, the embeddings presented in the proofs of
  Lemma \ref{lemma:step-public-in-LTLKone}
  and Lemma \ref{lemma:embedding-incr-in-LTLK} modify the agent sets.
  In the cases of the step and public-history semantics
  it suffices to deal with a single agent in the constructed LTLK structure.
  In the case of the incremental and decremental semantics,
  we add new agents which can be seen as copies of the agents $a\in \Ag$
  with different indistinguishable relations in the LTLK structure. 


\begin{lemma}
 \label{lemma:step-public-in-LTLKone}
   For $*\in \{\step,\public\}$,
   the $\StpLTL^*$ model checking problem 
   is polynomially reducible to the LTLK$_1$ model checking
   problem.
\end{lemma}

\begin{proof}
  We revisit the translation of
  $\StpLTL$ structures $\fT$ and $\StpLTL$ formulas $\phi$
  into an LTLK structure $\ltlk{\fT}$ and an LTLK formula $\ltlk{\phi}$
  presented in the proof of Lemma \ref{lemma:embdding-pobs-in-LTLK}
  for the pure observation-based semantics.

  For the step semantics, we redefine the LTLK structure 
  $\ltlk{\fT}$ as follows. The transition system $\cT'$
  is the same as in the proof of Lemma \ref{lemma:embdding-pobs-in-LTLK},
  but now we deal with a single agent, say $\alpha$, and define
  $\sim_{\alpha}$ to be the equivalence relation that identifies all states
  in $\cT'$.
  The translation of the $\StpLTL$ formula $\phi$ into an LTLK formula
  $\ltlkstep{\phi}$ is the same as for the pure observation-based setting,
  except that we replace $\psi= \StMod{a}{\varphi}$ with
  $\primed{\psi} = 
    \overline{K}_{\alpha}{(\primed{\varphi} \wedge \Box \neg \bot_a)}$.

  In this way, we obtain an LTLK formula $\ltlkstep{\phi}$ over a
  singleton agent set. 
  Such formulas have alternation depth at most 1.
  Thus, $\ltlkstep{\phi}$ is an LTLK$_1$ formula.
  Soundness in the sense that $\fT \stepmodels \phi$ if and only if
  $\ltlk{\fT} \LTLKmodels \ltlk{\phi}$ can be shown as in the proof
  of Lemma \ref{lemma:embdding-pobs-in-LTLK}
  (see Lemma \ref{app:lemma:embdding-pobs-in-LTLK}).

  For the public-history semantics,
  the reduction is almost the same as for the step semantics,
  except that we deal  with the equivalence relation
  $\sim_{\alpha}$ that identifies those states
  $\sigma$ and $\theta$ in $\cT'$ where 
  $L'(\sigma)=L'(\theta)$.
\end{proof}

The statement of Theorem \ref{thm:step-public-PSPACE-completeness}
is now a consequence of Lemma \ref{lemma:step-public-in-LTLKone}
in combination with the known PSPACE-completeness result for
LTLK$_1$.


%%%%%%%%%%%%%%%%%%%%%%%%%%%%%%%%%%%%%%%%%%%%%%%%%%%%%%%%%%%%%%%%%%%%%%%%%


\begin{lemma}
 \label{lemma:embedding-incr-in-LTLK}
  Let  $N=|\Ag|$, $d \geqslant 1$ and $*\in \{\decr,\incr\}$.
  The $\SLTL{*}{d}$ model checking problem 
  is polynomially reducible to the LTLK$_M$ model checking
  problem where $M = \min \{N,d\}$.
\end{lemma}

\begin{proof}
Let $\fT$ be a given $\StpLTL$ structure and $\phi$ a 
$\StpLTL_d$ formula.
We modify the construction presented in the proof of
Lemma \ref{lemma:embdding-pobs-in-LTLK} 
for the pure observation-based
semantics and redefine the LTLK structure $\ltlk{\fT}$ 
and the LTLK formula $\ltlk{\phi}$.

The transition system $\cT'$ of $\ltlk{\fT}$ is the same as in the
proof of Lemma \ref{lemma:embdding-pobs-in-LTLK}.
But now we deal with more agents. The idea is to add agents for the standpoint
subformulas $\chi = \StMod{a}{\varphi}$ of $\phi$ where $Q^{\tinyincr}_{\chi}$
is a proper superset of $P_a$.

We present the details for the incremental semantics.
The arguments for the decremental semantics are analogous and omitted here.

We consider the syntax tree of $\phi$. For each node $v$ in the syntax tree, 
let $\phi_v$ be the formula represented by $v$.
Let $V$ be the set of nodes $v$
where $\phi_v$ is a standpoint formula, say
$\phi_v = \StMod{a_v}{\phi_w}$ for the child $w$ of $v$. 
Given $v \in V$, let $A_v$ denote the set
of all agents $a \in \Ag$ such that the unique path from the root to  
$v$ contains at least one node $w \in V$ with $a_w=a$.
(With the notations used in the paragraph before
Lemma \ref{lemma:N-EXP-incremental-semantics} we have
$A_v=\Ag_v \cup \{a\}$.)

The observation set for the occurrence of $\phi_v$ as a subformula of $\phi$
represented by node $v$ is 
$\Obsset_{v}^{\tinyincr} = P_{A_v}$
where $P_A = \bigcup_{a \in A} P_a$  for $A \subseteq \Ag$. 
The set $Q_v^{\tinyincr}$ defined in 
Remark \ref{remark:context-dependency-syntax-tree} is
either $P_{A_v}$ or $P_{A_v}\setminus \{a_v\}$.
More precisely,
$Q_v^{\tinyincr} = \varnothing$ if $v$ is the root. For all other nodes $v$,
if $w$ is the father of $v$, then $Q_v^{\tinyincr} = P_{A_v} \setminus \{a_v\}$ 
if $a_v \notin P_{A_w}$ and $Q_v^{\tinyincr} = P_{A_v}$ if $a_v \in P_{A_w}$.

The new agent set of the constructed LTLK structure $\ltlk{\fT}$ is given by:
\begin{center}
  $\Ag' \ = \ 
  \bigl\{ A_v : v \in V, A_v \not= \varnothing \bigr\}$
\end{center}
Thus, $\Ag'$ is a subset of $2^{\Ag}$. 
Its size $|\Ag'|$ is bounded by the size of the syntax tree of $\phi$
and therefore linearly bounded in the length of $\phi$.

The equivalence relations $\sim_{A}$ for agent $A\in \Ag'$
is given by $\sigma \sim_{A} \theta$ if and only if 
$L'(\sigma) \cap P_A^{\bot} = L' (\theta) \cap  P_A^{\bot}$
where $P_A^{\bot}= \bigcup_{a\in A} P_A^{\bot} = P_A \cup \{\bot_a : a \in \Ag \}$.

Intuitively, agent $a$ in $\fT$
corresponds to agent $\{a\}$ in $\ltlk{\fT}$ while the new agents $A$ with
$|A|\geqslant 2$ in $\ltlk{\fT}$ can be seen as coalitions of 
agents who share their
information about the history.

The translation $\varphi \leadsto \primed{\varphi}$ of $\StpLTL$ formulas
$\varphi$ into ``equivalent'' LTLK formulas $\primed{\varphi}$ 
presented in the proof of Lemma \ref{lemma:embdding-pobs-in-LTLK}
is now replaced
with a translation of the $\StpLTL$ formulas $\phi_v$ for the nodes
in the syntax tree of $\phi$ into LTLK formulas.
If $v$ is a leaf (i.e., $\phi_v  \in \{\true\} \cup P$) then $\phi_v'=\phi_v$.
Let $v$ now be an inner node.
If 
$\phi_v = \phi_w \wedge \phi_u$ or $\phi_v = \phi_w \Until \phi_u$
(where $w$ and $u$ are the childs of $v$) then
$\phi_v' = \phi_w' \wedge \phi_u'$ or $\phi_v' = \phi_w' \Until \phi_u'$,
respectively.
The definition is analogous for the cases 
$\phi_v = \neg \phi_w$ or $\phi_v = \neXt \phi_w$.
Suppose now that $v \in V$, i.e., $\phi_v = \StMod{a_v}{\phi_w}$
for the child $w$ for $v$.
Then, we define:
\begin{center}
   $\phi_v' \  = \ 
    \overline{K}_{A_v}{(\phi_w' \wedge \Box \neg \bot_{a_v})}$
\end{center}
For each path $\pi$ in $\cT'$, each $n \in \Nat$ and node $v$ 
in the syntax tree
of $\phi$ we have (where we write $Q_v$ instead of $Q_v^{\tinyincr}$):
%
\begin{center}
  $(\pi,n) \LTLKmodels \phi_v'$
  \ iff \ 
  $(\proj{\trace(\suffix{\pi}{n})}{P},
    \proj{\trace(\prefix{\pi}{n})}{P}) \incrmodels{Q_v} \phi_v$
\end{center}
Finally, we define
$\ltlk{\phi}$ as in the proof of
Lemma \ref{lemma:embdding-pobs-in-LTLK} by
$\ltlk{\phi}= \Box \neg \bot_0 \to \phi'$
where $\phi' = \phi_v'$ for the root node $v$ for the syntax tree of $\phi$.
%
It is now easy to see that
$\fT \incrmodels{} \phi$ iff
$\ltlk{\fT} \LTLKmodels \ltlk{\phi}$.
Furthermore,
the alternation depth of 
$\ltlk{\phi}$ is bounded by both $N=|\Ag|$ and the alternation depth of $\phi$.
\end{proof}

%%%%%%%%%%

Corollary \ref{cor:incr-N-1-EXPSPACE} is now a consequence of
Lemma \ref{lemma:embedding-incr-in-LTLK} 
in combination with the results of
\cite{MC-LTLK-and-beyond-2024}.

%%%%%%%%%%%%%%%%%%%%%%%%%%%%%%%%%%%%%%%%%%%%%%%%%%%%%%%%%%%%%%%%%

\subsection{Generic model checking algorithm with improved space complexity}

\label{sec:m-minus-1-EXPSPACE}

The complexity of 
the generic $\StpLTL^*$ model checking algorithm of
Section \ref{sec:model-checking}
is $m$-fold exponentially time-bounded
with $m=1$ for $* \in \{\step,\public,\decr\}$ (see Lemma \ref{lemma:single-exp-run-time-step-public-decr}),  $m = |\Ag|$ for $* = \incr$ (see Lemma \ref{lemma:N-EXP-incremental-semantics})
and $m = \ad(\phi)$ for $*=\pobs$ (see Lemma \ref{d-exp-time-bound}).
%
To match the space bounds stated in
Corollary 5.5 for $\pobsmodels$ and $\incrmodels{}$, Theorem 5.6 and
Corollary \ref{cor:incr-N-1-EXPSPACE},
the algorithm of Section \ref{sec:model-checking}
can be adapted to obtain an $(m{-}1)$-EXPSPACE algorithm.

Let us briefly sketch the ideas for the pure observation-based semantics.
The idea for $\StpLTL_1$ formulas is to combine classical on-the-fly
automata-based LTL model checking techniques with an on-the-fly
construction of history automata, 
similar to the
techniques proposed in \cite{MC-LTLK-and-beyond-2024} for CTL*K.
This yields a polynomially-space bounded algorithm for $\StpLTL$ formulas
of alternation depth 1.

Suppose now that $d=\ad(\phi)\geqslant 2$.
Let $\chi_1,\ldots\chi_{\ell}$ be the maximal standpoint subformulas
of $\phi$ with $\ad(\chi_i) = d$, $i=1,\ldots,\ell$,
and $\chi_{\ell+1},\ldots,\chi_k$ the maximal standpoint subformulas
of $\phi$ with  $\ad(\chi_i) < d$, $i=\ell{+}1,\ldots,k$.
We apply the model checking
algorithm of Section \ref{sec:model-checking} to generate history-DFAs
$\cD_{\psi}$ for all standpoint subformulas $\psi$ of $\phi$ where either
$\psi \in \{\chi_{\ell {+}1},\ldots,\chi_k\}$ 
or $\psi$ is a standpoint subformula of one of the formulas
$\chi_i$, $i \in \{1,\ldots,\ell\}$ with $\ad(\psi) = d{-}1$ and maximal with
this property (i.e., there is no standpoint 
subformula $\psi'$ of $\chi_i$ where
$\psi$ is a proper subformula of $\psi'$ and $\ad(\psi') = d{-}1$).
Let $\Psi$ denote the set of these subformulas $\psi$ of $\phi$.
The sizes of the history-DFAs $\cD_{\psi}$ 
for $\psi \in \Psi$ are $(d{-}1)$-fold
exponentially bounded.
We now proceed in a similar way as we did in the induction step 
of the model checking algorithm
and introduce pairwise distinct, fresh atomic propositions $p_{\psi}$ 
for each $\psi \in \Psi$.
Then, $\phi' = \phi[\psi/p_{\psi} : \psi \in \Psi]$ is a $\StpLTL$ formula
over $\cP = P \cup \{p_{\psi} : \psi \in \Psi \}$
and has alternation depth 1. We then modify the
$\StpLTL$ structure $\fT$ by replacing
$\cT_0$ with the product of $\cT_0$ with the history-DFAs
for $\chi_1,\ldots,\chi_{\ell}$ and modifying the $\cT_a$'s accordingly.
The size of the resulting structure $\fT'$ 
is $(d{-}1)$-fold exponentially bounded.
We finally run a polynomially space-bounded algorithm to $\fT'$ and $\phi'$.
