\section{Further details and proofs for Section \ref{sec:logic}}

\label{sec:app-logic}

%%%%%%%%%%

\subsection{Alternation depth}

The \emph{alternation depth} $\ad(\varphi)$ of $\StpLTL$ formula $\varphi$ 
is the maximal number of alternations between standpoint modalities
for different agents in $\varphi$. 
For instance, 
$\ad(\StMod{a}{(p \wedge \neXt \DualStMod{b}{q})})=2$, while
$\ad(\StMod{a}{(p \wedge \neXt \DualStMod{a}{q})})= 
 \ad(\StMod{a}{p} \wedge \neXt \DualStMod{b}{q}) = 1$ 
if $a \not= b$.

\begin{definition}[Alternation depth of $\StpLTL$ formulas]
\label{def:alternation-depth}
{\rm
The alternation depth of $\StpLTL$ formulas is defined inductively:
\begin{itemize}
\item
   $\ad(\true)=\ad(p)=0$ for $p\in P$,
\item
   $\ad(\varphi_1 \wedge \varphi_2)=\ad(\varphi_1 \Until \varphi_2)=
      \max \{\ad(\varphi_1),\ad(\varphi_2)\}$,
\item
   $\ad(\neg \varphi)=\ad(\neXt \varphi)=\ad(\varphi)$.
\end{itemize}
%
For standpoint formulas $\phi = \StMod{a}{\varphi}$, 
the definition of $\ad(\phi)$ is as follows.
If $\varphi$ is an LTL formula then
$\ad(\StMod{a}{\varphi})= 1$.
%
Otherwise let $\chi_1=\StMod{b_1}{\psi}, \ldots, \chi_k=\StMod{b_k}{\psi_k}$ 
be the maximal standpoint subformulas of $\varphi$.
We may suppose an enumeration such that $a = b_1 = \ldots = b_{\ell}$ and
$a \notin \{b_{\ell +1},\ldots,b_k\}$.
Then, 
$\ad(\phi)$ is the maximum of
$\max \{ \ad(\chi_i): i =1,\ldots,\ell \}$ and
$\max \{ \ad(\chi_i): i =\ell{+}1,\ldots, k \} + 1$.
  }
\end{definition}

%%%%%%%%%

\subsection{Context-dependency of the decremental and incremental semantics}


   Figure \ref{fig:syntax-tree-incd-decr-semantics}  
   serves to 
   illustrate the definition of the decremental and incremental semantics
   using the sets
   $Q^{\incr}_v$ and  $Q^{\decr}_v$ for the nodes in
   the syntax tree of a given formula $\phi$ defined in
   Remark \ref{remark:context-dependency-syntax-tree}.

\begin{figure*}[ht]
	\begin{center}
		\begin{tikzpicture}[->,>=stealth',shorten >=1pt,auto, semithick]
			\tikzstyle{every state} = [text = black]
			
			\node[state,scale=0.75] (root) [fill = cyan!50, inner sep = 0pt] {$\lor$};
			\node[,scale=0.75] (temp) [below of = root, node distance = 1cm] {};
			\node[state,scale=0.75] (a) [left of = temp, node distance = 1.5cm, fill = cyan!50, inner sep = 0pt] {$\StMod{a}{}$}; 
			\node[state,scale=0.75] (b) [right of = temp, node distance = 1.5cm,fill=cyan!50, inner sep = 0pt] {$\StMod{b}{}$}; 
			\node[state,scale=0.75] (p1) [below of = a, node distance = 1.5cm, fill = magenta!50, inner sep = 0pt] {$p$}; 
			\node[state,scale=0.75] (and) [below of = b, node distance = 1.5cm, fill = green, inner sep = 0pt] {$\land$}; 
			\node[,scale=0.75] (temp2) [below of = and, node distance = 1cm, inner sep = 0pt] {};
			\node[state,scale=0.75] (q) [left of = temp2, node distance = 1cm, fill = green, inner sep = 0pt] {$q$}; 
			\node[state,scale=0.75] (next) [right of = temp2, node distance = 1cm, fill = green, inner sep = 0pt] {$\neXt$}; 
			\node[state,scale=0.75] (a2) [below of = next, node distance = 1.5cm , fill =green, inner sep = 0pt] {$\StMod{a}{}$}; 
			\node[state,scale=0.75] (p2) [below of = a2, node distance = 1.5cm, fill = yellow, inner sep = 0pt] {$p$}; 
			
			\path 
			(root) edge (a)
			(root) edge (b)
			(a) edge (p1)
			(b) edge (and)
			(and) edge (q)
			(and) edge (next)
			(next) edge (a2)
			(a2) edge (p2)
			;
			
			\node [right of = root, node distance = 0.75cm,scale=0.8] {$\decrmodels{P}$}; 
			\node [left of = a, node distance = 0.75cm,scale=0.8] {$\decrmodels{P}$}; 
			\node [left of = p1, node distance = 0.75cm,scale=0.8] {$\decrmodels{P_a}$}; 
			\node [right of = b, node distance = 0.75cm,scale=0.8] {$\decrmodels{P}$}; 
			\node [right of = and, node distance = 0.75cm,scale=0.8] {$\decrmodels{P_b}$}; 
			\node [left of = q, node distance = 0.75cm,scale=0.8] {$\decrmodels{P_b}$}; 
			\node [right of = next, node distance = 0.75cm,scale=0.8] {$\decrmodels{P_b}$}; 
			\node [right of = a2, node distance = 0.75cm,scale=0.8] {$\decrmodels{P_b}$}; 
			\node [right of = p2, node distance = 0.82cm,scale=0.8] {$\decrmodels{P_b \cap P_a}$}; 
			
			%%rhs
			
			\node[state,scale=0.75] (rootr) [fill = cyan!50, inner sep = 0pt, right of = root, node distance = 8cm] {$\lor$};
			\node[,scale=0.75] (tempr) [below of = rootr, node distance = 1cm] {};
			\node[state,scale=0.75] (ar) [left of = tempr, node distance = 1.5cm, fill = cyan!50, inner sep = 0pt] {$\StMod{a}{}$}; 
			\node[state,scale=0.75] (br) [right of = tempr, node distance = 1.5cm,fill=cyan!50, inner sep = 0pt] {$\StMod{b}{}$}; 
			\node[state,scale=0.75] (p1r) [below of = ar, node distance = 1.5cm, fill = magenta!50, inner sep = 0pt] {$p$}; 
			\node[state,scale=0.75] (andr) [below of = br, node distance = 1.5cm, fill = green, inner sep = 0pt] {$\land$}; 
			\node [,scale=0.75] (temp2r) [below of = andr, node distance = 1cm, inner sep = 0pt] {};
			\node[state,scale=0.75] (qr) [left of = temp2r, node distance = 1cm, fill = green, inner sep = 0pt] {$q$}; 
			\node[state,scale=0.75] (nextr) [right of = temp2r, node distance = 1cm, fill = green, inner sep = 0pt] {$\neXt$}; 
			\node[state,scale=0.75] (a2r) [below of = nextr, node distance = 1.5cm , fill =green, inner sep = 0pt] {$\StMod{a}{}$}; 
			\node[state,scale=0.75] (p2r) [below of = a2r, node distance = 1.5cm, fill = yellow, inner sep = 0pt] {$p$}; 
			
			\path 
			(rootr) edge (ar)
			(rootr) edge (br)
			(ar) edge (p1r)
			(br) edge (andr)
			(andr) edge (qr)
			(andr) edge (nextr)
			(nextr) edge (a2r)
			(a2r) edge (p2r)
			;
			
			\node [right of = rootr, node distance = 0.75cm,scale=0.75] {$\incrmodels{\varnothing}$}; 
			\node [left of = ar, node distance = 0.75cm,scale=0.75] {$\incrmodels{\varnothing}$}; 
			\node [left of = p1r, node distance = 0.75cm,scale=0.75] {$\incrmodels{P_a}$}; 
			\node [right of = br, node distance = 0.75cm,scale=0.75] {$\incrmodels{\varnothing}$}; 
			\node [right of = andr, node distance = .75cm,scale=0.75] {$\incrmodels{P_b}$}; 
			\node [left of = qr, node distance = 0.75cm,scale=0.75] {$\incrmodels{P_b}$}; 
			\node [right of = nextr, node distance = 0.75cm,scale=0.75] {$\incrmodels{P_b}$}; 
			\node [right of = a2r, node distance = 0.75cm,scale=0.75] {$\incrmodels{P_b}$}; 
			\node [right of = p2r, node distance = 0.82cm,scale=0.75] {$\incrmodels{P_b \cup P_a}$}; 
		\end{tikzpicture}
		\caption{Syntax tree for
			$\phi = \StMod{a}{p} \vee 
			\StMod{b}{(q \wedge \neXt \StMod{a}{p})}$
			and the corresponding satisfaction relations
			$\decrmodels{Q}$ and $\incrmodels{Q}$ for the subformulas.}
		\label{fig:syntax-tree-incd-decr-semantics} 
\end{center}
\end{figure*}

%%%%%%%%%%%%%%%%%%%%%%%%%%%%%%%%%%%%%%%%%%%%%%%%%%%%%%%%%%%%%%%%%%%%%%%%%%%

\subsection{Properties of $\StpLTL$}

\begin{lemma}[(Cf.~Lemma \ref{obs-a-lemma})]
 \label{app:obs-a-lemma}
    If $h_1, h_2 \in (2^P)^+$ with $\obs_a(h_1)=\obs_a(h_2)$  
    then
    $(*,h_1) \models \StMod{a}{\varphi}$ iff 
    $(*,h_2) \models \StMod{a}{\varphi}$.
\end{lemma}

\begin{proof}
By symmetry it suffices to show that
$(*,h_1) \models \StMod{a}{\varphi}$ implies
$(*,h_2) \models \StMod{a}{\varphi}$.
Suppose $(*,h_1) \models \StMod{a}{\varphi}$. 
Hence, there exist $h' \in (2^{P})^+$,
a state $t\in \Reach(\cT_a,h')$ 
and a trace
$f'\in \Traces^P(\cT_a,t)$ such that
$\last(h')=\first(f')$,
$\obs_a(h_1)=\obs_a(h')$ and $(f',h') \models \varphi$.
But then, $\obs_a(h_2)= \obs_a(h_1)=\obs_a(h')$
and $(*,h_2) \models \StMod{a}{\varphi}$.
\end{proof}

%%%%%%%%%%%%%%%%%%%%%%%%%%%%%%%%%%%%%%%%%%%%%%%%%%%%%%%%%%%%%%%%%%%%%

\begin{lemma}[Cf.~Lemma \ref{lemma:pobs=decr=incr-for-alternation-depth-1}]
\label{app:lemma:pobs=decr=incr-for-alternation-depth-1}
If $\phi$ is a $\StpLTL$ formula of alternation depth 1 then
$\fT \pobsmodels \phi$ 
iff $\fT \decrmodels{} \phi$ 
iff $\fT \incrmodels{} \phi$, while
$\fT \not\pobsmodels \phi$ 
and $\fT \models \phi$ 
(or vice versa)
is possible where ${\models} \in \{\stepmodels,\publicmodels\}$.
Likewise,
$\fT \not\publicmodels \phi$ 
and $\fT \stepmodels \phi$ (or vice versa) is possible.
\end{lemma}

\begin{proof}
The first part follows from the observation that
for every occurrence of a subformula of $\phi$ 
that is in the scope of a standpoint modality
$\StMod{a}{}$, the observation set used to extract information
from the history is $P_a$ under the pure observation-based,
the decremental and the incremental semantics.

To distinguish the pure observation-based from the public semantics,
consider $P=\{p\}$, $P_a=\varnothing$, $\phi = \StMod{a}{p}$ and suppose 
$p \notin L_0(\init_0)$. Then, $\fT \pobsmodels \phi$ 
(as $\cT_a$ has no information on $\cT_0$'s initial state and
may guess the satisfaction of $p$ in the initial state)
and $\fT \not\publicmodels \phi$ 
(as $\cT_a$ has observed that $p$ does not hold in $\cT_0$'s initial state).

To distinguish $\pobsmodels$ and $\publicmodels$ from $\stepmodels$,
regard 
$P=P_a=\{p\}$, $\phi = \StMod{a}{p}$ and suppose
that $p \in L_a(\init_a) \setminus L_0(\init_0)$. 
Then, $\fT \not\pobsmodels \phi$ and $\fT \not\publicmodels \phi$,
while $\fT \stepmodels \phi$.
\end{proof}


%%%%%%%%%%%%%%%%%%%%%%%%%%%%%%%%%%%%%%%%%%%%%%%%%%%%%%%%%%%%%%%%%%%%%%%%%%%%
%%%%%
%%%%%    Embedding POBS to LTLK
%%%%%
%%%%%%%%%%%%%%%%%%%%%%%%%%%%%%%%%%%%%%%%%%%%%%%%%%%%%%%%%%%%%%%%%%%%%%%%%%%%


\subsection{Embedding of $\StpLTL^{\tinypobs}$ into LTLK}


\begin{lemma}[Cf.~Lemma \ref{lemma:embdding-pobs-in-LTLK}]
	\label{app:lemma:embdding-pobs-in-LTLK}
	\label{app:lemma:embedding-pobs-in-LTLK}
	Given a pair $(\fT,\phi)$ consisting of a $\StpLTL$ structure 
	$\fT = (\cT_0, (\cT_a)_{a\in \Ag})$ and a $\StpLTL$ formula $\phi$, 
	one can construct in polynomial time
	an LTLK structure $\ltlk{\fT} = (\cT',(\sim_a)_{a\in \Ag})$
	and an LTLK formula $\ltlk{\phi}$ such that:
	\begin{enumerate}
		\item [(1)] 
		$\fT \pobsmodels \phi$ if and only if 
		$\ltlk{\fT} \LTLKmodels \ltlk{\phi}$ 
		\item [(2)] 
		$\phi$ and $\ltlk{\phi}$ have the same alternation depth.
	\end{enumerate}
\end{lemma}

\begin{proof}
	We may assume w.l.o.g. that the labelings of the initial states
	in $\cT_0$ and the $\cT_a$'s are consistent in the sense that
	$L_0(\init_0) \cap P_a = L_a(\init_a)$ for each $a \in \Ag$.
	If this is not the case, we extend $\cT_a$ for $a\in \Ag \cup \{0\}$ by
	a fresh initial state $\init_a'$ with $L_a(\init_a') = \varnothing$ and
	transitions $\init_a' \to_a \init_a$ and replace the given $\StpLTL$ formula
	$\phi$ with $\neXt \phi$.
	
	For each $a \in \Ag \cup \{0\}$, 
	we first switch from $\cT_a$ to its completion $\cT_a^{\bot}$ which extends
	$\cT_a$ by states $\bot_a^Q$ for $Q \subseteq P_a$.
	The state space of $\cT_a^{\bot}$ is 
	$S_a^{\bot} = S_a \cup \{\bot_a^Q : Q \subseteq P_a\}$, and the initial state is $\init_a$.
	The transition relation of $\cT_a^{\bot}$ extends $\to_a$ by:
	\begin{itemize}
		\item
		$s_a \to_a \bot_a^Q$ if there is no state $s'$ in $\cT_a$ with
		$L_a(s')=Q$ and $s \to_a s'$ 
		\item
		$\bot_a^R \to_a \bot_a^Q$ for all $R, Q \subseteq P_a$.
	\end{itemize}
	%
	The set of atomic propositions in $\cT_a^{\bot}$ is 
	$P_a^{\bot}=P_a \cup \{\bot_a\}$.
	The labeling of the original states is unchanged, i.e.,
	$L_a^{\bot}(s)=L_a(s)$, while
	for the fresh states $L_a^{\bot}(\bot_a^Q)=Q \cup \{\bot_a\}$.
	
	Thus, for each word $\rho = L_a(\init_a)\rho'$, where  $\rho'\in (2^{P_a})^{\omega}$ there exists a trace 
	$f \in \Traces(\cT^{\bot}_a)$
	with $\proj{f}{P_a}=\rho$. 
	(In this sense, 
	$\cT_a^{\bot}$ is the completion of $\cT_a$.) 
	Moreover, all $\rho \in \Traces(\cT_a^{\bot})$ satisfy the LTL formula
	$\Box \neg \bot_a \vee \Diamond \Box \bot_a$.
	The original traces of $\cT_a$ can be recovered from $\cT_a^{\bot}$
	as we have:
	\begin{center}
		$\Traces(\cT_a,s) \ = \ 
		\bigl\{ \proj{\rho}{P_a} :  
		\rho \in \Traces(\cT_a^{\bot},s), \ \rho \LTLmodels \Box \neg \bot_a  
		\bigr\}$
	\end{center}
	
	The transition system $\cT' = (S',\to',\init',P',L')$ 
	of the constructed LTLK structure $\ltlk{\fT}$
	arises by the synchronous product of
	the transition systems $\cT_a^{\bot}$. 
	That is, the state space $S'$ of $\cT'$ consists of all
	tuples $(s,(s_a)_{s\in \Ag}) \in S_0^{\bot} \times \prod_{a\in \Ag} S_a^{\bot}$ 
	such that $L_0^{\bot}(s) \cap P_a = L_a^{\bot}(s_a)\cap P_a$ for each $a \in \Ag$.
	The set $P'$ of atomic propositions is 
	$P'=P \cup \{\bot_a : a \in \Ag \cup \{0\}\}$.
	The labeling function is given by 
	$L'(s, (s_a)_{a\in \Ag}) = L_0(s) \cup 
	\{\bot_a : a \in \Ag \cup \{0\}, \bot_a \in L_a^{\bot}(s_a)  \}$.
	The initial state of $\cT'$ is 
	$\init' = (\init_0, (\init_a)_{a\in \Ag})$. By the assumption that
	the labelings of the initial states in $\fT$ are consistent, $\init'$ is indeed
	an element of $S'$.
	The transition relation $\to'$ in $\cT'$ is defined as follows. If
	$\sigma = (s,(s_a)_{a\in \Ag})$ and $\theta =(t,(t_a)_{a\in \Ag})$ are
	elements of $S'$ then
	$\sigma \to' \theta$
	if and only if $s \to_0 t$ in $\cT_0^{\bot}$ and $s_a \to_a t_a$ in 
	$\cT_a^{\bot}$.
	For $a \in \Ag$, the equivalence relation $\sim_a$ on $S'$ is given by:
	$\sigma \sim_a \theta$ if and only if 
	$L'(\sigma) \cap P_a^{\bot} = L'(\theta) \cap P_a^{\bot}$.
	By construction, for each trace $\rho \in (2^P)^{\omega}$ there
	is at least one path $\pi$ in $\cT'$ with $\proj{\trace(\pi)}{P}=\rho$. 
	
	
	The essential idea for
	translating the given $\StpLTL$ formula $\phi$ into an ``equivalent''
	LTLK formula $\ltlk{\phi}$ is to identify 
	$\DualStMod{a}{\varphi}$ with 
	$K_a (\Box \neg \bot_a \, \to \, \varphi)$.
	The precise definition of $\ltlk{\phi}$ is as follows.
	We first provide a translation $\varphi \leadsto \primed{\varphi}$
	of $\StpLTL$ formulas $\varphi$ into LTLK formulas $\primed{\varphi}$
	by structural induction:
	If $\varphi \in \{\true\} \cup P$ then $\primed{\varphi}=\varphi$.
	In the step of induction, we put
	$\primed{(\neg \varphi)}=\neg \primed{\varphi}$, 
	$\primed{(\varphi \wedge \psi)} = \primed{\varphi}\wedge \primed{\psi}$,
	$\primed{(\neXt \varphi)}=\neXt \primed{\varphi}$,
	$\primed{(\varphi \Until \psi)} = \primed{\varphi} \Until \primed{\psi}$ and
	$\primed{(\StMod{a}{\varphi})} = 
	\overline{K}_a (\primed{\varphi} \wedge \Box \neg \bot_a)$
	where $\overline{K}_a$ is the dual of $K_a$ given by
	$\overline{K}_a \psi = \neg K_a \neg \psi$.
	%
	Finally, we define $\ltlk{\phi}= \Box \neg \bot_0 \to \primed{\phi}$.
	
	Obviously, the translation of $(\fT,\phi)$ 
	into $(\ltlk{\fT},\ltlk{\phi})$ is polynomial.
	%
	It remains to show that $\fT \pobsmodels \phi$ 
	iff $\ltlk{\fT} \LTLKmodels \ltlk{\phi}$. 
	By structural induction we obtain that for each path $\pi$ in $\cT'$
	and each $n \in \Nat$ we have:
	%
	\begin{center}
		$(\pi,n) \LTLKmodels \primed{\varphi}$
		\ iff \ 
		$(\proj{\trace(\suffix{\pi}{n})}{P},
		\proj{\trace(\prefix{\pi}{n})}{P}) \pobsmodels \varphi$
	\end{center}
	%
	where $\LTLKmodels$ refers to the LTLK semantics over the
	structure $\ltlk{\fT}$ and
	$\pobsmodels$ to the pure observation-based semantics of $\StpLTL$
	over the original $\StpLTL$ structure $\fT$.
	Then, $\ltlk{\fT} \LTLKmodels \ltlk{\phi}$ iff 
	$(\pi,0) \LTLKmodels \Box \neg \bot_0 \to \primed{\phi}$
	for every path $\pi \in \Paths(\cT')$
	iff
	$(\pi,0) \LTLKmodels \primed{\phi}$
	for every path $\pi \in \Paths(\cT')$ 
	with $\trace(\pi) \LTLmodels \Box \neg \bot_0$
	iff
	$(f,\first(f)) \pobsmodels \phi$ for every
	trace $f \in \Traces(\cT_0)$
	iff $\fT \pobsmodels \phi$.
	Here, we use the fact that $\Traces(\cT_0)$ equals 
	$\{\proj{\trace(\pi)}{P} : 
	\pi \in \Paths(\cT'), \trace(\pi) \LTLmodels \Box \neg \bot_0 \}$.
\end{proof}
