
%%%%%%%%%%%%%%%%%%%%%%%%%%%%%%%%%%%%%%%%%%%%%%%%%%%%%%%%%%%%%%%%%%%
%%%%%%%%%%%%%%%%%%%%%%%%%%%%%%%%%%%%%%%%%%%%%%%%%%%%%%%%%%%%%%%%%%%
%%%%
%%%%     Syntax and semantics of StpLTL
%%%%
%%%%%%%%%%%%%%%%%%%%%%%%%%%%%%%%%%%%%%%%%%%%%%%%%%%%%%%%%%%%%%%%%%%
%%%%%%%%%%%%%%%%%%%%%%%%%%%%%%%%%%%%%%%%%%%%%%%%%%%%%%%%%%%%%%%%%%%



\section{$\StpLTL$: LTL with standpoint modalities}

\label{sec:logic}

Standpoint LTL ($\StpLTL$)
extends LTL by standpoint modalities $\StMod{a}{\varphi}$ 
where $a$ is an agent and $\varphi$ a formula.

\subsection{Syntax}

Given a finite set $P$ of atomic propositions and
a finite set $\Ag$ of agents, say $\Ag= \{a, b, \ldots\}$,
the syntax of $\StpLTL$ formulas over $P$ and $\Ag$ for $p \in P$ and $a \in \Ag$ is given by
\begin{center}
 $\begin{array}{lclcr}
  \varphi & ::= & 
  \true  \ \big|  \ p \ \big| \ \neg \varphi  \ \big| \ \varphi_1 \wedge \varphi_2
  \ \big| \ \neXt \varphi  \ \big|  \ \varphi_1 \Until \varphi_2  \ \big|  \ 
   \StMod{a}{\varphi}
 \end{array}$
\end{center}
The intuitive meaning of $\StMod{a}{\varphi}$
is that from agent $a$'s standpoint it is conceivable that $\varphi$ 
will hold, in the sense that there are indications from $a$'s view 
that there is a path
starting in the current state that fulfills $\varphi$. 
%
The dual standpoint modality is defined by
$\DualStMod{a}{\varphi} \ = \ \neg \StMod{a}{\neg \varphi}$ and
has the intuitive meaning that from the standpoint of $a$,
$\varphi$ is unequivocal.
That is, under $a$'s view
all paths starting in the current state fulfill $\varphi$.


Formulas of the shape $\StMod{a}{\varphi}$ are 
called \emph{standpoint formulas}.
If $\varphi$ is a $\StpLTL$ formula, then \emph{maximal standpoint subformulas} of $\varphi$
are subformulas that have the form $\StMod{a}{\phi}$ 
and that are not in the scope
of another standpoint operator.
For example, 
$\varphi = (p \wedge \neXt \chi_1) \vee \psi$
with $\chi_1 = \StMod{a}{ (\StMod{b}{q} \Until r)}$ and
 $\psi =
     \DualStMod{c}{ \neXt (r \wedge \StMod{a}{p \Until \StMod{a}{q}})}$
has two maximal standpoint subformulas, namely
$\chi_1$ and
$\chi_2 = 
 \StMod{c}{ \neg \neXt (r \wedge \StMod{a}{p \Until \StMod{a}{q}})}$.


The \emph{alternation depth} $\ad(\varphi)$ of $\varphi$
is the maximal number of alternations between standpoint modalities
for different agents. 
For instance, 
$\ad(\StMod{a}{(p \wedge \neXt \DualStMod{b}{q})})=2$, while
$\ad(\StMod{a}{(p \wedge \neXt \DualStMod{a}{q})})= 
 \ad(\StMod{a}{p} \wedge \neXt \DualStMod{b}{q}) = 1$ 
if $a \not= b$.
The precise definition is provided in
Definition \ref{def:alternation-depth} in the appendix.
%
Each $\StpLTL$ formula over a singleton agent set $\Ag=\{a\}$ has
alternation depth at most 1. 
%
For $d \in \Nat$, let $\StpLTL_d$ denote the sublogic of $\StpLTL$ where all formulas $\varphi$ satisfy $\ad(\varphi) \leq d$. In particular, $\StpLTL_0$ is LTL.

\begin{remark}
\label{remark:sharpening}
{\rm
The original papers \cite{SLTL-KR23,Complexity-SLTL-ECAI24} 
on standpoint LTL have an additional 
type of formula $a \preceq b$ 
where $a, b \in \Ag$.
The intuitive semantics of $a \preceq b$ is that 
the standpoint of $a$ is sharper than that of $b$.
In our setting, where we are given
transition systems for $a$ and $b$, formulas $a \preceq b$ are
either true or false depending on whether the set of traces
of the transition system representing $a$'s view is contained in
the set of $b$'s traces or not. 
As the trace inclusion problem is PSPACE-complete \cite{KanSmo90}
like the LTL model-checking problem
\cite{SistlaClarke-JACM85},
the complexity results of Section \ref{sec:complexity} 
are not affected when adding sharpening statements $a \preceq b$ to $\StpLTL$.
  }
\end{remark}
   


%%%%%%%%%%%%%%%%%%%%%%%%%%%%%%%%%%%%%%%%%%%%%%%%%%%%%%%%%%%%%%%%%%%%%%%%%%
%%%%%
%%%%%      subsection: semantics
%%%%%
%%%%%%%%%%%%%%%%%%%%%%%%%%%%%%%%%%%%%%%%%%%%%%%%%%%%%%%%%%%%%%%%%%%%%%%%%%


\subsection{Semantics of $\StpLTL$}
\label{sec:semantics}

We will consider different semantics of the standpoint modality
that differ in what the agents can observe from the history.


%%%%%%%%%%%%%%%%%%%%%%%%%%%%%%%%%%%%%%%%%%%%%%%%%%%%%%%%%%%%%%%%%%%%%%%%%%

\paragraph*{$\StpLTL$ structures.}
%
\label{sec:structures}
%
$\StpLTL$ structures are
tuples 
$\fT = \bigl(\cT_0,(\cT_a)_{a\in \Ag}\bigr)$ with $\cT_0 = (S_0,\to_0,\init_0,P_0,L_0)$ a 
transition system 
over the full set $P_0=P$ of atomic propositions, and
$\cT_a= (S_a,\to_a,\init_a,P_a,L_a)$ transition systems 
for the agents $a\in \Ag$ over some $P_a \subseteq P$.
For simplicity, we assume these transition systems to have
unique initial states.



%%%%%%%%%%%%%%%%%%%%%%%%%%%%%%%%%%%%%%%%%%%%%%%%%%%%%%%%%%%%%%%%%%%

\paragraph*{Semantics of the standpoint modalities.}

We consider five different semantics for
$\StMod{a}{\varphi}$, using satisfaction relations
$\stepmodels$ (step semantics as in \cite{SLTL-KR23,Complexity-SLTL-ECAI24}), 
$\probsmodels$ 
(pure observation-based semantics which is essentially the
perfect-recall partial information semantics for LTLK
\cite{MC-LTLK-and-beyond-2024} 
adapted for $\StpLTL$ structures),
$\publicmodels$ 
(public-history semantics which can be seen as a 
perfect-recall semantics where agents have full information about the
history),
$\decrmodels{Q}$ and $\incrmodels{Q}$
(variants of $\pobsmodels$ with decremental resp. incremental knowledge)
where $Q \subseteq P$.

We deal here with an interpretation over future-history pairs 
(see Section \ref{sec:prelim}).
Let $\models$ be one of the five satisfaction relations.
The semantics of $\StpLTL$
can be presented in a uniform manner
as shown in  Figure \ref{fig:semantics}.
%
For the dual standpoint operator we obtain:
$(f,h) \models \DualStMod{a}{\varphi}$ iff $(f',h') \models \varphi$ 
for all $h' \in (2^{P})^+$, $t\in \Reach(\cT_a,h')$ 
and $f'\in \Traces^P(\cT_a,t)$ with
$\last(h')=\first(f')$ and
$\obs_a(h)=\obs_a(h')$.

The five semantics rely on different observation functions
$\obs_a$. In all cases, 
$\obs_a$ is a projection $\obs_a: (2^P)^+ \to (2^{\Obsset_a})^+$, 
$\obs_a(h) = \proj{h}{\Obsset_a}$
for some $\Obsset_a \subseteq P$.
Intuitively, $\Obsset_a$ formalizes which of the propositions are visible to agent $a$
in the history. (For the decremental and
incremental semantics,
both $\Obsset_a$ and the induced observation function $\obs_a$
do not only depend on $a$, but on the context of the
formula $\StMod{a}{\varphi}$ as will be explained later.)
Before presenting the specific choices of $\Obsset_a$ in the five 
$\StpLTL$ semantics, we make some general observations:

\begin{lemma}
\label{future-ind-lemma}
  If $f_1, f_2\in (2^P)^{\omega}$ and $h \in (2^P)^+$ with 
  $\last(h)=\first(f_1)=\first(f_2)$ then
    $(f_1,h) \models \StMod{a}{\varphi}$  iff 
    $(f_2,h) \models \StMod{a}{\varphi}$. 
\end{lemma}
%
Lemma \ref{future-ind-lemma} permits to drop the $f$-component 
and to write $(*,h)\models \StMod{a}{\varphi}$
when $(f,h)\models \StMod{a}{\varphi}$ for some (each) future
$f$ with $\first(f)=\last(h)$. 
%
The truth values of standpoint formulas $\StMod{a}{\varphi}$
only depend on agent $a$'s observation of the history:


\begin{lemma}
 \label{obs-a-lemma}
    If $h_1, h_2 \in (2^P)^+$ with $\obs_a(h_1)=\obs_a(h_2)$  
    then
    $(*,h_1) \models \StMod{a}{\varphi}$ iff 
    $(*,h_2) \models \StMod{a}{\varphi}$.
\end{lemma}



\begin{remark}
\label{remark:standard-SLTL-semantics}
  {\rm
$\StpLTL$ 
structures defined in \cite{SLTL-KR23,Complexity-SLTL-ECAI24}
assign to each agent $a$ a nonempty subset $\lambda(a) \subseteq (2^P)^{\omega}$
and assume a universal agent, called *, such that 
$\lambda(a)\subseteq \lambda(*)$ for all other agents $a$.
The latter is irrelevant for our purposes.
Assuming transition system representations for the sets $\lambda(a)$
is natural for the model checking problem.
In contrast to \cite{SLTL-KR23,Complexity-SLTL-ECAI24}, 
we suppose here that the 
standpoint transition systems $\cT_a$ are defined over some $P_a \subseteq P$, which appears more natural for defining the information that an agent can extract
from the history.
In the semantics of $\StMod{a}{\varphi}$, we thus switch from
$\Traces(\cT_a,t) \subseteq (2^{P_a})^{\omega}$ to
$\Traces^P(\cT_a,t) \subseteq (2^{P})^{\omega}$ which essentially means
that $a$ may guess the truth values of the atomic propositions
in $P \setminus P_a$ to predict whether $\varphi$ can hold in the future.
Alternatively, one could define $\StpLTL$ structures
as tuples $(\cT_0, (\cT_a,P_a)_{a \in \Ag})$ where
$\cT_0$ is as before, the $\cT_a$'s are transition systems 
over some $R_a \subseteq P$, and $P_a \subseteq R_a$ where
the $P_a$ serves to define the functions
$\obs_a$ for the histories.
With $R_a=P$ 
and $\lambda(a)=\Traces(\cT_a)$,
$\StpLTL$ under $\stepmodels$ agrees with the logic 
considered 
in \cite{SLTL-KR23,Complexity-SLTL-ECAI24} (except for sharpening
statements; see Remark \ref{remark:sharpening}).
Our model checking algorithm can easily be
adapted for this more general type of $\StpLTL$ structures 
without affecting our complexity results.
   }
\end{remark}



%%%%%%%%%%%%%%   step semantics

\paragraph*{Step semantics:}
%
The semantics of the standpoint modality 
$\StMod{a}{\varphi}$ 
introduced in 
\cite{SLTL-KR23,Complexity-SLTL-ECAI24} 
relies on the assumption that the only information that the agents can extract
from the history is the number
of steps that have been performed in the past.
They formulate the semantics in terms of trace-position pairs 
$(\rho,n)\in (2^P)^{\omega}\times \Nat$ and define
$(\rho,n) \models \StMod{a}{\varphi}$ iff $(\rho',n)\models \varphi$
for some $\rho'\in \Traces(\cT_a)$.
Reformulated to our setting,
the set of propositions visible for agent $a$ in the history
is $\Obsset_a=\varnothing$, which yields the 
observation function $\obs_a : (2^P)^+ \to \Nat$, $\obs_a(h)=|h|$.
%
An equivalent formulation for future-history pairs is:
  $(f,h) \stepmodels \StMod{a}{\varphi}$ iff
     there exist a word $h' \in (2^{P})^+$,
     a state 
     $t\in \Reach(\cT_a,h')$ 
     and a trace
     $f'\in \Traces^P(\cT_a,t)$ such that 
     $\last(h')=\first(f')$,
     $|h|=|h'|$ and $(f',h') \stepmodels \varphi$.

%%%%%%%%    pure observation-based semantics

\paragraph*{Pure observation-based semantics:}
%
In the style of the (dual of the) classical K-modality in LTLK \cite{HalVar-STOC86,HalVar-JCSS89,MC-LTLK-and-beyond-2024} (cf. Section \ref{sec:embedding-StpLTL-into-LTLK}) we can deal with
the observation function that projects the given history $h$ to the
observations that agent $a$ can make when exactly the propositions in $P_a$
are visible for $a$. That is, for the pure observation-based semantics,  $\Obsset_a=P_a$. Then,
  $(f,h) \probsmodels \StMod{a}{\varphi}$ iff 
     there exist a word $h' \in (2^{P})^+$,
     a state 
     $t\in \Reach(\cT_a,h')$ 
     and a trace
     $f'\in \Traces^P(\cT_a,t)$ such that 
     $\last(h')=\first(f')$,
     $\proj{h}{P_a}=\proj{h'}{P_a}$ and $(f',h') \probsmodels \varphi$.

%%%%%%%%%%%%%%%  public-history semantics

\paragraph*{Public-history semantics:}
%
In the public-history semantics all agents $a$ have full information
about the history $h$, i.e., $\Obsset_a=P$ 
and 
$\obs_a(h)=h$. Then,
  $(f,h) \publicmodels \StMod{a}{\varphi}$ iff 
     there exist 
     a state $t\in \Reach(\cT_a,h)$ 
     and a trace
     $f'\in \Traces^P(\cT_a,t)$ such that 
     $\last(h)=\first(f')$ and
     $(f',h) \publicmodels \varphi$.


%%%%%%%%%%%%  decremental semantics

\paragraph*{Decremental semantics:}
%
\label{semantics-decremental-obs-based}
%
The decremental semantics is a variant of the pure observation-based 
semantics with a different meaning of nested standpoint subformulas:
A standpoint formula $\StMod{a}{\varphi}$ 
interpretes
maximal standpoint subformulas $\StMod{b}{\psi}$ of $\varphi$ from the view of
agent $a$. The assumption is that $a$ 
knows agent $b$'s transition system $\cT_b$ and thus
can make the same guesses for the future as $b$, but can access
only the truth values of 
joint atomic propositions in $P_a' \cap P_b$ from the history to make a guess
what $b$ has observed in the past.
Here, $P_a' \subseteq P_a$ is the set of atomic propositions accessible
when interpreting $\StMod{a}{\varphi}$ 
which can itself be a subformula
of a larger standpoint formula $\StMod{c}{\chi}$ 
(in which case $\StMod{a}{\varphi}$ is interpreted from the view of agent $c$
and
$P_a' \subseteq P_c \cap P_a$).
Formally, we use a parametric satisfaction relation
$\decrmodels{Q}$ for $Q \subseteq P$.
The intuitive meaning of $\StMod{a}{\varphi}$ under $\decrmodels{Q}$ is that  $a$ can extract from the history 
$h$ exactly the truth values of the propositions in 
$\Obsset_a^Q= Q \cap P_a$. So,
$(f,h) \decrmodels{Q} \StMod{a}{\varphi}$ iff
     there exist a word $h' \in (2^{P})^+$, 
     a state 
     $t\in \Reach(\cT_a,h')$ 
     and a trace
     $f'\in \Traces^P(\cT_a,t)$ such that 
     $\last(h')=\first(f')$,
     $\proj{h}{Q \cap P_a}= \proj{h'}{Q \cap P_a}$
     and $(f',h') \decrmodels{Q \cap P_a} \varphi$.
%
For the satisfaction over a $\StpLTL$ structure, we start with $Q=P$.

%%%%%%%%%%%%%%%%  incremental semantics

\paragraph*{Incremental semantics:}
%`
The incremental semantics also relies on a parametric satisfaction relation
$\incrmodels{Q}$ where $Q \subseteq P$.
The intuitive meaning of 
$\StMod{a}{\varphi}$ 
under $\incrmodels{Q}$ 
is 
that agent $a$ can extract from the history $h$ exactly the propositions
in $\Obsset_a^Q= Q \cup P_a$ and interpretes $\varphi$
over $Q \cup P_a$.
Thus, when interpreting nested standpoint subformulas
$\StMod{b}{\psi}$ of $\StMod{a}{\varphi}$  then agents $a$ and $b$ build a
coalition to extract the information from the history.
Formally, $(f,h) \incrmodels{Q} \StMod{a}{\varphi}$ iff
     there exist a word $h' \in (2^{P})^+$, 
     a state 
     $t\in \Reach(\cT_a,h')$ 
     and 
     a trace
     $f'\in \Traces^P(\cT_a,t)$ such that 
     $\last(h')=\first(f')$,
     $\proj{h}{Q \cup P_a}= \proj{h'}{Q \cup P_a}$
     and $(f',h') \incrmodels{Q \cup P_a} \varphi$.
%
For the satisfaction over a $\StpLTL$ structure, we start with $Q=\varnothing$.

\begin{example}
	\normalfont
Let us  illustrate  the differences between the semantics:
Consider two agents $a$ and $b$ with $P_a=\{p,r\}$, $P_b=\{q,r\}$, and the following standpoint transition systems:
\begin{center}
\begin{tikzpicture}[->,>=stealth',shorten >=1pt,auto, semithick]
				\tikzstyle{every state} = [text = black]
				
				%left
				\node[state,scale=0.6] (lroot1) [inner sep = 0pt, fill = yellow!50] {$\{p\}$};
				\node[state, scale=0.6] (l1) [below of = lroot1, node distance = 1.35cm, fill = magenta!50] {$\{r\}$};
				\node[scale = 0.6] (linit1) [left of = lroot1, node distance = 1.2cm] {};
				
				\node[state,scale=0.6] (lroot2) [inner sep = 0pt, right of = lroot1, node distance = 2.5cm] {$\emptyset$};
				\node[state, scale=0.6] (l2) [below of = lroot2, node distance = 1.35cm] {$\emptyset$};
				\node[scale = 0.6] (linit2) [left of = lroot2, node distance = 1.2cm] {};
				
				\path 
					(linit1) edge (lroot1)
					(lroot1) edge (l1)
					(l1) edge [loop left] (l1)
					(linit2) edge (lroot2)
					(lroot2) edge (l2)
					(l2) edge [loop right] (l2)
				
				;
				
				
				%%%right 
				\node[state,scale=0.6] (rroot1) [inner sep = 0pt, fill = cyan!50, right of = lroot2, node distance = 5cm] {$\{q\}$};
				\node[state, scale=0.6] (r1) [below of = rroot1, node distance = 1.35cm] {$\emptyset$};
				\node[scale = 0.6] (rinit1) [left of = rroot1, node distance = 1.2cm] {};
				
				\node[state,scale=0.6] (rroot2) [inner sep = 0pt, right of = rroot1, node distance = 2.5cm] {$\emptyset$};
				\node[state, scale=0.6] (r2) [below of = rroot2, node distance = 1.35cm, fill = magenta!50] {$\{r\}$};
				\node[scale = 0.6] (rinit2) [left of = rroot2, node distance = 1.2cm] {};
				
				\path 
				(rinit1) edge (rroot1)
				(rroot1) edge (r1)
				(r1) edge [loop left] (r1)
				(rinit2) edge (rroot2)
				(rroot2) edge (r2)
				(r2) edge [loop right] (r2)
				
				;
				
				\node [left of = linit1, node distance = 0.35cm, scale = 0.8] {$\cT_{a}$};
				\node [left of = rinit1, node distance = 0.35cm, scale = 0.8] {$\cT_{b}$};
				
			\end{tikzpicture}
\end{center}
%
We consider  formulas $\phi=\StMod{a}{\neXt \StMod{b}{x}}$ with $x\in P = P_a \cup P_b$.
As these formulas contain only the existential $\StMod{}{}$ modality, satisfaction becomes  easier if less restrictive  requirements are imposed  on histories.
So, for all $\rho=(f,h)$ and all $x$ as above:
$
\rho \publicmodels \phi \Rightarrow \rho \incrmodels{} \phi \Rightarrow \rho \probsmodels \phi \Rightarrow \rho \decrmodels{} \phi \Rightarrow \rho \stepmodels \phi$.

Now, we show that the reverse implications do not hold. Let $\rho=(f,h)$ with $h=\{p,q\}$ and $f=\{p,q\}\{r\}^\omega$.
For $\phi=\StMod{a}{\neXt \StMod{b}{r}}$, we have
$\rho \not\publicmodels \phi $: In the quantification of $\StMod{a}$, all possible future traces $f'$ have to start with $\{p,q\}$. So, according to $\cT_a$, the future has to contain $r$ in the second step.
When evaluating $\StMod{b}$, these first two steps have to be respected, i.e., the history under consideration is $h'=\{p,q\} X$ where $X$ contains $r$. But there is no trace in $\cT_b$ with $q$ in the first and $r$ in the second step.
However, $\rho \incrmodels{} \phi $ as we can choose any trace $\{p\}\{r\}...$ as the future trace in the quantification of $\StMod{a}$. When projecting to $P_b$ we obtain a trace in $\cT_b$ satisfying $r$ at the second step.

Next, consider $\phi=\StMod{a}{\neXt \StMod{b}{p}}$ and $\rho$ as above.
Then, $\rho \not\incrmodels{} \phi $ as the future trace $f'$ quantified in $ \StMod{a}$ cannot have $p$ in the second step because there is no such trace in $\cT_a$. In the incremental semantics, the history chosen for $\StMod{b}$ 
has to agree with $f'$  on $p$ in the second step and hence can also not contain $p$ there.
In contrast, $\rho \probsmodels \phi $ as here agent $b$ can ``choose''   the truth value of $p$ arbitrarily as $p\not \in P_b$.

With similar reasoning, $\probsmodels$ and $\decrmodels{}$ can be distinguished by the formula $\phi=\StMod{a}{\neXt \StMod{b}{(q\land r)}}$ and $\rho$ as above:
$\rho \not \probsmodels \phi$ as $q$ is  false on any state reachable in $\cT_b$ after one step. So, no matter which history $h'$ and future $f'$ is chosen in the quantification of
$ \StMod{a}$, there is no  future-history pair $(f'',h'')$ that agrees with a trace in $\cT_b$ on all atomic propositions in $P_b$ and is such that $q$ holds at the second position. 
In contrast, $\rho \decrmodels{} \phi$: We can choose $h'=\{p\}$ and $f'=\{p\}\{r\}^\omega$ for the quantification of $ \StMod{a}$ as $h'$ agrees with $h=\{p,q\}$ on all atomic propositions in $P_a$. Then, the quantification of $ \StMod{b}$ in $\phi$ can choose $h''=\emptyset \{r,q\}^\omega$ and $f''=\{r,q\}^\omega$ as this agrees with the trace of the right-hand side path in $\cT_b$ for all atomic propositions in $P_a\cap P_b = \{r\}$.

Finally, for $\decrmodels{}$ and  $\stepmodels$ consider  $\phi=\StMod{a}{\neXt \StMod{b}{\neg r}}$ and $\rho$ as above.
Under $\decrmodels{}$, in the first quantification of $ \StMod{a}$, the history $h'$ has to contain $p$ and potentially $q$ and so the future $f'$ has to contain $r$ in the second step as only the left initial state in $\cT_a$ is possible. So, the history $h''$ quantified in $ \StMod{b}$ also has to contain $r$ in the second step as $r\in P_a\cap P_b$ and so $\phi$ cannot hold, i.e., $\rho \not \decrmodels{} \phi$.
In the step semantics, however, the quantification for $\StMod{b}$ can simply choose history  $h''=\{q\}\emptyset$ and future  $f''=\emptyset^\omega$ and so $\rho \stepmodels \phi$. 
\end{example}

%%%%%%%%%%%%%%%%%%%%%%%%%%%%%%%%%%%%%%%%%%%%%%%%%%%%%%%%%%%%%%%%%%%%%%%%%

\paragraph*{Satisfaction of SLTL formulas over structures.}
%
Given a structure $\mathfrak{T}= \bigl(\cT_0,(\cT_a)_{a\in \Ag}\bigr)$
as in Section \ref{sec:structures}, an $\StpLTL$ formula $\phi$
and 
${\models}  \in \{{\stepmodels}, 
  {\probsmodels}, {\publicmodels}, {\decrmodels{}}, {\incrmodels{}} \}$:
\begin{center}
  $\mathfrak{T}\models \phi$
  \quad iff \quad
  $(f,\first(f)) \models_* \phi$   
  for all $f \in \Traces(\cT_0)$
\end{center}
where 
$\models_*$ equals $\models$ for the step, pure observation-based
and public-history semantics,
$\models_*$ equals $\decrmodels{P}$ for the decremental semantics
and $\models_*$ equals $\incrmodels{\varnothing}$ 
for the incremental semantics.


%%%%%%%%%%%%%%%%%%%%%%%%%%%%%%%%%%%%%%%%%%%%%%%%%%%%%%%%%%%%%%%%%%%%%%%%%

\begin{remark}
\label{remark:context-dependency-syntax-tree}
{\rm
The meanings of $\StMod{a}{\varphi}$ under 
$\decrmodels{Q}$ and $\incrmodels{Q}$ are context-dependent through
the parameter $Q$.
If, e.g.,  
$\phi = \StMod{a}{\varphi} \vee \StMod{b}{(q \wedge \neXt \StMod{a}{\varphi})}$ for $\varphi$ an $\StpLTL$ formula
then the first occurrence of $\StMod{a}{\varphi}$ 
is interpreted over $\decrmodels{P}$, 
while the second is interpreted over $\decrmodels{P_b}$.
The set $Q = Q_{\chi}^*$ with $*\in \{\decr,\incr\}$ 
over which occurrences of subformulas $\chi$ 
of $\phi$ are interpreted
can be derived from the syntax tree of $\phi$.
%

Formally, we assign a set $Q_v^*$ with $*\in \{\decr,\incr\}$ 
to each node $v$ in the tree. Let $\phi_v$ denote the subformula represented by $v$.
If $v$ is the root node then $\phi_v=\phi$ and
$Q_{v}^{\tinydecr}=P$ while
$Q_{v}^{\tinyincr}=\varnothing$.
Let now $v$ and $w$ be nodes such that $v$ 
is the father of $w$.
If $v$ is not labeled by a standpoint modality then
$Q_v^*=Q_w^*$. Otherwise $\phi_v$ has the form $\StMod{a}{\phi_w}$
for some $a\in \Ag$
and $Q_w^{\tinydecr} = Q_v^{\tinydecr} \cap P_a$ while 
$Q_w^{\tinyincr} = Q_v^{\tinyincr} \cup P_a$.
With abuse of notations,
we will simply write $Q_{\chi}^*$ instead of $Q_v^*$ for a node $v$ 
representing the particular occurrence of $\chi = \phi_v$ in $\phi$.
  }
\end{remark}


%%%%%%%%%%%%%%%%%%%%%%%%%%%%%%%%%%%%%%%%%%%%%%%%%%%%%%%%%%%%%%%%%%%%%%%%%

Obviously, the five semantics agree on the LTL fragment.
More precisely,
for $\models$ as above,  
$\fT$ a $\StpLTL$ structure and $\phi$ an LTL formula
we have 
$\fT \models \phi$ iff $\cT_0 \LTLmodels \phi$ in the sense that
$\Traces(\cT_0) \subseteq \{ f \in (2^P)^{\omega} : f \LTLmodels \phi\}$. 
%
For $\StpLTL$ formulas
with at least one standpoint subformula, but no alternation
of standpoint modalities,
the pure observation-based, decremental
and incremental semantics agree,
but might yield different truth values than the step or public-history 
semantics:

\begin{lemma}
\label{lemma:pobs=decr=incr-for-alternation-depth-1}
If $\phi$ is a $\StpLTL$ formula of alternation depth $1$ then
$\fT \pobsmodels \phi$ 
iff $\fT \decrmodels{} \phi$ 
iff $\fT \incrmodels{} \phi$, while
$\fT \not\pobsmodels \phi$ 
and $\fT \models \phi$ 
(or vice versa)
is possible where ${\models} \in \{\stepmodels,\publicmodels\}$.
Likewise,
$\fT \not\publicmodels \phi$ 
and $\fT \stepmodels \phi$ (or vice versa) is possible.
\end{lemma}

%%%%%%%%%%%%%%%%%%%%%%%%%%%%%%%%%%%%%%%%%%%%%%%%%%%%%%%%%%%%%%%%%%

\subsection{From pure observation-based $\StpLTL$ to LTLK}

\label{sec:embedding-StpLTL-into-LTLK}

LTLK (also called CKL$_m$) 
\cite{HalVar-STOC86,HalVar-JCSS89,MC-LTLK-and-beyond-2024} 
is an extension of LTL by an unary knowledge modality $K_a$ for 
every agent $a \in \Ag$ and a common knowledge operator $C_A$ for 
coalitions $A \subseteq \Ag$.
We drop the latter and deal with LTLK where the grammar 
for formulas is the same as for $\StpLTL$ when
$\StMod{a}{\varphi}$ is replaced with $K_a \varphi$.
The alternation depth of LTLK formulas and the sublogics LTLK$_d$ are defined as for $\StpLTL$.

LTLK structures are tuples 
$(\cT,(\sim_a)_{a\in \Ag})$ where 
$\cT = (S,\to,\init,R,L)$ is a transition system and the
$\sim_a$, $a \in \Ag$, are equivalence relations on $S$. The intended meaning
is that $s \sim_a t$ if agent $a$ cannot distinguish the states $s$ and $t$.

The perfect-recall semantics of LTLK extends
the standard LTL semantics formulated for path-position pairs $(\pi,n)$ 
consisting of a path $\pi = s_0 s_1 s_2 \ldots \in \Paths(\cT)$ 
and a position $n \in \Nat$
by $(\pi,n) \LTLKmodels K_a\varphi$ iff 
$(\pi',n) \LTLKmodels \varphi$ for all paths 
$\pi' = s_0' s_1' s_2' \ldots \in \Paths(\cT)$ with
$s_i \sim_a s_i'$ for $i=0,1,\ldots ,n$.
As such, the dual standpoint modality
$\DualStMod{a}{}$ under the pure observation-based semantics
resembles the $K_a$ modality of LTLK.
Indeed, $\StpLTL$ under $\probsmodels$ can be considered
as a special case of LTLK under the perfect-recall semantics:

\begin{lemma}
\label{lemma:embdding-pobs-in-LTLK}
\label{lemma:embedding-pobs-in-LTLK}
  Given a pair $(\fT,\phi)$ consisting of a $\StpLTL$ structure 
 $\fT = (\cT_0, (\cT_a)_{a\in \Ag})$ and a $\StpLTL$ formula $\phi$, 
 one can construct in polynomial time
 an LTLK structure $\ltlk{\fT} = (\cT',(\sim_a)_{a\in \Ag})$
 and an LTLK formula $\ltlk{\phi}$ such that:
\begin{enumerate}
\item [(1)] 
   $\fT \pobsmodels \phi$ if and only if 
   $\ltlk{\fT} \LTLKmodels \ltlk{\phi}$ 
\item [(2)] 
   $\phi$ and $\ltlk{\phi}$ have the same alternation depth.
\end{enumerate}
\end{lemma}



\begin{proofsketch}
	Assume w.l.o.g. that the initial state labelings in $\cT_0$ and $\cT_a$ are consistent for all $a$. 
   First, for each $a \in \Ag \cup \{0\}$, we extend $\cT_a$ to its completion $\cT_a^{\bot}$ by adding new states $\bot_a^Q$ for each $Q \subseteq P_a$, where the set of atomic propositions in $\cT_a^{\bot}$ is 
	$P_a^{\bot}=P_a \cup \{\bot_a\}$.
	The labeling of the original states is unchanged, i.e.,
	$L_a^{\bot}(s)=L_a(s)$, while
	for the fresh states $L_a^{\bot}(\bot_a^Q)=Q \cup \{\bot_a\}$. 
  %	
   We then construct the synchronous product $\cT' = S_0^{\bot}\times\prod_{a \in \Ag} \cT_a^{\bot}$	such that $L_0^{\bot}(s) \cap P_a = L_a^{\bot}(s_a)\cap P_a$ for each $a \in \Ag$.
	The set of atomic propositions is 
	$P'=P \cup \{\bot_a : a \in \Ag \cup \{0\}\}$, and 
	$L'(s, (s_a)_{a\in \Ag}) = L_0(s) \cup 
	\{\bot_a : a \in \Ag \cup \{0\}, \bot_a \in L_a^{\bot}(s_a)  \}$.
	The initial state of $\cT'$ is 
	$\init' = (\init_0, (\init_a)_{a\in \Ag})$, and the transitions are defined synchronously. The equivalence relations $\sim_a$ on $S'$ satisfy $\sigma \sim_a \theta$ iff $L'(\sigma) \cap P_a^{\bot} = L'(\theta) \cap P_a^{\bot}$.
	We then inductively
	translate $\StpLTL$ formulas $\varphi$ into ``equivalent'' LTLK formulas $\primed{\varphi}$:
	$
	\primed{\varphi} = \varphi, \ \primed{(\neg \varphi)} = \neg \primed{\varphi}, \  \primed{(\varphi \wedge \psi)} = \primed{\varphi} \wedge \primed{\psi}, \ \primed{(\neXt \varphi)} = \neXt\primed{\varphi}, \ \primed{(\varphi_1 \Until \varphi_2)} = \primed{\varphi_1} \! \! \! \Until \primed{\varphi_2}, \ \primed{(\StMod{a}{\varphi})} = \overline{K}_a (\primed{\varphi} \wedge \Box \neg \bot_a).
	$
	Finally, we define $\ltlk{\phi} = \Box \neg \bot_0 \to \primed{\phi}$.
\end{proofsketch}
