
%%%%%%%%%%%%%%%%%%%%%%%%%%%%%%%%%%%%%%%%%%%%%%%%%%%%%%%%%%%%%%%%%%%%%
%%%%%%%%%%%%%%%%%%%%%%%%%%%%%%%%%%%%%%%%%%%%%%%%%%%%%%%%%%%%%%%%%%%%%
%%%%%
%%%%%    Complexity-theoretic results
%%%%%
%%%%%%%%%%%%%%%%%%%%%%%%%%%%%%%%%%%%%%%%%%%%%%%%%%%%%%%%%%%%%%%%%%%%%
%%%%%%%%%%%%%%%%%%%%%%%%%%%%%%%%%%%%%%%%%%%%%%%%%%%%%%%%%%%%%%%%%%%%%



\section{Complexity-theoretic results}

\label{sec:complexity}

At first glance the time complexity of
our generic model checking algorithm 
is  $d$-fold exponential for
$\StpLTL$ formulas $\varphi$ where $d$ is the maximal nesting depth of standpoint modalities in $\varphi$. 
This yields a nonelementary upper bound for the
$\StpLTL$ model checking problem.

For $*\in \{\step,\pobs,\public,\decr,\incr\}$, let $\StpLTL^*$ denote $\StpLTL$ under the semantics w.r.t. $\models_*$.
We now investigate the complexity 
of the different $\StpLTL^*$ model checking problems. 

%\optional{%
%As $\StpLTL$ under all 5 semantics is an extension of LTL 
%and the LTL model checking
%problem is known to be PSPACE-complete \cite{SistlaClarke-JACM85},
%the $\StpLTL^*$ model checking problem is PSPACE-hard.}

%%%%%%%%%%%%%%%%%%%%%%%%%%%%%%%%%%%%%%%%%%%%%%%%%%%%%%%%%%%%%%%%%%%%%%

\subsubsection*{Bounds on the sizes of history-DFA and
                time bounds}

Obviously, it suffices to construct the reachable fragment of the
default history-DFA. Furthermore, we may apply a 
standard poly-time minimization
algorithm to the default history-DFA. Thus, to establish better complexity
bounds it suffices to provide elementary 
bounds on the sizes of minimal history-DFA
for standpoint subformulas.
%
Let us start with a simple observation on how to exploit the property that
the history-DFA $\cD_{\chi}$ are $\Obsset_{\chi}$-deterministic.

\begin{remark}
\label{remark:exploit-obs-determinism}
{\rm
With the notations of the step of induction for $\chi=\StMod{a}{\varphi}$,
the history-DFA $\cD_{\chi}$ for $\chi$ is obtained by 
(the reachable fragment) of $\pow(\cT_{\chi}', \Obsset_{\chi}, U_{\chi})$ where 
$U_{\chi}=\Sat_{\cT_{\chi}}(\exists \varphi')$.
Recall that $\cT_{\chi}'$ has been defined via a product construction
$\cT_a^R \bowtie \cD_1 \bowtie \ldots \bowtie \cD_k$ where the $\cD_j$'s
are history-DFAs for the maximal subformulas $\chi_j = \StMod{b_j}{\psi_j}$ 
of $\varphi$.

Suppose now that $i\in \{1,\ldots,k\}$ such that $\cD_i$
is $\Obsset_{\chi}$-deterministic.
Then, for all histories $h \in (2^P)^+$  and states
$((s,O),x_{1},\ldots,x_{k})$ 
in $\Reach(\cT_{\chi}',h)$ we have
$x_i = \delta_i(\init_i,h)$.
Hence, if $x$ is a reachable state in $\cD$, 
say $x=\delta_{\cD_{\chi}}(\init_{\cD},h)$, then for all states
$((s,O),x_{1},\ldots,x_{k}) \in x$ we have 
$x_i = \delta_i(\init_i,h)$.
Thus, we may redefine the default history-DFA $\cD_{\chi}$ as a product 
$\pow(\cT_a^R \bowtie \prod_{j \not= i} \cD_j,\Obsset_{\chi}) 
    \bowtie \cD_i$ 
rather than a powerset construction of
$\cT_a^R \bowtie \cD_1 \bowtie \ldots \bowtie \cD_k$.
%
From now on, we write $\pow(\cT,\Obsset)$ for the
structure $(\cX_{\cD},\delta_{\cD},\init_{\cD})$ defined as in
Definition \ref{def:default-history-DFA}. The acceptance condition 
$F_{\cD_{\chi}}$
is (re)defined as the set of all states $(x,x_i)$ 
in the above product (i.e., $x$ is a subset of the state space of 
$\cT_a^R \bowtie \prod_{j \not= i} \cD_j$ 
and $x_i$ a state in $\cD_i$) such that 
$\{ (\xi,x_i) : \xi \in x\} \cap U_{\chi} \neq \varnothing$.
%
Let $I$ denote the set of indices $i \in \{1,\ldots,k\}$ where
the history-DFA for $\chi_i=\StMod{b_i}{\psi_i}$ 
is $\Obsset_{\chi}$-deterministic.
W.l.o.g. $I=\{1,\ldots,\ell\}$.
Then, we can think of $\cD_{\chi}$ as (the reachable
fragment of) a product
$\pow(\cT_a^R \bowtie \cD_{\ell+1} \bowtie \ldots \bowtie \cD_k, \Obsset_{\chi}) 
    \bowtie \cD_1 \bowtie \ldots \bowtie \cD_{\ell}$.
  }
\end{remark}

If $\chi=\StMod{a}{\varphi}$ then
$\Obsset_{\chi_i} = \Obsset_{\chi}$
for all maximal standpoint subformulas $\chi_i = \StMod{b_i}{\psi_i}$
of $\varphi$ where $a=b_i$.
Thus, by induction on $d=\ad(\chi)$,
Remark \ref{remark:exploit-obs-determinism}
yields that
the size of the reachable fragment of the default 
history-DFA for a standpoint formula
$\chi$ is $d$-fold exponentially bounded. Hence:

\begin{lemma}
\label{d-exp-time-bound}
  Under all five semantics,
  the time complexity of the 
  algorithm of Section \ref{sec:model-checking}
  is at most $d$-fold exponential when $d=\ad(\phi)$.
\end{lemma}

This $d$-fold exponential upper time bound will now be improved
using further consequences of 
Remark \ref{remark:exploit-obs-determinism}.

We start with $\stepmodels$.
Here, we have $\Obsset_{\chi} = \varnothing$ for all $\chi$.
With the notations used in Remark \ref{remark:exploit-obs-determinism},
the history-DFA $\cD_1,\ldots,\cD_k$ are
$\Obsset_{\chi}$-deterministic. Thus, we can think of
$\cD_{\chi}$ as a product 
$\pow(\cT_a,\varnothing) \bowtie \cD_1 \bowtie \ldots \bowtie \cD_k$.
But now, each of the $\cD_i$'s also has this shape.
In particular, if \( \chi_i = \StMod{a}{\psi_i} \), then the projection of the first coordinate of the states and the transitions between them in the reachable fragment of \( \cD_i \) matches exactly those of \( \pow(\cT_a, \varnothing) \).
Thus, one can incorporate the information on final states
and drop $\pow(\cT_a,\varnothing)$ from the product.
As a consequence, the default history-DFA $\cD_{\chi}$ 
can be redefined for $\stepmodels$ such that
the state space of $\cD_{\chi}$ is contained in
$\prod_{b\in \Ag(\chi)} 2^{S_b}$ 
where $\Ag(\chi)$ denotes the set of agents $b\in \Ag$ 
such that $\chi$ has a (possibly non-maximal) standpoint subformula
of the form $\StMod{b}{\psi}$. 
%
  (See 
   Lemma \ref{lemma:step-semantics-history-DFA-single-exp} in
  the appendix for details.)
%
The situation is similar for $\publicmodels$
(where $\Obsset_{\chi} = P$ for all $\chi$) and
for $\decrmodels{}$ (where $\Obsset_{\chi_i} \subseteq \Obsset_{\chi}$
for all standpoint subformulas $\chi_i$ of $\chi$).
In the case of $\publicmodels$ there are history-DFA
where the state space
is contained in $(\prod_{b\in \Ag(\chi)} 2^{S_b}) \times 2^P$.
For $\decrmodels{}$, we assign history-DFA $\cD_v$ to the nodes
in the syntax tree of $v$ that have the shape
$\prod_{w} \pow(\cT_{a_w},\Obsset^{\tinydecr}_w)$
where $w$ ranges 
over all nodes in the syntax subtree of $v$ 
 such that the formula given by $w$ is a standpoint formula
$\StMod{a_w}{\varphi_w}$.
%
  (See 
   Lemma \ref{lemma:public-semantics-history-DFA-single-exp} 
   and Lemma \ref{lemma:decr-semantics-history-DFA-single-exp}
   in the appendix.)
%
For $\probsmodels$ under the additional assumption
that $P_a \cap P_b =\varnothing$ for $a,b\in \Ag$ with $a \not= b$,
one can simplify the transition rules for $\cT_{\chi}$
to obtain history-DFA $\cD_{\chi}$ for 
standpoint formulas $\chi=\StMod{a}{\varphi}$ 
where the state space is contained in 
$2^{S_a} \times \prod_{b \in \Ag(\varphi)} 2^{S_b}$.
%
  (See Lemma \ref{lemma:pobs-disjoint-semantics-history-DFA-single-exp}
  in the appendix.)



%%%%%%%%%%%%%%%%%%%%%%%%%%%%%%%%%%%%%%%%%%%%%


With these observations, we obtain (where we assume 
the above mentioned simplification of the default-history DFAs):

\begin{lemma}
 \label{lemma:single-exp-run-time-step-public-decr}
   For $*\in \{\step,\public,\decr\}$,
   the algorithm of Section \ref{sec:model-checking}
   for $\StpLTL^*$ model checking
   runs in (single) exponential time.
   %
   The same holds for $*=\pobs$ 
   under the additional assumption that the $P_a$'s are pairwise disjoint.
\end{lemma}

%%%%%%%%%%%%%%%%%%%%%%%%%%%%%%%%%%%%%%%%%%%%%%%%%%%%%%%%%%%

Let us now look at $\incrmodels{}$. 
Given a node $v$ in the syntax tree of $\phi$, let $\phi_v$ denote the subformula
of $\phi$ that is represented by $v$, let $\pi_v$ denote
the unique path $\pi_v=v_0 v_1 \ldots v_r$ in the syntax tree from
the root $v_0$ to $v_r = v$, and 
let $\Ag_v$ denote the set of agents
$b\in \Ag$ such that $\phi_w$ has the form $\StMod{b}{\psi_w}$ for some
$w \in \{v_1,\ldots, v_{r-1}\}$.
Then, $Q_{v}^{\tinyincr} = \bigcup_{b \in \Ag_v} P_b$
and $\Obsset_{v}^{\tinyincr} = Q_v^{\tinyincr} \cup P_a$ if
$\phi_v = \StMod{a}{\varphi}$ (see Remark \ref{remark:context-dependency-syntax-tree}).
Let 
$\pi_v'= w_1\ldots w_{\ell}$ be the sequence 
resulting from $\pi_v$
when all nodes $w$ where $\phi_w$ is not a standpoint subformula
are removed.
The observation sets along $\pi_v'$ are increasing, i.e.,
$\varnothing = \Obsset_{w_1}^{\tinyincr} \subseteq \Obsset_{w_2}^{\tinyincr}
    \subseteq \ldots \subseteq \Obsset_{w_{\ell}}^{\tinyincr} \subseteq P$.
Thus, the number of indices $j \in \{2,\ldots,\ell\}$ where
$\Obsset_{w_{j-1}}^{\tinyincr}$ and $\Obsset_{w_j}^{\tinyincr}$ 
are different is bounded
by $N{-}1$ where $N = |\Ag|$. If
$\Obsset_{w_{j-1}}^{\tinyincr} = \Obsset_{w_j}^{\tinyincr}$, 
the history-DFA constructed for $\phi_{w_j}$ is
$\Obsset_{w_{j-1}}^{\tinyincr}$-deterministic.
We can now again apply Remark \ref{remark:exploit-obs-determinism} 
to $\chi=\phi_{w_{j-1}}$ and $\chi_i = \phi_{w_j}$. This yields:


\begin{lemma}
 \label{lemma:N-EXP-incremental-semantics}
   With $N=|\Ag|$,
   our model checking algorithm for $\StpLTL^{\tinyincr}$ 
   is $N$-fold exponentially time bounded.
\end{lemma}

%%%%%%%%%%%%%%%%%%%%%%%%%%%%%%%%%%%%%%%%%%%%%%%%%%%%%%%%%%%%%%%%%%

\subsubsection*{Space bounds}


Lemma \ref{lemma:embdding-pobs-in-LTLK} 
shows that the $\SLTL{\pobs}{d}$ model checking problem 
is polynomially reducible to the LTLK$_{d}$ model checking
problem under the perfect-recall semantics.
The latter is known to be $(d{-}1)$-EXPSPACE-complete 
for $d \geqslant 2$ and PSPACE-complete for $d=1$
\cite{MC-LTLK-and-beyond-2024}.
%
In combination with the known
PSPACE lower bound for the LTL model checking problem
\cite{SistlaClarke-JACM85} and 
Lemmas \ref{lemma:pobs=decr=incr-for-alternation-depth-1}
and \ref{lemma:embdding-pobs-in-LTLK} 
we obtain:

\begin{corollary} 
 \label{cor:complexity-pobs}
    For $*\in \{\pobs,\decr,\incr\}$,
    the $\StpLTL^*_1$ model checking problem is PSPACE-complete.
    %
    For $d \geqslant 2$,
    the $\SLTL{$\pobs$}{d}$ model checking problem 
    is in $(d{-}1)$-EXPSPACE.
\end{corollary}

For the step and public-history semantics, the
reduction provided in the proof of Lemma \ref{lemma:embdding-pobs-in-LTLK}
can be adapted by dealing with an LTLK structure over a single agent.
Thus, the generated LTLK formula has alternation depth at most 1.
%
See Lemma \ref{lemma:step-public-in-LTLKone} in the appendix.


\begin{theorem}
 \label{thm:step-public-PSPACE-completeness}
   For $*\in \{\step,\public\}$,
   the $\StpLTL^*$ model checking problem 
   is polynomially reducible to the LTLK$_1$ model checking
   problem, and therefore PSPACE-complete.
\end{theorem}

Lemma \ref{lemma:embdding-pobs-in-LTLK} and 
Theorem \ref{thm:step-public-PSPACE-completeness} show that the
$\StpLTL^*$ model checking problem for $*\in \{\step,\public,\pobs\}$
can be viewed as an instance of the LTLK model checking problem.
An analogous statement holds for $\decrmodels{}$ and $\incrmodels{}$.
The idea is to modify the reduction described in the
proof of Lemma \ref{lemma:embdding-pobs-in-LTLK} by adding new
agents 
for all standpoint subformulas $\chi = \StMod{a}{\psi}$ where
$P_a \not= \Obsset_{\chi}$.
%
This yields a polynomial reduction of the 
$\SLTL{*}{d}$ model checking problem 
to the LTLK$_M$ model checking
problem where $M = \min \{|\Ag|,d\}$ and $*\in \{\incr, \decr\}$.
%
See Lemma \ref{lemma:embedding-incr-in-LTLK} in the
appendix.
%
With the above mentioned results of \cite{MC-LTLK-and-beyond-2024}, we can improve
the complexity-theoretic upper bound stated in
Lemma \ref{lemma:N-EXP-incremental-semantics}:


\begin{corollary}
\label{cor:incr-N-1-EXPSPACE}  
  The $\StpLTL^{\tinyincr}$ model checking problem is in
  $(N{-}1)$-EXPSPACE where $N = |\Ag|$.
\end{corollary}
