%%%%%%%%%%%%%%%%%%%%%%%%%%%%%%%%%%%%%%%%%%%%%%%%%%%%%%%%%%%%%%%%%%%
%%%%%%%%%%%%%%%%%%%%%%%%%%%%%%%%%%%%%%%%%%%%%%%%%%%%%%%%%%%%%%%%%%%
%%%%
%%%%     introduction
%%%%
%%%%%%%%%%%%%%%%%%%%%%%%%%%%%%%%%%%%%%%%%%%%%%%%%%%%%%%%%%%%%%%%%%%
%%%%%%%%%%%%%%%%%%%%%%%%%%%%%%%%%%%%%%%%%%%%%%%%%%%%%%%%%%%%%%%%%%%


\section{Introduction}

Automated reasoning about the dynamics of scenarios in which multiple agents with access to different information interact is a key problem
in artificial intelligence and  formal verification. Epistemic temporal logics  are prominent, expressive formalisms to specify properties of such scenarios (see, e.g., \cite{HalVar-STOC86,Halpern-Overview86,Reasoning-about-knowledge-MIT-2004,MC-LTLK-and-beyond-2024}).
The resulting algorithmic problems, however, often have  non-elementary  complexity or are even undecidable (see, e.g., \cite{vanderMeydenS1999,Dima2009,MC-LTLK-and-beyond-2024}).
%

Aiming to balance expressiveness and computational tractability, \cite{SL-FOIS21,phdGomez} defines
static \emph{standpoint logics} that extend propositional logic with modalities
$\StMod{a}{\varphi}$ expressing that ``according to agent $a$, it is conceivable that $\varphi$'' and the dual modalities
$\DualStMod{a}{\varphi}$ expressing that ``according to $a$, it is unequivocal that $\varphi$''.
Standpoint logics and their extensions have proven useful to, e.g., reason about inconsistent  formalizations of concepts 
in the medical domain to align different ontologies and in a forestry application, where different sources disagree about the global extent of forests \mbox{\cite{SL-FOIS21,phdGomez,Gomez2022}}.

Recently introduced combinations of linear temporal logic (LTL) with standpoint modalities
\cite{SLTL-KR23,Complexity-SLTL-ECAI24} enables reasoning about dynamical aspects of multi-agent systems.
The focus of \cite{SLTL-KR23,Complexity-SLTL-ECAI24}  
is the satisfiability problem for the resulting  \emph{standpoint LTL} ($\StpLTL$).
In this paper, we consider the model-checking problem that asks whether all executions of a transition system satisfy a given $\StpLTL$-formula. To the best of our knowledge, this problem has not been addressed in the literature.

Whether the formula $\StMod{a}{\varphi}$ holds  after some finite history, i.e., whether it is plausible for agent $a$ that property $\varphi$ holds on the future execution, depends on $a$'s standpoint
 as well as what was observable to $a$ from the history. 
 To illustrate this, consider a situation in which different political agents have different perceptions of how actions taken by the state influence future developments.
 These perceptions are the standpoints of the agents. After a series of events, i.e., a history, the agents deem different future developments possible according to their standpoint
 as well as the aspects of the history that they are actually aware of.  When reasoning about each other's standpoints, there are  different ways in which information might be exchanged between the agents.
 For example, in a discussion agents might learn about aspects of the history they were not aware of; or they might completely ignore what other agents have observed in the past.
 
 Besides the semantics for $\StpLTL$ proposed in \cite{SLTL-KR23,Complexity-SLTL-ECAI24}, we introduce four additional semantics that differ in the amount of information agent $a$ can access from the history and how information is transferred between agents.
  To formalize these semantics, we follow  a natural
approach   for standpoint logics that
uses separate
transition systems $\cT_a$ describing the executions that are consistent with $a$'s standpoint  as in 
\mbox{\cite{SLTL-KR23,Complexity-SLTL-ECAI24}}, together with a main transition system $\cT$ modeling the actual system.
%
Unlike \cite{SLTL-KR23,Complexity-SLTL-ECAI24}, 
we assume the labels of the states in $\cT_a$ to be from 
a subset $P_a$ of the set $P$ of atomic propositions of $\cT$.
The difference
to the  use of indistinguishablity
relations $\sim_a$ on states of a single transition system for all agents $a \in \Ag$, common in epistemic temporal logics,
 is mostly
of syntactic nature as we will show by translations of families of 
indistinguishablity
relations $(\sim_a)_{a \in \Ag}$ on a  transition system to 
separate transition systems $(\cT_a)_{a \in \Ag}$,  and vice versa.

Under all five semantics, the intuitive meaning of $\StMod{a}{\varphi}$ is that
there is a state $s$ in $\cT_a$, which is one of the 
potential current states from agent $a$'s view of the history, 
and a path $\pi$ in $\cT_a$ from $s$ such that $\pi$ satisfies $\varphi$
when agent $a$ makes nondeterministic guesses for truth values of 
the atomic propositions outside $P_a$. Informally, the different semantics are as follows:

\noindent
  -- The \emph{step semantics} $\stepmodels$   agrees with the 
  semantics  proposed 
  in \cite{SLTL-KR23,Complexity-SLTL-ECAI24}. 
  It assumes that only
  the number of steps  performed in the past are accessible
  to the agents.

\noindent
  -- The \emph{pure observation-based semantics} $\probsmodels$ 
  is in the spirit of
  the perfect-recall LTLK (LTL extended with knowledge operators) semantics  of \cite{MC-LTLK-and-beyond-2024} 
  where  $a$ can access 
  exactly the truth values of
  the atomic propositions in $P_a$ from the history.

\noindent
  -- The \emph{public-history semantics} $\publicmodels$ can be seen
  as a perfect-recall variant of the LTLK semantics 
  where all agents have
  full access to the history.

\noindent
  -- The \emph{decremental semantics} $\decrmodels{}$ is a variant of
  $\probsmodels$ where 
  standpoint subformulas $\StMod{b}{\psi}$ 
  of a  formula $\StMod{a}{\varphi}$ 
  are interpreted from the view of agent $a$, when $a$ knows the transitions
  system of $b$, but can access only 
  the atomic propositions
  in $P_a \cap P_b$ to guess what agent $b$ knows from the history.

\noindent
  -- The \emph{incremental semantics} $\incrmodels{}$ is as $\decrmodels{}$, but under the assumption that standpoint subformulas $\StMod{b}{\psi}$ 
  of a standpoint formula $\StMod{a}{\varphi}$ 
  are interpreted from the view of the coalition $\{a,b\}$, i.e.,
  that $a$ can access
  the atomic propositions
  in $P_a \cup P_b$ to guess what agent $b$ knows from the history.  


The decremental and incremental semantics share ideas of distributed knowledge and the ``everybody knows'' operator of epistemic logics \cite{Reasoning-about-knowledge-MIT-2004}.

\paragraph*{Main contributions.} Besides introducing the four new semantics for $\StpLTL$ (Section \ref{sec:logic}), our main contributions are 

\noindent
--  a generic model-checking algorithm that is applicable for all five semantics
  (Section \ref{sec:model-checking})

\noindent
-- complexity-theoretic results for the model checking problem of $\StpLTL$ under the different semantics (Section \ref{sec:complexity}).
More precisely we show
PSPACE-completeness for full $\StpLTL$ under $\stepmodels$ and $\publicmodels$, and for $\StpLTL$ formulas of alternation depth 1 under $\pobsmodels$, $\decrmodels{}$ and $\incrmodels{}$. This stands in contrast to the EXPSPACE-completeness of the satisfiability problem for SLTL under the step semantics \cite{Complexity-SLTL-ECAI24}.
Furthermore, our results yield an EXPTIME upper bound for $\decrmodels{}$. The same holds for $\pobsmodels$ under the additional assumption that the $P_a$'s are pairwise disjoint.
We show that $\StpLTL$ under all five semantics can be embedded into LTLK. For the case of $\incrmodels{}$, the embedding yields an $(N{-}1)$-EXPSPACE upper bound where $N=|\Ag|$. For the case of $\pobsmodels$ and the $\StpLTL$ fragment of alternation depth at most $d$, the embedding into LTLK implies 
$(d{-}1)$-EXPSPACE membership.

While our algorithm relies on similar ideas as the
LTLK model-checking algorithm as in \cite{MC-LTLK-and-beyond-2024} 
(even for the richer logic CTL*K), it
 exploits the simpler nature of
$\StpLTL$ compared to LTLK and generates smaller
history-automata than those that would have been constructed when
applying iteratively 
the powerset constructions of \cite{MC-LTLK-and-beyond-2024}.
As such, our algorithm can be seen as an adaption of 
\cite{MC-LTLK-and-beyond-2024} that takes a more fine-grained approach for the different $\StpLTL$ semantics resulting in the 
different complexity bounds described above.


%%%%%%%%%%%%%%%%%%%%%%%%%%%%%%%%%%%%%%%%%%%%%%%%%%%%



Omitted proofs and details can be found in the appendix.




