% !TeX root = 0_main

\section{\toolname: An Issue Report Management Tool}
\label{sec:tool}

\subsection{Supported Issue Management Tasks}
\toolname is an issue management assistant for developers, project managers, computer science students, and educators. \toolname currently supports three issue management tasks.
%in (1) identifying similar issues, (2) predicting issue severity, and (3) locating potential buggy code files. 

\subsubsection{Issue Severity Prediction}
\toolname classifies the reported issues based on their severity level. After a user creates a new issue, the tool tags it with one of five labels: Blocker, Critical, Major, Minor, or Trivial. This helps project managers and developers prioritize the issues that require immediate attention.
\looseness=-1

\subsubsection{Similar Issue Identification}
\toolname suggests similar issues as soon as a new issue is submitted, tagging the new issue with a ``Duplicate” label. Similar issues are suggested in a comment in the issue report.  
%The reporter will observe similar issues instantly by \toolname's comment and label. 
This feature minimizes redundant issue management efforts by suggesting related issues to developers and reporters, who are meant to inspect the results to determine if the new issue was reported before.
\looseness=-1 

\subsubsection{Buggy Code Localization}
\toolname suggests potential buggy code files based on the textual similarity between the reported issue and the code file of the system's latest version. When a new issue is reported, \toolname fetches the files of the latest system version and ranks them as a list of potential buggy code files. This feature suggests users code files that might require modification to solve the issues.

% \antu{Update this description like: ``With this feature, \toolname suggests similar issues as soon as a new issue is reported if there are any and assign the new issue ``Duplicate” label in such cases. The reporter will see the duplicate issues instantly in the comment box and developers will know the issue is a duplicate of previous issues with the help of the label. This will help both reporters and developers by eliminating redundant efforts spent on duplicate issues."}
% \antu{Update the next two features in this way: What \toolname will do with this feature? How it will show the output? How it will assist developers?}
% -> Rephrased accordingly

\begin{figure}[t]
    \centering
    \includegraphics[width=0.95\columnwidth, keepaspectratio]{ui_sprint.png}
    \caption{\toolname's GUI with a Usage Scenario}
    \label{fig:example-image}
\end{figure}

\subsection{Usage Scenario \& Graphical User Interface}

%\subsubsection{Usage Scenario}
% \antu{This section can be summarized to save space. Rewrite this: tell the same story, be to the point, avoid redundancy.}
% -> reduced this a little
\toolname can be easily installed in any repository. A user only needs to visit the installation website~\cite{sprintUrl}, click the `Install' button, and select the repositories where the tool will be used without requiring any configuration. 
\looseness=-1
% \antu{This paragraph can be simplified in one sentence: ``As a GitHub App, \toolname can be installed in any GitHub repository without requiring any additional configurations."}
% -> I wrote this line first. Professor Oscar suggested writing in detail about how the installation is done.

After installation, when a user reports a new issue  (see \textcircled{1} in \cref{fig:example-image}), \toolname's first step is to fetch the issue's title and description, and the code files of the system's latest version. \toolname then generates comments with feedback to the user by analyzing this information using state-of-the-art models (described in \cref{sec:architecture}). 

First, \toolname analyzes the new issue and classifies it into one of five severity levels (Blocker, Critical, Major, Minor, or Trivial~\cite{severityTypes}), assigning a label with predicted severity \textcircled{2}. This label follows a color-coding format from red to yellow, where red suggests the issue is very severe and yellow indicates the issue is trivial. 
Second, \toolname analyzes the new issue and all the existing issues in the project, suggesting which of these are similar to the new issue. Then, it generates a comment with a list of suggested similar issues \textcircled{3}, each including its issue ID, title, and URL. If one or more similar issues are found, the new issue is labeled as ``Duplicate" \textcircled{4}.  Third, \toolname identifies potential buggy code files by analyzing the reported issue's information and the paths and names of the repository's code files.
\toolname generates a comment displaying the list of files (with URLs) that may need modification to solve the issue \textcircled{5}.
%A generated comment \textcircled{5} displays the list of potential code files that may need modification to solve the issue. This list of issues and potentially buggy files contain their corresponding URLs. 
%Developers can click and open the URLs to examine them more closely. 
\toolname's features are independent of one another: no feature is dependent on the execution of others.
\looseness=-1
% ~\ahmed{``however, \toolname's architecture allows project owners to specify a particular order if desired, before deploying/installing the tool.'' 
% is this line needed since we don't provide users to allow execute features in custom order}.
% \os{quick question, what if the issue is not a bug, but a new feature request or a question or an enhancement? what do the suggested files represent?}. -Ans, currently we guide the prompt to treat the reported issue as issues since we don't analyze the input issue to check whether it is a feature request/question or not. I guess this can be a future work.

\toolname suggests severity labels, similar issues, and potential buggy code files for the issues created after installation, whenever a new issue is submitted.  
%\toolname's features only work on the reported issues created after the installation. 
\toolname does not generate comments or labels for the issues existing before installation because these suggestions might conflict with comments and labels manually created by developers and reporters. 
\toolname currently handles the code files of the system's latest version. A future tool improvement is to identify the affected system version specified in the issue and perform bug localization on that version's code files.
%\toolname currently handles the code files of the system's latest version. Analyzing the issue for the specific version affected by the problem and processing that version is in our backlog of improvements to the tool.
% \antu{I don't understand this. Shouldn't it consider previous issues (prior installation) to suggest duplicate issues?}
% -> it considers the previous issues to suggest duplicates. But when the tool is installed, it does not analyze the existing issues for the 3 features. (for example, we don't create comments for them. This para seems a little misinterpreting. So, I have rephrased this again clearly.


% \os{increase the font size, add the issue title, and rephrase the issue description to be more readable and better formatted, \eg no {\textbackslash}n, etc.}

% \os{this figure looks really ugly (the content looks disproportionate) and extremely hard to read, we definitely need a good figure. We should show a single bug (i.e., a single figure) with all three features shown and labeled with the numbers. Fig 3 can be optional because it is not a feature of the tool, but of github}}

% \antu{I think we scaled the figures. Let's not do that. The figures look ugly. And we should create a better figure by adding the screenshots (side-by-side or so) to save space. For taking the screenshots, we can resize the browser as needed to show more info in a small place.}

%\subsubsection{Graphical User Interface (GUI)}
%\toolname comments and labels are automatically generated after an issue report is submitted. 
%%In \toolname's GUI, all the features are demonstrated independently without dependency on each other. 
%For the similar issue detection task, if any similar issue exists, a list of duplicate issue IDs, titles, and URLs is displayed in a comment along with a ``duplicate'' label. The user can visit duplicate issues by clicking on the respective URL. For the severity prediction feature, the tool creates a label for the severity. This label follows a color-coding format from red to yellow, where red suggests the issue is very severe and yellow indicates the issue is merely trivial. 
%%This label can help users to categorize issues based on severity and priority. 
%For the bug localization task, a list of potential buggy code files with URLs is shown in another comment so that a user can inspect the files further. 
% \os{question: what if the bug is not for the current version but for another version (e.g., V1.1 or 1.2: suppose these were created long ago. Can the tool support bug localization on a specific version described in the issue? the answer is no, but explain that this feature will be implemented in future versions of the tool)}

% \os{question: since a project may use different labels for severity, can the tool allow the project to customize the label? the answer is no, but explain how this can be achieved}.

% \antu{Let's follow the same order of the features. These two paragraphs can be merged and shortened by rewriting.}
% -> did rewriting and merged into 1 para

