\section{Related Work}
\label{sec:related_work}

Existing tools support individual issue management tasks. Find Duplicates~\cite{find_duplicates}, Probot~\cite{probot}, NextBug~\cite{rocha2015nextbug} are plugins for GitHub, Jira~\cite{jira}, and Bugzilla~\cite{bugzilla} that identify related issues. These tools rely on information retrieval or classical machine learning models that process issue text to determine issue similarity. Priority Scheduler~\cite{priorityScheduler} is a Jira~\cite{jira} plugin that assigns priority based on project deadlines. \rev{BugLocalizer~\cite{thung2014buglocalizer} is a Bugzilla~\cite{bugzilla} extension that analyzes bug reports and source code similarities to identify buggy files.} PR-Agent~\cite{pr-agent} is a multi-featured paid tool that applies LLMs, \eg\ ChatGPT~\cite{chatgpt}, to perform tasks such as pull request change classification, automatic code review, and documentation. \rev{Additionally, there are tools for reporting~\cite{song2023burt,song2022toward}, identifying bug report components~\cite{song2020bee}, and assessing bug reproduction steps~\cite{mahmud:icpc2025,chaparro2019assessing}}. Researchers have proposed automated techniques for duplicate issue detection~\cite{fang2023representthemall,zhang2023duplicate,rodrigues2020soft}, severity prediction~\cite{fang2023representthemall,kim2022bug,ali2024bert}, bug localization~\cite{longCodeArena,mahmud2024using}, issue categorization~\cite{somasundaram2012automatic,catolino2019not}, and other issue management tasks~\cite{chaitra2022bug,saha2024toward}. We selected state-of-the-art models, namely RTA~\cite{fang2023representthemall} and LongCodeArena~\cite{longCodeArena} for \toolname's three features, based on a rigorous literature review found in our replication package~\cite{repl_pack}.
\looseness=-1

%\os{add one or two sentences stating that there is research that have proposed automated techniques, including X, Y, Z (enumerate several tasks, beyond the tasks that we focus on). And then say that we selected  state of the art models, namely X, Y, Z (add citations), for the three tasks the tools supports, based on a rigorous literature review, which is found in our replication package. }
Compared to prior tools, \toolname stands out as an open-source, easy-to-install solution that seamlessly integrates with GitHub and consolidates multiple features, making it a comprehensive assistant for issue management.
%Compared to prior tools, \toolname stands out as an open-source, easy-to-install tool seamlessly integrated with GitHub and combining multiple features into a single solution. This makes \toolname a comprehensive assistant for issue report management. 
%\rev{Notably, the runtime bug localization feature is particularly novel, as no prior tools localize buggy code files instantly just by analyzing the issue report and repository state.} 
Moreover, \toolname leverages state-of-the-art models that have demonstrated superior performance compared to prior proposed approaches.
\looseness=-1

%We developed \toolname to address several gaps, including the lack of multi-featured open-source issue management tools, outdated duplicate detection solutions, the absence of open-source severity prediction tools, and no existing runtime bug localization tools for instant buggy code file identification.