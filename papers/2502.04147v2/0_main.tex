\documentclass[10pt,conference]{IEEEtran}
%\raggedbottom
%\IEEEoverridecommandlockouts
% The preceding line is only needed to identify funding in the first footnote. If that is unneeded, please comment it out.
\def\method{\text MixMin~}
\def\methodnospace{\text MixMin}
\def\genmethod{$\mathbb{R}$\text Min~}
\def\genmethodnospace{ $\mathbb{R}$\text Min}

% \renewcommand{\newline}{\vspace{0.15em}\\}
%\usepackage[normalem]{ulem} 
%\renewcommand{\ULdepth}{1.75pt} 
%\renewcommand{\ULthickness}{0.2pt} 

% \usepackage{hyperref}
% \usepackage[]{footmisc}
% \usepackage{cite}

% \def\BibTeX{{\rm B\kern-.05em{\sc i\kern-.025em b}\kern-.08em
%     T\kern-.1667em\lower.7ex\hbox{E}\kern-.125emX}}

\begin{document}

\title{\toolname: An Assistant for Issue Report Management
%\thanks{This work is supported by U.S. NSF grants CCF-2239107 and CCF-1955853. The opinions, findings, and conclusions expressed in this paper are those of the authors and do not necessarily reflect the sponsors' opinions.}
}


\author{
\IEEEauthorblockN{Ahmed Adnan}
\IEEEauthorblockA{
\textit{University of Dhaka}\\
Dhaka, Bangladesh \\
\href{mailto:}{bsse1131@iit.du.ac.bd}}
\and
\IEEEauthorblockN{Antu Saha}
\IEEEauthorblockA{
\textit{William \& Mary}\\
Williamsburg, Virginia, USA \\
\href{mailto:}{asaha02@wm.edu}}
\and
\IEEEauthorblockN{Oscar Chaparro}
\IEEEauthorblockA{
\textit{William \& Mary}\\
Williamsburg, Virginia, USA \\
\href{mailto:}{oscarch@wm.edu}}
}



% \author{
%     \begin{tabular}{@{\hskip 1.5cm}c@{\hskip 1.5cm}c@{\hskip 1.5cm}c@{\hskip 1.5cm}} % Increases column spacing to 2cm
%         Ahmed Adnan & Antu Saha & Oscar Chaparro \\ 
%         \textit{University of Dhaka (Bangladesh)} & \textit{William \& Mary (USA)} & \textit{William \& Mary (USA)} \\ 
%         \href{mailto:bsse1131@iit.du.ac.bd}{bsse1131@iit.du.ac.bd} & 
%         \href{mailto:asaha02@wm.edu}{asaha02@wm.edu} & 
%         \href{mailto:oscarch@wm.edu}{oscarch@wm.edu} \\ 
%     \end{tabular}
% }

%\author{
%	
%	\IEEEauthorblockN{Author 1\IEEEauthorrefmark{1}, Author 2\IEEEauthorrefmark{2}, Author 2\IEEEauthorrefmark{3}, ...}
%    \IEEEauthorblockA{\IEEEauthorrefmark{1}\textit{Affiliation 1}
%		    (Country), 
%		    \IEEEauthorrefmark{2}\textit{Affiliation 2}
%		    (Country), 
%		    \IEEEauthorrefmark{3}\textit{Affiliation 3}
%		    (Country)
%		    %\\email1, email2, email3
%    }
%    
%}


\maketitle

\begin{abstract}
Managing issue reports is essential for the evolution and maintenance of software systems. However, manual issue management tasks such as triaging, prioritizing, localizing, and resolving issues are highly resource-intensive for projects with large codebases and users. To address this challenge, we present \toolname, a GitHub application that utilizes state-of-the-art deep learning techniques to streamline issue management tasks. \toolname assists developers by: (i) identifying existing issues similar to newly reported ones, (ii) predicting issue severity, and (iii) suggesting code files that likely require modification to solve the issues. We evaluated \toolname using existing datasets and methodologies, measuring its predictive performance, and conducted a user study with five professional developers to assess its usability and usefulness. The results show that \toolname is accurate, usable, and useful, providing evidence of its effectiveness in assisting developers in managing issue reports. \toolname is an open-source tool available at \href{https://github.com/sea-lab-wm/sprint_issue_report_assistant_tool/}{github.com/sea-lab-wm/sprint\_issue\_report\_assistant\_tool}.
\end{abstract}

%\begin{IEEEkeywords}
%keyword1, 2, 3, ...
%\end{IEEEkeywords}

\section{Introduction}
Backdoor attacks pose a concealed yet profound security risk to machine learning (ML) models, for which the adversaries can inject a stealth backdoor into the model during training, enabling them to illicitly control the model's output upon encountering predefined inputs. These attacks can even occur without the knowledge of developers or end-users, thereby undermining the trust in ML systems. As ML becomes more deeply embedded in critical sectors like finance, healthcare, and autonomous driving \citep{he2016deep, liu2020computing, tournier2019mrtrix3, adjabi2020past}, the potential damage from backdoor attacks grows, underscoring the emergency for developing robust defense mechanisms against backdoor attacks.

To address the threat of backdoor attacks, researchers have developed a variety of strategies \cite{liu2018fine,wu2021adversarial,wang2019neural,zeng2022adversarial,zhu2023neural,Zhu_2023_ICCV, wei2024shared,wei2024d3}, aimed at purifying backdoors within victim models. These methods are designed to integrate with current deployment workflows seamlessly and have demonstrated significant success in mitigating the effects of backdoor triggers \cite{wubackdoorbench, wu2023defenses, wu2024backdoorbench,dunnett2024countering}.  However, most state-of-the-art (SOTA) backdoor purification methods operate under the assumption that a small clean dataset, often referred to as \textbf{auxiliary dataset}, is available for purification. Such an assumption poses practical challenges, especially in scenarios where data is scarce. To tackle this challenge, efforts have been made to reduce the size of the required auxiliary dataset~\cite{chai2022oneshot,li2023reconstructive, Zhu_2023_ICCV} and even explore dataset-free purification techniques~\cite{zheng2022data,hong2023revisiting,lin2024fusing}. Although these approaches offer some improvements, recent evaluations \cite{dunnett2024countering, wu2024backdoorbench} continue to highlight the importance of sufficient auxiliary data for achieving robust defenses against backdoor attacks.

While significant progress has been made in reducing the size of auxiliary datasets, an equally critical yet underexplored question remains: \emph{how does the nature of the auxiliary dataset affect purification effectiveness?} In  real-world  applications, auxiliary datasets can vary widely, encompassing in-distribution data, synthetic data, or external data from different sources. Understanding how each type of auxiliary dataset influences the purification effectiveness is vital for selecting or constructing the most suitable auxiliary dataset and the corresponding technique. For instance, when multiple datasets are available, understanding how different datasets contribute to purification can guide defenders in selecting or crafting the most appropriate dataset. Conversely, when only limited auxiliary data is accessible, knowing which purification technique works best under those constraints is critical. Therefore, there is an urgent need for a thorough investigation into the impact of auxiliary datasets on purification effectiveness to guide defenders in  enhancing the security of ML systems. 

In this paper, we systematically investigate the critical role of auxiliary datasets in backdoor purification, aiming to bridge the gap between idealized and practical purification scenarios.  Specifically, we first construct a diverse set of auxiliary datasets to emulate real-world conditions, as summarized in Table~\ref{overall}. These datasets include in-distribution data, synthetic data, and external data from other sources. Through an evaluation of SOTA backdoor purification methods across these datasets, we uncover several critical insights: \textbf{1)} In-distribution datasets, particularly those carefully filtered from the original training data of the victim model, effectively preserve the model’s utility for its intended tasks but may fall short in eliminating backdoors. \textbf{2)} Incorporating OOD datasets can help the model forget backdoors but also bring the risk of forgetting critical learned knowledge, significantly degrading its overall performance. Building on these findings, we propose Guided Input Calibration (GIC), a novel technique that enhances backdoor purification by adaptively transforming auxiliary data to better align with the victim model’s learned representations. By leveraging the victim model itself to guide this transformation, GIC optimizes the purification process, striking a balance between preserving model utility and mitigating backdoor threats. Extensive experiments demonstrate that GIC significantly improves the effectiveness of backdoor purification across diverse auxiliary datasets, providing a practical and robust defense solution.

Our main contributions are threefold:
\textbf{1) Impact analysis of auxiliary datasets:} We take the \textbf{first step}  in systematically investigating how different types of auxiliary datasets influence backdoor purification effectiveness. Our findings provide novel insights and serve as a foundation for future research on optimizing dataset selection and construction for enhanced backdoor defense.
%
\textbf{2) Compilation and evaluation of diverse auxiliary datasets:}  We have compiled and rigorously evaluated a diverse set of auxiliary datasets using SOTA purification methods, making our datasets and code publicly available to facilitate and support future research on practical backdoor defense strategies.
%
\textbf{3) Introduction of GIC:} We introduce GIC, the \textbf{first} dedicated solution designed to align auxiliary datasets with the model’s learned representations, significantly enhancing backdoor mitigation across various dataset types. Our approach sets a new benchmark for practical and effective backdoor defense.




% !TeX root = 0_main

\section{\toolname: An Issue Report Management Tool}
\label{sec:tool}

\subsection{Supported Issue Management Tasks}
\toolname is an issue management assistant for developers, project managers, computer science students, and educators. \toolname currently supports three issue management tasks.
%in (1) identifying similar issues, (2) predicting issue severity, and (3) locating potential buggy code files. 

\subsubsection{Issue Severity Prediction}
\toolname classifies the reported issues based on their severity level. After a user creates a new issue, the tool tags it with one of five labels: Blocker, Critical, Major, Minor, or Trivial. This helps project managers and developers prioritize the issues that require immediate attention.
\looseness=-1

\subsubsection{Similar Issue Identification}
\toolname suggests similar issues as soon as a new issue is submitted, tagging the new issue with a ``Duplicate” label. Similar issues are suggested in a comment in the issue report.  
%The reporter will observe similar issues instantly by \toolname's comment and label. 
This feature minimizes redundant issue management efforts by suggesting related issues to developers and reporters, who are meant to inspect the results to determine if the new issue was reported before.
\looseness=-1 

\subsubsection{Buggy Code Localization}
\toolname suggests potential buggy code files based on the textual similarity between the reported issue and the code file of the system's latest version. When a new issue is reported, \toolname fetches the files of the latest system version and ranks them as a list of potential buggy code files. This feature suggests users code files that might require modification to solve the issues.

% \antu{Update this description like: ``With this feature, \toolname suggests similar issues as soon as a new issue is reported if there are any and assign the new issue ``Duplicate” label in such cases. The reporter will see the duplicate issues instantly in the comment box and developers will know the issue is a duplicate of previous issues with the help of the label. This will help both reporters and developers by eliminating redundant efforts spent on duplicate issues."}
% \antu{Update the next two features in this way: What \toolname will do with this feature? How it will show the output? How it will assist developers?}
% -> Rephrased accordingly

\begin{figure}[t]
    \centering
    \includegraphics[width=0.95\columnwidth, keepaspectratio]{ui_sprint.png}
    \caption{\toolname's GUI with a Usage Scenario}
    \label{fig:example-image}
\end{figure}

\subsection{Usage Scenario \& Graphical User Interface}

%\subsubsection{Usage Scenario}
% \antu{This section can be summarized to save space. Rewrite this: tell the same story, be to the point, avoid redundancy.}
% -> reduced this a little
\toolname can be easily installed in any repository. A user only needs to visit the installation website~\cite{sprintUrl}, click the `Install' button, and select the repositories where the tool will be used without requiring any configuration. 
\looseness=-1
% \antu{This paragraph can be simplified in one sentence: ``As a GitHub App, \toolname can be installed in any GitHub repository without requiring any additional configurations."}
% -> I wrote this line first. Professor Oscar suggested writing in detail about how the installation is done.

After installation, when a user reports a new issue  (see \textcircled{1} in \cref{fig:example-image}), \toolname's first step is to fetch the issue's title and description, and the code files of the system's latest version. \toolname then generates comments with feedback to the user by analyzing this information using state-of-the-art models (described in \cref{sec:architecture}). 

First, \toolname analyzes the new issue and classifies it into one of five severity levels (Blocker, Critical, Major, Minor, or Trivial~\cite{severityTypes}), assigning a label with predicted severity \textcircled{2}. This label follows a color-coding format from red to yellow, where red suggests the issue is very severe and yellow indicates the issue is trivial. 
Second, \toolname analyzes the new issue and all the existing issues in the project, suggesting which of these are similar to the new issue. Then, it generates a comment with a list of suggested similar issues \textcircled{3}, each including its issue ID, title, and URL. If one or more similar issues are found, the new issue is labeled as ``Duplicate" \textcircled{4}.  Third, \toolname identifies potential buggy code files by analyzing the reported issue's information and the paths and names of the repository's code files.
\toolname generates a comment displaying the list of files (with URLs) that may need modification to solve the issue \textcircled{5}.
%A generated comment \textcircled{5} displays the list of potential code files that may need modification to solve the issue. This list of issues and potentially buggy files contain their corresponding URLs. 
%Developers can click and open the URLs to examine them more closely. 
\toolname's features are independent of one another: no feature is dependent on the execution of others.
\looseness=-1
% ~\ahmed{``however, \toolname's architecture allows project owners to specify a particular order if desired, before deploying/installing the tool.'' 
% is this line needed since we don't provide users to allow execute features in custom order}.
% \os{quick question, what if the issue is not a bug, but a new feature request or a question or an enhancement? what do the suggested files represent?}. -Ans, currently we guide the prompt to treat the reported issue as issues since we don't analyze the input issue to check whether it is a feature request/question or not. I guess this can be a future work.

\toolname suggests severity labels, similar issues, and potential buggy code files for the issues created after installation, whenever a new issue is submitted.  
%\toolname's features only work on the reported issues created after the installation. 
\toolname does not generate comments or labels for the issues existing before installation because these suggestions might conflict with comments and labels manually created by developers and reporters. 
\toolname currently handles the code files of the system's latest version. A future tool improvement is to identify the affected system version specified in the issue and perform bug localization on that version's code files.
%\toolname currently handles the code files of the system's latest version. Analyzing the issue for the specific version affected by the problem and processing that version is in our backlog of improvements to the tool.
% \antu{I don't understand this. Shouldn't it consider previous issues (prior installation) to suggest duplicate issues?}
% -> it considers the previous issues to suggest duplicates. But when the tool is installed, it does not analyze the existing issues for the 3 features. (for example, we don't create comments for them. This para seems a little misinterpreting. So, I have rephrased this again clearly.


% \os{increase the font size, add the issue title, and rephrase the issue description to be more readable and better formatted, \eg no {\textbackslash}n, etc.}

% \os{this figure looks really ugly (the content looks disproportionate) and extremely hard to read, we definitely need a good figure. We should show a single bug (i.e., a single figure) with all three features shown and labeled with the numbers. Fig 3 can be optional because it is not a feature of the tool, but of github}}

% \antu{I think we scaled the figures. Let's not do that. The figures look ugly. And we should create a better figure by adding the screenshots (side-by-side or so) to save space. For taking the screenshots, we can resize the browser as needed to show more info in a small place.}

%\subsubsection{Graphical User Interface (GUI)}
%\toolname comments and labels are automatically generated after an issue report is submitted. 
%%In \toolname's GUI, all the features are demonstrated independently without dependency on each other. 
%For the similar issue detection task, if any similar issue exists, a list of duplicate issue IDs, titles, and URLs is displayed in a comment along with a ``duplicate'' label. The user can visit duplicate issues by clicking on the respective URL. For the severity prediction feature, the tool creates a label for the severity. This label follows a color-coding format from red to yellow, where red suggests the issue is very severe and yellow indicates the issue is merely trivial. 
%%This label can help users to categorize issues based on severity and priority. 
%For the bug localization task, a list of potential buggy code files with URLs is shown in another comment so that a user can inspect the files further. 
% \os{question: what if the bug is not for the current version but for another version (e.g., V1.1 or 1.2: suppose these were created long ago. Can the tool support bug localization on a specific version described in the issue? the answer is no, but explain that this feature will be implemented in future versions of the tool)}

% \os{question: since a project may use different labels for severity, can the tool allow the project to customize the label? the answer is no, but explain how this can be achieved}.

% \antu{Let's follow the same order of the features. These two paragraphs can be merged and shortened by rewriting.}
% -> did rewriting and merged into 1 para




\section{Architecture}
\label{sec:architecture}

\begin{figure}
    \centering
    \includegraphics[width=0.92\linewidth]{graphs/arch.pdf}
    \caption{
        \sysname{} proposes an LLM-based no-code application development framework using FaaS for infrastructure abstraction.
        The prompt constructor combines a user's application description with a system prompt for an LLM that generates application code.
        The function deployer uses that code to deploy a FaaS function on a FaaS platform.
    }
    \label{fig:arch}
\end{figure}

LLMs are excellent tools for transforming natural language software descriptions into executable code, but are by themselves unable to deploy and operate that code for users.
FaaS platforms can deploy small pieces of code as scalable, managed applications.
With \sysname{}, we propose combining these two technologies into an end-to-end no-code application development platform.
Our goal is to let non-technical users, i.e., individuals without experience in software development or operation, provide application descriptions in natural language and build fully-managed applications from those descriptions.
Examples for such applications can be found, e.g., in the context of smart home automation, simple extensions of enterprise applications, or custom information aggregation from news websites, social media, and web APIs~\cite{paper_bermbach2020_webapibenchmarking2}.
To support such applications, we design \sysname{} as shown in \cref{fig:arch}.

\sysname{} comprises three main components: an LLM for generating user-specified code, a FaaS platform for efficient function deployment, and a bridge that orchestrates prompt construction and function deployment.
Users provide their natural language application descriptions to \sysname{}, which combines them with a static system prompt in a \emph{prompt constructor}.
This structured prompt instructs an LLM to generate code based on the natural language description, including, e.g., details on programming language, application context, API references, and runtime environment.
\sysname{} then parses the LLM's answer for code in a \emph{function deployer}.
This generated code is deployed on the FaaS platform which abstracts the underlying application infrastructure complexities by providing containerized, auto-scaling environments for on-demand execution.



\section{Experiments}
\label{sec:experiments}



\begin{figure*}[t]
    \centering
    \includegraphics[width=1\linewidth]{images/Environments.pdf} 
    % \vspace{-20pt}
    \captionsetup{
    width=\textwidth,
    font=Smallfont,
    labelfont=Smallfont,
    textfont=Smallfont
    }
    \captionsetup{
    width=\textwidth,
    font=Smallfont,
    labelfont=Smallfont,
    textfont=Smallfont
    }
    \caption{Four different real-world experiment environments.}
    \label{fig:environments}
    % \vspace{-6pt}
\end{figure*}

\begin{figure*}[t]
    \centering
    \captionsetup{
    width=\textwidth,
    font=Smallfont,
    labelfont=Smallfont,
    textfont=Smallfont
    }
    % Top-left subfigure
    \begin{subfigure}[b]{0.45\textwidth}
        \centering
        \includegraphics[width=\textwidth]{images/fig_office.pdf}
        \caption{Office}
        \label{fig:subfig1}
    \end{subfigure}
    \hspace{0.02\textwidth}
    % Top-right subfigure
    \begin{subfigure}[b]{0.45\textwidth}
        \centering
        \includegraphics[width=\textwidth]{images/fig_apt.pdf}
        \caption{Apartment}
        \label{fig:subfig2}
    \end{subfigure}

    \vskip\baselineskip

    % Bottom-left subfigure
    \begin{subfigure}[b]{0.45\textwidth}
        \centering
        \includegraphics[width=\textwidth]{images/fig_outdoor.pdf}
        \caption{Outdoor}
        \label{fig:subfig3}
    \end{subfigure}
    \hspace{0.02\textwidth}
    % Bottom-right subfigure
    \begin{subfigure}[b]{0.45\textwidth}
        \centering
        \includegraphics[width=\textwidth]{images/fig_hallway.pdf}
        \caption{Hallway}
        \label{fig:subfig4}
    \end{subfigure}

    \caption{Top-down view of the trajectories comparison on the value maps with the detection results across the four different environments.}
    \label{fig:value_map}
\end{figure*}


\begin{table*}[ht]
\captionsetup{
    width=\textwidth,
    font=Smallfont,
    labelfont=Smallfont,
    textfont=Smallfont
    }
\caption{Vision-language navigation performance in 4 unseen environments (SR and SPL).}
\label{SOTAResults}
\centering
\begin{tabular}{lcccc|cccc}
\toprule
\multirow{2}{*}{\textbf{Method}} & \multicolumn{4}{c}{\textbf{SR (\%)}} & \multicolumn{4}{c}{\textbf{SPL}} \\
\cmidrule(lr){2-5} \cmidrule(lr){6-9}
 & Hallway & Office & Apartment & Outdoor & Hallway & Office & Apartment & Outdoor \\
\midrule
\textbf{Frontier Exploration}  
  & 40.0 & 41.7 & 55.6 & 33.3  
  & 0.239 & 0.317 & 0.363 & 0.189 \\

\textbf{VLFM} \cite{yokoyama2024vlfm}                 
  & 53.3 & 75.0 & 66.7 & 44.4  
  & 0.366 & 0.556 & 0.412 & 0.308 \\

\textbf{VL-Nav w/o IBTP}      
  & 66.7 & 83.3 & \underline{70.2} & \underline{55.6}  
  & 0.593 & 0.738 & 0.615 & \underline{0.573} \\

\textbf{VL-Nav w/o curiosity}      
  & \underline{73.3} & \underline{86.3} & 66.7 & \underline{55.6}  
  & \underline{0.612} & \underline{0.743} & \underline{0.631} & 0.498 \\

\textbf{VL-Nav}               
  & \textbf{86.7} & \textbf{91.7} & \textbf{88.9} & \textbf{77.8}  
  & \textbf{0.672} & \textbf{0.812} & \textbf{0.733} & \textbf{0.637} \\

\bottomrule
\end{tabular}
\end{table*}








\subsection{Experimental Setting}
\label{sec:experimental_setting}

We evaluate our approach in real-robot experiments against five methods: (1) classical frontier-based exploration, (2) VLFM \cite{yokoyama2024vlfm}, (3) VLNav without instance-based target points, (4) VLNav without curiosity terms, and (5) the full VLNav configuration. Because the original VLFM relies on BLIP-2 \cite{li2023blip}, which is too computationally heavy for real-time edge deployment, we use the YOLO-World \cite{cheng2024yolo} model instead to generate per-observation similarity scores for VLFM. Each method is tested under the same conditions to ensure a fair comparison of performance.

\paragraph{Environments:}
We consider four distinct environments (shown in \fref{fig:environments}), each with a specific combination of semantic complexity (\textit{High}, \textit{Medium}, or \textit{Low}) and size (\textit{Big}, \textit{Mid}, or \textit{Small}). Concretely, we use a Hallway (\textit{Medium \& Big}), an Office (\textit{High \& Mid}), an Outdoor area (\textit{Low \& Big}), and an Apartment (\textit{High \& Small}). In each environment, we evaluate five methods using three language prompts, yielding a diverse range of spatial layouts and semantic challenges. This setup provides a rigorous assessment of each method’s adaptability.

\paragraph{Language-Described Instance:}
We define nine distinct, uncommon human-described instances to serve as target objects or persons during navigation. Examples include phrases such as “tall white trash bin,” “there seems to be a man in white,” “find a man in gray,” “there seems to be a black chair,” “tall white board,” and “there seems to be a fold chair.” The variety in these descriptions ensures that the robot must rely on vision-language understanding to accurately locate these targets.

\noindent\textbf{Robots and Sensor Setup:} 
All experiments are conducted using a four-wheel Rover equipped with a Livox Mid-360 LiDAR. The LiDAR is tilted by approximately 23 degrees to the front to achieve a $\pm 30$ degrees vertical FOV coverage closely aligned with the forward camera’s view. An Intel RealSense D455 RGB-D camera, tilted upward by 7 degrees to detect taller objects, provides visual observation, though its depth data are not used for positioning or mapping. LiDAR measurements are a primary source of mapping and localization due to their higher accuracy. The whole VL-Nav system runs on an NVIDIA Jetson Orin NX on-board computer.




\subsection{Main Results}
\label{sec:main_results}

We validate the proposed VL-Nav system in real-robot experiments across four distinct environments (\textit{Hallway}, \textit{Office}, \textit{Apartment}, and \textit{Outdoor}), each featuring different semantic levels and sizes. Building on the motivation articulated in~\sref{sec:intro}, we focus on evaluating VL-Nav’s ability to (1) interpret fine-grained vision-language features and conduct robust VLN, (2) explore efficiently in unfamiliar spaces across various environments, and (3) run in real-time on resource-constrained platforms. \fref{fig:value_map} presents a top-down comparison of trajectories and detection results on the value map.

\begin{figure*}[t]
    \centering
    \includegraphics[width=1\linewidth]{images/result_plot.pdf} 
    % \vspace{-20pt}
    \captionsetup{
    width=\textwidth,
    font=Smallfont,
    labelfont=Smallfont,
    textfont=Smallfont
    }
    \caption{Plots of performance in different environments sizes and semantic comlexities.}
    \label{fig:results}
    % \vspace{-6pt}
\end{figure*}

\paragraph{Overall Performance:}
As reported in Table~\ref{SOTAResults}, our full \textbf{VL-Nav} consistently obtains the highest Success Rate (SR) and Success weighted by Path Length (SPL) across all four environments. In particular, VL-Nav outperforms classical exploration by a large margin, confirming the advantage of integrating CVL spatial reasoning with partial frontier-based search rather than relying solely on geometric exploration.

\paragraph{Effect of Instance-Based Target Points (IBTP):}
We note a marked improvement when enabling IBTP: the variant without IBTP lags behind, particularly in complex domains like the \textit{Apartment} and \textit{Office}. As discussed in \sref{sec:method}, IBTP allows VL-Nav to pursue and verify tentative detections with confidence above a threshold, mirroring human search behavior. This pragmatic mechanism prevents ignoring possible matches to the target description and reduces overall travel distance to confirm or discard candidate objects.

\paragraph{Curiosity Contributions:}
The \emph{curiosity Score} is also significant to VL-Nav’s performance. It merges two key components:
\begin{itemize}
    \item \textbf{Distance Weighting}: Preventing easily select very far way goals to reduce travel time and energy consumption which is extremely important for the efficiency (metrics SPL) in the large-size environments.
    \item \textbf{Unknown-Area Weighting}: Rewards navigation toward regions that yield more information.
\end{itemize}
Our ablations reveal that removing the distance-scoring element (\textit{VL-Nav w/o curiosity}) degrades both SR and SPL, particularly in the more cluttered environments. Meanwhile, dropping the instance-based target points (IBTP) similarly lowers performance, reflecting how each piece of CVL addresses a complementary aspect of semantic navigation.

\paragraph{Comparison to VLFM:}
Although the VLFM approach \cite{yokoyama2024vlfm} harnesses vision-language similarity value, it lacks the pixel-wise vision-language features, instance-based target points verification mechanism, and CVL-based spatial reasoning. Consequently, VL-Nav surpasses VLFM in both SR and SPL by effectively combining the pixel-wise vision language features and the curiosity cues via the CVL spatial reasoning. These gains are especially pronounced in semantic complex (\textit{Apartment}) and open-area (\textit{Outdoor}) environments, underscoring how our CVL spatial reasoning enhance vision-language navigation in complex settings and scenarios.



\paragraph{Summary of Findings:}
In conclusion, the experimental results confirm that VL-Nav delivers superior vision-language navigation across diverse, unseen real-world environments. By fusing frontier-based target points detection, instance-based target points, and the CVL spatial reasoning for goal selection, VL-Nav balances semantic awareness and exploration efficiency. The system’s robust performance, even in large or cluttered domains, highlights its potential as a practical solution for zero-shot vision-language navigation on low-power robots.


% !TEX root = ../main.tex

\section{Related work}
\label{sec:related_work}

\subsection{Visual unsupervised anomaly localization}

% In recent years the creation of the MVTec AD benchmark~\cite{mvtec} has given impetus to the development of new methods for visual unsupervised anomaly detection and localization. We review several main approaches which have representatives among top-5 methods on the localization track of the MVTec AD leaderboard
% The MVTec AD benchmark~\cite{mvtec}, developed in recent years, has been instrumental in propelling research towards new methods in visual unsupervised anomaly detection and localization.
In this section, we review several key approaches, each represented among the top five methods on the localization track of the MVTec AD benchmark~\cite{mvtec}, developed to stir progress in visual unsupervised anomaly detection and localization. 
% \footnote{\url{https://paperswithcode.com/sota/anomaly-detection-on-mvtec-ad}}.
% \paragraph{Synthetic anomalies} In unsupervised setting, real anomalies are either not present or not labeled in the training images. Some methods~\cite{memseg,mood_top1}, however, propose synthetic procedures that corrupt random regions in the images and train a segmentation model to predict the corrupted regions' masks.

\paragraph{Synthetic anomalies.} In unsupervised settings, real anomalies are typically absent or unlabeled in training images. To simulate anomalies, researchers synthetically corrupt random regions by replacing them with noise, random patterns from a special set~\cite{memseg}, or parts of other training images~\cite{mood_top1}. A segmentation model is trained to predict binary masks of corrupted regions, providing well-calibrated anomaly scores for individual pixels. While straightforward to train, these models may overfit to synthetic anomalies and struggle with real ones.
% . Unlabeled real anomalies in training images cannot be included in the binary masks, leading the model to predict zero scores for these regions and resulting in false negatives.

% One limitation of this approach is that the models may overfit to synthetic anomalies and generalize poorly to real anomalies. Another limitation is that training images may contain real anomalies which are unlabeled and cannot be included in the training binary masks. Thus, segmentation model is trained to predict zero scores for these regions which leads to false negatives.

% \paragraph{Reconstruction-based} Reconstruction-based methods build a generative model that takes an image $x$ as input and generates its normal (anomaly-free) version $\hat{x}$. Then anomaly scores are obtained as pixel-wise reconstruction errors between $x$ and $\hat{x}$. SotA methods from this family, e.g. DRAEM~\cite{draem}, DiffusionAD~\cite{diffusionad}, POUTA~\cite{pouta}, present a combination of reconstruction-based and synthetic-based approaches. First, they train a generative model to reconstruct synthetically corrupted image regions. Then, they train a segmentation model that takes a corrupted image and its reconstructed version as input and predicts the mask of the corrupted regions.

\paragraph{Reconstruction-based.} 
% In reconstruction-based methods, anomaly scores are obtained as reconstruction errors between the input image $x$ and generated normal (anomaly-free) counterpart $\hat{x}$.
% Reconstruction-based methods build a generative model that takes an image $x$ as input and generates its normal (anomaly-free) version $\hat{x}$. Then anomaly scores are obtained as reconstruction errors between $x$ and $\hat{x}$.
Trained solely on normal images, reconstruction-based approaches~\cite{autoencoder, vae, fanogan}, poorly reconstruct anomalous regions, allowing pixel-wise or feature-wise discrepancies to serve as anomaly scores. Later generative approaches~\cite{draem, diffusionad, pouta} integrate synthetic anomalies. The limitation stemming from anomaly-free train set assumption still persists -- if anomalous images are present, the model may learn to reconstruct anomalies as well as normal regions, undermining the ability to detect anomalies through differences between $x$ and $\hat{x}$.
% Early approaches, such as Autoencoders~\cite{autoencoder} and Variational Autoencoders~\cite{vae}, are trained solely on normal images. During inference, these models poorly reconstruct anomalous regions, allowing pixel-wise squared errors ${(x - \hat{x})^2}$ to serve as anomaly scores. Methods like f-AnoGAN~\cite{fanogan} enhance this by training W-GAN~\cite{wgan} $g$ to generate normal images and an encoder $f$ to map images to the GAN's latent space, ensuring ${\hat{x} = g(f(x)) \approx x}$. Anomalies are detected using a weighted average of reconstruction errors in pixel space and discrepancies in feature maps from GAN discriminator.

% State-of-the-art methods such as DRAEM~\cite{draem}, DiffusionAD~\cite{diffusionad}, and POUTA~\cite{pouta} integrate synthetic anomalies into the reconstruction process. They first train a generative model (autoencoder / diffusion model) to reconstruct synthetically corrupted regions. Then, they train a segmentation model that takes both the corrupted image and its reconstruction as input to predict masks of the corrupted regions.

% A major limitation of reconstruction-based methods is the assumption that the training set contains only normal images. If anomalous images are present, the generative model may learn to reconstruct anomalies as well as normal regions, undermining the ability to detect anomalies through differences between $x$ and $\hat{x}$.

% The earliest methods from this family are based on Autoencoder~\cite{autoencoder} or Variational Autoencoder~\cite{vae}, which are trained on anomaly-free images. At the inference stage, when it takes an image $x$ with anomalies it is intended to badly reconstruct the anomalous regions in $\hat{x}$, so that pixel-wise squared errors $(x - \hat{x})^2$ can be used as anomaly scores.

% Another method, f-AnoGAN~\cite{fanogan} at the first step trains W-GAN~\cite{wgan}, consisting of generator $g$ and discriminator $d$, to generate anomaly-free images $x \sim g(z)$ from latent variables $z \sim \mathcal{N}(0, I)$. Then, at the second step, it trains encoder $f$ to map anomaly-free images $x$ to the GAN's latent space, s.t. $\hat{x} = g(f(x)) \approx x$. At the inference stage, when $x$ is anomalous image, generator is assumed to generate its anomaly-free version $\hat{x}$, as it is trained only on normal images. Anomaly score are then obtained as a weighted average of reconstruction errors $(x - \hat{x})^2$ in pixel space and squared differences $(\varphi_d(x) - \varphi_d(x'))^2$ between feature maps $\varphi_d(x)$ and $\varphi_d(x')$ taken intermediate layers of GAN discriminator $d$.

% The SotA reconstruction-based methods, e.g. DRAEM~\cite{draem}, DiffusionAD~\cite{diffusionad}, POUTA~\cite{pouta}, present a combination with the approach based on synthetic anomalies. First, they train a generative model, e.g. autoencoder~\cite{draem,pouta} or diffusion model~\cite{diffusionad}, to reconstruct synthetically corrupted image regions. Then, they train a segmentation model that takes a corrupted image and its reconstructed version as input and predicts the mask of the corrupted regions.

% The main limitation of reconstruction-based methods is that they assume that training set does not contain anomalous images. Otherwise, generative model may learn to reconstruct anomalous regions as well as normal ones, which does not allow to detect anomalies by comparison of $x$ and $\hat{x}$.

\paragraph{Density-based.} Density-based methods for anomaly detection model the distribution of the training image patterns. As modeling of the joint distribution of raw pixel values is infeasible, these methods usually model the marginal or conditional distribution of pixel-wise deep feature vectors.

Some methods~\cite{ttr, pni} perform a non-parametric density estimation using memory banks. More scalable flow-based methods~\cite{fastflow,cflow,msflow}, leverage normalizing flows to assign low likelihoods to anomalies. From this family, we selected MSFlow as a representative baseline, because it is simpler than PNI, and yields similar top-5 results on the MVTec AD. 


\subsection{Medical unsupervised anomaly localization}
While there's no standard benchmark for pathology localization on CT images, MOOD~\cite{mood} offers a relevant benchmark with synthetic target anomalies. Unfortunately, at the time of preparing this work, the benchmark is closed for submissions, preventing us from evaluating our method on it. We include the top-performing method from MOOD~\cite{mood_top1} in our comparison, that relies on synthetic anomalies.

Other recognized methods for anomaly localization in medical images are reconstruction-based: variants of AE / VAE~\cite{autoencoder, dylov}, f-AnoGAN~\cite{fanogan}, and diffusion-based~\cite{latent_diffusion}. These approaches highly rely on the fact that the the training set consists of normal images only. However, it is challenging and costly to collect a large dataset of CT images of normal patients. While these methods work acceptable in the domain of 2D medical images and MRI, the capabilities of the methods have not been fully explored in a more complex CT data domain. We have adapted these methods to 3D.


\section{Conclusions}\label{sec-conclusions}
We present Interaction-aware Conformal Prediction (ICP) to explicitly address the mutual influence between robot and humans in crowd navigation problems. We achieve interaction awareness by proposing an iterative process of robot motion planning based on human motion uncertainty and conformal prediction of the human motion dependent on the robot motion plan. Our crowd navigation simulation experiments show ICP strikes a good balance of performance among navigation efficiency, social awareness, and uncertainty quantification compared to previous works. ICP generalizes well to navigation tasks across different crowd densities, and its fast runtime and manageable memory usage indicates potential for real-world applications.

In future work, we will address infeasible robot planning solutions with adaptive failure probability and conduct real-world crowd navigation experiments to evaluate the effectiveness of ICP. As ICP is a task-agnostic algorithm, we would like to explore its applications in manipulation settings, such as collaborative manufacturing.

\begin{credits}
\subsubsection{\ackname} This work was supported by the National Science Foundation under Grant No. 2143435 and by the National Science Foundation under Grant CCF 2236484.
\end{credits}

%\section*{Acknowledgments}
We thank an anonymous Prolific user whose answer to \autoref{xhw_study::question::perception} inspired our paper's title.

Work on this paper was funded by the Deutsche Forschungsgemeinschaft (DFG, German Research Foundation) under Germany's Excellence Strategy---\href{https://casa.rub.de}{EXC 2092 CASA}---390781972, through the DFG grant 389792660 as part of \href{https://perspicuous-computing.science}{TRR~248}, and by the Volkswagen Foundation grants AZ~9B830, AZ~98509, and AZ~98514 \href{https://explainable-intelligent.systems}{\enquote{Explainable Intelligent Systems}} (EIS). 

The Volkswagen Foundation and the DFG had no role in preparation, review, or approval of the manuscript; or the decision to submit the manuscript for publication. 
The authors declare no other financial interests.

\balance
\bibliographystyle{IEEEtran}
\bibliography{references}



\end{document}
