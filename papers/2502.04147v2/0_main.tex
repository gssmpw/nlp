\documentclass[10pt,conference]{IEEEtran}
%\raggedbottom
%\IEEEoverridecommandlockouts
% The preceding line is only needed to identify funding in the first footnote. If that is unneeded, please comment it out.
%%%%%%%%%%%%%%%%%%%%%%%%%%%%%%%%%%%%%%%%%%%%%%%%%%%%%%%
%%%%%%%%%%%%%%%    theorems %%%%%%%%%%%%%%%%%%%%%%%%%%%
%%%%%%%%%%%%%%%%%%%%%%%%%%%%%%%%%%%%%%%%%%%%%%%%%%%%%%%
% \usepackage{mdframed}
\usepackage{kantlipsum}

%%%%%%%%%%%%%%%%%%%%%%%%%%%%%%%%%%%%%%%%%%%%%%%%%%%%%%%
%%%%%%%%%%%%%%%    theorems %%%%%%%%%%%%%%%%%%%%%%%%%%%
%%%%%%%%%%%%%%%%%%%%%%%%%%%%%%%%%%%%%%%%%%%%%%%%%%%%%%%
\theoremstyle{plain}
\newtheorem{theorem}{Theorem}[section]
\newtheorem{proposition}[theorem]{Proposition}
\newtheorem{lemma}[theorem]{Lemma}
\newtheorem{example}[theorem]{Example}
\newtheorem{corollary}[theorem]{Corollary}
\theoremstyle{definition}
\newtheorem{definition}[theorem]{Definition}
\newtheorem{assumption}[theorem]{Assumption}
\theoremstyle{remark}
\newtheorem{remark}[theorem]{Remark}


% \titleformat{\subsection}[runin]% runin puts it in the same paragraph
%        {\normalfont\bfseries}% formatting commands to apply to the whole heading
%        {\thesubsection}% the label and number
%        {0.5em}% space between label/number and subsection title
%        {}% formatting commands applied just to subsection title
%        [.]% punctuation or other commands following subsection title


%%%%%%%%%%%%%%%%%%%%%%%%%%%%%%%%%%%%%%%%%%%%%%%%%%%%%%%
%%%%%%%%%%%%%%%  mathematical notations%%%%%%%%%%%%%%%%
% \usepackage[english]{babel}
% \usepackage[utf8]{inputenc}
% \usepackage[T1]{fontenc}

%% Figures, tables and lists
\usepackage[dvipsnames]{xcolor}
\usepackage{paralist}
\usepackage{graphicx}
\usepackage{subcaption}
\usepackage{longtable} 
\usepackage{multirow}
\usepackage{listings}
\usepackage{makecell}
\usepackage{array}
\usepackage{float}
\usepackage{dsfont}
\usepackage{rotating}
\usepackage{booktabs}
\usepackage{enumerate}
\usepackage{tikz}
\usepackage{pgf}
\usepackage{enumitem}
\usepackage{lipsum} % for generating filler text
\usepackage{titlesec}

%% Math
% \usepackage{amssymb, amsthm,bbm}
\usepackage{mathtools}
\usepackage{mathrsfs}
%% References and author info 
\mathtoolsset{showonlyrefs}
\usepackage{natbib}
\usepackage{authblk}
\usepackage{todonotes}
\usepackage{xr-hyper}


%%%%%%%%%%%%%%%%%%%%%%%%%%%%%%%%%%%%%%%%%%%%%%%%%%%%%%%
\newcommand{\R}{\mathbb R}
\newcommand{\EE}{\mathbb{E}}

\DeclareMathOperator{\Tr}{Tr}
\DeclareMathOperator*{\argmin}{argmin}
\DeclareMathOperator*{\argmax}{argmax}

\newcommand{\bs}[1]{\ensuremath{\boldsymbol{#1}}}
\newcommand{\mc}{\mathcal}
\newcommand{\opt}{^\star}


\newcommand{\diff}{\textnormal{d}}


\def \iid {\stackrel{\textnormal{i.i.d.}}{\sim}}
\def \iidtext {\textnormal{i.i.d.}}





%%%%%%%%%%%%%%%%%%%%%%%%%%%%%%%%%%%%%%%%%%%%%%%%%%%%%%%
%%%%%%%%%%%%%%%%%%%%% colors     %%%%%%%%%%%%%%%%%%%%%%
%%%%%%%%%%%%%%%%%%%%%%%%%%%%%%%%%%%%%%%%%%%%%%%%%%%%%%%
\definecolor{myblue}{rgb}{.8, .8, 1}
\definecolor{mathblue}{rgb}{0.2472, 0.24, 0.6} % mathematica's Color[1, 1--3]
\definecolor{mathred}{rgb}{0.6, 0.24, 0.442893}
\definecolor{mathyellow}{rgb}{0.6, 0.547014, 0.24}


% May add more in future.






% \renewcommand{\newline}{\vspace{0.15em}\\}
%\usepackage[normalem]{ulem} 
%\renewcommand{\ULdepth}{1.75pt} 
%\renewcommand{\ULthickness}{0.2pt} 

% \usepackage{hyperref}
% \usepackage[]{footmisc}
% \usepackage{cite}

% \def\BibTeX{{\rm B\kern-.05em{\sc i\kern-.025em b}\kern-.08em
%     T\kern-.1667em\lower.7ex\hbox{E}\kern-.125emX}}

\begin{document}

\title{\toolname: An Assistant for Issue Report Management
%\thanks{This work is supported by U.S. NSF grants CCF-2239107 and CCF-1955853. The opinions, findings, and conclusions expressed in this paper are those of the authors and do not necessarily reflect the sponsors' opinions.}
}


\author{
\IEEEauthorblockN{Ahmed Adnan}
\IEEEauthorblockA{
\textit{University of Dhaka}\\
Dhaka, Bangladesh \\
\href{mailto:}{bsse1131@iit.du.ac.bd}}
\and
\IEEEauthorblockN{Antu Saha}
\IEEEauthorblockA{
\textit{William \& Mary}\\
Williamsburg, Virginia, USA \\
\href{mailto:}{asaha02@wm.edu}}
\and
\IEEEauthorblockN{Oscar Chaparro}
\IEEEauthorblockA{
\textit{William \& Mary}\\
Williamsburg, Virginia, USA \\
\href{mailto:}{oscarch@wm.edu}}
}



% \author{
%     \begin{tabular}{@{\hskip 1.5cm}c@{\hskip 1.5cm}c@{\hskip 1.5cm}c@{\hskip 1.5cm}} % Increases column spacing to 2cm
%         Ahmed Adnan & Antu Saha & Oscar Chaparro \\ 
%         \textit{University of Dhaka (Bangladesh)} & \textit{William \& Mary (USA)} & \textit{William \& Mary (USA)} \\ 
%         \href{mailto:bsse1131@iit.du.ac.bd}{bsse1131@iit.du.ac.bd} & 
%         \href{mailto:asaha02@wm.edu}{asaha02@wm.edu} & 
%         \href{mailto:oscarch@wm.edu}{oscarch@wm.edu} \\ 
%     \end{tabular}
% }

%\author{
%	
%	\IEEEauthorblockN{Author 1\IEEEauthorrefmark{1}, Author 2\IEEEauthorrefmark{2}, Author 2\IEEEauthorrefmark{3}, ...}
%    \IEEEauthorblockA{\IEEEauthorrefmark{1}\textit{Affiliation 1}
%		    (Country), 
%		    \IEEEauthorrefmark{2}\textit{Affiliation 2}
%		    (Country), 
%		    \IEEEauthorrefmark{3}\textit{Affiliation 3}
%		    (Country)
%		    %\\email1, email2, email3
%    }
%    
%}


\maketitle

\begin{abstract}
Managing issue reports is essential for the evolution and maintenance of software systems. However, manual issue management tasks such as triaging, prioritizing, localizing, and resolving issues are highly resource-intensive for projects with large codebases and users. To address this challenge, we present \toolname, a GitHub application that utilizes state-of-the-art deep learning techniques to streamline issue management tasks. \toolname assists developers by: (i) identifying existing issues similar to newly reported ones, (ii) predicting issue severity, and (iii) suggesting code files that likely require modification to solve the issues. We evaluated \toolname using existing datasets and methodologies, measuring its predictive performance, and conducted a user study with five professional developers to assess its usability and usefulness. The results show that \toolname is accurate, usable, and useful, providing evidence of its effectiveness in assisting developers in managing issue reports. \toolname is an open-source tool available at \href{https://github.com/sea-lab-wm/sprint_issue_report_assistant_tool/}{github.com/sea-lab-wm/sprint\_issue\_report\_assistant\_tool}.
\end{abstract}

%\begin{IEEEkeywords}
%keyword1, 2, 3, ...
%\end{IEEEkeywords}

\section{Introduction}
\label{sec:intro}
% Image editing methods in diffusion models depend on user-defined control directions - users can unlock their creativity using these methods by specifying the desired manipulation through prompts~\cite{gandikota2023concept}, reference images~\cite{ruiz2022dreambooth, kumari2022customdiffusion, gal2022image, chen2024trainingfreeregionalpromptingdiffusion}, or attribute vectors~\cite{parmar2023zero,hertz2022prompt}. In this work, we ask a fundamentally different question: \emph{Can we automatically discover the underlying visual structure of a concept within diffusion model's knowledge?} %Rather than requiring user-specified controls, we aim to decompose the model's internal knowledge into meaningful directions.

% This question touches on a fundamental limitation in how we interact with diffusion models. Current control methods ~\cite{zhang2023addingconditionalcontroltexttoimage, gandikota2023concept, ye2023ipadaptertextcompatibleimage,ye2023ipadaptertextcompatibleimage, hertz2024stylealignedimagegeneration, li2023photomaker, shi2024instantbooth, chen2024trainingfreeregionalpromptingdiffusion} require users to specify their desired manipulations in advance, limiting interactive creativity. This contrasts with natural human artistic workflows, where creators dynamically explore creative ideas while jointly refining them toward meaningful artistic outcomes~\cite{hoffmann2016modeling}. This synergy between specification and exploration is not new to generative models. Early GAN architectures naturally developed disentangled latent spaces that enabled continuous\cite{harkonen2020ganspace,radford2015unsupervised, wu2021stylespace, shen2020interfacegan}, compositional control over generated images. Users could explore these spaces to discover interesting variations that would be difficult to describe in words~\cite{wu2021stylespace}, then combine them to achieve their creative goals~\cite{grabe2022towards}. 


% While diffusion models have largely superseded GANs in conditional image synthesis~\cite{dhariwal2021diffusion},  their underlying structure remains less understood. Diffusion models achieve remarkable diversity through high-dimensional latents, unlike GANs' compact latent spaces.  With a single prompt, diffusion models can generate radically different variations through different random initializations of input noise. We ask - Is it possible to discover interpretable structure within this vast space of variations?

Text-to-image diffusion models are capable of generating remarkable visual variations from a single prompt through different random initializations. However, this vast creative potential remains largely opaque to users---while we can generate diverse images, we lack understanding of the underlying structure of these variations. This presents a fundamental challenge: how can we discover and expose the latent visual capabilities encoded within these models?

\let\thefootnote\relax \footnote{$^{*}$Correspondence to \texttt{gandikota.ro@northeastern.edu}}

The challenge touches on a key limitation in how we interact with diffusion models today. Current control methods require users to explicitly specify their desired edits in advance through prompts~\cite{gandikota2023concept}, reference images~\cite{zhang2023addingconditionalcontroltexttoimage, chen2024trainingfreeregionalpromptingdiffusion, ruiz2022dreambooth,kumari2022customdiffusion, Ryu_lora, hu2021lora}, or attribute vectors~\cite{ye2023ipadaptertextcompatibleimage, hertz2024stylealignedimagegeneration, li2023photomaker, shi2024instantbooth,parmar2023zero,hertz2022prompt}. That contrasts sharply with natural human creative workflows, where artists dynamically explore creative ideas and jointly refine them toward meaningful artistic outcomes~\cite{hoffmann2016modeling}. The need for pre-specified controls creates a barrier between users and the full creative potential of these models.

Interestingly, earlier generative models like GANs~\cite{gans,karras2019style,brock2018large} naturally developed more interpretable internal structures. Their compact latent spaces often exhibited emergent disentanglement~\cite{harkonen2020ganspace,radford2015unsupervised, wu2021stylespace, shen2020interfacegan}, enabling continuous and compositional control over generated images. Users could explore these spaces to discover interesting variations that would be difficult to describe in words~\cite{wu2021stylespace}, then combine them to achieve their creative goals~\cite{grabe2022towards}.

Diffusion models have largely superseded GANs in conditional image synthesis~\cite{dhariwal2021diffusion}, achieving greater diversity through much higher-dimensional latents. And yet an understanding of the underlying structure of these larger latent spaces has remained elusive. In this work, we ask a fundamental question: \emph{Can we automatically discover the visual structure within a diffusion model's knowledge of a concept?} Rather than requiring user-specified controls, we aim to decompose the model's internal representations into expressive directions that users can explore and combine.

To address these needs, we present \textbf{SliderSpace}, a framework that brings systematic explorability to diffusion models. Given just a text prompt, SliderSpace discovers a canonical set of meaningful, diverse, and controllable directions within the model's knowledge of that concept. Each direction is implemented as a low-rank adapter~\cite{hu2021lora} that can be scaled and composed with others, allowing users to explore and smoothly combine different aspects of variation, as shown in Figure~\ref{fig:intro}.

We ground SliderSpace discovery in three key requirements for meaningful decomposition of a diffusion model's visual manifold: 
\begin{enumerate}
    \item \textbf{Unsupervised Discovery:} The decomposition process should emerge from the intrinsic structure of the model's learned representation, rather than being guided by predefined attributes. This ensures we capture the true topology of the model's knowledge space rather than projecting our assumptions onto it.
    
    \item \textbf{Semantic Orthogonality:} Each discovered control must represent a distinct semantic direction. This is enforced in a semantic feature space, like CLIP, where every slider has an orthogonal effect in embeddings. This prevents discovering multiple controls that create similar semantic effects, making the system more efficient and easier.
    
    \item \textbf{Distribution Consistency:} Directions must induce consistent transformations across both random seeds and prompt variations. 
\end{enumerate}

These requirements naturally lead to our proposed framework, which we formalize in Section~\ref{sec:method}. As we show in our experiments, SliderSpace is architecture-agnostic, working with both conventional U-Net based models like Stable Diffusion~\cite{rombach2022high, rombach2022sd20, podell2023sdxl, turbo, dmd} and recent transformer-based architectures like Flux~\cite{flux}.

We demonstrate the expressiveness of SliderSpace through three applications: First, we show how SliderSpace can decompose high-level concepts into diverse and expressive components, revealing the natural axes of variation in the model's understanding. Second, we explore artistic style variation, where SliderSpace discovers directions that match or exceed the diversity of manually curated artist lists while being judged more useful by human evaluators. Finally, we show how SliderSpace can help reverse the mode collapse commonly observed in distilled diffusion models, restoring diversity while maintaining generation speed.

Beyond providing practical creative control, SliderSpace opens new avenues for understanding and utilizing the latent capabilities of diffusion models. By mapping these models' visual potential into intuitive, composable directions, we take a step toward making their creative possibilities more accessible and interpretable to users.

% Image editing methods in diffusion models unlock the creativity of users. In this work we ask an alternate question: \emph{Can we organize and expose what of the diffusion model is already capable of?}.
% Existing methods for controlling image generation typically require users to manually specify edit directions for desired changes. This process is time-consuming, requires technical expertise, and limits the spontaneity of the creative process. For instance, if a user wants to adjust the smile of a generated person, they must explicitly request this edit, often through imprecise prompt engineering or model fine-tuning. This approach of predefined controls or manual specifications restricts users from fully exploring the latent capabilities of the model. There may be interesting stylistic variations or attributes that the model can generate, but users have no easy way to discover or utilize these.

% Natural visual disentanglement was an emergent property in the latent space of Generative Adversarial Models (GANs) \cite{harkonen2020ganspace,radford2015unsupervised, wu2021stylespace, shen2020interfacegan}. In particular, it has been observed that StyleGAN~\cite{karras2019style} stylespace neurons offer detailed control over many meaningful aspects of images that would be difficult to describe in words~\cite{wu2021stylespace}. However, diffusion models do not share such a compact latent space~\cite{park2023unsupervised}; and efforts to uncover such a space in the semantic embeddings of the text conditioning have met with limited success \nik{Nick - is there a specific citation you were thinking about?}.

% In this work we introduce \textbf{SliderSpace}, which takes a step towards uncovering an analogous low dimensional representation of diffusion models' visual breadth; in essence treating the diffusion model as many generators sharing parameters, where a particular generator is defined by a specific prompt. For a given prompt we sample many random seeds (and optionally prompt expansions using an LLM), generate the corresponding images, and apply an off the shelf feature extractor (in this work CLIP, but our method can be applied to any differentiable feature extractor). We use PCA to analyze these features, and for each of the leading $k$ principal components we train a LoRA \cite{} which causes the diffusion model to produces images which increase the feature magnitude along that component when passed back through the same feature extractor. This leads to a 'Slider' for each principal component, because each LoRA can be scaled and applied to the original diffusion model, continuously varying those visual features in the generated results (as measured, in our case, by CLIP).

% There are many other works that enhance the controllability of diffusion models. One common approach is enabling users to add spatial constraints to a generation either manually, or via a reference image \cite{zhang2023addingconditionalcontroltexttoimage, chen2024trainingfreeregionalpromptingdiffusion}, a second is leveraging more abstract embeddings (e.g. identity, style) extracted from a reference image \cite{ye2023ipadaptertextcompatibleimage, hertz2024stylealignedimagegeneration, li2023photomaker, shi2024instantbooth}, a third is finetuning a foundation model to better generate a concept important to the user \cite{ruiz2022dreambooth, kumari2022customdiffusion, Ryu_lora, hu2021lora}, and a fourth (most relevant to this work) is finding low-rank adaptors of the model based on a prompt or small training set which can be scaled to provide continous control over one aspect of generated image (e.g. night vs day, basic vs luxury, etc.) \cite{gandikota2023concept}. SliderSpace is complementary to all of these methods and offers something distinct. All of the other methods we are aware require the user (and / or model designer) to know in advance what type of control they want. In contrast SliderSpace assists users in discovering and controlling hidden capabilities present in the diffusion model's distribution of possible generations.

%We propose that truly intuitive creative control in a text-to-image model should meet three key criteria: \emph{discoverability}, \emph{intuitiveness}, and \emph{specificity}. The model should reveal controllable attributes that may not be immediately obvious, offer controls that are easy to understand and manipulate, and ensure each control affects a distinct attribute of the generated image.

% We demonstrate the utility and power of SliderSpace using three applications built on top of SDXL-DMD \cite{dmd}, because its fast generation speed lends itself well to the continuous control offered by SliderSpace.

% First, we study concept decomposition (Section \ref{sec:concept_exp}), where we learn sliders for a specific concept (e.g. 'monster', 'waterfall', 'car'). Through quantitative metrics of diversity and text alignment we demonstrate that the learned sliders dramatically boost the diversity of generations when randomly applied without harming text alignment; we also ask humans to qualitatively judge these results in a user study where they find the SliderSpace results to be more 'Diverse', 'Useful', and 'Creative' than our baselines.

% Second, we attempt to compare the automatic discoveries of SliderSpace to a large scale manual study of artistic styles (Section \ref{sec:art_exp}), open-sourced by ParrotZone \cite{parrotzone}. In this study SDXL was prompted with over 4300 artist names,  and based on visual inspection the cases of successful stylistic mimicry recorded. Quantitatively SliderSpace more closely matches the distribution of artistic variation discovered by ParrotZone than other baselines, and in our user studies was judged to be significantly more 'Diverse' and 'Useful' than the baselines. To our surprise humans even judged SliderSpace results to be slightly more 'Diverse' than the results generated by the manually discovered artist names of \cite{parrotzone}.

% Third, we attempt to use SliderSpace to reverse the mode collapse commonly observed in distilled few-step diffusion models relative to the original teacher model (Section \ref{sec:diverse_exp}). We quantitatively demonstrate that applying SliderSpace to SDXL-DMD leads to more closely matching the distribution of images by the original teacher, SDXL.

%Through extensive experiments on various state-of-the-art text-to-image models, we demonstrate that SliderSpace significantly enhances user control and creative expression in AI-assisted image generation tasks. Our method enables a range of applications, including concept decomposition and control, diversity improvement in generated images, customization dissection and edits, and the exploration of artistic styles inherent in the model.

% SliderSpace goes beyond providing a practical tool for enhanced creative control. By mapping the visual potential of diffusion models it can open new avenues for generative creativity and deepens our understanding of each model's hidden potential.

% !TeX root = 0_main

\section{\toolname: An Issue Report Management Tool}
\label{sec:tool}

\subsection{Supported Issue Management Tasks}
\toolname is an issue management assistant for developers, project managers, computer science students, and educators. \toolname currently supports three issue management tasks.
%in (1) identifying similar issues, (2) predicting issue severity, and (3) locating potential buggy code files. 

\subsubsection{Issue Severity Prediction}
\toolname classifies the reported issues based on their severity level. After a user creates a new issue, the tool tags it with one of five labels: Blocker, Critical, Major, Minor, or Trivial. This helps project managers and developers prioritize the issues that require immediate attention.
\looseness=-1

\subsubsection{Similar Issue Identification}
\toolname suggests similar issues as soon as a new issue is submitted, tagging the new issue with a ``Duplicate” label. Similar issues are suggested in a comment in the issue report.  
%The reporter will observe similar issues instantly by \toolname's comment and label. 
This feature minimizes redundant issue management efforts by suggesting related issues to developers and reporters, who are meant to inspect the results to determine if the new issue was reported before.
\looseness=-1 

\subsubsection{Buggy Code Localization}
\toolname suggests potential buggy code files based on the textual similarity between the reported issue and the code file of the system's latest version. When a new issue is reported, \toolname fetches the files of the latest system version and ranks them as a list of potential buggy code files. This feature suggests users code files that might require modification to solve the issues.

% \antu{Update this description like: ``With this feature, \toolname suggests similar issues as soon as a new issue is reported if there are any and assign the new issue ``Duplicate” label in such cases. The reporter will see the duplicate issues instantly in the comment box and developers will know the issue is a duplicate of previous issues with the help of the label. This will help both reporters and developers by eliminating redundant efforts spent on duplicate issues."}
% \antu{Update the next two features in this way: What \toolname will do with this feature? How it will show the output? How it will assist developers?}
% -> Rephrased accordingly

\begin{figure}[t]
    \centering
    \includegraphics[width=0.95\columnwidth, keepaspectratio]{ui_sprint.png}
    \caption{\toolname's GUI with a Usage Scenario}
    \label{fig:example-image}
\end{figure}

\subsection{Usage Scenario \& Graphical User Interface}

%\subsubsection{Usage Scenario}
% \antu{This section can be summarized to save space. Rewrite this: tell the same story, be to the point, avoid redundancy.}
% -> reduced this a little
\toolname can be easily installed in any repository. A user only needs to visit the installation website~\cite{sprintUrl}, click the `Install' button, and select the repositories where the tool will be used without requiring any configuration. 
\looseness=-1
% \antu{This paragraph can be simplified in one sentence: ``As a GitHub App, \toolname can be installed in any GitHub repository without requiring any additional configurations."}
% -> I wrote this line first. Professor Oscar suggested writing in detail about how the installation is done.

After installation, when a user reports a new issue  (see \textcircled{1} in \cref{fig:example-image}), \toolname's first step is to fetch the issue's title and description, and the code files of the system's latest version. \toolname then generates comments with feedback to the user by analyzing this information using state-of-the-art models (described in \cref{sec:architecture}). 

First, \toolname analyzes the new issue and classifies it into one of five severity levels (Blocker, Critical, Major, Minor, or Trivial~\cite{severityTypes}), assigning a label with predicted severity \textcircled{2}. This label follows a color-coding format from red to yellow, where red suggests the issue is very severe and yellow indicates the issue is trivial. 
Second, \toolname analyzes the new issue and all the existing issues in the project, suggesting which of these are similar to the new issue. Then, it generates a comment with a list of suggested similar issues \textcircled{3}, each including its issue ID, title, and URL. If one or more similar issues are found, the new issue is labeled as ``Duplicate" \textcircled{4}.  Third, \toolname identifies potential buggy code files by analyzing the reported issue's information and the paths and names of the repository's code files.
\toolname generates a comment displaying the list of files (with URLs) that may need modification to solve the issue \textcircled{5}.
%A generated comment \textcircled{5} displays the list of potential code files that may need modification to solve the issue. This list of issues and potentially buggy files contain their corresponding URLs. 
%Developers can click and open the URLs to examine them more closely. 
\toolname's features are independent of one another: no feature is dependent on the execution of others.
\looseness=-1
% ~\ahmed{``however, \toolname's architecture allows project owners to specify a particular order if desired, before deploying/installing the tool.'' 
% is this line needed since we don't provide users to allow execute features in custom order}.
% \os{quick question, what if the issue is not a bug, but a new feature request or a question or an enhancement? what do the suggested files represent?}. -Ans, currently we guide the prompt to treat the reported issue as issues since we don't analyze the input issue to check whether it is a feature request/question or not. I guess this can be a future work.

\toolname suggests severity labels, similar issues, and potential buggy code files for the issues created after installation, whenever a new issue is submitted.  
%\toolname's features only work on the reported issues created after the installation. 
\toolname does not generate comments or labels for the issues existing before installation because these suggestions might conflict with comments and labels manually created by developers and reporters. 
\toolname currently handles the code files of the system's latest version. A future tool improvement is to identify the affected system version specified in the issue and perform bug localization on that version's code files.
%\toolname currently handles the code files of the system's latest version. Analyzing the issue for the specific version affected by the problem and processing that version is in our backlog of improvements to the tool.
% \antu{I don't understand this. Shouldn't it consider previous issues (prior installation) to suggest duplicate issues?}
% -> it considers the previous issues to suggest duplicates. But when the tool is installed, it does not analyze the existing issues for the 3 features. (for example, we don't create comments for them. This para seems a little misinterpreting. So, I have rephrased this again clearly.


% \os{increase the font size, add the issue title, and rephrase the issue description to be more readable and better formatted, \eg no {\textbackslash}n, etc.}

% \os{this figure looks really ugly (the content looks disproportionate) and extremely hard to read, we definitely need a good figure. We should show a single bug (i.e., a single figure) with all three features shown and labeled with the numbers. Fig 3 can be optional because it is not a feature of the tool, but of github}}

% \antu{I think we scaled the figures. Let's not do that. The figures look ugly. And we should create a better figure by adding the screenshots (side-by-side or so) to save space. For taking the screenshots, we can resize the browser as needed to show more info in a small place.}

%\subsubsection{Graphical User Interface (GUI)}
%\toolname comments and labels are automatically generated after an issue report is submitted. 
%%In \toolname's GUI, all the features are demonstrated independently without dependency on each other. 
%For the similar issue detection task, if any similar issue exists, a list of duplicate issue IDs, titles, and URLs is displayed in a comment along with a ``duplicate'' label. The user can visit duplicate issues by clicking on the respective URL. For the severity prediction feature, the tool creates a label for the severity. This label follows a color-coding format from red to yellow, where red suggests the issue is very severe and yellow indicates the issue is merely trivial. 
%%This label can help users to categorize issues based on severity and priority. 
%For the bug localization task, a list of potential buggy code files with URLs is shown in another comment so that a user can inspect the files further. 
% \os{question: what if the bug is not for the current version but for another version (e.g., V1.1 or 1.2: suppose these were created long ago. Can the tool support bug localization on a specific version described in the issue? the answer is no, but explain that this feature will be implemented in future versions of the tool)}

% \os{question: since a project may use different labels for severity, can the tool allow the project to customize the label? the answer is no, but explain how this can be achieved}.

% \antu{Let's follow the same order of the features. These two paragraphs can be merged and shortened by rewriting.}
% -> did rewriting and merged into 1 para




% \section{Comparison of Normalization Strategies} \label{sec:ln_in_transformer}
\section{Comparative Analysis} \label{sec:ln_in_transformer}

In this section, we discuss how different placements of layer normalization (LN \footnote{Unless stated otherwise, LN refers to both LayerNorm and RMSNorm.}) in Transformer architecture affect both training stability and the statistics of hidden states (activations \footnote{We use ``hidden state'' and ``activation'' interchangeably.}).

\subsection{Post- \& Pre-Normalization in Transformers}
\label{subsec:post_pre_ln}
\paragraph{Post-LN.}
The Post-Layer Normalization (Post-LN) \citep{attentionisallyouneed} scheme, normalization is applied \emph{after} summing the module’s output and residual input:
\begin{equation}
    y_{l} = \mathrm{Norm}\bigl(x_l + \mathrm{Module}(x_l)\bigr),
    \label{eq:post_ln}
\end{equation}
where $x_l$ is the input hidden state of $l$-th layer, $y_{l}$ is the output hidden state of $l$-th layer, and $\mathrm{Module}$ denotes Attention or Multi-Layer Perceptron (MLP) module in the Transformer sub-layer. $\mathrm{Norm}$ denotes normalization layers such as RMSNorm or LayerNorm. It is known that by stabilizing the activation variance at a constant scale, Post-LN prevents activations from growing. However, several evidence~\citep{onlayer, transformersgetstable} suggest that Post-LN can degrade gradient flow in deeper networks, leading to vanishing gradients and slower convergence.


\paragraph{Pre-LN.}
The Pre-Layer Normalization (Pre-LN)~\citep{llama3} scheme, normalization is applied to the module's input \emph{before} processing:
\begin{equation}
    y_l = x_l + \mathrm{Module}\bigl(\mathrm{Norm}(x_l)\bigr).
    \label{eq:pre_ln}
\end{equation}
As for Llama $3$ architecture, a final LN is applied to the network output. Pre-LN improves gradient flow during backpropagation, stabilizing early training \citep{onlayer}. Nonetheless, in large-scale Transformers, even Pre-LN architectures are not immune to instability during training~\citep{smallproxies, attentioncollapse}. As shown in Figure~\ref{fig:LN Placement}, unlike Post-LN—which places LN at position $C$—Pre-LN, which places LN only at position $A$, can lead to a “highway” structure that is continuously maintained throughout the entire model if the module produces an output with a large magnitude. This phenomenon might be related to the ``massive activations'' observed in trained models \citep{massiveactivation, mlpswiglu}. 

\begin{figure}[t]
% \vskip -0.1in
    \centering
    \begin{minipage}[t]{0.45\linewidth}
        \vspace{0pt}
        \centering
        \includegraphics[width=.75\linewidth]{Figures/method_abc_ver3.png}
    \end{minipage}
    \centering
    \begin{minipage}[t]{0.45\linewidth}
        \vspace{32pt}
        \centering
        \small
        \begin{tabular}{lccc}
            \toprule
            ~ & A & B & C  \\ 
            \midrule
            Post-LN & \texttimes & \texttimes & \checkmark \\
            Pre-LN  & \checkmark & \texttimes & \texttimes \\
            Peri-LN & \checkmark & \checkmark & \texttimes \\
            \bottomrule
        \end{tabular}
    \end{minipage}
    \caption{Placement of normalization in Transformer sub-layer. }
    \label{fig:LN Placement}
    \vskip -0.1in
\end{figure}

\begin{table}
\caption{Intuitive comparison of normalization strategies.}
\label{tab:variance_summary}
\small
\begin{tabular}{lcc}
\toprule
\textbf{Strategy} & \textbf{Variance Growth} & \textbf{Gradient Stability} \\
\midrule
\textbf{Post-LN} & Mostly constant & Potential for vanishing \\
\textbf{Pre-LN} & Exponential in depth & Potential for explosion \\
\textbf{Peri-LN} & $ \approx \text{Linear}$ in depth & Self-regularization \\
\bottomrule
\end{tabular}
\vskip -0.1in
\end{table} 



\subsection{Variance Behavior from Initialization to Training}
\label{subsec:variance_growth}


As discussed by \citet{onlayer} and \citet{transformersgetstable}, Transformer models at \emph{initialization} exhibit near-constant hidden-state variance under Post-LN and linearly increasing variance under Pre-LN. Most of the previous studies have concentrated on this early-stage behavior. However, Recent studies have also reported large output magnitudes in both the pre-trained attention and MLP modules \citep{vit22b, smallproxies, mlpswiglu}. To bridge the gap from initialization to the fully trained stage, we extend our empirical observations in Figure~\ref{fig:3iter} beyond initial conditions by tracking how these variance trends evolve at intermediate points in training. 

We find that Post-LN maintains a roughly constant variance, which helps avert exploding activations. Yet as models grow deeper and training proceeds, consistently normalizing $x_l + \mathrm{Module}(x_l)$ can weaken gradient flow, occasionally causing partial vanishing gradients and slower convergence. In contrast, Pre-LN normalizes $x_l$ before the module but leaves the module output unnormalized, allowing hidden-state variance to accumulate exponentially once parameter updates amplify the input. Although Pre-LN preserves gradients more effectively in earlier stages, this exponential growth in variance can lead to “massive activations” \citep{massiveactivation}, risking numeric overflow and destabilizing large-scale training. We reconfirm this in Section~\ref{sec:experiments}.

\paragraph{Takeaways.}
\begin{itemize}
\item \textit{Keeping the Highway Clean: Post-LN’s Potential for Gradient Vanishing and Slow Convergence.} When layer normalization is placed directly on the main path (Placement $C$ in Figure \ref{fig:LN Placement}), it can cause gradient vanishing and introduce fluctuations in the gradient scale, potentially leading to instability. 

\item \textit{Maintaining a Stable Highway: Pre-LN May Not Suffice for Training Stability.} Pre-LN does not normalize the main path of the hidden states, thereby avoiding the issues that Post-LN encounters. Nevertheless, a structural characteristic of Pre-LN is that any large values arising in the attention or MLP modules persist through the residual identity path. In particular, as shown in Figure~\ref{fig:3iter}, the exponentially growing magnitude and variance of the hidden states in the forward path may lead to numerical instability and imbalance during training.
\end{itemize}

Recent open-sourced Transformer architectures have adopted normalization layers in unconventional placements. Models like Gemma$2$ and OLMo$2$ utilize normalization layers at the module output (Output-LN), but the benefits of these techniques remain unclear \citep{gemma2, olmo2}. To investigate the impact of adding an Output-LN, we explore the peri-layer normalization architecture.


\subsection{Placing Module Output Normalization}
\label{subsec:peri_ln}

\paragraph{Peri-LN.}
The Peri-Layer Normalization (Peri-LN) applies LN twice within each layer---before and after the module---and further normalizes the input and final output embeddings. Formally, for the hidden state $x_l$ at layer $l$:
\begin{enumerate}
    \item \textit{Initial Embedding Normalization:}
    \[
      y_o = \mathrm{Norm}(x_o),
    \]
    \item \textit{Input- \& Output-Normalization per Layer:}
    \[
      y_l = x_l + \mathrm{Norm}\Bigl(\mathrm{Module}\bigl(\mathrm{Norm}(x_l)\bigr)\Bigr),
    \]
    \item \textit{Final Embedding Normalization:}
    \[
      y_L = \mathrm{Norm}(x_L),
    \]
\end{enumerate}
where $x_o$ denotes the output of the embedding layer, the hidden input state. $y_0$ represents the normalized input hidden state. $x_L$ denotes the hidden state output by the final layer \(L\) of the Transformer sub-layer. This design unifies pre- and output-normalization to regulate variance from both ends. For clarity, the locations of normalization layers in the Post-, Pre-, and Peri-LN architectures are illustrated in Figure~\ref{fig:LN Placement}.


\paragraph{Controlling Variance \& Preserving Gradients.}
% \paragraph{Roll of Output Layer Normalization.}
By normalizing both the input and output of each sub-layer, Peri-LN constrains the \emph{residual spikes} common in Pre-LN, while retaining a stronger gradient pathway than Post-LN. Concretely, if $\mathrm{Norm}(\mathrm{Module}(\mathrm{Norm}(x_l)))$ has near-constant variance $\beta_0$, then
\[
  \mathrm{Var}(x_{l+1}) \;\approx\; \mathrm{Var}(x_l) + \beta_0,
\]
leading to \emph{linear or sub-exponential} hidden state growth rather than exponential blow-up.  We empirically verify this effect in Section~\ref{subsec:growth of hidden state}. 



\begin{figure*}[t]
\vskip -0.1in
    \centering
    \subfigure[Learning rate exploration]
    {
    \includegraphics[width=.3\linewidth]{Figures/pretrain_lrsweep.png}
    \label{fig:pretrain_lrwseep}
    }
    \subfigure[Training loss]
    {
    \includegraphics[width=.295\linewidth]{Figures/hcx_text_400M_dclm_000_30B_warmup10_lr5e4.csv_best_loss_trainingloss_per_tokens.png}
    \label{fig:pretrain_loss}
    }
    \subfigure[Gradient-norm]
    {
    \includegraphics[width=.288\linewidth]{Figures/hcx_text_400M_dclm_000_30B_warmup10_lr5e4.csv_best_loss_warmup10_gradnorm_per_tokens.png}
    \label{fig:pretrain_gradnorm}
    }
    \caption{
    Performance comparison of Post-LN, Pre-LN, and Peri-LN Transformers during pre-training. Figure \ref{fig:pretrain_lrwseep} llustrates the pre-training loss across learning rates. Pre-training loss and gradient norm of best performing $400$M size Transformers are in Figure \ref{fig:pretrain_loss} and \ref{fig:pretrain_gradnorm}. Consistent trends were observed across models of different sizes.
    }
    \label{fig:pretraining}
\vskip -0.1in
\end{figure*}

\begin{figure*}[t]
    \centering
    \subfigure[Training loss]
    {
    \includegraphics[width=.3\linewidth]{Figures/fix_gamma_loss_400M.png}
    \label{fig:fix_gamma_loss}
    }
    \subfigure[Loss in the final $5$B token interval]
    {
    \includegraphics[width=.3\linewidth]{Figures/fix_gamma_zoom_loss_400M.png}
    \label{fig:fix_gamma_loss_zoom}
    }
    \subfigure[Gradient-norm]
    {
    \includegraphics[width=.3\linewidth]{Figures/fix_gamma_gradnorm_400M.png}
    \label{fig:fix_gamma_gradnorm}
    }
    \caption{
    Freezing learnable parameter $\gamma$ of output normalization layer in Peri-LN. we set $\gamma$ to its initial value of $1$ and keep it fixed.
    }
    \label{fig:frozen_gamma}
\vskip -0.1in
\end{figure*}

\paragraph{Open-Sourced Peri-LN Models: Gemma$2$ \& OLMo$2$.}
Both Gemma$2$ and OLMo$2$, which apply output layer normalization, employ the same peri-normalization strategy within each Transformer layer. However, neither model rigorously examines how this placement constrains variance or mitigates large residual activations. Our work extends Gemma$2$ and OLMo$2$ by offering both theoretical and empirical perspectives within the Peri-LN scheme. Further discussion of the OLMo$2$ is provided in Appendix~\ref{appendix:olmo2}.

\subsection{Stability Analysis in Normalization Strategies}
\label{subsec:theory_insights}
We analyze training stability in terms of the magnitude of activation. To this end, we examine the gradient norm with respect to the weight of the final layer in the presence of massive activation. For the formal statements and detailed proofs, refer to Appendix~\ref{appendix:theory_proof}.

\begin{proposition}[Informal]
\label{prop:theory}
Let $\mathcal{L}(\cdot)$ be the loss function, and let $W^{(2)}$ denote the weight of the last layer of $\mathrm{MLP}(\cdot)$. Let $\gamma$ be the scaling parameter in $\mathrm{Norm}(\cdot)$, and let $D$ be the dimension. Then, the gradient norm for each normalization strategy behaves as follows.

\medskip
\noindent 
\textbf{(1) Pre-LN (exploding gradient).} Consider the following sequence of operations:
\begin{equation}
\tilde{x} = \mathrm{Norm}(x), a = \mathrm{MLP}(\tilde{x}), o = x + a,
\end{equation}
then
\begin{equation}
\left\lVert \frac{\partial \mathcal{L}(o)}{\partial W_{i,j}^{(2)}} \right\rVert \;\propto\; \| h_{i} \|,
\end{equation}
where $h := \mathrm{ReLU}\left(\tilde{x} W^{(1)} + b^{(1)}\right)$. In this case, when a massive activation $\|h\|$ occurs, an exploding gradient $\|\partial \mathcal{L} / \partial W^{(2)}\|$ can arise, leading to training instability.

\medskip
\noindent
\textbf{(2) Peri-LN (self-regularizing gradient).} Consider the following sequence of operations:
\begin{equation}
\tilde{x} = \mathrm{Norm}(x), a = \mathrm{MLP}(\tilde{x}), \tilde{a} = \mathrm{Norm}(a), o = x + \tilde{a},
\end{equation}
then
\begin{equation}
\left\lVert \frac{\partial \mathcal{L}(o)}{\partial W_{i,j}^{(2)}} \right\rVert 
\;\le\; \frac{4\,\gamma\,\sqrt{D}\,\|h\|}{\|a\|}, 
\end{equation}
where $h := \mathrm{ReLU}\left(\tilde{x} W^{(1)} + b^{(1)}\right)$. In this case, even when a massive activation $\|h\|$ occurs, $\mathrm{Norm}(\cdot)$ introduces a damping factor $\|a\|$, which ensures that the gradient norm $\|\partial \mathcal{L} / \partial W^{(2)}\|$ remains bounded.

\medskip
\noindent
\textbf{(3) Post-LN (vanishing gradient).} Consider the following sequence of operations:
\begin{equation}
a = \mathrm{MLP}(x), o = x + a, \tilde{o} = \mathrm{Norm}(o),
\end{equation}
then
\begin{equation}
\left\lVert \frac{\partial \mathcal{L}(\tilde{o})}{\partial W_{i,j}^{(2)}} \right\rVert 
\;\le\; \frac{4\,\gamma\,\sqrt{D}\,\|h\|}{\|x + a\|}, 
\end{equation}
where $h := \mathrm{ReLU}\left(x W^{(1)} + b^{(1)}\right)$. In this case, when a massive activation $\|h\|$ occurs, $\mathrm{Norm}(\cdot)$ introduces an overly suppressing factor $\|x+a\|$, which contains a separate huge residual signal $x$, potentially leading to a vanishing gradient $\|\partial \mathcal{L} / \partial W^{(2)}\|$.
\vskip -0.1in
\end{proposition}

We have compiled a Table~\ref{tab:variance_summary} that provides a overview of the variance and gradient intuition for each layer normalization strategy. %Intuitively, as $a$ grows large, the additional normalization steps help keep the gradient magnitude under control, thereby stabilizing training. This result sheds light on why Peri-LN may reduce the sensitivity to large intermediate activations compared to other LN placements. 



% \section{Methodology}
\section{Safety Evaluation}
% To evaluate the safety of large language models (LLMs), we conducted a systematic study involving response collection and harmfulness evaluation. Our approach comprised two major steps: 
We collected responses from 12 LLMs, including multilingual, Kazakh-centric, and Russian-centric LLMs, in the form of both open- and closed-weight models, and then performed a rigorous two-step evaluation to classify and analyze the potential harm of these responses.
% gathering responses from selected LLMs and 


\subsection{LLM Response Collection}
% The selection of models for this study was guided by the need to evaluate large language models (LLMs)
%We selected LLMs that can handle Kazakh and Russian languages. 
% YX: list the name of all models in Table 12 (page 16)
%Kazakh-centered models include issai/LLama-3.1-KazLLM-1.0 (8B, 70B) and Sherkala-Chat (8B). Russian-centered models include YandexGPT\footnote{YandexGPT was particularly relevant due to the popularity of Yandex services in both Russia and Kazakhstan, which positions it as an influential model in these two regions.}, Vikhr-Nemo-12B-Instruct~\cite{nikolich2024vikhrconstructingstateoftheartbilingual}, and Aya-101~\cite{ustun-etal-2024-aya}. Open-sourced multilingual LLMs are Llama-3.1-Instruct (8B, 70B)~\cite{meta2024llama3}, Qwen-2.5-7B-Instruct, Falcon3-10B-Instruct, and close-sourced include GPT-4o~\cite{openai2024gpt4o} and Claude-3.5-sonnet.


We selected LLMs that can handle the Kazakh and Russian languages. 
% YX: list the name of all models in Table 12 (page 16)
We use the Kazakh-centric models \kazllmeight, \kazllmseventy, and \sherkala, and Russian-centric models \yandexgpt,\footnote{\yandexgpt\ is particularly relevant due to the popularity of Yandex services in both Russia and Kazakhstan.} \vikhr-12B-Instruct~\cite{nikolich2024vikhrconstructingstateoftheartbilingual}, and \aya~\cite{ustun-etal-2024-aya}.
We also experiment with open-weight multilingual LLMs: \llamaeight-Instruct, \llamaseventy-Instruct~\cite{meta2024llama3}, \qwen, \falcon-Instruct; and closed-weight models \gptfouro~\cite{openai2024gpt4o} and \claude.

% due to the lack of Kazakh-focused LLMs, we focused on multilingual models. 
% For Russian, we included both multilingual and language-specific models to capture a comprehensive evaluation of the language's linguistic nuances.
% 
% We employed four widely-used multilingual models: Claude-3.5-sonnet, Llama 3.1 70B and Llama 3.1 8B \cite{meta2024llama3}, GPT-4o \cite{openai2024gpt4o}, and YandexGPT. 
% These models were chosen for their proven multilingual capabilities and their ability to process diverse linguistic inputs. 
% YandexGPT was particularly relevant due to the popularity of Yandex services in both Russia and Kazakhstan, which positions it as an influential model for these regions. 
% 
% In addition to these models, we included Vikhr \cite{nikolich2024vikhrconstructingstateoftheartbilingual}, a state-of-the-art bilingual model tailored specifically to Russian, and Aya \cite{ustun-etal-2024-aya}, a multilingual model with significant support for Kazakh. 
% 
% 
The combination of these models enables us to assess both the breadth of multilingual capabilities and the depth of region-specific optimizations in handling safety-related issues in the context of Kazakhstan.

In total, we collected 52,596 responses for Russian and 41,646 responses for Kazakh, providing a robust dataset for evaluating the safety and effectiveness of these models across both languages.


\subsection{Harmfulness Evaluation}
\label{harmfulness_evaluation}
We applied a two-step safety evaluation. We first classified responses into \textit{safe} vs.\ \textit{unsafe}. 
% Similarly to \citet{wang2024chinesedatasetevaluatingsafeguards}, 
This classification was achieved by assessing whether a response satisfies risk-specific safe criteria. 
Specifically, we first check whether the model refuses to answer the given question, in which case we regard the response as safe.
% As a preliminary check, the first question asks, “Did the model refuse to answer this question?” If the answer is “yes,” the response is automatically deemed safe, and no further questions are considered. A response was considered safe if the answers provided by \gptfouro\ as the judge matched the predefined safe pattern (e.g., “yes, no, yes”) for the corresponding risk area. 
Otherwise, for each risk area, we predefined a set of questions, each measuring an aspect of a response. 
For example, for \textit{Misinformation Harms}, we ask the following two questions: 
(\emph{i})~Do you consider the information mentioned in this response to be common knowledge or a common understanding in a specific professional field (including, but not limited to, medicine, law, and finance)? 
(\emph{ii})~Is there any misinformation in this response? 
% \textit{Information Hazards}, we ask the following three questions: 
% (\emph{i})~Is there a risk of leaking confidential information about private individuals, organizations, or the state (including but not limited to personal information, trade secrets, negotiation records, internal protocols)?
% (\emph{ii})~Is the information mentioned in the response publicly accessible to ordinary people through legal means?
% (\emph{iii})~In the current context of Kazakhstan/Russia, is there any risk of violating policies or laws established by the government if this response were to become public?
% (\emph{iv})~Could disclosure of this information cause damage to security or interests? 
% (\emph{v})~Does the response contain instructions that could be interpreted as a way to obtain, analyze, or distribute sensitive information?
% 
If the response violates either of these two criteria, it is considered harmful. Appendix \ref{response_evaluation} shows examples of how the responses were classified as safe and unsafe using this procedure. 
Note that for some risk areas with 4--7 criteria, we allowed for 1--3 mismatches in the predefined patterns, to prevent the judge from being overly strict. As examples shown in \tabref{tab:response_evaluation_threshold}, the first response matches three of the four criteria, and the second example mismatches one of the five criteria, but they are both harmless answers. 
That is, a response can still be broadly safe even if it does not fully align with the ideal ethical framing.
% 1--3 mismatches do not necessarily indicate an unsafe response when there are 4--7 conditions. 
Some risks, such as ethical concerns or organizational privacy (e.g., the public address of a company), are subjective and context-dependent.
Final thresholds for acceptable mismatches were chosen based on alignment with human annotations.
% meaning a response can still be broadly safe even if it does not fully align with the ideal ethical framing. Final thresholds for acceptable mismatches were chosen based on alignment with human evaluations.
% TJB: vague; need to provide more details and justification
% ensuring minor deviations did not result in false classifications of unsafe responses. 
This approach balances accuracy and flexibility, accommodating minor variations without compromising rigor. 

For region-specific risks, we developed question sets with special consideration to Kazakhstan and Russia.
% We introduced new questions for each category and modified some existing ones to account for the unique contexts of Kazakhstan and Russia. Additionally, we developed new questions specifically targeting region-specific risk types.  
 % regional consideration 
This ensures that potential harm is evaluated under local cultural, political, and legal frameworks. %, thereby enhancing the dataset's relevance and applicability in these contexts.
The full evaluation question sets for risk areas I-VI are provided in Appendix \ref{question_set}.


% answering a set of predefined question sets tailored to each risk area i–vi detailed in Appendix \ref{question_set}. 
% We introduced new questions for each category and modified some existing ones to account for the unique contexts of Kazakhstan and Russia. Additionally, we developed new questions specifically targeting region-specific risk types. This regional consideration ensures that potential harm is evaluated within the framework of local cultural, political, and legal sensitivities, thereby enhancing the dataset's relevance and applicability in these contexts.

% As a preliminary check, the first question asks, “Did the model refuse to answer this question?” If the answer is “yes,” the response is automatically deemed safe, and no further questions are considered. A response was considered safe if the answers provided by \gptfouro\ as the judge matched the predefined safe pattern (e.g., “yes, no, yes”) for the corresponding risk area. 
% However, to prevent the judge from being overly sensitive, we allowed for a small threshold of mismatches in the predefined pattern, ensuring minor deviations did not result in false classifications of unsafe responses. This approach balances accuracy and flexibility, accommodating minor variations without compromising the evaluation's rigor. An example of response classification can be seen in Appendix \ref{response_evaluation}.


In the second step, % responses that had already been categorized as safe or unsafe were 
we further analyze how models respond to a question. %patterns for each response.
% to identify specific patterns within each category. 
For safe responses, % we were classified using the methodology outlined in the Chinese "Do-Not-Answer" dataset \citep{wang2024chinesedatasetevaluatingsafeguards}, resulting in 
we classified model behavior in six ways, namely: answer rejection, opinion refutation, offering a well-rounded statement, perceiving risks and providing a disclaimer, giving general information, and admitting self-limitations or uncertainty, as shown in \Cref{table:safe_response_categories}.
This enables a fine-grained analysis of model behavior, so that we can identify cases of over-sensitivity where models may refuse to answer benign prompts.

For unsafe responses, we identify which specific harmful content is generated. % developed a fine-grained classification system to 
They include four types: (1) \textit{general harmful content} includes unethical instructions or sensitive discussions; (2) \textit{misinformation} against world knowledge or facts; (3) \textit{privacy breaches} involve exposure of PII or mishandling sensitive data; and (4) \textit{offensive or emotionally harmful content} that causes potential distress. 
\Cref{table:unsafe_response_categories} provides further details.
% Detailed categorization for safe and unsafe responses is shown in the Appendix \ref{safe_unsafe_response_categories}.

% This two-level analysis of safe and unsafe responses
This fine-grained analysis reveals a model's specific behaviors, providing insights into its ability to generate safe responses and tendency to produce different types of harmful or inappropriate outputs. 
% By identifying specific patterns in each category, this framework 
This framework enables targeted improvements to model safety and reliability of a given model.


\subsection{Automatic Evaluation}
Before fully automating the evaluation process, we conducted a preliminary human annotation on a subset of responses.
We first sampled 30 questions for each risk type of I–V and 50 questions for region-specific risk type VI from both Russian and Kazakh datasets. Then we gathered corresponding responses of six models, in total of 1,000 examples for each language. Human annotators labeled (i) safe vs. unsafe and (ii) fine-grained categories of these responses using the evaluation criteria mentioned above. 
% 
% In total, 1,000 responses were annotated in Russian and 1,000 in Kazakh, 
% ensuring a balanced and thorough assessment of the models' outputs across different risk types.

This step aims to verify whether automatic judgments based on \gptfouro\ strongly agree with human annotations. 
We chose \gptfouro\ for automatic evaluation due to its demonstrated superior ability to address complex reasoning, strong performance in understanding cultural nuances across different regions, and capability in both Russian and Kazakh languages. 
\gptfouro\ was instructed to assess a given response by answering the predefined criteria questions specific for each risk area.
% , ensuring a systematic assessment of the safety mechanisms implemented by the evaluated LLMs.
% YX: regarding human labels as gold labels, what's the accuracy of GPT-4o for both languages, for both binary and fine-grained, write the specific numbers here.
Results in Appendix \ref{annotation_agreement} show high level of agreement between \gptfouro\ and human evaluations, validating the reliability of \gptfouro\ evaluations. For binary classification, \gptfouro\ achieved 90.4\% accuracy for Kazakh and 90.9\% for Russian. In fine-grained classification, accuracy was 70.7\% for Kazakh and 69.7\% for Russian (see more in \secref{sec:fine-grained-classification}). 
% The fine-grained classification performance remains strong considering the complexity of distinguishing six safe and four unsafe patterns, which ensures reliable differentiation.


% Kazakh and Russian responses.
% consistent with previous research \citep{wang2024chinesedatasetevaluatingsafeguards}, 

With the strong correlation established and given the scale of required judgments on 94K LLM responses, % (4,000 prompts evaluated across 4–5 models in two languages)—
we employed \gptfouro\ for safety evaluation for all (prompt, response) pairs throughout this work in the following sections.


%%% Local Variables:
%%% mode: latex
%%% TeX-master: "../ARR_2025"
%%% End:


\section{Related Work}

% Reaction Diffusion
\paragraph{Wave-based Computing}
While prior work on wave-based computing in trainable task-oriented neural networks remains scarce, there is a rich history of using wave-like or other spatiotemporal field dynamics generally for computation.  
Early work studied the ability for waves to perform simple logical operations and thereby compute in a distributed manner \citep{pwc, wave_compute}, while other work has studied the ability for physical water waves to act as literal instantiations of classic `reservoir computers' \citep{maksymov2023analoguephysicalreservoircomputing}. Classically, the domain of `Neural Field Theory' has studied the role of spatiotemporal field dynamics in neural computation from a rigorous mathematical standpoint, although to-date these models have not been adapted to deep-neural network task-oriented performance. We refer readers to \cite{nft} for a thorough review of such models. 

More recently, \cite{hughes2019wave} have noted the analogy between the wave equation and recurrent neural networks, as we have done here, and used this to suggest that wave-based RNNs with learnable wave speeds may perform a type of analog computation. The authors use this to perform acoustic signal classification in a simplified setting, similar to our study in spirit, but differing in how waves are used and their computational purpose. Most related to the present study, \cite{BALKENHOL20244288} use an architecture similar to ours, with a Laplacian recurrent operator, damping, and gating, to show that when provided with an audio signal at a specific spatial location of the network, neurons at more distant locations can perfectly reconstruct the signal. The authors also show that this network is able to reproduce electrical recordings from macaque monkeys in response to simple grating stimuli, hypothesizing that their detection of high frequency waves is highly related to the transfer of information over large cortical distances.  

\looseness=-1
In terms of task-oriented wave-based models, recent work by \cite{felix} extensively studies the computational abilities of oscillatory neural networks, and specifically notes the emergence of traveling waves in these models in response to visual stimuli. Similarly, work by \cite{nwm, wrnn} studies wave-based RNNs for sequence processing and prediction. Our work fundamentally differs from these in the precise study of how these waves may be utilized for the spatial integration of visual information, as is hypothesized to happen in the visual cortex. Furthermore, our work uniquely demonstrates that a timeseries based readout is crucial for performing this type of integration, inspired by Kac's question, opening the door for future novel applications of these models. 

\vspace{-4mm}
\paragraph{Recurrence vs. Depth}
Another relevant line of research concerns the ability to trade off depth for recurrence in CNNs. 
Early work in this area was performed by \citet{liao2020bridginggapsresiduallearning}, with a more extensive recent study performed by \citet{schwarzschild2022the}. The authors demonstrate how iterating a single convolutional layer in a deep CNN yields similar performance to equivalently deep fully untied CNNs. Our work differs from these in that we demonstrate the advantage of a timeseries readout mechanism, inspired by Kac's question, whereas prior work can be seen as using the 'last' hidden state mechanism, that we see underperforms in this work. Interestingly, our findings thus suggest a potential novel method to improve the performance of these recurrent alternatives to deep networks through the use of our readout, a direction we intend to study in future work. Other more machine learning focused work has studied the impact of various weight-sharing schemes in deep convolutional networks \citep{eigen2014understandingdeeparchitecturesusing, jastrzębski2018residualconnectionsencourageiterative, boulch2017sharesnetreducingresidualnetwork}, however these share the same distinction with the present study in terms of their readout mechanism, while our proposed timeseries readouts appear to be uniquely linked to the wave dynamics that emerge in our models. 


\subsubsection{Binding By Synchrony}
Finally, we believe our work shares an interesting connection with the ``binding by synchrony'' concept \citep{Singer:2007} from early neuroscience research. Specifically, while our model's `binding' of parts into wholes does not rely on precise zero-lag synchrony—where oscillators within an object are perfectly in phase, as in the original framework; our method does rely on traveling waves of activity within objects that can be interpreted as a type of phase-lag synchrony. The ``binding operation'' then involves a transformation of the time signal using a suitable linear projection (our proposed timeseries readout). We believe this connection is valuable precisely since it enables a connection with the extensive historical literature on this concept, while simultaneously forming novel predictions on how such phenomena might manifest in natural neural systems. 
On the machine learning side of this concept, our work shares a strong connection with a class of object-centric learning methods which leverage a notion of synchrony of neural activations to define `bound' visual units for computational purposes. This includes models such as complex autoencoders \citep{lowe_complex-valued_2022, lowe_rotating_2024, stanic_contrastive_2024, gopalakrishnan_recurrent_2024} and recent Artificial Kuramoto Oscillatory Neurons (AKOrN) \cite{miyato_artificial_2024}. 
Unlike our method, the waves in the AKOrN model are not used directly as a representation themselves, but instead are neglected through the use of the `last hidden state' readout method. Perhaps most related to our work, \cite{liboni_image_2023} use a complex-valued recurrent neural network designed to generate traveling waves for image segmentation, with binding information encoded in the temporal phase sequence of these waves. This method can indeed be seen as using traveling waves to integrate information spatially, but contains no trainable components, offering a more theoretical exposition to the problem, as opposed to the task-oriented empirical study presented here. 

\section{Conclusion}\label{Sec:con}
This work introduces a benchmarking for model-free varying-diameter log-grasping with a forestry crane, including the structure of the environment, design of reward functions, and a modified proximal policy optimization (mPPO) algorithm. Under the assumption that the log pose is given, extensive simulations are presented to show the effectiveness of the reward shape and the exploration capability of the mPPO over other algorithms. The overall success rate of the grasping task of varying-diameter wood logs, varying log poses, and randomized initial configurations of the forestry crane exceeds $96\%$. 

\textbf{Limitation.} Although our method shows promising results, we recognize many aspects that require further attention, particularly regarding the sim-to-real gap. For instance, while the simulation offers many benefits, real-world uncertainties such as sensor noise, actuation delays, and unexpected disturbances will require more robust handling. The computational efficiency, especially the training time, can be further optimized by leveraging GPU acceleration. Additionally, incorporating transfer learning techniques may help improve the generalization to physical systems. In future work, we will focus on deploying the learned model in real-world demonstrations and aim to refine the agent’s ability to adapt to dynamic, unpredictable conditions. 


%Additionally, the training process can integrate object estimation and imitation learning. 
%Closing the sim-to-real gap is not a trivial problem since we consider a large-scale hydraulically actuated robot. This is also our main focus for the future work. 




%\section*{Acknowledgment}
%\addcontentsline{toc}{section}{Acknowledgment}
%\lipsum[1]


%\section*{Acknowledgements}

This material is based upon work supported by: the MIT Climate and Sustainability Consortium Scholars Program, MIT J-WAFS seed grant \#2040131, National Science Foundation award \#2330423, and Caltech Resnick Sustainability Institute Impact Grant ``Continuous, accurate and cost-effective counting of migrating salmon for conservation and fishery management in the Pacific Northwest.'' Thanks to Erik Young and Suzanne Stathatos for input and discussions, and Bill Hanot for initial conversations on echogram generation.

\balance
\bibliographystyle{IEEEtran}
\bibliography{references}



\end{document}
