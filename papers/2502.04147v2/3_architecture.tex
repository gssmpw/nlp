% !TeX root = 0_main
\section{\toolname’s Architecture \& Implementation}
\label{sec:architecture}

\subsection{Architecture}
\toolname's architecture, shown in \cref{fig:architecture}, consists of three main components: (1) the  Issue Indexer, (2) the GitHub Event Listener, and (3) the Issue Management Components.
\subsubsection{Issue Indexer}
When \toolname is installed in one or more repositories, this component fetches all the existing issues from those repositories using GitHub Webhooks~\cite{webhookDoc} and stores them in a local relational database. This component applies page-based indexing to partition issues into manageable groups for efficient fetching. The database is meant to facilitate quick access and analysis of issues. This database is permanently synchronized with GitHub to provide 24/7 support since new issues are reported continuously and concurrently.
% \os{how is it synchronized? every day? every hour? every minute?}.
% \os{explain why the indexer uses a local database to index the issues (ie. why is this needed?). Also, how does the database synchronize with github, as new issues may reported continuously?} 
\subsubsection{GitHub Event Handler}
This component is responsible for listening to and handling the GitHub repository events when a new issue is submitted. This component fetches a newly reported issue along with the latest version's code files and sends them to the Issue Management Components for further analysis. After the analysis, this component processes the feedback, formats it appropriately, and posts the generated comments and labels to the reported issue as the final output.
% \os{does the controller have a listener or something that receives
% a request from Github whenever a new issue is submitted? is
% ”Controller” a good name? Controller sounds more general, like
% it controls all the communication between SPRINT and github,
% but this is not true, because the indexer also communicates
% with gibhut and it seems the controller only handles the event
% when a new issue is created and runs the issue management
% models} -ans: I renamed it with GitHub Event Handler and made some edits in the text

\begin{figure}[t]
    \centering
    \includegraphics[width=0.5\textwidth]{architecture.PNG} 
    \caption{Overview of \toolname's Architecture}
    \label{fig:architecture}
\end{figure}
% \os{the text is hard to read, increase the font size and the boxes' size (there is a lot of blank space; we should take advantage of it). Also, consider using rounded box corners: they'd probably look nicer than sharp squared corners. Increasing the Sprint logo, and the issues and code files icons.}

\subsubsection{Issue Management Components}
% \antu{This section can be shortened.}
There are three components for issue management in the current version of \toolname.

\underline{Similar Issue Detection:} This component takes a newly created issue (title \& description) from the GitHub Event Handler and compares it with each issue stored in the database by analyzing textual similarity between them. For this task, \toolname uses the RTA classification model~\cite{fang2023representthemall}, fine-tuned on the RTA duplicate bug report training dataset~\cite{representThemAllDataset}. \toolname uses a Process Pool Executor~\cite{processPool}, a multiprocessing component, to analyze multiple issue pairs concurrently. After the analysis, this component returns the duplicate issues to the GitHub Event Handler.
\looseness=-1
% \os{Explain why you used RTA and not other models
% -ans: I described in evaluation using a general line that since these models predictive accuracy were better, we chose these models. I didn't specify it here due to the page shortage}
% \antu{Fine-tuned on test dataset?} -> Fixed

\underline{Issue Severity Prediction:} This component receives a newly reported issue (issue title \& description) from the GitHub Event Handler and utilizes the RTA classification model~\cite{fang2023representthemall}, fine-tuned on RTA severity prediction training dataset~\cite{representThemAllDataset}, to classify the issue into one of five severity levels: blocker, critical, major, minor, and trivial. This component returns the predicted issue severity level to the Event Handler.
% \os{explain why you used this model and not others for severity prediction}
\rev{For similar issue detection and severity prediction tasks, we selected RTA~\cite{fang2023representthemall} because of its state-of-the-art predictive performance on large open-source issue reports and its speed and efficiency in issue analysis. It learns the fundamental representation of bug reports via a dynamic masked language model and contrastive learning objectives in a self-supervised manner.}
%\rev{We used RTA, a transformer-based embedding model for these two tasks due to its state-of-the-art predictive performance on large open-source issue report datasets and its fast and efficient issue analysis.}

\underline{Bug Localization:} This component receives a newly created issue (title \& description) and the code files of the latest system version from the GitHub Event Handler. For bug localization, \toolname uses the bug localization approach proposed by Bogomolov \etal~\cite{longCodeArena}: a Llama-2-7b-chat~\cite{llama-2} model, fine-tuned on their bug localization dataset~\cite{longCodeBLDataset}. 
%\rev{\{review 1\} \(\Rightarrow\) This dataset had separate training, validation, and test data. To prevent data leakage, we made sure that no test data was used during the fine-tuning of the model.}
The bug localization component constructs a prompt including the issue contents and the list of code file paths and names. With this prompt, the model generates a ranked list of potential buggy code files that might need further inspection to solve the issue and returns it to the Event Handler. The prompt asks the model to compare the textual semantics between the issue content and repository code file paths/names to predict potential buggy code files.
% \os{explain how the model can generate buggy file suggestions: just by comparing textually the issue text and the file paths?}.
% \antu{I think we used their approach which uses LLAMA. We need to be clear here. Current description is confusing. It seems we are using both their model and LLAMA.
% } -> Rephrased

\subsection{\toolname's Extensibility \& Scalability}
\toolname's architecture separates the issue management components from the components used for GitHub integration and event handling. The features are implemented as modular APIs with comprehensive documentation to simplify tool feature addition and enhancement. Our tool also conforms with the plugin architecture~\cite{pluginArchitecture} to enable easy extensibility. The event handler serves as the host component, while the issue management component APIs serve as the plugin interface. As per plugin architecture, \toolname's GitHub application supports dynamic loading and communication protocols, ensuring isolation between the host and plugins. Moreover, in the tool repository~\cite{repl_pack}, we provided scripts that can be used to fine-tune other transformer-based models for issue management tasks. With this, project owners can easily integrate the resulting models into our tool by replacing the model paths in the tool's configuration file.
% \antu{Is it about fine-tuning the underlying models with the user's dataset? If so, we should emphasize this more. I think we would get a review like `` Okay fine, but your tool will not work better for users' repos as the models are fine-tuned with different datasets".}
% -> Described a little more
% \os{we should describe the process of replacing a model or adding a new model. what are the requirements/conditions and/or API? what are the steps? how is easy it is to do it? once again, if we can't describe this, then we probably should not oversell this (it'd be a pity, because this is a key feature of the tool that we wanted to sell -- I hope we can still sell this)}
% \os{Unfortunately, this doesn't sound very convincing: it is too high level. Concrete tactics to enable extensibility need to be explained. 
% Does the tool provide an API with good documentation? Also a good internal documentation? Does it support a plugin architecture \url{https://en.wikipedia.org/wiki/Plug-in_(computing)
% Ans: yes, the feature components are implemented as APIs and well documented. The tool also conforms with the plugin architecture (our event handler part acts as Host Application, our feature component APIs act as Plugin Interface, our GitHub application has Dynamic Loading Mechanism and Communication Protocol support. Also, our host app and plugin interface is isolated.
% }\os{https://medium.com/omarelgabrys-blog/plug-in-architecture-dec207291800}?}

\toolname's backend utilizes Python's Process Pool Executor, which applies multiprocessing to concurrently handle multiple issues from various repositories. Project owners can configure the pool size based on their computational resources and workload requirements, by changing the configuration file, allowing \toolname to be responsive even during simultaneous requests. This multiprocessing technique also facilitates duplicate detection by notably reducing processing time and ensuring fast responses since pairwise issue comparison is computationally intensive.
\looseness=-1
% \os{is there a configuration file that allows a person to change the pool size, depending on their preferences or system/hardware constraints? 
% Ans: yes, we have set a value in our .env file. The users can update it to increase/decrease no. of processor pools}.

% \antu{We should talk about this in intro as well.}

\subsection{Implementation}
\toolname is implemented using Python's Flask framework~\cite{pythonFlask}. \toolname is developed as a GitHub Application, using GitHub’s Webhooks that allow integration with GitHub’s issues tracker~\cite{webhookDoc}. For data storage, \toolname uses the relational database system, SQLite3. Currently, \toolname is running on a modest server and can support five to six repositories concurrently.  For production, it needs to be deployed in a robust server, possibly in a cloud infrastructure. Though \toolname is tailored for handling GitHub issue management, most of its backend components can be easily adapted for other issue trackers.
\looseness=-1
% \os{After reading the whole paper, I got a question: what's the hardware used by the tool to support multiple repositories? I think we need to say that currently the tool runs in a modest server for illustration purposes, but for production, it needs to be installed in a robust server, probably using Cloud services}