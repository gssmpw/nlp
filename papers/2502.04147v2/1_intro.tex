% !TeX root = 0_main

\section{Introduction}
\label{sec:intro}
% \antu{Please make sure we answer all these questions and points in the first three paragraphs:\\
% * Para-1: Why issue reports and issue management are important? How does it help issue resolutions?\\
% * Para-2: Why automatic management is needed? Why a tool is needed and how a tool can assist developers in issue management and resolution?\\
% (We can merge para-1 and 2 later if needed.)\\
% * Para-3: What are the limitations of the existing tools? How do we aim to address the limitations?\\
% (This is a very important paragraph.)
% }

Issue reports are essential artifacts to identify, track, and resolve issues in software systems~\cite{blImportance, detmisinginfoinbr, bettenburg2008makes,Saha:icse25}. Given their importance, managing issue reports is critical for maintaining and evolving software systems~\cite{benefitsofissuemanage, brAnalysis}. Key issue management activities include labeling issues that report similar problems, categorizing them based on severity, and identifying potential buggy locations in the code~\cite{dupbrharmful, otoom2019automated, blImportance}. While important, manual issue management is a time-consuming and challenging process, especially for projects that typically receive hundreds of issues affecting various software components.  

To help developers manage issues, researchers have proposed a variety of techniques to automate issue report management tasks~\cite{practitionerPerceiveAutomation}.  Practitioners have also proposed a variety of tools~\cite{find_duplicates,probot,rocha2015nextbug,priorityScheduler,pr-agent}, standalone or integrated into existing systems (\eg issue trackers), to support these tasks. However, most of these tools are designed to support specific tasks and their underlying models are not easy to update. As such, these tools do not integrate different techniques and support for various issue management tasks in a single, easy-to-extend, comprehensive solution.
\looseness=-1

%As a result, automated techniques have been proposed in the field of software engineering,  becoming indispensable for effective and efficient issue management and resolution~\cite{practitionerPerceiveAutomation}.

%However, many of these techniques are not implemented as tools that users and developers can use to automate these tasks. Most existing solutions are not open-source and are designed for specific use cases~\cite{find_duplicates, Findbugs, zoho, buglocator}. Furthermore, these tools are often not extensible since feature addition or enhancement is hard.  As a result, they quickly become outdated, failing to keep pace with the evolving technology. Most existing issue management tools also don't provide concurrent support to multiple users and projects.
% \antu{Maybe we should add a line about scalability (\ie\ adding more models for each feature as required) and specify why it is important (\ie\ different techniques are emerging day-by-day which are better than the previous ones; if we can change or add more models, we will be able to use the most effective model).}
% -> Added scalability line. Discussed adding more models in the architecture section.

To address these limitations, we introduce \toolname, an integrated GitHub application for issue management. \toolname is designed as a comprehensive solution that can be easily extended and adapted to support multiple issue management tasks. In its current version, \toolname leverages state-of-the-art deep learning techniques to assist developers in (i) identifying issues similar to the newly reported issue, (ii) predicting issue severity, and (iii) localizing the potential code files that require modification to solve the reported problem. \toolname gives the developers suggestions as comments in issue reports and as labels attached to the issues to facilitate issue management. \toolname is extensible due to its modularized plugin-based architecture, which includes well-designed APIs and scripts that enable easy feature integration and extensibility. \toolname is also scalable, allowing multiple users and repositories to leverage its features concurrently. 

We evaluated the predictive performance of \toolname's underlying state-of-the-art models by replicating the evaluations of their respective papers. Additionally, we conducted a user study with five professional developers experienced in issue management, who found \toolname easy to use, useful, and practical for issue management and resolution. \toolname is an open-source tool hosted on GitHub~\cite{repl_pack} and can be easily installed in any GitHub repository.
% \antu{Same comment as above. Add another line about scalability. This can also justify our model selection. Because other models can be used easily if needed.}
% -> Added
% \antu{Are we sure we are choosing the best models? And ensuring the selection of the best models is not the goal of the replication. We need to update this line.}
% -> Rephrased
% \antu{We should add what are the findings/conclusions of the user study.}
% -> Added