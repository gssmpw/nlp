\section{Experimental Methods in Proteomics and Their Limitations}
\label{sec:experiment}

Traditional experimental techniques such as X-ray crystallography, nuclear magnetic resonance (NMR) spectroscopy, and cryo-electron microscopy (cryo-EM) in protein science have laid the foundation for studying protein structure and functions. However, computational approaches and also embrace the progress of AI development. This section briefly covers methods, which are essential for determining protein structures and functions.

\smallskip \noindent \textbf{X-ray Crystallography} is a widely utilized method for determining the 3D structures of proteins~\citep{jones1986using}. 
In this method, X-rays are directed at a crystallized sample, and the resulting diffraction patterns are analyzed to reveal the arrangement of atoms within the crystal. 
This process provides detailed insights into the protein's electron density and overall structure. 
However, crystallization can be challenging, especially for large, flexible, or membrane-associated proteins. The technique typically offers a static snapshot of the protein, which may not fully capture its dynamic nature in solution.
Advancements in AI have led to the development of structure prediction tools like AlphaFold~\citep{jumper2021highly} and RoseTTAFold~\citep{baek2021accurate}. For instance, the crystal structure of the KlNmd4 protein is predicted to consist of a single PIN domain~\citep{barbarin2021artificial}. The study demonstrates that the high-quality models significantly accelerate the determination of KlNmd4's structure, while existing models fail to achieve similar results. 


\smallskip \noindent \textbf{Nuclear Magnetic Resonance (NMR) Spectroscopy} is a non-destructive technique for determining the structure, dynamics, and interactions of molecules at the atomic level under near-physiological conditions~\citep{shukla2023biomolecular}. 
It provides 3D structural data of proteins in solution and captures real-time dynamics, making it highly effective for studying protein flexibility and weak protein-ligand interactions. NMR exploits the magnetic properties of atomic nuclei (e.g., hydrogen nuclei in proteins) to provide detailed information about the local chemical environment. 


With the development of AI, deep learning methods are more and more promising to advance the reconstruction of sparsely sampled data in NMR spectroscopy, particularly in the context of non-uniform sampling. The input data typically consists of sparsely sampled NMR spectra, while the output is the fully sampled spectrum, reconstructed either in the time~\citep{hansen2019using,karunanithy2021fid} or frequency domain~\citep{qu2020accelerated,luo2020fast}. For time-domain reconstructions, neural networks effectively predict the missing data points. In frequency-domain reconstructions, they excel at removing artifacts caused by sparse and non-Nyquist sampling. Studies across various research groups have consistently demonstrated the high accuracy of DNN-based reconstructions, even under conditions of extremely sparse sampling, highlighting the potential of deep learning to enhance data acquisition and analysis in NMR.

However, NMR has limited size range: NMR is mostly suitable for proteins smaller than ~30–50 kDa (larger proteins become challenging due to signal overlap). Protein sample preparation and data collection can also be expensive and take weeks to months.



\smallskip \noindent \textbf{Cryo-EM} is a structural biology technique that enables the direct observation of conformational heterogeneity in individual dynamic macromolecules~\citep{lyumkis2019challenges}. Researchers aim to reconstruct high-resolution 3D structural landscapes from numerous 2D observed projections, which may represent different conformational states. However, the cryo-EM reconstruction task is challenging because each particle's pose is unknown during imaging.
Recently, deep learning methods have demonstrated powerful capabilities in representing heterogeneity within datasets by mapping them onto nonlinear manifold embeddings. On the one hand, CryoDRGN~\citep{zhong2021cryodrgn} is a pioneering work that captures this heterogeneity by employing variational autoencoders (VAEs) to map the data into a low-dimensional latent space. A generative decoder then reconstructs a 3D volume from a sampled point in this latent space.
CryoGAN~\citep{gupta2021cryogan} introduces an entirely new possibility to learn to reconstruct in a distributional sense with a generative adversarial framework. Because of its likelihood-free nature, CryoGAN does not require any additional processing steps such as pose estimation and can be directly applied to cryo-EM measurements. This greatly simplifies the reconstruction procedure. On the other hand, E2gmm~\citep{chen2021deep} models the 3D structure using a set of Gaussians to automatically resolve the structural heterogeneity, whereas 3DFlex~\citep{punjani20233dflex} employs a neural network to fit the 3D displacement field of each particle by concurrently exploring its deformation field and refining a canonical density. More recently, CryoSTAR~\citep{li2024cryostar} resolves continuous conformational heterogeneity by constructing reasonable coarse-grained models, meanwhile, density maps are also estimated for different conformations. It meticulously preserves local structures, minimizes erroneous solutions, and ultimately achieves enhanced, accelerated convergence. Overall, the current trend is to incorporate atomic information to better activate deep models, aiming for more precise 3D structures that better comply with natural laws.

