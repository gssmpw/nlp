\section{Datasets}
\label{sec:dataset}
Datasets are crucial for training and evaluating \proteinllms. They are categorized into pretraining datasets, comprising unlabeled protein sequences for self-supervised learning, and benchmark datasets, which contain labeled sequences for supervised fine-tuning and evaluation on specific biological tasks.

\subsection{Pretraining Datasets}

\smallskip \noindent \textbf{UniProtKB}: A comprehensive protein sequence and annotation database composed of two main components: \textit{Swiss-Prot}~\cite{boutet2016uniprotkb}, a manually curated, high-quality dataset with reliable annotations and \textit{TrEMBL}~\cite{m1999edittotrembl}, an automatically annotated dataset providing broader coverage. 

\smallskip \noindent \textbf{UniRef Clusters}~\cite{suzek2015uniref}: A collection of clustered protein sequences designed to reduce data redundancy and improve computational efficiency. Provided by the UniProt database, UniRef is organized into three hierarchical levels: UniRef100, UniRef90, and UniRef50. UniRef100 contains a non-redundant set of all UniProt protein sequences where the latter two are created by clustering sequences with at least 90\% and 50\% sequence identity.

\smallskip \noindent \textbf{Pfam}~\cite{finn2006pfam}: A database of protein families and domains widely used for annotation and analysis of protein sequences. Each Pfam entry represents a group of related protein sequences defined by a multiple sequence alignment and a corresponding profile hidden Markov model (HMM). It provides insights into protein structure, function, and evolution, helping researchers identify conserved domains, predict functions, and classify proteins across organisms.

\smallskip \noindent \textbf{PDB}~\cite{bank1971crystallography}: The Protein Data Bank is a repository for the 3D structural data of large biological molecules, such as proteins and nucleic acids. It provides valuable resources for understanding the structural aspects of proteins, which can be beneficial for training models that incorporate structural information.

\smallskip \noindent \textbf{AlphaFoldDB}~\cite{tunyasuvunakool2021highly}: The AlphaFold Protein Structure Database offers predicted protein structures generated by the AlphaFold model containing over 200 million entries.

\subsection{Benchmark Datasets}

\smallskip \noindent \textbf{CASP}~\cite{kryshtafovych2019critical}: Critical Assessment of Structure Prediction is a biennial competition that evaluates methods for protein structure prediction. Participants predict 3D structures of proteins from their sequences, compared against experimental results.

\smallskip \noindent \textbf{ProteinGym}~\cite{notin2024proteingym}: A large-scale benchmark platform for protein design and fitness prediction. It includes over 250 Deep Mutational Scanning (DMS) assays, encompassing millions of mutated protein sequences, and curated clinical datasets with expert annotations. By integrating zero-shot and supervised evaluation frameworks, ProteinGym allows systematic comparison of over 70 machine learning models. It provides standardized metrics for tasks like mutation effect prediction and protein design, fostering innovation in computational biology and protein engineering.

\smallskip \noindent \textbf{TAPE}~\cite{rao2019evaluating}: A benchmark designed to evaluate protein sequence embeddings in biologically relevant tasks using machine learning. It includes five tasks covering structure prediction, evolutionary understanding, and protein engineering. TAPE leverages self-supervised learning, enabling models to learn from unlabeled protein sequences, and offers standardized datasets and metrics for systematic comparisons. It aims to advance protein representation learning by addressing gaps in generalization and real-world applicability.

\smallskip \noindent \textbf{PEER}~\cite{xu2022peer}: A comprehensive and multi-task benchmark designed to evaluate protein sequence understanding. It includes tasks such as protein function prediction, localization prediction, structure prediction, protein-protein interaction prediction, and protein-ligand interaction prediction. 

\smallskip \noindent \textbf{ProteinLMBench}~\cite{shen2024fine}: A benchmark dataset comprising 944 manually verified multiple-choice questions aimed at assessing the protein understanding capabilities of LLMs. It incorporates protein-related details and sequences in multiple languages, setting a new standard for evaluating LLMs’ abilities in protein comprehension.

