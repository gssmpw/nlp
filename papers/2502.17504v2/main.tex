% This must be in the first 5 lines to tell arXiv to use pdfLaTeX, which is strongly recommended.
\pdfoutput=1
% In particular, the hyperref package requires pdfLaTeX in order to break URLs across lines.

\documentclass[11pt]{article}

% Change "review" to "final" to generate the final (sometimes called camera-ready) version.
% Change to "preprint" to generate a non-anonymous version with page numbers.
% \usepackage[review]{acl}
\usepackage[preprint]{acl}

% Standard package includes
\usepackage{times}
\usepackage{latexsym}
\usepackage{enumitem}
\usepackage{booktabs}
\usepackage[edges]{forest}
\usepackage{array}
% For proper rendering and hyphenation of words containing Latin characters (including in bib files)
\usepackage[T1]{fontenc}
% For Vietnamese characters
% \usepackage[T5]{fontenc}
% See https://www.latex-project.org/help/documentation/encguide.pdf for other character sets

% This assumes your files are encoded as UTF8
\usepackage[utf8]{inputenc}
\usepackage{booktabs}
\usepackage{tabularx}
\usepackage{multirow}
\usepackage{rotating}

% This is not strictly necessary, and may be commented out,
% but it will improve the layout of the manuscript,
% and will typically save some space.
\usepackage{microtype}

% This is also not strictly necessary, and may be commented out.
% However, it will improve the aesthetics of text in
% the typewriter font.
\usepackage{inconsolata}

%Including images in your LaTeX document requires adding
%additional package(s)
\usepackage{graphicx}
\usepackage{amsmath}

\usepackage{tikz}
% \usepackage{forest}
\usepackage{xcolor}
\usepackage{multirow}
\usepackage{natbib}
\usepackage{mydef}
\usepackage{xspace}
\usepackage{pifont}

\newcommand{\proteinllms}{Protein LLMs\xspace}

% If the title and author information does not fit in the area allocated, uncomment the following
%
%\setlength\titlebox{<dim>}
%
% and set <dim> to something 5cm or larger.

% \title{Protein Language Models: A Comprehensive Survey of Architectures, Datasets, Applications, and Evaluation}
% \title{A Comprehensive Survey of Protein Large Language Models}
\title{Protein Large Language Models: A Comprehensive Survey}

% : Architectures, Datasets, Applications, and Evaluation

% Alternative titles
% Protein LLMs in Focus: Innovations, Applications, and Evaluation
% Exploring Protein Language Models: From Pretraining to Practical Applications

% Author information can be set in various styles:
% For several authors from the same institution:
% \author{Author 1 \and ... \and Author n \\
%         Address line \\ ... \\ Address line}
% if the names do not fit well on one line use
%         Author 1 \\ {\bf Author 2} \\ ... \\ {\bf Author n} \\
% For authors from different institutions:
% \author{Author 1 \\ Address line \\  ... \\ Address line
%         \And  ... \And
%         Author n \\ Address line \\ ... \\ Address line}
% To start a separate ``row'' of authors use \AND, as in
% \author{Author 1 \\ Address line \\  ... \\ Address line
%         \AND
%         Author 2 \\ Address line \\ ... \\ Address line \And
%         Author 3 \\ Address line \\ ... \\ Address line}


\author{%
    \begin{tabular}{c}
        Yijia Xiao\textsuperscript{\ding{171}} \thanks{Contact Email: \texttt{yijia.xiao@cs.ucla.edu}}, Wanjia Zhao\textsuperscript{\ding{168}}, Junkai Zhang\textsuperscript{\ding{171}}, 
        Yiqiao Jin\textsuperscript{\ding{170}}, Han Zhang\textsuperscript{\ding{171}}, \\
        Zhicheng Ren\textsuperscript{\ding{171}},  
        Renliang Sun\textsuperscript{\ding{171}}, Haixin Wang\textsuperscript{\ding{171}}, 
        Guancheng Wan\textsuperscript{\ding{171}}, Pan Lu\textsuperscript{\ding{168}}, \\
        Xiao Luo\textsuperscript{\ding{171}},  
        Yu Zhang\textsuperscript{\ding{169}}, James Zou\textsuperscript{\ding{168}}, Yizhou Sun\textsuperscript{\ding{171}}, Wei Wang\textsuperscript{\ding{171}} \\
    \end{tabular}\\[3.5ex]
    \textsuperscript{\ding{171}}UCLA, \quad 
    \textsuperscript{\ding{168}}Stanford, \quad 
    \textsuperscript{\ding{170}}Georgia Tech, \quad 
    \textsuperscript{\ding{169}}Texas A\&M  \\[1ex]
    \texttt{\url{https://github.com/Yijia-Xiao/Protein-LLM-Survey}}
}


%%%%%%%%%%%---SETME-----%%%%%%%%%%%%%
%replace @@ with the submission number submission site.
\newcommand{\thiswork}{INF$^2$\xspace}
%%%%%%%%%%%%%%%%%%%%%%%%%%%%%%%%%%%%


%\newcommand{\rev}[1]{{\color{olivegreen}#1}}
\newcommand{\rev}[1]{{#1}}


\newcommand{\JL}[1]{{\color{cyan}[\textbf{\sc JLee}: \textit{#1}]}}
\newcommand{\JW}[1]{{\color{orange}[\textbf{\sc JJung}: \textit{#1}]}}
\newcommand{\JY}[1]{{\color{blue(ncs)}[\textbf{\sc JSong}: \textit{#1}]}}
\newcommand{\HS}[1]{{\color{magenta}[\textbf{\sc HJang}: \textit{#1}]}}
\newcommand{\CS}[1]{{\color{navy}[\textbf{\sc CShin}: \textit{#1}]}}
\newcommand{\SN}[1]{{\color{olive}[\textbf{\sc SNoh}: \textit{#1}]}}

%\def\final{}   % uncomment this for the submission version
\ifdefined\final
\renewcommand{\JL}[1]{}
\renewcommand{\JW}[1]{}
\renewcommand{\JY}[1]{}
\renewcommand{\HS}[1]{}
\renewcommand{\CS}[1]{}
\renewcommand{\SN}[1]{}
\fi

%%% Notion for baseline approaches %%% 
\newcommand{\baseline}{offloading-based batched inference\xspace}
\newcommand{\Baseline}{Offloading-based batched inference\xspace}


\newcommand{\ans}{attention-near storage\xspace}
\newcommand{\Ans}{Attention-near storage\xspace}
\newcommand{\ANS}{Attention-Near Storage\xspace}

\newcommand{\wb}{delayed KV cache writeback\xspace}
\newcommand{\Wb}{Delayed KV cache writeback\xspace}
\newcommand{\WB}{Delayed KV Cache Writeback\xspace}

\newcommand{\xcache}{X-cache\xspace}
\newcommand{\XCACHE}{X-Cache\xspace}


%%% Notions for our methods %%%
\newcommand{\schemea}{\textbf{Expanding supported maximum sequence length with optimized performance}\xspace}
\newcommand{\Schemea}{\textbf{Expanding supported maximum sequence length with optimized performance}\xspace}

\newcommand{\schemeb}{\textbf{Optimizing the storage device performance}\xspace}
\newcommand{\Schemeb}{\textbf{Optimizing the storage device performance}\xspace}

\newcommand{\schemec}{\textbf{Orthogonally supporting Compression Techniques}\xspace}
\newcommand{\Schemec}{\textbf{Orthogonally supporting Compression Techniques}\xspace}



% Circular numbers
\usepackage{tikz}
\newcommand*\circled[1]{\tikz[baseline=(char.base)]{
            \node[shape=circle,draw,inner sep=0.4pt] (char) {#1};}}

\newcommand*\bcircled[1]{\tikz[baseline=(char.base)]{
            \node[shape=circle,draw,inner sep=0.4pt, fill=black, text=white] (char) {#1};}}

\begin{document}
\maketitle

\begin{abstract}
Protein-specific large language models (\proteinllms) are revolutionizing protein science by enabling more efficient protein structure prediction, function annotation, and design. While existing surveys focus on specific aspects or applications, this work provides the first comprehensive overview of \proteinllms, covering their architectures, training datasets, evaluation metrics, and diverse applications. Through a systematic analysis of over 100 articles, we propose a structured taxonomy of state-of-the-art \proteinllms, analyze how they leverage large-scale protein sequence data for improved accuracy, and explore their potential in advancing protein engineering and biomedical research.  Additionally, we discuss key challenges and future directions, positioning \proteinllms as essential tools for scientific discovery in protein science. Resources are maintained at \url{https://github.com/Yijia-Xiao/Protein-LLM-Survey}.
\end{abstract}



\section{Introduction}

``\textit{Proteins are the machinery of life, and understanding their language unlocks the secrets of biology.}''
\rightline{--- David Baker (Nobel Prize laureate 2024)}
\\

Proteins are essential biological molecules, driving functions such as catalyzing biochemical reactions, maintaining cell structure, and enabling cellular communication. Understanding their sequence-structure-function relationships is central to biological research. However, traditional experimental methods, including X-ray crystallography, NMR spectroscopy, and cryo-electron microscopy, are time-consuming and labor-intensive, posing bottlenecks for large-scale applications.

Recent advancements in language modeling have revolutionized computational biology, offering powerful tools for protein analysis. Protein large language models (\textbf{\proteinllms}) share several foundational similarities with LLMs: 1) \emph{Training objectives and learning paradigms}, both LLMs and \proteinllms are trained in a self-supervised manner on large-scale datasets using objectives such as masked language modeling~\cite{devlin2019bert},  auto-regressive modeling~\cite{luo2022biogpt}, or sentence permutation~\cite{lewis2019bart, yuan2022biobart}, learning to predict missing or next elements in sequences from the vocabulary. While LLMs predict missing words or phrases within textual data~\cite{reimers2019sentence, liu2019roberta, touvron2023llama}, \proteinllms predict amino acids or subsequences within protein sequences. 2) \emph{Pretraining data.} \proteinllms adopt a data-driven paradigm to learn directly from large-scale protein datasets~\cite{liu2024timemattersexaminetemporal, jones2024examiningimbalanceeffectsperformance}. The datasets for training \proteinllms consist of vast collections of protein sequences, analogous to the textual corpora used for LLMs. This eliminates the need for explicit feature engineering, allowing \proteinllms to learn intricate patterns, such as structural motifs, evolutionary relationships, and functional insights, similar to how LLMs capture semantic and syntactic structures in language.

This paradigm shift has led to the emergence of highly effective models that can predict protein folding, annotate biological functions, and even design novel proteins with desired characteristics. Beyond their predictive capabilities, \proteinllms also provide interactive interfaces that allow users to upload protein sequences or structural files (e.g., PDB format), pose questions, and interact with the model in a conversational manner~\cite{liu2024prott3,xiao2024proteingpt}, proving deeper insights into protein structure, function, and design.

We present the first dedicated survey of \proteinllms, analyzing their unique architectures, training methodologies, and practical applications in protein research. While previous studies have explored the applications of various computational methods for protein research~\cite{xinhui2024generative, wu2022survey} or discussed the role of language models in general scientific domains such as biomedicine~\cite{wang2023pre} and chemistry~\cite{liao2024words}, this survey focuses specifically on \proteinllms--a rapidly evolving area at the intersection of computational biology and NLP.  

The key contributions are as follows:


\begin{itemize}[leftmargin=1em, noitemsep, topsep=0pt]
    \item \textbf{Architectural Overview.} A structured taxonomy of state-of-the-art \proteinllms (Figure \ref{fig:taxonomy}) detailing their unique architectures for protein understanding (\S\ref{sec:llm_understanding}) and generation (\S\ref{sec:llm_generation}), highlighting how these models surpass traditional experimental methods in both efficiency and accuracy (Appendix \S\ref{sec:experiment}). 
    \item \textbf{Data Insights.} A comprehensive summary of datasets for pretraining, fine-tuning, and benchmarking \proteinllms, providing critical insights into data curation strategies and their impact on model performance (\S\ref{sec:dataset}).
    \item \textbf{Evaluation Protocols.} A thorough discussion of methodologies for assessing the performance and impact of \proteinllms, including comprehensive new benchmarking strategies (\S\ref{sec:eval} and Appendix \S\ref{sec:app_eval}).
    \item \textbf{Applications.} A detailed exploration of practical applications in protein prediction, annotation, and design, remarkably highlighting recent innovative advancements and showcasing the transformative potential of \proteinllms in advancing biomedical research.
\end{itemize}

\begin{figure*}[htbp]
    \centering
    \includegraphics[width=\textwidth]{fig/Tasks.pdf}
    \caption{An Overview of Tasks in Protein Large Language Models.}
    \label{fig:methods}
\end{figure*}


\section{LLM Methods for Protein Understanding and Prediction}


\label{sec:llm_understanding}

\subsection{Problem Definition}

A protein, composed of amino acids (residues), can be represented as a sequence \( [x_1, \dots, x_L] \) in the residue token space \( \mathcal{P} \), where \(L\) denotes its length. According to Anfinsen's dogma, a protein’s primary sequence determines its structure and function. General problems in protein understanding and prediction are as follows:


\noindent \textit{I. Sequence-to-Property Prediction:} \( f_\theta: \mathcal{P} \rightarrow \mathcal{R}^+ \) mapping sequences to numerical properties, such as stability or fluorescence intensity.

\noindent \textit{II. Sequence-to-Label Prediction:} \( f_\theta: \mathcal{P} \rightarrow \mathcal{L} \) mapping sequences to categorical labels, including secondary structure types, contact maps, or functional annotations.

\noindent \textit{III. Sequence-to-Structure Prediction} \( f_\theta: \mathcal{P} \rightarrow \mathcal{S} \) mapping sequences to the 3D folding structures (i.e. tertiary structures). 

\noindent \textit{IV. Sequence-to-Text Understanding:} \( f_\theta: \mathcal{P} \rightarrow \mathcal{T} \), where \( \mathcal{T} \) represents generated textual descriptions of protein sequences.


\subsection{Protein Sequence Models}


\noindent \textbf{Individual Protein Sequences Models.}
Protein language models process amino acid sequences into meaningful representations for downstream tasks including structure and function prediction. Like NLP models, they are usually first pretrained on large sequence datasets with masked language modeling (MLM) objective; and then the protein sequences' embeddings are adapted for downstream tasks.
Initially, researchers leveraged long short-term memory (LSTM) architectures to learn representation of proteins \citep{alley2019unified,bepler2019learning,zhou2020mutation}. 
Following the breakthrough of transformer architectures \citep{vaswani2017attention} in NLP, transformer-based protein language models emerged as the new paradigm. Large-scale transformer models, scaling up to billions of parameters and trained on millions of protein sequences, have demonstrated remarkable effectiveness for protein understanding and prediction tasks \citep{rao2019evaluating,elnaggar2021prottrans, xiao2021modeling,hu2022exploring}, and 3D structure folding \citep{chowdhury2022single,fang2022helixfold,chen2024xtrimopglm}.
The interpretability of these \proteinllms has also been explored, with \citep{vig2020bertology} analyzing learned representations through the lens of attention.
Beyond general-purpose protein language models, several works have focused on domain-specific applications. For instance, \citet{hie2021learning} applied BiLSTM to model viral escape patterns; TCR-BERT \citep{wu2024tcr} specialized in T-cell receptor (TCR) analysis for improved TCR-antigen binding prediction; PeptideBERT \citep{guntuboina2023peptidebert} focused on predicting key properties of peptides; \citet{kroll2023turnover,yu2023enzyme} adapted ESM-1b for enzymatic function prediction.
 

\noindent \textbf{Multiple Sequence Alignments (MSA) Models.} MSA aligns homologous proteins within sequence space by mapping their residues to the coordinate framework of a designated seed sequence. MSA reveals evolutionary relationships between proteins and thus serves as a cornerstone of computational biology, particularly for mutation effects prediction \citep{ram2022few,hawkins2021msa}. The MSA Transformer \citep{rao2021msa} processed MSAs instead of single sequences. It used a modified axial attention mechanism \citep{ho2019axial,child2019generating} to model both intra- and inter-sequence relationships. In contrast, Tranception \citep{notin2022tranception}, was trained on individual non-aligned sequences but could leverage aligned sequences during inference. It extracted patterns from contiguous protein subsequences and improves fitness prediction by integrating MSAs retrieved at inference time. In specific subdomains, \citet{lin2023deep} developed a transfer learning framework that utilized ESM-MSA-1b for transmembrane protein complexes. Additionally, vcMSA~\citep{mcwhite2023leveraging} and Poet~\citep{truong2023poet} leveraged protein LLMs to identify MSAs or homologous sequences.

\noindent \textbf{Evolutionary Scale Modeling (ESM) Series.} ESM is a family of transformer models for protein modeling. 
ESM-1b~\cite{rives2021biological}, the first model in the series with up to 669.2 million parameters, was trained on 250 million protein sequences using a masked language modeling (MLM) objective and contains up to 669.2 million parameters. 
Building on this,  ESM-1v~\citep{meier2021language} focused on predicting the effects of mutations in zero-shot setting, while incorporating the MSA Transformer \citep{rao2021msa} for few-shot mutation prediction. 
Thanks to the success of AlphaFold2 \citep{jumper2021highly}, ESM-IF \citep{hsu2022learning} utilized predicted structures to train large models combining Geometric Vector Perceptron \citep{jing2020learning} with GNN or transformer on the inverse folding task that predicts protein strings from the 3D structures. The new general-purpose language protein model ESM-2 \citep{lin2023evolutionary} further scaled up the model size to 15 billion parameters and incorporated a folding head to create an end-to-end single-sequence structure prediction model ESMFold. The latest model ESM-3~\citep{hayes2025simulating} is a multimodal generative model with 98 billion parameters that could reason over protein sequences, structures, and functions. Using a chain-of-thought approach, it successfully designed a novel fluorescent protein far from any known fluorescent proteins.
\begin{figure*}[htbp]
    \centering
    \includegraphics[width=\textwidth]{fig/Methods.pdf}
    \caption{An Overview of Methods of Protein Large Language Models.}
    \label{fig:methods}
\end{figure*}

\subsection{Structure-Integrated and Knowledge-Enhanced Models}

Beyond residue sequences, many models integrate additional information, such as structure data or external knowledge, to enhance protein understanding and prediction ability.

\noindent \textbf{Structure-Integrated Models}: Structural information plays an important role in protein understanding, as a protein's functions are determined by its structures. Therefore, many works have incorporated structural information to enhance protein modeling ability. 
Some works utilized structure information as additional inputs \citep{chen2024endowing,tan2024simple}. For instance, \citet{zhang2023systematic} fused global structure information captured by structure encoder (GVP, GearNet \cite{zhang2022protein}, or CDConv \citep{fan2022continuous}) into representations of ESM-2; SaProt \citep{su2023saprot} incorporated local structural information for each amino acid, derived from Foldseek \citep{van2022foldseek}, to generate structure-aware tokens.
Alternatively, other works injected the structure information only in the training stage by either additional training tasks \citet{wang2022multi,sun2024structure,zhang2024structure} or contrastive learning \citep{wang2025s}.
Some studies have also leveraged pretrained protein language models to improve structure models~\citep{wu2023integration, zheng2024ccpl}.


\noindent \textbf{Knowledge-Enhanced Models}: Beyond large protein sequence datasets, information in other formats can further enhance a model’s understanding of proteins in the training stage. 
OntoProtein \citep{zhang2022ontoprotein} and KeAP \citep{zhou2023protein} incorporated knowledge graphs data during training by additional MLM objectives and/or contrastive learning to inject factual biological knowledge into the pre-trained \proteinllms.
ProteinBERT \citep{brandes2022proteinbert} performed dual-task learning during pretraining to learn both protein sequence modeling and Gene Ontology (GO) annotation prediction. It utilized a specialized BERT architecture with parallel input pathways for sequences and annotations.
To leverage the rich information in textual descriptions or other modalities, ProteinCLIP \citep{wu2024proteinclip} and MolBind \citep{xiao2024molbind} applied contrastive learning between protein sequences and textual descriptions and/or molecular to learn improved embeddings.


\subsection{Protein Description and Annotation Models}


The previously mentioned models have primarily focused on learning protein representations and utilizing them for classification, regression, or 3D structure folding tasks. To enhance expressiveness and understanding, more recent models have been trained on both protein sequences and textual data, allowing them to integrate NLP capabilities with protein representation learning \citep{wang2023instructprotein, liu2024prott3, zhuo2024protllm,jin2024prollm}.
\citet{xu2022protranslator} proposed ProTranslator, a bilingual translation framework between protein sequences and GO functions with textual descriptions. ProTranslator encoded and aligned the textual definitions of GO functions and protein sequences within the same low-dimensional space, facilitating the annotation of novel GO functions and the generation of textual descriptions for proteins. BioTranslator \citep{xu2023multilingual} further improved ProTranslator by extending the bilingual framework to a multilingual translation framework, embedding text and multiple biomedical modalities into a shared space.
ProtST~\cite{xu2023protst} was a framework designed to jointly learn from protein sequences and their associated biomedical text descriptions. It integrated protein language models (e.g., ESM or ProtBERT) with biomedical language models (e.g., PubMedBERT) to fuse sequence and text information through pre-training tasks. Prot2Text~\citep{abdine2024prot2text} combined ESM-2 with a structure encoder (RGCN) and extended function prediction from categorical classification to free-text descriptions. BioT5 and BioT5+~\citep{pei2023biot5,pei2024biot5+} further unified molecular information within a more comprehensive training framework.

There have also been several interactive LLMs for protein understanding. These models enhanced pretrained LLMs with protein comprehension by integrating a protein processing module~\citep{wu2024structure, wang2024protchatgpt,wang2024long}. For instance, ProteinChat~\citep{guo2023proteinchat} allowed users to input protein structures and query them using texts. ProteinGPT~\citep{xiao2024proteingpt} extended this capability by supporting both protein sequences and structures as inputs. In these models, protein data were processed through \proteinllms to generate embeddings, which were then projected to the natural language embedding space. The backbone LLMs integrated these adapted embeddings with user’s queries to produce meaningful answers.
\begin{figure*}[ht]
\centering
\begin{forest}
  for tree={
    forked edges,
    grow=east,
    reversed=true,
    anchor=base west,
    parent anchor=east,
    child anchor=west,
    base=middle,
    font=\scriptsize,
    rectangle,
    draw=black,
    edge=black!50, 
    rounded corners,
    align=center,
    minimum width=1em,
    s sep=5pt,
    inner xsep=2.5pt,
    inner ysep=1pt
  },
  [Protein Large Language Models,rotate=90,anchor=north,edge=black!50,fill=myblue,draw=black
    [Problem,edge=black!50, fill=mypurple, minimum height=1.2em
      [Protein Understanding \& Prediction,text width=9.6em,fill=mypurple
        [\textbf{Pretraining Dataset:} UniRef Clusters~\cite{suzek2015uniref}{,} Pfam~\cite{finn2006pfam}{,} \\UniProtKB~\cite{boutet2016uniprotkb,m1999edittotrembl},text width=23em, fill=mypurple]
       [\textbf{Benchmark Dataset:} CASP~\cite{kryshtafovych2019critical}{,} TAPE\citep{rao2019evaluating}{,} \\ProteinLMBench~\cite{shen2024fine}{,} PEER~\cite{xu2022peer},text width=23em, fill=mypurple]
      ]
      [Protein Engineering \& \\Generation  \& Translation ,text width=6.8em,fill=mypurple
        [\textbf{Pretraining Dataset:} AlphaFoldDB~\cite{tunyasuvunakool2021highly},text width=25.8em, fill=mypurple]
       [\textbf{Benchmark Dataset:} ProteinGym~\cite{notin2024proteingym}{,} ProteinLMBench~\cite{shen2024fine},text width=25.8em, fill=mypurple]]
      ]
    [Method, edge=black!50, fill=myred, minimum height=1.2em
      [LLM Methods for \\ProteinUnderstanding \\and Prediction, edge=black!50, fill=myred
        [Protein Sequence Models, text width=6.6em, fill=myred
          [ UniRep~\citep{alley2019unified}{,} SSA \citep{bepler2019learning}{,}\\ MuPIPR \citep{zhou2020mutation}{,} ProtTrans\citep{elnaggar2021prottrans}{,}\\ AminoBERT\citep{chowdhury2022single}{,} Provis \citep{vig2020bertology}{,}  \\xTrimoPGLM~\citep{chen2024xtrimopglm}{,} CSCS~\citep{hie2021learning}{,}\\TCR-BERT \citep{wu2024tcr}{,} PeptideBert \citep{guntuboina2023peptidebert}{,}\\
          ESM-1b \citep{rives2021biological}{,} ESM-1v \citep{meier2021language}{,} \\
          AlphaFold2 \citep{jumper2021highly}{,} ESM-IF \citep{hsu2022learning}{,}\\
          ESM-2 \citep{lin2023evolutionary}{,} ProtTrans \citep{elnaggar2021prottrans}{,} \\
          ProteinLM \citep{xiao2021modeling}{,} 
 ProteinBERT~\citep{brandes2022proteinbert}{,}\\
          MSA Transformer \citep{rao2021msa}{,} Tranception \citep{notin2022tranception}{,} \\vcMSA~\citep{mcwhite2023leveraging}{,} Poet~\citep{truong2023poet} , text width=18.9em, fill=myred]
        ]
        [Structure-Integrated \&\\ Knowledge-Enhanced \\Models, text width=5.7em, fill=myred
          [OntoProtein \citep{zhang2022ontoprotein}{,} ESM-GearNet \citep{zhang2023systematic}{,}\\
          SaProt \citep{su2023saprot}{,} ProteinBERT \citep{brandes2022proteinbert}{,}\\ProteinCLIP \citep{wu2024proteinclip} , text width=19.8em, fill=myred]
        ]
        [Protein Description \&\\ Annotation Models, text width=5.6em, fill=myred
          [ProTranslator~\citep{xu2022protranslator}{,} BioTranslator~\citep{xu2023multilingual}{,}\\ProtST~\cite{xu2023protst}{,} Prot2Text~\citep{abdine2024prot2text}{,} \\BioT5~\citep{pei2023biot5}{,} ProtChatGPT~\citep{wang2024protchatgpt}{,}\\ProteinChat~\citep{guo2023proteinchat}{,} ProteinGPT~\cite{xiao2024proteingpt},text width=19.8em, fill=myred]
        ]
      ]
      [LLM Methods for \\Protein Engineering\\\& Generation \& Translation, edge=black!50, fill=myred
        % [Encoder-based Models, text width=6em, fill=myred
        %   [ProtST~\cite{xu2023protst}, text width=7.0em, fill=myred]
        % ]
        [Protein Engineering Models, text width=7.4em, fill=myred
        [
        ProteinDT~\citep{liu2023text}{,} PLMeAE~\cite{plmeae}{,} \\ Toursynbio~\cite{shen2024toursynbio}, text width=16.3em, fill=myred
        ]
        ]
        [Protein Generation Models, text width=7.2em, fill=myred
          [ProGen \citep{madani2023large}{,} ProtGPT2 \citep{ferruz2022protgpt2}{,}\\
          ProGen2 \citep{nijkamp2023progen2}{,} ProLLaMA~\citep{lv2024prollama}{,}\\Ankh \citep{elnaggar2023ankhoptimizedproteinlanguage}{,}
          PAAG \citep{yuan2024annotation} {,} \\ Pinal \citep{dai2024toward} {,} IgLM \citep{shuai2023iglm}{,}\\
          PALM-H3 \citep{he2024novo}{,}  LM-D~\citep{10.5555/3618408.3620189}, text width=16.6em, fill=myred]
        ]
        [Protein Translation Models, text width=7em, fill=myred
          [ProstT5 \citep{heinzinger2023bilingual}{,} Fold2Seq \citep{cao2021fold2seq} {,} \\
          ProtAgents \citep{ghafarollahi2024protagents},
           text width=16.8em, fill=myred]
        ]
      ]
      [Traditional experimental methods, edge=black!50, fill=myred
        [X-ray \citep{jones1986using}{,} NMR \citep{shukla2023biomolecular}{,} Cryo-EM \citep{lyumkis2019challenges}{,}\\
        CryoDRGN~\citep{zhong2021cryodrgn}{,} CryoGAN~\citep{gupta2021cryogan}{,}\\
        CryoSTAR~\citep{li2024cryostar}{,} E2gmm~\citep{chen2021deep}, text width=24.1em, fill=myred]
      ]
    ]
    % [Capability,edge=black!50, fill=mygreen, minimum height=1.2em
    %   [Structure Prediction,text width=5.5em,fill=mygreen
    %     [SSA \citep{bepler2019learning}{,} ESM-1b \citep{rives2021biological}{,} ESM-1v \citep{meier2021language}{,} \\AlphaFold2~\citep{jumper2021highly}{,}
    %     ESM-IF~\citep{hsu2022learning}{,} ESM-2 \citep{lin2023evolutionary}{,}\\AminoBERT\citep{chowdhury2022single}{,} 
    %     Prottrans \citep{elnaggar2021prottrans}{,} ProtST~\cite{xu2023protst}{,}\\ProteinBERT~\citep{brandes2022proteinbert}{,} 
    %     MSA Transformer~\citep{rao2021msa}{,} Tranception \citep{notin2022tranception}, text width=28em, fill=mygreen]
    %   ]
    %   [Function Prediction,text width=5.5em,fill=mygreen
    %     [OntoProtein \citep{zhang2022ontoprotein}{,}  ESM-GearNet \citep{zhang2023systematic}{,}ProTranslator~\citep{xu2022protranslator}{,}\\ Prot2Text~\citep{abdine2024prot2text}{,}
    %     SaProt \citep{su2023saprot}, text width=28em, fill=mygreen]
    %   ]
    %   [Sequence Generation,text width=5.5em,fill=mygreen
    %     [ProLLaMA~\citep{lv2024prollama}{,} ProGen \citep{madani2023large}{,} ProtGPT2 \citep{ferruz2022protgpt2}{,} \\ProteinDT~\citep{liu2023text}{,}ProGen2 \citep{nijkamp2023progen2}{,}
    %     IgLM \citep{shuai2023iglm}{,}\\ PALM-H3 \citep{he2024novo}, text width=24em, fill=mygreen]
    %   ]
    %   [Sequence Translation,text width=5.5em,fill=mygreen
    %     [ProTranslator~\citep{xu2022protranslator}{,}ProstT5 \citep{heinzinger2023bilingual}{,} Fold2Seq \citep{cao2021fold2seq}{,} \\Ankh \citep{elnaggar2023ankhoptimizedproteinlanguage}{,}
    %     ProtChatGPT \citep{wang2024protchatgpt}{,} ProteinChat~\citep{guo2023proteinchat}{,} \\ProteinGPT \citep{xiao2024proteingpt}{,}BioT5~\citep{pei2023biot5}{,} BioTranslator~\citep{xu2023multilingual}, text width=26em, fill=mygreen]
    %   ]
    % ]
    [Evaluation Metrics,edge=black!50, fill=mygreen, minimum height=1.2em
        [Structure Prediction Metrics, text width=7.6em,fill=mygreen
        [RMSD~\cite{li2013difficulty}{,} GDT-TS~\cite{zemla2003lga}{,} TM-Score~\cite{zhang2004scoring}{,}\\lDDT~\cite{mariani2013lddt}{,} pLDDT ~\cite{guo2022alphafold2}, text width=22.3em, fill=mygreen]
        ]
        [Function Prediction Metrics, text width=7.6em,fill=mygreen
        [Accuracy{,} Precision{,} Recall{,} F1-score{,} AUC{,} BLEU \cite{papineni2002bleu}{,} \\ROUGE-L \cite{lin2004rouge}{,} METEOR \cite{banerjee2005meteor}{,} \\Subcellular Localization~\cite{briesemeister2010yloc}{,} Stability~\cite{cheng2006prediction}{,}\\Homology Detection~\cite{altschul1990basic,hamamsy2024protein}{,} \\ Solubility~\cite{hebditch2017protein}{,} Mutation Effect Prediction~\cite{mansoor2022accurate},text width=22.3em, fill=mygreen]
	]
        [Sequence Generation Metrics, text width=7.6em,fill=mygreen
	    [Perplexity~\cite{hesslow2022rita}{,} Diversity~\cite{bywater2015prediction,mcgee2021generative}{,}\\Novelty~\cite{truong2023poet}{,} Fr\'echet Protein Distance~\citep{jiang2008protein}{,} \\Foldability~\cite{baek2021accurate,magliery2015protein}{,} Recovery~\cite{watson2023novo}, text width=22.3em, fill=mygreen]
            ]
      ]
  ]
\end{forest}
\caption{Taxonomy of Protein Large Language Models.}
% \YS{what's the difference between problem and capability? People might think where does protein folding fit in, due to the popularity of AlphaFold.}\WJ{added detailed dataset and benchmark for problems}
\label{fig:taxonomy}
\end{figure*}




\section{LLM Methods for Protein Engineering, Generation and Translation}
\label{sec:llm_generation}


Protein engineering and generation aims to design protein sequences with desired attributes (e.g. structures and properties). Given the desired attributes \(T\) and reference protein sequence \(\mathcal{S}\) (optional), the model is expected to output a protein sequence \(\mathcal{S}'\) with desired attributes. Key tasks include:

\noindent \textit{I. Protein Engineering:} 
\( f_\theta: (\mathcal{S}, T) \rightarrow \mathcal{S}' \) modifies protein \(\mathcal{S}\) toward the desired attributes \(T\), yielding the engineered protein \(\mathcal{S}'\).

\noindent \textit{II. Protein Generation:} 
\( f_\theta: (T, R) \rightarrow \mathcal{P} \) generates proteins with attributes \(T\) by sampling from the protein space using random seeds \(R\).



\noindent \textit{III. Protein Translation:} \( f_\theta: (\mathcal{P},T) \rightarrow \mathcal{P'} \) translates a protein \( \mathcal{P} \) into an alternative representation \( \mathcal{P}' \)  based on the target translation parameters \( T \). 



\subsection{Protein Engineering Models}

ProteinDT~\cite{liu2023text} is a multimodal protein design framework that robustly integrates textual protein knowledge with sequence-based generative modeling. ProteinDT employs contrastive alignment and a facilitator module, enabling zero-shot text-to-protein generation and editing. %For editing tasks, it takes an initial protein sequence and a descriptive text prompt as inputs, and outputs a modified protein sequence that aligns with the target property without any task-specific training. 
Meanwhile, PLMeAE~\cite{plmeae} is a closed-loop protein engineering framework that integrates protein language models with an automated biofoundry within a Design-Build-Test-Learn cycle.
Furthermore, Toursynbio~\cite{shen2024toursynbio} introduces an agent that is capable of facilitating the modification and engineering of wet lab proteins.


\subsection{Protein Generation Models}

Protein generation models are designed to create novel protein sequences for specific engineering applications, often leveraging large-scale datasets of existing proteins with known amino acid sequences and properties. These models typically employ decoder-based architectures to generate functional protein sequences conditioned on various biological annotations. For example, ProGen~\cite{madani2023large} is a GPT-based generative protein engineering model that treats protein engineering as an unsupervised sequence generation process, and generates functional protein sequences conditioned on annotations like molecular function or taxonomy. The model is trained on diverse, non-redundant protein sequences from databases such as UniProt and Pfam, utilizing associated tags for conditional generation. ProtGPT2~\cite{ferruz2022protgpt2} is another model that generates de novo protein sequences with natural amino acid compositions using autoregressive modeling. In particular, they noticed that the generated sequences could explore a few uncharted areas of the protein sequence space. 
ProGen2~\cite{nijkamp2023progen2} is an extended version of ProGen, featuring a larger model size and a more extensive training dataset to enhance sequence diversity.  
Notably, ProGen2 can predict protein fitness without requiring additional fine-tuning. Recently, ProLLaMA~\citep{lv2024prollama} proposed a multi-task protein language model to handle both protein sequence generation and protein understanding tasks. Built on LLaMA2, ProLLaMA introduces a two-stage training framework: (1) continued pre-training on protein sequences, and (2) instruction tuning with a 13-million-sample dataset for multitasking capabilities.

Beyond conventional decoder-based approaches, Ankh~\citep{elnaggar2023ankhoptimizedproteinlanguage} employs an encoder-decoder architecture that optimizes efficiency by reducing parameters while maintaining high-quality protein generation. PAAG~\citep{yuan2024annotation} is another encoder-decoder architecture which focuses on the alignment between textual annotations and protein sequences at multiple levels before generating new sequences.  
Pinal~\citep{dai2024toward} does not directly generate protein sequences from text. Instead, it first constrains the protein design space by generating structure tokens, then predicts sequences based on those constraints to improve foldability and function alignment.


While many of these models are designed for general protein generation, some focus on specialized applications such as antibody design. IgLM~\cite{shuai2023iglm} employs autoregressive sequence generation conditioned on an antibody's sequence chain type and species of origin. %A key feature is its ability to generate infilled residue spans located at indicated positions within the entire antibody sequence. 
As a further step, PALM-H3~\cite{he2024novo} specifically targets SARS-CoV-2 antibody generation, highlighting how protein generation language models can be tailored for highly specific protein design tasks.


\subsection{Protein Translation Models}


Protein translation models are specifically developed to handle tasks that require translating between different protein representations, which could be helpful in protein design. 

ProstT5~\cite{heinzinger2023bilingual} addresses the task of simultaneously modeling the dual nature of proteins — their linear one-dimensional (1D) sequences and three-dimensional (3D) structures — using a bilingual language model based on T5~\cite{raffel2020exploring} and ProtT5~\cite{pokharel2022improving}. It extracts features and patterns from both the sequence and the structure data %, enabling improved protein design, remote homology detection, and structure-sequence translation. 
Fold2Seq~\cite{cao2021fold2seq} is another model that learns structure-sequence relationships of proteins. %Using a multi-step training framework that involves fold-to-sequence reconstruction, fold classification, and cross-domain losses, 
The model could guide designs of protein sequences conditioned on desired structural folds. Recently, ProtAgents \cite{ghafarollahi2024protagents}, a multiagent framework, has been proposed to handle 1D sequence generation and 3D fold generation simultaneously. LM-DESIGN~\citep{10.5555/3618408.3620189} is a method for reprogramming protein language models (pLMs) to design protein sequences for given structural folds.


\section{Datasets}
\label{sec:dataset}
Datasets are crucial for training and evaluating \proteinllms. They are categorized into pretraining datasets, comprising unlabeled protein sequences for self-supervised learning, and benchmark datasets, which contain labeled sequences for supervised fine-tuning and evaluation on specific biological tasks.

\subsection{Pretraining Datasets}

\smallskip \noindent \textbf{UniProtKB}: A comprehensive protein sequence and annotation database composed of two main components: \textit{Swiss-Prot}~\cite{boutet2016uniprotkb}, a manually curated, high-quality dataset with reliable annotations and \textit{TrEMBL}~\cite{m1999edittotrembl}, an automatically annotated dataset providing broader coverage. 

\smallskip \noindent \textbf{UniRef Clusters}~\cite{suzek2015uniref}: A collection of clustered protein sequences designed to reduce data redundancy and improve computational efficiency. Provided by the UniProt database, UniRef is organized into three hierarchical levels: UniRef100, UniRef90, and UniRef50. UniRef100 contains a non-redundant set of all UniProt protein sequences where the latter two are created by clustering sequences with at least 90\% and 50\% sequence identity.

\smallskip \noindent \textbf{Pfam}~\cite{finn2006pfam}: A database of protein families and domains widely used for annotation and analysis of protein sequences. Each Pfam entry represents a group of related protein sequences defined by a multiple sequence alignment and a corresponding profile hidden Markov model (HMM). It provides insights into protein structure, function, and evolution, helping researchers identify conserved domains, predict functions, and classify proteins across organisms.

\smallskip \noindent \textbf{PDB}~\cite{bank1971crystallography}: The Protein Data Bank is a repository for the 3D structural data of large biological molecules, such as proteins and nucleic acids. It provides valuable resources for understanding the structural aspects of proteins, which can be beneficial for training models that incorporate structural information.

\smallskip \noindent \textbf{AlphaFoldDB}~\cite{tunyasuvunakool2021highly}: The AlphaFold Protein Structure Database offers predicted protein structures generated by the AlphaFold model containing over 200 million entries.

\subsection{Benchmark Datasets}

\smallskip \noindent \textbf{CASP}~\cite{kryshtafovych2019critical}: Critical Assessment of Structure Prediction is a biennial competition that evaluates methods for protein structure prediction. Participants predict 3D structures of proteins from their sequences, compared against experimental results.

\smallskip \noindent \textbf{ProteinGym}~\cite{notin2024proteingym}: A large-scale benchmark platform for protein design and fitness prediction. It includes over 250 Deep Mutational Scanning (DMS) assays, encompassing millions of mutated protein sequences, and curated clinical datasets with expert annotations. By integrating zero-shot and supervised evaluation frameworks, ProteinGym allows systematic comparison of over 70 machine learning models. It provides standardized metrics for tasks like mutation effect prediction and protein design, fostering innovation in computational biology and protein engineering.

\smallskip \noindent \textbf{TAPE}~\cite{rao2019evaluating}: A benchmark designed to evaluate protein sequence embeddings in biologically relevant tasks using machine learning. It includes five tasks covering structure prediction, evolutionary understanding, and protein engineering. TAPE leverages self-supervised learning, enabling models to learn from unlabeled protein sequences, and offers standardized datasets and metrics for systematic comparisons. It aims to advance protein representation learning by addressing gaps in generalization and real-world applicability.

\smallskip \noindent \textbf{PEER}~\cite{xu2022peer}: A comprehensive and multi-task benchmark designed to evaluate protein sequence understanding. It includes tasks such as protein function prediction, localization prediction, structure prediction, protein-protein interaction prediction, and protein-ligand interaction prediction. 

\smallskip \noindent \textbf{ProteinLMBench}~\cite{shen2024fine}: A benchmark dataset comprising 944 manually verified multiple-choice questions aimed at assessing the protein understanding capabilities of LLMs. It incorporates protein-related details and sequences in multiple languages, setting a new standard for evaluating LLMs’ abilities in protein comprehension.


\section{Evaluation Metrics}
\label{sec:eval}
Comprehensive evaluation is essential for applying \proteinllms, which are assessed on tasks like structure prediction, function prediction, and sequence generation. Appendix~\ref{sec:eval} provides detailed descriptions of structure and function prediction metrics, as well as sequence generation metrics for generative \proteinllms.

\subsection{Structure Prediction Metrics}
\texttt{Root Mean Square Deviation (RMSD)} measures the distance between predicted and actual atomic coordinates, with lower values indicating better accuracy~\cite{li2013difficulty}. \texttt{Global Distance Test (GDT-TS)} calculates the percentage of alpha-carbon atoms within 1, 2, 4, and 8 \r{A} thresholds, reflecting structural similarity~\cite{zemla2003lga}. \texttt{Template Modeling (TM) Score} evaluates global structural similarity (scores between 0 and 1) via 
\begin{gather}
\text{TM}=\max \left[ \frac{1}{L_{\text{tgt}}} \sum_{i}^{L_{\text{com}}} \frac{1}{1+\left( \frac{d_{i}}{\scriptscriptstyle d_{0}(L_{\text{tgt}})} \right)^{2}} \right], \\
d_0(L_{\text{tgt}}) = 1.24 \sqrt[3]{L_{\text{tgt}} - 15} - 1.8.
\end{gather}
\texttt{Local Distance Difference Test (lDDT)} quantifies local accuracy by comparing interatomic distances~\cite{mariani2013lddt}, and \texttt{Predicted Local Distance Difference Test (pLDDT)} provides per-residue confidence scores (0–100) without a reference structure, as used in AlphaFold~\cite{guo2022alphafold2,jumper2021highly}.

\subsection{Function Prediction Metrics}

Protein function prediction determines biological roles, including biomolecular interactions~\cite{radivojac2013large}. Machine learning metrics include classification measures (precision, recall, F-1 score, accuracy, AUC) and generative metrics such as BLEU~\cite{papineni2002bleu}, ROUGE-L~\cite{lin2004rouge}, and METEOR~\cite{banerjee2005meteor}. These evaluation methods offer quantitative benchmarks crucial for model validation and biological inference.

\texttt{Subcellular Localization} predicts proteins' cellular positions to infer functions~\cite{briesemeister2010yloc,holm2020dali}. \texttt{Homology Detection} identifies evolutionary relationships using sequence alignment methods like BLAST~\cite{altschul1990basic} or deep learning approaches such as TM-vec~\cite{hamamsy2024protein}. \texttt{Stability} and \texttt{Solubility} assessments evaluate whether a protein can function effectively in its environment~\cite{cheng2006prediction,hebditch2017protein}, while \texttt{Mutation Effect Prediction} gauges the impact of amino acid changes on protein properties~\cite{mansoor2022accurate}. These integrative metrics underpin the development of robust protein prediction systems and support advancements in drug design and molecular biology.

\vspace{-5pt}
\section{Discussion}
\textbf{Conclusion.}
In this work, we propose the \textit{\methodname{}} metric, $M_{AP}$, to evaluate preference data quality in alignment.
By measuring the gap from the model's current implicit reward margin to the target explicit reward margin, $M_{AP}$ quantifies the discrepancy between the current model and the aligned optimum, thereby indicating the potential for alignment enhancement.
Extensive experiments validate the efficacy of $M_{AP}$ across various training settings under offline and self-play preference learning scenarios.

\textbf{Limitations and future work}. 
Despite the performance improvements, $M_{AP}$ requires tuning a parameter $\beta$ to combine the explicit and implicit margins; future work could explore how to set this ratio automatically.
Additionally, while our experiments focus on the widely applied DPO and SimPO objectives, a broader investigation with alternative preference learning methods is crucial in future works.

% \section{Conclusion}
% In this paper, we introduce the \methodname{} metric to evaluate preference data quality in LLM alignment.
% By measuring the discrepancy between the model's current implicit reward margin to the target explicit reward margin, this metric quantifies the gap between the current model and the aligned optimum, thereby indicating the potential for alignment enhancement.
% Empirical results demonstrate that training on data selected by our metric consistently improves alignment performance, outperforming existing metrics across different base models and training objectives.
% Moreover, this metric extends to data generation scenarios (\ie self-play alignment): by identifying high-quality data from the intrinsic self-generated context, our metric yields superior results across various training settings, providing a comprehensive solution for enhancing LLM alignment through optimized
% preference data generation, selection, and utilization.


\section*{Impact Statement}
This paper presents work whose goal is to advance the field of Machine Learning. There are many potential societal consequences of our work, none which we feel must be specifically highlighted here.


% \bibliography{anthology,custom}
% Custom bibliography entries only
\bibliography{main}
\newpage
\appendix

\label{sec:appendix}
\subsection{Lloyd-Max Algorithm}
\label{subsec:Lloyd-Max}
For a given quantization bitwidth $B$ and an operand $\bm{X}$, the Lloyd-Max algorithm finds $2^B$ quantization levels $\{\hat{x}_i\}_{i=1}^{2^B}$ such that quantizing $\bm{X}$ by rounding each scalar in $\bm{X}$ to the nearest quantization level minimizes the quantization MSE. 

The algorithm starts with an initial guess of quantization levels and then iteratively computes quantization thresholds $\{\tau_i\}_{i=1}^{2^B-1}$ and updates quantization levels $\{\hat{x}_i\}_{i=1}^{2^B}$. Specifically, at iteration $n$, thresholds are set to the midpoints of the previous iteration's levels:
\begin{align*}
    \tau_i^{(n)}=\frac{\hat{x}_i^{(n-1)}+\hat{x}_{i+1}^{(n-1)}}2 \text{ for } i=1\ldots 2^B-1
\end{align*}
Subsequently, the quantization levels are re-computed as conditional means of the data regions defined by the new thresholds:
\begin{align*}
    \hat{x}_i^{(n)}=\mathbb{E}\left[ \bm{X} \big| \bm{X}\in [\tau_{i-1}^{(n)},\tau_i^{(n)}] \right] \text{ for } i=1\ldots 2^B
\end{align*}
where to satisfy boundary conditions we have $\tau_0=-\infty$ and $\tau_{2^B}=\infty$. The algorithm iterates the above steps until convergence.

Figure \ref{fig:lm_quant} compares the quantization levels of a $7$-bit floating point (E3M3) quantizer (left) to a $7$-bit Lloyd-Max quantizer (right) when quantizing a layer of weights from the GPT3-126M model at a per-tensor granularity. As shown, the Lloyd-Max quantizer achieves substantially lower quantization MSE. Further, Table \ref{tab:FP7_vs_LM7} shows the superior perplexity achieved by Lloyd-Max quantizers for bitwidths of $7$, $6$ and $5$. The difference between the quantizers is clear at 5 bits, where per-tensor FP quantization incurs a drastic and unacceptable increase in perplexity, while Lloyd-Max quantization incurs a much smaller increase. Nevertheless, we note that even the optimal Lloyd-Max quantizer incurs a notable ($\sim 1.5$) increase in perplexity due to the coarse granularity of quantization. 

\begin{figure}[h]
  \centering
  \includegraphics[width=0.7\linewidth]{sections/figures/LM7_FP7.pdf}
  \caption{\small Quantization levels and the corresponding quantization MSE of Floating Point (left) vs Lloyd-Max (right) Quantizers for a layer of weights in the GPT3-126M model.}
  \label{fig:lm_quant}
\end{figure}

\begin{table}[h]\scriptsize
\begin{center}
\caption{\label{tab:FP7_vs_LM7} \small Comparing perplexity (lower is better) achieved by floating point quantizers and Lloyd-Max quantizers on a GPT3-126M model for the Wikitext-103 dataset.}
\begin{tabular}{c|cc|c}
\hline
 \multirow{2}{*}{\textbf{Bitwidth}} & \multicolumn{2}{|c|}{\textbf{Floating-Point Quantizer}} & \textbf{Lloyd-Max Quantizer} \\
 & Best Format & Wikitext-103 Perplexity & Wikitext-103 Perplexity \\
\hline
7 & E3M3 & 18.32 & 18.27 \\
6 & E3M2 & 19.07 & 18.51 \\
5 & E4M0 & 43.89 & 19.71 \\
\hline
\end{tabular}
\end{center}
\end{table}

\subsection{Proof of Local Optimality of LO-BCQ}
\label{subsec:lobcq_opt_proof}
For a given block $\bm{b}_j$, the quantization MSE during LO-BCQ can be empirically evaluated as $\frac{1}{L_b}\lVert \bm{b}_j- \bm{\hat{b}}_j\rVert^2_2$ where $\bm{\hat{b}}_j$ is computed from equation (\ref{eq:clustered_quantization_definition}) as $C_{f(\bm{b}_j)}(\bm{b}_j)$. Further, for a given block cluster $\mathcal{B}_i$, we compute the quantization MSE as $\frac{1}{|\mathcal{B}_{i}|}\sum_{\bm{b} \in \mathcal{B}_{i}} \frac{1}{L_b}\lVert \bm{b}- C_i^{(n)}(\bm{b})\rVert^2_2$. Therefore, at the end of iteration $n$, we evaluate the overall quantization MSE $J^{(n)}$ for a given operand $\bm{X}$ composed of $N_c$ block clusters as:
\begin{align*}
    \label{eq:mse_iter_n}
    J^{(n)} = \frac{1}{N_c} \sum_{i=1}^{N_c} \frac{1}{|\mathcal{B}_{i}^{(n)}|}\sum_{\bm{v} \in \mathcal{B}_{i}^{(n)}} \frac{1}{L_b}\lVert \bm{b}- B_i^{(n)}(\bm{b})\rVert^2_2
\end{align*}

At the end of iteration $n$, the codebooks are updated from $\mathcal{C}^{(n-1)}$ to $\mathcal{C}^{(n)}$. However, the mapping of a given vector $\bm{b}_j$ to quantizers $\mathcal{C}^{(n)}$ remains as  $f^{(n)}(\bm{b}_j)$. At the next iteration, during the vector clustering step, $f^{(n+1)}(\bm{b}_j)$ finds new mapping of $\bm{b}_j$ to updated codebooks $\mathcal{C}^{(n)}$ such that the quantization MSE over the candidate codebooks is minimized. Therefore, we obtain the following result for $\bm{b}_j$:
\begin{align*}
\frac{1}{L_b}\lVert \bm{b}_j - C_{f^{(n+1)}(\bm{b}_j)}^{(n)}(\bm{b}_j)\rVert^2_2 \le \frac{1}{L_b}\lVert \bm{b}_j - C_{f^{(n)}(\bm{b}_j)}^{(n)}(\bm{b}_j)\rVert^2_2
\end{align*}

That is, quantizing $\bm{b}_j$ at the end of the block clustering step of iteration $n+1$ results in lower quantization MSE compared to quantizing at the end of iteration $n$. Since this is true for all $\bm{b} \in \bm{X}$, we assert the following:
\begin{equation}
\begin{split}
\label{eq:mse_ineq_1}
    \tilde{J}^{(n+1)} &= \frac{1}{N_c} \sum_{i=1}^{N_c} \frac{1}{|\mathcal{B}_{i}^{(n+1)}|}\sum_{\bm{b} \in \mathcal{B}_{i}^{(n+1)}} \frac{1}{L_b}\lVert \bm{b} - C_i^{(n)}(b)\rVert^2_2 \le J^{(n)}
\end{split}
\end{equation}
where $\tilde{J}^{(n+1)}$ is the the quantization MSE after the vector clustering step at iteration $n+1$.

Next, during the codebook update step (\ref{eq:quantizers_update}) at iteration $n+1$, the per-cluster codebooks $\mathcal{C}^{(n)}$ are updated to $\mathcal{C}^{(n+1)}$ by invoking the Lloyd-Max algorithm \citep{Lloyd}. We know that for any given value distribution, the Lloyd-Max algorithm minimizes the quantization MSE. Therefore, for a given vector cluster $\mathcal{B}_i$ we obtain the following result:

\begin{equation}
    \frac{1}{|\mathcal{B}_{i}^{(n+1)}|}\sum_{\bm{b} \in \mathcal{B}_{i}^{(n+1)}} \frac{1}{L_b}\lVert \bm{b}- C_i^{(n+1)}(\bm{b})\rVert^2_2 \le \frac{1}{|\mathcal{B}_{i}^{(n+1)}|}\sum_{\bm{b} \in \mathcal{B}_{i}^{(n+1)}} \frac{1}{L_b}\lVert \bm{b}- C_i^{(n)}(\bm{b})\rVert^2_2
\end{equation}

The above equation states that quantizing the given block cluster $\mathcal{B}_i$ after updating the associated codebook from $C_i^{(n)}$ to $C_i^{(n+1)}$ results in lower quantization MSE. Since this is true for all the block clusters, we derive the following result: 
\begin{equation}
\begin{split}
\label{eq:mse_ineq_2}
     J^{(n+1)} &= \frac{1}{N_c} \sum_{i=1}^{N_c} \frac{1}{|\mathcal{B}_{i}^{(n+1)}|}\sum_{\bm{b} \in \mathcal{B}_{i}^{(n+1)}} \frac{1}{L_b}\lVert \bm{b}- C_i^{(n+1)}(\bm{b})\rVert^2_2  \le \tilde{J}^{(n+1)}   
\end{split}
\end{equation}

Following (\ref{eq:mse_ineq_1}) and (\ref{eq:mse_ineq_2}), we find that the quantization MSE is non-increasing for each iteration, that is, $J^{(1)} \ge J^{(2)} \ge J^{(3)} \ge \ldots \ge J^{(M)}$ where $M$ is the maximum number of iterations. 
%Therefore, we can say that if the algorithm converges, then it must be that it has converged to a local minimum. 
\hfill $\blacksquare$


\begin{figure}
    \begin{center}
    \includegraphics[width=0.5\textwidth]{sections//figures/mse_vs_iter.pdf}
    \end{center}
    \caption{\small NMSE vs iterations during LO-BCQ compared to other block quantization proposals}
    \label{fig:nmse_vs_iter}
\end{figure}

Figure \ref{fig:nmse_vs_iter} shows the empirical convergence of LO-BCQ across several block lengths and number of codebooks. Also, the MSE achieved by LO-BCQ is compared to baselines such as MXFP and VSQ. As shown, LO-BCQ converges to a lower MSE than the baselines. Further, we achieve better convergence for larger number of codebooks ($N_c$) and for a smaller block length ($L_b$), both of which increase the bitwidth of BCQ (see Eq \ref{eq:bitwidth_bcq}).


\subsection{Additional Accuracy Results}
%Table \ref{tab:lobcq_config} lists the various LOBCQ configurations and their corresponding bitwidths.
\begin{table}
\setlength{\tabcolsep}{4.75pt}
\begin{center}
\caption{\label{tab:lobcq_config} Various LO-BCQ configurations and their bitwidths.}
\begin{tabular}{|c||c|c|c|c||c|c||c|} 
\hline
 & \multicolumn{4}{|c||}{$L_b=8$} & \multicolumn{2}{|c||}{$L_b=4$} & $L_b=2$ \\
 \hline
 \backslashbox{$L_A$\kern-1em}{\kern-1em$N_c$} & 2 & 4 & 8 & 16 & 2 & 4 & 2 \\
 \hline
 64 & 4.25 & 4.375 & 4.5 & 4.625 & 4.375 & 4.625 & 4.625\\
 \hline
 32 & 4.375 & 4.5 & 4.625& 4.75 & 4.5 & 4.75 & 4.75 \\
 \hline
 16 & 4.625 & 4.75& 4.875 & 5 & 4.75 & 5 & 5 \\
 \hline
\end{tabular}
\end{center}
\end{table}

%\subsection{Perplexity achieved by various LO-BCQ configurations on Wikitext-103 dataset}

\begin{table} \centering
\begin{tabular}{|c||c|c|c|c||c|c||c|} 
\hline
 $L_b \rightarrow$& \multicolumn{4}{c||}{8} & \multicolumn{2}{c||}{4} & 2\\
 \hline
 \backslashbox{$L_A$\kern-1em}{\kern-1em$N_c$} & 2 & 4 & 8 & 16 & 2 & 4 & 2  \\
 %$N_c \rightarrow$ & 2 & 4 & 8 & 16 & 2 & 4 & 2 \\
 \hline
 \hline
 \multicolumn{8}{c}{GPT3-1.3B (FP32 PPL = 9.98)} \\ 
 \hline
 \hline
 64 & 10.40 & 10.23 & 10.17 & 10.15 &  10.28 & 10.18 & 10.19 \\
 \hline
 32 & 10.25 & 10.20 & 10.15 & 10.12 &  10.23 & 10.17 & 10.17 \\
 \hline
 16 & 10.22 & 10.16 & 10.10 & 10.09 &  10.21 & 10.14 & 10.16 \\
 \hline
  \hline
 \multicolumn{8}{c}{GPT3-8B (FP32 PPL = 7.38)} \\ 
 \hline
 \hline
 64 & 7.61 & 7.52 & 7.48 &  7.47 &  7.55 &  7.49 & 7.50 \\
 \hline
 32 & 7.52 & 7.50 & 7.46 &  7.45 &  7.52 &  7.48 & 7.48  \\
 \hline
 16 & 7.51 & 7.48 & 7.44 &  7.44 &  7.51 &  7.49 & 7.47  \\
 \hline
\end{tabular}
\caption{\label{tab:ppl_gpt3_abalation} Wikitext-103 perplexity across GPT3-1.3B and 8B models.}
\end{table}

\begin{table} \centering
\begin{tabular}{|c||c|c|c|c||} 
\hline
 $L_b \rightarrow$& \multicolumn{4}{c||}{8}\\
 \hline
 \backslashbox{$L_A$\kern-1em}{\kern-1em$N_c$} & 2 & 4 & 8 & 16 \\
 %$N_c \rightarrow$ & 2 & 4 & 8 & 16 & 2 & 4 & 2 \\
 \hline
 \hline
 \multicolumn{5}{|c|}{Llama2-7B (FP32 PPL = 5.06)} \\ 
 \hline
 \hline
 64 & 5.31 & 5.26 & 5.19 & 5.18  \\
 \hline
 32 & 5.23 & 5.25 & 5.18 & 5.15  \\
 \hline
 16 & 5.23 & 5.19 & 5.16 & 5.14  \\
 \hline
 \multicolumn{5}{|c|}{Nemotron4-15B (FP32 PPL = 5.87)} \\ 
 \hline
 \hline
 64  & 6.3 & 6.20 & 6.13 & 6.08  \\
 \hline
 32  & 6.24 & 6.12 & 6.07 & 6.03  \\
 \hline
 16  & 6.12 & 6.14 & 6.04 & 6.02  \\
 \hline
 \multicolumn{5}{|c|}{Nemotron4-340B (FP32 PPL = 3.48)} \\ 
 \hline
 \hline
 64 & 3.67 & 3.62 & 3.60 & 3.59 \\
 \hline
 32 & 3.63 & 3.61 & 3.59 & 3.56 \\
 \hline
 16 & 3.61 & 3.58 & 3.57 & 3.55 \\
 \hline
\end{tabular}
\caption{\label{tab:ppl_llama7B_nemo15B} Wikitext-103 perplexity compared to FP32 baseline in Llama2-7B and Nemotron4-15B, 340B models}
\end{table}

%\subsection{Perplexity achieved by various LO-BCQ configurations on MMLU dataset}


\begin{table} \centering
\begin{tabular}{|c||c|c|c|c||c|c|c|c|} 
\hline
 $L_b \rightarrow$& \multicolumn{4}{c||}{8} & \multicolumn{4}{c||}{8}\\
 \hline
 \backslashbox{$L_A$\kern-1em}{\kern-1em$N_c$} & 2 & 4 & 8 & 16 & 2 & 4 & 8 & 16  \\
 %$N_c \rightarrow$ & 2 & 4 & 8 & 16 & 2 & 4 & 2 \\
 \hline
 \hline
 \multicolumn{5}{|c|}{Llama2-7B (FP32 Accuracy = 45.8\%)} & \multicolumn{4}{|c|}{Llama2-70B (FP32 Accuracy = 69.12\%)} \\ 
 \hline
 \hline
 64 & 43.9 & 43.4 & 43.9 & 44.9 & 68.07 & 68.27 & 68.17 & 68.75 \\
 \hline
 32 & 44.5 & 43.8 & 44.9 & 44.5 & 68.37 & 68.51 & 68.35 & 68.27  \\
 \hline
 16 & 43.9 & 42.7 & 44.9 & 45 & 68.12 & 68.77 & 68.31 & 68.59  \\
 \hline
 \hline
 \multicolumn{5}{|c|}{GPT3-22B (FP32 Accuracy = 38.75\%)} & \multicolumn{4}{|c|}{Nemotron4-15B (FP32 Accuracy = 64.3\%)} \\ 
 \hline
 \hline
 64 & 36.71 & 38.85 & 38.13 & 38.92 & 63.17 & 62.36 & 63.72 & 64.09 \\
 \hline
 32 & 37.95 & 38.69 & 39.45 & 38.34 & 64.05 & 62.30 & 63.8 & 64.33  \\
 \hline
 16 & 38.88 & 38.80 & 38.31 & 38.92 & 63.22 & 63.51 & 63.93 & 64.43  \\
 \hline
\end{tabular}
\caption{\label{tab:mmlu_abalation} Accuracy on MMLU dataset across GPT3-22B, Llama2-7B, 70B and Nemotron4-15B models.}
\end{table}


%\subsection{Perplexity achieved by various LO-BCQ configurations on LM evaluation harness}

\begin{table} \centering
\begin{tabular}{|c||c|c|c|c||c|c|c|c|} 
\hline
 $L_b \rightarrow$& \multicolumn{4}{c||}{8} & \multicolumn{4}{c||}{8}\\
 \hline
 \backslashbox{$L_A$\kern-1em}{\kern-1em$N_c$} & 2 & 4 & 8 & 16 & 2 & 4 & 8 & 16  \\
 %$N_c \rightarrow$ & 2 & 4 & 8 & 16 & 2 & 4 & 2 \\
 \hline
 \hline
 \multicolumn{5}{|c|}{Race (FP32 Accuracy = 37.51\%)} & \multicolumn{4}{|c|}{Boolq (FP32 Accuracy = 64.62\%)} \\ 
 \hline
 \hline
 64 & 36.94 & 37.13 & 36.27 & 37.13 & 63.73 & 62.26 & 63.49 & 63.36 \\
 \hline
 32 & 37.03 & 36.36 & 36.08 & 37.03 & 62.54 & 63.51 & 63.49 & 63.55  \\
 \hline
 16 & 37.03 & 37.03 & 36.46 & 37.03 & 61.1 & 63.79 & 63.58 & 63.33  \\
 \hline
 \hline
 \multicolumn{5}{|c|}{Winogrande (FP32 Accuracy = 58.01\%)} & \multicolumn{4}{|c|}{Piqa (FP32 Accuracy = 74.21\%)} \\ 
 \hline
 \hline
 64 & 58.17 & 57.22 & 57.85 & 58.33 & 73.01 & 73.07 & 73.07 & 72.80 \\
 \hline
 32 & 59.12 & 58.09 & 57.85 & 58.41 & 73.01 & 73.94 & 72.74 & 73.18  \\
 \hline
 16 & 57.93 & 58.88 & 57.93 & 58.56 & 73.94 & 72.80 & 73.01 & 73.94  \\
 \hline
\end{tabular}
\caption{\label{tab:mmlu_abalation} Accuracy on LM evaluation harness tasks on GPT3-1.3B model.}
\end{table}

\begin{table} \centering
\begin{tabular}{|c||c|c|c|c||c|c|c|c|} 
\hline
 $L_b \rightarrow$& \multicolumn{4}{c||}{8} & \multicolumn{4}{c||}{8}\\
 \hline
 \backslashbox{$L_A$\kern-1em}{\kern-1em$N_c$} & 2 & 4 & 8 & 16 & 2 & 4 & 8 & 16  \\
 %$N_c \rightarrow$ & 2 & 4 & 8 & 16 & 2 & 4 & 2 \\
 \hline
 \hline
 \multicolumn{5}{|c|}{Race (FP32 Accuracy = 41.34\%)} & \multicolumn{4}{|c|}{Boolq (FP32 Accuracy = 68.32\%)} \\ 
 \hline
 \hline
 64 & 40.48 & 40.10 & 39.43 & 39.90 & 69.20 & 68.41 & 69.45 & 68.56 \\
 \hline
 32 & 39.52 & 39.52 & 40.77 & 39.62 & 68.32 & 67.43 & 68.17 & 69.30  \\
 \hline
 16 & 39.81 & 39.71 & 39.90 & 40.38 & 68.10 & 66.33 & 69.51 & 69.42  \\
 \hline
 \hline
 \multicolumn{5}{|c|}{Winogrande (FP32 Accuracy = 67.88\%)} & \multicolumn{4}{|c|}{Piqa (FP32 Accuracy = 78.78\%)} \\ 
 \hline
 \hline
 64 & 66.85 & 66.61 & 67.72 & 67.88 & 77.31 & 77.42 & 77.75 & 77.64 \\
 \hline
 32 & 67.25 & 67.72 & 67.72 & 67.00 & 77.31 & 77.04 & 77.80 & 77.37  \\
 \hline
 16 & 68.11 & 68.90 & 67.88 & 67.48 & 77.37 & 78.13 & 78.13 & 77.69  \\
 \hline
\end{tabular}
\caption{\label{tab:mmlu_abalation} Accuracy on LM evaluation harness tasks on GPT3-8B model.}
\end{table}

\begin{table} \centering
\begin{tabular}{|c||c|c|c|c||c|c|c|c|} 
\hline
 $L_b \rightarrow$& \multicolumn{4}{c||}{8} & \multicolumn{4}{c||}{8}\\
 \hline
 \backslashbox{$L_A$\kern-1em}{\kern-1em$N_c$} & 2 & 4 & 8 & 16 & 2 & 4 & 8 & 16  \\
 %$N_c \rightarrow$ & 2 & 4 & 8 & 16 & 2 & 4 & 2 \\
 \hline
 \hline
 \multicolumn{5}{|c|}{Race (FP32 Accuracy = 40.67\%)} & \multicolumn{4}{|c|}{Boolq (FP32 Accuracy = 76.54\%)} \\ 
 \hline
 \hline
 64 & 40.48 & 40.10 & 39.43 & 39.90 & 75.41 & 75.11 & 77.09 & 75.66 \\
 \hline
 32 & 39.52 & 39.52 & 40.77 & 39.62 & 76.02 & 76.02 & 75.96 & 75.35  \\
 \hline
 16 & 39.81 & 39.71 & 39.90 & 40.38 & 75.05 & 73.82 & 75.72 & 76.09  \\
 \hline
 \hline
 \multicolumn{5}{|c|}{Winogrande (FP32 Accuracy = 70.64\%)} & \multicolumn{4}{|c|}{Piqa (FP32 Accuracy = 79.16\%)} \\ 
 \hline
 \hline
 64 & 69.14 & 70.17 & 70.17 & 70.56 & 78.24 & 79.00 & 78.62 & 78.73 \\
 \hline
 32 & 70.96 & 69.69 & 71.27 & 69.30 & 78.56 & 79.49 & 79.16 & 78.89  \\
 \hline
 16 & 71.03 & 69.53 & 69.69 & 70.40 & 78.13 & 79.16 & 79.00 & 79.00  \\
 \hline
\end{tabular}
\caption{\label{tab:mmlu_abalation} Accuracy on LM evaluation harness tasks on GPT3-22B model.}
\end{table}

\begin{table} \centering
\begin{tabular}{|c||c|c|c|c||c|c|c|c|} 
\hline
 $L_b \rightarrow$& \multicolumn{4}{c||}{8} & \multicolumn{4}{c||}{8}\\
 \hline
 \backslashbox{$L_A$\kern-1em}{\kern-1em$N_c$} & 2 & 4 & 8 & 16 & 2 & 4 & 8 & 16  \\
 %$N_c \rightarrow$ & 2 & 4 & 8 & 16 & 2 & 4 & 2 \\
 \hline
 \hline
 \multicolumn{5}{|c|}{Race (FP32 Accuracy = 44.4\%)} & \multicolumn{4}{|c|}{Boolq (FP32 Accuracy = 79.29\%)} \\ 
 \hline
 \hline
 64 & 42.49 & 42.51 & 42.58 & 43.45 & 77.58 & 77.37 & 77.43 & 78.1 \\
 \hline
 32 & 43.35 & 42.49 & 43.64 & 43.73 & 77.86 & 75.32 & 77.28 & 77.86  \\
 \hline
 16 & 44.21 & 44.21 & 43.64 & 42.97 & 78.65 & 77 & 76.94 & 77.98  \\
 \hline
 \hline
 \multicolumn{5}{|c|}{Winogrande (FP32 Accuracy = 69.38\%)} & \multicolumn{4}{|c|}{Piqa (FP32 Accuracy = 78.07\%)} \\ 
 \hline
 \hline
 64 & 68.9 & 68.43 & 69.77 & 68.19 & 77.09 & 76.82 & 77.09 & 77.86 \\
 \hline
 32 & 69.38 & 68.51 & 68.82 & 68.90 & 78.07 & 76.71 & 78.07 & 77.86  \\
 \hline
 16 & 69.53 & 67.09 & 69.38 & 68.90 & 77.37 & 77.8 & 77.91 & 77.69  \\
 \hline
\end{tabular}
\caption{\label{tab:mmlu_abalation} Accuracy on LM evaluation harness tasks on Llama2-7B model.}
\end{table}

\begin{table} \centering
\begin{tabular}{|c||c|c|c|c||c|c|c|c|} 
\hline
 $L_b \rightarrow$& \multicolumn{4}{c||}{8} & \multicolumn{4}{c||}{8}\\
 \hline
 \backslashbox{$L_A$\kern-1em}{\kern-1em$N_c$} & 2 & 4 & 8 & 16 & 2 & 4 & 8 & 16  \\
 %$N_c \rightarrow$ & 2 & 4 & 8 & 16 & 2 & 4 & 2 \\
 \hline
 \hline
 \multicolumn{5}{|c|}{Race (FP32 Accuracy = 48.8\%)} & \multicolumn{4}{|c|}{Boolq (FP32 Accuracy = 85.23\%)} \\ 
 \hline
 \hline
 64 & 49.00 & 49.00 & 49.28 & 48.71 & 82.82 & 84.28 & 84.03 & 84.25 \\
 \hline
 32 & 49.57 & 48.52 & 48.33 & 49.28 & 83.85 & 84.46 & 84.31 & 84.93  \\
 \hline
 16 & 49.85 & 49.09 & 49.28 & 48.99 & 85.11 & 84.46 & 84.61 & 83.94  \\
 \hline
 \hline
 \multicolumn{5}{|c|}{Winogrande (FP32 Accuracy = 79.95\%)} & \multicolumn{4}{|c|}{Piqa (FP32 Accuracy = 81.56\%)} \\ 
 \hline
 \hline
 64 & 78.77 & 78.45 & 78.37 & 79.16 & 81.45 & 80.69 & 81.45 & 81.5 \\
 \hline
 32 & 78.45 & 79.01 & 78.69 & 80.66 & 81.56 & 80.58 & 81.18 & 81.34  \\
 \hline
 16 & 79.95 & 79.56 & 79.79 & 79.72 & 81.28 & 81.66 & 81.28 & 80.96  \\
 \hline
\end{tabular}
\caption{\label{tab:mmlu_abalation} Accuracy on LM evaluation harness tasks on Llama2-70B model.}
\end{table}

%\section{MSE Studies}
%\textcolor{red}{TODO}


\subsection{Number Formats and Quantization Method}
\label{subsec:numFormats_quantMethod}
\subsubsection{Integer Format}
An $n$-bit signed integer (INT) is typically represented with a 2s-complement format \citep{yao2022zeroquant,xiao2023smoothquant,dai2021vsq}, where the most significant bit denotes the sign.

\subsubsection{Floating Point Format}
An $n$-bit signed floating point (FP) number $x$ comprises of a 1-bit sign ($x_{\mathrm{sign}}$), $B_m$-bit mantissa ($x_{\mathrm{mant}}$) and $B_e$-bit exponent ($x_{\mathrm{exp}}$) such that $B_m+B_e=n-1$. The associated constant exponent bias ($E_{\mathrm{bias}}$) is computed as $(2^{{B_e}-1}-1)$. We denote this format as $E_{B_e}M_{B_m}$.  

\subsubsection{Quantization Scheme}
\label{subsec:quant_method}
A quantization scheme dictates how a given unquantized tensor is converted to its quantized representation. We consider FP formats for the purpose of illustration. Given an unquantized tensor $\bm{X}$ and an FP format $E_{B_e}M_{B_m}$, we first, we compute the quantization scale factor $s_X$ that maps the maximum absolute value of $\bm{X}$ to the maximum quantization level of the $E_{B_e}M_{B_m}$ format as follows:
\begin{align}
\label{eq:sf}
    s_X = \frac{\mathrm{max}(|\bm{X}|)}{\mathrm{max}(E_{B_e}M_{B_m})}
\end{align}
In the above equation, $|\cdot|$ denotes the absolute value function.

Next, we scale $\bm{X}$ by $s_X$ and quantize it to $\hat{\bm{X}}$ by rounding it to the nearest quantization level of $E_{B_e}M_{B_m}$ as:

\begin{align}
\label{eq:tensor_quant}
    \hat{\bm{X}} = \text{round-to-nearest}\left(\frac{\bm{X}}{s_X}, E_{B_e}M_{B_m}\right)
\end{align}

We perform dynamic max-scaled quantization \citep{wu2020integer}, where the scale factor $s$ for activations is dynamically computed during runtime.

\subsection{Vector Scaled Quantization}
\begin{wrapfigure}{r}{0.35\linewidth}
  \centering
  \includegraphics[width=\linewidth]{sections/figures/vsquant.jpg}
  \caption{\small Vectorwise decomposition for per-vector scaled quantization (VSQ \citep{dai2021vsq}).}
  \label{fig:vsquant}
\end{wrapfigure}
During VSQ \citep{dai2021vsq}, the operand tensors are decomposed into 1D vectors in a hardware friendly manner as shown in Figure \ref{fig:vsquant}. Since the decomposed tensors are used as operands in matrix multiplications during inference, it is beneficial to perform this decomposition along the reduction dimension of the multiplication. The vectorwise quantization is performed similar to tensorwise quantization described in Equations \ref{eq:sf} and \ref{eq:tensor_quant}, where a scale factor $s_v$ is required for each vector $\bm{v}$ that maps the maximum absolute value of that vector to the maximum quantization level. While smaller vector lengths can lead to larger accuracy gains, the associated memory and computational overheads due to the per-vector scale factors increases. To alleviate these overheads, VSQ \citep{dai2021vsq} proposed a second level quantization of the per-vector scale factors to unsigned integers, while MX \citep{rouhani2023shared} quantizes them to integer powers of 2 (denoted as $2^{INT}$).

\subsubsection{MX Format}
The MX format proposed in \citep{rouhani2023microscaling} introduces the concept of sub-block shifting. For every two scalar elements of $b$-bits each, there is a shared exponent bit. The value of this exponent bit is determined through an empirical analysis that targets minimizing quantization MSE. We note that the FP format $E_{1}M_{b}$ is strictly better than MX from an accuracy perspective since it allocates a dedicated exponent bit to each scalar as opposed to sharing it across two scalars. Therefore, we conservatively bound the accuracy of a $b+2$-bit signed MX format with that of a $E_{1}M_{b}$ format in our comparisons. For instance, we use E1M2 format as a proxy for MX4.

\begin{figure}
    \centering
    \includegraphics[width=1\linewidth]{sections//figures/BlockFormats.pdf}
    \caption{\small Comparing LO-BCQ to MX format.}
    \label{fig:block_formats}
\end{figure}

Figure \ref{fig:block_formats} compares our $4$-bit LO-BCQ block format to MX \citep{rouhani2023microscaling}. As shown, both LO-BCQ and MX decompose a given operand tensor into block arrays and each block array into blocks. Similar to MX, we find that per-block quantization ($L_b < L_A$) leads to better accuracy due to increased flexibility. While MX achieves this through per-block $1$-bit micro-scales, we associate a dedicated codebook to each block through a per-block codebook selector. Further, MX quantizes the per-block array scale-factor to E8M0 format without per-tensor scaling. In contrast during LO-BCQ, we find that per-tensor scaling combined with quantization of per-block array scale-factor to E4M3 format results in superior inference accuracy across models. 

\section{Evaluation Metrics}
\label{sec:app_eval}
Comprehensive and accurate evaluation is essential for understanding and applying \proteinllms. Currently, these models are commonly assessed on tasks such as structure prediction, function prediction, and sequence generation.

\subsection{Structure Prediction Metrics}
Structure prediction evaluates how accurately a model predicts a protein’s three-dimensional structure from its sequence~\cite{kuhlman2019advances}. Common metrics include:

\smallskip\noindent \textit{\textbf{Root Mean Square Deviation (RMSD)}} measures the distance between the predicted and actual atomic coordinates. Lower RMSD indicates higher structural accuracy~\cite{li2013difficulty}.


\smallskip\noindent \textit{\textbf{Global Distance Test (GDT-TS)}} calculates the percentage of alpha-carbon atoms within thresholds (1, 2, 4, and 8 \r{A}) of the reference structure after iterative superimposition~\cite{zemla2003lga}.

GDT-TS usually uses thresholds of 1, 2, 4, and 8 \r{A}. The higher the GDT-TS score, the closer the predicted structure is to the reference structure.


\smallskip\noindent \textit{\textbf{Template Modeling (TM) Score}} evaluates the global structural similarity of proteins with values ranging from 0 to 1~\cite{zhang2004scoring}. 
\begin{gather}
\text{TM}=\max \left[ \frac{1}{L_{\text{tgt}}} \sum_{i}^{L_{\text{com}}} \frac{1}{1+\left( \frac{d_{i}}{\scriptscriptstyle d_{0}(L_{\text{tgt}})} \right)^{2}} \right], \\
d_0(L_{\text{tgt}}) = 1.24 \sqrt[3]{L_{\text{tgt}} - 15} - 1.8.
\end{gather}
Here, $L_{\text{tgt}}$ is the length of the target protein amino acid sequence. $L_{\text{com}}$ is the number of residues in the template and target structures. $d_i$ represents the distance between the $i$-th residue pair in the template structure and the target structure. Higher scores indicate closer similarity. 

\smallskip\noindent \textit{\textbf{lDDT}}, Local Distance Difference Test, evaluates the local accuracy of protein structure prediction by comparing distances between atom pairs in the predicted structures and those in the reference structures~\cite{mariani2013lddt}.

A distance is considered preserved if it falls within a specified threshold. lDDT is calculated as the proportion of preserved distances, with higher values indicating better local accuracy. 

\smallskip\noindent \textit{\textbf{pLDDT}}, Predicted Local Distance Difference Test, is a per-residue measure of local confidence~\cite{guo2022alphafold2}. pLDDT evaluates the local quality of the predicted structure without a reference structure. Its computation usually relies on models such as AlphaFold \cite{jumper2021highly}, which learns patterns from large-scale protein data. Scores range from 0 to 100, with higher scores indicating greater confidence and more accurate predictions.

\subsection{Function Prediction Metrics}
Protein function prediction aims to determine biological roles, including interactions with other biomolecules \cite{radivojac2013large}. The evaluation methods involve machine learning performance metrics and biomedical relevance validation.

Machine learning evaluation metrics can be categorized into classification task metrics and generative task metrics. For classification tasks, such as protein classification and interaction prediction, standard metrics can be adopted, such as precision, recall, F-1 scores, accuracy, and area under the curve (AUC). For generative tasks, such as question answering, evaluation is performed by measuring the alignment between the LLM's output and the ground truth using metrics such as BLEU \cite{papineni2002bleu}, ROUGE-L \cite{lin2004rouge}, and METEOR \cite{banerjee2005meteor}.


In addition to machine learning metrics, there are also biometric-related evaluation metrics:

\smallskip\noindent \textit{\textbf{Subcellular Localization}} refers to the specific location of proteins within a cell \cite{briesemeister2010yloc}. The location of a protein is closely related to the function it performs, so by predicting the subcellular localization of a protein, it is possible to speculate on the biological function it may have \cite{holm2020dali}.

\smallskip\noindent \textit{\textbf{Homology Detection}} aims to identify proteins that share an evolutionary relationship (homologous) with the target protein, usually reflected in similarities in  sequences, structure, and functions. Traditional methods such as BLAST \cite{altschul1990basic} 
perform sequence alignment to identify homologs by comparing the query sequence against a database.

Recent deep learning approaches such as TM-vec \cite{hamamsy2024protein} focus on structural similarity and generate vector representations of proteins.

\smallskip\noindent \textit{\textbf{Stability}} of the protein is critical for many applications, such as drug development. Predicting the stability of a protein can help determine whether the protein can perform its function efficiently in the cellular environment \cite{cheng2006prediction}.

\smallskip\noindent \textit{\textbf{Solubility}} reflects the solubility characteristics of a protein in a particular solvent. Predictions of solubility can help to understand whether a protein can exist and function properly within a cell \cite{hebditch2017protein}.

\smallskip\noindent \textit{\textbf{Mutation Effect Prediction}} of proteins refers to the assessment of the impact on various properties, structures, and functions of proteins when their amino acid sequences are changed \cite{mansoor2022accurate}. Commonly used methods include molecular dynamics-based methods, deep learning-based prediction models, and structural comparison methods.

\subsection{Sequence Generation Metrics} 
Protein sequence generation is the process of creating new protein sequences using specific methods, models, or algorithms \cite{anand2022protein}. Common evaluation methods include:

\smallskip\noindent \textit{\textbf{Perplexity (PPL)}} can be used to measure how accurately a model predicts amino acids \cite{hesslow2022rita}. The lower the perplexity, the more accurate the prediction.

\smallskip\noindent \textit{\textbf{Novelty}} refers to the degree of uniqueness of the generated protein sequence compared to a database of known protein sequences \cite{truong2023poet}.

\smallskip\noindent \textit{\textbf{Fr\'echet Protein Distance (FPD)}} is used to measure the similarity between the distribution represented by the generated protein sequence and the distribution of the real protein sequence~\citep{jiang2008protein}, denoted as: 
\begin{equation}
\delta_{\mathcal{F}}(f, g) = \inf_{\alpha, \beta} \max_{s \in [0,1]} \text{dist}(f(\alpha(s)), g(\beta(s)))
\end{equation}
where $\alpha$ and $\beta$ are continuous non-decreasing functions. The sequence distribution can be denoted by $f$ and $g$.

\smallskip\noindent \textit{\textbf{Diversity}} is designed to evaluate the degree of difference between a range of protein sequences generated by a model. Rich diversity means that the model is capable of generating a variety of different sequences. Common methods include Shannon Entropy \cite{bywater2015prediction} and Hamming Distance \cite{mcgee2021generative}.

\smallskip\noindent \textit{\textbf{Foldability}} focuses on whether the generated protein sequence can be folded into a stable three-dimensional structure. Measuring foldability is usually performed with tools such as RoseTTAFold \cite{baek2021accurate} or computational methods based on physicochemical principles \cite{magliery2015protein} to predict the likelihood that the generated sequence will form a stable structure.

\smallskip\noindent \textit{\textbf{Recovery}} is focused on the ability of a model to predict the corresponding sequence for a given structure accurately \cite{watson2023novo}. Evaluating recovery includes methods sequence comparison, structure comparison, functionality comparison, etc.





\end{document}