\section{Related Work}
\label{sec:related}

In ____ authors propose a method for extracting high-level activities from low-level event logs of program execution. To achieve this, they developed a tool called \textsc{Procfiler}, which collects events that occur during the execution of programs written in the C\# programming language in the .NET CLR runtime environment and creates a log from them. They then apply a predefined hierarchy to raise the abstraction level of the events. An example of such a hierarchy is shown in \Cref{fig:hierarchy}. The root is an artificially added, most abstract event. For instance, \textit{AssemblyLoader/Start\_System\_Threading} represents a very specific and detailed event. This event can be abstracted to a higher level by naming it \textit{AssemblyLoader/Start}.

\begin{figure}
  \centering
  \includegraphics[scale=0.3]{figures/activity_hierarchy.png}
  \caption{An example of a predefined hierarchy used to raise the abstraction level of low-level event logs, where "root" is the most abstract event and \textit{AssemblyLoader/Start\_System\_Threading} is the most specific, detailed event.}
  \Description{}
  \label{fig:hierarchy}
\end{figure}

In our project, we utilized the results of the Procfiler tool, specifically logs with the lowest level of abstraction, meaning they contain events that are the leaves in the hierarchical tree shown in Figure 1.

For the task of anomaly detection in event logs, supervised learning methods have been applied, treating this task as a binary classification problem ____. However, this significantly reduces the applicability of such methods in real systems, as it requires a prelabeled dataset, and more importantly, limits the ability to detect previously unseen anomalies. There are works that use unsupervised learning approaches ____ based on LSTM, as well as the \textsc{BERT} model ____, which is based on the transformer architecture and employs preliminary tokenization of the event log.

To the best of our knowledge, no previous study has investigated the applicability of NLP methods for automated pattern and anomaly detection for the domain of low-level event .NET CLR logs.