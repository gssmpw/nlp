\begin{abstract}

As autonomous driving systems (ADSes) become increasingly complex and integral to daily life, the importance of understanding the nature and mitigation of software bugs in these systems has grown correspondingly. 
\yuntianc{Addressing the challenges of software maintenance in autonomous driving systems (e.g., handling real-time system decisions and ensuring safety-critical reliability) is crucial due to the unique combination of real-time decision-making requirements and the high stakes of operational failures in ADSes.}
The potential of automated tools in this domain is promising, yet there remains a gap in our comprehension of the challenges faced and the strategies employed during manual debugging and repair of such systems.
In this paper, we present an empirical study that investigates \bfps in ADSes, with the aim of improving reliability and safety. We have analyzed the commit histories and bug reports of two major autonomous driving projects, Apollo and Autoware, from \numbug bug fixes with the study of bug symptoms, root causes, and \bfps.
Our study reveals several dominant \bfps, including those related to path planning, data flow, and configuration management. Additionally, we find that the frequency distribution of \bfps varies significantly depending on their nature and types and that certain categories of bugs are recurrent and more challenging to exterminate. 
Based on our findings, we propose a hierarchy of ADS bugs and two taxonomies of \numsyn syntactic \bfps and \numsem semantic \bfps that offer \yuntianyihl{guidance} for bug identification and resolution. 
\yuntianyihl{We also contribute a benchmark of \numbug ADS bug-fix instances.}

\end{abstract}
