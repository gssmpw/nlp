\section{Background}\label{sec:background}

\subsection{Autonomous Driving Systems}

An ADS aims to achieve high automation levels for vehicles to automatically run on roads. 
The Society of Automotive Engineers defines six levels of autonomous driving, from Level 0, with no assistance systems, to Level 5, which represents fully autonomous driving \cite{RodelSMT14}. 
Baidu Apollo \cite{apollo} and Autoware~\cite{autoware} achieve Level 4~\cite{apollo_level_4}, which refers to a high degree of automation where the vehicle can handle all aspects of driving in certain environments without human intervention, but a human override is still an option.
The core modules and components are listed in \autoref{tab:modules_ads}.


\begin{table}[ht]
    \centering
    \caption{Modules in Autonomous Driving Systems}
    \begin{adjustbox}{width=0.9\linewidth}
    \begin{tabular}{|l|l|}
    \hline
    \textbf{Module} & \textbf{Description} \\
    \hline
    Planning & Makes decisions for the AV to execute, such as cruising or stopping. \\
    \hline
    Perception & Processes data from the surrounding environment detected by sensors. \\
    \hline
    Prediction & Receives obstacle information and predicts its future motion. \\
    \hline
    Control & Enforces the planned trajectory with lateral and longitudinal control. \\
    \hline
    Localization & Provides location, heading, and velocity information of the AV. \\
    \hline
    Simulator & Tests the ADS in a virtual environment, replicating real-world scenarios. \\
    \hline
    Sensing & Detects obstacles and traffic to understand the environment. \\
    \hline
    CAN Bus & Handles communication between software and vehicle. \\
    \hline
    HMI & Collects and visualizes the status and interfaces of the system. \\
    \hline
    HD Map & Includes lane geometries and locations of traffic control devices. \\
    \hline
    System & Coordinates the integration and operation of all AV modules. \\
    \hline
    Infrastructure & Includes robotics middleware to support communication among modules. \\
    \hline
    Utilities\&Tools & Provides necessary utilities, such as sensor calibration tools. \\
    \hline
    Docker & Includes the Docker image housing an instance of the ADS. \\
    \hline
    Build & Compiles and integrates the AV software, enabling smooth deployment. \\
    \hline
    Drivers & Contains the hardware drivers necessary for operating the AV. \\
    \hline
    Documentation & Provides guides, specifications, and manuals detailing the ADS. \\
    \hline
    \end{tabular}
    \end{adjustbox}
    \label{tab:modules_ads}
\end{table}


\subsection{Bug Study of Autonomous Driving Systems}

Understanding the nature of defects in ADSes is essential for enhancing their safety and performance. A comprehensive analysis of prior work \cite{GarciaF0AXC20} provides a foundational understanding by categorizing ADS bugs from the root causes that reflect errors in code and symptoms manifested as incorrect behaviors or errors during runtime as presented in \autoref{tab:root_causes_ads} and \autoref{tab:symptoms_ads}.

\vspace{-1.5ex}
\begin{table}[ht!]
    \centering
    \begin{minipage}{0.52\linewidth}
        \centering
        \caption{Root Causes of Bugs in the ADS}
        \begin{adjustbox}{width=\linewidth}
        \begin{tabular}{|l|l|}
        \hline
        \textbf{Type} & \textbf{Description} \\
        \hline
        Alg & Flawed logic that requires a comprehensive correction. \\
        \hline
        Num & Incorrect numerical calculations, values, or usage. \\
        \hline
        Assi & Wrong variable assignments or initializations. \\
        \hline
        MCC & Absence of necessary conditional statements. \\
        \hline
        Data & Incorrect data structure definitions or pointer misuse. \\
        \hline
        Exter-API & Incorrect usage of interfaces of other systems or libraries. \\
        \hline
        Inter-API & Misuse of interfaces of other components within the ADS. \\
        \hline
        ICL & Incorrect condition logic or faulty conditional expressions. \\
        \hline
        Conc & Mismanagement of concurrency mechanisms like threads. \\
        \hline
        Mem & Misuse of memory or improper memory management. \\
        \hline
        Doc & Incorrect manuals, tutorials, or code comments. \\
        \hline
        Config & Incorrect system setup or build settings in configurations. \\
        \hline
        OT & Infrequent and uncategorized issues. \\
        \hline
        \end{tabular}
        \end{adjustbox}
        \label{tab:root_causes_ads}
    \end{minipage}
    \hfill
    \begin{minipage}{0.47\linewidth}
        \centering
        \caption{Symptoms of Bugs in the ADS}
        \begin{adjustbox}{width=\linewidth}
        \begin{tabular}{|l|l|}
        \hline
        \textbf{Type} & \textbf{Description} \\
        \hline
        Crashes & Improper termination of an ADS. \\
        \hline
        Hangs & The system becomes unresponsive but remains running. \\
        \hline
        Build & Errors hinder the compilation or installation of ADS modules. \\
        \hline
        DGUI & Erroneous output on a GUI, visualization, or HMI. \\
        \hline
        Cam & Errors prevent image capture by the camera. \\
        \hline
        Stop & Incorrect behaviors during stopping or parking. \\
        \hline
        LPN & Incorrect lane positioning, maintenance, or navigation. \\
        \hline
        SVC & Incorrect AV speed and velocity management. \\
        \hline
        TLP & Incorrect behaviors involving handling of traffic lights. \\
        \hline
        Lau & ADS or component fails to start or launch. \\
        \hline
        Turn & AV behaves incorrectly when making a turn. \\
        \hline
        Traj & Incorrect trajectory prediction results. \\
        \hline
        IO & Incorrect input/output operations to files or devices. \\
        \hline
        LOC & Inaccurate multi-sensor fusion-based localization. \\
        \hline
        SS & Behaviors that affect the safety or security of its passengers. \\
        \hline
        OP & Incorrect processing or handling of obstacles on the road. \\
        \hline
        Logic & Incorrect behaviors that do not terminate the program. \\
        \hline
        Doc & Errors in the documentation or comments. \\
        \hline
        UN & Symptoms that are unreported and cannot be identified. \\
        \hline
        OT & Infrequent symptoms that do not fit into above categories. \\
        \hline
        \end{tabular}
        \end{adjustbox}
        \label{tab:symptoms_ads}
    \end{minipage}
\end{table}


\subsection{\BFPs}

The concept of \bfps (BFPs) emerges as a critical tool in the software engineering domain, offering a structured approach to diagnosing and rectifying common defects in software systems~\cite{PanKW09,CamposM17}. These patterns, distilled from the collective experience of developers, encapsulate proven solutions to recurring problems, thereby streamlining the debugging and maintenance processes. In the context of ADS, where software reliability and safety are essential, the identification and application of BFPs become even more vital.
Autonomous driving systems, with their integration of complex algorithms, sensor data processing, and real-time decision-making, present a unique set of challenges for software maintenance. The diversity of bugs in ADS---ranging from algorithmic inaccuracies and sensor data misinterpretations to issues in system integration and performance---necessitates a specialized set of BFPs tailored to these specific problems. For instance, a \bfp aimed at addressing inaccuracies in environmental perception might involve strategies for sensor data fusion enhancement or algorithmic refinement for object detection.
