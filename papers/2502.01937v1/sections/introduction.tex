\section{Introduction}\label{sec:introduction}

The rise of Autonomous Vehicles (AVs) signifies a crucial transformation in the realm of transportation. With more than 50 companies such as Ford, Toyota, Tesla, and Waymo~\cite{waymo,av_ford,av_intel,40_plus_corporations,18_av_companies,tesla_sold_2_million_cars,toyota_av,gm-av_charging} actively developing AVs, these vehicles are swiftly becoming a significant part of our daily lives.
This surge in development and deployment underscores the importance of understanding and maintaining the software that powers these complex cyber-physical systems. 
In the intricate ecosystem of AV software development, a critical yet often overlooked aspect is the pattern of bug fixes. While the initial focus of AV development has been on achieving functional and Operational Design Domain (ODD)~\cite{czarnecki2018operational} goal which specifies the operating conditions under which an ADS can operate safely, it is equally important to understand and repair the software bugs that inevitably arise. 
\yuntianyihl{We define a \bfp (BFP) as a recurring repair solution applied to similar bug types, capturing the modifications needed to resolve the bug.}
The significance of \bfps lies in their ability to reveal underlying system vulnerabilities and inform more effective automated repair strategies. This understanding is not only critical for enhancing software reliability and performance but also crucial for ensuring the safety of AV operations. As AVs find their way onto public roads, the stakes of software malfunctions escalate, necessitating a thorough and detailed study of bug repair.

One of the critical gaps in previous research~\cite{PanKW09,ZhongS15,SotoTWGL16,CamposM17,IslamZ20} on \bfps is the limited focus on syntactic \bfps, primarily at the code or statement level, often overlooking the semantic aspects (e.g., affected algorithms or domain-specific components) of the projects under study.
Although some existing empirical studies have investigated \bfps or strategies in a variety of domains including the deep learning stack~\cite{Huang0WCM023}, federated learning systems~\cite{DuCC0C023}, and deep learning libraries~\cite{IslamPNR20}, none of them have focused on \bfps in the ADS domain.

Our research centers on two widely-used~\cite{udacity_apollo,baidu_volvo_ford}, production-grade (i.e., used by some commercial companies) ADSes---Baidu Apollo~\cite{baidu_apollo}, which have reached mass production agreements with Ford and Volvo~\cite{baidu_volvo_ford}, and Autoware~\cite{autoware}, which was selected by the US Department of Transportation for intelligent transportation solutions~\cite{carma_github}. 
These systems represent the forefront of ADS technology and offer a large-scale dataset of bug history for understanding \bfps in real-world, high-stakes, or virtual-simulation environments. The exploration of BFPs in these systems is crucial, as it provides insights into the challenges and strategies employed in maintaining software that directly impacts the safety and reliability of AVs.
The importance of this study is further highlighted by the safety-critical nature of autonomous driving. As AVs continue to integrate into the fabric of daily transportation, the implications of software faults range from minor malfunctions to catastrophic failures, highlighting the necessity of meticulous software maintenance and bug fixing. While automated tools for bug detection and repair show promise, a comprehensive understanding of BFPs, encompassing both syntactic and semantic aspects, is crucial to advance the ADS domain.

In addressing this need, our study not only investigates the common syntactic BFPs in Apollo and Autoware but also delves into the semantic information underlying these patterns. This approach allows for a more nuanced understanding of the bugs, their causes, and the most effective strategies for their repair. 
We further introduce the concept of modularization granularity as a structured hierarchy of ADS bugs at various levels of abstraction. This hierarchy is essential for the systematic analysis and debugging of ADS and serves as a foundational structure for developing robust software systems. From both syntactic and semantic levels, our study formulates a more effective response to the complexities of ADS development and bug repair, potentially leading to resilient and fault-tolerant system designs.
Our proposed hierarchical classification of ADS \bfp study includes \numsyn syntactic and \numsem semantic \bfps, as well as root causes, symptoms, modules, sub-modules, related algorithms, \yuntianyihl{and detailed \bfas}. This taxonomy is instrumental for developers, testers, and researchers in developing automated bug detection and repair tools for ADS.
\yuntianc{Bug-fix patterns are the building blocks for repairing programs.}
Our taxonomy and statistics provide an in-depth understanding of common BFPs in ADSes, which ensures comprehensive coverage and paves the way for targeted improvements and innovations in debugging or repair techniques tailored for ADS.
\yuntianc{Specifically, our taxonomy can inform the design of new testing techniques, such as mutation testing, and the benchmark we propose can serve as a valuable resource for failure prediction research.}
The methodology of this study involves a detailed analysis of the collected data to identify common BFPs and understand their underlying causes and symptoms. 
Our research advances software engineering for ADS in the following key aspects:
\begin{itemize}[leftmargin=*,nosep]
\item We present the first empirical study of \bfps in the ADS domain through an analysis of \numbug fixed bugs from 15,099 pull requests in two open-source ADSes, offering a foundational understanding of the challenges and strategies in this domain. 

\item We \yuntianyihl{conduct the first study that} introduces and differentiates the concepts of syntactic and semantic \bfps, which serve as a practical guide for ADS software maintenance and identify opportunities for improved bug identification and repair for future work. 

\yuntianyihl{\item We propose a hierarchy of ADS \bfp study classified by modularization granularity. We also contribute a benchmark of \numbug ADS bug-fix instances, which is publicly accessible~\cite{bfp_artifacts}.} 
\end{itemize}
