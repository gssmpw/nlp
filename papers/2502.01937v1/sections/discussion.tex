\section{Discussion}\label{sec:discussion}

The prevalence of Data Flow Correction (DFC) underscores the pivotal role of accurate data-flow 
handling in the 
ADS (Findings \ref{finding:rq1_dfc}, \ref{finding:rq4_dfc},
and \ref{finding:rq5_dfc}). 
DFC's significance across various operational contexts reveals that data-flow issues are a major factor in system bugs, necessitating advanced testing and repair strategies. 
The need for improved tracking and visualization of 
data-flow
is further backed by our study's results, as ADS engineers appear to already struggle from these issues (\autoref{finding:rq4_dti}). 


In terms of software testing, our study reveals the need for a greater focus on ADS module integration testing (\autoref{finding:rq1_ppo_mii}) and for more general software testing focused on understudied scenario types, e.g., AV stopping and parking (Findings \ref{finding:rq4_sme} and \ref{finding:rq4_stop}). In fact, the many sub-modules of Planning indicate that there are a wide variety of opportunities to potentially handle many sub-functionalities of Planning and test them in tandem (e.g., obstacle processing and avoidance (Findings \ref{finding:rq2_obs} and \ref{finding:rq3_alg}) and parking (Findings \ref{finding:rq2_combined} and \ref{finding:rq4_sme})) and with other modules (e.g., Planning and Control along the interactions between software, hardware, and the physical environment (\autoref{finding:rq1_ppo_mii})).

The manner in which ADS domain-specific BFPs tend to coalesce in semantically similar ADS modules, while domain-independent BFPs cross-cut those modules (\autoref{finding:rq5_specific_independent}), along with certain ADS modules (e.g., Planning) being architected into many sub-modules (\autoref{finding:rq5_planning}), suggests that ADS engineers highly value properly-encapsulated and well-architected ADS modules. We are not aware of a previous study that correlated bug fixing and software architecture in such a manner. Future research may consider whether machine learning-based modules (e.g., Prediction and Perception) may benefit in terms of bug fixing from explicit modularization and encapsulation.

\yuntianyihl{The dataset in this study could serve as a benchmark to facilitate future automated program repair research in the ADS domain. 
The high occurrences of certain \bfas, such as \textit{Adjust Return Values} and \textit{Fix API Misuse} (\autoref{finding:rq2_di_bfa}), suggest that developing specialized automated repair tools for these categories could substantially reduce manual debugging efforts. Additionally, the prominence of logical condition modifications implies that enhancing logic-checking mechanisms could improve overall system reliability and performance.
By identifying recurring actions, this study provides actionable insights for prioritizing repair strategies that target the most frequent and critical bug types in ADS.
}

