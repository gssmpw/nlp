\subsection{\BFPs at the Syntactic Level}
\label{subsec:syntactic_taxonomy}

We adopted the classification from previous work~\cite{PanKW09,IslamZ20} as it comprehensively covered the syntactic BFPs observed in Apollo and Autoware. We also added new patterns like \yuntianyihl{Syntax (SYN),} \textit{Document Fix (DOC)}, and \textit{Library (LIB)} to describe bug repairs that are missing in previous studies but relevant for ADSes. We enumerate the syntactic \bfps in the following paragraphs.

\patternpara{If-related (IF)}
Modifications in the conditional logic of code through addition, modification, or removal of \textit{if} statements to ensure that operations are executed under the correct conditions.

\patternpara{Assignment (AS)}
Modifications in the assignment statements, including changes in the expressions on the right-hand side of an assignment, to ensure correct value assignment and operation.

\patternpara{Method Call (MC)}
The MC pattern involves adjustments in the way methods are invoked, which include changing the number and type of parameters or the values passed to it.

\patternpara{Method Declaration (MD)}
Changes in Method Declarations involve alterations in the method's interface, such as the number of parameters, parameter types, or return type. 

\patternpara{Sequence (SQ)}
Changes in the order of a series of operations or method calls. 

\patternpara{Loop (LOOP)}
Modifications in the conditions or the structure of loop statements. 

\patternpara{Return (RT)}
This pattern encompasses modifications in the return statements of functions, ensuring that functions return the correct values or types as expected.

\patternpara{Local Variable (LV)}
Changes to the declaration or initialization of local variables within functions. 

\patternpara{Non-source-code Variable (NV)} 
NV adjustments pertain to changes in parameters that are not part of the source code but are essential for the configuration of the software.

\patternpara{Library (LIB)}
Fixes involve the addition, removal, or modification of libraries used in the project. 

\patternpara{Syntax (SYN)}
This pattern addresses issues like incorrect indentation \yuntianyihl{in Python}, syntax misuse, or typographical errors \yuntianyihl{that can affect the code execution}.

\patternpara{Try-Catch (TRY)}
This pattern involves the addition or removal of try-catch blocks, which are vital for error handling and exception management without crashing.

\patternpara{Switch (SW)}
Add or remove switch branches or change the conditions within the switch block. 

\patternpara{Class Field (CF)}
CF fixes pertain to modifications in the class fields or attributes that are present in any object of a class, and whose lifetime is the same as the object lifetime. 

\patternpara{Document Fix (DOC)}
Corrections or updates in documentation or comments within the code. 
