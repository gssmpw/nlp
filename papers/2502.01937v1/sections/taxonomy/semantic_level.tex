\subsection{\BFPs at the Semantic Level}
\label{subsec:semantic_taxonomy}

The semantic level of \bfps in ADS delves into the deeper, meaning-oriented aspects of software bugs, focusing on how these bugs affect the behavior and functionality of the system as a whole. Unlike syntactic \bfps, which are concerned with the structure and form of the code, semantic \bfps address the logic, algorithms, and operational semantics of the ADS software, sometimes including multiple syntactic patterns within a single semantic pattern. To comprehensively understand these patterns, we categorize them into \textit{domain-specific} and \textit{domain-independent} \bfps. Domain-specific patterns address issues unique to autonomous driving technology, while domain-independent patterns apply to a broader range of software systems. 
We enumerate our identified ADS semantic \bfps in the remainder of this section.

\subsubsection{Domain-Specific Semantic \BFPs.}

\yuntianyihl{Domain-specific patterns are uniquely tailored to address the problems inherent to ADSes. 
These patterns often highlight the intricate interactions between sub-modules and external factors like road conditions, traffic regulations, and sensor data.}

\patternpara{Path Planning Optimization (PPO)}
Refine the algorithms related to path and velocity planning in ADS, including enhancements to the lane-change strategies, obstacle avoidance, or navigation in complex environments.
The code snippet provided illustrates a specific bug fix of PPO:

\begin{lstlisting}
+ overwriteStopPoint(clipped,traj_smoothed);
  traj_smoothed.insert(traj_smoothed.begin(),traj_resampled->begin(),traj_resampled->begin()+ *traj_resampled_closest);
- overwriteStopPoint(*traj_resampled,traj_smoothed);
\end{lstlisting}

\noindent 
In this fix, the function \texttt{overwriteStopPoint} is being called with updated parameters while being moved forward.
By adjusting the source of the stop point data and how the trajectory is assembled, the updated code aims to improve the vehicle's path planning accuracy, making the stop points more precise and better aligned with the actual route and obstacles. 

\patternpara{Sensor Data Interpretation (SDI)}
Address issues in interpreting data from sensor input (e.g., \lidar, radar, and cameras), ensuring accurate and reliable environmental perception.

\patternpara{Control System Adjustment (CSA)}
Optimize vehicle's control systems, such as steering, braking, and throttle, which are vital to ensure smooth and safe vehicle operation.

\patternpara{Predictive Algorithm Enhancement (PAE)}
Improve the algorithms responsible for predicting potential hazards or traffic conditions for proactive safety measures and efficient route planning.

\patternpara{SLAM Algorithm Refinement (SAR)}
Refine Simultaneous Localization and Mapping (SLAM) algorithms used in the Localization module to enhance map creation and position estimation. 

\patternpara{Environmental Adaptability (EA)}
EA involves enhancements in the system's ability to adapt to different environmental conditions, such as varying weather, lighting, or road surface conditions, which are critical for the reliability of ADSes.
The code snippet below illustrates an EA bug fix:

\begin{lstlisting}
  if ((rclcpp::Time(msg->stamp)-rclcpp::Time(latest_perception_msg_.stamp)).seconds() > perception_time_tolerance_) {
-   latest_external_msg_.signals.clear();}
+   latest_perception_msg_.signals.clear();}
\end{lstlisting}

\noindent 
The repair suggests the original code was incorrectly resetting wrong signals when a delay in perception message updates was detected. By clearing correct signals, the system ensures outdated perception data does not interfere with the ability to adapt to current environmental conditions.

\patternpara{Firmware-Software Harmonization (FSH)}
Align the vehicle's firmware with higher-level software systems, ensuring that hardware-software interactions are seamless and efficient.

\patternpara{Module Integration and Interaction (MII)}
Improve the integration and interaction among various ADS modules for ensuring cohesive and harmonious system operations.

\patternpara{Safety Protocol Enhancement (SPE)}
Enhance safety features to ensure compliance with emerging safety standards and regulations, as well as to address newly identified safety concerns.

\patternpara{Communication Protocol Refinement (CPR)}
Enhance communications for vehicle-to-vehicle (V2V) and vehicle-to-everything (V2X) interactions to ensure robust and secure data transmission.

\patternpara{Real-time Data Processing Improvement (RDPI)}
Improvements in algorithms for ensuring efficient handling of real-time data across modules and sub-modules for better vehicle control.

\patternpara{Autonomous Decision-Making Improvement (ADMI)}
Enhancements in ADS decision-making algorithms, ensuring that the vehicle makes safe and logical decisions in real-time traffic scenarios.

\patternpara{Simulation Performance Optimization (SPO)}
Improve the simulator's performance to enable more efficient virtual testing. This pattern could involve code optimizations to reduce computational load, enhance rendering speeds, and decrease latency in the simulation environment. 


%%%%%%%%%%%%%%%%%%%%%%%%%%%%%%
\subsubsection{Domain-Independent Semantic \BFPs.}

Domain-independent semantic \bfps play a vital role in the overall functionality and robustness of ADSes. These patterns are not unique to ADS but are critical for ensuring the software's operational efficacy and reliability.

\patternpara{Data Flow Correction (DFC)}
Address issues in the way data is passed and used throughout the system.
Rectify the data flow between functions to ensure data is correctly processed and utilized.

\patternpara{Logic Amendment (LA)}
Changes in the underlying logic of the code, which is applied when existing logic leads to incorrect or undesired behaviors to realign with the intended functionality.

\patternpara{State Management Enhancement (SME)}
Refine how states (e.g., vehicle state or sensor state) are tracked, updated, and managed within the system to ensure correct state-dependent functionalities.

\patternpara{Interface Consistency (IC)}
Interface consistency fixes are applied to standardize and streamline the interactions between different software modules or components. 

\patternpara{Resource Management Improvement (RMI)}
Optimize the utilization of system resources like memory, processing power, and network bandwidth that can impact system performance.

\patternpara{Concurrency Control (CC)}
Address issues arising from simultaneous operations or multi-threaded environments, ensuring concurrent processes do not lead to conflicts, deadlocks, or inconsistency.

\patternpara{Error Handling Refinement (EHR)}
Improve the system's ability to detect, report, and recover from errors. Robust error handling can prevent cascading failures and improve system resilience.

\patternpara{Configuration and Environment Management (CEM)}
Update configuration or environment options to ensure that the ADS operates correctly and is adaptable to different operational contexts.

\patternpara{Security Strengthening (SS)}
Enhance the system's defense against external threats and vulnerabilities, including security flaws, which is important in ADS due to its autonomous nature.

\patternpara{User Interface Adjustment (UIA)}
Changes to the human-machine interface of ADS, aimed at improving usability, providing clearer information, or enhancing the overall user experience.

\patternpara{Dependency Update (DU)}
Update or modify external libraries or components to ensure compatibility and that the system benefits from the latest improvements of its dependencies.

\patternpara{Build and Compilation Enhancement (BCE)}
This pattern aims to improve the efficiency, reliability, and accuracy of the build and compilation processes in software development.

\patternpara{Debugging Tools Improvement (DTI)}
DTI involves integrating better logging systems, visual debugging aids, and automated analysis tools to help identify and diagnose issues.

\patternpara{Documentation Update and Clarification (DUC)}
This pattern involves correcting and updating the documentation to ensure that it reflects the current state of the project accurately. 
