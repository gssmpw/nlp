\yuntianyihl{

\subsection{\textbf{RQ2: \BFA Frequency}}

This research question explores the prevalence of domain-specific and domain-independent \bfas in ADS, aiming to categorize them into frequently occurring strategies and identify patterns that could guide targeted repair methods. By quantifying and analyzing these actions, we aim to discern which \bfas are most effective and frequently utilized across ADSes.

\begin{table}[ht]
\centering
\caption{\yuntianyihl{Domain-Independent \BFAs with Occurrences}}
\vspace{-0.5ex}
\begin{adjustbox}{width=\linewidth}
\begin{tabular}{|c|c|l|}
\hline
\textbf{Bug-Fix Action} & \textbf{\#N} & \textbf{Description} \\
\hline
\textbf{\makecell{Adjust Return Values}} & 187 & Add, delete, or change the return values, which is crucial for fixing incorrect output.  \\
\hline
\textbf{\makecell{Update System Config}}  & 177 & Add, remove, or change the system configuration or settings to ensure correct behavior. \\
\hline
\textbf{Fix API Misuse} & 166 & Correction of misused API functions to improve system interoperability. \\
\hline
\textbf{Add Condition Logic} & 147 & Add conditional statements to handle special cases or edge conditions in the code logic. \\
\hline
\textbf{Change Condition Logic} & 123 & Modify existing conditional logic to correct bugs or optimize decision-making pathways. \\
\hline
\textbf{\makecell{Update Comments}} & 109 & Update code comments or documentation to clarify or correct misunderstandings. \\
\hline
\textbf{\makecell{Modify Debug Handler}} & 96 & Enhance the error handling or debugging mechanisms to improve diagnostic ability. \\
\hline
\textbf{\makecell{Update Data Values}} & 84 & Adjustments to how data values are handled or calculated within the code. \\
\hline
\textbf{Modify Syntax Error} & 66 & Corrections of syntax errors or typos that may cause compilation or runtime errors. \\
\hline
\textbf{Move Statements} & 62 & Relocation of code statements for optimization or to resolve execution order issues. \\
\hline
\textbf{Modify GUI Property} & 58 & Changes or additions to graphical user interface elements or properties. \\
\hline
\textbf{Change Library Import} & 49 & Add, delete, or change the libraries importing to resolve dependency issues. \\
\hline
\textbf{Optimize Script Tools} & 46 & Improve the efficiency or functionality of scripts or tools used within the system. \\
\hline
\textbf{Update Dependency} & 42 & Add or remove project dependencies to resolve conflicts or ensure proper library linking. \\
\hline
\textbf{Modify Parameter Path} & 39 & Changes to parameter paths within functions or systems to resolve accessibility issues. \\
\hline
\textbf{\makecell{Add/Delete/Modify \\Parameter Declaration}} & 39 & Adjust variable or parameter declarations to fix scoping or initialization problems. \\
\hline
\textbf{Change Data Processing} & 31 & Modifications to how data is processed, often to fix errors or improve efficiency. \\
\hline
\textbf{Fix Memory Issue} & 30 & Resolution of memory management issues, such as leaks or improper allocation. \\
\hline
\textbf{Modify Command} & 29 & Update commands or options to resolve functional issues or improve user interaction. \\
\hline
\textbf{Change Data Type} & 28 & Changes in data types to resolve type mismatches or improve data integrity. \\
\hline
\end{tabular}
\end{adjustbox}
\label{table:di_bfa}
\end{table}

\autoref{table:di_bfa} presents the data of frequent domain-independent \bfas that occurred more than 20 times. The most common actions involve adjustments to return values (187 occurrences) and system configurations or settings (177 occurrences), which emphasize the need for fine-tuning output and system behavior. This reflects the dynamic nature of ADS, where precise control over return values and system configurations is essential for performance and safety.
API misuse fixes (166 occurrences) are also frequent, suggesting that proper API usage is critical in ADS development. Errors in API functions can lead to cascading issues in the system, necessitating frequent corrections. Additionally, logical condition changes and additions (270 occurrences in total) indicate that much of the bug-fixing effort in ADS involves improving decision-making logic, highlighting the complexity of handling edge cases and optimizing pathways.


The frequency of certain \bfas, such as return value adjustments and API misuse corrections, suggests that developing specialized automated repair tools for these categories could substantially reduce manual debugging efforts. Additionally, the prominence of logical condition modifications implies that enhancing logic-checking mechanisms could improve overall system reliability and performance.
By identifying recurring actions, this study provides actionable insights for prioritizing repair strategies that target the most frequent and critical bug types in ADS.



\vspace{-1ex}
\begin{finding}
\label{finding:rq2_di_bfa}
\yuntianyihl{
    The most frequent \bfas in ADS are related to adjustments of return values and configurations, indicating the critical need for precise control over output and system behavior. Frequent API misuse fixes also highlight the importance of proper API usage, as errors can propagate and cause system-wide issues. Logical condition changes and additions point to the complexity of decision-making logic in ADS, emphasizing the need to handle edge cases.
}
\end{finding}
\vspace{-1ex}
}


As shown in \autoref{fig:ds_bfa}, we also examine the frequency of domain-specific \bfas in ADSes, focusing on the most frequent actions that occur more than 20 times and highlighting their importance within the ADS context. 
The domain-specific data clearly illustrates that Planning is the most critical ADS module, with a diverse range of sub-modules contributing to the system's overall functionality and 12 frequent \bfas, such as \textit{Modify Avoidance Processing} (98 occurrences) and \textit{Adjust Stop/Park Handling} (88 occurrences), point to its centrality in ADS operations. Planning sub-modules are integral to both path planning and lane-change processing, such as obstacle avoidance and velocity handling, reflecting their crucial role in vehicle decision-making. In addition to Planning, the Perception module stands out as the second most important and complex module of the ADS, with 4 frequent \bfas. The Perception module plays a critical role in integrating data from object detection sensors and software fusion algorithms, making it essential for accurately interpreting the environment. The complexity of this module lies in its dual functionality—relying on both hardware (sensors) and software (fusion algorithms)—to ensure real-time object detection, mapping, and environmental understanding within the ADS.

\begin{figure}[ht]
    \centering
    \noindent\makebox[\linewidth]{\includegraphics[width=0.8\linewidth]{images/rq2/ds-bfa.pdf}}
    \caption{\yuntianyihl{Domain-Specific \BFAs with Occurrences}}
    \label{fig:ds_bfa}
\end{figure}


Object and obstacle processing emerges as one of the most important ADS functionalities, with \bfas like 98 occurrences of \textit{Modify Avoidance Processing}, 30 of \textit{Modify Obstacle Prediction}, and 29 of \textit{Modify Object Detection} being highly frequent. This demonstrates that ensuring system's ability to detect, predict, and avoid obstacles is a top priority. These actions span multiple ADS modules, including Perception (object detection), Planning (avoidance), and Prediction (obstacle prediction), highlighting the complexity and interdependence of object handling across the ADS.

\vspace{-1ex}
\begin{finding}
\label{finding:rq2_obs}
\yuntianyihl{
    Object and obstacle processing is a major focus, 
    emphasizing the importance of detecting, predicting, and avoiding obstacles. The frequent occurrence of object-processing actions across multiple ADS modules reflects the interdependence and complexity of handling obstacles, highlighting the priority of maintaining robust object-handling mechanism.
}
\end{finding}
\vspace{-1ex}

Autonomous vehicle maneuvers play an important role in ADS path planning. Frequent \bfas like \textit{Adjust Stop/Park Handling} (88 occurrences), \textit{Adjust Velocity/Speed Handling} (76 occurrences), and \textit{Modify Turning Logic} (23 occurrences) stress the importance of refining these maneuvers to avoid accidents and ensure smooth navigation. The complexity of handling speed and velocity, in particular, indicates a need for ongoing refinement in dynamic vehicle control.

Lane-related \bfas also feature prominently, with \textit{Adjust Lane-change Rule} (60 occurrences), \textit{Modify Goal Planning} (22 occurrences), \textit{Modify Start Planning} (22 occurrences), and \textit{Adjust Drivable Area} (20 occurrences) indicating the system's reliance on precise lane generation and start and destination planning for safe and efficient travel. Lane changes, drivable area processing, and goal planning are all essential for adaptive route selection, further emphasizing the importance of these components within the broader ADS planning framework.

The \textit{Modify Simulation Visualization} (42 occurrences) demonstrates the critical need for simulator reliability in ADS development. Simulations are used to test and verify system behaviors before deployment in the real world. A stable and accurate simulation environment is vital for scenario-based testing, such as obstacle detection and collision avoidance, to ensure that the ADS performs as expected in diverse conditions. High-frequency modifications in this area signal the need to continuously refine simulation environments for more reliable and user-friendly ADS simulators.


\vspace{-1ex}
\begin{finding}
\label{finding:rq2_combined}
    Frequent maneuver-related bug-fix actions (187 occurrences) highlight the need for precise control to ensure safe navigation and prevent accidents. Lane-related actions (102 occurrences) underscore the importance of accurate lane and route planning for adaptive and efficient travel. The simulation-related action (42 occurrences) emphasizes the critical role of simulation in verifying ADS behaviors under various conditions. Frequent adjustments to the simulator highlight the ongoing need to enhance its stability and accuracy for reliable testing.
\end{finding}
\vspace{-1ex}
