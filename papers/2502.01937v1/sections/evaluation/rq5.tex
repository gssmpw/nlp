\subsection{\textbf{RQ5: \BFPs Across ADS Modules}}

This RQ investigates the degree of repair efforts associated with ADS modules. 
\autoref{fig:module_semantic_combined} presents the occurrences of semantic \bfps within distinct modules of Apollo and Autoware. 


\begin{wrapfigure}[15]{r}{0.52\textwidth}
    \centering
    \vspace{-2ex}
    \noindent\makebox[\linewidth]{\includegraphics[width=\linewidth]{images/rq3_4_5/pie_subcomponent_combined.pdf}}
    \vspace{-4ex}
    \caption{Distribution of Sub-modules in Planning Module (Labels less than 1.0\% are hidden)}
    \label{fig:pie_subcomponent_combined}
\end{wrapfigure}

In the ADS domain, the Planning module is distinguished by its coverage of 22 types of semantic BFPs, with PPO manifesting 548 times, making it one of the most frequently occurring patterns in ADS modules.
The Planning module is also the most complicated module, with a total of 50 sub-modules for Apollo and 51 for Autoware. With its diverse sub-modules like trajectory generation and velocity planning, the Planning's complexity and centrality in ADS underscore the necessity for improved design and modularization for enhanced maintenance and functionality.

The Planning's intricacy is further illuminated by the frequent occurrence of PPO and its subdivision into numerous sub-modules, as highlighted in \autoref{fig:pie_subcomponent_combined}, 
\yuntianyihl{such as the \textit{Behavior Path Planner} (51.0\%), whose dominance signifies its critical role in determining the optimal path. Its frequent interaction with other subsystems makes it a focal point for potential errors. The \textit{Behavior Velocity Planner} at 18.5\% further emphasizes the complexity of real-time decision-making in velocity adjustments, affecting safety and performance. \textit{The Motion Velocity Smoother} at 5.5\% plays a crucial role in ensuring the vehicle's movements remain smooth and predictable, reducing jerky motions that could compromise passenger comfort and safety.}
This intricate structure \yuntianyihl{of interdependent sub-modules} and the transition of Apollo to fragment modules into smaller packages in \textit{v9.0}~\cite{apollo_architecture, apollo_v9} reflects the growing need for refined architectural and modularization approaches in ADS, particularly to enhance the Planning's efficiency and maintainability \yuntianyihl{in handling real-world challenges like obstacle avoidance and smooth velocity transitions. The Planning's complexity, as evidenced by the high frequency of semantic bugs in its core sub-modules, indicates that further research into modularization could lead to significant performance improvements and error reduction.}

\vspace{-1ex}
\begin{finding}
\label{finding:rq5_planning}
    The prominent occurrence of 548 instances of semantic \bfps and about 50 sub-modules in the Planning module underscores its centrality and complexity in both ADSes. 
    \yuntianyihl{Key sub-modules about path, velocity, and avoidance planning contribute significantly to this complexity due to their high functional load and frequent interaction with other sub-modules. The varied roles of these sub-modules highlight the need for a more refined software architecture and improved granular modularization to facilitate easier maintenance and system upgrades.}
\end{finding}
\vspace{-1ex}


Domain-specific semantic BFPs are predominantly associated with particular modules, whereas domain-independent BFPs are more evenly spread across various modules. For instance, SLAM Algorithm Refinement (SAR) is specific to the Localization and Environmental Adaptability (EA) to the Perception. 
This distribution implies domain-specific issues are closely tied to the functionalities of their respective modules, reflecting adherence to the principle of single responsibility~\cite{martin2000design}, where each module is closely tied to distinct functionalities, necessitating precise, module-specific solutions. 
Conversely, domain-independent patterns, embodying cross-cutting concerns~\cite{crosscutting_concerns,EaddyZSGMNA08}, pervade various modules, suggesting a need for overarching strategies that address these broader issues. 


\vspace{-1ex}
\begin{finding}
\label{finding:rq5_specific_independent}
    While domain-specific bug-fix patterns tend to coalesce to their most semantically similar ADS module, domain-independent ADS bug-fix patterns cross-cut modules. 
\end{finding}
\vspace{-1ex}

Configuration and Environment Management (CEM) is present across all modules, with a notable prevalence in the Perception module (35 occurrences). This indicates that configuration and environmental factors are universally important across modules, with the Perception module being particularly sensitive to these aspects. The pervasive nature of CEM patterns suggests that consistent and accurate configuration and environment management are critical for the seamless operation of autonomous driving systems, especially in processing and interpreting sensory data.


Data Flow Correction (DFC), with a total of 330 occurrences in all modules, stands out as the most common domain-independent \bfp, indicating that issues related to data handling and processing are pervasive across various modules. 
This prevalence suggests that ensuring the accuracy and efficiency of data flow is a critical concern in ADSes, reflecting the need for robust mechanisms to manage and correct data flow across the system's diverse components. 

\vspace{-1ex}
\begin{finding}
\label{finding:rq5_dfc}
    The prominence of Data Flow Correction (DFC) as a domain-independent BFP highlights the critical need for accurate and efficient data flow management across the diverse components of ADSes. This prevalence indicates a pressing demand for advanced APR heuristics and debugging tools specifically designed to address the complex data-flow challenges in ADS.
\end{finding}
\vspace{-1ex}

