\subsection{\textbf{RQ4: \BFPs Across Symptoms}}


As shown in \autoref{fig:symptom_semantic_combined}, this research question illustrates the relationship between bug symptoms observed in Apollo and Autoware and semantic BFPs, with 2473 symptom instances in total.


The occurrence of domain-specific \bfp Path Planning Optimization (PPO) and domain-independent pattern Data Flow Correction (DFC) across various symptoms, such as Crashes, Stop and parking (Stop), Lane Positioning and Navigating (LPN), Speed and Velocity Control (SVC), Turning (Turn), Trajectory (Traj), Obstacle Processing (OP), and Logic, signifies the critical nature of these patterns in addressing diverse and complex issues in different driving scenarios. 

\vspace{-1ex}
\begin{finding}
\label{finding:rq4_dfc}
    The frequent occurrence of Data Flow Correction (DFC) (364 occurrences) in diverse symptoms underscores the importance of data-flow handling for reliable autonomous vehicle operations, which suggests the need for robust data-flow testing and correction for ADSes.
\end{finding}
\vspace{-1ex}



The 30 instances of Debugging Tools Improvement (DTI) bug fixes, primarily in Display and GUI (DGUI) symptoms, demonstrate that ADS developers value debugging methods that are specialized to their domain, but further value effective visualizations for debugging concerns. 
Existing debugging techniques~\cite{ArcainiCILZAHV21,MINNERUP201644,apollo_debug,autoware_debug}, such as log analysis, breakpoint debugging, and unit testing, provide a broad foundation for software diagnosis. However, the absence of dedicated research on convenient debugging tools for ADS reveals a critical gap, where the intricate and real-time nature of ADS presents unique challenges such as complex sensor data integration, safety-critical decision-making, and dynamic environmental interaction. Improved visualization tools for debugging may consider offering a more intuitive understanding of ADS behavior and error states.

\vspace{-1ex}
\begin{finding}
\label{finding:rq4_dti}
    The prevalence of 30 Debugging Tools Improvement (DTI) bug fixes in DGUI symptoms highlights the critical need for ongoing improvements in debugging tools and user interfaces within autonomous vehicle systems. 
\end{finding}
\vspace{-1ex}

\begin{figure}[ht!]
    \centering
    \begin{minipage}{0.525\linewidth}
        \centering
        \includegraphics[width=\linewidth]{images/rq3_4_5/symptom_semantic_combined.pdf}
        \vspace{-5ex}
        \caption{Relation between Symptoms and Semantic \BFPs for Apollo and Autoware}
        \label{fig:symptom_semantic_combined}
    \end{minipage}
    \hfill
    \begin{minipage}{0.465\linewidth}
        \centering
        \includegraphics[width=\linewidth]{images/rq3_4_5/module_semantic_combined.pdf}
        \vspace{-5ex}
        \caption{Relation between Modules and Semantic \BFPs for Apollo and Autoware}
        \label{fig:module_semantic_combined}
    \end{minipage}
\end{figure}

State Management Enhancement (SME) \bfps are most commonly observed in Stop and Parking (Stop) (23 occurrences), Obstacle Processing (OP) (23 occurrences), and Lane Positioning and Navigating (LPN) (20 occurrences) symptoms. 
Effective state management is crucial for ensuring seamless transitions between different operational modes, such as transitioning from driving to parking or navigating through obstacles. This necessity points to the intricate interplay between state control mechanisms and the vehicle's capacity to interpret and react to dynamic driving environments, thereby stressing the importance of sophisticated state handling strategies to mitigate potential risks and enhance driving safety.
These scenarios, which necessitate precise state transitions such as moving from forward to reverse for parking or adjusting to dynamic obstacles, highlight the critical role of advanced state management.

\vspace{-1ex}
\begin{finding}
    \label{finding:rq4_sme}
    The concentration of State Management Enhancement (SME) patterns in scenarios like Stop and Parking, Obstacle Processing, and Lane Positioning and Navigating underscores the importance of robust state management for AV adaptability to diverse driving conditions. 
\end{finding}
\vspace{-1ex}


The significant count of 178 STOP-related \bfps in AVs underscores the complex and persistent issues in stopping and parking functionalities, revealing systemic vulnerabilities and the need for refined parking algorithms and community concerns. These patterns suggest that current methodologies may fall short in addressing the intricate interactions among subsystems essential for efficient stop and parking operations. Few existing papers focus on enhancing the parking and stopping of the ADS~\cite{IbischSANTSSK13,BanzhafNKZ17}, this gap in research and development highlights the urgency of advancing testing frameworks and enhancing safety standards to address the operational complexities and potential hazards inherent in AV stop and parking mechanisms.

\vspace{-1ex}
\begin{finding}
\label{finding:rq4_stop}
    The presence of 178 STOP-related \bfps and few AV studies focusing on such functionalities reveals a major gap of \yuntianyihl{stop and parking optimization} in ADS research.
\end{finding}
\vspace{-1ex}
