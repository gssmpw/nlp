\section{Threats to Validity}\label{sec:threats}

\noindent \textbf{Internal Threats.} 
The main internal threat to the validity of this study is the potential subjective bias or errors in bug classification. To mitigate this, we began our labeling process using established classification schemes from the literature~\cite{ThungWLJ12, SeamanSREFGG08} and employed an open-coding scheme~\cite{blair2015reflexive} to expand these initial frameworks. We also concentrated on real bug fixes by selecting only closed and merged pull requests. Each bug was independently examined and labeled by two ADS developers, both of whom have contributed to Apollo and Autoware, with any discrepancies resolved through discussion until consensus was achieved. In addition, discussions with Apollo and Autoware developers were held to refine our classification, further reducing subjective biases and errors.

Additionally, our approach to identifying bugs 
relies heavily on keywords 
in the context of pull requests. This method might result in false positives since developers sometimes mistakenly mark feature additions as bug fixes.
We mitigated this by filtering out false positives during labeling.

\noindent \textbf{External Threats.} 
The primary external threat concerns the reliability of our dataset. To address this, our data collection includes all pull requests, commits, and issues from the inception of the subject ADSes until the major version release at the end of 2023. 
This comprehensive dataset aligns with similar methods~\cite{GarciaF0AXC20,IslamNPR19,VasilescuYWDF15,ZhangCCXZ18,FrancoGR17} used in other bug studies to identify bug-fix pull requests. 
Another threat is the open-source nature of the ADSes under study. Our study involves two widely-used representatives---Apollo and Autoware, developed independently and featuring extensive code bases, which may not reflect the bug-fix patterns prevalent in closed-source environments due to differing development processes. 
\yuntianyihl{As for the generalizability, despite the study encompassing only two systems, they have significant industry and research usage, including the US government, Google, Ford, and various car manufacturers~\cite{40_plus_corporations,baidu_volvo_ford,baidu_apolong,carma_github,18_av_companies,carma_overview,waymo_8_miles,lyft_ces}, and have been developed, iterated, and studied for years, accumulating more research studies, issues, and pull requests than any other small-scale open-source ADS projects. 
\yuntianc{The two projects we selected represent the forefront of open-source ADS development.}
Furthermore, most ADSes, including Apollo and Autoware, share the similar reference architecture and are developed from the ROS architecture~\cite{ros_icra}.}


Besides, the number of labeled bugs in our study (\numbug) is comparable to other \bfp studies in different domains (e.g., 970 in deep neural networks~\cite{IslamPNR20}, 446 in deep learning stack~\cite{Huang0WCM023}, and 395 in federated learning systems~\cite{DuCC0C023}), underscoring the representativeness of our findings.
