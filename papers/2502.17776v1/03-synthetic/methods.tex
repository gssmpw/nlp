\subsection{Synthetic Query Generation}
\begin{table*}[]
\centering
\begin{tabular}{ccccccc c ccc}
   &&&&&&&\multicolumn{3}{c}{Validation Results}\\
   \cline{8-10}
   & \multicolumn{4}{c}{Prompt \& Model Design}   &&& \multicolumn{1}{c}{MRR-based} & \multicolumn{1}{c}{NDCG-based} & \multicolumn{1}{c}{\makecell{Linguistic\\Similarity}} \\
   \cline{2-6} \cline{8-10}
   ID & \makecell{Prompt\\Template} & \makecell{System\\Role}   & \makecell{Wikipedia\\Summary} & \makecell{Instruction\\Type} & Temperature && $\tau$/$r$ $\uparrow$ & $\tau$/$r$ $\uparrow$ & EMD $\downarrow$ \\
   \hline
1 & V0 & Writing Assistant  & w/o  & 9 rules                 & 0.5  && 0.2641 / 0.6101 & 0.3092 / 0.4247 & 0.0899  \\
2 & V1 & Writing Assistant  & w/   & 9 rules                 & 0.5  && 0.4923 / 0.5692 & 0.5077 / 0.6475 & 0.1012  \\
3 & V2 & Searcher role play  & w/   & 6-shot                  & 0.5  && 0.4632 / 0.5391 & 0.4430 / 0.5492 & 0.0299  \\
4 & V2 & Searcher role play  & w/   & 6-shot                  & 0.7  && 0.4538 / 0.4234 & 0.4310 / 0.4650 & 0.0276  \\
5 & V3 & Searcher role play  & w/o  & 6-shot                  & 0.5  && 0.1524 / 0.5293 & 0.2045 / 0.5995 & 0.0584  \\
6 & V3 & Searcher role play  & w/o  & 6-shot                  & 0.7  && 0.1440*/ 0.5173 & 0.1960*/ 0.5942 & 0.0534  \\
7 & V4 & Searcher role play & w/   & 13 rules                & 0.3  && 0.6927 / 0.9143 & 0.6757 / 0.8975 & 0.0490  \\
8 & V4 & Searcher role play & w/   & 13 rules                & 0.5  && 0.6463 / 0.9173 & 0.6235 / 0.8998 & 0.0542  \\
9 & V4 & Searcher role play & w/   & 13 rules                & 0.7  && 0.6472 / 0.9336 & 0.6785 / 0.9303 & 0.0577  \\
10 & V5 & Searcher role play & w/   & 14 rules               & 0.1 && 0.6719 / 0.9388 & 0.6500 / 0.9375 & 0.0542   \\
11 & V5 & Searcher role play & w/   & 14 rules               & 0.3 && 0.6558 / 0.9396 & 0.6728 / 0.9342 & 0.0620   \\
12 & V5 & Searcher role play & w/   & 14 rules               & 0.5 && 0.7013 / \textbf{0.9536} & 0.7127 / \textbf{0.9500} & 0.0590   \\
13 & V6 & Searcher role play & w/   & 7 Musts + 7 Coulds     & 0.3 && \textbf{0.7573} / 0.9166 & \textbf{0.7194} / 0.8973 & \textbf{0.0264} 
\end{tabular}
\caption{
Experiments on LLM-elicited query generation in the Movie domain.
$\tau$ and $r$ represent Kendall's Tau and Pearson's r correlation values, respectively, while EMD (Earth Mover’s Distance) measures linguistic similarity between CQA-based and LLM-elicited queries. All correlation values have a p-value < 0.01, except for those marked with (*).
}
\label{tab:prompt-versions}
\end{table*}

\begin{figure} 
\centering
\includegraphics[trim=110 243 78 210, clip, width=\columnwidth]{03-synthetic/graphics/generation-sequence.pdf}
\caption{
LLM-elicited TOT query generation process. Given an arbitrary entity for which we want to generate a TOT query, we retrieve its Wikipedia page, summarize it into a few paragraphs, and construct a prompt template using the summary. This prompt is then used to query an LLM to generate a synthetic TOT query.
}
\label{fig:query-gen-process}
\end{figure}

Figure \ref{fig:query-gen-process} illustrates the LLM-elicited TOT query generation process. Given an arbitrary entity for which we want to generate a TOT query, we retrieve the entity’s Wikipedia page, summarize it into a few paragraphs, and construct a prompt template using the summary. This prompt is then used to query an LLM to generate a synthetic TOT query.

The query generation process consists of the following steps:
\begin{enumerate}
\item \textit{Sampling a target entity}:\\
A random entity is selected as the target for TOT query generation. In the Movie domain, the MS-TOT test set serves as the pool of entities, while for the Landmark and Person domains, we manually curated the entity pool.

\item \textit{Retrieving and summarizing the entity’s Wikipedia page}:\\
The entity’s Wikipedia page is retrieved and used to construct a summarization prompt for the GPT-4o model\footnote{GPT4o-2024-05-13}. To ensure the content fits within the model’s token limit, the page is truncated if necessary. The model processes the page into a two-paragraph summary. For Movie domain entities, we check for a dedicated \textit{Plot} section and ensure its content is included in the summarization prompt, as it often provides essential context for generating realistic TOT queries.

\item \textit{Constructing a domain- and entity-specific prompt}:\\
Following summarization, a domain- and entity-specific prompt is constructed, incorporating the summarized information to generate a TOT query. The same GPT-4o model is used for query elicitation. 
We tested various prompt configurations and temperature settings to identify the best-performing strategy in the Movie domain, then applied it to the Landmark and Person domains.\footnote{Details on the different prompt versions we tested are available at \url{https://github.com/kimdanny/llm-tot-query-elicitation}.}


\item \textit{Ensuring entity name anonymity in the generated query}:\\
An essential consideration in the query generation process is ensuring that the generated TOT query does not contain the target entity’s name. To address this, we implement a name-checking step: if the generated query includes the entity name, we retry the generation using the same prompt, with a maximum of three retries before discarding the query.
\end{enumerate}



