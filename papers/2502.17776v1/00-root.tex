\documentclass[sigconf,natbib=true]{acmart}
\usepackage{fdiaz}

%%%%%%%%%%%%%%%%%%%%%%%%%%%%%%%%%%%%%%%%%%%%%%%%%%%%%%%
% BEGIN ACM GARBAGE
%%%%%%%%%%%%%%%%%%%%%%%%%%%%%%%%%%%%%%%%%%%%%%%%%%%%%%%
\usepackage{booktabs} % For formal tables
% Copyright
\setcopyright{rightsretained}
\usepackage{makecell}
\usepackage{tcolorbox}
\usepackage{subcaption}
\usepackage{tablefootnote}
\usepackage{arydshln}
\usepackage{colortbl}
\usepackage{xcolor}

\definecolor{lightgray}{gray}{0.9}
\definecolor{Gray}{gray}{0.9}
\definecolor{highlight}{rgb}{0.9, 0.9, 0.9} % Light gray



%% \BibTeX command to typeset BibTeX logo in the docs
\AtBeginDocument{%
  \providecommand\BibTeX{{%
    Bib\TeX}}}

\setcopyright{acmlicensed}
\copyrightyear{2025}
\acmYear{2025}
\acmDOI{XXXXXXX.XXXXXXX}
%% These commands are for a PROCEEDINGS abstract or paper.
\acmConference[Conference acronym 'XX]{Make sure to enter the correct
  conference title from your rights confirmation email}{June 03--05,
  2025}{Woodstock, NY}
\acmISBN{978-1-4503-XXXX-X/2025/06}

%% The code below is generated by the tool at http://dl.acm.org/ccs.cfm.
%% Please copy and paste the code instead of the example below.
%%
\begin{CCSXML}
<ccs2012>
<concept>
<concept_id>10002951.10003317.10003359.10003360</concept_id>
<concept_desc>Information systems~Test collections</concept_desc>
<concept_significance>500</concept_significance>
</concept>
<concept>
<concept_id>10002951.10003317.10003371</concept_id>
<concept_desc>Information systems~Specialized information retrieval</concept_desc>
<concept_significance>500</concept_significance>
</concept>
<concept>
<concept_id>10003120.10003123</concept_id>
<concept_desc>Human-centered computing~Interaction design</concept_desc>
<concept_significance>500</concept_significance>
</concept>
</ccs2012>
\end{CCSXML}
\ccsdesc[500]{Information systems~Test collections}
\ccsdesc[500]{Information systems~Specialized information retrieval}
\ccsdesc[500]{Human-centered computing~Interaction design}

\keywords{Tip-of-the-Tongue Known-Item Retrieval, Synthetic Query Generation, Human Query Elicitation}

%%%%%%%%%%%%%%%%%%%%%%%%%%%%%%%%%%%%%%%%%%%%%%%%%%%%%%%
% END ACM GARBAGE
%%%%%%%%%%%%%%%%%%%%%%%%%%%%%%%%%%%%%%%%%%%%%%%%%%%%%%%


\begin{document}

\title{Tip of the Tongue Query Elicitation for Simulated Evaluation}


\author{Yifan He$^{*\dagger}$}
\affiliation{%
  \institution{Carnegie Mellon University}
  \city{Pittsburgh}
  \state{PA}
  \country{USA}
}
\email{yifanhe@cs.cmu.edu}

\author{To Eun Kim$^*$}
\affiliation{%
  \institution{Carnegie Mellon University}
  \city{Pittsburgh}
  \state{PA}
  \country{USA}
}
\email{toeunk@cs.cmu.edu}

\thanks{$^*$ Equal Contribution.}
\thanks{$^\dagger$ Now at Meta.}

\author{Fernando Diaz}
\affiliation{%
  \institution{Carnegie Mellon University}
  \city{Pittsburgh}
  \state{PA}
  \country{USA}
}
\email{diazf@acm.org}

\author{Jaime Arguello}
\affiliation{%
  \institution{UNC Chapel Hill}
  \city{Chapel Hill}
  \state{NC}
  \country{USA}
}
\email{jarguell@email.unc.edu}

\author{Bhaskar Mitra}
\affiliation{%
  \institution{Microsoft Research}
  \city{Montréal}
  \state{QC}
  \country{Canada}
}
\email{bmitra@microsoft.com}

\newcommand{\movie}{Movie\xspace}
\newcommand{\landmark}{Landmark\xspace}
\newcommand{\person}{Person\xspace}


\newcommand{\gptfouro}{GPT-4o\xspace}
\newcommand{\gptfouromini}{GPT-4o-mini\xspace}

During the early stages of interface design, designers need to produce multiple sketches to explore a design space.  Design tools often fail to support this critical stage, because they insist on specifying more details than necessary. Although recent advances in generative AI have raised hopes of solving this issue, in practice they fail because expressing loose ideas in a prompt is impractical. In this paper, we propose a diffusion-based approach to the low-effort generation of interface sketches. It breaks new ground by allowing flexible control of the generation process via three types of inputs: A) prompts, B) wireframes, and C) visual flows. The designer can provide any combination of these as input at any level of detail, and will get a diverse gallery of low-fidelity solutions in response. The unique benefit is that large design spaces can be explored rapidly with very little effort in input-specification. We present qualitative results for various combinations of input specifications. Additionally, we demonstrate that our model aligns more accurately with these specifications than other models. 

% OLD ABSTRACT
%When sketching Graphical User Interfaces (GUIs), designers need to explore several aspects of visual design simultaneously, such as how to guide the user’s attention to the right aspects of the design while making the intended functionality visible. Although current Large Language Models (LLMs) can generate GUIs, they do not offer the finer level of control necessary for this kind of exploration. To address this, we propose a diffusion-based model with multi-modal conditional generation. In practice, our model optionally takes semantic segmentation, prompt guidance, and flow direction to generate multiple GUIs that are aligned with the input design specifications. It produces multiple examples. We demonstrate that our approach outperforms baseline methods in producing desirable GUIs and meets the desired visual flow.

% Designing visually engaging Graphical User Interfaces (GUIs) is a challenge in HCI research. Effective GUI design must balance visual properties, like color and positioning, with user behaviors to ensure GUIs easy to comprehend and guide attention to critical elements. Modern GUIs, with their complex combinations of text, images, and interactive components, make it difficult to maintain a coherent visual flow during design.
% Although current Large Language Models (LLMs) can generate GUIs, they often lack the fine control necessary for ensuring a coherent visual flow. To address this, we propose a diffusion-based model that effectively handles multi-modal conditional generation. Our model takes semantic segmentation, optional prompt guidance, and ordered viewing elements to generate high-fidelity GUIs that are aligned with the input design specifications.
% We demonstrate that our approach outperforms baseline methods in producing desirable GUIs and meets the desired visual flow. Moreover, a user study involving XX designers indicates that our model enhances the efficiency of the GUI design ideation process and provides designers with greater control compared to existing methods.    



% %%%%%%%%%%%%%%%%%%%%%%%%%%%%%%%%%%%%%%%%%%%%%%%%%%%%%%
% % Writing Clinic Comments:
% %%%%%%%%%%%%%%%%%%%%%%%%%%%%%%%%%%%%%%%%%%%%%%%%%%%%%%
% % Define: Effective UI design
% % Motivate GANs and write in full form.
% % LLMs vs ControlNet vs GANs
% % Say something about the Figma plugin?
% % Write the work is novel or what has been done before
% % What is desirable UI and how to evalutate that?
% % Visual Flow - main theme (center around it)
% % Re-Title: use word Flow!
% % Use ControlNet++ & SPADE for abstract.
% % Write about input/output. 
% % Why better than previous work?
% %%%%%%%%%%%%%%%%%%%%%%%%%%%%%%%%%%%%%%%%%%%%%%%%%%%%%

% % v2:
% % \noindent \textcolor{red}{\textbf{NEW Abstract!} (Post Writing Clinic 1 - 25-Jun)}

% % \noindent \textcolor{red}{----------------------------------------------------------------------}

% % \noindent Designing user interfaces (UIs) is a time-consuming process, particularly for novice designers. 
% % Creating UI designs that are effective in market funneling or any other designer defined goal requires a good understanding of the visual flow to guide users' attention to UI elements in the desired order. 
% % While current Large Language Models (LLMs) can generate UIs from just prompts, they often lack finer pixel-precise control and fail to consider visual flow. 
% % In this work, we present a UI synthesis method that incorporates visual flow alongside prompts and semantic layouts. 
% % Our efficient approach uses a carefully designed Generative Adversarial Network (GAN) optimized for scenarios with limited data, making it more suitable than diffusion-based and large vision-language models.
% % We demonstrate that our method produces more "desirable" UIs according to the well-known contrast, repetition, alignment, and proximity principles of design. 
% % We further validate our method through comprehensive automatic non-reference, human-preference aligned network scoring and subjective human evaluations.
% % Finally, an evaluation with xx non-expert designers using our contributed Figma plugin shows that <method-name> improves the time-efficiency as well as the overall quality of the UI design development cycle.

% % \noindent \textcolor{red}{----------------------------------------------------------------------}


% \noindent \textcolor{blue}{\textbf{NEW Abstract!} (Pre Writing Clinic 9-July)}

% \noindent \textcolor{blue}{----------------------------------------------------------------------}

% \noindent Exploring different graphical user interface (GUI) design ideas is time-consuming, particularly for novice designers. 
% Given the segmentation masks, design requirement as prompt, and/or preferred visual flow, we aim to facilitate creative exploration for GUI design and generate different UI designs for inspiration.
% While current Vision Language Models (VLMs) can generate GUIs from just prompts, they often lack control over visual concepts and flow that are difficult to convey through language during the generation process. 
% In this work, we present FlowGenUI, a semantic map-guided GUI synthesis method that optionally incorporates visual flow information based on the user's choice alongside language prompts. 
% We demonstrate that our model not only creates more realistic GUIs but also creates "predictable" (how users pay attention to and order of looking at GUI elements) GUIs.
% Our approach uses Stable Diffusion (SD), a large paired image-text pretrained diffusion model with a rich latent space that we steer toward realistic GUIs using a trainable copy of SD's encoder for every condition (segmentation masks, prompts, and visual flow). 
% We further provide a semantic typography feature to create custom text-fonts and styles while also alleviating SD's inherent limitations in drawing coherent, meaningful and correct aspect-ratio text. 
% Finally, a subjective evaluation study of XX non-expert and expert designers demonstrates the efficiency and fidelity of our method.


% This process encourages creativity and prevents designers from falling into habitual patterns.


% ------------------------------------------------------------------
% Joongi Why is it important to create realistic GUI?
% I do not see how the Visual Flow given on the left hand side is reflected in the results on the right hand side. 
% I’d avoid making unsubstantiated claims about designers (falling into habitual patterns).
% The UIs you generate do not “align with users’ attention patterns” but rather try to control it (that’s what visual flow means)
% ------------------------------------------------------------------
% Comments - Writing Clinic - 9th July:
% Improve title. More names: FlowGen
% Figure 1: Use an inference time hand-drawn mask
% Figure 1: Show both workflows. Add a designer --> Input.
% Figure 1: Make them more diverse
% ------------------------------------------------------------------
% Designing graphical user interfaces (GUIs) requires human creativity and time. Designers often fall into habitual patterns, which can limit the exploration of new ideas. 
% To address this, we introduce FlowGenUI, a method that facilitates creative exploration and generates diverse GUI designs for inspiration. By using segmentation masks, design requirements as prompts, and/or selected visual flows, our approach enhances control over the visual concepts and flows during the generation process, which current Vision Language Models (VLMs) often lack.
% FlowGenUI uses Stable Diffusion (SD), a largely pretrained text-to-image diffusion model, and guides it to create realistic GUIs. 
% We achieve this by using a trainable copy of SD's encoder for each condition (segmentation masks, prompts, and visual flow). 
% This method enables the creation of more realistic and predictable GUIs that align with users' attention patterns and their preferred order of viewing elements.
% We also offer a semantic typography feature that creates custom text fonts and styles while addressing SD's limitations in generating coherent, meaningful, and correctly aspect-ratio text.
% Our approach's efficiency and fidelity are evaluated through a subjective user study involving XX designers. 
% The results demonstrate the effectiveness of FlowGenUI in generating high-quality GUI designs that meet user requirements and visual expectations.

% ---------------------------------------


%A critical and general issue remains while using such deep generative priors: creating coherent, meaningful and correct aspect-ratio text. 
%We tackle this issue within our framework and additionally provide a semantic typography feature to create custom text-fonts and styles. 


% %Creating UI designs that are effective in market funneling or any other designer-defined goal requires a good understanding of the visual flow to guide users' attention to UI elements in the desired order. 
% %While current largely pre-trained Vision Language Models (VLMs) can generate GUIs from just prompts, they often lack finer or pixel-precise control which can be crucial for many easy-to-understand visual concepts but difficult to convey through language. 
% % However, obtaining such pixe-level labels is an extremely expensive so we
% % For example - overlaying text on images with certain aspect ratios and two equally separated buttons 
% Additionally, all prior GUI generation work fails to consider visual flow information during the generation process. 
% We demonstrate that visual flow-informed generation not only creates more realistic and human-friendly GUIs but also creates "predictable" (how users pay attention to and order of looking at GUI elements) UIs that could be beneficial for designers for tasks like creating effective market funnels.
% In this work, we present a semantic map-guided GUI synthesis method that optionally incorporates visual flow information based on the user's choice alongside language prompts. 
% Our approach uses Stable Diffusion, a large (billions) paired image-text pretrained diffusion model with a rich latent space that we steer toward realistic GUIs using an ensemble of ControlNets. 
% % TODO: Mention it in 1 sentence:
% A critical and general issue remains while using such deep generative priors: creating coherent, meaningful and correct aspect-ratio text. 
% We tackle this issue within our framework and additionally provide a semantic typography feature to create custom text-fonts and styles. 
% To evaluate our method, we demonstrate that our method produces more "desirable" UIs according to the well-known contrast, repetition, alignment, and proximity principles of design. 
% % We further validate our method through comprehensive automatic non-reference and human-preference aligned scores. (TODO: Maybe Unskip if we get UIClip from Jason!)
% % TODO: Re-word this and only keep ideation cycles and time-efficiency.
% Finally, a subjective evaluation study of XX non-expert and expert designers demonstrates the efficiency and fidelity of our method.
% % improves the time-efficiency by quick iterations of the UI design ideation process.
% %Finally, an evaluation with xx non-expert designers using our contributed <method-name> improves the time-efficiency by quick iterations of the UI design ideation cycle.

%\noindent \textcolor{blue}{----------------------------------------------------------------------}


%In an evaluation with xx designers, we found that GenerativeLayout: 1) enhances designers' exploration by expanding the coverage of the design space, 2) reduces the time required for exploration, and 3) maintains a perceived level of control similar to that of manual exploration.



% Present-day graphical user interfaces (GUIs) exhibit diverse arrangements of text, graphics, and interactive elements such as buttons and menus, but representations of GUIs have not kept up. They do not encapsulate both semantic and visuo-spatial relationships among elements. %\color{red} 
% To seize machine learning's potential for GUIs more efficiently, \papername~ exploits graph neural networks to capture individual elements' properties and their semantic—visuo-spatial constraints in a layout. The learned representation demonstrated its effectiveness in multiple tasks, especially generating designs in a challenging GUI autocompletion task, which involved predicting the positions of remaining unplaced elements in a partially completed GUI. The new model's suggestions showed alignment and visual appeal superior to the baseline method and received higher subjective ratings for preference. 
% Furthermore, we demonstrate the practical benefits and efficiency advantages designers perceive when utilizing our model as an autocompletion plug-in.


% Overall pipeline: Maybe drop semantic typography / visual flow?
\maketitle

% humans are sensitive to the way information is presented.

% introduce framing as the way we address framing. say something about political views and how information is represented.

% in this paper we explore if models show similar sensitivity.

% why is it important/interesting.



% thought - it would be interesting to test it on real world data, but it would be hard to test humans because they come already biased about real world stuff, so we tested artificial.


% LLMs have recently been shown to mimic cognitive biases, typically associated with human behavior~\citep{ malberg2024comprehensive, itzhak-etal-2024-instructed}. This resemblance has significant implications for how we perceive these models and what we can expect from them in real-world interactions and decisionmaking~\citep{eigner2024determinants, echterhoff-etal-2024-cognitive}.

The \textit{framing effect} is a well-known cognitive phenomenon, where different presentations of the same underlying facts affect human perception towards them~\citep{tversky1981framing}.
For example, presenting an economic policy as only creating 50,000 new jobs, versus also reporting that it would cost 2B USD, can dramatically shift public opinion~\cite{sniderman2004structure}. 
%%%%%%%% 图1:  %%%%%%%%%%%%%%%%
\begin{figure}[t]
    \centering
    \includegraphics[width=\columnwidth]{Figs/01.pdf}
    \caption{Performance comparison (Top-1 Acc (\%)) under various open-vocabulary evaluation settings where the video learners except for CLIP are tuned on Kinetics-400~\cite{k400} with frozen text encoders. The satisfying in-context generalizability on UCF101~\cite{UCF101} (a) can be severely affected by static bias when evaluating on out-of-context SCUBA-UCF101~\cite{li2023mitigating} (b) by replacing the video background with other images.}
    \label{fig:teaser}
\end{figure}


Previous research has shown that LLMs exhibit various cognitive biases, including the framing effect~\cite{lore2024strategic,shaikh2024cbeval,malberg2024comprehensive,echterhoff-etal-2024-cognitive}. However, these either rely on synthetic datasets or evaluate LLMs on different data from what humans were tested on. In addition, comparisons between models and humans typically treat human performance as a baseline rather than comparing patterns in human behavior. 
% \gabis{looks good! what do we mean by ``most studies'' or ``rarely'' can we remove those? or we want to say that we don't know of previous work doing both at the same time?}\gili{yeah the main point is that some work has done each separated, but not all of it together. how about now?}

In this work, we evaluate LLMs on real-world data. Rather than measuring model performance in terms of accuracy, we analyze how closely their responses align with human annotations. Furthermore, while previous studies have examined the effect of framing on decision making, we extend this analysis to sentiment analysis, as sentiment perception plays a key explanatory role in decision-making \cite{lerner2015emotion}. 
%Based on this, we argue that examining sentiment shifts in response to reframing can provide deeper insights into the framing effect. \gabis{I don't understand this last claim. Maybe remove and just say we extend to sentiment analysis?}

% Understanding how LLMs respond to framing is crucial, as they are increasingly integrated into real-world applications~\citep{gan2024application, hurlin2024fairness}.
% In some applications, e.g., in virtual companions, framing can be harnessed to produce human-like behavior leading to better engagement.
% In contrast, in other applications, such as financial or legal advice, mitigating the effect of framing can lead to less biased decisions.
% In both cases, a better understanding of the framing effect on LLMs can help develop strategies to mitigate its negative impacts,
% while utilizing its positive aspects. \gabis{$\leftarrow$ reading this again, maybe this isn't the right place for this paragraph. Consider putting in the conclusion? I think that after we said that people have worked on it, we don't necessarily need this here and will shorten the long intro}


% If framing can influence their outputs, this could have significant societal effects,
% from spreading biases in automated decision-making~\citep{ghasemaghaei2024understanding} to reducing public trust in AI-generated content~\citep{afroogh2024trust}. 
% However, framing is not inherently negative -- understanding how it affects LLM outputs can offer valuable insights into both human and machine cognition.
% By systematically investigating the framing effect,


%It is therefore crucial to systematically investigate the framing effect, to better understand and mitigate its impact. \gabis{This paragraph is important - I think that right now it's saying that we don't want models to be influenced by framing (since we want to mitigate its impact, right?) When we talked I think we had a more nuanced position?}




To better understand the framing effect in LLMs in comparison to human behavior,
we introduce the \name{} dataset (Section~\ref{sec:data}), comprising 1,000 statements, constructed through a three-step process, as shown in Figure~\ref{fig:fig1}.
First, we collect a set of real-world statements that express a clear negative or positive sentiment (e.g., ``I won the highest prize'').
%as exemplified in Figure~\ref{fig:fig1} -- ``I won the highest prize'' positive base statement. (2) next,
Second, we \emph{reframe} the text by adding a prefix or suffix with an opposite sentiment (e.g., ``I won the highest prize, \emph{although I lost all my friends on the way}'').
Finally, we collect human annotations by asking different participants
if they consider the reframed statement to be overall positive or negative.
% \gabist{This allows us to quantify the extent of \textit{sentiment shifts}, which is defined as labeling the sentiment aligning with the opposite framing, rather then the base sentiment -- e.g., voting ``negative'' for the statement ``I won the highest prize, although I lost all my friends on the way'', as it aligns with the opposite framing sentiment.}
We choose to annotate Amazon reviews, where sentiment is more robust, compared to e.g., the news domain which introduces confounding variables such as prior political leaning~\cite{druckman2004political}.


%While the implications of framing on sensitive and controversial topics like politics or economics are highly relevant to real-world applications, testing these subjects in a controlled setting is challenging. Such topics can introduce confounding variables, as annotators might rely on their personal beliefs or emotions rather than focusing solely on the framing, particularly when the content is emotionally charged~\cite{druckman2004political}. To balance real-world relevance with experimental reliability, we chose to focus on statements derived from Amazon reviews. These are naturally occurring, sentiment-rich texts that are less likely to trigger strong preexisting biases or emotional reactions. For instance, a review like ``The book was engaging'' can be framed negatively without invoking specific cultural or political associations. 

 In Section~\ref{sec:results}, we evaluate eight state-of-the-art LLMs
 % including \gpt{}~\cite{openai2024gpt4osystemcard}, \llama{}~\cite{dubey2024llama}, \mistral{}~\cite{jiang2023mistral}, \mixtral{}~\cite{mistral2023mixtral}, and \gemma{}~\cite{team2024gemma}, 
on the \name{} dataset and compare them against human annotations. We find  that LLMs are influenced by framing, somewhat similar to human behavior. All models show a \emph{strong} correlation ($r>0.57$) with human behavior.
%All models show a correlation with human responses of more than $0.55$ in Pearson's $r$ \gabis{@Gili check how people report this?}.
Moreover, we find that both humans and LLMs are more influenced by positive reframing rather than negative reframing. We also find that larger models tend to be more correlated with human behavior. Interestingly, \gpt{} shows the lowest correlation with human behavior. This raises questions about how architectural or training differences might influence susceptibility to framing. 
%\gabis{this last finding about \gpt{} stands in opposition to the start of the statement, right? Even though it's probably one of the largest models, it doesn't correlate with humans? If so, better to state this explicitly}

This work contributes to understanding the parallels between LLM and human cognition, offering insights into how cognitive mechanisms such as the framing effect emerge in LLMs.\footnote{\name{} data available at \url{https://huggingface.co/datasets/gililior/WildFrame}\\Code: ~\url{https://github.com/SLAB-NLP/WildFrame-Eval}}

%\gabist{It also raises fundamental philosophical and practical questions -- should LLMs aim to emulate human-like behavior, even when such behavior is susceptible to harmful cognitive biases? or should they strive to deviate from human tendencies to avoid reproducing these pitfalls?}\gabis{$\leftarrow$ also following Itay's comment, maybe this is better in the dicsussion, since we don't address these questions in the paper.} %\gabis{This last statement brings the nuance back, so I think it contradicts the previous parapgraph where we talked about ``mitigating'' the effect of framing. Also, I think it would be nice to discuss this a bit more in depth, maybe in the discussion section.}






\section{Related Work}
%%%%%%%%%%%%%%
% Know-Item Retrieval and Query Simulation
%%%%%%%%%%%%%%
\subsection{Query Simulation and Know-Item Retrieval}

Query simulation methods have been used for various purposes, including document expansion \cite{nogueira2019docT5query} and synthetic test collection generation \cite{Rahmani24synthetic}. In the context of known-item retrieval, these methods have been explored to improve retrieval strategies \cite{OgilvieCallan03combining} and evaluation frameworks \cite{Azzopardi06testbeds, hagen2015corpus}.



%% Query Simulation
\textit{Simulating} the known-item queries has long been an active research area \cite{balog2006overviewWebclef, Azzopardi07SimulatedQueries, Kim09desktop, Elsweiler2011Seeding}.
Early work \cite{Azzopardi07SimulatedQueries} generated synthetic queries using term-based likelihood models, selecting query terms based on their likelihood within a randomly chosen document. Later studies adapted this approach for desktop search \cite{Kim09desktop} and email re-finding \cite{Elsweiler2011Seeding}, demonstrating its effectiveness for simulated evaluations of know-item retrieval models.
%
The \textit{validation} of these query simulators has also been a key focus.
System ranking correlation \cite{balog2006overviewWebclef}, retrieval score distribution comparisons \cite{Azzopardi07SimulatedQueries}, and synthetic versus human query resemblance \cite{Kim09desktop} have been used to assess their reliability.


While valuable, known-item search queries differ significantly from TOT queries, which are longer and more complex. Despite progress in simulating known-item queries, TOT retrieval remains unexplored. This paper bridges that gap by introducing novel TOT query elicitation methods and adapting established validation techniques \cite{zeigler2000theory} to ensure alignment with real-world queries, enabling scalable and accurate simulated evaluations.






%%%%%%%%%%%%%%
% TOT Datasets
%%%%%%%%%%%%%%
\subsection{TOT Datasets}
Several datasets have been developed to support research on TOT retrieval, primarily collected from online CQA platforms and focused on specific domains. MS-TOT \cite{arguello-movie-identification} was constructed from the \textit{IRememberThisMovie} website and human-annotated with tags in the Movie domain. It also includes qualitative coding of TOT queries and demonstrates significant room for improvement in current retrieval technologies for such information needs. Similarly, \citet{gameTOT} collected TOT queries from Reddit's \textit{/r/tipofmyjoystick} subreddit in the Game domain, providing coded tag information. Other datasets include Reddit-TOMT \cite{Bhargav-2022-wsdm}, focused on movies and books from Reddit's \textit{/r/tipofmytongue} subreddit; TOT-Music \cite{Bhargav23MusicTOT}, targeting the Music domain from the same subreddit; and Whatsthatbook \cite{lin-etal-2023-whatsthatbook}, sourced from \textit{GoodReads}, focused on the Book domain.



In response to the domain specificity of these datasets, recent efforts have aimed to expand TOT datasets across multiple areas. \citet{Meier21-complex-reddit} expanded to general casual leisure domains using data from six Reddit subreddits, including games, books, and music, although other identified domains, such as videos and people, remain underrepresented. Similarly, TOMT-KIS \cite{frobe2023-performance-pred} extended the collection from \textit{/r/tipofmytongue} by adapting \citet{Bhargav-2022-wsdm}'s approach with fewer filtering restrictions, resulting in 1.28 million TOT queries. However, only 47\% of these queries have identified answers, and the dataset continues to exhibit severe domain skewness toward a few topics. 


In this work, we develop and validate TOT query elicitation methods using the Movie domain for robust evaluation, then expand to Landmark and Person to assess applicability across underrepresented domains.



\section{Method}
Overall, we elicit TOT queries from both LLMs and humans and validate them using two methods: system rank correlation (\S\ref{subsubsec:sys-rank-correlation}) and linguistic similarity (\S\ref{subsubsec:ling-sim}).

\subsection{Query Elicitation}
For LLM-elicited queries, we generate synthetic queries by exploring various prompting strategies. We experiment with different prompting conditions and model hyperparameters to identify the most effective prompt that yields the best validation results.  
This procedure is detailed in Section \S\ref{sec:llm-elicitation}.  

For human-elicited queries, we designed an interface that places participants in a TOT state using visual stimuli, such as movie stills, landmarks, and celebrity images. Participants then compose TOT queries as they would when posting on a CQA website.  
This procedure is described in Section \S\ref{sec:human-elicitation}.


\subsection{Query Validation}

\subsubsection{\textbf{System Rank Correlation}}\label{subsubsec:sys-rank-correlation}
\begin{figure} 
\centering
\includegraphics[trim=140 130 90 120, clip, width=\columnwidth]{03-synthetic/graphics/tau-validation.pdf}
\caption{
Validation of elicited queries using system rank correlation. We evaluate 40 different retrieval models using both CQA-based and elicited queries, ranking them based on search performance measured by MRR and NDCG. We then compute Kendall’s Tau and Pearson correlation to assess the agreement between the rankings derived from the two query sets.
}
\label{fig:validation-sys-rank}
\end{figure}

To validate the effectiveness of elicited TOT queries, we measure the correlation between its rankings of retrieval systems and rankings based on CQA queries for the same entities. This evaluation assesses whether retrieval models maintain consistent performance across different query sources but on the same entities. If the rankings derived from elicited queries strongly correlate with those from CQA-based queries, it indicates that the elicited queries capture similar retrieval challenges and retrieval effectiveness. A high correlation suggests that our synthetic and human-elicited queries can serve as reliable substitutes for traditionally collected CQA-based queries in evaluating retrieval systems.

To compute these correlations, we run the queries on 40 different retrieval models, comprising lexical and dense retrievers. The lexical retrievers include BM25 \cite{Robertson1995OkapiBM25} and language models with Dirichlet priors \cite{zhai2001DirichletSmoothing} using varying parameters. The dense retrievers include models of different sizes and performance levels, such as MiniLM-L6 and MiniLM-L12 \cite{miniLM}. To further introduce variation in systems, we include retrieval models with intentionally degraded performance by reinitializing the weights of certain layers in dense retrievers. Additionally, we incorporate an API-based closed-source LLM as one of the retrieval systems, specifically GPT-3.5-Turbo-Instruct, following the prompting format used in the baseline of the TREC 2024 TOT track to function as a ranker.
Furthermore, we include the top-ranked retrieval system from the TREC 2023 TOT track \cite{luis24totDPR, arguello2023overview} to provide a strong performance reference point.

Figure \ref{fig:validation-sys-rank} illustrates our validation strategy, showing how retrieval system rankings across different query sets provide insight into the reliability and effectiveness of our elicited queries.



%%%%%%%%%%%%%%%%%%%%%%%%
\subsubsection{\textbf{Linguistic Similarity}}\label{subsubsec:ling-sim}
\begin{figure} 
\centering
\includegraphics[width=0.9\columnwidth]{04-human/graphics/ms-tot-gold-pred.pdf}
\caption{
Validation of automatic annotation using the MS-TOT dataset.
Results show that GPT-4o-mini performs well in annotating TOT queries, closely aligning with human annotators. It achieves high accuracy and exhibits low Earth Mover’s Distance (EMD), indicating strong agreement with expert annotations.}
\label{fig:automatic-annotation-validation}
\end{figure}

Elicited TOT queries can exhibit substantial variation in writing style, word choice, experiences, and the presence of distorted memories \cite{Meier21-complex-reddit, arguello-movie-identification}. This diversity is inherent to the nature of TOT queries and is a valid characteristic of real-world TOT retrieval scenarios. Consequently, evaluating linguistic similarity between CQA-based and elicited queries using traditional methods, such as vector-based semantic similarity or lexicon-based similarity, may be ineffective.



To address this, we utilize a predefined set of TOT-specific linguistic codes introduced by \citet{arguello-movie-identification}. These handcrafted codes provide sentence-level annotations of TOT queries in the Movie domain, categorizing linguistic phenomena into eight top-level groups: `movie', `context', `previous-search', `social', `uncertainty', `opinion', `emotion', and `relative-comparison'. By leveraging this framework, we compare the linguistic distribution of codes between CQA-based and elicited queries rather than relying on direct semantic or lexical overlap.


To conduct this analysis, we annotate our elicited TOT queries in the Movie domain using these linguistic codes and compare the percentage distributions of codes found in CQA-based and elicited queries. To automate this process, we develop a language model-based automatic code annotator, prompting GPT-4o-mini\footnote{Temperature is set to 0 for reproducibility and consistency.} to produce JSON-formatted sentence-level annotations.



Before applying this annotator to LLM- and human-elicited queries, we first validate its performance on the MS-TOT dataset, where sentence-level gold annotations are available. We evaluate the annotator’s performance by computing precision and recall as prediction accuracy measures. Additionally, to assess annotation quality at a broader level, we compute query-level precision and recall, which measure the accuracy of identifying unique codes that appear within a multi-sentence query.

To further validate the annotator, we compare the distribution of annotated codes against the gold annotations using Earth Mover’s Distance (EMD), which quantifies how different the predicted code distribution is from the reference distribution. Figure \ref{fig:automatic-annotation-validation} presents the validation results of our automatic annotator on the MS-TOT dataset. Our evaluation shows that the annotator achieves sufficiently high accuracy and low EMD, confirming its suitability as an automatic labeler.



With this validation, we apply the automatic annotator to both LLM- and human-elicited queries in the Movie domain and analyze how their linguistic code distributions compare to those in CQA-based queries. However, we conduct this linguistic similarity validation only in the Movie domain, as there are no existing comprehensive handcrafted linguistic codes available for the Landmark and Person domains.

\section{TOT Query Elicitation from LLMs}\label{sec:llm-elicitation}

\subsection{Greedies}
We have two greedy methods that we're using and testing, but they both have one thing in common: They try every node and possible resistances, and choose the one that results in the largest change in the objective function.

The differences between the two methods, are the function. The first one uses the median (since we want the median to be >0.5, we just set this as our objective function.)

Second one uses a function to try to capture more nuances about the fact that we want the median to be over 0.5. The function is:

\[
\text{score}(\text{opinion}) =
\begin{cases} 
\text{maxScore}, & \text{if } \text{opinion} \geq 0.5 \\
\min\left(\frac{50}{0.5 - \text{opinion}}, \frac{\text{maxScore}}{2}\right), & \text{if } \text{opinion} < 0.5 
\end{cases}
\] 

Where we set maxScore to be $10000$.

\subsection{find-c}
Then for the projected methods where we use Huber-Loss, we also have a $find-c$ version (temporary name). This method initially finds the c for the rest of the run. 

The way it does it it randomly perturbs the resistances and opinions of every node, then finds the c value that most closely approximates the median for all of the perturbed scenarios (after finding the stable opinions). 


\begin{table*}[t]
\centering
\fontsize{11pt}{11pt}\selectfont
\begin{tabular}{lllllllllllll}
\toprule
\multicolumn{1}{c}{\textbf{task}} & \multicolumn{2}{c}{\textbf{Mir}} & \multicolumn{2}{c}{\textbf{Lai}} & \multicolumn{2}{c}{\textbf{Ziegen.}} & \multicolumn{2}{c}{\textbf{Cao}} & \multicolumn{2}{c}{\textbf{Alva-Man.}} & \multicolumn{1}{c}{\textbf{avg.}} & \textbf{\begin{tabular}[c]{@{}l@{}}avg.\\ rank\end{tabular}} \\
\multicolumn{1}{c}{\textbf{metrics}} & \multicolumn{1}{c}{\textbf{cor.}} & \multicolumn{1}{c}{\textbf{p-v.}} & \multicolumn{1}{c}{\textbf{cor.}} & \multicolumn{1}{c}{\textbf{p-v.}} & \multicolumn{1}{c}{\textbf{cor.}} & \multicolumn{1}{c}{\textbf{p-v.}} & \multicolumn{1}{c}{\textbf{cor.}} & \multicolumn{1}{c}{\textbf{p-v.}} & \multicolumn{1}{c}{\textbf{cor.}} & \multicolumn{1}{c}{\textbf{p-v.}} &  &  \\ \midrule
\textbf{S-Bleu} & 0.50 & 0.0 & 0.47 & 0.0 & 0.59 & 0.0 & 0.58 & 0.0 & 0.68 & 0.0 & 0.57 & 5.8 \\
\textbf{R-Bleu} & -- & -- & 0.27 & 0.0 & 0.30 & 0.0 & -- & -- & -- & -- & - &  \\
\textbf{S-Meteor} & 0.49 & 0.0 & 0.48 & 0.0 & 0.61 & 0.0 & 0.57 & 0.0 & 0.64 & 0.0 & 0.56 & 6.1 \\
\textbf{R-Meteor} & -- & -- & 0.34 & 0.0 & 0.26 & 0.0 & -- & -- & -- & -- & - &  \\
\textbf{S-Bertscore} & \textbf{0.53} & 0.0 & {\ul 0.80} & 0.0 & \textbf{0.70} & 0.0 & {\ul 0.66} & 0.0 & {\ul0.78} & 0.0 & \textbf{0.69} & \textbf{1.7} \\
\textbf{R-Bertscore} & -- & -- & 0.51 & 0.0 & 0.38 & 0.0 & -- & -- & -- & -- & - &  \\
\textbf{S-Bleurt} & {\ul 0.52} & 0.0 & {\ul 0.80} & 0.0 & 0.60 & 0.0 & \textbf{0.70} & 0.0 & \textbf{0.80} & 0.0 & {\ul 0.68} & {\ul 2.3} \\
\textbf{R-Bleurt} & -- & -- & 0.59 & 0.0 & -0.05 & 0.13 & -- & -- & -- & -- & - &  \\
\textbf{S-Cosine} & 0.51 & 0.0 & 0.69 & 0.0 & {\ul 0.62} & 0.0 & 0.61 & 0.0 & 0.65 & 0.0 & 0.62 & 4.4 \\
\textbf{R-Cosine} & -- & -- & 0.40 & 0.0 & 0.29 & 0.0 & -- & -- & -- & -- & - & \\ \midrule
\textbf{QuestEval} & 0.23 & 0.0 & 0.25 & 0.0 & 0.49 & 0.0 & 0.47 & 0.0 & 0.62 & 0.0 & 0.41 & 9.0 \\
\textbf{LLaMa3} & 0.36 & 0.0 & \textbf{0.84} & 0.0 & {\ul{0.62}} & 0.0 & 0.61 & 0.0 &  0.76 & 0.0 & 0.64 & 3.6 \\
\textbf{our (3b)} & 0.49 & 0.0 & 0.73 & 0.0 & 0.54 & 0.0 & 0.53 & 0.0 & 0.7 & 0.0 & 0.60 & 5.8 \\
\textbf{our (8b)} & 0.48 & 0.0 & 0.73 & 0.0 & 0.52 & 0.0 & 0.53 & 0.0 & 0.7 & 0.0 & 0.59 & 6.3 \\  \bottomrule
\end{tabular}
\caption{Pearson correlation on human evaluation on system output. `R-': reference-based. `S-': source-based.}
\label{tab:sys}
\end{table*}



\begin{table}%[]
\centering
\fontsize{11pt}{11pt}\selectfont
\begin{tabular}{llllll}
\toprule
\multicolumn{1}{c}{\textbf{task}} & \multicolumn{1}{c}{\textbf{Lai}} & \multicolumn{1}{c}{\textbf{Zei.}} & \multicolumn{1}{c}{\textbf{Scia.}} & \textbf{} & \textbf{} \\ 
\multicolumn{1}{c}{\textbf{metrics}} & \multicolumn{1}{c}{\textbf{cor.}} & \multicolumn{1}{c}{\textbf{cor.}} & \multicolumn{1}{c}{\textbf{cor.}} & \textbf{avg.} & \textbf{\begin{tabular}[c]{@{}l@{}}avg.\\ rank\end{tabular}} \\ \midrule
\textbf{S-Bleu} & 0.40 & 0.40 & 0.19* & 0.33 & 7.67 \\
\textbf{S-Meteor} & 0.41 & 0.42 & 0.16* & 0.33 & 7.33 \\
\textbf{S-BertS.} & {\ul0.58} & 0.47 & 0.31 & 0.45 & 3.67 \\
\textbf{S-Bleurt} & 0.45 & {\ul 0.54} & {\ul 0.37} & 0.45 & {\ul 3.33} \\
\textbf{S-Cosine} & 0.56 & 0.52 & 0.3 & {\ul 0.46} & {\ul 3.33} \\ \midrule
\textbf{QuestE.} & 0.27 & 0.35 & 0.06* & 0.23 & 9.00 \\
\textbf{LlaMA3} & \textbf{0.6} & \textbf{0.67} & \textbf{0.51} & \textbf{0.59} & \textbf{1.0} \\
\textbf{Our (3b)} & 0.51 & 0.49 & 0.23* & 0.39 & 4.83 \\
\textbf{Our (8b)} & 0.52 & 0.49 & 0.22* & 0.43 & 4.83 \\ \bottomrule
\end{tabular}
\caption{Pearson correlation on human ratings on reference output. *not significant; we cannot reject the null hypothesis of zero correlation}
\label{tab:ref}
\end{table}


\begin{table*}%[]
\centering
\fontsize{11pt}{11pt}\selectfont
\begin{tabular}{lllllllll}
\toprule
\textbf{task} & \multicolumn{1}{c}{\textbf{ALL}} & \multicolumn{1}{c}{\textbf{sentiment}} & \multicolumn{1}{c}{\textbf{detoxify}} & \multicolumn{1}{c}{\textbf{catchy}} & \multicolumn{1}{c}{\textbf{polite}} & \multicolumn{1}{c}{\textbf{persuasive}} & \multicolumn{1}{c}{\textbf{formal}} & \textbf{\begin{tabular}[c]{@{}l@{}}avg. \\ rank\end{tabular}} \\
\textbf{metrics} & \multicolumn{1}{c}{\textbf{cor.}} & \multicolumn{1}{c}{\textbf{cor.}} & \multicolumn{1}{c}{\textbf{cor.}} & \multicolumn{1}{c}{\textbf{cor.}} & \multicolumn{1}{c}{\textbf{cor.}} & \multicolumn{1}{c}{\textbf{cor.}} & \multicolumn{1}{c}{\textbf{cor.}} &  \\ \midrule
\textbf{S-Bleu} & -0.17 & -0.82 & -0.45 & -0.12* & -0.1* & -0.05 & -0.21 & 8.42 \\
\textbf{R-Bleu} & - & -0.5 & -0.45 &  &  &  &  &  \\
\textbf{S-Meteor} & -0.07* & -0.55 & -0.4 & -0.01* & 0.1* & -0.16 & -0.04* & 7.67 \\
\textbf{R-Meteor} & - & -0.17* & -0.39 & - & - & - & - & - \\
\textbf{S-BertScore} & 0.11 & -0.38 & -0.07* & -0.17* & 0.28 & 0.12 & 0.25 & 6.0 \\
\textbf{R-BertScore} & - & -0.02* & -0.21* & - & - & - & - & - \\
\textbf{S-Bleurt} & 0.29 & 0.05* & 0.45 & 0.06* & 0.29 & 0.23 & 0.46 & 4.2 \\
\textbf{R-Bleurt} & - &  0.21 & 0.38 & - & - & - & - & - \\
\textbf{S-Cosine} & 0.01* & -0.5 & -0.13* & -0.19* & 0.05* & -0.05* & 0.15* & 7.42 \\
\textbf{R-Cosine} & - & -0.11* & -0.16* & - & - & - & - & - \\ \midrule
\textbf{QuestEval} & 0.21 & {\ul{0.29}} & 0.23 & 0.37 & 0.19* & 0.35 & 0.14* & 4.67 \\
\textbf{LlaMA3} & \textbf{0.82} & \textbf{0.80} & \textbf{0.72} & \textbf{0.84} & \textbf{0.84} & \textbf{0.90} & \textbf{0.88} & \textbf{1.00} \\
\textbf{Our (3b)} & 0.47 & -0.11* & 0.37 & 0.61 & 0.53 & 0.54 & 0.66 & 3.5 \\
\textbf{Our (8b)} & {\ul{0.57}} & 0.09* & {\ul 0.49} & {\ul 0.72} & {\ul 0.64} & {\ul 0.62} & {\ul 0.67} & {\ul 2.17} \\ \bottomrule
\end{tabular}
\caption{Pearson correlation on human ratings on our constructed test set. 'R-': reference-based. 'S-': source-based. *not significant; we cannot reject the null hypothesis of zero correlation}
\label{tab:con}
\end{table*}

\section{Results}
We benchmark the different metrics on the different datasets using correlation to human judgement. For content preservation, we show results split on data with system output, reference output and our constructed test set: we show that the data source for evaluation leads to different conclusions on the metrics. In addition, we examine whether the metrics can rank style transfer systems similar to humans. On style strength, we likewise show correlations between human judgment and zero-shot evaluation approaches. When applicable, we summarize results by reporting the average correlation. And the average ranking of the metric per dataset (by ranking which metric obtains the highest correlation to human judgement per dataset). 

\subsection{Content preservation}
\paragraph{How do data sources affect the conclusion on best metric?}
The conclusions about the metrics' performance change radically depending on whether we use system output data, reference output, or our constructed test set. Ideally, a good metric correlates highly with humans on any data source. Ideally, for meta-evaluation, a metric should correlate consistently across all data sources, but the following shows that the correlations indicate different things, and the conclusion on the best metric should be drawn carefully.

Looking at the metrics correlations with humans on the data source with system output (Table~\ref{tab:sys}), we see a relatively high correlation for many of the metrics on many tasks. The overall best metrics are S-BertScore and S-BLEURT (avg+avg rank). We see no notable difference in our method of using the 3B or 8B model as the backbone.

Examining the average correlations based on data with reference output (Table~\ref{tab:ref}), now the zero-shoot prompting with LlaMA3 70B is the best-performing approach ($0.59$ avg). Tied for second place are source-based cosine embedding ($0.46$ avg), BLEURT ($0.45$ avg) and BertScore ($0.45$ avg). Our method follows on a 5. place: here, the 8b version (($0.43$ avg)) shows a bit stronger results than 3b ($0.39$ avg). The fact that the conclusions change, whether looking at reference or system output, confirms the observations made by \citet{scialom-etal-2021-questeval} on simplicity transfer.   

Now consider the results on our test set (Table~\ref{tab:con}): Several metrics show low or no correlation; we even see a significantly negative correlation for some metrics on ALL (BLEU) and for specific subparts of our test set for BLEU, Meteor, BertScore, Cosine. On the other end, LlaMA3 70B is again performing best, showing strong results ($0.82$ in ALL). The runner-up is now our 8B method, with a gap to the 3B version ($0.57$ vs $0.47$ in ALL). Note our method still shows zero correlation for the sentiment task. After, ranks BLEURT ($0.29$), QuestEval ($0.21$), BertScore ($0.11$), Cosine ($0.01$).  

On our test set, we find that some metrics that correlate relatively well on the other datasets, now exhibit low correlation. Hence, with our test set, we can now support the logical reasoning with data evidence: Evaluation of content preservation for style transfer needs to take the style shift into account. This conclusion could not be drawn using the existing data sources: We hypothesise that for the data with system-based output, successful output happens to be very similar to the source sentence and vice versa, and reference-based output might not contain server mistakes as they are gold references. Thus, none of the existing data sources tests the limits of the metrics.  


\paragraph{How do reference-based metrics compare to source-based ones?} Reference-based metrics show a lower correlation than the source-based counterpart for all metrics on both datasets with ratings on references (Table~\ref{tab:sys}). As discussed previously, reference-based metrics for style transfer have the drawback that many different good solutions on a rewrite might exist and not only one similar to a reference.


\paragraph{How well can the metrics rank the performance of style transfer methods?}
We compare the metrics' ability to judge the best style transfer methods w.r.t. the human annotations: Several of the data sources contain samples from different style transfer systems. In order to use metrics to assess the quality of the style transfer system, metrics should correctly find the best-performing system. Hence, we evaluate whether the metrics for content preservation provide the same system ranking as human evaluators. We take the mean of the score for every output on each system and the mean of the human annotations; we compare the systems using the Kendall's Tau correlation. 

We find only the evaluation using the dataset Mir, Lai, and Ziegen to result in significant correlations, probably because of sparsity in a number of system tests (App.~\ref{app:dataset}). Our method (8b) is the only metric providing a perfect ranking of the style transfer system on the Lai data, and Llama3 70B the only one on the Ziegen data. Results in App.~\ref{app:results}. 


\subsection{Style strength results}
%Evaluating style strengths is a challenging task. 
Llama3 70B shows better overall results than our method. However, our method scores higher than Llama3 70B on 2 out of 6 datasets, but it also exhibits zero correlation on one task (Table~\ref{tab:styleresults}).%More work i s needed on evaluating style strengths. 
 
\begin{table}%[]
\fontsize{11pt}{11pt}\selectfont
\begin{tabular}{lccc}
\toprule
\multicolumn{1}{c}{\textbf{}} & \textbf{LlaMA3} & \textbf{Our (3b)} & \textbf{Our (8b)} \\ \midrule
\textbf{Mir} & 0.46 & 0.54 & \textbf{0.57} \\
\textbf{Lai} & \textbf{0.57} & 0.18 & 0.19 \\
\textbf{Ziegen.} & 0.25 & 0.27 & \textbf{0.32} \\
\textbf{Alva-M.} & \textbf{0.59} & 0.03* & 0.02* \\
\textbf{Scialom} & \textbf{0.62} & 0.45 & 0.44 \\
\textbf{\begin{tabular}[c]{@{}l@{}}Our Test\end{tabular}} & \textbf{0.63} & 0.46 & 0.48 \\ \bottomrule
\end{tabular}
\caption{Style strength: Pearson correlation to human ratings. *not significant; we cannot reject the null hypothesis of zero corelation}
\label{tab:styleresults}
\end{table}

\subsection{Ablation}
We conduct several runs of the methods using LLMs with variations in instructions/prompts (App.~\ref{app:method}). We observe that the lower the correlation on a task, the higher the variation between the different runs. For our method, we only observe low variance between the runs.
None of the variations leads to different conclusions of the meta-evaluation. Results in App.~\ref{app:results}.



\section{TOT Query Elicitation from Human}\label{sec:human-elicitation}


The study of the tip-of-the-tongue phenomenon has been a long-standing area of research in psychology \cite{burke1991tip, jones1989back}. Many studies have attempted to induce TOT states in participants using auditory \cite{reefer1995name} or visual stimuli \cite{tranel2005landmarks}, enabling researchers to examine recognizability (whether a subject recognizes an entity from the stimulus) and retrievability (whether they can recall the entity's name or title).

Building on these methodologies, we employ visual stimuli in the Movie, Landmark, and Person domains to develop an interface that places participants in a TOT state and allows them to compose TOT queries about the entities they struggle to recall.

In this section, we describe the design process of our interface for eliciting human-written TOT queries from trained contracted participants, along with an analysis of the collected human-elicited queries.

\subsection{Greedies}
We have two greedy methods that we're using and testing, but they both have one thing in common: They try every node and possible resistances, and choose the one that results in the largest change in the objective function.

The differences between the two methods, are the function. The first one uses the median (since we want the median to be >0.5, we just set this as our objective function.)

Second one uses a function to try to capture more nuances about the fact that we want the median to be over 0.5. The function is:

\[
\text{score}(\text{opinion}) =
\begin{cases} 
\text{maxScore}, & \text{if } \text{opinion} \geq 0.5 \\
\min\left(\frac{50}{0.5 - \text{opinion}}, \frac{\text{maxScore}}{2}\right), & \text{if } \text{opinion} < 0.5 
\end{cases}
\] 

Where we set maxScore to be $10000$.

\subsection{find-c}
Then for the projected methods where we use Huber-Loss, we also have a $find-c$ version (temporary name). This method initially finds the c for the rest of the run. 

The way it does it it randomly perturbs the resistances and opinions of every node, then finds the c value that most closely approximates the median for all of the perturbed scenarios (after finding the stable opinions). 


\begin{table*}[t]
\centering
\fontsize{11pt}{11pt}\selectfont
\begin{tabular}{lllllllllllll}
\toprule
\multicolumn{1}{c}{\textbf{task}} & \multicolumn{2}{c}{\textbf{Mir}} & \multicolumn{2}{c}{\textbf{Lai}} & \multicolumn{2}{c}{\textbf{Ziegen.}} & \multicolumn{2}{c}{\textbf{Cao}} & \multicolumn{2}{c}{\textbf{Alva-Man.}} & \multicolumn{1}{c}{\textbf{avg.}} & \textbf{\begin{tabular}[c]{@{}l@{}}avg.\\ rank\end{tabular}} \\
\multicolumn{1}{c}{\textbf{metrics}} & \multicolumn{1}{c}{\textbf{cor.}} & \multicolumn{1}{c}{\textbf{p-v.}} & \multicolumn{1}{c}{\textbf{cor.}} & \multicolumn{1}{c}{\textbf{p-v.}} & \multicolumn{1}{c}{\textbf{cor.}} & \multicolumn{1}{c}{\textbf{p-v.}} & \multicolumn{1}{c}{\textbf{cor.}} & \multicolumn{1}{c}{\textbf{p-v.}} & \multicolumn{1}{c}{\textbf{cor.}} & \multicolumn{1}{c}{\textbf{p-v.}} &  &  \\ \midrule
\textbf{S-Bleu} & 0.50 & 0.0 & 0.47 & 0.0 & 0.59 & 0.0 & 0.58 & 0.0 & 0.68 & 0.0 & 0.57 & 5.8 \\
\textbf{R-Bleu} & -- & -- & 0.27 & 0.0 & 0.30 & 0.0 & -- & -- & -- & -- & - &  \\
\textbf{S-Meteor} & 0.49 & 0.0 & 0.48 & 0.0 & 0.61 & 0.0 & 0.57 & 0.0 & 0.64 & 0.0 & 0.56 & 6.1 \\
\textbf{R-Meteor} & -- & -- & 0.34 & 0.0 & 0.26 & 0.0 & -- & -- & -- & -- & - &  \\
\textbf{S-Bertscore} & \textbf{0.53} & 0.0 & {\ul 0.80} & 0.0 & \textbf{0.70} & 0.0 & {\ul 0.66} & 0.0 & {\ul0.78} & 0.0 & \textbf{0.69} & \textbf{1.7} \\
\textbf{R-Bertscore} & -- & -- & 0.51 & 0.0 & 0.38 & 0.0 & -- & -- & -- & -- & - &  \\
\textbf{S-Bleurt} & {\ul 0.52} & 0.0 & {\ul 0.80} & 0.0 & 0.60 & 0.0 & \textbf{0.70} & 0.0 & \textbf{0.80} & 0.0 & {\ul 0.68} & {\ul 2.3} \\
\textbf{R-Bleurt} & -- & -- & 0.59 & 0.0 & -0.05 & 0.13 & -- & -- & -- & -- & - &  \\
\textbf{S-Cosine} & 0.51 & 0.0 & 0.69 & 0.0 & {\ul 0.62} & 0.0 & 0.61 & 0.0 & 0.65 & 0.0 & 0.62 & 4.4 \\
\textbf{R-Cosine} & -- & -- & 0.40 & 0.0 & 0.29 & 0.0 & -- & -- & -- & -- & - & \\ \midrule
\textbf{QuestEval} & 0.23 & 0.0 & 0.25 & 0.0 & 0.49 & 0.0 & 0.47 & 0.0 & 0.62 & 0.0 & 0.41 & 9.0 \\
\textbf{LLaMa3} & 0.36 & 0.0 & \textbf{0.84} & 0.0 & {\ul{0.62}} & 0.0 & 0.61 & 0.0 &  0.76 & 0.0 & 0.64 & 3.6 \\
\textbf{our (3b)} & 0.49 & 0.0 & 0.73 & 0.0 & 0.54 & 0.0 & 0.53 & 0.0 & 0.7 & 0.0 & 0.60 & 5.8 \\
\textbf{our (8b)} & 0.48 & 0.0 & 0.73 & 0.0 & 0.52 & 0.0 & 0.53 & 0.0 & 0.7 & 0.0 & 0.59 & 6.3 \\  \bottomrule
\end{tabular}
\caption{Pearson correlation on human evaluation on system output. `R-': reference-based. `S-': source-based.}
\label{tab:sys}
\end{table*}



\begin{table}%[]
\centering
\fontsize{11pt}{11pt}\selectfont
\begin{tabular}{llllll}
\toprule
\multicolumn{1}{c}{\textbf{task}} & \multicolumn{1}{c}{\textbf{Lai}} & \multicolumn{1}{c}{\textbf{Zei.}} & \multicolumn{1}{c}{\textbf{Scia.}} & \textbf{} & \textbf{} \\ 
\multicolumn{1}{c}{\textbf{metrics}} & \multicolumn{1}{c}{\textbf{cor.}} & \multicolumn{1}{c}{\textbf{cor.}} & \multicolumn{1}{c}{\textbf{cor.}} & \textbf{avg.} & \textbf{\begin{tabular}[c]{@{}l@{}}avg.\\ rank\end{tabular}} \\ \midrule
\textbf{S-Bleu} & 0.40 & 0.40 & 0.19* & 0.33 & 7.67 \\
\textbf{S-Meteor} & 0.41 & 0.42 & 0.16* & 0.33 & 7.33 \\
\textbf{S-BertS.} & {\ul0.58} & 0.47 & 0.31 & 0.45 & 3.67 \\
\textbf{S-Bleurt} & 0.45 & {\ul 0.54} & {\ul 0.37} & 0.45 & {\ul 3.33} \\
\textbf{S-Cosine} & 0.56 & 0.52 & 0.3 & {\ul 0.46} & {\ul 3.33} \\ \midrule
\textbf{QuestE.} & 0.27 & 0.35 & 0.06* & 0.23 & 9.00 \\
\textbf{LlaMA3} & \textbf{0.6} & \textbf{0.67} & \textbf{0.51} & \textbf{0.59} & \textbf{1.0} \\
\textbf{Our (3b)} & 0.51 & 0.49 & 0.23* & 0.39 & 4.83 \\
\textbf{Our (8b)} & 0.52 & 0.49 & 0.22* & 0.43 & 4.83 \\ \bottomrule
\end{tabular}
\caption{Pearson correlation on human ratings on reference output. *not significant; we cannot reject the null hypothesis of zero correlation}
\label{tab:ref}
\end{table}


\begin{table*}%[]
\centering
\fontsize{11pt}{11pt}\selectfont
\begin{tabular}{lllllllll}
\toprule
\textbf{task} & \multicolumn{1}{c}{\textbf{ALL}} & \multicolumn{1}{c}{\textbf{sentiment}} & \multicolumn{1}{c}{\textbf{detoxify}} & \multicolumn{1}{c}{\textbf{catchy}} & \multicolumn{1}{c}{\textbf{polite}} & \multicolumn{1}{c}{\textbf{persuasive}} & \multicolumn{1}{c}{\textbf{formal}} & \textbf{\begin{tabular}[c]{@{}l@{}}avg. \\ rank\end{tabular}} \\
\textbf{metrics} & \multicolumn{1}{c}{\textbf{cor.}} & \multicolumn{1}{c}{\textbf{cor.}} & \multicolumn{1}{c}{\textbf{cor.}} & \multicolumn{1}{c}{\textbf{cor.}} & \multicolumn{1}{c}{\textbf{cor.}} & \multicolumn{1}{c}{\textbf{cor.}} & \multicolumn{1}{c}{\textbf{cor.}} &  \\ \midrule
\textbf{S-Bleu} & -0.17 & -0.82 & -0.45 & -0.12* & -0.1* & -0.05 & -0.21 & 8.42 \\
\textbf{R-Bleu} & - & -0.5 & -0.45 &  &  &  &  &  \\
\textbf{S-Meteor} & -0.07* & -0.55 & -0.4 & -0.01* & 0.1* & -0.16 & -0.04* & 7.67 \\
\textbf{R-Meteor} & - & -0.17* & -0.39 & - & - & - & - & - \\
\textbf{S-BertScore} & 0.11 & -0.38 & -0.07* & -0.17* & 0.28 & 0.12 & 0.25 & 6.0 \\
\textbf{R-BertScore} & - & -0.02* & -0.21* & - & - & - & - & - \\
\textbf{S-Bleurt} & 0.29 & 0.05* & 0.45 & 0.06* & 0.29 & 0.23 & 0.46 & 4.2 \\
\textbf{R-Bleurt} & - &  0.21 & 0.38 & - & - & - & - & - \\
\textbf{S-Cosine} & 0.01* & -0.5 & -0.13* & -0.19* & 0.05* & -0.05* & 0.15* & 7.42 \\
\textbf{R-Cosine} & - & -0.11* & -0.16* & - & - & - & - & - \\ \midrule
\textbf{QuestEval} & 0.21 & {\ul{0.29}} & 0.23 & 0.37 & 0.19* & 0.35 & 0.14* & 4.67 \\
\textbf{LlaMA3} & \textbf{0.82} & \textbf{0.80} & \textbf{0.72} & \textbf{0.84} & \textbf{0.84} & \textbf{0.90} & \textbf{0.88} & \textbf{1.00} \\
\textbf{Our (3b)} & 0.47 & -0.11* & 0.37 & 0.61 & 0.53 & 0.54 & 0.66 & 3.5 \\
\textbf{Our (8b)} & {\ul{0.57}} & 0.09* & {\ul 0.49} & {\ul 0.72} & {\ul 0.64} & {\ul 0.62} & {\ul 0.67} & {\ul 2.17} \\ \bottomrule
\end{tabular}
\caption{Pearson correlation on human ratings on our constructed test set. 'R-': reference-based. 'S-': source-based. *not significant; we cannot reject the null hypothesis of zero correlation}
\label{tab:con}
\end{table*}

\section{Results}
We benchmark the different metrics on the different datasets using correlation to human judgement. For content preservation, we show results split on data with system output, reference output and our constructed test set: we show that the data source for evaluation leads to different conclusions on the metrics. In addition, we examine whether the metrics can rank style transfer systems similar to humans. On style strength, we likewise show correlations between human judgment and zero-shot evaluation approaches. When applicable, we summarize results by reporting the average correlation. And the average ranking of the metric per dataset (by ranking which metric obtains the highest correlation to human judgement per dataset). 

\subsection{Content preservation}
\paragraph{How do data sources affect the conclusion on best metric?}
The conclusions about the metrics' performance change radically depending on whether we use system output data, reference output, or our constructed test set. Ideally, a good metric correlates highly with humans on any data source. Ideally, for meta-evaluation, a metric should correlate consistently across all data sources, but the following shows that the correlations indicate different things, and the conclusion on the best metric should be drawn carefully.

Looking at the metrics correlations with humans on the data source with system output (Table~\ref{tab:sys}), we see a relatively high correlation for many of the metrics on many tasks. The overall best metrics are S-BertScore and S-BLEURT (avg+avg rank). We see no notable difference in our method of using the 3B or 8B model as the backbone.

Examining the average correlations based on data with reference output (Table~\ref{tab:ref}), now the zero-shoot prompting with LlaMA3 70B is the best-performing approach ($0.59$ avg). Tied for second place are source-based cosine embedding ($0.46$ avg), BLEURT ($0.45$ avg) and BertScore ($0.45$ avg). Our method follows on a 5. place: here, the 8b version (($0.43$ avg)) shows a bit stronger results than 3b ($0.39$ avg). The fact that the conclusions change, whether looking at reference or system output, confirms the observations made by \citet{scialom-etal-2021-questeval} on simplicity transfer.   

Now consider the results on our test set (Table~\ref{tab:con}): Several metrics show low or no correlation; we even see a significantly negative correlation for some metrics on ALL (BLEU) and for specific subparts of our test set for BLEU, Meteor, BertScore, Cosine. On the other end, LlaMA3 70B is again performing best, showing strong results ($0.82$ in ALL). The runner-up is now our 8B method, with a gap to the 3B version ($0.57$ vs $0.47$ in ALL). Note our method still shows zero correlation for the sentiment task. After, ranks BLEURT ($0.29$), QuestEval ($0.21$), BertScore ($0.11$), Cosine ($0.01$).  

On our test set, we find that some metrics that correlate relatively well on the other datasets, now exhibit low correlation. Hence, with our test set, we can now support the logical reasoning with data evidence: Evaluation of content preservation for style transfer needs to take the style shift into account. This conclusion could not be drawn using the existing data sources: We hypothesise that for the data with system-based output, successful output happens to be very similar to the source sentence and vice versa, and reference-based output might not contain server mistakes as they are gold references. Thus, none of the existing data sources tests the limits of the metrics.  


\paragraph{How do reference-based metrics compare to source-based ones?} Reference-based metrics show a lower correlation than the source-based counterpart for all metrics on both datasets with ratings on references (Table~\ref{tab:sys}). As discussed previously, reference-based metrics for style transfer have the drawback that many different good solutions on a rewrite might exist and not only one similar to a reference.


\paragraph{How well can the metrics rank the performance of style transfer methods?}
We compare the metrics' ability to judge the best style transfer methods w.r.t. the human annotations: Several of the data sources contain samples from different style transfer systems. In order to use metrics to assess the quality of the style transfer system, metrics should correctly find the best-performing system. Hence, we evaluate whether the metrics for content preservation provide the same system ranking as human evaluators. We take the mean of the score for every output on each system and the mean of the human annotations; we compare the systems using the Kendall's Tau correlation. 

We find only the evaluation using the dataset Mir, Lai, and Ziegen to result in significant correlations, probably because of sparsity in a number of system tests (App.~\ref{app:dataset}). Our method (8b) is the only metric providing a perfect ranking of the style transfer system on the Lai data, and Llama3 70B the only one on the Ziegen data. Results in App.~\ref{app:results}. 


\subsection{Style strength results}
%Evaluating style strengths is a challenging task. 
Llama3 70B shows better overall results than our method. However, our method scores higher than Llama3 70B on 2 out of 6 datasets, but it also exhibits zero correlation on one task (Table~\ref{tab:styleresults}).%More work i s needed on evaluating style strengths. 
 
\begin{table}%[]
\fontsize{11pt}{11pt}\selectfont
\begin{tabular}{lccc}
\toprule
\multicolumn{1}{c}{\textbf{}} & \textbf{LlaMA3} & \textbf{Our (3b)} & \textbf{Our (8b)} \\ \midrule
\textbf{Mir} & 0.46 & 0.54 & \textbf{0.57} \\
\textbf{Lai} & \textbf{0.57} & 0.18 & 0.19 \\
\textbf{Ziegen.} & 0.25 & 0.27 & \textbf{0.32} \\
\textbf{Alva-M.} & \textbf{0.59} & 0.03* & 0.02* \\
\textbf{Scialom} & \textbf{0.62} & 0.45 & 0.44 \\
\textbf{\begin{tabular}[c]{@{}l@{}}Our Test\end{tabular}} & \textbf{0.63} & 0.46 & 0.48 \\ \bottomrule
\end{tabular}
\caption{Style strength: Pearson correlation to human ratings. *not significant; we cannot reject the null hypothesis of zero corelation}
\label{tab:styleresults}
\end{table}

\subsection{Ablation}
We conduct several runs of the methods using LLMs with variations in instructions/prompts (App.~\ref{app:method}). We observe that the lower the correlation on a task, the higher the variation between the different runs. For our method, we only observe low variance between the runs.
None of the variations leads to different conclusions of the meta-evaluation. Results in App.~\ref{app:results}.
\section{Discussion}\label{sec:resource}


\textbf{Resource Contribution}.
Using the LLM-elicitation method from Experiment ID 13 (Prompt Version 6), we generated synthetic TOT queries for entities collected through the visual stimuli selection process, resulting in 1,687 queries in the Movie domain, 330 in the Landmark domain, and 1,946 in the Person domain. Additionally, through the human-elicitation method, we collected 584 human-elicited TOT queries spanning all three domains.


The full release of these queries is scheduled for the TREC 2025 TOT track, where they will be included as part of the official test collection. However, for TREC 2024, and in this work, we have released 450 synthetic queries (150 per domain) from the full set of generated queries. Each query in the dataset is accompanied by its corresponding Wikidata ID, domain name, and entity name, ensuring clear entity association for retrieval experiments and analysis. Alongside these queries, we provide the source code for query generation and experimentation, as well as the visual stimuli entity set with corresponding image URLs and Wikidata ID. We also release the MTurk-based human query collection interface, allowing researchers to replicate or extend the human TOT query elicitation process.



\textbf{Availability}.
At the time of review, LLM-elicited queries are publicly available as part of the TREC 2024 TOT track test collection\footnote{\url{https://github.com/kimdanny/llm-tot-query-elicitation}} and can also be accessed at the track website\footnote{\url{https://trec-tot.github.io/guidelines-2024}}. The human-elicited queries will be released as part of the TREC 2025 TOT track, aligning with the SIGIR 2025 conference.
%
Although the human-elicited queries are not yet publicly available, we have released the source code for the human query collection interface used in pilot testing on Amazon MTurk\footnote{\url{https://github.com/kimdanny/human-tot-query-elicitation-mturk}}. This allows researchers to explore and reproduce the query elicitation process for future development.

Both datasets are, and will continue to be, freely available under open licensing terms, ensuring unrestricted access for academic researchers and industry practitioners to support research and development in TOT retrieval.



\textbf{Utility}.
%
The resource is well-documented and designed for easy integration into retrieval experiments. Queries are provided in pure text format for compatibility with retrieval models, and a baseline implementation with tools for data loading and retrieval is available in the public repositories.
Additionally, this paper details the data provenance, processing, and experimentation steps, ensuring that future researchers can expand the dataset or adopt the TOT query elicitation method for other domains, supporting reproducibility and innovation.




\textbf{Novelty and Predicted Impact}.
Our work represents a major shift in TOT query collection methodology, moving beyond CQA-based datasets to LLM- and human-elicited queries, providing a scalable and flexible alternative for TOT dataset creation. Unlike previous approaches, our method eliminates the need for manual labeling, avoids data restrictions, and mitigates domain skewness in CQA datasets, which overrepresent casual leisure topics like movies and books. By incorporating underrepresented domains such as Person and Landmark, our dataset extends beyond traditional leisure-focused queries, supporting the development of general-domain TOT retrieval systems while enabling simulated evaluation independent of CQA constraints.

While TOT retrieval builds on known-item retrieval research, our dataset and methodology provide new tools for evaluating and training retrieval systems to handle TOT queries more effectively. We anticipate its long-term value and plan to incrementally expand domain coverage through future TREC tracks, ensuring broader applicability and comprehensive evaluation in TOT retrieval research.

\textbf{Methodological Implications}.
While our findings show that both LLM- and human-elicited queries are effective, they serve complementary roles: LLMs offer scalability and efficiency, whereas human queries may provide authentic linguistic patterns and user behaviors. A hybrid approach can balance dataset diversity and efficiency, leading to more comprehensive evaluations of TOT retrieval systems.

Beyond TOT retrieval, our elicitation methods could support vague or exploratory search scenarios, where users struggle to articulate precise queries. Additionally, LLM-based query generation could aid low-resource domains, simulating real-world search behaviors where query logs are scarce or unavailable, broadening its impact across information retrieval research.

\textbf{Limitation and Future Work}.
A limitation of our current LLM-elicitation method is that prompts are domain-specific, limiting their generalizability. Future work should develop generalized prompting strategies to elicit TOT queries across diverse search contexts without extensive manual tuning.
%
Additionally, expanding the methodology to multi-turn interactions could better reflect real-world TOT search behavior, where users iteratively refine queries as they recover missing information. Simulating this step-by-step recall process could improve query realism and retrieval effectiveness, further advancing TOT retrieval research.
\section{Conclusion}\label{sec:conclusion}
This work introduces a novel approach to TOT query elicitation, leveraging LLMs and human participants to move beyond the limitations of CQA-based datasets. Through system rank correlation and linguistic similarity validation, we demonstrate that LLM- and human-elicited queries can effectively support the simulated evaluation of TOT retrieval systems. Our findings highlight the potential for expanding TOT retrieval research into underrepresented domains while ensuring scalability and reproducibility. The released datasets and source code provide a foundation for future research, enabling further advancements in TOT retrieval evaluation and system development.
\newpage
\section*{Acknowledgments}

\bibliographystyle{ACM-Reference-Format}
\bibliography{XX-references.bib}

\newpage
\appendix
% \section{Framework Details}
% Our framework is described in Algorithm~\ref{algorithm}, and compared with former baselines in Table~\ref{table:comparison}. Distinct with several methods generating Python code for visualization directly, we use VQL as an intermediate representation to bridge natural language queries and visualization code. Additionally, our framework can be easily optimized by adding some useful tools such as Retrieval Augmented Generation. Moreover, our method supports handling multi-table data and the visualization can be customized according to humans' preferences. Our framework utilizes the agent-based collaborative workflow, which consists of data preprocessing, generation, and error correction, organized with the modular design.

% \begin{algorithm}
% \small
% \caption{\system Framework}
% \label{algorithm}
% \begin{algorithmic}[1]
% \Function{\nlvis}{$Q$, $S$}
%     \State Initialize $Mem \gets \{Q,S\}$
%     \State $(S', A) \gets \textsc{Processor}(Mem)$
%     \State $Mem.update(S', A)$
%     \State $V \gets \textsc{Composer}(Mem)$
%     \State $Mem.update(V)$
%     \State $Chart, isValid \gets \textsc{Validator}(Mem)$
%     \While{not $isValid$}
%         \State $V \gets \textsc{Refine}(Mem)$
%         \State $Mem.update(V)$
%         \State $Chart, isValid \gets \textsc{Validator}(Mem)$
%     \EndWhile
%     \State \Return $Chart$
% \EndFunction
% \end{algorithmic}

% \end{algorithm}




% \begin{table*}[!t]
%     \centering
    
%     \vspace{-1em}
%     \scalebox{0.68}{
%     \begin{tabular}{lccccccc}
%         \toprule[1.5pt]
%         \multirow{3}{*}{\textbf{Framework}} & \multicolumn{2}{c}{\textbf{System Features}} & \multicolumn{2}{c}{\textbf{Visualization Capabilities}} & \multicolumn{3}{c}{\textbf{Agentic Workflow}} \\
%         \cmidrule(lr){2-3} \cmidrule(lr){4-5} \cmidrule(lr){6-8}
%         & \textbf{VQL as} & \textbf{Extensible} & \textbf{Multi-Table} & \textbf{Customizable} & \textbf{Data} & \textbf{Modular} & \textbf{Error-} \\
%         & \textbf{Thoughts} & \textbf{Optimization} & \textbf{Support} & \textbf{Styling} & \textbf{Preprocess} & \textbf{Design} & \textbf{Correction} \\
%         \midrule
%         Chat2VIS~\cite{chat2vis} & \textcolor{red}{\ding{56}} & \textcolor{red}{\ding{56}} & \textcolor{red}{\ding{56}} & \textcolor{red}{\ding{56}} & \textcolor{green!60!black}{\ding{52}} & \textcolor{red}{\ding{56}} & \textcolor{red}{\ding{56}} \\
%         Mirror~\cite{mirror} & \textcolor{red}{\ding{56}} & \textcolor{red}{\ding{56}} & \textcolor{red}{\ding{56}} & \textcolor{red}{\ding{56}} & \textcolor{red}{\ding{56}} & \textcolor{green!60!black}{\ding{52}} & \textcolor{red}{\ding{56}} \\
        
%         LIDA~\cite{lida} & \textcolor{red}{\ding{56}} & \textcolor{green!60!black}{\ding{52}} & \textcolor{red}{\ding{56}} & \textcolor{green!60!black}{\ding{52}} & \textcolor{green!60!black}{\ding{52}} & \textcolor{green!60!black}{\ding{52}} & \textcolor{red}{\ding{56}} \\
%         CoML4VIS~\cite{coml} & \textcolor{red}{\ding{56}} & \textcolor{red}{\ding{56}} & \textcolor{green!60!black}{\ding{52}} & \textcolor{red}{\ding{56}} & \textcolor{green!60!black}{\ding{52}} & \textcolor{red}{\ding{56}} & \textcolor{red}{\ding{56}} \\
        
%         Prompt4VIS~\cite{prompt4vis} & \textcolor{green!60!black}{\ding{52}} & \textcolor{red}{\ding{56}} & \textcolor{green!60!black}{\ding{52}} & \textcolor{red}{\ding{56}} & \textcolor{green!60!black}{\ding{52}} & \textcolor{green!60!black}{\ding{52}} & \textcolor{red}{\ding{56}} \\
        
%         CoT-Vis~\cite{cotvis} & \textcolor{green!60!black}{\ding{52}} & \textcolor{red}{\ding{56}} & \textcolor{red}{\ding{56}} & \textcolor{red}{\ding{56}} & \textcolor{green!60!black}{\ding{52}} & \textcolor{red}{\ding{56}} & \textcolor{red}{\ding{56}} \\

%         \midrule
%         \SystemName (Ours) & \textcolor{green!60!black}{\ding{52}} & \textcolor{green!60!black}{\ding{52}} & \textcolor{green!60!black}{\ding{52}} & \textcolor{green!60!black}{\ding{52}} & \textcolor{green!60!black}{\ding{52}} & \textcolor{green!60!black}{\ding{52}} & \textcolor{green!60!black}{\ding{52}} \\
%         \bottomrule[1.5pt]
%     \end{tabular}}
% \caption{Comparison of various \nlvis frameworks. }  \label{table:comparison}
% \vspace{-1em}
% \end{table*}

\section{Detailed Experiment Setups}
\label{detailed_experiment_setups}
\paragraph{Baselines.}
\label{detailed_baselines}
% We implemented our experiment compared with three recent baselines. Note that, we also tried to use Code Interpreter as a baseline, but due to the rate limit of API constraint, the evaluation failed to generate visualizations via direct .csv files.
This study compares our approach with three state-of-the-art baselines. We also attempted to include Code Interpreter as a baseline; however, API rate limitations prevent the direct generation of visualizations from CSV files.

\begin{itemize}[leftmargin=*, itemsep=0pt] 
    \item \textbf{Chat2Vis} \cite{chat2vis}: It generates data visualizations by leveraging prompt engineering to translate natural language descriptions into visualizations. It uses a language-based table description, which includes column types and sample values, to inform the visualization generation process.\item \textbf{LIDA} \cite{lida}: It structures visualization generation as a four-step process, where each step builds on the previous one to incrementally translate natural language inputs into visualizations. It uses a JSON format to describe column statistics and samples, making it adaptable across various visualization tasks.
    \item \textbf{CoML4Vis} \cite{coml}: 
    % Building on a data science code generation framework, CoML4Vis 
    It utilizes a few-shot prompt that integrates multiple tables into a single visualization task. It summarizes data table information, including column names and samples, and then applies a few-shot prompt to guide visualization generation.
\end{itemize}

\paragraph{Metrics.}
\label{detailed_metrics}
Our evaluation framework involves five main metrics:
\begin{itemize}[leftmargin=*, itemsep=0pt] 
    \item \textbf{Invalid Rate} represents the percentage of visualizations that fail to render due to issues like incorrect API usage or other code errors.
    \item \textbf{Illegal Rate} indicates the percentage of visualizations that do not meet query requirements, which can include incorrect data transformations, mismatched chart types, or improper visualizations.
    \item \textbf{Readability Score} is the average score (range 1-5) assigned by a vision language model, like GPT-4V, for valid and legal visualizations, assessing their visual clarity and ease of interpretation.
    \item \textbf{Pass Rate} measures the proportion of visualizations in the evaluation set that are both valid (able to render) and legal (meet the query requirements).
    \item \textbf{Quality Score} is set to 0 for invalid or illegal visualizations; otherwise, it is equal to the readability score, providing an overall assessment of visualization quality factoring in both functionality and clarity.
\end{itemize}
To thoroughly evaluate each main metric, we further break them down into the following detailed assessment criteria:
\begin{itemize}[leftmargin=4mm, itemsep=0.05mm] 
    \item \textbf{Code Execution Check} verifies that the Python code generated by the model can be successfully executed.
    \item \textbf{Surface-form Check} ensures that the generated code includes necessary elements to produce a visualization like function calls to display the chart.
    \item \textbf{Chart Type Check} verifies whether the extracted chart type from the visualization matches the ground truth.
    \item \textbf{Data Check} assesses if the data used in the visualization matches the ground truth, taking into consideration potential channel swaps based on specified channels.
    \item \textbf{Order Check} evaluates whether the sorting of visual elements follows the specified query requirements.
    \item \textbf{Layout Check} examines issues like text overflow or element overlap within visualizations.
    \item \textbf{Scale \& Ticks Check} ensures that scales and ticks are appropriately chosen, avoiding unconventional representations.
    \item \textbf{Overall Readability Rating} integrates various readability checks to provide a comprehensive score considering layout, scale, text clarity, and arrangement.
\end{itemize}

% For all evaluation results, these metrics are averaged across the dataset to provide an overarching view of model performance. These metrics collectively ensure that visualizations are not only correct in terms of execution but also effective in communicating the intended data narratives.
The evaluation metrics are averaged across the dataset to provide a comprehensive overview of the model's performance. Together, these metrics ensure that the visualizations are both accurate in execution and effective in conveying the intended data narratives.



\begin{table}[!t]
\centering
\setlength{\belowcaptionskip}{0em} 
% \vspace{-1em}
\begin{tabular}{lcc}
\toprule[1.5pt]
\textbf{Model} & \textbf{P-corr} & \textbf{P-value} \\
\midrule
GPT-4o-mini & \textbf{0.6503} & 0.000 \\
GPT-4o & 0.5648 & 0.000 \\
\bottomrule[1.5pt]
\end{tabular}
\caption{ The Pearson correlations of GPT-4o-mini and GPT-4o with human judgments on readability scores. }
\label{tab:pearson_corr}
\vspace{-1em}
\end{table}

\begin{table*}[!ht]
\centering

\vspace{-1em}
\begin{tabular}{l|ccc|ccc}
\toprule
\multirow{2}{*}{Method} & \multicolumn{3}{c|}{Single Table} & \multicolumn{3}{c}{Multiple Tables} \\
\cmidrule(l){2-4} \cmidrule(l){5-7}
 & prompt & response & total & prompt & response & total \\
\midrule
LIDA & 1386.23 & 237.90 & 1624.13 & \multicolumn{3}{c}{N/A} \\
Chat2Vis & 414.35 & 451.30 & 865.65 & \multicolumn{3}{c}{N/A} \\
CoML4Vis & 2614.76 & 279.86 & 2894.62 & 3069.62 & 307.67 & 3377.29 \\
\system & 5122.99 & 777.63 & 5900.62 & 5613.96 & 1014.10 & 6628.06 \\
\bottomrule
\end{tabular}
\caption{Token usage comparison for different methods. N/A indicates that LIDA and Chat2Vis cannot handle multiple table scenarios.}
\label{tab:token_usage}
\end{table*}

\begin{table}[ht]
\centering
\scalebox{1}{
\begin{tabular}{l|ccc}
\toprule
Agent & \#Input & \#Output & \#Total \\
\midrule
Processor & 1486.07 & 569.58 & 1755.65\\
Composer & 3268.32 & 221.74 & 3490.07 \\
Validator & 1051.82 & 127.85 & 1179.67  \\
\bottomrule
\end{tabular}}
\caption{Token usage of three agents in \system.} \label{tab:token_agent} 
\vspace{-1em}
\end{table}

\paragraph{Implement Details.}
Our system is implemented in Python 3.9, utilizing GPT-4o \citep{openai_gpt4o_2024}, GPT-4o-mini~\cite{openai2024gpt4omini}, and GPT-3.5-turbo~\cite{chatgpt3.5} as the backbone model for all approaches, with the temperature set to 0 for consistent outputs. GPT-4o-mini serves as the vision language model for readability evaluation. We interact with these models through the Azure OpenAI API. The specific prompt templates for each agent, crucial for guiding their respective roles in the visualization generation process, are detailed in Appendix~\ref{prompt_details}. Token usages of \system and baselines are demonstrated in Table~\ref{tab:token_usage}, and usage for each agent in our \system is shown in Table~\ref{tab:token_agent}. Additionally, our evaluations are conducted in VisEval Benchmark (with MIT license).

\paragraph{Human Annotation.}
\label{human}
The annotation is conducted by 5 authors of this paper independently. As acknowledged, the diversity of annotators plays a crucial role in reducing bias and enhancing the reliability of the benchmark. These annotators have knowledge in the data visualization domain, with different genders, ages, and educational backgrounds. The educational backgrounds of annotators are above undergraduate. To ensure the annotators can proficiently mark the data, we provide them with detailed tutorials, teaching them how to judge the quality of data visualization. We also provide them with detailed criteria and task requirements in each annotation process shown in Figure~\ref{fig:annotation}. Two experiments requiring human annotation are detailed as follows:

\begin{figure}[!ht]
    \centering
    \includegraphics[width=\linewidth]{figure/score_distribution.pdf}
    \caption{Comparison of score density distribution between GPT-4o, GPT-4o-mini and human average score.}
    \label{fig:score_distribution}
\end{figure}

\begin{table*}[!ht]
\centering
\begin{tabular}{l|ccc}
\toprule
& Invalid Rate & Illegal Rate & Pass Rate \\
\midrule
\system & 4.66\% & 23.97\% & 71.35\% \\
w. CoT for Validator & 5.82\% & 23.39\% & 70.78\% \\
w. original schema for Validator & 4.80\% & 24.22\% & 70.97\% \\
\bottomrule
\end{tabular}
\caption{Additional exploration for Validator (using GPT-3.5-turbo).} 
\vspace{-1em} 
\label{tab:ablation_validator}
\end{table*}

\begin{itemize}[leftmargin=*, itemsep=0pt]
    \item \textbf{Pearson Correlation of Visual Language Model.} We conduct human annotation frameworks to compare the ability of the visual language model for MLLM-as-a-Judge~\cite{chen2024mllm}, providing the readability score. Our annotation framework is shown in Figure~\ref{fig:annotation}. The final Pearson scores are demonstrated in Table~\ref{tab:pearson_corr}, with its density distribution in Figure~\ref{fig:score_distribution}. The detailed instructions can be found in Figure~\ref{fig:scoring_instructions}.
    \item \textbf{Qualitative comparison to calculate ELO Scores.} We conduct human-judgments evaluations to compare which visualization generated by different models meets the query requirement more precisely. The leaderboard is shown in Table~\ref{tab:elo_rankings}, and Figure~\ref{fig:elo} shows the judgment framework. Each model starts with a base ELO score of 1500. After each pairwise comparison, the scores are updated based on the outcome and the current scores of the models involved. The hyperparameters are set as follows: the $K$-factor is set to 32, which determines the maximum change in rating after a single comparison. We conduct two sets of evaluations: one for single-table queries and another for multiple-table queries, with 1000 bootstrap iterations for each set to ensure statistical robustness. For each model's ELO rating, we report the 95\% confidence intervals computed through bootstrap resampling, providing a measure of rating stability. The evaluation process involves presenting human judges with a query and two visualizations, asking them to select the one that better meets the query requirements. This process is repeated across all model pairs and queries in our test set. The detailed guidance provides to the human evaluators can be found in Figure~\ref{fig:evaluation_instructions}, which outlines the criteria for judging visualization quality and relevance to the given query.


\end{itemize}

\begin{figure}[!ht]
	\centering
    \setlength{\belowcaptionskip}{-1em}
	\includegraphics[width=0.98\linewidth,scale=1.0]
    {./figure/library.pdf}
    \vspace{-1em}
	\caption{Performance of different models using \texttt{Matplotlib} and \texttt{Seaborn} libraries, using GPT-3.5-turbo.
    % \yao{larger fontsize?}
    }
\label{fig: library}
\end{figure}

\begin{figure*}[!h]
    \centering
    \includegraphics[width=0.98\linewidth]{figure/annotation.pdf}
    \caption{Screenshot of human annotation process in readability score.}
    \label{fig:annotation}
\end{figure*}

\begin{figure*}[ht]
\centering
\vspace{1em}
\begin{tcolorbox}[enhanced,attach boxed title to top center={yshift=-3mm,yshifttext=-1mm},boxrule=0.9pt, 
  colback=gray!00,colframe=black!50,colbacktitle=gray,
  title=Readability Scoring Instruction,
  boxed title style={size=small,colframe=gray} ]
\small
\textbf{Scoring Instructions:} Please evaluate the charts based on the following criteria, with a score range from 1 to 5, where 1 indicates very poor quality and 5 indicates excellent quality. You should focus on the following aspects:

\vspace{0.5em}
\textbf{1. Chart Colors:}
\begin{itemize}
    \item Are the colors clear and natural, effectively conveying the information?
    \item Color blindness accessibility: Are the color combinations easy to distinguish, especially for users with color blindness?
\end{itemize}

\vspace{0.5em}
\textbf{2. Title and Axis Labels:}
\begin{itemize}
    \item Ensure the chart has a clear title.
    \item Do the X-axis and Y-axis labels exist, and are they complete?
    \item Check if the labels are difficult to read, e.g., are they written vertically instead of horizontally?
    \item The title should not be a direct question; instead, it should describe the data or trends being presented.
\end{itemize}

\vspace{0.5em}
\textbf{3. Legend Completeness:}
\begin{itemize}
    \item Is the legend complete, and does it clearly indicate the color labels for different data series?
    \item Ensure each color has a corresponding legend, making it easy for users to understand what the data represents.
\end{itemize}

\vspace{0.5em}
\textbf{Scoring Scale:}
\begin{itemize}
    \item \textbf{1 Point:} Very poor, unable to understand or severely lacking information.
    \item \textbf{2 Points:} Poor quality, multiple issues present, difficult to extract information.
    \item \textbf{3 Points:} Fair, conveys some information but still has room for improvement.
    \item \textbf{4 Points:} Good, generally clear charts with minor areas for improvement.
    \item \textbf{5 Points:} Excellent, outstanding chart design with clear and effective information presentation.
\end{itemize}

Please consider the above factors when assessing the charts and provide the appropriate score. Thank you for your cooperation and effort!
\end{tcolorbox}
\vspace{-7pt}
\caption{Instructions for human annorators in annotating readability scoring.}
\label{fig:scoring_instructions}
\vspace{1em}
\end{figure*}

\begin{figure*}[!ht]
    \centering
    \includegraphics[width=0.98\linewidth]{figure/elo.pdf}
    \caption{Screenshot of ELO score evaluation framework for Human-as-a-Judge.}
    \label{fig:elo}
\end{figure*}

\begin{figure*}[ht]
\centering
\vspace{1em}
\begin{tcolorbox}[enhanced,attach boxed title to top center={yshift=-3mm,yshifttext=-1mm},boxrule=0.9pt, 
  colback=gray!00,colframe=black!50,colbacktitle=gray,
  title=Visualization Comparison Guidance,
  boxed title style={size=small,colframe=gray} ]
\small
Welcome to the visualization comparison evaluation. Your task is to judge which model-generated visualization better meets the requirements of the natural language query.

\vspace{0.5em}
\textbf{Evaluation criteria:}
\begin{enumerate}
    \item \textbf{Appropriateness of chart type:} Check if the selected chart type is suitable for expressing the data and relationships required by the query.
    \item \textbf{Data completeness:} Ensure the chart includes all necessary data required by the query.
    \item \textbf{Readability:} Assess the clarity of the chart, accuracy of labels, and overall layout.
    \item \textbf{Aesthetics:} Consider if the chart's color scheme, proportions, and overall design are visually pleasing.
    \item \textbf{Information conveyance:} Judge if the chart effectively conveys the main information or insights required by the query.
\end{enumerate}

\vspace{0.5em}
\textbf{Evaluation process:}
\begin{enumerate}
    \item Carefully read the natural language query.
    \item Observe the visualization results generated by two models.
    \item Based on the above criteria, choose the better visualization or select a tie if they are equally good.
    \item If neither visualization satisfies the query requirements well, please choose the relatively better one.
\end{enumerate}

Remember, your evaluation will help us improve and compare different visualization models. Thank you for your participation!
\end{tcolorbox}
\vspace{-7pt}
\caption{Instructions for human annorators in visualization comparison.}
\label{fig:evaluation_instructions}
\vspace{1em}
\end{figure*}


\section{Additional Experiment Results}
\label{additional_experiment_result}

We also conducted a comparison experiment of different methods using matplotlib or seaborn library. Figure~\ref{fig: library} demonstrates the results, indicating that our method outperforms obviously other baselines not only with matplotlib but also seaborn.

In addition, we test techniques in the Validator Agent, such as Chain-of-Thought. As is shown in Table~\ref{tab:ablation_validator}, integrating Chain-of-Thought reasoning, may affect its performance badly, likely due to the simple refining task with complex reasoning. Moreover, using the original schema to check for false schema filtering seems to be useless in this case.

\section{Evaluation Results with Detailed Metrics}
We demonstrated the main results in Table~\ref{tab:performance_comparison}, and here we reported more detailed results of other metrics in Table~\ref{tab:detailed_results}, which underscored the error rates for each stage, including \textit{Invalid}, \textit{Illegal}, and \textit{Low Readability}. 

\begin{table*}[!ht]
\centering
\footnotesize
\scalebox{0.98}{
\begin{tabular}{ll|cc|cccc|cc}
\toprule[1.5pt]
\multirow{2}{*}{Method} & \multirow{2}{*}{Dataset} & \multicolumn{2}{c|}{Invalid} & \multicolumn{4}{c|}{Illegal} & \multicolumn{2}{c}{Low Readability} \\
&  & Execution & Surface. & Decon. & Chart Type & Data & Order & Layout & Scale\&Ticks \\
\midrule
\multicolumn{10}{c}{ \textbf{\textit{GPT-4o}}}\\
\midrule
\multirow{3}{*}{CoML4Vis} & All & 1.15 & 0.00 & 0.26 & 1.75 & 14.28 & 10.36 & 32.02 & 32.55 \\
& Single & 0.67 & 0.00 & 0.43 & 1.93 & 13.54 & 10.16 & 31.08 & 32.76 \\
& Multiple & 1.87 & 0.00 & 0.00 & 1.48 & 15.39 & 10.66 & 33.43 & 32.23 \\
\cmidrule{2-10}
\multirow{3}{*}{LIDA} & All & 6.61 & 0.00 & 1.60 & 3.24 & 40.53 & 4.07 & 32.68 & 15.77 \\
& Single & 1.13 & 0.00 & 2.11 & 0.89 & 12.26 & 6.79 & 53.93 & 26.22 \\
& Multiple & 14.80 & 0.00 & 0.79 & 8.51 & 80.53 & 0.00 & 1.24 & 0.21 \\
\cmidrule{2-10}
\multirow{3}{*}{Chat2Vis} & All & 16.05 & 0.00 & 0.62 & 3.99 & 30.14 & 5.96 & 2.37 & 20.88 \\
& Single & 0.86 & 0.00 & 0.75 & 2.30 & 10.78 & 9.73 & 3.97 & 34.63 \\
& Multiple & 38.74 & 0.00 & 0.43 & 6.51 & 59.08 & 0.32 & 0.00 & 0.34 \\
\cmidrule{2-10}
\multirow{3}{*}{nvAgent} & All & 0.97 & 0.00 & 0.08 & 1.28 & 11.07 & 4.05 & 5.07 & 40.03 \\
& Single & 0.72 & 0.00 & 0.14 & 1.27 & 9.88 & 3.60 & 3.92 & 39.36 \\
& Multiple & 1.34 & 0.00 & 0.00 & 1.30 & 12.84 & 4.73 & 6.79 & 41.03 \\
\midrule
\multicolumn{10}{c}{ \textbf{\textit{GPT-4o-mini}}}\\
\midrule
\multirow{3}{*}{CoML4Vis} & All & 4.23 & 0.00 & 0.20 & 2.31 & 16.64 & 11.83 & 35.23 & 29.35 \\
& Single & 0.36 & 0.00 & 0.26 & 2.32 & 13.80 & 11.67 & 35.92 & 32.22 \\
& Multiple & 10.01 & 0.00 & 0.10 & 2.31 & 20.87 & 12.07 & 34.19 & 25.05 \\
\cmidrule{2-10}
\multirow{3}{*}{LIDA} & All & 12.50 & 0.00 & 0.40 & 4.92 & 40.02 & 5.80 & 27.87 & 17.05 \\
& Single & 9.09 & 0.00 & 0.44 & 2.53 & 12.91 & 9.68 & 45.69 & 28.32 \\
& Multiple & 17.61 & 0.00 & 0.33 & 8.51 & 80.53 & 0.00 & 1.24 & 0.21 \\
\cmidrule{2-10}
\multirow{3}{*}{Chat2Vis} & All & 15.45 & 0.17 & 0.17 & 4.21 & 31.90 & 8.20 & 2.14 & 18.97 \\
& Single & 2.14 & 0.29 & 0.41 & 2.53 & 11.99 & 9.68 & 45.69 & 28.32 \\
& Multiple & 35.78 & 0.00 & 0.00 & 6.70 & 61.66 & 0.00 & 0.92 & 0.32 \\
\cmidrule{2-10}
\multirow{3}{*}{nvAgent} & All & 5.14 & 0.00 & 0.00 & 2.40 & 16.33 & 10.61 & 41.06 & 27.00 \\
& Single & 1.97 & 0.00 & 0.14 & 2.97 & 15.21 & 7.49 & 39.30 & 32.39 \\
& Multiple & 8.15 & 0.00 & 0.00 & 2.31 & 20.87 & 12.07 & 34.19 & 25.05 \\
\midrule
\multicolumn{10}{c}{ \textbf{\textit{GPT-3.5-turbo}}}\\
\midrule
\multirow{3}{*}{CoML4Vis} & All & 9.28 & 0.00 & 0.62 & 1.91 & 15.83 & 12.86 & 25.09 & 27.73 \\ 
& Single & 6.17 & 0.00 & 0.89 & 2.50 & 14.71 & 13.20 & 26.10 & 29.93 \\ 
& Multiple & 13.92 & 0.00 & 0.21 & 1.04 & 17.51 & 12.36 & 23.57 & 24.43 \\ 
\cmidrule{2-10} 
\multirow{3}{*}{LIDA} & All & 53.43 & 0.00 & 1.27 & 3.56 & 22.33 & 0.53 & 14.90 & 6.62 \\ 
& Single & 47.32 & 0.00 & 1.91 & 2.81 & 13.03 & 0.89 & 24.43 & 11.05 \\ 
& Multiple & 62.57 & 0.00 & 0.32 & 4.68 & 36.23 & 0.00 & 0.65 & 0.00 \\ 
\cmidrule{2-10} 
\multirow{3}{*}{Chat2Vis} & All & 18.68 & 0.00 & 0.28 & 3.66 & 32.47 & 7.20 & 25.45 & 20.15 \\ 
& Single & 3.90 & 0.00 & 0.47 & 2.78 & 15.62 & 12.01 & 41.74 & 33.38 \\ 
& Multiple & 40.77 & 0.00 & 0.00 & 4.97 & 57.66 & 0.00 & 1.12 & 0.37 \\ 
\cmidrule{2-10} 
\multirow{3}{*}{nvAgent} & All & 4.66 & 0.00 & 0.08 & 3.06 & 18.24 & 5.64 & 5.25 & 35.34 \\ 
& Single & 2.98 & 0.00 & 0.14 & 2.84 & 15.08 & 5.69 & 3.62 & 37.57 \\ 
& Multiple & 7.18 & 0.00 & 0.00 & 3.38 & 22.95 & 5.56 & 7.69 & 32.02 \\
\bottomrule[1.5pt]
\end{tabular}
}
\caption{Detailed error rates (\%) for different methods.} 
\label{tab:detailed_results}
\end{table*}

\section{Case Study}
\label{example}
% To demonstrate our approach's effectiveness, we present several illustrative examples. Figure~\ref{fig:nl_vql} shows how our system translates natural language into a structured VQL representation. Figure~\ref{python code} and Figure~\ref{fig:example_chart} demonstrate the complete pipeline from query to visualization.
Figure~\ref{fig:nl_vql} shows an example of a natural language query with its corresponding VQL representation. The output Python code for visualization and the final bar chart are demonstrated in Figure~\ref{python code} and Figure~\ref{fig:example_chart}, respectively.
Furthermore, we provide a case study of \system performance on four hardness-level NL2Vis problems in VisEval in Figure \ref{hardness case}.

The easy case demonstrates accurate grouping in scatter plot relationships. The medium case shows correct handling of multi-table joins for continent-wise statistics. The hard case exhibits temporal data visualization with proper filtering. The extra hard case showcases complex operations including weekday binning and stacked visualization. These cases highlight our system's consistent performance across varying task complexities, particularly excelling in multiple table scenarios and complex aggregations.

\begin{figure*}[htbp]
\centering
\begin{tcolorbox}[enhanced,attach boxed title to top center={yshift=-3mm,yshifttext=-1mm},boxrule=0.9pt, 
  colback=gray!00,colframe=black!50,colbacktitle=gray,
  title=An Example of Natural Language Query and  Corresponding VQL,,
  boxed title style={size=small,colframe=gray} ]

\textbf{Natural Language Query:}\\
How many documents are stored? Bin the store date by weekday in a bar chart.\\
\tcbline
\textbf{Corresponding VQL:}\\
Visualize BAR \\
SELECT Date\_Stored, COUNT(Document\_ID)\\
FROM All\_Documents \\
GROUP BY Date\_Stored \\
BIN Date\_Stored BY WEEKDAY\\
\end{tcolorbox}
\caption{The natural language query case and its corresponding output VQL representation.}
\label{fig:nl_vql}
\end{figure*}

\lstset{
    basicstyle=\ttfamily\small,
    breaklines=true,
    numbers=left,
    numberstyle=\tiny,
    frame=single,
    showstringspaces=false,
    tabsize=4,
    keywordstyle=\color{blue},
    commentstyle=\color{green!60!black},
    stringstyle=\color{purple},
    breakatwhitespace=false,
    keepspaces=true
}
\begin{figure*}[htbp]
\begin{lstlisting}[
    language=Python,
    frame=single,
    numbers=left,
    numberstyle=\tiny,
    basicstyle=\tiny\ttfamily,    
    breaklines=true,
    breakatwhitespace=false,
    showstringspaces=false,
    framesep=1mm,                 % 减小框架边距
    framexleftmargin=1mm,         % 减小框架左边距
    columns=flexible,             % 灵活列宽
    keepspaces=true
]
import matplotlib.pyplot as plt 
import pandas as pd 
import os 
import duckdb

# Set data folder path
data_folder = 'E:/visEval_dataset/databases/cre_Doc_Tracking_DB'  

# Connect to database 
con = duckdb.connect(database=':memory:')  

# Read all CSV files and create view
csv_files = [f for f in os.listdir(data_folder) if f.endswith('.csv')] 
for file in csv_files:
    table_name = os.path.splitext(file)[0]
    con.execute(f"CREATE VIEW {table_name} AS SELECT * FROM read_csv_auto('{os.path.join(data_folder, file)}')")  
    
# Execute SQL query 
sql = f''' 
    SELECT Date_Stored, COUNT(Document_ID) AS count_Document_ID 
    FROM All_Documents 
    GROUP BY Date_Stored 
''' 
df = con.execute(sql).fetchdf() 
con.close()  

# Rename columns 
df.columns = ['Date_Stored','count_Document_ID'] 

# Apply binning operation
flag = True 
df['Date_Stored'] = pd.to_datetime(df['Date_Stored']) 
df['Date_Stored'] = df['Date_Stored'].dt.day_name()  

# Group by and calculate count 
if flag:
    df = df.groupby('Date_Stored').sum().reset_index() 

# Ensure all seven days of the week are included 
weekday_order = ['Monday', 'Tuesday', 'Wednesday', 'Thursday', 
                 'Friday', 'Saturday', 'Sunday'] 
df = df.set_index('Date_Stored').reindex(weekday_order, fill_value=0).reset_index()
df['Date_Stored'] = pd.Categorical(df['Date_Stored'], 
                                  categories=weekday_order, ordered=True) 
df = df.sort_values('Date_Stored')

# Create visualization 
fig, ax = plt.subplots(1, 1, figsize=(10, 4)) 
ax.spines['top'].set_visible(False) 
ax.spines['right'].set_visible(False) 
ax.bar(df['Date_Stored'], df['count_Document_ID']) 
ax.set_xlabel('Date_Stored') 
ax.set_ylabel('count_Document_ID') 
ax.set_title(f'BAR Chart of count_Document_ID by Date_Stored') 
plt.xticks(rotation=45) 
plt.tight_layout()  
plt.show()
\end{lstlisting}
\caption{An example of python code generating module within \system.}
\label{python code}
\end{figure*}


\begin{figure*}[!ht]
    \centering
    \includegraphics[width=0.98\linewidth,scale=1.0]{figure/bar_chart.pdf}
    \caption{An example of generated bar chart using \system.}
    \label{fig:example_chart}
\end{figure*}

\begin{figure*}[htbp]
\centering
\begin{tcolorbox}[enhanced,attach boxed title to top center={yshift=-3mm,yshifttext=-1mm},boxrule=0.9pt, 
  colback=gray!00,colframe=black!50,colbacktitle=gray,
  title=Examples of \textsc{nvAgent} performance on different hardness levels,
  boxed title style={size=small,colframe=gray} ]
  
\textbf{Hardness Level:} Easy \\
\begin{minipage}{0.45\linewidth}
    \textbf{Dataset}: \textit{Single}\\
    \textbf{Input Tables}: basketball\_match\\
    \textbf{Input Query}: Show the relation between acc percent and all\_games\_percent for each ACC\_Home using a grouped scatter chart.\\
\end{minipage}\hfill
\begin{minipage}{0.45\linewidth}
    \centering
    \textbf{Response}:
    \includegraphics[width=\linewidth]{figure/easy_3085.pdf} 
\end{minipage}
\tcbline

\textbf{Hardness Level:} Medium \\
\begin{minipage}{0.45\linewidth}
    \textbf{Dataset}: \textit{Multiple}\\
    \textbf{Input Tables}: car\_makers, car\_names, cars\_data, continents, countries, model\_list\\
    \textbf{Input Query}: Display a pie chart for what is the name of each continent and how many car makers are there in each one?\\
\end{minipage}\hfill
\begin{minipage}{0.55\linewidth}
    \centering
    \textbf{Response}:
    \includegraphics[width=\linewidth]{figure/medium_433.pdf} 
\end{minipage}
\tcbline

\textbf{Hardness Level:} Hard \\[1em]
\begin{minipage}{0.45\linewidth}
    \textbf{Dataset}: \textit{Multiple}\\
    \textbf{Input Tables}: advisor, classroom, course, department, instructor, prereq, section, student, takes, teaches, time\_slot\\
    \textbf{Input Query}: Find the number of courses offered by Psychology department in each year with a line chart.\\
\end{minipage}\hfill
\begin{minipage}{0.45\linewidth}
    \centering
    \textbf{Response}:
    \includegraphics[width=\linewidth]{figure/hard_611.pdf} 
\end{minipage}
\tcbline

\textbf{Hardness Level:} Extra Hard \\[1em]
\begin{minipage}{0.45\linewidth}
    \textbf{Dataset}: \textit{Multiple}\\
    \textbf{Input Tables}: Accounts, Documents, Documents\_with\_Expenses, Projects, Ref- \_Budget\_Codes, Ref\_Document\_Types, Statements\\
    \textbf{Input Query}: How many documents are created in each day? Bin the document date by weekday and group by document type description with a stacked bar chart, I want to sort Y in desc order.\\
\end{minipage}\hfill
\begin{minipage}{0.45\linewidth}
    \centering
    \textbf{Response}:
    \includegraphics[width=\linewidth]{figure/extra_851.pdf} 
\end{minipage}

\end{tcolorbox}
    \caption{Examples of \textsc{nvAgent}'s performance on different hardness levels in VisEval (easy, medium, hard, and extra hard.}
    \label{hardness case}
\end{figure*}


\clearpage
\onecolumn
\section{Prompts Details}
\label{prompt_details}
We provide detailed prompt design of our \system as follows.



\begin{promptbox}[Prompt template for Processor Agent]
You are an experienced and professional database administrator. Given a database schema and a user query, your task is to analyze the query, filter the relevant schema, generate an optimized representation, and classify the query difficulty. \\
\\
Now you can think step by step, following these instructions below. \\
\textbf{[Instructions]} \\
1. Schema Filtering: \\
\text{\ \ \ \ }- Identify the tables and columns that are relevant to the user query.\\
\text{\ \ \ \ }- Only exclude columns that are completely irrelevant.\\
\text{\ \ \ \ }- The output should be \{\{tables: [columns]\}\}.\\
\text{\ \ \ \ }- Keep the columns needed to be primary keys and foreign keys in the filtered schema.\\
\text{\ \ \ \ }- Keep the columns that seem to be similar with other columns of another table.\\
\\
2. New Schema Generation:\\
\text{\ \ \ \ }- Generate a new schema of the filtered schema, based on the given database schema and your filtered schema.\\
\\
3. Augmented Explanation:\\
\text{\ \ \ \ }- Provide a concise summary of the filtered schema to give additional knowledge.\\
\text{\ \ \ \ }- Include the number of tables, total columns, and any notable relationships or patterns.\\
\\
4. Classification:\\
For the database new schema, classify it as SINGLE or MULTIPLE based on the tables number.\\
\text{\ \ \ \ }- if tables number >= 2: predict MULTIPLE\\
\text{\ \ \ \ }- elif only one table: predict SINGLE\\
\\
==============================\\
Here is a typical example:\\
\textbf{[Database Schema]}\\
\textbf{[DB\_ID]} dorm\_1\\
\textbf{[Schema]}\\
\# Table: Student\\
\text{[}\\
  \text{\ \ \ \ }(stuid, And This is a id type column),\\
  \text{\ \ \ \ }(lname, Value examples: [`Smith', `Pang', `Lee', `Adams', `Nelson', `Wilson'].),\\
  \text{\ \ \ \ }(fname, Value examples: [`Eric', `Lisa', `David', `Sarah', `Paul', `Michael'].),\\
  \text{\ \ \ \ }(age, Value examples: [18, 20, 17, 19, 21, 22].),\\
  \text{\ \ \ \ }(sex, Value examples: [`M', `F'].),\\
  \text{\ \ \ \ }(major, Value examples: [600, 520, 550, 50, 540, 100].),\\
  \text{\ \ \ \ }(advisor, And this is a number type column),\\
  \text{\ \ \ \ }(city code, Value examples: [`PIT', `BAL', `NYC', `WAS', `HKG', `PHL'].)\\
\text{]}\\
% \end{promptbox}
% \end{figure*}
% \begin{figure*}[!h]
% \begin{promptbox}[Prompt template for Processor Agent]
\# Table: Dorm\\
\text{[}\\
  \text{\ \ \ \ }(dormid, And This is a id type column),\\
  \text{\ \ \ \ }(dorm name, Value examples: [`Anonymous Donor Hall', `Bud Jones Hall', `Dorm-plex 2000', `Fawlty Towers', `Grad Student Asylum', `Smith Hall'].),\\
  \text{\ \ \ \ }(student capacity, Value examples: [40, 85, 116, 128, 256, 355].),
  (gender, Value examples: [`X', `F', `M'].)\\
\text{]}\\
\# Table: Dorm\_amenity\\
\text{[}\\
  \text{\ \ \ \ }(amenid, And This is a id type column),\\
  \text{\ \ \ \ }(amenity name, Value examples: [`4 Walls', `Air Conditioning', `Allows Pets', `Carpeted Rooms', `Ethernet Ports', `Heat'].)\\
\text{]}\\
\# Table: Has\_amenity\\
\text{[}\\
  \text{\ \ \ \ }(dormid, And This is a id type column),\\
  \text{\ \ \ \ }(amenid, And This is a id type column)\\
\text{]}\\
\# Table: Lives\_in\\
\text{[}\\
  \text{\ \ \ \ }(stuid, And This is a id type column),\\
  \text{\ \ \ \ }(dormid, And This is a id type column),\\
  \text{\ \ \ \ }(room number, And this is a number type column)\\
\text{]}\\
\\
\textbf{[Query]}\\
Find the first name of students who are living in the Smith Hall, and count them by a pie chart\\
\\
Now we can think step by step\\
\textbf{[Filtered Schema]}\\
\{\\
  "Student": ["stuid", "fname"],\\
  "Dorm": ["dormid", "dorm name"],\\
  "Lives\_in": ["stuid", "dormid"]\\
\}\\
\\
\textbf{[New Schema]}\\
\# Table: Student\\
\text{[}\\
  (stuid, And This is a id type column),\\
  (fname, Value examples: [`Eric', `Lisa', `David', `Sarah', `Paul', `Michael'].),\\
\text{]}\\
% \end{promptbox}
% \end{figure*}
% \begin{figure*}[!h]
% \begin{promptbox}[Prompt template for Processor Agent]
\# Table: Dorm
\text{[}\\
  (dormid, And This is a id type column),\\
  (dorm name, Value examples: [`Anonymous Donor Hall', `Bud Jones Hall', `Dorm-plex 2000', `Fawlty Towers', `Grad Student Asylum', `Smith Hall'].),\\
\text{]}\\
\# Table: Lives\_in\\
\text{[}\\
  (stuid, And This is a id type column),\\
  (dormid, And This is a id type column),\\
\text{]}\\
\textbf{[Augmented Explanation]}\\
The filtered schema consists of 3 tables (Student, Dorm, and Lives\_in) with a total of 6 relevant columns. There is a many-to-one relationship between Student and Dorm through the Lives\_in junction table. The query involves joining these three tables to find students living in a specific dorm (Smith Hall).\\
\\
Key points:\\
1. The Lives\_in table acts as a bridge between Student and Dorm, allowing for the association of students with their dorms.\\
2. The `dorm name' column in the Dorm table is crucial for filtering the specific dorm (Smith Hall).\\
3. The `fname' column from the Student table is required for the final output.\\
\\
\textbf{[Classification]}\\
MULTIPLE\\
\\
==============================\\
Here is a new question:\\
\\
\textbf{[DB\_ID]} \{db\_id\}\\
\textbf{[Database Schema]}\\
\{db\_schema\}\\
\\
\textbf{[Query]}\\
\{query\}\\
\\
Now give your answer following this format strictly without other explanation:\\
\\
\textbf{[Filtered Schema]}\\
\\
\textbf{[New Schema]}\\
\\
\textbf{[Augmented Explanation]}\\
\\
\textbf{[Classification]}\\
\\
\end{promptbox}
% \end{figure*}

% \subsection{Composer Agent Prompt}
% \label{composer_prompt}
% \begin{figure*}[!h]
\begin{promptbox}[Prompt template for multiple classification]
Given a [Database schema] with [Augmented Explanation] and a [Question], generate a valid VQL (Visualization Query Language) sentence. VQL is similar to SQL but includes visualization components. \\
\\
Now you can think step by step, following these instructions below. \\
\textbf{[Background]} \\
VQL Structure:\\
Visualize [TYPE] SELECT [COLUMNS] FROM [TABLES] [JOIN] [WHERE] [GROUP BY] [ORDER BY] [BIN BY]\\
\\
You can consider a VQL sentence as "VIS TYPE + SQL + BINNING"\\
You must consider which part in the sketch is necessary, which is unnecessary, and construct a specific sketch for the natural language query.\\
\\
Key Components:\\
1. Visualization Type: bar, pie, line, scatter, stacked bar, grouped line, grouped scatter\\
2. SQL Components: SELECT, FROM, JOIN, WHERE, GROUP BY, ORDER BY\\
3. Binning: BIN [COLUMN] BY [INTERVAL], [INTERVAL]: [YEAR, MONTH, DAY, WEEKDAY]\\
\\
When generating VQL, we should always consider special rules and constraints:\\
\textbf{[Special Rules]} \\
a. For simple visualizations:\\
    \text{\ \ \ \ }- SELECT exactly TWO columns, X-axis and Y-axis(usually aggregate function)\\
b. For complex visualizations (STACKED BAR, GROUPED LINE, GROUPED SCATTER):\\
    \text{\ \ \ \ }- SELECT exactly THREE columns in this order!!!:\\
        \text{\ \ \ \ }\text{\ \ \ \ }1. X-axis\\
        \text{\ \ \ \ }\text{\ \ \ \ }2. Y-axis (aggregate function)\\
        \text{\ \ \ \ }\text{\ \ \ \ }3. Grouping column\\
c. When "COLORED BY" is mentioned in the question:\\
    \text{\ \ \ \ }- Use complex visualization type(STACKED BAR for bar charts, GROUPED LINE for line charts, GROUPED SCATTER for scatter charts)\\
    \text{\ \ \ \ }- Make the "COLORED BY" column the third SELECT column\\
    \text{\ \ \ \ }- Do NOT include "COLORED BY" in the final VQL\\     
d. Aggregate Functions:\\
    \text{\ \ \ \ }- Use COUNT for counting occurrences\\
    \text{\ \ \ \ }- Use SUM only for numeric columns\\
    \text{\ \ \ \ }- When in doubt, prefer COUNT over SUM\\
e. Time based questions:\\
    \text{\ \ \ \ }- Always use BIN BY clause at the end of VQL sentence\\
    \text{\ \ \ \ }- When you meet the questions including "year", "month", "day", "weekday"\\
    \text{\ \ \ \ }- Avoid using window function, just use BIN BY to deal with time base queries\\
% \end{promptbox}
% \end{figure*}
% \begin{figure*}[!h]
% \begin{promptbox}[Prompt template for multiple classification]
\textbf{[Constraints]} \\
- In SELECT <column>, make sure there are at least two selected!!!\\
- In FROM <table> or JOIN <table>, do not include unnecessary table\\
- Use only table names and column names from the given database schema\\
- Enclose string literals in single quotes\\
- If [Value examples] of <column> has `None' or None, use JOIN <table> or WHERE <column> is NOT NULL is better\\
- Ensure GROUP BY precedes ORDER BY for distinct values\\
- NEVER use window functions in SQL\\
\\
Now we could think step by step:\\
1. First choose visualize type and binning, then construct a specific sketch for the natural language query\\
2. Second generate SQL components following the sketch.\\
3. Third add Visualize type and BINNING into the SQL components to generate final VQL\\
\\
==============================\\
Here is a typical example:\\
\textbf{[Database Schema]}\\
\# Table: Orders, (orders)\\
\text{[}\\
  \text{\ \ \ \ }(order\_id, order id, And this is a id type column),\\
  \text{\ \ \ \ }(customer\_id, customer id, And this is a id type column),\\
  \text{\ \ \ \ }(order\_date, order date, Value examples: [`2023-01-15', `2023-02-20', `2023-03-10'].),\\
  \text{\ \ \ \ }(total\_amount, total amount, Value examples: [100.00, 200.00, 300.00, 400.00, 500.00].)\\
\text{]}\\
\# Table: Customers, (customers)\\
\text{[}\\
  \text{\ \ \ \ }(customer\_id, customer id, And this is a id type column),\\
  \text{\ \ \ \ }(customer\_name, customer name, Value examples: [`John', `Emma', `Michael', `Sophia', `William'].),\\
  \text{\ \ \ \ }(customer\_type, customer type, Value examples: [`Regular', `VIP', `New'].)\\
\text{]}\\
\textbf{[Augmented Explanation]}\\
The filtered schema consists of 2 tables (Orders and Customers) with a total of 7 relevant columns. There is a one-to-many relationship between Customers and Orders through the customer\_id foreign key.\\
\\
Key points:\\
1. The Orders table contains information about individual orders, including the order date and total amount.\\
2. The Customers table contains customer information, including their name and type (Regular, VIP, or New).\\
3. The customer\_id column links the two tables, allowing us to associate orders with specific customers.\\
% \end{promptbox}
% \end{figure*}
% \begin{figure*}[!h]
% \begin{promptbox}[Prompt template for multiple classification]
4. The order\_date column in the Orders table will be used for monthly grouping and binning.\\
5. The total\_amount column in the Orders table needs to be summed for each group.\\
6. The customer\_type column in the Customers table will be used for further grouping and as the third dimension in the stacked bar chart.\\
\\

The query involves joining these two tables to analyze order amounts by customer type and month, which requires aggregation and time-based binning.\\
\\
\textbf{[Question]}\\
Show the total order amount for each customer type by month in a stacked bar chart.\\
\\
Decompose the task into sub tasks, considering [Background] [Special Rules] [Constraints], and generate the VQL after thinking step by step:\\
\\
\textbf{Sub task 1:} First choose visualize type and binning, then construct a specific sketch for the natural language query\\
Visualize type: STACKED BAR, BINNING: True\\
VQL Sketch:\\
Visualize STACKED BAR SELECT \_ , \_ , \_ FROM \_ JOIN \_ ON \_ GROUP BY \_ BIN \_ BY MONTH\\
\\
\textbf{Sub task 2:} Second generate SQL components following the sketch.\\
Let's think step by step:\\
1. We need to select 3 columns for STACKED BAR chart, order\_date as X-axis, SUM(total\_amout) as Y-axis, customer\_type as group column.\\
2. We need to join the Orders and Customers tables.\\
3. We need to group by customer type.\\
4. We do not need to use any window function for MONTH.\\
\\
\text{sql}\\
```sql\\
SELECT O.order\_date, SUM(O.total\_amount), C.customer\_type\\
FROM Orders AS O\\
JOIN Customers AS C ON O.customer\_id = C.customer\_id\\
GROUP BY C.customer\_type\\
```\\
\\
\textbf{Sub task 3:} Third add Visualize type and BINNING into the SQL components to generate final VQL\\
\textbf{Final VQL:}\\
Visualize STACKED BAR SELECT O.order\_date, SUM(O.total\_amount), C.customer\_type FROM Orders O JOIN Customers C ON O.customer\_id = C.customer\_id GROUP BY C.customer\_type BIN O.order\_date BY MONTH\\
\\
% \end{promptbox}
% \end{figure*}
% \begin{figure*}[!h]
% \begin{promptbox}[Prompt template for multiple classification]
==============================\\
Here is a new question:\\
\\
\textbf{[Database Schema]}\\
\{desc\_str\}\\
\\
\textbf{[Augmented Explanation]}\\
\{augmented\_explanation\}\\
\\
\textbf{[Query]}\\
\{query\}\\
\\
Now, please generate a VQL sentence for the database schema and question after thinking step by step.\\

\end{promptbox}
% \end{figure*}


% \begin{figure*}[!h]
\begin{promptbox}[Prompt template for single classification]
Given a [Database schema] with [Augmented Explanation] and a [Question], generate a valid VQL (Visualization Query Language) sentence. VQL is similar to SQL but includes visualization components. \\
\\
Now you can think step by step, following these instructions below. \\
\textbf{[Background]} \\
VQL Structure:\\
Visualize [TYPE] SELECT [COLUMNS] FROM [TABLES] [JOIN] [WHERE] [GROUP BY] [ORDER BY] [BIN BY]\\
\\
You can consider a VQL sentence as "VIS TYPE + SQL + BINNING"\\
You must consider which part in the sketch is necessary, which is unnecessary, and construct a specific sketch for the natural language query.\\
\\
Key Components:\\
1. Visualization Type: bar, pie, line, scatter, stacked bar, grouped line, grouped scatter\\
2. SQL Components: SELECT, FROM, JOIN, WHERE, GROUP BY, ORDER BY\\
3. Binning: BIN [COLUMN] BY [INTERVAL], [INTERVAL]: [YEAR, MONTH, DAY, WEEKDAY]\\
\\
When generating VQL, we should always consider special rules and constraints:\\
\textbf{[Special Rules]} \\
a. For simple visualizations:\\
    \text{\ \ \ \ }- SELECT exactly TWO columns, X-axis and Y-axis(usually aggregate function)\\
b. For complex visualizations (STACKED BAR, GROUPED LINE, GROUPED SCATTER):\\
    \text{\ \ \ \ }- SELECT exactly THREE columns in this order!!!:\\
        \text{\ \ \ \ }\text{\ \ \ \ }1. X-axis\\
        \text{\ \ \ \ }\text{\ \ \ \ }2. Y-axis (aggregate function)\\
        \text{\ \ \ \ }\text{\ \ \ \ }3. Grouping column\\
c. When "COLORED BY" is mentioned in the question:\\
    \text{\ \ \ \ }- Use complex visualization type(STACKED BAR for bar charts, GROUPED LINE for line charts, GROUPED SCATTER for scatter charts)\\
    \text{\ \ \ \ }- Make the "COLORED BY" column the third SELECT column\\
    \text{\ \ \ \ }- Do NOT include "COLORED BY" in the final VQL\\     
d. Aggregate Functions:\\
    \text{\ \ \ \ }- Use COUNT for counting occurrences\\
    \text{\ \ \ \ }- Use SUM only for numeric columns\\
    \text{\ \ \ \ }- When in doubt, prefer COUNT over SUM\\
e. Time based questions:\\
    \text{\ \ \ \ }- Always use BIN BY clause at the end of VQL sentence\\
    \text{\ \ \ \ }- When you meet the questions including "year", "month", "day", "weekday"\\
    \text{\ \ \ \ }- Avoid using window function, just use BIN BY to deal with time base queries\\
% \end{promptbox}
% \end{figure*}
% \begin{figure*}[!h]
% \begin{promptbox}[Prompt template for single classification]
\textbf{[Constraints]} \\
- In SELECT <column>, make sure there are at least two selected!!!\\
- In FROM <table> or JOIN <table>, do not include unnecessary table\\
- Use only table names and column names from the given database schema\\
- Enclose string literals in single quotes\\
- If [Value examples] of <column> has `None' or None, use JOIN <table> or WHERE <column> is NOT NULL is better\\
- Ensure GROUP BY precedes ORDER BY for distinct values\\
- NEVER use window functions in SQL\\
\\
Now we could think step by step:\\
1. First choose visualize type and binning, then construct a specific sketch for the natural language query\\
2. Second generate SQL components following the sketch.\\
3. Third add Visualize type and BINNING into the SQL components to generate final VQL\\
\\
==============================\\
Here is a typical example:\\
\textbf{[Database Schema]}\\
\# Table: course, (course)\\
\text{[}\\
  \text{\ \ \ \ }(course\_id, course id, Value examples: [101, 696, 656, 659]. And this is an id type column),\\
  \text{\ \ \ \ }(title, title, Value examples: [`Geology', `Differential Geometry', `Compiler Design', `International Trade', `Composition and Literature', `Environmental Law'].),\\
  \text{\ \ \ \ }(dept\_name, dept name, Value examples: [`Cybernetics', `Finance', `Psychology', `Accounting', `Mech. Eng.', `Physics'].),\\
  \text{\ \ \ \ }(credits, credits, Value examples: [3, 4].)\\
\text{]}\\
\# Table: section, (section)\\
\text{[}\\
  \text{\ \ \ \ }(course\_id, course id, Value examples: [362, 105, 960, 468]. And this is an id type column),\\
  \text{\ \ \ \ }(sec\_id, sec id, Value examples: [1, 2, 3]. And this is an id type column),\\
  \text{\ \ \ \ }(semester, semester, Value examples: [`Fall', `Spring'].),\\
  \text{\ \ \ \ }(year, year, Value examples: [2002, 2006, 2003, 2007, 2010, 2008].),\\
  \text{\ \ \ \ }(building, building, Value examples: [`Saucon', `Taylor', `Lamberton', `Power', `Fairchild', `Main'].),\\
  \text{\ \ \ \ }(room\_number, room number, Value examples: [180, 183, 134, 143].),\\
  \text{\ \ \ \ }(time\_slot\_id, time slot id, Value examples: [`D', `J', `M', `C', `E', `F']. And this is an id type column)\\
\text{]}\\
\textbf{[Augmented Explanation]}\\
The filtered schema consists of 2 tables (course and section) with a total of 11 relevant columns. There is a one-to-many relationship between course and section through the course\_id foreign key.\\
\\
% \end{promptbox}
% \end{figure*}
% \begin{figure*}[!h]
% \begin{promptbox}[Prompt template for single classification]
Key points:\\
1. The course table contains information about individual courses, including the course title, department, and credits.\\
2. The section table contains information about specific sections of courses, including the semester, year, building, room number, and time slot.\\
3. The course\_id column links the two tables, allowing us to associate sections with specific courses.\\
4. The dept\_name column in the course table will be used to filter for Psychology department courses.\\
5. The year column in the section table will be used for yearly grouping and binning.\\
6. We need to count the number of courses offered each year, which requires aggregation and time-based binning.\\
\\
The query involves joining these two tables to analyze the number of courses offered by the Psychology department each year, which requires aggregation and time-based binning.\\
\\
\textbf{[Question]}\\
Find the number of courses offered by Psychology department in each year with a line chart.\\
\\
Decompose the task into sub tasks, considering [Background] [Special Rules] [Constraints], and generate the VQL after thinking step by step:\\
\\
\textbf{Sub task 1:} First choose visualize type and binning, then construct a specific sketch for the natural language query\\
Visualize type: LINE, BINNING: True\\
VQL Sketch:\\
Visualize LINE SELECT \_ , \_ FROM \_ JOIN \_ ON \_ WHERE \_ BIN \_ BY YEAR\\
\\
\textbf{Sub task 2:} Second generate SQL components following the sketch.\\
Let's think step by step:\\
1. We need to select 2 columns for LINE chart, year as X-axis, COUNT(year) as Y-axis.\\
2. We need to join the course and section tables to get the number of courses offered by the Psychology department in each year.\\
3. We need to filter the courses by the Psychology department.\\
4. We do not need to use any window function for YEAR.\\
\\
\text{sql}\\
```sql\\
SELECT S.year, COUNT(S.year)\\
FROM course AS C\\
JOIN section AS S ON C.course\_id = S.course\_id\\
WHERE C.dept\_name = `Psychology'\\
```\\
\\
% \end{promptbox}
% \end{figure*}
% \begin{figure*}[!h]
% \begin{promptbox}[Prompt template for single classification]
\textbf{Sub task 3:} Third add Visualize type and BINNING into the SQL components to generate final VQL\\
\textbf{Final VQL:}\\
Visualize LINE SELECT S.year, COUNT(S.year) FROM course C JOIN section S ON C.course\_id = S.course\_id WHERE C.dept\_name = `Psychology' BIN S.year BY YEAR\\
\\
==============================\\
Here is a new question:\\
\\
\textbf{[Database Schema]}\\
\{desc\_str\}\\
\\
\textbf{[Augmented Explanation]}\\
\{augmented\_explanation\}\\
\\
\textbf{[Query]}\\
\{query\}\\
\\
Now, please generate a VQL sentence for the database schema and question after thinking step by step.\\

\end{promptbox}
% \end{figure*}

% \subsection{Validator Agent Prompt}
% \label{validator_prompt}
% \begin{figure*}
\begin{promptbox}[Prompt template for Validator Agent]
As an AI assistant specializing in data visualization and VQL (Visualization Query Language), your task is to refine a VQL query that has resulted in an error. Please approach this task systematically, thinking step by step.\\
\textbf{[Background]}\\
VQL Structure:\\
Visualize [TYPE] SELECT [COLUMNS] FROM [TABLES] [JOIN] [WHERE] [GROUP BY] [ORDER BY] [BIN BY]\\
\\
You can consider a VQL sentence as "VIS TYPE + SQL + BINNING"\\
\\
Key Components:\\
1. Visualization Type: bar, pie, line, scatter, stacked bar, grouped line, grouped scatter\\
2. SQL Components: SELECT, FROM, JOIN, WHERE, GROUP BY, ORDER BY\\
3. Binning: BIN [COLUMN] BY [INTERVAL], [INTERVAL]: [YEAR, MONTH, DAY, WEEKDAY]\\
\\
When refining VQL, we should always consider special rules and constraints:\\
\textbf{[Special Rules]} \\
a. For simple visualizations:\\
    \text{\ \ \ \ }- SELECT exactly TWO columns, X-axis and Y-axis(usually aggregate function)\\
b. For complex visualizations (STACKED BAR, GROUPED LINE, GROUPED SCATTER):\\
    \text{\ \ \ \ }- SELECT exactly THREE columns in this order!!!:\\
        \text{\ \ \ \ }\text{\ \ \ \ }1. X-axis\\
        \text{\ \ \ \ }\text{\ \ \ \ }2. Y-axis (aggregate function)\\
        \text{\ \ \ \ }\text{\ \ \ \ }3. Grouping column\\
c. When "COLORED BY" is mentioned in the question:\\
    \text{\ \ \ \ }- Use complex visualization type(STACKED BAR for bar charts, GROUPED LINE for line charts, GROUPED SCATTER for scatter charts)\\
    \text{\ \ \ \ }- Make the "COLORED BY" column the third SELECT column\\
    \text{\ \ \ \ }- Do NOT include "COLORED BY" in the final VQL\\     
d. Aggregate Functions:\\
    \text{\ \ \ \ }- Use COUNT for counting occurrences\\
    \text{\ \ \ \ }- Use SUM only for numeric columns\\
    \text{\ \ \ \ }- When in doubt, prefer COUNT over SUM
% \end{promptbox}
% \end{figure*}

% \begin{figure*}
% \begin{promptbox}[Prompt template for Validator Agent]
e. Time based questions:\\
    \text{\ \ \ \ }- Always use BIN BY clause at the end of VQL sentence\\
    \text{\ \ \ \ }- When you meet the questions including "year", "month", "day", "weekday"\\
    \text{\ \ \ \ }- Avoid using time function, just use BIN BY to deal with time base queries\\
\\
\textbf{[Constraints]} \\
- In FROM <table> or JOIN <table>, do not include unnecessary table\\
- Use only table names and column names from the given database schema\\
- Enclose string literals in single quotes\\
- If [Value examples] of <column> has `None' or None, use JOIN <table> or WHERE <column> is NOT NULL is better\\
- ENSURE GROUP BY clause cannot contain aggregates\\
- NEVER use date functions in SQL\\
\\
\textbf{[Query]} \\
\{query\}\\
\\
\textbf{[Database info]} \\
\{db\_info\}\\
\\
\textbf{[Current VQL]} \\
\{vql\}\\
\\
\textbf{[Error]} \\
\{error\}\\
\\
Now, please analyze and refine the VQL, please provide:\\
\\
\textbf{[Explanation]}\\
\text{[}Provide a detailed explanation of your analysis process, the issues identified, and the changes made. Reference specific steps where relevant.\text{]}\\
\\
\textbf{[Corrected VQL]}\\
\text{[}Present your corrected VQL here. Ensure it's on a single line without any line breaks.\text{]}\\
\\
Remember:\\
- The SQL components must be parseable by DuckDB.\\
- Do not change rows when you generate the VQL.\\
- Always verify your answer carefully before submitting.\\
\end{promptbox}
% \end{figure*}

\end{document}
