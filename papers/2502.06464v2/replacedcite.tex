\section{Related Work}
SR was introduced by Gale and Shapley____ who showed that SR instances need not admit stable matchings. Knuth____ explicitly posed the question of whether there is an efficient algorithm to find a stable matching or report that none exists. Irving____ devised an algorithm that finds a stable matching or correctly determines that none exists in $O(n^2)$ time (assuming each preference list entry can be accessed in $O(1)$ time). Gusfield and Irving's influential book____ gives a comprehensive overview of early algorithmic work on SR, including structural aspects of the set of stable matchings for an SR instance. Additionally,____ listed two questions related to efficiently deciding SR solvability. The first question asked if it is possible to generate a ``succinct certificate'' for non-solvability of an SR instance. This question was positively answered by Tan____ who demonstrated that a ``stable partition'' (of size $O(n)$) can be computed in $O(n^2)$ time. Given the partition, solvability of an SR instance can be verified in $O(n)$ time. Gusfield and Irving's second question asked if it is possible to decide SR solvability in $o(n^2)$ time directly. Our lower bound gives a negative answer to this question. Given Tan's result, our lower bound also implies that finding a stable partition requires $\Omega(n^2)$ time. Several previous works____ proved $\Omega(n^2)$ lower bounds for \emph{finding} stable marriages in SM (and \emph{a fortiori} for finding stable matchings in SR), but none of these results imply lower bounds for the decision problem of SR solvability. We refer the reader to the text of Manlove____ for an account of more recent developments related to SR.

Our main lower bound result follows from a reduction from the two-party communication complexity of the disjointness function (see Section~\ref{sec:communication} for a formal definition). The two-party communication complexity model was introduced by Yao in____. The (randomized) two-party communication complexity of the disjointness function was first characterized by Kalyanasundaram and Schintger____, and subsequently a simpler argument was found by Razborov____. 

Most closely related to the present paper is the work of Gonczarowski et al.____. In____, the authors apply the communication complexity of set disjointness to obtain lower bounds for finding and verifying (bipartite) stable marriages. Indeed, our techniques in this paper closely follow that of____. Reductions from communication complexity to obtain (sublinear time) lower bounds were employed by____ for property testing. The technique was further codified in the work of Eden and Rosenbaum____ who proved a variety of lower bounds for sublinear time algorithms. Our description of our lower bound in terms of an ``embedding'' of set disjointness follows the general approach and vocabulary of____.