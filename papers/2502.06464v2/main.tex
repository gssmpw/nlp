
\documentclass[a4paper,11pt]{article}

% Essential packages
\usepackage[utf8]{inputenc}  % UTF-8 encoding
\usepackage[T1]{fontenc}     % Proper font encoding
\usepackage{lmodern}         % Improved font rendering
\usepackage{geometry}        % Page layout
\geometry{margin=1in}       % 1-inch margins

% Math and symbols
\usepackage{amsmath, amssymb, amsthm} % Advanced math support

% Graphics and colors
\usepackage{graphicx}        % Include images
\usepackage{xcolor}          % Define colors

% Hyperlinks
\usepackage{hyperref}        % Clickable links
\hypersetup{
    colorlinks=true,
    linkcolor=blue,
    urlcolor=blue,
    citecolor=red
}

% Algorithms!
\usepackage{algorithm}
\usepackage{algpseudocode}

% Pictures & figures
\usepackage{tikz}
\usepackage{subcaption}

% Misc
\usepackage{multicol}
\usepackage{booktabs}


%
\setlength\unitlength{1mm}
\newcommand{\twodots}{\mathinner {\ldotp \ldotp}}
% bb font symbols
\newcommand{\Rho}{\mathrm{P}}
\newcommand{\Tau}{\mathrm{T}}

\newfont{\bbb}{msbm10 scaled 700}
\newcommand{\CCC}{\mbox{\bbb C}}

\newfont{\bb}{msbm10 scaled 1100}
\newcommand{\CC}{\mbox{\bb C}}
\newcommand{\PP}{\mbox{\bb P}}
\newcommand{\RR}{\mbox{\bb R}}
\newcommand{\QQ}{\mbox{\bb Q}}
\newcommand{\ZZ}{\mbox{\bb Z}}
\newcommand{\FF}{\mbox{\bb F}}
\newcommand{\GG}{\mbox{\bb G}}
\newcommand{\EE}{\mbox{\bb E}}
\newcommand{\NN}{\mbox{\bb N}}
\newcommand{\KK}{\mbox{\bb K}}
\newcommand{\HH}{\mbox{\bb H}}
\newcommand{\SSS}{\mbox{\bb S}}
\newcommand{\UU}{\mbox{\bb U}}
\newcommand{\VV}{\mbox{\bb V}}


\newcommand{\yy}{\mathbbm{y}}
\newcommand{\xx}{\mathbbm{x}}
\newcommand{\zz}{\mathbbm{z}}
\newcommand{\sss}{\mathbbm{s}}
\newcommand{\rr}{\mathbbm{r}}
\newcommand{\pp}{\mathbbm{p}}
\newcommand{\qq}{\mathbbm{q}}
\newcommand{\ww}{\mathbbm{w}}
\newcommand{\hh}{\mathbbm{h}}
\newcommand{\vvv}{\mathbbm{v}}

% Vectors

\newcommand{\av}{{\bf a}}
\newcommand{\bv}{{\bf b}}
\newcommand{\cv}{{\bf c}}
\newcommand{\dv}{{\bf d}}
\newcommand{\ev}{{\bf e}}
\newcommand{\fv}{{\bf f}}
\newcommand{\gv}{{\bf g}}
\newcommand{\hv}{{\bf h}}
\newcommand{\iv}{{\bf i}}
\newcommand{\jv}{{\bf j}}
\newcommand{\kv}{{\bf k}}
\newcommand{\lv}{{\bf l}}
\newcommand{\mv}{{\bf m}}
\newcommand{\nv}{{\bf n}}
\newcommand{\ov}{{\bf o}}
\newcommand{\pv}{{\bf p}}
\newcommand{\qv}{{\bf q}}
\newcommand{\rv}{{\bf r}}
\newcommand{\sv}{{\bf s}}
\newcommand{\tv}{{\bf t}}
\newcommand{\uv}{{\bf u}}
\newcommand{\wv}{{\bf w}}
\newcommand{\vv}{{\bf v}}
\newcommand{\xv}{{\bf x}}
\newcommand{\yv}{{\bf y}}
\newcommand{\zv}{{\bf z}}
\newcommand{\zerov}{{\bf 0}}
\newcommand{\onev}{{\bf 1}}

% Matrices

\newcommand{\Am}{{\bf A}}
\newcommand{\Bm}{{\bf B}}
\newcommand{\Cm}{{\bf C}}
\newcommand{\Dm}{{\bf D}}
\newcommand{\Em}{{\bf E}}
\newcommand{\Fm}{{\bf F}}
\newcommand{\Gm}{{\bf G}}
\newcommand{\Hm}{{\bf H}}
\newcommand{\Id}{{\bf I}}
\newcommand{\Jm}{{\bf J}}
\newcommand{\Km}{{\bf K}}
\newcommand{\Lm}{{\bf L}}
\newcommand{\Mm}{{\bf M}}
\newcommand{\Nm}{{\bf N}}
\newcommand{\Om}{{\bf O}}
\newcommand{\Pm}{{\bf P}}
\newcommand{\Qm}{{\bf Q}}
\newcommand{\Rm}{{\bf R}}
\newcommand{\Sm}{{\bf S}}
\newcommand{\Tm}{{\bf T}}
\newcommand{\Um}{{\bf U}}
\newcommand{\Wm}{{\bf W}}
\newcommand{\Vm}{{\bf V}}
\newcommand{\Xm}{{\bf X}}
\newcommand{\Ym}{{\bf Y}}
\newcommand{\Zm}{{\bf Z}}

% Calligraphic

\newcommand{\Ac}{{\cal A}}
\newcommand{\Bc}{{\cal B}}
\newcommand{\Cc}{{\cal C}}
\newcommand{\Dc}{{\cal D}}
\newcommand{\Ec}{{\cal E}}
\newcommand{\Fc}{{\cal F}}
\newcommand{\Gc}{{\cal G}}
\newcommand{\Hc}{{\cal H}}
\newcommand{\Ic}{{\cal I}}
\newcommand{\Jc}{{\cal J}}
\newcommand{\Kc}{{\cal K}}
\newcommand{\Lc}{{\cal L}}
\newcommand{\Mc}{{\cal M}}
\newcommand{\Nc}{{\cal N}}
\newcommand{\nc}{{\cal n}}
\newcommand{\Oc}{{\cal O}}
\newcommand{\Pc}{{\cal P}}
\newcommand{\Qc}{{\cal Q}}
\newcommand{\Rc}{{\cal R}}
\newcommand{\Sc}{{\cal S}}
\newcommand{\Tc}{{\cal T}}
\newcommand{\Uc}{{\cal U}}
\newcommand{\Wc}{{\cal W}}
\newcommand{\Vc}{{\cal V}}
\newcommand{\Xc}{{\cal X}}
\newcommand{\Yc}{{\cal Y}}
\newcommand{\Zc}{{\cal Z}}

% Bold greek letters

\newcommand{\alphav}{\hbox{\boldmath$\alpha$}}
\newcommand{\betav}{\hbox{\boldmath$\beta$}}
\newcommand{\gammav}{\hbox{\boldmath$\gamma$}}
\newcommand{\deltav}{\hbox{\boldmath$\delta$}}
\newcommand{\etav}{\hbox{\boldmath$\eta$}}
\newcommand{\lambdav}{\hbox{\boldmath$\lambda$}}
\newcommand{\epsilonv}{\hbox{\boldmath$\epsilon$}}
\newcommand{\nuv}{\hbox{\boldmath$\nu$}}
\newcommand{\muv}{\hbox{\boldmath$\mu$}}
\newcommand{\zetav}{\hbox{\boldmath$\zeta$}}
\newcommand{\phiv}{\hbox{\boldmath$\phi$}}
\newcommand{\psiv}{\hbox{\boldmath$\psi$}}
\newcommand{\thetav}{\hbox{\boldmath$\theta$}}
\newcommand{\tauv}{\hbox{\boldmath$\tau$}}
\newcommand{\omegav}{\hbox{\boldmath$\omega$}}
\newcommand{\xiv}{\hbox{\boldmath$\xi$}}
\newcommand{\sigmav}{\hbox{\boldmath$\sigma$}}
\newcommand{\piv}{\hbox{\boldmath$\pi$}}
\newcommand{\rhov}{\hbox{\boldmath$\rho$}}
\newcommand{\upsilonv}{\hbox{\boldmath$\upsilon$}}

\newcommand{\Gammam}{\hbox{\boldmath$\Gamma$}}
\newcommand{\Lambdam}{\hbox{\boldmath$\Lambda$}}
\newcommand{\Deltam}{\hbox{\boldmath$\Delta$}}
\newcommand{\Sigmam}{\hbox{\boldmath$\Sigma$}}
\newcommand{\Phim}{\hbox{\boldmath$\Phi$}}
\newcommand{\Pim}{\hbox{\boldmath$\Pi$}}
\newcommand{\Psim}{\hbox{\boldmath$\Psi$}}
\newcommand{\Thetam}{\hbox{\boldmath$\Theta$}}
\newcommand{\Omegam}{\hbox{\boldmath$\Omega$}}
\newcommand{\Xim}{\hbox{\boldmath$\Xi$}}


% Sans Serif small case

\newcommand{\Gsf}{{\sf G}}

\newcommand{\asf}{{\sf a}}
\newcommand{\bsf}{{\sf b}}
\newcommand{\csf}{{\sf c}}
\newcommand{\dsf}{{\sf d}}
\newcommand{\esf}{{\sf e}}
\newcommand{\fsf}{{\sf f}}
\newcommand{\gsf}{{\sf g}}
\newcommand{\hsf}{{\sf h}}
\newcommand{\isf}{{\sf i}}
\newcommand{\jsf}{{\sf j}}
\newcommand{\ksf}{{\sf k}}
\newcommand{\lsf}{{\sf l}}
\newcommand{\msf}{{\sf m}}
\newcommand{\nsf}{{\sf n}}
\newcommand{\osf}{{\sf o}}
\newcommand{\psf}{{\sf p}}
\newcommand{\qsf}{{\sf q}}
\newcommand{\rsf}{{\sf r}}
\newcommand{\ssf}{{\sf s}}
\newcommand{\tsf}{{\sf t}}
\newcommand{\usf}{{\sf u}}
\newcommand{\wsf}{{\sf w}}
\newcommand{\vsf}{{\sf v}}
\newcommand{\xsf}{{\sf x}}
\newcommand{\ysf}{{\sf y}}
\newcommand{\zsf}{{\sf z}}


% mixed symbols

\newcommand{\sinc}{{\hbox{sinc}}}
\newcommand{\diag}{{\hbox{diag}}}
\renewcommand{\det}{{\hbox{det}}}
\newcommand{\trace}{{\hbox{tr}}}
\newcommand{\sign}{{\hbox{sign}}}
\renewcommand{\arg}{{\hbox{arg}}}
\newcommand{\var}{{\hbox{var}}}
\newcommand{\cov}{{\hbox{cov}}}
\newcommand{\Ei}{{\rm E}_{\rm i}}
\renewcommand{\Re}{{\rm Re}}
\renewcommand{\Im}{{\rm Im}}
\newcommand{\eqdef}{\stackrel{\Delta}{=}}
\newcommand{\defines}{{\,\,\stackrel{\scriptscriptstyle \bigtriangleup}{=}\,\,}}
\newcommand{\<}{\left\langle}
\renewcommand{\>}{\right\rangle}
\newcommand{\herm}{{\sf H}}
\newcommand{\trasp}{{\sf T}}
\newcommand{\transp}{{\sf T}}
\renewcommand{\vec}{{\rm vec}}
\newcommand{\Psf}{{\sf P}}
\newcommand{\SINR}{{\sf SINR}}
\newcommand{\SNR}{{\sf SNR}}
\newcommand{\MMSE}{{\sf MMSE}}
\newcommand{\REF}{{\RED [REF]}}

% Markov chain
\usepackage{stmaryrd} % for \mkv 
\newcommand{\mkv}{-\!\!\!\!\minuso\!\!\!\!-}

% Colors

\newcommand{\RED}{\color[rgb]{1.00,0.10,0.10}}
\newcommand{\BLUE}{\color[rgb]{0,0,0.90}}
\newcommand{\GREEN}{\color[rgb]{0,0.80,0.20}}

%%%%%%%%%%%%%%%%%%%%%%%%%%%%%%%%%%%%%%%%%%
\usepackage{hyperref}
\hypersetup{
    bookmarks=true,         % show bookmarks bar?
    unicode=false,          % non-Latin characters in AcrobatÕs bookmarks
    pdftoolbar=true,        % show AcrobatÕs toolbar?
    pdfmenubar=true,        % show AcrobatÕs menu?
    pdffitwindow=false,     % window fit to page when opened
    pdfstartview={FitH},    % fits the width of the page to the window
%    pdftitle={My title},    % title
%    pdfauthor={Author},     % author
%    pdfsubject={Subject},   % subject of the document
%    pdfcreator={Creator},   % creator of the document
%    pdfproducer={Producer}, % producer of the document
%    pdfkeywords={keyword1} {key2} {key3}, % list of keywords
    pdfnewwindow=true,      % links in new window
    colorlinks=true,       % false: boxed links; true: colored links
    linkcolor=red,          % color of internal links (change box color with linkbordercolor)
    citecolor=green,        % color of links to bibliography
    filecolor=blue,      % color of file links
    urlcolor=blue           % color of external links
}
%%%%%%%%%%%%%%%%%%%%%%%%%%%%%%%%%%%%%%%%%%%



% Colors
\definecolor{KleinBlue}{RGB}{0, 47, 167}
% define functional colors
% the base color
\colorlet{BaseBlue}{KleinBlue}

% analogous colors
\colorlet{AnalogousGreen}{BaseBlue>wheel,5,6}
\colorlet{AnalogousViolet}{BaseBlue>wheel,1,6}

% triad
\colorlet{TriadRed}{BaseBlue>wheel,4,12}
\colorlet{TriadGreen}{BaseBlue>wheel,8,12}

% tetrad colors
\colorlet{TetradMagenta}{BaseBlue>wheel,1,4}
\colorlet{TetradGreen}{BaseBlue>wheel,3,4}

% complementary colors
\colorlet{ComplementaryBrown}{BaseBlue>wheel,1,2}


% Convert wheel-based colors to RGB
\convertcolorspec{named}{BaseBlue}{rgb}\basebluergb
\convertcolorspec{named}{AnalogousGreen}{rgb}\analogousgreenrgb
\convertcolorspec{named}{AnalogousViolet}{rgb}\analogousvioletrgb
\convertcolorspec{named}{TriadRed}{rgb}\triadredrgb
\convertcolorspec{named}{TriadGreen}{rgb}\triadgreenrgb
\convertcolorspec{named}{TetradMagenta}{rgb}\tetradmagentarbg
\convertcolorspec{named}{TetradGreen}{rgb}\tetradgreenrgb
\convertcolorspec{named}{ComplementaryBrown}{rgb}\complementarybrownrgb

\definecolor{rb-blue}{rgb}{\basebluergb}
\colorlet{rb-blue-light}{rb-blue!10!white}
\colorlet{rb-blue-pastel}{rb-blue!50!white}

\definecolor{rb-red}{rgb}{\triadredrgb}
\colorlet{rb-red-light}{rb-red!10!white}
\colorlet{rb-red-pastel}{rb-red!50!white}
\colorlet{rb-red-bright}{rb-red!50!red}

\definecolor{rb-green}{rgb}{\triadgreenrgb}
\colorlet{rb-green-light}{rb-green!10!white}
\colorlet{rb-green-pastel}{rb-green!50!white}

\definecolor{rb-brown}{rgb}{\complementarybrownrgb}
\colorlet{rb-brown-pastel}{rb-brown!50!white}
\colorlet{rb-brown-light}{rb-brown!15!white}
\colorlet{rb-brown-dark}{rb-brown!20!black}

\definecolor{rb-magenta}{rgb}{\tetradmagentarbg}
\colorlet{rb-magenta-light}{rb-magenta!10!white}
\colorlet{rb-magenta-pastel}{rb-magenta!50!white}

% \definecolor{rb-yellow}{rgb}{\tetradgreenrgb}
% \colorlet{rb-yellow-light}{TetradGreen!10!white}
% \colorlet{rb-yellow-pastel}{TetradGreen!50!white}

\definecolor{rb-violet}{rgb}{\analogousvioletrgb}
\colorlet{rb-violet-light}{rb-violet!10!white}
\colorlet{rb-violet-pastel}{rb-violet!50!white}

\definecolor{rb-mint}{rgb}{\analogousgreenrgb}
\colorlet{rb-mint-light}{rb-mint!10!white}
\colorlet{rb-mint-pastel}{rb-mint!50!white}

% Theorem environments
\newtheorem{thm}{Theorem}
\newtheorem{prop}{Proposition}[section]
\newtheorem{lem}[prop]{Lemma}
\newtheorem{dfn}[prop]{Definition}
\newtheorem{cor}[prop]{Corollary}
\newtheorem{eg}[prop]{Example}
\newtheorem{rem}[prop]{Remark}


\title{A Quadratic Lower Bound for Stable Roommates Solvability}
\author{Will Rosenbaum\\
\texttt{W.Rosenbaum@liverpool.ac.uk}\\
Department of Computer Science\\
University of Liverpool
}
\date{\today}

\begin{document}

\maketitle

\begin{abstract}
    In their seminal work on the Stable Marriage Problem (SM), Gale and Shapley introduced a generalization of SM referred to as the Stable Roommates Problem (SR). An instance of SR consists of a set of $2n$ agents, and each agent has preferences in the form of a ranked list of all other agents. The goal is to find a one-to-one matching between the agents that is stable in the sense that no pair of agents have a mutual incentive to deviate from the matching. Unlike the (bipartite) stable marriage problem, in SR, stable matchings need not exist. Irving devised an algorithm that finds a stable matching or reports that none exists in $O(n^2)$ time. In their influential 1989 text, Gusfield and Irving posed the question of whether $\Omega(n^2)$ time is required for SR solvability---the task of deciding if an SR instance admits a stable matching.

    In this paper we provide an affirmative answer to Gusfield and Irving's question. We show that any (randomized) algorithm that decides SR solvability requires $\Omega(n^2)$ adaptive Boolean queries to the agents' preferences (in expectation). Our argument follows from a reduction from the communication complexity of the set disjointness function. The query lower bound implies quadratic time lower bounds for Turing machines, and memory access lower bounds for random access machines. Thus, we establish that Irving's algorithm is optimal (up to a logarithmic factor) in a very strong sense.
\end{abstract}

\section{Introduction}


\begin{figure}[t]
\centering
\includegraphics[width=0.6\columnwidth]{figures/evaluation_desiderata_V5.pdf}
\vspace{-0.5cm}
\caption{\systemName is a platform for conducting realistic evaluations of code LLMs, collecting human preferences of coding models with real users, real tasks, and in realistic environments, aimed at addressing the limitations of existing evaluations.
}
\label{fig:motivation}
\end{figure}

\begin{figure*}[t]
\centering
\includegraphics[width=\textwidth]{figures/system_design_v2.png}
\caption{We introduce \systemName, a VSCode extension to collect human preferences of code directly in a developer's IDE. \systemName enables developers to use code completions from various models. The system comprises a) the interface in the user's IDE which presents paired completions to users (left), b) a sampling strategy that picks model pairs to reduce latency (right, top), and c) a prompting scheme that allows diverse LLMs to perform code completions with high fidelity.
Users can select between the top completion (green box) using \texttt{tab} or the bottom completion (blue box) using \texttt{shift+tab}.}
\label{fig:overview}
\end{figure*}

As model capabilities improve, large language models (LLMs) are increasingly integrated into user environments and workflows.
For example, software developers code with AI in integrated developer environments (IDEs)~\citep{peng2023impact}, doctors rely on notes generated through ambient listening~\citep{oberst2024science}, and lawyers consider case evidence identified by electronic discovery systems~\citep{yang2024beyond}.
Increasing deployment of models in productivity tools demands evaluation that more closely reflects real-world circumstances~\citep{hutchinson2022evaluation, saxon2024benchmarks, kapoor2024ai}.
While newer benchmarks and live platforms incorporate human feedback to capture real-world usage, they almost exclusively focus on evaluating LLMs in chat conversations~\citep{zheng2023judging,dubois2023alpacafarm,chiang2024chatbot, kirk2024the}.
Model evaluation must move beyond chat-based interactions and into specialized user environments.



 

In this work, we focus on evaluating LLM-based coding assistants. 
Despite the popularity of these tools---millions of developers use Github Copilot~\citep{Copilot}---existing
evaluations of the coding capabilities of new models exhibit multiple limitations (Figure~\ref{fig:motivation}, bottom).
Traditional ML benchmarks evaluate LLM capabilities by measuring how well a model can complete static, interview-style coding tasks~\citep{chen2021evaluating,austin2021program,jain2024livecodebench, white2024livebench} and lack \emph{real users}. 
User studies recruit real users to evaluate the effectiveness of LLMs as coding assistants, but are often limited to simple programming tasks as opposed to \emph{real tasks}~\citep{vaithilingam2022expectation,ross2023programmer, mozannar2024realhumaneval}.
Recent efforts to collect human feedback such as Chatbot Arena~\citep{chiang2024chatbot} are still removed from a \emph{realistic environment}, resulting in users and data that deviate from typical software development processes.
We introduce \systemName to address these limitations (Figure~\ref{fig:motivation}, top), and we describe our three main contributions below.


\textbf{We deploy \systemName in-the-wild to collect human preferences on code.} 
\systemName is a Visual Studio Code extension, collecting preferences directly in a developer's IDE within their actual workflow (Figure~\ref{fig:overview}).
\systemName provides developers with code completions, akin to the type of support provided by Github Copilot~\citep{Copilot}. 
Over the past 3 months, \systemName has served over~\completions suggestions from 10 state-of-the-art LLMs, 
gathering \sampleCount~votes from \userCount~users.
To collect user preferences,
\systemName presents a novel interface that shows users paired code completions from two different LLMs, which are determined based on a sampling strategy that aims to 
mitigate latency while preserving coverage across model comparisons.
Additionally, we devise a prompting scheme that allows a diverse set of models to perform code completions with high fidelity.
See Section~\ref{sec:system} and Section~\ref{sec:deployment} for details about system design and deployment respectively.



\textbf{We construct a leaderboard of user preferences and find notable differences from existing static benchmarks and human preference leaderboards.}
In general, we observe that smaller models seem to overperform in static benchmarks compared to our leaderboard, while performance among larger models is mixed (Section~\ref{sec:leaderboard_calculation}).
We attribute these differences to the fact that \systemName is exposed to users and tasks that differ drastically from code evaluations in the past. 
Our data spans 103 programming languages and 24 natural languages as well as a variety of real-world applications and code structures, while static benchmarks tend to focus on a specific programming and natural language and task (e.g. coding competition problems).
Additionally, while all of \systemName interactions contain code contexts and the majority involve infilling tasks, a much smaller fraction of Chatbot Arena's coding tasks contain code context, with infilling tasks appearing even more rarely. 
We analyze our data in depth in Section~\ref{subsec:comparison}.



\textbf{We derive new insights into user preferences of code by analyzing \systemName's diverse and distinct data distribution.}
We compare user preferences across different stratifications of input data (e.g., common versus rare languages) and observe which affect observed preferences most (Section~\ref{sec:analysis}).
For example, while user preferences stay relatively consistent across various programming languages, they differ drastically between different task categories (e.g. frontend/backend versus algorithm design).
We also observe variations in user preference due to different features related to code structure 
(e.g., context length and completion patterns).
We open-source \systemName and release a curated subset of code contexts.
Altogether, our results highlight the necessity of model evaluation in realistic and domain-specific settings.





\section{Background}\label{sec:backgrnd}

\subsection{Cold Start Latency and Mitigation Techniques}

Traditional FaaS platforms mitigate cold starts through snapshotting, lightweight virtualization, and warm-state management. Snapshot-based methods like \textbf{REAP} and \textbf{Catalyzer} reduce initialization time by preloading or restoring container states but require significant memory and I/O resources, limiting scalability~\cite{dong_catalyzer_2020, ustiugov_benchmarking_2021}. Lightweight virtualization solutions, such as \textbf{Firecracker} microVMs, achieve fast startup times with strong isolation but depend on robust infrastructure, making them less adaptable to fluctuating workloads~\cite{agache_firecracker_2020}. Warm-state management techniques like \textbf{Faa\$T}~\cite{romero_faa_2021} and \textbf{Kraken}~\cite{vivek_kraken_2021} keep frequently invoked containers ready, balancing readiness and cost efficiency under predictable workloads but incurring overhead when demand is erratic~\cite{romero_faa_2021, vivek_kraken_2021}. While these methods perform well in resource-rich cloud environments, their resource intensity challenges applicability in edge settings.

\subsubsection{Edge FaaS Perspective}

In edge environments, cold start mitigation emphasizes lightweight designs, resource sharing, and hybrid task distribution. Lightweight execution environments like unikernels~\cite{edward_sock_2018} and \textbf{Firecracker}~\cite{agache_firecracker_2020}, as used by \textbf{TinyFaaS}~\cite{pfandzelter_tinyfaas_2020}, minimize resource usage and initialization delays but require careful orchestration to avoid resource contention. Function co-location, demonstrated by \textbf{Photons}~\cite{v_dukic_photons_2020}, reduces redundant initializations by sharing runtime resources among related functions, though this complicates isolation in multi-tenant setups~\cite{v_dukic_photons_2020}. Hybrid offloading frameworks like \textbf{GeoFaaS}~\cite{malekabbasi_geofaas_2024} balance edge-cloud workloads by offloading latency-tolerant tasks to the cloud and reserving edge resources for real-time operations, requiring reliable connectivity and efficient task management. These edge-specific strategies address cold starts effectively but introduce challenges in scalability and orchestration.

\subsection{Predictive Scaling and Caching Techniques}

Efficient resource allocation is vital for maintaining low latency and high availability in serverless platforms. Predictive scaling and caching techniques dynamically provision resources and reduce cold start latency by leveraging workload prediction and state retention.
Traditional FaaS platforms use predictive scaling and caching to optimize resources, employing techniques (OFC, FaasCache) to reduce cold starts. However, these methods rely on centralized orchestration and workload predictability, limiting their effectiveness in dynamic, resource-constrained edge environments.



\subsubsection{Edge FaaS Perspective}

Edge FaaS platforms adapt predictive scaling and caching techniques to constrain resources and heterogeneous environments. \textbf{EDGE-Cache}~\cite{kim_delay-aware_2022} uses traffic profiling to selectively retain high-priority functions, reducing memory overhead while maintaining readiness for frequent requests. Hybrid frameworks like \textbf{GeoFaaS}~\cite{malekabbasi_geofaas_2024} implement distributed caching to balance resources between edge and cloud nodes, enabling low-latency processing for critical tasks while offloading less critical workloads. Machine learning methods, such as clustering-based workload predictors~\cite{gao_machine_2020} and GRU-based models~\cite{guo_applying_2018}, enhance resource provisioning in edge systems by efficiently forecasting workload spikes. These innovations effectively address cold start challenges in edge environments, though their dependency on accurate predictions and robust orchestration poses scalability challenges.

\subsection{Decentralized Orchestration, Function Placement, and Scheduling}

Efficient orchestration in serverless platforms involves workload distribution, resource optimization, and performance assurance. While traditional FaaS platforms rely on centralized control, edge environments require decentralized and adaptive strategies to address unique challenges such as resource constraints and heterogeneous hardware.



\subsubsection{Edge FaaS Perspective}

Edge FaaS platforms adopt decentralized and adaptive orchestration frameworks to meet the demands of resource-constrained environments. Systems like \textbf{Wukong} distribute scheduling across edge nodes, enhancing data locality and scalability while reducing network latency. Lightweight frameworks such as \textbf{OpenWhisk Lite}~\cite{kravchenko_kpavelopenwhisk-light_2024} optimize resource allocation by decentralizing scheduling policies, minimizing cold starts and latency in edge setups~\cite{benjamin_wukong_2020}. Hybrid solutions like \textbf{OpenFaaS}~\cite{noauthor_openfaasfaas_2024} and \textbf{EdgeMatrix}~\cite{shen_edgematrix_2023} combine edge-cloud orchestration to balance resource utilization, retaining latency-sensitive functions at the edge while offloading non-critical workloads to the cloud. While these approaches improve flexibility, they face challenges in maintaining coordination and ensuring consistent performance across distributed nodes.


\section{The lower bound}\label{sec:lb}
In this section, we provide a sample complexity lower bound proof with regard to the error tolerance $\eps$. In fact, we prove \Cref{thm:main}, which is a formal version of \Cref{thm:main-lb}. %All distributions constructed in this section will satisfy \Cref{assump:moment} and \Cref{assump:smooth}.

% A sampling algorithm here is a procedure with query access to $f$, $\grad f$ or $\grad^2 f$, which outputs a sample point based on a series of queries. We aim to prove the following theorem in this note.
% \htodo{Here $c$ is the constant in \Cref{lem:disjointcap}.}

Recall that we use $\@D_{L,M}$ to denote the collection of distributions which are $L$-log-smooth and have second moment at most $M$. 
\begin{theorem}\label{thm:main}
    % There exist a universal constant $C>0$ such that 
    For any $L,M>0$ satisfying $LM\ge d$ and for any $\eps\in(0,1/32)$, $d\geq 5$, if a sampling algorithm $\+A$ always terminates within 
\[
    \frac{\eps (d-2)^{\frac{3}{2}}}{8}\cdot \tp{\frac{9}{256} \cdot \frac{LM}{d\eps} \cdot \frac{1}{\log \frac{LM}{d\eps}}}^{\frac{d-1}{2}}
    % \frac{\eps}{4}\cdot \tp{\frac{C\cdot LM}{2 d\eps} \cdot \frac{1}{\log \frac{LM}{d\eps}}}^{\frac{d-1}{2}}% \approx \frac{\eps}{4}\exp\set{\frac{d}{2}\cdot \Omega\tp{\log \frac{LM}{d\eps}}}
\]
queries on every input instance in $\@D_{L,M}$, then there must exist some distribution $\mu\in \@D_{L,M}$ such that when the underlying instance is $\mu$, the distribution of $\+A$'s output, denoted as $\tilde \mu$, is $\eps$ away from $\mu$ in total variation distance, i.e., $\DTV(\mu,\tilde \mu)\geq \eps$.
\end{theorem}
%\htodo{Actually we consider those distributions with second moment $O(M)$ and $O(L)$-smooth. Not exactly $M$ and $L$.}

%\mn{When $d$ is even, $\Gamma\tp{\frac{d}{2}+1} = \tp{\frac{d}{2}}!$. When $d$ is odd, $\Gamma\tp{\frac{d}{2}+1} = \frac{\sqrt{\pi}}{2^d}\cdot \frac{d!}{\tp{\frac{d-1}{2}}!}$.}

\subsection{The base instance}

We first construct a base distribution $\mu_0$. Let $R= \tp{\frac{M}{\eps}}^{\frac{1}{2}}$, and let
\[
    \mathfrak{g}_{[\frac{R}{4},\frac{R}{2}]}(x)=q_{\!{mol}}\tp{\frac{ \|x\|^2- \frac{R^2}{16}}{\frac{R^2}{4} - \frac{R^2}{16}}} \quad \mbox{and} \quad \mathfrak{g}_{[R,2R]}(x)=q_{\!{mol}}\tp{\frac{\|x\|^2-R^2}{4R^2-R^2}},
\]
% Let $\alpha\in (0,2)$ be a constant to be determined later.
With constant function $h_1 \equiv \log \tp{\!{vol}(\+B_{3 R})} + \log \frac{1}{\eps}$ and function $h_0(x)=\frac{d\|x\|^2}{2M} + \frac{d}{2}\log \frac{2\pi M}{d}$, define the function $f_0$ as
\[
    \forall x\in \bb R^d, f_0(x) = \begin{cases} h_0(x), & \|x\|\leq \frac{R}{4} \\
    \mathfrak{g}_{[\frac{R}{4},\frac{R}{2}]}(x)\cdot h_1 + \tp{1-\mathfrak{g}_{[\frac{R}{4},\frac{R}{2}]}(x)}\cdot h_0(x), & \frac{R}{4}<\|x\| \leq \frac{R}{2}\\
    h_1, & \frac{R}{2} <\|x\|\leq R\\
    \mathfrak{g}_{[R,2R]}(x)\cdot h_0(x) + \tp{1-\mathfrak{g}_{[R,2R]}(x)}\cdot h_1, & R<\|x\|\leq 2R \\
    h_0(x), & \|x\|>2R
    \end{cases}.
\]

%\mn{Here we use a constant function $h_1$ rather than using Gaussian directly. This is crucial.}
Consider the distribution $\mu_0$ with density $p_{\mu_0} \propto \exp\tp{-f_0(x)}$ and its normalizing factor $Z_0 = \int_{\bb R^d} \exp\tp{-f_0(x)} \d x$.
\begin{lemma}\label{lem:Z_0}
    The normalizing constant $1-16\eps \leq Z_0 \leq  1+\eps$. 
\end{lemma}
\begin{proof}
    On one hand, from Markov's inequality,
    $$
        Z_0\geq \int_{\+B_{\frac{R}{4}}} e^{-f_0(x)} \dd x = \int_{\+B_{\frac{R}{4}}} e^{-h_0(x)} \dd x = 1- \Pr[X\sim \+N\tp{0,\frac{M}{d}\cdot \!{Id}_d}]{\|X\|^2\geq \frac{R^2}{16}} \geq 1-16\eps.
    $$
    On the other hand, 
    $$
        Z_0\leq \int_{\bb R^d} e^{-h_0(x)} \dd x + \vol (\+B_{2R})\cdot e^{-h_1} \leq 1 + \eps\cdot \frac{\vol (\+B_{2R})}{\vol(\+B_{3R})}\leq 1+\eps.
    $$
    
    % Note that $h_1=\log \frac{1}{\eps} + \frac{d}{2}\log\tp{\pi \alpha^2 R^2} - \log \Gamma\tp{\frac{d}{2}+1}$.
    
    % Note that for each fixed $x\in \bb R^d$, $f_0$ is a non-decreasing function wrt $\alpha$. So $Z_0$ is decreasing as $\alpha$ increases. When $\alpha=1$,
\end{proof}

% We fixed the value of $\alpha$ to be the one in \Cref{lem:Z_0}.
\begin{lemma}\label{lem:propertymu0}
    The distribution $\mu_0$ is $\+O(L)$-log-smooth and has second moment $\+O(M)$.
\end{lemma}
%\htodo{$O(M)$ and $O(L)$, not exactly $M$ and $L$} 
\begin{proof}
    We first calculate the second moment of $\mu_0$. We have
    $$
        \E[\mu_0]{\|X\|^2} \leq \frac{\E[X\sim \+N\tp{0,\frac{M}{d}\cdot \!{Id}_d}]{\|X\|^2} + \vol (\+B_{2R})\cdot e^{-h_1}\cdot 4R^2}{Z_0} \leq \frac{M + \eps\cdot \frac{\vol (\+B_{2R})}{\vol (\+B_{3R})}\cdot 4R^2}{Z_0} \leq \frac{3M}{Z_0}\leq 6M,
    $$
    where the last inequality is due to \Cref{lem:Z_0} and the fact that $\eps<\frac{1}{32}$.
    % \htodo{We may need $\eps<\frac{1}{32}$.}

    For the smoothness, we only need to check $\|\grad^2 f_0(x)\|$ for those $x$ with $\|x\|\in (\frac{R}{4},\frac{R}{2}]$ and $\|x\|\in (R,2R]$ since clearly $f_0\in C^2(\bb R^d)$. %\ctodo{need $f_0\in C^2$ here.}

    First, for $\|x\|\in (\frac{R}{4},\frac{R}{2}]$,
    \[
        \grad f_0(x) = \grad \mathfrak{g}_{[\frac{R}{4},\frac{R}{2}]}(x) \cdot (h_1 - h_0(x) ) + (1-\mathfrak{g}_{[\frac{R}{4},\frac{R}{2}]}(x))\cdot \grad h_0(x), 
    \]
    and
    \begin{align*}
        \grad^2 f_0(x) &= \underbrace{\grad^2 \mathfrak{g}_{[\frac{R}{4},\frac{R}{2}]}(x) \cdot (h_1 - h_0(x))}_{\mbox{(a)}} - \underbrace{\tp{\grad \mathfrak{g}_{[\frac{R}{4},\frac{R}{2}]}(x) \cdot \grad h_0(x)^{\top} +  \grad h_0(x)\cdot \grad \mathfrak{g}_{[\frac{R}{4},\frac{R}{2}]}(x)^{\top}}}_{\mbox{(b)}}\\
        &\quad + \underbrace{ \tp{1-\mathfrak{g}_{[\frac{R}{4},\frac{R}{2}]}(x)}\cdot \grad^2 h_0(x)}_{\mbox{(c)}}.
        \end{align*}
    Recall that $LM\geq d$, so it is easy to get $0\preceq \mbox{(c)}\preceq L\cdot \!{Id}_d$. Since 
    \[
        \grad \mathfrak{g}_{[\frac{R}{4},\frac{R}{2}]}(x) = \frac{2x}{\frac{R^2}{4} - \frac{R^2}{16}} \cdot q_{\!{mol}}'\tp{\frac{ \|x\|^2- \frac{R^2}{16}}{\frac{R^2}{4} - \frac{R^2}{16}}},
    \]
    we have
    \[
        \mbox{(b)} = \frac{4d \cdot xx^{\top}}{M\tp{\frac{R^2}{4} - \frac{R^2}{16}}} \cdot q_{\!{mol}}'\tp{\frac{ \|x\|^2- \frac{R^2}{16}}{\frac{R^2}{4} - \frac{R^2}{16}}}.
    \]
    % \mn{For $x,y\in \bb R^d$, we can show that $\|x\|\|y\|\!{Id}_d - xy^T\succeq 0$: for any $z\in \bb R^d$,
    % \begin{align*}
    %     &\phantom{{}={}}z^T\tp{\|x\|\|y\|\!{Id}_d - xy^T}z \\
    %     & =\|x\|\|y\|\|z\|^2 - (z^Tx)(y^Tz)\\
    %     &\geq \|x\|\|y\|\|z\|^2 - \|x\|\|y\|\|z\|^2\\
    %     &= 0.
    % \end{align*}}
    Therefore, $-\+O(L)\cdot\!{Id}_d \preceq \mbox{(b)} \preceq \+O(L)\cdot\!{Id}_d$.
    By direct calculation,
    \[
        \grad^2 \mathfrak{g}_{[\frac{R}{4},\frac{R}{2}]}(x) = \frac{4xx^{\top}}{\tp{\frac{R^2}{4} - \frac{R^2}{16}}^2} \cdot q''_{\!{mol}}\tp{\frac{ \|x\|^2- \frac{R^2}{16}}{\frac{R^2}{4} - \frac{R^2}{16}}} + \frac{2\!{Id}_d}{\frac{R^2}{4} - \frac{R^2}{16}} \cdot q'_{\!{mol}}\tp{\frac{ \|x\|^2- \frac{R^2}{16}}{\frac{R^2}{4} - \frac{R^2}{16}}}
    \]
    and 
    \begin{align}
        \abs{h_1 - h_0(x)} &= \abs{\log \frac{1}{\eps} + \frac{d}{2}\log\tp{9\pi R^2} - \log \Gamma\tp{\frac{d}{2}+1} - \frac{d\|x\|^2}{2M} - \frac{d}{2}\log \frac{2\pi M}{d}} \notag\\
        &= \abs{\log \frac{1}{\eps} + \frac{d}{2}\log\frac{9d}{2\eps} - \log \Gamma\tp{\frac{d}{2}+1} - \frac{d\|x\|^2}{2M}} \label{eq:1}
        % &\leq \log \frac{1}{\eps} + \frac{d}{2}\log\tp{9\pi R^2} - \log \Gamma\tp{\frac{d}{2}+1} + \frac{R^2 d}{8M} + \frac{d}{2}\log \frac{2\pi M}{d}.
    \end{align}
    % \htodo{Here we need $R^2=\frac{M}{\eps}\log \frac{M}{\eps}$?}
    From Stirling's formula, we know that for any $d>0$,
    \[
        \log\sqrt{\pi d} + \frac{d}{2}\log \frac{d}{2e} \leq \log \Gamma\tp{\frac{d}{2}+1}\leq \log\sqrt{\pi d} + \frac{d}{2}\log \frac{d}{2e} + 1.
    \]
    % \[
    %     \log \Gamma\tp{\frac{d}{2}+1} \leq \begin{cases}
    %         \log\sqrt{\pi d} + \frac{d}{2}\log \frac{d}{2e} + 1, & d \mbox{ is even}\\
    %         \log \sqrt{\frac{2\pi d}{d-1}} + \frac{d+1}{2}\log \frac{d}{2e} + 1, & d \mbox{ is odd}
    %     \end{cases},
    % \]
    % and 
    % \[
    %     \log \Gamma\tp{\frac{d}{2}+1} \geq \begin{cases}
    %         \log\sqrt{\pi d} + \frac{d}{2}\log \frac{d}{2e} , & d \mbox{ is even}\\
    %         \log \sqrt{\frac{2\pi d}{d-1}} + \frac{d+1}{2}\log \frac{d}{2e} , & d \mbox{ is odd}
    %     \end{cases}.
    % \]
    Back to \Cref{eq:1}, we have
    \begin{align*}
        \abs{h_1 - h_0(x)}&\leq \abs{\log \frac{1}{\eps} + \frac{d}{2}\log\frac{9e}{\eps} - \frac{d\|x\|^2}{2M} -\log\sqrt{\pi d}} + 1 \\
        &\leq \log \frac{1}{\eps} + \frac{d}{2}\log\frac{9e}{\eps} + \frac{dR^2}{8M} + \log\sqrt{\pi d} +1.
    \end{align*}
    Since $LM\geq d$, we have $-\+O(L)\cdot\!{Id}_d \preceq \mbox{(a)}\preceq \+O(L)\cdot\!{Id}_d$.
    
    For $\|x\|\in (R,2R]$, 
    \[
        \grad^2 f_0(x) = \grad^2 \mathfrak{g}_{[R,2R]}(x)(h_0(x)-h_1) + 2\grad \mathfrak{g}_{[R,2R]}(x) \grad h_0(x)^{\top} + \mathfrak{g}_{[R,2R]}(x) \cdot \grad^2 h_0(x).
    \]
    The remaining calculations are similar. %\ctodo{Maybe say more here.}
\end{proof}


\subsection{Perturb the base instance}\label{sec:hardinstance}
We then construct instances via perturbing $\mu_0$. Let $r_1= \sqrt{\frac{d}{L}\log \frac{LM}{d\eps}}$, $r_2=\sqrt{2}r_1$. Let $h_2\defeq h_1 - \gamma$, where $\gamma$ is a value to be determined later.
% \htodo{Assume the value of $\eps,L,M,d$ satisfy $4r_2\leq R$ and $r_1=\Omega(1)$.}

Note that when $\eps<1/32$, we have $4r_2\leq R$. For a point $v\in \bb R^d$ with $\|v\|=\frac{3R}{4}$, let $\mathfrak{g}_v(x) = q_{\!{mol}}\tp{\frac{\|x-v\|^2-r_1^2}{r_2^2-r_1^2}}$ and $f_v(x)=\mathfrak{g}_v(x)f_0(x)+(1-\mathfrak{g}_v(x))h_2$. This means that, outside the ball $\+B_{r_2}(v)$, $f_v\equiv f_0$, and inside the ball $\+B_{r_1}(v)$, $f_v\equiv h_2$. Define density of the distribution $\mu_v$ over $\bb R^d$ as $p_{\mu_v}\propto e^{-f_v}$. Let $Z_v = \int_{\bb R^d} e^{-f_v(x)} \dd x$.

\begin{lemma}\label{lem:gamma}
    There exists a $\gamma>0$ such that the following holds at the same time:
    \begin{itemize}
        \item  $\int_{\+B_{r_2}(v)} \tp{e^{-f_v(x)} - e^{-h_1}}  \dd x = 9\eps$;
        %\item $9 \tp{\frac{3R}{r_2}}^d \leq e^{\gamma}\leq 18\tp{\frac{3R}{r_1}}^d $;
        % \item $\eps \leq \DTV(\mu_0,\mu_v)\leq 10\eps$;
        \item $\frac{9\eps e^{h_1}}{\!{vol}(\+B_{r_2})}\le e^\gamma-1 \le \frac{9\eps e^{h_1}}{\!{vol}(\+B_{r_1})}$.
        \item $Z_0\leq Z_v\leq 1+10\eps$.
    \end{itemize}
    % There exists a $\gamma$ with $\eps\abs{h_1-\gamma} \leq ()$, such that $\eps \leq \DTV(\mu_0,\mu_v)\leq 10\eps$ for each $\|v\| = \frac{3R}{4}$. This $\gamma$ also satisfies that $Z_0\leq Z_v\leq 1+10\eps$
\end{lemma}

Before proving the lemma, let us examine the information it brings. Recall that $f_0(x)\equiv h_1$ when $x\in \+B_{r_2}(v)$ and we would like to perturb $h_1$ by amount of $\gamma$ to obtain $f_v$ so that there will be $\Theta(\eps)$ more probability mass in $\+B_{r_2}(v)$. For fixed $r_1$ and $r_2$, the lemma says that the order of $\gamma$ is roughly proportional to $h_1$. 

\begin{proof}[Proof of \Cref{lem:gamma}]
    Consider the value $\int_{\+B_{r_2}(v)} \tp{e^{-f_v(x)} - e^{-h_1}}\dd x$. It is continuous and increasing in $\gamma$ when $\gamma\ge 0$. When $\gamma = 0$, $\int_{\+B_{r_2}(v)} \tp{e^{-f_v(x)} - e^{-h_1}}\dd x = 0$. When $\gamma \to \infty $, this value goes to $\infty$. So we can find a $\gamma$ such that $\int_{\+B_{r_2}(v)} \tp{e^{-f_v(x)} - e^{-h_1}}\dd x = 9\eps$ holds exactly.

    For such a $\gamma$, we have
    \[
        Z_v = \int_{\bb R^d} e^{-f_v(x)} \dd x \leq \int_{\bb R^d} e^{-f_0(x)} \dd x + \int_{\+B_{r_2}(v)} \tp{ e^{-f_v(x)} - e^{-h_1}}\dd x = Z_0+9\eps \leq 1+10\eps.
    \]
    Also
    $$
        Z_v = \int_{\bb R^d} e^{-f_v(x)} \dd x \geq \int_{\bb R^d} e^{-f_0(x)} \dd x = Z_0.
    $$
    
    % Then we caculate $\DTV(\mu_0,\mu_v)$. On one hand, since $\+B(v,r_1)\subseteq \+B(v,r_2)\subseteq \set{x\in \bb R^d:\ \|x\| \in (\frac{R}{2},R]}$,
    % \begin{align*}
    %     \DTV(\mu_0,\mu_v) &= \frac{1}{2}\int_{\bb R^d} \abs{\frac{e^{-f_0(x)}}{Z_0} - \frac{e^{-f_v(x)}}{Z_v}} \dd x \\ 
    %     &\geq \frac{1}{2}\int_{\+B(v,r_2)} \abs{\frac{e^{-f_0(x)}}{Z_0} - \frac{e^{-f_v(x)}}{Z_v}} \dd x \\
    %     &=  \frac{1}{2}\int_{\+B(v,r_2)}  \abs{\frac{e^{-f_0(x)}}{Z_v} - \frac{e^{-f_v(x)}}{Z_v}} - \abs{\frac{e^{-f_0(x)}}{Z_0} -  \frac{e^{-f_0(x)}}{Z_v}}  \dd x  \\
    %     &= \frac{1}{2}\tp{\frac{9\eps}{Z_v} - e^{-h_1}\cdot \abs{\frac{1}{Z_0} - \frac{1}{Z_v}} \cdot \vol\tp{\+B(v,r_2)}} \\
    %     &\geq \eps.
    % \end{align*}

    % On the other hand, \begin{align*}
    %     \DTV(\mu_0,\mu_v) &= \frac{1}{2}\int_{\bb R^d} \abs{\frac{e^{-f_0(x)}}{Z_0} - \frac{e^{-f_v(x)}}{Z_v}} \dd x \\ 
    %     &\leq \frac{1}{2}\int_{\bb R^d }  \abs{\frac{e^{-f_0(x)}}{Z_v} - \frac{e^{-f_v(x)}}{Z_v}} + \abs{\frac{e^{-f_0(x)}}{Z_0} -  \frac{e^{-f_0(x)}}{Z_v}}  \dd x \\
    %     &\leq \frac{1}{2}\tp{\frac{9\eps}{Z_v} + Z_0\cdot \abs{\frac{1}{Z_0} - \frac{1}{Z_v}}}\\
    %     &\leq 10\eps.
    % \end{align*}
    
    It remains to calculate $e^\gamma$. We have that
    \[
        \vol\tp{\+B_{r_1}(v)} \cdot e^{-h_1} \tp{e^\gamma - 1}\leq \int_{\+B_{r_2}(v)} \tp{e^{-f_v(x)} - e^{-h_1}}  \dd x =9\eps \leq \vol\tp{\+B_{r_2}(v)} \cdot e^{-h_1} \tp{e^\gamma - 1}.
    \]
\end{proof}

\begin{corollary}\label{cor:gamma-bound}
    For our choice of $h_1$, $r_1$ and $r_2$, it holds that
    \[
        9\tp{\frac{3R}{r_2}}^d \leq e^{\gamma}\leq 18\tp{\frac{3R}{r_1}}^d.
    \]
\end{corollary}
\begin{proof}
    Recall that $e^{h_1}=\eps^{-1}\vol\tp{\+B_{3R}}$ and $\vol\tp{\+B_r} = \frac{\tp{\pi R^2}^{\frac{d}{2}}}{\Gamma\tp{\frac{d}{2}+1}}$. We have
    \[
        e^\gamma-1 \leq  9\tp{\frac{3R}{r_1}}^d \mbox{ and } e^\gamma-1 \geq 9 \tp{\frac{3R}{r_2}}^d.
    \]
    Therefore
    \[
        18\tp{\frac{3R}{r_1}}^d \geq e^\gamma\geq 9 \tp{\frac{3R}{r_2}}^d.
    \]
\end{proof}

\subsection{Properties of the perturbed distributions}
For every $v$ with $\|v\|= \frac{3R}{4}$, we first analyze the smoothness and second moment of the distribution $\mu_v$.

\begin{lemma}\label{lem:moment}
    For $\|v\|=\frac{3R}{4}$, $\E[\mu_v]{\|X\|^2}= \+O\tp{M}$.
\end{lemma}
\begin{proof}
    Direct calculation gives
    \begin{align*}
        \E[\mu_v]{\|X\|^2} & \leq \frac{\E[X\sim \+N\tp{0,\frac{M}{d}\cdot \!{Id}_d}]{\|X\|^2} + \vol (B_{2R})\cdot e^{-h_1}\cdot 4R^2 + \int_{\+B_{r_2}(v)} \tp{e^{-f_v(x)} - e^{-h_1}} \|X\|^2 \dd x}{Z_v} \\
        &\leq \frac{3M + R^2\cdot 9\eps}{Z_v} \leq 24 M.
    \end{align*}
    % \htodo{Here only $\tilde O(M)$. It seems that we cannot guarantee the $O(L)$-smooth of $f_0$ and $O(M)$ second moment of $\mu_v$. Or equivalently, we can choose $M=\frac{M_0}{\log \frac{M_0}{\eps}}$. We require $LM_0\geq d\log\frac{M_0}{\eps}$.}
\end{proof}

\begin{lemma}\label{smooth}
  For $\|v\|=\frac{3R}{4}$, the function $f_v$ is $\+O(L)$-smooth. 
\end{lemma}
\begin{proof}
    We only need to consider those $x\in \+B_{r_2}(v)\setminus \+B_{r_1}(v)$. For such $x$, $f_0(x)=h_1$. Therefore,
    \[
        \grad^2 f_v(x) = \gamma\cdot \grad^2 g_v(x).
    \]
    Note that
    \[
        \grad g_v(x) = \frac{2(x-v)}{r_2^2-r_1^2}\cdot q_{\!{mol}}'\tp{\frac{\|x-v\|^2-r_1^2}{r_2^2-r_1^2}}
    \]
    and
    \[
        \grad^2 g_v(x) = \frac{4(x-v)(x-v)^{\top}}{\tp{r_2^2-r_1^2}^2}\cdot q''_{\!{mol}}\tp{y=\frac{\|x-v\|^2-r_1^2}{r_2^2-r_1^2}} + \frac{2\!{Id}_d}{r_2^2-r_1^2}\cdot q'_{\!{mol}}\tp{\frac{\|x-v\|^2-r_1^2}{r_2^2-r_1^2}}.
    \]
    From \Cref{cor:gamma-bound}, 
    \[
        \gamma \leq \log 18 + \frac{d}{2}\log\frac{9LM}{\eps d} - \frac{d}{2}\log\log \frac{LM}{d\eps}.
    \]
    Therefore, $-\+O(L)\cdot\!{Id}_d \preceq \grad^2 f_v(x) \preceq \+O(L)\cdot\!{Id}_d $.
\end{proof}

We remark that the constants hidden in the $\+O(\cdot)$ in the above two lemmas are universal constants and do not depend on $d$ and $\eps$. 


\begin{lemma}\label{lem:TV}
    For $u,v\in \bb R^d$ such that $\|v\|=\|u\|=\frac{3R}{4}$ and $\+B_{r_2}(u)\cap \+B_{r_2}(v)=\emptyset$, $\DTV\tp{\mu_u,\mu_v}> 4\eps$.
\end{lemma}
\begin{proof}
    By the definition of total variation distance, 
    \begin{align*}
        \DTV(\mu_u,\mu_v) & = \frac{1}{2}\int_{\bb R^d} \abs{\frac{e^{-f_u(x)}}{Z_u} - \frac{e^{-f_v(x)}}{Z_v}} \dd x \\
        \mr{$Z_u=Z_v$} & =\frac{1}{2Z_v} \tp{\int_{\+B_{r_2}(u)}\abs{e^{-f_u(x)} - e^{-h_1}} \dd x + \int_{\+B_{r_2}(v)}\abs{e^{-f_v(x) }- e^{-h_1}} \dd x} \\
        &= \frac{9\eps}{Z_v} > 4\eps.
    \end{align*}
\end{proof}

\subsection{The number of disjoint $\+B_{r_2}(v)$'s}

\begin{lemma}\label{lem:disjointcap}
    Suppose $d\geq 5$. There exist $n=\frac{(d-1)\sqrt{d-2}}{2}\cdot \tp{\frac{3R}{8\sqrt{2}r_2}}^{d-1}$ vectors $v_1,v_2,\dots,v_n \in \bb R^d$ such that
    \begin{itemize}
        \item for each $i\in[n]$, $\|v_i\| = \frac{3R}{4}$;
        \item for each $i,j\in [n]$, if $i\ne j$, then $\+B_{r_2}(v_i)\cap \+B_{r_2}(v_j)=\emptyset$.
    \end{itemize}
\end{lemma}
\begin{proof}
    Let $S$ be the sphere $\set{x\in \bb R^d: \|x\| = \frac{3R}{4}}$. For two vectors $x,y\in \bb R^d$, let $\theta(x,y)$ represent the angle between $x$ and $y$. We first try to find $n$ disjoint caps $C_1,C_2,\dots,C_n$ on $S$. Denoting $v_i$ as the central vector of cap $C_i$, $C_i=\set{x\in \bb R^d: \|x\|=\frac{3R}{4}, \cos(\theta(x,v_i))\geq \ell}$ with $\ell=\frac{\sqrt{(\tp{\frac{3R}{4}}^2-2r_2^2)}}{\frac{3R}{4}}$. 
    
    In contrast, suppose we can only find $n'<\frac{(d-2)^{\frac{3}{2}}}{2}\cdot \tp{\frac{3R}{8\sqrt{2}r_2}}^{d-1}$ such disjoint caps $\set{C_i}_{1\leq i\leq n'}$ with central vectors $\set{v_i}_{1\leq i\leq n'}$. Then for any $w\in S$, there exist $i\in[n']$ and $x\in S$ such that $\cos(\theta(x,v_i))\geq \ell$ and $\cos(\theta(x,w))\geq \ell$. Otherwise we can find a new cap with center $w$.
    
    Via \Cref{lem:cos}, we know that $\cos(\theta(w,v_i))\geq \ell^2 - \tp{1-\ell^2}  \ell'$ with $\ell'=\frac{\tp{\frac{3R}{4}}^2-4r_2^2}{\tp{\frac{3R}{4}}^2}$. 
    This means we can find $n'$ larger caps with central vectors $\set{v_i}_{1\leq i\leq n'}$ and angle $\arccos(\ell')$ to cover the sphere.
    From \cite{L11}, however, the area of a cap with angle $\theta = \arccos(\ell')$ is $\frac{\Gamma\tp{\frac{d-1}{2}}}{\sqrt{\pi}\Gamma\tp{\frac{d}{2}}} \int_{0}^{\theta} \sin^{d-2}(\phi)\d \phi$ times of the total sphere. We have
    % \ctodo{Find the constant $c$.}
    \[
         \int_{0}^{\theta} \sin^{d-2}(\phi)\d \phi \leq \frac{1}{\ell'} \int_{0}^{\theta} \sin^{d-2}(\phi)\cos (\phi)\d \phi = \frac{1}{\ell'} \int_0^{\sin(\theta)} s^{d-2} \d s = \frac{1}{\ell'} \frac{\tp{\sin \theta}^{d-1}}{d-1}.
    \]
    Since $\sin \theta = \sqrt{1-\tp{\ell'}^2}\leq \frac{2\sqrt{2}r_2}{\frac{3R}{4}}$, the ratio between the area of this larger cap and the sphere can be bounded by
    \begin{align*}
        \frac{\Gamma\tp{\frac{d-1}{2}}}{\sqrt{\pi}\Gamma\tp{\frac{d}{2}}} \int_{0}^{\theta} \sin^{d-2}(\phi)\d \phi & \leq \frac{1}{\ell' (d-1)}\cdot \frac{\Gamma\tp{\frac{d-1}{2}}}{\sqrt{\pi}\Gamma\tp{\frac{d}{2}}}\cdot \tp{\frac{2\sqrt{2}r_2}{\frac{3R}{4}}}^{d-1} \\
        \mr{$R\geq 4r_2$}
        &\leq \frac{9}{5\sqrt{\pi}(d-1)} \cdot \frac{\Gamma\tp{\frac{d-1}{2}}}{\Gamma\tp{\frac{d}{2}}} \cdot \tp{\frac{8\sqrt{2}r_2}{3R}}^{d-1} \\
        \mr{Gautschi's inequality}
        &\leq \frac{9\sqrt{2}}{5\sqrt{\pi}}\cdot \frac{1}{(d-1)\sqrt{d-2}}\cdot \tp{\frac{8\sqrt{2}r_2}{3R}}^{d-1}\\
        &\leq \frac{2}{(d-1)\sqrt{d-2}} \tp{\frac{8\sqrt{2}r_2}{3R}}^{d-1}.
    \end{align*}
    
    % So there exists some universal constant $c'$ such that the ratio between the area of this larger cap and the sphere is no larger than $\tp{\frac{c' r_2}{R}}^{d-1}$. 
    % By choosing $c= \frac{1}{c'}$, 
    This will lead to a conflict since $n'$ such caps cannot cover the sphere. Therefore, we can find $n= \frac{(d-1)\sqrt{d-2}}{2}\cdot \tp{\frac{3R}{8\sqrt{2}r_2}}^{d-1}$ such $C_i$'s.
    
    Furthermore, from \Cref{lem:cosinBall}, $\+B_{r_2}(v_i)\cap \+B_{r_2}(v_j)=\emptyset$.
\end{proof}

\subsection{Proof of the lower bound}
\begin{theorem}[\Cref{thm:main} restated]
    % There exist a universal constant $C>0$ such that 
    For any $L,M>0$ satisfying $LM\ge d$ and for any $\eps\in(0,1/32)$, $d\geq 5$, if a sampling algorithm $\+A$ always terminates within 
    \[
        \frac{\eps (d-2)^{\frac{3}{2}}}{8}\cdot \tp{\frac{9}{256} \cdot \frac{LM}{d\eps} \cdot \frac{1}{\log \frac{LM}{d\eps}}}^{\frac{d-1}{2}}
        % \frac{\eps}{4}\cdot \tp{\frac{C\cdot LM}{2 d\eps} \cdot \frac{1}{\log \frac{LM}{d\eps}}}^{\frac{d-1}{2}}
        % % \approx \frac{\eps}{4}\exp\set{\frac{d}{2}\cdot \Omega\tp{\log \frac{LM}{d\eps}}}
    \]
    queries on every input instance in $\@D_{L,M}$, then there must exist some distribution $\mu\in \@D_{L,M}$ such that when the underlying instance is $\mu$, the distribution of $\+A$'s output, denoted as $\tilde \mu$, is $\eps$ away from $\mu$ in total variation distance, i.e., $\DTV(\mu,\tilde \mu)\geq \eps$.
\end{theorem}
\begin{proof}
    Let $v_1,v_2,\dots,v_n$ be the $n$ vectors in \Cref{lem:disjointcap}. For each $i\in [n]$, construct a distribution $\mu_{v_i}$ with density $p_i \propto e^{-f_{v_i}}$ as described in \Cref{sec:hardinstance}. From the discussion in previous sections, we can assume that every $\mu_{v_i}$ as well as the base instance are $L$-log-smooth and have second moment at most $M$. For simplicity, we write $\mu_{v_i}$ as $\mu_i$.

    We use $\Pr[\mu]{\cdot}$ and $\E[\mu]{\cdot}$ to denote the probability and expectation when the underlying instance is some distribution $\mu$. Let $\+E_{k,i}$ be the event that the algorithm $\+A$ queries a value in zone $\+B_{r_2}(v_i)$ in the $k$-th query and it is the first time that $\+A$ queries the points in $\+B_{r_2}(v_i)$. If the algorithm terminates before the $k$-th query, we regard $\+E_{k,i}$ as an impossible event.

    Assume $\+A$ always terminates in $N$ queries for some 
    $$
        N< \frac{\eps (d-2)^{\frac{3}{2}}}{8}\cdot \tp{\frac{9}{256}\cdot \frac{LM}{d\eps} \cdot \frac{1}{\log \frac{LM}{d\eps}}}^{\frac{d-1}{2}} < \frac{\eps}{4}\cdot \frac{(d-1)\sqrt{d-2}}{2}\cdot \tp{\frac{3R}{8\sqrt{2}r_2}}^{d-1} = \frac{\eps n}{4}.
    $$ 
    Let $\+E_{i}$ be the event that $\+A$ queries the points in $\+B_{r_2}(v_i)$ at least once. Then
    \begin{align*}
        \sum_{i=1}^n \Pr[\mu_0]{\+E_i}& = \sum_{i=1}^n \sum_{k=1}^N \Pr[\mu_0]{\+E_{k,i}} = \sum_{k=1}^N \sum_{i=1}^n \Pr[\mu_0]{\+E_{k,i}} \leq N<\frac{\eps n}{4}.
    \end{align*}
    So there exists some $i_0,j_0\in[n]$ and $i_0\neq j_0$ such that $\Pr[\mu_0]{\+E_{i_0}}<\frac{\eps}{3}$ and $\Pr[\mu_0]{\+E_{j_0}}<\frac{\eps}{3}$. Otherwise $\sum_{i=1}^n \Pr[\mu_0]{\+E_i}\geq \frac{\eps(n-1)}{3}\geq \frac{\eps n}{4}$ since $n\geq 4$ when $d\geq 5$. From union bound, $\Pr[\mu_0]{\ol{\+E_{i_0}}\cap \ol{\+E_{j_0}}}> 1-\frac{2\eps}{3}$. 
    
    We know that on $\bb R^d\setminus \+B_{r_2}(v_i)$, $f_{v_i}(x)=f_0(x)$ for each $i\in[n]$. Therefore, via coupling arguments,
    \[
        \Pr[\mu_{i_0}]{\ol{\+E_{i_0}}\cap \ol{\+E_{j_0}}} = \Pr[\mu_{j_0}]{\ol{\+E_{i_0}}\cap \ol{\+E_{j_0}}} = \Pr[\mu_0]{\ol{\+E_{i_0}}\cap \ol{\+E_{j_0}}}> 1-\frac{2\eps}{3}.
    \]

    Let $\+E=\+E_{i_0}\cup \+E_{j_0}$. Let $\tilde \mu_0^{\ol{\+E}}$ be the output distribution of $\+A$ with input distribution $\mu_0$ when $\ol{\+E}$ happens. 
    Since
    \begin{align*}
        4\eps\leq \DTV(\mu_{i_0},\mu_{j_0}) \leq \DTV(\mu_{i_0},\tilde \mu_0^{\ol{\+E}}) + \DTV(\mu_{j_0},\tilde \mu_0^{\ol{\+E}}),
    \end{align*}
    we have either $\DTV(\mu_{i_0},\tilde \mu_0^{\ol{\+E}})>2\eps$ or $\DTV(\mu_{j_0},\tilde \mu_0^{\ol{\+E}})>2\eps$. W.l.o.g., assume $\DTV(\mu_{i_0},\tilde \mu_0^{\ol{\+E}})>2\eps$. Assume the output distribution of $\+A$ is $\tilde \mu_{i_0}$ when the input is $\mu_{i_0}$. Let $\tilde \mu_{i_0}^{\+E}$ and $\tilde \mu_{i_0}^{\ol{\+E}}$ be $\tilde \mu_{i_0}$ conditioned on $\+E$ and $\ol{\+E}$ respectively and denote their density functions as $\tilde p_{i_0}^{\+E}$, $\tilde p_{i_0}^{\ol{\+E}}$ and $\tilde p_{i_0}$. Then $\tilde \mu_{i_0}^{\ol{\+E}}=\tilde \mu_0^{\ol{\+E}}$. 
    Then we have
    \begin{align*}
        \DTV(\mu_{i_0},\tilde \mu_{i_0}) &= \frac{1}{2}\int_{\bb R^d} \abs{p_{i_0}(x) - \tilde p_{i_0} (x)} \dd x\\
        &=\frac{1}{2}\int_{\bb R^d} \abs{p_{i_0}(x) - \Pr[\mu_{i_0}]{\+E}\tilde p_{i_0}^{\+E}(x) - \Pr[\mu_{i_0}]{\ol{\+E}}\tilde p_{i_0}^{\ol{\+E}}(x)} \dd x \\
        &\geq \Pr[\mu_{i_0}]{\ol{\+E}}\cdot \frac{1}{2} \int_{\bb R^d}\abs{p_{i_0}(x) - \tilde p_{i_0}^{\ol{\+E}}(x)} \dd x  - \Pr[\mu_{i_0}]{\+E}\cdot \frac{1}{2} \int_{\bb R^d}\abs{p_{i_0}(x) - \tilde p_{i_0}^{\+E}(x)} \dd x \\
        & = \Pr[\mu_{i_0}]{\ol{\+E}}\cdot \DTV(\mu_{i_0}, \tilde \mu_{i_0}^{\ol{\+E}}) - \Pr[\mu_{i_0}]{\+E}\cdot \DTV(\mu_{i_0}, \tilde \mu_{i_0}^{\+E}) \\
        & = \Pr[\mu_{i_0}]{\ol{\+E}}\cdot \DTV(\mu_{i_0}, \tilde \mu^{\ol{\+E}}_0) - \Pr[\mu_{i_0}]{\+E}\cdot \DTV(\mu_{i_0}, \tilde \mu_{i_0}^{\+E}) \\
        &\geq \tp{1-\frac{2\eps}{3}}\cdot 2\eps - \frac{2\eps}{3}\cdot 1\\
        &>\eps.
    \end{align*}
    This means, on instance $\mu_{i_0}$, the algorithm $\+A$ will fail to output a distribution which is $\eps$-close to $\mu_{i_0}$ in total variation distance.
\end{proof}




\paragraph{Acknowledgements} I am thankful to Christine Cheng for reigniting my interest in the stable roommates problem and to David Manlove for his suggestions to improve this manuscript.

\bibliographystyle{plain}
\bibliography{references}


\end{document}
