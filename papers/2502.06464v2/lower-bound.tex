\section{The Lower Bound}\label{sec:lb}

In this section we give the main argument for our lower bound for SR solvability. As described above, the idea is to give an embedding of the two party set disjointness problem into SR. The embedding will be as follows. For any positive integer $n$, we will construct SR instances with $4 n$ agents. To this end, let $N = n^2$. To simplify notation in the construction, we index bits of the inputs $x, y$ with two indices from the range $[n]$, i.e., $x = (x_{ij})$ with $1 \leq i, j \leq n$.

Given any pair of Boolean vectors $x, y$, we will define an SR instance $R(x, y)$ as follows. $R(x, y)$ contains $4 n$ agents partitioned into $4$ sets, each of size $n$: $S = A \cup B \cup C \cup D$. We will write the agents in set $A$ as $A = \set{a_1, a_2, \ldots, a_n}$, and similarly with $B$, $C$, and $D$. The preferences of the agents in set $A$ are determined from $x$, and the preferences of agents of $B$ are determined from $y$. The preferences of agents in sets $C$ and $D$ are fixed independent of $x$ and $y$. The preferences are determined as follows.

\begin{figure}[p]
    \centering
    \begin{subfigure}{0.45\textwidth}
        \begin{center}
            \begin{tikzpicture}
                % Define a uniform style for all nodes
                \tikzstyle{uniformNode} = [thick, circle, minimum size=0.75cm, inner sep=0pt, line width=1.5pt]
                % Define the positions of the nodes
                \foreach \i in {1,2,3} {
                    \node[draw=rb-blue, style=uniformNode] (a\i) at (0, -\i) {$a_\i$};
                    \node[draw=rb-red, style=uniformNode] (b\i) at (4, -\i) {$b_\i$};
                    \node[draw=rb-blue-pastel, style=uniformNode] (c\i) at (0, -\i-4) {$c_\i$};
                    \node[draw=rb-violet-pastel, style=uniformNode] (d\i) at (4, -\i-4) {$d_\i$};
                }
                % Add edges from each c\i to a\i that are bent to the left
                \foreach \i in {1,2,3} {
                    \draw[draw=rb-blue-pastel, thick, bend left=45, line width=1.5pt] (c\i) to (a\i);
                    \draw[draw=rb-violet-pastel, thick, bend right=45, line width=1.5pt] (d\i) to (b\i);
                }

                \draw[draw=rb-blue, thick, line width=1.5pt] (a1) to (b1);
                \draw[draw=rb-blue, thick, line width=1.5pt] (a1) to (b2);
                \draw[draw=rb-blue, thick, line width=1.5pt] (a3) to (b2);

                \draw[draw=rb-red, thick, line width=1.5pt] (b2) to (c2);
                \draw[draw=rb-red, thick, line width=1.5pt] (b3) to (c3);

                \draw[draw=rb-red, thick, line width=1.5pt] (b2) to (a2);
                \draw[draw=rb-red, thick, line width=1.5pt] (b3) to (a3);
            \end{tikzpicture}
        \end{center}
        \caption{Initial preferred pairs in the table.}
        %\label{fig:img1}
    \end{subfigure}
    \hfill
    \begin{subfigure}{0.45\textwidth}
        \begin{center}
            \begin{tikzpicture}
                % Define a uniform style for all nodes
                \tikzstyle{uniformNode} = [thick, circle, minimum size=0.75cm, inner sep=0pt, line width=1.5pt]
                % Define the positions of the nodes
                \foreach \i in {1,2,3} {
                    \node[draw=rb-blue, style=uniformNode] (a\i) at (0, -\i) {$a_\i$};
                    \node[draw=rb-red, style=uniformNode] (b\i) at (4, -\i) {$b_\i$};
                    \node[draw=rb-blue-pastel, style=uniformNode] (c\i) at (0, -\i-4) {$c_\i$};
                    \node[draw=rb-violet-pastel, style=uniformNode] (d\i) at (4, -\i-4) {$d_\i$};
                }
                % Add edges from each c\i to a\i that are bent to the left
                \foreach \i in {1,2,3} {
                    \draw[draw=rb-blue-pastel, thick, bend left=45, line width=1.5pt] (c\i) to (a\i);
                    \draw[draw=rb-violet-pastel, thick, bend right=45, line width=1.5pt] (d\i) to (b\i);
                }

                \draw[draw=rb-blue, thick, line width=1.5pt] (a1) to (b1);
                \draw[draw=rb-blue, thick, line width=1.5pt] (a1) to (b2);
                \draw[draw=rb-blue, thick, line width=1.5pt] (a3) to (b2);

                \draw[draw=rb-red, thick, line width=1.5pt] (b2) to (c2);
                \draw[draw=rb-red, thick, line width=1.5pt] (b3) to (c3);

                %\draw[draw=rb-red, thick, line width=1.5pt] (b2) to (a2);
                %\draw[draw=rb-red, thick, line width=1.5pt] (b3) to (a3);
            \end{tikzpicture}
        \end{center}
        \caption{Pairs removed after proposals from $C$.}
        %\label{fig:img2}
    \end{subfigure}\\

    \vspace{1cm}
    
    \begin{subfigure}{0.45\textwidth}
        \begin{center}
            \begin{tikzpicture}
                % Define a uniform style for all nodes
                \tikzstyle{uniformNode} = [thick, circle, minimum size=0.75cm, inner sep=0pt, line width=1.5pt]
                % Define the positions of the nodes
                \foreach \i in {1,2,3} {
                    \node[draw=rb-blue, style=uniformNode] (a\i) at (0, -\i) {$a_\i$};
                    \node[draw=rb-red, style=uniformNode] (b\i) at (4, -\i) {$b_\i$};
                    \node[draw=rb-blue-pastel, style=uniformNode] (c\i) at (0, -\i-4) {$c_\i$};
                    \node[draw=rb-violet-pastel, style=uniformNode] (d\i) at (4, -\i-4) {$d_\i$};
                }
                % Add edges from each c\i to a\i that are bent to the left
                \foreach \i in {1,2,3} {
                    \draw[draw=rb-blue-pastel, thick, bend left=45, line width=1.5pt] (c\i) to (a\i);
                    \draw[draw=rb-violet-pastel, thick, bend right=45, line width=1.5pt] (d\i) to (b\i);
                }

                %\draw[draw=rb-blue, thick, line width=1.5pt] (a1) to (b1);
                %\draw[draw=rb-blue, thick, line width=1.5pt] (a1) to (b2);
                %\draw[draw=rb-blue, thick, line width=1.5pt] (a3) to (b2);

                \draw[draw=rb-red, thick, line width=1.5pt] (b2) to (c2);
                \draw[draw=rb-red, thick, line width=1.5pt] (b3) to (c3);

                %\draw[draw=rb-red, thick, line width=1.5pt] (b2) to (a2);
                %\draw[draw=rb-red, thick, line width=1.5pt] (b3) to (a3);
            \end{tikzpicture}
        \end{center}
        \caption{Pairs removed after proposals from $D$.}
        %\label{fig:img3}
    \end{subfigure}
    \hfill
    \begin{subfigure}{0.45\textwidth}
        \begin{center}
            \begin{tikzpicture}
                % Define a uniform style for all nodes
                \tikzstyle{uniformNode} = [thick, circle, minimum size=0.75cm, inner sep=0pt, line width=1.5pt]
                % Define the positions of the nodes
                \foreach \i in {1,2,3} {
                    \node[draw=rb-blue, style=uniformNode] (a\i) at (0, -\i) {$a_\i$};
                    \node[draw=rb-red, style=uniformNode] (b\i) at (4, -\i) {$b_\i$};
                    \node[draw=rb-blue-pastel, style=uniformNode] (c\i) at (0, -\i-4) {$c_\i$};
                    \node[draw=rb-violet-pastel, style=uniformNode] (d\i) at (4, -\i-4) {$d_\i$};
                }
                % Add edges from each c\i to a\i that are bent to the left
                \foreach \i in {1,2,3} {
                    \draw[draw=rb-blue-pastel, thick, bend left=45, line width=1.5pt] (c\i) to (a\i);
                    \draw[draw=rb-violet-pastel, thick, bend right=45, line width=1.5pt] (d\i) to (b\i);
                }

                %\draw[draw=rb-blue, thick, line width=1.5pt] (a1) to (b1);
                %\draw[draw=rb-blue, thick, line width=1.5pt] (a1) to (b2);
                %\draw[draw=rb-blue, thick, line width=1.5pt] (a3) to (b2);

                %\draw[draw=rb-red, thick, line width=1.5pt] (b2) to (c2);
                %\draw[draw=rb-red, thick, line width=1.5pt] (b3) to (c3);

                %\draw[draw=rb-red, thick, line width=1.5pt] (b2) to (a2);
                %\draw[draw=rb-red, thick, line width=1.5pt] (b3) to (a3);
            \end{tikzpicture}
        \end{center}
        \caption{Remaining pairs after proposals from $A$.}
        %\label{fig:img4}
    \end{subfigure}
    \caption{An illustration of the embedding of disjointness for $N = 3 \times 3$ and $n = 3$. This instance corresponds to $x_{1,1} = x_{1,2} = x_{3, 2} = 1$, while the remaining values of $x_{ij} = 0$, and $y_{2,2} = y_{3,3} = 1$ with the remaining values of $y_{ij} = 0$. Thus, $\disj(x, y) = 1.$ The edges between agents are colored by agent preferences: the dark blue edges from the $a_i$ correspond to their most preferred partners (preferred above $c_i$). Similarly, the dark red edges from the $b_j$ correspond to their most preferred edges. The light blue edges from the $c_i$ and violet edges from the $d_j$ correspond to those agent's most preferred partners. Other possible partners are not depicted. Sub-figure~(a) represents the remaining pairs in the preference table before the first rounds of proposals, while (b), (c), and (d) depict the remaining pairs after each round of proposals. The final figure depicts the unique stable matching for this instance.}
    \label{fig:disjoint}
\end{figure}

\begin{multicols}{2}
\begin{itemize}
    \item An agent $a_i \in A$ prefers in order
    \begin{enumerate}
        \item all $b_j$ with $x_{i j} = 1$ (arbitrary order)
        \item $c_i$
        \item all other agents in arbitrary order
    \end{enumerate}
    \item An agent $b_j \in B$ prefers in order
    \begin{enumerate}
        \item all $c_i$ with $y_{i j} = 1$ (arbitrary order)
        \item all $a_i$ with $y_{i j} = 1$ (arbitrary order)
        \item $d_j$
        \item all other agents in arbitrary order
    \end{enumerate}
    \columnbreak

    \item An agent $c_i \in C$ prefers in order
    \begin{enumerate}
        \item $a_i$
        \item all agents $b_j \in B$ in arbitrary order
        \item all other agents in arbitrary order
    \end{enumerate}
    \item An agent $d_j \in D$ prefers in order
    \begin{enumerate}
        \item $b_j$
        \item all other agents in arbitrary order
    \end{enumerate}
    \vfill
\end{itemize}
\end{multicols}

With the SR instances $R(x, y)$ defined as above, we state the main property regarding the solvability of $R(x, y)$.

\begin{prop}
    \label{prop:embedding}
    Suppose $x, y \in \set{0, 1}^{n \times n}$, and let $R(x, y)$ be the SR instance defined above. 
    \begin{enumerate}
        \item If $\disj(x, y) = 1$ (i.e., $x$ and $y$ are disjoint), then $R(x, y)$ admits a unique stable matching $M$, namely 
        \begin{equation}
            M = \set{\set{a_i, c_i} \sucht i \in [n]} \cup \set{\set{b_j, d_j} \sucht j \in [n]}
        \end{equation}
        \item If $x$ and $y$ are uniquely intersecting (i.e., $\sum_{i,j} x_{ij} y_{ij} = 1$), and hence $\disj(x, y) = 0$, then $R(x, y)$ is not solvable.
    \end{enumerate}
    In particular, the association $(x, y) \mapsto R(x, y)$ is an embedding of disjointness into SR solvability.
\end{prop}
\begin{proof}
    To prove the proposition, we consider an execution of the Phase~1 algorithm, Algorithm~\ref{alg:phase-1}. We first consider the case where $\disj(x, y) = 1$ (case~1 of the proposition). Consider an execution of Algorithm~\ref{alg:phase-1} with the agents proposing in the following order. By Lemma~\ref{lem:phase-1}, the resulting preference table does not depend on the order of the proposals. We illustrate an example of the execution of such an instance with $n = 3$ in Figure~\ref{fig:disjoint}.
    \begin{enumerate}
        \item Agents in $C$ propose. Specifically, each $c_i \in C$ proposes to $a_i$. Since $a_i$ receives a proposal from $c_i$, in Line~\ref{ln:remove}, only $b_j$s with $x_{ij} = 1$ will remain on $a$'s preference list (above $c_i$). Since $\disj(x, y) = 1$, $x_{ij} = 1$ implies $y_{ij} = 0$, hence all pairs $\set{a_i, b_j}$ with $y_{ij} = 1$ are removed.
        \item Agents in $D$ propose. Specifically, each $d_j \in D$ proposes to $b_j$. When $b_j$ receives the proposal from $d_j$, only $c_i$s with $y_{ij} = 1$ remain on $b_j$'s preference list, as all $a_i$'s with $y_{ij} = 1$ were removed after the proposals in Step~1. Since $\disj(x, y) = 1$, if $y_{ij} = 1$, then $x_{ij} = 0$. Therefore, all remaining pairs of the form $\set{a_i, b_j}$ are removed from the preference table.
        \item Agents in $A$ propose. At this point, each $a_i$'s preference list consists of a single entry, $c_i$, so $a_i$ proposes to $c_i$. Since $a_i$ is first on $c_i$'s preference list, all remaining pairs involving $c_i$ are removed from the preference table. In particular, all pairs of the form $\set{c_i, b_j}$ are removed.
        \item Agents in $B$ propose. At this point, each $b_j$'s preference list consists of a single entry, $d_j$, so $b_j$ proposes to $d_j$.
    \end{enumerate}
    After agents from $B$ propose, all agents are semiengaged so the algorithm terminates. The preference table consists only of the matching $M$ in the proposition's statement, which is therefore a stable matching (by Lemma~\ref{lem:phase-1}).
    
    Now consider the case where $x$ and $y$ are uniquely intersecting. Let $k, \ell \in [n]$ be the indices for which $x_{k \ell} = y_{k \ell} = 1$, while $x_{i j} y_{i j} = 0$ for all $(i, j) \neq (k, \ell)$. As above, we consider an execution of Algorithm~\ref{alg:phase-1}. We illustrate an example of the execution of such an instance in Figure~\ref{fig:intersecting}.
    \begin{enumerate}
        \item Agents in $C$ propose. In this case, pairs $\set{a_i, b_j}$ with $y_{ij} = 1$ are removed from the preference table except for $\set{a_k, b_\ell}$. Further, all pairs $\set{a_i, d_j}$ are removed.
        \item Agents in $D$ propose. Now all pairs $\set{a_i, b_j}$ with $x_{ij} = 1$ are removed from the preference table except for $\set{a_k, b_\ell}$.
        \item Agents in $A$ propose. All $a_{i}$ with $i \neq k$ propose to $c_i$, while $a_k$ proposes to $b_\ell$. Subsequently, $b_\ell$ rejects $d_\ell$. As each agent $c_i$ with $i \neq k$ receives a proposal from $a_i$, each $\set{c_i, d_\ell}$ is removed from the preference table.
        \item Agents in $B$ propose. In this case, each $b_j$ with $j \neq \ell$ proposes to $d_j$, while $b_\ell$ proposes to $c_k$. Subsequently, all pairs $\set{d_j, d_\ell}$ are removed from the preference table, as is the pair $\set{c_k, d_\ell}$.
    \end{enumerate}
    At this point all pairs involving $d_\ell$ have been removed, so $d_\ell$'s preference list is empty. Therefore, by Lemma~\ref{lem:phase-1}, $R(x, y)$ does not admit a stable matching.
\end{proof}

\begin{figure}[p]
    \centering
    \begin{subfigure}{0.45\textwidth}
        \begin{center}
            \begin{tikzpicture}
                % Define a uniform style for all nodes
                \tikzstyle{uniformNode} = [thick, circle, minimum size=0.75cm, inner sep=0pt, line width=1.5pt]
                % Define the positions of the nodes
                \foreach \i in {1,2,3} {
                    \node[draw=rb-blue, style=uniformNode] (a\i) at (0, -\i) {$a_\i$};
                    \node[draw=rb-red, style=uniformNode] (b\i) at (4, -\i) {$b_\i$};
                    \node[draw=rb-blue-pastel, style=uniformNode] (c\i) at (0, -\i-4) {$c_\i$};
                    \node[draw=rb-violet-pastel, style=uniformNode] (d\i) at (4, -\i-4) {$d_\i$};
                }
                % Add edges from each c\i to a\i that are bent to the left
                \foreach \i in {1,2,3} {
                    \draw[draw=rb-blue-pastel, thick, bend left=45, line width=1.5pt] (c\i) to (a\i);
                    \draw[draw=rb-violet-pastel, thick, bend right=45, line width=1.5pt] (d\i) to (b\i);
                }

                \draw[draw=rb-blue, thick, line width=1.5pt] (a1) to (b1);
                \draw[draw=black, thick, line width=1.5pt] (a2) to (b2);
                \draw[draw=rb-blue, thick, line width=1.5pt] (a3) to (b2);

                \draw[draw=rb-red, thick, line width=1.5pt] (b2) to (c2);
                \draw[draw=rb-red, thick, line width=1.5pt] (b3) to (c3);

                %\draw[draw=rb-red, thick, line width=1.5pt] (b2) to (a2);
                \draw[draw=rb-red, thick, line width=1.5pt] (b3) to (a3);
            \end{tikzpicture}
        \end{center}
        \caption{Initial preferred pairs in the table.}
        %\label{fig:img1}
    \end{subfigure}
    \hfill
    \begin{subfigure}{0.45\textwidth}
        \begin{center}
            \begin{tikzpicture}
                % Define a uniform style for all nodes
                \tikzstyle{uniformNode} = [thick, circle, minimum size=0.75cm, inner sep=0pt, line width=1.5pt]
                % Define the positions of the nodes
                \foreach \i in {1,2,3} {
                    \node[draw=rb-blue, style=uniformNode] (a\i) at (0, -\i) {$a_\i$};
                    \node[draw=rb-red, style=uniformNode] (b\i) at (4, -\i) {$b_\i$};
                    \node[draw=rb-blue-pastel, style=uniformNode] (c\i) at (0, -\i-4) {$c_\i$};
                    \node[draw=rb-violet-pastel, style=uniformNode] (d\i) at (4, -\i-4) {$d_\i$};
                }
                % Add edges from each c\i to a\i that are bent to the left
                \foreach \i in {1,2,3} {
                    \draw[draw=rb-blue-pastel, thick, bend left=45, line width=1.5pt] (c\i) to (a\i);
                    \draw[draw=rb-violet-pastel, thick, bend right=45, line width=1.5pt] (d\i) to (b\i);
                }

                \draw[draw=rb-blue, thick, line width=1.5pt] (a1) to (b1);
                \draw[draw=black, thick, line width=1.5pt] (a2) to (b2);
                \draw[draw=rb-blue, thick, line width=1.5pt] (a3) to (b2);

                \draw[draw=rb-red, thick, line width=1.5pt] (b2) to (c2);
                \draw[draw=rb-red, thick, line width=1.5pt] (b3) to (c3);

                %\draw[draw=rb-red, thick, line width=1.5pt] (b2) to (a2);
                %\draw[draw=rb-red, thick, line width=1.5pt] (b3) to (a3);
            \end{tikzpicture}
        \end{center}
        \caption{Pairs removed after proposals from $C$.}
        %\label{fig:img2}
    \end{subfigure}\\

    \vspace{1cm}
    
    \begin{subfigure}{0.45\textwidth}
        \begin{center}
            \begin{tikzpicture}
                % Define a uniform style for all nodes
                \tikzstyle{uniformNode} = [thick, circle, minimum size=0.75cm, inner sep=0pt, line width=1.5pt]
                % Define the positions of the nodes
                \foreach \i in {1,2,3} {
                    \node[draw=rb-blue, style=uniformNode] (a\i) at (0, -\i) {$a_\i$};
                    \node[draw=rb-red, style=uniformNode] (b\i) at (4, -\i) {$b_\i$};
                    \node[draw=rb-blue-pastel, style=uniformNode] (c\i) at (0, -\i-4) {$c_\i$};
                    \node[draw=rb-violet-pastel, style=uniformNode] (d\i) at (4, -\i-4) {$d_\i$};
                }
                % Add edges from each c\i to a\i that are bent to the left
                \foreach \i in {1,2,3} {
                    \draw[draw=rb-blue-pastel, thick, bend left=45, line width=1.5pt] (c\i) to (a\i);
                    \draw[draw=rb-violet-pastel, thick, bend right=45, line width=1.5pt] (d\i) to (b\i);
                }

                %\draw[draw=rb-blue, thick, line width=1.5pt] (a1) to (b1);
                \draw[draw=black, thick, line width=1.5pt] (a2) to (b2);
                %\draw[draw=rb-blue, thick, line width=1.5pt] (a3) to (b2);

                \draw[draw=rb-red, thick, line width=1.5pt] (b2) to (c2);
                \draw[draw=rb-red, thick, line width=1.5pt] (b3) to (c3);

                %\draw[draw=rb-red, thick, line width=1.5pt] (b2) to (a2);
                %\draw[draw=rb-red, thick, line width=1.5pt] (b3) to (a3);
            \end{tikzpicture}
        \end{center}
        \caption{Pairs removed after proposals from $D$.}
        %\label{fig:img3}
    \end{subfigure}
    \hfill
    \begin{subfigure}{0.45\textwidth}
        \begin{center}
            \begin{tikzpicture}
                % Define a uniform style for all nodes
                \tikzstyle{uniformNode} = [thick, circle, minimum size=0.75cm, inner sep=0pt, line width=1.5pt]
                % Define the positions of the nodes
                \foreach \i in {1,2,3} {
                    \node[draw=rb-blue, style=uniformNode] (a\i) at (0, -\i) {$a_\i$};
                    \node[draw=rb-red, style=uniformNode] (b\i) at (4, -\i) {$b_\i$};
                    \node[draw=rb-blue-pastel, style=uniformNode] (c\i) at (0, -\i-4) {$c_\i$};
                    \node[draw=rb-violet-pastel, style=uniformNode] (d\i) at (4, -\i-4) {$d_\i$};
                }
                % Add edges from each c\i to a\i that are bent to the left
                \foreach \i in {1,2,3} {
                    \draw[draw=rb-blue-pastel, thick, bend left=45, line width=1.5pt] (c\i) to (a\i);
                }                
                \foreach \i in {1,3} {
                    \draw[draw=rb-blue-pastel, thick, bend left=45, line width=1.5pt] (c\i) to (a\i);
                    \draw[draw=rb-violet-pastel, thick, bend right=45, line width=1.5pt] (d\i) to (b\i);
                }

                %\draw[draw=rb-blue, thick, line width=1.5pt] (a1) to (b1);
                \draw[draw=black, thick, line width=1.5pt] (a2) to (b2);
                %\draw[draw=rb-blue, thick, line width=1.5pt] (a3) to (b2);

                \draw[draw=rb-red, thick, line width=1.5pt] (b2) to (c2);
                %\draw[draw=rb-red, thick, line width=1.5pt] (b3) to (c3);

                %\draw[draw=rb-red, thick, line width=1.5pt] (b2) to (a2);
                %\draw[draw=rb-red, thick, line width=1.5pt] (b3) to (a3);
            \end{tikzpicture}
        \end{center}
        \caption{Remaining pairs after proposals from $A$.}
        %\label{fig:img4}
    \end{subfigure}
    \caption{An illustration of the embedding of disjointness for $N = 3 \times 3$ and $n = 6$. This instance corresponds to $x_{1,1} = x_{2,2} = x_{3, 2} = 1$, while the remaining values of $x_{ij} = 0$, and $y_{2,2} = y_{3,3} = 1$ with the remaining values of $y_{ij} = 0$. Thus, $\disj(x, y) = 1$ with $x_{2,2} = y_{2,2} = 1$. Sub-figure~(a) represents the remaining pairs in the preference table before the first rounds of proposals, while (b), (c), and (d) depict the remaining pairs after each round of proposals. Note that in all of the figures, the pair $\set{a_2, b_2}$ is preferred by $a_2$ to $c_2$ and preferred by $b_2$ to $d_2$. Therefore, this edge is not removed after either the $C$ proposals nor the $D$ proposals in figures~(b) and~(c). Subsequently, when agents in $A$ propose, $a_2$ proposes to $b_2$, after which $b_2$ rejects $d_2$. At this point $d_2$'s preference list is empty, hence the instance does not admit a stable matching.}
    \label{fig:intersecting}
\end{figure}


We now state and prove our main lower bound for the stable roommates problem. In the formal statement, we allow for any procedure that performs arbitrary adaptive Boolean queries to the agents' preference lists, and queries can even be made to ``batches'' of agents---i.e., a fixed partition of the agents---so long as no batch contains more than $n/2$ agents. For example, this allows queries of the form, ``Does any agent in set $A$ prefer agent $b$ to $b'$?'' Specifically, our argument holds for any Boolean query that does not involve agents from \emph{both} sets $A$ and $B$ in the construction above. An upper bound on the batch size is necessary allowing a batch size of $n$ would allow for the query ``Does $S$ admit a stable matching?'' as a single query.

\begin{thm}\label{thm:main-lb}
    Any randomized (or deterministic) mechanism that decides SR solvability on instances with $n$ agents using adaptive Boolean query access to the agents' preferences requires $\Omega(n^2)$ Boolean queries in expectation. This bound applies even if queries can be made to fixed batches of agents of size up to $n/2$.
\end{thm}
\begin{proof}
    Let $\calA$ be any algorithm that determines SR solvability using $q$ queries for SR instances of size $n$. Then we can use $\calA$ to define a two party communication protocol for $\disj$ using the embedding of Proposition~\ref{prop:embedding} as follows. Suppose Alice and Bob hold inputs $x, y \in \set{0,1}^{n \times n}$, respectively, which are promised to be either disjoint or uniquely intersecting. Note that the set disjointness instance has size $N = n^2$. Alice forms preferences of agents in the set $A$ as in the embedding, and Bob does the same for $B$. Thus, the instance has $4n$ agents in total. Note that the preferences of agents in $C$ and $D$ are independent of $x$ and $y$, hence they are known to both Alice and Bob. Alice and Bob then simulate the algorithm $\calA$: the response to any query that $\calA$ makes to agents in $A \cup C \cup D$ can be computed by Alice, who then sends the Boolean result to Bob as a single bit. Similarly, the response to any query made to $B \cup C \cup D$ can be computed by Bob, who then sends the response to Alice. When $\calA$ terminates after $q$ queries, both Alice and Bob can compute the output of $\calA$ from the communication transcript, hence they both know whether or not the $R(x, y)$ is solvable. By Proposition~\ref{prop:embedding}, this output determines $\disj(x, y)$.

    Since this communication protocol solves an arbitrary instance of $\disj$ (with the unique intersection promise) with input size $n^2$ using $q$ bits of communication, we have $q = \Omega(n^2)$ by Theorem~\ref{thm:disjointness}.
\end{proof}

Finally, we argue that the query lower bound of Theorem~\ref{thm:main-lb} implies the computational lower bounds listed in Corollary~\ref{cor:lb}.

\begin{proof}[Proof of Corollary~\ref{cor:lb} (Sketch)]
    The lower bound for Turing machines follows from the observation that each ``read'' operation performed by a Turing machine can be modelled as a response to a single Boolean query (the value of the bit that is read). Thus, the query lower bound implies that any Turing machine the decides SR solvability must read $\Omega(n^2)$ bits of its input, hence its running time is $\Omega(n^2)$. 
    
    Similarly, for random access machines (RAMs), each memory access to a word of size $O(\log n)$ bits can be simulated by $O(\log n)$ Boolean queries. Hence we obtain a $\Omega(n^2/\log n)$ lower bound on the number of memory accesses.

    We observe that the arguments above make no reference to the representation of the input to the problem, and the query lower bound argument holds for any Boolean queries made to the input so long as a single query's response does not depend on the preferences of agents from both $A$ and $B$. Following an arbitrary preprocessing of the agents' preferences where agents in $A$ are processed independently from agents in $B$, each bit of the resulting encoding is determined by either preferences in $A \cup C \cup D$ or preferences in $B \cup C \cup D$. In the former case, reading a single bit of the processed input can be simulated with a single Boolean query to $A$ (as preferences in $C$ and $D$ are fixed) in the former case and $B$ in the latter case. Thus, the lower bounds apply to preprocessed inputs as well.
\end{proof}

% Finally, we state an immediate corollary of Theorem~\ref{thm:main-lb} for the running time of any algorithm that solves SR solvability. Once again, we allow for arbitrary preprocessing of batches of agents' preferences, subject to the batches being of size at most $n/4$. 

% \begin{cor}
%     Any randomized or deterministic algorithm that solves SR solvability requires $\Omega(n^2)$ accesses to the agents' preferences on instances with $n$ agents in the worst case. This bound holds even if the agents' preferences are represented or preprocessed in an arbitrary manner, so long as as the preprocessing is applied independently to batches of size at most $n/4$.
% \end{cor}

% This batch size bound is necessary, as without it, one could simply record the satisfiability of the instance with a single bit as part of the encoded instance, yielding an $O(1)$ time algorithm for satisfiability.




