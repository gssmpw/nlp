\section{Introduction}

Since its formalization by Gale and Shapley in 1962~\cite{GS62}, the Stable Marriage Problem (SM) and its variants has become a central topic in economics, computer science, and mathematics. The rich theoretical landscape of stable allocation problems and their diverse applications gained further widespread recognition with the 2012 Nobel Prize in Economics, which was awarded to Roth and Shapley for their contributions to this field. In Gale and Shapley's original formulation of the problem, an SM instance consists of two disjoint sets of agents, traditionally referred to as men and women. Each agent has preferences in the form of a ranked list of agents of the opposite gender. The goal is to find a marriage (i.e., a one-to-one correspondence) between men and women that is stable in the sense that no man-woman pair have a mutual incentive to deviate from the marriage. The seminal result of Gale and Shapley proves that a stable marriage always exists and describes an efficient algorithm for finding one.

In~\cite{GS62}, Gale and Shapley also describe a generalization of SM in which there is no bipartition of the agent set into men and women. In this variant, each agent ranks all other agents---not just those of the opposite gender. Again, the goal is to find a stable matching: a one-to-one correspondence between agents such that no pair has a mutual incentive to deviate from their assigned partners (see Section~\ref{sec:sr} for formal definitions). This generalization of SM is referred to as the Stable Roommates Problem (SR).

Since its introduction by Gale and Shapley, SR has been the subject of a large volume of theoretical work and SR has found many applications. As the problem's name suggests, SR naturally models situations in which ``peers'' should be matched with one another, such as identifying roommates in housing searches~\cite{Perach2008-stable}. Variants of SR can be used to model pair-wise kidney exchanges for organ donation~\cite{Roth2005-pairwise}. The general study of this problem has led to the development of algorithms to facilitate kidney exchange markets for organ donation in several countries~\cite{PairedDonation, LivingKidneyDonation, deKlerka2005-dutch}. SR algorithms have also been employed for peer-to-peer file sharing networks~\cite{Lebedev2007-p2p, Mathieu2010-acyclic}, and forming pairs of players in chess tournaments~\cite{Kujansuu1999-stable}. We refer the readers to~\cite{Manlove2013-algorithmics} for a thorough overview of SR and applications.

Unlike the bipartite SM---where stable marriages are guaranteed to exist for all preferences---Gale and Shapley observed that SR instances need not admit stable matchings (see Section~\ref{sec:unsolvable} for an explicit example). The question of whether or not a given instance admits a stable matching is known as the SR solvability problem. Gale and Shapley~\cite{GS62} and subsequently Knuth~\cite{Knuth76} asked whether there exists an efficient algorithm for SR solvability (and finding a stable matching if one exists), where Knuth suggested the problem might be NP-complete. In 1985, Irving~\cite{Irving1985-stable} devised an efficient algorithm for SR solvability. Irving's algorithm finds a stable matching or reports that none exists in $O(n^2)$ time on instances with $2n$ agents.\footnote{Observe that since the input consists of $2n$ lists each of length $2n - 1$, $O(n^2)$ is \emph{linear} in the input size.} In 1990, Ng and Hirschberg~\cite{HN90} showed that $\Omega(n^2)$ time is necessary to find a stable marriage (in the bipartite stable marriage problem). Their result implies the same lower bound holds for \emph{finding} a stable matching for SR, but the lower bound does not apply to the decision problem of determining whether or not an SR instance admits a stable matching. In their influential 1989 text on stable matching problems, Gusfield and Irving~\cite{GI89} listed finding a lower bound (or $o(n^2)$ time algorithm) for deciding SR solvability as one of 12 open questions for future research. This question was again listed as an open problem in Manlove's comprehensive 2012 text on algorithmic aspects of stable matchings~\cite{Manlove2013-algorithmics}. To our knowledge, the question has remained open since.

\subsection{Our Contributions}

 Our main result is to prove an $\Omega(n^2)$ lower bound for any (randomized or deterministic) algorithm deciding SR solvability. While Gusfield and Irving's question is phrased in terms of running time lower bounds for a particular representation of the agents' preferences, we give a more general result that bounds the number of Boolean queries necessary to decide SR solvability.

\begin{thm}\label{thm:informal}[Informal, c.f.\ Theorem~\ref{thm:main-lb}]
    Any algorithm that decides SR solvability requires $\Omega(n^2)$ Boolean queries to the agents' preferences for instances with $2n$ agents. This lower bound applies to randomized protocols (in expectation) and allows for arbitrary Boolean queries made to individual agents' preferences as well as queries made to predetermined batches of agents (i.e., queries that involve more than one agent's preferences).
\end{thm}

One advantage of phrasing our lower bound in terms of (arbitrary) Boolean queries is that the lower bound extends to many models of computation simultaneously and it is agnostic to the representation of the preferences.

\begin{cor}\label{cor:lb}
    The following lower bounds hold for deciding SR solvability of instances with $2n$ agents:
    \begin{enumerate}
        \item Any (multi-tape, probabilistic) Turing machine that decides SR solvability requires $\Omega(n^2)$ time (in expectation).
        \item Any (randomized) random access machine (RAM) with word size $O(\log n)$ bits that decides SR solvability requires $\Omega(n^2 / \log n)$ memory accesses (in expectation).
        %\item Any streaming algorithm for SR solvability requires $\Omega(n^2)$ memory.
    \end{enumerate}
    The lower bounds of~1 and~2 hold even if the agents' preferences are preprocessed arbitrarily in batches of size up to $n/2$ (where separate batches are preprocessed independently of each other).
\end{cor}

Another interpretation of our lower bound is for mechanisms that decide SR solvability by eliciting agents' \emph{implicit} preferences. In practice, agents' preferences may not be known explicitly, even to the agents themselves.\footnote{For example, consider the eponymous application of determining suitable roommates for housing. The question of determining if $a$ prefers potential roommate $b$ to $c$ may rely on knowing a significant amount of information about $b$ and $c$ that is not initially known to $a$. In cases of assigning university housing (where potential roommates may not know each other beforehand) responses to questionnaires are often used as a proxy for explicit preferences, with the hope that responses to the questions elicit sufficient information about the ``true'' preferences to determine roommate compatibility.} Thus, it is desirable to design a mechanism (such as questionnaires) that elicits sufficient information about the agents' preferences to determine SR solvability. Theorem~\ref{thm:informal} implies that any such mechanism requires $\Omega(n^2)$ Boolean queries. Thus, determining SR solvability is essentially as hard as learning the agents' preferences.

In order to prove Theorem~\ref{thm:informal}, we employ a reduction from the two-party communication complexity of the set disjointness function, $\disj$. This technique was previously applied by Gonczarowski et al.~\cite{Gonczarowski2019-stable} to obtain lower bounds for the (bipartite) SM. Given the framework of query lower bounds via communication complexity~\cite{Blais2012-property,Eden2018-lower}, our construction and argument are quite simple. The basic idea is to define a family of SR instances that correspond to inputs to the disjointness function. That is, given an input $(x, y)$ to $\disj$, we define an SR instance $R(x, y)$ such that $R(x, y)$ admits a stable matching if and only if $\disj(x, y) = 1$. We then argue that the correspondence $(x, y) \mapsto R(x, y)$ has the property that any algorithm that decides SR solvability using $q$ queries implies the existence of a communication protocol for $\disj$ using $q$ bits of communication. Our main result follows from well-known communication complexity lower bounds for $\disj$~\cite{KS92,Razborov92}. We describe this technique more thoroughly in Section~\ref{sec:communication} before defining the reduction formally in Section~\ref{sec:lb}.

\begin{rem}
    Our $\Omega(n^2)$ query lower bound is only tight to Irving's algorithm~\cite{Irving1985-stable} up to a logarithmic factor. This is because an SR instance of size $2n$ contains $2n$ lists of $2n - 1$ agents. If each agent's identity is encoded with $\Theta(\log n)$ bits, the total size of preference lists is $\Theta(n^2 \log n)$ bits. Irving's algorithm allows for access to a single entry on a preference list as a unit cost operation, whereas this operation would correspond to $\Theta(\log n)$ Boolean queries in our model. Thus, in our model, Irving's algorithm uses $\Theta(n^2 \log n)$ Boolean queries. We leave it as an open question whether the lower bound can be improved to $\Omega(n^2 \log n)$ Boolean queries (i.e., a lower bound that is truly linear in the instance size) or if the logarithmic factor in the running time can be improved.
\end{rem}

\subsection{Related Work}

SR was introduced by Gale and Shapley~\cite{GS62} who showed that SR instances need not admit stable matchings. Knuth~\cite{Knuth76} explicitly posed the question of whether there is an efficient algorithm to find a stable matching or report that none exists. Irving~\cite{Irving1985-stable} devised an algorithm that finds a stable matching or correctly determines that none exists in $O(n^2)$ time (assuming each preference list entry can be accessed in $O(1)$ time). Gusfield and Irving's influential book~\cite{GI89} gives a comprehensive overview of early algorithmic work on SR, including structural aspects of the set of stable matchings for an SR instance. Additionally,~\cite{GI89} listed two questions related to efficiently deciding SR solvability. The first question asked if it is possible to generate a ``succinct certificate'' for non-solvability of an SR instance. This question was positively answered by Tan~\cite{Tan1991-necessary} who demonstrated that a ``stable partition'' (of size $O(n)$) can be computed in $O(n^2)$ time. Given the partition, solvability of an SR instance can be verified in $O(n)$ time. Gusfield and Irving's second question asked if it is possible to decide SR solvability in $o(n^2)$ time directly. Our lower bound gives a negative answer to this question. Given Tan's result, our lower bound also implies that finding a stable partition requires $\Omega(n^2)$ time. Several previous works~\cite{HN90,CL10,Segal03,Gonczarowski2019-stable} proved $\Omega(n^2)$ lower bounds for \emph{finding} stable marriages in SM (and \emph{a fortiori} for finding stable matchings in SR), but none of these results imply lower bounds for the decision problem of SR solvability. We refer the reader to the text of Manlove~\cite{Manlove2013-algorithmics} for an account of more recent developments related to SR.

Our main lower bound result follows from a reduction from the two-party communication complexity of the disjointness function (see Section~\ref{sec:communication} for a formal definition). The two-party communication complexity model was introduced by Yao in~\cite{Yao79}. The (randomized) two-party communication complexity of the disjointness function was first characterized by Kalyanasundaram and Schintger~\cite{KS92}, and subsequently a simpler argument was found by Razborov~\cite{Razborov92}. 

Most closely related to the present paper is the work of Gonczarowski et al.~\cite{Gonczarowski2019-stable}. In~\cite{Gonczarowski2019-stable}, the authors apply the communication complexity of set disjointness to obtain lower bounds for finding and verifying (bipartite) stable marriages. Indeed, our techniques in this paper closely follow that of~\cite{Gonczarowski2019-stable}. Reductions from communication complexity to obtain (sublinear time) lower bounds were employed by~\cite{Blais2012-property} for property testing. The technique was further codified in the work of Eden and Rosenbaum~\cite{Eden2018-lower} who proved a variety of lower bounds for sublinear time algorithms. Our description of our lower bound in terms of an ``embedding'' of set disjointness follows the general approach and vocabulary of~\cite{Eden2018-lower}.