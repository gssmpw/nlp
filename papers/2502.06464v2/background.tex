\section{Background and Definitions}

\subsection{The Stable Roommates Problem}\label{sec:sr}

Formally, an instance of the Stable Roommates problem (hereafter, SR), consists of a set $S$ of $2n$ \dft{agents} together with \dft{preferences} for each agent in the form a ranked list of all other agents. We say that $a$ \dft{prefers} agent $b$ to $c$ if $b$ appears before $c$ on $a$'s preference list. A matching $M = \set{\set{a_1, b_1}, \set{a_2, b_2}, \ldots, \set{a_n, b_n}}$ is a partition of $S$ into subsets of size $2$. Given an SR instance and a matching $M$, we say that a pair $\set{a_i, b_j} \notin M$ is a \dft{blocking pair} if $a_i$ prefers $b_j$ to $b_i$ and $b_j$ prefers $a_i$ to $a_j$. A matching that does not induce any blocking pairs is said to be \dft{stable}.

An SR instance is \dft{solvable} if it admits a stable matching. The \dft{SR Solvability} problem is the problem of deciding if a given SR instance is solvable.

An efficient ($O(n^2)$ time) algorithm for solving SR Solvability was proposed by Irving in~\cite{Irving1985-stable}. We will not require a description of the full algorithm, but our analysis will rely on an initial processing of the preference lists from Irving's algorithm. Our terminology and exposition of Irving's algorithm follow Gusfield and Irving's text~\cite[Chapter~4]{GI89}. 

The algorithm maintains a \dft{preference table} that initially lists all of the agents' preferences. As the algorithm proceeds, pairs of agents that cannot appear in any stable matching are removed. When a pair $\set{a, b}$ is removed from the table, $a$ is removed from $b$'s preference list and $b$ is removed from $a$'s preference list, while the relative order of other agents in the table are unchanged. The initial phase of Irving's algorithm, which we refer to as the Phase~1 algorithm (Algorithm~\ref{alg:phase-1}) proceeds as follows. Each agent is initially ``free.'' Free agents are selected in an arbitrary order to propose to the next remaining agent on their preference list. When $a$ proposes to $b$, $a$ becomes \dft{semiengaged} to $b$ and $a$ is no longer free. Upon receiving a proposal from $a$, $b$ rejects any agent currently semiengaged to $b$, and all pairs of agents $\set{a', b}$ such that $b$ prefers $a$ to $a'$ are removed from the preference table. This process continues until either all agents are semiengaged or all free agents have empty preference lists.

\begin{algorithm}
    \caption{Phase 1 Algorithm~\cite{Irving1985-stable,GI89} \label{alg:phase-1}}
    \begin{algorithmic}[1]
        \Procedure{PhaseOne}{}
        \State Assign each agent to be free
        \While{Some free agent $x$ has a nonempty list}
        \State $y \gets$ first agent on $x$'x preference list \Comment{$x$ proposes to $y$}
        \If{some $z$ is semiengaged to $y$}
        \State assign $z$ to be free \Comment{$y$ rejects $z$}
        \EndIf
        \State assign $x$ to be semiengaged to $y$
        \ForAll{successors $x'$ of $x$ on $y$'s list}
        \State remove $\set{x', y}$ from the preference table \label{ln:remove}
        \EndFor
        \EndWhile
        \EndProcedure
    \end{algorithmic}
\end{algorithm}

Irving showed that when the Phase~1 algorithm terminates, the resulting preference table has the following properties.

\begin{lem}[{Irving~\cite{Irving1985-stable}, c.f.~\cite[Section 4.2.2]{GI89}}]\label{lem:phase-1}
    Consider the preference table resulting from an execution of Algorithm~\ref{alg:phase-1} on an SR instance. Then:
    \begin{enumerate}
        \item The resulting preference table is independent of the order in which free agents propose (\cite[Lemma 4.2.1]{GI89}).
        \item Any pair $\set{a, b}$ that is not contained in the table is not contained in any stable matching. In particular, if any preference list in the table is empty, then the instance is not solvable (\cite[Lemma~4.2.3]{GI89})
        \item If $M$ is a matching such that all pairs in $M$ appear in the preference table, then no pair $\set{a, b}$ not appearing the preference table can block $M$. In particular, if the preference table consists only of a matching $M$, $M$ is the unique stable matching for the instance (c.f., \cite[Lemmas~4.2.1 and~4.2.4]{GI89})

    \end{enumerate}
\end{lem}

In our lower bound construction we will use only Lemma~\ref{lem:phase-1} to argue the (non)solvability of the required SR instances.

\subsubsection{An Unsolvable SR Instance}\label{sec:unsolvable}

In order to motivate the construction used to achieve our main result, it is helpful to understand the following non-solvable instance of SR.

\begin{eg}\label{eg:unsolvable}
    Consider agent set $S = \set{a, b, c, d}$ with the following preferences
    \begin{center}
        \begin{tabular}{c|ccc}
            agent & \multicolumn{3}{c}{preferences}\\
            \hline
            $a$ & $b$ & $c$ & $d$\\
            $b$ & $c$ & $a$ & $d$\\
            $c$ & $a$ & $b$ & $d$\\
            $d$ & \multicolumn{3}{c}{arbitrary}
        \end{tabular}
    \end{center}
    To see why this instance is not solvable, observe that agents $a$, $b$, and $c$ form a ``directed triangle'' where they all rank one another in the top two, but the most preferred agents form a cycle $a \to b \to c \to a$. Thus in any matching $M$, $d$'s partner in $M$ will form a blocking pair with one of the other two matched agents. For example, if $\set{a, d} \in M$, then $\set{a, c}$ forms a blocking pair.
\end{eg}

The idea of our lower bound construction is to define a family of SR instances in which there are many possible sub-instances homomorphic to Example~\ref{eg:unsolvable}. For this family, determining SR solvability is equivalent to finding if a given instance contains such a sub-instance. We argue that determining the existence of such a sub-instance requires $\Omega(n^2)$ accesses to the agent's preferences using a reduction from communication complexity.

\subsection{Communication Complexity and Embeddings}
\label{sec:communication}

Our proof of Theorem~\ref{thm:informal} employs a reduction from two-party communication complexity~\cite{Yao79}. In this computational model, the goal is to compute a value $f(x, y)$ where $f$ is some Boolean function, and $x, y \in \set{0, 1}^N$ are inputs. The inputs are distributed between two parties, traditionally referred to as Alice and Bob. Alice knows only $x$, while Bob knows only $y$. The goal is for Alice and Bob to both learn the value of $f(x, y)$ while exchanging the fewest possible number of bits. The (deterministic) \dft{communication complexity} of $f$ is defined to be the minimum number of bits needed for the worst-case input for any communication protocol that computes $f$. Randomized communication complexity is defined analogously using randomized communication protocols that are allowed to err with some small probability. We refer the reader to the text of Kushilevitz and Nisan~\cite{KN97} for a formal description of the communication models and a thorough exposition of fundamental results.

In this paper, we focus on the communication complexity of the \dft{disjointness function} defined by
\begin{equation}
    \disj(x, y) = 
    \begin{cases}
        1 &\text{ if } \sum_{i = 1}^N x_i y_i = 0\\
        0 &\text{ otherwise.}
    \end{cases}
\end{equation}


The (randomized) communication complexity of the disjointness function was first characterized by Kalyanasundaram and Schintger~\cite{KS92}, and a simpler argument was found by Razborov~\cite{Razborov92} soon after. 

\begin{thm}[{\cite{KS92,Razborov92}}]\label{thm:disjointness}
    Any (deterministic or randomized) communication protocol for $\disj$ requires $\Omega(N)$ bits of communication. This lower bound holds if inputs are restricted to be either disjoint ($\sum_i x_i y_i = 0$) or uniquely intersecting ($\sum_i x_i y_i = 1$).
\end{thm}

%[Explain promise problem more?]

Our strategy in obtaining our main result is to define a reduction from $\disj$ to SR solvability. That is, we define a function that maps each input $(x, y)$ for the two party disjointness function to an SR instance $R(x,y)$. This mapping $(x, y) \mapsto R(x, y)$ is an \dft{embedding} of set disjointness in the sense defined by Eden and Rosenbaum~\cite{Eden2018-lower}. This means that the function has the following properties:
\begin{enumerate}
    \item $R(x, y)$ is solvable if and only if $\disj(x, y) = 1$.
    \item The response to any allowed Boolean query to $R(x, y)$ can be computed by Alice (who knows only $x$) and Bob (who knows only $y$) using $O(1)$ bits of communication.
\end{enumerate}

Under a generalization of these conditions, Eden and Rosenbaum~\cite{Eden2018-lower} showed that communication lower bounds imply query lower bounds. We emphasize that this technique yields adaptive query lower bounds: future queries are allowed to depend on the responses to previous queries. Specializing their result to our setting gives the following result.

\begin{thm}[{c.f.~\cite{Eden2018-lower}}]\label{thm:communication-embedding}
    Suppose there exists an embedding from $\disj$ instances of size $N$ to SR solvability instances of size $n$ satisfying the conditions above. Then any randomized or deterministic protocol for SR solvability for instances of size $n$ must use $\Omega(N)$ adaptive queries in expectation.
\end{thm}

We elaborate on what the ``allowed queries'' for our embedding are. In general one cannot allow for arbitrary queries made to arbitrary sets of agents. Indeed, this would allow queries such as ``Is the instance solvable?'' which trivially solves SR solvability with a single query. At the other extreme, one could restrict queries of particular forms such as ``Does agent $a$ prefer agent $b$ to agent $c$?'' Our lower bound holds for arbitrary Boolean queries made to any fixed subsets of agents we call \dft{batches}, so long as each batch contains at most $n/4$ agents. This allows for arbitrary Boolean queries to individual agents' preferences, but also queries whose responses depend on multiple agents within a batch, such as ``Does any agent in batch $A$ prefer agent $b$ to agent $c$?''
