\clearpage
\appendix

\definecolor{lightgray}{gray}{0.95}
\definecolor{deepblue}{RGB}{70,130,180}
\definecolor{deepgray}{RGB}{119,136,153}
\lstdefinestyle{prompt}{
    basicstyle=\ttfamily\fontsize{7pt}{8pt}\selectfont,
    frame=none,
    breaklines=true,
    backgroundcolor=\color{lightgray},
    breakatwhitespace=true,
    breakindent=0pt,
    escapeinside={(*@}{@*)},
    numbers=none,
    numbersep=5pt,
    xleftmargin=5pt,
    aboveskip=2pt,
    belowskip=2pt,
}
\tcbset{
  aibox/.style={
    top=10pt,
    colback=white,
    % colframe=black,
    % colbacktitle=black,
    enhanced,
    center,
    % attach boxed title to top left={yshift=-0.1in,xshift=0.15in},
    % boxed title style={boxrule=0pt,colframe=white,},
  }
}
\newtcolorbox{AIbox}[2][]{aibox, title=#2,#1}



\section{Additional Experiments Details}

\paragraph{Backbones.} To thoroughly validate the effectiveness of ViDoRAG, we conducted experiments on various models across various baselines, including both closed-source and open-source models: GPT-4o, Qwen2.5-7B, Llama3.2-3B, Qwen2.5-VL-7B\cite{yang2024qwen2}, Llama3.2-Vision-90B. For OCR-based pipelines, we use PPOCR\cite{ma2019paddlepaddle} to recognize text within documents. Optionally, VLMs can also be employed for text recognition, as their OCR capabilities are quite strong.

\paragraph{Experimental Environments.}
We conducted our experiments on a server equipped with 8 A100 GPUs and 96 CPU cores. Open-source models require substantial computational resources.

\paragraph{Retrieval Implementation Details.} Due to the context length limitations of the model, we use the Top-$2K$ pages to fit the GMM and we restrict the output chunks of the GMM algorithm to be between $K/2$ and $K$, we set $K=10$ in practice. 

\section{More Details on Datasets}
\label{appendix:dataset_composition}

\subsection{Annotation Case}
\lstdefinestyle{mystyle}{
    language=Python,
    % commentstyle=\color{codegreen},
    % keywordstyle=\color{magenta},
    % numberstyle=\tiny\color{codegray},
    % stringstyle=\color{codepurple},
    basicstyle=\ttfamily \lst@ifdisplaystyle\tiny\fi,
    breakatwhitespace=false,
    breaklines=true,
    captionpos=b,
    keepspaces=true,
    numbers=left,
    numbersep=5pt,
    xleftmargin=0pt,  % 去除左侧缩进
    showspaces=false,  % 不显示空格
    showstringspaces=false,  % 不显示字符串中的空格
    showtabs=false,
    tabsize=2,
    columns=flexible,  % 使列宽自动调整
    moredelim=[is][\bfseries]{<highlight>}{</highlight>}
}

\begin{figure}[!h]
\begin{tcolorbox}[title={\textbf{\small Annotated Data Format}}]
{\small
\begin{lstlisting}[style=mystyle]
## JSON Format
{
    "uid": "04d8bb0db929110f204723c56e5386c1d8d21587_2",
    "query": "What is the temperature of Steam explosion of Pretreatment for Switchgrass and Sugarcane bagasse preparation?",
    "reference_answer": "195-205 Centigrade",
    "meta_info": {
        "file_name": "04d8bb0db929110f204723c586c1d8d21587.pdf",
        "reference_page": [
            10
        ], # may contain multiple pages
        "source_type": "2d_layout",
        "query_type": "Multi-Hop"
    }
}
\end{lstlisting}
}
\end{tcolorbox}
\caption{Annotation case in ViDoSeek.}
% \label{fig:annotated}
\end{figure}

\subsection{Details on ViDoSeek}
\paragraph{More Dataset Statistics.}
The statistical about ViDoSeek is presented in Table \ref{tab:data_statistic_slide}. We categorize queries from a logical reasoning perspective into single-hop and multi-hop. Text, Table, Chart and Layout represent different sources of reference.
\begin{table}[!h]
    \small
    \centering
    \caption{\textbf{Statistics of ViDoSeek.}}
    % \resizebox{1.0\textwidth}{!}{
    \label{tab:data_statistic}
    \begin{tabular}{lcc}
    \toprule
    \textsc{\textbf{Statistic}} & \textsc{\textbf{Number}}\\
    \midrule
    Total Questions & 1142\\
    \midrule
    Single-Hop & 645 \\
    Multi-Hop & 497 \\
    \midrule
    Pure Text & 80\\
    Chart & 157 \\
    Table & 175 \\
    Layout & 730\\
    \bottomrule
    \end{tabular}
% }
\end{table}
\paragraph{Dataset Difficulty.}
ViDoSeek sets itself apart with its heightened difficulty level, attributed to the multi-document context and the intricate nature of its content types, particularly the Layout category. The dataset contains both single-hop and multi-hop queries, presenting a diverse set of challenges. Consequently, ViDoSeek serves as a more comprehensive and demanding benchmark for RAG systems compared to previous works.

\subsection{Details on SlideVQA-Refined}
\paragraph{Dataset Statistics.}
We supplemented our experiments with the SlideVQA dataset to demonstrate the scalability of our method. 
SlideVQA categorizes queries from a logical reasoning perspective into single-hop and multi-hop. 
Non-span, single-span, and multi-span respectively refer to answers derived from a single information-dense sentence, reference information that is sparse but located on the same page, and reference information distributed across different pages.
The statistical information about dataset is presented in Table \ref{tab:data_statistic_slide}.

\begin{table}[!h]
    \small
    \centering
    \caption{\textbf{Statistics of SlideVQA-Refined.}}
    % \resizebox{1.0\textwidth}{!}{
    \label{tab:data_statistic_slide}
    \begin{tabular}{lcc}
    \toprule
    \textsc{\textbf{Statistic}} & \textsc{\textbf{Number}}\\
    \midrule
    Total Questions & 2020\\
    \midrule
    Single-Hop & 1486 \\
    Multi-Hop & 534 \\
    \midrule
    Non-Span & 358\\
    Single-Spin & 1347 \\
    Multi-Span & 315 \\
    \bottomrule
    \end{tabular}
% }
\end{table}
\paragraph{Dataset Difficulty.} The SlideVQA dataset focuses on evaluating the RAG system's ability to understand both visually sparse and visually dense information. When multi-hop questions involve reference information spread across different pages, it presents a significant challenge to the RAG system, further demonstrating the effectiveness of our approach.



\section{Data Construction Details}
\label{appendix:data_construction_pipeline}


To construct the ViDoSeek dataset, we developed a four-step pipeline to ensure that the queries meet our requirements. 
\paragraph{Step 1. Document Collecting.}
We collected English-language slides containing 25 to 50 pages, covering 12 domains such as economics, technology, literature, and geography, etc.

\paragraph{Step 2. Query Creation.}

To make the queries more suitable for RAG over a large-scale collection, our experts constructed queries based on the following requirements: (\romannumeral1) Each query must have a unique answer when paired with the document. (\romannumeral2) The query must include unique keywords that point to the specific document and pages. (\romannumeral3) The query should require external knowledge. Additionally, we encouraged constructing queries in various forms and with different sources and reasoning types to better reflect real-world scenarios. Our queries not only focus on types of references, including text, tables, charts, and layouts, but also provide a classification of reasoning types, including single-hop and multi-hop.

\paragraph{Step 3. Quality Review.}
To effectively evaluate the generation and retrieval quality of our RAG system, we require queries that yield unique answers, preferably located on a specific page or within a few pages. However, in large-scale retrieval and generation tasks, relying solely on manual annotation is challenging due to human cognitive limitations. To address this, we propose a review module that automatically identifies problematic queries. This module consists of two steps: (\romannumeral1) We prompt LLMs to filter out queries that may have multiple answers across the document collection; for example, the question \emph{What is the profit for this company in 2024?} might have a unique answer within a single document but could yield multiple answers in a multi-document setting. (\romannumeral2) For the remaining queries, we retrieve the top-\emph{k} slides for each query and use a VLM to determine whether each slide can answer the query. If only the golden page can answer the question, we consider it to meet the requirements. If pages other than the golden page can answer the query, we have experts manually evaluate and refine them.

\paragraph{Step 4. Multimodal Refine.}
% refine问题 同时保证问题中不含答案
In this final step, we refine the queries that did not meet our standards during the quality review. The goal is to adjust these queries so they satisfy the following requirements: (\romannumeral1) The refined query should point to specific pages within the large collection with minimal additional information; (\romannumeral2) The refined query must retain its original meaning. 
We use carefully designed VLM-based agents to assist us throughout the entire dataset construction pipeline. The prompt is presented in Fig. \ref{fig: reviewer} and Fig. \ref{fig: multi_reviewer}, respectively. We will first perform filtering based on semantics, and then conduct a fine-grained review using a multimodal reviewer.



\section{More Details about Multi-Agent Generation with Iterative Reasoning}
\label{appendix: gen}
We designed prompts to drive VLMs-based agents, and through our experiments, we found that some open-source models require the design of few-shot examples to learn specific thought patterns. See detailed prompts in Fig. \ref{fig: seeker}, Fig.\ref{fig: inspector} and Fig.\ref{fig: answer}.

\begin{figure*}[!ht] 
\begin{AIbox}{Query Reviewer Prompt.}
{\color{black}\bf \large System Prompt:} 
\vspace{1mm}
\\
\textbf{Task}  \\
I have some QA data here, and you can observe that the questions can be divided into two categories:\\
The category \#A: When you see this question alone without a given document, you are sure to find a unique document in a corpus to provide a unique answer. The question having some key words to help you locate the document from corpus.\\
The category \#B: When you see this question alone without a given document, you will find hard to locate a document to give a deterministic answer for this question, because you will find multiple candidate documents in a corpus, which may lead to different answers for this question. The question do not have any special key words to help you locate the document from corpus.

\textbf{Examples}\\
The number mentioned on the right of the leftside margin? \#B\\
What is the date mentioned in the second table? \#B\\
What is the full form of PUF? \#A\\
What is the number at the bottom of the page, in bold? \#B\\
Who presented the results on cabin air quality study in commercial aircraft? \#A\\
What is the name of the corporation? \#B\\
Which part of Virginia is this letter sent from? \#B\\
who were bothered by cigarette odors? \#A\\
which cigarette would be better if offered on a thicker cigarette? \#A\\
Cigarettes will be produced and submitted to O/C Panel for what purpose? \#A\\
What is the heading of first table? \#B\\
What is RIP-6 value for KOOL KS? \#A\\
Which test is used to evaluate ART menthol levels that has been shipped? \#A\\
How much percent had not noticed any difference in the odor of VSSS? \#A\\
What is the cigarette code of RIP-6(W/O Filter) 21/4SE? \#A\\
what mm Marlboro Menthol were subjectively smoked by the Richmond Panel? \#A\\
What are the steps of Weft Preparation between Spinning bobbin and Weaving? \#A\\
What level comes between Middle Managers and Non-managerial Employees? \#A\\
What are the six parts of COLLABORATION MODEL of the organization where James has a role of leading the UK digital strategy? \#A

\tcblower
{\color{black}\bf \large User Prompt:}\\
Query: {\color{deepblue}\bf \{Query Description\}} 
\end{AIbox}
\vspace{-1em}
\caption{Prompt of Query Reviewer.}
\label{fig: reviewer}
\end{figure*}


\begin{figure*}[!ht] 
\begin{AIbox}{Multi-Modal Reviewer Prompt.}
{\color{black}\bf \large System Prompt:} 
\vspace{1mm}
\\
Please check the image, tell me whether the image can answer my question.

\tcblower
{\color{black}\bf \large User Prompt:}\\
Query: {\color{deepblue}\bf \{Query Description\}}\\
Image: {\color{deepblue}\bf \{Relevant Image\}} 
\end{AIbox}
\vspace{-1em}
\caption{Prompt of Multi-Modal Reviewer.}
\label{fig: multi_reviewer}
\end{figure*}

\begin{figure*}[!ht] 
\begin{AIbox}{Multi-Modal Query Refiner Prompt.}
{\color{black}\bf \large System Prompt:} 
\vspace{1mm}
\\
\textbf{Task}  \\
Rewrite the following question so that it contains specific keywords that clearly point to the provided document, ensuring that it would likely match this document alone within a larger corpus.\\
\\
\textbf{Instruction}\\
- Do not add any additional information or context to the question.\\
- You should not change the meaning of the question.\\
- If the question is already specific and unique, you may leave it unchanged.\\
- Please make the sentences you have rewritten more diverse and fluent. \\
\\
\textbf{Examples}\\
- Original question: GIS data integration is part of which process?\\
- Rewritten question: Citizen Science shows which process the GIS data integration is part of?\\

- Original question: What percentage of apps ranked in the top five for including what resulted in a 10,3\% Ranking Increase?\\
- Rewritten question: According to the App Store Optimization what percentage of apps ranked in the top five for including what resulted in a 10,3\% Ranking Increase?\\

- Original question: Who is the author of the book, the title of which is the same as the section title of the presentation?\\
- Rewritten question: Who is the author of the book, the title of which is the same as the section title of the presentation by Michael Sahota and Olaf Lewitz?\\

- Original question: Which region of the world accounts for the highest percentage of revenues in the year 12\% GROWTH is achieved?\\
- Rewritten question: Which region of the world accounts for the highest percentage of revenues in the year 12\% GROWTH is achieved?\\

- Original question: What directly follows "conduct market research to refine" in the figure?\\
- Rewritten question: What directly follows "conduct market research to refine" in the figure within the Social Velocity Strategic Plan Process?\\

- Original question: How can the company which details 24 countries in the report be contacted?\\
- Rewritten question: How can the company which details 24 countries in the Global Digital Statistics 2014 report, be contacted?\\

- Original question: What substances are involved in the feeding of substrates?\\
- Rewritten question: What substances are involved in the feeding of substrates during the production of penicillin?\\

\tcblower
{\color{black}\bf \large User Prompt:}\\
Query: {\color{deepblue}\bf \{Query Description\}} \\
Document: {\color{deepblue}\bf \{Document Description\}} \\
Image: {\color{deepblue}\bf \{Image File\}} \\

\end{AIbox}
\vspace{-1em}
\caption{Prompt of Multi-Modal Refiner.}
\label{fig: refiner}
\end{figure*}

\begin{figure*}[!ht] 
\begin{AIbox}{Seeker Agent Prompt.}
{\color{black}\bf \large System Prompt:} 
\vspace{1mm}
\\
\textbf{Character Introduction}  \\
You are an artificial intelligence assistant with strong ability to find references to problems through images. The images are numbered in order, starting from zero and numbered as 0, 1, 2 ... Now please tell me what information you can get from all the images first, then help me choose the number of the best picture that can answer the question. 

\textbf{Response Format}  \\
The number of the image is starting from zero, and counting from left to right and top to bottom, and you should response with the image number in the following format:
\begin{lstlisting}[style=prompt]
{
    "reason": Evaluate the relevance of the image to the question step by step,
    "summary": Extract the information related to the problem,
    "choice": List[int]
}
\end{lstlisting}

\textbf{Response Example}  \# open-source models sometimes need few-shot instructions.
\begin{lstlisting}[style=prompt]
Example 1: Question: Who is the person playing a musical instrument in restaurant? 
Response to Example 1: 
{
    "reason": "Image 0 shows that KFC on Renmin Road has a birthday party on February 3rd. I can know that there are musical instruments playing in Shanghai hotels during meals from Image 1. Image 2 shows that this is an invitation letter for the music performance of the New Year's Concert at Qintai Art Museum on December 31st. The question is related to the restaurant, and Image 2 is not relevant to the question.",
    "summary": "KFC on Renmin Road has a birthday party on February 3rd;Shanghai hotels have musical instruments playing during meals;The Qintai Art Museum will hold a New Year's concert on December 31st.",
    "choice": [0, 1]
}

Example 2: Question: What time is the train departing from hangzhou to beijing?
Response to Example 2:
{
    "reason": "Image 0 shows that Beijing has a temperature of 18 degrees Celsius. Image 0 is a train ticket from hangzhou to beijing showing a departure time of 14:30. Image 1 is a photo of a train station clock, but it's blurry and hard to read the exact time. Image 2 shows a train schedule with multiple departure times listed. Image 3 is the timetable of Hangzhou Xiaoshan International Airport, and this image is not related to the issue. I think Image 0 is the most relevant to the question.",
    "summary": "The train ticket shows a departure time of 14:30;The train station clock is blurry;Train schedule shows time.",
    "choice": [0]
}

Example 3: Question: Where can I find a bookstore that sells rare books? 
Response to Example 3: 
{
    "reason": "Image 0 is a street view of a shopping mall with various stores, but no bookstores are visible. Image 1 shows a sign for a bookstore called "Rare Finds Bookstore" specializing in rare books. Image 2 is a map with multiple bookstores marked, but it doesn't specify if they sell rare books. Image 3 is a photo of a library, which is not a place to buy books. Image 5 is a rare books list, which includes the names and prices of various books. ",
    "summary": "The shopping mall has no visible bookstores;Rare Finds Bookstore specializes in rare books;Map shows multiple bookstores but doesn't specify rarity;Library is not for buying books;The price list includes the prices and names of rare books.",
    "choice": [1, 5]
}
\end{lstlisting}


\tcblower
{\color{black}\bf \large User Prompt:}\\
Query: {\color{deepblue}\bf \{Query Description\}} \\
Images: {\color{deepblue}\bf \{Candidate Images\}} \\
Reflection:  {\color{deepblue}\bf \{Feedback From Inspector\}} 

\end{AIbox}
\vspace{-1em}
\caption{Prompt of Seeker Agent.}
\label{fig: seeker}
\end{figure*}

\begin{figure*}[!ht] 
\begin{AIbox}{Inspector Agent Prompt.}
{\color{black}\bf \large System Prompt:} 
\vspace{1mm}
\\
\textbf{Character Introduction}  \\
You are an artificial intelligence assistant with strong ability to answer questions through images. Please provide the answer to the question based on the information provided.

\textbf{Task Description}  \\
- If the images can answer the question, please answer the question directly.\\
- If the images are not enough to answer the question, please tell me which pictures are related to the question.

\textbf{Response Format}  \\
- If the images can answer the question, please answer the question directly:
\begin{lstlisting}[style=prompt]
{
    "reason": Solve the question step by step,
    "answer": Answer the question briefly with several words,
    "reference": List[int]
}
\end{lstlisting}

- If the images are not enough to answer the question, please tell me what additional information you need, and tell me which pictures are related to the question:
\begin{lstlisting}[style=prompt]
{
    "reason": Evaluate the relevance of the image to the question one by one, and solve the question step by step,
    "information": Carefully clarify the information required,
    "choice": List[int]
}

\end{lstlisting}

\textbf{Response Example}  \# open-source models sometimes need few-shot instructions.
\begin{lstlisting}[style=prompt]
- Example 1:
{
    "reason": "The image only provides information about the Bohr Model and does not include details about subshells in the Modern Quantum Cloud Model.",
    "information": "More information about the Bohr Model.",
    "choice": []
}

- Example 2:
{
    "reason": "The images provide information about the #swallowaware campaign, including its aims and how they were measured. However, specific details on the success metrics are not clearly visible in the provided images.",
    "information": "More information about the success metrics of the #swallowaware campaign.",
    "choice": [0, 1]
}

- Example 3:
{
    "reason": "We first found the restaurant name on the menu, and then we located the restaurant in the city center on the map.",
    "answer": "city center",
    "reference": [2, 3]
}

- Example 4:
{
    "reason": "The entire process, from input, processing to output, ultimately produces a product with a purity of 42%.",
    "answer": "42%",
    "reference": [0]
}
\end{lstlisting}


\tcblower
{\color{black}\bf \large User Prompt:}\\
Query: {\color{deepblue}\bf \{Query Description\}} \\
Plan: {\color{deepblue}\bf \{Thought From Last Step.\}} \\
Images: {\color{deepblue}\bf \{Images Pending Review.\}} 

\end{AIbox}
\vspace{-1em}
\caption{Prompt of Inspector Agent.}
\label{fig: inspector}
\end{figure*}

\begin{figure*}[!ht] 
\begin{AIbox}{Answer Agent Prompt.}
{\color{black}\bf \large System Prompt:} 
\vspace{1mm}
\\
\textbf{Character Introduction}  \\
You are an artificial intelligence assistant with strong ability to answer questions through images. Please provide the answer to the question based on the information provided and tell me which pictures are your references.

\textbf{Response Format}  \\
Please provide the answer in JSON format:
\begin{lstlisting}[style=prompt]
{
    "reason": Solve the question step by step,
    "answer": Answer the question briefly with several words,
    "reference": List[int]
}
\end{lstlisting}

\tcblower
{\color{black}\bf \large User Prompt:}\\
Query: {\color{deepblue}\bf \{Query Description\}} \\
Draft Answer: {\color{deepblue}\bf \{Draft Answer From Inspector\}} \\
Images: {\color{deepblue}\bf \{Reference Images\}} 


\end{AIbox}
\vspace{-1em}
\caption{Prompt of Answer Agent.}
\label{fig: answer}
\end{figure*}

