\begin{table}[!t]
    \small
    \centering
    \caption{Retrieval Performance on ViDoSeek.}
    \resizebox{1\linewidth}{!}{
    \label{tab:ret}
    \begin{tabular}{lcccc}
    \toprule
    \textbf{Retriever} & \textbf{Recall@1} & \textbf{Recall@3} & \textbf{Recall@5} & \textbf{MRR@5}\\
    \midrule
    BM25 & 55.2 & 77.4 & 84.5 & 66.5 \\
    BGE-M3\cite{chen2024bge} & 60.2 & 79.3 & 87.6 & 70.5\\
    NV-Embed-V2\cite{lee2024nv} & 64.1 & 83.5  & 90.3 & 74.7 \\
    \midrule
    VisRAG-Ret\cite{yu2024visrag} & 64.4 & 84.1 & 91.2 & 75.2\\
    ColPali\cite{faysse2024colpali} & 70.6 & 87.9 & 92.8 & 79.6\\
    ColQwen2\cite{faysse2024colpali} & 75.4 & 89.7 & 95.1 & 83.3 \\
    \bottomrule
    \end{tabular}
}
\end{table}
\begin{figure}[!h]
    \centering 
    \includegraphics[width=0.95\linewidth]{figure/recall.pdf}
    \caption{Retrieval performance across different retrievers and hybrid retrieval, along with ablations on GMM. 
    }
    \label{fig:ret}
\end{figure}

\subsection{Retrieval Evaluation}
In Table \ref{tab:ret}, we report the detailed performance for various retrievers, including OCR-based and visual-based. 
Due to the uncertainty of dynamical retrieval across queries, we use the average length of results for analysis.
Our goal is to incorporate more relevant information within a shorter context while minimizing the impact of noise and reducing computational cost without losing valuable information. 
Dynamic retrieval can achieve better recall performance with a smaller context length, while hybrid retrieval combines the results of two pipelines achieving state-of-the-art performance.

\section{Analysis}
\subsection{Ablations}
Table \ref{tab:ablation} presents the impact of different retrievers and generation methods on performance. We have decomposed the dynamic retrieval into two components, Dynamic and Hybrid. Naive refers to the method of direct input, which is most commonly used as baselines. Dynamic indicates using GMM to fit the optimal recall distribution based solely on the visual pipeline. Hybrid refers to merging the visual and the textual retrieval results directly, which leads to suboptimal results due to long contexts. Experiments demonstrate that the effectiveness and scalability of our improvements on retrieval and generation modules, as well as their combination, can comprehensively enhance end-to-end performance from various perspectives.


\paragraph{Synthetic multiple} 
Thus far, we have exclusively considered that the number of generated synthetic records equals the number of records in the real data, \ie, $N = \synthetic{N}$. We now consider the case when more synthetic data is made available to a data-based adversary ($\synthetic{\mathcal{A}}$). Specifically, we denote the \emph{synthetic multiple} $m = \nicefrac{\synthetic{N}}{N}$ and evaluate how different MIAs perform for varying values of $m$.
%
Figure~\ref{fig:synthetic_multiple} shows how the ROC AUC score varies as $m$ increases. As expected, the ROC AUC score for the attack that uses membership signals computed using a 2-gram model trained on synthetic data increases when more synthetic data is available. In contrast, attacks based on similarity metrics do not seem to benefit significantly from this additional synthetic data.

\begin{figure}[htb]
  \centering
  \begin{subfigure}{0.4\textwidth}
    \centering
    \includegraphics[width=\textwidth]{figures/synthetic_multiple_sst2.pdf}
  \end{subfigure}
  \hspace{0.05\textwidth}
  \begin{subfigure}{0.4\textwidth}
    \includegraphics[width=\textwidth]{figures/synthetic_multiple_agnews.pdf}
  \end{subfigure}
  \caption{ROC AUC score for increasing value of the synthetic multiple $m$ across data-based attack methods for SST-2 (left) and AG News (right). Canaries are synthetically generated with target perplexity of $\mathcal{P}_{\textrm{target}}=250$,  with a natural label, with no in-distribution prefix ($F=0$), and inserted $n_\textrm{rep}=12$ times.} 
  \label{fig:synthetic_multiple}
\end{figure} 


\paragraph{Hyperparameters in data-based attacks}
The data-based attacks that we presented in Sec.~\ref{sec:membership_method} rely on certain hyperparameters.
%
The attack that uses $n$-gram models to compute membership signals is parameterized by the order $n$. Using a too small value for $n$ might not suffice to capture the information leaked from canaries into the synthetic data used to train the $n$-gram model. When using a too large order $n$, on the other hand, we would expect less overlap between $n$-grams present in the synthetic data and the canaries, lowering the membership signal.

Further, the similarity-based methods rely on the computation of the mean similarity of the closest $k$ synthetic records to the a canary. When $k$ is very small, \eg $k=1$, the method takes into account a single synthetic record, potentially missing on leakage of membership information from other close synthetic data records. When $k$ becomes too large, larger regions of the synthetic data are taken into account, which might dilute the membership signal among the noise.

Table~\ref{tab:ablations_synthetic} reports the ROC AUC scores of data-based attacks for different values of the hyperparameters $n$ and $k$ when using standard canaries (Sec.~\ref{sec:baseline_results}). We find that for both datasets, training a $2$-gram model on the synthetic data to compute the membership signal yields the best performance. For the data-based MIAs relying on the similarity between the canary and the synthetic records, both when considering Jaccard distance and cosine distance in the embedding space, we find that considering the $k=25$ closest synthetic records yields the best performance. 

\begin{table}[ht]
    \centering
    \begin{tabular}{ccc@{\hskip 20pt}cc@{\hskip 20pt}cc}
    \toprule
         & \multicolumn{2}{c}{$n$-gram} 
         & \multicolumn{2}{c}{$\textsc{SIM}_\textrm{Jac}$} 
         & \multicolumn{2}{c}{$\textsc{SIM}_\textrm{emb}$} \\
        \cmidrule(lr){2-3} \cmidrule(lr){4-5} \cmidrule(lr){6-7}
        Dataset & $n$ & AUC & $k$ & AUC & $k$ & AUC\\
        \midrule
        \multirow{4}{*}{\parbox{1.8cm}{\centering SST-2}} 
        & $1$ & $0.415$ & $1$ & $0.520$ & $1$ & $0.516$ \\ 
        & $2$ & \bm{$0.616$} & $5$ & $0.535$ & $5$ & $0.516$ \\ 
        & $3$ & $0.581$ & $10$ & $0.538$ & $10$ & $0.519$ \\ 
        & $4$ & $0.530$ & $25$ & \bm{$0.547$} & $25$ & \bm{$0.530$} \\   
        \midrule
        \multirow{4}{*}{\parbox{1.8cm}{\centering AG News}} 
        & $1$ & $0.603$ & $1$ & $0.522$ & $1$ & $0.503$ \\ 
        & $2$ & \bm{$0.644$} & $5$ & $0.525$ & $5$ & $0.498$ \\ 
        & $3$ & $0.567$ & $10$ & $0.537$ & $10$ & $0.503$ \\ 
        & $4$ & $0.527$ & $25$ & \bm{$0.552$} & $25$ & \bm{$0.506$} \\        
        \bottomrule
    \end{tabular}
    \caption{Ablation over hyperparameters of data-based MIAs. We report ROC AUC scores across different values of the hyperparameters $n$ and $k$ (see Sec.~\ref{sec:membership_method}). Canaries are synthetically generated with target perplexity $\mathcal{P}_\textrm{target}=250$, with a natural label, with no in-distribution prefix ($F=0$), and inserted $n_\textrm{rep}=12$ times.}
    \label{tab:ablations_synthetic}
\end{table}

\subsection{Time Efficiency}
\label{sec:analysis:time}
\paragraph{How does dynamic retrieval balance latency and accuracy?}
In traditional RAG systems, using a small top-K value may result in missing critical information, whereas employing a larger value can introduce noise and increase computational overhead. ViDoRAG dynamically determines the number of documents to retrieve based on the similarity distribution between the query and the corpus. This approach ensures that only the most relevant documents are retrieved, thereby reducing unnecessary computations from overly long contexts and accelerating the generation process. As shown in Table \ref{tab:time_recall}, we compare retrieval with and without GMM based on the Naive method. The experiments indicate that GMM may reduce recall due to distribution bias. However, because it significantly shortens the generation context, it effectively improves performance in end-to-end evaluations.
\begin{table}[!t]
    \small
    \centering
    \caption{Evaluation of Dynamic Retrieval Methods.}
    \resizebox{0.71\linewidth}{!}{
    \label{tab:time_recall}
    \begin{tabular}{lcc}
    \toprule
    \textbf{Method} & \textbf{Accuracy} $\uparrow$ & \textbf{Avg. Pages} $\downarrow$\\
    \midrule
    w/o GMM &  72.1 & 10\\
    w/ GMM & \textbf{72.8} & \textbf{6.76} \\
    \bottomrule
    \end{tabular}
}
\end{table}

\paragraph{Latency Analysis of the Multi-Agent Generation.}
There is an increase in delay due to the iterative nature of the multi-agent system, as shown in Fig. \ref{fig:gen_time}. Each agent performs specific tasks in a sequential manner, which adds a small overhead compared to traditional straightforward RAG. However, despite the increase in latency, the overall performance improves due to the higher quality of generated answers, making the trade-off between latency and accuracy highly beneficial for complex RAG tasks.

\begin{figure}[!h]
    \centering 
    \includegraphics[width=0.93\linewidth]{figure/time.pdf}
    \caption{Latency Analysis on Generation.}
    \label{fig:gen_time}
\end{figure}

\subsection{Modalities and Strategies of Generation}
As shown in Fig. \ref{fig:compare_radar}, the vision-based pipeline outperforms the text-based pipeline across all types, even for queries related to text content. Generally speaking, due to models' inherent characteristics, the reasoning ability of LLMs is stronger than that of VLMs. However, the lack of visual information makes it difficult for models to identify the intrinsic connections between pieces of information. This also poses a challenge for the generation of content based on visually rich documents. While obtaining visual information, VidoRAG further enhances the reasoning capabilities of VLMs, striking a balance between accuracy and computational load.

\begin{figure}[!h]
    \centering 
    \includegraphics[width=0.99\linewidth]{figure/compare_radar.pdf}
    \caption{Performance across different types of queries on our ViDoSeek and the refined SlideVQA datasets.}
    \label{fig:compare_radar}
\end{figure}

\begin{figure}[!h]
    \centering 
    \includegraphics[width=0.91\linewidth]{figure/scale.pdf}
    \caption{Scaling behavior with ViDoRAG.}
    \label{fig:scale}
\end{figure}

\subsection{Performance with Test-time Scaling}

Fig. \ref{fig:scale} illustrates the number of interaction rounds between the seeker and inspector within ViDoRAG based on different models. 
Due to the limited instruction capabilities of some models, we sampled 200 queries for the experiment.
Models with stronger performance require fewer reasoning iterations, while weaker models often need additional time to process and reach a conclusion.
Conditioning the model on a few demonstrations of the task at inference time has been proven to be a computationally efficient approach to enhance model performance\cite{brown2020language,min2021metaicl}. 
The results indicate that predefining tasks and breaking down complex tasks into simpler ones is an effective method for scaling inference.
