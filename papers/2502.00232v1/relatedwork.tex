\section{Related Works}
Oil-water classification using hyperspectral images (HSI) has gained increasing attention due to its environmental significance, particularly in detecting and monitoring oil spills. This section focuses on recent advancements in HSI classification, with an emphasis on works closely aligned with the Hyperspectral Oil Spill Detection (HOSD) dataset, introduced by Puhong Duan.\cite{duan2023hosd}
\newline
The HOSD dataset, contains 18 hyperspectral images captured by the ARVIS sensor during the Deepwater Horizon oil spill. These images cover a spectral range of 365 nm to 2500 nm and are accompanied by reference maps that classify each pixel as either oil or non-oil. It is a valuable resource for advancing oil-water classification research.

Early work on the HOSD dataset explored unsupervised detection techniques for pixel level classification, such as the isolation forest (iForest)-based framework proposed by \cite{duan2023hosd}. This method employed a Gaussian statistical model to preprocess noisy spectral bands, followed by dimensionality reduction using kernel principal component analysis (KPCA). Probabilistic outputs from the iForest were refined using clustering algorithms and a support vector machine (SVM), achieving competitive image-wise classification accuracy. While effective, the approach did not integrate spatial relationships, which are critical for context-aware analysis in oil-water classification \cite{duan2023hosd}.

To address the limitations of pixel-wise methods, multiscale spectral-spatial learning frameworks have emerged as a promising direction for HSI classification. A notable contribution in this area is the multiscale spectral-spatial CNN (HyMSCN), which introduced an image-based classification framework designed to improve processing efficiency by integrating features from multiple receptive fields \cite{xu2021hymscn}. Unlike patch-based approaches, this method minimized redundancy in testing and effectively fused multiscale features to enhance classification accuracy. While the approach achieved strong performance on general-purpose hyperspectral datasets, it was not tailored for domain-specific datasets like HOSD, highlighting the need for specialized frameworks that address the unique spectral and spatial characteristics of oil spill imagery \cite{xu2021hymscn}.

Building upon the strengths of spectral-spatial integration, some efforts have emphasized the importance of contextual learning in HSI classification. The contextual CNN(HybridSN) proposed by \cite{zhang2020contextualcnn} demonstrated how a multi-scale convolutional filter bank could effectively exploit spatio-spectral relationships, producing a unified feature map for accurate pixel-wise classification. By leveraging deeper and wider architectures, this approach achieved high-ranking performance on standard datasets, such as Indian Pines and Salinas, underscoring the potential of contextual learning in enhancing classification accuracy \cite{zhang2020contextualcnn}.

While these approaches provide valuable insights, our method simplifies the training process by adopting a CNN on 2D probabilistic images generated by Random Forest rather than directly processing 3D hyperspectral data. This hybrid framework bridges the gap between probabilistic modeling and spatial feature learning, achieving superior oil-water classification performance on the HOSD dataset. By combining the strengths of Random Forest in both pixel-level and tile-wise classification with CNNs’ capacity to incorporate spatial context, the proposed approach offers a practical and efficient solution for context-aware analysis of hyperspectral images.