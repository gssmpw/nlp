
%% The first command in your LaTeX source must be the \documentclass
%% command.
%%
%% For submission and review of your manuscript please change the
%% command to \documentclass[manuscript, screen, review]{acmart}.
%%
%% When submitting camera ready or to TAPS, please change the command
%% to \documentclass[sigconf]{acmart} or whichever template is required
%% for your publication.
%%
%%
\PassOptionsToPackage{table}{xcolor}  
\documentclass[manuscript,screen]{acmart}
\let\Bbbk\relax
\usepackage{amsmath}    % For mathematical expressions
\usepackage{tabularx} 
\usepackage{rotating}
\usepackage{booktabs}
\usepackage{subcaption} 
\usepackage[utf8]{inputenc}
\usepackage{longtable}
\usepackage{csquotes}

\geometry{margin=1in}
%%
%% \BibTeX command to typeset BibTeX logo in the docs
\AtBeginDocument{%
  \providecommand\BibTeX{{%
    Bib\TeX}}}

%% Rights management information.  This information is sent to you
%% when you complete the rights form.  These commands have SAMPLE
%% values in them; it is your responsibility as an author to replace
%% the commands and values with those provided to you when you
%% complete the rights form.
% \setcopyright{acmlicensed}
% \copyrightyear{2024}
% \acmYear{2024}
% \acmDOI{XXXXXXX.XXXXXXX}

%% These commands are for a PROCEEDINGS abstract or paper.
% \acmConference[CHI 2025]{Make sure to enter the correct
%   conference title from your rights confirmation emai}{June 03--05,
%   2018}{Woodstock, NY}
%%
%%  Uncomment \acmBooktitle if the title of the proceedings is different
%%  from \enquote{Proceedings of ...}!
%%
%%\acmBooktitle{Woodstock '18: ACM Symposium on Neural Gaze Detection,
%%  June 03--05, 2018, Woodstock, NY}

\begin{document}

%%
%% The \enquote{title} command has an optional parameter,
\title{Bridging Voting and Deliberation with Algorithms: Field Insights from vTaiwan and Kultur Komitee}
%%
\author{Joshua C. Yang}
\email{joyang@ethz.ch}
\orcid{0009-0004-4044-6338}
\affiliation{%
  \institution{ETH Zurich}
  \city{Zurich}
  \country{Switzerland}
}

\author{Fynn Bachmann}
\email{fbachmann@ifi.uzh.ch}
\orcid{0000-0001-9571-5671}
\affiliation{%
  \institution{University of Zurich}
  \city{Zurich}
  \country{Switzerland}
}

\begin{abstract} 

Democratic processes increasingly aim to integrate large-scale voting with face-to-face deliberation, addressing the challenge of reconciling individual preferences with collective decision-making. This work introduces new methods that use algorithms and computational tools to bridge online voting with face-to-face deliberation, tested in two real-world scenarios: \textbf{Kultur Komitee 2024 (KK24)} and \textbf{vTaiwan}. These case studies highlight the practical applications and impacts of the proposed methods.
%
We present three key contributions: (1) \textbf{Radial Clustering for Preference Based Subgroups}, which enables both in-depth and broad discussions in deliberative settings by computing homogeneous and heterogeneous group compositions with balanced and adjustable group sizes; (2) \textbf{Human-in-the-loop MES}, a practical method that enhances the Method of Equal Shares (MES) algorithm with real-time digital feedback. This builds algorithmic trust by giving participants full control over how much decision-making is delegated to the voting aggregation algorithm as compared to deliberation; and (3) the \textbf{ReadTheRoom} deliberation method, which uses opinion space mapping to identify agreement and divergence, along with spectrum-based preference visualisation to track opinion shifts during deliberation. This approach enhances transparency by clarifying collective sentiment and fosters collaboration by encouraging participants to engage constructively with differing perspectives.
%
By introducing these actionable frameworks, this research extends in-person deliberation with scalable digital methods that address the complexities of modern decision-making in participatory processes.

\end{abstract}

\begin{CCSXML}
<ccs2012>
 <concept>
  <concept_id>10003120.10003121.10003125.10010597</concept_id>
  <concept_desc>Human-centered computing~Social computing theory, methods, and systems</concept_desc>
  <concept_significance>500</concept_significance>
 </concept>
 <concept>
  <concept_id>10010147.10010178.10010224.10010245.10010250</concept_id>
  <concept_desc>Computing methodologies~Voting methods</concept_desc>
  <concept_significance>300</concept_significance>
 </concept>
 <concept>
  <concept_id>10003752.10010070.10010099.10010100</concept_id>
  <concept_desc>Information systems~Collaborative and social computing systems and tools</concept_desc>
  <concept_significance>300</concept_significance>
 </concept>
 <concept>
  <concept_id>10003120.10003123</concept_id>
  <concept_desc>Human-centered computing~Interaction design</concept_desc>
  <concept_significance>200</concept_significance>
 </concept>
</ccs2012>
\end{CCSXML}

\ccsdesc[500]{Human-centered computing~Social computing theory, methods, and systems}
\ccsdesc[300]{Computing methodologies~Voting methods}
\ccsdesc[300]{Information systems~Collaborative and social computing systems and tools}
\ccsdesc[200]{Human-centered computing~Interaction design}

\keywords{Participatory budgeting, citizens' assemblies, Method of Equal Shares, group deliberation, democratic innovations, social computing, voting systems}

\maketitle

\section{Introduction}

In an era where democratic decision-making faces increasing complexity, computational tools have emerged as essential enablers of fair and scalable processes. Modern challenges in democracy often involve balancing individual preferences with collective decisions \cite{weyl2024}, particularly in large-scale contexts. As \citet{hendriks2024} point out, each democratic format offers unique strengths and limitations. Deliberative designs are effective in building meaningful discussions, but often engage a limited number of participants and struggle to achieve large-scale impact \cite{michels2011}. Conversely, plebiscitary designs, or voting, can involve broad citizen participation and produce decisive collective outcomes, but often lack the depth of deliberative dialogue. To address these trade-offs, hybrid democratic innovations have emerged, combining the strengths of both approaches \cite{felicetti2021,hendriks2023}.

Deliberative democracy has long focused on the role of informed discussions in determining collective decisions~\cite{bachtiger_deliberative_2018}. Recent studies, however, emphasise a complementary dynamic: voting can enhance deliberation~\cite{bachtiger_deliberation_2018-3,bachtiger_democratic_2018}. Namely, voting complements and enhances deliberation in seven key ways, as pointed out by \citet{chambers2023}: it (i) provides a feasible and fair closure mechanism; (ii) ensures equal recognition and status among participants; (iii) politicises deliberation by internalising conflict; (iv) induces authenticity by encouraging participants to reveal their preferences; (v) preserves dissent; (vi) defines issues to focus deliberation effectively; and, (vii) in public voting contexts, fosters accountability for claims. Understanding these dynamics helps identify designs that better integrate deliberation and voting to capitalise on their combined strengths. 

Fair mechanisms for aggregating diverse preferences are essential for proportional representation in collective decision-making, ensuring outcomes reflect the priorities of all groups rather than only favouring majorities \cite{aziz2020}. Algorithms like the Method of Equal Shares (MES) \cite{peters2021} enable fair resource allocation while protecting minority voices. However, real-world implementation faces challenges, including bridging technological literacy gaps, managing real-time deliberations, and building trust in algorithmic outcomes. To address these issues, accessible and user-friendly tools are vital for translating theoretical fairness into practical, inclusive applications across diverse contexts \cite{yang2024}. 


This research offers practical insights into designing democratic innovations that integrate voting and deliberation. In particular, we introduce three computational methods applied in two real-world settings: \textbf{Kultur Komitee Winterthur 2024 (KK24)} and \textbf{vTaiwan}. The contributions of this paper are:

\begin{enumerate}
    \item We introduce \textbf{Radial Clustering for Preference Based Subgroups}, which computes homogeneous or heterogeneous subgroups for deliberation based on prior voting results. This clustering method ensures that subgroups have balanced sizes by slicing the (two-dimensional) preference space into even parts. 
    \item We extend the \textbf{Method of Equal Shares (MES)} with a \textbf{Human-in-the-Loop} approach, allowing participants to adjust the overall budget and explore funding scenarios, balancing algorithmic decision-making with deliberation.
    \item We present the \textbf{ReadTheRoom Deliberation} method, which uses opinion mapping to identify divisive statements and spectrum-based voting to visualise shifts in preferences. This feedback loop fosters reflection, collaboration, and open-minded discussions.
    \item We document two participatory case studies: \textbf{KK24} and \textbf{vTaiwan}. KK24 demonstrates the impact of a \textit{Budget Assembly} for inclusive funding decisions, while vTaiwan shows how combining online and offline deliberation enhances participation and mutual learning.
\end{enumerate}



\subsection{Case Studies}

\subsubsection{\textbf{Kultur Komitee: Democratising Art and Culture through Participatory Budgeting in Winterthur, Switzerland}}

Established in 2019, the \textit{Kultur Komitee} [\enquote{Culture Committee}] in Winterthur, Switzerland, empowers citizens to allocate cultural funding and shape the city’s cultural landscape. Combining elements of participatory budgeting and citizens' assemblies, this new \textit{Budget Assembly} integrates face-to-face deliberations with the tangible outcomes of participatory decision-making.

The 2023/2024 cycle (KK24) marked the fourth iteration of the \textit{Kultur Komitee} process, involving 37 randomly selected citizens. Invitations were distributed via the city’s postal service to ensure diverse participation, with 300 invitations sent and 37 accepted. The process began with a kick-off event in September 2023 where committee members developed shared goals and evaluation criteria. An open call for proposals followed, resulting in 134 submissions, which were refined into 56 shortlisted projects for deliberation. In March 2024, participants reviewed these projects individually via approval voting on an online platform, ensuring a streamlined deliberation process while capturing individual preferences. Finally, on 13 April 2024, the committee met in Winterthur to finalise funding decisions using the Human-in-the-Loop \textit{Method of Equal Shares} (MES) combined with in-person deliberations. CHF 381,500 was ultimately awarded to selected projects. This paper focuses on the later stage of the process involving online voting and the final in-person deliberation.

Out of the 37 participants involved in the process, 35 attended the final deliberative workshop (\(N = 35\)), with a mean age of 41.2 years, ranging from 20 to 82. The gender distribution consisted of 13 males and 22 females. Most participants were born in Switzerland, while 8 were from outside Switzerland. By the end of the process, 35 projects were funded, receiving grants ranging from CHF 2,500 to CHF 30,000, totalling CHF 381,500. During the process of Human-in-the-loop MES, it was decided that 50\% of the budget would be used for MES calculation, and the other half would be allocated with deliberation. Eventually, 18 projects were selected using MES, and 17 projects through deliberation. KK24 serves as a model for integrating participatory and deliberative practices in resource allocation. For more details, we refer to the KK24 factsheet \footnote{https://kulturkomitee.win/media/digital\_kk23\_24\_factsheet.pdf}.

\subsubsection{\textbf{vTaiwan: Citizen-led Deliberation Process for Public Issues}}

Launched in 2014, vTaiwan is a decentralised open consultation process combining online and offline interactions to connect citizens and government for the deliberation on national issues. It serves as a model to involve government ministries, elected representatives, scholars, experts, business leaders, civil society organisations, and citizens in crafting digital legislation. One of vTaiwan's key tools is \textit{Polis}~\cite{small2021}, a digital platform for opinion collection and consensus formation. The tool facilitates large-scale conversations and has been pivotal in achieving \enquote{rough consensus} on various policy issues at the national level, overcoming scalability challenges in deliberative democracy. Since its inception, vTaiwan has engaged over 200,000 participants and contributed to 26 pieces of legislation on the digital economy and other key issues. 

In December 2024, vTaiwan and the Taiwan Network Information Center (TWNIC) co-hosted a \textit{Roundtable Discussion} on \textit{\enquote{What aspects should Taiwan consider when regulating artificial intelligence (AI)?}} This event addressed legal and ethical challenges of generative AI and gathered input for Taiwan’s proposed Basic Law on Artificial Intelligence. Using Polis and Mentimeter, public sentiment was visualised, areas of consensus and disagreement were highlighted, and opinion shifts were tracked during workshops. This deliberative process provided actionable insights for AI governance while showcasing the effectiveness of participatory approaches to address complex policy challenges.

\section{Literature Review}

\subsection{Hybrid Democratic Innovation}

Hybrid Democratic Innovations (HDIs) combine concentrated deliberation with large-scale voting \cite{hendriks2024} to enhance inclusivity, reflection, and policy impact. Early examples of HDIs were given by \citet{fishkin1991} with deliberative polling. This method has been described by \citet{mansbridge2010} to capture informed public opinion through random sampling, structured deliberation, and feedback. Building on this, \citet{felicetti2021} proposed \enquote{democratic assemblage,} emphasising adaptive systemic integration of deliberative and participatory processes. This has been illustrated by \citet{gastil2021}, who showed how digital platforms and game mechanics can enhance participatory budgeting through motivation and engagement. Practical applications include the three-step model of \citet{Itten2022} for climate policymaking, which combines mini-publics and maxi-publics to produce widely accepted recommendations, and \citet{hendriks2024}’s study of Antwerp’s Citizens’ Budget, showcasing how \enquote{participatory budgeting new-style} can align deliberation with large-scale voting to uphold democratic values. Referendum processes have similarly benefited, as \citet{witting2023} identify how deliberation mitigates deficits of referenda. \citet{hendriks2023} propose deliberative referenda to connect citizens’ assemblies with direct voting, thereby enhancing public debate and systemic legitimacy. Finally, \citet{chambers2023} theorise the complementary role of voting in deliberation to provide closure, equality, and accountability, highlighting the importance of hybrid approaches in advancing democratic innovation.

\subsection{Online Tools for Deliberative Democracy}

Digital tools have emerged as critical innovations in deliberative democracy, enabling scalable citizen participation in decision-making~\cite{klein_towards_2017,davies_online_2020,mikhaylovskaya_enhancing_2024}. These tools integrate deliberation with digital methods for visualising, moderating, and summarising discussions. A prominent example is Polis~\cite{small2021}, an open-source platform where participants collect and rate statements on given topics. The tool visualises the high-dimensional opinion space and computes consensus statements to foster common ground in the deliberation. Being applied to many larger participatory processes such as vTaiwan and the German \enquote{Aufstehen} [\enquote{Stand up}] movement, Polis demonstrates how digital tools can scale deliberation beyond traditional in-person settings. Another successful example of such an online tool is the Stanford Deliberation Platform~\cite{fishkin_deliberative_2019}, where automatic moderation ensures fair deliberation of small groups in video calls. This platform has also been used in the \enquote{America in One Room} deliberation. To summarise discussions, tools like Wikum~\cite{zhang_wikum_2017} and Kialo~\cite{chaudoin_revolutionizing_2017} have been developed.  These tools offer thread-like discussions with features such as nested argumentation and can be used to challenge reasoning in deliberation. Decision-making platforms such as Decidim~\cite{aragon_deliberative_2017} facilitate participatory processes for cities, while initiatives like Decide Madrid and vTaiwan exemplify algorithmic empowerment, decentralising power and fostering pluralistic policymaking~\cite{tseng2022}. This emphasises the possible role of algorithms in enhancing democratic transparency and inclusivity. \citet{umbelino2021} introduce ProtoTeams, a gamified approach to forming diverse groups for real-time collaborative discussions, illustrating how team composition and dynamics influence decision-making outcomes. However, scaling deliberation introduces challenges, as many platforms prioritise technical solutions, often overlooking factors like cultural and linguistic diversity, social inequalities, and inclusion across genders and abilities~\cite{shortall_reason_2022}. Addressing these aspects is also essential to ensure digital tools support a broader and more diverse range of participants.

\subsection{Computational Social Choice, Voting and Participatory Budgeting}

Computational social choice provides theoretical and practical tools for designing voting systems that address fairness, proportionality, and collective decision-making in contexts such as participatory budgeting (PB) \cite{brandt2016}. Fairness in this context refers to the equal representation of diverse voter preferences and the proportional allocation of resources or decision-making power, ensuring that no group is disproportionately advantaged or excluded \cite{fain2016}. Recent advancements have emphasised the integration of computational methods with participatory processes. \citet{aziz2020} provide a comprehensive survey of PB models, highlighting the importance of preference elicitation, voter incentives, and welfare objectives. Furthermore, the subfield of multi-winner voting contributes insights into proportionality and robustness. \citet{brill2023} proposes robust axioms for proportional representation, ensuring fairness for cohesive voter groups. \citet{faliszewski2023} further demonstrate the utility of proportional rules in PB elections. \citet{peters2021} formalises axioms for proportional representation in PB with additive utilities, proposing rules like the \textbf{Method of Equal Shares (MES)} to achieve fairness and efficiency. The basic idea of the Method of Equal Shares is that each voter is assigned an equal part of the budget. This part of the budget can only be used to fund projects that the voter has voted for. The method goes through all project proposals, beginning with the projects with the highest number of votes. It selects a project if it can be funded using the budget shares of those who voted for the project. The method then divides the cost of a project equally among its supporters. \citet{faliszewski2023} demonstrates how MES achieves fairer outcomes and avoids divisive districts while improving overall utility. Additionally, \citet{yang2024} explore digital PB platforms, identifying design principles that reduce cognitive load while enhancing fairness and legitimacy, where participants favour expressive input formats and fair budget allocations. Building on these contributions, our paper explores how combining deliberation and computational tools for democratic decision-making should be designed to achieve proportional fairness, transparency, and practicality.

\section{Methodology}

This study employs a mixed methods approach, combining computational techniques, digital tools, and participatory frameworks to explore the integration of voting and deliberation. Our methodology is structured around three algorithmic methods: Radial Clustering for Preference Based Subgroups, Human-in-the-loop MES, and ReadTheRoom Deliberation. As these methods have been co-designed and co-developed specifically to address the real-world issues and needs of KK24 and vTaiwan, they are also applied and evaluated in the context of these two case studies. 

\subsection{Radial Clustering for Preference Based Subgroups}

\subsubsection{\textbf{Motivation}}
Group deliberation often leads to middle-ground solutions that fail to address niche interests, such as local or specific community needs. This tendency for deliberation to gravitate towards consensus solutions has been noted in the literature, with scholars arguing that structured approaches are needed to ensure the representation of marginalised or niche perspectives~\cite{abdullah2016, karpowitz2014, sunstein2017}. \citet{fraser1990} highlights the importance of creating spaces for marginalised groups to deliberate autonomously, while \citet{mansbridge1994} emphasises the role of \enquote{enclaves of protected discourse} to empower these groups in broader discussions. \citet{sunstein2017} introduces the term \enquote{Enclave Deliberation,} noting both its potential to address inequalities and the risks of group polarisation, which can be mitigated through careful integration into wider deliberative processes.

Radial Clustering for Preference Based Subgroups, hereby referred to as just \enquote{Radial Clustering}, builds on the concept of Enclave Deliberation by structuring initial discussions in homogeneous groups. In these groups, participants with similar voting patterns deliberate to ensure niche interests are well-represented and given a platform. This stage creates an environment where marginalised perspectives can emerge without being overshadowed by dominant views. In the second round, participants transition to heterogeneous groups to engage with diverse perspectives, building broader understanding and identifying projects that benefit larger segments of citizens. By combining enclave-inspired homogeneous groups with broader heterogeneous deliberation, Radial Clustering aim to ensure both equality and inclusivity in the decision-making process.

Standard clustering methods, such as k-Means~\cite{macqueen1967some}, often create uneven groups or exclude outliers. While extensions exist that adjust the loss function to balance cluster sizes~\cite{franti_balanced_2014}, these complex algorithms might not be very comprehensive for participants in the deliberation. Our suggestion, Radial clustering, ensures equal group sizes with a straightforward and visually understandable algorithm: It projects the participants to a two-dimensional opinion space and divides it radially into \enquote{pizza slices}. Despite not being the most optimal clustering mathematically, this algorithm achieves balanced assignments and ensures visually transparent explainability, making it practical for real-world settings.

\begin{figure}[ht]
    \begin{minipage}[t]{0.49\linewidth}
        \centering
        \includegraphics[width=\linewidth]{PizzaClust.pdf}
        \caption{PCA visualisation of voters based on their votes on projects. Participants were grouped by calculating their proximity in the reduced two-dimensional space, with each group occupying a distinct pizza-slice-shaped area. The size of each slice is determined by the angle required to accommodate the pre-determined number of participants in the group.}
        \label{fig:voter_clusters}
    \end{minipage}
    \hfill
    \begin{minipage}[t]{0.49\linewidth}
        \centering
        \includegraphics[width=\linewidth]{kk_mes.pdf}
        \caption{Real-time Human-in-the-loop MES results interface used during the KK24 workshop in Winterthur Switzerland. The total budget could be adjusted to see what projects can be funded. The green bar represents the total number of votes, and the orange bar shows the project cost. Funded projects are highlighted in black.}
        \label{fig:MES_results}
    \end{minipage}%
\end{figure}

\subsubsection{\textbf{Process}}
In KK24, the organising committee decided that six groups with 6-7 participants each were needed for group deliberations. The steps for group assignment and deliberation were as follows:

\begin{enumerate}
\item \textbf{Dimensionality Reduction and Mapping:}
The group composition was based on participants' voting similarity in a pre-survey (i.e., approval votes for 50+ projects). These high-dimensional data were reduced using Principal Component Analysis (PCA), projecting participants' preferences into a two-dimensional space. Each participant's angular position in this space was calculated relative to the mean position:
\[
\theta = \arctan \left( \frac{PC2_i - \text{mean}(PC2)}{PC1_i - \text{mean}(PC1)} \right),
\]
where \(PC1_i\) and \(PC2_i\) represent the coordinates of participant \(i\). The angles \(\theta\) were normalised to 0-360 degrees.

\item \textbf{Radial Partitioning and Group Assignment:}
The angular space \([0, 360^\circ]\) was divided into six sectors, each having 6-7 participants. To achieve this, each sector was initialised spanning \(\frac{360}{6}\) degrees. Participants were assigned to sectors based on their angular positions. The angles of each sector were then iteratively adjusted to arrive at balanced group sizes.

\item \textbf{Two-Round Deliberation with Homogeneous and Heterogeneous Groups:}
The Radial clustering method was used to assign participants to six homogeneous groups (A/B/C/D/E/F) in the first round. In the second round, participants were reassigned to six heterogeneous groups (1/2/3/4/5/6), ensuring diversity by mixing members from different homogeneous groups.

In both rounds, group members discussed, selected, and ranked five projects. Rankings were converted into Borda scores (e.g., 1st: 5 points, 2nd: 4 points, etc.), and the points from all groups were aggregated across both rounds. Projects were then selected based on total points until the budget was fully allocated.
\end{enumerate}

\subsubsection{\textbf{Evaluation Metric}}

To evaluate the alignment between individual votes and group decisions, we calculate a weighted, normalised alignment score (\(A_i\)) for each voter in both homogeneous and heterogeneous rounds. For a voter \(i\), the score is defined as \( A_i = \frac{\sum_{j \in P_i} w_j}{|S_i|} \), where \(P_i\) is the set of projects the voter supported and were also selected by the group, \(w_j\) is the weight assigned to project \(j\), and \(|S_i|\) is the total number of projects the voter supported. This calculation is performed separately for the homogeneous and heterogeneous rounds to obtain two distinct alignment scores for each voter.

\subsection{Human-in-the-Loop Method of Equal Shares (Human-in-the-Loop MES)}

\subsubsection{\textbf{Motivation}}
The Human-in-the-Loop MES is a practical extension of the Method of Equal Shares algorithm, co-designed with KK24 for their needs, to enhance transparency and build algorithmic trust by incorporating real-time digital feedback. This method allows participants to control the extent to which decision-making is delegated to the voting aggregation algorithm versus reserved for deliberation. 

\subsubsection{\textbf{Method of Equal Shares (MES)}}

The Greedy method, commonly used in standard participatory budgeting, selects projects with the highest votes but ignores proportionality. While simple and efficient, it favours majority preferences and overlooks minority groups. The Method of Equal Shares (MES), proposed by \citet{peters2021}, ensures proportional and fair resource allocation in collective decisions. Each voter $i \in N$ is assigned an equal starting budget $b_i \geq 0$, determined iteratively. Here, $u_i(p)$ denotes the utility voter $i$ derives from project $p$, representing the perceived benefit or preference for the project. A project $p$ is $q$-affordable for $q \geq 0$ if its cost can be covered by voters contributing up to $q$ monetary units per unit of utility: MES proceeds as follows:
\begin{enumerate}
\item \textbf{Check Affordability:} Identify if any project is $q$-affordable for some $q$. A project is $q$-affordable when its total cost $c_p$ can be funded by voters, each contributing no more than $q$ monetary units per unit of utility. If no project is affordable, the process ends, returning the set of winners $W$.
\item \textbf{Select and Deduct:} Add the most affordable project $p$ to the set of winners $W$, and update each voter's budget:

This ensures that voters contribute only as much as their remaining budget allows, with contributions capped by their proportional share based on $q \cdot u_i(p)$.
\end{enumerate}

The starting budget begins at $B/n$, where $B$ is the total budget and $n$ is the number of voters. Incremental adjustments to the budget allow flexibility to explore alternative project selections without exceeding the total budget $B$. By iteratively selecting projects that align contributions with utility and budget constraints, MES guarantees proportionality. This ensures that groups with shared preferences are fairly represented in the selected set of projects.

\subsubsection{\textbf{The Human-in-the-Loop extension}} 

The Human-in-the-Loop MES process begins by aggregating votes using the MES algorithm applied to a partial budget. An interactive interface enables participants to adjust a slider controlling the total budget allocation from 0 to 380,000 CHF. This feature provides real-time visualisation of the projects that would be funded under different budget scenarios using MES calculations. After exploring the outcomes, participants collaboratively decide on a specific budget allocation for the MES calculation. In this study, 190,000 CHF --- 50\% of the total budget --- was allocated to the MES calculation, resulting in 18 projects being funded. 

Following the initial selection of projects based on the MES algorithm, participants engage in a deliberation round. During this round, they discuss whether the budget of any selected projects should be adjusted or whether any projects from the MES set should be eliminated. This additional step ensures that the final set of projects aligns with participant preferences and collective priorities, providing a balance between algorithmic decision-making and human judgment.

\subsection{ReadTheRoom Deliberation}

\subsubsection{\textbf{Motivation}} Conventional deliberations often lack concrete, data-supported, and actionable outcomes for effective policymaking. Discussions can become directionless, reiterating widely accepted opinions or focusing on tangential topics, thereby undervaluing citizens' time and effort. Moreover, the process designs frequently assume static opinions, failing to capture mutual learning and opinion shifts during deliberations. The proposed \textit{ReadTheRoom} deliberation introduces a structured approach to document opinion changes, providing insights into how public sentiment evolves through discussion and evidence-based results for policy-making.

\begin{figure}[h!]
\centering
\begin{minipage}[t]{0.49\linewidth}
    \centering
    \includegraphics[width=\textwidth]{polis_vtaiwan.pdf}
    \caption{The vTaiwan Deliberation Workshop on AI Regulation held in Taipei, Taiwan, on December 20, 2024. The moderator presented the Polis report showcasing the \enquote{group informed consensus }, calculated using the online votes collected prior to the event. (Photo credit: vTaiwan)}
    \label{fig:polis}
\end{minipage}%
\hfill
\begin{minipage}[t]{0.49\linewidth}
    \centering
    \includegraphics[width=\textwidth]{readtheroom.pdf}
    \caption{The vTaiwan ReadTheRoom game uses an interactive voting platform like Mentimeter to encourage participants to be open to different ideas. (Photo credit: vTaiwan)}
    \label{fig:readtheroom}
\end{minipage}
\label{fig:vtaiwan_tools}
\end{figure}

\subsubsection{\textbf{Online Phase: Wiki-Survey with Polis}} 

The \textit{Online Phase} uses \textbf{Polis}, a participatory wiki-survey platform, to enable scalable deliberation. Participants can submit statements and vote by agreeing, disagreeing, or abstaining, fostering the identification of consensus and divisive opinions. As described in \textit{The Computational Democracy Project} webpage \footnote{https://compdemocracy.org/algorithms/} and work by \citet{small2021}, Polis uses clustering algorithms to analyse responses and generate real-time reports. Fine-grained clustering applies K-means at $K=100$ for real-time visualisation, while coarse-grained clustering iterates K-means for $K=2$ to $5$, optimised using silhouette coefficients to ensure stability. The final report highlights group consensus and divisive statements, such as the identification of five opinion groups in the AI regulation deliberation in the vTaiwan Polis report\footnote{https://polis.tw/report/r3dvith8ntmwywyf4nctc}.


\subsubsection{\textbf{Offline Phase: Decision Tree and Face-to-Face Deliberation}}

Building on the insights from the \textit{Online Phase}, the proposed \textit{Offline Phase} integrates digital data into structured physical deliberations. This phase, co-designed with vTaiwan, ensures that discussions are grounded in evidence while allowing participants to learn and refine their opinions through direct interaction.

\begin{enumerate}
    \item \textbf{Decision Tree Creation:} The clusters and divisive statements identified in the Polis report are used to construct a decision tree. This tree visually represents how the crowd is divided by key issues, serving as a roadmap for discussions.
    \item \textbf{Deliberation Event:} Structured deliberations based on divisive statements unfold in the following steps:
    \begin{enumerate}
        \item \textbf{Pre-Discussion Voting:} Participants use Mentimeter to vote on divisive statements via a 5-point Likert scale (e.g., from strongly disagree to strongly agree).
        \item \textbf{Discussion Round:} Moderators facilitate dialogue by inviting participants from diverse opinion groups to share their perspectives and reasoning. This encourages open dialogue and mutual understanding.
        \item \textbf{Post-Discussion Voting:} Participants re-vote on the same statements after discussions. Moderators prompt participants to reflect on and share reasons for any changes in their stance.
    \end{enumerate}
\end{enumerate}

In this iteration, 104 participants voted online using Polis by 17th December 2024, while 44 self-invited participants attended a physical deliberation workshop on 20th December in Taipei. Although the two groups overlap, they are not subsets of each other, as some workshop attendees did not participate in the online voting. The two phases were conducted independently, with the online results serving to collect public opinion and guide the physical deliberation. No demographic data was collected.

\subsubsection{{\textbf{Evaluation Metrics}}}
To evaluate the impact of deliberation, we used four key metrics. The Percentage of Voters Changed captures the proportion of participants who shifted their stance. The \textit{Polarisation Index}, measuring opinion dispersion, is calculated as the standard deviation of responses divided by the response range. The \textit{Consensus Index} quantifies group alignment as the proportion of responses matching the majority opinion. Lastly, the \textit{Mean Opinion Change} evaluates shifts in sentiment as the difference between the average post-deliberation and pre-deliberation scores. These metrics assess how deliberation influences opinions, reduces polarisation, and fosters consensus.

\section{Data and Implementation}

We apply a data-driven approach to quantitative voting data to understand voting patterns and preferences. This provides insights into the efficacy of these participatory mechanisms. Complementary, we evaluate qualitative feedback that was collected through interviews and surveys by the organising committees. This offers a deeper understanding of how these voting patterns influence deliberative outcomes, adding an additional layer of context to the quantitative data. The responses in German (KK24) and Traditional Chinese (vTaiwan) were translated using DeepL before the analysis. 

As researchers advising citizen groups in this participatory research, we ensured academic integrity by clearly defining our roles as advisors and observers. In both roles, the processes were co-designed with the organising committees through multiple design meetings from late 2023 to 2024, which shaped the methods and tools to suit their needs. We guided the application of methods, built interfaces and explanations, and facilitated technical implementation while the organising committees and participants then took their decisions autonomously. The data analysed in this study are all secondary and anonymised, independently collected by the organising committees of \textbf{KK24} and \textbf{vTaiwan} during their processes. No primary data collection was conducted by the research team. For \textbf{KK24}, the voting data and survey responses were gathered with signed consent after participants were briefed on the purpose, data use, privacy, and withdrawal rights. Data collected by vTaiwan using Polis and Mentimeter included no personal information, was also obtained with participant consent, and is publicly available as open data on vTaiwan's GitHub\footnote{github.com/v-taiwan/241220-AI-Regulation}.

\section{Results}

\subsection{Radial Clustering for Preference Based Subgroups}
\subsubsection{\textbf{Heterogeneous deliberation outcome mirror the voting outcome more closely}}

\begin{figure}[ht]
    \centering
    \includegraphics[width=0.8\textwidth]{points_votes.pdf}
    \caption{Votes vs Deliberation Points in Homogeneous and Heterogeneous Rounds. 
    The figure illustrates the relationship between the individual votes projects received during the online voting phase (x-axis) and the aggregated points assigned by the groups during the physical group deliberation (y-axis). The left panel represents the homogeneous round, where deliberation groups consisted of participants with similar voting patterns. The right panel represents the heterogeneous round, where groups were composed of participants with diverse voting patterns. Each dot corresponds to a project, with its position determined by the number of votes it received and the deliberation points assigned to it. Regression lines and confidence intervals are shown to highlight trends.}
    \label{fig:points_votes}
\end{figure}

The p-values in Figure~\ref{fig:points_votes} reveal a significant correlation between individual voting outcomes and group-assigned deliberation points in the heterogeneous round ($p = 0.000527$), but a non-significant correlation in the homogeneous round ($p = 0.0943$). This suggests that in heterogeneous deliberation groups, where diverse perspectives are expected to emerge, the final group decisions closely mirror individual online voting outcomes. We also observed more groups using sticky notes as votes to make collective decisions in the heterogeneous round, as shown in Figure~\ref{fig:notes}. This pattern is further illustrated in Appendix Table~\ref{tab:final_projects}. Among the 17 projects selected during the deliberation process, 7 overlap with those that would have been chosen if online voting alone determined the funding allocation for this portion of the budget. Compared to a hypothetical pure voting scenario, heterogeneous deliberation introduces fewer unique projects. Only 3 projects were selected primarily due to heterogeneous deliberation (tagged HT), while 7 projects were selected primarily due to homogeneous deliberation (tagged HM). This suggests that homogeneous deliberation incorporates a wider range of projects that diverge from individual voting outcomes.

One possible explanation is that participants with diverse preferences in heterogeneous groups find it challenging to reach consensus, often resorting to in-group voting to resolve differences. These findings highlight the need to critically assess the impact of group dynamics in deliberative processes within systems where voting already aggregates preferences effectively.

\subsubsection{\textbf{Heterogeneous deliberation tends to fund more costly projects}}
As detailed in Table~\ref{tab:deliberation_analysis} in the appendix, the hypothetical outcomes represent the project selections if the 190,000 CHF budget were allocated to projects based solely on decisions from homogeneous or heterogeneous groups. Heterogeneous deliberation funded fewer but more expensive projects than homogeneous deliberation, as evidenced by higher mean (18,990 CHF vs. 13,350 CHF) and median (13,000 CHF vs. 11,450 CHF) project costs. These findings suggest that diverse perspectives may lead to prioritising larger or higher-impact projects. 

Although a Mann-Whitney U Test revealed that the difference in project cost distributions between the two groups is not statistically significant (U = 49.0, p = 0.22715) given the smaller data sample, the trend toward funding costlier projects under heterogeneous deliberation remains noteworthy. This observation may have important implications for resource allocation strategies in participatory processes.

\begin{figure}[ht]
    \begin{minipage}[t]{0.49\linewidth}
        \centering
        \includegraphics[width=\linewidth]{kk_deliberation.jpg}
        \caption{In the heterogeneous deliberation round, more participants started using the sticky notes available to indicate their individual support for the project and subsequently decide what projects they should select collectively as a group. Names are blurred out for privacy reasons.}
        \label{fig:notes}
    \end{minipage}
    \hfill
    \begin{minipage}[t]{0.49\linewidth}
        \centering
        \includegraphics[width=\linewidth]{mes_adjust.jpg}
        \caption{After the budget delegated to the algorithm was decided and the MES selection of projects was confirmed, participants went through the budget of each project to decide whether the proposed budget was suitable or not. They collectively voted on whether the budget should be reduced for the selected projects.}
        \label{fig:adjust}
    \end{minipage}%
\end{figure}

\subsubsection{\textbf{More participants perceived homogeneous deliberation to be easier, with outcomes reflecting their preferences}}

As shown in Figure~\ref{fig:survey_group}, participant responses highlight notable differences in the perceived ease of decision-making and the extent to which outcomes reflected their preferences across the two rounds of deliberation. The first round, organised using Radial clustering to create homogeneous groups, was perceived as easier, with 83\% of participants agreeing that reaching a decision was easy (52\% \enquote{Somewhat agree} and 31\% \enquote{Strongly agree}). 65\% of the participants agreed in the second round, where Radial clustering was used to form heterogeneous groups (31\% \enquote{Somewhat agree}, and 34\% \enquote{Strongly agree}). Similarly, decisions made by the homogeneous groups were more likely to reflect participants' preferences, with 76\% agreeing (62\% \enquote{Somewhat agree} and 14\% \enquote{Strongly agree}), compared to 55\% in the heterogeneous groups (34\% \enquote{Somewhat agree} and 21\% \enquote{Strongly agree}). These results suggest that using Radial clustering to organise groupings creates a perceivable difference for participants, with homogeneous groupings leading to smoother decision-making and outcomes more aligned with individual preferences.

\begin{figure}[h!]
  \includegraphics[width=0.8\textwidth]{socio.pdf}
  \caption{Alignment Changes Across Homogeneous and Heterogeneous Rounds by Gender and Age Group. This figure shows the alignment of participants during the homogeneous and heterogeneous rounds. Each subplot represents a demographic group (gender or age group), with the age groups created by dividing the sample into thirds. Grey lines connect individual participant alignments, with larger grey dots marking individual alignments in each round. Violin plots display the distribution of alignments with embedded box plots showing medians and interquartile ranges. Red dots indicate mean alignments for each round, connected by a solid red line to illustrate mean trends. The Wilcoxon signed-rank test p-values are displayed at the bottom of each subplot to assess the statistical significance of the differences between rounds.}
  \label{fig:socio}
\end{figure}

\subsubsection{\textbf{Younger participants exhibit significant alignment differences between deliberation rounds}}

Figure \ref{fig:socio} compares alignment, the proportion across homogeneous and heterogeneous deliberation rounds, segmented by gender and age groups. In the homogeneous round, younger participants (Age $\leq 33$) and female participants exhibit higher median alignment (0.33 and 0.35, respectively), while older participants (Age $>$ 42) and male participants have lower alignment medians (0.23 and 0.32, respectively). In the heterogeneous round, alignment for younger participants and females decreases (to 0.28 and 0.30), whereas older participants and males experience increases (to 0.34 and 0.37, respectively).

Due to the smaller sample sizes within each group, only the difference in alignment for younger participants meets the threshold for statistical significance ($p=0.047$). For the other groups, the observed differences do not reach statistical significance ($p=0.534$ for males, $p=0.117$ for females, $p=0.066$ for middle-aged participants, and $p=0.093$ for older participants). 

The results suggest that homogeneous deliberation may create a setting where younger and perhaps also female participants exhibit higher alignment, potentially because of fewer conflicting perspectives or reduced influence from dominant voices. However, the lower alignment observed for these groups in heterogeneous deliberations could also indicate a greater openness to new ideas or a tendency to adjust preferences in response to diverse viewpoints. These interpretations remain speculative, as the specific reasons driving these patterns are unclear and require further investigation.

\subsubsection{\textbf{The perceptions of group dynamics of deliberation were mixed}}

While most participants did not specifically comment on differences between the two rounds of deliberation, they perceived the deliberations differently. Some participants noted difficulties in ensuring equal participation within small groups. P01 observed, \textit{\enquote{When working in groups, the stronger voices sometimes had more say. I then tried to mediate, which was sometimes difficult.}} P07 echoed these challenges but appreciated the structure provided by facilitators, noting, \textit{\enquote{The main challenge to contribute my opinion that I have developed or to stand up for it / to share myself  in small groups... However, (KK24) they have made this as simple as possible with the structure.}} Others highlighted the inclusivity fostered by group dynamics. P04 remarked positively, \textit{\enquote{Through the different groups, there was always the opportunity to share one’s voice.}} Similarly, P21 appreciated the multiple avenues for participation, stating, \textit{\enquote{I find my opportunity to participate in decision-making to be appropriate overall since various opportunities for participation (voting, polls, meetings in smaller groups) were provided.}} Participants also reflected on the balance between individual preferences and collective outcomes. P16 praised the alignment between decision-making and group interests, stating, \textit{\enquote{The decision-making process was good and based on the group's own interests.}} P27 emphasised the effective integration of individual and group perspectives, describing it as a \textit{\enquote{Very good mix of individual opinions and discussion.}} While homogeneous groups may foster alignment by reducing conflicting perspectives, heterogeneous groups encourage engagement with diverse viewpoints, which may lead to some of these mixed feelings regarding the group dynamic. Future efforts should address the influence of dominant voices while enhancing structures that support fair participation and constructive dialogue.


% P01: When working in small groups, the stronger voices sometimes had more say. I then tried to mediate, which was sometimes difficult.
% P04: Through the different groups there was always the opportunity to share one’s voice
% P07: (The main challenge is) To contribute myself and my opinion that I have developed (heart for certain projects) or to stand up for it / to share myself (also in small groups). However, Noemi \& Mia have made this as simple as possible with the structure
% P13: I found that the selection - at least in my case - reflected/complemented the results of the shortlist very well. There were certainly projects that I personally wasn't particularly interested in, but I found the majority of the projects that were funded to be worthwhile.
% P16: The decision-making process was good and based on the group's own interests. The last workshop in particular was great.
% P21: Since the group decides and there are no individual decisions, I find my opportunity to participate in decision-making to be appropriate overall, since various opportunities for participation (voting, polls, meetings in smaller groups) were provided.
% P27: Very good mix of individual opinions and discussion.


\subsection{Human-in-the-loop MES}

\begin{figure}[ht]
    \centering
    \includegraphics[width=0.8\textwidth]{mes_grdy.pdf}
    \caption{
        Outcome comparison of MES (\textit{mes190k}), as implemented in KK24, and a hypothetical Greedy Method baseline (\textit{grdy190k}) under a budget constraint of 190,000 CHF. The left panel shows the number of projects won per voter, and the right panel shows the budget allocation per voter. Individual voter trajectories are represented by connecting lines, where light green lines indicate an increase in the outcome under MES compared to the Greedy Method, and grey lines indicate no increase. Boxplots summarise the distributions, showing the median (central line), interquartile range (box), and whiskers extending to 1.5 times the IQR or the data extremes. Mean (green) and median (red) markers are overlaid, and the Gini coefficient in the right panel quantifies the fairness of budget allocations.
    }
    \label{fig:mes}
\end{figure}

\subsubsection{\textbf{MES distributes budget more fairly and fund more projects}}

In Figure~\ref{fig:mes}, under the same budget constraint of 190,000 CHF, the Method of Equal Shares (MES) funds more affordable projects than the conventional and commonly used Greedy Method (taking projects with the highest votes), leading to a higher number of projects won per voter. While the budget allocation per voter is similar in both methods, MES distributes resources more fairly, giving more to voters who receive less under the Greedy Method. This redistribution lowers the Gini coefficient from 0.17 to 0.14, reflecting a fairer outcome. The light green lines highlight these changes, showing how MES benefits voters disadvantaged by the Greedy Method while staying within the same budget.


\subsubsection{\textbf{Participants value Human-in-the-loop MES for balancing algorithmic decisions and human input}}

After being shown different MES budget scenarios, participants opted to allocate 50\% of the budget using MES. The interactive interface displayed 18 selected projects, and participants were divided into randomly assigned groups to discuss potential vetoes or budget adjustments. None of the groups vetoed any projects, indicating approval of the algorithmically selected set. However, participants proposed reducing the budget for the project \enquote{Kultur-Agenda,} an online calendar for art and cultural events, from 18,000 CHF to 9,000 CHF. This additional step allowed participants to refine algorithmic decisions, balancing automation with human judgment.

Figures ~\ref{fig:survey_portion} and ~\ref{fig:survey_mes} in the appendix present participant responses to two key survey questions about MES fairness and the balance between voting and deliberation. The responses indicate that participants generally perceived MES as fair, with 62\% rating it \enquote{Very fair} and 23\% rating it \enquote{Somewhat fair.} A majority expressed support for maintaining the 50:50 ratio of individual voting to group deliberation, with an overwhelming 81\% preferring the same 50:50 proportion and only 15\% and 3\% advocating for having a higher portion of voting or deliberation, respectively. %There is strong support for MES and the current 50:50 balance between voting and deliberation.

In the open-ended feedback questions, participants generally expressed appreciation for the fairness and structure of MES but highlighted areas for improvement in preparation and communication. Many valued its ability to ensure all votes are counted equally, as P02 noted, \enquote{\textit{It is pragmatic and valuable to count every vote that is cast. Allows less verbal people to have equal weight given to their voice.}} Others praised its practical approach to budget management, with P09 stating, \enquote{\textit{The Method of Equal Shares provided me with a reassuring guide.}} P19 shared, \enquote{\textit{It's nice that half of the budget is distributed automatically, and the rest is then discussed and decided in detail.}} P03 commented on the balance of methods, saying, \enquote{\textit{The good thing was the mix of analogue and digital.}} P27 added, \enquote{\textit{The best thing was the mixture of individual decisions and group decisions}.} P21 acknowledged the trade-offs inherent in collective decision-making, commenting, \textit{\enquote{Since it was a large group, the opinions of individuals may have been lost a little, as many things were decided by voting, but I still felt heard and taken seriously.}} Suggestions for enhancement included utilising online voting more extensively (P20: \enquote{\textit{More online voting; With this voting method (MES), there is no external influence.}}).

While participants’ feedback did not focus much on the flexibility of Human-in-the-loop MES, since the 50:50 voting-deliberation ratio was established early on in the discussion, they frequently highlighted the value of combining voting with deliberation. This mix was seen as an effective way to balance individual input with collective decision-making, ensuring fairness and practicality.

% P02: Pragmatic and valuable to count every vote that is cast. Allows less verbal people to have equal weight given to their voice.
% P03: (The best thing was) The mix of analog and digital.
% P09: The Method of Equal Shares provided me with a reassuring guide.
% P19: (The best thing was) That half of the budget is distributed automatically and the rest is then discussed and decided in detail.
% P20: There is no outside influence in this voting method. Fewer group discussions. Use more online voting.
% P23: (MES) Was introduced abruptly, with an explanation, but I would have liked to have informed myself about it beforehand
% P27: (The best thing was) Mixture of individual decisions and group decisions.

\subsection{The ReadTheRoom Deliberation Method}
\begin{figure}[h!]
  \includegraphics[width=0.9\textwidth]{decision_tree.jpg}
  \caption{Translated Decision Tree Screenshot from the vTaiwan AI Regulation Deliberation Workshop on December 20, 2024. Based on data collected through the Polis wiki-survey, this decision tree visualises how different opinion groups diverge in their opinions on AI governance. The opinion space is summarised using ChatGPT, interpreting key statements that stand out within each group. This illustration was used during the workshop to explain to the participants why certain statements were used as the central theme of the deliberative event.}
  \label{fig:tree}
\end{figure}


\subsubsection{\textbf{Voting shows how deliberation reduces polarisation and builds consensus}}

Using the decision tree in Figure ~\ref{fig:tree}, participants started deliberating the top five statements that divided the online opinion, gathered using Polis beforehand. Figure~\ref{fig:change} illustrates the impact of deliberation on participant opinions across the five statements, providing insights into opinion shifts using four key metrics. The \textit{percentage of voters changed} shows significant variability, with 53.1\% of participants changing their stance on Statement 2 (AI companies should disclose algorithms and data), the highest among the statements. In contrast, only 30.0\% of participants changed their opinion on Statement 3 (AI’s industrial impact should be solved by the market), indicating more entrenched positions. The \textit{Polarisation Index} highlights how deliberation reduced opinion dispersion, particularly for Statement 2 (0.75), whereas Statement 3 retained high polarisation (0.95), reflecting persistent disagreement. The \textit{Consensus Index} reveals increased alignment for most statements; for example, Statement 5 (AI developers must be accountable for harm caused by AI content) improved from 0.44 to 0.47, while Statement 4 (Governments should prevent AI reinforcing stereotypes) rose from 0.41 to 0.43. Finally, \textit{Mean Opinion Changes} capture shifts in group sentiment, such as Statement 4 shifting significantly from \textit{-0.17} to \textit{0.69}, demonstrating a strong movement toward agreement. Conversely, Statement 2 shifted slightly toward disagreement (\textit{-0.03} to \textit{-0.34}). Together, these metrics reveal that deliberation effectively fosters opinion shifts, reduces polarisation, and builds consensus, though its impact varies depending on the statement and participants' initial positions.

\begin{figure}[ht!]
  \includegraphics[width=0.8\textwidth]{change.pdf}
  \caption{Opinion Changes Before and After Deliberation. This figure shows the change in opinion before and after deliberation for each statement. Each row represents a statement, with box plots and jittered points displaying the distribution of opinions on a Likert scale (-2 to 2, from Strongly Disagree to Strongly Agree). Blue corresponds to opinions before deliberation, and orange corresponds to opinions after deliberation. Arrows indicate the direction of the mean opinion change, with annotated mean values before and after deliberation. Statistics on the right-hand side include the percentage of voters who changed their opinion, the Polarisation Index (standard deviation after divided by before), and the Consensus Index (1 divided by 1 + standard deviation) for both before and after deliberation.}
  \label{fig:change}
\end{figure}
\subsubsection{\textbf{Participants were able to voice their diverse opinions with ReadTheRoom method}}

In the survey (Figure~\ref{fig:survey} in the appendix) conducted at the end of the deliberation with 5-point Likert scale (-2 to 2), the ReadTheRoom Deliberation method demonstrates its strength in fostering mutual learning among participants. The highest mean (1.60) for \enquote{The discussion today helped me learn new perspectives} highlights the process’s effectiveness in broadening participants’ understanding. The positive responses suggest that the deliberation created an environment conducive to knowledge exchange and mutual understanding. Additionally, the significant agreement on \enquote{real-time voting} (a core aspect of the ReadTheRoom method) allowing diverse views to be expressed (1.57) and the constructive consensus-building process (1.13) further illustrates the method’s ability to encourage participants to reflect on diverse perspectives and adapt their views. By visualising opinion changes transparently, the deliberation fosters a shared learning experience where participants collectively engage with divisive issues and work toward consensus. Most participants also see the point of having a more data-supported deliberation outcome, as they agree that the combination of voting and deliberation makes the outcomes more convincing (1.47).

Feedback from participants also shows that the method was able to prompt meaningful dialogue. P05 remarked, \enquote{\textit{There were many deep experiences and opinions from various fields!}}. P08 noted, \enquote{\textit{Great, the data generated today is very valuable!}}. P10 expressed, \enquote{\textit{I learned a lot about different positions and their concerns, and I gained a lot!}}. P13 shared, \enquote{\textit{My understanding of related issues became much clearer}}. Some feedback recommended that vTaiwan use even more digital tools and demand more from citizen input. P21 suggested, \enquote{\textit{I recommend asking participants to also document changes in their attitudes simultaneously}}. While P39 proposed, \enquote{\textit{More digital tools can be used to continue the dialogue online for deeper discussions in the future}}.

\section{Discussion and Conclusion}
\subsection{Practical Impact}

This study examined three digital methods --- {Radial Clustering for Preference Based Subgroups}, {Human-in-the-loop MES}, and {ReadTheRoom} Deliberation --- implemented in two distinct participatory processes. While conducted on a relatively modest scale, these method offers innovative enhancements with broader implications.

First, Radial clustering facilitated the creation of small, evenly sized breakout groups during the deliberation phase of KK24. Differences in outcomes between deliberation rounds underscore the impact of varying group compositions on decision-making dynamics. This method aligns with the emphasis on bridging enclaves with the wider deliberative process \cite{sunstein2017}. 

Crucially, this approach demonstrates how Enclave Deliberation can be achieved without stereotyping or profiling citizens, relying solely on real preferences derived from online voting data. This aligns with \citet{abdullah2016}'s emphasis that enclaves should \enquote{represent marginalised perspectives or social locations rather than essentialised identities}. Such an approach avoids reductive assumptions about individuals' views based on their demographic characteristics, instead empowering disadvantaged groups to contribute autonomously and effectively to broader deliberations. Future research could examine the impact of this clustering approach on the quality and outcomes of deliberation in larger-scale settings.

Second, Human-in-the-loop MES offered a novel approach to adjusting project budgets within participatory budgeting frameworks. This approach is especially valuable for scenarios with flexible budgets, a concept rarely addressed in participatory budgeting literature due to the challenges of bridging theoretical frameworks with real-world possibilities. However, in contexts where deliberation is integral to the participatory process, Human-in-the-loop MES enables adaptable budgeting with transparent and engaging tools for monitoring and exploration. This approach could be applied to other algorithm-supported collective decisions, ensuring citizens retain control not only over outcomes but also over the extent of algorithmic involvement in the process.

Lastly, vTaiwan's {ReadTheRoom deliberation} merged the online opinion space with the physical deliberation space through a gamified spectrum, using Likert-scale questions to position participants from disagreement to agreement in two rounds: before and after the discussion. This approach ensured that participants with differing stances had a platform to voice their views while generating quantifiable data to support deliberation and provide policymakers with clear insights into citizen opinions. The interactive design encouraged participants to engage dynamically, listen, learn, and adapt their views, making deliberation both enjoyable and collaborative. By integrating real-time public input and voting data, ReadTheRoom added legitimacy to the process, enabling participants to justify their positions with evidence and giving policymakers actionable insights.

% Lastly, ReadTheRoom Deliberation enabled the identification of pivotal talking points and excluded already established consensual statements in vTaiwan’s deliberations, incorporating real-time feedback from the general public via wiki surveys such as Polis. While in-person deliberations (such as citizens' assemblies) are traditionally limited by size and representation, ReadTheRoom extends these processes to a broader audience, adding depth and legitimacy to the outcomes. Here, the general public can inform the representatives, who can then justify certain positions based on the live-collected evidence in the database. However, when establishing such processes in high-stakes scenarios, it is crucial to ensure that interest groups do not hijack the general public's surveys, introducing unwanted bias into the deliberation.

\subsection{Limitations}

Throughout the application and research, we identified limitations affecting both the methods and their impact. One overarching limitation of this study is its reliance on case studies with small sample sizes. While these offer valuable insights, they limit the generalisability of findings and may not fully capture the complexities of larger-scale applications. However, we chose this approach because these methods are highly relevant for real-world situations, where small group settings and practical constraints often shape deliberative processes. Future research should validate these methods in more diverse and extensive settings to ensure broader applicability.

For group composition with Radial clustering, the quantitative differences in agreement between the two conditions (homogeneous and heterogeneous) could have been influenced by confounding factors such as gender and age. Group demographics may have impacted dynamics beyond the intended effects of voting-based clustering. Additionally, the algorithm’s use of a low-dimensional space (e.g., PCA) prioritised visual clarity over precision, potentially losing critical information. While the method ensures equal group sizes, future work should refine the algorithm to balance transparency with and clustering accuracy.

Human-in-the-loop MES allowed flexibility in adjusting budget ratios and modifying individual project budgets, but some challenges were identified. The KK24 organisers set a 50:50 default budget ratio, potentially creating a default bias. Although participants considered alternatives, they ultimately retained the default. Future research should explore whether removing preset defaults changes decision-making outcomes. Additionally, while participants could reduce some project budgets, this flexibility might compromise the proportionality MES aims to ensure, potentially leading to a sub-optimal allocation of resources. Pre-adjusting project budgets could enhance fairness, but it may also place an excessive burden on participants.

In the ReadTheRoom Deliberation, participants voted on the same statements twice: before and after the discussion. Some participants raised concerns about the definitions and wording of these statements, suggesting that updating them in real-time to reflect ongoing conversations could improve the process. A potential future direction could involve using AI-driven summarisation of deliberations based on real-time transcripts to dynamically update the statements. This approach could make responses more relevant and insightful. While the interactive design effectively fosters discussion, a notable limitation of real-time visualisation is its potential to influence individual voting, creating a bias toward prevailing opinions. To mitigate this, incorporating some level of secret voting could be essential, particularly in high-stakes political scenarios.


\subsection{Concluding Remarks}

Deliberation and voting are not opposing endpoints but rather complementary points on a spectrum of democratic innovation \cite{weyl2024}. By combining computational methods like {Radial clustering}, {Human-in-the-loop MES}, and {ReadTheRoom}, this study shows how algorithms can bridge the gap between deliberation and voting, addressing the trade-offs between depth and breadth in democratic processes. The advancements presented here point toward a future where richer collaboration and broader participation are not mutually exclusive; algorithms and thoughtful design can help democratic innovations overcome traditional constraints for fairer, more impactful decision-making.

\begin{acks}

We thank the entire \textit{KK24 committee} and \textit{vTaiwan community}, as well as all the citizens who participated, for their invaluable contributions to this research and to participatory democracy. Special thanks to \textit{Noemi Scheurer} and \textit{Mia Odermatt}, the main organisers of Kultur Komitee Winterthur, for their openness to experimenting with innovative methods. We also thank \textit{Jia-Wei Cui}, the current organiser and moderator of vTaiwan, for his commitment, and \textit{Yi-Ting Lien} for co-hosting these deliberation workshops and connecting vTaiwan with TWNIC. Additionally, FB gratefully acknowledges the financial support of the Swiss National Science Foundation (SNSF) under grant ID CRSII5-205975.
\end{acks}

\bibliographystyle{ACM-Reference-Format}
\bibliography{Library}

\clearpage

\section{Appendices}
\appendix

\section{Radial Clustering for Preference Based Subgroups}

\begin{figure}[H]
\centering
\begin{minipage}[t]{0.53\linewidth} 
    \vspace{30pt} 
    \centering
    \begin{tabular}{l|c|c}
    \textbf{Metric} & \textbf{Homogeneous} & \textbf{Heterogeneous} \\
    \hline
    Total Budget & 186,900 & 189,900 \\
    Number of Projects & 14 & 10 \\
    \textbf{Mean Cost} & \textbf{13,350.00} & \textbf{18,990.00} \\
    Median Cost & 11,450.0 & 13,000.0 \\
    Cost Std. Dev. & 8,581.71 & 12,657.05 \\
    \hline
    Mann-Whitney U & \multicolumn{2}{c}{49.0 (p-value: 0.22715)} \\
    \end{tabular}
    \caption{Comparison of cost distributions for homogeneous and heterogeneous deliberation groups. The table includes key metrics such as total budget, project cost statistics, and voting averages, along with the results of the Mann-Whitney U Test.}
    \label{tab:deliberation_analysis}
\end{minipage}%
\hfill
\begin{minipage}[t]{0.45\linewidth} 
    \vspace{0pt}
    \centering
    \includegraphics[width=\linewidth]{survey_group.pdf}
    \caption{Participant responses on the ease of decision-making and alignment with preferences during the first (homogeneous) and second (heterogeneous) rounds of deliberation.}
    \label{fig:survey_group}
\end{minipage}
\end{figure}

\definecolor{LightBlue}{RGB}{240, 250, 240} % For tag 'HM'
\definecolor{LightRed}{RGB}{220, 240, 250}  % For tag 'HT'
\definecolor{LightGrey}{RGB}{245, 245, 245} % For tag 'HM/HT'

\subsection{KK24 Outcome Table}

\begin{longtable}{>{\raggedright\arraybackslash}p{1cm} p{8cm} r r r c c}
\caption{Overview of Final Projects Categorised by Deliberation Outcome Tags\\
\smallskip
\small
\textbf{HM} -- Projects that won mainly due to Homogeneous deliberation, where HM points exceed HT points; \\
\textbf{HT} -- Projects that won mainly due to Heterogeneous deliberation, where HT points exceed HM points; \\
\textbf{HM/HT} -- Projects where HM points equal HT points. \\
This table provides a breakdown of the 17 final projects selected during the deliberation process, highlighting the deliberation method that played the most significant role in their selection. Projects tagged as HM reflect the outcomes of discussions within more like-minded groups, HT tags indicate the influence of diverse perspectives from heterogeneous groups, and HM/HT tags indicate balanced influence from both.}

\label{tab:final_projects} \\
\toprule
\textbf{ID} & \textbf{Project Name} & \textbf{Cost (CHF)} & \textbf{HM Pts} & \textbf{HT Pts} & \textbf{Tag} & \textbf{Vote} \\
\midrule
\endfirsthead

\toprule
\textbf{ID} & \textbf{Project Name} & \textbf{Cost (CHF)} & \textbf{HM Pts} & \textbf{HT Pts} & \textbf{Tag} & \textbf{Vote} \\
\midrule
\endhead

\bottomrule
\endfoot

% HM tagged projects (LightBlue)
\rowcolor{LightBlue}
5 & Wir, hier. Briefe an Winterthur & 24,500 & 5 & 3 & HM &  \\
\rowcolor{LightBlue}
14 & Haltestelle 21: Pedibus de la Culture – De Pedibus vo Winti & 20,000 & 3 & 2 & HM &  \\
\rowcolor{LightBlue}
20 & Cashflow & 25,000 & 7 & 2 & HM &  \\
\rowcolor{LightBlue}
65 & Clown-Theater Schanz \& Ganz & 20,000 & 4 & 3 & HM &  \\
\rowcolor{LightBlue}
78 & Aufführung des Oratoriums \enquote{Die Schöpfung} von Joseph Haydn & 5,000 & 5 & 2 & HM &  \\
\rowcolor{LightBlue}
93 & Gesehen & 13,000 & 9 & 8 & HM &  \\
\rowcolor{LightBlue}
142 & \enquote{copy paste} & 4,500 & 2 & 1 & HM &  \\

% HT tagged projects (LightRed)
\rowcolor{LightRed}
9 & F --- E Filmpreis & 28,000 & 3 & 5 & HT &  \\
\rowcolor{LightRed}
76 & Peter und der Wolf – Konzertsaison 24/25 & 40,000 & 1 & 3 & HT &  \\
\rowcolor{LightRed}
147 & Konzertreihe Salle Bolivar & 10,000 & 1 & 4 & HT &  \\
\rowcolor{LightRed}
16 & Dokfilm: Die Unsichtbaren (Arbeitstitel). & 20,000 & 6 & 10 & HT & \checkmark \\
\rowcolor{LightRed}
102 & Afro-Classics im Rahmen des Afro-Pfingsten Festivals 2025 & 8,000 & 7 & 8 & HT & \checkmark \\
\rowcolor{LightRed}
117 & lauschig – wOrte im Freien 2024 & 40,000 & 0 & 10 & HT & \checkmark \\
\rowcolor{LightRed}
126 & Connection instead Addiction & 13,000 & 13 & 16 & HT & \checkmark \\

% Projects with equal HM and HT points (LightGrey)
\rowcolor{LightGrey}
81 & AlbTraumWelt & 9,900 & 3 & 3 & HM/HT & \checkmark \\

% Voted projects tagged appropriately
\rowcolor{LightBlue}
12 & Bambolini! & 5,000 & 7 & 2 & HM & \checkmark \\
\rowcolor{LightRed}
112 & Sub Factory & 8,000 & 5 & 6 & HT & \checkmark \\
\end{longtable}



\section{Human-in-the-loop MES}

\begin{figure}[H]
    \centering
    \includegraphics[width=0.9\linewidth]{mes_ui.pdf}
    \caption{Screenshots of the Human-in-the-loop MES interface used in the KK24 workshop for participants to clearly understand the implications and the final selections of projects with MES calculation and ajustable budget}
    \label{fig:ui}
\end{figure}

\begin{figure}[H]
\centering
\begin{minipage}[t]{0.49\linewidth}
\centering
\includegraphics[width=\linewidth]{survey_portion.pdf}
\caption{Participant responses to how the ratio between voting, calculated using the MES algorithm, and deliberation should be adjusted in future processes.}
\label{fig:survey_portion}
\end{minipage}%
\hfill
\begin{minipage}[t]{0.49\linewidth}
\centering
\includegraphics[width=\linewidth]{survey\_mes.pdf}
\caption{Participant responses to the fairness of the Method of Equal Shares.}
\label{fig:survey_mes}
\end{minipage}
\end{figure}





\section{ReadTheRoom}

\begin{figure}[H]
  \includegraphics[width=0.8\textwidth]{survey_vtaiwan.pdf}
  \caption{\textbf{Survey Responses from the vTaiwan Deliberation.} This diagram shows participant responses to six survey questions after deliberation. The vertical axis lists the questions, and the horizontal axis represents a Likert scale from -2 (\textit{Strongly Disagree}) to 2 (\textit{Strongly Agree}). Grey areas indicate the kde distribution of responses, coloured dots represent individual responses (blue for disagreement, grey for neutrality, orange to red for agreement), and red circles show the mean response with annotated values.}
  \label{fig:survey}
\end{figure}

\end{document}
\endinput
%%
%% End of file `sample-sigconf-authordraft.tex'.
