\section{Literature Review}
\subsection{Hybrid Democratic Innovation}

Hybrid Democratic Innovations (HDIs) combine concentrated deliberation with large-scale voting ____ to enhance inclusivity, reflection, and policy impact. Early examples of HDIs were given by ____ with deliberative polling. This method has been described by ____ to capture informed public opinion through random sampling, structured deliberation, and feedback. Building on this, ____ proposed \enquote{democratic assemblage,} emphasising adaptive systemic integration of deliberative and participatory processes. This has been illustrated by ____, who showed how digital platforms and game mechanics can enhance participatory budgeting through motivation and engagement. Practical applications include the three-step model of ____ for climate policymaking, which combines mini-publics and maxi-publics to produce widely accepted recommendations, and ____’s study of Antwerp’s Citizens’ Budget, showcasing how \enquote{participatory budgeting new-style} can align deliberation with large-scale voting to uphold democratic values. Referendum processes have similarly benefited, as ____ identify how deliberation mitigates deficits of referenda. ____ propose deliberative referenda to connect citizens’ assemblies with direct voting, thereby enhancing public debate and systemic legitimacy. Finally, ____ theorise the complementary role of voting in deliberation to provide closure, equality, and accountability, highlighting the importance of hybrid approaches in advancing democratic innovation.

\subsection{Online Tools for Deliberative Democracy}

Digital tools have emerged as critical innovations in deliberative democracy, enabling scalable citizen participation in decision-making____. These tools integrate deliberation with digital methods for visualising, moderating, and summarising discussions. A prominent example is Polis____, an open-source platform where participants collect and rate statements on given topics. The tool visualises the high-dimensional opinion space and computes consensus statements to foster common ground in the deliberation. Being applied to many larger participatory processes such as vTaiwan and the German \enquote{Aufstehen} [\enquote{Stand up}] movement, Polis demonstrates how digital tools can scale deliberation beyond traditional in-person settings. Another successful example of such an online tool is the Stanford Deliberation Platform____, where automatic moderation ensures fair deliberation of small groups in video calls. This platform has also been used in the \enquote{America in One Room} deliberation. To summarise discussions, tools like Wikum____ and Kialo____ have been developed.  These tools offer thread-like discussions with features such as nested argumentation and can be used to challenge reasoning in deliberation. Decision-making platforms such as Decidim____ facilitate participatory processes for cities, while initiatives like Decide Madrid and vTaiwan exemplify algorithmic empowerment, decentralising power and fostering pluralistic policymaking____. This emphasises the possible role of algorithms in enhancing democratic transparency and inclusivity. ____ introduce ProtoTeams, a gamified approach to forming diverse groups for real-time collaborative discussions, illustrating how team composition and dynamics influence decision-making outcomes. However, scaling deliberation introduces challenges, as many platforms prioritise technical solutions, often overlooking factors like cultural and linguistic diversity, social inequalities, and inclusion across genders and abilities____. Addressing these aspects is also essential to ensure digital tools support a broader and more diverse range of participants.

\subsection{Computational Social Choice, Voting and Participatory Budgeting}

Computational social choice provides theoretical and practical tools for designing voting systems that address fairness, proportionality, and collective decision-making in contexts such as participatory budgeting (PB) ____. Fairness in this context refers to the equal representation of diverse voter preferences and the proportional allocation of resources or decision-making power, ensuring that no group is disproportionately advantaged or excluded ____. Recent advancements have emphasised the integration of computational methods with participatory processes. ____ provide a comprehensive survey of PB models, highlighting the importance of preference elicitation, voter incentives, and welfare objectives. Furthermore, the subfield of multi-winner voting contributes insights into proportionality and robustness. ____ proposes robust axioms for proportional representation, ensuring fairness for cohesive voter groups. ____ further demonstrate the utility of proportional rules in PB elections. ____ formalises axioms for proportional representation in PB with additive utilities, proposing rules like the \textbf{Method of Equal Shares (MES)} to achieve fairness and efficiency. The basic idea of the Method of Equal Shares is that each voter is assigned an equal part of the budget. This part of the budget can only be used to fund projects that the voter has voted for. The method goes through all project proposals, beginning with the projects with the highest number of votes. It selects a project if it can be funded using the budget shares of those who voted for the project. The method then divides the cost of a project equally among its supporters. ____ demonstrates how MES achieves fairer outcomes and avoids divisive districts while improving overall utility. Additionally, ____ explore digital PB platforms, identifying design principles that reduce cognitive load while enhancing fairness and legitimacy, where participants favour expressive input formats and fair budget allocations. Building on these contributions, our paper explores how combining deliberation and computational tools for democratic decision-making should be designed to achieve proportional fairness, transparency, and practicality.