\section{Challenges, Opportunities and Future Work}

Our research reveals that many open-source projects require more information when addressing security-related issues, sparking a broader conversation about best practices in security management, especially for OSS projects.
This study investigates the security policies of open-source projects, exploring their characteristics and relationships to project attributes. 

We believe that this work represents a crucial first step in understanding the requirements for effective security measures, their impact on project outcomes, and which policies are actually followed. Additionally, we aim to explore how software communities, such as library ecosystems like NPM, PyPI and so on, conform to these policies.
Future studies will build upon this research by conducting an in-depth analysis of the relationship between security issue resolution and discussions with developers. Furthermore, a larger-scale study of software projects at the ecosystem level will provide valuable insights into how open-source communities navigate and adhere to security policies.



% \pooh{For future work,\\ 
% -the low-hanging fruit one is going to be exploring other ecosystems.\\
% -Investigate the characteristics of security policy from GitHub alternative (other git hosting service. Im not sure the population would be sufficient).\\
% -Tracking for the adoption of practices after this paper was published? maybe this one is too far fetch.
% }


% This research offers a focused analysis of security policies in open-source projects on GitHub, exploring both the structure and content of these documents. By categorizing policy elements and examining their associations, we identified common patterns and key areas frequently addressed in these policies. Our findings highlight that, while the inclusion of security policies is on the rise, the content often varies in depth and scope, suggesting that more standardized guidelines could benefit project maintainers by providing clear expectations for essential security topics.

% Through a structured categorization process involving multiple researchers, we achieved a high level of reliability, ensuring that our results accurately reflect common practices across diverse projects. This process, validated with inter-rater reliability measures, strengthens our conclusions and provides a reliable framework that future researchers can build on. Additionally, our use of association rule mining revealed valuable insights into frequently linked policy elements, offering a clearer view of how security practices are organized within open-source documentation.

% Overall, this study underscores the need for clear and comprehensive security policies in open-source projects. Our findings provide practical recommendations for developers and maintainers, helping them prioritize essential security elements and improve policy consistency. Future research could extend this work by exploring security policies in different ecosystems, examining how various platforms, programming languages, and community norms influence security practices. This broader perspective could further support the open-source community in developing adaptable and effective security standards across ecosystems.


% \textbf{Future Work}

% \section*{Acknowledgment}
% The preferred spelling of the word ``acknowledgment'' in America is without % an ``e'' after the ``g''. Avoid the stilted expression ``one of us (R. B. % G.) thanks $\ldots$''. Instead, try ``R. B. G. thanks$\ldots$''. Put sponsor % acknowledgments in the unnumbered footnote on the first page.


