\section{Methodology}
\label{3_methodology}
\usetikzlibrary{shapes, arrows.meta, positioning}
In this section, we present the methodology



\subsection{Overview} 
Figure \ref{fig:overviewFig} shows an overview of our study framework. The study begins with the collection of vulnerability report data from GitHub Advisory, specifically focusing on PyPI packages. While this initial scope centers on PyPI, future work will aim to broaden the range of analyzed packages. From the vulnerability reports, we extract a list of relevant PyPI projects and gather the corresponding GitHub repositories for further analysis. Once the repositories are collected, we identify those containing a security policy file (e.g., SECURITY.md), which serves as the primary data source for our study. Following data collection, we preprocess the repositories to extract relevant content from the security policies, such as policy categories and specific features characterizing each repository. This refined dataset enables us to address our research questions. For RQ1, we manually examine and categorize the security policy content. For RQ2, we apply association rule mining\cite{ReadME}
% \morakot{cite}\pooh{cited} 
to analyze patterns within the dataset. For RQ3, we conduct a correlation analysis to explore the relationship between project characteristics and identified security policy categories. A detailed description of our study’s methodology follows.

% , gathering a total of 2,311 reports on PyPI packages. We can then retreive the list of 679 unique PyPI projects existing in GitHub Advisory. We can then identify 



% \morakot{some of these details we can move to their subsection e.g. the number reporting. the overview should be just describe the methodology in general}
% Our study methodology is outlined in \ref{fig:enter-label}, illustrating the process of analyzing security policies across PyPI projects. We begin with data collection from GitHub Advisory, gathering a total of 2,311 reports on PyPI packages. Through data preprocessing, these reports are filtered to identify 679 unique PyPI packages, of which 303 contain an explicit security policy. From these policies, we extract a total of 722 content blocks, which represent distinct sections of security-relevant information. To categorize and understand the themes within these policies, we employ content snowballing and manual analysis, aiming to answer our first research question (RQ1) by identifying the primary security policy categories. Additionally, we use association rule mining to detect common patterns and relationships among these categories, addressing our second research question (RQ2) on frequently occurring security policy themes. Finally, collocation analysis is applied to examine the spatial relationships and co-occurrence of security policy elements within project features, helping us answer our third research question (RQ3) on how security policies are positioned in relation to project components. This multi-step approach allows us to uncover and characterize patterns in the content and structure of security policies across open-source projects.

% \morakot{need framework figure}

% - figure overview framework





% \begin{figure}
%     \centering
%     \includegraphics[width=1\linewidth]{2RqFlow.png}
%     \caption{Framework Overview}
%     \label{fig:enter-label}
% \end{figure}

% \morakot{use section command}
\subsection{Data Collection} 

Our study examines the security policies of open-source projects within the PyPI ecosystem, specifically focusing on projects listed in GitHub’s advisory database. Data collection begins with querying the GitHub Advisory API\footnote{\url{https://docs.github.com/en/graphql/reference/queries\#securityadvisories}}
% \morakot{add footnote to the API manual}\pooh{added}
to retrieve a total of 2,311 advisory reports on PyPI packages. From this data, we extract a list of unique PyPI projects, accounting for instances where multiple reports pertain to a single project. This refinement results in 679 unique projects.
A security policy file in a repository can be implemented in various ways, including placement within the project repository or within the \texttt{.github} repository. It can be located in directories such as the project root, the \texttt{.github} folder, or the \texttt{docs} folder. The security policy file can be labeled as ``security'' (case-insensitive) and may exist in \texttt{.md} or \texttt{.rst} formats. During data preprocessing, we filter the reports to identify these unique PyPI packages, of which 303 contain a clear security policy. We then proceed to gather the corresponding GitHub repositories for further analysis.



% //Pooh, I know that "including, but not limited to" is not academic but I want to show the possible location of policy. cause as far as I know, I could not find any site mentioning this. and importantly I could not proof that this is all the possible locations it ban be.

% \vspace{10pt}

% \begin{figure}[h!]
% \centering
% \begin{tikzpicture}[node distance=2cm, font=\small]
% % Nodes
% \node (start) [startstop] {Start};
% \node (repo) [decision, below of=start, yshift=0.25cm] {Repository};
% \node (projRepo) [process, below of=repo, xshift=-1.5cm, yshift=0.5cm] {Project Repository};
% \node (dotGithubRepo) [process, below of=repo, xshift=1.5cm, yshift=0.5cm] {.github Repository};

% \node (filepath) [decision, below of=repo, yshift=-1cm] {Directory};

% \node (githubDir) [process, below of=filepath, yshift=.35cm] {/.github};
% \node (rootDir) [process, left of=githubDir, xshift=-.5cm] {/\textit{root}};
% \node (docsDir) [process, right of=githubDir, xshift=.5cm] {/docs};

% \node (filename) [decision, below of=githubDir, yshift=.25cm] {File Type};

% \node (securityMd) [process, below of=filename, yshift=.5cm ,xshift=-1.5cm] {.md};

% \node (securityRst) [process, below of=filename, yshift=.5cm ,xshift=1.5cm] {.rst};

% % Arrows
% \draw [arrow] (start) -- (repo);
% \draw [arrow] (repo) --  (projRepo);
% \draw [arrow] (repo) -- (dotGithubRepo);

% \draw [arrow] (projRepo) -- (filepath);
% \draw [arrow] (dotGithubRepo) -- (filepath);

% \draw [arrow] (filepath) -- (rootDir);
% \draw [arrow] (filepath) -- (githubDir);
% \draw [arrow] (filepath) -- (docsDir);

% \draw [arrow] (rootDir) -- (filename);
% \draw [arrow] (githubDir) -- (filename);
% \draw [arrow] (docsDir) -- (filename);

% \draw [arrow] (filename) -- (securityMd);
% \draw [arrow] (filename) -- (securityRst);

% \end{tikzpicture}
% \caption{Possibility of Security Policy Source File\chaiyong{Not sure we need this figure. If we need to save space, this may go.}}
% \label{fig:security-policy}
% \end{figure}






% https://www.tablesgenerator.com/

% \begin{table}[tb]
% \centering
% \begin{tabular}{lll|r|l}
% \cline{1-4}
% \multicolumn{3}{c|}{Extraction Snapshot of Reports} & \multicolumn{1}{c|}{Oct 2024} &  \\ \cline{1-4}
% \multicolumn{3}{l|}{PyPI Report from GitHub's Advisory} & 2311 reports &  \\ \cline{2-4}
% \multicolumn{1}{l|}{} & \multicolumn{2}{l|}{Unique PyPI Project from GitHub's Advisory} & 679 Projects &  \\ \cline{3-4}
% \multicolumn{1}{l|}{} & \multicolumn{1}{l|}{} & PyPI projects with security policy & 303 Projects   &  \\ \cline{1-4}
% \end{tabular}
% \vspace{10pt}
% \caption{Summary of Extracted PyPI projects.\chaiyong{Strange table format. I think this is better represented using a figure showing the filtering of the data from left $\rightarrow$ right.}}
% \label{tab:pypi_data_tables}
% \end{table}


\subsection{Data Preprocessing} 


After acquiring the security policies from unique PyPI projects listed in GitHub's Advisory reports, we parsed the titles and content based on markdown formatting. Specifically, we identified each security policy \texttt{content block} as starting with a title marked by a hashtag symbol (\#) followed by its associated content. This approach allowed us to build datasets containing the security policy content blocks within the PyPI ecosystem. Figure \ref{fig:data_filtering} shows the data filtering process, where each content block is extracted, cleaned, and structured for further analysis. From these policies, we extract a total of 722 content blocks, which represent distinct sections of security-relevant information. 


\begin{figure}[H]
    \centering
    \begin{tikzpicture}[
        block/.style={
            rectangle, draw, thin, minimum height=2.5cm, minimum width=1cm, anchor = base,
            align=center, font=\small, fill=gray!10
        },
    ]

    % Blocks with captions and numbers
    \node[block] (stage1) {PyPI\\Vulnerability\\Reports from \\GitHub\\Advisory\\\rule{1.5cm}{0.4pt}\\2311 reports\\};
    
    \node[block, right=of stage1, xshift=-.65cm] (stage2) {Unique PyPI\\Projects with\\Vulnerability\\Reports\\\\\rule{1.5cm}{0.4pt}\\679 Projects\\};
    
    \node[block, right=of stage2, xshift=-.65cm] (stage3) {PyPI Projects\\Containing\\Security\\Policies\\\\\rule{1.5cm}{0.4pt}\\303 Projects\\};
    
    \node[block, right=of stage3, xshift=-.65cm] (stage4) { Extracted\\Security \\Policy\\Content\\ Blocks\\\ \rule{1.5cm}{0.4pt}\\722 Content \\Blocks};

    % Arrows
    \draw[arrow] (stage1) -- (stage2);
    \draw[arrow] (stage2) -- (stage3);
    \draw[arrow] (stage3) -- (stage4);

    \end{tikzpicture}
    \caption{Summary of data collection and preprocessing steps}
    \label{fig:data_filtering}
    
\end{figure}


% The projects without security policies are ruled out. We obtained 303 unique PyPI projects with security policy.
% \vspace{10pt}

% \begin{table}[h!]
% \centering
% \begin{tabular}{llll|r|l}
% \cline{1-5}
% \multicolumn{4}{c|}{Extraction Snapshot of Reports} & \multicolumn{1}{c|}{Oct 2024} &  \\ \cline{1-5}
% \multicolumn{4}{l|}{PyPI Report from GitHub's Advisory} & 2311 Reports &  \\ \cline{2-5}
% \multicolumn{1}{l|}{} & \multicolumn{3}{l|}{Unique PyPI Project from GitHub's Advisory} & 679 Projects &  \\ \cline{3-5}
% \multicolumn{1}{l|}{} & \multicolumn{1}{l|}{} & \multicolumn{2}{l|}{Unique PyPI projects with security policy} & 303 Projects &  \\ \cline{4-5}
% \multicolumn{1}{l|}{} & \multicolumn{1}{l|}{} & \multicolumn{1}{l|}{} & \textbf{PyPI Security Policy content block} & \multicolumn{1}{l|}{\textbf{722 Content Blocks}} &  \\ \cline{1-5}
% \end{tabular}
% \vspace{10pt}
% \caption{Summary of Extracted PyPI Content Blocks.\chaiyong{Similar to the Table I.}}
% \label{tab:pypi_data_tables_2}
% \end{table}


\subsection{Data Analysis} 

\subsubsection{RQ1: What are the primary categories of security policies commonly declared in software repositories?}

To address RQ1, we conducted a manual analysis using Cohen's Kappa \cite{Cohen} 
% \morakot{cite cohen kappa original paper}\pooh{cited}
to ensure reliability. Our process involved an iterative approach, where we examined documents one by one, refining and expanding categories as each new document was analyzed. This iterative method allowed for a comprehensive and detailed set of categories to emerge naturally from the data. For each security policy content block, two researchers independently categorized the content without prior knowledge of each other's categorization choices. Following the initial categorization, we calculated Cohen's Kappa to assess the reliability of the manual analysis, ensuring consistency and minimizing the influence of individual biases. 

In cases where disagreements occurred, the assigned pair engaged in a discussion session to review and resolve conflicts, reaching a consensus on the final categorization. This collaborative resolution process reinforced the reliability of the categorized content blocks. After finalizing the categorizations, we calculated the frequency of each category to identify the most commonly declared security policy types in software repositories. 


% To answer the RQ1, we used a Manual Analysis analysis with Cohen’s Kappa,  \morakot{cite the cohen kappa paper}. We perform an iterative process used to identify categories by reviewing documents one at a time, refining and expanding categories as each new document is analyzed. This allows for a comprehensive set of themes to emerge naturally from the data.

% Manual Analysis with Multiple Investigators: For each security policy content block, a pair of researchers were assigned to independently categorize the content, with no prior knowledge of the other’s categorization choices. After initial categorization, we calculated Cohen’s Kappa to assess manual analysis reliability, ensuring the categorization was consistent and not influenced by individual biases.

% For content blocks where disagreements happen, the assigned pair conducted a discussion session to review and resolve these conflicts, reaching a consensus on the final categorization. This collaborative resolution process strengthened the reliability of the categorized content block.

% We can then determine the frequency count of the appearing of each categories to identify categories of security policies that mostly declared. 


% \chaiyong{inter-rater agreement is not an approach. The actual approach is manual analysis with multiple investigators.}
% \chaiyong{I think this can be removed. You need to explain instead how the two investigators work. Do they look at the policy independently? Do they meet to resolve the conflicts at the end? etc.} \pooh{resolved}

\subsubsection{RQ2: Do security policies follow the GitHub template?}

To address RQ2, we applied association rule mining to identify common patterns within security policies across repositories \cite{AssociationRuleMiningBook, ReadME}. Association rule mining is a data-mining technique that uncovers patterns of co-occurrence among categories in a dataset. By applying this method, we can detect commonly grouped elements, revealing structural similarities across security policies in different repositories.

In our analysis, we used the support metric to measure the significance of these patterns. The support metric indicates the proportion of repositories in the dataset where a specific combination of categories is present. To calculate the percentage of repositories adhering to GitHub's recommended security policy template, we identified itemsets that included the category found within GitHub's security policy template. The support value for this itemset was then multiplied by 100 to express the adherence rate as a percentage. Similarly, the percentage of repositories not adhering to the template was computed as 100 Adherence Percentage

This approach allows us to quantitatively assess the extent to which repositories follow GitHub's template and identify additional patterns or extensions commonly included by maintainers.


% This approach provides insight into the common structures and priorities in security policies within the PyPI ecosystem.

% To address RQ2, we applied association rule mining to identify common patterns within security policies across repositories \cite{AssociationRuleMiningBook, ReadME}. Association Rule Mining is a data-mining technique used to identify patterns of co-occurrence among categories in a dataset. Applying metrics like support frequently reveals co-existing elements, highlighting common structures across items in the data. [explain in our context as well]

% To measure the significance of these patterns, we used the support metric, which indicates how frequently certain category combinations occur across the dataset. High-support patterns reveal commonly co-existing elements in security policies, helping us identify the foundational components that are widely shared in open-source repositories.
\begin{table*}[!]
\centering
\resizebox{\textwidth}{!}{
\begin{tabular}{p{3.5cm}rp{5cm}p{8.5cm}}
\toprule
\textbf{Category} & \textbf{Count} & \textbf{Definition} & \textbf{Example} \\
\midrule
\textbf{Reporting mechanism} & 282 & A set of instructions outlining the procedure for reporting security vulnerabilities to the relevant developers. & {``To report a vulnerability, please do not share it publicly on GitHub nor the community slack channel. Instead, contact the \censortext \ \ team directly via email first: \censortext \ ''} \\

\textbf{Generic policy} & 211 & Outline objective and foundation of security policy. & ``This document describes model security and code security in \censortext. It also provides guidelines on how to report vulnerabilities in \censortext.'' \\

\textbf{User guideline} & 133 & Security-related and project-specific e.g. Instructions for users, including secure usage of dependencies, recommended configurations, and general project usage. & ``\censortext \ \ is not intended to be deployed on a public-facing server. By default, the web UI is only exposed on localhost, so normally this is not a problem. Do not give someone access to the web UI unless you trust them with everything else that is on that machine.'' \\

\textbf{Scope of practice} & 127 &The list of project versions still under maintenance. & ``These \censortext \ \ releases are currently supported with security updates: Version 1.2.x - \checkmark\  Supported Version \textless{}1.1 - \texttimes\ Not supported'' \\

\textbf{Project practice} & 20 & Detailed procedural steps that will be executed when a vulnerability is reported. & ``Each report is acknowledged and analyzed by security response members within five (5) working days. Any vulnerability information shared with security response members stays within the \censortext \ \ project and will not be disseminated to other projects unless it is necessary to get the issue fixed.'' \\

\textbf{History of vulnerability} & 8 & A list of disclosed vulnerabilities. & ``We maintain Security Advisories on the \censortext \ \ project \censortext \ \ repository, where we will post a summary, remediation, and mitigation details for any patch containing security fixes." \\

\textbf{Additional information} & 7 & Non-security related and project specific e.g. Access for further details that are non-related to the security, Copyright section, project contribution, acknowledgment. & ``We prefer all communications to be in English'' \\

\textbf{Information on maintainer} & 4 & Maintainer information e.g. name, email, role, responsibility. & ``\censortext \ \ is always open to feedback, questions, and suggestions. If you would like to talk to us, please feel free to email us at \censortext \ \, and our PGP key is at \censortext \ .'' \\

\textbf{Others} & 0 & None of all above. & - \\

\bottomrule
\end{tabular}
}
\caption{Categories of Security Policy Titles}
\label{Categories of Security Policy Titles}
\end{table*}


\subsubsection{RQ3: Does the occurrence of security policy categories correlate with characteristics of software repositories?}

In this study, we categorize repositories based on a set of traditional attributes, with plans to expand this feature set in future research. The initial features include project age (in days), number of commits, number of contributors, number of subscribers, number of pull requests, total number of issues, number of open issues, repository size (in kilobytes), number of stars, and number of forks. We then apply point-biserial correlation analysis \cite{Point‐biserial} to examine the correlation between project features and security policy categories. This method is appropriate for identifying correlations between continuous variables (e.g., project features) and binary indicators representing the presence of specific categories. Thus, we can observe which project characteristics are most strongly associated with certain security policy categories.
% For instance, projects with a higher number of contributors or commits may be more likely to implement comprehensive security policies. 


% To explore how security policies correlate \cite{Point‐biserial} with specific repository characteristics, we conducted a point-biserial correlation analysis between project features and security policy categories.

% This approach show the strength and direction of relationships between continuous project features and binary security policy categories. By identifying significant correlations, we aim to reveal which repository characteristics are associated with the presence of specific security policy elements.
% \textbf{Overview of Git Version Control Data}
% \ani{xxx}