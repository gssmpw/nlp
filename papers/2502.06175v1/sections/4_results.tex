\section{Results}
\label{4_results}


\subsection{RQ1: What are the primary categories of security policies commonly declared in software repositories?}

% \begin{table}[h]
% \centering
% \begin{tabular}{lp{4.75cm}}
% \toprule
% \textbf{Category} & \textbf{Definition} \\
% \midrule
% \textbf{Generic policy} & Outline objective and foundation of security policy. \\
% \textbf{Reporting mechanism} & A set of instructions outlining the procedure for reporting security vulnerabilities to the relevant developers. \\
% \textbf{Information on maintainer} & Maintainer information e.g. name, email, role, responsibility. \\
% \textbf{Scope of practice} & The list of project versions still under maintenance. \\
% \textbf{Project practice} & Detailed procedural steps that will be executed when a vulnerability is reported. \\
% \textbf{History of vulnerability} & A list of disclosed vulnerabilities. \\
% \textbf{User guideline} & Security related and project specific e.g. Instructions for users, including secure usage of dependencies, recommended configurations, and general project usage. \\
% \textbf{Additional information} & Non-security related and project specific e.g. Access for further details that are non-related to the security, Copyright section, project contribution, acknowledgement. \\
% \textbf{Others} & None of all above. \\
% \bottomrule
% \end{tabular}
% \captionsetup{justification=centering}
% \caption{Categories of Security Policy Titles \\(addressing RQ1)}
% \label{Categories}
% \end{table}



From the manual validation of security policy content blocks, we can identify nine distinct categories of security policies. The Cohen’s Kappa value was 0.9, indicating almost perfect agreement among the researchers. Table \ref{Categories of Security Policy Titles} presents definitions for each category and the counts of each identified category.

The \texttt{Reporting mechanism} category has the highest count (282), highlighting its importance across projects. This suggests that a large number of repositories prioritize clear instructions for reporting vulnerabilities, emphasizing communication channels that allow users to report security issues effectively. This aligns with best practices, as reporting mechanisms are essential for timely vulnerability disclosure and management. The \texttt{User guideline} category, appearing 133 times, also shows a strong presence, suggesting that many projects provide users with specific security instructions. This may include advice on secure configurations or recommendations to avoid common security pitfalls, reflecting the maintainers’ focus on enhancing the overall security posture of their software.

In contrast, categories like \texttt{Project practice} (20), \texttt{History of vulnerability} (8), \texttt{Additional information} (7), and \texttt{Information on maintainer} (4) appear far less frequently. The low counts suggest that specific project practices, past vulnerability history, and additional details about maintainers are not commonly documented within formal security policies. This raises the concern that these categories may not be receiving adequate attention in current security policies, potentially overlooking elements that could enhance overall security practices. It prompts us to consider whether these categories should be leveraged more effectively to provide a comprehensive view of security practices.

% \begin{table}[h]
% \centering
% \begin{tabular}{lr}
% \toprule
% \textbf{Category} & \textbf{Count} \\
% \midrule
% \textbf{Reporting mechanism} & 282 \\
% \textbf{Generic policy} & 211 \\
% \textbf{User guideline} & 133 \\
% \textbf{Scope of practice} & 127 \\
% \textbf{Project practice} & 20 \\
% \textbf{History of vulnerability} & 8 \\
% \textbf{Additional information} & 7 \\
% \textbf{Information on maintainer} & 4 \\
% \textbf{Others} & 0 \\
% \midrule
% \textbf{Total} & 722 \\
% \bottomrule
% \end{tabular}
% \captionsetup{justification=centering}
% \caption{Distribution of Security Policy Categories in PyPI Population (addressing RQ1)}

% \label{tab:pypi_category_distribution}
% \end{table}

\begin{tcolorbox}[] \textbf{RQ1 Summary}:
Nine security policy categories were identified, with \texttt{Reporting mechanism} (282), \texttt{Generic policy} (211), \texttt{User guideline} (133), and \texttt{Scope of practice} (127) being the most common, highlighting their critical roles.
\end{tcolorbox}

% The findings reveal that "Reporting mechanism," "Generic policy," and "Scope of practice" are the most prevalent categories, with counts of 282, 211, and 127, respectively. This prevalence suggests that a significant number of projects prioritize providing structured instructions for reporting vulnerabilities, establishing general security guidelines, and specifying the versions supported with ongoing maintenance. Together, these categories indicate that many maintainers are following the GitHub Security template, which emphasizes clear communication channels, general security principles, and versioning practices to help users manage security-related concerns effectively. In contrast, categories such as "Project practice" and "History of vulnerability" appear less frequently Further, "Additional information" and "Information on maintainer" are rarely documented, . This rarity implies that non-security-specific details or maintainer contact information may be deemed less critical or might be documented outside of formal security policies.







% To address the main categories of security policies in software repositories, we identified nine distinct categories based on the content within each policy document. Table \ref{tab:security_policy_categories} provides definitions for each category, while Table \ref{tab:pypi_category_distributionsensor} outlines their distribution and includes representative examples.

% The findings reveal that "Reporting mechanism," "Generic policy," and "Scope of practice" are the most prevalent categories, with counts of 282, 211, and 127, respectively. This prevalence suggests that a significant number of projects prioritize providing structured instructions for reporting vulnerabilities, establishing general security guidelines, and specifying the versions supported with ongoing maintenance. Together, these categories indicate that many maintainers are following the GitHub Security template, which emphasizes clear communication channels, general security principles, and versioning practices to help users manage security-related concerns effectively.

% In contrast, categories such as "Project practice" and "History of vulnerability" appear less frequently, with only 20 and 8 occurrences respectively, suggesting that fewer projects have formalized procedural steps for handling vulnerabilities or maintain a public record of past security issues. This lower prevalence may imply that, while vulnerability reporting is prioritized, there is limited visibility into internal practices and historical security documentation.

% Further, "Additional information" and "Information on maintainer" are rarely documented, with only 7 and 4 occurrences, respectively. This rarity implies that non-security-specific details or maintainer contact information may be deemed less critical or might be documented outside of formal security policies. Notably, the "Others" category had no occurrences, suggesting that the categories defined in this study comprehensively cover the types of content found within security policies, validating the applicability of our categorization framework for open-source security policy documentation.

% Overall, while most projects emphasize reporting mechanisms, general security guidelines, and versioning practices, there appears to be less focus on transparency regarding internal procedural practices and historical vulnerability records. These findings provide insight into common categories across open-source repositories and highlight areas where security policies could be expanded to enhance transparency and user guidance.




% \morakot{this content of the investigation approach should be moved to the data analysis approach according to Aj.Aun's comment}\pooh{rewrote this part(Q1)}
% \morakot{discuss the finding a bit e.g., what is the majority of content, what are the common content, what is the least or smallest group of content}



% \subsection{RQ2: Are there any correlations or associations among the categories of security policies in software repositories?}

%  The results of our association rule mining show in Figure \ref{fig:PyPIARM_FixedSize}. There are four categories exhibit strong support and frequently co-occur within security policies: \texttt{Generic Policy}, \texttt{Reporting Mechanism}, \texttt{Scope of Practice}, and \texttt{User Guideline}. Notably, these categories align with GitHub's security policy template, which is recommended by GitHub

% However, the \texttt{User Guideline} category does not appear in GitHub's template, yet our analysis shows that maintainers frequently include it alongside GitHub's recommended categories. This suggests that maintainers see value in providing specific security instructions for users, considering it an important addition to core policy elements and a way to enhance security practices across repositories.



\subsection{RQ2: Do security policies follow the GitHub template?}

The results of our association rule mining, reveal that 39.93\% of repositories adhere to GitHub's recommended security policy template by including all three core categories: \texttt{Generic Policy}, \texttt{Reporting Mechanism}, and \texttt{Scope of Practice}. However, the remaining 60.07\% of repositories either omit one or more of these categories or incorporate additional elements, indicating that not all maintainers strictly follow the template.

Interestingly, our analysis highlights the frequent inclusion of the \texttt{User Guideline} category, which is absent from GitHub's template but co-occurs frequently alongside the recommended categories. This suggests that maintainers see value in providing specific security instructions for users, considering it an important enhancement to the core policy elements is shown in Figure \ref{fig:PyPIARM_FixedSize}.

These findings indicate that while a significant proportion of repositories align with the GitHub template, many maintainers adapt their policies by tailoring them to the needs of their projects. The inclusion of categories like \texttt{User Guideline} reflects the evolving practices of security policy maintenance, demonstrating an effort to address user-specific concerns and improve overall security.

\begin{tcolorbox}[] \textbf{RQ2 Summary}: Our findings suggest that while a significant portion of developers align with GitHub's security policy guidelines, with 39.93\% of repositories incorporating all three core elements (\texttt{Generic Policy}, \texttt{Reporting Mechanism}, and \texttt{Scope of Practice}), the majority (60.07\%) deviate by omitting one or more of these elements or adding additional categories. This highlights varying levels of adherence and customization among open-source projects, including the notable addition of a \texttt{User Guideline}, which reflects its acknowledged importance. \end{tcolorbox}

% After completing RQ1, where we identified nine distinct categories of security policies, we applied these categories back to the PyPI dataset. For each unique package, we categorized its security policy content blocks based on our predefined categories, resulting in a structured set of categories per package. This structured dataset allowed us to examine patterns and commonalities across repositories.

% To evaluate the significance of these patterns, we focused on the \textit{support} metric. Support measures the frequency of a specific category or combination of categories in the dataset. By analyzing support, we identified the most common and relevant patterns within security policies, providing insights into the foundational components that security policies across repositories often share.

% As shown in Figure~\ref{fig:PyPIARM_FixedSize}, the results of our association rule mining reveal that four categories exhibit strong support and frequently co-occur within security policies: \textit{Generic Policy}, \textit{Reporting Mechanism}, \textit{Scope of Practice}, and \textit{User Guideline}. Interestingly, three of these categories—Generic Policy, Reporting Mechanism, and Scope of Practice—are also found in GitHub's security policy template, which GitHub recommends maintainers use as a foundation for creating their own policies.

% However, the \textit{User Guideline} category, which provides security-related instructions for users, is notably absent from GitHub's template. Despite this, our analysis shows that maintainers frequently include user guidelines alongside the three main categories advised by GitHub. This suggests that maintainers recognize the importance of providing users with specific security guidelines, viewing it as a valuable addition to the standard policy elements. These findings highlight how maintainers go beyond template recommendations to address user needs, enhancing security practices across repositories.

\begin{figure}[h]
    \centering
    \includegraphics[width=.9\linewidth 
    % ,height=0.28\textheight
    ]{PyPIARM_FixedSize2.png}
    \captionsetup{justification=centering}
    \caption{Results from applying association rule mining\\(addressing RQ2)}
    \label{fig:PyPIARM_FixedSize}
\end{figure}

\subsection{RQ3: Does the occurrence of security policy categories correlate with characteristics of software repositories?}

Figure \ref{fig:RQ3Correlations} shows the results of the point-biserial correlation analysis. A positive correlation is observed between repository size and the presence of the \texttt{History of Vulnerability} category, suggesting that larger repositories are more likely to document their vulnerability history. This may be due to the increased complexity and greater likelihood of past security incidents that require tracking and documentation. Similarly, a positive correlation exists between the number of stars and the \texttt{User Guideline} category, indicating that more popular projects, as measured by star count, are more likely to provide specific security instructions for users. Conversely, a negative correlation is seen between the total number of issues and the \texttt{Scope of Practice} category, suggesting that projects with a high volume of issues may place less emphasis on clearly defining the scope of their security coverage.


\begin{figure}
    \centering
    \includegraphics[width=1\linewidth
    ,height=0.22\textheight
    ]{RQ3.png}
    \captionsetup{justification=centering}
    \caption{Correlations between project features and the occurrence of security policy categories (addressing RQ3)}
    \label{fig:RQ3Correlations}
\end{figure}

\begin{tcolorbox}[] \textbf{RQ3 Summary}: The analysis reveals correlations between repository characteristics and security policy categories. Larger repositories correlate with the \texttt{History of Vulnerability} category, while popular projects (based on stars) are more likely to include \texttt{User Guideline}. Conversely, repositories with many issues tend to emphasize the \texttt{Scope of Practice} category less. \end{tcolorbox}

% 
% After categorizing security policies into nine distinct categories in RQ1, we applied point-biserial correlation analysis to understand how these security policy categories correlate with various repository characteristics as shown in figure \ref{fig:RQ3Correlations}.

% Our analysis reveals notable correlations between repository characteristics and specific elements within security policies, particularly focusing on popularity and scope of practice.

% Repositories with a high number of stars, reflecting significant community attention and user interest, tend to include a \textit{User Guideline} section in their security policies. This guideline provides essential instructions and best practices, which help users understand secure usage and contribute responsibly to the project. The presence of user guidelines in highly-starred repositories suggests that maintainers recognize the importance of offering clear, accessible security information to a broad user base. This addition is likely an effort to ensure that a diverse audience, drawn to the project by its popularity, adheres to secure practices.

% Furthermore, our analysis indicates that repositories lacking a \textit{Scope of Practice} in their security policies tend to experience more open issues. Without a defined scope, users may not know which versions are actively maintained, leading to confusion and potentially unnecessary issue reports. By not specifying which parts of the project are covered under security maintenance, maintainers may inadvertently invite a higher volume of issues from users unsure about the project’s support boundaries. This suggests that a well-defined scope of practice can help manage user expectations and streamline issue reporting, ultimately reducing maintenance overhead.

% In addition to these observations, our analysis shows a significant correlation between project size and the inclusion of a \textit{History of Vulnerability} section in security policies. Larger projects, with extensive codebases and complex dependencies, are more likely to experience and document vulnerabilities over time. This documentation provides a clear record of past security issues and their resolutions, helping maintainers and contributors monitor the security landscape of the project. Including a history of vulnerabilities supports transparency and accountability, which can be particularly valuable for large projects where users and developers benefit from a documented security history.

% These findings highlight how maintainers adapt their security policies based on repository characteristics. Popular repositories often include user guidelines to meet the needs of a large audience, ensuring users have access to essential security practices. Projects with well-defined scopes of practice help clarify support boundaries, reducing confusion and unnecessary issue reports from users. Additionally, larger projects are more likely to document a history of vulnerabilities, providing a transparent record of past security issues and resolutions. This approach supports both community engagement and efficient project management, aligning security policies with the unique demands of each repository.

