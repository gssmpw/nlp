\section{GitHub Security Policy}



\label{2_background}

% In this section, we review prior research on GitHub’s security policy feature and its role in open-source software, focusing on key findings regarding its impact on vulnerability handling and community trust. We also identify gaps in the current literature related to security policy adoption and effectiveness in open-source projects.


% \subsection{GitHub Security Policy}
GitHub introduced the security policy feature to help projects formalize their approach to vulnerability handling. This policy is typically provided as a SECURITY.md file, located either in the project’s main directory or the .github directory. This file guides users and contributors on reporting security vulnerabilities and outlines any established guidelines for handling these issues. In an official blog post from May 2019, GitHub highlighted the role of the SECURITY.md file in specifying a repository's security policy. When included, this file appears in the repository’s \textit{Security} tab and within the new issue workflow, facilitating more accessible reporting procedures for contributors \cite{githubSecurityPolicy2019}. In November 2022, GitHub emphasized the significance of the SECURITY.md file by further highlighting the security policy on the repository overview in the sidebar. This adjustment further emphasizes its accessibility to users and contributors \cite{githubSecurityPolicy2022}. According to GitHub’s documentation,\footnote{\url{https://docs.github.com/en/code-security/getting-started/adding-a-security-policy-to-your-repository}
} a clear security policy can improve responsiveness to security reports, assist maintainers in prioritizing critical issues, and foster trust. GitHub also offers a security policy template to help maintainers create their policies. 

% \chaiyong{Do we know since when? We may need a citation here.}
% \pooh{I had found the blog post.}

% \pooh{because its quite recent? (is 2 years recent?) should I emphasize how recent the precise it is?}




% Figure \ref{figure:template} shows the template includes three sections: \textit{Security Policy}, \textit{Supported Versions} (listing versions currently maintained with security updates), and \textit{Reporting a Vulnerability}, which specifies the procedure for reporting security issues.
% {

% \begin{figure}

%     \centering
% \begin{mdframed}[linewidth=1pt, roundcorner=10pt]

% \# Security Policy

% \#\# Supported Versions

% Use this section to tell people about which versions of your project are
% currently being supported with security updates.
% \vspace{10pt}

% \begin{tabular}{|l|c|}
% \hline
% \textbf{Version} & \textbf{Supported} \\
% \hline
% 5.1.x & \checkmark \\
% 5.0.x & \texttimes \\
% 4.0.x & \checkmark \\
% \textless{} 4.0 & \texttimes \\
% \hline
% \end{tabular}

% \vspace{10pt}
% \#\# Reporting a Vulnerability

% Use this section to tell people how to report a vulnerability.

% Tell them where to go, how often they can expect to get an update on a
% reported vulnerability, what to expect if the vulnerability is accepted or
% declined, etc.
% \end{mdframed}
% \caption{Security Policy Template provided by GitHub}
% \label{figure:template}
% \end{figure}

% % \morakot{add footnote as a url to the page of github template guidline}
% % \pooh{the template guideline can be access via a security tab inside a project, therefor you need a repository to view it. to compensate, I insert a fig1 for the template with a footnote to documentation instead.}

% % \morakot{add example of sec.md}

% \subsection{Related works}

% Existing research underscores the critical role of security policies in effectively managing vulnerabilities within open-source software. Nusrat Zahan et al. \cite{zahan2023softwaresecuritypracticesyield} recommend for a comprehensive approach to security practices, including the implementation of security policies that incorporate best practices in project management and secure development. This viewpoint highlights the necessity of a proactive and holistic strategy for security, aimed at addressing vulnerabilities throughout the software development lifecycle.


% % Most related work explores the importance of security policies in managing vulnerabilities within open-source software. Nusrat Zahan et al. \cite{zahan2023softwaresecuritypracticesyield} suggest that security practices, including security policies, should adopt a comprehensive approach that encompasses best practices in project management and secure development. This perspective emphasizes the need for a more proactive and holistic approach to security, addressing vulnerabilities throughout the software lifecycle.

% Ayala, Garcia, et al. \cite{10190609} conducted an empirical study of GitHub workflows and security policies in popular repositories, finding that only 37\% of repositories have workflows enabled, and only 7\% have defined security policies. The study also highlights limited adoption of CodeQL, GitHub's static analysis tool for vulnerability detection, with only 13.5\% of applicable repositories utilizing it. Zahan et al. \cite{10163720} analyze security practices in npm and PyPI using the OpenSSF Scorecard, which assesses projects based on security metrics such as workflow configuration and vulnerability reporting. Their study reveals notable security gaps and low adoption rates of best practices, finding that security policies recommended by GitHub are implemented in fewer than 4\% of projects. 


% % \begin{figure}
% %     \centering
% %     \includegraphics[width=0.5\linewidth]{3RQsFlowFinal.png}
% %     \caption{Enter Caption}
% %     \label{fig:enter-label}
% % \end{figure}



% These findings emphasize the need for open-source project maintainers to prioritize workflow configuration, automated static analysis, and security policy definition to mitigate vulnerabilities and strengthen the security of widely-used software in the supply chain. Additionally, they underscore the importance of consistent security measures, suggesting that adopting standardized security policies could significantly improve open-source security. Our work aims to address these gaps by proposing solutions that enhance security practices and support maintainers in implementing effective, standardized policies across open-source projects.

% \morakot{missing}