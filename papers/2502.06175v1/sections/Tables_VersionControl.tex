\begin{table}[tb]
\centering
\begin{tabular}{lp{4.75cm}}
\toprule
\textbf{Category} & \textbf{Definition} \\
\midrule
\textbf{Generic policy} & Outline objective and foundation of security policy. \\
\textbf{Reporting mechanism} & A set of instructions outlining the procedure for reporting security vulnerabilities to the relevant developers. \\
\textbf{Information on maintainer} & Maintainer information e.g. name, email, role, responsibility. \\
\textbf{Scope of practice} & The list of project versions still under maintenance. \\
\textbf{Project practice} & Detailed procedural steps that will be executed when a vulnerability is reported. \\
\textbf{History of vulnerability} & A list of disclosed vulnerabilities. \\
\textbf{User guideline} & Security related and project specific e.g. Instructions for users, including secure usage of dependencies, recommended configurations, and general project usage. \\
\textbf{Additional information} & Non-security related and project specific e.g. Access for further details that are non-related to the security, Copyright section, project contribution, acknowledgement. \\
\textbf{Others} & None of all above. \\
\bottomrule
\end{tabular}
\caption{Categories of Security Policy Titles}
\label{}
\end{table}

\begin{table*}[!]
\centering
\begin{tabular}{lrl}
\toprule
\textbf{Category} & \textbf{Count} & \textbf{Example} \\
\midrule
\textbf{Reporting mechanism} & 282 & \makecell[l]{"To report a vulnerability, please do not share it publicly on GitHub
nor the community slack channel.\\ Instead, contact the \censortext \ \ team
directly via email first: \censortext"} \\\\
\textbf{Generic policy} & 211 & \makecell[l]{"This document describes model security and code security in \censortext.\\ It also provides guidelines on how to report vulnerabilities in \censortext."} \\\\
\textbf{User guideline} & 133 & \makecell[l]{"\censortext \ \ is not intended to be deployed on a public-facing server.\\ By default the web UI is only exposed on localhost, so normally this is not a problem.\\
Do not give someone access to the web UI unless you trust them with everything else that is on that machine."} \\\\
\textbf{Scope of practice} & 127 & \makecell[l]{"These \censortext \ \ releases are currently supported with security updates:\\
Version 2.0.x - \texttimes\  (not released)\\
Version 1.2.x - \checkmark\  Supported\\
Version 1.1.x - \texttimes\  Not supported\\
Version \textless{}1.1 - \texttimes\ Not supported"} \\\\
\textbf{Project practice} & 20 & \makecell[l]{"Each report is acknowledged and analyzed by security response members within five (5) working days.\\
Any vulnerability information shared with security response members stays within the \censortext \ \ project and-\\ will not be disseminated to other projects unless it is necessary to get the issue fixed.\\
As the security issue moves from triage, to identified fix, to release planning we will keep the reporter updated."} \\\\
\textbf{History of vulnerability} & 8 & \makecell[l]{"We maintain Security Advisories on the \censortext \ \ project \censortext \ \  repository,\\
where we will post a summary, remediation, and mitigation details\\ for any patch containing security fixes."} \\\\
\textbf{Additional information} & 7 & "We prefer all communications to be in English." \\\\
\textbf{Information on maintainer} & 4 & \makecell[l]{"\censortext \ \ is always open to feedback, questions, and suggestions.\\ If you would like to talk to us, please feel free to email us at \censortext \ \,\\ and our PGP key is at \censortext \ ."} \\\\
\textbf{Others} & 0 & -- \\
\midrule
\textbf{Total} & 722 &  \\
\bottomrule
\end{tabular}


\caption{Distribution of Security Policy Categories in PyPI Population}
\label{}
\end{table*}


\begin{table*}[!]
\centering
\begin{tabular}{lll}
\toprule
\textbf{Category} & \textbf{Definition} & \textbf{Example} \\
\midrule
\textbf{Reporting mechanism} & \makecell[tl]{A set of instructions outlining\\ the procedure for reporting\\ security vulnerabilities to the\\ relevant developers.} & \makecell[tl]{``To report a vulnerability, please do not share it publicly on GitHub nor\\ the community slack channel. Instead, contact the \censortext \\ team directly \\via email first: \censortext \ ''} \\

\textbf{Generic policy} & \makecell[tl]{Outline objective and foundation \\of security policy.} & \makecell[tl]{``This document describes model security and code security in \censortext.\\ It also provides guidelines on how to report vulnerabilities in \censortext.''} \\

\textbf{User guideline} & \makecell[tl]{Security-related and project-\\specific e.g. Instructions for\\ users, including secure usage\\ of dependencies, recommended\\ configurations, and general\\ project usage.} & \makecell[tl]{``\censortext \ \ is not intended to be deployed on a public-facing server.\\ By default the web UI is only exposed on localhost, so normally this is \\not a problem. Do not give someone access to the web UI unless you\\ trust them with everything else that is on that machine.''} \\

\textbf{Scope of practice} & \makecell[tl]{The list of project versions still\\ under maintenance.} & \makecell[tl]{``These \censortext \ \ releases are currently supported with security updates:\\ Version 1.2.x - \checkmark\  Supported\\ Version \textless{}1.1 - \texttimes\ Not supported''} \\

\textbf{Project practice} & \makecell[tl]{Detailed procedural steps that will\\ be executed when a vulnerability \\is reported.} & \makecell[tl]{``Each report is acknowledged and analyzed by security response members\\ within five (5) working days. Any vulnerability information shared with\\ security response members stays within the \censortext \ \ project and will not be\\ disseminated to other projects unless it is necessary to get the issue fixed.''} \\

\textbf{History of vulnerability} & \makecell[tl]{A list of disclosed vulnerabilities.} & \makecell[tl]{``We maintain Security Advisories on the \censortext \ \ project \censortext \ \ repository,\\ where we will post a summary, remediation, and mitigation details\\ for any patch containing security fixes."} \\

\textbf{Additional information} & \makecell[tl]{Non-security related and project\\ specific e.g. Access for further\\ details that are non-related to the\\ security, Copyright section, project\\ contribution, acknowledgment.} & \makecell[tl]{``We prefer all communications to be in English''} \\

\textbf{Information on maintainer} & \makecell[tl]{Maintainer information e.g. name, \\email, role, responsibility.} & \makecell[tl]{``\censortext \ \ is always open to feedback, questions, and suggestions.\\ If you would like to talk to us, please feel free to email us at \censortext \ \,\\ and our PGP key is at \censortext \ .''} \\

\textbf{Others} & \makecell[tl]{None of all above.} & - \\

\bottomrule
\end{tabular}
\caption{Categories of Security Policy Titles}
\label{}
\end{table*}