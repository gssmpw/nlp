\section{Introduction}
\label{1_introduction}

Open-source software has become a fundamental component of technological advancement and is used by individuals and organizations worldwide. This impact is especially evident in fields like data science, where open-source tools provide foundational support and enable complex analyses that would be challenging without community-driven resources. \cite{Malone2020Doing}
% \morakot{cite}
This reusable code is publicly available in online repositories and delivered by software management systems, such as npm for Node.js or Python Package Index (PyPI)\cite{PyPI} for Python packages. These repositories host an extensive collection of reusable code, offering numerous benefits by providing reusable code to save time and effort in software development, these libraries are widely reused among developers. 
\cite{DBLP:journals/corr/abs-1907-11073}
% \morakot{cite}
With this widespread usage comes a dependence on packages, increasing the spread of vulnerability problems through reused packages.

Existing research underscores the critical role of security policies in effectively managing vulnerabilities within open-source software. Nusrat Zahan et al. \cite{zahan2023softwaresecuritypracticesyield} recommend for a comprehensive approach to security practices, including the implementation of security policies that incorporate best practices in project management and secure development. This viewpoint highlights the necessity of a proactive and holistic strategy for security, aimed at addressing vulnerabilities throughout the software development lifecycle.
ecurity, addressing vulnerabilities throughout the software lifecycle.
Ayala, Garcia, et al. \cite{10190609} conducted an empirical study of GitHub workflows and security policies in popular repositories, finding that only 37\% of repositories have workflows enabled, and only 7\% have defined security policies. The study also highlights limited adoption of CodeQL, GitHub's static analysis tool for vulnerability detection, with only 13.5\% of applicable repositories utilizing it. Zahan et al. \cite{10163720} analyze security practices in npm and PyPI using the OpenSSF Scorecard, which assesses projects based on security metrics such as workflow configuration and vulnerability reporting. Their study reveals notable security gaps and low adoption rates of best practices, finding that security policies recommended by GitHub are implemented in fewer than 4\% of projects. 
These findings emphasize the need for open-source project maintainers to prioritize workflow configuration, automated static analysis, and security policy definition to mitigate vulnerabilities and strengthen the security of widely-used software in the supply chain. Additionally, they underscore the importance of consistent security measures, suggesting that adopting standardized security policies could significantly improve open-source security.
The security of the software packages is the backbone of numerous applications used in the industry. It is essential to ensure that these packages maintain a security posture because they are publicly accessible and unaddressed vulnerabilities can have significant consequences for their users. Fixing the vulnerabilities rapidly after their discovery is essential, especially after the vulnerabilities have been publicly disclosed. Alfadel et al. \cite{EmpiricalAnalysisOfSecurityVulnerabilitiesInPythonPackages} analyzed the security vulnerability in PyPI for Python packages, and found that the vulnerability in PyPI packages has been increasing over time, with some projects taking a long time to disclose and fix the vulnerabilities.

\begin{figure*}[ht!] 
    \centering
    \includegraphics[width=0.8\linewidth, 
    % height= 0.19\linewidth
    ]
    {3RQsFlowFinal_Fixed.png}
    \caption{Overview of the study}
    \label{fig:overviewFig}
\end{figure*}

Given the critical need for maintaining security in open-source software, platforms hosting these repositories have a role in supporting security practices. GitHub, one of the largest platforms for open-source code, provides maintainers with tools to establish and communicate security guidelines directly within the project repository. This includes the ability to add a SECURITY.md file \cite{GitMd}, a dedicated document for outlining the project's vulnerability disclosure process, reporting steps, and security-related guidelines. By offering this feature, GitHub encourages maintainers to adopt a proactive approach to managing security, helping developers address vulnerabilities efficiently and transparently. A study by Ayala and Garcia~\cite{10190609} suggests that maintainers should prioritize adding a security policy to their projects, as this can help prevent vulnerabilities from lingering in the source code. While GitHub's guidelines emphasize the importance of a reporting mechanism through tools like SECURITY.md, Nusrat et al.~\cite{zahan2023softwaresecuritypracticesyield} suggest that security practices including security policy should cover a broader scope, covering all aspects of best practices in project management and secure software development. This holistic approach can better support maintainers in proactively managing vulnerabilities. 




Our study builds on this premise by investigating the specific contents and structures that constitute effective security policies. By identifying patterns within these policies, we aim to provide open-source pracitioners (e.g., maintainers) with insights that can optimize security management and reduce project vulnerabilities. Despite the crucial role of security policies in managing vulnerabilities, there is limited research study focused on the characteristics of these policies. The primary goal of our study is to identify patterns in defining effective security policies, which can serve as a valuable reference for open-source maintainers. This paper conducts a preliminary investigation into the characteristics, content, and structures that define security policies in software repositories. We first examine the main categories of security policies, identify common patterns within them, and analyze how these policies correlate with specific repository characteristics. By addressing these aspects, this study aims to provide a foundation for developing strategies that enhance security management practices in software development environments. We thus define three research questions to guide our study as follows:


\begin{enumerate}
\item RQ1: What are the primary categories of security policies commonly declared in software repositories?

% \item RQ2: Are there any correlations or associations among the categories of security policies in software repositories?

\item RQ2: Do security policies follow the GitHub template?

\item RQ3: Does the occurrence of security policy categories correlate with characteristics of software repositories?
\end{enumerate}

Our preliminary study of over 300 security policies in open-source projects reveals nine distinct categories of security policy content. Most security policies include a reporting mechanism but rarely outline the procedural steps that will be followed once a vulnerability is reported. Analysis of the co-occurrence of these categories also indicates that user guidelines, such as instructions for secure dependency usage, recommended configurations, and general project usage, are more commonly present in projects that incorporate all basic categories recommended by GitHub. The remainder of this paper provides an in-depth discussion of our study's methodology, findings, and implications.


% Thus, this paper aims to investigate the contents and structures that constitute well-defined security policies, finding how these can be optimized for better security management.
% \morakot{says something about the big goal of the study. the big goal is that we want to find a good pattern in definding security policy} \morakot{says that thus this paper aim to initially investigate ....}

% This gap in the literature, especially regarding the content and structure of security policies, motivates our investigation into their defining characteristics.\chaiyong{I think this repeats the previous paragraph. Need paraphrasing or removal.}

% We define three research questions to guide our preliminary study:

% \morakot{We define three resaerch questions to guild our preliminary study as follows:}

% \begin{enumerate}
% \item \textbf{RQ1:} What are the main categories of security policies in software repositories?

% \item \textbf{RQ2:} What common patterns can be observed within security policies across repositories?

% \item \textbf{RQ3:} How do security policies correlate with specific repository characteristics?
% \end{enumerate}

% \morakot{RQ1: What are the main categories of security policies in software repositories?
% RQ2: What common patterns can be observed within security policies across repositories?
% RQ3: How do security policies correlate with specific repository characteristics?}



%  \morakot{says briefly about our finding}

% \morakot{We thus conducted the student on xxxx Github repositoies with secrueirty.md. We found that ......}

% To address these research questions, we analyzed 303 security policies within 679 PyPI projects found in GitHub’s advisories report. Our study categorizes security policy contents into nine distinct categories and examines the frequency and context of these categories to identify those most commonly utilized across projects. Through this analysis, we aim to provide insights that can help maintainers enhance the security readiness of their repositories by establishing clear, structured, and comprehensive security guidelines.

% \morakot{The rest of this paper is organized as follows. Section II provides existent studies on commit message generation and applications of LLMs. Section III presents the research questions of this work and the overview of our approach. Section IV is the results and discussion.}