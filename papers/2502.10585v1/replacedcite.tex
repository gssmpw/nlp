\section{Related Work}
Traditionally, Brito et.al ____ proposed a model predictive contouring control (MPCC) based  collision avoidance method that compared both hard and chance constraints while avoiding humans. However, the method does not consider the future predicted states of humans while planning. In ____,  the authors proposed a Decentralized Structural-Recurrent Neural Network (DS-RNN) which trained a policy on the reciprocal collision avoidance (ORCA) dataset while considering the future predicted states of humans  and then successfully transferred the policy from simulation to real-world for robot navigation among humans.  Similarly, Chalaki et.al ____ introduced multi-robot cooperative navigation that used deep learning based human trajectory prediction using Social-LSTM and game-theoretic planning with collision avoidance  using MPC to plan. However, the prior works only considered deterministic single-step or multi-step future prediction and did not account for  uncertainty arising from stochastic human motion in planning.


 ____ presented one of the seminal works in social navigation   that enhances the safety of autonomous systems by incorporating uncertainty estimates into navigation policies. Using Monte Carlo (MC) Dropout ____ and bootstrapping, the method provided computationally efficient uncertainty-aware collision avoidance around pedestrians.
Similarly, Zhang et. al ____ presented a method that combines Multi-modal Motion Predictions (MMPs) using deep learning with predictive control to enable safe, collision-free navigation for mobile robots in dynamic indoor environments. Alternatively, researchers also used   Bayesian filters such as Kalman filter to predict and propagate uncertainty of surrounding obstacles into planning.  Zu et.at ____ used extended Kalman filter (EKF) to propagate uncertainty of robot and surrounding obstacles formulating  a chance constrained trajectory optimisation for multi-agent collision avoidance. Similarly, 
____  proposed a method that combines dynamic control barrier function (D-CBF)  with model predictive control (MPC) to ensure safety-critical dynamic obstacle avoidance with demonstrations through simulations and real-world experiments. However, one major drawback of these works is the use of Bayes filters to predict the future trajectory of surrounding obstacles which can be short-sighted and fail to capture long-term non-linearities.  Alternatively,   data-driven prediction methods can address such drawbacks and may be necessary for accurate incorporation of prediction into planning. 

Along similar lines, ____  presented a framework for planning in unknown dynamic environments with probabilistic safety guarantees using conformal prediction. By integrating trajectory predictions and uncertainty quantification into a Model Predictive Controller (MPC), the approach ensures provably safe navigation in  pedestrian-filled intersections within a simulated environment. 
Recently, ____  introduced a distributionally robust chance-constrained model predictive control (DRCC-MPC) for safe robot navigation in human-populated environments. It uses a probabilistic risk metric to account for uncertainties in human motion, ensuring robustness against prediction errors. Our work is aligned along similar lines where we use data-driven deep ensemble based probabilistic  model to predict the trajectory of  surrounding humans ____ and incorporate the information into planning. Also, we compare  robot navigation under different constraints such as CBF, hard constraint and chance constraints on their effectiveness for safe navigation.



\subsection*{\textbf{Contributions.}} The main contributions of the paper are:
\begin{itemize}
    \item   We present a uncertainty-aware planning algorithm that combines data-driven uncertainty-inclusive trajectory forecasting model with MPC for robot social navigation.

    \item We provide a comprehensive list of simulations comparing hard constraint, chance constraint and control barrier function for robot navigation on  publicly available pedestrian datasets such as ETH and UCY.

    \item We also validate our planning algorithm by incorporating out of distribution prediction for multiple pedestrians in a narrow corridor through offline navigation. 

\end{itemize}

% \subsection*{\textbf{Organisation of the paper.}}
% We formulate data-driven trajectory prediction of surrounding humans with uncertainty quantification in Section III. Section IV proposes to integrate the prediction uncertainty of surrounding humans into the planning architecture of robot for safe navigation. Chapter V provides the implementation details discussing the  simulations and experimental results. Finally,  we conclude the paper and provide limitations as well as the future scope of the current  work in chapter VI.


% Lately, researchers have also looking into interaction-aware planning as the prior works considered the movement of humans as independent and robot complied to human motion. This assumption of  independent  human motion may still lead to "freezing robot problem" in dense human crowds.  To address this, ____  introduced interacting Gaussian processes that model cooperative collision avoidance and goal-driven human behavior. The proposed method significantly improves navigation performance, proving crucial for safe and efficient robot navigation in complex, crowded environments. Similarly, ____ developed probabilistic collision checking that accounts for  robot and obstacle's geometry and uncertainty and generated closed form solutions where robot and obstacle states are described by  independent and dependent (interactive robot-obstacle models) Gaussian distributions.