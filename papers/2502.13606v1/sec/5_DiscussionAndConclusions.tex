\section{Discussion \& Conclusions}
In this study, we introduce a novel method called LaVCa, which leverages LLMs to produce data-driven, natural-language descriptions of voxel selectivity in the human visual cortex. The voxel captions generated by LaVCa exhibit higher accuracy and greater semantic diversity than those generated by the existing approach, BrainSCUBA. We attribute this improvement to our mechanism for integrating multiple keywords extracted by advanced LLMs, which enables a more comprehensive capture of the diverse selectivity patterns across voxels.

Despite the overall improvement in brain activity prediction, we observe that face-selective regions do not achieve accuracy levels as high as those in other ROIs (Figure \ref{results:ROI_caption_shuffle}). One reason may be that our current approach uses a Multimodal LLM (MLLM) to produce relatively simple captions for optimal images, often omitting important local features (e.g., ``eyes,'' ``nose'') and focusing on more global terms (e.g., ``face,'' ``person''). Consequently, the subsequent summarization step lacks access to these local details. Because our method relies on language descriptions, it has inherent limitations in capturing the fine-grained, local selectivity of these voxels. Incorporating recent techniques that visually interpret local voxel selectivity ~\cite{luo2024brain} could help address this gap.

Furthermore, while our current study describes voxel selectivity primarily in response to visual stimuli in the occipital cortex, there exist ``multimodal voxels'' in the brain that are simultaneously activated by auditory and linguistic information, and higher-order cognitive processes such as calculations, memory retrieval, and reasoning ~\cite{nakai2020quantitative, nakai2022representations}. Designing stimuli and experimental tasks encompassing diverse sensory inputs (e.g., auditory, textual) and cognitive challenges (e.g., recalling past events, performing reasoning tasks) is essential when interpreting such voxels. Because our approach uses LLM-based textual summarization, it can be adapted to represent a wide range of stimuli and cognitive states in text form, providing a unified framework for multimodal integration. Looking ahead, by jointly modeling images, semantic information, auditory features, and cognitive tasks, we anticipate capturing the brain’s integrated representation of both sensory and higher-order cognitive functions with greater accuracy.

