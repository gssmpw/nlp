\begin{table}[!h]

\begin{subtable}[t]{0.48\textwidth}
\centering
\resizebox{\linewidth}{!}{
\begin{tabular}[h]{c|ccccc}
\toprule
Hyper-Parameter $\mu$ & 0.01 & 0.1 & 0.5 & 1 & 100\\
\midrule
OOD Avg. & 60.95 & 59.96 & 61.17 & 60.31 & 60.40\\
\midrule
ID & 92.52 & 92.68 & 92.72 & 92.65 & 92.45\\
\bottomrule
\end{tabular}}
\caption{DomainNet-oVQA hyper-parameter ($\mu$) sweep.}
\label{tab:dom_hyper}
\end{subtable}
\hfill
\begin{subtable}[t]{0.48\textwidth}
\centering
\resizebox{\linewidth}{!}{
\begin{tabular}[h]{c|ccccc}
\toprule
Hyper-Parameter $\mu$ & 0.01 & 0.1 & 0.5 & 1 & 100\\
\midrule
OOD Avg. & 63.15 & 62.78 & 63.50 & 63.52 & 63.44\\
\midrule
ID & 86.91 & 86.85 & 87.40 & 87.08 & 87.18\\
\bottomrule
\end{tabular}}
\caption{VQA hyper-parameter ($\mu$) sweep.}
\label{tab:vqa_hyper}
\end{subtable}
\caption{\textbf{Sensitivity Analysis of Hyper-Parameter $\mu$ on ID and OOD performance.} We sweep $\mu \in \{0.01, 0.1, 0.5, 1, 10\}$. For DomainNet-oVQA and VQA experiments, both ID and average OOD performance fluctuates slightly and are robust to the change of $\mu$ over a wide range.}
\label{tab:hyper}
\end{table}

 % Stronger regularizations (larger values) decrease deviation, simultaneously improving OOD performance. The ID performance is not impacted significantly.