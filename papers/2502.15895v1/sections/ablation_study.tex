\section{Hyper-Parameter Tuning and Ablation Study}

\subsection{Variation of Projection Strength Throughout Training}

\begin{figure}[!h]
    \centering
    % First image on the left
    \begin{minipage}[b]{0.45\textwidth}
        \centering
        \includegraphics[width=\textwidth]{figures/domainnet_sensitivity.png}
        % \caption{First Image}
    \end{minipage}
    \hspace{0.05\textwidth}  % Space between the images
    % Second image on the right
    \begin{minipage}[b]{0.45\textwidth}
        \centering
        \includegraphics[width=\textwidth]{figures/vqa_sensitivity.png}
        % \caption{Second Image}
    \end{minipage}
    
    \caption{Variation of the Average Projection Strength $\omega$ of all Layers over Iterations. We present the results of $\mu \in \{0.01, 0.1, 0.5, 1, 100\}$. We use the sliding window with window sizes of 50 and 200 to visualize the results for DomainNet-oVQA (Real) and VQAv2. The projection strength $\omega$ is dynamic over iterations, growing from small to large and converging in the end.}
    \label{fig:variation}
\end{figure}

We visualize the variation of the average projection strength $\omega$ of all layers over iterations for five different hyper-parameters $\mu \in \{0.01, 0.1, 0.5, 1, 100\}$ in Fig.~\ref{fig:variation}. As we increase $\mu$, the projection strength $\omega$ becomes larger. For all cases, the projection strength $\omega$ starts from zero and converges at the end of the training. This aligns with our intuition that the projection strength should vary over time to learn dynamic priority of the two objectives during different stage of training.

\subsection{Impact of Hyper-Parameter Sensitivity on Robustness}
\label{sec:sensitivity_analysis}

\begin{tikzpicture}
    \begin{axis}[
        width=\linewidth,
        ylabel style={font=\scriptsize,yshift=-0.6em},
        y tick label style={font=\scriptsize},
        x tick label style={font=\scriptsize},
        ybar,
        %axis lines=left,  
        ymajorgrids,
        symbolic x coords={XGBoost, gMLP, PedCA-FT},
        %xtick={XGBoost, LightGBM, {ours}},
        ylabel={Sensitivity},
        ymin=0,
        ymax=55,
        bar shift=0pt,
        %bar width=0.5cm,
        nodes near coords, 
        nodes near coords style={font=\scriptsize}, 
        %enlargelimits=0.10,
    ]
        \addplot[
            fill=Set2-A,
            ybar,
            error bars/.cd,
            y dir=both,
            y explicit,
        ] coordinates {
            (XGBoost, 39.04) += (0, 8.1) -= (0, 7.53)
        };
        \addplot[
            fill=Set2-B,
            ybar,
            error bars/.cd,
            y dir=both,
            y explicit,
        ] coordinates {
            (gMLP, 8.22) += (0, 5.6) -= (0, 3.46)
        };
        \addplot[
            fill=Set2-C,
            ybar,
            error bars/.cd,
            y dir=both,
            y explicit,
        ] coordinates {
            (PedCA-FT, 42.47) += (0, 8.11) -= (0, 4.73)
        };
    \end{axis}
\end{tikzpicture}

We further perform the sensitivity analysis of the hyper-parameter $\mu$ on ID and average OOD performance for DomainNet-oVQA and VQA experiments. Results from Tab.~\ref{tab:hyper} show that both ID and OOD results fluctuate slightly even when $\mu$ spans over a wide range from 0.01 to 100. This again proves that \emph{DiGrap} is more controllable and less sensitive to hyper-parameter change.

\subsection{Ablating Fixed and Trainable Projection Strength}
\label{sec:compare_fixed_trainable}
\begin{table}[!h]
\centering
\resizebox{0.85\linewidth}{!}{ % Adjust the table to fit the page width
\begin{tabular}{c|c|ccccc|c}
    \toprule
    &\multicolumn{1}{c|}{ID} & \multicolumn{5}{c|}{OOD} &  \\
    & Real & Sketch & Painting & Infograph & Clipart & Quickdraw & OOD Avg. \\
    \midrule
    DiGraP ($\omega=0.1$)& 92.49 & 71.86 & 71.36 & \underline{56.34} & 83.15 & 19.22 & 60.39 \\
    DiGraP ($\omega=0.5$)  & \underline{92.63} & 71.24 & 71.46 & 56.08 & \underline{83.17} & 18.61 & 60.11 \\
    DiGraP ($\omega=0.9$) &  92.53 & \textbf{73.09} & \underline{71.73} & \underline{56.34} & 82.85 & \underline{20.11} & \underline{60.82}\\
    \midrule
    DiGraP (trainable)  & \textbf{92.72} & \underline{72.56} & \textbf{72.31} & \textbf{57.52} & \textbf{83.32} & \textbf{20.13} & \textbf{61.17}\\
    \bottomrule
\end{tabular}}
\caption{\textbf{Comparing Fixed and Trainable Projection Strength $\omega$ on DomainNet-oVQA.} \textbf{Bold}: best. \underline{Underline}: second best. Trainable projection strength outperforms different fixed projection strengths on both ID and average OOD.}
\label{tab:ablate_fixed_omega}
\end{table}


To validate the effectiveness of trainable projection strength $\omega$, we conduct analysis to compare with fixed projection strength with different values $\omega \in \{0.1, 0.5, 0.9\}$. Tab.~\ref{tab:ablate_fixed_omega} shows that trainable \emph{DiGraP} outperforms the others and achieves the best ID and average OOD results.


% \subsection{Ablating Constrained and Unconstrained Projection strength}