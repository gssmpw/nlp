\section{Directional Gradient Projection for Robust Fine-tuning}
In this section, we first describe the intuition and the mathematical motivation behind \emph{DiGraP}. Then, we provide our method's concrete algorithmic design.

\begin{figure}[!h]
     \centering
     \includegraphics[width=0.85\textwidth]{figures/method.pdf}
     \caption{\textbf{Directional Gradient Projection.} (Left) When the gradients are non-conflicting, we do not perform projection. (Right) When the gradients are conflicting, we project the gradient for the original loss function according to the gradient for the regularization term. We add $\omega \in [0,1]$ to control the projection strength, where $\omega=0$ is an unconstrained update and $\omega=1$ represents a full projection to the orthogonal direction of the gradient for the regularization term.}
     \label{fig:domainnet-ovqa}
\end{figure}


\subsection{Robust Fine-tuning as a Multi-objective Optimization Problem}
In order to adapt the model to the downstream tasks but preserve the power of the pre-trained model (e.g. robustness to distribution shifts), methods such as L2-SP~\citep{li_explicit_2018} impose a regularization term on the distance between the fine-tuned and pre-trained weights% . These methods have the advantage of being simple to implement and do not require more computionally heavy methods requiring bi-level optimization
\zkn{What about positioning w.r.t. TPGM/FTP? Perhaps it can be added in related work that those methods have the complexity of bi-level optimization, etc. Depending on that narrative, we can perhaps not mention here. }. Formally,
\begin{align}
    \label{eq:l2_sp}
    \mathcal{L(\theta)} = \Tilde{\mathcal{L}}(\theta) + \frac{\lambda}{2}\|\theta-\theta_0\|^2_2
\end{align}
where $\theta$ denotes the fine-tuned weights, $\theta_0$ the pre-trained weights, $\Tilde{\mathcal{L}}(\theta)$ the original loss function\zkn{maybe mention e.g. cross-entropy to emphasize no regularization}, and $\lambda$ the hyper-parameter for regularization strength, i.e., weight decay. In this case, $\|\theta-\theta_0\|^2_2$ serves as a constraint so that the updated model will not deviate from the initialization too much, thus we can maintain some strengths from the pre-trained model. However, L2-SP is not intuitive to tune $\lambda$ which often spans over a wide range\zkn{Should we mention that this is also harder to tune if it's applied differently across layers?}: a small $\lambda$ may achieve better in-distribution performance but leads to poor OOD robustness, while a large $\lambda$ results in underfitting. Besides, L2-SP is also harder to tune if applied differently across layers. 

In this work, we instead propose\zkn{Use explicit words like ``we propose'' to make it clear what your contribution is. I added this.} to view robust fine-tuning from a \textit{multi-objective optimization} perspective, leading to a more explicit method to balance this trade-off. Specifically, there are two objectives that we want to optimize,
\begin{align}
    \label{eq:two_obj}
    \textbf{Objective}_1 = \Tilde{\mathcal{L}}(\theta), \textbf{Objective}_2 = \frac{1}{2}\|\theta-\theta_0\|^2_2
\end{align}
where the first objective represents the original loss function and the second objective represents the distance between the fine-tuned and pre-trained weights. Our goal is to minimize both at the same time to achieve ID generalization and OOD robustness.


\subsection{Projecting Conflicting Gradients}
Viewed from this perspective, we can leverage prior multi-objective methods towards our problem.\zkn{Added transition} PCGrad~\citep{yu2020gradientsurgerymultitasklearning} hypothesizes that the key optimization issue in multi-objective learning arises from conflicting gradients, where gradients for different objectives point away from each other. Thus, optimizing one of them will lead to the suboptimality of the others.
They propose a form of gradient surgery by projecting a task’s gradient onto the normal plane of the gradient of any other task that has a conflicting gradient, therefore benefiting all objectives. 

Inspired by PCGrad, we propose the following algorithm for robust finetuning: When the gradients between the two objectives are in conflict, i.e. their cosine similarity is negative, we project the gradient for the original loss function to the orthogonal direction of the gradient for the regularization term. Specifically, the gradients for the two objectives and the projection of the first gradient in the direction of the second gradient are respectively:
\begin{align}
    \label{eq:grad}
    \tilde{g}_{1} = \nabla_{\theta} \Tilde{\mathcal{L}}(\theta),
    ~\tilde{g}_{2} = \theta - \theta_0,
    ~\tilde{g}_{1}^{\text{proj}} = \frac{\tilde{g}_{1} \tilde{g}_{2}}{\|\tilde{g}_{2}\|^2} \tilde{g}_{2}
\end{align}
We add a hyper-parameter $\omega \in [0,1]$ to further control the projection strength. $\omega=0$ is equivalent to an unconstrained gradient update, while $\omega=1$ is the same as a full orthogonal projection. The final projected gradient is the following:
\begin{align}
    \label{eq:final_grad}
    g = \tilde{g}_1 - \omega \tilde{g}_{1}^{\text{proj}} = \tilde{g}_1 -\omega \frac{\tilde{g}_{1} \tilde{g}_{2}}{\|\tilde{g}_{2}\|^2} \tilde{g}_{2},~\omega \in [0,1]
\end{align}

Note that for L2-SP, the gradient for the regularized loss function is formulated similarly:
\begin{align}
    \label{l2sp_grad}
    g = \nabla_{\theta} \mathcal{L(\theta)} = \nabla_{\theta} \Tilde{\mathcal{L}}(\theta) + \lambda (\theta-\theta_0) = \tilde{g}_1 + \lambda \tilde{g}_2
\end{align}

Thus, \emph{DiGraP} is equivalent to L2-SP with different $\lambda$ for every layer\zkn{This is not clear; you don't have $t$ subscript above; do you mean that part of your proposed algorithm is to make this layer/iteration dependent? Also, wasn't L2-SP also applied per layer at least? Need to clarify here and be more precise. }. In summary, for each layer $i$:
\begin{itemize}
    \item \textbf{Gradients are non-conflicting} ($\tilde{g}_{1}^i \tilde{g}_{2}^i \geq 0$): $\lambda^i = 0$
    \item \textbf{Gradients are conflicting} ($\tilde{g}_{1}^i \tilde{g}_{2}^i < 0$): $\lambda^i = -\omega^i \frac{\tilde{g}_{1}^i \tilde{g}_{2}^i}{\|\tilde{g}_{2}^i\|^2}$
\end{itemize}

Compared to L2-SP, the hyper-parameter $\omega$ in \emph{DiGraP} is within the range between 0 and 1, which is more intuitive to tune. Furthermore, even with one fixed $\omega$, the regularization strength $\lambda$ varies across both layers and iterations, making the fine-tuning process more flexible to fit the training data.


\subsection{Layer-wise Trainable Directional Gradient Projection}
We emphasize that the regularization problem in Eq.~\ref{eq:l2_sp} is still not fully equivalent to the multi-objective optimization in Eq.~\ref{eq:two_obj}\zkn{This is confusing; 3.2 makes it seem like you're already converting eq. 1 into a multi-objective problem inspired by pccgrad. Now you're going back to eq. 1 and 2. }. Specifically, for a multi-objective optimization problem, we want to optimize all objective functions, i.e., to minimize both $\Tilde{\mathcal{L}}(\theta)$ and $\frac{1}{2}\|\theta-\theta_0\|^2_2$ in our case. However, for a regularization problem, the regularization term does not necessarily decrease. Instead, it acts as a constraint on the original loss function and the regularization term is smaller compared to the one in a model trained without regularization. Projecting the original gradient to the orthogonal direction of the gradient for the regularization term will potentially lead to underfitting. It is especially detrimental at the beginning of the training, where the fine-tuned weights are close to the pre-trained weights, thus it is more benefitial for the model to stick to its original gradient descent direction.

As a result, we aim for the projection strength $\omega$ to be dynamic throughout the training process. Intuitively, $\omega$ should start small during the early iterations, allowing the model to prioritize fitting to the downstream task. As training progresses and the fine-tuned model diverges further from its initial state, $\omega$ should gradually increase to guide the fine-tuned gradient direction towards alignment with the regularization gradient direction. {In Sec.~\ref{sec:sensitivity_analysis} we will visualize the variation of projection strength $\omega$ throughout training to further validate this motivation.} \zkn{Are you adding analysis later confirming this? That's crucial if this is one of the motivations.}

To achieve this, we make the projection strength $\omega$ trainable, allowing it to adapt throughout the training process\zkn{Does this also remove the extra hyper-parameter (which would be a benefit)? If so, mention}.
For the $t$ step of unconstrained gradient descent with the learning rate of $\alpha$, the model weights update as follows\zkn{It's not clear what this is. A definition/background for the derivative in eq 8? Need some text/transition. Same with next eq 7 description as well. },
\begin{align}
    \label{eq:unconstrained_gd}
    \tilde{\theta}_{t} = \theta_{t-1} - \alpha \tilde{g}_{t,1}
\end{align}
where $\tilde{\theta}_{t}$, $\theta_{t-1}$ and $\tilde{g}_{t,1}$ denote the unconstrained model weights at current step $t$, the model updates of previous step $t-1$ and the gradient for the original loss function at current step $t$.

For one step of directional gradient descent, the model weights update as follows,
\begin{align}
    \label{eq:directional_gd}
    \theta_{t} = \theta_{t-1} - \alpha (\tilde{g}_{t,1} - \omega_t \frac{\tilde{g}_{t, 1} \tilde{g}_{t, 2}}{\|\tilde{g}_{t, 2}\|^2} \tilde{g}_{t, 2})
    = \tilde{\theta}_{t} + \alpha \omega_t \frac{\tilde{g}_{t, 1} \tilde{g}_{t, 2}}{\|\tilde{g}_{t, 2}\|^2} \tilde{g}_{t, 2}
\end{align}
where $\theta_{t}$, $\omega_t$ and $\tilde{g}_{t, 2}$ denote the constrained model weights, the projection strength and the gradient for the regularization term at current step $t$. $\tilde{\theta}_{t}$ and $\tilde{g}_{t,1}$ are the same as the ones in Eq.~\ref{eq:unconstrained_gd}.

The derivative of the original loss function $\Tilde{\mathcal{L}}(\theta)$ w.r.t. $\omega$ is as follows using the chain rule:
\begin{align}
    \label{eq:omega_lr}
    \nabla\omega := \frac{\partial \Tilde{\mathcal{L}}(\theta_{t-1})}{\partial \omega}
    = \frac{\partial \Tilde{\mathcal{L}}(\theta_{t-1})}{\theta} \frac{\partial \theta_{t-1}}{\partial \omega}
    = \alpha \tilde{g}_{t,1} \tilde{g}_{t-1, 1}^{\text{proj}}
\end{align}

We initialize $\omega_0=0$ for the first iteration and update with learning rate of $\mu$. We also add normalization on $\nabla\omega$ for numerical stability. For \emph{DiGraP}, instead of tuning the weight decay $\lambda$ in L2-SP, we only tune the learning rate $\mu$ of the projection strength $\omega$\zkn{Is this less sensitive? We should have sensitivity analysis}. {We argue that tuning $\mu$ is less sensitive and will provide sensitivity analysis in Sec.~\ref{sec:sensitivity_analysis}. We also compare fixed and trainable projection strength $\omega$ in Sec.~\ref{sec:compare_fixed_trainable}.} The final algorithm of \emph{DiGraP} is illustrated in Alg.~\ref{algo:digrap}.




Based on the elements reviewed in \Cref{sec:related_work}, we contribute quantization schemes tailored for small values of $M$. This section is structured as follows: %
In \Cref{sec:wfr}, we propose simulating the WFR gradient flow to minimize the $\MMD$ using a weighted mixture of $M$ Diracs. In \Cref{sec:msip}, we use a sufficient condition for the WFR gradient flow's steady state to derive a fixed-point iteration. 
Moreover, we demonstrate that this iteration can be expressed as a preconditioned gradient descent for a function related to the $\MMD$. Further, the iteration is a natural extension of the mean shift algorithm.


\subsection{Simulating the WFR gradient flow using ODEs}\label{sec:wfr}
Without loss of generality on the time-scale, we consider \eqref{eq:WFR_for_MMD} with a reaction speed $\beta\equiv 1.$
We seek $\mut$ as a mixture of $M$ Diracs, i.e.,
\(
    \mut = \sum_{i = 1}^{M} \wit \delta_{\yit},
\)
where $\yonet, \dots, \ymt \in \mathcal{X} $ and $\wonet, \dots, \wmt \in \RR$. 


\begin{proposition}\label{prop:coupled_ode_WFR}
Define the system of differential equations
\begin{equation}\label{eq:WFR_particle_equation}
\begin{aligned}
    \dotyit &= - \alpha\nabla\Ffv[\mu_t](\yit) =-\alpha\left(\sum\limits_{m = 1}^{M} \wit \nabla_2 \kappa(\yjt, \yit) - \nabla v_{0}(\yit)\right)  \\
     \dotwit &= -  \wit\Ffv[\mut](\yit) =- \wit \left(\sum\limits_{m = 1}^{M} \wit \kappa(\yjt, \yit) - v_{0}(\yit) \right)
\end{aligned}
\end{equation}
where $i \in [M]$, and  $\vzero$ as defined in \eqref{eq:mke_def}.
If the trajectory $(\mut)_{t \geq 0}$ satisfies \eqref{eq:WFR_particle_equation}, then $(\mut)_{t \geq 0}$ weakly satisfies \eqref{eq:WFR_for_MMD}.
\end{proposition}


The proof and a more detailed account of \Cref{prop:coupled_ode_WFR} are given in \Cref{proof:coupled_ode_WFR}.
We can use standard ODE solvers to efficiently approximate $\mut$ provided the function $\vzero$ can be evaluated. For instance, when $\pi$ is an empirical measure of $N$ atoms in $\RR^d$, the system of equations \eqref{eq:WFR_particle_equation} can be implemented in 
$O(d(M+N)M)$ operations\footnote{This assumes that evaluating $\kappa$ is $O(1)$ and $\nabla_2\kappa$ is $O(d)$.}.


Informally, if $\wizero > 0$, we are ensured that $\wit \geq 0$ for sufficiently small $t$ %
by the structure of \eqref{eq:WFR_particle_equation}. To understand this, we consider the sign of the term $\Ffv[\mu_t](\yit)$: If this term is negative, then $\wit$ will increase locally in $t$, drifting away from $0$. However, if the term is positive, the weight $\wit$ locally decreases toward a steady state $\wit\equiv 0$. This argument only holds for operator $\Ffv$ that is continuously differentiable in $\mut$.













\subsection{A fixed-point scheme for steady-state solutions}\label{sec:msip}










While the solution to \eqref{eq:WFR_particle_equation} is an entire trajectory $(\mut)_{t\geq 0}$, we are primarily interested in $\mu_{\infty}$; if it exists, it should be a minimizer to the functional $\mathcal{F}$. To find $\mu_\infty$, we look for a steady state solution satisfying $\dotyit\equiv 0$ and $\dotwit\equiv 0$, which is ensured by the sufficient condition for all $i \in [M]$
\begin{align}
       \sum\limits_{m = 1}^{M} \wms \kappa(\yms,\yi) & = \int_{\mathcal{X}} \kappa(x,\yi) \dpix \label{eq:mixture_diracs_kbar_eq_1}\\
       \sum\limits_{m = 1}^{M} \wms \nabla_{2} \kappa(\yms,\yi) & = \int_{\mathcal{X}} \nabla_{2} \kappa(x,\yi) \dpix. \label{eq:MMD_gradient_mixture}
\end{align}

The condition~\eqref{eq:mixture_diracs_kbar_eq_1} is prevalent in the literature of kernel-based quadrature, and reflects that the quadrature rule defined by the measure $\mu = \sum_{m = 1 }^{M} \wms \delta_{\yms}$ is exact on the subspace spanned by the functions $\kappa(\cdot,\yi)$. Note that equation 
\eqref{eq:mixture_diracs_kbar_eq_1} enforces optimal kernel quadrature weights $    \hat{\bm{w}}(\Y) = \bm{K}(\Y)^{-1} \vzerov(\Y)$, as in~\eqref{eq:optimal_w}, and the resulting quadrature rule thus satisfies the optimality property \eqref{eq:MMD_optimality_okq}. 
Now, we explore the second condition~\eqref{eq:MMD_gradient_mixture}, less known in the literature, when making \Cref{assumption:gradient_kappa}.
\begin{proposition}\label{prop:steady_state_equation_on_y}
Let $\y \in \mathcal{X}^{M}$. Under \Cref{assumption:gradient_kappa} and assuming that the gradient of $\kappa$ is bounded on $\X$, the equations \eqref{eq:mixture_diracs_kbar_eq_1} and \eqref{eq:MMD_gradient_mixture} imply that 
\begin{equation}\label{eq:mixture_diracs_kbar_eq_2_reformulation}
     \bm{\bar{K}}(\y)\bm{W}(\y)\y  = \bvonem(\y) +  \bar{\bm{u}}(\y),
\end{equation}
where $\bm{\bar{K}}(\y) \in \RR^{M \times M}$ is the kernel matrix with the kernel $\bar{\kappa}$, diagonal matrix $\bm{W}(\y)$ has entries corresponding to those of $\hw(\y)$ given by \eqref{eq:optimal_w}, and
$\bvonem(\y), \bar{\bm{u}}(\y) \in \RR^{M \times d }$ are the matrices with rows defined as
\begin{align}\label{eq:u_vonebar_def}
(\bvonem(\y))_{i,:} &:=  \int_{\mathcal{X}} x \bar{\kappa}(x,\yi)\dpix,\\
(\bar{\bm{u}}(\y))_{i,:} &:= \Big(\sum\limits_{m=1}^M \wms\bar{\kappa}(\yms,\yi) - \int_{\mathcal{X}} \bar{\kappa}(x,\yi)\dpix\Big)\yi.\nonumber
\end{align}
\end{proposition}



The equation \eqref{eq:mixture_diracs_kbar_eq_2_reformulation} does not admit a tractable closed-form solution for $\y$. We propose a sequence $(\yt)_{t}$ satisfying
\begin{equation}\label{eq:msip_fixed_point_def}
     \bm{\bar{K}}(\yt) \bm{W}(\yt) \ytplus = \hvonem(\yt),
\end{equation}
where 
\begin{equation}\label{eq:v_hat_def}
    \hvonem(\y):=  \bvonem(\y) +  \bar{\bm{u}}(\y).
\end{equation}
Under the assumption that the kernel matrices $\bm{K}(\yt)$ and $\bm{\bar{K}}(\yt)$ are non-singular, the relationship \eqref{eq:msip_fixed_point_def} is expressed as a fixed-point iteration $\ytplus =  \bm{\Psi}_{\MSIP}(\yt)$, where
\begin{equation}\label{eq:fixed_point_general_case}
    \bm{\Psi}_{\MSIP}(\Y): = \bm{W}(\Y)^{-1} \bm{\bar{K}}(\Y)^{-1} \hvonem(\Y).
\end{equation}

We call $\bm{\Psi}_{\MSIP}$ the \textit{mean shift interacting particles} map. When $\kappa$ is the squared exponential kernel, there exists a scalar $\lambda \in \mathbb{R}$ such that $\bar{\kappa} = \lambda \kappa$; then, the condition~\eqref{eq:MMD_optimality_okq} implies $\bar{\bm{u}}(\y) = 0$, and \eqref{eq:fixed_point_general_case} simplifies to 
\begin{equation}
    \ytplus = \bm{W}(\yt)^{-1} \bm{K}(\yt)^{-1} \vonem(\yt),
\end{equation}
where $(\vonem(\y))_{i,:} :=  \int_{\mathcal{X}} x \kappa(\yi,x)\dpix$.
The proof of \Cref{prop:steady_state_equation_on_y} is given in \Cref{proof:steady_state_equation_on_y}.
 






\paragraph{Damped fixed-point iteration}
In the following, we prove that the fixed-point algorithm defined using \eqref{eq:fixed_point_general_case} can be seen as a preconditioned gradient descent iteration on the function 
$F_M: \mathcal{X}_{\neq}^{M} \rightarrow \mathbb{R}$ given by
\begin{equation}\label{eq:opt_mmd_functional}
        F_{M}(\yone, \dots, \ym) := \inf\limits_{\w \in \mathbb{R}^{M}} \F \left[ \sum\limits_{i=1}^{M}\wi \delta_{\yi} \right].
    \end{equation}
We start with the following result.
\begin{theorem}\label{thm:gradient_opt_mmd}
    Under \Cref{assumption:gradient_kappa}, the gradient of $F$, seen as an $M \times d$ matrix, is given by 
\begin{align}\label{eq:gradien_opt_mmd_under_assumption}
\nabla F_{M}(\y) & =  \bm{W}(\y)\left(\bm{\bar{K}}(\y)\bm{W}(\y)\y - \hat{\bm{v}}_1(\y) \right)\\
&= \bm{W}(\y) \bm{\bar{K}}(\y) \bm{W}(\y) \Big( \y - \bm{\Psi}_{\MMS}(\y) \Big), \nonumber
\end{align}
with $\bm{\Psi}_{\MMS}(\y)$ is defined in \eqref{eq:fixed_point_general_case}, where the latter equality holds when $\bm{W}(\y)\bm{\bar{K}}(\Y)\bm{W}(\y)$ is invertible.
In particular, for the Gaussian kernel, the expression simplifies to 
\begin{equation*}\label{eq:gradien_opt_mmd_gaussian_kernel}
  \nabla F_{M}(\y) =  \bm{W}(\y) \bm{K}(\y) \bm{W}(\y) \Big( \y - \bm{\Psi}_{\MMS}(\y) \Big).
\end{equation*}
\end{theorem}
The identity \eqref{eq:gradien_opt_mmd_under_assumption} proves that the fixed points of the function $\bm{\Psi}_{\MMS}$ are critical points of the function $F_M$. This result is reminiscent of \eqref{eq:gradient_G_M_lloyd} proved in \cite{DuFaGu99}, which gives a similar characterization of the fixed points of Lloyd's map $\bm{\Psi}_{\mathrm{Lloyd}}$. Following the same reasoning as in \cite{PoCaPa24}, 
we deduce from~\eqref{eq:gradien_opt_mmd_under_assumption} that the fixed point iteration defined by \eqref{eq:fixed_point_general_case} is a preconditioned gradient descent, with $\bm{M}(\y):=  \mathbf{W}(\y) \bm{\bar{K}}(\y) \mathbf{W}(\y)$ as the preconditioning matrix.
 Indeed, if the matrix $\bm{M}(\yt)$ is non-singular, combining \eqref{eq:fixed_point_general_case} and \eqref{eq:gradien_opt_mmd_under_assumption} yields
\begin{equation}\label{eq:fixed_point_as_precond_grad_descent}
  \ytplus   = \yt - \bm{M}(\yt)^{-1} \nabla F_{M}(\yt).
\end{equation}
Now, if we choose a step size $\eta$ for  \eqref{eq:fixed_point_as_precond_grad_descent}, we get 
\begin{equation}\label{eq:fixed_point_as_precond_grad_descent_alpha}
\begin{split}
  \ytplus &= \yt - \eta\bm{M}(\yt)^{-1} \nabla F_{M}(\yt)\\
  &= (1-\eta) \yt + \eta \bm{\Psi}_{\MMS}(\yt),
  \end{split}
\end{equation}
which is equivalent to a damped fixed-point iteration on $\bm{\Psi}_{\MMS}$ \cite{Ke18}. 

\paragraph{Mean shift through the lens of MMD minimization}
In the case $M = 1$, we see that the classical mean shift algorithm identifies critical points of the function $F_1$.
\begin{corollary}\label{cor:ms_as_MSIP} Under \Cref{assumption:gradient_kappa}, we have
\begin{equation}\label{eq:gradient_F_1}
    \nabla F_{1}(y) = \frac{v_{0}(y)}{\kappa(y,y)} \left(\bar{v}_{0}(y) y - \bar{v}_{1}(y) \right),
\end{equation}
where $\bar{v}_{0}(y):= \int_{\X} \bar{\kappa}(x,y) \dpix \in \mathbb{R}$. 
\end{corollary}
Thus, a critical point $y$ of $F_1$ is either a zero of $v_{0}$ or satisfies
\begin{equation}\label{eq:goq_for_K_1}
      y = \frac{\bar{v}_{1}(y)}{\bar{v}_{0}(y)} = \frac{\int_{\mathcal{X}} x \bar{\kappa}(x, y)  \dpix}{\int_{\mathcal{X}} \bar{\kappa}(x, y) \dpix},
\end{equation}
if the ratio is well-defined. 
In particular, when $\pi$ is taken to be an empirical measure $\pi = \frac{1}{N} \sum_{n=1}^{N} \delta_{\xns}$,
any point $y^\star$ that satisfies $\Psi_{\mathrm{MS}}(y^\star)\equiv y^\star$ as in \eqref{eq:ms_equation} is a critical point of $F_{1}$. Then, the mean shift algorithm becomes a special case of the fixed-point iteration defined in \Cref{sec:msip} when $M = 1$. 
The proof of \Cref{cor:ms_as_MSIP} is given in \Cref{proof:mean_shift_as_mmd_min} as a direct consequence of \Cref{thm:gradient_opt_mmd}.  

We now contrast \eqref{eq:goq_for_K_1} with an unweighted Wasserstein gradient flow for the MMD, as given in \citep{ArKoSaGr19}, when $M=1$. In particular, the path traced using the $W_2$ geometry will use the gradient of the first variation given by $\nabla \Ffv(y) = y \bar{v}_{0}(y) -  \bar{v}_{1}(y)$. When $y$ is far from the data, both $\bvzero(y)$ and $\bvonev(y)$ will vanish and thus the particle's speed vanishes along with them. On the other hand, mean shift ameliorates this through the inverse operation in \eqref{eq:goq_for_K_1}, magnifying the gradient considerably.































{In summary, our unique contribution lies in adapting gradient projection to the specific challenges of robust fine-tuning. While PCGrad was designed for multi-task learning, \emph{DiGraP} extends these principles to the fine-tuning of pre-trained models by introducing a hyper-optimizer and tailoring gradient projection to balance both ID and OOD performance. This refinement allows us to address the unique trade-offs in robust fine-tuning, which are distinct from those in multi-task optimization. }

\subsection{Compatability with Parameter-Efficient Fine-tuning (PEFT) Methods}

\emph{DiGraP} is further compatible with PEFT methods such as LoRA~\citep{hu_lora_2021}, which is a prevalent fine-tuning strategy for large foundation models. PEFT methods generally update new parameters to add to the original weights. In this case, instead of optimizing the distance between fine-tuned and pre-trained weights $\frac{1}{2}\|\theta - \theta_0\|^2$, we minimize the distance between the updated weight and origin $\frac{1}{2}\|\theta\|^2$ in PEFT. Thus, when combined with PEFT, \emph{DiGraP} does not need to save an additional pre-trained copy and requires the same amount of memory as PEFT. In Sec.~\ref{sec:ref_vqa} and Sec.~\ref{sec:ft_vqa}, we demonstrate that \emph{DiGraP} can further improve the results of LoRA on VQA tasks.

% We emphasize that our unique contribution lies in adapting gradient projection to the specific challenges of robust fine-tuning. While PCGrad was designed for multi-task learning, \emph{DiGraP} extends these principles to the fine-tuning of pre-trained models by introducing a hyper-optimizer and tailoring gradient projection to balance both ID and OOD performance. This refinement allows us to address the unique trade-offs in robust fine-tuning, which are distinct from those in multi-task optimization. 

