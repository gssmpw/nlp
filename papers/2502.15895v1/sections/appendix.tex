\newpage
\section{Appendix}
% \subsection{ImageNet Experiments}
% \begin{table}[H]
    \centering
    \resizebox{0.9\linewidth}{!}{
    \begin{tabular}{@{}cc|c|cccc|c@{}}
        \toprule
        & & \multicolumn{1}{c|}{ID} & \multicolumn{4}{c|}{OOD} \\
        &  & \multicolumn{1}{c|}{ImageNet} & ImageNet-2 & ImageNet-A & ImageNet-R & ImageNet-S & OOD Avg. \\ \midrule
        \multicolumn{1}{l|}{\multirow{8}{*}{\rotatebox[origin=c]{90}{$1^\text{st}$ Question}}} & Zero-Shot & 30.17 & 25.89 & 10.23 & 32.39 & 28.76 & 24.32 \\
        \multicolumn{1}{l|}{} & Vanilla FT / L2-SP & 85.66 & 77.11 & 39.45 & 73.52 & 66.97 & 64.26 \\
        \multicolumn{1}{l|}{} & Linear Prob. & 80.27 & 71.38 & 37.91 & 67.45 & 63.92 \\
        \multicolumn{1}{l|}{} & LP-FT & 85.74 & 77.00 & 39.05 & 73.12 & 66.49 & \\
        \multicolumn{1}{l|}{} & FTP & 81.46 & 74.11 & 47.10 & 78.18 & 66.36 \\
        \multicolumn{1}{l|}{} & Adam-SPD & 85.67 & 76.85 & 41.61 & 74.87 & 67.30 \\
        % \cmidrule{2-8}
        \multicolumn{1}{l|}{} & WiSE-FT & & & & & \\
        \multicolumn{1}{l|}{} & DiGraP (0.01) & \textbf{85.79} & 76.97 & \textbf{41.69} & \textbf{73.90} & \textbf{67.32} & \textbf{64.97} \\ \midrule
        \multicolumn{1}{l|}{\multirow{8}{*}{\rotatebox[origin=c]{90}{Follow-up Question}}} & Zero-Shot & 40.50 & 35.24 & 21.12 & 45.99 & 36.63 \\
        \multicolumn{1}{l|}{} & Vanilla FT / L2-SP & & & & & \\
        \multicolumn{1}{l|}{} & Linear Prob. & & & & & \\
        \multicolumn{1}{l|}{} & LP-FT & & & & & \\
        \multicolumn{1}{l|}{} & FTP &  & & & & \\
        \multicolumn{1}{l|}{} & Adam-SPD & & & & & \\
        % \cmidrule{2-8}
        \multicolumn{1}{l|}{} & WiSE-FT & & & & & \\
        \multicolumn{1}{l|}{} & DiGraP (0.01) & & & & & \\ \bottomrule
    \end{tabular}}
        
    \caption{ImageNet-oVQA Fine-Tuning Results. PaliGemma-3B fine-tuned on ImageNet-oVQA and evaluated on ImageNet(variants)-oVQA as a VQA task. We use LoRA for efficiency. Note that L2-SP reduces to Vinilla FT with AdamW under LoRA.
    }
    \label{tab:imagenet_ft}
\end{table}
% \subsection{DomainNet CLIP-ViT}
% \begin{table}[!h]
    \centering
    \resizebox{1\linewidth}{!}{ % 
    \begin{tabular}{c|c|cccc|ccc}
    \toprule
         & ID & \multicolumn{4} {c|} {OOD} & \multicolumn{3} {c} {Statistics} \\
        Method & Real & Sketch & Painting & Infograph & Clipart & OOD Avg. & ID $\Delta$ (\%) &  OOD $\Delta$ (\%) \\
    \midrule
        % Zero Shot & 79.4 & 50.3 & 62.4 & 37.1 & 59.8 & 52.4 & 0.00 & 0.00\\
        Vanilla FT & \underline{86.5} & 45.9 & \textbf{58.9} & \underline{34.8} & 59.7 & 49.82 & 0 & 0 \\
        LP & 83.2 & 38.6 & 51.4 & 27.2 & 50.7 & 41.97 & \textcolor{red}{-3.82} &  \textcolor{red}{-15.76} \\
        L2-SP & 86.4 & 46.0 & \textbf{58.9} & \textbf{35.0} & 59.8 & \underline{49.92} & \textcolor{red}{\underline{-0.12}} & \textcolor{green}{\underline{0.20}} \\
        LP-FT & 84.4 & 36.5 & 50.5 & 27.2 & 51.4 & 41.40 & \textcolor{red}{-2.43} & \textcolor{red}{-16.90}  \\
        FTP & \textbf{86.7} & \underline{49.1} & 55.8 & 32.2 & 61.4 & 49.63 & \textcolor{green}{\textbf{0.23}} & \textcolor{red}{-0.38} \\
        DiGraP (1) & 86.2 & \textbf{51.3} & \underline{57.3} & {34.0} & \textbf{62.4} & \textbf{51.25} & \textcolor{red}{-0.35} & \textcolor{green}{\textbf{2.87}} \\
    \bottomrule
    \end{tabular}}
    \caption{\textbf{OOD Results on DomainNet}. CLIP ViT-Base finetuned on DomainNet (Real) dataset and evaluated on other domains. \textbf{Bold}: best. \underline{Underline}: second best.}
    \label{tab:domainnet_clipvit}
\end{table}

% \subsection{DomainNet-oVQA LlaVA}
% \begin{table}[!h]
\centering
\resizebox{1\linewidth}{!}{ % Adjust the table to fit the page width
\begin{tabular}{c|c|ccccc|ccc}
    \toprule
    &\multicolumn{1}{c|}{ID} & \multicolumn{5}{c|}{OOD} & \multicolumn{3}{c}{Statistics} \\
    & Real & Sketch & Painting & Infograph & Clipart & Quickdraw & OOD Avg. & ID $\Delta$ ($\%$) & OOD $\Delta$ ($\%$)\\
    \midrule
    Zero-Shot & 67.71 & 53.45 & 56.15 & 37.90 & 58.32 & 15.87 & 44.34 & - & - \\
    Vanilla FT\tablefootnote{Same as L2-SP~\citep{li_explicit_2018} under LoRA~\citep{hu_lora_2021}} & \textbf{86.03} & \underline{63.93} & \underline{59.47} & \underline{45.50} & \underline{72.85} & \underline{15.94} & 51.54 & 0.00 & 0.00\\
    % Vanilla FT\tablefootnote{Same as L2-SP~\citep{li_explicit_2018} under LoRA~\citep{hu_lora_2021}} & 92.57 & 70.65 & \underline{71.17} & \underline{54.75} & \underline{82.68} & 18.73 & 59.60 & 0.00 & 0.00\\
    Linear Prob. & 76.84 & 51.41 & 50.53 & 32.67 & 60.86 & 8.72 & 40.84 & \textcolor{red}{-10.68} & \textcolor{red}{-20.76} \\
    LP-FT & 79.76 & 56.12 & 55.77 & 37.10 & 66.03 & 12.82 & 45.57 & \textcolor{red}{-7.29} & \textcolor{red}{-11.58}\\
    % WiSE-FT & 2.24 & 3.00 & 2.30 & 4.59 & 2.11 & 0.96 & 2.59 \\
    FTP & 77.70 & 55.96 & 43.74 & 39.05 & 62.61 & 15.78 & 43.43 & \textcolor{red}{-9.68} & \textcolor{red}{-15.74}\\
    % Adam-SPD  & 92.72 & \textbf{73.91} & 72.97 & 57.11 & \textbf{83.80} & 20.67 & 61.69\\
    \midrule
    % DiGraP (0.5)  & \underline{85.49} & \textbf{64.30} & \textbf{59.93} & \textbf{45.53} & \underline{72.40} & \textbf{17.02} & \textbf{51.84} & \textcolor{green}{\textbf{}} & \textcolor{green}{\textbf{}}\\
    DiGraP (0.1)  & \underline{85.68} & \textbf{64.40} & \textbf{60.26} & \textbf{45.94} & \textbf{72.95} & \textbf{16.69} & \textbf{52.05} & \textcolor{red}{\textbf{-0.41}} & \textcolor{green}{\textbf{0.98}}\\
    % no upper bound
    % DiGraP (0.5)  & \textbf{92.83} & \textbf{73.30} & \textbf{73.60} & \textbf{57.71} & \textbf{83.64} & \textbf{22.92} & \textbf{62.23}\\
    \bottomrule
\end{tabular}
}
\caption{\textbf{DomainNet-oVQA Fine-Tuning Results.} LLaVA-7B~\citep{liu2023visualinstructiontuning} fine-tuned on DomainNet-oVQA (Real) and evaluated on other domains as a VQA task. We use LoRA for efficiency
%\zk{Need details on how you port the method to LORA-based finetuning, at least in supplementary}
. Note that L2-SP reduces to Vanilla FT with AdamW under LoRA. \emph{DiGraP} achieves SOTA results on both ID and OOD performance. \textbf{Bold}: best. \underline{Underline}: second best.
}
\label{tab:domainnet_ft_llava}
\end{table}


% \subsection{VQA LlaVA}
% \begin{table}[!h]
        \centering
        \resizebox{\linewidth}{!}{
        
        \begin{tabular}{@{}c|cccccccc|cccc|c@{}}
        \toprule
        \multirow{3}{*}{} &
          \multicolumn{1}{c|}{ID} &
          \multicolumn{7}{c|}{Near OOD} &
          \multicolumn{4}{c|}{Far OOD} &
          \multicolumn{1}{c}{} \\ \cmidrule(l){2-13} 
         &
          \multicolumn{1}{c|}{} &
          \multicolumn{2}{c|}{Vision} &
          \multicolumn{1}{c|}{Question} &
          \multicolumn{1}{c|}{Answer} &
          \multicolumn{1}{c|}{Multimodal} &
          \multicolumn{1}{c|}{Adversarial} &
           \multicolumn{1}{c|}{} &
           &
           &
           \multicolumn{1}{c|}{} &
           \\
         &
          \multicolumn{1}{c|}{VQAv2 (val)} &
          IV-VQA &
          \multicolumn{1}{c|}{CV-VQA} &
          \multicolumn{1}{c|}{VQA-Rep.} &
          \multicolumn{1}{c|}{VQA-CP v2} &
          \multicolumn{1}{c|}{VQA-CE} &
          \multicolumn{1}{c|}{AdVQA} & 
          Avg. &
          TextVQA &
          VizWiz &
          \multicolumn{1}{c|}{OK-VQA} &
          \multicolumn{1}{c|}{Avg.} &
          OOD Avg.\\ \midrule
        Zero-Shot &
          \multicolumn{1}{c|}{3.27} &
          6.34 &
          \multicolumn{1}{c|}{4.40} &
          \multicolumn{1}{c|}{2.92} &
          \multicolumn{1}{c|}{4.28} &
          \multicolumn{1}{c|}{1.46} &
          \multicolumn{1}{c|}{1.22} &
          3.44 &
          1.10 &
          0.24 &
          \multicolumn{1}{c|}{0.71} &
          \multicolumn{1}{c|}{0.68} &
          2.52
          \\ 
          
        Vanilla FT\tablefootnote{Same as L2-SP~\citep{li_explicit_2018} under LoRA~\citep{hu_lora_2021}} &
          \multicolumn{1}{c|}{\underline{72.49}} &
          \underline{82.23} &
          \multicolumn{1}{c|}{\textbf{58.61}} &
          \multicolumn{1}{c|}{\underline{63.93}} &
          \multicolumn{1}{c|}{\underline{69.80}} &
          \multicolumn{1}{c|}{\underline{45.60}} &
          \multicolumn{1}{c|}{\underline{40.22}} &
          \underline{60.07} &
          \underline{37.16} &
          12.11 &
          \multicolumn{1}{c|}{\underline{36.74}}  &
          \multicolumn{1}{c|}{\underline{28.67}} &
          \underline{49.60}
          \\ 

        % Linear Prob. &
        %   \multicolumn{1}{c|}{78.24} &
        %   87.83 &
        %   \multicolumn{1}{c|}{63.87} &
        %   \multicolumn{1}{c|}{69.61} &
        %   \multicolumn{1}{c|}{78.48} &
        %   \multicolumn{1}{c|}{61.66} &
        %   \multicolumn{1}{c|}{42.90} &
        %   67.39 &
        %   29.61 &
        %   18.80 &
        %   \multicolumn{1}{c|}{42.27}  &
        %   \multicolumn{1}{c|}{30.23} &
        %   55.00
        %   \\

        LP-FT &
          \multicolumn{1}{c|}{53.01}  &
          39.26 &
          \multicolumn{1}{c|}{38.54} &
          \multicolumn{1}{c|}{27.93} &
          \multicolumn{1}{c|}{33.14} &
          \multicolumn{1}{c|}{9.24} &
          \multicolumn{1}{c|}{23.66} &
          28.63 &
          7.80 &
          5.16 &
          \multicolumn{1}{c|}{9.95}  &
          \multicolumn{1}{c|}{7.64} &
          21.63
          \\

        WiSE-FT &
          \multicolumn{1}{c|}{60.47} &
          63.98 &
          \multicolumn{1}{c|}{46.39} &
          \multicolumn{1}{c|}{50.26} &
          \multicolumn{1}{c|}{55.79} &
          \multicolumn{1}{c|}{20.23} &
          \multicolumn{1}{c|}{23.35} &
          43.33 &
          10.10 &
          3.15 &
          \multicolumn{1}{c|}{13.97}  &
          \multicolumn{1}{c|}{9.07} &
          31.98
          \\

        FTP &
          \multicolumn{1}{c|}{67.95} &
          80.65 &
          \multicolumn{1}{c|}{57.33} &
          \multicolumn{1}{c|}{61.66} &
          \multicolumn{1}{c|}{68.05} &
          \multicolumn{1}{c|}{43.70} &
          \multicolumn{1}{c|}{39.53} &
          58.49 &
          33.88 &
          \textbf{12.98} &
          \multicolumn{1}{c|}{31.77}  &
          \multicolumn{1}{c|}{26.21} &
          47.73
          \\

        DiGraP (0.01)  &
          \multicolumn{1}{c|}{\textbf{72.54}} &
           \textbf{83.64} &
          \multicolumn{1}{c|}{\underline{56.56}} &
          \multicolumn{1}{c|}{\textbf{64.70}} &
          \multicolumn{1}{c|}{\textbf{69.71}} &
          \multicolumn{1}{c|}{\textbf{45.15}} &
          \multicolumn{1}{c|}{\textbf{43.10}} &
          \textbf{60.48} &
          \textbf{38.22} &
          \underline{12.97} &
          \multicolumn{1}{c|}{\textbf{37.43}}  &
          \multicolumn{1}{c|}{\textbf{29.54}} &
          \textbf{50.17}
          \\
        
        \bottomrule
        \end{tabular}}
          \caption{\textbf{Visual Question Answering Fine-Tuning Results using LLaVA-7B~\citep{liu2023visualinstructiontuning}.} We sample 10\% of the VQAv2 training and validation set. We fine-tune using LoRA with a rank of 4 and target on the $W_q, W_v$. \emph{DiGraP} outperforms baselines across ID and near OOD and is competitive on far OOD datasets using LoRA. Note that Vanilla FT with AdamW reduces to L2-SP under LoRA. Trainable projection strength \textbf{Bold}: best. \underline{Underline}: second best.}
          \label{tab:vqa_llava}
      \end{table}

\subsection{Additional Results}
\label{sec:more_results}

\noindent \textbf{1) More Backbones: CLIP, LLaVA.} We include experiments of fine-tuning on DomainNet with CLIP ViT-Base~\citep{radford_learning_2021} (Tab.~\ref{tab:domainnet_clipvit}), DomainNet-oVQA (Tab.~\ref{tab:domainnet_ft_llava}), and VQA (Tab.~\ref{tab:vqa_llava}) with LLaVA-7B~\citep{liu2023visualinstructiontuning}. \textit{DiGraP} consistently achieves the best performance across all experiments.

\begin{table}[!h]
        \centering
        \resizebox{\linewidth}{!}{
        
        \begin{tabular}{@{}c|cccccccc|cccc|c@{}}
        \toprule
        \multirow{3}{*}{} &
          \multicolumn{1}{c|}{ID} &
          \multicolumn{7}{c|}{Near OOD} &
          \multicolumn{4}{c|}{Far OOD} &
          \multicolumn{1}{c}{} \\ \cmidrule(l){2-13} 
         &
          \multicolumn{1}{c|}{} &
          \multicolumn{2}{c|}{Vision} &
          \multicolumn{1}{c|}{Question} &
          \multicolumn{1}{c|}{Answer} &
          \multicolumn{1}{c|}{Multimodal} &
          \multicolumn{1}{c|}{Adversarial} &
           \multicolumn{1}{c|}{} &
           &
           &
           \multicolumn{1}{c|}{} &
           \\
         &
          \multicolumn{1}{c|}{VQAv2 (val)} &
          IV-VQA &
          \multicolumn{1}{c|}{CV-VQA} &
          \multicolumn{1}{c|}{VQA-Rep.} &
          \multicolumn{1}{c|}{VQA-CP v2} &
          \multicolumn{1}{c|}{VQA-CE} &
          \multicolumn{1}{c|}{AdVQA} & 
          Avg. &
          TextVQA &
          VizWiz &
          \multicolumn{1}{c|}{OK-VQA} &
          \multicolumn{1}{c|}{Avg.} &
          OOD Avg.\\ \midrule
        Zero-Shot &
          \multicolumn{1}{c|}{3.27} &
          6.34 &
          \multicolumn{1}{c|}{4.40} &
          \multicolumn{1}{c|}{2.92} &
          \multicolumn{1}{c|}{4.28} &
          \multicolumn{1}{c|}{1.46} &
          \multicolumn{1}{c|}{1.22} &
          3.44 &
          1.10 &
          0.24 &
          \multicolumn{1}{c|}{0.71} &
          \multicolumn{1}{c|}{0.68} &
          2.52
          \\ 
          
        Vanilla FT\tablefootnote{Same as L2-SP~\citep{li_explicit_2018} under LoRA~\citep{hu_lora_2021}} &
          \multicolumn{1}{c|}{\underline{72.49}} &
          \underline{82.23} &
          \multicolumn{1}{c|}{\textbf{58.61}} &
          \multicolumn{1}{c|}{\underline{63.93}} &
          \multicolumn{1}{c|}{\underline{69.80}} &
          \multicolumn{1}{c|}{\underline{45.60}} &
          \multicolumn{1}{c|}{\underline{40.22}} &
          \underline{60.07} &
          \underline{37.16} &
          12.11 &
          \multicolumn{1}{c|}{\underline{36.74}}  &
          \multicolumn{1}{c|}{\underline{28.67}} &
          \underline{49.60}
          \\ 

        % Linear Prob. &
        %   \multicolumn{1}{c|}{78.24} &
        %   87.83 &
        %   \multicolumn{1}{c|}{63.87} &
        %   \multicolumn{1}{c|}{69.61} &
        %   \multicolumn{1}{c|}{78.48} &
        %   \multicolumn{1}{c|}{61.66} &
        %   \multicolumn{1}{c|}{42.90} &
        %   67.39 &
        %   29.61 &
        %   18.80 &
        %   \multicolumn{1}{c|}{42.27}  &
        %   \multicolumn{1}{c|}{30.23} &
        %   55.00
        %   \\

        LP-FT &
          \multicolumn{1}{c|}{53.01}  &
          39.26 &
          \multicolumn{1}{c|}{38.54} &
          \multicolumn{1}{c|}{27.93} &
          \multicolumn{1}{c|}{33.14} &
          \multicolumn{1}{c|}{9.24} &
          \multicolumn{1}{c|}{23.66} &
          28.63 &
          7.80 &
          5.16 &
          \multicolumn{1}{c|}{9.95}  &
          \multicolumn{1}{c|}{7.64} &
          21.63
          \\

        WiSE-FT &
          \multicolumn{1}{c|}{60.47} &
          63.98 &
          \multicolumn{1}{c|}{46.39} &
          \multicolumn{1}{c|}{50.26} &
          \multicolumn{1}{c|}{55.79} &
          \multicolumn{1}{c|}{20.23} &
          \multicolumn{1}{c|}{23.35} &
          43.33 &
          10.10 &
          3.15 &
          \multicolumn{1}{c|}{13.97}  &
          \multicolumn{1}{c|}{9.07} &
          31.98
          \\

        FTP &
          \multicolumn{1}{c|}{67.95} &
          80.65 &
          \multicolumn{1}{c|}{57.33} &
          \multicolumn{1}{c|}{61.66} &
          \multicolumn{1}{c|}{68.05} &
          \multicolumn{1}{c|}{43.70} &
          \multicolumn{1}{c|}{39.53} &
          58.49 &
          33.88 &
          \textbf{12.98} &
          \multicolumn{1}{c|}{31.77}  &
          \multicolumn{1}{c|}{26.21} &
          47.73
          \\

        DiGraP (0.01)  &
          \multicolumn{1}{c|}{\textbf{72.54}} &
           \textbf{83.64} &
          \multicolumn{1}{c|}{\underline{56.56}} &
          \multicolumn{1}{c|}{\textbf{64.70}} &
          \multicolumn{1}{c|}{\textbf{69.71}} &
          \multicolumn{1}{c|}{\textbf{45.15}} &
          \multicolumn{1}{c|}{\textbf{43.10}} &
          \textbf{60.48} &
          \textbf{38.22} &
          \underline{12.97} &
          \multicolumn{1}{c|}{\textbf{37.43}}  &
          \multicolumn{1}{c|}{\textbf{29.54}} &
          \textbf{50.17}
          \\
        
        \bottomrule
        \end{tabular}}
          \caption{\textbf{Visual Question Answering Fine-Tuning Results using LLaVA-7B~\citep{liu2023visualinstructiontuning}.} We sample 10\% of the VQAv2 training and validation set. We fine-tune using LoRA with a rank of 4 and target on the $W_q, W_v$. \emph{DiGraP} outperforms baselines across ID and near OOD and is competitive on far OOD datasets using LoRA. Note that Vanilla FT with AdamW reduces to L2-SP under LoRA. Trainable projection strength \textbf{Bold}: best. \underline{Underline}: second best.}
          \label{tab:vqa_llava}
      \end{table}
\begin{table}[!h]
\centering
\resizebox{1\linewidth}{!}{ % Adjust the table to fit the page width
\begin{tabular}{c|c|ccccc|ccc}
    \toprule
    &\multicolumn{1}{c|}{ID} & \multicolumn{5}{c|}{OOD} & \multicolumn{3}{c}{Statistics} \\
    & Real & Sketch & Painting & Infograph & Clipart & Quickdraw & OOD Avg. & ID $\Delta$ ($\%$) & OOD $\Delta$ ($\%$)\\
    \midrule
    Zero-Shot & 67.71 & 53.45 & 56.15 & 37.90 & 58.32 & 15.87 & 44.34 & - & - \\
    Vanilla FT\tablefootnote{Same as L2-SP~\citep{li_explicit_2018} under LoRA~\citep{hu_lora_2021}} & \textbf{86.03} & \underline{63.93} & \underline{59.47} & \underline{45.50} & \underline{72.85} & \underline{15.94} & 51.54 & 0.00 & 0.00\\
    % Vanilla FT\tablefootnote{Same as L2-SP~\citep{li_explicit_2018} under LoRA~\citep{hu_lora_2021}} & 92.57 & 70.65 & \underline{71.17} & \underline{54.75} & \underline{82.68} & 18.73 & 59.60 & 0.00 & 0.00\\
    Linear Prob. & 76.84 & 51.41 & 50.53 & 32.67 & 60.86 & 8.72 & 40.84 & \textcolor{red}{-10.68} & \textcolor{red}{-20.76} \\
    LP-FT & 79.76 & 56.12 & 55.77 & 37.10 & 66.03 & 12.82 & 45.57 & \textcolor{red}{-7.29} & \textcolor{red}{-11.58}\\
    % WiSE-FT & 2.24 & 3.00 & 2.30 & 4.59 & 2.11 & 0.96 & 2.59 \\
    FTP & 77.70 & 55.96 & 43.74 & 39.05 & 62.61 & 15.78 & 43.43 & \textcolor{red}{-9.68} & \textcolor{red}{-15.74}\\
    % Adam-SPD  & 92.72 & \textbf{73.91} & 72.97 & 57.11 & \textbf{83.80} & 20.67 & 61.69\\
    \midrule
    % DiGraP (0.5)  & \underline{85.49} & \textbf{64.30} & \textbf{59.93} & \textbf{45.53} & \underline{72.40} & \textbf{17.02} & \textbf{51.84} & \textcolor{green}{\textbf{}} & \textcolor{green}{\textbf{}}\\
    DiGraP (0.1)  & \underline{85.68} & \textbf{64.40} & \textbf{60.26} & \textbf{45.94} & \textbf{72.95} & \textbf{16.69} & \textbf{52.05} & \textcolor{red}{\textbf{-0.41}} & \textcolor{green}{\textbf{0.98}}\\
    % no upper bound
    % DiGraP (0.5)  & \textbf{92.83} & \textbf{73.30} & \textbf{73.60} & \textbf{57.71} & \textbf{83.64} & \textbf{22.92} & \textbf{62.23}\\
    \bottomrule
\end{tabular}
}
\caption{\textbf{DomainNet-oVQA Fine-Tuning Results.} LLaVA-7B~\citep{liu2023visualinstructiontuning} fine-tuned on DomainNet-oVQA (Real) and evaluated on other domains as a VQA task. We use LoRA for efficiency
%\zk{Need details on how you port the method to LORA-based finetuning, at least in supplementary}
. Note that L2-SP reduces to Vanilla FT with AdamW under LoRA. \emph{DiGraP} achieves SOTA results on both ID and OOD performance. \textbf{Bold}: best. \underline{Underline}: second best.
}
\label{tab:domainnet_ft_llava}
\end{table}

\begin{table}[!h]
    \centering
    \resizebox{1\linewidth}{!}{ % 
    \begin{tabular}{c|c|cccc|ccc}
    \toprule
         & ID & \multicolumn{4} {c|} {OOD} & \multicolumn{3} {c} {Statistics} \\
        Method & Real & Sketch & Painting & Infograph & Clipart & OOD Avg. & ID $\Delta$ (\%) &  OOD $\Delta$ (\%) \\
    \midrule
        % Zero Shot & 79.4 & 50.3 & 62.4 & 37.1 & 59.8 & 52.4 & 0.00 & 0.00\\
        Vanilla FT & \underline{86.5} & 45.9 & \textbf{58.9} & \underline{34.8} & 59.7 & 49.82 & 0 & 0 \\
        LP & 83.2 & 38.6 & 51.4 & 27.2 & 50.7 & 41.97 & \textcolor{red}{-3.82} &  \textcolor{red}{-15.76} \\
        L2-SP & 86.4 & 46.0 & \textbf{58.9} & \textbf{35.0} & 59.8 & \underline{49.92} & \textcolor{red}{\underline{-0.12}} & \textcolor{green}{\underline{0.20}} \\
        LP-FT & 84.4 & 36.5 & 50.5 & 27.2 & 51.4 & 41.40 & \textcolor{red}{-2.43} & \textcolor{red}{-16.90}  \\
        FTP & \textbf{86.7} & \underline{49.1} & 55.8 & 32.2 & 61.4 & 49.63 & \textcolor{green}{\textbf{0.23}} & \textcolor{red}{-0.38} \\
        DiGraP (1) & 86.2 & \textbf{51.3} & \underline{57.3} & {34.0} & \textbf{62.4} & \textbf{51.25} & \textcolor{red}{-0.35} & \textcolor{green}{\textbf{2.87}} \\
    \bottomrule
    \end{tabular}}
    \caption{\textbf{OOD Results on DomainNet}. CLIP ViT-Base finetuned on DomainNet (Real) dataset and evaluated on other domains. \textbf{Bold}: best. \underline{Underline}: second best.}
    \label{tab:domainnet_clipvit}
\end{table}

\noindent \textbf{2) Language-only Experiments.} We also conduct a language-only experiment in Tab.~\ref{tab:sa_comparison}, where we use BOSS~\citep{yuan2023revisitingoutofdistributionrobustnessnlp}, an NLP benchmark suite designed for OOD robustness evaluation. This suite includes both ID and OOD language-only datasets across multiple tasks (e.g., Sentiment Analysis, Toxic Detection). 

\begin{table}[ht]
\centering
\begin{tabular}{c|c|ccc|ccc}
\toprule
% {Task} & \multicolumn{4}{c}{{SA}} \\ \midrule
\multirow{2}{*}{{Dataset}} & \multicolumn{1}{c|}{ID} & \multicolumn{3}{c|}{{OOD}} & \multicolumn{3}{c} {Statistics} \\
 & \multicolumn{1}{c|}{AZ} & {DS} & {SE} & {SST} & OOD Avg. & ID $\Delta$ (\%) & OOD $\Delta$ (\%) \\ \midrule
Vanilla FT & \underline{85.57} & 43.63 & 48.47 & \underline{67.29} & \underline{53.13} & \underline{0.00} & \underline{0.00} \\
L2-SP & 84.87 & {43.40} & 48.65 & 65.07 & {52.37} & \textcolor{red}{-0.82} & \textcolor{red}{-1.43} \\
FTP & 84.73 & \underline{43.84} & \underline{48.69} & 66.73 & 53.09 & \textcolor{red}{-0.98} & \textcolor{red}{-0.08} \\
DiGraP (8e-3) & \textbf{85.60} & \textbf{44.10} & \textbf{49.07} & \textbf{67.39} & \textbf{53.52} & \textcolor{green}{\textbf{0.04}} & \textcolor{green}{\textbf{0.73}} \\ 
\bottomrule
\end{tabular}
\caption{\textbf{BOSS~\citep{yuan2023revisitingoutofdistributionrobustnessnlp} Performance Using T5~\citep{raffel2023exploringlimitstransferlearning} Fine-Tuned with AZ.} We leverage BOSS, an
NLP benchmark suite for OOD
robustness evaluation. We focus on the Sentiment Analysis task which contains Amazon (AZ) as the ID and DynaSent (DS), SemEval and SST as the OOD. \textbf{Bold}: best. \underline{Underline}: second best.}
\label{tab:sa_comparison}
\end{table}


\noindent \textbf{3) LP-FT Controlled Version.} In Tab.~\ref{tab:domainnet_moco_appendix}, we conduct an LP-FT-C experiment (a controlled version of LP-FT), similar to TPGM-C in~\citet{tian_trainable_2023}, by increasing the regularization strength to ensure that the ID performance matches that of DiGraP. Despite this adjustment, \textit{DiGraP} still outperforms LP-FT-C on OOD datasets.

\begin{table}[h]
\caption{\textbf{DomainNet Results using MOCO-V3 
%\zk{Are you planning to add CLIP, etc.? Since TPGM/FTP had some of these, it would be good to add at least in supplementary or if that's not a later deadline be prepared for rebuttal}
 pre-trained ResNet50 with Real Data.} \emph{DiGraP} outperforms baselines on average OOD. \textbf{Bold}: best. \underline{Underline}: second best.
}
\centering

\resizebox{0.85\linewidth}{!}{
\begin{tabular}{c|c|cccc|ccc} 

\toprule
&\multicolumn{1}{c|}{ID} & \multicolumn{4}{c|}{OOD} & \multicolumn{3}{c}{Statistics} \\
&Real & Sketch & Painting & Infograph & Clipart & OOD Avg. & ID $\Delta$ ($\%$) & OOD $\Delta$ ($\%$)\\
\midrule
Vanilla FT & 81.99 &	31.52 &	42.89 &	18.51 &	44.98 &	34.47&	0.00&	0.00 \\
Linear Prob. & 73.01 &	24.10 &	39.56 &	12.27 &	30.38 &	26.58&	\color{red}-10.96&	\color{red}-22.90\\
Partial Fusion & 78.27 &	27.72 &	39.74 &	15.56 &	38.18 &	30.30&	\color{red}-4.55&	\color{red}-12.11 
\\
L2-SP & 81.51 &	34.91 &	45.76 &	18.97 &	45.29 &	36.23&	\color{red}-0.59&	\color{green}5.09 \\
MARS-SP & 81.89  &	34.44  &	45.05  &	19.97  &	46.36  &	36.45& \color{red}-0.13&	\color{green}5.74 \\
LP-FT & \textbf{82.92} & 34.50 &	45.42 &	20.12 &	47.11 &	36.79&	\color{green}\textbf{1.13}&	\color{green}6.72\\
TPGM & \underline{82.66} & 35.35 &	46.20 &	20.13 &	45.75 &	36.86&	\color{green}\underline{0.82}&	\color{green}6.91\\
FTP & 82.17	&\underline{36.26}	&\underline{46.58}	&\underline{20.67} &46.97	& \underline{37.62}	&\color{green}0.22	&\color{green}\underline{9.13}\\
\midrule
LP-FT-C (reg=5) & 82.18 & 35.13 &	46.02 &	20.36 &	\textbf{47.61} & 37.28 &	\color{green}{0.23} &	\color{green}{8.15}\\
DiGraP (0.1) & 82.20 & \textbf{36.43} &	\textbf{46.75} &	\textbf{21.40} &	\underline{47.46} & \textbf{38.01} &	\color{green}0.26&	\color{green}\textbf{10.27} \\
\bottomrule				
\end{tabular}
}
\label{tab:domainnet_moco_appendix}
\end{table}

\noindent \textbf{4) Sensitivity Analysis on Single-Modal Tasks.} In Tab.~\ref{tab:hyper_img}, we include the hyper-parameter tuning experiment for DomainNet on CLIP ViT-Base. The results demonstrate that \textit{DiGraP} remains robust to hyperparameter variations on both ID and OOD datasets in the uni-modal image classification task.

\begin{table}[!h]

\begin{subtable}[t]{1\textwidth}
\centering
% \resizebox{\linewidth}{!}{
\begin{tabular}[h]{c|ccccc}
\toprule
Hyper-Parameter $\mu$ & 0.01 & 0.1 & 0.5 & 1 & 10\\
\midrule
OOD Avg. & 50.86 & 51.04 & 50.79 & 51.25 & 51.14\\
\midrule
ID & 86.00 & 86.12 & 86.13 & 86.14 & 86.12\\
\bottomrule
\end{tabular}
% }
\caption{DomainNet hyper-parameter ($\mu$) sweep.}
\label{tab:dom_hyper}
\end{subtable}

\caption{\textbf{Sensitivity Analysis of Hyper-Parameter $\mu$ on ID and OOD performance.} We sweep $\mu \in \{0.01, 0.1, 0.5, 1, 10\}$. For DomainNet with CLIP ViT-Base experiments, both ID and average OOD performance fluctuates slightly and are robust to the change of $\mu$ over a wide range.}
\label{tab:hyper_img}
\end{table}

 % Stronger regularizations (larger values) decrease deviation, simultaneously improving OOD performance. The ID performance is not impacted significantly.

\noindent \textbf{5) Visualization of the Variation in Regularization Strength across Layers over Epochs.} We present a visualization of the variation in regularization strength ($\lambda$) across different layers over epochs in Fig.~\ref{fig:layer_epoch_reg_strength}. The results show that the regularization strength evolves dynamically during training, starting small, increasing over iterations, and eventually converging. In the vision layers (blue), early layers tend to experience weaker regularization compared to later layers throughout the training process. Conversely, the language layers (orange) display a more uniform regularization strength, with comparable levels observed between early and later layers. The weaker regularization in early vision layers likely allows them to preserve foundational low-level features, while stronger regularization in later layers encourages the model to focus on high-level semantic representations.

% \begin{figure}[!h]
%     \centering
%     % First image on the left
%     \begin{subfigure}{0.33\linewidth}
%         \centering
%         \includegraphics[width=\linewidth]{figures/3d_plot.png}
%         \caption{TextVQA, PT}
%     \end{subfigure}%
%     \begin{subfigure}{0.33\linewidth}
%         \centering
%         \includegraphics[width=\linewidth]{figures/3d_plot_layers-2.png}
%         \caption{TextVQA, FT}
%     \end{subfigure}%
%     \begin{subfigure}{0.33\linewidth}
%         \centering
%         \includegraphics[width=\linewidth]{figures/3d_plot_iteration-2.png}
%         \caption{TextVQA, LP}
%     \end{subfigure}%
    
%     \caption{Variation of the Average Projection Strength $\omega$ of all Layers over Iterations. We present the results of $\mu \in \{0.01, 0.1, 0.5, 1, 100\}$. We use the sliding window with window sizes of 50 and 200 to visualize the results for DomainNet-oVQA (Real) and VQAv2. The projection strength $\omega$ is dynamic over iterations, growing from small to large and converging in the end.}
%     \label{fig:variation}
% \end{figure}

\begin{figure}[!h]
    \centering
    \begin{subfigure}{0.6\textwidth}
        \centering
        \includegraphics[width=\textwidth]{figures/3d_plot.png}
        \caption{3D Overview of the Variation of Regularization Strength across Layers and Epochs.}
        \label{fig:3d}
    \end{subfigure}%
    \begin{subfigure}{0.4\textwidth}
        \centering
        \includegraphics[width=\textwidth]{figures/3d_plot_layers-2.png}
        \caption{Layer-Wise Regularization Strength}
        \label{fig:layer}
        \vspace{0.5cm} % Adjust space between the two figures on the right
        \includegraphics[width=\textwidth]{figures/3d_plot_iteration-2.png}
        \caption{Dynamic Regularization Across Epochs}
        \label{fig:epoch}
    \end{subfigure}
    \caption{\textbf{Visualization of the Variation in Regularization Strength ($\lambda$) across Layers over Epochs.} Fig.~\ref{fig:3d} is an 3D view of R=regularization strength dynamics across layers and epochs. The X-axis represents training epochs, the Y-axis represents model layers (vision in blue, language in orange), and the Z-axis represents the regularization strength $\lambda$ applied to each layer. Fig.~\ref{fig:layer} projects the data onto the plane formed by layers (Y) and regularization strength (Z). Fig.~\ref{fig:epoch} projects the data onto the plane formed by epochs (X) and regularization strength (Z).}
    \label{fig:layer_epoch_reg_strength}
\end{figure}

\subsection{Training Details}
\label{sec:training}

\paragraph{Image Classification.} For \emph{DiGraP}, we fine-tune the model using SGD with a learning rate of $1e-2$ and $\mu=0.1$ with a batchsize of 256. The regularization hyper-parameter is found through cross-validation, and the model with the best ID validation accuracy is taken. We use 4 RTX 2080 GPUs for each experiment. \zkn{Might be better to have this before sec. 4.1}

\paragraph{Fine-Tuning DomainNet-oVQA.} We use the model pretrained with $224*224$ input images and $128$ token input/output text sequences and fine-tune with the precision of bfloat16. We use the LAVIS~\citep{li2022lavislibrarylanguagevisionintelligence} public repository to fine-tune all methods. Standard hyper-parameters are used for all: learning rate ($1e-3$), weight-decay ($1e-4$), optimizer (AdamW), scheduler (Linear Warmup With Cosine Annealing), warm-up learning rate ($1e-4$), minimum learning rate ($1e-4$), accumulation steps ($2$), beam size (5). The model is trained for $10$ epochs with a batch size of $128$ for Tab.~\ref{tab:domainnet_ft}. For LoRA~\citep{hu_lora_2021}, we limit our study to only adapting the attention weights and freeze the MLP modules for parameter-efficiency, specifically apply LoRA to $W_q, W_k, W_v, W_o$ with $r=8$ in Tab.~\ref{tab:domainnet_ft}. We use $\lambda=0.5$ for all \emph{DiGraP} results in Tab.~\ref{tab:domainnet_ft}. The regularization hyper-parameter is found through cross-validation, and the model with the best ID validation accuracy is taken. We use 8 A40 GPU for each experiment.

\paragraph{Fine-tuning VQA.} We use the model pretrained with $224*224$ input images and $128$ token input/output text sequences and fine-tune with the precision of bfloat16. We use the LAVIS~\citep{li2022lavislibrarylanguagevisionintelligence} public repository to fine-tune all methods. Standard hyper-parameters are used for all: learning rate ($1e-3$), weight-decay ($1e-4$), optimizer (AdamW), scheduler (Linear Warmup With Cosine Annealing), warm-up learning rate ($1e-4$), minimum learning rate ($1e-4$), accumulation steps ($2$), beam size (5). The model is trained for $10$ epochs with a batch size of $128$ for Tab.~\ref{tab:main_result}. For LoRA~\citep{hu_lora_2021}, we limit our study to only adapting the attention weights and freeze the MLP modules for parameter-efficiency, specifically apply LoRA to $W_q, W_k, W_v, W_o$ with $r=8$ in Tab.~\ref{tab:main_result}. We use $\lambda=0.5$ for all \emph{DiGraP} results in Tab.~\ref{tab:main_result}. The regularization hyper-parameter is found through cross-validation, and the model with the best ID validation accuracy is taken. We use 8 A40 GPU for each experiment.

\subsection{Measuring OOD Distance}
\label{sec:ood_dist}

We follow procedures similar to typical feature-based OOD detection methods \citep{Shi_2024_WACV}\zkn{cite}. Specifically, given our input training split \( X_{\text{in}}^{\text{train}} \), we compute feature representations \( z \) of the training samples to estimate the empirical mean \( \mu \) and covariance matrix \( \Sigma \). For each test split, we compute the test set shift relative to the training domain using the Mahalanobis distance metric defined in Eq.~\ref{eq:maha_distance}. The overall shift score for each test dataset, denoted as \( S_{\text{maha}} \), is calculated as the average 
\( S_{\text{Maha}} \) across all samples. Let \( q \) denote the question, \( v \) the image (vision input), and \( a \) the answer. The input features used in measuring shifts include uni-modal embeddings \( f(v) \), \( f(q) \) and joint embeddings \( f(q,v) \).

\begin{equation}
S_{\text{Maha}}(z_{\text{test}}) = \sqrt{(z_{\text{test}} - \mu)^\top \Sigma^{-1} (z_{\text{test}} - \mu)}
\label{eq:maha_distance}
\end{equation}

We utilize the vanilla fine-tuned PaliGemma model on the VQAv2 training dataset as our feature encoder. For the image embedding \( f(v) \), we obtain it via masking out the question input tokens and mean-pooling the image portion from the final layer of the model before the language model head. Similarly, to get \( f(q) \), we mask out the image tokens and extract the question portion from the final layer. To obtain \(f(v,q)\), we pass in both image and text tokens as input, compute the average embedding for both modalities and then taking the overall mean. 

\subsubsection{Correlation between Uni- \& Multi-Modal Shifts per Dataset}

Fig.~\ref{fig:crosscorrheatmap} shows the heatmap of the correlation between uni-modal and multi-modal shifts per dataset. Question-joint shift correlations are higher than image-joint shift correlations across all VQA datasets and fine-tuning methods. However, pre-train model maintains similar correlation between both modalities. Vanilla FT and SPD exhibits the lowest question-joint shift correlation shown by the darkest row color across all fine-tuning methods in \ref{fig:ques_ft_correlation_heatmap}. Whilst, SPD shows the lowest image-joint shift correlation across the datasets in \ref{fig:img_final_correlation_heatmap}.

\begin{figure}[!h]
    \centering
    \begin{subfigure}[t]{0.8\linewidth}
        \centering
        \includegraphics[width=\linewidth]{figures/corr_heatmap/ques_ft_correlation_heatmap.png}
        \caption{Question-Joint shift correlation heatmap}
        \label{fig:ques_ft_correlation_heatmap}
    \end{subfigure}
    % \hfill
    \begin{subfigure}[t]{0.8\linewidth}
        \centering
        \includegraphics[width=\linewidth]{figures/corr_heatmap/img_final_correlation_heatmap.png}
        \caption{Image-Joint shift correlation heatmap}
        \label{fig:img_final_correlation_heatmap}
    \end{subfigure}
    \caption{Heatmap of correlation between uni-modal and multi-modal shifts per dataset.}
    \label{fig:crosscorrheatmap}
\end{figure}


% \clearpage

\subsubsection{Histograms for Evaluating Different Distribution Shifts}



\begin{figure*}[!h]
    \centering
    
    % Subfigure for VQAv2 val
    \begin{subfigure}[b]{0.3\linewidth}
        \centering
        \includegraphics[width=\linewidth]{figures/histograms/visual/lora_img_final_coco_vqa_raw_val.png}
        \caption{VQAv2 Val}
        \label{fig:VQAv2_val}
    \end{subfigure}
    \hfill
    % Subfigure for IV VQA
    \begin{subfigure}[b]{0.3\linewidth}
        \centering
        \includegraphics[width=\linewidth]{figures/histograms/visual/lora_img_final_coco_iv-vqa_val.png}
        \caption{IV VQA}
        \label{fig:ivvqa}
    \end{subfigure}
    \hfill
    % Subfigure for CV VQA
    \begin{subfigure}[b]{0.3\linewidth}
        \centering
        \includegraphics[width=\linewidth]{figures/histograms/visual/lora_img_final_coco_cv-vqa_val.png}
        \caption{CV VQA}
        \label{fig:vcvvqa}
    \end{subfigure}

    \vspace{0.5cm} % Space between rows

    % Subfigure for VQA Rephrasings
    \begin{subfigure}[b]{0.3\linewidth}
        \centering
        \includegraphics[width=\linewidth]{figures/histograms/visual/lora_img_final_coco_vqa_rephrasings_val.png}
        \caption{VQA Rephrasings}
        \label{fig:vvqa_rephrasings}
    \end{subfigure}
    \hfill
    % Subfigure for VQA CP v2
    \begin{subfigure}[b]{0.3\linewidth}
        \centering
        \includegraphics[width=\linewidth]{figures/histograms/visual/lora_img_final_coco_vqa_cp_val.png}
        \caption{VQA CP v2}
        \label{fig:vvqa_cp_v2}
    \end{subfigure}
    \hfill
    % Subfigure for VQA CE
    \begin{subfigure}[b]{0.3\linewidth}
        \centering
        \includegraphics[width=\linewidth]{figures/histograms/visual/lora_img_final_coco_vqa_ce_val.png}
        \caption{VQA CE}
        \label{fig:vvqa_ce}
    \end{subfigure}

    \vspace{0.5cm} % Space between rows

    % Subfigure for ADVQA
    \begin{subfigure}[b]{0.3\linewidth}
        \centering
        \includegraphics[width=\linewidth]{figures/histograms/visual/lora_img_final_coco_advqa_val.png}
        \caption{ADVQA}
        \label{fig:vadvqa}
    \end{subfigure}
    \hfill
    % Subfigure for Text VQA
    \begin{subfigure}[b]{0.3\linewidth}
        \centering
        \includegraphics[width=\linewidth]{figures/histograms/visual/lora_img_final_textvqa_val.png}
        \caption{Text VQA}
        \label{fig:vtextvqa}
    \end{subfigure}
    \hfill
    % Subfigure for VizWiz
    \begin{subfigure}[b]{0.3\linewidth}
        \centering
        \includegraphics[width=\linewidth]{figures/histograms/visual/lora_img_final_vizwiz_val.png}
        \caption{VizWiz}
        \label{fig:vvizwiz}
    \end{subfigure}

    \vspace{0.5cm} % Space between rows

    % Subfigure for OK VQA
    \begin{subfigure}[b]{0.3\linewidth}
        \centering
        \includegraphics[width=\linewidth]{figures/histograms/visual/lora_img_final_coco_okvqa_val.png}
        \caption{OK VQA}
        \label{fig:vokvqa}
    \end{subfigure}
    \caption{Histogram for Vanilla FT Visual Shifts: We depict the \( S_{\text{Maha}} \) score on the visual modality for each sample in the VQAv2 train split in blue and the corresponding test samples in orange. There's minimal visual shifts for all VQA datasets from the VQAv2 train, except for Figure \subref{fig:vvizwiz} which shows evidence of greater shifts between the orange distribution and the blue distribution. }
    \label{fig:v_histograms}
\end{figure*}



\begin{figure*}[!h]
    \centering

    % Subfigure for VQAv2 val
    \begin{subfigure}[b]{0.3\linewidth}
        \centering
        \includegraphics[width=\linewidth]{figures/histograms/question/lora_ques_ft_coco_vqa_raw_val.png}
        \caption{VQAv2 Val}
        \label{fig:vqvqav2_val}
    \end{subfigure}
    \hfill
    % Subfigure for IV VQA
    \begin{subfigure}[b]{0.3\linewidth}
        \centering
        \includegraphics[width=\linewidth]{figures/histograms/question/lora_ques_ft_coco_iv-vqa_val.png}
        \caption{IV VQA}
        \label{fig:vqivvqa}
    \end{subfigure}
    \hfill
    % Subfigure for CV VQA
    \begin{subfigure}[b]{0.3\linewidth}
        \centering
        \includegraphics[width=\linewidth]{figures/histograms/question/lora_ques_ft_coco_cv-vqa_val.png}
        \caption{CV VQA}
        \label{fig:vqcvvqa}
    \end{subfigure}

    \vspace{0.5cm} % Space between rows

    % Subfigure for VQA Rephrasings
    \begin{subfigure}[b]{0.3\linewidth}
        \centering
        \includegraphics[width=\linewidth]{figures/histograms/question/lora_ques_ft_coco_vqa_rephrasings_val.png}
        \caption{VQA Rephrasings}
        \label{fig:vqvqa_rephrasings}
    \end{subfigure}
    \hfill
    % Subfigure for VQA CP v2
    \begin{subfigure}[b]{0.3\linewidth}
        \centering
        \includegraphics[width=\linewidth]{figures/histograms/question/lora_ques_ft_coco_vqa_cp_val.png}
        \caption{VQA CP v2}
        \label{fig:vqvqa_cp_v2}
    \end{subfigure}
    \hfill
    % Subfigure for VQA CE
    \begin{subfigure}[b]{0.3\linewidth}
        \centering
        \includegraphics[width=\linewidth]{figures/histograms/question/lora_ques_ft_coco_vqa_ce_val.png}
        \caption{VQA CE}
        \label{fig:vqvqa_ce}
    \end{subfigure}

    \vspace{0.5cm} % Space between rows

    % Subfigure for ADVQA
    \begin{subfigure}[b]{0.3\linewidth}
        \centering
        \includegraphics[width=\linewidth]{figures/histograms/question/lora_ques_ft_coco_advqa_val.png}
        \caption{ADVQA}
        \label{fig:vqadvqa}
    \end{subfigure}
    \hfill
    % Subfigure for Text VQA
    \begin{subfigure}[b]{0.3\linewidth}
        \centering
        \includegraphics[width=\linewidth]{figures/histograms/question/lora_ques_ft_textvqa_val.png}
        \caption{Text VQA}
        \label{fig:vqtextvqa}
    \end{subfigure}
    \hfill
    % Subfigure for VizWiz
    \begin{subfigure}[b]{0.3\linewidth}
        \centering
        \includegraphics[width=\linewidth]{figures/histograms/question/lora_ques_ft_vizwiz_val.png}
        \caption{VizWiz}
        \label{fig:vqvizwiz}
    \end{subfigure}

    \vspace{0.5cm} % Space between rows

    % Subfigure for OK VQA
    \begin{subfigure}[b]{0.3\linewidth}
        \centering
        \includegraphics[width=\linewidth]{figures/histograms/question/lora_ques_ft_coco_okvqa_val.png}
        \caption{OK VQA}
        \label{fig:vqokvqa}
    \end{subfigure}

    \caption{Histogram for Vanilla FT Question Shifts: We depict the \( S_{\text{Maha}} \) score on the question modality for each sample in the VQAv2 train split in blue and the corresponding test samples in orange. Similar to the visual shift histograms, far OODs (Figures \subref{fig:vqtextvqa}, \subref{fig:vqvizwiz}, \subref{fig:vqokvqa}) also show evidence of greater shifts between the orange distribution and the blue distribution than near OODs.}
    \label{fig:vq_histograms}
\end{figure*}


\begin{figure*}[!h]
    \centering

    % Subfigure for VQAv2 val
    \begin{subfigure}[b]{0.3\linewidth}
        \centering
        \includegraphics[width=\linewidth]{figures/histograms/joint/lora_joint_coco_vqa_raw_val.png}
        \caption{VQAv2 Val}
        \label{fig:vqaVQAv2_val}
    \end{subfigure}
    \hfill
    % Subfigure for IV VQA
    \begin{subfigure}[b]{0.3\linewidth}
        \centering
        \includegraphics[width=\linewidth]{figures/histograms/joint/lora_joint_coco_iv-vqa_val.png}
        \caption{IV VQA}
        \label{fig:vqaivvqa}
    \end{subfigure}
    \hfill
    % Subfigure for CV VQA
    \begin{subfigure}[b]{0.3\linewidth}
        \centering
        \includegraphics[width=\linewidth]{figures/histograms/joint/lora_joint_coco_cv-vqa_val.png}
        \caption{CV VQA}
        \label{fig:vqacvvqa}
    \end{subfigure}

    \vspace{0.5cm} % Space between rows

    % Subfigure for VQA Rephrasings
    \begin{subfigure}[b]{0.3\linewidth}
        \centering
        \includegraphics[width=\linewidth]{figures/histograms/joint/lora_joint_coco_vqa_rephrasings_val.png}
        \caption{VQA Rephrasings}
        \label{fig:vqavqa_rephrasings}
    \end{subfigure}
    \hfill
    % Subfigure for VQA CP v2
    \begin{subfigure}[b]{0.3\linewidth}
        \centering
        \includegraphics[width=\linewidth]{figures/histograms/joint/lora_joint_coco_vqa_cp_val.png}
        \caption{VQA CP v2}
        \label{fig:vqavqa_cp_v2}
    \end{subfigure}
    \hfill
    % Subfigure for VQA CE
    \begin{subfigure}[b]{0.3\linewidth}
        \centering
        \includegraphics[width=\linewidth]{figures/histograms/joint/lora_joint_coco_vqa_ce_val.png}
        \caption{VQA CE}
        \label{fig:vqavqa_ce}
    \end{subfigure}

    \vspace{0.5cm} % Space between rows

    % Subfigure for ADVQA
    \begin{subfigure}[b]{0.3\linewidth}
        \centering
        \includegraphics[width=\linewidth]{figures/histograms/joint/lora_joint_coco_advqa_val.png}
        \caption{ADVQA}
        \label{fig:vqaadvqa}
    \end{subfigure}
    \hfill
    % Subfigure for Text VQA
    \begin{subfigure}[b]{0.3\linewidth}
        \centering
        \includegraphics[width=\linewidth]{figures/histograms/joint/lora_joint_textvqa_val.png}
        \caption{Text VQA}
        \label{fig:vqatextvqa}
    \end{subfigure}
    \hfill
    % Subfigure for VizWiz
    \begin{subfigure}[b]{0.3\linewidth}
        \centering
        \includegraphics[width=\linewidth]{figures/histograms/joint/lora_joint_vizwiz_val.png}
        \caption{VizWiz}
        \label{fig:vqavizwiz}
    \end{subfigure}

    \vspace{0.5cm} % Space between rows

    % Subfigure for OK VQA
    \begin{subfigure}[b]{0.3\linewidth}
        \centering
        \includegraphics[width=\linewidth]{figures/histograms/joint/lora_joint_coco_okvqa_val.png}
        \caption{OKVQA}
        \label{fig:vqaokvqa}
    \end{subfigure}

    \caption{Histogram for Vanilla FT V+Q Shifts : We depict the \( S_{\text{Maha}} \) score on the V+Q shift for each sample in the VQAv2 train split in blue and the corresponding test samples in orange. For all test splits, V+Q shifts show a greater degree of shift compared to the corresponding visual and question shift.}
    \label{fig:vqa_histograms}
\end{figure*}


% VQAV2 val 
% ivvqa 
% cvvqa 
% vqa_rephrasings
% vqa_cp v2 
% vqa_ce
% advqa 
% textvqa
% vizwiz
% okvqa













