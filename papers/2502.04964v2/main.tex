\documentclass{article}

% Recommended, but optional, packages for figures and better typesetting:
\usepackage{microtype}
\usepackage{graphicx}
\usepackage{subfigure}
\usepackage{booktabs} % for professional tables
\usepackage[table]{xcolor}

% hyperref makes hyperlinks in the resulting PDF.
% If your build breaks (sometimes temporarily if a hyperlink spans a page)
% please comment out the following usepackage line and replace
% \usepackage{icml2025} with \usepackage[nohyperref]{icml2025} above.
\usepackage{hyperref}

\usepackage{placeins}

% Attempt to make hyperref and algorithmic work together better:
\newcommand{\theHalgorithm}{\arabic{algorithm}}

% Use the following line for the initial blind version submitted for review:
% \usepackage{icml2025}

% If accepted, instead use the following line for the camera-ready submission:
\usepackage[accepted]{icml2025}

% For theorems and such
\usepackage{amsmath}
\usepackage{amssymb}
\usepackage{mathtools}
\usepackage{amsthm}
\usepackage{easyeqn}
\usepackage{natbib}
\usepackage{hyperref}
\usepackage{ulem}
\usepackage{bm}
\usepackage{multirow}


% if you use cleveref..
\usepackage[capitalize,noabbrev]{cleveref}

%%%%%%%%%%%%%%%%%%%%%%%%%%%%%%%%
% THEOREMS
%%%%%%%%%%%%%%%%%%%%%%%%%%%%%%%%
\theoremstyle{plain}
\newtheorem{theorem}{Theorem}[section]
\newtheorem{proposition}[theorem]{Proposition}
\newtheorem{lemma}[theorem]{Lemma}
\newtheorem{corollary}[theorem]{Corollary}
\theoremstyle{definition}
\newtheorem{definition}[theorem]{Definition}
\newtheorem{assumption}[theorem]{Assumption}
\theoremstyle{remark}
\newtheorem{remark}[theorem]{Remark}

% Todonotes is useful during development; simply uncomment the next line
%    and comment out the line below the next line to turn off comments
%\usepackage[disable,textsize=tiny]{todonotes}
%\usepackage[textsize=tiny]{todonotes}

\newcommand{\todo}[1]{\textcolor{red}{TODO: #1}}

% L

% Commands %%%%%%%%%%%%%%%%%%%%%%%
% Notations
\newcommand{\uv}{\mathbf{u}}
\newcommand{\vv}{\mathbf{v}}
\newcommand{\xv}{\mathbf{x}}
\newcommand{\yv}{\mathbf{y}}
\newcommand{\thetav}{\bm{\theta}}
\newcommand{\EE}{\mathbb{E}}
\newcommand{\BC}{\mathcal{B}}
\newcommand{\DC}{\mathcal{D}}
\newcommand{\CC}{\mathcal{C}}
\newcommand{\HC}{\mathcal{H}}
\newcommand{\IC}{\mathcal{I}}
\newcommand{\KC}{\mathcal{K}}
\newcommand{\MC}{\mathcal{M}}
\newcommand{\YC}{\mathcal{Y}}
\newcommand{\PT}{P(True)\xspace}
\DeclareMathOperator*{\argmax}{argmax}


\icmltitlerunning{CoCoA: Uncertainty via Confidence and Consistency of LLM Outputs}

\begin{document}

\twocolumn[
\icmltitle{CoCoA: A Generalized Approach to Uncertainty Quantification\\ by Integrating Confidence and Consistency of LLM Outputs}


\icmlsetsymbol{equal}{*}

\begin{icmlauthorlist}
  \icmlauthor{Roman Vashurin}{equal,mbzuai}
  \icmlauthor{Maiya Goloburda}{equal,mbzuai}
  \icmlauthor{Preslav Nakov}{mbzuai}
  \icmlauthor{Artem Shelmanov}{mbzuai}
  \icmlauthor{Maxim Panov}{mbzuai}
\end{icmlauthorlist}

\icmlaffiliation{mbzuai}{Mohamed bin Zayed University of Artificial Intelligence}
\icmlcorrespondingauthor{Roman Vashurin}{Roman.Vashurin@mbzuai.ac.ae}

\vskip 0.3in
]

\printAffiliationsAndNotice{\icmlEqualContribution} 
\begin{abstract}
  Uncertainty quantification (UQ) methods for Large Language Models (LLMs) encompasses a variety of approaches, with two major types being particularly prominent: information-based, which focus on model confidence expressed as token probabilities, and consistency-based, which assess the semantic relationship between multiple outputs generated using repeated sampling. Several recent methods have combined these two approaches and shown impressive performance in various applications. However, they sometimes fail to outperform much simpler baseline methods. Our investigation reveals distinctive characteristics of LLMs as probabilistic models, which help to explain why these UQ methods underperform in certain tasks. Based on these findings, we propose a new way of synthesizing model confidence and output consistency that leads to a family of efficient and robust UQ methods. We evaluate our approach across a variety of tasks such as question answering, abstractive summarization, and machine translation, demonstrating sizable improvements over state-of-the-art UQ approaches.
\end{abstract}



% !TEX root = ../main.tex

\section{Introduction}
\label{sec:intro}
  Large Language Models (LLMs) have revolutionized natural language processing (NLP), enabling advancements in information retrieval, question answering, machine translation, and other language-driven applications. As these models become an integral part of everyday life, ensuring the reliability of their outputs is crucial, especially in high-stakes scenarios where errors or uncertainty can have serious consequences. One way to address this challenge is through uncertainty quantification (UQ), which measures how confident a model is in its outputs, and makes possible the rejection of generations with a high risk of being incorrect.

  UQ for LLMs is a rapidly advancing research area, with new methods for estimating uncertainty emerging each year. The large portion of novel techniques is based on two fundamental approaches: information-theoretic analysis and the assessment of output consistency.

  \begin{figure}[t!]
    \centering
    \includegraphics[width=\linewidth]{img_inconsistent_probabilities_final.pdf}
    \caption{Example of inconsistent probabilities assigned to semantically identical answers by an LLM, demonstrating the limitation of relying solely on sequence-level information.}
  \label{fig:inconsistent_probability}
  \end{figure}
  
  Information-theoretic methods quantify the confidence of a model by analyzing the probability distributions it induces for predictions~\citep{malinin2020uncertainty, fomicheva-etal-2020-unsupervised}. A key limitation of these methods is that they cannot account for the semantic variability across multiple possible outputs for the same input. Specifically, the model may generate answers with the same meaning but with very different assigned probabilities; see Figure~\ref{fig:inconsistent_probability}. LLMs are trained to predict the next token in a sequence based on patterns observed in vast amounts of data, resulting in varying probabilities for semantically equivalent output sequences.
 
  In contrast, consistency-based methods directly analyze the semantic relationships between the sampled outputs~\citep{lin2023generating, fomicheva-etal-2020-unsupervised}. Information-theoretic and consistency-based methods have complementary strengths: the former provides insights into the model's internal confidence, while the latter captures the uncertainty as objective variability of meaning among sampled outputs. For this reason, recent state-of-the-art methods aimed to unify these approaches~\citep{kuhn2023semantic, duan-etal-2024-shifting}. Although such methods show good performance in various applications, they sometimes fail to outperform their simpler counterparts in certain scenarios~\citep{vashurin2024benchmarkinguncertaintyquantificationmethods}.

  Our investigation revealed distinctive characteristics of LLMs as probabilistic models, shedding light on why current state-of-the-art UQ methods that attempt to integrate both approaches often underperform in certain tasks. Specifically, we highlight the complexity of the token prediction process and the absence of a unified framework that simultaneously addresses model confidence and output variability, both of which can limit the effectiveness of existing UQ techniques. This insight drives our proposal for a novel family of methods that integrate model confidence with output consistency, resulting in more efficient and robust UQ techniques. Our approach combines the strengths of both information-based and consistency-based methods, providing a more comprehensive and accurate assessment of uncertainty.

  Our main contributions can be summarized as follows.
  \begin{itemize}
    \item We identify key limitations in current UQ methods for LLMs, particularly in addressing both token- and sequence-level confidence and output consistency.

    \item We present a family of \textit{\underline{Co}nfidence and \underline{Co}nsistency-based \underline{A}pproaches} (\texttt{CoCoA}) to UQ, offering a new way to merge information- and consistency-based measures for uncertainty quantification in LLMs.
        
    \item We evaluate our approaches across a variety of NLP tasks, including question answering, summarization, and translation. Our experiments demonstrate sizable improvements in the reliability and robustness of UQ compared to state-of-the-art methods.
  \end{itemize}


% !TEX root = ../main.tex

  \begin{figure*}[t!]
    \centering
    \includegraphics[width=\linewidth]{img_final_diagram.pdf}
    \caption{Illustration of the method: the LLM generates a response, evaluates the similarity to alternatives, computes the uncertainty, and combines the uncertainty score with the similarity measures. High similarity to alternatives reduces the uncertainty, while low similarity keeps it high.}
    \label{fig:method}
  \end{figure*}


\section{Background}
  In this section, we introduce key concepts related to uncertainty quantification for LLMs, outline existing methods, discuss their limitations, and highlight the motivation for our approach.

  First and foremost, it is important to establish the concept of an \textit{uncertainty function}. Let $\yv = f(\xv)$ denote the output of an LLM given an input sequence $\xv$. The model defines a probabilistic output distribution $p(\yv \mid \xv)$, from which outputs can be sampled. An \textit{uncertainty function} \( U \) is a mapping that quantifies the level of uncertainty $u$ associated with the output of a model \( \yv \), conditioned on the input sequence \( \xv \), which we denote as
  \begin{equation}
    u = U(\yv \mid \xv).
  \end{equation}
  %
  With this notion in mind, we now discuss existing methods for uncertainty quantification, outline their limitations, and describe how our approach addresses them.

\subsection{Single-Sequence Information-Based Methods}
  Information-based methods rely on a single sample from the LLM and estimate the uncertainty of the generated sequence by aggregating the uncertainty scores of individual tokens. One of the simplest techniques of this kind is
  \textit{Maximum Sequence Probability (MSP)}: 
  \begin{equation}
    u_{\text{MSP}} = - \log p\bigl(\yv \mid \xv\bigr).
  \end{equation}
  %
  Several other measures fall into this category, including \textit{Perplexity} and \textit{Mean Token Entropy}~\citep{fomicheva-etal-2020-unsupervised}; see Appendix~\ref{suppl:confidence} for details. While using only a single sample makes them computationally efficient, these techniques face three major challenges:
  \begin{enumerate}
    \item First, LLMs provide us with the probability of a specific answer, even though the same meaning could often be expressed in multiple ways. Therefore, to obtain a proper probability for the meaning of an answer, we need to marginalize over its various possible rephrasings. However, this is not feasible if we generate only a single sample. 

    \item Second, answers of LLMs in general have very low probabilities due to the size of the vocabulary and the length of the generated sequences. Therefore, their estimates by LLM are not very reliable in general.

    \item Third, these methods do not provide information about the flatness or the variability of the answer distribution. 
  \end{enumerate}
  

\subsection{Consistency-Based Methods}
  The aforementioned issues lead to the development of consistency-based methods based on repetitive sampling from the LLM.
  Consider that we have sampled a set of outputs \( \bigl\{\yv^{(i)}\bigr\}_{i=1}^M\), where \({\yv^{(i)} \sim p(\yv \mid \xv)}\). 
  Consistency-based uncertainty quantification methods rely only on the diversity of answers \(\yv^{(i)}\) sampled from the LLM. The idea is that if the model outputs similar answers for the same prompt over and over again, it is confident in its predictions; otherwise, it is uncertain. These techniques do not require knowledge about the probability distribution of the tokens and can be applied in the black-box setting, when only the generated tokens are available. This case is quite common when LLMs are deployed as a service and are accessible through a limited API.
  
  Formally, given $M$ samples from the model, consistency-based methods compute a similarity matrix \(G\), where each element \(g_{ij}\) represents some form of similarity between the sampled outputs \(\yv^{(i)}\) and \(\yv^{(j)}\): 
  \begin{equation}
    g\bigl(\yv^{(i)}, \yv^{(j)}\bigr) \in [0, 1].
  \end{equation}
  %
  The value \(g\bigl(\yv^{(i)}, \yv^{(j)}\bigr) = 1\) indicates the complete equivalence between \(\yv^{(i)}\) and \(\yv^{(j)}\), and \(g\bigl(\yv^{(i)}, \yv^{(j)}\bigr) = 0\) indicates that there is no similarity. 
  
  Similarity could be computed in various ways. For example, {\it Lexical Similarity}~\citep{fomicheva-etal-2020-unsupervised} is the surface form similarity, which calculates the overlap of words or phrases in the generations. More advanced techniques propose various methods for taking into account the semantic similarity of the generated answers by hard or soft clustering~\citep{lin2023generating}. For example, the \textit{Degree Matrix} approach considers a similarity matrix $G$, which is computed using a model for Natural Language Inference (NLI), which predicts the probabilities of entailment, ${p}_{\mathrm{entail}}(\yv, \yv')$, and contradiction, ${p}_{\mathrm{contra}}(\yv, \yv')$, between pairs of sentences $\yv$ and $\yv'$. The similarity between two sequences is then defined as either $g_{\mathrm{entail}}(\yv, \yv') = {p}_{\mathrm{entail}}(\yv, \yv') \quad \text{or} \quad g_{\mathrm{contra}}(\yv, \yv') = 1 - {p}_{\mathrm{contra}}(\yv, \yv')$. \citet{lin2023generating} proposes to compute an averaged similarity matrix as  $g_{ij} = \bigl(g\bigl(\yv^{(i)}, \yv^{(j)}\bigr) + g\bigl(\yv^{(j)}, \yv^{(i)}\bigr)\bigr)/2$. The diagonal matrix $D$ is then defined with elements  $D_{ii} = \sum_{j = 1}^M g_{ij}$. The uncertainty from this soft clustering of meanings can then be defined as follows:
  \begin{equation}
    U_{\mathrm{DegMat}} = 1 - \frac{\mathrm{trace}(D)}{M^2}. 
  \end{equation}
  %
  The advantage of these techniques is that by generating multiple samples and analyzing their semantic similarity, they can obtain empirical probabilities for \textit{meanings} instead of individual answers. The main drawback is that they discard the useful information that comes from the probability distribution represented by the LLM, including estimates of probabilities of specific answers. 


\subsection{Information-Based Methods with Repeated Sampling}
  The natural idea is to somehow benefit from having multiple samples from the model while using important information contained in the output probabilities estimated by an LLM. Below, we examine several approaches that have sought to achieve this.

\paragraph{Averaging uncertainties.}
  The uncertainty scores can be aggregated using simple Monte Carlo averaging:
  \begin{equation}
    u_{\text{MC}} = \frac{1}{M} \sum_{i=1}^M u_i.
  \label{eq:mc_uncertainty}
  \end{equation}
  %
  For the case when using the MSP uncertainty measure, i.e., when $u_i = -\log p\bigl(\yv^{(i)} \mid \xv\bigr)$, we obtain $u_{\text{MC}} = -\frac{1}{M} \sum_{i=1}^M \log p\bigl(\yv^{(i)} \mid \xv\bigr)$. The other notable example is the Monte Carlo Sequence Entropy~\citep{kuhn2023semantic}.

  While simple averaging represents a natural way to aggregate uncertainties, it has certain issues related to the nature of LLMs. First of all, in the vast majority of applications, an LLM-based system should produce a single output $\yv_*$ for an input query. When we consider $u_{\text{MC}}$, we essentially perform averaging of uncertainties of different sequences, thus somewhat assessing the uncertainty related to the entire generative distribution $p(\yv \mid \xv)$ for the input $\xv$, but not for a particular generated sequence $\yv_*$. This averaged uncertainty might not be adequate for this particular sequence and, remarkably, often performs worse than the uncertainty $u_* = U(\yv_* \mid \xv)$, which is related solely to the output $\yv_*$.
  Moreover, although intuitive, this na\"{i}ve aggregation method assumes that all outputs contribute equally to the final uncertainty estimate, regardless of their semantic relationships. This can lead to inconsistencies when semantically equivalent outputs have varying uncertainty scores or when outputs with low similarity are treated as equally important. 


\paragraph{Semantically weighted averaging.}
  The basic idea of aggregation approaches like Semantic Entropy~\cite{kuhn2023semantic} or SAR~\cite{duan-etal-2024-shifting} is to perform a weighted averaging of output probabilities and give more weight to sequences semantically similar to the response shown to a user. All recently proposed techniques, such as SAR and Semantic Entropy, can be unified into a semantically-aware Generalized Monte Carlo uncertainty estimate, defined as
  \begin{equation}
    u_{\text{GMCU}} = \frac{1}{M}\sum_{i = 1}^M h \Biggl(\sum_{j = 1}^M g_{ij} \, p_j\Biggr).
  \label{eq:gmcu}
  \end{equation}
  %
  Here, the inner summation aggregates sequence probabilities \(p_j\) weighted by their semantic similarity to the \(i\)-th output, and the outer summation averages these contributions across all samples. The function $h(\cdot)$ provides an additional layer of flexibility, transforming the reweighted uncertainty scores, making the method a generalized framework for uncertainty quantification. Existing methods, such as Semantic Entropy and SAR, can be considered as special cases of this more comprehensive approach, where the functions \(h\) and \( g \) are chosen appropriately.

  Unfortunately, methods that fall under GMCU, while offering benefits, also inherit the aforementioned issues from both categories of methods:
  \begin{enumerate}
    \item The term \(\sum_{j = 1}^M g_{ij} \, p_j\) aims to average the probabilities of semantically similar sequences to obtain more robust estimate of the probability. However, due to the extreme instability of the LLM probabilities, as shown in Figure~\ref{fig:inconsistent_probability}, the aggregated probabilities often perform worse than non-aggregated baselines.

    \item The outer summation in~\eqref{eq:gmcu}, similarly to the case of simple Monte Carlo averaging~\eqref{eq:mc_uncertainty}, often fails to outperform the uncertainty $u_* = U(\yv_* \mid \xv)$ of a single generated sequence $\yv_*$.
  \end{enumerate}

  Let us note that all uncertainty functions discussed so far have the following properties: 
  \begin{itemize}
    \item \textbf{Non-Negativity.} The uncertainty function produces nonnegative values, that is, \(U(\yv) \geq 0\) for all \(\yv\).
    
    \item \textbf{Monotonicity.} Higher values of the uncertainty function \(U\) indicate higher uncertainty, i.e.,~if output \(\yv^{(1)}\) is considered more uncertain than output \(\yv^{(2)}\), then \(U\bigl(\yv^{(1)}\bigr) \geq U\bigl(\yv^{(2)}\bigr)\).
  \end{itemize}
  %
  These properties become important to perform the kind of synthesis of confidence and consistency we propose in the following section.

\section{CoCoA: Bridging Confidence and Consistency for Better Uncertainty Quantification}
  We start by summarizing the benefits and drawbacks of various uncertainty quantification approaches discussed above:
  \begin{enumerate}
    \item Both information-based and (semantic) consistency-based methods provide grounded and useful uncertainty quantification measures.

    \item Output probabilities $p(\yv^{(j)} \mid \xv), \, j = 1, \dots, M$ might have substantially different values for semantically equivalent outputs, which questions the usefulness of (weighted) averaging these probabilities for uncertainty quantification.

    \item For various methods based on the aggregation over multiple samples, the result might be suboptimal due to the noise related to the averaging over all generated outputs. Focusing solely on a particular output sequence and its relation to other generated outputs might be beneficial.
  \end{enumerate}
  %
  In what follows, we present a family of UQ \textit{\underline{Co}nfidence and \underline{Co}nsistency-based \underline{A}pproaches} (\texttt{CoCoA}), offering a new way to merge information- and consistency-based measures for uncertainty quantification in LLMs.

  Let us consider an actual output sequence $\yv_*$ and a set of sampled sequences $\yv^{(i)}, \, i = 1, \dots, M$. Here, $\yv_*$ might be one of the sequences $\yv^{(i)}$ or might be generated separately. In what follows, we will consider several possible cases, including $\yv_*$ being a random sequence from a set $\{\yv^{(i)}\}$, $\yv_*$ being a sequence from a set $\{\yv^{(i)}\}$ having the highest probability, and, finally, $\yv_*$ being a sequence found via the beam search procedure.

  First, consider an information-based uncertainty score of the output $\yv_*$:
  \begin{equation}
    u_*^{\text{info}} = U^{\text{info}}(\yv_* \mid \xv),
  \end{equation}
  where $U^{\text{info}}$ might be MSP, perplexity, mean token entropy, or another uncertainty measure related solely to the generated sequence $\yv_*$. 

  We quantify the consistency-based uncertainty via a direct measurement of the semantic similarity of generated sequence $\yv_*$ to sampled sequences:
  \begin{equation}
    u_*^{\text{cons}} = \frac{1}{M} \sum_{i = 1}^M (1 - g_{*i}),
    \label{eq:ave_dissim}
  \end{equation}
  where $g_{*i} = g\bigl(\yv_*, \yv^{(i)}\bigr)$. This formulation satisfies the desired properties of the uncertainty function -- that is their values are nonnegative and their values increase with increased inconsistency (decreasing value of $g_{*i}$). In our ablation study, we will show that such an uncertainty measure reliably outperforms consistency-based measures that aggregate the pairwise similarities of all the samples (see Appendix~\ref{sec:sum_cocoa}).

  Finally, we need to aggregate $u_*^{\text{info}}$ and $u_*^{\text{cons}}$ into a single uncertainty measure. We propose to aggregate them in a multiplicative way:
  \begin{equation}
    u_*^{\texttt{CoCoA}} = u_*^{\text{info}} \cdot u_*^{\text{cons}}
  \label{eq:cocoa}
  \end{equation}
  %
  This formulation preserves the non-negativity and the monotonicity properties while integrating both global (semantic) and local (model-specific) uncertainty signals. It ensures that uncertainty is amplified for sequences that are both intrinsically uncertain (high \(u_*^{\text{info}}\)) and semantically inconsistent with the dataset (high \(u_*^{\text{cons}}\)), while keeping it low for the opposite scenario (see Figure~\ref{fig:method}).

  Although the choice of the multiplicative aggregation function \texttt{CoCoA} is heuristic, it provides a practical and effective way to combine information- and consistency-based uncertainty signals in LLMs. In our ablation study, we also compare the multiplicative formulation in~\eqref{eq:cocoa} to a simpler additive variant, \(u_*^{\text{info}} + u_*^{\text{cons}}\) (see Appendix~\ref{sec:sum_cocoa}). Empirically, the multiplicative combination is better at capturing the joint impact of both information-based and consistency-based uncertainty, yielding more reliable estimation across all tasks.


  % \textcolor{red}{The text below is currently not used.}
  %   Given the semantic similarity function \( g(\cdot, \cdot) \), we calculate pairwise similarity scores for all $M$ sampled outputs $\{\yv_1, \dots, \yv_M\}$. The result is a similarity matrix $G$ of size $M \times M$, where each element $g_{ij}$ represents the semantic similarity between $\yv^{(i)}$ and $\yv^{(j)}$. This matrix serves as the foundation for incorporating semantic relationships into the uncertainty enrichment process.

  % $g_{ij}$ = g\bigl(\yv^{(i)}, \yv^{(j)}\bigr)

  % Assymetric/

  % \subsection{Generalized Monte-Carlo Uncertainty}
  %   Accurately quantifying uncertainty in large language models (LLMs) requires an approach that not only accounts for the probabilistic nature of the model's output distribution $p_{\thetav}(\yv \mid \xv)$ but also reflects the semantic relationships among the generated outputs. Traditional Monte Carlo methods estimate uncertainty by uniformly aggregating scores across multiple outputs, treating all samples as equally informative. However, this approach fails to consider the underlying structure of the sampled outputs. For example, when the majority of sampled outputs are semantically similar, while one is a clear outlier, the aggregation process should assign greater weight to the consistent outputs to ensure the uncertainty estimate accurately reflects the predominant patterns in the model's predictions. A robust uncertainty estimation method should assign higher uncertainty to semantically incoherent or nonsensical outputs and lower uncertainty to outputs that are consistent and meaningful.

  %   To address this issue, we propose a Generalized Monte-Carlo Uncertainty (GMCU) method that takes semantic similarity into account when aggregating uncertainty scores. Instead of treating all outputs as equally important, GMCU assigns more weight to outputs that are semantically divergent, ensuring that the final uncertainty score reflects both the diversity of the predictions and their underlying meanings. 

  % % To address this limitation, we propose an enriched uncertainty estimation method that aggregates uncertainty scores across multiple sampled outputs, using semantic relationships between them. The key idea is to sample multiple outputs from \( p_{\thetav}(\yv \mid \xv) \) and compute an aggregated uncertainty score that accounts for their semantic equivalence.

  % % We propose a way to enrich this estimate $u$ by sampling the output distribution of an LLM repeatedly and aggregating uncertainty scores of each of the samples, with respect to their semantic equivalence. More precisely, given $M$ samples from the model and a semantic equivalence function $g_{ij} = g\bigl(\yv^{(i)}, \yv^{(j)}\bigr) \in [0,1]$, that produces similarity score between two samples $\yv^{(i)}, \yv^{(j)}$, we build a semantically-aware Monte-Carlo estimate like this:

  % Any uncertainty metric $u$ can be enriched by sampling the output distribution of a large language model (LLM) multiple times and aggregating the uncertainty scores of the sampled outputs while accounting for their semantic relationships. Specifically, given $M$ samples from the model, compute a similarity matrix \( G \), where each element \( g_{ij} \) represents the semantic similarity between the sampled outputs \( \yv^{(i)} \) and \( \yv^{(j)} \), formally defined as \( g_{ij} = g\bigl(\yv^{(i)}, \yv^{(j)}\bigr)\). Using this similarity matrix, the semantically-aware Generalized Monte Carlo uncertainty estimate is defined as:
  % \begin{equation}
  %   u_{GMCU} = \frac{1}{M}\sum_{i = 1}^M h \Biggl(\sum_{j = 1}^M g_{ij} u_j\Biggr),
  % \end{equation}
  % %
  % Here, the inner summation aggregates uncertainty scores \( u_j \) weighted by their semantic similarity to the \( i \)-th output, and the outer summation averages these contributions across all samples. The function $h(\cdot)$ provides an additional layer of flexibility, transforming the reweighted uncertainty scores, making the method a generalized framework for uncertainty estimation. Existing methods, such as Semantic Entropy, can be considered as specific cases of this more comprehensive approach, where \( h(\cdot) \) and \( g(\cdot, \cdot) \) are chosen appropriately to match their definitions.

  % This formulation ensures that the final uncertainty estimate reflects both the probabilistic variability of the model's outputs and their semantic coherence, as captured by the similarity matrix \( G \).

  %For a specific sequence \(i\), its uncertainty can be viewed as a measure of the model's inconsistency with itself - i.e. how similar is this specific sequence relative to other possible generations? 

  % This diversity-based metric thus captures the model's self-inconsistency for a given sequence in the context of the sample. These two formulations focus on two potential aspects of consistency:
  % \begin{enumerate}
  %   \item The first formulation, \( u_i^{\text{diversity}} = \frac{1}{M-1} \sum_{j \neq i} (1 - g_{ij}) \), rewards consistent outputs by bringing the value closer to 0 as the model becomes more consistent with itself and reaches 0 when all samples are identical.
  %   % % \item The second formulation, \( U_i^{\text{diversity}} = \frac{n-1}{\sum_{j \neq i} G_{ij}} \), penalizes inconsistency more strongly, with the score increasing dramatically as inconsistency grows. In the extreme case where the outputs are entirely inconsistent (i.e., all pairwise similarities \( G_{ij} = 0 \) for \( j \neq i \)), the denominator becomes zero, making the score undefined. To address this edge case, a small positive constant \( \epsilon = 10^{-10} \) is added to the denominator, ensuring numerical stability:
  %   % \[
  %   % U_i^{\text{diversity}} = \frac{n-1}{\sum_{j \neq i} G_{ij} + \epsilon}.
  %   % \]
  % \end{enumerate}

% \subsection{Alternative approach to why it looks like this}
%   \begin{itemize}
%     \item Sample scores in the LLM output are noisy and inconsistent, which is why one-sample estimates of MCSE and MCNSE often outperform full-sample ones. (MP $>$ MCSE, PPL $>$ MCNSE, see some of the results in the tables).
%     \item Semantically-enriched full-sample methods (SE, SAR) mitigate this to a certain degree by adding semantic consistency into consideration.
%     \item We suggest a one-sample approach to estimating semantically-weighted Monte-Carlo entropy of LLM output. Consider the following estimate:
%     \[
%       U_{SW-MCSE} = -\frac{1}{M}\sum_{i = 1}^M \biggl[\log p_i\sum_{j = 1}^M (1 - g_{ij})\biggr].
%     \]
%     %
%     We suggest using one-sample version of this estimate, centered at the most probable sample:
%     \[
%       U_{ours} = -\log p_{i}\sum_{j = 1}^M (1 - g_{ij}),
%     \]
%     \[
%       i = \argmax_i(p_i).
%     \]
%   \end{itemize}

 
% \subsection{Semantic Entropy}
%   \begin{equation}
%     u_{SE} = - \frac{1}{M}\sum_{i = 1}^M \log \Biggl(\sum_{j = 1}^M g_{ij} p_j\Biggr),
%   \end{equation}

% \subsection{SAR}
%   \begin{equation}
%     u_{SAR} = - \frac{1}{M}\sum_{i = 1}^M \log \left(\tilde{p}_i 
%     + \frac{\sum_{j = 1, j \neq i}^M g_{ij} \tilde{p}_j}{t}\right),
%   \end{equation}

% \subsection{MP/PPLGSU}
%   \begin{equation}
%     u_{GSU} = \frac{1}{M}\sum_{i = 1}^M \left(\frac{\sum_{j = 1}^M g_{ij}(-\log\tilde{p}_j)}{\sum_{j = 1}^M g_{ij}}\right),
%   \end{equation}

% \subsection{MP/PPLGSUexp}
%   \begin{equation}
%     u_{GSU} = \frac{1}{M}\sum_{i = 1}^M \left(\frac{\sum_{j = 1}^M g_{ij}(-\tilde{p}_j)}{\sum_{j = 1}^M g_{ij}}\right),
%   \end{equation}

% \subsection{MTEGSU}
%   \begin{equation}
%     u_{GSU} = \frac{1}{M}\sum_{i = 1}^M \left(\frac{\sum_{j = 1}^M g_{ij} u_{MTE}}{\sum_{j = 1}^M g_{ij}}\right),
%   \end{equation}

% \subsection{Semantic Average MP/PPL}
%   \begin{equation}
%     u_{SA} = \frac{\sum_{j = 1}^M g_{0j}(-\log\tilde{p}_j)}{\sum_{j = 1}^M g_{0j}},
%   \end{equation}

% \subsection{Semantic Average MP/PPLexp}
%   \begin{equation}
%     u_{SA} =\frac{\sum_{j = 1}^M g_{0j}(-\tilde{p}_j)}{\sum_{j = 1}^M g_{0j}},
% \end{equation}

% \subsection{Semantic Average MTE}
% \begin{equation}
%     u_{SA} = \frac{\sum_{j = 1}^M g_{0j} u_{MTE}}{\sum_{j = 1}^M g_{0j}},
%   \end{equation}

% \subsection{EnrichedAveSimilarity}
%   Average similarity of a sample j to other model outputs can be defined as: 
%   \begin{equation}
%     \text{Average Similarity}_j = \frac{\sum_{j \neq i} g_{ji}}{M - 1},
%   \end{equation}

%   For each sample j in ${s_1, \dots, s_n}$ and it's corresponding single-sequence metrics ${u_1, \dots, u_n}$
%   \begin{equation}
%     u^{enriched}_{j} = u_j \cdot \frac{1}{\text{Average Similarity}_j}
%   \end{equation}

% This setup explicitly punishes inconsistency in answer.
%   Problem - values from 0 to +infinity. 

% \subsection{EnrichedAveDissimilarity}
%   Average dissimilarity of a sample j to other model outputs can be defined as: 
%   \begin{equation}
%     \text{Average Dissimilarity}_j = \frac{\sum_{j \neq i} (1-g_{ji})}{M - 1},
%   \end{equation}

%   For each sample j in ${s_1, \dots, s_n}$ and it's corresponding single-sequence metrics ${u_1, \dots, u_n}$
%   \begin{equation}
%     u^{enriched}_{j} = u_j \cdot \textit{Average Dissimilarity}_j
%   \end{equation}

%   Values from 0 to 1. This setup rewards consistency. 

% First, suppose we have a distance (or dissimilarity) function 
% \begin{equation}
%     D(\mathbf{y}, \mathbf{y}') \in [0, 1]  
% \end{equation}
  
% defined between two model outputs \(\mathbf{y}\) and \(\mathbf{y}'\). We consider the 
% distribution \(p_{\theta}(\mathbf{y}\mid \mathbf{x})\) of outputs produced by a large language model (LLM). 
% The \emph{expected dissimilarity} between a fixed output \(\mathbf{y}_j\) and a random draw \(\mathbf{y}'\) 
% from that distribution is:

% \begin{equation}
%   \mathbb{E}
%   \bigl[D(\mathbf{y}_j, \mathbf{y}')\bigr]
%   \;=\;
%   \int 
%     D\bigl(\mathbf{y}_j, \mathbf{y}'\bigr)\, 
%     p_{\theta}\bigl(\mathbf{y}'\!\mid\!\mathbf{x}\bigr)\,
%   d \mathbf{y}'
% \end{equation}

% Since we typically cannot compute this integral in closed form, we approximate it 
% via Monte Carlo sampling. Specifically, we draw \(\mathbf{y}_1, \ldots, \mathbf{y}_M\) 
% i.i.d.\ from \(p_{\theta}(\mathbf{y}\mid \mathbf{x})\). Then, for each sampled output \(\mathbf{y}_j\), 
% we estimate its expected distance to a fresh random draw by the average distance 
% to the other samples:

% \begin{equation}
%   \mathbb{E}
%   \bigl[D(\mathbf{y}_j, \mathbf{y}')\bigr]
%   \;\approx\;
%   \frac{1}{M-1}
%   \sum_{\substack{i \neq j}}^{M}
%   D\bigl(\mathbf{y}_j, \mathbf{y}_i\bigr).
% \end{equation}


% \input{sections_related_works}

% !TEX root = ../main.tex

\section{Experiments}
\subsection{Experimental Setup}
  To evaluate the effectiveness of the proposed method, we extended the \texttt{LM-Polygraph} library~\citep{vashurin2024benchmarkinguncertaintyquantificationmethods,fadeeva-etal-2023-lm} by implementing our approach within its framework. Since the library already includes tools for calculating other uncertainty scores, it provided a convenient and efficient environment for setting up and running experiments. The primary objective of our experiments is to evaluate whether our method offers improved performance in key tasks such as question answering (QA), summarization (SUM), and machine translation (MT), compared to existing baselines.

  \begin{figure}[t!]
    \centering
    \includegraphics[trim={0.cm 0.cm 0.cm 0.cm},clip,width=0.99\linewidth]{img_prr.jpg}
    \caption{Prediction-Rejection Ratio (PRR) Curve. The curve illustrates the quality of the non-rejected predictions as a function of the rejection rate. \textit{Oracle} represents the optimal rejection strategy, \textit{Random} is a random rejection, and \textit{UE} is rejection based on the evaluated uncertainty estimation method.}
  \label{fig:prr}
  \end{figure}
  
\paragraph{Datasets.}
  For QA, we selected diverse datasets to capture a variety of challenges: TriviaQA~\citep{joshi-etal-2017-triviaqa}, an open-domain factual QA dataset; CoQA~\citep{coqa}, a conversational QA benchmark requiring multi-turn contextual understanding; MMLU~\citep{mmlu}, a multi-task dataset spanning 57 topics to test broad knowledge; and GSM8k~\citep{gsm8k}, which focuses on grade-school math problems requiring logical reasoning. For translation, we evaluated our method on WMT14 French-English~\citep{wmt14} and WMT19 German-English~\citep{wmt19translate}. Finally, for summarization, we used XSum~\citep{xsum}, a dataset of complex documents paired with concise abstractive summaries.

  For all datasets, we follow~\citep{vashurin2024benchmarkinguncertaintyquantificationmethods} for selecting the subsets, for prompt formatting, and for number and sourcing of few-shot examples.

\paragraph{Models.}
  We evaluated our method using the base versions of three open-weights language models: LLaMA 3.1 8B~\cite{touvron2023llama}, Mistral 7B~\cite{mistral}, and Falcon~3~7B~\cite{Falcon3}. The open-source nature of these models provides full access to their token probabilities, which are essential for implementing our UQ method. For all models, we consider base versions, without instruction-tuning.

  
% \begin{table*}[th!]
%     \centering
%     \renewcommand{\arraystretch}{1.2} % Adjust row height
%     \scalebox{0.85}{
%     \begin{tabular}{lccccccccc}
%     \bottomrule
%     \multirow{2}{*}{\textbf{Metric}}  & \multicolumn{3}{c}{\textbf{Llama}} & \multicolumn{3}{c}{\textbf{Mistral}} & \multicolumn{3}{c}{\textbf{Falcon}} \\  
%     \cmidrule(lr){2-4} \cmidrule(lr){5-7} \cmidrule(lr){8-10}
%     & \textbf{QA} & \textbf{NNMT} & \textbf{SUM} 
%     & \textbf{QA} & \textbf{NNMT} & \textbf{SUM}  
%     & \textbf{QA} & \textbf{NNMT} & \textbf{SUM}  \\
%     \midrule
%     $\text{MCSE}$ & 0.357 & 0.380 & 0.192 & 0.453 & 0.406 & 0.162 & 0.460 & 0.409 & 0.128 \\
% $\text{MCNSE}$ & 0.380 & 0.429 & 0.186 & 0.466 & 0.489 & 0.196 & 0.530 & 0.424 & 0.153 \\
% $\text{Semantic Entropy}$ & 0.397 & 0.411 & 0.194 & 0.482 & 0.438 & 0.164 & 0.479 & 0.440 & 0.134 \\
% $\text{SAR}$ & 0.479 & 0.506 & 0.159 & 0.542 & 0.576 & 0.175 & 0.590 & 0.488 & 0.193 \\
%     \midrule
% $\text{MSP}$ & 0.395 & 0.376 & \underline{0.464} & 0.444 & 0.252 & 0.330 & 0.343 & 0.381 & 0.099 \\
% $\text{CoCoA}_{MSP}$ & 0.484  \(\uparrow\)   & \underline{0.607}  \(\uparrow\)   & \textbf{0.484}  \(\uparrow\)   & 0.526  \(\uparrow\)   & \underline{0.721}  \(\uparrow\)   & 0.366  \(\uparrow\)   & 0.529  \(\uparrow\)   & \underline{0.631}  \(\uparrow\)   & 0.210  \(\uparrow\)   \\
%     \midrule
% $\text{Perplexity}$ & 0.532 & 0.563 & 0.458 & 0.587 & 0.686 & 0.365 & 0.627 & 0.589 & 0.275 \\
% $\text{CoCoA}_{Perplexity}$ & \textbf{0.571}  \(\uparrow\)   & \textbf{0.617}  \(\uparrow\)   & 0.450    & \textbf{0.613}  \(\uparrow\)   & \textbf{0.745}  \(\uparrow\)   & \underline{0.372}  \(\uparrow\)   & \textbf{0.647}  \(\uparrow\)   & \textbf{0.648}  \(\uparrow\)   & \textbf{0.310}  \(\uparrow\)   \\
%     \midrule
% $\text{MeanTokenEntropy}$ & 0.477 & 0.469 & 0.449 & 0.559 & 0.637 & 0.350 & 0.602 & 0.492 & 0.186 \\
% $\text{CoCoA}_{MeanTokenEntropy}$ & \underline{0.548}  \(\uparrow\)   & 0.579  \(\uparrow\)   & 0.451  \(\uparrow\)   & \underline{0.600}  \(\uparrow\)   & 0.720  \(\uparrow\)   & \textbf{0.373}  \(\uparrow\)   & \underline{0.641}  \(\uparrow\)   & 0.614  \(\uparrow\)   & \underline{0.289}  \(\uparrow\)   \\
%     \bottomrule
%     \end{tabular}}
%     \caption{Results for Evaluated Sequence - Best Sample: Mean PRR across datasets for each task. The best performing method is in bold, and the second-best is underscored. Arrows indicate improvement in CoCoA over the base version.}
%     \label{tab:best_sample_results}
%     \end{table*}


 \begin{table*}[th!]
    \centering
    \renewcommand{\arraystretch}{1.2} % Adjust row height
    %\scalebox{0.85}{
    \begin{tabular}{lccccccccc}
    \bottomrule
    \multirow{2}{*}{\textbf{Metric}} & \multicolumn{3}{c}{\textbf{Llama}} & \multicolumn{3}{c}{\textbf{Mistral}} & \multicolumn{3}{c}{\textbf{Falcon}} \\  
    \cmidrule(lr){2-4} \cmidrule(lr){5-7} \cmidrule(lr){8-10}
    & \textbf{QA} & \textbf{NMT} & \textbf{SUM} 
    & \textbf{QA} & \textbf{NMT} & \textbf{SUM}  
    & \textbf{QA} & \textbf{NMT} & \textbf{SUM}  \\
    \midrule
    $\text{MCSE}$ & 0.357 & 0.380 & 0.192 & 0.453 & 0.406 & 0.162 & 0.460 & 0.409 & 0.128 \\
$\text{MCNSE}$ & 0.380 & 0.429 & 0.186 & 0.466 & 0.489 & 0.196 & 0.530 & 0.424 & 0.153 \\
$\text{Semantic Entropy}$ & 0.397 & 0.411 & 0.194 & 0.482 & 0.438 & 0.164 & 0.479 & 0.440 & 0.134 \\
$\text{SAR}$ & 0.479 & 0.506 & 0.159 & 0.542 & 0.576 & 0.175 & 0.590 & 0.488 & 0.193 \\
$\text{DegMat}$ & 0.422 & 0.342 & 0.191 & 0.465 & 0.425 & 0.205 & 0.543 & 0.386 & 0.177 \\
$\text{EigValLaplacian}$ & 0.388 & 0.274 & 0.190 & 0.426 & 0.366 & 0.197 & 0.498 & 0.336 & 0.174 \\
    \midrule
$\text{MSP}$ & 0.395 & 0.376 & \underline{0.464} & 0.444 & 0.252 & 0.330 & 0.343 & 0.381 & 0.099 \\
$\text{CoCoA}_{MSP} $  & 0.484  \(\uparrow\)   & \underline{0.607}  \(\uparrow\)   & \textbf{0.484}  \(\uparrow\)   & 0.526  \(\uparrow\)   & \underline{0.721}  \(\uparrow\)   & 0.366  \(\uparrow\)   & 0.529  \(\uparrow\)   & \underline{0.631}  \(\uparrow\)   & 0.210  \(\uparrow\)   \\
    \midrule
$\text{PPL}$ & 0.532 & 0.563 & 0.458 & 0.587 & 0.686 & 0.365 & 0.627 & 0.589 & 0.275 \\
$\text{CoCoA}_{PPL} $ & \textbf{0.571}  \(\uparrow\)   & \textbf{0.617}  \(\uparrow\)   & 0.450   & \textbf{0.613}  \(\uparrow\)   & \textbf{0.745}  \(\uparrow\)   & \underline{0.372}  \(\uparrow\)   & \textbf{0.647}  \(\uparrow\)   & \textbf{0.648}  \(\uparrow\)   & \textbf{0.310}  \(\uparrow\)   \\
    \midrule
$\text{NMTE}$ & 0.477 & 0.469 & 0.449 & 0.559 & 0.637 & 0.350 & 0.602 & 0.492 & 0.186 \\
$\text{CoCoA}_{NMTE}$ & \underline{0.548}  \(\uparrow\)   & 0.579  \(\uparrow\)   & 0.451  \(\uparrow\)   & \underline{0.600}  \(\uparrow\)   & 0.720  \(\uparrow\)   & \textbf{0.373}  \(\uparrow\)   & \underline{0.641}  \(\uparrow\)   & 0.614  \(\uparrow\)   & \underline{0.289}  \(\uparrow\)   \\
    \bottomrule
    \end{tabular}%}
    \caption{Results for Evaluated Sequence -- Best Sample: Mean PRR across datasets for each task. The best performing method is in bold, and the second-best is underscored. Arrows indicate improvement in \texttt{CoCoA} over the base version.}
    \label{tab:best_sample_results}
    \end{table*}    

     

\paragraph{Similarity Function.}
  To measure the similarity, we use the RoBERTa-large cross-encoder model, fine-tuned on the Semantic Textual Similarity benchmark dataset~\citep{liu2019robertarobustlyoptimizedbert,DBLP:journals/corr/abs-1908-10084,huggingface:dataset:stsb_multi_mt}. This model is widely regarded as one of the most reliable and commonly used approaches for evaluating sentence similarity. The cross-encoder processes two sequences jointly and directly outputs a similarity score ranging from 0 to 1, providing a nuanced measure.

  Appendix~\ref{sec:ablation_sim_mat} contains comparative experiments with cross-encoder and other choices of the similarity function, substantiating this choice.


\paragraph{Baselines.} 
  We compare the performance of the proposed method against a diverse set of baseline and state-of-the-art UQ scores, including confidence-based, consistency-based, and hybrid approaches. Information-based methods include Maximum Sequence Probability (MSP), Perplexity (PPL), Mean Token Entropy (MTE), Monte Carlo Sequence Entropy (MCSE), and Monte Carlo Normalized Sequence Entropy (MCNSE). Consistency-based methods include the Degree Matrix (DegMat) and the Sum of Eigenvalues of the Graph Laplacian (EigValLaplacian). Finally, Hybrid methods include Semantic Entropy and SAR. All formulations for these baselines can be found in Appendix~\ref{sec:appendix_methods}.

\paragraph{Evaluation measure.}
  As our evaluation measure, we choose the Prediction Rejection Ratio (PRR), which measures the effectiveness of the uncertainty scores for identifying high-quality predictions~\citep{malinin2020uncertainty}. PRR operates by progressively rejecting predictions with uncertainty scores above a threshold $a$ and observing how the average quality $Q(f(\xv_i), \yv_i)$ of the remaining predictions changes. The metric is calculated as the ratio of two areas: the area between the Prediction Rejection (PR) curves for the evaluated uncertainty score and a random baseline, and the area between the oracle (the ideal uncertainty score that perfectly ranks instances by quality) and the random baseline. Formally, PRR is defined as follows:
    \begin{equation}
    PRR = \frac{\text{AUC}_{\text{unc}}-\text{AUC}_{\text{rnd}}}{\text{AUC}_{\text{oracle}}-\text{AUC}_{\text{rnd}}}.
  \label{eq:prr}
  \end{equation}
  %
  Higher PRR values indicate better alignment of uncertainty scores with prediction quality, approaching the performance of an oracle. To ensure practical applicability, we compute PRR only up to a rejection threshold of 50\%, preventing cases where excessive rejection artificially inflates quality measures. Figure~\ref{fig:prr} gives a visual representation of the PRR calculation, highlighting the relationship between the uncertainty threshold and the quality measures.

  
    \begin{table*}[th!]
    \centering
    \renewcommand{\arraystretch}{1.2} % Adjust row height
    %\scalebox{0.85}{
    \begin{tabular}{lccccccccc}
    \bottomrule
    \textbf{Metric} & \multicolumn{3}{c}{\textbf{Llama}} & \multicolumn{3}{c}{\textbf{Mistral}} & \multicolumn{3}{c}{\textbf{Falcon}} \\  
    \cmidrule(lr){2-4} \cmidrule(lr){5-7} \cmidrule(lr){8-10}
    & \textbf{QA} & \textbf{NMT} & \textbf{SUM} 
    & \textbf{QA} & \textbf{NMT} & \textbf{SUM}  
    & \textbf{QA} & \textbf{NMT} & \textbf{SUM}  \\
    \midrule
    $\text{MCSE}$ & 0.310 & 0.323 & 0.033 & 0.389 & 0.304 & 0.007 & 0.414 & 0.317 & 0.159 \\
$\text{MCNSE}$ & 0.309 & 0.393 & 0.022 & 0.384 & 0.410 & 0.009 & 0.405 & 0.422 & 0.108 \\
$\text{Semantic Entropy}$ & 0.356 & 0.343 & 0.033 & 0.423 & 0.327 & 0.008 & 0.439 & 0.348 & 0.164 \\
$\text{SAR}$ & 0.414 & 0.455 & 0.077 & 0.462 & 0.435 & 0.094 & 0.481 & 0.458 & 0.144 \\
$\text{DegMat}$ & 0.406 & 0.302 & 0.081 & 0.423 & 0.305 & 0.137 & 0.483 & 0.353 & 0.201 \\
$\text{EigValLaplacian}$ & 0.375 & 0.238 & 0.079 & 0.391 & 0.267 & 0.132 & 0.459 & 0.312 & 0.201 \\ \midrule
$\text{MSP}$ & 0.409 & 0.399 & 0.328 & 0.475 & 0.383 & 0.287 & 0.475 & 0.356 & 0.201 \\
$\text{CoCoA}_{MSP}$ & \underline{0.451}  \(\uparrow\)   & \textbf{0.519}  \(\uparrow\)   & 0.378  \(\uparrow\)   & \textbf{0.509}  \(\uparrow\)   & \textbf{0.497}  \(\uparrow\)   & \textbf{0.330}  \(\uparrow\)   & 0.511  \(\uparrow\)   & 0.505  \(\uparrow\)   & \textbf{0.257}  \(\uparrow\)   \\ \midrule
$\text{PPL}$ & 0.381 & 0.386 & 0.369 & 0.424 & 0.427 & 0.204 & 0.456 & 0.450 & 0.155 \\
$\text{CoCoA}_{PPL}$ & \textbf{0.454}  \(\uparrow\)   & \underline{0.481}  \(\uparrow\)   & \textbf{0.387}  \(\uparrow\)   & \underline{0.494}  \(\uparrow\)   & \underline{0.472}  \(\uparrow\)   & 0.286  \(\uparrow\)   & \underline{0.523}  \(\uparrow\)   & \underline{0.508}  \(\uparrow\)   & \underline{0.229}  \(\uparrow\)   \\ \midrule
$\text{NMTE}$ & 0.353 & 0.382 & 0.357 & 0.417 & 0.438 & 0.182 & 0.456 & 0.473 & 0.152 \\
$\text{CoCoA}_{NMTE}$ & 0.447  \(\uparrow\)   & 0.478  \(\uparrow\)   & \underline{0.380}  \(\uparrow\)   & 0.492  \(\uparrow\)   & 0.469  \(\uparrow\)   & \underline{0.288}  \(\uparrow\)   & \textbf{0.527}  \(\uparrow\)   & \textbf{0.508}  \(\uparrow\)   & 0.228  \(\uparrow\)   \\
    \bottomrule
    \end{tabular}%}
    \caption{Results for Evaluated Sequence -- Greedy Sample: Mean PRR across datasets for each task. The best-performing method is shown in bold, and the second-best is underscored. Arrows indicate improvement in \texttt{CoCoA} over the base version.}
    \label{tab:greedy_results}
    \end{table*}

\paragraph{Quality Measures.}
  The Predictive Rejection Ratio (PRR) requires an appropriate quality measure for each specific task to effectively evaluate the model output. For question-answering tasks, we use \textit{Accuracy} to directly evaluate whether the generated answers match the ground truth in short-form QA tasks (e.g., MMLU), and we use the \textit{AlignScore} between correct answer and generated sequence for assessing the performance for long-form QA tasks~\citep{zha2023alignscore}. For summarization tasks, we use  \textit{AlignScore} to measure the alignment between the output summary and the input document. It serves as a quality indicator by evaluating the relevance and the overlap between the generated content with the source text. For translation tasks, we use \textit{COMET}, as it captures both semantic adequacy and fluency, ensuring that translations are accurate and linguistically appropriate~\citep{rei-etal-2020-comet}. 

\paragraph{Generation Setup.}
  We discuss the generation parameters, the decoding strategy and sample selection procedure in depth in Appendix~\ref{sec:experimental_results}. In short, we report evaluation results in two distinct setups: greedy decoding and stochastic sampling with focus on the most probable sequence among the generated outputs (\textit{best-sample}). These two setups offer the highest-quality outputs and are the most reasonable generation approaches in practice. 


\subsection{Results}
\paragraph{Main results.}
  Tables~\ref{tab:best_sample_results} and \ref{tab:greedy_results} show the PRR scores under the \textit{best‐sample} and  \textit{greedy} generation setup. We calculate a single representative PRR for each task -- question answering, neural machine translation (NMT), and summarization (SUM) -- by averaging the results across all relevant datasets (e.g., TriviaQA, MMLU, CoQA, GSM8k for QA). This aggregated score provides a concise measure of the performance for each model for each task. Detailed results for each dataset separately can be found in Appendix~\ref{sec:experimental_results}. 

  We can see that our \texttt{CoCoA} methods are \textit{the best} across all tasks and models in our experiments. They outperform existing consistency‐based and hybrid state‐of‐the‐art approaches, like Semantic Entropy and SAR. In addition, our proposed \texttt{CoCoA} approach consistently surpasses the baseline UE metrics: for example, $\texttt{CoCoA}_{PPL}$ outperforms standard $\text{Perplexity}$, illustrating the advantage of combining token‐level confidence with semantic consistency. This pattern holds for other information‐based metrics as well, demonstrating that using the consistency between multiple sampled outputs reliably enhances uncertainty quantification.

\paragraph{Ablation study.}
  As a part of our ablation study (see Appendix~\ref{suppl:ave_dissim_ue}), we evaluate the performance of the average dissimilarity component (\( u_*^{\text{cons}} \)) independently to assess its effectiveness as a standalone uncertainty measure and to investigate whether it could potentially outweigh the contribution of the information-based component (\( u_*^{\text{info}} \)) in the enriched uncertainty measure. This evaluation enables us to isolate and better understand the complementary roles and the relative importance of each component. Our experiments demonstrate that the combination of consistency- and confidence-based metrics outperforms the pure consistency-based measure on a vast majority of tasks. Notably, in the few cases where the pure consistency messure outperforms the combined approach, the performance difference is minimal. It is possible that a more suitable choice of a similarity measure or confidence-based metric for the task could further improve the performance.
 
  This leads us to the next part of our ablation study, where we investigate the impact of different similarity measures (see Appendix~\ref{sec:ablation_sim_mat}). We find that for some tasks, the similarity score computed by the Cross-encoder does not yield optimal performance. For example, for question-answering tasks on CoQA and Trivia, NLI-derived similarity performs better than the Cross-encoder similarity and outperforms the pure consistency-based uncertainty discussed above.
  
  The next section of our ablation study focuses on alternative forms of combining model confidence $u_*^{\text{info}}$ and consistency $u_*^{\text{cons}}$ (see Apendix~\ref{sec:sum_cocoa}). First, we consider an additive form of combining them: $\mathrm{U}_{\text{AdditiveCoCoA}} = u_*^{\text{info}} + u_*^{\text{cons}}$. The results show that this additive formulation does not perform as well compared to the multiplicative one. The additive form tends to underemphasize the interaction between the two components, which is critical for capturing the nuanced relationships between confidence and consistency. 

  We also consider an alternative formulation of the consistency term $u_*^{\text{cons}} $, as the average of the full pairwise dissimilarity. In this formulation,  $u_*^{\text{cons}}$ represents the average inconsistency across all samples rather than focusing solely on the dissimilarity of the evaluated sequence with the other samples. Our experiments demonstrate that this formulation is not very strong. By distributing the consistency computation across all samples, it loses focus on the specific sequence being evaluated.

 Lastly, in Appendix~\ref{sec:sum_cocoa}, we also consider alternative formulations of the information-based metric that do not rely on logarithmic transformations. While we primarily use logarithms due to their numerical stability, we explore an alternative approach by converting these values back to probabilities and analyzing their impact on uncertainty estimation. Our findings indicate that both formulations exhibit consistent performance and yield similar results. This suggests that while logarithmic transformations enhance numerical stability, the choice between log-based and probability-based formulations does not affect much the overall performance.



% !TEX root = ../main.tex

\section{Limitations}

While our proposed \texttt{CoCoA} approach demonstrates robust empirical performance, several important considerations remain.

\paragraph{Task and Domain Dependency.}
  Our method relies on both an information-based confidence score and a semantic similarity function. The effectiveness of each can vary across models, tasks, and domains. For open-ended tasks with multiple equally valid outputs (e.g., creative generation), consistent rephrasing may inflate the perceived certainty. Conversely, in domains that demand highly precise factual or logical correctness (e.g., math problem solving), small deviations in reasoning can lead to large outcome differences that are not fully captured by a generic similarity measure. Adapting both the confidence measure and the similarity function to specific domains or prompt types is an important direction for future work.

\paragraph{Limited Sample Size.}
  \texttt{CoCoA} estimates the model's consistency by sampling multiple outputs and comparing them. In practice, generating a large number of samples can be computationally expensive and may increase the inference latency. Consequently, our experiments (like many sampling-based approaches) rely on relatively small sample sets. Although even a handful of samples can provide a meaningful estimate of consistency, it may not fully capture the diversity of the underlying distribution for certain tasks or for more complex prompts.

\paragraph{Quality Metric.}
  Finally, the \texttt{CoCoA}'s performance assessment depends on quality metrics (e.g., COMET for machine translation, and Accuracy for QA) that may not capture every nuance of textual outputs. Automatic metrics can have blind spots, particularly in evaluating coherence, factual correctness, or subtle aspects of style. Further refining or extending quality metrics to account for deeper reasoning, factual faithfulness, and stylistic appropriateness would better align uncertainty scores with real-world perceptions of model correctness.


% !TEX root = ../main.tex

\section{Conclusion}
  We presented \texttt{CoCoA}, a unified approach that integrates \textbf{Co}nfidence and \textbf{Co}nsistency for uncertainty quantification in LLMs. By combining token-level confidence scores with semantic similarity between multiple sampled outputs, CoCoA offers a more holistic view of uncertainty than either approach alone. In extensive evaluations on question answering, summarization, and translation, our approach outperformed existing baselines and state-of-the-art UQ methods. Moreover, \texttt{CoCoA}'s flexible design allows easy adaptation to a variety of tasks and settings. 
  
  Moving forward, several directions are open for further exploration. These include incorporating more adaptive sampling strategies that efficiently capture the model output space, refining semantic similarity functions for domain-specific tasks, and improving calibration techniques to strengthen the confidence metrics of the model.


% !TEX root = ../main.tex

\vspace{-2mm}
\section*{Impact Statement}
As large language models (LLMs) are increasingly integrated into everyday applications, ensuring the robustness of their outputs is essential, particularly in high-stakes domains such as healthcare and the legal sector. In addition to flagging low-quality responses, clear confidence estimates can help reduce overreliance on automated systems and foster safer, more responsible LLM usage. Here, we introduced \texttt{CoCoA}, a new framework that enhances uncertainty quantification in LLMs. While \texttt{CoCoA} improves reliability, its use requires careful domain adaptation to align with specific task requirements.


% Fix from misc to article / inproceedings 
\bibliography{custom}
\bibliographystyle{acl_natbib}


\newpage
\clearpage

\appendix
\onecolumn

\section{Decoding Strategy and Sample Selection}
\label{sec:generation_params}
  Modern LLMs are capable of producing output using a wide range of decoding strategies, and it is not readily apparent which one to use as a foundation for UQ experiments. On top of that, when sampling multiple outputs stochastically, one has to decide which sample to select for comparison with the target sequence and UQ purposes. 

  To facilitate the choice of decoding and sample selection strategies for our experiments, we conducted an evaluation of model performance with different approaches to both. Table~\ref{tab:base_quality} shows average values of corresponding quality metrics for all combinations of models and datasets. We considered 4 approaches for the selection of output that subsequently is used to calculate the quality of generation.

  \begin{itemize}
    \item \textbf{Greedy decoding} produces single output by selecting top-1 candidate token at each generation step, thus not further selection of sample is needed.

    \item \textbf{Random sample} corresponds to the case where random output is selected among the number of samples produced by repeatedly prompting the model with the same question. In practice we use first generated sample, highlighting model performance when stochastic decoding is done only once.

    \item \textbf{Best (normalized) sample} selects the output with highest model-assigned (length-normalized) probability among several sampled outputs.
  \end{itemize}

  We note that selecting a random sample from the model outputs incurs a significant drop in the quality of results on several datasets, most prominently on GSM8k. Based on these observations, we evaluate the efficacy of UE on two setups: greedy decoding and stochastic sampling with a focus on the highest-probability sample.

  In all experiments, we performed stochastic sampling with temperature $t=1.0$, top-k equal to 50, and top-p equal to $1.0$.

  
% \begin{table*}[th!]
%     \centering
%     \renewcommand{\arraystretch}{1.2} % Adjust row height
%     \scalebox{0.85}{
%     \begin{tabular}{lccccccccc}
%     \bottomrule
%     \multirow{2}{*}{\textbf{Dataset}}  & \multicolumn{3}{c}{\textbf{Llama}} & \multicolumn{3}{c}{\textbf{Mistral}} & \multicolumn{3}{c}{\textbf{Falcon}} \\  
%     \cmidrule(lr){2-4} \cmidrule(lr){5-7} \cmidrule(lr){8-10}
%     & \textbf{Greedy} & \textbf{Random} & \textbf{Best} 
%     & \textbf{Greedy} & \textbf{Random} & \textbf{Best} 
%     & \textbf{Greedy} & \textbf{Random} & \textbf{Best} 
%     \\
%     \midrule
% Trivia	& 0.686 & 0.625 & 0.687	&0.743 & 0.655 & 0.750	& 0.557 & 0.473 & 0.568 \\
% Mmlu	& 0.590 & 0.430 & 0.597	& 0.633 & 0.558 & 0.632	& 0.713 & 0.639 & 0.713 \\
% Coqa	& 0.499 & 0.359 & 0.529	& 0.574 & 0.403 & 0.591	& 0.512 & 0.408 & 0.560 \\
% Gsm8k	& 0.548 & 0.234 & 0.261	& 0.382 & 0.169 & 0.190	& 0.776 & 0.313 & 0.205 \\
% Xsum	& 0.848 & 0.608 & 0.825	& 0.803 & 0.578 & 0.775	& 0.842 & 0.734 & 0.782 \\
% Wmt14 (Fr-En)	& 0.863 & 0.819 & 0.852	& 0.863 & 0.812 & 0.830	& 0.867 & 0.833 & 0.857 \\
% Wmt19 (De-En)	& 0.870 & 0.816 & 0.854	& 0.864 & 0.805 & 0.836	& 0.846 & 0.807 & 0.826 \\
%     \bottomrule
%     \end{tabular}}
%     \caption{Base quality metrics for models for different evaluated sequence choice.}
%     \label{tab:base_quality}
%     \end{table*}
    
\begin{table*}[ht!]
\footnotesize
\centering
   \scalebox{0.75}{
\begin{tabular}{llcccccccccccc}
\toprule
\multirow{2}{*}{\textbf{Dataset}} 
& \multirow{2}{*}{\textbf{Metric}} 
& \multicolumn{4}{c}{\textbf{Llama}} & \multicolumn{4}{c}{\textbf{Mistral}} & \multicolumn{4}{c}{\textbf{Falcon}} \\  
\cmidrule(lr){3-6} \cmidrule(lr){7-10} \cmidrule(lr){11-14}
& & \textbf{Greedy} & \textbf{Random} & \textbf{Best} & \textbf{Best Norm} 
 & \textbf{Greedy} & \textbf{Random} & \textbf{Best}  & \textbf{Best Norm} 
& \textbf{Greedy} & \textbf{Random} & \textbf{Best} & \textbf{Best Norm} \\
\midrule
Trivia & Algin Score & 0.686 & 0.625 & 0.687 & \textbf{0.689} & 0.743 & 0.655 & 0.750 & \textbf{0.751} & 0.557 & 0.473 & \textbf{0.568} & 0.562 \\
MMLU & Accuracy& 0.590 & 0.430 & \textbf{0.597} & 0.587 & \textbf{0.633} & 0.558 & 0.632 & 0.632 & \textbf{0.713} & 0.639 & \textbf{0.713} & \textbf{0.713} \\
CoQa & Algin Score & 0.499 & 0.359 & \textbf{0.529} & 0.462 & 0.574 & 0.403 & \textbf{0.591} & 0.528 & 0.512 & 0.408 & \textbf{0.560} & 0.471 \\
GSM8k &Accuracy & \textbf{0.548} & 0.234 & 0.261 & 0.432 & \textbf{0.382} & 0.169 & 0.190 & 0.290 & \textbf{0.776} & 0.313 & 0.205 & 0.593 \\
Xsum & Align Score& \textbf{0.848} & 0.608 & 0.825 & 0.830 & \textbf{0.803} & 0.578 & 0.775 & 0.777 & \textbf{0.842} & 0.734 & 0.782 & 0.809 \\
WMT14FrEn & Comet & \textbf{0.863} & 0.819 & 0.852 & 0.859 & \textbf{0.863} & 0.812 & 0.830 & 0.855 & \textbf{0.867} & 0.833 & 0.857 & 0.863 \\
WMT19DeEn & Comet & \textbf{0.870} & 0.816 & 0.854 & 0.860 & \textbf{0.864} & 0.805 & 0.836 & 0.851 & \textbf{0.846} & 0.807 & 0.826 & 0.838 \\
\bottomrule
\end{tabular}}
\caption{Base quality metrics for models for different evaluated sequence choice.}
\label{tab:base_quality}
\end{table*}



\newpage

\section{Ablation}
\label{sec:appendix_ablation}

\subsection{Average Dissimilarity as Uncertainty Measure}
\label{suppl:ave_dissim_ue}
  Tables~\ref{tab:ablation_dissim_greedy} and~\ref{tab:ablation_dissim_best} report PRRs of \texttt{CoCoA}-family methods with uncertainty estimates based solely on average dissimilarity of samples, as proposed in equation~\eqref{eq:ave_dissim} We observe that it is still widely beneficial to synthesize consistency of outputs with model confidence, even when limiting consistency evaluation to the particular sample to be scored.

  % \begin{table*}[h!]
% \footnotesize
% \centering
% \caption{Comparison of PRRs of CoCoA-family methods with similarity of greedy output and other samples taken in isolation.}
% \begin{tabular}{lrrrrrrr}
%  & XSum & WMT14FrEn & WMT19DeEn & CoQa & Trivia & MMLU & GSM8k \\

% \midrule

% & \multicolumn{7}{c}{Mistral7b-Base} \\

% \midrule

% AveDissimilarity & \cellcolor[rgb]{0.615686274509804, 0.6490196078431373, 0.8764705882352941} 0.051& \cellcolor[rgb]{0.615686274509804, 0.6490196078431373, 0.8764705882352941} 0.285& \cellcolor[rgb]{0.615686274509804, 0.6490196078431373, 0.8764705882352941} 0.500& \cellcolor[rgb]{0.9450980392156862, 0.711764705882353, 0.6666666666666666} 0.379& \cellcolor[rgb]{0.615686274509804, 0.6490196078431373, 0.8764705882352941} 0.647& \cellcolor[rgb]{0.615686274509804, 0.6490196078431373, 0.8764705882352941} 0.423& \cellcolor[rgb]{0.615686274509804, 0.6490196078431373, 0.8764705882352941} 0.435\\
% $\text{CoCoA}_{MSP}$ & \cellcolor[rgb]{0.8529411764705882, 0.5078431372549019, 0.5745098039215686} 0.330& \cellcolor[rgb]{0.8529411764705882, 0.5078431372549019, 0.5745098039215686} 0.396& \cellcolor[rgb]{0.8529411764705882, 0.5078431372549019, 0.5745098039215686} 0.598& \cellcolor[rgb]{0.8529411764705882, 0.5078431372549019, 0.5745098039215686} 0.383& \cellcolor[rgb]{0.9352941176470588, 0.6901960784313725, 0.6509803921568628} 0.670& \cellcolor[rgb]{0.8529411764705882, 0.5078431372549019, 0.5745098039215686} 0.466& \cellcolor[rgb]{0.8529411764705882, 0.5078431372549019, 0.5745098039215686} 0.517\\
% $\text{CoCoA}_{PPL}$ & \cellcolor[rgb]{0.9490196078431372, 0.7196078431372549, 0.6725490196078432} 0.286& \cellcolor[rgb]{0.9627450980392157, 0.7490196078431373, 0.6941176470588235} 0.375& \cellcolor[rgb]{0.984313725490196, 0.8392156862745098, 0.7823529411764706} 0.568& \cellcolor[rgb]{0.9333333333333333, 0.9313725490196079, 0.9313725490196079} 0.369& \cellcolor[rgb]{0.8529411764705882, 0.5078431372549019, 0.5745098039215686} 0.674& \cellcolor[rgb]{0.8529411764705882, 0.5078431372549019, 0.5745098039215686} 0.466& \cellcolor[rgb]{0.8745098039215686, 0.9137254901960784, 0.9823529411764705} 0.467\\
% $\text{CoCoA}_{MTE}$ & \cellcolor[rgb]{0.9450980392156862, 0.711764705882353, 0.6666666666666666} 0.288& \cellcolor[rgb]{0.9666666666666667, 0.7588235294117647, 0.703921568627451} 0.374& \cellcolor[rgb]{0.984313725490196, 0.8647058823529412, 0.8117647058823529} 0.564& \cellcolor[rgb]{0.615686274509804, 0.6490196078431373, 0.8764705882352941} 0.355& \cellcolor[rgb]{0.8784313725490196, 0.5843137254901961, 0.5941176470588235} 0.673& \cellcolor[rgb]{0.9666666666666667, 0.907843137254902, 0.8764705882352941} 0.447& \cellcolor[rgb]{0.984313725490196, 0.8450980392156863, 0.788235294117647} 0.491\\
% \midrule

% & \multicolumn{7}{c}{Llama8b-Base} \\

% \midrule

% AveDissimilarity & \cellcolor[rgb]{0.615686274509804, 0.6490196078431373, 0.8764705882352941} 0.024& \cellcolor[rgb]{0.615686274509804, 0.6490196078431373, 0.8764705882352941} 0.389& \cellcolor[rgb]{0.615686274509804, 0.6490196078431373, 0.8764705882352941} 0.453& \cellcolor[rgb]{0.8529411764705882, 0.5078431372549019, 0.5745098039215686} 0.375& \cellcolor[rgb]{0.8529411764705882, 0.5078431372549019, 0.5745098039215686} 0.614& \cellcolor[rgb]{0.615686274509804, 0.6490196078431373, 0.8764705882352941} 0.392& \cellcolor[rgb]{0.615686274509804, 0.6490196078431373, 0.8764705882352941} 0.368\\
% $\text{CoCoA}_{MSP}$ & \cellcolor[rgb]{0.8705882352941177, 0.5607843137254902, 0.5882352941176471} 0.378& \cellcolor[rgb]{0.8529411764705882, 0.5078431372549019, 0.5745098039215686} 0.456& \cellcolor[rgb]{0.8529411764705882, 0.5078431372549019, 0.5745098039215686} 0.582& \cellcolor[rgb]{0.8156862745098039, 0.8764705882352941, 1.0} 0.349& \cellcolor[rgb]{0.7588235294117647, 0.8274509803921568, 0.9941176470588236} 0.597& \cellcolor[rgb]{0.8529411764705882, 0.5078431372549019, 0.5745098039215686} 0.485& \cellcolor[rgb]{0.6470588235294118, 0.696078431372549, 0.9176470588235294} 0.372\\
% $\text{CoCoA}_{PPL}$ & \cellcolor[rgb]{0.8529411764705882, 0.5078431372549019, 0.5745098039215686} 0.387& \cellcolor[rgb]{0.9352941176470588, 0.6901960784313725, 0.6509803921568628} 0.448& \cellcolor[rgb]{0.9196078431372549, 0.9313725490196079, 0.9470588235294117} 0.514& \cellcolor[rgb]{0.6254901960784314, 0.6666666666666666, 0.8921568627450981} 0.338& \cellcolor[rgb]{0.615686274509804, 0.6490196078431373, 0.8764705882352941} 0.593& \cellcolor[rgb]{0.9823529411764705, 0.8705882352941177, 0.8176470588235294} 0.452& \cellcolor[rgb]{0.9549019607843138, 0.7313725490196078, 0.6803921568627451} 0.433\\
% $\text{CoCoA}_{MTE}$ & \cellcolor[rgb]{0.8647058823529412, 0.5431372549019607, 0.5843137254901961} 0.380& \cellcolor[rgb]{0.9470588235294117, 0.7156862745098039, 0.6686274509803921} 0.446& \cellcolor[rgb]{0.9019607843137255, 0.9254901960784314, 0.9627450980392157} 0.511& \cellcolor[rgb]{0.615686274509804, 0.6490196078431373, 0.8764705882352941} 0.337& \cellcolor[rgb]{0.884313725490196, 0.9176470588235294, 0.9764705882352941} 0.601& \cellcolor[rgb]{0.6784313725490196, 0.7392156862745098, 0.9490196078431372} 0.402& \cellcolor[rgb]{0.8529411764705882, 0.5078431372549019, 0.5745098039215686} 0.447\\
% \midrule

% & \multicolumn{7}{c}{Falcon7b-Base} \\

% \midrule

% AveDissimilarity & \cellcolor[rgb]{0.615686274509804, 0.6490196078431373, 0.8764705882352941} 0.226& \cellcolor[rgb]{0.615686274509804, 0.6490196078431373, 0.8764705882352941} 0.337& \cellcolor[rgb]{0.615686274509804, 0.6490196078431373, 0.8764705882352941} 0.496& \cellcolor[rgb]{0.8529411764705882, 0.5078431372549019, 0.5745098039215686} 0.408& \cellcolor[rgb]{0.615686274509804, 0.6490196078431373, 0.8764705882352941} 0.656& \cellcolor[rgb]{0.615686274509804, 0.6490196078431373, 0.8764705882352941} 0.485& \cellcolor[rgb]{0.615686274509804, 0.6490196078431373, 0.8764705882352941} 0.426\\
% $\text{CoCoA}_{MSP}$ & \cellcolor[rgb]{0.8529411764705882, 0.5078431372549019, 0.5745098039215686} 0.257& \cellcolor[rgb]{0.8921568627450981, 0.6098039215686275, 0.6058823529411764} 0.433& \cellcolor[rgb]{0.8666666666666667, 0.5529411764705883, 0.5862745098039216} 0.578& \cellcolor[rgb]{0.6529411764705882, 0.703921568627451, 0.9235294117647059} 0.396& \cellcolor[rgb]{0.8549019607843137, 0.5176470588235295, 0.5764705882352941} 0.684& \cellcolor[rgb]{0.8529411764705882, 0.5078431372549019, 0.5745098039215686} 0.529& \cellcolor[rgb]{0.6862745098039216, 0.7490196078431373, 0.9549019607843138} 0.436\\
% $\text{CoCoA}_{PPL}$ & \cellcolor[rgb]{0.6901960784313725, 0.7509803921568627, 0.9568627450980393} 0.229& \cellcolor[rgb]{0.8705882352941177, 0.5607843137254902, 0.5882352941176471} 0.436& \cellcolor[rgb]{0.8529411764705882, 0.5078431372549019, 0.5745098039215686} 0.580& \cellcolor[rgb]{0.9686274509803922, 0.7666666666666667, 0.7098039215686275} 0.406& \cellcolor[rgb]{0.9803921568627451, 0.8137254901960784, 0.7529411764705882} 0.677& \cellcolor[rgb]{0.8529411764705882, 0.5078431372549019, 0.5745098039215686} 0.529& \cellcolor[rgb]{0.9784313725490196, 0.888235294117647, 0.8431372549019608} 0.478\\
% $\text{CoCoA}_{MTE}$ & \cellcolor[rgb]{0.6666666666666666, 0.7235294117647059, 0.9372549019607843} 0.228& \cellcolor[rgb]{0.8529411764705882, 0.5078431372549019, 0.5745098039215686} 0.439& \cellcolor[rgb]{0.8725490196078431, 0.5705882352941176, 0.5901960784313726} 0.577& \cellcolor[rgb]{0.615686274509804, 0.6490196078431373, 0.8764705882352941} 0.395& \cellcolor[rgb]{0.8529411764705882, 0.5078431372549019, 0.5745098039215686} 0.685& \cellcolor[rgb]{0.9823529411764705, 0.8235294117647058, 0.7627450980392156} 0.517& \cellcolor[rgb]{0.8529411764705882, 0.5078431372549019, 0.5745098039215686} 0.510\\
% \midrule

% \end{tabular}
% \label{tab:ablation_dissim_greedy}
% \end{table*}


\begin{table*}[h!]
\footnotesize
\centering

\begin{tabular}{lrrrrrrr}
\toprule
    \multirow{2}{*}{\textbf{Method}}  & \multicolumn{7}{c}{\textbf{Dataset}}  \\ 
      \cmidrule(lr){2-8}  \\
  & XSum & WMT14FrEn & WMT19DeEn & CoQa & Trivia & MMLU & GSM8k \\
  \midrule

\rowcolor[gray]{0.9} & \multicolumn{7}{c}{Mistral7b-Base} \\

\midrule

AveDissimilarity & 0.051& 0.285& 0.500& \underline{0.379}& 0.647& 0.423& 0.435\\
$\text{CoCoA}_{MSP}$ & \textbf{0.330}& \textbf{0.396}& \textbf{0.598}& \textbf{0.383}& 0.670& \textbf{0.466}& \textbf{0.517}\\
$\text{CoCoA}_{PPL}$ & 0.286& \underline{0.375}& \underline{0.568}& 0.369& \textbf{0.674}& \textbf{0.466}& 0.467\\
$\text{CoCoA}_{MTE}$ & \underline{0.288}& 0.374& 0.564& 0.355& \underline{0.673}& 0.447& \underline{0.491}\\
\midrule

\rowcolor[gray]{0.9} & \multicolumn{7}{c}{Llama8b-Base} \\

\midrule

AveDissimilarity & 0.024& 0.389& 0.453& \textbf{0.375}& \textbf{0.614}& 0.392& 0.368\\
$\text{CoCoA}_{MSP}$ & 0.378& \textbf{0.456}& \textbf{0.582}& \underline{0.349}& 0.597& \textbf{0.485}& 0.372\\
$\text{CoCoA}_{PPL}$ & \textbf{0.387}& \underline{0.448}& \underline{0.514}& 0.338& 0.593& \underline{0.452}& \underline{0.433}\\
$\text{CoCoA}_{MTE}$ & \underline{0.380}& 0.446& 0.511& 0.337& \underline{0.601}& 0.402& \textbf{0.447}\\
\midrule

\rowcolor[gray]{0.9} & \multicolumn{7}{c}{Falcon7b-Base} \\

\midrule

AveDissimilarity & 0.226& 0.337& 0.496& \textbf{0.408}& 0.656& 0.485& 0.426\\
$\text{CoCoA}_{MSP}$ & \textbf{0.257}& 0.433& \underline{0.578}& 0.396& \underline{0.684}& \textbf{0.529}& 0.436\\
$\text{CoCoA}_{PPL}$ & \underline{0.229}& \underline{0.436}& \textbf{0.580}& \underline{0.406}& 0.677& \textbf{0.529}& \underline{0.478}\\
$\text{CoCoA}_{MTE}$ & 0.228& \textbf{0.439}& 0.577& 0.395& \textbf{0.685}& 0.517& \textbf{0.510}\\
\bottomrule

\end{tabular}
\caption{Comparison of PRRs of \texttt{CoCoA}-family methods with similarity of greedy output and other samples taken in isolation.}
\label{tab:ablation_dissim_greedy}
\end{table*}

  % \begin{table*}[h!]
% \footnotesize
% \centering
% \caption{Comparison of PRRs of CoCoA-family methods with similarity of samples with best sample taken in isolation.}
% \begin{tabular}{lrrrrrrr}
%  & XSum & WMT14FrEn & WMT19DeEn & CoQa & Trivia & MMLU & GSM8k \\

% \midrule

% & \multicolumn{7}{c}{Mistral7b-Base} \\

% \midrule

% AveDissimilarity & \cellcolor[rgb]{0.615686274509804, 0.6490196078431373, 0.8764705882352941} 0.071& \cellcolor[rgb]{0.615686274509804, 0.6490196078431373, 0.8764705882352941} 0.670& \cellcolor[rgb]{0.615686274509804, 0.6490196078431373, 0.8764705882352941} 0.708& \cellcolor[rgb]{0.9588235294117647, 0.9156862745098039, 0.8921568627450981} 0.405& \cellcolor[rgb]{0.615686274509804, 0.6490196078431373, 0.8764705882352941} 0.614& \cellcolor[rgb]{0.615686274509804, 0.6490196078431373, 0.8764705882352941} 0.423& \cellcolor[rgb]{0.9764705882352941, 0.796078431372549, 0.7352941176470589} 0.846\\
% $\text{CoCoA}_{MSP}$ & \cellcolor[rgb]{0.8705882352941177, 0.5607843137254902, 0.5882352941176471} 0.366& \cellcolor[rgb]{0.9823529411764705, 0.8725490196078431, 0.8215686274509804} 0.712& \cellcolor[rgb]{0.9235294117647059, 0.9313725490196079, 0.9431372549019608} 0.730& \cellcolor[rgb]{0.8529411764705882, 0.5078431372549019, 0.5745098039215686} 0.430& \cellcolor[rgb]{0.9274509803921569, 0.6764705882352942, 0.6431372549019608} 0.644& \cellcolor[rgb]{0.8529411764705882, 0.5078431372549019, 0.5745098039215686} 0.466& \cellcolor[rgb]{0.615686274509804, 0.6490196078431373, 0.8764705882352941} 0.562\\
% $\text{CoCoA}_{PPL}$ & \cellcolor[rgb]{0.8529411764705882, 0.5078431372549019, 0.5745098039215686} 0.372& \cellcolor[rgb]{0.8529411764705882, 0.5078431372549019, 0.5745098039215686} 0.735& \cellcolor[rgb]{0.8529411764705882, 0.5078431372549019, 0.5745098039215686} 0.755& \cellcolor[rgb]{0.9372549019607843, 0.9294117647058824, 0.9254901960784314} 0.402& \cellcolor[rgb]{0.8529411764705882, 0.5078431372549019, 0.5745098039215686} 0.648& \cellcolor[rgb]{0.8529411764705882, 0.5078431372549019, 0.5745098039215686} 0.466& \cellcolor[rgb]{0.8529411764705882, 0.5078431372549019, 0.5745098039215686} 0.937\\
% $\text{CoCoA}_{MTE}$ & \cellcolor[rgb]{0.8529411764705882, 0.5078431372549019, 0.5745098039215686} 0.373& \cellcolor[rgb]{0.9666666666666667, 0.907843137254902, 0.8764705882352941} 0.708& \cellcolor[rgb]{0.9411764705882353, 0.9274509803921569, 0.9196078431372549} 0.732& \cellcolor[rgb]{0.615686274509804, 0.6490196078431373, 0.8764705882352941} 0.373& \cellcolor[rgb]{0.907843137254902, 0.6392156862745098, 0.6196078431372549} 0.645& \cellcolor[rgb]{0.9627450980392157, 0.9137254901960784, 0.8862745098039215} 0.447& \cellcolor[rgb]{0.8549019607843137, 0.5176470588235295, 0.5764705882352941} 0.935\\
% \midrule

% & \multicolumn{7}{c}{Llama8b-Base} \\

% \midrule

% AveDissimilarity & \cellcolor[rgb]{0.615686274509804, 0.6490196078431373, 0.8764705882352941} 0.030& \cellcolor[rgb]{0.615686274509804, 0.6490196078431373, 0.8764705882352941} 0.473& \cellcolor[rgb]{0.615686274509804, 0.6490196078431373, 0.8764705882352941} 0.598& \cellcolor[rgb]{0.8529411764705882, 0.5078431372549019, 0.5745098039215686} 0.395& \cellcolor[rgb]{0.8529411764705882, 0.5078431372549019, 0.5745098039215686} 0.600& \cellcolor[rgb]{0.615686274509804, 0.6490196078431373, 0.8764705882352941} 0.353& \cellcolor[rgb]{0.984313725490196, 0.8470588235294118, 0.7901960784313725} 0.795\\
% $\text{CoCoA}_{MSP}$ & \cellcolor[rgb]{0.8529411764705882, 0.5078431372549019, 0.5745098039215686} 0.484& \cellcolor[rgb]{0.9705882352941176, 0.7725490196078432, 0.7156862745098039} 0.529& \cellcolor[rgb]{0.884313725490196, 0.5941176470588235, 0.5980392156862745} 0.685& \cellcolor[rgb]{0.9725490196078431, 0.7764705882352941, 0.7176470588235294} 0.384& \cellcolor[rgb]{0.9509803921568627, 0.9215686274509803, 0.9058823529411765} 0.587& \cellcolor[rgb]{0.8529411764705882, 0.5078431372549019, 0.5745098039215686} 0.452& \cellcolor[rgb]{0.615686274509804, 0.6490196078431373, 0.8764705882352941} 0.513\\
% $\text{CoCoA}_{PPL}$ & \cellcolor[rgb]{0.9019607843137255, 0.6294117647058823, 0.615686274509804} 0.450& \cellcolor[rgb]{0.8529411764705882, 0.5078431372549019, 0.5745098039215686} 0.544& \cellcolor[rgb]{0.8529411764705882, 0.5078431372549019, 0.5745098039215686} 0.689& \cellcolor[rgb]{0.8588235294117647, 0.9058823529411765, 0.9882352941176471} 0.364& \cellcolor[rgb]{0.615686274509804, 0.6490196078431373, 0.8764705882352941} 0.573& \cellcolor[rgb]{0.984313725490196, 0.8372549019607843, 0.7784313725490196} 0.422& \cellcolor[rgb]{0.8529411764705882, 0.5078431372549019, 0.5745098039215686} 0.925\\
% $\text{CoCoA}_{MTE}$ & \cellcolor[rgb]{0.9019607843137255, 0.6294117647058823, 0.615686274509804} 0.451& \cellcolor[rgb]{0.984313725490196, 0.8686274509803922, 0.8156862745098039} 0.520& \cellcolor[rgb]{0.9019607843137255, 0.9254901960784314, 0.9627450980392157} 0.638& \cellcolor[rgb]{0.615686274509804, 0.6490196078431373, 0.8764705882352941} 0.346& \cellcolor[rgb]{0.8509803921568627, 0.9019607843137255, 0.9921568627450981} 0.582& \cellcolor[rgb]{0.6745098039215687, 0.7333333333333333, 0.9450980392156862} 0.363& \cellcolor[rgb]{0.8960784313725491, 0.615686274509804, 0.6078431372549019} 0.900\\
% \midrule

% & \multicolumn{7}{c}{Falcon7b-Base} \\

% \midrule

% AveDissimilarity & \cellcolor[rgb]{0.9823529411764705, 0.8235294117647058, 0.7627450980392156} 0.282& \cellcolor[rgb]{0.615686274509804, 0.6490196078431373, 0.8764705882352941} 0.491& \cellcolor[rgb]{0.615686274509804, 0.6490196078431373, 0.8764705882352941} 0.651& \cellcolor[rgb]{0.9509803921568627, 0.9215686274509803, 0.9058823529411765} 0.416& \cellcolor[rgb]{0.615686274509804, 0.6490196078431373, 0.8764705882352941} 0.627& \cellcolor[rgb]{0.615686274509804, 0.6490196078431373, 0.8764705882352941} 0.484& \cellcolor[rgb]{0.8823529411764706, 0.5901960784313726, 0.596078431372549} 0.979\\
% $\text{CoCoA}_{MSP}$ & \cellcolor[rgb]{0.615686274509804, 0.6490196078431373, 0.8764705882352941} 0.210& \cellcolor[rgb]{0.9568627450980393, 0.7352941176470589, 0.6823529411764706} 0.564& \cellcolor[rgb]{0.984313725490196, 0.8313725490196079, 0.7725490196078432} 0.698& \cellcolor[rgb]{0.8529411764705882, 0.5078431372549019, 0.5745098039215686} 0.428& \cellcolor[rgb]{0.8529411764705882, 0.5078431372549019, 0.5745098039215686} 0.659& \cellcolor[rgb]{0.8529411764705882, 0.5078431372549019, 0.5745098039215686} 0.530& \cellcolor[rgb]{0.615686274509804, 0.6490196078431373, 0.8764705882352941} 0.498\\
% $\text{CoCoA}_{PPL}$ & \cellcolor[rgb]{0.8529411764705882, 0.5078431372549019, 0.5745098039215686} 0.310& \cellcolor[rgb]{0.8529411764705882, 0.5078431372549019, 0.5745098039215686} 0.579& \cellcolor[rgb]{0.8529411764705882, 0.5078431372549019, 0.5745098039215686} 0.717& \cellcolor[rgb]{0.9254901960784314, 0.9313725490196079, 0.9411764705882353} 0.415& \cellcolor[rgb]{0.9588235294117647, 0.9156862745098039, 0.8921568627450981} 0.644& \cellcolor[rgb]{0.8529411764705882, 0.5078431372549019, 0.5745098039215686} 0.530& \cellcolor[rgb]{0.8529411764705882, 0.5078431372549019, 0.5745098039215686} 1.000\\
% $\text{CoCoA}_{MTE}$ & \cellcolor[rgb]{0.9686274509803922, 0.7666666666666667, 0.7098039215686275} 0.289& \cellcolor[rgb]{0.984313725490196, 0.8529411764705882, 0.796078431372549} 0.551& \cellcolor[rgb]{0.8764705882352941, 0.9156862745098039, 0.9803921568627451} 0.678& \cellcolor[rgb]{0.615686274509804, 0.6490196078431373, 0.8764705882352941} 0.402& \cellcolor[rgb]{0.9745098039215686, 0.8980392156862745, 0.8588235294117647} 0.646& \cellcolor[rgb]{0.9823529411764705, 0.8196078431372549, 0.7607843137254902} 0.517& \cellcolor[rgb]{0.8529411764705882, 0.5078431372549019, 0.5745098039215686} 0.998\\
% \midrule

% \end{tabular}
% \label{tab:ablation_dissim_best}
% \end{table*}


\begin{table*}[h!]
\footnotesize
\centering

\begin{tabular}{lrrrrrrr}
\toprule
    \multirow{2}{*}{\textbf{Method}}  & \multicolumn{7}{c}{\textbf{Dataset}}  \\ 
      \cmidrule(lr){2-8}  \\
  & XSum & WMT14FrEn & WMT19DeEn & CoQa & Trivia & MMLU & GSM8k \\
  \midrule

\rowcolor[gray]{0.9} & \multicolumn{7}{c}{Mistral7b-Base} \\

\midrule

AveDissimilarity & 0.071& 0.670& 0.708& \underline{0.405}& 0.614& 0.423& 0.846\\
$\text{CoCoA}_{MSP}$ & 0.366& \underline{0.712}& 0.730& \textbf{0.430}& 0.644& \textbf{0.466}& 0.562\\
$\text{CoCoA}_{PPL}$ & \underline{0.372}& \textbf{0.735}& \textbf{0.755}& 0.402& \textbf{0.648}& \textbf{0.466}& \textbf{0.937}\\
$\text{CoCoA}_{MTE}$ & \textbf{0.373}& 0.708& \underline{0.732}& 0.373& \underline{0.645}& 0.447& \underline{0.935}\\
\midrule

\rowcolor[gray]{0.9} & \multicolumn{7}{c}{Llama8b-Base} \\

\midrule

AveDissimilarity & 0.030& 0.473& 0.598& \textbf{0.395}& \textbf{0.600}& 0.353& 0.795\\
$\text{CoCoA}_{MSP}$ & \textbf{0.484}& \underline{0.529}& \underline{0.685}& \underline{0.384}& \underline{0.587}& \textbf{0.452}& 0.513\\
$\text{CoCoA}_{PPL}$ & 0.450& \textbf{0.544}& \textbf{0.689}& 0.364& 0.573& \underline{0.422}& \textbf{0.925}\\
$\text{CoCoA}_{MTE}$ & \underline{0.451}& 0.520& 0.638& 0.346& 0.582& 0.363& \underline{0.900}\\
\midrule

\rowcolor[gray]{0.9} & \multicolumn{7}{c}{Falcon7b-Base} \\

\midrule

AveDissimilarity & 0.282& 0.491& 0.651& \underline{0.416}& 0.627& 0.484& 0.979\\
$\text{CoCoA}_{MSP}$ & 0.210& \underline{0.564}& \underline{0.698}& \textbf{0.428}& \textbf{0.659}& \textbf{0.530}& 0.498\\
$\text{CoCoA}_{PPL}$ & \textbf{0.310}& \textbf{0.579}& \textbf{0.717}& 0.415& 0.644& \textbf{0.530}& \textbf{1.000}\\
$\text{CoCoA}_{MTE}$ & \underline{0.289}& 0.551& 0.678& 0.402& \underline{0.646}& 0.517& \underline{0.998}\\
\bottomrule

\end{tabular}
\caption{Comparison of PRRs of \texttt{CoCoA}-family methods with similarity of samples with best sample taken in isolation.}
\label{tab:ablation_dissim_best}
\end{table*}



\newpage
\subsection{Choice of Similarity Function}
  \label{sec:ablation_sim_mat}
  For sample consistency estimation, one could come up with a variety of similarity functions $g(\yv, \yv')$. We perform a comparison of the effectiveness of \texttt{CoCoA}-family methods using several such functions. We consider the following functions:
  \begin{itemize}
    \item AlignScore~\cite{zha2023alignscore} with \texttt{AlignScore-large} model
    \item RougeL~\cite{lin-2004-rouge}
    \item NLI~\cite{he2021deberta} based on \texttt{microsoft/deberta-large-mnli} model
    \item CrossEncoder~\cite{liu2019robertarobustlyoptimizedbert} based on \texttt{cross-encoder/stsb-roberta-large} model.
  \end{itemize}

  Tables~\ref{suppl:ablation_sim_mat_greedy} and~\ref{suppl:ablation_sim_mat_best} report these results. There exists a considerable variation of relative effectiveness of proposed methods with various similarity function choices, depending on a task at hand. We opt to report all results in other sections with CrossEncoder-based similarity as it by itself provides a significant improvement over baselines, and for consistency and ease of comparison reasons. However, when applying these methods to a particular task, we encourage users to select appropriate underlying similarity function for best results.

\newpage

\FloatBarrier
  % \begin{table*}[h!]
% \footnotesize
% \centering
% \caption{Greedy PRRs for falcon7b, comparison between CoCoA family methods based on different choices of similarity function}
% \begin{tabular}{lrrrrrrr}
%  & XSum & WMT14FrEn & WMT19DeEn & CoQa & Trivia & MMLU & GSM8k \\

% \midrule

% & \multicolumn{7}{c}{Mistral7b-Base} \\

% \midrule

% & \multicolumn{7}{c}{$\text{CoCoA}_{MSP}$}\\
% \midrule

% AlignScore & \cellcolor[rgb]{0.8529411764705882, 0.5078431372549019, 0.5745098039215686} 0.334& \cellcolor[rgb]{0.615686274509804, 0.6490196078431373, 0.8764705882352941} 0.293& \cellcolor[rgb]{0.615686274509804, 0.6490196078431373, 0.8764705882352941} 0.445& \cellcolor[rgb]{0.615686274509804, 0.6490196078431373, 0.8764705882352941} 0.354& \cellcolor[rgb]{0.7450980392156863, 0.8137254901960784, 0.9882352941176471} 0.655& \cellcolor[rgb]{0.7019607843137254, 0.7666666666666667, 0.9666666666666667} 0.466& \cellcolor[rgb]{0.9823529411764705, 0.8196078431372549, 0.7607843137254902} 0.550\\
% RougeL & \cellcolor[rgb]{0.615686274509804, 0.6490196078431373, 0.8764705882352941} 0.289& \cellcolor[rgb]{0.9803921568627451, 0.884313725490196, 0.8372549019607843} 0.358& \cellcolor[rgb]{0.984313725490196, 0.8588235294117647, 0.803921568627451} 0.546& \cellcolor[rgb]{0.8235294117647058, 0.8823529411764706, 0.9980392156862745} 0.369& \cellcolor[rgb]{0.615686274509804, 0.6490196078431373, 0.8764705882352941} 0.649& \cellcolor[rgb]{0.615686274509804, 0.6490196078431373, 0.8764705882352941} 0.466& \cellcolor[rgb]{0.9333333333333333, 0.9313725490196079, 0.9313725490196079} 0.536\\
% NLI & \cellcolor[rgb]{0.888235294117647, 0.9196078431372549, 0.9725490196078431} 0.308& \cellcolor[rgb]{0.7392156862745098, 0.807843137254902, 0.9862745098039216} 0.313& \cellcolor[rgb]{0.7470588235294118, 0.8156862745098039, 0.9901960784313726} 0.477& \cellcolor[rgb]{0.8529411764705882, 0.5078431372549019, 0.5745098039215686} 0.403& \cellcolor[rgb]{0.8529411764705882, 0.5078431372549019, 0.5745098039215686} 0.677& \cellcolor[rgb]{0.8529411764705882, 0.5078431372549019, 0.5745098039215686} 0.470& \cellcolor[rgb]{0.8529411764705882, 0.5078431372549019, 0.5745098039215686} 0.568\\
% CrossEncoder & \cellcolor[rgb]{0.9450980392156862, 0.711764705882353, 0.6666666666666666} 0.327& \cellcolor[rgb]{0.8529411764705882, 0.5078431372549019, 0.5745098039215686} 0.397& \cellcolor[rgb]{0.8529411764705882, 0.5078431372549019, 0.5745098039215686} 0.595& \cellcolor[rgb]{0.9588235294117647, 0.9156862745098039, 0.8921568627450981} 0.381& \cellcolor[rgb]{0.9686274509803922, 0.7666666666666667, 0.7098039215686275} 0.671& \cellcolor[rgb]{0.6254901960784314, 0.6666666666666666, 0.8921568627450981} 0.466& \cellcolor[rgb]{0.615686274509804, 0.6490196078431373, 0.8764705882352941} 0.505\\

% \midrule

% & \multicolumn{7}{c}{$\text{CoCoA}_{PPL}$}\\
% \midrule

% AlignScore & \cellcolor[rgb]{0.8529411764705882, 0.5078431372549019, 0.5745098039215686} 0.307& \cellcolor[rgb]{0.615686274509804, 0.6490196078431373, 0.8764705882352941} 0.308& \cellcolor[rgb]{0.615686274509804, 0.6490196078431373, 0.8764705882352941} 0.489& \cellcolor[rgb]{0.9372549019607843, 0.6941176470588235, 0.6529411764705882} 0.373& \cellcolor[rgb]{0.9117647058823529, 0.9294117647058824, 0.9549019607843138} 0.666& \cellcolor[rgb]{0.7019607843137254, 0.7666666666666667, 0.9666666666666667} 0.466& \cellcolor[rgb]{0.8529411764705882, 0.5078431372549019, 0.5745098039215686} 0.536\\
% RougeL & \cellcolor[rgb]{0.615686274509804, 0.6490196078431373, 0.8764705882352941} 0.226& \cellcolor[rgb]{0.8725490196078431, 0.5705882352941176, 0.5901960784313726} 0.369& \cellcolor[rgb]{0.9588235294117647, 0.9156862745098039, 0.8921568627450981} 0.531& \cellcolor[rgb]{0.615686274509804, 0.6490196078431373, 0.8764705882352941} 0.352& \cellcolor[rgb]{0.615686274509804, 0.6490196078431373, 0.8764705882352941} 0.653& \cellcolor[rgb]{0.615686274509804, 0.6490196078431373, 0.8764705882352941} 0.466& \cellcolor[rgb]{0.6176470588235294, 0.6529411764705882, 0.8803921568627451} 0.466\\
% NLI & \cellcolor[rgb]{0.6666666666666666, 0.7235294117647059, 0.9372549019607843} 0.233& \cellcolor[rgb]{0.692156862745098, 0.7549019607843137, 0.9588235294117647} 0.316& \cellcolor[rgb]{0.707843137254902, 0.7725490196078432, 0.9686274509803922} 0.501& \cellcolor[rgb]{0.8529411764705882, 0.5078431372549019, 0.5745098039215686} 0.376& \cellcolor[rgb]{0.8529411764705882, 0.5078431372549019, 0.5745098039215686} 0.682& \cellcolor[rgb]{0.8529411764705882, 0.5078431372549019, 0.5745098039215686} 0.470& \cellcolor[rgb]{0.7450980392156863, 0.8137254901960784, 0.9882352941176471} 0.480\\
% CrossEncoder & \cellcolor[rgb]{0.984313725490196, 0.8529411764705882, 0.796078431372549} 0.281& \cellcolor[rgb]{0.8529411764705882, 0.5078431372549019, 0.5745098039215686} 0.371& \cellcolor[rgb]{0.8529411764705882, 0.5078431372549019, 0.5745098039215686} 0.565& \cellcolor[rgb]{0.9450980392156862, 0.9254901960784314, 0.9137254901960784} 0.365& \cellcolor[rgb]{0.984313725490196, 0.8294117647058824, 0.7686274509803922} 0.674& \cellcolor[rgb]{0.6254901960784314, 0.6666666666666666, 0.8921568627450981} 0.466& \cellcolor[rgb]{0.615686274509804, 0.6490196078431373, 0.8764705882352941} 0.465\\

% \midrule

% & \multicolumn{7}{c}{$\text{CoCoA}_{MTE}$}\\
% \midrule

% AlignScore & \cellcolor[rgb]{0.8529411764705882, 0.5078431372549019, 0.5745098039215686} 0.302& \cellcolor[rgb]{0.615686274509804, 0.6490196078431373, 0.8764705882352941} 0.299& \cellcolor[rgb]{0.615686274509804, 0.6490196078431373, 0.8764705882352941} 0.477& \cellcolor[rgb]{0.8529411764705882, 0.5078431372549019, 0.5745098039215686} 0.366& \cellcolor[rgb]{0.888235294117647, 0.9196078431372549, 0.9725490196078431} 0.664& \cellcolor[rgb]{0.8509803921568627, 0.9019607843137255, 0.9921568627450981} 0.450& \cellcolor[rgb]{0.8529411764705882, 0.5078431372549019, 0.5745098039215686} 0.555\\
% RougeL & \cellcolor[rgb]{0.615686274509804, 0.6490196078431373, 0.8764705882352941} 0.212& \cellcolor[rgb]{0.8529411764705882, 0.5078431372549019, 0.5745098039215686} 0.377& \cellcolor[rgb]{0.9745098039215686, 0.8941176470588236, 0.8529411764705882} 0.528& \cellcolor[rgb]{0.615686274509804, 0.6490196078431373, 0.8764705882352941} 0.345& \cellcolor[rgb]{0.615686274509804, 0.6490196078431373, 0.8764705882352941} 0.652& \cellcolor[rgb]{0.7392156862745098, 0.807843137254902, 0.9862745098039216} 0.449& \cellcolor[rgb]{0.7098039215686275, 0.7764705882352941, 0.9705882352941176} 0.497\\
% NLI & \cellcolor[rgb]{0.6627450980392157, 0.7156862745098039, 0.9333333333333333} 0.219& \cellcolor[rgb]{0.7313725490196078, 0.8, 0.9823529411764705} 0.313& \cellcolor[rgb]{0.6941176470588235, 0.7588235294117647, 0.9607843137254902} 0.488& \cellcolor[rgb]{0.9666666666666667, 0.7588235294117647, 0.703921568627451} 0.362& \cellcolor[rgb]{0.8529411764705882, 0.5078431372549019, 0.5745098039215686} 0.681& \cellcolor[rgb]{0.8529411764705882, 0.5078431372549019, 0.5745098039215686} 0.453& \cellcolor[rgb]{0.6470588235294118, 0.696078431372549, 0.9176470588235294} 0.490\\
% CrossEncoder & \cellcolor[rgb]{0.9725490196078431, 0.7764705882352941, 0.7176470588235294} 0.282& \cellcolor[rgb]{0.9313725490196079, 0.6823529411764706, 0.6450980392156863} 0.368& \cellcolor[rgb]{0.8529411764705882, 0.5078431372549019, 0.5745098039215686} 0.560& \cellcolor[rgb]{0.8176470588235294, 0.8784313725490196, 1.0} 0.351& \cellcolor[rgb]{0.984313725490196, 0.8333333333333334, 0.7745098039215687} 0.673& \cellcolor[rgb]{0.615686274509804, 0.6490196078431373, 0.8764705882352941} 0.448& \cellcolor[rgb]{0.615686274509804, 0.6490196078431373, 0.8764705882352941} 0.486\\
% \midrule

% & \multicolumn{7}{c}{Llama8b-Base} \\

% \midrule

% & \multicolumn{7}{c}{$\text{CoCoA}_{MSP}$}\\
% \midrule

% AlignScore & \cellcolor[rgb]{0.9666666666666667, 0.7627450980392156, 0.7058823529411765} 0.367& \cellcolor[rgb]{0.615686274509804, 0.6490196078431373, 0.8764705882352941} 0.331& \cellcolor[rgb]{0.615686274509804, 0.6490196078431373, 0.8764705882352941} 0.452& \cellcolor[rgb]{0.615686274509804, 0.6490196078431373, 0.8764705882352941} 0.308& \cellcolor[rgb]{0.9764705882352941, 0.792156862745098, 0.7333333333333333} 0.596& \cellcolor[rgb]{0.8529411764705882, 0.5078431372549019, 0.5745098039215686} 0.484& \cellcolor[rgb]{0.984313725490196, 0.8686274509803922, 0.8156862745098039} 0.401\\
% RougeL & \cellcolor[rgb]{0.615686274509804, 0.6490196078431373, 0.8764705882352941} 0.336& \cellcolor[rgb]{0.9352941176470588, 0.9313725490196079, 0.9294117647058824} 0.393& \cellcolor[rgb]{0.984313725490196, 0.8313725490196079, 0.7725490196078432} 0.545& \cellcolor[rgb]{0.7686274509803922, 0.8372549019607843, 0.996078431372549} 0.321& \cellcolor[rgb]{0.615686274509804, 0.6490196078431373, 0.8764705882352941} 0.563& \cellcolor[rgb]{0.615686274509804, 0.6490196078431373, 0.8764705882352941} 0.474& \cellcolor[rgb]{0.703921568627451, 0.7705882352941177, 0.9686274509803922} 0.375\\
% NLI & \cellcolor[rgb]{0.7509803921568627, 0.8196078431372549, 0.9901960784313726} 0.344& \cellcolor[rgb]{0.7235294117647059, 0.792156862745098, 0.9784313725490196} 0.352& \cellcolor[rgb]{0.6843137254901961, 0.7450980392156863, 0.9529411764705882} 0.467& \cellcolor[rgb]{0.8529411764705882, 0.5078431372549019, 0.5745098039215686} 0.364& \cellcolor[rgb]{0.8529411764705882, 0.5078431372549019, 0.5745098039215686} 0.606& \cellcolor[rgb]{0.8588235294117647, 0.9058823529411765, 0.9882352941176471} 0.478& \cellcolor[rgb]{0.8529411764705882, 0.5078431372549019, 0.5745098039215686} 0.419\\
% CrossEncoder & \cellcolor[rgb]{0.8529411764705882, 0.5078431372549019, 0.5745098039215686} 0.375& \cellcolor[rgb]{0.8529411764705882, 0.5078431372549019, 0.5745098039215686} 0.454& \cellcolor[rgb]{0.8529411764705882, 0.5078431372549019, 0.5745098039215686} 0.583& \cellcolor[rgb]{0.9784313725490196, 0.8019607843137255, 0.7411764705882353} 0.350& \cellcolor[rgb]{0.9627450980392157, 0.7490196078431373, 0.6941176470588235} 0.598& \cellcolor[rgb]{0.9764705882352941, 0.8921568627450981, 0.8490196078431372} 0.480& \cellcolor[rgb]{0.615686274509804, 0.6490196078431373, 0.8764705882352941} 0.367\\

% \midrule

% & \multicolumn{7}{c}{$\text{CoCoA}_{PPL}$}\\
% \midrule

% AlignScore & \cellcolor[rgb]{0.8529411764705882, 0.5078431372549019, 0.5745098039215686} 0.422& \cellcolor[rgb]{0.615686274509804, 0.6490196078431373, 0.8764705882352941} 0.346& \cellcolor[rgb]{0.707843137254902, 0.7725490196078432, 0.9686274509803922} 0.450& \cellcolor[rgb]{0.9803921568627451, 0.8764705882352941, 0.8274509803921568} 0.337& \cellcolor[rgb]{0.9156862745098039, 0.6529411764705882, 0.6274509803921569} 0.596& \cellcolor[rgb]{0.8529411764705882, 0.5078431372549019, 0.5745098039215686} 0.453& \cellcolor[rgb]{0.8529411764705882, 0.5078431372549019, 0.5745098039215686} 0.446\\
% RougeL & \cellcolor[rgb]{0.615686274509804, 0.6490196078431373, 0.8764705882352941} 0.370& \cellcolor[rgb]{0.9803921568627451, 0.8803921568627451, 0.8333333333333334} 0.408& \cellcolor[rgb]{0.9803921568627451, 0.884313725490196, 0.8372549019607843} 0.486& \cellcolor[rgb]{0.615686274509804, 0.6490196078431373, 0.8764705882352941} 0.319& \cellcolor[rgb]{0.615686274509804, 0.6490196078431373, 0.8764705882352941} 0.552& \cellcolor[rgb]{0.615686274509804, 0.6490196078431373, 0.8764705882352941} 0.441& \cellcolor[rgb]{0.7666666666666667, 0.8352941176470589, 0.996078431372549} 0.418\\
% NLI & \cellcolor[rgb]{0.6529411764705882, 0.703921568627451, 0.9235294117647059} 0.374& \cellcolor[rgb]{0.6666666666666666, 0.7235294117647059, 0.9372549019607843} 0.354& \cellcolor[rgb]{0.615686274509804, 0.6490196078431373, 0.8764705882352941} 0.438& \cellcolor[rgb]{0.8529411764705882, 0.5078431372549019, 0.5745098039215686} 0.348& \cellcolor[rgb]{0.8529411764705882, 0.5078431372549019, 0.5745098039215686} 0.600& \cellcolor[rgb]{0.8490196078431372, 0.9, 0.9921568627450981} 0.446& \cellcolor[rgb]{0.615686274509804, 0.6490196078431373, 0.8764705882352941} 0.409\\
% CrossEncoder & \cellcolor[rgb]{0.7372549019607844, 0.8058823529411765, 0.9862745098039216} 0.380& \cellcolor[rgb]{0.8529411764705882, 0.5078431372549019, 0.5745098039215686} 0.444& \cellcolor[rgb]{0.8529411764705882, 0.5078431372549019, 0.5745098039215686} 0.514& \cellcolor[rgb]{0.984313725490196, 0.8529411764705882, 0.796078431372549} 0.339& \cellcolor[rgb]{0.9509803921568627, 0.7235294117647059, 0.6745098039215687} 0.593& \cellcolor[rgb]{0.9294117647058824, 0.9313725490196079, 0.9352941176470588} 0.447& \cellcolor[rgb]{0.9490196078431372, 0.9235294117647059, 0.9098039215686274} 0.429\\

% \midrule

% & \multicolumn{7}{c}{$\text{CoCoA}_{MTE}$}\\
% \midrule

% AlignScore & \cellcolor[rgb]{0.8529411764705882, 0.5078431372549019, 0.5745098039215686} 0.419& \cellcolor[rgb]{0.615686274509804, 0.6490196078431373, 0.8764705882352941} 0.340& \cellcolor[rgb]{0.6862745098039216, 0.7490196078431373, 0.9549019607843138} 0.438& \cellcolor[rgb]{0.8784313725490196, 0.5843137254901961, 0.5941176470588235} 0.339& \cellcolor[rgb]{0.9313725490196079, 0.6823529411764706, 0.6450980392156863} 0.605& \cellcolor[rgb]{0.8529411764705882, 0.5078431372549019, 0.5745098039215686} 0.411& \cellcolor[rgb]{0.8529411764705882, 0.5078431372549019, 0.5745098039215686} 0.459\\
% RougeL & \cellcolor[rgb]{0.615686274509804, 0.6490196078431373, 0.8764705882352941} 0.362& \cellcolor[rgb]{0.9764705882352941, 0.792156862745098, 0.7333333333333333} 0.417& \cellcolor[rgb]{0.9803921568627451, 0.8764705882352941, 0.8274509803921568} 0.481& \cellcolor[rgb]{0.615686274509804, 0.6490196078431373, 0.8764705882352941} 0.319& \cellcolor[rgb]{0.615686274509804, 0.6490196078431373, 0.8764705882352941} 0.560& \cellcolor[rgb]{0.615686274509804, 0.6490196078431373, 0.8764705882352941} 0.390& \cellcolor[rgb]{0.9294117647058824, 0.9313725490196079, 0.9352941176470588} 0.440\\
% NLI & \cellcolor[rgb]{0.6588235294117647, 0.7137254901960784, 0.9294117647058824} 0.366& \cellcolor[rgb]{0.6235294117647059, 0.6627450980392157, 0.888235294117647} 0.342& \cellcolor[rgb]{0.615686274509804, 0.6490196078431373, 0.8764705882352941} 0.428& \cellcolor[rgb]{0.8529411764705882, 0.5078431372549019, 0.5745098039215686} 0.340& \cellcolor[rgb]{0.8529411764705882, 0.5078431372549019, 0.5745098039215686} 0.612& \cellcolor[rgb]{0.807843137254902, 0.8705882352941177, 1.0} 0.396& \cellcolor[rgb]{0.615686274509804, 0.6490196078431373, 0.8764705882352941} 0.420\\
% CrossEncoder & \cellcolor[rgb]{0.7411764705882353, 0.8117647058823529, 0.9882352941176471} 0.374& \cellcolor[rgb]{0.8529411764705882, 0.5078431372549019, 0.5745098039215686} 0.441& \cellcolor[rgb]{0.8529411764705882, 0.5078431372549019, 0.5745098039215686} 0.511& \cellcolor[rgb]{0.9254901960784314, 0.6725490196078432, 0.6392156862745098} 0.337& \cellcolor[rgb]{0.9686274509803922, 0.7666666666666667, 0.7098039215686275} 0.601& \cellcolor[rgb]{0.7176470588235294, 0.7862745098039216, 0.9764705882352941} 0.394& \cellcolor[rgb]{0.9784313725490196, 0.8862745098039215, 0.8392156862745098} 0.444\\
% \midrule

% & \multicolumn{7}{c}{Falcon7b-Base} \\

% \midrule

% & \multicolumn{7}{c}{$\text{CoCoA}_{MSP}$}\\
% \midrule

% AlignScore & \cellcolor[rgb]{0.8529411764705882, 0.5078431372549019, 0.5745098039215686} 0.278& \cellcolor[rgb]{0.615686274509804, 0.6490196078431373, 0.8764705882352941} 0.306& \cellcolor[rgb]{0.615686274509804, 0.6490196078431373, 0.8764705882352941} 0.475& \cellcolor[rgb]{0.615686274509804, 0.6490196078431373, 0.8764705882352941} 0.361& \cellcolor[rgb]{0.615686274509804, 0.6490196078431373, 0.8764705882352941} 0.677& \cellcolor[rgb]{0.6588235294117647, 0.7137254901960784, 0.9294117647058824} 0.528& \cellcolor[rgb]{0.9117647058823529, 0.9294117647058824, 0.9549019607843138} 0.470\\
% RougeL & \cellcolor[rgb]{0.615686274509804, 0.6490196078431373, 0.8764705882352941} 0.205& \cellcolor[rgb]{0.984313725490196, 0.8529411764705882, 0.796078431372549} 0.394& \cellcolor[rgb]{0.7686274509803922, 0.8372549019607843, 0.996078431372549} 0.499& \cellcolor[rgb]{0.8647058823529412, 0.907843137254902, 0.9862745098039216} 0.378& \cellcolor[rgb]{0.6843137254901961, 0.7450980392156863, 0.9529411764705882} 0.678& \cellcolor[rgb]{0.615686274509804, 0.6490196078431373, 0.8764705882352941} 0.527& \cellcolor[rgb]{0.615686274509804, 0.6490196078431373, 0.8764705882352941} 0.417\\
% NLI & \cellcolor[rgb]{0.8960784313725491, 0.9235294117647059, 0.9686274509803922} 0.236& \cellcolor[rgb]{0.8901960784313725, 0.9215686274509803, 0.9705882352941176} 0.361& \cellcolor[rgb]{0.8431372549019608, 0.8960784313725491, 0.9941176470588236} 0.511& \cellcolor[rgb]{0.8529411764705882, 0.5078431372549019, 0.5745098039215686} 0.407& \cellcolor[rgb]{0.9470588235294117, 0.7156862745098039, 0.6686274509803921} 0.684& \cellcolor[rgb]{0.8529411764705882, 0.5078431372549019, 0.5745098039215686} 0.532& \cellcolor[rgb]{0.8529411764705882, 0.5078431372549019, 0.5745098039215686} 0.532\\
% CrossEncoder & \cellcolor[rgb]{0.984313725490196, 0.8647058823529412, 0.8117647058823529} 0.253& \cellcolor[rgb]{0.8529411764705882, 0.5078431372549019, 0.5745098039215686} 0.436& \cellcolor[rgb]{0.8529411764705882, 0.5078431372549019, 0.5745098039215686} 0.577& \cellcolor[rgb]{0.9745098039215686, 0.7823529411764706, 0.7235294117647059} 0.396& \cellcolor[rgb]{0.8529411764705882, 0.5078431372549019, 0.5745098039215686} 0.685& \cellcolor[rgb]{0.903921568627451, 0.9274509803921569, 0.9607843137254902} 0.529& \cellcolor[rgb]{0.6764705882352942, 0.7352941176470589, 0.9470588235294117} 0.428\\

% \midrule

% & \multicolumn{7}{c}{$\text{CoCoA}_{PPL}$}\\
% \midrule

% AlignScore & \cellcolor[rgb]{0.8529411764705882, 0.5078431372549019, 0.5745098039215686} 0.252& \cellcolor[rgb]{0.615686274509804, 0.6490196078431373, 0.8764705882352941} 0.340& \cellcolor[rgb]{0.615686274509804, 0.6490196078431373, 0.8764705882352941} 0.523& \cellcolor[rgb]{0.8529411764705882, 0.5078431372549019, 0.5745098039215686} 0.410& \cellcolor[rgb]{0.9725490196078431, 0.7803921568627451, 0.7215686274509804} 0.678& \cellcolor[rgb]{0.6588235294117647, 0.7137254901960784, 0.9294117647058824} 0.528& \cellcolor[rgb]{0.8529411764705882, 0.5078431372549019, 0.5745098039215686} 0.521\\
% RougeL & \cellcolor[rgb]{0.615686274509804, 0.6490196078431373, 0.8764705882352941} 0.170& \cellcolor[rgb]{0.984313725490196, 0.8254901960784313, 0.7666666666666667} 0.409& \cellcolor[rgb]{0.7764705882352941, 0.8450980392156863, 0.9980392156862745} 0.537& \cellcolor[rgb]{0.615686274509804, 0.6490196078431373, 0.8764705882352941} 0.389& \cellcolor[rgb]{0.615686274509804, 0.6490196078431373, 0.8764705882352941} 0.668& \cellcolor[rgb]{0.615686274509804, 0.6490196078431373, 0.8764705882352941} 0.527& \cellcolor[rgb]{0.615686274509804, 0.6490196078431373, 0.8764705882352941} 0.439\\
% NLI & \cellcolor[rgb]{0.796078431372549, 0.8607843137254902, 1.0} 0.193& \cellcolor[rgb]{0.7803921568627451, 0.8470588235294118, 0.9980392156862745} 0.364& \cellcolor[rgb]{0.7, 0.7647058823529411, 0.9647058823529412} 0.531& \cellcolor[rgb]{0.9117647058823529, 0.6470588235294118, 0.6254901960784314} 0.408& \cellcolor[rgb]{0.8529411764705882, 0.5078431372549019, 0.5745098039215686} 0.680& \cellcolor[rgb]{0.8529411764705882, 0.5078431372549019, 0.5745098039215686} 0.532& \cellcolor[rgb]{0.9823529411764705, 0.8196078431372549, 0.7607843137254902} 0.499\\
% CrossEncoder & \cellcolor[rgb]{0.984313725490196, 0.8509803921568627, 0.7941176470588235} 0.226& \cellcolor[rgb]{0.8529411764705882, 0.5078431372549019, 0.5745098039215686} 0.437& \cellcolor[rgb]{0.8529411764705882, 0.5078431372549019, 0.5745098039215686} 0.579& \cellcolor[rgb]{0.9803921568627451, 0.807843137254902, 0.7470588235294118} 0.405& \cellcolor[rgb]{0.9764705882352941, 0.796078431372549, 0.7352941176470589} 0.677& \cellcolor[rgb]{0.903921568627451, 0.9274509803921569, 0.9607843137254902} 0.529& \cellcolor[rgb]{0.8901960784313725, 0.9215686274509803, 0.9705882352941176} 0.474\\

% \midrule

% & \multicolumn{7}{c}{$\text{CoCoA}_{MTE}$}\\
% \midrule

% AlignScore & \cellcolor[rgb]{0.8529411764705882, 0.5078431372549019, 0.5745098039215686} 0.253& \cellcolor[rgb]{0.615686274509804, 0.6490196078431373, 0.8764705882352941} 0.337& \cellcolor[rgb]{0.615686274509804, 0.6490196078431373, 0.8764705882352941} 0.519& \cellcolor[rgb]{0.8529411764705882, 0.5078431372549019, 0.5745098039215686} 0.403& \cellcolor[rgb]{0.9784313725490196, 0.8019607843137255, 0.7411764705882353} 0.683& \cellcolor[rgb]{0.6529411764705882, 0.703921568627451, 0.9235294117647059} 0.515& \cellcolor[rgb]{0.8529411764705882, 0.5078431372549019, 0.5745098039215686} 0.554\\
% RougeL & \cellcolor[rgb]{0.615686274509804, 0.6490196078431373, 0.8764705882352941} 0.170& \cellcolor[rgb]{0.9274509803921569, 0.6764705882352942, 0.6431372549019608} 0.426& \cellcolor[rgb]{0.8627450980392157, 0.907843137254902, 0.9882352941176471} 0.540& \cellcolor[rgb]{0.615686274509804, 0.6490196078431373, 0.8764705882352941} 0.382& \cellcolor[rgb]{0.615686274509804, 0.6490196078431373, 0.8764705882352941} 0.673& \cellcolor[rgb]{0.615686274509804, 0.6490196078431373, 0.8764705882352941} 0.514& \cellcolor[rgb]{0.615686274509804, 0.6490196078431373, 0.8764705882352941} 0.472\\
% NLI & \cellcolor[rgb]{0.7686274509803922, 0.8372549019607843, 0.996078431372549} 0.190& \cellcolor[rgb]{0.788235294117647, 0.8549019607843137, 0.9980392156862745} 0.364& \cellcolor[rgb]{0.6843137254901961, 0.7450980392156863, 0.9529411764705882} 0.525& \cellcolor[rgb]{0.9745098039215686, 0.7862745098039216, 0.7274509803921568} 0.398& \cellcolor[rgb]{0.8529411764705882, 0.5078431372549019, 0.5745098039215686} 0.687& \cellcolor[rgb]{0.8529411764705882, 0.5078431372549019, 0.5745098039215686} 0.521& \cellcolor[rgb]{0.9392156862745098, 0.9294117647058824, 0.9235294117647059} 0.514\\
% CrossEncoder & \cellcolor[rgb]{0.9823529411764705, 0.8725490196078431, 0.8215686274509804} 0.223& \cellcolor[rgb]{0.8529411764705882, 0.5078431372549019, 0.5745098039215686} 0.438& \cellcolor[rgb]{0.8529411764705882, 0.5078431372549019, 0.5745098039215686} 0.575& \cellcolor[rgb]{0.9803921568627451, 0.8803921568627451, 0.8333333333333334} 0.395& \cellcolor[rgb]{0.9352941176470588, 0.6901960784313725, 0.6509803921568628} 0.685& \cellcolor[rgb]{0.8823529411764706, 0.9176470588235294, 0.9784313725490196} 0.517& \cellcolor[rgb]{0.8764705882352941, 0.9156862745098039, 0.9803921568627451} 0.505\\
% \midrule

% \end{tabular}
% \label{tab:ablation_sim_mat_greedy}
% \end{table*}

\begin{table*}[th!]
\footnotesize
\centering

\begin{tabular}{lrrrrrrr}
\toprule
    \multirow{2}{*}{\textbf{Method}}  & \multicolumn{7}{c}{\textbf{Dataset}}  \\ 
      \cmidrule(lr){2-8}  \\
  & XSum & WMT14FrEn & WMT19DeEn & CoQa & Trivia & MMLU & GSM8k \\
  \midrule

\rowcolor[gray]{0.9} & \multicolumn{7}{c}{Mistral7b-Base} \\

\midrule

& \multicolumn{7}{c}{$\text{CoCoA}_{MSP}$}\\
\midrule

AlignScore & \textbf{0.334}& 0.293& 0.445& 0.354& 0.655& 0.466& 0.550\\
RougeL & 0.289& 0.358& 0.546& 0.369& 0.649& 0.466& 0.536\\
NLI & 0.308& 0.313& 0.477& \textbf{0.403}& 0.677& \textbf{0.470}& \textbf{0.568}\\
CrossEncoder & \underline{0.327}& \textbf{0.397}& \textbf{0.595}& \underline{0.381}& 0.671& 0.466& 0.505\\

\midrule

& \multicolumn{7}{c}{$\text{CoCoA}_{PPL}$}\\
\midrule

AlignScore & 0.307& 0.308& 0.489& 0.373& 0.666& 0.466& 0.536\\
RougeL & 0.226& 0.369& 0.531& 0.352& 0.653& 0.466& 0.466\\
NLI & 0.233& 0.316& 0.501& 0.376& \textbf{0.682}& \textbf{0.470}& 0.480\\
CrossEncoder & 0.281& 0.371& \underline{0.565}& 0.365& 0.674& 0.466& 0.465\\

\midrule

& \multicolumn{7}{c}{$\text{CoCoA}_{MTE}$}\\
\midrule

AlignScore & 0.302& 0.299& 0.477& 0.366& 0.664& 0.450& \underline{0.555}\\
RougeL & 0.212& \underline{0.377}& 0.528& 0.345& 0.652& 0.449& 0.497\\
NLI & 0.219& 0.313& 0.488& 0.362& \underline{0.681}& 0.453& 0.490\\
CrossEncoder & 0.282& 0.368& 0.560& 0.351& 0.673& 0.448& 0.486\\
\midrule

\rowcolor[gray]{0.9} & \multicolumn{7}{c}{Llama8b-Base} \\

\midrule

& \multicolumn{7}{c}{$\text{CoCoA}_{MSP}$}\\
\midrule

AlignScore & 0.367& 0.331& 0.452& 0.308& 0.596& \textbf{0.484}& 0.401\\
RougeL & 0.336& 0.393& \underline{0.545}& 0.321& 0.563& 0.474& 0.375\\
NLI & 0.344& 0.352& 0.467& \textbf{0.364}& \underline{0.606}& 0.478& 0.419\\
CrossEncoder & 0.375& \textbf{0.454}& \textbf{0.583}& \underline{0.350}& 0.598& \underline{0.480}& 0.367\\

\midrule

& \multicolumn{7}{c}{$\text{CoCoA}_{PPL}$}\\
\midrule

AlignScore & \textbf{0.422}& 0.346& 0.450& 0.337& 0.596& 0.453& \underline{0.446}\\
RougeL & 0.370& 0.408& 0.486& 0.319& 0.552& 0.441& 0.418\\
NLI & 0.374& 0.354& 0.438& 0.348& 0.600& 0.446& 0.409\\
CrossEncoder & 0.380& \underline{0.444}& 0.514& 0.339& 0.593& 0.447& 0.429\\

\midrule

& \multicolumn{7}{c}{$\text{CoCoA}_{MTE}$}\\
\midrule

AlignScore & \underline{0.419}& 0.340& 0.438& 0.339& 0.605& 0.411& \textbf{0.459}\\
RougeL & 0.362& 0.417& 0.481& 0.319& 0.560& 0.390& 0.440\\
NLI & 0.366& 0.342& 0.428& 0.340& \textbf{0.612}& 0.396& 0.420\\
CrossEncoder & 0.374& 0.441& 0.511& 0.337& 0.601& 0.394& 0.444\\
\midrule

\rowcolor[gray]{0.9} & \multicolumn{7}{c}{Falcon7b-Base} \\

\midrule

& \multicolumn{7}{c}{$\text{CoCoA}_{MSP}$}\\
\midrule

AlignScore & \textbf{0.278}& 0.306& 0.475& 0.361& 0.677& 0.528& 0.470\\
RougeL & 0.205& 0.394& 0.499& 0.378& 0.678& 0.527& 0.417\\
NLI & 0.236& 0.361& 0.511& 0.407& 0.684& \textbf{\underline{0.532}}& 0.532\\
CrossEncoder & \underline{0.253}& 0.436& \underline{0.577}& 0.396& \underline{0.685}& 0.529& 0.428\\

\midrule

& \multicolumn{7}{c}{$\text{CoCoA}_{PPL}$}\\
\midrule

AlignScore & 0.252& 0.340& 0.523& \textbf{0.410}& 0.678& 0.528& 0.521\\
RougeL & 0.170& 0.409& 0.537& 0.389& 0.668& 0.527& 0.439\\
NLI & 0.193& 0.364& 0.531& \underline{0.408}& 0.680& \textbf{0.532}& 0.499\\
CrossEncoder & 0.226& \underline{0.437}& \textbf{0.579}& 0.405& 0.677& 0.529& 0.474\\

\midrule

& \multicolumn{7}{c}{$\text{CoCoA}_{MTE}$}\\
\midrule

AlignScore & \underline{0.253}& 0.337& 0.519& 0.403& 0.683& 0.515& \textbf{0.554}\\
RougeL & 0.170& 0.426& 0.540& 0.382& 0.673& 0.514& 0.472\\
NLI & 0.190& 0.364& 0.525& 0.398& \textbf{0.687}& 0.521& 0.514\\
CrossEncoder & 0.223& \textbf{0.438}& 0.575& 0.395& \underline{0.685}& 0.517& 0.505\\
\bottomrule

\end{tabular}
% \subsection{}
\caption{Comparison of PRRs of \texttt{CoCoA}-family methods with different choices of similarity function with greedy sample taken in isolation.}
\label{suppl:ablation_sim_mat_greedy}
\end{table*}

\FloatBarrier

\FloatBarrier
  \newpage
  % \begin{table*}
% \footnotesize
% \centering
% \caption{Best Sample PRRs for falcon7b, comparison between CoCoA family methods based on different choices of similarity function}
% \begin{tabular}{lrrrrrrr}
%  & XSum & WMT14FrEn & WMT19DeEn & CoQa & Trivia & MMLU & GSM8k \\

% \midrule

% & \multicolumn{7}{c}{Mistral7b-Base} \\

% \midrule

% & \multicolumn{7}{c}{$\text{CoCoA}_{MSP}$}\\
% \midrule

% AlignScore & \cellcolor[rgb]{0.8529411764705882, 0.5078431372549019, 0.5745098039215686} 0.393& \cellcolor[rgb]{0.615686274509804, 0.6490196078431373, 0.8764705882352941} 0.448& \cellcolor[rgb]{0.615686274509804, 0.6490196078431373, 0.8764705882352941} 0.491& \cellcolor[rgb]{0.615686274509804, 0.6490196078431373, 0.8764705882352941} 0.399& \cellcolor[rgb]{0.6941176470588235, 0.7588235294117647, 0.9607843137254902} 0.626& \cellcolor[rgb]{0.6784313725490196, 0.7392156862745098, 0.9490196078431372} 0.467& \cellcolor[rgb]{0.7313725490196078, 0.8, 0.9823529411764705} 0.476\\
% RougeL & \cellcolor[rgb]{0.6647058823529411, 0.7196078431372549, 0.9352941176470588} 0.344& \cellcolor[rgb]{0.9686274509803922, 0.903921568627451, 0.8705882352941177} 0.602& \cellcolor[rgb]{0.9, 0.9254901960784314, 0.9647058823529412} 0.597& \cellcolor[rgb]{0.9117647058823529, 0.9294117647058824, 0.9549019607843138} 0.420& \cellcolor[rgb]{0.615686274509804, 0.6490196078431373, 0.8764705882352941} 0.622& \cellcolor[rgb]{0.615686274509804, 0.6490196078431373, 0.8764705882352941} 0.466& \cellcolor[rgb]{0.9745098039215686, 0.7862745098039216, 0.7274509803921568} 0.538\\
% NLI & \cellcolor[rgb]{0.615686274509804, 0.6490196078431373, 0.8764705882352941} 0.340& \cellcolor[rgb]{0.9803921568627451, 0.8803921568627451, 0.8333333333333334} 0.615& \cellcolor[rgb]{0.9176470588235294, 0.9294117647058824, 0.9490196078431372} 0.604& \cellcolor[rgb]{0.8529411764705882, 0.5078431372549019, 0.5745098039215686} 0.445& \cellcolor[rgb]{0.8529411764705882, 0.5078431372549019, 0.5745098039215686} 0.651& \cellcolor[rgb]{0.8529411764705882, 0.5078431372549019, 0.5745098039215686} 0.470& \cellcolor[rgb]{0.615686274509804, 0.6490196078431373, 0.8764705882352941} 0.456\\
% CrossEncoder & \cellcolor[rgb]{0.9274509803921569, 0.9313725490196079, 0.9392156862745098} 0.366& \cellcolor[rgb]{0.8529411764705882, 0.5078431372549019, 0.5745098039215686} 0.712& \cellcolor[rgb]{0.8529411764705882, 0.5078431372549019, 0.5745098039215686} 0.730& \cellcolor[rgb]{0.984313725490196, 0.8627450980392157, 0.8098039215686275} 0.430& \cellcolor[rgb]{0.9745098039215686, 0.7862745098039216, 0.7274509803921568} 0.644& \cellcolor[rgb]{0.6254901960784314, 0.6666666666666666, 0.8921568627450981} 0.466& \cellcolor[rgb]{0.8529411764705882, 0.5078431372549019, 0.5745098039215686} 0.562\\

% \midrule

% & \multicolumn{7}{c}{$\text{CoCoA}_{PPL}$}\\
% \midrule

% AlignScore & \cellcolor[rgb]{0.8529411764705882, 0.5078431372549019, 0.5745098039215686} 0.474& \cellcolor[rgb]{0.615686274509804, 0.6490196078431373, 0.8764705882352941} 0.619& \cellcolor[rgb]{0.615686274509804, 0.6490196078431373, 0.8764705882352941} 0.657& \cellcolor[rgb]{0.9823529411764705, 0.8176470588235294, 0.7568627450980392} 0.408& \cellcolor[rgb]{0.8549019607843137, 0.9019607843137255, 0.9901960784313726} 0.638& \cellcolor[rgb]{0.6784313725490196, 0.7392156862745098, 0.9490196078431372} 0.467& \cellcolor[rgb]{0.615686274509804, 0.6490196078431373, 0.8764705882352941} 0.910\\
% RougeL & \cellcolor[rgb]{0.615686274509804, 0.6490196078431373, 0.8764705882352941} 0.362& \cellcolor[rgb]{0.9705882352941176, 0.7725490196078432, 0.7156862745098039} 0.710& \cellcolor[rgb]{0.9745098039215686, 0.8941176470588236, 0.8529411764705882} 0.717& \cellcolor[rgb]{0.615686274509804, 0.6490196078431373, 0.8764705882352941} 0.391& \cellcolor[rgb]{0.615686274509804, 0.6490196078431373, 0.8764705882352941} 0.627& \cellcolor[rgb]{0.615686274509804, 0.6490196078431373, 0.8764705882352941} 0.466& \cellcolor[rgb]{0.8529411764705882, 0.5078431372549019, 0.5745098039215686} 0.950\\
% NLI & \cellcolor[rgb]{0.6568627450980392, 0.7098039215686275, 0.9274509803921569} 0.370& \cellcolor[rgb]{0.9313725490196079, 0.9333333333333333, 0.9333333333333333} 0.677& \cellcolor[rgb]{0.796078431372549, 0.8607843137254902, 1.0} 0.684& \cellcolor[rgb]{0.8529411764705882, 0.5078431372549019, 0.5745098039215686} 0.414& \cellcolor[rgb]{0.8529411764705882, 0.5078431372549019, 0.5745098039215686} 0.657& \cellcolor[rgb]{0.8529411764705882, 0.5078431372549019, 0.5745098039215686} 0.470& \cellcolor[rgb]{0.9725490196078431, 0.7764705882352941, 0.7176470588235294} 0.941\\
% CrossEncoder & \cellcolor[rgb]{0.6666666666666666, 0.7235294117647059, 0.9372549019607843} 0.372& \cellcolor[rgb]{0.8529411764705882, 0.5078431372549019, 0.5745098039215686} 0.735& \cellcolor[rgb]{0.8529411764705882, 0.5078431372549019, 0.5745098039215686} 0.755& \cellcolor[rgb]{0.9137254901960784, 0.9294117647058824, 0.9529411764705882} 0.402& \cellcolor[rgb]{0.984313725490196, 0.8372549019607843, 0.7784313725490196} 0.648& \cellcolor[rgb]{0.6254901960784314, 0.6666666666666666, 0.8921568627450981} 0.466& \cellcolor[rgb]{0.984313725490196, 0.8509803921568627, 0.7941176470588235} 0.937\\

% \midrule

% & \multicolumn{7}{c}{$\text{CoCoA}_{MTE}$}\\
% \midrule

% AlignScore & \cellcolor[rgb]{0.8529411764705882, 0.5078431372549019, 0.5745098039215686} 0.492& \cellcolor[rgb]{0.615686274509804, 0.6490196078431373, 0.8764705882352941} 0.547& \cellcolor[rgb]{0.615686274509804, 0.6490196078431373, 0.8764705882352941} 0.590& \cellcolor[rgb]{0.9705882352941176, 0.7705882352941177, 0.711764705882353} 0.383& \cellcolor[rgb]{0.807843137254902, 0.8705882352941177, 1.0} 0.633& \cellcolor[rgb]{0.8176470588235294, 0.8784313725490196, 1.0} 0.449& \cellcolor[rgb]{0.615686274509804, 0.6490196078431373, 0.8764705882352941} 0.914\\
% RougeL & \cellcolor[rgb]{0.615686274509804, 0.6490196078431373, 0.8764705882352941} 0.355& \cellcolor[rgb]{0.9058823529411765, 0.6352941176470588, 0.6176470588235294} 0.695& \cellcolor[rgb]{0.984313725490196, 0.8607843137254902, 0.8058823529411765} 0.684& \cellcolor[rgb]{0.615686274509804, 0.6490196078431373, 0.8764705882352941} 0.366& \cellcolor[rgb]{0.615686274509804, 0.6490196078431373, 0.8764705882352941} 0.624& \cellcolor[rgb]{0.7254901960784313, 0.7941176470588235, 0.9803921568627451} 0.448& \cellcolor[rgb]{0.8529411764705882, 0.5078431372549019, 0.5745098039215686} 0.959\\
% NLI & \cellcolor[rgb]{0.6490196078431373, 0.7, 0.9196078431372549} 0.364& \cellcolor[rgb]{0.984313725490196, 0.8509803921568627, 0.7941176470588235} 0.656& \cellcolor[rgb]{0.9235294117647059, 0.9313725490196079, 0.9431372549019608} 0.658& \cellcolor[rgb]{0.8529411764705882, 0.5078431372549019, 0.5745098039215686} 0.387& \cellcolor[rgb]{0.8529411764705882, 0.5078431372549019, 0.5745098039215686} 0.656& \cellcolor[rgb]{0.8529411764705882, 0.5078431372549019, 0.5745098039215686} 0.453& \cellcolor[rgb]{0.6666666666666666, 0.7235294117647059, 0.9372549019607843} 0.918\\
% CrossEncoder & \cellcolor[rgb]{0.692156862745098, 0.7549019607843137, 0.9588235294117647} 0.373& \cellcolor[rgb]{0.8529411764705882, 0.5078431372549019, 0.5745098039215686} 0.708& \cellcolor[rgb]{0.8529411764705882, 0.5078431372549019, 0.5745098039215686} 0.732& \cellcolor[rgb]{0.8254901960784313, 0.884313725490196, 0.9980392156862745} 0.373& \cellcolor[rgb]{0.984313725490196, 0.8588235294117647, 0.803921568627451} 0.645& \cellcolor[rgb]{0.615686274509804, 0.6490196078431373, 0.8764705882352941} 0.447& \cellcolor[rgb]{0.9117647058823529, 0.9294117647058824, 0.9549019607843138} 0.935\\
% \midrule

% & \multicolumn{7}{c}{Llama8b-Base} \\

% \midrule

% & \multicolumn{7}{c}{$\text{CoCoA}_{MSP}$}\\
% \midrule

% AlignScore & \cellcolor[rgb]{0.8529411764705882, 0.5078431372549019, 0.5745098039215686} 0.520& \cellcolor[rgb]{0.615686274509804, 0.6490196078431373, 0.8764705882352941} 0.332& \cellcolor[rgb]{0.615686274509804, 0.6490196078431373, 0.8764705882352941} 0.491& \cellcolor[rgb]{0.615686274509804, 0.6490196078431373, 0.8764705882352941} 0.354& \cellcolor[rgb]{0.9725490196078431, 0.7803921568627451, 0.7215686274509804} 0.587& \cellcolor[rgb]{0.8529411764705882, 0.5078431372549019, 0.5745098039215686} 0.457& \cellcolor[rgb]{0.615686274509804, 0.6490196078431373, 0.8764705882352941} 0.401\\
% RougeL & \cellcolor[rgb]{0.6666666666666666, 0.7235294117647059, 0.9372549019607843} 0.471& \cellcolor[rgb]{0.984313725490196, 0.8372549019607843, 0.7784313725490196} 0.470& \cellcolor[rgb]{0.9333333333333333, 0.9313725490196079, 0.9313725490196079} 0.588& \cellcolor[rgb]{0.7725490196078432, 0.8392156862745098, 0.996078431372549} 0.362& \cellcolor[rgb]{0.615686274509804, 0.6490196078431373, 0.8764705882352941} 0.551& \cellcolor[rgb]{0.615686274509804, 0.6490196078431373, 0.8764705882352941} 0.446& \cellcolor[rgb]{0.9333333333333333, 0.6862745098039216, 0.6470588235294118} 0.499\\
% NLI & \cellcolor[rgb]{0.615686274509804, 0.6490196078431373, 0.8764705882352941} 0.466& \cellcolor[rgb]{0.9607843137254902, 0.9137254901960784, 0.888235294117647} 0.442& \cellcolor[rgb]{0.9019607843137255, 0.9254901960784314, 0.9627450980392157} 0.577& \cellcolor[rgb]{0.8529411764705882, 0.5078431372549019, 0.5745098039215686} 0.386& \cellcolor[rgb]{0.8529411764705882, 0.5078431372549019, 0.5745098039215686} 0.597& \cellcolor[rgb]{0.6254901960784314, 0.6666666666666666, 0.8921568627450981} 0.446& \cellcolor[rgb]{0.9764705882352941, 0.8921568627450981, 0.8490196078431372} 0.470\\
% CrossEncoder & \cellcolor[rgb]{0.8313725490196079, 0.888235294117647, 0.996078431372549} 0.484& \cellcolor[rgb]{0.8529411764705882, 0.5078431372549019, 0.5745098039215686} 0.529& \cellcolor[rgb]{0.8529411764705882, 0.5078431372549019, 0.5745098039215686} 0.685& \cellcolor[rgb]{0.8901960784313725, 0.6058823529411764, 0.6039215686274509} 0.384& \cellcolor[rgb]{0.9705882352941176, 0.7725490196078432, 0.7156862745098039} 0.587& \cellcolor[rgb]{0.9627450980392157, 0.9137254901960784, 0.8862745098039215} 0.452& \cellcolor[rgb]{0.8529411764705882, 0.5078431372549019, 0.5745098039215686} 0.513\\

% \midrule

% & \multicolumn{7}{c}{$\text{CoCoA}_{PPL}$}\\
% \midrule

% AlignScore & \cellcolor[rgb]{0.8529411764705882, 0.5078431372549019, 0.5745098039215686} 0.546& \cellcolor[rgb]{0.615686274509804, 0.6490196078431373, 0.8764705882352941} 0.406& \cellcolor[rgb]{0.615686274509804, 0.6490196078431373, 0.8764705882352941} 0.561& \cellcolor[rgb]{0.8529411764705882, 0.5078431372549019, 0.5745098039215686} 0.376& \cellcolor[rgb]{0.9274509803921569, 0.6764705882352942, 0.6431372549019608} 0.577& \cellcolor[rgb]{0.8529411764705882, 0.5078431372549019, 0.5745098039215686} 0.429& \cellcolor[rgb]{0.615686274509804, 0.6490196078431373, 0.8764705882352941} 0.875\\
% RougeL & \cellcolor[rgb]{0.6215686274509804, 0.6588235294117647, 0.8862745098039215} 0.452& \cellcolor[rgb]{0.9627450980392157, 0.7529411764705882, 0.6980392156862745} 0.518& \cellcolor[rgb]{0.9764705882352941, 0.8901960784313725, 0.8470588235294118} 0.639& \cellcolor[rgb]{0.615686274509804, 0.6490196078431373, 0.8764705882352941} 0.352& \cellcolor[rgb]{0.615686274509804, 0.6490196078431373, 0.8764705882352941} 0.532& \cellcolor[rgb]{0.615686274509804, 0.6490196078431373, 0.8764705882352941} 0.417& \cellcolor[rgb]{0.8529411764705882, 0.5078431372549019, 0.5745098039215686} 0.931\\
% NLI & \cellcolor[rgb]{0.6627450980392157, 0.7156862745098039, 0.9333333333333333} 0.458& \cellcolor[rgb]{0.8980392156862745, 0.9235294117647059, 0.9666666666666667} 0.466& \cellcolor[rgb]{0.796078431372549, 0.8607843137254902, 1.0} 0.597& \cellcolor[rgb]{0.9607843137254902, 0.9137254901960784, 0.888235294117647} 0.365& \cellcolor[rgb]{0.8529411764705882, 0.5078431372549019, 0.5745098039215686} 0.583& \cellcolor[rgb]{0.6725490196078432, 0.7294117647058823, 0.9411764705882353} 0.418& \cellcolor[rgb]{0.984313725490196, 0.8686274509803922, 0.8156862745098039} 0.912\\
% CrossEncoder & \cellcolor[rgb]{0.615686274509804, 0.6490196078431373, 0.8764705882352941} 0.450& \cellcolor[rgb]{0.8529411764705882, 0.5078431372549019, 0.5745098039215686} 0.544& \cellcolor[rgb]{0.8529411764705882, 0.5078431372549019, 0.5745098039215686} 0.689& \cellcolor[rgb]{0.9352941176470588, 0.9313725490196079, 0.9294117647058824} 0.364& \cellcolor[rgb]{0.9686274509803922, 0.7666666666666667, 0.7098039215686275} 0.573& \cellcolor[rgb]{0.9019607843137255, 0.9254901960784314, 0.9627450980392157} 0.422& \cellcolor[rgb]{0.9215686274509803, 0.6647058823529411, 0.6352941176470588} 0.925\\

% \midrule

% & \multicolumn{7}{c}{$\text{CoCoA}_{MTE}$}\\
% \midrule

% AlignScore & \cellcolor[rgb]{0.8529411764705882, 0.5078431372549019, 0.5745098039215686} 0.561& \cellcolor[rgb]{0.615686274509804, 0.6490196078431373, 0.8764705882352941} 0.325& \cellcolor[rgb]{0.615686274509804, 0.6490196078431373, 0.8764705882352941} 0.497& \cellcolor[rgb]{0.8529411764705882, 0.5078431372549019, 0.5745098039215686} 0.365& \cellcolor[rgb]{0.9470588235294117, 0.7156862745098039, 0.6686274509803921} 0.589& \cellcolor[rgb]{0.8529411764705882, 0.5078431372549019, 0.5745098039215686} 0.380& \cellcolor[rgb]{0.615686274509804, 0.6490196078431373, 0.8764705882352941} 0.821\\
% RougeL & \cellcolor[rgb]{0.615686274509804, 0.6490196078431373, 0.8764705882352941} 0.448& \cellcolor[rgb]{0.9333333333333333, 0.6862745098039216, 0.6470588235294118} 0.496& \cellcolor[rgb]{0.984313725490196, 0.8254901960784313, 0.7666666666666667} 0.598& \cellcolor[rgb]{0.615686274509804, 0.6490196078431373, 0.8764705882352941} 0.336& \cellcolor[rgb]{0.615686274509804, 0.6490196078431373, 0.8764705882352941} 0.539& \cellcolor[rgb]{0.6490196078431373, 0.7, 0.9196078431372549} 0.361& \cellcolor[rgb]{0.8529411764705882, 0.5078431372549019, 0.5745098039215686} 0.921\\
% NLI & \cellcolor[rgb]{0.615686274509804, 0.6490196078431373, 0.8764705882352941} 0.449& \cellcolor[rgb]{0.9784313725490196, 0.888235294117647, 0.8431372549019608} 0.446& \cellcolor[rgb]{0.9254901960784314, 0.9313725490196079, 0.9411764705882353} 0.565& \cellcolor[rgb]{0.7901960784313725, 0.8568627450980392, 1.0} 0.344& \cellcolor[rgb]{0.8529411764705882, 0.5078431372549019, 0.5745098039215686} 0.598& \cellcolor[rgb]{0.615686274509804, 0.6490196078431373, 0.8764705882352941} 0.359& \cellcolor[rgb]{0.9745098039215686, 0.8980392156862745, 0.8588235294117647} 0.881\\
% CrossEncoder & \cellcolor[rgb]{0.6294117647058823, 0.6705882352941176, 0.8941176470588236} 0.451& \cellcolor[rgb]{0.8529411764705882, 0.5078431372549019, 0.5745098039215686} 0.520& \cellcolor[rgb]{0.8529411764705882, 0.5078431372549019, 0.5745098039215686} 0.638& \cellcolor[rgb]{0.8392156862745098, 0.8941176470588236, 0.996078431372549} 0.346& \cellcolor[rgb]{0.9823529411764705, 0.8176470588235294, 0.7568627450980392} 0.582& \cellcolor[rgb]{0.7196078431372549, 0.788235294117647, 0.9764705882352941} 0.363& \cellcolor[rgb]{0.9666666666666667, 0.7588235294117647, 0.703921568627451} 0.900\\
% \midrule

% & \multicolumn{7}{c}{Falcon7b-Base} \\

% \midrule

% & \multicolumn{7}{c}{$\text{CoCoA}_{MSP}$}\\
% \midrule

% AlignScore & \cellcolor[rgb]{0.984313725490196, 0.8509803921568627, 0.7941176470588235} 0.181& \cellcolor[rgb]{0.615686274509804, 0.6490196078431373, 0.8764705882352941} 0.378& \cellcolor[rgb]{0.615686274509804, 0.6490196078431373, 0.8764705882352941} 0.473& \cellcolor[rgb]{0.615686274509804, 0.6490196078431373, 0.8764705882352941} 0.410& \cellcolor[rgb]{0.615686274509804, 0.6490196078431373, 0.8764705882352941} 0.654& \cellcolor[rgb]{0.6568627450980392, 0.7098039215686275, 0.9274509803921569} 0.528& \cellcolor[rgb]{0.615686274509804, 0.6490196078431373, 0.8764705882352941} 0.239\\
% RougeL & \cellcolor[rgb]{0.6254901960784314, 0.6666666666666666, 0.8921568627450981} 0.122& \cellcolor[rgb]{0.9568627450980393, 0.7352941176470589, 0.6823529411764706} 0.531& \cellcolor[rgb]{0.9235294117647059, 0.9313725490196079, 0.9431372549019608} 0.581& \cellcolor[rgb]{0.8568627450980392, 0.903921568627451, 0.9901960784313726} 0.420& \cellcolor[rgb]{0.6764705882352942, 0.7352941176470589, 0.9470588235294117} 0.655& \cellcolor[rgb]{0.615686274509804, 0.6490196078431373, 0.8764705882352941} 0.528& \cellcolor[rgb]{0.9823529411764705, 0.8235294117647058, 0.7627450980392156} 0.426\\
% NLI & \cellcolor[rgb]{0.615686274509804, 0.6490196078431373, 0.8764705882352941} 0.120& \cellcolor[rgb]{0.9803921568627451, 0.8784313725490196, 0.8313725490196079} 0.496& \cellcolor[rgb]{0.9745098039215686, 0.8980392156862745, 0.8588235294117647} 0.607& \cellcolor[rgb]{0.8529411764705882, 0.5078431372549019, 0.5745098039215686} 0.437& \cellcolor[rgb]{0.9529411764705882, 0.7274509803921568, 0.6784313725490196} 0.658& \cellcolor[rgb]{0.8529411764705882, 0.5078431372549019, 0.5745098039215686} 0.533& \cellcolor[rgb]{0.9490196078431372, 0.7196078431372549, 0.6725490196078432} 0.458\\
% CrossEncoder & \cellcolor[rgb]{0.8529411764705882, 0.5078431372549019, 0.5745098039215686} 0.210& \cellcolor[rgb]{0.8529411764705882, 0.5078431372549019, 0.5745098039215686} 0.564& \cellcolor[rgb]{0.8529411764705882, 0.5078431372549019, 0.5745098039215686} 0.698& \cellcolor[rgb]{0.9823529411764705, 0.8725490196078431, 0.8215686274509804} 0.428& \cellcolor[rgb]{0.8529411764705882, 0.5078431372549019, 0.5745098039215686} 0.659& \cellcolor[rgb]{0.888235294117647, 0.9196078431372549, 0.9725490196078431} 0.530& \cellcolor[rgb]{0.8529411764705882, 0.5078431372549019, 0.5745098039215686} 0.498\\

% \midrule

% & \multicolumn{7}{c}{$\text{CoCoA}_{PPL}$}\\
% \midrule

% AlignScore & \cellcolor[rgb]{0.8529411764705882, 0.5078431372549019, 0.5745098039215686} 0.384& \cellcolor[rgb]{0.615686274509804, 0.6490196078431373, 0.8764705882352941} 0.454& \cellcolor[rgb]{0.615686274509804, 0.6490196078431373, 0.8764705882352941} 0.586& \cellcolor[rgb]{0.8529411764705882, 0.5078431372549019, 0.5745098039215686} 0.440& \cellcolor[rgb]{0.8529411764705882, 0.5078431372549019, 0.5745098039215686} 0.648& \cellcolor[rgb]{0.6568627450980392, 0.7098039215686275, 0.9274509803921569} 0.528& \cellcolor[rgb]{0.615686274509804, 0.6490196078431373, 0.8764705882352941} 0.994\\
% RougeL & \cellcolor[rgb]{0.615686274509804, 0.6490196078431373, 0.8764705882352941} 0.280& \cellcolor[rgb]{0.9274509803921569, 0.6764705882352942, 0.6431372549019608} 0.565& \cellcolor[rgb]{0.9803921568627451, 0.8803921568627451, 0.8333333333333334} 0.668& \cellcolor[rgb]{0.615686274509804, 0.6490196078431373, 0.8764705882352941} 0.410& \cellcolor[rgb]{0.615686274509804, 0.6490196078431373, 0.8764705882352941} 0.637& \cellcolor[rgb]{0.615686274509804, 0.6490196078431373, 0.8764705882352941} 0.528& \cellcolor[rgb]{0.8529411764705882, 0.5078431372549019, 0.5745098039215686} 1.000\\
% NLI & \cellcolor[rgb]{0.6313725490196078, 0.6725490196078432, 0.8980392156862745} 0.283& \cellcolor[rgb]{0.9254901960784314, 0.9313725490196079, 0.9411764705882353} 0.515& \cellcolor[rgb]{0.9823529411764705, 0.8725490196078431, 0.8215686274509804} 0.671& \cellcolor[rgb]{0.9137254901960784, 0.9294117647058824, 0.9529411764705882} 0.424& \cellcolor[rgb]{0.907843137254902, 0.6392156862745098, 0.6196078431372549} 0.647& \cellcolor[rgb]{0.8529411764705882, 0.5078431372549019, 0.5745098039215686} 0.533& \cellcolor[rgb]{0.984313725490196, 0.8254901960784313, 0.7666666666666667} 0.998\\
% CrossEncoder & \cellcolor[rgb]{0.8019607843137255, 0.8666666666666667, 1.0} 0.310& \cellcolor[rgb]{0.8529411764705882, 0.5078431372549019, 0.5745098039215686} 0.579& \cellcolor[rgb]{0.8529411764705882, 0.5078431372549019, 0.5745098039215686} 0.717& \cellcolor[rgb]{0.703921568627451, 0.7705882352941177, 0.9686274509803922} 0.415& \cellcolor[rgb]{0.984313725490196, 0.8450980392156863, 0.788235294117647} 0.644& \cellcolor[rgb]{0.888235294117647, 0.9196078431372549, 0.9725490196078431} 0.530& \cellcolor[rgb]{0.8529411764705882, 0.5078431372549019, 0.5745098039215686} 1.000\\

% \midrule

% & \multicolumn{7}{c}{$\text{CoCoA}_{MTE}$}\\
% \midrule

% AlignScore & \cellcolor[rgb]{0.8529411764705882, 0.5078431372549019, 0.5745098039215686} 0.292& \cellcolor[rgb]{0.615686274509804, 0.6490196078431373, 0.8764705882352941} 0.386& \cellcolor[rgb]{0.615686274509804, 0.6490196078431373, 0.8764705882352941} 0.498& \cellcolor[rgb]{0.8529411764705882, 0.5078431372549019, 0.5745098039215686} 0.435& \cellcolor[rgb]{0.8529411764705882, 0.5078431372549019, 0.5745098039215686} 0.648& \cellcolor[rgb]{0.6470588235294118, 0.696078431372549, 0.9176470588235294} 0.515& \cellcolor[rgb]{0.615686274509804, 0.6490196078431373, 0.8764705882352941} 0.972\\
% RougeL & \cellcolor[rgb]{0.7627450980392156, 0.8313725490196079, 0.9941176470588236} 0.222& \cellcolor[rgb]{0.8784313725490196, 0.5843137254901961, 0.5941176470588235} 0.545& \cellcolor[rgb]{0.9764705882352941, 0.8921568627450981, 0.8490196078431372} 0.607& \cellcolor[rgb]{0.615686274509804, 0.6490196078431373, 0.8764705882352941} 0.400& \cellcolor[rgb]{0.615686274509804, 0.6490196078431373, 0.8764705882352941} 0.633& \cellcolor[rgb]{0.615686274509804, 0.6490196078431373, 0.8764705882352941} 0.515& \cellcolor[rgb]{0.8549019607843137, 0.5176470588235295, 0.5764705882352941} 0.998\\
% NLI & \cellcolor[rgb]{0.615686274509804, 0.6490196078431373, 0.8764705882352941} 0.201& \cellcolor[rgb]{0.984313725490196, 0.8529411764705882, 0.796078431372549} 0.498& \cellcolor[rgb]{0.9745098039215686, 0.7862745098039216, 0.7274509803921568} 0.636& \cellcolor[rgb]{0.9, 0.9254901960784314, 0.9647058823529412} 0.415& \cellcolor[rgb]{0.9529411764705882, 0.7274509803921568, 0.6784313725490196} 0.645& \cellcolor[rgb]{0.8529411764705882, 0.5078431372549019, 0.5745098039215686} 0.521& \cellcolor[rgb]{0.9686274509803922, 0.9058823529411765, 0.8745098039215686} 0.987\\
% CrossEncoder & \cellcolor[rgb]{0.8725490196078431, 0.5705882352941176, 0.5901960784313726} 0.289& \cellcolor[rgb]{0.8529411764705882, 0.5078431372549019, 0.5745098039215686} 0.551& \cellcolor[rgb]{0.8529411764705882, 0.5078431372549019, 0.5745098039215686} 0.678& \cellcolor[rgb]{0.6529411764705882, 0.703921568627451, 0.9235294117647059} 0.402& \cellcolor[rgb]{0.9372549019607843, 0.6941176470588235, 0.6529411764705882} 0.646& \cellcolor[rgb]{0.8725490196078431, 0.9117647058823529, 0.9823529411764705} 0.517& \cellcolor[rgb]{0.8529411764705882, 0.5078431372549019, 0.5745098039215686} 0.998\\
% \midrule

% \end{tabular}
% \label{tab:ablation_sim_mat_best}
% \end{table*}

\begin{table*}[h!]
\footnotesize
\centering

\begin{tabular}{lrrrrrrr}
\toprule
    \multirow{2}{*}{\textbf{Method}}  & \multicolumn{7}{c}{\textbf{Dataset}}  \\ 
      \cmidrule(lr){2-8}  \\
  & XSum & WMT14FrEn & WMT19DeEn & CoQa & Trivia & MMLU & GSM8k \\
  \midrule

\rowcolor[gray]{0.9} & \multicolumn{7}{c}{Mistral7b-Base} \\

\midrule

& \multicolumn{7}{c}{$\text{CoCoA}_{MSP}$}\\
\midrule

AlignScore & 0.393& 0.448& 0.491& 0.399& 0.626& 0.467& 0.476\\
RougeL & 0.344& 0.602& 0.597& 0.420& 0.622& 0.466& 0.538\\
NLI & 0.340& 0.615& 0.604& \textbf{0.445}& 0.651& \textbf{0.470}& 0.456\\
CrossEncoder & 0.366& \underline{0.712}& 0.730& \underline{0.430}& 0.644& 0.466& 0.562\\

\midrule

& \multicolumn{7}{c}{$\text{CoCoA}_{PPL}$}\\
\midrule

AlignScore & \underline{0.474}& 0.619& 0.657& 0.408& 0.638& 0.467& 0.910\\
RougeL & 0.362& 0.710& 0.717& 0.391& 0.627& 0.466& \underline{0.950}\\
NLI & 0.370& 0.677& 0.684& 0.414& \textbf{0.657}& \textbf{0.470}& 0.941\\
CrossEncoder & 0.372& \textbf{0.735}& \textbf{0.755}& 0.402& 0.648& 0.466& 0.937\\

\midrule

& \multicolumn{7}{c}{$\text{CoCoA}_{MTE}$}\\
\midrule

AlignScore & \textbf{0.492}& 0.547& 0.590& 0.383& 0.633& 0.449& 0.914\\
RougeL & 0.355& 0.695& 0.684& 0.366& 0.624& 0.448& \textbf{0.959}\\
NLI & 0.364& \underline{0.656}& 0.658& 0.387& 0.656& 0.453& 0.918\\
CrossEncoder & 0.373& 0.708& \underline{0.732}& 0.373& 0.645& 0.447& 0.935\\
\midrule

\rowcolor[gray]{0.9} & \multicolumn{7}{c}{Llama8b-Base} \\

\midrule

& \multicolumn{7}{c}{$\text{CoCoA}_{MSP}$}\\
\midrule

AlignScore & 0.520& 0.332& 0.491& 0.354& 0.587& \textbf{0.457}& 0.401\\
RougeL & 0.471& 0.470& 0.588& 0.362& 0.551& 0.446& 0.499\\
NLI & 0.466& 0.442& 0.577& \textbf{0.386}& \underline{0.597}& 0.446& 0.470\\
CrossEncoder & 0.484& \underline{0.529}& \underline{0.685}& \underline{0.384}& 0.587& \underline{0.452}& 0.513\\

\midrule

& \multicolumn{7}{c}{$\text{CoCoA}_{PPL}$}\\
\midrule

AlignScore & \underline{0.546}& 0.406& 0.561& 0.376& 0.577& 0.429& 0.875\\
RougeL & 0.452& 0.518& 0.639& 0.352& 0.532& 0.417& \textbf{0.931}\\
NLI & 0.458& 0.466& 0.597& 0.365& 0.583& 0.418& 0.912\\
CrossEncoder & 0.450& \textbf{0.544}& \textbf{0.689}& 0.364& 0.573& 0.422& \underline{0.925}\\

\midrule

& \multicolumn{7}{c}{$\text{CoCoA}_{MTE}$}\\
\midrule

AlignScore & \textbf{0.561}& 0.325& 0.497& 0.365& 0.589& 0.380& 0.821\\
RougeL & 0.448& 0.496& 0.598& 0.336& 0.539& 0.361& 0.921\\
NLI & 0.449& 0.446& 0.565& 0.344& \textbf{0.598}& 0.359& 0.881\\
CrossEncoder & 0.451& 0.520& 0.638& 0.346& 0.582& 0.363& 0.900\\
\midrule

\rowcolor[gray]{0.9} & \multicolumn{7}{c}{Falcon7b-Base} \\

\midrule

& \multicolumn{7}{c}{$\text{CoCoA}_{MSP}$}\\
\midrule

AlignScore & 0.181& 0.378& 0.473& 0.410& 0.654& 0.528& 0.239\\
RougeL & 0.122& 0.531& 0.581& 0.420& 0.655& 0.528& 0.426\\
NLI & 0.120& 0.496& 0.607& \underline{0.437}& \underline{0.658}& \textbf{0.533}& 0.458\\
CrossEncoder & 0.210& 0.564& \underline{0.698}& 0.428& \textbf{0.659}& 0.530& 0.498\\

\midrule

& \multicolumn{7}{c}{$\text{CoCoA}_{PPL}$}\\
\midrule

AlignScore & \textbf{0.384}& 0.454& 0.586& \textbf{0.440}& 0.648& 0.528& 0.994\\
RougeL & 0.280& \underline{0.565}& 0.668& 0.410& 0.637& 0.528& \textbf{1.000}\\
NLI & 0.283& 0.515& 0.671& 0.424& 0.647& \textbf{0.533}& 0.998\\
CrossEncoder & \underline{0.310}& \textbf{0.579}& \textbf{0.717}& 0.415& 0.644& 0.530& \textbf{1.000}\\

\midrule

& \multicolumn{7}{c}{$\text{CoCoA}_{MTE}$}\\
\midrule

AlignScore & 0.292& 0.386& 0.498& 0.435& 0.648& 0.515& 0.972\\
RougeL & 0.222& 0.545& 0.607& 0.400& 0.633& 0.515& 0.998\\
NLI & 0.201& 0.498& 0.636& 0.415& 0.645& 0.521& 0.987\\
CrossEncoder & 0.289& 0.551& 0.678& 0.402& 0.646& 0.517& 0.998\\
\bottomrule

\end{tabular}

\caption{Comparison of PRRs of \texttt{CoCoA}-family methods with different choices of similarity function with best sample taken in isolation.}
\label{suppl:ablation_sim_mat_best}
\end{table*}

\FloatBarrier



\newpage

\subsection{Different Ways of Combining Confidence and Consistency}
\label{sec:sum_cocoa}

  We justify the particular form of equation~\eqref{eq:cocoa} by considering alternative ways to combine sample-focused confidence with consistency estimation. Results are presented in Tables~\ref{tab:ablation_cocoa_greedy} and~\ref{tab:ablation_cocoa_best}. In particular, we investigate the performance of the additive approach (AdditiveCoCoA):
  \begin{equation}
    \mathrm{U}_{AdditiveCoCoA} = u_*^{\text{info}} + u_*^{\text{cons}},
  \end{equation}
  and the same multiplicative combination, replacing sample-focused dissimilarity from~\eqref{eq:ave_dissim} with the average of the full pairwise dissimilarity matrix \eqref{eq:degmat}:
  \begin{equation}
    \mathrm{U}_{FullSampleCoCoA} = u_*^{\text{info}} \cdot U_{Deg}.
  \end{equation}
  %
  It is evident that on average the multiplicative form proposed in equation~\eqref{eq:cocoa} with both confidence and consistency terms focused on a single sample is the better performing variant.

  % \begin{table*}
% \footnotesize
% \caption{Greedy PRRs for falcon7b on all tasks}
% \begin{tabular}{lrrrrrrr}
%  & XSum & WMT14FrEn & WMT19DeEn & CoQa & Trivia & MMLU & GSM8k \\

% \midrule

% & \multicolumn{7}{c}{Mistral7b-Base} \\

% \midrule

% $\text{AdditiveCoCoA}_{MSP}$ & \cellcolor[rgb]{0.9450980392156862, 0.7098039215686275, 0.6647058823529411} 0.290& \cellcolor[rgb]{0.7294117647058823, 0.796078431372549, 0.9823529411764705} 0.319& \cellcolor[rgb]{0.615686274509804, 0.6490196078431373, 0.8764705882352941} 0.459& \cellcolor[rgb]{0.615686274509804, 0.6490196078431373, 0.8764705882352941} 0.351& \cellcolor[rgb]{0.615686274509804, 0.6490196078431373, 0.8764705882352941} 0.654& \cellcolor[rgb]{0.8529411764705882, 0.5078431372549019, 0.5745098039215686} 0.471& \cellcolor[rgb]{0.9098039215686274, 0.9274509803921569, 0.9568627450980393} 0.472\\
% $\text{FullSampleCoCoA}_{MSP}$ & \cellcolor[rgb]{0.8784313725490196, 0.5843137254901961, 0.5941176470588235} 0.319& \cellcolor[rgb]{0.9313725490196079, 0.6823529411764706, 0.6450980392156863} 0.385& \cellcolor[rgb]{0.8921568627450981, 0.6098039215686275, 0.6058823529411764} 0.590& \cellcolor[rgb]{0.7196078431372549, 0.788235294117647, 0.9764705882352941} 0.357& \cellcolor[rgb]{0.9333333333333333, 0.6862745098039216, 0.6470588235294118} 0.668& \cellcolor[rgb]{0.9607843137254902, 0.9137254901960784, 0.888235294117647} 0.467& \cellcolor[rgb]{0.9450980392156862, 0.7098039215686275, 0.6647058823529411} 0.505\\
% $\text{ProbCoCoA}_{MSP}$ & \cellcolor[rgb]{0.615686274509804, 0.6490196078431373, 0.8764705882352941} 0.059& \cellcolor[rgb]{0.615686274509804, 0.6490196078431373, 0.8764705882352941} 0.302& \cellcolor[rgb]{0.9, 0.9254901960784314, 0.9647058823529412} 0.520& \cellcolor[rgb]{0.8529411764705882, 0.5078431372549019, 0.5745098039215686} 0.390& \cellcolor[rgb]{0.8529411764705882, 0.5078431372549019, 0.5745098039215686} 0.671& \cellcolor[rgb]{0.615686274509804, 0.6490196078431373, 0.8764705882352941} 0.461& \cellcolor[rgb]{0.615686274509804, 0.6490196078431373, 0.8764705882352941} 0.435\\
% $\text{CoCoA}_{MSP}$ & \cellcolor[rgb]{0.8529411764705882, 0.5078431372549019, 0.5745098039215686} 0.330& \cellcolor[rgb]{0.8529411764705882, 0.5078431372549019, 0.5745098039215686} 0.396& \cellcolor[rgb]{0.8529411764705882, 0.5078431372549019, 0.5745098039215686} 0.598& \cellcolor[rgb]{0.9568627450980393, 0.7392156862745098, 0.6862745098039216} 0.383& \cellcolor[rgb]{0.8666666666666667, 0.5529411764705883, 0.5862745098039216} 0.670& \cellcolor[rgb]{0.9156862745098039, 0.9294117647058824, 0.9509803921568627} 0.466& \cellcolor[rgb]{0.8529411764705882, 0.5078431372549019, 0.5745098039215686} 0.517\\

% \midrule

% $\text{AdditiveCoCoA}_{PPL}$ & \cellcolor[rgb]{0.615686274509804, 0.6490196078431373, 0.8764705882352941} 0.262& \cellcolor[rgb]{0.8529411764705882, 0.5078431372549019, 0.5745098039215686} 0.392& \cellcolor[rgb]{0.9764705882352941, 0.796078431372549, 0.7352941176470589} 0.564& \cellcolor[rgb]{0.9215686274509803, 0.6647058823529411, 0.6352941176470588} 0.369& \cellcolor[rgb]{0.615686274509804, 0.6490196078431373, 0.8764705882352941} 0.671& \cellcolor[rgb]{0.7156862745098039, 0.7823529411764706, 0.9745098039215686} 0.464& \cellcolor[rgb]{0.8529411764705882, 0.5078431372549019, 0.5745098039215686} 0.494\\
% $\text{FullSampleCoCoA}_{PPL}$ & \cellcolor[rgb]{0.9, 0.9254901960784314, 0.9647058823529412} 0.277& \cellcolor[rgb]{0.7019607843137254, 0.7666666666666667, 0.9666666666666667} 0.373& \cellcolor[rgb]{0.615686274509804, 0.6490196078431373, 0.8764705882352941} 0.551& \cellcolor[rgb]{0.615686274509804, 0.6490196078431373, 0.8764705882352941} 0.334& \cellcolor[rgb]{0.8588235294117647, 0.9058823529411765, 0.9882352941176471} 0.672& \cellcolor[rgb]{0.8529411764705882, 0.5078431372549019, 0.5745098039215686} 0.467& \cellcolor[rgb]{0.615686274509804, 0.6490196078431373, 0.8764705882352941} 0.435\\
% $\text{ProbCoCoA}_{PPL}$ & \cellcolor[rgb]{0.8529411764705882, 0.5078431372549019, 0.5745098039215686} 0.297& \cellcolor[rgb]{0.615686274509804, 0.6490196078431373, 0.8764705882352941} 0.369& \cellcolor[rgb]{0.9352941176470588, 0.6901960784313725, 0.6509803921568628} 0.566& \cellcolor[rgb]{0.8529411764705882, 0.5078431372549019, 0.5745098039215686} 0.373& \cellcolor[rgb]{0.8529411764705882, 0.5078431372549019, 0.5745098039215686} 0.674& \cellcolor[rgb]{0.615686274509804, 0.6490196078431373, 0.8764705882352941} 0.464& \cellcolor[rgb]{0.984313725490196, 0.8470588235294118, 0.7901960784313725} 0.475\\
% $\text{CoCoA}_{PPL}$ & \cellcolor[rgb]{0.984313725490196, 0.8549019607843137, 0.8} 0.286& \cellcolor[rgb]{0.7803921568627451, 0.8470588235294118, 0.9980392156862745} 0.375& \cellcolor[rgb]{0.8529411764705882, 0.5078431372549019, 0.5745098039215686} 0.568& \cellcolor[rgb]{0.9254901960784314, 0.6725490196078432, 0.6392156862745098} 0.369& \cellcolor[rgb]{0.9803921568627451, 0.8058823529411765, 0.7450980392156863} 0.674& \cellcolor[rgb]{0.984313725490196, 0.8372549019607843, 0.7784313725490196} 0.466& \cellcolor[rgb]{0.9588235294117647, 0.9156862745098039, 0.8921568627450981} 0.467\\

% \midrule

% $\text{AdditiveCoCoA}_{MTE}$ & \cellcolor[rgb]{0.615686274509804, 0.6490196078431373, 0.8764705882352941} -0.279& \cellcolor[rgb]{0.615686274509804, 0.6490196078431373, 0.8764705882352941} -0.058& \cellcolor[rgb]{0.615686274509804, 0.6490196078431373, 0.8764705882352941} -0.072& \cellcolor[rgb]{0.615686274509804, 0.6490196078431373, 0.8764705882352941} 0.098& \cellcolor[rgb]{0.615686274509804, 0.6490196078431373, 0.8764705882352941} 0.312& \cellcolor[rgb]{0.615686274509804, 0.6490196078431373, 0.8764705882352941} 0.079& \cellcolor[rgb]{0.615686274509804, 0.6490196078431373, 0.8764705882352941} 0.187\\
% $\text{FullSampleCoCoA}_{MTE}$ & \cellcolor[rgb]{0.8705882352941177, 0.5607843137254902, 0.5882352941176471} 0.274& \cellcolor[rgb]{0.8607843137254902, 0.5352941176470588, 0.5823529411764706} 0.368& \cellcolor[rgb]{0.8764705882352941, 0.5784313725490197, 0.592156862745098} 0.543& \cellcolor[rgb]{0.9568627450980393, 0.7392156862745098, 0.6862745098039216} 0.309& \cellcolor[rgb]{0.8607843137254902, 0.5352941176470588, 0.5823529411764706} 0.668& \cellcolor[rgb]{0.8607843137254902, 0.5352941176470588, 0.5823529411764706} 0.442& \cellcolor[rgb]{0.9274509803921569, 0.6764705882352942, 0.6431372549019608} 0.456\\
% $\text{CoCoA}_{MTE}$ & \cellcolor[rgb]{0.8529411764705882, 0.5078431372549019, 0.5745098039215686} 0.288& \cellcolor[rgb]{0.8529411764705882, 0.5078431372549019, 0.5745098039215686} 0.374& \cellcolor[rgb]{0.8529411764705882, 0.5078431372549019, 0.5745098039215686} 0.564& \cellcolor[rgb]{0.8529411764705882, 0.5078431372549019, 0.5745098039215686} 0.355& \cellcolor[rgb]{0.8529411764705882, 0.5078431372549019, 0.5745098039215686} 0.673& \cellcolor[rgb]{0.8529411764705882, 0.5078431372549019, 0.5745098039215686} 0.447& \cellcolor[rgb]{0.8529411764705882, 0.5078431372549019, 0.5745098039215686} 0.491\\
% \midrule

% & \multicolumn{7}{c}{Llama8b-Base} \\

% \midrule

% $\text{AdditiveCoCoA}_{MSP}$ & \cellcolor[rgb]{0.9411764705882353, 0.7019607843137254, 0.6588235294117647} 0.330& \cellcolor[rgb]{0.615686274509804, 0.6490196078431373, 0.8764705882352941} 0.345& \cellcolor[rgb]{0.615686274509804, 0.6490196078431373, 0.8764705882352941} 0.462& \cellcolor[rgb]{0.615686274509804, 0.6490196078431373, 0.8764705882352941} 0.301& \cellcolor[rgb]{0.615686274509804, 0.6490196078431373, 0.8764705882352941} 0.566& \cellcolor[rgb]{0.8529411764705882, 0.5078431372549019, 0.5745098039215686} 0.502& \cellcolor[rgb]{0.615686274509804, 0.6490196078431373, 0.8764705882352941} 0.326\\
% $\text{FullSampleCoCoA}_{MSP}$ & \cellcolor[rgb]{0.8921568627450981, 0.6098039215686275, 0.6058823529411764} 0.358& \cellcolor[rgb]{0.9666666666666667, 0.7588235294117647, 0.703921568627451} 0.434& \cellcolor[rgb]{0.9470588235294117, 0.7156862745098039, 0.6686274509803921} 0.564& \cellcolor[rgb]{0.9058823529411765, 0.9274509803921569, 0.9588235294117647} 0.333& \cellcolor[rgb]{0.9372549019607843, 0.9294117647058824, 0.9254901960784314} 0.589& \cellcolor[rgb]{0.984313725490196, 0.8588235294117647, 0.803921568627451} 0.488& \cellcolor[rgb]{0.9745098039215686, 0.8960784313725491, 0.8549019607843137} 0.354\\
% $\text{ProbCoCoA}_{MSP}$ & \cellcolor[rgb]{0.615686274509804, 0.6490196078431373, 0.8764705882352941} 0.031& \cellcolor[rgb]{0.9509803921568627, 0.9215686274509803, 0.9058823529411765} 0.405& \cellcolor[rgb]{0.6588235294117647, 0.7137254901960784, 0.9294117647058824} 0.471& \cellcolor[rgb]{0.8529411764705882, 0.5078431372549019, 0.5745098039215686} 0.371& \cellcolor[rgb]{0.8529411764705882, 0.5078431372549019, 0.5745098039215686} 0.612& \cellcolor[rgb]{0.615686274509804, 0.6490196078431373, 0.8764705882352941} 0.461& \cellcolor[rgb]{0.9176470588235294, 0.6568627450980392, 0.6294117647058823} 0.368\\
% $\text{CoCoA}_{MSP}$ & \cellcolor[rgb]{0.8529411764705882, 0.5078431372549019, 0.5745098039215686} 0.378& \cellcolor[rgb]{0.8529411764705882, 0.5078431372549019, 0.5745098039215686} 0.456& \cellcolor[rgb]{0.8529411764705882, 0.5078431372549019, 0.5745098039215686} 0.582& \cellcolor[rgb]{0.984313725490196, 0.8529411764705882, 0.796078431372549} 0.349& \cellcolor[rgb]{0.984313725490196, 0.8470588235294118, 0.7901960784313725} 0.597& \cellcolor[rgb]{0.9686274509803922, 0.9058823529411765, 0.8745098039215686} 0.485& \cellcolor[rgb]{0.8529411764705882, 0.5078431372549019, 0.5745098039215686} 0.372\\

% \midrule

% $\text{AdditiveCoCoA}_{PPL}$ & \cellcolor[rgb]{0.615686274509804, 0.6490196078431373, 0.8764705882352941} 0.368& \cellcolor[rgb]{0.8784313725490196, 0.9156862745098039, 0.9784313725490196} 0.431& \cellcolor[rgb]{0.9745098039215686, 0.8941176470588236, 0.8529411764705882} 0.504& \cellcolor[rgb]{0.984313725490196, 0.8313725490196079, 0.7725490196078432} 0.336& \cellcolor[rgb]{0.9764705882352941, 0.792156862745098, 0.7333333333333333} 0.595& \cellcolor[rgb]{0.8529411764705882, 0.5078431372549019, 0.5745098039215686} 0.455& \cellcolor[rgb]{0.8764705882352941, 0.5784313725490197, 0.592156862745098} 0.437\\
% $\text{FullSampleCoCoA}_{PPL}$ & \cellcolor[rgb]{0.8529411764705882, 0.5078431372549019, 0.5745098039215686} 0.389& \cellcolor[rgb]{0.615686274509804, 0.6490196078431373, 0.8764705882352941} 0.420& \cellcolor[rgb]{0.615686274509804, 0.6490196078431373, 0.8764705882352941} 0.487& \cellcolor[rgb]{0.615686274509804, 0.6490196078431373, 0.8764705882352941} 0.314& \cellcolor[rgb]{0.615686274509804, 0.6490196078431373, 0.8764705882352941} 0.580& \cellcolor[rgb]{0.8960784313725491, 0.9235294117647059, 0.9686274509803922} 0.450& \cellcolor[rgb]{0.615686274509804, 0.6490196078431373, 0.8764705882352941} 0.399\\
% $\text{ProbCoCoA}_{PPL}$ & \cellcolor[rgb]{0.9784313725490196, 0.888235294117647, 0.8431372549019608} 0.381& \cellcolor[rgb]{0.9156862745098039, 0.6529411764705882, 0.6274509803921569} 0.445& \cellcolor[rgb]{0.8960784313725491, 0.615686274509804, 0.6078431372549019} 0.513& \cellcolor[rgb]{0.8529411764705882, 0.5078431372549019, 0.5745098039215686} 0.345& \cellcolor[rgb]{0.8529411764705882, 0.5078431372549019, 0.5745098039215686} 0.599& \cellcolor[rgb]{0.615686274509804, 0.6490196078431373, 0.8764705882352941} 0.446& \cellcolor[rgb]{0.8529411764705882, 0.5078431372549019, 0.5745098039215686} 0.438\\
% $\text{CoCoA}_{PPL}$ & \cellcolor[rgb]{0.9352941176470588, 0.6901960784313725, 0.6509803921568628} 0.387& \cellcolor[rgb]{0.8529411764705882, 0.5078431372549019, 0.5745098039215686} 0.448& \cellcolor[rgb]{0.8529411764705882, 0.5078431372549019, 0.5745098039215686} 0.514& \cellcolor[rgb]{0.9725490196078431, 0.7764705882352941, 0.7176470588235294} 0.338& \cellcolor[rgb]{0.984313725490196, 0.8588235294117647, 0.803921568627451} 0.593& \cellcolor[rgb]{0.984313725490196, 0.8313725490196079, 0.7725490196078432} 0.452& \cellcolor[rgb]{0.9333333333333333, 0.6862745098039216, 0.6470588235294118} 0.433\\

% \midrule

% $\text{AdditiveCoCoA}_{MTE}$ & \cellcolor[rgb]{0.615686274509804, 0.6490196078431373, 0.8764705882352941} -0.331& \cellcolor[rgb]{0.615686274509804, 0.6490196078431373, 0.8764705882352941} -0.042& \cellcolor[rgb]{0.615686274509804, 0.6490196078431373, 0.8764705882352941} -0.122& \cellcolor[rgb]{0.615686274509804, 0.6490196078431373, 0.8764705882352941} 0.089& \cellcolor[rgb]{0.615686274509804, 0.6490196078431373, 0.8764705882352941} 0.321& \cellcolor[rgb]{0.615686274509804, 0.6490196078431373, 0.8764705882352941} -0.122& \cellcolor[rgb]{0.615686274509804, 0.6490196078431373, 0.8764705882352941} 0.117\\
% $\text{FullSampleCoCoA}_{MTE}$ & \cellcolor[rgb]{0.8529411764705882, 0.5078431372549019, 0.5745098039215686} 0.383& \cellcolor[rgb]{0.9058823529411765, 0.6352941176470588, 0.6176470588235294} 0.410& \cellcolor[rgb]{0.8862745098039215, 0.6, 0.6019607843137255} 0.481& \cellcolor[rgb]{0.9274509803921569, 0.6764705882352942, 0.6431372549019608} 0.308& \cellcolor[rgb]{0.8862745098039215, 0.6, 0.6019607843137255} 0.588& \cellcolor[rgb]{0.9019607843137255, 0.6294117647058823, 0.615686274509804} 0.363& \cellcolor[rgb]{0.9196078431372549, 0.6607843137254902, 0.6333333333333333} 0.414\\
% $\text{CoCoA}_{MTE}$ & \cellcolor[rgb]{0.8549019607843137, 0.5176470588235295, 0.5764705882352941} 0.380& \cellcolor[rgb]{0.8529411764705882, 0.5078431372549019, 0.5745098039215686} 0.446& \cellcolor[rgb]{0.8529411764705882, 0.5078431372549019, 0.5745098039215686} 0.511& \cellcolor[rgb]{0.8529411764705882, 0.5078431372549019, 0.5745098039215686} 0.337& \cellcolor[rgb]{0.8529411764705882, 0.5078431372549019, 0.5745098039215686} 0.601& \cellcolor[rgb]{0.8529411764705882, 0.5078431372549019, 0.5745098039215686} 0.402& \cellcolor[rgb]{0.8529411764705882, 0.5078431372549019, 0.5745098039215686} 0.447\\
% \midrule

% & \multicolumn{7}{c}{Falcon7b-Base} \\

% \midrule

% $\text{AdditiveCoCoA}_{MSP}$ & \cellcolor[rgb]{0.615686274509804, 0.6490196078431373, 0.8764705882352941} 0.203& \cellcolor[rgb]{0.615686274509804, 0.6490196078431373, 0.8764705882352941} 0.318& \cellcolor[rgb]{0.615686274509804, 0.6490196078431373, 0.8764705882352941} 0.409& \cellcolor[rgb]{0.615686274509804, 0.6490196078431373, 0.8764705882352941} 0.350& \cellcolor[rgb]{0.615686274509804, 0.6490196078431373, 0.8764705882352941} 0.674& \cellcolor[rgb]{0.8529411764705882, 0.5078431372549019, 0.5745098039215686} 0.533& \cellcolor[rgb]{0.615686274509804, 0.6490196078431373, 0.8764705882352941} 0.379\\
% $\text{FullSampleCoCoA}_{MSP}$ & \cellcolor[rgb]{0.8764705882352941, 0.9156862745098039, 0.9803921568627451} 0.225& \cellcolor[rgb]{0.9098039215686274, 0.6431372549019608, 0.6235294117647059} 0.423& \cellcolor[rgb]{0.8823529411764706, 0.5901960784313726, 0.596078431372549} 0.571& \cellcolor[rgb]{0.9647058823529412, 0.9117647058823529, 0.8823529411764706} 0.388& \cellcolor[rgb]{0.8666666666666667, 0.9098039215686274, 0.9862745098039216} 0.678& \cellcolor[rgb]{0.8862745098039215, 0.6, 0.6019607843137255} 0.533& \cellcolor[rgb]{0.9, 0.9254901960784314, 0.9647058823529412} 0.404\\
% $\text{ProbCoCoA}_{MSP}$ & \cellcolor[rgb]{0.888235294117647, 0.9196078431372549, 0.9725490196078431} 0.226& \cellcolor[rgb]{0.8960784313725491, 0.9235294117647059, 0.9686274509803922} 0.367& \cellcolor[rgb]{0.9784313725490196, 0.8862745098039215, 0.8392156862745098} 0.515& \cellcolor[rgb]{0.8529411764705882, 0.5078431372549019, 0.5745098039215686} 0.416& \cellcolor[rgb]{0.9784313725490196, 0.8862745098039215, 0.8392156862745098} 0.680& \cellcolor[rgb]{0.615686274509804, 0.6490196078431373, 0.8764705882352941} 0.526& \cellcolor[rgb]{0.9568627450980393, 0.7352941176470589, 0.6823529411764706} 0.426\\
% $\text{CoCoA}_{MSP}$ & \cellcolor[rgb]{0.8529411764705882, 0.5078431372549019, 0.5745098039215686} 0.257& \cellcolor[rgb]{0.8529411764705882, 0.5078431372549019, 0.5745098039215686} 0.433& \cellcolor[rgb]{0.8529411764705882, 0.5078431372549019, 0.5745098039215686} 0.578& \cellcolor[rgb]{0.984313725490196, 0.8372549019607843, 0.7784313725490196} 0.396& \cellcolor[rgb]{0.8529411764705882, 0.5078431372549019, 0.5745098039215686} 0.684& \cellcolor[rgb]{0.9274509803921569, 0.9313725490196079, 0.9392156862745098} 0.529& \cellcolor[rgb]{0.8529411764705882, 0.5078431372549019, 0.5745098039215686} 0.436\\

% \midrule

% $\text{AdditiveCoCoA}_{PPL}$ & \cellcolor[rgb]{0.9686274509803922, 0.9058823529411765, 0.8745098039215686} 0.222& \cellcolor[rgb]{0.9803921568627451, 0.8058823529411765, 0.7450980392156863} 0.433& \cellcolor[rgb]{0.8529411764705882, 0.5078431372549019, 0.5745098039215686} 0.580& \cellcolor[rgb]{0.8529411764705882, 0.5078431372549019, 0.5745098039215686} 0.413& \cellcolor[rgb]{0.9745098039215686, 0.7862745098039216, 0.7274509803921568} 0.677& \cellcolor[rgb]{0.615686274509804, 0.6490196078431373, 0.8764705882352941} 0.525& \cellcolor[rgb]{0.8529411764705882, 0.5078431372549019, 0.5745098039215686} 0.489\\
% $\text{FullSampleCoCoA}_{PPL}$ & \cellcolor[rgb]{0.615686274509804, 0.6490196078431373, 0.8764705882352941} 0.204& \cellcolor[rgb]{0.615686274509804, 0.6490196078431373, 0.8764705882352941} 0.425& \cellcolor[rgb]{0.615686274509804, 0.6490196078431373, 0.8764705882352941} 0.565& \cellcolor[rgb]{0.615686274509804, 0.6490196078431373, 0.8764705882352941} 0.393& \cellcolor[rgb]{0.615686274509804, 0.6490196078431373, 0.8764705882352941} 0.669& \cellcolor[rgb]{0.8529411764705882, 0.5078431372549019, 0.5745098039215686} 0.533& \cellcolor[rgb]{0.615686274509804, 0.6490196078431373, 0.8764705882352941} 0.437\\
% $\text{ProbCoCoA}_{PPL}$ & \cellcolor[rgb]{0.8529411764705882, 0.5078431372549019, 0.5745098039215686} 0.235& \cellcolor[rgb]{0.984313725490196, 0.8372549019607843, 0.7784313725490196} 0.433& \cellcolor[rgb]{0.9803921568627451, 0.807843137254902, 0.7470588235294118} 0.576& \cellcolor[rgb]{0.9411764705882353, 0.7019607843137254, 0.6588235294117647} 0.410& \cellcolor[rgb]{0.8529411764705882, 0.5078431372549019, 0.5745098039215686} 0.680& \cellcolor[rgb]{0.8352941176470589, 0.8921568627450981, 0.996078431372549} 0.528& \cellcolor[rgb]{0.9392156862745098, 0.6980392156862745, 0.6549019607843137} 0.482\\
% $\text{CoCoA}_{PPL}$ & \cellcolor[rgb]{0.9627450980392157, 0.7529411764705882, 0.6980392156862745} 0.229& \cellcolor[rgb]{0.8529411764705882, 0.5078431372549019, 0.5745098039215686} 0.436& \cellcolor[rgb]{0.8666666666666667, 0.5529411764705883, 0.5862745098039216} 0.580& \cellcolor[rgb]{0.984313725490196, 0.8647058823529412, 0.8117647058823529} 0.406& \cellcolor[rgb]{0.9803921568627451, 0.8137254901960784, 0.7529411764705882} 0.677& \cellcolor[rgb]{0.9607843137254902, 0.9137254901960784, 0.888235294117647} 0.529& \cellcolor[rgb]{0.9686274509803922, 0.7666666666666667, 0.7098039215686275} 0.478\\

% \midrule

% $\text{AdditiveCoCoA}_{MTE}$ & \cellcolor[rgb]{0.615686274509804, 0.6490196078431373, 0.8764705882352941} 0.001& \cellcolor[rgb]{0.615686274509804, 0.6490196078431373, 0.8764705882352941} -0.103& \cellcolor[rgb]{0.615686274509804, 0.6490196078431373, 0.8764705882352941} -0.106& \cellcolor[rgb]{0.615686274509804, 0.6490196078431373, 0.8764705882352941} 0.114& \cellcolor[rgb]{0.615686274509804, 0.6490196078431373, 0.8764705882352941} 0.041& \cellcolor[rgb]{0.615686274509804, 0.6490196078431373, 0.8764705882352941} 0.138& \cellcolor[rgb]{0.615686274509804, 0.6490196078431373, 0.8764705882352941} 0.221\\
% $\text{FullSampleCoCoA}_{MTE}$ & \cellcolor[rgb]{0.9313725490196079, 0.6823529411764706, 0.6450980392156863} 0.201& \cellcolor[rgb]{0.8705882352941177, 0.5607843137254902, 0.5882352941176471} 0.425& \cellcolor[rgb]{0.8725490196078431, 0.5705882352941176, 0.5901960784313726} 0.557& \cellcolor[rgb]{0.9, 0.6254901960784314, 0.6137254901960785} 0.377& \cellcolor[rgb]{0.8607843137254902, 0.5352941176470588, 0.5823529411764706} 0.675& \cellcolor[rgb]{0.8529411764705882, 0.5078431372549019, 0.5745098039215686} 0.519& \cellcolor[rgb]{0.9411764705882353, 0.7019607843137254, 0.6588235294117647} 0.470\\
% $\text{CoCoA}_{MTE}$ & \cellcolor[rgb]{0.8529411764705882, 0.5078431372549019, 0.5745098039215686} 0.228& \cellcolor[rgb]{0.8529411764705882, 0.5078431372549019, 0.5745098039215686} 0.439& \cellcolor[rgb]{0.8529411764705882, 0.5078431372549019, 0.5745098039215686} 0.577& \cellcolor[rgb]{0.8529411764705882, 0.5078431372549019, 0.5745098039215686} 0.395& \cellcolor[rgb]{0.8529411764705882, 0.5078431372549019, 0.5745098039215686} 0.685& \cellcolor[rgb]{0.8549019607843137, 0.5176470588235295, 0.5764705882352941} 0.517& \cellcolor[rgb]{0.8529411764705882, 0.5078431372549019, 0.5745098039215686} 0.510\\
% \midrule

% \end{tabular}
% \end{table*}


\begin{table*}[h!]
\footnotesize
\centering

\begin{tabular}{lrrrrrrr}
\toprule
    \multirow{2}{*}{\textbf{Method}}  & \multicolumn{7}{c}{\textbf{Dataset}}  \\ 
      \cmidrule(lr){2-8}  \\
  & XSum & WMT14FrEn & WMT19DeEn & CoQa & Trivia & MMLU & GSM8k \\
  \midrule

\rowcolor[gray]{0.9} & \multicolumn{7}{c}{Mistral7b-Base} \\

\midrule

$\text{AdditiveCoCoA}_{MSP}$ & 0.290& 0.319& 0.459& 0.351& 0.654& \textbf{0.471}& 0.472\\
$\text{FullSampleCoCoA}_{MSP}$ & \underline{0.319}& 0.385& \underline{0.590}& 0.357& 0.668& \underline{0.467}& \underline{0.505}\\
$\text{ProbCoCoA}_{MSP}$ & 0.059& 0.302& 0.520& \textbf{0.390}& 0.671& 0.461& 0.435\\
$\text{CoCoA}_{MSP}$ & \textbf{0.330}& \textbf{0.396}& \textbf{0.598}& \underline{0.383}& 0.670& 0.466& \textbf{0.517}\\

\midrule

$\text{AdditiveCoCoA}_{PPL}$ & 0.262& \underline{0.392}& 0.564& 0.369& 0.671& 0.464& 0.494\\
$\text{FullSampleCoCoA}_{PPL}$ & 0.277& 0.373& 0.551& 0.334& 0.672& \underline{0.467}& 0.435\\
$\text{ProbCoCoA}_{PPL}$ & 0.297& 0.369& 0.566& 0.373& \textbf{0.674}& 0.464& 0.475\\
$\text{CoCoA}_{PPL}$ & 0.286& 0.375& 0.568& 0.369& \textbf{0.674}& 0.466& 0.467\\

\midrule

$\text{AdditiveCoCoA}_{MTE}$ & -0.279& -0.058& -0.072& 0.098& 0.312& 0.079& 0.187\\
$\text{FullSampleCoCoA}_{MTE}$ & 0.274& 0.368& 0.543& 0.309& 0.668& 0.442& 0.456\\
$\text{CoCoA}_{MTE}$ & 0.288& 0.374& 0.564& 0.355& 0.673& 0.447& 0.491\\
\midrule

\rowcolor[gray]{0.9} & \multicolumn{7}{c}{Llama8b-Base} \\

\midrule

$\text{AdditiveCoCoA}_{MSP}$ & 0.330& 0.345& 0.462& 0.301& 0.566& \textbf{0.502}& 0.326\\
$\text{FullSampleCoCoA}_{MSP}$ & 0.358& 0.434& \underline{0.564}& 0.333& 0.589& \underline{0.488}& 0.354\\
$\text{ProbCoCoA}_{MSP}$ & 0.031& 0.405& 0.471& \textbf{0.371}& \textbf{0.612}& 0.461& 0.368\\
$\text{CoCoA}_{MSP}$ & 0.378& \textbf{0.456}& \textbf{0.582}& \underline{0.349}& 0.597& 0.485& 0.372\\

\midrule

$\text{AdditiveCoCoA}_{PPL}$ & 0.368& 0.431& 0.504& 0.336& 0.595& 0.455& 0.437\\
$\text{FullSampleCoCoA}_{PPL}$ & \textbf{0.389}& 0.420& 0.487& 0.314& 0.580& 0.450& 0.399\\
$\text{ProbCoCoA}_{PPL}$ & 0.381& 0.445& 0.513& 0.345& 0.599& 0.446& \underline{0.438}\\
$\text{CoCoA}_{PPL}$ & \underline{0.387}& \underline{0.448}& 0.514& 0.338& 0.593& 0.452& 0.433\\

\midrule

$\text{AdditiveCoCoA}_{MTE}$ & -0.331& -0.042& -0.122& 0.089& 0.321& -0.122& 0.117\\
$\text{FullSampleCoCoA}_{MTE}$ & 0.383& 0.410& 0.481& 0.308& 0.588& 0.363& 0.414\\
$\text{CoCoA}_{MTE}$ & 0.380& 0.446& 0.511& 0.337& \underline{0.601}& 0.402& \textbf{0.447}\\
\midrule

\rowcolor[gray]{0.9} & \multicolumn{7}{c}{Falcon7b-Base} \\

\midrule

$\text{AdditiveCoCoA}_{MSP}$ & 0.203& 0.318& 0.409& 0.350& 0.674& \textbf{0.533}& 0.379\\
$\text{FullSampleCoCoA}_{MSP}$ & 0.225& 0.423& 0.571& 0.388& 0.678& \textbf{0.533}& 0.404\\
$\text{ProbCoCoA}_{MSP}$ & 0.226& 0.367& 0.515& \textbf{0.416}& 0.680& 0.526& 0.426\\
$\text{CoCoA}_{MSP}$ & \textbf{0.257}& 0.433& 0.578& 0.396& \underline{0.684}& 0.529& 0.436\\

\midrule

$\text{AdditiveCoCoA}_{PPL}$ & 0.222& 0.433& \textbf{0.580}& \underline{0.413}& 0.677& 0.525& \underline{0.489}\\
$\text{FullSampleCoCoA}_{PPL}$ & 0.204& 0.425& 0.565& 0.393& 0.669& \textbf{0.533}& 0.437\\
$\text{ProbCoCoA}_{PPL}$ & \underline{0.235}& 0.433& 0.576& 0.410& 0.680& 0.528& 0.482\\
$\text{CoCoA}_{PPL}$ & 0.229& \underline{0.436}& \textbf{0.580}& 0.406& 0.677& 0.529& 0.478\\

\midrule

$\text{AdditiveCoCoA}_{MTE}$ & 0.001& -0.103& -0.106& 0.114& 0.041& 0.138& 0.221\\
$\text{FullSampleCoCoA}_{MTE}$ & 0.201& 0.425& 0.557& 0.377& 0.675& 0.519& 0.470\\
$\text{CoCoA}_{MTE}$ & 0.228& \textbf{0.439}& 0.577& 0.395& \textbf{0.685}& 0.517& \textbf{0.510}\\
\bottomrule

\end{tabular}
\caption{Comparison of PRRs of \texttt{CoCoA}-family methods with alternative formulations with greedy sample taken in isolation.}
\label{tab:ablation_cocoa_greedy}
\end{table*}

  \newpage
  % \begin{table*}
% \footnotesize
% \caption{Best Sample PRRs for falcon7b on all tasks}
% \begin{tabular}{lrrrrrrr}
%  & XSum & WMT14FrEn & WMT19DeEn & CoQa & Trivia & MMLU & GSM8k \\

% \midrule

% & \multicolumn{7}{c}{Mistral7b-Base} \\

% \midrule

% $\text{AdditiveCoCoA}_{MSP}$ & \cellcolor[rgb]{0.9254901960784314, 0.6725490196078432, 0.6392156862745098} 0.333& \cellcolor[rgb]{0.615686274509804, 0.6490196078431373, 0.8764705882352941} 0.239& \cellcolor[rgb]{0.615686274509804, 0.6490196078431373, 0.8764705882352941} 0.310& \cellcolor[rgb]{0.615686274509804, 0.6490196078431373, 0.8764705882352941} 0.406& \cellcolor[rgb]{0.615686274509804, 0.6490196078431373, 0.8764705882352941} 0.631& \cellcolor[rgb]{0.8529411764705882, 0.5078431372549019, 0.5745098039215686} 0.472& \cellcolor[rgb]{0.615686274509804, 0.6490196078431373, 0.8764705882352941} 0.311\\
% $\text{FullSampleCoCoA}_{MSP}$ & \cellcolor[rgb]{0.8823529411764706, 0.5901960784313726, 0.596078431372549} 0.354& \cellcolor[rgb]{0.9823529411764705, 0.8745098039215686, 0.8254901960784313} 0.543& \cellcolor[rgb]{0.9745098039215686, 0.8941176470588236, 0.8529411764705882} 0.565& \cellcolor[rgb]{0.7745098039215687, 0.8431372549019608, 0.9980392156862745} 0.412& \cellcolor[rgb]{0.9274509803921569, 0.6764705882352942, 0.6431372549019608} 0.643& \cellcolor[rgb]{0.9137254901960784, 0.9294117647058824, 0.9529411764705882} 0.468& \cellcolor[rgb]{0.7529411764705882, 0.8215686274509804, 0.9921568627450981} 0.428\\
% $\text{ProbCoCoA}_{MSP}$ & \cellcolor[rgb]{0.615686274509804, 0.6490196078431373, 0.8764705882352941} 0.076& \cellcolor[rgb]{0.8960784313725491, 0.615686274509804, 0.6078431372549019} 0.684& \cellcolor[rgb]{0.8666666666666667, 0.5529411764705883, 0.5862745098039216} 0.721& \cellcolor[rgb]{0.8862745098039215, 0.6, 0.6019607843137255} 0.428& \cellcolor[rgb]{0.8960784313725491, 0.615686274509804, 0.6078431372549019} 0.643& \cellcolor[rgb]{0.615686274509804, 0.6490196078431373, 0.8764705882352941} 0.464& \cellcolor[rgb]{0.8529411764705882, 0.5078431372549019, 0.5745098039215686} 0.846\\
% $\text{CoCoA}_{MSP}$ & \cellcolor[rgb]{0.8529411764705882, 0.5078431372549019, 0.5745098039215686} 0.366& \cellcolor[rgb]{0.8529411764705882, 0.5078431372549019, 0.5745098039215686} 0.712& \cellcolor[rgb]{0.8529411764705882, 0.5078431372549019, 0.5745098039215686} 0.730& \cellcolor[rgb]{0.8529411764705882, 0.5078431372549019, 0.5745098039215686} 0.430& \cellcolor[rgb]{0.8529411764705882, 0.5078431372549019, 0.5745098039215686} 0.644& \cellcolor[rgb]{0.7823529411764706, 0.8490196078431372, 0.9980392156862745} 0.466& \cellcolor[rgb]{0.9176470588235294, 0.9294117647058824, 0.9490196078431372} 0.562\\

% \midrule

% $\text{AdditiveCoCoA}_{PPL}$ & \cellcolor[rgb]{0.615686274509804, 0.6490196078431373, 0.8764705882352941} 0.368& \cellcolor[rgb]{0.8901960784313725, 0.6058823529411764, 0.6039215686274509} 0.737& \cellcolor[rgb]{0.9431372549019608, 0.7058823529411765, 0.6607843137254902} 0.751& \cellcolor[rgb]{0.8529411764705882, 0.5078431372549019, 0.5745098039215686} 0.406& \cellcolor[rgb]{0.615686274509804, 0.6490196078431373, 0.8764705882352941} 0.644& \cellcolor[rgb]{0.615686274509804, 0.6490196078431373, 0.8764705882352941} 0.465& \cellcolor[rgb]{0.8529411764705882, 0.5078431372549019, 0.5745098039215686} 0.939\\
% $\text{FullSampleCoCoA}_{PPL}$ & \cellcolor[rgb]{0.8529411764705882, 0.5078431372549019, 0.5745098039215686} 0.383& \cellcolor[rgb]{0.615686274509804, 0.6490196078431373, 0.8764705882352941} 0.714& \cellcolor[rgb]{0.615686274509804, 0.6490196078431373, 0.8764705882352941} 0.723& \cellcolor[rgb]{0.615686274509804, 0.6490196078431373, 0.8764705882352941} 0.379& \cellcolor[rgb]{0.8960784313725491, 0.615686274509804, 0.6078431372549019} 0.649& \cellcolor[rgb]{0.8529411764705882, 0.5078431372549019, 0.5745098039215686} 0.468& \cellcolor[rgb]{0.615686274509804, 0.6490196078431373, 0.8764705882352941} 0.933\\
% $\text{ProbCoCoA}_{PPL}$ & \cellcolor[rgb]{0.6568627450980392, 0.7098039215686275, 0.9274509803921569} 0.369& \cellcolor[rgb]{0.8529411764705882, 0.5078431372549019, 0.5745098039215686} 0.738& \cellcolor[rgb]{0.8529411764705882, 0.5078431372549019, 0.5745098039215686} 0.756& \cellcolor[rgb]{0.9568627450980393, 0.7392156862745098, 0.6862745098039216} 0.401& \cellcolor[rgb]{0.8529411764705882, 0.5078431372549019, 0.5745098039215686} 0.649& \cellcolor[rgb]{0.9647058823529412, 0.9098039215686274, 0.8803921568627451} 0.467& \cellcolor[rgb]{0.8352941176470589, 0.8921568627450981, 0.996078431372549} 0.935\\
% $\text{CoCoA}_{PPL}$ & \cellcolor[rgb]{0.7862745098039216, 0.8529411764705882, 0.9980392156862745} 0.372& \cellcolor[rgb]{0.9372549019607843, 0.6941176470588235, 0.6529411764705882} 0.735& \cellcolor[rgb]{0.8764705882352941, 0.5784313725490197, 0.592156862745098} 0.755& \cellcolor[rgb]{0.9411764705882353, 0.7019607843137254, 0.6588235294117647} 0.402& \cellcolor[rgb]{0.9705882352941176, 0.7705882352941177, 0.711764705882353} 0.648& \cellcolor[rgb]{0.9235294117647059, 0.9313725490196079, 0.9431372549019608} 0.466& \cellcolor[rgb]{0.984313725490196, 0.8647058823529412, 0.8117647058823529} 0.937\\

% \midrule

% $\text{AdditiveCoCoA}_{MTE}$ & \cellcolor[rgb]{0.615686274509804, 0.6490196078431373, 0.8764705882352941} 0.368& \cellcolor[rgb]{0.8529411764705882, 0.5078431372549019, 0.5745098039215686} 0.723& \cellcolor[rgb]{0.9803921568627451, 0.884313725490196, 0.8372549019607843} 0.702& \cellcolor[rgb]{0.6176470588235294, 0.6529411764705882, 0.8803921568627451} 0.332& \cellcolor[rgb]{0.7549019607843137, 0.8254901960784313, 0.9921568627450981} 0.643& \cellcolor[rgb]{0.8529411764705882, 0.5078431372549019, 0.5745098039215686} 0.452& \cellcolor[rgb]{0.8529411764705882, 0.5078431372549019, 0.5745098039215686} 0.942\\
% $\text{FullSampleCoCoA}_{MTE}$ & \cellcolor[rgb]{0.8529411764705882, 0.5078431372549019, 0.5745098039215686} 0.380& \cellcolor[rgb]{0.615686274509804, 0.6490196078431373, 0.8764705882352941} 0.661& \cellcolor[rgb]{0.615686274509804, 0.6490196078431373, 0.8764705882352941} 0.653& \cellcolor[rgb]{0.615686274509804, 0.6490196078431373, 0.8764705882352941} 0.331& \cellcolor[rgb]{0.615686274509804, 0.6490196078431373, 0.8764705882352941} 0.643& \cellcolor[rgb]{0.615686274509804, 0.6490196078431373, 0.8764705882352941} 0.442& \cellcolor[rgb]{0.615686274509804, 0.6490196078431373, 0.8764705882352941} 0.929\\
% $\text{CoCoA}_{MTE}$ & \cellcolor[rgb]{0.8686274509803922, 0.9117647058823529, 0.984313725490196} 0.373& \cellcolor[rgb]{0.9764705882352941, 0.796078431372549, 0.7352941176470589} 0.708& \cellcolor[rgb]{0.8529411764705882, 0.5078431372549019, 0.5745098039215686} 0.732& \cellcolor[rgb]{0.8529411764705882, 0.5078431372549019, 0.5745098039215686} 0.373& \cellcolor[rgb]{0.8529411764705882, 0.5078431372549019, 0.5745098039215686} 0.645& \cellcolor[rgb]{0.9431372549019608, 0.9274509803921569, 0.9176470588235294} 0.447& \cellcolor[rgb]{0.8941176470588236, 0.9215686274509803, 0.9705882352941176} 0.935\\
% \midrule

% & \multicolumn{7}{c}{Llama8b-Base} \\

% \midrule

% $\text{AdditiveCoCoA}_{MSP}$ & \cellcolor[rgb]{0.8823529411764706, 0.5901960784313726, 0.596078431372549} 0.466& \cellcolor[rgb]{0.615686274509804, 0.6490196078431373, 0.8764705882352941} 0.349& \cellcolor[rgb]{0.615686274509804, 0.6490196078431373, 0.8764705882352941} 0.425& \cellcolor[rgb]{0.615686274509804, 0.6490196078431373, 0.8764705882352941} 0.333& \cellcolor[rgb]{0.615686274509804, 0.6490196078431373, 0.8764705882352941} 0.555& \cellcolor[rgb]{0.8529411764705882, 0.5078431372549019, 0.5745098039215686} 0.473& \cellcolor[rgb]{0.615686274509804, 0.6490196078431373, 0.8764705882352941} 0.285\\
% $\text{FullSampleCoCoA}_{MSP}$ & \cellcolor[rgb]{0.8647058823529412, 0.5431372549019607, 0.5843137254901961} 0.476& \cellcolor[rgb]{0.9803921568627451, 0.8803921568627451, 0.8333333333333334} 0.462& \cellcolor[rgb]{0.9784313725490196, 0.8019607843137255, 0.7411764705882353} 0.619& \cellcolor[rgb]{0.9117647058823529, 0.9294117647058824, 0.9549019607843138} 0.363& \cellcolor[rgb]{0.9019607843137255, 0.9254901960784314, 0.9627450980392157} 0.574& \cellcolor[rgb]{0.9705882352941176, 0.7725490196078432, 0.7156862745098039} 0.464& \cellcolor[rgb]{0.7313725490196078, 0.8, 0.9823529411764705} 0.379\\
% $\text{ProbCoCoA}_{MSP}$ & \cellcolor[rgb]{0.615686274509804, 0.6490196078431373, 0.8764705882352941} 0.035& \cellcolor[rgb]{0.9705882352941176, 0.7705882352941177, 0.711764705882353} 0.491& \cellcolor[rgb]{0.9803921568627451, 0.807843137254902, 0.7470588235294118} 0.617& \cellcolor[rgb]{0.8529411764705882, 0.5078431372549019, 0.5745098039215686} 0.398& \cellcolor[rgb]{0.8529411764705882, 0.5078431372549019, 0.5745098039215686} 0.598& \cellcolor[rgb]{0.615686274509804, 0.6490196078431373, 0.8764705882352941} 0.433& \cellcolor[rgb]{0.8529411764705882, 0.5078431372549019, 0.5745098039215686} 0.795\\
% $\text{CoCoA}_{MSP}$ & \cellcolor[rgb]{0.8529411764705882, 0.5078431372549019, 0.5745098039215686} 0.484& \cellcolor[rgb]{0.8529411764705882, 0.5078431372549019, 0.5745098039215686} 0.529& \cellcolor[rgb]{0.8529411764705882, 0.5078431372549019, 0.5745098039215686} 0.685& \cellcolor[rgb]{0.9686274509803922, 0.7666666666666667, 0.7098039215686275} 0.384& \cellcolor[rgb]{0.9784313725490196, 0.8019607843137255, 0.7411764705882353} 0.587& \cellcolor[rgb]{0.9235294117647059, 0.9313725490196079, 0.9431372549019608} 0.452& \cellcolor[rgb]{0.903921568627451, 0.9274509803921569, 0.9607843137254902} 0.513\\

% \midrule

% $\text{AdditiveCoCoA}_{PPL}$ & \cellcolor[rgb]{0.9705882352941176, 0.7705882352941177, 0.711764705882353} 0.454& \cellcolor[rgb]{0.9450980392156862, 0.9254901960784314, 0.9137254901960784} 0.536& \cellcolor[rgb]{0.9725490196078431, 0.9, 0.8607843137254902} 0.673& \cellcolor[rgb]{0.984313725490196, 0.8529411764705882, 0.796078431372549} 0.358& \cellcolor[rgb]{0.8529411764705882, 0.5078431372549019, 0.5745098039215686} 0.575& \cellcolor[rgb]{0.9725490196078431, 0.9, 0.8607843137254902} 0.425& \cellcolor[rgb]{0.9627450980392157, 0.7490196078431373, 0.6941176470588235} 0.923\\
% $\text{FullSampleCoCoA}_{PPL}$ & \cellcolor[rgb]{0.8529411764705882, 0.5078431372549019, 0.5745098039215686} 0.459& \cellcolor[rgb]{0.615686274509804, 0.6490196078431373, 0.8764705882352941} 0.525& \cellcolor[rgb]{0.615686274509804, 0.6490196078431373, 0.8764705882352941} 0.649& \cellcolor[rgb]{0.615686274509804, 0.6490196078431373, 0.8764705882352941} 0.343& \cellcolor[rgb]{0.615686274509804, 0.6490196078431373, 0.8764705882352941} 0.556& \cellcolor[rgb]{0.8529411764705882, 0.5078431372549019, 0.5745098039215686} 0.430& \cellcolor[rgb]{0.615686274509804, 0.6490196078431373, 0.8764705882352941} 0.914\\
% $\text{ProbCoCoA}_{PPL}$ & \cellcolor[rgb]{0.615686274509804, 0.6490196078431373, 0.8764705882352941} 0.438& \cellcolor[rgb]{0.8529411764705882, 0.5078431372549019, 0.5745098039215686} 0.547& \cellcolor[rgb]{0.8529411764705882, 0.5078431372549019, 0.5745098039215686} 0.689& \cellcolor[rgb]{0.8529411764705882, 0.5078431372549019, 0.5745098039215686} 0.364& \cellcolor[rgb]{0.8647058823529412, 0.5431372549019607, 0.5843137254901961} 0.574& \cellcolor[rgb]{0.615686274509804, 0.6490196078431373, 0.8764705882352941} 0.419& \cellcolor[rgb]{0.9490196078431372, 0.7196078431372549, 0.6725490196078432} 0.923\\
% $\text{CoCoA}_{PPL}$ & \cellcolor[rgb]{0.9725490196078431, 0.9, 0.8607843137254902} 0.450& \cellcolor[rgb]{0.9274509803921569, 0.6764705882352942, 0.6431372549019608} 0.544& \cellcolor[rgb]{0.8607843137254902, 0.5352941176470588, 0.5823529411764706} 0.689& \cellcolor[rgb]{0.8607843137254902, 0.5352941176470588, 0.5823529411764706} 0.364& \cellcolor[rgb]{0.9274509803921569, 0.6764705882352942, 0.6431372549019608} 0.573& \cellcolor[rgb]{0.8235294117647058, 0.8823529411764706, 0.9980392156862745} 0.422& \cellcolor[rgb]{0.8529411764705882, 0.5078431372549019, 0.5745098039215686} 0.925\\

% \midrule

% $\text{AdditiveCoCoA}_{MTE}$ & \cellcolor[rgb]{0.8529411764705882, 0.5078431372549019, 0.5745098039215686} 0.457& \cellcolor[rgb]{0.9666666666666667, 0.907843137254902, 0.8764705882352941} 0.496& \cellcolor[rgb]{0.6294117647058823, 0.6705882352941176, 0.8941176470588236} 0.579& \cellcolor[rgb]{0.615686274509804, 0.6490196078431373, 0.8764705882352941} 0.304& \cellcolor[rgb]{0.615686274509804, 0.6490196078431373, 0.8764705882352941} 0.561& \cellcolor[rgb]{0.9098039215686274, 0.6431372549019608, 0.6235294117647059} 0.361& \cellcolor[rgb]{0.8529411764705882, 0.5078431372549019, 0.5745098039215686} 0.901\\
% $\text{FullSampleCoCoA}_{MTE}$ & \cellcolor[rgb]{0.984313725490196, 0.8529411764705882, 0.796078431372549} 0.455& \cellcolor[rgb]{0.615686274509804, 0.6490196078431373, 0.8764705882352941} 0.464& \cellcolor[rgb]{0.615686274509804, 0.6490196078431373, 0.8764705882352941} 0.577& \cellcolor[rgb]{0.7450980392156863, 0.8137254901960784, 0.9882352941176471} 0.313& \cellcolor[rgb]{0.6490196078431373, 0.7, 0.9196078431372549} 0.563& \cellcolor[rgb]{0.615686274509804, 0.6490196078431373, 0.8764705882352941} 0.341& \cellcolor[rgb]{0.615686274509804, 0.6490196078431373, 0.8764705882352941} 0.878\\
% $\text{CoCoA}_{MTE}$ & \cellcolor[rgb]{0.615686274509804, 0.6490196078431373, 0.8764705882352941} 0.451& \cellcolor[rgb]{0.8529411764705882, 0.5078431372549019, 0.5745098039215686} 0.520& \cellcolor[rgb]{0.8529411764705882, 0.5078431372549019, 0.5745098039215686} 0.638& \cellcolor[rgb]{0.8529411764705882, 0.5078431372549019, 0.5745098039215686} 0.346& \cellcolor[rgb]{0.8529411764705882, 0.5078431372549019, 0.5745098039215686} 0.582& \cellcolor[rgb]{0.8529411764705882, 0.5078431372549019, 0.5745098039215686} 0.363& \cellcolor[rgb]{0.8666666666666667, 0.5529411764705883, 0.5862745098039216} 0.900\\
% \midrule

% & \multicolumn{7}{c}{Falcon7b-Base} \\

% \midrule

% $\text{AdditiveCoCoA}_{MSP}$ & \cellcolor[rgb]{0.615686274509804, 0.6490196078431373, 0.8764705882352941} 0.100& \cellcolor[rgb]{0.615686274509804, 0.6490196078431373, 0.8764705882352941} 0.397& \cellcolor[rgb]{0.615686274509804, 0.6490196078431373, 0.8764705882352941} 0.394& \cellcolor[rgb]{0.615686274509804, 0.6490196078431373, 0.8764705882352941} 0.393& \cellcolor[rgb]{0.615686274509804, 0.6490196078431373, 0.8764705882352941} 0.649& \cellcolor[rgb]{0.8529411764705882, 0.5078431372549019, 0.5745098039215686} 0.534& \cellcolor[rgb]{0.615686274509804, 0.6490196078431373, 0.8764705882352941} -0.156\\
% $\text{FullSampleCoCoA}_{MSP}$ & \cellcolor[rgb]{0.7686274509803922, 0.8372549019607843, 0.996078431372549} 0.144& \cellcolor[rgb]{0.9647058823529412, 0.7568627450980392, 0.7} 0.531& \cellcolor[rgb]{0.984313725490196, 0.8372549019607843, 0.7784313725490196} 0.607& \cellcolor[rgb]{0.9607843137254902, 0.9137254901960784, 0.888235294117647} 0.416& \cellcolor[rgb]{0.9294117647058824, 0.9313725490196079, 0.9352941176470588} 0.654& \cellcolor[rgb]{0.9411764705882353, 0.7019607843137254, 0.6588235294117647} 0.533& \cellcolor[rgb]{0.8117647058823529, 0.8745098039215686, 1.0} 0.189\\
% $\text{ProbCoCoA}_{MSP}$ & \cellcolor[rgb]{0.8529411764705882, 0.5078431372549019, 0.5745098039215686} 0.282& \cellcolor[rgb]{0.9784313725490196, 0.8019607843137255, 0.7411764705882353} 0.522& \cellcolor[rgb]{0.9156862745098039, 0.6529411764705882, 0.6274509803921569} 0.670& \cellcolor[rgb]{0.8529411764705882, 0.5078431372549019, 0.5745098039215686} 0.434& \cellcolor[rgb]{0.9058823529411765, 0.6352941176470588, 0.6176470588235294} 0.658& \cellcolor[rgb]{0.615686274509804, 0.6490196078431373, 0.8764705882352941} 0.529& \cellcolor[rgb]{0.8529411764705882, 0.5078431372549019, 0.5745098039215686} 0.978\\
% $\text{CoCoA}_{MSP}$ & \cellcolor[rgb]{0.9745098039215686, 0.8960784313725491, 0.8549019607843137} 0.210& \cellcolor[rgb]{0.8529411764705882, 0.5078431372549019, 0.5745098039215686} 0.564& \cellcolor[rgb]{0.8529411764705882, 0.5078431372549019, 0.5745098039215686} 0.698& \cellcolor[rgb]{0.9490196078431372, 0.7196078431372549, 0.6725490196078432} 0.428& \cellcolor[rgb]{0.8529411764705882, 0.5078431372549019, 0.5745098039215686} 0.659& \cellcolor[rgb]{0.6666666666666666, 0.7235294117647059, 0.9372549019607843} 0.530& \cellcolor[rgb]{0.9666666666666667, 0.907843137254902, 0.8764705882352941} 0.498\\

% \midrule

% $\text{AdditiveCoCoA}_{PPL}$ & \cellcolor[rgb]{0.615686274509804, 0.6490196078431373, 0.8764705882352941} 0.297& \cellcolor[rgb]{0.9549019607843138, 0.7313725490196078, 0.6803921568627451} 0.582& \cellcolor[rgb]{0.9784313725490196, 0.7980392156862746, 0.7392156862745098} 0.706& \cellcolor[rgb]{0.8529411764705882, 0.5078431372549019, 0.5745098039215686} 0.417& \cellcolor[rgb]{0.8431372549019608, 0.8960784313725491, 0.9941176470588236} 0.643& \cellcolor[rgb]{0.615686274509804, 0.6490196078431373, 0.8764705882352941} 0.526& \cellcolor[rgb]{0.615686274509804, 0.6490196078431373, 0.8764705882352941} 1.000\\
% $\text{FullSampleCoCoA}_{PPL}$ & \cellcolor[rgb]{0.6333333333333333, 0.6764705882352942, 0.9} 0.297& \cellcolor[rgb]{0.615686274509804, 0.6490196078431373, 0.8764705882352941} 0.560& \cellcolor[rgb]{0.615686274509804, 0.6490196078431373, 0.8764705882352941} 0.670& \cellcolor[rgb]{0.615686274509804, 0.6490196078431373, 0.8764705882352941} 0.405& \cellcolor[rgb]{0.615686274509804, 0.6490196078431373, 0.8764705882352941} 0.641& \cellcolor[rgb]{0.8529411764705882, 0.5078431372549019, 0.5745098039215686} 0.533& \cellcolor[rgb]{0.615686274509804, 0.6490196078431373, 0.8764705882352941} 1.000\\
% $\text{ProbCoCoA}_{PPL}$ & \cellcolor[rgb]{0.8529411764705882, 0.5078431372549019, 0.5745098039215686} 0.311& \cellcolor[rgb]{0.8529411764705882, 0.5078431372549019, 0.5745098039215686} 0.587& \cellcolor[rgb]{0.8529411764705882, 0.5078431372549019, 0.5745098039215686} 0.718& \cellcolor[rgb]{0.9764705882352941, 0.7901960784313725, 0.7294117647058823} 0.414& \cellcolor[rgb]{0.8529411764705882, 0.5078431372549019, 0.5745098039215686} 0.648& \cellcolor[rgb]{0.984313725490196, 0.8254901960784313, 0.7666666666666667} 0.531& \cellcolor[rgb]{0.615686274509804, 0.6490196078431373, 0.8764705882352941} 1.000\\
% $\text{CoCoA}_{PPL}$ & \cellcolor[rgb]{0.8980392156862745, 0.6215686274509804, 0.6098039215686275} 0.310& \cellcolor[rgb]{0.984313725490196, 0.8254901960784313, 0.7666666666666667} 0.579& \cellcolor[rgb]{0.8764705882352941, 0.5784313725490197, 0.592156862745098} 0.717& \cellcolor[rgb]{0.9686274509803922, 0.7666666666666667, 0.7098039215686275} 0.415& \cellcolor[rgb]{0.9470588235294117, 0.9254901960784314, 0.9117647058823529} 0.644& \cellcolor[rgb]{0.9490196078431372, 0.9235294117647059, 0.9098039215686274} 0.530& \cellcolor[rgb]{0.615686274509804, 0.6490196078431373, 0.8764705882352941} 1.000\\

% \midrule

% $\text{AdditiveCoCoA}_{MTE}$ & \cellcolor[rgb]{0.8176470588235294, 0.8784313725490196, 1.0} 0.253& \cellcolor[rgb]{0.8529411764705882, 0.5078431372549019, 0.5745098039215686} 0.554& \cellcolor[rgb]{0.9823529411764705, 0.8705882352941177, 0.8176470588235294} 0.634& \cellcolor[rgb]{0.6352941176470588, 0.6803921568627451, 0.903921568627451} 0.383& \cellcolor[rgb]{0.615686274509804, 0.6490196078431373, 0.8764705882352941} 0.630& \cellcolor[rgb]{0.8529411764705882, 0.5078431372549019, 0.5745098039215686} 0.523& \cellcolor[rgb]{0.9431372549019608, 0.7058823529411765, 0.6607843137254902} 0.997\\
% $\text{FullSampleCoCoA}_{MTE}$ & \cellcolor[rgb]{0.615686274509804, 0.6490196078431373, 0.8764705882352941} 0.237& \cellcolor[rgb]{0.615686274509804, 0.6490196078431373, 0.8764705882352941} 0.502& \cellcolor[rgb]{0.615686274509804, 0.6490196078431373, 0.8764705882352941} 0.554& \cellcolor[rgb]{0.615686274509804, 0.6490196078431373, 0.8764705882352941} 0.383& \cellcolor[rgb]{0.8666666666666667, 0.9098039215686274, 0.9862745098039216} 0.636& \cellcolor[rgb]{0.8588235294117647, 0.9058823529411765, 0.9882352941176471} 0.519& \cellcolor[rgb]{0.615686274509804, 0.6490196078431373, 0.8764705882352941} 0.989\\
% $\text{CoCoA}_{MTE}$ & \cellcolor[rgb]{0.8529411764705882, 0.5078431372549019, 0.5745098039215686} 0.289& \cellcolor[rgb]{0.8980392156862745, 0.6215686274509804, 0.6098039215686275} 0.551& \cellcolor[rgb]{0.8529411764705882, 0.5078431372549019, 0.5745098039215686} 0.678& \cellcolor[rgb]{0.8529411764705882, 0.5078431372549019, 0.5745098039215686} 0.402& \cellcolor[rgb]{0.8529411764705882, 0.5078431372549019, 0.5745098039215686} 0.646& \cellcolor[rgb]{0.615686274509804, 0.6490196078431373, 0.8764705882352941} 0.517& \cellcolor[rgb]{0.8529411764705882, 0.5078431372549019, 0.5745098039215686} 0.998\\
% \midrule

% \end{tabular}
% \end{table*}



\begin{table*}[h!]
\footnotesize
\centering

\begin{tabular}{lrrrrrrr}
\toprule
    \multirow{2}{*}{\textbf{Method}}  & \multicolumn{7}{c}{\textbf{Dataset}}  \\ 
      \cmidrule(lr){2-8}  \\
  & XSum & WMT14FrEn & WMT19DeEn & CoQa & Trivia & MMLU & GSM8k \\
  \midrule

\rowcolor[gray]{0.9} & \multicolumn{7}{c}{Mistral7b-Base} \\

\midrule

$\text{AdditiveCoCoA}_{MSP}$ & 0.333& 0.239& 0.310& 0.406& 0.631& \textbf{0.472}& 0.311\\
$\text{FullSampleCoCoA}_{MSP}$ & 0.354& 0.543& 0.565& 0.412& 0.643& \underline{0.468}& 0.428\\
$\text{ProbCoCoA}_{MSP}$ & 0.076& 0.684& 0.721& \underline{0.428}& 0.643& 0.464& 0.846\\
$\text{CoCoA}_{MSP}$ & 0.366& 0.712& 0.730& \textbf{0.430}& 0.644& 0.466& 0.562\\

\midrule

$\text{AdditiveCoCoA}_{PPL}$ & 0.368& \underline{0.737}& 0.751& 0.406& 0.644& 0.465& \underline{0.939}\\
$\text{FullSampleCoCoA}_{PPL}$ & \textbf{0.383}& 0.714& 0.723& 0.379& \textbf{0.649}& \underline{0.468}& 0.933\\
$\text{ProbCoCoA}_{PPL}$ & 0.369& \textbf{0.738}& \textbf{0.756}& 0.401& \textbf{0.649}& 0.467& 0.935\\
$\text{CoCoA}_{PPL}$ & 0.372& 0.735& \underline{0.755}& 0.402& 0.648& 0.466& 0.937\\

\midrule

$\text{AdditiveCoCoA}_{MTE}$ & 0.368& 0.723& 0.702& 0.332& 0.643& 0.452& \textbf{0.942}\\
$\text{FullSampleCoCoA}_{MTE}$ & \underline{0.380}& 0.661& 0.653& 0.331& 0.643& 0.442& 0.929\\
$\text{CoCoA}_{MTE}$ & 0.373& 0.708& 0.732& 0.373& 0.645& 0.447& 0.935\\
\midrule

\rowcolor[gray]{0.9} & \multicolumn{7}{c}{Llama8b-Base} \\

\midrule

$\text{AdditiveCoCoA}_{MSP}$ & 0.466& 0.349& 0.425& 0.333& 0.555& \textbf{0.473}& 0.285\\
$\text{FullSampleCoCoA}_{MSP}$ & \underline{0.476}& 0.462& 0.619& 0.363& 0.574& \underline{0.464}& 0.379\\
$\text{ProbCoCoA}_{MSP}$ & 0.035& 0.491& 0.617& \textbf{0.398}& \textbf{0.598}& 0.433& 0.795\\
$\text{CoCoA}_{MSP}$ & \textbf{0.484}& 0.529& 0.685& \underline{0.384}& \underline{0.587}& 0.452& 0.513\\

\midrule

$\text{AdditiveCoCoA}_{PPL}$ & 0.454& 0.536& 0.673& 0.358& 0.575& 0.425& \underline{0.923}\\
$\text{FullSampleCoCoA}_{PPL}$ & 0.459& 0.525& 0.649& 0.343& 0.556& 0.430& 0.914\\
$\text{ProbCoCoA}_{PPL}$ & 0.438& \textbf{0.547}& \textbf{0.689}& 0.364& 0.574& 0.419& \underline{0.923}\\
$\text{CoCoA}_{PPL}$ & 0.450& \underline{0.544}& \textbf{0.689}& 0.364& 0.573& 0.422& \textbf{0.925}\\

\midrule

$\text{AdditiveCoCoA}_{MTE}$ & 0.457& 0.496& 0.579& 0.304& 0.561& 0.361& 0.901\\
$\text{FullSampleCoCoA}_{MTE}$ & 0.455& 0.464& 0.577& 0.313& 0.563& 0.341& 0.878\\
$\text{CoCoA}_{MTE}$ & 0.451& 0.520& 0.638& 0.346& 0.582& 0.363& 0.900\\
\midrule

\rowcolor[gray]{0.9} & \multicolumn{7}{c}{Falcon7b-Base} \\

\midrule

$\text{AdditiveCoCoA}_{MSP}$ & 0.100& 0.397& 0.394& 0.393& 0.649& \textbf{0.534}& -0.156\\
$\text{FullSampleCoCoA}_{MSP}$ & 0.144& 0.531& 0.607& 0.416& 0.654& \underline{0.533}& 0.189\\
$\text{ProbCoCoA}_{MSP}$ & 0.282& 0.522& 0.670& \textbf{0.434}& \underline{0.658}& 0.529& 0.978\\
$\text{CoCoA}_{MSP}$ & 0.210& 0.564& 0.698& \underline{0.428}& \textbf{0.659}& 0.530& 0.498\\

\midrule

$\text{AdditiveCoCoA}_{PPL}$ & 0.297& \underline{0.582}& 0.706& 0.417& 0.643& 0.526& \textbf{1.000}\\
$\text{FullSampleCoCoA}_{PPL}$ & 0.297& 0.560& 0.670& 0.405& 0.641& \underline{0.533}& \textbf{1.000}\\
$\text{ProbCoCoA}_{PPL}$ & \textbf{0.311}& \textbf{0.587}& \textbf{0.718}& 0.414& 0.648& 0.531& \textbf{1.000}\\
$\text{CoCoA}_{PPL}$ & \underline{0.310}& 0.579& \underline{0.717}& 0.415& 0.644& 0.530& \textbf{1.000}\\

\midrule

$\text{AdditiveCoCoA}_{MTE}$ & 0.253& 0.554& 0.634& 0.383& 0.630& 0.523& 0.997\\
$\text{FullSampleCoCoA}_{MTE}$ & 0.237& 0.502& 0.554& 0.383& 0.636& 0.519& 0.989\\
$\text{CoCoA}_{MTE}$ & 0.289& 0.551& 0.678& 0.402& 0.646& 0.517& 0.998\\
\bottomrule

\end{tabular}
\caption{Comparison of PRRs of \texttt{CoCoA}-family methods with alternative formulations of with best sample taken in isolation.}
\label{tab:ablation_cocoa_best}
\end{table*}



\newpage
\section{Detailed Description of Uncertainty Quantification Methods}
\label{sec:appendix_methods}
  In this section, we provide a detailed description of the uncertainty quantification methods used in this study. 

\subsection{Information-Based Methods}
\label{suppl:confidence}
  Information-based methods are commonly used to estimate uncertainty by analyzing the probability distributions of tokens within a given output. These methods examine different levels of model generation, such as the model's confidence in producing a specific sequence, its ability to predict individual tokens at each generation step, and the variability in the token-level predictions across the sequence.

  \textit{Maximum Sequence Probability (MSP)} is one of the simplest and most direct methods for estimating uncertainty. It measures the probability of the most likely output sequence given a specific input. Thus, uncertainty is quantified by calculating the probability of the sequence with the highest likelihood, under the assumption that the model is most confident in this output. It is defined as:
  \begin{equation}
    U_{MSP}(\yv \mid \xv, \thetav) = - \log P(\yv \mid \xv).
  \label{eq:msp}
  \end{equation}

  \textit{Perplexity (PPL)} is another widely used metric for estimating uncertainty in language models~\citep{fomicheva-etal-2020-unsupervised}. It measures the model's confidence by evaluating the average likelihood of generating the sequence tokens: 
  \begin{equation}
    U_\mathrm{PPL}(\yv, \xv) = -\frac{1}{L} \log P(\yv \mid \xv).
  \label{eq:ppl}
  \end{equation}
  
 \textit{Mean Token Entropy} takes a broader view of uncertainty by considering the token-level predictions across the entire sequence~\citep{fomicheva-etal-2020-unsupervised}. Instead of evaluating the model's confidence in a single output or individual token predictions, Mean Token Entropy calculates the average entropy of the token probability distributions for each token in the sequence: 
  \begin{equation}
    U_{\HC_T}(\yv, \xv) = \frac{1}{L} \sum_{l = 1}^L \HC(y_l \mid \yv_{<l}, \xv),
  \label{eq:entropy}
  \end{equation}
  where $\HC(y_l \mid \yv_{<l}, \xv)$ is an entropy of the token distribution $P(y_l \mid \yv_{<l}, \xv)$.

  The \textit{TokenSAR} method, introduced in~\cite{duan-etal-2024-shifting}, generalizes length-normalized log probability by computing a weighted average of the negative log probabilities of generated tokens, where weights are based on token relevance to the overall text. Using a similarity function $g(\cdot, \cdot)$ and token relevance function $R_T(y_k, \yv, \xv) = 1 - g(\xv \cup \yv, \xv \cup \yv \setminus y_k)$, the uncertainty estimate is calculated as:
  \begin{equation}
    U_\mathrm{TokenSAR}(\yv, \xv) = -\sum_{l = 1}^L \tilde{\mathrm{R}}_T(y_l, \yv, \xv) \log P(y_l \mid \yv_{<l}, \xv),
  \end{equation}
  where 
  \begin{equation}
  \tilde{\mathrm{R}}_T(y_k, \yv, \xv) = \frac{\mathrm{R}_T(y_k, \yv, \xv)}{\sum\nolimits_{l = 1}^L \mathrm{R}_T(y_l, \yv, \xv)}.
  \end{equation}
  %
  This measure is central for computing \textit{SAR} uncertainty measure.  


\subsection{Consistency-Based Methods}
\label{suppl:consistency}
  Consistency-based methods assess the uncertainty of a language model by evaluating the semantic consistency of its predictions across multiple outputs for the same prompt. The core idea is that semantically similar outputs indicate higher confidence, while diverse or conflicting outputs suggest greater uncertainty. Since language models can express the same meaning in different surface forms, these methods construct a semantic similarity matrix  $G = (g_{ij})$, where each entry represents the degree of similarity between pairs of responses. By clustering responses into groups with equivalent meanings, these methods provide a semantic measure of the model's consistency.

  \citet{lin2023generating} offers two similarity measures to evaluate the similarity of sequences. The first is the Jaccard similarity, which treats sequences as sets of words and calculates the proportion of shared words to the total number of unique words in both sequences: $ g(\yv, \yv') = |\yv \cap \yv'| / |\yv \cup \yv'|$.
  
  Natural Language Inference (NLI) provides another method for computing similarity between sequences. We use the DeBERTa-large NLI model~\cite{he2021deberta}, following~\citet{kuhn2023semantic}. For each pair of sequences, an NLI model predicts two probabilities: \( {p}_{\mathrm{entail}}(\yv, \yv') \), indicating entailment, and \( {p}_{\mathrm{contra}}(\yv, \yv') \), indicating contradiction. Similarity is then defined as either $g_{\mathrm{entail}}(\yv, \yv') = {p}_{\mathrm{entail}}(\yv, \yv')$ or $ g_{\mathrm{contra}}(\yv, \yv') = 1 - {p}_{\mathrm{contra}}(\yv, \yv')$.  


  Among the simplest consistency-based approaches are the \textit{Number of Semantic Sets} and the \textit{Sum of Eigenvalues of the Graph Laplacian}~\citep{lin2023generating}. \textit{Number of Semantic Sets} estimates how many distinct ``meanings'' the model produces by clustering its outputs with an NLI model. The number of semantic sets is initially equal to the total number of generated answers, \( M \). Two sentences are grouped into the same cluster if the following conditions are satisfied: $ {p}_{\text{entail}}(\yv^{i}, \yv^{j}) > {p}_{\text{contra}}(\yv^{i}, \yv^{j}) \quad \text{and} \quad {p}_{\text{entail}}(\yv^{j}, \yv^{i}) > {p}_{\text{contra}}(\yv^{j}, \yv^{i})$. This computation is performed for all pairs of answers, and the final number of distinct clusters is denoted by $ U_{\text{NumSemSets}} $.

  \textit{Sum of Eigenvalues of the Graph Laplacian} examines global diversity: it constructs a similarity matrix among the sampled outputs and computes a continuous uncertainty score from the eigenvalues of the Laplacian of that similarity graph. \citet{lin2023generating} proposes computing an averaged similarity matrix as $g_{ij} = \bigl(g\bigl(\yv^{(i)}, \yv^{(j)}\bigr) + g\bigl(\yv^{(j)}, \yv^{(i)}\bigr)\bigr) / 2$. The Laplacian for the matrix $G$ is defined as  $L = I - D^{-\frac{1}{2}} G D^{-\frac{1}{2}}$, where $D$ is a diagonal matrix with elements $D_{ii} = \sum_{j = 1}^M g_{ij}$. Consequently, the following formula is derived: 
  \begin{equation}
    U_{\text{EigV}} = \sum_{i = 1}^M \max(0, 1 - \lambda_i).
  \end{equation}

  Both \textit{Number of Semantic Sets} and \textit{Sum of Eigenvalues of the Graph Laplacian} effectively capture overall variation in generated text but cannot produce an individual uncertainty score for each output. To address this, \citet{lin2023generating} proposes to use the diagonal \textit{Degree Matrix} $D$ which represents the total similarity of each answer with all others. The corrected trace of $D$ provides an average pairwise distance between answers, and uncertainty is computed as:
  \begin{equation}
    U_{\mathrm{DegMat}} = 1 - \mathrm{trace}(D) / M^2. 
    \label{eq:degmat}
  \end{equation}
  
\subsection{Information-Based Methods with Repeated Sampling}
  In this section we detail methods that integrate model confidence with consistency.

  We can compute the entropy on the sequence level $\EE \bigl[-\log P(\yv \mid \xv)\bigr]$, where the expectation is taken over the sequences $\yv$ randomly generated from the distribution $P(\yv \mid \xv)$. Unfortunately, while for token level, we have an exact way of computing the entropy, for the sequence level, we need to adhere to some approximations. In practice, we can use Monte-Carlo integration, i.e. generate several sequences $\yv^{(i)}, \, i = 1, \dots, M$ via random sampling and compute \textit{Monte Carlo Sequence Entropy}:
  \begin{equation}
    U_{\HC_S}(\xv) = -\frac{1}{M} \sum_{i = 1}^M \log P(\yv^{(i)} \mid \xv).
  \label{eq:seq_entropy}
  \end{equation}
  %
  We can replace $P(\yv^{(i)} \mid \xv)$ with its length-normalized version $\bar{P}(\yv^{(i)} \mid \xv)$ leading to a more reliable uncertainty measure in some cases.
  
  \textit{Semantic Entropy}~\cite{kuhn2023semantic} addresses the issue of generated sequences with similar meanings but differing probabilities according to the model, which can heavily influence the resulting entropy value~\eqref{eq:seq_entropy}. The method clusters generated sequences $\yv^{(i)}, \, i = 1, \dots, M$ into semantically homogeneous groups $\CC_k, ~ k = 1, \dots, K$ (where $K \le M$) using a bi-directional entailment algorithm. Probabilities of sequences are averaged within each cluster. The entropy estimate is then defined as:
  \begin{equation}
    U_\mathrm{SE}(\xv) = -\sum_{k = 1}^K \frac{|\CC_k|}{M} \log \hat{P}_k(\xv),
  \end{equation}
  where $\hat{P}_k(\xv) = \sum_{\yv \in \CC_k} P(\yv \mid \xv)$ represents the aggregated probability for cluster $\CC_k$.

  \textit{SentenceSAR}~\cite{duan-etal-2024-shifting} enhances the probability of sentences that are more relevant.  It uses a sentence relevance measure $g\bigl(\yv^{(j)}, \yv^{(k)}\bigr)$ to evaluate the relevance of $\yv^{(j)}$ with respect to $\yv^{(k)}$. SentenceSAR is calculated as:
  
  \begin{equation}
    U_\mathrm{SentSAR}(\xv) = -\frac{1}{M} \sum_{i = 1}^M \log \Bigl(P(\yv^{(i)} \mid \xv) + \frac{1}{t} \mathrm{R}_S (\yv^{(i)}, \xv)\Bigr),
  \end{equation}
  where $t$ is a temperature parameter used to control the scale of shifting to relevance, and
  \begin{equation}
    \mathrm{R}_S (\yv^{(j)}, \xv) \! = \sum_{k \neq j} g\bigl(\yv^{(j)}, \yv^{(k)}\bigr) P\bigl(\yv^{(k)} \mid \xv \bigr).
  \end{equation}

  The combination of SentenceSAR and TokenSAR results in a unified method called \textit{SAR}~\cite{duan-etal-2024-shifting}. In this approach, the generative probability $P(\yv \mid \xv)$ in the SentenceSAR formula is replaced with the token-shifted probability $P'(\yv \mid \xv) = \exp\bigl\{-\mathrm{TokenSAR}(\yv, \xv)\bigr\}$, creating a comprehensive measure that integrates both sentence- and token-level adjustments.

\newpage

\section{Detailed Experimental Results}
\label{sec:experimental_results}
  In this section, we present detailed experimental results, which were used for computing values in Tables~\ref{tab:best_sample_results} and~\ref{tab:greedy_results}.

  
\begin{table*}[ht!]
\footnotesize
\centering

\begin{tabular}{lrrrrrrr}
\toprule
    \multirow{2}{*}{\textbf{Method}}  & \multicolumn{7}{c}{\textbf{Dataset}}  \\ 
      \cmidrule(lr){2-8}  \\
  & XSum & WMT14FrEn & WMT19DeEn & CoQa & Trivia & MMLU & GSM8k \\
  \midrule

\rowcolor[gray]{0.9} & \multicolumn{7}{c}{Mistral7b-Base} \\

\midrule

MCSE & 0.007& 0.257& 0.350& 0.247& 0.496& 0.337& 0.475\\
MCNSE & 0.009& 0.342& 0.478& 0.238& 0.540& 0.356& 0.401\\
Semantic Entropy & 0.008& 0.271& 0.382& 0.271& 0.562& 0.387& 0.472\\
SAR & 0.094& 0.353& 0.517& 0.313& 0.644& 0.419& 0.471\\
DegMat & 0.137& 0.229& 0.382& 0.336& 0.646& 0.410& 0.299\\
EigValLaplacian & 0.132& 0.207& 0.328& 0.301& 0.624& 0.398& 0.241\\

\midrule

MSP & 0.287& 0.315& 0.451& 0.326& 0.628& \textbf{0.474}& 0.471\\
$\text{CoCoA}_{MSP}$ & \textbf{0.330}& \textbf{0.396}& \textbf{0.598}& \textbf{0.383}& 0.670& 0.466& \textbf{0.517}\\

\midrule

PPL & 0.204& 0.365& 0.489& 0.281& 0.632& \textbf{0.474}& 0.311\\
$\text{CoCoA}_{PPL}$ & 0.286& 0.375& \underline{0.568}& \underline{0.369}& \textbf{0.674}& 0.466& 0.467\\

\midrule

MTE & 0.182& \underline{0.392}& 0.484& 0.243& 0.619& 0.456& 0.350\\
$\text{CoCoA}_{MTE}$ & \underline{0.288}& 0.374& 0.564& 0.355& \underline{0.673}& 0.447& \underline{0.491}\\
\midrule

\rowcolor[gray]{0.9} & \multicolumn{7}{c}{Llama8b-Base} \\

\midrule

MCSE & 0.033& 0.293& 0.354& 0.237& 0.482& 0.171& 0.351\\
MCNSE & 0.022& 0.370& 0.415& 0.219& 0.501& 0.170& 0.344\\
Semantic Entropy & 0.033& 0.297& 0.389& 0.272& 0.549& 0.229& 0.375\\
SAR & 0.077& 0.427& 0.483& 0.311& 0.595& 0.352& 0.398\\
DegMat & 0.081& 0.250& 0.355& \textbf{0.353}& \textbf{0.622}& 0.342& 0.309\\
EigValLaplacian & 0.079& 0.198& 0.278& 0.332& \underline{0.604}& 0.292& 0.273\\

\midrule

MSP & 0.328& 0.342& 0.456& 0.277& 0.526& \textbf{0.508}& 0.324\\
$\text{CoCoA}_{MSP}$ & 0.378& \textbf{0.456}& \textbf{0.582}& \underline{0.349}& 0.597& \underline{0.485}& 0.372\\

\midrule

PPL & 0.369& 0.351& 0.422& 0.253& 0.507& 0.461& 0.303\\
$\text{CoCoA}_{PPL}$ & \textbf{0.387}& \underline{0.448}& \underline{0.514}& 0.338& 0.593& 0.452& \underline{0.433}\\

\midrule

MTE & 0.357& 0.357& 0.408& 0.239& 0.497& 0.350& 0.326\\
$\text{CoCoA}_{MTE}$ & \underline{0.380}& 0.446& 0.511& 0.337& 0.601& 0.402& \textbf{0.447}\\
\midrule

\rowcolor[gray]{0.9} & \multicolumn{7}{c}{Falcon7b-Base} \\

\midrule

MCSE & 0.159& 0.297& 0.337& 0.258& 0.549& 0.420& 0.427\\
MCNSE & 0.108& 0.371& 0.474& 0.293& 0.586& 0.442& 0.299\\
Semantic Entropy & 0.164& 0.307& 0.389& 0.294& 0.581& 0.463& 0.418\\
SAR & 0.144& 0.398& 0.517& 0.381& 0.649& 0.508& 0.387\\
DegMat & 0.201& 0.274& 0.431& \textbf{0.407}& 0.651& 0.480& 0.395\\
EigValLaplacian & 0.201& 0.229& 0.394& 0.381& 0.645& 0.454& 0.358\\

\midrule

MSP & 0.201& 0.312& 0.400& 0.321& 0.662& \textbf{0.539}& 0.377\\
$\text{CoCoA}_{MSP}$ & \textbf{0.257}& 0.433& \underline{0.578}& 0.396& \underline{0.684}& 0.529& 0.436\\

\midrule

PPL & 0.155& 0.375& 0.525& 0.316& 0.644& \textbf{0.539}& 0.326\\
$\text{CoCoA}_{PPL}$ & \underline{0.229}& \underline{0.436}& \textbf{0.580}& \underline{0.406}& 0.677& 0.529& \underline{0.478}\\

\midrule

MTE & 0.152& 0.409& 0.537& 0.291& 0.633& 0.533& 0.367\\
$\text{CoCoA}_{MTE}$ & 0.228& \textbf{0.439}& 0.577& 0.395& \textbf{0.685}& 0.517& \textbf{0.510}\\
\bottomrule

\end{tabular}
\caption{Detailed experimental results with greedy sample taken in isolation.}
\label{tab:experimental_results_greedy}
\end{table*}

  % \begin{table*}
% \footnotesize
% \caption{Best Sample PRRs for falcon7b on all tasks}
% \begin{tabular}{lrrrrrrr}
%  & XSum & WMT14FrEn & WMT19DeEn & CoQa & Trivia & MMLU & GSM8k \\

% \midrule

% & \multicolumn{7}{c}{Mistral7b-Base} \\

% \midrule

% MCSE & \cellcolor[rgb]{0.8470588235294118, 0.8980392156862745, 0.9941176470588236} 0.177& \cellcolor[rgb]{0.6784313725490196, 0.7392156862745098, 0.9490196078431372} 0.406& \cellcolor[rgb]{0.7803921568627451, 0.8470588235294118, 0.9980392156862745} 0.407& \cellcolor[rgb]{0.6843137254901961, 0.7450980392156863, 0.9529411764705882} 0.289& \cellcolor[rgb]{0.615686274509804, 0.6490196078431373, 0.8764705882352941} 0.492& \cellcolor[rgb]{0.615686274509804, 0.6490196078431373, 0.8764705882352941} 0.339& \cellcolor[rgb]{0.9666666666666667, 0.7627450980392156, 0.7058823529411765} 0.693\\
% MCNSE & \cellcolor[rgb]{0.9352941176470588, 0.9313725490196079, 0.9294117647058824} 0.213& \cellcolor[rgb]{0.9196078431372549, 0.9313725490196079, 0.9470588235294117} 0.471& \cellcolor[rgb]{0.984313725490196, 0.8627450980392157, 0.8098039215686275} 0.507& \cellcolor[rgb]{0.615686274509804, 0.6490196078431373, 0.8764705882352941} 0.277& \cellcolor[rgb]{0.807843137254902, 0.8705882352941177, 1.0} 0.529& \cellcolor[rgb]{0.7627450980392156, 0.8313725490196079, 0.9941176470588236} 0.358& \cellcolor[rgb]{0.9627450980392157, 0.7490196078431373, 0.6941176470588235} 0.700\\
% SemanticEntropy & \cellcolor[rgb]{0.8490196078431372, 0.9, 0.9921568627450981} 0.179& \cellcolor[rgb]{0.7823529411764706, 0.8490196078431372, 0.9980392156862745} 0.434& \cellcolor[rgb]{0.8764705882352941, 0.9156862745098039, 0.9803921568627451} 0.442& \cellcolor[rgb]{0.8431372549019608, 0.8960784313725491, 0.9941176470588236} 0.312& \cellcolor[rgb]{0.9254901960784314, 0.9313725490196079, 0.9411764705882353} 0.554& \cellcolor[rgb]{0.9745098039215686, 0.8941176470588236, 0.8529411764705882} 0.389& \cellcolor[rgb]{0.9784313725490196, 0.7980392156862746, 0.7392156862745098} 0.675\\
% SAR & \cellcolor[rgb]{0.8529411764705882, 0.5078431372549019, 0.5745098039215686} 0.330& \cellcolor[rgb]{0.8529411764705882, 0.5078431372549019, 0.5745098039215686} 0.563& \cellcolor[rgb]{0.8529411764705882, 0.5078431372549019, 0.5745098039215686} 0.590& \cellcolor[rgb]{0.984313725490196, 0.8313725490196079, 0.7725490196078432} 0.347& \cellcolor[rgb]{0.8529411764705882, 0.5078431372549019, 0.5745098039215686} 0.620& \cellcolor[rgb]{0.8529411764705882, 0.5078431372549019, 0.5745098039215686} 0.421& \cellcolor[rgb]{0.8529411764705882, 0.5078431372549019, 0.5745098039215686} 0.780\\
% DegMat & \cellcolor[rgb]{0.6588235294117647, 0.7137254901960784, 0.9294117647058824} 0.112& \cellcolor[rgb]{0.807843137254902, 0.8705882352941177, 1.0} 0.439& \cellcolor[rgb]{0.788235294117647, 0.8549019607843137, 0.9980392156862745} 0.410& \cellcolor[rgb]{0.8529411764705882, 0.5078431372549019, 0.5745098039215686} 0.376& \cellcolor[rgb]{0.8607843137254902, 0.5352941176470588, 0.5823529411764706} 0.618& \cellcolor[rgb]{0.9372549019607843, 0.6941176470588235, 0.6529411764705882} 0.410& \cellcolor[rgb]{0.7588235294117647, 0.8274509803921568, 0.9941176470588236} 0.454\\
% EigValLaplacian & \cellcolor[rgb]{0.615686274509804, 0.6490196078431373, 0.8764705882352941} 0.094& \cellcolor[rgb]{0.615686274509804, 0.6490196078431373, 0.8764705882352941} 0.388& \cellcolor[rgb]{0.615686274509804, 0.6490196078431373, 0.8764705882352941} 0.344& \cellcolor[rgb]{0.984313725490196, 0.8686274509803922, 0.8156862745098039} 0.342& \cellcolor[rgb]{0.9450980392156862, 0.711764705882353, 0.6666666666666666} 0.600& \cellcolor[rgb]{0.9803921568627451, 0.8117647058823529, 0.7509803921568627} 0.399& \cellcolor[rgb]{0.615686274509804, 0.6490196078431373, 0.8764705882352941} 0.361\\

% \midrule

% MaximumSequenceProbability & \cellcolor[rgb]{0.615686274509804, 0.6490196078431373, 0.8764705882352941} -0.141& \cellcolor[rgb]{0.615686274509804, 0.6490196078431373, 0.8764705882352941} 0.212& \cellcolor[rgb]{0.615686274509804, 0.6490196078431373, 0.8764705882352941} 0.291& \cellcolor[rgb]{0.615686274509804, 0.6490196078431373, 0.8764705882352941} 0.388& \cellcolor[rgb]{0.615686274509804, 0.6490196078431373, 0.8764705882352941} 0.607& \cellcolor[rgb]{0.8529411764705882, 0.5078431372549019, 0.5745098039215686} 0.476& \cellcolor[rgb]{0.615686274509804, 0.6490196078431373, 0.8764705882352941} 0.307\\
% $\text{CoCoA}_{MSP}$ & \cellcolor[rgb]{0.8529411764705882, 0.5078431372549019, 0.5745098039215686} 0.163& \cellcolor[rgb]{0.8529411764705882, 0.5078431372549019, 0.5745098039215686} 0.712& \cellcolor[rgb]{0.8529411764705882, 0.5078431372549019, 0.5745098039215686} 0.730& \cellcolor[rgb]{0.8529411764705882, 0.5078431372549019, 0.5745098039215686} 0.430& \cellcolor[rgb]{0.8529411764705882, 0.5078431372549019, 0.5745098039215686} 0.644& \cellcolor[rgb]{0.615686274509804, 0.6490196078431373, 0.8764705882352941} 0.466& \cellcolor[rgb]{0.8529411764705882, 0.5078431372549019, 0.5745098039215686} 0.562\\

% \midrule

% Perplexity & \cellcolor[rgb]{0.615686274509804, 0.6490196078431373, 0.8764705882352941} 0.180& \cellcolor[rgb]{0.615686274509804, 0.6490196078431373, 0.8764705882352941} 0.695& \cellcolor[rgb]{0.615686274509804, 0.6490196078431373, 0.8764705882352941} 0.676& \cellcolor[rgb]{0.615686274509804, 0.6490196078431373, 0.8764705882352941} 0.327& \cellcolor[rgb]{0.615686274509804, 0.6490196078431373, 0.8764705882352941} 0.615& \cellcolor[rgb]{0.8529411764705882, 0.5078431372549019, 0.5745098039215686} 0.476& \cellcolor[rgb]{0.615686274509804, 0.6490196078431373, 0.8764705882352941} 0.931\\
% $\text{CoCoA}_{PPL}$ & \cellcolor[rgb]{0.8529411764705882, 0.5078431372549019, 0.5745098039215686} 0.303& \cellcolor[rgb]{0.8529411764705882, 0.5078431372549019, 0.5745098039215686} 0.735& \cellcolor[rgb]{0.8529411764705882, 0.5078431372549019, 0.5745098039215686} 0.755& \cellcolor[rgb]{0.8529411764705882, 0.5078431372549019, 0.5745098039215686} 0.402& \cellcolor[rgb]{0.8529411764705882, 0.5078431372549019, 0.5745098039215686} 0.648& \cellcolor[rgb]{0.615686274509804, 0.6490196078431373, 0.8764705882352941} 0.466& \cellcolor[rgb]{0.8529411764705882, 0.5078431372549019, 0.5745098039215686} 0.937\\

% \midrule

% MeanTokenEntropy & \cellcolor[rgb]{0.615686274509804, 0.6490196078431373, 0.8764705882352941} 0.119& \cellcolor[rgb]{0.615686274509804, 0.6490196078431373, 0.8764705882352941} 0.668& \cellcolor[rgb]{0.615686274509804, 0.6490196078431373, 0.8764705882352941} 0.606& \cellcolor[rgb]{0.615686274509804, 0.6490196078431373, 0.8764705882352941} 0.254& \cellcolor[rgb]{0.615686274509804, 0.6490196078431373, 0.8764705882352941} 0.594& \cellcolor[rgb]{0.8529411764705882, 0.5078431372549019, 0.5745098039215686} 0.457& \cellcolor[rgb]{0.615686274509804, 0.6490196078431373, 0.8764705882352941} 0.932\\
% $\text{CoCoA}_{MTE}$ & \cellcolor[rgb]{0.8529411764705882, 0.5078431372549019, 0.5745098039215686} 0.285& \cellcolor[rgb]{0.8529411764705882, 0.5078431372549019, 0.5745098039215686} 0.708& \cellcolor[rgb]{0.8529411764705882, 0.5078431372549019, 0.5745098039215686} 0.732& \cellcolor[rgb]{0.8529411764705882, 0.5078431372549019, 0.5745098039215686} 0.373& \cellcolor[rgb]{0.8529411764705882, 0.5078431372549019, 0.5745098039215686} 0.645& \cellcolor[rgb]{0.615686274509804, 0.6490196078431373, 0.8764705882352941} 0.447& \cellcolor[rgb]{0.8529411764705882, 0.5078431372549019, 0.5745098039215686} 0.935\\
% \midrule

% & \multicolumn{7}{c}{Llama8b-Base} \\

% \midrule

% MCSE & \cellcolor[rgb]{0.615686274509804, 0.6490196078431373, 0.8764705882352941} 0.063& \cellcolor[rgb]{0.984313725490196, 0.8607843137254902, 0.8058823529411765} 0.366& \cellcolor[rgb]{0.7823529411764706, 0.8490196078431372, 0.9980392156862745} 0.395& \cellcolor[rgb]{0.7098039215686275, 0.7764705882352941, 0.9705882352941176} 0.259& \cellcolor[rgb]{0.615686274509804, 0.6490196078431373, 0.8764705882352941} 0.465& \cellcolor[rgb]{0.615686274509804, 0.6490196078431373, 0.8764705882352941} 0.158& \cellcolor[rgb]{0.9764705882352941, 0.8901960784313725, 0.8470588235294118} 0.546\\
% MCNSE & \cellcolor[rgb]{0.7196078431372549, 0.788235294117647, 0.9764705882352941} 0.100& \cellcolor[rgb]{0.984313725490196, 0.8254901960784313, 0.7666666666666667} 0.377& \cellcolor[rgb]{0.9784313725490196, 0.888235294117647, 0.8431372549019608} 0.480& \cellcolor[rgb]{0.615686274509804, 0.6490196078431373, 0.8764705882352941} 0.239& \cellcolor[rgb]{0.7, 0.7647058823529411, 0.9647058823529412} 0.484& \cellcolor[rgb]{0.6372549019607843, 0.6843137254901961, 0.9058823529411765} 0.165& \cellcolor[rgb]{0.9254901960784314, 0.6725490196078432, 0.6392156862745098} 0.634\\
% SemanticEntropy & \cellcolor[rgb]{0.615686274509804, 0.6490196078431373, 0.8764705882352941} 0.063& \cellcolor[rgb]{0.984313725490196, 0.8450980392156863, 0.788235294117647} 0.371& \cellcolor[rgb]{0.9313725490196079, 0.9333333333333333, 0.9333333333333333} 0.451& \cellcolor[rgb]{0.8568627450980392, 0.903921568627451, 0.9901960784313726} 0.286& \cellcolor[rgb]{0.903921568627451, 0.9274509803921569, 0.9607843137254902} 0.528& \cellcolor[rgb]{0.8117647058823529, 0.8745098039215686, 1.0} 0.213& \cellcolor[rgb]{0.984313725490196, 0.8686274509803922, 0.8156862745098039} 0.559\\
% SAR & \cellcolor[rgb]{0.8529411764705882, 0.5078431372549019, 0.5745098039215686} 0.280& \cellcolor[rgb]{0.8529411764705882, 0.5078431372549019, 0.5745098039215686} 0.441& \cellcolor[rgb]{0.8529411764705882, 0.5078431372549019, 0.5745098039215686} 0.571& \cellcolor[rgb]{0.984313725490196, 0.8450980392156863, 0.788235294117647} 0.327& \cellcolor[rgb]{0.9647058823529412, 0.7568627450980392, 0.7} 0.578& \cellcolor[rgb]{0.8529411764705882, 0.5078431372549019, 0.5745098039215686} 0.340& \cellcolor[rgb]{0.8529411764705882, 0.5078431372549019, 0.5745098039215686} 0.668\\
% DegMat & \cellcolor[rgb]{0.7941176470588235, 0.8588235294117647, 1.0} 0.122& \cellcolor[rgb]{0.7823529411764706, 0.8490196078431372, 0.9980392156862745} 0.274& \cellcolor[rgb]{0.8254901960784313, 0.884313725490196, 0.9980392156862745} 0.409& \cellcolor[rgb]{0.8529411764705882, 0.5078431372549019, 0.5745098039215686} 0.367& \cellcolor[rgb]{0.8529411764705882, 0.5078431372549019, 0.5745098039215686} 0.606& \cellcolor[rgb]{0.9274509803921569, 0.6764705882352942, 0.6431372549019608} 0.320& \cellcolor[rgb]{0.7019607843137254, 0.7666666666666667, 0.9666666666666667} 0.397\\
% EigValLaplacian & \cellcolor[rgb]{0.7509803921568627, 0.8196078431372549, 0.9901960784313726} 0.109& \cellcolor[rgb]{0.615686274509804, 0.6490196078431373, 0.8764705882352941} 0.216& \cellcolor[rgb]{0.615686274509804, 0.6490196078431373, 0.8764705882352941} 0.333& \cellcolor[rgb]{0.9686274509803922, 0.7666666666666667, 0.7098039215686275} 0.340& \cellcolor[rgb]{0.9392156862745098, 0.6980392156862745, 0.6549019607843137} 0.587& \cellcolor[rgb]{0.9803921568627451, 0.8784313725490196, 0.8313725490196079} 0.274& \cellcolor[rgb]{0.615686274509804, 0.6490196078431373, 0.8764705882352941} 0.351\\

% \midrule

% MaximumSequenceProbability & \cellcolor[rgb]{0.615686274509804, 0.6490196078431373, 0.8764705882352941} -0.197& \cellcolor[rgb]{0.615686274509804, 0.6490196078431373, 0.8764705882352941} 0.339& \cellcolor[rgb]{0.615686274509804, 0.6490196078431373, 0.8764705882352941} 0.413& \cellcolor[rgb]{0.615686274509804, 0.6490196078431373, 0.8764705882352941} 0.304& \cellcolor[rgb]{0.615686274509804, 0.6490196078431373, 0.8764705882352941} 0.514& \cellcolor[rgb]{0.8529411764705882, 0.5078431372549019, 0.5745098039215686} 0.483& \cellcolor[rgb]{0.615686274509804, 0.6490196078431373, 0.8764705882352941} 0.281\\
% $\text{CoCoA}_{MSP}$ & \cellcolor[rgb]{0.8529411764705882, 0.5078431372549019, 0.5745098039215686} 0.012& \cellcolor[rgb]{0.8529411764705882, 0.5078431372549019, 0.5745098039215686} 0.529& \cellcolor[rgb]{0.8529411764705882, 0.5078431372549019, 0.5745098039215686} 0.685& \cellcolor[rgb]{0.8529411764705882, 0.5078431372549019, 0.5745098039215686} 0.384& \cellcolor[rgb]{0.8529411764705882, 0.5078431372549019, 0.5745098039215686} 0.587& \cellcolor[rgb]{0.615686274509804, 0.6490196078431373, 0.8764705882352941} 0.452& \cellcolor[rgb]{0.8529411764705882, 0.5078431372549019, 0.5745098039215686} 0.513\\

% \midrule

% Perplexity & \cellcolor[rgb]{0.615686274509804, 0.6490196078431373, 0.8764705882352941} -0.076& \cellcolor[rgb]{0.615686274509804, 0.6490196078431373, 0.8764705882352941} 0.504& \cellcolor[rgb]{0.615686274509804, 0.6490196078431373, 0.8764705882352941} 0.622& \cellcolor[rgb]{0.615686274509804, 0.6490196078431373, 0.8764705882352941} 0.294& \cellcolor[rgb]{0.615686274509804, 0.6490196078431373, 0.8764705882352941} 0.483& \cellcolor[rgb]{0.8529411764705882, 0.5078431372549019, 0.5745098039215686} 0.441& \cellcolor[rgb]{0.615686274509804, 0.6490196078431373, 0.8764705882352941} 0.912\\
% $\text{CoCoA}_{PPL}$ & \cellcolor[rgb]{0.8529411764705882, 0.5078431372549019, 0.5745098039215686} 0.065& \cellcolor[rgb]{0.8529411764705882, 0.5078431372549019, 0.5745098039215686} 0.544& \cellcolor[rgb]{0.8529411764705882, 0.5078431372549019, 0.5745098039215686} 0.689& \cellcolor[rgb]{0.8529411764705882, 0.5078431372549019, 0.5745098039215686} 0.364& \cellcolor[rgb]{0.8529411764705882, 0.5078431372549019, 0.5745098039215686} 0.573& \cellcolor[rgb]{0.615686274509804, 0.6490196078431373, 0.8764705882352941} 0.422& \cellcolor[rgb]{0.8529411764705882, 0.5078431372549019, 0.5745098039215686} 0.925\\

% \midrule

% MeanTokenEntropy & \cellcolor[rgb]{0.615686274509804, 0.6490196078431373, 0.8764705882352941} -0.153& \cellcolor[rgb]{0.615686274509804, 0.6490196078431373, 0.8764705882352941} 0.437& \cellcolor[rgb]{0.615686274509804, 0.6490196078431373, 0.8764705882352941} 0.501& \cellcolor[rgb]{0.615686274509804, 0.6490196078431373, 0.8764705882352941} 0.239& \cellcolor[rgb]{0.615686274509804, 0.6490196078431373, 0.8764705882352941} 0.458& \cellcolor[rgb]{0.615686274509804, 0.6490196078431373, 0.8764705882352941} 0.326& \cellcolor[rgb]{0.615686274509804, 0.6490196078431373, 0.8764705882352941} 0.884\\
% $\text{CoCoA}_{MTE}$ & \cellcolor[rgb]{0.8529411764705882, 0.5078431372549019, 0.5745098039215686} 0.035& \cellcolor[rgb]{0.8529411764705882, 0.5078431372549019, 0.5745098039215686} 0.520& \cellcolor[rgb]{0.8529411764705882, 0.5078431372549019, 0.5745098039215686} 0.638& \cellcolor[rgb]{0.8529411764705882, 0.5078431372549019, 0.5745098039215686} 0.346& \cellcolor[rgb]{0.8529411764705882, 0.5078431372549019, 0.5745098039215686} 0.582& \cellcolor[rgb]{0.8529411764705882, 0.5078431372549019, 0.5745098039215686} 0.363& \cellcolor[rgb]{0.8529411764705882, 0.5078431372549019, 0.5745098039215686} 0.900\\
% \midrule

% & \multicolumn{7}{c}{Falcon7b-Base} \\

% \midrule

% MCSE & \cellcolor[rgb]{0.9019607843137255, 0.9254901960784314, 0.9627450980392157} 0.053& \cellcolor[rgb]{0.984313725490196, 0.8627450980392157, 0.8098039215686275} 0.399& \cellcolor[rgb]{0.7862745098039216, 0.8529411764705882, 0.9980392156862745} 0.419& \cellcolor[rgb]{0.615686274509804, 0.6490196078431373, 0.8764705882352941} 0.285& \cellcolor[rgb]{0.615686274509804, 0.6490196078431373, 0.8764705882352941} 0.535& \cellcolor[rgb]{0.615686274509804, 0.6490196078431373, 0.8764705882352941} 0.421& \cellcolor[rgb]{0.7764705882352941, 0.8450980392156863, 0.9980392156862745} 0.598\\
% MCNSE & \cellcolor[rgb]{0.9588235294117647, 0.7411764705882353, 0.6882352941176471} 0.149& \cellcolor[rgb]{0.9803921568627451, 0.8764705882352941, 0.8274509803921568} 0.395& \cellcolor[rgb]{0.9333333333333333, 0.9313725490196079, 0.9313725490196079} 0.452& \cellcolor[rgb]{0.7686274509803922, 0.8372549019607843, 0.996078431372549} 0.318& \cellcolor[rgb]{0.9274509803921569, 0.9313725490196079, 0.9392156862745098} 0.588& \cellcolor[rgb]{0.7745098039215687, 0.8431372549019608, 0.9980392156862745} 0.443& \cellcolor[rgb]{0.9588235294117647, 0.7411764705882353, 0.6882352941176471} 0.771\\
% SemanticEntropy & \cellcolor[rgb]{0.9, 0.9254901960784314, 0.9647058823529412} 0.052& \cellcolor[rgb]{0.9705882352941176, 0.7705882352941177, 0.711764705882353} 0.420& \cellcolor[rgb]{0.9607843137254902, 0.9137254901960784, 0.888235294117647} 0.460& \cellcolor[rgb]{0.7725490196078432, 0.8392156862745098, 0.996078431372549} 0.319& \cellcolor[rgb]{0.803921568627451, 0.8686274509803922, 1.0} 0.566& \cellcolor[rgb]{0.9235294117647059, 0.9313725490196079, 0.9431372549019608} 0.463& \cellcolor[rgb]{0.703921568627451, 0.7705882352941177, 0.9686274509803922} 0.567\\
% SAR & \cellcolor[rgb]{0.8529411764705882, 0.5078431372549019, 0.5745098039215686} 0.196& \cellcolor[rgb]{0.8529411764705882, 0.5078431372549019, 0.5745098039215686} 0.455& \cellcolor[rgb]{0.8529411764705882, 0.5078431372549019, 0.5745098039215686} 0.521& \cellcolor[rgb]{0.9823529411764705, 0.8176470588235294, 0.7568627450980392} 0.385& \cellcolor[rgb]{0.8529411764705882, 0.5078431372549019, 0.5745098039215686} 0.642& \cellcolor[rgb]{0.8529411764705882, 0.5078431372549019, 0.5745098039215686} 0.509& \cellcolor[rgb]{0.8529411764705882, 0.5078431372549019, 0.5745098039215686} 0.826\\
% DegMat & \cellcolor[rgb]{0.6254901960784314, 0.6666666666666666, 0.8921568627450981} -0.056& \cellcolor[rgb]{0.8568627450980392, 0.903921568627451, 0.9901960784313726} 0.350& \cellcolor[rgb]{0.7980392156862746, 0.8627450980392157, 1.0} 0.422& \cellcolor[rgb]{0.8529411764705882, 0.5078431372549019, 0.5745098039215686} 0.422& \cellcolor[rgb]{0.8862745098039215, 0.6, 0.6019607843137255} 0.637& \cellcolor[rgb]{0.984313725490196, 0.8588235294117647, 0.803921568627451} 0.480& \cellcolor[rgb]{0.8549019607843137, 0.9019607843137255, 0.9901960784313726} 0.633\\
% EigValLaplacian & \cellcolor[rgb]{0.615686274509804, 0.6490196078431373, 0.8764705882352941} -0.061& \cellcolor[rgb]{0.615686274509804, 0.6490196078431373, 0.8764705882352941} 0.289& \cellcolor[rgb]{0.615686274509804, 0.6490196078431373, 0.8764705882352941} 0.382& \cellcolor[rgb]{0.9705882352941176, 0.7705882352941177, 0.711764705882353} 0.393& \cellcolor[rgb]{0.9627450980392157, 0.7490196078431373, 0.6941176470588235} 0.622& \cellcolor[rgb]{0.8588235294117647, 0.9058823529411765, 0.9882352941176471} 0.454& \cellcolor[rgb]{0.615686274509804, 0.6490196078431373, 0.8764705882352941} 0.522\\

% \midrule

% MaximumSequenceProbability & \cellcolor[rgb]{0.615686274509804, 0.6490196078431373, 0.8764705882352941} -0.011& \cellcolor[rgb]{0.615686274509804, 0.6490196078431373, 0.8764705882352941} 0.385& \cellcolor[rgb]{0.615686274509804, 0.6490196078431373, 0.8764705882352941} 0.378& \cellcolor[rgb]{0.615686274509804, 0.6490196078431373, 0.8764705882352941} 0.369& \cellcolor[rgb]{0.615686274509804, 0.6490196078431373, 0.8764705882352941} 0.638& \cellcolor[rgb]{0.8529411764705882, 0.5078431372549019, 0.5745098039215686} 0.540& \cellcolor[rgb]{0.615686274509804, 0.6490196078431373, 0.8764705882352941} -0.175\\
% $\text{CoCoA}_{MSP}$ & \cellcolor[rgb]{0.8529411764705882, 0.5078431372549019, 0.5745098039215686} 0.119& \cellcolor[rgb]{0.8529411764705882, 0.5078431372549019, 0.5745098039215686} 0.564& \cellcolor[rgb]{0.8529411764705882, 0.5078431372549019, 0.5745098039215686} 0.698& \cellcolor[rgb]{0.8529411764705882, 0.5078431372549019, 0.5745098039215686} 0.428& \cellcolor[rgb]{0.8529411764705882, 0.5078431372549019, 0.5745098039215686} 0.659& \cellcolor[rgb]{0.615686274509804, 0.6490196078431373, 0.8764705882352941} 0.530& \cellcolor[rgb]{0.8529411764705882, 0.5078431372549019, 0.5745098039215686} 0.498\\

% \midrule

% Perplexity & \cellcolor[rgb]{0.615686274509804, 0.6490196078431373, 0.8764705882352941} 0.252& \cellcolor[rgb]{0.615686274509804, 0.6490196078431373, 0.8764705882352941} 0.541& \cellcolor[rgb]{0.615686274509804, 0.6490196078431373, 0.8764705882352941} 0.637& \cellcolor[rgb]{0.615686274509804, 0.6490196078431373, 0.8764705882352941} 0.353& \cellcolor[rgb]{0.615686274509804, 0.6490196078431373, 0.8764705882352941} 0.614& \cellcolor[rgb]{0.8529411764705882, 0.5078431372549019, 0.5745098039215686} 0.540& \cellcolor[rgb]{0.615686274509804, 0.6490196078431373, 0.8764705882352941} 1.000\\
% $\text{CoCoA}_{PPL}$ & \cellcolor[rgb]{0.8529411764705882, 0.5078431372549019, 0.5745098039215686} 0.328& \cellcolor[rgb]{0.8529411764705882, 0.5078431372549019, 0.5745098039215686} 0.579& \cellcolor[rgb]{0.8529411764705882, 0.5078431372549019, 0.5745098039215686} 0.717& \cellcolor[rgb]{0.8529411764705882, 0.5078431372549019, 0.5745098039215686} 0.415& \cellcolor[rgb]{0.8529411764705882, 0.5078431372549019, 0.5745098039215686} 0.644& \cellcolor[rgb]{0.615686274509804, 0.6490196078431373, 0.8764705882352941} 0.530& \cellcolor[rgb]{0.615686274509804, 0.6490196078431373, 0.8764705882352941} 1.000\\

% \midrule

% MeanTokenEntropy & \cellcolor[rgb]{0.615686274509804, 0.6490196078431373, 0.8764705882352941} 0.124& \cellcolor[rgb]{0.615686274509804, 0.6490196078431373, 0.8764705882352941} 0.475& \cellcolor[rgb]{0.615686274509804, 0.6490196078431373, 0.8764705882352941} 0.510& \cellcolor[rgb]{0.615686274509804, 0.6490196078431373, 0.8764705882352941} 0.317& \cellcolor[rgb]{0.615686274509804, 0.6490196078431373, 0.8764705882352941} 0.573& \cellcolor[rgb]{0.8529411764705882, 0.5078431372549019, 0.5745098039215686} 0.534& \cellcolor[rgb]{0.615686274509804, 0.6490196078431373, 0.8764705882352941} 0.984\\
% $\text{CoCoA}_{MTE}$ & \cellcolor[rgb]{0.8529411764705882, 0.5078431372549019, 0.5745098039215686} 0.284& \cellcolor[rgb]{0.8529411764705882, 0.5078431372549019, 0.5745098039215686} 0.551& \cellcolor[rgb]{0.8529411764705882, 0.5078431372549019, 0.5745098039215686} 0.678& \cellcolor[rgb]{0.8529411764705882, 0.5078431372549019, 0.5745098039215686} 0.402& \cellcolor[rgb]{0.8529411764705882, 0.5078431372549019, 0.5745098039215686} 0.646& \cellcolor[rgb]{0.615686274509804, 0.6490196078431373, 0.8764705882352941} 0.517& \cellcolor[rgb]{0.8529411764705882, 0.5078431372549019, 0.5745098039215686} 0.998\\
% \midrule

% \end{tabular}
% \end{table*}


% \begin{table*}[ht!]
% \footnotesize
% \centering

% \begin{tabular}{lrrrrrrr}
% \toprule
%     \multirow{2}{*}{\textbf{Method}}  & \multicolumn{7}{c}{\textbf{Dataset}}  \\ 
%       \cmidrule(lr){2-8}  \\
%   & XSum & WMT14FrEn & WMT19DeEn & CoQa & Trivia & MMLU & GSM8k \\

% \rowcolor[gray]{0.9} & \multicolumn{7}{c}{Mistral7b-Base} \\

% \midrule

% MCSE & 0.177& 0.406& 0.407& 0.289& 0.492& 0.339& 0.693\\
% MCNSE & 0.213& 0.471& 0.507& 0.277& 0.529& 0.358& 0.700\\
% Semantic Entropy & 0.179& 0.434& 0.442& 0.312& 0.554& 0.389& 0.675\\
% SAR & \textbf{0.330}& 0.563& 0.590& 0.347& 0.620& 0.421& 0.780\\
% Degree Matrix & 0.112& 0.439& 0.410& 0.376& 0.618& 0.410& 0.454\\
% Eigenvalue Laplacian & 0.094& 0.388& 0.344& 0.342& 0.600& 0.399& 0.361\\

% \midrule

% Maximum Sequence Probability & -0.141& 0.212& 0.291& 0.388& 0.607& \textbf{0.476}& 0.307\\
% $\text{CoCoA}_{MSP}$ & 0.163& \underline{0.712}& 0.730& \textbf{0.430}& 0.644& 0.466& 0.562\\

% \midrule

% Perplexity & 0.180& 0.695& 0.676& 0.327& 0.615& \textbf{0.476}& 0.931\\
% $\text{CoCoA}_{PPL}$ & \underline{0.303}& \textbf{0.735}& \textbf{0.755}& \underline{0.402}& \textbf{0.648}& 0.466& \textbf{0.937}\\

% \midrule

% Mean Token Entropy & 0.119& 0.668& 0.606& 0.254& 0.594& 0.457& 0.932\\
% $\text{CoCoA}_{MTE}$ & 0.285& 0.708& \underline{0.732}& 0.373& \underline{0.645}& 0.447& \underline{0.935}\\
% \midrule

% \rowcolor[gray]{0.9} & \multicolumn{7}{c}{Llama8b-Base} \\

% \midrule

% MCSE & 0.063& 0.366& 0.395& 0.259& 0.465& 0.158& 0.546\\
% MCNSE & 0.100& 0.377& 0.480& 0.239& 0.484& 0.165& 0.634\\
% Semantic Entropy & 0.063& 0.371& 0.451& 0.286& 0.528& 0.213& 0.559\\
% SAR & \textbf{0.280}& 0.441& 0.571& 0.327& 0.578& 0.340& 0.668\\
% Degree Matrix & \underline{0.122}& 0.274& 0.409& \underline{0.367}& \textbf{0.606}& 0.320& 0.397\\
% Eigenvalue Laplacian & 0.109& 0.216& 0.333& 0.340& \underline{0.587}& 0.274& 0.351\\

% \midrule

% Maximum Sequence Probability & -0.197& 0.339& 0.413& 0.304& 0.514& \textbf{0.483}& 0.281\\
% $\text{CoCoA}_{MSP}$ & 0.012& \underline{0.529}& \underline{0.685}& \textbf{0.384}& \underline{0.587}& \underline{0.452}& 0.513\\

% \midrule

% Perplexity & -0.076& 0.504& 0.622& 0.294& 0.483& 0.441& \underline{0.912}\\
% $\text{CoCoA}_{PPL}$ & 0.065& \textbf{0.544}& \textbf{0.689}& 0.364& 0.573& 0.422& \textbf{0.925}\\

% \midrule

% Mean Token Entropy & -0.153& 0.437& 0.501& 0.239& 0.458& 0.326& 0.884\\
% $\text{CoCoA}_{MTE}$ & 0.035& 0.520& 0.638& 0.346& 0.582& 0.363& 0.900\\
% \midrule

% \rowcolor[gray]{0.9} & \multicolumn{7}{c}{Falcon7b-Base} \\

% \midrule

% MCSE & 0.053& 0.399& 0.419& 0.285& 0.535& 0.421& 0.598\\
% MCNSE & 0.149& 0.395& 0.452& 0.318& 0.588& 0.443& 0.771\\
% Semantic Entropy & 0.052& 0.420& 0.460& 0.319& 0.566& 0.463& 0.567\\
% SAR & 0.196& 0.455& 0.521& 0.385& 0.642& 0.509& 0.826\\
% Degree Matrix & -0.056& 0.350& \underline{0.422}& 0.422& 0.637& 0.480& 0.633\\
% Eigenvalue Laplacian & -0.061& 0.289& 0.382& 0.393& 0.622& 0.454& 0.522\\

% \midrule

% Maximum Sequence Probability & -0.011& 0.385& 0.378& 0.369& 0.638& \textbf{0.540}& -0.175\\
% $\text{CoCoA}_{MSP}$ & 0.119& \underline{0.564}& \underline{0.698}& \textbf{0.428}& \textbf{0.659}& 0.530& 0.498\\

% \midrule

% Perplexity & 0.252& 0.541& 0.637& 0.353& 0.614& \textbf{0.540}& \textbf{1.000}\\
% $\text{CoCoA}_{PPL}$ & \textbf{0.328}& \textbf{0.579}& \textbf{0.717}& 0.415& 0.644& 0.530& \textbf{1.000}\\

% \midrule

% Mean Token Entropy & 0.124& 0.475& 0.510& 0.317& 0.573& 0.534& 0.984\\
% $\text{CoCoA}_{MTE}$ & \underline{0.284}& 0.551& 0.678& 0.402& \underline{0.646}& 0.517& 0.998\\
% \bottomrule

% \end{tabular}
% \caption{Detailed experimental results for Best Sample Evaluated Sequence.}
% \end{table*}


\begin{table*}[ht!]
\footnotesize
\centering

\begin{tabular}{lrrrrrrr}
\toprule
    \multirow{2}{*}{\textbf{Method}}  & \multicolumn{7}{c}{\textbf{Dataset}}  \\ 
      \cmidrule(lr){2-8}  \\
  & XSum & WMT14FrEn & WMT19DeEn & CoQa & Trivia & MMLU & GSM8k \\
  \midrule

\rowcolor[gray]{0.9} & \multicolumn{7}{c}{Mistral7b-Base} \\

\midrule

MCSE & 0.162& 0.406& 0.407& 0.289& 0.492& 0.339& 0.693\\
MCNSE & 0.196& 0.471& 0.507& 0.277& 0.529& 0.358& 0.700\\
Semantic Entropy & 0.164& 0.434& 0.442& 0.312& 0.554& 0.389& 0.675\\
SAR & 0.175& 0.563& 0.590& 0.347& 0.620& 0.421& 0.780\\
DegMat & 0.205& 0.439& 0.410& 0.376& 0.618& 0.410& 0.454\\
EigValLaplacian & 0.197& 0.388& 0.344& 0.342& 0.600& 0.399& 0.361\\

\midrule

MSP & 0.330& 0.212& 0.291& 0.388& 0.607& \textbf{0.476}& 0.307\\
$\text{CoCoA}_{MSP}$ & 0.366& \underline{0.712}& 0.730& \textbf{0.430}& 0.644& 0.466& 0.562\\

\midrule

PPL & 0.365& 0.695& 0.676& 0.327& 0.615& \textbf{0.476}& 0.931\\
$\text{CoCoA}_{PPL}$ & \underline{0.372}& \textbf{0.735}& \textbf{0.755}& \underline{0.402}& \textbf{0.648}& 0.466& \textbf{0.937}\\

\midrule

MTE & 0.350& 0.668& 0.606& 0.254& 0.594& 0.457& 0.932\\
$\text{CoCoA}_{MTE}$ & \textbf{0.373}& 0.708& \underline{0.732}& 0.373& \underline{0.645}& 0.447& \underline{0.935}\\
\midrule

\rowcolor[gray]{0.9} & \multicolumn{7}{c}{Llama8b-Base} \\

\midrule

MCSE & 0.192& 0.366& 0.395& 0.259& 0.465& 0.158& 0.546\\
MCNSE & 0.186& 0.377& 0.480& 0.239& 0.484& 0.165& 0.634\\
Semantic Entropy & 0.194& 0.371& 0.451& 0.286& 0.528& 0.213& 0.559\\
SAR & 0.159& 0.441& 0.571& 0.327& 0.578& 0.340& 0.668\\
DegMat & 0.191& 0.274& 0.409& \underline{0.367}& \textbf{0.606}& 0.320& 0.397\\
EigValLaplacian & 0.190& 0.216& 0.333& 0.340& \underline{0.587}& 0.274& 0.351\\

\midrule

MSP & \underline{0.464}& 0.339& 0.413& 0.304& 0.514& \textbf{0.483}& 0.281\\
$\text{CoCoA}_{MSP}$ & \textbf{0.484}& \underline{0.529}& \underline{0.685}& \textbf{0.384}& \underline{0.587}& \underline{0.452}& 0.513\\

\midrule

PPL & 0.458& 0.504& 0.622& 0.294& 0.483& 0.441& \underline{0.912}\\
$\text{CoCoA}_{PPL}$ & 0.450& \textbf{0.544}& \textbf{0.689}& 0.364& 0.573& 0.422& \textbf{0.925}\\

\midrule

MTE & 0.449& 0.437& 0.501& 0.239& 0.458& 0.326& 0.884\\
$\text{CoCoA}_{MTE}$ & 0.451& 0.520& 0.638& 0.346& 0.582& 0.363& 0.900\\
\midrule

\rowcolor[gray]{0.9} & \multicolumn{7}{c}{Falcon7b-Base} \\

\midrule

MCSE & 0.128& 0.399& 0.419& 0.285& 0.535& 0.421& 0.598\\
MCNSE & 0.153& 0.395& 0.452& 0.318& 0.588& 0.443& 0.771\\
Semantic Entropy & 0.134& 0.420& 0.460& 0.319& 0.566& 0.463& 0.567\\
SAR & 0.193& 0.455& 0.521& 0.385& 0.642& 0.509& 0.826\\
DegMat & 0.177& 0.350& \underline{0.422}& 0.422& 0.637& 0.480& 0.633\\
EigValLaplacian & 0.174& 0.289& 0.382& 0.393& 0.622& 0.454& 0.522\\

\midrule

MSP & 0.099& 0.385& 0.378& 0.369& 0.638& \textbf{0.540}& -0.175\\
$\text{CoCoA}_{MSP}$ & 0.210& \underline{0.564}& \underline{0.698}& \textbf{0.428}& \textbf{0.659}& 0.530& 0.498\\

\midrule

PPL & 0.275& 0.541& 0.637& 0.353& 0.614& \textbf{0.540}& \textbf{1.000}\\
$\text{CoCoA}_{PPL}$ & \textbf{0.310}& \textbf{0.579}& \textbf{0.717}& 0.415& 0.644& 0.530& \textbf{1.000}\\

\midrule

MTE & 0.186& 0.475& 0.510& 0.317& 0.573& 0.534& 0.984\\
$\text{CoCoA}_{MTE}$ & \underline{0.289}& 0.551& 0.678& 0.402& \underline{0.646}& 0.517& 0.998\\
\bottomrule

\end{tabular}
\caption{Detailed experimental results with best sample taken in isolation.}
\label{tab:experimental_results_best}
\end{table*}

  % 

\begin{table*}[ht!]
\footnotesize
\centering

\begin{tabular}{lrrrrrrr}
\toprule
    \multirow{2}{*}{\textbf{Method}}  & \multicolumn{7}{c}{\textbf{Dataset}}  \\ 
      \cmidrule(lr){2-8}  \\
  & XSum & WMT14FrEn & WMT19DeEn & CoQa & Trivia & MMLU & GSM8k \\
  \midrule

\rowcolor[gray]{0.9} & \multicolumn{7}{c}{Mistral7b-Base} \\

\midrule

MCSE & 0.072& 0.270& 0.338& 0.320& 0.555& 0.404& 0.713\\
MCNSE & 0.080& 0.377& 0.482& 0.325& 0.592& 0.420& 0.616\\
Semantic Entropy & 0.075& 0.289& 0.369& 0.339& 0.611& 0.450& 0.665\\
SAR & 0.164& 0.418& 0.534& 0.401& 0.662& 0.467& 0.692\\
DegMat & 0.161& 0.309& 0.384& 0.428& 0.670& 0.460& 0.462\\
EigValLaplacian & 0.161& 0.302& 0.346& 0.409& 0.659& 0.448& 0.387\\

\midrule

MSP & 0.239& 0.379& 0.475& 0.366& 0.545& 0.618& 0.644\\
$\text{CoCoA}_{MSP}$ & \underline{0.321}& 0.620& 0.695& 0.504& 0.680& \textbf{0.625}& 0.688\\

\midrule

PPL & 0.237& 0.631& 0.664& 0.400& 0.551& 0.618& 0.711\\
$\text{CoCoA}_{PPL}$ & \textbf{0.326}& \textbf{0.673}& \textbf{0.738}& \textbf{0.529}& \underline{0.685}& \textbf{0.625}& \underline{0.777}\\

\midrule

MTE & 0.215& 0.613& 0.658& 0.338& 0.642& 0.470& 0.705\\
$\text{CoCoA}_{MTE}$ & 0.314& \underline{0.657}& \underline{0.727}& \underline{0.512}& \textbf{0.725}& 0.545& \textbf{0.778}\\
\midrule

\rowcolor[gray]{0.9} & \multicolumn{7}{c}{Llama8b-Base} \\

\midrule

MCSE & 0.086& 0.270& 0.313& 0.299& 0.488& 0.344& 0.498\\
MCNSE & 0.100& 0.423& 0.475& 0.274& 0.504& 0.355& 0.417\\
Semantic Entropy & 0.086& 0.278& 0.359& 0.337& 0.552& 0.404& 0.507\\
SAR & 0.168& 0.461& 0.536& 0.385& 0.604& 0.483& 0.503\\
DegMat & 0.128& 0.307& 0.402& 0.445& \underline{0.620}& 0.440& 0.445\\
EigValLaplacian & 0.123& 0.272& 0.344& 0.439& 0.611& 0.389& 0.423\\

\midrule

MSP & 0.257& 0.451& 0.472& 0.319& 0.476& 0.755& 0.471\\
$\text{CoCoA}_{MSP}$ & \textbf{0.322}& 0.591& 0.696& 0.460& 0.603& \textbf{0.796}& 0.542\\

\midrule

PPL & 0.254& 0.564& 0.634& 0.355& 0.489& 0.732& 0.640\\
$\text{CoCoA}_{PPL}$ & \underline{0.312}& \textbf{0.626}& \textbf{0.719}& \textbf{0.499}& 0.617& \underline{0.780}& \underline{0.736}\\

\midrule

MTE & 0.230& 0.602& 0.628& 0.316& 0.548& 0.552& 0.686\\
$\text{CoCoA}_{MTE}$ & 0.299& \underline{0.618}& \underline{0.703}& \underline{0.491}& \textbf{0.652}& 0.679& \textbf{0.747}\\
\midrule

\rowcolor[gray]{0.9} & \multicolumn{7}{c}{Falcon7b-Base} \\

\midrule

MCSE & 0.118& 0.323& 0.316& 0.319& 0.601& 0.496& 0.397\\
MCNSE & 0.110& 0.449& 0.497& 0.341& 0.642& 0.516& 0.444\\
Semantic Entropy & 0.122& 0.355& 0.368& 0.342& 0.633& 0.527& 0.426\\
SAR & 0.138& 0.475& 0.545& 0.396& 0.690& 0.551& 0.517\\
DegMat & 0.159& 0.362& 0.429& 0.435& \underline{0.691}& 0.538& 0.444\\
EigValLaplacian & 0.155& 0.339& 0.409& 0.413& 0.688& 0.508& 0.396\\

\midrule

MSP & 0.137& 0.476& 0.435& 0.374& 0.548& \textbf{0.691}& 0.192\\
$\text{CoCoA}_{MSP}$ & \underline{0.201}& 0.648& 0.671& 0.499& 0.654& 0.684& 0.317\\

\midrule

PPL & 0.160& 0.581& 0.608& 0.378& 0.522& \textbf{0.691}& 0.728\\
$\text{CoCoA}_{PPL}$ & \textbf{0.204}& \textbf{0.656}& \textbf{0.715}& \textbf{0.515}& 0.650& 0.684& \textbf{0.802}\\

\midrule

MTE & 0.145& 0.607& 0.624& 0.348& 0.625& 0.548& 0.647\\
$\text{CoCoA}_{MTE}$ & 0.195& \underline{0.652}& \underline{0.698}& \underline{0.500}& \textbf{0.697}& 0.618& \underline{0.747}\\
\bottomrule

\end{tabular}
\caption{Detailed experimental results with random sample taken in isolation.}
\label{tab:experimental_results_random}
\end{table*}


\end{document}

\typeout{get arXiv to do 4 passes: Label(s) may have changed. Rerun}
