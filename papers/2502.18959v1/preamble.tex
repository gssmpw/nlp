%%%%%%%%%%%%%%%%%%%%%%%%%%%%%%%%%%%%%%%%%%%%%%%%
\usepackage[a4paper]{geometry}		
%
%% for example, change the margins to 2 inches all round
%%\geometry{top=1.25cm, bottom=-.6cm, left=1.5cm, right=1.5cm}
\geometry{top=2.540cm, bottom=2.540cm, left=2.540cm, right=2.540cm} 

% Specify the dimensions of each page

% \oddsidemargin .25in    %   Note \oddsidemargin = \evensidemargin
% \evensidemargin .25in
% \marginparwidth 0.07 true in
% %\marginparwidth 0.75 true in
% %\topmargin 0 true pt           % Nominal distance from top of page to top of
% %\topmargin 0.125in
% \topmargin -0.5in
% \addtolength{\headsep}{0.25in}
% \textheight 8.5 true in       % Height of text (including footnotes & figures)
% \textwidth 6.0 true in        % Width of text line.
% \widowpenalty=10000
% \clubpenalty=10000



% \usepackage[section]{placeins}
\setlength{\textfloatsep}{8pt plus 2pt minus 2pt}
%%% 8-12pt
%%%%%%%%%%%%%%%%%%%55
\usepackage{amsmath,amsfonts,amssymb}
% \usepackage{mathrsfs} % \mathscr
\usepackage[scr=rsfs]{mathalpha} % \mathscr
\usepackage{mathtools}
\usepackage{dsfont} % \mathds{1}
\usepackage{bm}
% \usepackage[realmainfile]{currfile}
% \def\mycurrentfilename{\urlstyle{rm}\expandafter\nolinkurl\expandafter{\currfilename}}
% % use \currfilebase (main) 
% % or \currfilename (main.tex)
% \usepackage{fancyhdr,lastpage}
% % \setlength{\headheight}{14.0pt}
% \pagestyle{fancy}
% % \fancyhf{} % Clear all header and footer fields
% \fancyhead[R]{ \large  Page \thepage\ of \pageref{LastPage}} % Left side of the header
% \fancyhead[C]{}
% % \fancyhead[C]{ \large  Page \thepage\ of \pageref{LastPage};\quad  You are editing \qquad
% % \color{red}\bf 
% % \mycurrentfilename} % Center of the header
% \fancyhead[L]{ \large   You are editing \quad
% \color{red}\bf 
% \mycurrentfilename} % Right side of the header



% \usepackage{MnSymbol}
%\usepackage{indentfirst}
%\setlength{\parindent}{2em}% indent
%\usepackage{natbib} % references
%\usepackage[numbers,sort]{natbib} # author year
\usepackage{xurl,xcolor,enumerate}
% \usepackage[nocompress,sort]{cite} %for numbers
% %%%%%%%%%%%%%%%%%%%%%%%%%%%%%%%%%%
% \usepackage[bookmarks,colorlinks]{hyperref}
% \hypersetup{colorlinks=true,citecolor=cyan,linkcolor=cyan}
% \definecolor{darkblue}{rgb}{0,0.22,0.66}
% %{0.0, 0.0, 0.8}
% \hypersetup{citecolor=cyan,linkcolor=darkblue}
%%%%%%%%%%%%%%%%%%%%%%%%%%%%%%%%%%%%%%%%%%%%
%%%% table setting
\usepackage{tabularx,multirow,array,booktabs}
\usepackage{colortbl}%%% color in tables
%%%%%%%%%%%%%%%%%%%%%%%%%%%%%%%%%%%%%%%%%%%%%%%%%%%%
%%%%%% figure setting
\usepackage{graphicx}%% 
\graphicspath{{figures/}}%
% \graphicspath{{figures/}{figuresOld/}}%
\usepackage{float}%% 
\usepackage{subcaption} %% subfigure env
% \captionsetup{subrefformat=parens} 
\renewcommand\thesubfigure{(\alph{subfigure})}
\captionsetup[subfigure]{labelformat=simple, labelsep=space, 
% labelfont=bf
}
% \usepackage{subcaption} %% subfigure env
% \captionsetup{subrefformat=parens} 
\captionsetup[subfigure]{aboveskip=3pt}
\usepackage{caption}
\captionsetup[figure]{aboveskip=7pt}
\captionsetup[figure]{belowskip=2pt}
\captionsetup[table]{aboveskip=5pt}
%%%%%%%%%%%%%%%%%%%%%%%%%%%%%%%%%%
%%%%%%%%%%%%%%%%%%%%%%%%%%%%%%%%%%%%%
%\makeatletter\@addtoreset{equation}{section}\makeatother
%%% \thequation resets when \thesection updates
%\renewcommand{\theequation}{\arabic{section}.\arabic{equation}}
% %%%%%%%%%%%%%%%%%%%%%%%%%%%%%%%%%
% %%%%%%% set page properties
% \setlength{\oddsidemargin}{0.5cm}
% \setlength{\evensidemargin}{0.5cm}
% \setlength{\textwidth}{15.8cm}
% \setlength{\topmargin}{0.4cm}
% \setlength{\headheight}{0.0cm}% 
% \setlength{\headsep}{0.0cm}
% \setlength{\textheight}{23.5cm}
% Specify the dimensions of each page
%%%%%%%%%%%%%%%%%%%%%%%%%%%%%%%
%%%%%%%%%%%%%%%%%%%%%%%%%%%%%%%%%%
%% Theorem environment below
\usepackage{amsthm}
\theoremstyle{plain}
\newtheorem{theorem}{Theorem}[section]%  
\newtheorem{corollary}[theorem]{Corollary} %
\newtheorem*{main}{Main Theorem}% 
\newtheorem{lemma}[theorem]{Lemma}
\newtheorem*{lemma*}{Lemma}
\newtheorem{problem}[theorem]{Problem}
\newtheorem{claim}[theorem]{Claim}
\newtheorem{proposition}[theorem]{Proposition}
\theoremstyle{definition}
\newtheorem{definition}[theorem]{Definition}
\theoremstyle{remark}
\newtheorem*{notation}{Notation}
\newtheorem{remark}[theorem]{Remark}
\newtheorem*{remark*}{Remark}
%%%%%%%%%%%%%%%%%%%%%%%%%%%%%%%%%%%%%%%%%%%%%%%%%%%%%%%%%%%%%%%%%%%%%%%%%%%%
%%%% New commands below
%%%% math fonts
\usepackage{etoolbox,xparse,dsfont,bm,amssymb}
\newcommand{\myMFabc}[4]{\expandafter#1\csname#3#4\endcsname{{#2{#4}}}}
%\forcsvlist{\myMF{\newcommand}{\bm}{bm}}{x,y}
\newcommand{\myMFcmd}[4]{\expandafter#1\csname#3#4\endcsname{{#2{\csname#4\endcsname}}}}
%\forcsvlist{\myMFcmd{\newcommand}{\bm}{bm}}{theta,phi}


\newcommand{\MFabc}[3][\newcommand]{
    \def\doOld##1##2{\forcsvlist{\myMFabc{#1}{##1}{##2}}{#3}}
    \providecommand{\do}{do}
    \RenewDocumentCommand \do { >{\SplitList{,}} m } { \doOld##1 }
    \docsvlist{#2}
}
% \MFabc{ {\bm,bm}, {\mathcal,cal} }{A,B,C}
\newcommand{\MFcmd}[3][\newcommand]{
    \def\doOld##1##2{\forcsvlist{\myMFcmd{#1}{##1}{##2}}{#3}}
    \providecommand{\do}{do}
    \RenewDocumentCommand \do { >{\SplitList{,}} m } { \doOld##1 }
    \docsvlist{#2}
}

%%% 0 and 1
\newcommand{\bmzero}{{\bm{0}}}
\newcommand{\bmone}{{\bm{1}}}
\def\bbone{{\ensuremath{\mathds{1}}}}
\let\one\bbone
%  mathfrak, e.g., \frakT = \mathfrak{T}
\MFabc{ {\mathfrak,frak}, }{T,u,U,o,O,E,m}
%  mathbb, e.g., \R = \mathbb{R}
\MFabc{ {\mathbb, }, }{A,N,Z,R,C,Q}

%%% all capital letters
% \mathcal, \mathscr, and \mathbb 
% e.g., \calA=\mathcal{A}
% \bm\mathcal, e.g., \calbmA = \bm{\mathcal{A}}
\newcommand{\hatbm}[1]{\widehat{\bm{#1}}}
\newcommand{\tildebm}[1]{\widetilde{\bm{#1}}}
\newcommand{\bmcal}[1]{\bm{\mathcal{#1}}}
\newcommand{\caltilde}[1]{\mathcal{\widetilde{#1}}}
\newcommand{\calhat}[1]{\mathcal{\widehat{#1}}} 
\newcommand{\scrtilde}[1]{\widetilde{\mathscr{#1}\mspace{1mu}\mspace{-1mu}}}% \mspace{1mu}\mspace{-1mu} to avoid spacing conflict \mathscr and \widetilde
\newcommand{\scrhat}[1]{\mathscr{\widehat{#1}\mspace{1mu}\mspace{-1mu}}}
\newcommand{\bmcalhat}[1]{\bm{\mathcal{\widehat{#1}}}}
\newcommand{\bmcaltilde}[1]{\bm{\mathcal{\widetilde{#1}}}}
%%%%%%
\MFabc{ {\mathcal,cal}, {\mathscr,scr}, {\mathbb,bb}, {\bmcal,bmcal}, {\bmcal,calbm}, {\caltilde,caltilde}, {\caltilde,tildecal}, {\calhat,calhat}, {\calhat,hatcal}, {\scrtilde,scrtilde}, {\scrtilde,tildescr}, {\scrhat,scrhat}, {\scrhat,hatscr}, {\bmcalhat,bmcalhat}, {\bmcalhat,bmhatcal}, 
{\bmcalhat,calbmhat}, {\bmcalhat,calhatbm}, {\bmcalhat,hatbmcal}, {\bmcalhat,hatcalbm}, {\bmcaltilde,bmcaltilde}, {\bmcaltilde,bmtildecal}, {\bmcaltilde,calbmtilde}, {\bmcaltilde,caltildebm}, {\bmcaltilde,tildebmcal}, {\bmcaltilde,tildecalbm}}{A,B,C,D,E,F,G,H,I,J,K,L,M,N,O,P,Q,R,S,T,U,V,W,X,Y,Z}

\renewcommand{\scrtildeA}{\widetilde{\mathscr{A}\mspace{2mu}}\mspace{-6.1mu}}% \mspace{2mu} and\mspace{-6.1mu} to avoid spacing conflict \mathscr and \widetilde
\renewcommand{\scrhatA}{\mathscr{\widehat{A}\mspace{2mu}}\mspace{-6.1mu}}
\let\tildescrA\scrtildeA
\let\hatscrA\scrhatA

%%% all letters
% bm, \widehat, \widetilde, and
% \mathsf shortcuts -- for random variables, 
% e.g., \bmA=\bm{A}, \bmalpha=\bm{\alpha}
%%%%%%%%%
\MFabc{{\bm,bm}, {\mathsf,sf}, {\widehat,hat}, {\widetilde,tilde}, {\hatbm,hatbm}, {\hatbm,bmhat}, {\tildebm,tildebm}, {\tildebm,bmtilde}}{a,b,c,d,e,f,g,h,i,j,k,l,m,n,o,p,q,r,s,t,u,v,w,x,y,z,A,B,C,D,E,F,G,H,I,J,K,L,M,N,O,P,Q,R,S,T,U,V,W,X,Y,Z}

%%% all Greek alphabet, e.g., \bmalpha = \bm{\alpha}
\let\eps\varepsilon
\MFcmd{{\bm,bm}, {\widehat,hat}, {\widetilde,tilde}, {\tildebm, tildebm}, {\tildebm, bmtilde}, {\hatbm, hatbm}, {\hatbm, bmhat}}{alpha,beta,gamma,delta,epsilon,zeta,eta,theta,iota,kappa,lambda,mu,nu,xi,omicron,pi,rho,sigma,tau,upsilon,phi,chi,psi,omega,Alpha,Beta,Gamma,Delta,Epsilon,Zeta,Eta,Theta,Iota,Kappa,Lambda,Mu,Nu,Xi,Omicron,Pi,Rho,Sigma,Tau,Upsilon,Phi,Chi,Psi,Omega,varrho,varphi,vartheta,varepsilon,varsigma,ell,eps}



%%%%%%%%%% activation functions
%\newcommand{\actdef}[1]{\lowercase{\expandafter\def\csname#1\endcsname}{\texttt{#1}}}
\newcommand{\actdef}[1]{\expandafter\def\csname#1\endcsname{{\ensuremath{\mathtt{#1}}}}}
\forcsvlist{\actdef}{ReLU, LReLU, LeakyReLU, ELU, GELU, SiLU, Softplus, dGELU, dSiLU, dSoftplus, Tanh, Sigmoid, Arctan, Softsign, SRS, dSRS, Swish, dSwish, Mish, dMish, SELU, CELU, dSELU, Sin,SinLU, SinTU, PSinTU, sine, cosine, Sine, Cosine, EUAF}

% \newcommand{\actdefs}[1]{\expandafter\def\csname#1s\endcsname{{\ensuremath{\mathtt{#1}}s}}}
% \forcsvlist{\actdefs}{ELU, GELU, SiLU, dGELU, dSiLU, SRS, dSRS, dSwish}


%%%%%%%%%%%%%%%%%%%%%%%%%%%%%%%%%%%%%%%%%%%
%%%%%%%%%%%%%%%%%%%%%%%%%%%%%%%%%%%%%%%%%%%%
\newlength{\myLength}
\newcommand{\negphantom}[2][1]{\settowidth{\myLength}{#2}\hspace{-#1\myLength}}
\newcommand{\posphantom}[2][1]{\settowidth{\myLength}{#2}\hspace{#1\myLength}}
%%%%%%%%%%%%%%%%%%%%%%%%%%%%%%%%%%%

%\newcommand{\nn}{{\hspace{0.6pt}\mathcal{N}\hspace{-1.80pt}\mathcal{N}\hspace{-1.3pt}}}
% \newcommand{\nn}[6][]{\ensuremath{
% 		{\hspace{0.6pt}\mathcal{N}\hspace{-1.98pt}\mathcal{N}\hspace{-0.725pt}}_{\hspace{-0.75pt}#2\hspace{-0.02pt}}%  NN
% 		#1\{#3,\hspace{1.987pt} #4;\hspace{3.0297pt} \R^{#5}\hspace{-1.0298pt}\to\hspace{-0.98pt}\R^{#6}#1\}
% }}
% \newcommand{\mmnn}{\ensuremath{	{\hspace{0.6pt}\mathcal{M}\hspace{-1.98pt}\mathcal{M}\hspace{-1.98pt}\mathcal{N}\hspace{-1.98pt}\mathcal{N}\hspace{-0.725pt}}
% }}

\newcommand{\mn}[7][]{\ensuremath{	{\hspace{0.6pt}\mathcal{M}\hspace{-0.0pt}\mathcal{N}\hspace{-0.8pt}}_{#2}%  NN
		#1\{#3,\hspace{2.2pt} #4,\hspace{2.2pt} #5;\hspace{3.0297pt} {#6}\hspace{-1.0298pt}\to\hspace{-0.98pt}{#7}#1\}
}}


\let\mmnn\mn

\newcommand{\nn}[6][]{\ensuremath{	{\hspace{0.6pt}\mathcal{N}\hspace{-1.28pt}\mathcal{N}\hspace{-0.725pt}}_{#2}%  NN
		#1\{#3,\hspace{1.987pt} #4;\hspace{3.0297pt} {#5}\hspace{-1.0298pt}\to\hspace{-0.98pt}{#6}#1\}
}}

\renewcommand{\nn}[6][]{\ensuremath{	{\hspace{0.6pt}\mathcal{F}\hspace{-0.828pt}\mathcal{N}\hspace{-0.725pt}}_{#2}%  NN
		#1\{#3,\hspace{1.987pt} #4;\hspace{3.0297pt} {#5}\hspace{-1.0298pt}\to\hspace{-0.98pt}{#6}#1\}
}}
% \let\nn\mn
\let\fcnn\nn

% \NewDocumentCommand{\nn}{o o m m m m}{\ensuremath{		{\hspace{0.6pt}\mathcal{N}\hspace{-1.98pt}\mathcal{N}\hspace{-0.725pt}}_{\hspace{-0.75pt}#2\hspace{-0.02pt}}%  NN
%     		#1\{#3,\hspace{1.987pt} #4;\hspace{3.0297pt} \R^{#5}\hspace{-1.0298pt}\to\hspace{-0.98pt}\R^{#6}#1\}
% }}

% \NewDocumentCommand{\nnOneD}{o o m m m m}{\ensuremath{
% {\hspace{0.6pt}\mathcal{N}\hspace{-1.98pt}\mathcal{N}\hspace{-0.725pt}}_{#2}%  NN
% 		#1\{#3,\hspace{1.987pt} #4;\hspace{3.0297pt} {#5}\hspace{-1.0298pt}\to\hspace{-0.98pt}{#6}#1\}
% }}


%\newcommand{\tbinom}[2]{{\textstyle\binom{#1}{#2}}}
% \newcommand{\ts}{{T}}
\let\ts\intercal
\def\ts{\tn{$\textsf{T}$}}
\def\ts{\textsf{\textup{T}}}

% \newcommand{\cpwl}{\tn{CPwL}}
\def\cpwl{\textsf{\textup{CPwL}}}
\let\cpl\cpwl
\let\CPwL\cpwl
\newcommand{\din}{{d_\tn{in}}}
\newcommand{\dout}{{d_\tn{out}}}
\newcommand{\dprime}{{\prime\prime}}
\newcommand{\tprime}{{\prime\prime\prime}}
\newcommand{\bin}{\tn{bin}\hspace{1.2pt}}
\newcommand{\mystep}[2]{\par \vspace{0.25cm}\noindent\textbf{\hspace{8pt}Step }$#1\colon$ #2 \vspace{0.18cm} \par }

\newcommand{\mycase}[2]{\par \vspace{0.25cm}\noindent\textbf{\hspace{8pt}Case }$#1\colon$ #2\par \vspace{0.18cm} \par}

% \newcommand{\myto}[2][1]{\mathop{\raisebox{-0pt}{\scalebox{#2}[#1]{$\longrightarrow$}}}}
\usepackage{tikz}
\newcommand{\myto}[2][1]{\mathop{
		\vcenter{\hbox{\scalebox{1}[#1]{\tikz{\draw[->,line width=0.72pt] (0,0.5) to (0.69*#2,0.5);}}}}
}}





%% MnSymbol comflicts with dsfont,  $\mathds{1}$ conflicts with $\mathcal{A}$
%% \newlength{\myonewidth}
%% \def\one{  \settowidth{\myonewidth}{$1$}
%% 	\mbox{1\hspace{-0.733\myonewidth}1}    }
%\def\one{{\ensuremath{\mathds{1}}}}
%%%%%%%%%%%%%%%%%%%%%%%
%%%% math resize
%% \newcommand{\myMathResize}[2][0.9]{
%% 	\scalebox{#1}[#1]{\(\displaystyle #2\)}
% }
%\newcommand{\setMathResizeRate}[1]{\def\mathResizeRate{#1}}
%\setMathResizeRate{0.8}
%\newcommand{\mathResize}[2][\mathResizeRate]{
%	\scalebox{#1}[#1]{\(\displaystyle #2\)}
%}
%
%%
%%%%%%%%%%%%%%%%%%%%%%%%%%%%%%%%%%%%%%%%%%%%%%%%%%%%%
%\newcommand*\circled[1]{\tikz[baseline=(char.base)]{
%            \node[shape=circle,draw,inner sep=1pt] (char) {#1};}}
%%\renewcommand{\thefootnote}{\textcircled{\raisebox{-0.5pt}{\arabic{footnote}}}}
%% \renewcommand{\thefootnote}{\textcircled{\raisebox{-0.5pt}{\arabic{footnote}}}}
%\renewcommand{\thefootnote}{\textcircled{$\vcenter{\hbox{\arabic{footnote}}}$}}
%
%\interfootnotelinepenalty=10000
%%%%%%%%%%%%%%%%%%%%%%%%%%%%%%%%%%%%%%%%%%%%5
\newenvironment{keywords}{\par \noindent\textbf{Key words}.}{\par}
%%%%%%%%%%%%%%%%%%%%%%%%%%%%%%%%%%%%%
%%%%%%%%%%%%%%%%%%%%%%%%%%%%%%%%%%%%%%%%%%%%%%%%%%%%%%%%%
\DeclareMathOperator*{\argmax}{arg\,max}
\DeclareMathOperator*{\argmin}{arg\,min}
% \DeclareMathOperator*{\sine}{sine}
% \DeclareMathOperator*{\cosine}{cosine}
%\newcommand{\dataset}{{\cal D}}
%\usepackage{makecell}
%\usepackage{amsopn}
%\DeclareMathOperator{\diag}{diag}
%\newcommand{\fracpartial}[2]{\frac{\partial #1}{\partial  #2}}
%\usepackage{url}
%% \newcommand*{\email}[1]{\href{mailto:#1}{\nolinkurl{#1}}}
%\newcommand{\subjclass}[2][1991]{%
%  \let\@oldtitle\@title%
%  \gdef\@title{\@oldtitle\footnotetext{#1 \emph{Mathematics subject classification.} #2}}%
%}
%%%%%%%%%%%%%%%%%%%%%%%%%%%%%%%%%%%%%%%%%%%%%%%5

%%% see web http://phaseportrait.blogspot.com/2007/08/lineno-and-amsmath-compatibility.html
%%%%line number below
\usepackage[left,mathlines]{lineno}
\usepackage{refcount}
%\renewcommand\linenumberfont{\rmfamily\upshape\scriptsize}

%%%%%%%%%%%%%%%%%%%%%%%%%%%%%%%%%%%%%%%%%%
%%%%%% %%%%% line numbers (cross-ref) properties setting
\renewcommand\thelinenumber{{\rmfamily\upshape\normalsize\color{blue}\arabic{linenumber}}}
%%%%%%%%%%%%%%%%%%%%%%%%%%%%%%%%%%%%%%%%%
%%%%% line numbers properties setting
\definecolor{mylinenumbercolor}{HTML}{BEBEBE}
\renewcommand\LineNumber{{\rmfamily\upshape\footnotesize\color{mylinenumbercolor}\arabic{linenumber}}}
%\linenumbers
%%%%%%%%%%%%%%%%%%%%%%%%%%%%%%%%%%%%%%%%%%
\newcommand{\mylinelabel}[1]{\phantomsection\label{#1zsj}\linelabel{#1}}
%\newcommand{\mylineref}[1]{\hyperref[#1zsj]{\ref{#1}}}
%%%%%%%%% modify line numbers
\makeatletter
\newcommand*\patchAmsMathEnvironmentForLineno[1]{%
	\expandafter\let\csname old#1\expandafter\endcsname\csname #1\endcsname
	\expandafter\let\csname oldend#1\expandafter\endcsname\csname end#1\endcsname
	\renewenvironment{#1}%
	{\linenomath\csname old#1\endcsname}%
	{\csname oldend#1\endcsname\endlinenomath}}% 
\newcommand*\patchBothAmsMathEnvironmentsForLineno[1]{%
	\patchAmsMathEnvironmentForLineno{#1}%
	\patchAmsMathEnvironmentForLineno{#1*}}%
\patchBothAmsMathEnvironmentsForLineno{equation}%
\patchBothAmsMathEnvironmentsForLineno{align}%
\patchBothAmsMathEnvironmentsForLineno{flalign}%
\patchBothAmsMathEnvironmentsForLineno{alignat}%
\patchBothAmsMathEnvironmentsForLineno{gather}%
\patchBothAmsMathEnvironmentsForLineno{multline}%
\makeatother
%%%%%%%%%%%%%%%%%%%%%%%%%%%%%%%%%%%%%%%%%%%%%%%%%%%%%%%%

\newcommand{\mailto}[2][]{\href{mailto:#2?cc=#1}{\color{black}#2}}
%%  \email[cc addresses]{addresses}
%%  Multiple addresses are separated by comma (,)

% Sets running headers as well as PDF title and authors
%\headers{Deep Network Approximation Characterized by Number of Neurons}{Z. Shen, H. Yang, and S. Zhang}
% Title. If the supplement option is on, then "Supplementary Material"
% is automatically inserted before the title.
% \title{Universal Approximation of Fixed-Size Neural Networks With a Simple Activation Function\thanks{Submitted to the editors DATE.}}
%\title{Deep Network Approximation: Achieving Arbitrary Accuracy with Fixed Number of Neurons
%}
%\title{On generalizing ReLU network approximation results to other activation functions}


%\date{\today}

%  \subjclass{68T07; 41A46; 41A63}
%%% Local Variables: 
%%% mode:latex
%%% TeX-master: "ex_article"
%%% End: 

% \usepackage{doi}\def\doitext{doi: }
%%%%%%%%%%%%%%%%%%%%%%%%%%
\let\tilde\widetilde
\let\hat\widehat
\let\epsilon\varepsilon
\let\eps\varepsilon
\let\subset\subseteq
\let\tn\textnormal
%%%% 
% \let\mycdots\cdots
% \def\cdots{\mathop{\mycdots}}
\newcommand{\customcdots}{%
	\mathinner{\mkern-0.1mu\cdotp\mkern-0.3mu\cdotp\mkern-0.3mu\cdotp\mkern-0.1mu}%
}
\let\cdots\customcdots
\let\myforall\forall
\def\forall{{\myforall\, }}
\let\myexists\exists
\def\exists{{\myexists\, }}
\let\emptyset\varnothing
%%%%
\long\def\black#1{{\color{black}#1}}
\long\def\red#1{{\color{red}#1}}
\long\def\green#1{{\color{green}#1}}
\long\def\blue#1{{\color{blue}#1}}
\long\def\cyan#1{{\color{cyan}#1}}

% define colors used in table
\definecolor{mygray}{RGB}{230,230,230}
\definecolor{myorange}{HTML}{ff7f0e}

%%%%%%%%%%%%%%%%%%%%%%%%%%%%%%%%%
%%%% redefine proof env
% Note that 
%\newenvironment{foo}[1]{start code with #1}{end code}
% is equvalent to
%\newcommand\foo[1]{start code with #1}
%\def\endfoo{end code}
% \renewenvironment{proof}{\zsj}{\endzsj}
% \LetLtxMacro{\proof}{\myproof}
% \let\endproof\endmyproof
% %%%%%%%%%%%% redef proof env
% \makeatletter
% \renewenvironment{proof}[1][\proofname]{\par
%     \pushQED{\qed}%
%     \normalfont \topsep6\p@\@plus6\p@\relax
%     \trivlist
%     \item\relax
%     {\itshape
%     % \bfseries
%     #1\@addpunct{.}}\hspace\labelsep\ignorespaces
%     }{%
%     \popQED\endtrivlist\@endpefalse
%     }
% \makeatother
% \renewcommand{\qedsymbol}{\BlackBox\hspace{0pt}}
\let\cite\citep
%%%\clearpage
% \bibliographystyle{siam}
% \bibliographystyle{plainnat}
% \bibliographystyle{plain}	
% \bibliographystyle{plainnat}
\usepackage{doi}
\def\doitext{DOI: }
% \def\url#1{\href{#1}{\nolinkurl{#1}}}
% \urlstyle{sf}


% % % create title
% \makeatletter
% % \providecommand{\maketitle}{}
% % \renewcommand{\maketitle}{%
% %   \par
% %   \begingroup
% %     \renewcommand{\thefootnote}{\fnsymbol{footnote}}
% %     % for perfect author name centering
% %     \renewcommand{\@makefnmark}{\hbox to \z@{$^{\@thefnmark}$\hss}}
% %     % The footnote-mark was overlapping the footnote-text,
% %     % added the following to fix this problem               (MK)
% %     \long\def\@makefntext##1{%
% %       \parindent 1em\noindent
% %       \hbox to 1.8em{\hss $\m@th ^{\@thefnmark}$}##1
% %     }
% %     \thispagestyle{empty}
% %     \@maketitle
% %     \@thanks
% %     \@notice
% %   \endgroup
% %   \let\maketitle\relax
% %   \let\thanks\relax
% % }

% % rules for title box at top of first page
% \newcommand{\@toptitlebar}{
%   \hrule height 4\p@
%   \vskip 0.25in
%   \vskip -\parskip%
% }
% \newcommand{\@bottomtitlebar}{
%   \vskip 0.29in
%   \vskip -\parskip
%   \hrule height 1\p@
%   \vskip 0.09in%
% }

% % create title (includes both anonymized and non-anonymized versions)
% \providecommand{\@maketitle}{}
% \renewcommand{\@maketitle}{%
%   \vbox{%
%     \hsize\textwidth
%     \linewidth\hsize
%     \vskip 0.1in
%     \@toptitlebar
%     \centering
%     {\LARGE\bf \@title\par}
%     \@bottomtitlebar
%       \def\And{%
%         \end{tabular}\hfil\linebreak[0]\hfil%
%         \begin{tabular}[t]{c}\bf\rule{\z@}{24\p@}\ignorespaces%
%       }
%       \def\AND{%
%         \end{tabular}\hfil\linebreak[4]\hfil%
%         \begin{tabular}[t]{c}\bf\rule{\z@}{24\p@}\ignorespaces%
%       }
%       \begin{tabular}[t]{c}\bf\rule{\z@}{24\p@}\@author\end{tabular}%
%     \vskip 0.3in \@minus 0.1in
%   }
% }
% \makeatother


%%%%%%%%%%%%%%%%%%%%%%%%%%%%%%%%%%%%%%%%%%%%%%%%%%%%%%%%
%%%% define title page, following JMLR template
\makeatletter
\long\def\@makefntext#1{\@setpar{\@@par\@tempdima \hsize 
		\advance\@tempdima-15pt\parshape \@ne 15pt \@tempdima}\par
	\parindent 2em\noindent \hbox to \z@{\hss{\textsuperscript{\@thefnmark}} \hfil}#1}
%%%%%%%%%%%%%%%%%
\newlength\aftertitskip     \newlength\beforetitskip
\newlength\interauthorskip  \newlength\aftermaketitskip
%% Changeable parameters.
\setlength\aftertitskip{0.1in plus 0.2in minus 0.2in}
\setlength\beforetitskip{0.05in plus 0.08in minus 0.08in}
\setlength\interauthorskip{0.08in plus 0.1in minus 0.1in}
\setlength\aftermaketitskip{0.3in plus 0.1in minus 0.1in}
%%%%%%%%%%%%%%
%% overall definition of maketitle, @maketitle does the real work
\def\maketitle{\par
	\begingroup
	\def\thefootnote{\color{black}\fnsymbol{footnote}}
	\def\@makefnmark{\hbox to 0pt{$^{\@thefnmark}$\hss}}
	\@maketitle \@thanks
	\endgroup
	\setcounter{footnote}{0}
	\let\maketitle\relax \let\@maketitle\relax
	\gdef\@thanks{}\gdef\@author{}\gdef\@title{}\let\thanks\relax}

\def\@startauthor{\noindent \normalsize\bf}
\def\@endauthor{}
\def\@starteditor{\noindent \small {\bf Editor:~}}
\def\@endeditor{\normalsize}
\def\@maketitle{\vbox{\hsize\textwidth
		\linewidth\hsize \vskip \beforetitskip
		{\begin{center} \Large\bf \@title \par \end{center}} \vskip \aftertitskip
		{\def\and{\unskip\enspace{\rm and}\enspace}%
			\def\addr{\small\it}%
			\def\email{\hfill\small\sc}%
            % \def\email{\hfill\small}%
			\def\name{\normalsize\bf}%
			\def\AND{\@endauthor\rm\hss \vskip \interauthorskip \@startauthor}
			\@startauthor \@author \@endauthor}
		\vskip \aftermaketitskip
}}
\makeatother
% %%%%%%%%%%%%% title page setting
\addtolength\aftertitskip{0.285in}
\addtolength\aftermaketitskip{0.15in}
\addtolength\interauthorskip{4pt}
% %%%%%%%%%%%%%%%%%%%%