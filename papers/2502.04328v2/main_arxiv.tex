% CVPR 2025 Paper Template; see https://github.com/cvpr-org/author-kit

\documentclass[10pt,twocolumn,letterpaper]{article}

%%%%%%%%% PAPER TYPE  - PLEASE UPDATE FOR FINAL VERSION
% \usepackage{cvpr}              % To produce the CAMERA-READY version
% \usepackage[review]{cvpr}      % To produce the REVIEW version
\usepackage[pagenumbers]{cvpr} % To force page numbers, e.g. for an arXiv version

% Import additional packages in the preamble file, before hyperref
%
% --- inline annotations
%
\newcommand{\red}[1]{{\color{red}#1}}
\newcommand{\todo}[1]{{\color{red}#1}}
\newcommand{\TODO}[1]{\textbf{\color{red}[TODO: #1]}}
% --- disable by uncommenting  
% \renewcommand{\TODO}[1]{}
% \renewcommand{\todo}[1]{#1}



\newcommand{\VLM}{LVLM\xspace} 
\newcommand{\ours}{PeKit\xspace}
\newcommand{\yollava}{Yo’LLaVA\xspace}

\newcommand{\thisismy}{This-Is-My-Img\xspace}
\newcommand{\myparagraph}[1]{\noindent\textbf{#1}}
\newcommand{\vdoro}[1]{{\color[rgb]{0.4, 0.18, 0.78} {[V] #1}}}
% --- disable by uncommenting  
% \renewcommand{\TODO}[1]{}
% \renewcommand{\todo}[1]{#1}
\usepackage{slashbox}
% Vectors
\newcommand{\bB}{\mathcal{B}}
\newcommand{\bw}{\mathbf{w}}
\newcommand{\bs}{\mathbf{s}}
\newcommand{\bo}{\mathbf{o}}
\newcommand{\bn}{\mathbf{n}}
\newcommand{\bc}{\mathbf{c}}
\newcommand{\bp}{\mathbf{p}}
\newcommand{\bS}{\mathbf{S}}
\newcommand{\bk}{\mathbf{k}}
\newcommand{\bmu}{\boldsymbol{\mu}}
\newcommand{\bx}{\mathbf{x}}
\newcommand{\bg}{\mathbf{g}}
\newcommand{\be}{\mathbf{e}}
\newcommand{\bX}{\mathbf{X}}
\newcommand{\by}{\mathbf{y}}
\newcommand{\bv}{\mathbf{v}}
\newcommand{\bz}{\mathbf{z}}
\newcommand{\bq}{\mathbf{q}}
\newcommand{\bff}{\mathbf{f}}
\newcommand{\bu}{\mathbf{u}}
\newcommand{\bh}{\mathbf{h}}
\newcommand{\bb}{\mathbf{b}}

\newcommand{\rone}{\textcolor{green}{R1}}
\newcommand{\rtwo}{\textcolor{orange}{R2}}
\newcommand{\rthree}{\textcolor{red}{R3}}
\usepackage{amsmath}
%\usepackage{arydshln}
\DeclareMathOperator{\similarity}{sim}
\DeclareMathOperator{\AvgPool}{AvgPool}

\newcommand{\argmax}{\mathop{\mathrm{argmax}}}     



\usepackage{mdwmath}
\usepackage{eqparbox}
\usepackage{epsfig}
\usepackage{graphicx}
\usepackage{amsmath}
\usepackage{amssymb}
\usepackage{pifont}
\usepackage{url}            % simple URL typesetting
\usepackage{booktabs}       % professional-quality tables
\usepackage{tabu}           % tabular
\usepackage{multirow}
\usepackage{graphicx}
\usepackage{algorithm}
\usepackage{algorithmicx}
\usepackage{algpseudocode}
\usepackage{amsmath,amssymb}
% \usepackage[table,dvipsnames]{xcolor}
\usepackage{array}

\newcommand{\cmark}{\ding{51}}%
\newcommand{\xmark}{\ding{55}}%
\usepackage{cite}

\usepackage{makecell}
\usepackage{tabularx}
\usepackage{pifont}
\usepackage{multicol}
\usepackage{array}
\usepackage{adjustbox}

\usepackage{url}            % simple URL typesetting
\usepackage{booktabs}       % professional-quality tables
\usepackage{amsfonts}       % blackboard math symbols
\usepackage{nicefrac}       % compact symbols for 1/2, etc.
\usepackage{microtype}      % microtypography
% \usepackage{xcolor}         % colors
\usepackage{graphicx}
\usepackage{amsmath}
\usepackage{bm}
\usepackage{multirow}
\usepackage{tabu}
\usepackage{siunitx}
\usepackage{adjustbox}
\usepackage{subcaption}
\usepackage{siunitx}
\usepackage{colortbl}
\usepackage{color}
\usepackage{pifont}
\definecolor{Gray}{gray}{0.9}
\definecolor{Lightorange}{RGB}{255,214,169}
\definecolor{Cyan}{rgb}{0.88,1,1}
\definecolor{reminder}{RGB}{255,0,0}


\usepackage{tabu}           % tabular
\usepackage{multirow}
\usepackage{booktabs}
\usepackage{makecell}
\usepackage{pifont}


\newcommand{\paragrapha}[2][3pt]{\vspace{#1}\noindent\textbf{#2}}

\newcolumntype{x}[1]{>{\centering\arraybackslashå}p{#1pt}}
\newlength\savewidth\newcommand\shline{\noalign{\global\savewidth\arrayrulewidth
  \global\arrayrulewidth 1pt}\hline\noalign{\global\arrayrulewidth\savewidth}}
\newcommand\hshline{\noalign{\global\savewidth\arrayrulewidth
  \global\arrayrulewidth 0.6pt}\hline\noalign{\global\arrayrulewidth\savewidth}}
\newcommand{\tablestyle}[2]{\setlength{\tabcolsep}{#1}\renewcommand{\arraystretch}{#2}\centering\footnotesize}

\newcommand{\PreserveBackslash}[1]{\let\temp=\\#1\let\\=\temp}
\newcolumntype{C}[1]{>{\PreserveBackslash\centering}p{#1}}
\newcolumntype{L}[1]{>{\PreserveBackslash\raggedright}p{#1}}

% Optional math commands from https://github.com/goodfeli/dlbook_notation.
% %%%%% NEW MATH DEFINITIONS %%%%%

\usepackage{amsmath,amsfonts,bm}
\usepackage{derivative}
% Mark sections of captions for referring to divisions of figures
\newcommand{\figleft}{{\em (Left)}}
\newcommand{\figcenter}{{\em (Center)}}
\newcommand{\figright}{{\em (Right)}}
\newcommand{\figtop}{{\em (Top)}}
\newcommand{\figbottom}{{\em (Bottom)}}
\newcommand{\captiona}{{\em (a)}}
\newcommand{\captionb}{{\em (b)}}
\newcommand{\captionc}{{\em (c)}}
\newcommand{\captiond}{{\em (d)}}

% Highlight a newly defined term
\newcommand{\newterm}[1]{{\bf #1}}

% Derivative d 
\newcommand{\deriv}{{\mathrm{d}}}

% Figure reference, lower-case.
\def\figref#1{figure~\ref{#1}}
% Figure reference, capital. For start of sentence
\def\Figref#1{Figure~\ref{#1}}
\def\twofigref#1#2{figures \ref{#1} and \ref{#2}}
\def\quadfigref#1#2#3#4{figures \ref{#1}, \ref{#2}, \ref{#3} and \ref{#4}}
% Section reference, lower-case.
\def\secref#1{section~\ref{#1}}
% Section reference, capital.
\def\Secref#1{Section~\ref{#1}}
% Reference to two sections.
\def\twosecrefs#1#2{sections \ref{#1} and \ref{#2}}
% Reference to three sections.
\def\secrefs#1#2#3{sections \ref{#1}, \ref{#2} and \ref{#3}}
% Reference to an equation, lower-case.
\def\eqref#1{equation~\ref{#1}}
% Reference to an equation, upper case
\def\Eqref#1{Equation~\ref{#1}}
% A raw reference to an equation---avoid using if possible
\def\plaineqref#1{\ref{#1}}
% Reference to a chapter, lower-case.
\def\chapref#1{chapter~\ref{#1}}
% Reference to an equation, upper case.
\def\Chapref#1{Chapter~\ref{#1}}
% Reference to a range of chapters
\def\rangechapref#1#2{chapters\ref{#1}--\ref{#2}}
% Reference to an algorithm, lower-case.
\def\algref#1{algorithm~\ref{#1}}
% Reference to an algorithm, upper case.
\def\Algref#1{Algorithm~\ref{#1}}
\def\twoalgref#1#2{algorithms \ref{#1} and \ref{#2}}
\def\Twoalgref#1#2{Algorithms \ref{#1} and \ref{#2}}
% Reference to a part, lower case
\def\partref#1{part~\ref{#1}}
% Reference to a part, upper case
\def\Partref#1{Part~\ref{#1}}
\def\twopartref#1#2{parts \ref{#1} and \ref{#2}}

\def\ceil#1{\lceil #1 \rceil}
\def\floor#1{\lfloor #1 \rfloor}
\def\1{\bm{1}}
\newcommand{\train}{\mathcal{D}}
\newcommand{\valid}{\mathcal{D_{\mathrm{valid}}}}
\newcommand{\test}{\mathcal{D_{\mathrm{test}}}}

\def\eps{{\epsilon}}


% Random variables
\def\reta{{\textnormal{$\eta$}}}
\def\ra{{\textnormal{a}}}
\def\rb{{\textnormal{b}}}
\def\rc{{\textnormal{c}}}
\def\rd{{\textnormal{d}}}
\def\re{{\textnormal{e}}}
\def\rf{{\textnormal{f}}}
\def\rg{{\textnormal{g}}}
\def\rh{{\textnormal{h}}}
\def\ri{{\textnormal{i}}}
\def\rj{{\textnormal{j}}}
\def\rk{{\textnormal{k}}}
\def\rl{{\textnormal{l}}}
% rm is already a command, just don't name any random variables m
\def\rn{{\textnormal{n}}}
\def\ro{{\textnormal{o}}}
\def\rp{{\textnormal{p}}}
\def\rq{{\textnormal{q}}}
\def\rr{{\textnormal{r}}}
\def\rs{{\textnormal{s}}}
\def\rt{{\textnormal{t}}}
\def\ru{{\textnormal{u}}}
\def\rv{{\textnormal{v}}}
\def\rw{{\textnormal{w}}}
\def\rx{{\textnormal{x}}}
\def\ry{{\textnormal{y}}}
\def\rz{{\textnormal{z}}}

% Random vectors
\def\rvepsilon{{\mathbf{\epsilon}}}
\def\rvphi{{\mathbf{\phi}}}
\def\rvtheta{{\mathbf{\theta}}}
\def\rva{{\mathbf{a}}}
\def\rvb{{\mathbf{b}}}
\def\rvc{{\mathbf{c}}}
\def\rvd{{\mathbf{d}}}
\def\rve{{\mathbf{e}}}
\def\rvf{{\mathbf{f}}}
\def\rvg{{\mathbf{g}}}
\def\rvh{{\mathbf{h}}}
\def\rvu{{\mathbf{i}}}
\def\rvj{{\mathbf{j}}}
\def\rvk{{\mathbf{k}}}
\def\rvl{{\mathbf{l}}}
\def\rvm{{\mathbf{m}}}
\def\rvn{{\mathbf{n}}}
\def\rvo{{\mathbf{o}}}
\def\rvp{{\mathbf{p}}}
\def\rvq{{\mathbf{q}}}
\def\rvr{{\mathbf{r}}}
\def\rvs{{\mathbf{s}}}
\def\rvt{{\mathbf{t}}}
\def\rvu{{\mathbf{u}}}
\def\rvv{{\mathbf{v}}}
\def\rvw{{\mathbf{w}}}
\def\rvx{{\mathbf{x}}}
\def\rvy{{\mathbf{y}}}
\def\rvz{{\mathbf{z}}}

% Elements of random vectors
\def\erva{{\textnormal{a}}}
\def\ervb{{\textnormal{b}}}
\def\ervc{{\textnormal{c}}}
\def\ervd{{\textnormal{d}}}
\def\erve{{\textnormal{e}}}
\def\ervf{{\textnormal{f}}}
\def\ervg{{\textnormal{g}}}
\def\ervh{{\textnormal{h}}}
\def\ervi{{\textnormal{i}}}
\def\ervj{{\textnormal{j}}}
\def\ervk{{\textnormal{k}}}
\def\ervl{{\textnormal{l}}}
\def\ervm{{\textnormal{m}}}
\def\ervn{{\textnormal{n}}}
\def\ervo{{\textnormal{o}}}
\def\ervp{{\textnormal{p}}}
\def\ervq{{\textnormal{q}}}
\def\ervr{{\textnormal{r}}}
\def\ervs{{\textnormal{s}}}
\def\ervt{{\textnormal{t}}}
\def\ervu{{\textnormal{u}}}
\def\ervv{{\textnormal{v}}}
\def\ervw{{\textnormal{w}}}
\def\ervx{{\textnormal{x}}}
\def\ervy{{\textnormal{y}}}
\def\ervz{{\textnormal{z}}}

% Random matrices
\def\rmA{{\mathbf{A}}}
\def\rmB{{\mathbf{B}}}
\def\rmC{{\mathbf{C}}}
\def\rmD{{\mathbf{D}}}
\def\rmE{{\mathbf{E}}}
\def\rmF{{\mathbf{F}}}
\def\rmG{{\mathbf{G}}}
\def\rmH{{\mathbf{H}}}
\def\rmI{{\mathbf{I}}}
\def\rmJ{{\mathbf{J}}}
\def\rmK{{\mathbf{K}}}
\def\rmL{{\mathbf{L}}}
\def\rmM{{\mathbf{M}}}
\def\rmN{{\mathbf{N}}}
\def\rmO{{\mathbf{O}}}
\def\rmP{{\mathbf{P}}}
\def\rmQ{{\mathbf{Q}}}
\def\rmR{{\mathbf{R}}}
\def\rmS{{\mathbf{S}}}
\def\rmT{{\mathbf{T}}}
\def\rmU{{\mathbf{U}}}
\def\rmV{{\mathbf{V}}}
\def\rmW{{\mathbf{W}}}
\def\rmX{{\mathbf{X}}}
\def\rmY{{\mathbf{Y}}}
\def\rmZ{{\mathbf{Z}}}

% Elements of random matrices
\def\ermA{{\textnormal{A}}}
\def\ermB{{\textnormal{B}}}
\def\ermC{{\textnormal{C}}}
\def\ermD{{\textnormal{D}}}
\def\ermE{{\textnormal{E}}}
\def\ermF{{\textnormal{F}}}
\def\ermG{{\textnormal{G}}}
\def\ermH{{\textnormal{H}}}
\def\ermI{{\textnormal{I}}}
\def\ermJ{{\textnormal{J}}}
\def\ermK{{\textnormal{K}}}
\def\ermL{{\textnormal{L}}}
\def\ermM{{\textnormal{M}}}
\def\ermN{{\textnormal{N}}}
\def\ermO{{\textnormal{O}}}
\def\ermP{{\textnormal{P}}}
\def\ermQ{{\textnormal{Q}}}
\def\ermR{{\textnormal{R}}}
\def\ermS{{\textnormal{S}}}
\def\ermT{{\textnormal{T}}}
\def\ermU{{\textnormal{U}}}
\def\ermV{{\textnormal{V}}}
\def\ermW{{\textnormal{W}}}
\def\ermX{{\textnormal{X}}}
\def\ermY{{\textnormal{Y}}}
\def\ermZ{{\textnormal{Z}}}

% Vectors
\def\vzero{{\bm{0}}}
\def\vone{{\bm{1}}}
\def\vmu{{\bm{\mu}}}
\def\vtheta{{\bm{\theta}}}
\def\vphi{{\bm{\phi}}}
\def\va{{\bm{a}}}
\def\vb{{\bm{b}}}
\def\vc{{\bm{c}}}
\def\vd{{\bm{d}}}
\def\ve{{\bm{e}}}
\def\vf{{\bm{f}}}
\def\vg{{\bm{g}}}
\def\vh{{\bm{h}}}
\def\vi{{\bm{i}}}
\def\vj{{\bm{j}}}
\def\vk{{\bm{k}}}
\def\vl{{\bm{l}}}
\def\vm{{\bm{m}}}
\def\vn{{\bm{n}}}
\def\vo{{\bm{o}}}
\def\vp{{\bm{p}}}
\def\vq{{\bm{q}}}
\def\vr{{\bm{r}}}
\def\vs{{\bm{s}}}
\def\vt{{\bm{t}}}
\def\vu{{\bm{u}}}
\def\vv{{\bm{v}}}
\def\vw{{\bm{w}}}
\def\vx{{\bm{x}}}
\def\vy{{\bm{y}}}
\def\vz{{\bm{z}}}

% Elements of vectors
\def\evalpha{{\alpha}}
\def\evbeta{{\beta}}
\def\evepsilon{{\epsilon}}
\def\evlambda{{\lambda}}
\def\evomega{{\omega}}
\def\evmu{{\mu}}
\def\evpsi{{\psi}}
\def\evsigma{{\sigma}}
\def\evtheta{{\theta}}
\def\eva{{a}}
\def\evb{{b}}
\def\evc{{c}}
\def\evd{{d}}
\def\eve{{e}}
\def\evf{{f}}
\def\evg{{g}}
\def\evh{{h}}
\def\evi{{i}}
\def\evj{{j}}
\def\evk{{k}}
\def\evl{{l}}
\def\evm{{m}}
\def\evn{{n}}
\def\evo{{o}}
\def\evp{{p}}
\def\evq{{q}}
\def\evr{{r}}
\def\evs{{s}}
\def\evt{{t}}
\def\evu{{u}}
\def\evv{{v}}
\def\evw{{w}}
\def\evx{{x}}
\def\evy{{y}}
\def\evz{{z}}

% Matrix
\def\mA{{\bm{A}}}
\def\mB{{\bm{B}}}
\def\mC{{\bm{C}}}
\def\mD{{\bm{D}}}
\def\mE{{\bm{E}}}
\def\mF{{\bm{F}}}
\def\mG{{\bm{G}}}
\def\mH{{\bm{H}}}
\def\mI{{\bm{I}}}
\def\mJ{{\bm{J}}}
\def\mK{{\bm{K}}}
\def\mL{{\bm{L}}}
\def\mM{{\bm{M}}}
\def\mN{{\bm{N}}}
\def\mO{{\bm{O}}}
\def\mP{{\bm{P}}}
\def\mQ{{\bm{Q}}}
\def\mR{{\bm{R}}}
\def\mS{{\bm{S}}}
\def\mT{{\bm{T}}}
\def\mU{{\bm{U}}}
\def\mV{{\bm{V}}}
\def\mW{{\bm{W}}}
\def\mX{{\bm{X}}}
\def\mY{{\bm{Y}}}
\def\mZ{{\bm{Z}}}
\def\mBeta{{\bm{\beta}}}
\def\mPhi{{\bm{\Phi}}}
\def\mLambda{{\bm{\Lambda}}}
\def\mSigma{{\bm{\Sigma}}}

% Tensor
\DeclareMathAlphabet{\mathsfit}{\encodingdefault}{\sfdefault}{m}{sl}
\SetMathAlphabet{\mathsfit}{bold}{\encodingdefault}{\sfdefault}{bx}{n}
\newcommand{\tens}[1]{\bm{\mathsfit{#1}}}
\def\tA{{\tens{A}}}
\def\tB{{\tens{B}}}
\def\tC{{\tens{C}}}
\def\tD{{\tens{D}}}
\def\tE{{\tens{E}}}
\def\tF{{\tens{F}}}
\def\tG{{\tens{G}}}
\def\tH{{\tens{H}}}
\def\tI{{\tens{I}}}
\def\tJ{{\tens{J}}}
\def\tK{{\tens{K}}}
\def\tL{{\tens{L}}}
\def\tM{{\tens{M}}}
\def\tN{{\tens{N}}}
\def\tO{{\tens{O}}}
\def\tP{{\tens{P}}}
\def\tQ{{\tens{Q}}}
\def\tR{{\tens{R}}}
\def\tS{{\tens{S}}}
\def\tT{{\tens{T}}}
\def\tU{{\tens{U}}}
\def\tV{{\tens{V}}}
\def\tW{{\tens{W}}}
\def\tX{{\tens{X}}}
\def\tY{{\tens{Y}}}
\def\tZ{{\tens{Z}}}


% Graph
\def\gA{{\mathcal{A}}}
\def\gB{{\mathcal{B}}}
\def\gC{{\mathcal{C}}}
\def\gD{{\mathcal{D}}}
\def\gE{{\mathcal{E}}}
\def\gF{{\mathcal{F}}}
\def\gG{{\mathcal{G}}}
\def\gH{{\mathcal{H}}}
\def\gI{{\mathcal{I}}}
\def\gJ{{\mathcal{J}}}
\def\gK{{\mathcal{K}}}
\def\gL{{\mathcal{L}}}
\def\gM{{\mathcal{M}}}
\def\gN{{\mathcal{N}}}
\def\gO{{\mathcal{O}}}
\def\gP{{\mathcal{P}}}
\def\gQ{{\mathcal{Q}}}
\def\gR{{\mathcal{R}}}
\def\gS{{\mathcal{S}}}
\def\gT{{\mathcal{T}}}
\def\gU{{\mathcal{U}}}
\def\gV{{\mathcal{V}}}
\def\gW{{\mathcal{W}}}
\def\gX{{\mathcal{X}}}
\def\gY{{\mathcal{Y}}}
\def\gZ{{\mathcal{Z}}}

% Sets
\def\sA{{\mathbb{A}}}
\def\sB{{\mathbb{B}}}
\def\sC{{\mathbb{C}}}
\def\sD{{\mathbb{D}}}
% Don't use a set called E, because this would be the same as our symbol
% for expectation.
\def\sF{{\mathbb{F}}}
\def\sG{{\mathbb{G}}}
\def\sH{{\mathbb{H}}}
\def\sI{{\mathbb{I}}}
\def\sJ{{\mathbb{J}}}
\def\sK{{\mathbb{K}}}
\def\sL{{\mathbb{L}}}
\def\sM{{\mathbb{M}}}
\def\sN{{\mathbb{N}}}
\def\sO{{\mathbb{O}}}
\def\sP{{\mathbb{P}}}
\def\sQ{{\mathbb{Q}}}
\def\sR{{\mathbb{R}}}
\def\sS{{\mathbb{S}}}
\def\sT{{\mathbb{T}}}
\def\sU{{\mathbb{U}}}
\def\sV{{\mathbb{V}}}
\def\sW{{\mathbb{W}}}
\def\sX{{\mathbb{X}}}
\def\sY{{\mathbb{Y}}}
\def\sZ{{\mathbb{Z}}}

% Entries of a matrix
\def\emLambda{{\Lambda}}
\def\emA{{A}}
\def\emB{{B}}
\def\emC{{C}}
\def\emD{{D}}
\def\emE{{E}}
\def\emF{{F}}
\def\emG{{G}}
\def\emH{{H}}
\def\emI{{I}}
\def\emJ{{J}}
\def\emK{{K}}
\def\emL{{L}}
\def\emM{{M}}
\def\emN{{N}}
\def\emO{{O}}
\def\emP{{P}}
\def\emQ{{Q}}
\def\emR{{R}}
\def\emS{{S}}
\def\emT{{T}}
\def\emU{{U}}
\def\emV{{V}}
\def\emW{{W}}
\def\emX{{X}}
\def\emY{{Y}}
\def\emZ{{Z}}
\def\emSigma{{\Sigma}}

% entries of a tensor
% Same font as tensor, without \bm wrapper
\newcommand{\etens}[1]{\mathsfit{#1}}
\def\etLambda{{\etens{\Lambda}}}
\def\etA{{\etens{A}}}
\def\etB{{\etens{B}}}
\def\etC{{\etens{C}}}
\def\etD{{\etens{D}}}
\def\etE{{\etens{E}}}
\def\etF{{\etens{F}}}
\def\etG{{\etens{G}}}
\def\etH{{\etens{H}}}
\def\etI{{\etens{I}}}
\def\etJ{{\etens{J}}}
\def\etK{{\etens{K}}}
\def\etL{{\etens{L}}}
\def\etM{{\etens{M}}}
\def\etN{{\etens{N}}}
\def\etO{{\etens{O}}}
\def\etP{{\etens{P}}}
\def\etQ{{\etens{Q}}}
\def\etR{{\etens{R}}}
\def\etS{{\etens{S}}}
\def\etT{{\etens{T}}}
\def\etU{{\etens{U}}}
\def\etV{{\etens{V}}}
\def\etW{{\etens{W}}}
\def\etX{{\etens{X}}}
\def\etY{{\etens{Y}}}
\def\etZ{{\etens{Z}}}

% The true underlying data generating distribution
\newcommand{\pdata}{p_{\rm{data}}}
\newcommand{\ptarget}{p_{\rm{target}}}
\newcommand{\pprior}{p_{\rm{prior}}}
\newcommand{\pbase}{p_{\rm{base}}}
\newcommand{\pref}{p_{\rm{ref}}}

% The empirical distribution defined by the training set
\newcommand{\ptrain}{\hat{p}_{\rm{data}}}
\newcommand{\Ptrain}{\hat{P}_{\rm{data}}}
% The model distribution
\newcommand{\pmodel}{p_{\rm{model}}}
\newcommand{\Pmodel}{P_{\rm{model}}}
\newcommand{\ptildemodel}{\tilde{p}_{\rm{model}}}
% Stochastic autoencoder distributions
\newcommand{\pencode}{p_{\rm{encoder}}}
\newcommand{\pdecode}{p_{\rm{decoder}}}
\newcommand{\precons}{p_{\rm{reconstruct}}}

\newcommand{\laplace}{\mathrm{Laplace}} % Laplace distribution

\newcommand{\E}{\mathbb{E}}
\newcommand{\Ls}{\mathcal{L}}
\newcommand{\R}{\mathbb{R}}
\newcommand{\emp}{\tilde{p}}
\newcommand{\lr}{\alpha}
\newcommand{\reg}{\lambda}
\newcommand{\rect}{\mathrm{rectifier}}
\newcommand{\softmax}{\mathrm{softmax}}
\newcommand{\sigmoid}{\sigma}
\newcommand{\softplus}{\zeta}
\newcommand{\KL}{D_{\mathrm{KL}}}
\newcommand{\Var}{\mathrm{Var}}
\newcommand{\standarderror}{\mathrm{SE}}
\newcommand{\Cov}{\mathrm{Cov}}
% Wolfram Mathworld says $L^2$ is for function spaces and $\ell^2$ is for vectors
% But then they seem to use $L^2$ for vectors throughout the site, and so does
% wikipedia.
\newcommand{\normlzero}{L^0}
\newcommand{\normlone}{L^1}
\newcommand{\normltwo}{L^2}
\newcommand{\normlp}{L^p}
\newcommand{\normmax}{L^\infty}

\newcommand{\parents}{Pa} % See usage in notation.tex. Chosen to match Daphne's book.

\DeclareMathOperator*{\argmax}{arg\,max}
\DeclareMathOperator*{\argmin}{arg\,min}

\DeclareMathOperator{\sign}{sign}
\DeclareMathOperator{\Tr}{Tr}
\let\ab\allowbreak


\usepackage{url}

% It is strongly recommended to use hyperref, especially for the review version.
% hyperref with option pagebackref eases the reviewers' job.
% Please disable hyperref *only* if you encounter grave issues, 
% e.g. with the file validation for the camera-ready version.
%
% If you comment hyperref and then uncomment it, you should delete *.aux before re-running LaTeX.
% (Or just hit 'q' on the first LaTeX run, let it finish, and you should be clear).
\definecolor{cvprblue}{rgb}{0.21,0.49,0.74}
\usepackage[pagebackref,breaklinks,colorlinks,allcolors=cvprblue]{hyperref}

%%%%%%%%% PAPER ID  - PLEASE UPDATE
\def\paperID{xxxx} % *** Enter the Paper ID here
\def\confName{CVPR}
\def\confYear{2025}

%%%%%%%%% TITLE - PLEASE UPDATE
% \title{\emph{\color{Orange}Ola}: Omni-Modal Language Model with Progressive Modality Alignment}

\title{\emph{\color{Orange}Ola}: Pushing the Frontiers of Omni-Modal Language Model with\\ Progressive Modality Alignment}

%%%%%%%%% AUTHORS - PLEASE UPDATE
% \author{First Author\\
% Institution1\\
% Institution1 address\\
% {\tt\small firstauthor@i1.org}
% % For a paper whose authors are all at the same institution,
% % omit the following lines up until the closing ``}''.
% % Additional authors and addresses can be added with ``\and'',
% % just like the second author.
% % To save space, use either the email address or home page, not both
% \and
% Second Author\\
% Institution2\\
% First line of institution2 address\\
% {\tt\small secondauthor@i2.org}
% }
\def\spaces{~~~~~~}
\author{Zuyan Liu\textsuperscript{1,2}\thanks{Authors contributed equally to this research.~~\textsuperscript{\dag}Corresponding authors.}\spaces{}Yuhao Dong\textsuperscript{3,2}\footnotemark[1]\spaces{}Jiahui Wang\textsuperscript{1}\spaces{} \\
Ziwei Liu\textsuperscript{3}\spaces{}Winston Hu\textsuperscript{2}\spaces{}Jiwen Lu\textsuperscript{1}$^{\dagger}$\spaces{}Yongming Rao\textsuperscript{2,1}$^{\dagger}$ \\ \\
\textsuperscript{1}~Tsinghua University\spaces{}\textsuperscript{2}~Tencent Hunyuan Research\spaces{}\textsuperscript{3}~S-Lab, NTU\\ \\
\textbf{\color{Orange}\url{https://ola-omni.github.io/}}
}

\begin{document}
% \maketitle

\twocolumn[{
    \renewcommand\twocolumn[1][]{#1}
    \maketitle
    \begin{center}
  % \fbox{\rule{0pt}{1.6in} \rule{0.9\linewidth}{0pt}}
        \includegraphics[width=1.0\linewidth]{figs/fig-sota.pdf}
        \captionof{figure}{\textbf{\textit{Ola} pushes the frontiers of the omni-modal language model across image, video and audio understanding benchmarks. } We compare \emph{Ola} with existing state-of-the-art open-sourced multimodal models and GPT-4o on their abilities in mainstream image, video, and audio benchmarks. For fair comparisons, we select around 7B versions of existing MLLMs. \emph{Ola} can achieve outperforming performance against omni-modal and specialized MLLMs in all modalities thanks to our progressive alignment strategy. ``$\times$'' indicates that the model is not capable of the task and ``$-$'' indicates the result is lacking. The score for LibriSpeech is inverted as lower is better for the WER metric.}
        \label{fig:teaser}
    \end{center}
}]

\let\thefootnote\relax\footnotetext{*~Equal Contribution.~~\textsuperscript{\dag}~Corresponding authors.}

% \begin{figure*}[!h]
%   \centering
%   \fbox{\rule{0pt}{1.6in} \rule{0.9\linewidth}{0pt}}
%    %\includegraphics[width=0.8\linewidth]{egfigure.eps}
%    \caption{Example of caption.
%    It is set in Roman so that mathematics (always set in Roman: $B \sin A = A \sin B$) may be included without an ugly clash.}
%    \label{fig:overview}
% \end{figure*}


\begin{abstract}


The choice of representation for geographic location significantly impacts the accuracy of models for a broad range of geospatial tasks, including fine-grained species classification, population density estimation, and biome classification. Recent works like SatCLIP and GeoCLIP learn such representations by contrastively aligning geolocation with co-located images. While these methods work exceptionally well, in this paper, we posit that the current training strategies fail to fully capture the important visual features. We provide an information theoretic perspective on why the resulting embeddings from these methods discard crucial visual information that is important for many downstream tasks. To solve this problem, we propose a novel retrieval-augmented strategy called RANGE. We build our method on the intuition that the visual features of a location can be estimated by combining the visual features from multiple similar-looking locations. We evaluate our method across a wide variety of tasks. Our results show that RANGE outperforms the existing state-of-the-art models with significant margins in most tasks. We show gains of up to 13.1\% on classification tasks and 0.145 $R^2$ on regression tasks. All our code and models will be made available at: \href{https://github.com/mvrl/RANGE}{https://github.com/mvrl/RANGE}.

\end{abstract}

  
\section{Introduction}
Backdoor attacks pose a concealed yet profound security risk to machine learning (ML) models, for which the adversaries can inject a stealth backdoor into the model during training, enabling them to illicitly control the model's output upon encountering predefined inputs. These attacks can even occur without the knowledge of developers or end-users, thereby undermining the trust in ML systems. As ML becomes more deeply embedded in critical sectors like finance, healthcare, and autonomous driving \citep{he2016deep, liu2020computing, tournier2019mrtrix3, adjabi2020past}, the potential damage from backdoor attacks grows, underscoring the emergency for developing robust defense mechanisms against backdoor attacks.

To address the threat of backdoor attacks, researchers have developed a variety of strategies \cite{liu2018fine,wu2021adversarial,wang2019neural,zeng2022adversarial,zhu2023neural,Zhu_2023_ICCV, wei2024shared,wei2024d3}, aimed at purifying backdoors within victim models. These methods are designed to integrate with current deployment workflows seamlessly and have demonstrated significant success in mitigating the effects of backdoor triggers \cite{wubackdoorbench, wu2023defenses, wu2024backdoorbench,dunnett2024countering}.  However, most state-of-the-art (SOTA) backdoor purification methods operate under the assumption that a small clean dataset, often referred to as \textbf{auxiliary dataset}, is available for purification. Such an assumption poses practical challenges, especially in scenarios where data is scarce. To tackle this challenge, efforts have been made to reduce the size of the required auxiliary dataset~\cite{chai2022oneshot,li2023reconstructive, Zhu_2023_ICCV} and even explore dataset-free purification techniques~\cite{zheng2022data,hong2023revisiting,lin2024fusing}. Although these approaches offer some improvements, recent evaluations \cite{dunnett2024countering, wu2024backdoorbench} continue to highlight the importance of sufficient auxiliary data for achieving robust defenses against backdoor attacks.

While significant progress has been made in reducing the size of auxiliary datasets, an equally critical yet underexplored question remains: \emph{how does the nature of the auxiliary dataset affect purification effectiveness?} In  real-world  applications, auxiliary datasets can vary widely, encompassing in-distribution data, synthetic data, or external data from different sources. Understanding how each type of auxiliary dataset influences the purification effectiveness is vital for selecting or constructing the most suitable auxiliary dataset and the corresponding technique. For instance, when multiple datasets are available, understanding how different datasets contribute to purification can guide defenders in selecting or crafting the most appropriate dataset. Conversely, when only limited auxiliary data is accessible, knowing which purification technique works best under those constraints is critical. Therefore, there is an urgent need for a thorough investigation into the impact of auxiliary datasets on purification effectiveness to guide defenders in  enhancing the security of ML systems. 

In this paper, we systematically investigate the critical role of auxiliary datasets in backdoor purification, aiming to bridge the gap between idealized and practical purification scenarios.  Specifically, we first construct a diverse set of auxiliary datasets to emulate real-world conditions, as summarized in Table~\ref{overall}. These datasets include in-distribution data, synthetic data, and external data from other sources. Through an evaluation of SOTA backdoor purification methods across these datasets, we uncover several critical insights: \textbf{1)} In-distribution datasets, particularly those carefully filtered from the original training data of the victim model, effectively preserve the model’s utility for its intended tasks but may fall short in eliminating backdoors. \textbf{2)} Incorporating OOD datasets can help the model forget backdoors but also bring the risk of forgetting critical learned knowledge, significantly degrading its overall performance. Building on these findings, we propose Guided Input Calibration (GIC), a novel technique that enhances backdoor purification by adaptively transforming auxiliary data to better align with the victim model’s learned representations. By leveraging the victim model itself to guide this transformation, GIC optimizes the purification process, striking a balance between preserving model utility and mitigating backdoor threats. Extensive experiments demonstrate that GIC significantly improves the effectiveness of backdoor purification across diverse auxiliary datasets, providing a practical and robust defense solution.

Our main contributions are threefold:
\textbf{1) Impact analysis of auxiliary datasets:} We take the \textbf{first step}  in systematically investigating how different types of auxiliary datasets influence backdoor purification effectiveness. Our findings provide novel insights and serve as a foundation for future research on optimizing dataset selection and construction for enhanced backdoor defense.
%
\textbf{2) Compilation and evaluation of diverse auxiliary datasets:}  We have compiled and rigorously evaluated a diverse set of auxiliary datasets using SOTA purification methods, making our datasets and code publicly available to facilitate and support future research on practical backdoor defense strategies.
%
\textbf{3) Introduction of GIC:} We introduce GIC, the \textbf{first} dedicated solution designed to align auxiliary datasets with the model’s learned representations, significantly enhancing backdoor mitigation across various dataset types. Our approach sets a new benchmark for practical and effective backdoor defense.



\begin{figure*}[t]
  \centering
    \includegraphics[width=1\linewidth]{visuals/final_registration.png}
    \caption{Target measurement process on low-cost scan data using ICP and Coloured ICP. (1) Initialisation: The source point cloud (checkerboard) is misaligned with the target point cloud. (2) Initial Registration using Point-to-Plane ICP: Standard ICP leads to suboptimal registration. (3) Final Registration using Coloured ICP: Colour information is incorporated after pre-processing with RANSAC and Binarisation with Otsu Thresholding for real data, resulting in improved alignment.}
    \label{fig:Registration_visualisation}
\end{figure*}

\subsection{Iterative Closest Point (ICP) Algorithm}
The Iterative Closest Point (ICP) algorithm has been a fundamental technique in 3D computer vision and robotics for point cloud. Originally proposed by \cite{besl_method_1992}, ICP aims to minimise the distance between two datasets, typically referred to as the source and the target. The algorithm operates in an iterative manner, identifying correspondences by matching each source point with its nearest target point \citep{survey_ICP}. It then computes the rigid transformation, usually involving both rotation and translation, to achieve the best alignment of these matched points \citep{survey_ICP}. This process is repeated until convergence, where the change in the alignment parameters or the overall alignment error becomes smaller than a predefined threshold.

One key advantage of the ICP framework lies in its simplicity: the algorithm is conceptually straightforward, and its basic version is relatively easy to implement. However, traditional ICP can be sensitive to local minima, often requiring a good initial alignment \citep{zhang2021fast}. Furthermore, outliers, noise, and partial overlaps between datasets can significantly degrade its performance \citep{zhang2021fast, bouaziz2013sparse}. Over the years, various modifications and improvements \citep{gelfand2005robust, rusu2009fast, aiger20084, gruen2005least, fitzgibbon2003robust} have been proposed to mitigate these issues. Among the most common strategies are robust cost functions \citep{fitzgibbon2003robust}, weighting schemes for correspondences \citep{rusu2009fast}, and more sophisticated techniques \citep{gelfand2005robust, bouaziz2013sparse} to reject outliers. 

In addition, there is significant interest in integrating additional information into the ICP pipeline. Instead of solely relying on geometric cues such as point coordinates or surface normals, recent approaches have proposed incorporating colour (RGB) or intensity data to enhance correspondence accuracy. These methods \citep{park_colored_2017, 5980407}, commonly known as "Colored ICP" employ differences in pixel intensities or colour values as additional constraints. This is particularly beneficial in situations where geometric attributes alone are inadequate for accurate alignment or where surfaces possess complex texture patterns that can assist in the matching process.

\subsection{Applications of Target Measurement}

One approach relies on the use of physical checkerboard targets for registration. \cite{fryskowska2019} analyse checkerboard target identification for terrestrial laser scanning. They propose a geometric method to determine the target centre with higher precision, demonstrating that their approach can reduce errors by up to 6 mm compared to conventional automatic methods.

\cite{becerik2011assessment} examines data acquisition errors in 3D laser scanning for construction by evaluating how different target types (paper, paddle, and sphere) and layouts impact registration accuracy in both indoor and outdoor environments and presents guidelines for optimal target configuration.

\citet{Liang2024} propose to use Coloured ICP to measure target centres for checkerboard targets, similar to our investigation. They use data from a survey-grade terrestrial laser scanner. Their intended application is structural bridge monitoring purposes. They report an average accuracy of the measurement below 1.3 millimetres.

Where targets cannot be placed in the scene, the intensity information form the scanner can still be used to identify distinctive points. For point cloud data that is captured with a regular pattern, standard image processing can be used in a similar way to target detection. For example, \citet{wendt_automation_2004} proposes to use the SUSAN operator on a co-registered image from a camera, \citet{bohm_automatic_2007} proposes to use the SIFT operator on the LIDAR reflectance directly and \citet{theiler_markerless_2013} propose to use a Difference-of-Gaussian approach on the reflectance information.
Most of these methods extract image features to find reliable 3D correspondences for the purpose of registration.

In the following we describe our approach to the measurement of the target centre. In contrast to most proposed methods above we focus on unordered point clouds, where raster-based methods are not available, and low-cost sensors, where increased measurement noise and outliers are expected. As we are not aware of a commercial reference solution to this problem, we start by conducting a series of synthetic experiments to explore the viability and accuracy potential of the approach.



%The reviewed studies primarily rely on physical targets or target-free methods and do not utilise 3D synthetic point cloud checkerboards. In contrast, our approach introduces synthetic point cloud checkerboards, which offer controlled and consistent target geometry and reduce variability caused by physical targets. This innovation has significant potential for commercialisation and industrial application.

\section{Study Design}
% robot: aliengo 
% We used the Unitree AlienGo quadruped robot. 
% See Appendix 1 in AlienGo Software Guide PDF
% Weight = 25kg, size (L,W,H) = (0.55, 0.35, 06) m when standing, (0.55, 0.35, 0.31) m when walking
% Handle is 0.4 m or 0.5 m. I'll need to check it to see which type it is.
We gathered input from primary stakeholders of the robot dog guide, divided into three subgroups: BVI individuals who have owned a dog guide, BVI individuals who were not dog guide owners, and sighted individuals with generally low degrees of familiarity with dog guides. While the main focus of this study was on the BVI participants, we elected to include survey responses from sighted participants given the importance of social acceptance of the robot by the general public, which could reflect upon the BVI users themselves and affect their interactions with the general population \cite{kayukawa2022perceive}. 

The need-finding processes consisted of two stages. During Stage 1, we conducted in-depth interviews with BVI participants, querying their experiences in using conventional assistive technologies and dog guides. During Stage 2, a large-scale survey was distributed to both BVI and sighted participants. 

This study was approved by the University’s Institutional Review Board (IRB), and all processes were conducted after obtaining the participants' consent.

\subsection{Stage 1: Interviews}
We recruited nine BVI participants (\textbf{Table}~\ref{tab:bvi-info}) for in-depth interviews, which lasted 45-90 minutes for current or former dog guide owners (DO) and 30-60 minutes for participants without dog guides (NDO). Group DO consisted of five participants, while Group NDO consisted of four participants.
% The interview participants were divided into two groups. Group DO (Dog guide Owner) consisted of five participants who were current or former dog guide owners and Group NDO (Non Dog guide Owner) consisted of three participants who were not dog guide owners. 
All participants were familiar with using white canes as a mobility aid. 

We recruited participants in both groups, DO and NDO, to gather data from those with substantial experience with dog guides, offering potentially more practical insights, and from those without prior experience, providing a perspective that may be less constrained and more open to novel approaches. 

We asked about the participants' overall impressions of a robot dog guide, expectations regarding its potential benefits and challenges compared to a conventional dog guide, their desired methods of giving commands and communicating with the robot dog guide, essential functionalities that the robot dog guide should offer, and their preferences for various aspects of the robot dog guide's form factors. 
For Group DO, we also included questions that asked about the participants' experiences with conventional dog guides. 

% We obtained permission to record the conversations for our records while simultaneously taking notes during the interviews. The interviews lasted 30-60 minutes for NDO participants and 45-90 minutes for DO participants. 

\subsection{Stage 2: Large-Scale Surveys} 
After gathering sufficient initial results from the interviews, we created an online survey for distributing to a larger pool of participants. The survey platform used was Qualtrics. 

\subsubsection{Survey Participants}
The survey had 100 participants divided into two primary groups. Group BVI consisted of 42 blind or visually impaired participants, and Group ST consisted of 58 sighted participants. \textbf{Table}~\ref{tab:survey-demographics} shows the demographic information of the survey participants. 

\subsubsection{Question Differentiation} 
Based on their responses to initial qualifying questions, survey participants were sorted into three subgroups: DO, NDO, and ST. Each participant was assigned one of three different versions of the survey. The surveys for BVI participants mirrored the interview categories (overall impressions, communication methods, functionalities, and form factors), but with a more quantitative approach rather than the open-ended questions used in interviews. The DO version included additional questions pertaining to their prior experience with dog guides. The ST version revolved around the participants' prior interactions with and feelings toward dog guides and dogs in general, their thoughts on a robot dog guide, and broad opinions on the aesthetic component of the robot's design. 

\section{Experiments}
\label{sec:exp}
\subsection{Experimental settings}

\noindent\textbf{Benchmark.}  We conduct experiments on two established 3D occupancy benchmarks: (i) nuScenes~\cite{nuScenes}, which provides instance-level annotations with manually labeled 3D bounding boxes (position/size/orientation) for dynamic objects, and (ii) Occ3D~\cite{Occ3D}, which generates voxel-level occupancy labels (0.4m resolution) through automated LiDAR point cloud aggregation and mesh reconstruction, including occlusion states. Both benchmarks share identical scene configurations of 1,050 driving scenes, each containing up to 40 timestamped frames. Every frame includes six synchronized camera views (front, front-left, front-right, back, back-left, back-right) at 1600$\times$900 resolution. In our experiments, we extend single-frame baselines~\cite{MonoScene,surroundOcc,viewformer} by aggregating features from $N$ historical keyframes. Additionally, we extract unlabeled intermediate frames from the ``sweeps'' folder~\cite{nuScenes} to provide implicit motion cues, enabling self-supervised temporal consistency learning.

\noindent\textbf{Implementation details.} For the nuScenes benchmark~\cite{nuScenes}, we follow the parameter settings of SurroundOcc~\cite{surroundOcc}, using $Cam=6$, $p=6$, $v=32$,$L=116$, and $W=200$. For the Occ3D benchmark~\cite{Occ3D}, we adopt ViewFormer's~\cite{viewformer} standard setup with $Cam=6$, $p=6$, $v=32$, $L=32$, and $W=88$. The output of the occupancy result on both benchmarks is formatted into a vector with dimensions $[200, 200, 16]$. In this vector, the first two dimensions (200 and 200) represent the length and width, while the third (16) indicates the height. The occupancy result covers a range from -50 meters to 50 meters in both width and length, and the vertical height varies from -5 meters to 3 meters. Each voxel corresponds to a cube measuring 0.5 meters on each side. Occupied voxels are categorized into one of 17~\cite{nuScenes,surroundOcc} and 18~\cite{Occ3D} semantic classes.
More details on implementation can be found in the supplementary material.

\subsection{Evaluation Metrics}

To validate the temporal consistency and occupancy accuracy of moving and static objects, objects are divided into two general classes~\cite{Cam4docc}: General Moving Objects (GMO) and General Static Objects (GSO). Detailed classification classes are introduced in the supplementary material.

\noindent\textbf{Occupancy Accuracy Metric.} To ensure rigorous evaluation across different benchmarks, we employ both Intersection over Union (IoU) and Mean Intersection over Union (mIoU) metrics. These metrics are widely adopted in 3D semantic occupancy prediction tasks~\cite{PASCAL, Microsoft_COCO, Cityscapes_dataset, Mask_R_CNN}. The mIoU are calculated separately for three category groups: All classes, GMO classes, and GSO classes.

\noindent\textbf{Temporal Consistency Metric.} \label{para:consistency_metric} To evaluate the effect achieved by integrating \ours\ with baseline models, we propose a temporal consistency metric. We aim to detect and measure changes in a scene from one frame to the next. This metric reflects the stability of prediction results, which directly impacts the user's visual experience. Let $\sigma_{i,n}^{(x,y,z)}$ denote the semantic label of the $n$-th voxel point (with coordinates $(x,y,z)$) in frame $i$, and define the indicator function $\delta(e_1,e_2) = \mathbb{I}(e_1 \neq e_2)$. 

In the occupancy results of frames $i$ and $j$, voxels at corresponding positions may undergo changes, which are categorized into two types: ``Static Object Change"~(SOC) and ``Moving Object Change"~(MOC). The definitions of these changes are as table \ref{tab:moc-soc}.

\begin{wrapfigure}[7]{l}{80mm}
\centering
% \setlength{\tabcolsep}{8pt}
\captionsetup{type=table}
    \begin{tabular}{c|c}
    \toprule
    \textbf{Type} & \textbf{Condition} \\  
    \midrule
    MOC & $\sigma_{i,n}^{(x,y,z)} \in \text{GMO} \lor \sigma_{j,n}^{(x,y,z)} \in \text{GMO}$ \\
    % \midrule
    SOC & $\sigma_{i,n}^{(x,y,z)} \wedge \sigma_{j,n}^{(x,y,z)} \in \text{GSO}$ \\  
    \bottomrule
    \end{tabular}
    \vspace{8pt}
    \caption{Definition of MOC and SOC. $N_{mc}$/$N_{sc}$ denote the number of MOC/SOC voxels, respectively.}
    \label{tab:moc-soc}
    % \vspace{-10pt}
\end{wrapfigure}

Based on these definitions, we can define disparity metrics~($\Delta_{m}$/$\Delta_{s}$) to quantify temporal inconsistencies across frames~($i$ and $j$). The process is defined as:
\begin{equation}
\begin{dcases}
\Delta_{m}(i,j) = \dfrac{1}{N_{mc}} \sum\limits_{n=1}^{N_{mc}} \delta\left(\sigma_{i,n}^{(x,y,z)}, \sigma_{j,n}^{(x,y,z)}\right) \\
\Delta_{s}(i,j) = \dfrac{1}{N_{sc}} \sum\limits_{n=1}^{N_{sc}} \delta\left(\sigma_{i,n}^{(x,y,z)}, \sigma_{j,n}^{(x,y,z)}\right).
\end{dcases}
\label{eq:disparity}
\end{equation}

The temporal consistency metrics -- $S_m$ (moving) and $S_s$ (static) -- are derived through aggregation of $\Delta_{m}$ and $\Delta_{s}$ across sequential frames. Formally, we have:
\begin{equation}
S_{m/s} = 1 - \dfrac{1}{M-1} \sum\limits_{k=1}^{M-1} \Delta_{m/s}(k,k+1),
\label{eq:consistency_scores}
\end{equation}
where $M$ is the scene's total frame count. Final metrics $\overline{S_m}$/$\overline{S_s}$ average across all scenes. A higher temporal consistency score indicates that the predictions within the scene are smoother and more consistent over time.

\begin{table}[t]
    % \centering
    \footnotesize 
    \begin{minipage}[t]{0.48\textwidth}
        \centering
        % \captionsetup{justification=centering, singlelinecheck=false}
        \setlength{\tabcolsep}{2pt}
        \renewcommand{\arraystretch}{1.25}
        \begin{tabular}{r|cccc|cc}
            \toprule
            \multicolumn{1}{r|}{\multirow{2}{*}[-0.4em]{Method}} & \multicolumn{1}{c|}{\multirow{2}{*}[-0.4em]{IoU~$\uparrow$}} & \multicolumn{3}{c|}{mIoU~$\uparrow$} & \multicolumn{1}{c}{\multirow{2}{*}[-0.4em]{$\overline{S_m}\uparrow$}} & \multicolumn{1}{c}{\multirow{2}{*}[-0.4em]{$\overline{S_s}\uparrow$}} \\ \cmidrule(lr){3-5}
            \multicolumn{1}{c|}{} & \multicolumn{1}{c|}{} & All & GMO & GSO & \multicolumn{1}{c}{} & \multicolumn{1}{c}{} \\ 
            \midrule        
            Atlas~\cite{Atlas} & 28.66 & 15.00 & 12.64 & 17.35 & \text{--} & \text{--}  \\
            BEVFormer{~\cite{BEVFormer}} & 30.50 & 16.75 & 14.17 & 19.33 & \text{--} & \text{--}  \\
            TPVFormer~\cite{TPVFormer} & 30.86 & 17.10 & 14.04 & 20.15 & \text{--} & \text{--} \\
            BEVDet4D-Occ~\cite{bevdet4d} & 24.26 & 14.22 & 11.10 & 17.34 & \text{--} & \text{--} \\
            MonoScene~\cite{MonoScene} & 10.04 & 1.15 & 0.24 & 2.07 & 46.53 & 81.77 \\
            % Cam4DOcc~\cite{Cam4docc} & 23.92 & 7.12 & 4.71 & 10.17 & 60.34 & 91.15 \\
            SurroundOcc~\cite{surroundOcc} & 31.49 & 20.30  & \cellcolor{gray!20}18.39 & 22.20 & 58.33 & 91.71 \\
            \midrule
            \makecell[r]{MonoScene \\ \textbf{+\ours}} & \makecell{13.10\\\textbf{+3.06}} & \makecell{1.69\\\textbf{+0.54}}  & \makecell{0.34\\\textbf{+0.10}} & \makecell{3.04\\\textbf{+0.98}} & \makecell{54.21\\\textbf{+7.68}} & \makecell{83.84\\\textbf{+2.07}} \\
            \midrule
            \makecell[r]{SurroundOcc \\ \textbf{+\ours}} & \cellcolor{gray!20}\makecell{33.12 \\ \textbf{+1.63}} & \cellcolor{gray!20}\makecell{20.67\\\textbf{+0.37}}  & \makecell{18.26\\-0.13} & \cellcolor{gray!20}\makecell{23.08\\\textbf{+0.88}} & \cellcolor{gray!20}\makecell{60.64\\\textbf{+2.31}} & \cellcolor{gray!20}\makecell{92.54\\\textbf{+0.83}} \\
            \bottomrule
        \end{tabular}
        \vspace{2mm}
        \caption{Occupancy prediction accuracy on \textbf{nuScenes benchmark~\cite{nuScenes}}. For a fair comparison, we ensure that all models have uniform input data. The best performance is highlighted in gray.}
        \label{tab:main-res-a}
    \end{minipage}\hfill
    \begin{minipage}[t]{0.48\textwidth}
        \centering
        % \captionsetup{justification=centering, singlelinecheck=false}
        \setlength{\tabcolsep}{2pt}
        \begin{tabular}{r|cccc|cc}
            \toprule
            \multicolumn{1}{r|}{\multirow{2}{*}[-0.4em]{Method}} & \multicolumn{1}{c|}{\multirow{2}{*}[-0.4em]{IoU~$\uparrow$}} & \multicolumn{3}{c|}{mIoU~$\uparrow$} & \multicolumn{1}{c}{\multirow{2}{*}[-0.4em]{$\overline{S_m}\uparrow$}} & \multicolumn{1}{c}{\multirow{2}{*}[-0.4em]{$\overline{S_s}\uparrow$}} \\ \cmidrule(lr){3-5}
            \multicolumn{1}{c|}{} & \multicolumn{1}{c|}{} & All & GMO & GSO & \multicolumn{1}{c}{} & \multicolumn{1}{c}{} \\ 
            \midrule    
            MonoScene~\cite{MonoScene} & \text{--} & 6.06 & 5.36 & 6.68 & \text{--} & \text{--} \\
            OccFormer~\cite{OccFormer} & \text{--} & 21.93 & 21.78 & 22.06 & \text{--} & \text{--} \\
            % CTF-Occ~\cite{Occ3D} & 28.53 & 27.42 & 29.52 & \text{--} & \text{--} \\
            FB-OCC~\cite{fb_occ} & \text{--} & 39.11  & 33.74 & 43.88 & \text{--} & \text{--} \\
            SparseOcc~\cite{SparseOcc_Liu} & \text{--} & 30.10  & \text{--} & \text{--} & \text{--} & \text{--} \\
            BEVDet4D-Occ~\cite{bevdet4d} & \text{--} & 39.30  & 29.09 & 42.16 & \text{--} & \text{--} \\ 
            OPUS-L~\cite{opus} & \text{--} & 36.20  & 31.25 & 40.44 & \text{--} & \text{--} \\      
            SurroundOcc~\cite{surroundOcc} & 51.89 & 7.24  & 0.36 & 13.35 & 65.35 & 89.54 \\
            ViewFormer~\cite{viewformer} & 70.39 & 40.46  & 33.73 & 46.45 & 67.26 & 86.06 \\
            \midrule
            \makecell[r]{SurroundOcc \\ \textbf{+\ours}} & \makecell{52.13\\\textbf{+0.24}}& \makecell{10.33\\\textbf{+3.09}}  & \makecell{1.98\\\textbf{+1.62}} & \makecell{17.76\\\textbf{+4.41}} & \makecell{69.60\\\textbf{+4.25}} & \cellcolor{gray!20}\makecell{90.91\\\textbf{+1.37}} \\
            \midrule
            \makecell[r]{ViewFormer \\ \textbf{+\ours}} & \cellcolor{gray!20}\makecell{70.63\\\textbf{+0.24}} & \cellcolor{gray!20}\makecell{41.30\\\textbf{+0.84}}  & \cellcolor{gray!20}\makecell{34.33\\\textbf{+0.60}} & \cellcolor{gray!20}\makecell{47.50\\\textbf{+1.05}} & \cellcolor{gray!20}\makecell{70.13\\\textbf{+2.87}} & \makecell{87.10\\\textbf{+1.04}} \\
            \bottomrule
        \end{tabular}
        \vspace{2mm}
        \caption{Occupancy prediction accuracy on \textbf{Occ3D benchmark~\cite{Occ3D}}. For a fair comparison, we ensure that all models have uniform input data. The best performance is highlighted in gray.}
        \label{tab:main-res-b}
    \end{minipage}
    \vspace{-3mm}
    % \caption{Occupancy prediction accuracy on two benchmarks. The best performance is highlighted in gray.}
    \label{tab:main-res}
    \vspace{-5mm}
\end{table}

\subsection{Comparison Results}

\noindent\textbf{Occupancy accuracy on nuScenes.} We compare our method against several SOTA models, including Atlas~\cite{Atlas}, BEVFormer~\cite{BEVFormer}, TPVFormer~\cite{TPVFormer}, MonoScene~\cite{MonoScene}, and SurroundOcc~\cite{surroundOcc}. For a fair comparison, all methods are trained on the same ground truth and follow the same training procedure. By combining methods such as MonoScene~\cite{MonoScene} and SurroundOcc~\cite{surroundOcc} with \ours, we evaluate the effect of \ours\ in performance enhancement. The results presented in \cref{tab:main-res-a} show that our performance improvement is significant. Notably, the incorporation of \ours\ into SurroundOcc~\cite{surroundOcc} has led to improved metrics that surpass those of all other models listed in this table. The results are improved by 1.63\% and 0.37\% compared with SurroundOcc~\cite{surroundOcc} in IoU and mIoU~(All), respectively.

\noindent\textbf{Occupancy accuracy on Occ3D.} We also conduct experiments on Occ3D~\cite{Occ3D} in \cref{tab:main-res-b}. To validate \ours, we conducted two sets of experiments: First, integrating \ours\ with the 3D VONs~\cite{surroundOcc,viewformer} improved one of the original models'~\cite{viewformer} performance by 0.24\% in IoU and 0.84\% in mIoU. Second, \ours\ consistently outperforms existing history-aware VONs~\cite{opus,bevdet,SparseOcc_Liu,fb_occ} by over 2\% mIoU, demonstrating the efficacy of the \ours.


\noindent\textbf{Temporal Consistency.} The results of $\overline{S_m}$ and $\overline{S_s}$ shown in \cref{tab:main-res} indicate that the integration of \ours\ improved the temporal consistency of occupancy across all frames in all scenes for all models, demonstrating \ours' effectiveness. This enhancement can be attributed to the incorporation of previous keyframes from the dataset~\cite{nuScenes,Occ3D}, along with the addition of intermediate frames from the ``sweeps''~\cite{nuScenes} directory for the SFE and MFE modules. These elements provide critical historical information and motion clues for the model.

\subsection{Ablation study}

Our ablation experiments are all conducted on the nuScenes benchmark~\cite{nuScenes}. The results are presented in~\cref{tab:ablation-studies}.

\begin{wrapfigure}{l}{90mm}
\centering
\captionsetup{type=table}
    \begin{subtable}[t]{0.48\textwidth}
    \centering
    \footnotesize
    \begin{tabular}{c|ccc|cccc}
    \toprule 
    Idx. & Pre & Cur & Mid & IoU$\uparrow$ & mIoU$\uparrow$ & $\overline{S_m}\uparrow$ & $\overline{S_s}\uparrow$ \\
    \midrule
    \textbf{M0} & \ding{55} & \ding{55} & \ding{55} & 31.49 & 20.30 & 58.33 & 91.71 \\
    \textbf{M1} & \ding{55} & \ding{51} & \ding{51} & 33.04 & 20.04 & 60.59 & 92.25 \\
    \textbf{M2} & \ding{51} & \ding{55} & \ding{51} & 33.05 & 19.98 & 60.09 & 92.44 \\
    \textbf{M3} & \ding{51} & \ding{51} & \ding{55} & 32.88 & 20.10 & 60.24 & 92.24 \\
    \textbf{M4} & \ding{51} & \ding{51} & \ding{51} & 31.97 & 20.11 & 60.19 & 92.01 \\
    \textbf{M5} & \ding{51} & \ding{51} & \ding{51} & \textbf{33.12} & \textbf{20.67} & \textbf{60.64} & \textbf{92.54} \\
    \bottomrule
    \end{tabular}
    \vspace{1mm}
    \caption{Ablation study of \ours. \textbf{Cur}, \textbf{Pre} and \textbf{Mid} represent the $f_{cur}$, $f_{pre}$ and $f_{mid}$ input, respectively, in the MSI.}
    \label{tab:ablation-modules}
    \end{subtable}
% \hfill
    \vspace{2mm}
    
    \begin{subtable}[t]{0.48\textwidth}
    \centering
    \footnotesize
    \setlength{\tabcolsep}{7pt}
    \begin{tabular}{c|c|cccc}
    \toprule 
    Idx. & \makecell{Type of\\motion info.} & IoU~$\uparrow$ & mIoU~$\uparrow$ & $\overline{S_m}\uparrow$ & $\overline{S_s}\uparrow$ \\
    \midrule
    \textbf{I0} & -  & 31.49 & 20.30 & 58.33 & 91.71 \\
    \textbf{I1} & Raw Image  & 32.39 & 19.45 & 59.01 & 91.15 \\
    \textbf{I2} & Optical Flow  & 32.80 & 20.27 & 60.53 & 92.13 \\
    \textbf{I3} & Frame Diff.  & \textbf{33.12} & \textbf{20.67}  & \textbf{60.64} & \textbf{92.54} \\
    \bottomrule
    \end{tabular}
    \vspace{1mm}
    \caption{Effect of different types of motion information.}
    \label{tab:ablation-motion}
    \end{subtable}
    % \vspace{-2mm}
\caption{Ablation studies on \ours\ modules and motion information. Best results are \textbf{bolded}.}
\label{tab:ablation-studies}
\end{wrapfigure}

\noindent\textbf{Different combinations of \ours.} \label{para:aba-comb}\cref{tab:ablation-modules} presents the performance results of different combination of \ours's components for $N$=$1$. In \cref{tab:ablation-modules}, there are 6 different combinations: \textbf{M0} shows results from SurroundOcc~\cite{surroundOcc}, which represents the basic model without our method. \textbf{M1} means the model variant in which the part responsible for processing previous keyframes is removed, thereby excluding the input data \(F_{pre}^{1}\). \textbf{M2} refers to the model variant that omits the current feature \(F_{cur}\). \textbf{M3} indicates the model configuration that has \(F_{mid}^{1}\) removed. \textbf{M4} indicates that $I_{cur}$ is used to compute MHAM's query, while $I_{pre}$ and $I_{mid}$ are utilized to compute MHAM's key and value, which differs from the standard design. \textbf{M5} represents the full model with all components included.
% 验证了我们三种输入的必要性
\cref{tab:ablation-modules} clearly demonstrates that the removal of any single input from \ours\ module significantly reduces performance both in prediction accuracy and in temporal consistency. This validates the necessity of the three inputs. Furthermore, the comparison between \textbf{M4} and \textbf{M5} confirms that the cues provided by the previous keyframes and the intermediate frames are crucial for occupancy prediction.

\noindent\textbf{Impact of different types of motion information.} \label{para:motion-extracting} This experiment was conducted on the MFE module to investigate the effects of various types of motion information for $N=1$. The results are presented in \cref{tab:ablation-motion}. Specifically, \textbf{I0} served as the base model~\cite{surroundOcc} without using any motion information. \textbf{I1} employed raw intermediate frames as the input for the MFE. \textbf{I2} used optical flow~\cite{OpticalFlow} as the motion information input. \textbf{I3} used frame difference~\cite{Frame_difference} to capture motion information. It is clear that \textbf{I1} surpasses \textbf{I0} in terms of IoU metrics; however, it exhibits the lowest performance in mIoU, $\overline{S_m}$, and $\overline{S_s}$ metrics compared with \textbf{I1}, \textbf{I2}, and \textbf{I3}. This discrepancy is mainly because of the substantial amount of irrelevant information in the raw, intermediate frames, which complicates the extraction of motion features by the MFE. In addition, the results show that \textbf{I3} significantly outperforms \textbf{I2} in both IoU and mIoU metrics and slightly improves in $\overline{S_m}$ and $\overline{S_s}$ metrics. This indicates that frame difference more effectively captures sudden changes in a scene, such as the abrupt appearance of pedestrians or vehicles exiting intersections, while optical flow may experience delays in processing these sudden events. Furthermore, given the lightweight design of \ours, the frame difference method~\cite{Frame_difference} reduces data processing complexity by only processing simple differential data, thereby contributing to computing speed.

\begin{wrapfigure}[22]{l}{90mm}
    \centering
    \includegraphics[width=90mm]{assets/case_single_light.png}
    \caption{
    Several challenging scenarios are presented: pedestrians are partially occluded by vehicles in the first and second columns, and the road boundary appears visually obscure in the third column. Ours achieves more accurate predictions, while SOTA methods display significant artifacts.
    }
    \label{fig:case-extra}
\end{wrapfigure}

\noindent\textbf{Impact of different numbers of previous keyframes.}\label{para:ntrack} We conduct ablation experiments on $N$ to explore the performance of the model when $N$=$0$, $N$=$1$ and $N$=$2$. $N$=$0$ represents SurroundOcc~\cite{surroundOcc}, which does not use any previous keyframes. Detailed experiment results are documented in the supplementary material.

\begin{figure}[!t]
\centering
% \fbox{\rule{0pt}{2in} \rule{0.9\linewidth}{0pt}}
\includegraphics[width=\linewidth]{assets/case_big_light.png}
% \vspace{-7mm}
\caption{
Comparison under a T-junction scenario, where a pedestrian is partially and dynamically occluded in certain frames. Ours showcases robust predictions, with the pedestrian being consistently tracked, while SOTA methods show a flickering phenomenon.
}
\label{fig:case-study-big}
\vspace{-20pt}
\end{figure}

\subsection{Case analysis}
To visually evaluate the effectiveness of our method~(SurroundOcc+\ours), we compare it with the SOTA 3D VONs~\cite{surroundOcc} and the SOTA history-aware VONs~\cite{bevdet4d}.

\noindent\textbf{Temporal visualization case.} As shown in~\cref{fig:case-study-big} (Scene 277, Frames \#7-\#11), a pedestrian traversing the sidewalk parallel to the ego-motion trajectory is intermittently occluded by roadside vegetation. SurroundOcc~\cite{surroundOcc} exhibits severe instability in predictions (missing in Frames \#7/\#9), revealing fundamental limitations in temporal modeling. BEVDet4D-Occ~\cite{bevdet4d} alleviates this issue through data fusion but still suffers from occasional inconsistencies, such as detection dropout in Frame \#8. In contrast, our method completely eliminates flickering artifacts and maintains consistent detection across all occlusion states.

\noindent\textbf{Extra single frame visualization case.} ~\cref{fig:case-extra} highlights challenging scenarios: 
(i) Vehicle-pedestrian occlusion (Scene-0911 Frame \#15, Scene-0928 Frame \#14): Both SurroundOcc~\cite{surroundOcc} and BEVDet4D-Occ~\cite{bevdet4d} fail to recover the occluded pedestrian’s occupancy, while our method successfully localizes the target with precise geometry.
(ii) Curved road prediction (Scene-0923 Frame \#28): Our approach correctly anticipates the right-turn road geometry where baselines produce fragmented or erroneous occupancy, achieving superior shape consistency with real-world conditions.




\subsection{Overhead analysis}

For a fair comparison, all overhead analysis experiments are performed on a single NVIDIA L20 GPU.

\begin{figure}[htbp]
    \centering
    \begin{minipage}{0.48\textwidth}
        \footnotesize
        \begin{tabular}{r|ccc}
            \toprule
            Model & mIoU~$\uparrow$ & \makecell{Memory (MB)\\ Train~/~Test}~$\downarrow$ & Latency~$\downarrow$ \\
            \midrule
            FB-Occ~\cite{fb_occ} & 39.11 & 32,915~/~5,933 & 0.09s \\
            % SparseOcc~\cite{SparseOcc_Liu} & 30.10 & $>$49,140~/~7,147 & \cellcolor{Gray} 0.05s \\
            OPUS-L~\cite{opus} & 36.20 & OOM~/~10,579 & 0.16s \\
            OPUS-T~\cite{opus} & 33.20 & 48,532~/~6,711 & \textbf{0.03s} \\
            BEVDet4D-Occ~\cite{bevdet4d} & 39.30 & 22,833~/~4,689 & 0.26s \\
            \midrule
            ViewFormer+Ours & \textbf{41.30} & \textbf{16,619~/~4,687} & 0.12s \\
            \bottomrule
        \end{tabular}
        \vspace{2mm}
        \caption{Comparison of computational overhead. All models are benchmarked with ResNet-50 backbones. Our result (ViewFormer+\ours) in this table is measured for $N = 1$. OOM indicates out of CUDA memory. Best results are \textbf{bolded}.}
        \label{tab:efficiency}
    \end{minipage}\hfill
    \begin{minipage}{0.48\textwidth}
        \centering
        \includegraphics[width=\linewidth]{assets/bubble_0304.png}  % 替换成你的图片
        % \vspace{-8mm}
        \caption{Comparison of memory and latency overheads. Lower-left positions indicate superior performance with reduced memory consumption and faster inference. Large circles indicate better mIoU quality.}
        \label{fig:bubble}
    \end{minipage}
\end{figure}



As illustrated in \cref{tab:efficiency} and \cref{fig:bubble}, we conducted a comparative study to evaluate the computational overhead of our model against existing temporal methods~\cite{bevdet4d,opus,fb_occ}. The analysis focuses on GPU memory consumption during the training/testing phases and per-sample inference latency. The result shows that our method establishes an optimal accuracy-memory balance, achieving state-of-the-art mIoU while maintaining minimal GPU memory consumption alongside sustained computational efficiency that avoids runtime bottlenecks. For quantitative benchmarking, we compare two baseline frameworks:
\begin{itemize}
    \item ViewFormer on Occ3D: (i) Training memory: ViewFormer+\ours\ requires 16 GB of GPU memory, with the \ours\ module consuming only 0.22 GB, accounting for \textbf{1.4\%} of total usage; (ii) Inference latency: Full sample processing takes 0.1218s, where \ours\ contributes merely 0.0043s, accounting for \textbf{3.5\%} of total computation.
    \item SurroundOcc on nuScenes: (i) Training memory: SurroundOcc+\ours\ consumes 39 GB of GPU memory, with \ours\ occupying only 0.69 GB, which is \textbf{1.8\%} of total memory; (ii) Inference latency: Complete sample inference requires 0.9200s, while \ours\ takes 0.0065s, contributing to \textbf{0.7\%} of total latency.
\end{itemize}

These measurements confirm that our architecture introduces negligible computational overhead while delivering competitive performance.
\section{Conclusion}
We introduce a novel approach, \algo, to reduce human feedback requirements in preference-based reinforcement learning by leveraging vision-language models. While VLMs encode rich world knowledge, their direct application as reward models is hindered by alignment issues and noisy predictions. To address this, we develop a synergistic framework where limited human feedback is used to adapt VLMs, improving their reliability in preference labeling. Further, we incorporate a selective sampling strategy to mitigate noise and prioritize informative human annotations.

Our experiments demonstrate that this method significantly improves feedback efficiency, achieving comparable or superior task performance with up to 50\% fewer human annotations. Moreover, we show that an adapted VLM can generalize across similar tasks, further reducing the need for new human feedback by 75\%. These results highlight the potential of integrating VLMs into preference-based RL, offering a scalable solution to reducing human supervision while maintaining high task success rates. 

\section*{Impact Statement}
This work advances embodied AI by significantly reducing the human feedback required for training agents. This reduction is particularly valuable in robotic applications where obtaining human demonstrations and feedback is challenging or impractical, such as assistive robotic arms for individuals with mobility impairments. By minimizing the feedback requirements, our approach enables users to more efficiently customize and teach new skills to robotic agents based on their specific needs and preferences. The broader impact of this work extends to healthcare, assistive technology, and human-robot interaction. One possible risk is that the bias from human feedback can propagate to the VLM and subsequently to the policy. This can be mitigated by personalization of agents in case of household application or standardization of feedback for industrial applications. 


% WARNING: do not forget to delete the supplementary pages from your submission 
\appendix
\section*{Appendix}

\section{More Details}

We provide more details that are not implemented in the main paper. Specifically, we provide the detailed architecture for the model, more training details about \textit{Ola}'s progressively modality alignment, and details of the data preparation procedure. 

\subsection{Model Details}

The visual encoder for \textit{Ola} is based on the SigLIP-400M~\citep{zhai2023siglip} backbone and is further fine-tuned for native-resolution visual inputs. The patch size for the visual encoder is set to 16, the hidden dimension is 1152, and the MLP hidden dimension is 4304. The SigLIP-400M model consists of 27 transformer blocks and 16 attention heads. The audio encoder for \textit{Ola} is built on the whisper-v3-large~\citep{radford2022whisper} model. The length of the input audio tensor for the whisper model is fixed at 480,000; therefore, we chunk the entire audio tensor into pieces and concatenate the audio features. The mel size for the whisper-v3-large model is set to 128, and the hidden dimension for the speech features after the whisper-v3-large model is 1280.

For the connector layer, we utilize a 2-layer MLP for feature projection. We initialize two separate MLPs for visual and audio features, respectively. The input dimension matches that of the visual or audio encoder, and the output dimension matches the LLM dimension. For the \textit{Local-Global Attention Pooling} layer, we use a predictor to calculate the score based on the concatenated features. Therefore, the dimension for the predictor is 2$\times$ to 1$\times$ dimension.

We integrate Qwen-2.5-7B~\citep{qwen2.5} into the \textit{Ola} large language model. The Qwen-2.5-7B model has a hidden LLM dimension of 3,584 and an intermediate size of 18,944. It consists of 28 transformer layers. The basic architecture for the speech decoder mirrors that of the Qwen-2.5-7B LLM, but we use only a 2-layer transformer block for the speech decoder. During generation, the maximum number of classification categories for the unit speech tensor is set to 1,000. We use a pre-trained speech vocoder to convert the unit speech tensor into speech waveforms. During inference, speech outputs are generated whenever a punctuation mark is detected, ensuring that the features of streaming outputs are preserved.

\subsection{Training Details}

As stated in the main paper, the progressive modality alignment procedure is conducted in four stages. The first image-text stage involves adapter pre-training and supervised fine-tuning. The adapter pre-training stage is conducted on 808k image captioning data collected from the LAION datasets. During pre-training, we unfreeze the parameters for the connector while keeping other parameters frozen. We set the training batch size to 256 and the overall learning rate to 1e-3. The supervised fine-tuning stage is conducted on 7.3 million image-text data pairs. During image-text training, input images are maintained at their original aspect ratio, with the maximum image size restricted to 1536×1536. We set the batch size to 128 and the overall learning rate to 2e-5. The stage 1 experiment is conducted on 64 NVIDIA A800 GPUs.

\begin{figure*}[t]
\centering
\includegraphics[width=\textwidth]{figs/fig-supp1.pdf}
\caption{\textbf{Showcases on Text and Audio Understanding.}}
\label{fig:supp1}
\end{figure*}

Stage 2 integrates both image and video data for supervised fine-tuning, following most of the training strategies from Stage 1. The total amount of training data is 2.7 million, comprising 800k image-text pairs sampled from Stage 1 and 1.9M video data collected from open-source datasets. We set the training batch size to 256 and the overall learning rate to 2e-5. The model's maximum sequence length is set to 16k. The maximum number of frames is set to 64. The Stage 2 experiment is conducted on 64 NVIDIA A800 GPUs. 

Stage 3 involves joint training in the audio domain. We conduct a projector alignment procedure similar to the image pre-training for initializing the speech adapter. During the speech adapter pre-training, we unfreeze the parameters of the speech adapter while freezing the other parameters. We set the batch size to 256 and the overall learning rate to 1e-3. The pre-training phase is conducted on the 370k LibriTTS~\citep{zen2019libritts} dataset. After pre-training, we integrate audio-video joint alignment by combining image data, video data, and pure audio data. We use 600k image data, 1.1M audio data, and 243k video-audio training data. The training batch size is set to 128, and the overall learning rate is set to 1e-5. We maintain the original audio data for inputs in the audio and video data and append the necessary prompts for instructions. Specifically, for the ASR tasks, we set the ASR prompt as \textit{"Please give the ASR results of the given speech."} For the audio instruction tasks, we set the instruction-following prompt as \textit{"Please directly answer the questions in the user's speech."} We maintain the maximum frame number for the video and set the maximum speech chunk number to 20. The Stage 3 experiment is conducted on 64 NVIDIA A800 GPUs.

\subsection{Data Collection Details}


\begin{figure*}[t]
\centering
\includegraphics[width=\textwidth]{figs/fig-supp2.pdf}
\caption{\textbf{Showcases on Video Understanding.}}
\label{fig:supp2}
\end{figure*}

We provide details on collecting video-audio relevant data. Our data comes from two sources with high-quality raw videos: LLaVA-Video-178k~\citep{zhang2024llavanextvideo}, which contains 178k raw videos, and FineVideo~\citep{Farré2024FineVideo}, which contains 42k raw videos. For the open-sourced video data from LLaVA-Video-178k, we first use the Whisper~\citep{radford2022whisper} model to generate subtitles. We find that the videos include content in other languages and videos without valid audio, so we design a filtering method for better results. Specifically, we first assess the ratio of English words in the generated subtitles and discard those with a lower ratio, indicating subtitles in other languages. We also discard extremely short subtitles. Then, we use a large language model, Qwen-2.5-72B, to further filter the subtitles. The model is asked to identify meaningless sentences with the following prompt: \textit{"I will give you a subtitle generated from a video. Identify whether the subtitle is complete, fluent, and informative. Answer directly with yes or no and do not add other explanations."} After this procedure, we gathered 41k valid videos. For the videos in FineVideo, as they are already well-processed, we directly use the subtitles for the following steps. We utilize Qwen-2-72B to generate audio-relevant question-answer pairs based on the given videos and subtitles. The prompt for the vision-language model is: \textit{"Please generate at least three questions and answers based on the information in the subtitle. You can refer to the video for additional context. The questions and answers must be highly relevant to the subtitle and video and should not include fabricated content."} We then generate 243k cross-modal video-audio data points from the 81k collected videos. This data is used for stage 3 training for omni-modal alignment.

\section{More Showcases}

\subsection{Text and Audio Understanding}

In this subsection, we provide more practical text-audio understanding samples for visualizations. The inputs of the text-audio understanding are a mixture of audio and text instructions, which can strongly test the cross-modal capability for \textit{Ola} model. Results are shown in Fig.~\ref{fig:supp1}, we provide results on music-related, speech-related, and sound-related audio inputs and \textit{Ola} excels at all the circumstances with a strong performance on mixed audio and text understanding. 

\subsection{Video Understanding}

In this subsection, we provide more results on video understanding tasks and provide comparisons with state-of-the-art vision LLM. Results are shown in Fig.~\ref{fig:supp2}. With the capability to recognize video, audio, and text jointly, \textit{Ola} can gather more information from the video. 


{
    \small
    \bibliographystyle{ieeenat_fullname}
    \bibliography{main}
}

\end{document}
