\documentclass[conference, review]{IEEEtran}
\usepackage{amsmath}
%qed at the end of proofs (for LNCS)
%\let\proof\relax\let\endproof\relax
\usepackage{amsthm}

%\usepackage{txfonts}  %英文をTimes Romanのようなフォントにする.
%\usepackage[top=30truemm,bottom=30truemm,left=25truemm,right=25truemm]{geometry} %余白の設定
%\usepackage{algpseudocode,algorithm}
%hyperref
%    \usepackage[unicode=true,psdextra]{hyperref}
%bm command
    \usepackage{bm}
%itembox
    \usepackage{ascmac}
%for mathbb
%    \usepackage{amssymb}
%    \usepackage{stmaryrd}
% default mathcal
    \DeclareMathAlphabet{\mathcal}{OMS}{cmsy}{m}{n}
%Tables
    \usepackage{booktabs}
    \usepackage{array}
        \setlength{\extrarowheight}{-1pt}
    \newcolumntype{C}[1]{>{\centering\let\newline\\\arraybackslash\hspace{0pt}}m{#1}}
    \usepackage{float}
    \usepackage{multicol}
    \usepackage{multirow} % Required for multirows
%layout sizes
    \usepackage{layout}
%colors
    \usepackage{xcolor}
    \usepackage{colortbl}
%Description options
%    \usepackage{enumitem}
% appendix
    \usepackage[title,titletoc]{appendix}
% maxsizebox
    \usepackage{adjustbox}
%line number for revision
    \usepackage{lineno}
%arrangement: spaces, etc
%    \usepackage[final]{microtype}
%for cites
    \usepackage{cite}
% Complexity terms
    \usepackage[full]{complexity}
%algorithm
    \usepackage[linesnumbered,lined,ruled,noend,vlined]{algorithm2e}
    \makeatletter
    \@ifpackageloaded{algorithm2e}{
    \SetKwProg{Fn}{Function}{}{}
    \SetKwProg{Procedure}{Procedure}{}{}
    \SetKwProg{Subprocedure}{Subprocedure}{}{}
    \SetKwComment{tcc}{//}{}
    \SetKwFunction{Output}{Output}%
    \SetKw{Continue}{continue} 
    \SetKwInOut{AlgInput}{Input}
    \SetKwInOut{AlgOutput}{Output}
    \SetKwInOut{AlgPrecondition}{Pre-conditions}
    \SetKwInOut{AlgInvariant}{Invariants}
    }{}
    \makeatother
%cref
%    \usepackage[capitalise]{cleveref}
    \crefname{problem}{Problem}{Problems}
    \crefname{itembox}{Problem}{Problems}
    \crefname{algocf}{algorithm}{algorithms}
    \Crefname{algocf}{Algorithm}{Algorithms}
    %\crefalias{AlgoLine}{line}
    \crefname{AlgoLine}{line}{lines}
    \Crefname{AlgoLine}{Line}{Lines}
    \crefname{observation}{observation}{observations}
    \Crefname{observation}{Observation}{Observations}
    \crefname{mtheorem}{theorem}{theorems}
    \Crefname{mtheorem}{Theorem}{Theorems}
    \crefname{mlemma}{lemma}{lemmas}
    \Crefname{mlemma}{Lemma}{Lemmas}
    \crefname{mcorollary}{corollary}{corollaries}
    \Crefname{mcorollary}{Corollary}{Corollaries}

%restate theorems
%for copying text
%    \usepackage{clipboard}
%    \newtheorem{rdef}{Definition}
%    \newtheorem{rlemma}{Lemma}
%    \newtheorem{rthm}{Theorem}
%    \newenvironment{repdefinition}[1]
%      {\renewcommand\therdef{\ref*{#1}}\rdef}
%      {\endrdef}
%    \newenvironment{replemma}[1]
%      {\renewcommand\therlemma{\ref*{#1}}\rlemma}
%      {\endrlemma}
%    \newenvironment{reptheorem}[1]
%      {\renewcommand\therthm{\ref*{#1}}\rthm}
%      {\endrthm}


% send proofs to appendix
% appendix
    \usepackage[title]{appendix}
    %\renewcommand{\thechapter}{1}
    %used to move proofs to the appendix.
    \ifConf
         \usepackage{apxproof}
    \fi
    %used to move proofs to the main sections
    \ifFull
       \usepackage[appendix=inline]{apxproof}
    \fi
    \renewenvironment{proofsketch}{\begin{axp@oldproof}[Sketch of Proof]}{\end{axp@oldproof}}
    %used to hide the appendix
    %    \usepackage[appendix=strip]{apxproof}
    \newtheoremrep{mtheorem}[theorem]{\ifConf$\star\,$\fi Theorem}
    \newtheoremrep{mlemma}[lemma]{\ifConf$\star\,$\fi Lemma}
    \newtheoremrep{mcorollary}[corollary]{\ifConf$\star\,$\fi Corollary}
    %format the appendix bibliography
        \renewcommand{\appendixbibliographystyle}{plainurl}
    % changes the counter of appendix section from \Alph to \arabic
        \renewcommand{\appendixprelim}{\renewcommand{\thesection}{Appendix \arabic{section}}\setcounter{section}{0}\renewcommand{\appendix}{}\clearpage\onecolumn}
        \renewcommand{\appendixbibliographyprelim}{\clearpage\onecolumn}
        %\renewcommand{\appendixrefname}{Appendix}

%To prevent placing floats before a section.
    \usepackage[section]{placeins}

    
\setcopyright{rightsretained}
\copyrightyear{2025}
\acmYear{2025}
\acmDOI{XXXXXXX.XXXXXXX}

%% These commands are for a PROCEEDINGS abstract or paper.
\acmConference[Conference acronym 'XX]{Make sure to enter the correct
  conference title from your rights confirmation emai}{June 03--05,
  2018}{Woodstock, NY}
%%
%%  Uncomment \acmBooktitle if the title of the proceedings is different
%%  from ``Proceedings of ...''!
%%
%%\acmBooktitle{Woodstock '18: ACM Symposium on Neural Gaze Detection,
%%  June 03--05, 2018, Woodstock, NY}
\acmISBN{978-1-4503-XXXX-X/18/06}

\newcommand{\pimSAWidth}[0]{w_s}
\newcommand{\pimSAHeight}[0]{h_s}

\newcommand{\ifm}[0]{\mathbf{I}}
\newcommand{\ifmWidth}[0]{w_i}
\newcommand{\ifmHeight}[0]{h_i}
\newcommand{\numInChannel}[0]{m}

\newcommand{\kernel}[0]{\mathbf{K}}
\newcommand{\kernelWidth}[0]{k}
\newcommand{\kernelStride}[0]{s}
\newcommand{\numOutChannel}[0]{n}

\newcommand{\ofm}[0]{\mathbf{O}}
\newcommand{\ofmWidth}[0]{w_o}
\newcommand{\ofmHeight}[0]{h_o}

\newcommand{\pw}[0]{\mathbf{P}}
\newcommand{\pwWidth}[0]{w_p}
\newcommand{\pwHeight}[0]{h_p}
\newcommand{\pwStride}[0]{\sigma}

\newcommand{\pimInput}[0]{\mathbf{x}}
\newcommand{\pimWeight}[0]{\mathbf{M}}
\newcommand{\pimWeightWidth}[0]{w_\mu}
\newcommand{\pimWeightHeight}[0]{h_\mu}
\newcommand{\pimArrayWidth}[0]{w_\text{PIM}}
\newcommand{\pimArrayHeight}[0]{h_\text{PIM}}
\newcommand{\pimOutput}[0]{\mathbf{y}}

\newcommand{\rowRedRate}[0]{R}
\newcommand{\numRedRow}[0]{N_r}
\newcommand{\numOmWeight}[0]{N_o}

\newcommand{\utilMatrix}[0]{\mathbf{U}}
\newcommand{\utilElement}[0]{u}
\newcommand{\numWeightsOnRow}[0]{N_w}
\newcommand{\utilThresh}[0]{\theta_u}

\newcommand{\cdMask}[0]{\mathbf{M}}

\newcommand{\baseKernel}[0]{\mathbf{K}_b}
\newcommand{\baseKernelWidth}[0]{w_b}
\newcommand{\baseKernelHeight}[0]{h_b}

\newcommand{\compWeightMask}[0]{\mathbf{C}}

\newcommand{\subKernel}[0]{\mathbf{K}_s}
\newcommand{\subKernelStride}[0]{\sigma_s}
%%%%%%%%%%%%%%%%%%%%%%%%%%%%%%%%%%%%%%%%%%%%%%%%%%%%%%%
%%%%%%%%%%%%%%%    theorems %%%%%%%%%%%%%%%%%%%%%%%%%%%
%%%%%%%%%%%%%%%%%%%%%%%%%%%%%%%%%%%%%%%%%%%%%%%%%%%%%%%
% \usepackage{mdframed}
\usepackage{kantlipsum}

%%%%%%%%%%%%%%%%%%%%%%%%%%%%%%%%%%%%%%%%%%%%%%%%%%%%%%%
%%%%%%%%%%%%%%%    theorems %%%%%%%%%%%%%%%%%%%%%%%%%%%
%%%%%%%%%%%%%%%%%%%%%%%%%%%%%%%%%%%%%%%%%%%%%%%%%%%%%%%
\theoremstyle{plain}
\newtheorem{theorem}{Theorem}[section]
\newtheorem{proposition}[theorem]{Proposition}
\newtheorem{lemma}[theorem]{Lemma}
\newtheorem{example}[theorem]{Example}
\newtheorem{corollary}[theorem]{Corollary}
\theoremstyle{definition}
\newtheorem{definition}[theorem]{Definition}
\newtheorem{assumption}[theorem]{Assumption}
\theoremstyle{remark}
\newtheorem{remark}[theorem]{Remark}


% \titleformat{\subsection}[runin]% runin puts it in the same paragraph
%        {\normalfont\bfseries}% formatting commands to apply to the whole heading
%        {\thesubsection}% the label and number
%        {0.5em}% space between label/number and subsection title
%        {}% formatting commands applied just to subsection title
%        [.]% punctuation or other commands following subsection title


%%%%%%%%%%%%%%%%%%%%%%%%%%%%%%%%%%%%%%%%%%%%%%%%%%%%%%%
%%%%%%%%%%%%%%%  mathematical notations%%%%%%%%%%%%%%%%
% \usepackage[english]{babel}
% \usepackage[utf8]{inputenc}
% \usepackage[T1]{fontenc}

%% Figures, tables and lists
\usepackage[dvipsnames]{xcolor}
\usepackage{paralist}
\usepackage{graphicx}
\usepackage{subcaption}
\usepackage{longtable} 
\usepackage{multirow}
\usepackage{listings}
\usepackage{makecell}
\usepackage{array}
\usepackage{float}
\usepackage{dsfont}
\usepackage{rotating}
\usepackage{booktabs}
\usepackage{enumerate}
\usepackage{tikz}
\usepackage{pgf}
\usepackage{enumitem}
\usepackage{lipsum} % for generating filler text
\usepackage{titlesec}

%% Math
% \usepackage{amssymb, amsthm,bbm}
\usepackage{mathtools}
\usepackage{mathrsfs}
%% References and author info 
\mathtoolsset{showonlyrefs}
\usepackage{natbib}
\usepackage{authblk}
\usepackage{todonotes}
\usepackage{xr-hyper}


%%%%%%%%%%%%%%%%%%%%%%%%%%%%%%%%%%%%%%%%%%%%%%%%%%%%%%%
\newcommand{\R}{\mathbb R}
\newcommand{\EE}{\mathbb{E}}

\DeclareMathOperator{\Tr}{Tr}
\DeclareMathOperator*{\argmin}{argmin}
\DeclareMathOperator*{\argmax}{argmax}

\newcommand{\bs}[1]{\ensuremath{\boldsymbol{#1}}}
\newcommand{\mc}{\mathcal}
\newcommand{\opt}{^\star}


\newcommand{\diff}{\textnormal{d}}


\def \iid {\stackrel{\textnormal{i.i.d.}}{\sim}}
\def \iidtext {\textnormal{i.i.d.}}





%%%%%%%%%%%%%%%%%%%%%%%%%%%%%%%%%%%%%%%%%%%%%%%%%%%%%%%
%%%%%%%%%%%%%%%%%%%%% colors     %%%%%%%%%%%%%%%%%%%%%%
%%%%%%%%%%%%%%%%%%%%%%%%%%%%%%%%%%%%%%%%%%%%%%%%%%%%%%%
\definecolor{myblue}{rgb}{.8, .8, 1}
\definecolor{mathblue}{rgb}{0.2472, 0.24, 0.6} % mathematica's Color[1, 1--3]
\definecolor{mathred}{rgb}{0.6, 0.24, 0.442893}
\definecolor{mathyellow}{rgb}{0.6, 0.547014, 0.24}


% May add more in future.






    
\begin{document}
\title{\myTitleFirstPage}

\hyphenation{ConvMapSIM}
\myAuthorBlock

\maketitle

\begin{abstract}
In this study, we address the challenge of low-rank model compression in the context of in-memory computing (IMC) architectures.
% , crucial for reducing computation cycles and efficiently utilizing IMC array areas. 
Traditional pruning approaches, while effective in model size reduction, necessitate additional peripheral circuitry to manage complex dataflows and mitigate dislocation issues, leading to increased area and energy overheads.
% , especially when model sparsity does not meet a specific threshold. 
To circumvent these drawbacks, we propose leveraging low-rank compression techniques, which, unlike pruning, streamline the dataflow and seamlessly integrate with IMC architectures. However, low-rank compression presents its own set of challenges, namely i) suboptimal IMC array utilization and ii) compromised accuracy. To address these issues, we introduce a novel approach i) employing shift and duplicate kernel (SDK) mapping technique, which exploits idle IMC columns for parallel processing, and ii) group low-rank convolution, which mitigates the information imbalance in the decomposed matrices. Our experimental results
% , using ResNet-20 and Wide ResNet16-4 networks on CIFAR-10 and CIFAR-100 datasets, 
demonstrate that our proposed method 
% not only matches the performance of existing pruning techniques on ResNet-20 but also 
achieves up to 2.5$\times$ speedup or +20.9\% accuracy boost over existing pruning techniques.
% on Wide ResNet16-4. 
\end{abstract}

\section{Introduction}
The advent of in-memory computing (IMC) architecture heralds a transformative shift in the computing domain, primarily driven by the escalating demands for processing large-scale data on complex deep neural networks. 
By unifying computation and data storage, IMC overcomes the von Neumann bottleneck inherent in traditional architectures that separate memory and processing units. 
% Unlike traditional architectures that separate memory and processing units, leading to the well-known von Neumann bottleneck, the IMC architecture coalesces computation and data storage into a single unified framework. 
This integration facilitates direct matrix-vector multiplication (MVM) within the memory itself, exploiting the parallel computation capabilities for expedited processing at lower energy costs \cite{pipelayer}. 

Despite these advantages, IMC architectures face challenges in handling convolution operations, which require reshaping of convolutional weights and input data for MVM compatibility. The image-to-column (im2col) method \cite{im2col} unrolls convolutional weights into IMC columns for MVM but often suffers from low column utilization.
% , especially with a small number of output channels. 
To address this, techniques such as shift and duplicate kernel (SDK) \cite{sdk} and variable-window SDK (VW-SDK) \cite{vwc-sdk} have been proposed. These methods enhance array utilization and computational performance by enabling input data reuse and parallel processing, effectively exploiting idle columns where duplicated kernels are situated.

% However, inherent limitations of IMC architecture in handling convolution operations necessitate transforming convolutional weights and input data for compatibility with MVM. The image to column (im2col) \cite{im2col} mapping method is one example, which unrolls convolutional weights into IMC columns to perform MVM with input data. Despite its utility, im2col often experiences low column utilization, especially with a small number of output channels. To mitigate these inefficiencies, strategies like shift and duplicate kernel (SDK) \cite{sdk} and variable-window SDK (VW-SDK) \cite{vwc-sdk} have been developed. These approaches enhance overall array utilization, and therefore computing performance, by enabling input data reuse and parallel processing that exploits idle columns where duplicated kernels are situated. 

\begin{figure}
    \centering
    \includegraphics[width=0.9\columnwidth]{fig/figure_1_v2.pdf}
    \caption{Conventional model compression methods for IMC arrays and the proposed low-rank compression method. }
    \vspace{-1em}
    \label{fig:paper-overview}
\end{figure}

While mapping techniques \cite{im2col, sdk, vwc-sdk} improve array utilization, they do not compress the weights themselves for additional performance gains. Pruning methods \cite{vwc-sdk, patdnn, pairs, exploring, unst_pruning}, particularly structured pruning \cite{pairs, exploring} tailored to the unique hardware constraints of IMC arrays, emerged as a promising solution. Structured pruning reduces the computational workload by omitting non-essential weights in a way that complements the IMC's MVM functionality. 
% Techniques such as pattern-based pruning and channel pruning exemplify structured pruning methodologies designed to minimize computational demands with negligible drops in accuracy.

However, pruning techniques encounter hurdles due to the necessity of additional peripheral circuitry, such as zero-skipping hardware \cite{zero-skipping, islped_row_merging} or multiplexers \cite{exploring}, to translate model sparsity into performance benefits (see Fig. \ref{fig:paper-overview}). Zero-skipping hardware leverages sparsity by deactivating unnecessary wordlines—rows containing zero-valued weights—while multiplexers realign input data with pruned weights to counteract dislocation. These requirements introduce extra area and energy overheads, hindering the practical adoption of pruning methods despite their theoretical advantages.

% However, the adoption of pruning techniques is not without its own set of hurdles. One significant drawback is the need for additional peripheral circuitry such, as zero-skipping hardware \cite{zero-skipping} or multiplexer \cite{exploring}, to convert the model sparsity into tangible performance benefits (see Fig. \ref{fig:paper-overview}). 
% Zero-skipping hardware \cite{zero-skipping} makes use of model sparsity by turning off unnecessary wordlines, namely the rows only containing zero-valued weights.
% On the other hand, a multiplexer circuit facilitates the alignment of the input data with the pruned and compressed weight, to counteract the dislocation \cite{exploring}.
% These requirements not only incur extra area and energy overheads but also create barriers to the widespread adoption of pruning techniques in practical settings, limiting their utility despite theoretical advantages.

To overcome the drawbacks associated with pruning, this research advocates for low-rank matrix decomposition technique as an alternative for compressing neural network weights. Unlike pruning, low-rank compression does not necessitate complex peripheral circuitry or realignment mechanisms, offering a more straightforward integration into IMC arrays. However, this method typically involves a trade-off between compression and accuracy, with low-rank compression often resulting in lower performance compared to pruning. Moreover, low-rank compressed matrices frequently lead to suboptimal utilization of IMC arrays. Our work, as shown in Fig. \ref{fig:paper-overview}d, introduces new techniques, namely, SDK and group low-rank compression, aiming to balance accuracy retention, and IMC array utilization. 
Our experimental results show that our proposed method can achieve up to 2.5$\times$ speedup and +20.9\% accuracy boost on Wide ResNet16-4 versus pruning methods. 
% Through these innovations, we seek to pave the way for more effective applications of IMC technology in the computing landscape.

\section{Backgrounds and Related Works}
\mysubsectionNospace{IMC and Convolutional Weight Mapping.} 
IMC architecture marks a paradigm shift towards memory-centric computing, where MVM operation is performed directly within the memory that hosts the deep learning model parameters. 
% IMC can be broadly categorized into analog and digital implementations, each with its unique operational principles and advantages. Analog IMC architectures leverage the resistive crossbar arrays and their analog properties to conduct MVM by exploiting Ohm’s law and Kirchhoff’s law. On the other hand, digital IMC architectures perform MVM operations with the help of adder trees and SRAM devices. 
% Despite their fundamental differences, both analog and digital IMC variants efficiently perform MVM operations with high parallelism by concurrently activating multiple word lines in the memory array, fully harnessing IMC's inherent parallel processing capabilities. 
While IMC architecture is adept at MVM operations, it is not inherently equipped for convolution operations. To address this, convolutional weight mapping methods such as image to column (im2col) have been employed. Im2col, as illustrated in Fig. \ref{fig:conv_weight_mapping}a and c, maps a sliding window of the input feature map (IFM) to the input port of the IMC array. Concurrently, it unrolls and maps each output channel of the kernel to the columns of the IMC array, thus facilitating the convolution operation in the form of MVM.
However, since the number of utilized columns equals the number of output channels, the array utilization of im2col mapping is contingent upon the number of output channels. Hence, im2col mapping often delivers suboptimal array utilization with smaller convolutional filters and consequently resulting in additional computing cycles.

To address the low array utilization issue of im2col \cite{im2col}, Zhang et al. \cite{sdk} and Rhe et al. \cite{vwc-sdk} proposed shift and duplicate kernel (SDK) mapping method. The SDK method uses parallel window (PW) and duplicated kernels to facilitate parallel processing of multiple sliding windows concurrently—unlike the single window processing inherent to the im2col method. By situating duplicated kernels in previously idle columns of the IMC array, the SDK method significantly enhances array utilization. The extent of this enhancement is governed by the size of the PW; for instance, employing a 4$\times$4 PW allows for the duplication of three additional kernels, thereby increasing the number of simultaneously processed sliding windows. 
% For optimal efficiency, the dimension of the PW should be tailored to fully exploit all available columns within the IMC array. 
However, as illustrated in Fig. \ref{fig:conv_weight_mapping}b and d, SDK mapping introduces structural sparsity by its very nature. Specifically, larger PW sizes improve idle column utilization, but at the expense of increased sparsity within the rows. 
% Therefore, it is crucial to ascertain the optimal dimension of the PW that harmonizes the trade-off between minimizing sparsity across both rows and columns, ensuring a balanced optimization of array utilization and computational efficiency.
Based on the generated mapping, the computing cycle of IMC array can be calculated, as proposed by Rhe et al. \cite{vwc-sdk}, using array row (AR) and array column (AC) cycles. AR cycle defines the number of arrays required to process the rows in a mapping, and AC cycle the columns.

% Based on the number of rows and the number of columns utilized or spanned by the weight mappings, the numbers of computing cycles can be calculated. 
% To this end, we calculate the array row (AR) and array column (AC) cycles, as proposed by Rhe et al. \cite{vw-sdk}. 
% AR cycle defines the number of arrays required to process the rows in a mapping, and AC cycle the columns.
% or the number of cycles required to compute all the multiplication between the weights and the input sliding window. On the other hand, the AC cycle defines the number of cycles required to generate partial products for all output channels. 
% Then the computing cycle is given by the multiple of AR cycle, AC cycle, and the number of outputs processed by the mapping (i.e., the number of outputs for im2col is fixed to 1 whereas SDK depends on the dimensions of PW).
% \begin{equation}
%     C_r = \left\lceil \frac{N_r}{h} \right\rceil, \quad C_c = \left\lceil \frac{2 N_c b_c}{w} \right\rceil,
% \end{equation}
% Here, $b_c$ is the bit-precision of a single memory cell. Here, the numerator of $C_c$ represents the number of columns required in a PIM array to host the mapping subject to the requirements for multi-bit and negative value representation. 


% IMC represents a paradigm shift in computing architecture, integrating processing capabilities directly within memory arrays to enhance computational efficiency and reduce data movement. This hardware architecture can be broadly classified into analog and digital versions, both of which leverage the physical properties of memory devices to perform computations. Despite their differences, both analog and digital IMC architectures fundamentally operate as matrix-vector multipliers, a unifying abstraction that simplifies the understanding of their computational capabilities and applications.

% The performance metrics of IMC arrays, particularly energy consumption and latency, can be intricately analyzed through the notions of array row cycles and array column cycles. These cycles represent the operational timing and energy requirements for processing data across the rows and columns of an IMC array, providing a framework for quantifying the efficiency of IMC hardware. By optimizing these cycles, IMC architectures can significantly reduce the overall computational energy and latency, offering a promising solution for accelerating data-intensive applications.

% Among the various convolutional weight mapping methods, im2col stands out for its simplicity, transforming convolution operations into matrix multiplications to facilitate computation on IMC arrays. However, its major drawback lies in the inefficient column utilization, leading to suboptimal performance. In response to this limitation, advanced mapping strategies such as Shifted and Duplicated Kernel (SDK) and Variable Windows and Channels SDK (VWC-SDK) \cite{vwc-sdk} have been developed. These methods aim to improve upon im2col by optimizing the layout of convolutional kernels on IMC arrays, thereby enhancing array utilization and computational efficiency.

\begin{figure}
    \centering
    \includegraphics[width=0.9\columnwidth]{fig/figure_2.pdf}
    \vspace{-1em}
    \caption{Convolutional weight mapping methods.}
    \vspace{-1em}
    \label{fig:conv_weight_mapping}
\end{figure}
\mysubsection{Pruning Methods and Challenges on IMC.} Weight pruning technique \cite{unst_pruning, vwc-sdk} is a strategy to reduce computational requirements by eliminating redundant or non-contributory weights within neural networks.
% Fine-grained pruning method \cite{unst_pruning}, which targets an unimportant weight element irregularly, often complicates hardware implementation due to its irregularity. 
% Conversely, a coarse-grained pruning method \cite{vwc-sdk} aims for hardware-friendly regularity by removing entire filters or channels, facilitating a more structured approach to weight reduction.
% Standing in the middle of these approaches, pattern-based pruning method \cite{patdnn} employs kernel-wise patterns, notably a 4-entry kernel (only four weight elements of the kernel remain un-pruned) to maintain a balance between network efficiency and hardware compatibility \cite{patdnn}.
Within the IMC community, recent pruning techniques are tailored to compress the weight matrix to fit the constraint of IMC arrays, thereby enhancing computational efficiency.
For example, Rhe et al. \cite{vwc-sdk} have proposed the column-wise pruning method to exploit the structural column sparsity of the weight matrix through channel pruning, achieving 1.38$\times$ inference speed in ResNet-20.
Similarly, pattern-based pruning has been used to compress the weight matrix in the row direction, which shows up to 4$\times$ higher compression rate \cite{pairs}.
Although these pruning techniques have shown promising results in compressing the weight matrix and improving computation performance on IMC arrays, these methods necessitate additional peripheral circuits, such as Multiplexer (MUX) \cite{exploring} and Demultiplexer (DEMUX) \cite{structured}, which aims to remap the data path of the input feature for realignment with the sparsity pattern of the pruned model \cite{pimprune, flexible}.
Consequently, while these pruning methods have been instrumental in enhancing the computational efficiency of IMC arrays, the necessity for supplementary peripheral circuits impedes their real-life adoption. 


% To extend weight pruning towards the SDK mapping, Rhe et al. \cite{pairs} designed the pattern shapes to effectively compress the SDK-based weight matrix, which shows up to 4$\times$ higher compression rate.
% Although these pattern-based pruning techniques have shown promising results in compressing weight matrices and improving computation performance on IMC arrays, there exists a challenge that as pruning intensifies, it can lead to a decrease in inference accuracy.


\mysubsection{Low-Rank Compression.} 
% A notable observation that has emerged in the study of DNN is that its layers often exhibit low-rank properties. 
% This implies that the essential information within these matrices can be compactly represented using a subset of their original parameters without a significant loss of information and accuracy. 
To exploit the low-rank properties inherent in neural network weights, various low-rank compression techniques have been applied with considerable success \cite{denton_low-rank}. 
% These methods, including Singular Value Decomposition (SVD), Tucker decomposition and their constrained variants, aim to approximate the original weight matrices/tensors with products of smaller matrices/tensors with lower ranks. 
While the low-rank compression method is often considered to be less effective compared to other compression methods, it holds a significant advantage in terms of fast inference, especially on GPUs \cite{trp_low-rank}. This is attributed to its use of dense matrices, which exhibit local, regular, and parallelizable memory access patterns, facilitating quicker computations. 
% To capitalize on the parallelism benefits of low-rank compression, research has advanced low-rank decomposition techniques for optimizing pre-trained networks post-training \cite{yani_low-rank}. This includes 1xD and Dx1 convolutional kernels for faster, yet accurate, convolutional neural network operations. 
% % Innovations have also leveraged gradient information to prioritize neural network channels during decomposition, enhancing the process by focusing on essential data. 
% However, studies have shown that even minor approximation errors can significantly diminish accuracy through network propagation \cite{trp_low-rank}. 
% In response, innovative strategies promoting rank deficiency during training have been introduced. 
% % Crucially, the art of rank selection has evolved, introducing methods to identify the optimal rank configuration that adeptly balances hardware efficiency with task performance, ensuring a tailored approach to network optimization.
The previous research efforts \cite{yani_low-rank,trp_low-rank}, while pioneering in advancing low-rank compression techniques for deep neural networks, are mostly tailored for optimization on GPUs.
However, this focus has inadvertently left a gap in the exploration of low-rank compression techniques for other forms of hardware, particularly IMC arrays. IMC arrays, known for their potential to significantly reduce energy consumption and latency in performing matrix operations, present a unique architecture that could benefit from specialized compression methods. Yet, the application of low-rank compression within the context of IMC arrays remains unexplored, signifying a critical research gap. This oversight underscores the need for a dedicated investigation into how low-rank compression techniques can be adapted or reimagined to exploit the distinctive advantages and architecture of IMC arrays, a challenge that our current research endeavors to address.

\begin{figure}
    \centering
    \includegraphics[width=0.8\columnwidth]{fig/figure_3.pdf}
    \vspace{-1em}
    \caption{Low-rank matrix decomposition.}
    \vspace{-1em}
    \label{fig:low-rank_decomposition}
\end{figure}


% However, the use of low-rank compression in the context of in-memory computing architectures yet remain uncharted. 

% Low rank compression is a technique aimed at reducing the computational complexity of neural networks by approximating weight matrices with lower rank representations. This approach seeks to capture the most significant features of the weight matrices while discarding redundancies, effectively compressing the network without substantially compromising its predictive performance. Core to low rank compression is the concept of matrix factorization, which decomposes a matrix into a product of two or more matrices of lower dimensions.

\section{Motivation}
Given a weight matrix $\mathbf{W} \in \mathbb{R}^{m \times n}$, low-rank decomposition approximates it as $\hat{\mathbf{W}} = \mathbf{L}\mathbf{R}$, where $\mathbf{L} \in \mathbb{R}^{m \times k}$ and $\mathbf{R} \in \mathbb{R}^{k \times n}$. Each element $w_{i,j}$ is the dot product of the $i^\text{th}$ row of $\mathbf{L}$ and the $j^\text{th}$ column of $\mathbf{R}$ (see Fig. \ref{fig:low-rank_decomposition}). The parameter $k$ balances approximation accuracy and computational savings; smaller $k$ means more compression but potentially more information loss.
% Before delving into the specifics of the proposed method, we first provide a rigorous mathematical overview of low-rank matrix factorization, along with the challenges encountered in its transplantation to the IMC architecture. Let us consider a weight matrix $\mathbf{W} \in \mathbb{R}^{m \times n}$, where $\mathbf{W}$ represents an $m \times n$ matrix. The essence of low-rank decomposition lies in identifying two matrices, $\mathbf{L} \in \mathbb{R}^{m \times k}$ and $\mathbf{R} \in \mathbb{R}^{k \times n}$, such that their product, denoted by $\hat{\mathbf{W}} = \mathbf{L}\mathbf{R}$, closely approximates $\mathbf{W}$. The principle of low-rank decomposition is graphically illustrated in Fig. \ref{fig:low-rank_decomposition}. Here, it can be easily appreciated that the element of the weight matrix, $w_{i,j}$, is given by the dot product of $i^\text{th}$ row vector of $\mathbf{L}$ (i.e., $\mathbf{l}_i$) and $j^\text{th}$ column vector of $\mathbf{R}$ (i.e., $\mathbf{r}_j$). To be more specific, the vector $\mathbf{l}_i$ encapsulates the features of the $i^\text{th}$ row of the matrix $\mathbf{W}$ and $\mathbf{r}_j$ the $j^\text{th}$ column.
% The parameter $k$, representing the rank of the decomposed matrix, acts as a crucial hyper-parameter and a bottlenecking factor that serves as a control knob to trade-off approximation accuracy versus memory/computational savings.
% More specifically, a lower $k$ means a more compact representation but potentially more information loss.
% , whereas a higher $k$ allows for a more accurate approximation at the cost of reduced computational savings. 
% Note that if the chosen $k$ matches the rank of the original weight matrix $\mathbf{W}$, the decomposition results in perfect reconstruction of the original weight matrix, i.e., $\mathbf{W} = \hat{\mathbf{W}}$, with no loss in approximation. 
The adaptation of the low-rank compression technique, which involves decomposing a larger matrix into two smaller ones, to IMC architecture presents two significant impediments: low array utilization and diminished accuracy in machine learning tasks. Fig. \ref{fig:challenges} illustrates the computational difficulties of applying low-rank matrix compression within an IMC framework. For instance, when original, uncompressed convolutional weights are mapped onto IMC arrays using the prevalent im2col mapping strategy, a rectangular-shaped weight matrix, $\mathbf{W}$, is produced. This matrix extends across more rows than columns and requires three computing cycles to generate a single output, as shown in Fig. \ref{fig:challenges}a. Conversely, Fig.  \ref{fig:challenges}b showcases low-rank compression on IMC arrays, where $\mathbf{W}$ is decomposed into two matrices that do not fully utilize the IMC array's capacity. This decomposition, intended to reduce computational load, paradoxically introduces an additional computing cycle due to low array utilization.

Moreover, the inherent rectangular shape of convolutional kernels leads to a significant imbalance in information encoding between the $\mathbf{L}$ and $\mathbf{R}$ matrices. This imbalance causes a notable loss of information in the weight matrix's rows, thereby reducing computation accuracy, a crucial aspect for the effectiveness of neural network models. To address the first challenge of low array utilization, we propose the integration of SDK mapping with the low-rank compression technique. This approach enhances array utilization through input data reuse and the added parallelism of duplicated kernels. For the second challenge, concerning reduced machine learning task accuracy, we introduce grouped low-rank decomposition. This method partitions the weight matrix into multiple groups prior to low-rank compression, effectively mitigating the information imbalance initially present in $\mathbf{L}$, while capturing essential weight features with a minimal increase in parameters.

\begin{figure}
    \centering
    \includegraphics[width=0.9\columnwidth]{fig/figure_4.pdf}
    \vspace{-1em}
    \caption{Motivation of our research.}
    \vspace{-1em}
    \label{fig:challenges}
\end{figure}

% Although the decomposition is aimed at reducing the overall computational burden, it introduces additional computing cycle due the 



% \textbf{Challenge 1 - Low array utilization due to skinny and fat matrices}. 

% \textbf{Challenge 2 - Information loss due to rectangular convolutional kernels}. 


% Despite its potential for reducing model size and complexity, low rank compression presents distinct challenges when applied to IMC arrays. Chief among these is the issue of low array utilization, as the compressed matrices often do not align well with the hardware's parallel computing architecture, leading to underutilized computational resources. Moreover, the accuracy of machine learning tasks may suffer as a result of the compression, especially in applications requiring high precision. These challenges necessitate innovative solutions to harness the benefits of low rank compression while mitigating its drawbacks for efficient IMC array implementation.

\begin{figure*}
    \centering
    \includegraphics[width=1.7\columnwidth]{fig/figure_5.pdf}
    \vspace{-1em}
    \caption{Overview of the proposed techniques for low-rank compression on IMC arrays.}
    \vspace{-1em}
    \label{fig:proposed}
\end{figure*}


\section{Proposed Method}
\mysubsectionNospace{Group Low-Rank Compression.} 
To address the severe accuracy degradation and low row utilization issues associated with the $\mathbf{L}$ matrix in traditional low-rank approximations, we propose a group low-rank decomposition technique, as illustrated in Fig.~\ref{fig:proposed}a. 
In this approach, the weight matrix is partitioned into $g$ submatrices or groups, denoted in block matrix notation as $\mathbf{W} = [\mathbf{W}_1, \mathbf{W}_2, \dots, \mathbf{W}_g]$, where $g$ is the number of groups. Each submatrix $\mathbf{W}_i$ is then independently compressed using low-rank decomposition:
% To address the severe accuracy drop issue and low row utilization issue associated with $\mathbf{L}$ matrix, we propose group low-rank decomposition technique, as shown in Fig. \ref{fig:proposed}a, where the im2col mapping is split into groups to generate multiple matrices (i.e., $\mathbf{W} = [\mathbf{W}_1, \mathbf{W}_2, \cdots, \mathbf{W}_g]$, where $g$ is the number of groups). The split weight matrices are then compressed with low-rank decomposition independently to approximate the MVM between the im2col mapping and the input:
\begin{equation}
    \mathcal{D}_g(\mathbf{W}) := \begin{bmatrix}
        \mathcal{D}(\mathbf{W}_1),\;
        \mathcal{D}(\mathbf{W}_2),\; 
        \cdots,\;
        \mathcal{D}(\mathbf{W}_g)
    \end{bmatrix}.
\end{equation}
Here, $\mathcal{D}_g(\cdot)$ denotes the grouped low-rank decomposition operator for a specified number of groups $g$, and $\mathcal{D}(\cdot)$ represents the traditional low-rank decomposition operator without matrix partitioning, such that $\mathcal{D}(\mathbf{W}_i) := \mathbf{L}_i \mathbf{R}_i$.
% where $\mathcal{D}_g(\cdot)$ is the grouped low-rank decomposition operator for a given number of groups, $g$; $\mathcal{D}(\cdot)$  is the traditional low-rank decomposition operator which does not involve splitting of the matrix into groups (i.e., $\mathcal{D}(\mathbf{W}_i) = \mathbf{L}_i\mathbf{R}_i$). 

\vspace{1em}\noindent\textbf{Theorem 1.} \textit{Given a weight matrix $\mathbf{W}$ and a target rank $k$, the reconstruction error of its group low-rank approximation, $\varepsilon_g := ||\mathbf{W} - \mathcal{D}_g(\mathbf{W})||_\mathrm{F}$, is upper-bounded by that of the traditional low-rank approximation, $\varepsilon := ||\mathbf{W} - \mathcal{D}(\mathbf{W})||_\mathrm{F}$, for an arbitrary number of groups, $g$:} 
\begin{equation}
    \underbrace{||\mathbf{W} - \mathcal{D}_g(\mathbf{W})||_\mathrm{F} }_{\varepsilon_g}
    \leq 
    \underbrace{||\mathbf{W} - \mathcal{D}(\mathbf{W})||_\mathrm{F}}_{\varepsilon}
    \label{eqn:theorem1}
\end{equation}
\textit{where both reconstruction errors are measured in Frobenius norm, denoted by $||\cdot||_\mathrm{F}$.}

\vspace{1em}\noindent\textit{Proof.} We begin by approximating $\mathbf{W}$ using truncated singular value decomposition (SVD), i.e., $\mathcal{D}(\mathbf{W}) = \mathbf{U}\mathbf{\Sigma}\mathbf{V}^\top$. We know that this is an optimal approximation with respect to the Frobenius norm according to the Eckart-Young-Mirsky theorem. The decomposed matrices can be expressed in a block matrix form:
\begin{equation}
    \mathcal{D}(\mathbf{W}) = \underbrace{\mathbf{L}}_{\mathbf{U}\mathbf{\Sigma}}
    \underbrace{
        \begin{bmatrix}
            \mathbf{R}_1,\:
            \mathbf{R}_2,\:
            \cdots,\:
            \mathbf{R}_g,
        \end{bmatrix}
    }_{\mathbf{V}^\top}
\end{equation}
where $\mathbf{L}\!=\!\mathbf{U}\mathbf{\Sigma}$ and $\mathbf{R}_i$ is the $i$-th submatrix of $\mathbf{V}^\top$ which is partitioned into $g$ groups. 

Following the distributive property of block matrices, $\mathbf{L}$ is multiplied with all $\mathbf{R}_i$ matrices, approximating $\mathbf{W}_i$ (i.e., $\mathbf{W_i} \approx \mathbf{L}\mathbf{R}_i$). However, according to the Eckart–Young–Mirsky theorem, we know that $\mathbf{L}\mathbf{R}_i$ is not necessarily the optimal approximation of $\mathbf{W}_i$ since $\mathbf{L}\mathbf{R}_i$ may not be the SVD of $\mathbf{W}_i$. Hence, 
\begin{equation}
    ||\mathbf{W}_i - \mathcal{D}(\mathbf{W}_i)||_\mathrm{F}
    \leq
    ||\mathbf{W}_i - \mathbf{L}\mathbf{R}_i||_\mathrm{F}  \quad\quad \forall i
    \label{eqn:4}
\end{equation}
where $\mathcal{D}(\mathbf{W}_i)$ is the truncated SVD of $\mathbf{W}_i$. Note that RHS represents the reconstruction error of $\mathbf{W}_i$ of the group low-rank compression method, and LHS represents that of the traditional method. 

The Eq. (\ref{eqn:4}) implies that the inequality should hold also for the summation over all $i$ of the squares of the norms:
\begin{equation}
    \underbrace{\sum_{i = 1}^{g} ||\mathbf{W}_i - \mathcal{D}(\mathbf{W}_i)||_\mathrm{F}^2}_{\varepsilon_g^2}
    \leq
    \underbrace{\sum_{i = 1}^{g} ||\mathbf{W}_i - \mathbf{L}\mathbf{R}_i||_\mathrm{F}^2 }_{\varepsilon^2}
\end{equation}
where LHS is the square of $\varepsilon_g$ and RHS is the square of $\varepsilon$.
Since the square function is monotonic for non-negative real numbers and the Frobenius norms are also non-negative, the inequality holds even after taking the square root of both sides. Doing so yields Eq. \ref{eqn:theorem1} and concludes the proof. $\hfill\ensuremath{\Box}$

By Theorem 1, the proposed method guarantees a smaller reconstruction error than the traditional low-rank compression method, promising an improved accuracy performance. Although the performance boost comes at a cost of additional $\mathbf{L}_i$ matrices, note that these matrices are mapped to the idle rows. Therefore, in the context of IMC arrays, the proposed group low-rank compression could potentially offer accuracy gains at no cost, if the number of groups is chosen wisely. Nonetheless, the proposed Theorem 1 is significant as it is universally applicable to all matrices and neural network layers such as convolutional layers and linear layers. 

% \vspace{1em}With group low-rank compression, we can effectively relieve the previously mentioned imbalance in information encoding, with additional $\mathbf{L}_i$ matrices. Consequently, the information loss in the rows of the weight matrix should be mitigated, leading to better accuracy. Although the performance boost comes at a cost of $g-1$ additional $\mathbf{L}_i$ matrices and parameters, note that these matrices are mapped to the idle rows. Therefore, in the context of IMC arrays, the proposed group low-rank compression could potentially offer accuracy gains at no cost, if the number of groups is chosen wisely. 

\mysubsection{SDK for Low-Rank Compression.}
To improve on the low column utilization issue, we seek to integrate SDK mapping \cite{vwc-sdk} together with the low-rank decomposition technique. Since the SDK mapping inherently uses more columns than the im2col mapping, its low-rank decomposed version should also utilize more columns for parallel processing. However, the formulation to derive the low-rank decomposition of SDK mapping is non-trivial. To this end, we first propose a rigorous mathematical description of the SDK mapping method and then derive a low-rank decomposition formula with respect to the SDK mapping. 

\vspace{1em}\noindent\textbf{Theorem 2.} \textit{Given a weight matrix $\mathbf{W}$, and its low-rank decomposed matrices, $\mathbf{L}$ and $\mathbf{R}$, low-rank approximation of the SDK mapping of $\mathbf{W}$ is given by:}
\begin{equation}
    \mathcal{D}(\,\operatorname{SDK}(\mathbf{W})\,) = (\mathbf{I}_N \otimes \mathbf{L}) \operatorname{SDK}(\mathbf{R})
    \label{eqn:theorem2}
\end{equation}
\textit{where $\mathbf{I}_N$ is the identity matrix of size $N \times N$, $N$ is the number of parallel outputs in the SDK mapping, $\otimes$ denotes the Kronecker product, and $\operatorname{SDK}(\cdot)$ denotes the SDK operator that generates SDK mapping for a given matrix.}

% The idea is to shift and duplicate the $\mathbf{R}$ matrix to the idle columns
% One straightforward approach to improving array utilization is to apply low-rank decomposition on an SDK mapping instead of im2col; SDK mapping uses more columns, therefore the low-rank decomposed version should have better column utilization. However, this na\"ive approach has several drawbacks that make its implementation impractical. Firstly, as we have highlighted in the related works section, low-rank decomposing post-training often results in severe accuracy drop. Hence, it is imperative to train the deep learning model with low-rank constraints or properties. 

\begin{figure*}[t]
\centering
\includegraphics[trim={0cm 0cm 0cm 0cm}, clip, width=0.85\linewidth]{fig/results_v3.pdf}
\vspace{-1em}
\caption{
Accuracy and computing cycle of pattern-pruning methods vs. the proposed method evaluated for ResNet-20 and WRN16-4 with varying array sizes. 
}
\vspace{-1em}
\label{fig:result}
\end{figure*}

\vspace{1em}\noindent\textit{Proof.} A convolutional kernel matricized by im2col mapping method can be described as $\mathbf{W} = [\mathbf{w}_1, \mathbf{w}_2, \cdots, \mathbf{w}_m]^\top$ where $\mathbf{w}_i \in \mathbb{R}^{1 \times n}$ is a vectorized output channel of a convolutional kernel. 
% A convolutional filter can be matricized through convolutional weight mapping techniques such as im2col:
% \begin{equation}
%     \mathbf{W} = [\mathbf{w}_1, \mathbf{w}_2, \cdots, \mathbf{w}_m]^\top
% \end{equation}
% where $\mathbf{w}_i$ is a vectorized kernel. 
Then the SDK mapping, $\operatorname{SDK}(\mathbf{W}) \in \mathbb{R}^{Nn \times b}$ can be expressed as a linear transformation of $\mathbf{W}$. 
\begin{equation}
    \label{eqn:sdk}
    \begin{split}
        \operatorname{SDK}(\mathbf{W}) & = [\mathbf{P}_1 \mathbf{W}^\top, \mathbf{P}_2 \mathbf{W}^\top, \cdots, \mathbf{P}_N \mathbf{W}^\top]^\top \\
        & = 
        \underbrace{
        \begin{bmatrix}
            \mathbf{W} & \mathbf{0} & \cdots & \mathbf{0} \\
            \mathbf{0} & \mathbf{W} & \cdots & \mathbf{0} \\
            \vdots & \vdots & \ddots & \vdots \\
            \mathbf{0} & \mathbf{0} & \cdots & \mathbf{W} \\
        \end{bmatrix}}_{\mathbf{W}_b \in \mathbb{R}^{Nm \times Nn}}
        \underbrace{
        \begin{bmatrix}
            \mathbf{P}_1^\top \\
            \mathbf{P}_2^\top \\ 
            \vdots \\
            \mathbf{P}_N^\top 
        \end{bmatrix}}_{\mathbf{P} \in \mathbb{R}^{Nn \times b}}
    \end{split}
\end{equation}
where $\mathbf{P}_s \in \mathbb{R}^{b \times n}$ is the $s$-th padding matrix. $N$ is the total number of parallel outputs, which is determined by the PW dimension, and $b$ is the input dimension of the flattened PW. 
% Note that, here we examine $\tilde{\mathbf{W}}$ as a block matrix, and $\mathbf{W}$ can be factored out in the form of a block diagonal matrix, $\mathbf{W}_b$. 
The role of the padding matrix is to insert zero column vectors into $\mathbf{W}$, such that the elements of the kernels are appropriately shifted and aligned with the PW input. $\mathbf{P}_s$ can be built from a square identity matrix followed by the insertion of zero row vectors in a specific pattern that is dictated by the SDK mapping. Then the element of $\mathbf{P}_s$ at index $i,j$ is defined as:
\begin{equation}
    [\mathbf{P}_s]_{i,j} =
    \begin{cases}
        1 \quad \text{if } \, i = f(j) \\
        0 \quad \text{otherwise}
    \end{cases}
\end{equation}
where $f(\cdot)$ is a mapping function that describes the insertion locations of the zero column vectors. 

Now we can substitute low-rank compressed matrix of the im2col mapping, $\mathbf{W} = \mathbf{LR}$, in to equation (\ref{eqn:sdk}).
\begin{equation}
    \operatorname{SDK}(\mathbf{W}) = [\mathbf{P}_1 \mathbf{R}^\top \mathbf{L}^\top, \mathbf{P}_2 \mathbf{R}^\top \mathbf{L}^\top, \cdots, \mathbf{P}_N \mathbf{R}^\top \mathbf{L}^\top]^\top
\end{equation}
Then, instead of factoring out the entire $\mathbf{LR}$, which would give us the equivalent formulation as in  (\ref{eqn:sdk}), we can solely factor out $\mathbf{L}$ in the form of block diagonal matrix: 
\begin{equation}
    \operatorname{SDK}(\mathbf{W}) = 
    \underbrace{
    \begin{bmatrix}
        \mathbf{L} & \mathbf{0} & \cdots & \mathbf{0} \\
        \mathbf{0} & \mathbf{L} & \cdots & \mathbf{0} \\
        \vdots & \vdots & \ddots & \vdots \\
        \mathbf{0} & \mathbf{0} & \cdots & \mathbf{L} \\
    \end{bmatrix}}_{\tilde{\mathbf{L}} \in \mathbb{R}^{Nm \times Nk}}
    \underbrace{
    \begin{bmatrix}
        \mathbf{R} \mathbf{P}_1^\top \\
        \mathbf{R} \mathbf{P}_2^\top \\ 
        \vdots \\
        \mathbf{R} \mathbf{P}_N^\top 
    \end{bmatrix}}_{\tilde{\mathbf{R}} \in \mathbb{R}^{Nk \times b}}
    \label{eqn:10}
\end{equation}
where the $\tilde{\mathbf{L}}$ and $\tilde{\mathbf{R}}$ are the low-rank decomposed matrices of the SDK mapping. We can see that $\tilde{\mathbf{L}}$ can be compactly denoted as $\mathbf{I}_N \otimes \mathbf{L}$ and $\tilde{\mathbf{R}} = \operatorname{SDK} (\mathbf{R})$. Plugging in the two expressions for $\tilde{\mathbf{L}}$ and $\tilde{\mathbf{R}}$ into Eq. (\ref{eqn:10}) yields Eq. (\ref{eqn:theorem2}) and concludes the proof. $\hfill\ensuremath{\Box}$

The proposed method is graphically illustrated in Fig. \ref{fig:proposed}b.

\section{Experiments and Results}
\mysubsectionNospace{Experimental Setup.}
To evaluate and demonstrate the effectiveness of the proposed method, we employed ResNet-20 and Wide-ResNet16-4 (WRN16-4) for image classification tasks on CIFAR-10 and CIFAR-100 datasets, respectively. 
% Our key performance metrics are top-1 accuracy and computing cycles on IMC array as previously presented in the related works section. 
Here, ResNet-20 was trained with expansion parameter set to 1 (i.e., the first basic block has 16 input/output channels). 
% On top of those networks, we implement the proposed low-rank decomposed convolution layer, which replaces the existing full-rank convolution layer, with the help of group convolution for efficient training and inference. For comparisons, we also implemented state-of-the-art pattern pruning solutions. 
% All deep learning models were implemented and trained with PyTorch framework. The models were trained with an SGD optimizer paired with a cosine annealing strategy. The initial learning rate was set to 1e-1, minimum learning rate 1e-4, scheduler cycle 20, and weight decay 5e-4. The batch size was set to 512. 
Weights and activations of all deep learning models were both quantized 4 bit, and the models were trained following the quantization aware training framework proposed in \cite{anyprecision}. 
% To be more specific, uncompressed convolutional weights, $\mathbf{W}$, and low-rank decomposed weights, $\mathbf{L}$ and $\mathbf{R}$, are all quantized to 4 bit precision. 
We did not compress the very first convolution layer and the last linear layer, as they are known to be highly sensitive to perturbations and are often processed on digital computing units that support floating point operations \cite{vwc-sdk, sdk}. 
Proposed low-rank compressed models were trained from scratch for 250 epochs, where as the pattern-pruned counterparts were fine-tuned for 20 epochs from a pre-trained model. The pre-trained model was trained for 200 epochs. We experimented for three trials using different seeds.
% with the identical optimizer and learning rate scheduler as stated above. 

\begin{figure}
    \centering
    \includegraphics[width=0.85\columnwidth]{fig/hw_perf.pdf}
    \vspace{-1em}
    \caption{Energy consumption of the pattern-pruning methods vs. the proposed methods evaluated for ResNet-20 and WRN16-4 with varying array sizes.}
    \vspace{-1em}
    \label{fig:hw-perf}
\end{figure}

\mysubsection{Comparisons with pattern pruning methods.} 
% Our experiments with ResNet-20 and WRN16-4 demonstrate that the proposed low-rank compression method retains high accuracy while significantly reducing the number of computing cycles, achieving up to 2.5$\times$ speed-up,  20.9\% accuracy improvement, or 71\% improvement in energy saving against pattern pruned counterparts. 
% Table \ref{tab:pattern_pruning} presents the baseline models that were compressed with the latest pattern pruning techniques. 
% Pattern pruned models were trained with varying values of entries from 8 to 1 which indicate the number of unpruned or non-zero elements in a kernel. 
% "\textit{Baseline}" in the first row represents an unpruned baseline model that will serve as a point of comparison. 
% The table shows the accuracy and computing cycle for the pattern-pruned models for the two networks. We have evaluated the computing cycles for varying array sizes of 32$\times$32, 64$\times$64, and 128$\times$128. However, we have omitted the results for the array size of 128$\times$128 due to lack of space. It is worth noting that, in the case of ResNet-20, the pruned model performs extremely well even at low entries (within 3\% accuracy drop compared to the baseline), and only shows a significant accuracy drop with 1 entry pattern. However, the same cannot be said for WRN16-4, which experiences considerable accuracy degradation (over 3\% drop) starting from 4 or 5 entry patterns. 
% \begin{table}[]
%     \centering
%     \vspace{-1em}
%     \caption{Results on pattern pruning}
%     \vspace{-1em}
%     \includegraphics[width=\columnwidth]{tab/table_1.pdf}
%     \vspace{-2em}
%     \label{tab:pattern_pruning}
% \end{table} 
Table \ref{tab:ours} presents the model accuracy and computing cycle for a low-rank compressed model for different combinations of group and rank. 
% The first group of rows (labeled "\textit{w/o SDK}" and group = 1) represent the baseline low-rank compressed models where the proposed SDK mapping and group low-rank compression were not implemented. 
% Whereas the second group of rows (labeled "\textit{w/ SDK}") represents the models trained and mapped with the proposed two techniques. 
The rank of each layer was configured uniformly to the number of output channels, $m$, divided by a constant factor, in this case, 2, 4, 8, and 16. 
Also, the number of groups is set to either 1, 2, 4, or 8. 
Fig. \ref{fig:result} presents a comprehensive overview and comparisons of the proposed low-rank compression method versus the existing pattern-pruning approaches. Baseline accuracies and computing cycles of unpruned models are presented by the orange dotted line. The first row of figures presents experiments conducted on ResNet-20, whereas the second row shows the results for WRN16-4. For pattern-pruning baselines, we plotted the results for entries ranging from 1 to 8, whereas, for our proposed method, we selectively plotted the combinations of rank and group that form the Pareto front for conciseness and clarity. The result demonstrates the effectiveness of the proposed compression method, achieving on-par performance with pattern-pruning approaches on ResNet-20, and significantly outperforming them on WRN16-4. From Fig. \ref{fig:result}d, we can see that our proposed approach can achieve up 2.5$\times$ speedup and +20.9\% accuracy boost compared to the pruning counterparts. 

To evaluate and compare the hardware performance, we have built a simulator based on NeuroSIM~\cite{bib:neurosim}  and ConvMapSIM~\cite{bib:convmapsim} that measures the energy consumption of the proposed and pattern pruning methods. We measured the energy consumption for both ResNet-20 and WRN16-4 networks for varying array sizes. Fig. \ref{fig:hw-perf} plots the normalized energy consumption of the two compression methods against im2col method. For low-rank compressed models, we employ the model with group = 4 and rank = $m/8$, which exhibits high accuracy (less than 1 or 2\% drop from the uncompressed model) while achieving significant computing cycle reduction. For pattern-pruned models, we employ the model pruned with 6-entries, which achieves almost identical accuracy performance as our low-rank model. The results show that the proposed method is more energy-efficient than the pattern-pruned models for both networks across all array dimensions. For smaller arrays, the proposed method could improve energy saving by up to 71\% when compared against the pattern-pruning method and up to 80\% against im2col method. 

We highlight once again, that unlike pruning approaches that necessitate additional peripheral circuitry to combat misalignment and dislocation issues, the proposed method can be adopted on any IMC array and is free of such overheads, and yet achieves better performance in both accuracy and computing cycles. This result is significant and underscores the potential impact of the proposed method when integrated with various deep learning networks and IMC architectures. 

\begin{figure}
    \centering
    \includegraphics[width=0.85\columnwidth]{fig/vs_quant.pdf}
    \vspace{-1em}
    \caption{Comparison of accuracy and computing cycle performance between low-rank compression models and quantized models.}
    \vspace{-1em}
    \label{fig:vs_quant}
\end{figure}

\begin{figure}
    \centering
    \includegraphics[width=0.85\columnwidth]{fig/ablation2.pdf}
    \vspace{-1em}
    \caption{Comparison of accuracy and computing cycle performance between low-rank compression models with and without the proposed techniques.}
    \vspace{-1em}
    \label{fig:ablation}
\end{figure}

\mysubsection{Comparisons with quantization methods.} To enrich our evaluation, we also compare our proposed method against quantization with varying bit precision. 
We trained dedicated 1, 2, 3, and 4 bit quantized models of ResNet-20 using a QAT framework and a DoReFa quantizer.
The accuracies and computing cycles of the quantized models for array dimensions of 64$\times$64 and 128$\times$128 are plotted in Fig. \ref{fig:vs_quant}. It can be seen that the proposed low-rank compression method outperforms quantized models, achieving up to 1.8$\times$ speed-up. 

\mysubsection{Comparisons with traditional low-rank compression.} 
As shown in Fig. \ref{fig:ablation} and Table \ref{tab:ours}, the proposed method consistently outperforms the traditional low-rank compressed baseline models (where the proposed SDK mapping and group low-rank compression technique are not applied). 
Whereas the prior low-rank method can reduce the computing cycles to 54K and 40K in the WRN16-4 and ResNet-20 networks, respectively, the proposed method significantly reduces them to 37K in WRN16-4 and 25K in ResNet-20. This is equivalent to 1.5$\times$ and 1.6$\times$ speedup in WRN16-4 and ResNet-20, respectively, due to better array utilization with SDK mapping. The gain is more notable on larger arrays, where SDK mapping can be better explored for more parallel computation.  
% Furthermore, when comparing the proposed method to the prior one at equivalent levels of inference accuracy, it achieves computing speeds that are 1.5$\times$ and 1.6$\times$ faster in WRN16-4 and ResNet-20, respectively.
On the other hand, the proposed method also boasts significant boosts in accuracy even at lower values of rank, thanks to the use of group low-rank compression. It can be seen from Table \ref{tab:ours}, that with the increasing number of groups, even with just 2, we witness significant mitigation of accuracy drop. 

\begin{table}[]
    \centering
    \caption{Results on low-rank compression}
    \vspace{-1em}
    \includegraphics[width=0.9\columnwidth]{tab/table_2.pdf}
    \vspace{-2em}
    \label{tab:ours}
\end{table}

\section{Conclusion}
In this study, we tackled the challenge of efficiently compressing models tailored to IMC architectures to enhance computational efficiency without the significant area and energy overheads typical of traditional pruning methods. Our approach introduced low-rank compression techniques integrated with novel SDK and group low-rank convolution strategies, mitigating issues such as suboptimal IMC array utilization and accuracy compromises. Through rigorous experiments on ResNet-20 and WRN16-4 using CIFAR-10 and CIFAR-100 datasets, our method demonstrated its potential by matching or surpassing the performance of existing pruning techniques while significantly reducing computational cycles. This research not only offers a viable alternative to conventional pruning but also opens new avenues for optimizing deep neural networks for IMC architectures, offering 
paving the way for their more efficient deployment in real-world applications.

\section*{Acknowledgement}
This work was partly supported by
the National Research Foundation of Korea (NRF) grant (No. RS-2024-00345732); % 신진
the Institute for Information \& communications Technology Planning \& Evaluation (IITP) grants (RS-2020-II201821, IITP-2021-0-02052, RS-2019-II190421, RS-2021-II212068); % 명품, ITRC, AI, 혁신 
% the Ministry of Trade, Industry \& Energy (MOTIE, Korea) grant (RS-2023-00235718), % 산자부
the Technology Innovation Program (RS-2023-00235718, 23040-15FC) funded by the Ministry of Trade, Industry \& Energy (MOTIE, Korea) (1415187505); % 산자부
Samsung Electronics Co., Ltd (IO230404-05747-01); % 삼성전략
and the BK21-FOUR Project.

% the MSIT (Ministry of Science and ICT), Korea, under the ICT Creative Consilience program (IITP-2023-2020-0-01821) supervised by the IITP (Institute for Information \& communications Technology Planning \& Evaluation);
% the MSIT, Korea, under the ITRC (Information Technology Research Center) support program (IITP-2021-0-02052) supervised by the IITP;
% IITP grant (RS-2019-II190421, AI Graduate School Support Program(Sungkyunkwan University));
% IITP Grant (Artificial Intelligence Innovation Hub) under Grant RS-2021-II212068;
% Samsung Electronics Co., Ltd (IO230404-05747-01); and
% the Technology Innovation Program (or Industrial Strategic Technology Development Program) (RS-2023-00235718, 23040-15FC) funded By the Ministry of Trade, Industry \& Energy (MOTIE, Korea) (1415187505).


\bibliographystyle{IEEEtran}
\bibliography{IEEEabrv,mybibfile}

\end{document}
