%%%%%%%% ICML 2025 EXAMPLE LATEX SUBMISSION FILE %%%%%%%%%%%%%%%%%

\documentclass{article}

% Recommended, but optional, packages for figures and better typesetting:
\usepackage{microtype}
\usepackage{graphicx}
\usepackage{subfigure}
\usepackage{booktabs} % for professional tables

% hyperref makes hyperlinks in the resulting PDF.
% If your build breaks (sometimes temporarily if a hyperlink spans a page)
% please comment out the following usepackage line and replace
% \usepackage{icml2025} with \usepackage[nohyperref]{icml2025} above.
\usepackage{hyperref}


% Attempt to make hyperref and algorithmic work together better:
\newcommand{\theHalgorithm}{\arabic{algorithm}}

% Use the following line for the initial blind version submitted for review:
% \usepackage{icml2025}

% If accepted, instead use the following line for the camera-ready submission:
\usepackage[accepted]{icml2025}

% For theorems and such
\usepackage{amsmath}
\usepackage{amssymb}
\usepackage{mathtools}
\usepackage{amsthm}
\usepackage{xspace}
\usepackage{natbib}
\usepackage{float}
\usepackage{bm}
\usepackage{authblk}
% \usepackage{algpseudocode}
% if you use cleveref..
\usepackage[capitalize,noabbrev]{cleveref}

%%%%%%%%%%%%%%%%%%%%%%%%%%%%%%%%
% THEOREMS
%%%%%%%%%%%%%%%%%%%%%%%%%%%%%%%%
\theoremstyle{plain}
\newtheorem{theorem}{Theorem}[section]
\newtheorem{proposition}[theorem]{Proposition}
\newtheorem{lemma}[theorem]{Lemma}
\newtheorem{corollary}[theorem]{Corollary}
\theoremstyle{definition}
\newtheorem{definition}[theorem]{Definition}
\newtheorem{assumption}[theorem]{Assumption}
\theoremstyle{remark}
\newtheorem{remark}[theorem]{Remark}

% Todonotes is useful during development; simply uncomment the next line
%    and comment out the line below the next line to turn off comments
%\usepackage[disable,textsize=tiny]{todonotes}
\usepackage[textsize=tiny]{todonotes}
\newcommand{\tao}[1]{\textcolor{blue}{\textbf{Tao:} #1}}
\newcommand{\tailin}[1]{\textcolor{red}{\textbf{Tailin:} #1}}
\newcommand{\jiashu}[1]{\textcolor{green!50!black}{\textbf{Jiashu:} #1}}
\newcommand{\ruiqi}[1]{\textcolor{orange!85!black}{\textbf{Ruiqi:} #1}}
\newcommand{\proj}{T-SCEND\xspace}
\newcommand{\vy}{\mathbf{y}}
\newcommand{\vx}{\mathbf{x}}
\def\x{{\boldsymbol{x}}}
\def\eps{{\boldsymbol{\epsilon}}}
\def\C{{\mathcal{C}}}
\def\E{{\mathbb{E}}}
\def\l{{\mathcal{L}}}

\usepackage{amsmath}
\DeclareMathOperator*{\argmax}{arg\,max}
\DeclareMathOperator*{\argmin}{arg\,min}
% The \icmltitle you define below is probably too long as a header.
% Therefore, a short form for the running title is supplied here:
% \icmltitlerunning{Submission and Formatting Instructions for ICML 2025}

\begin{document}
\twocolumn[
\icmltitle{\proj: Test-time Scalable MCTS-enhanced Diffusion Model}

% It is OKAY to include author information, even for blind
% submissions: the style file will automatically remove it for you
% unless you've provided the [accepted] option to the icml2025
% package.

% List of affiliations: The first argument should be a (short)
% identifier you will use later to specify author affiliations
% Academic affiliations should list Department, University, City, Region, Country
% Industry affiliations should list Company, City, Region, Country

% You can specify symbols, otherwise they are numbered in order.
% Ideally, you should not use this facility. Affiliations will be numbered
% in order of appearance and this is the preferred way.
\icmlsetsymbol{equal}{*}
\icmlsetsymbol{correspond}{$\dagger$}

\begin{icmlauthorlist}
\icmlauthor{Tao Zhang}{equal,westlake}
\icmlauthor{Jia-Shu Pan}{equal,westlake}
\icmlauthor{Ruiqi Feng}{westlake}
\icmlauthor{Tailin Wu}{equal,correspond,westlake}
\end{icmlauthorlist}
\icmlaffiliation{westlake}{Department of Artificial Intelligence, Westlake University}
\icmlcorrespondingauthor{Tailin Wu}{wutailin@westlake.edu.cn}

% You may provide any keywords that you
% find helpful for describing your paper; these are used to populate
% the "keywords" metadata in the PDF but will not be shown in the document
% \icmlkeywords{Machine Learning, ICML}

\vskip 0.3in
]

% this must go after the closing bracket ] following \twocolumn[ ...

% This command actually creates the footnote in the first column
% listing the affiliations and the copyright notice.
% The command takes one argument, which is text to display at the start of the footnote.
% The \icmlEqualContribution command is standard text for equal contribution.
% Remove it (just {}) if you do not need this facility.
% \setcounter{footnote}{*}
% \footnotetext{Equal contribution. School of Engineering, Westlake University. Correspondence to: Tailin Wu \texttt{<wutailin@westlake.edu.cn>}.}
%\printAffiliationsAndNotice{}  % leave blank if no need to mention equal contribution
\printAffiliationsAndNotice{\icmlEqualContribution} % otherwise use the standard text.

\begin{abstract}
  In this work, we present a novel technique for GPU-accelerated Boolean satisfiability (SAT) sampling. Unlike conventional sampling algorithms that directly operate on conjunctive normal form (CNF), our method transforms the logical constraints of SAT problems by factoring their CNF representations into simplified multi-level, multi-output Boolean functions. It then leverages gradient-based optimization to guide the search for a diverse set of valid solutions. Our method operates directly on the circuit structure of refactored SAT instances, reinterpreting the SAT problem as a supervised multi-output regression task. This differentiable technique enables independent bit-wise operations on each tensor element, allowing parallel execution of learning processes. As a result, we achieve GPU-accelerated sampling with significant runtime improvements ranging from $33.6\times$ to $523.6\times$ over state-of-the-art heuristic samplers. We demonstrate the superior performance of our sampling method through an extensive evaluation on $60$ instances from a public domain benchmark suite utilized in previous studies. 


  
  % Generating a wide range of diverse solutions to logical constraints is crucial in software and hardware testing, verification, and synthesis. These solutions can serve as inputs to test specific functionalities of a software program or as random stimuli in hardware modules. In software verification, techniques like fuzz testing and symbolic execution use this approach to identify bugs and vulnerabilities. In hardware verification, stimulus generation is particularly vital, forming the basis of constrained-random verification. While generating multiple solutions improves coverage and increases the chances of finding bugs, high-throughput sampling remains challenging, especially with complex constraints and refined coverage criteria. In this work, we present a novel technique that enables GPU-accelerated sampling, resulting in high-throughput generation of satisfying solutions to Boolean satisfiability (SAT) problems. Unlike conventional sampling algorithms that directly operate on conjunctive normal form (CNF), our method refines the logical constraints of SAT problems by transforming their CNF into simplified multi-level Boolean expressions. It then leverages gradient-based optimization to guide the search for a diverse set of valid solutions.
  % Our method specifically takes advantage of the circuit structure of refined SAT instances by using GD to learn valid solutions, reinterpreting the SAT problem as a supervised multi-output regression task. This differentiable technique enables independent bit-wise operations on each tensor element, allowing parallel execution of learning processes. As a result, we achieve GPU-accelerated sampling with significant runtime improvements ranging from $10\times$ to $1000\times$ over state-of-the-art heuristic samplers. Specifically, we demonstrate the superior performance of our sampling method through an extensive evaluation on $60$ instances from a public domain benchmark suite utilized in previous studies.

\end{abstract}

\begin{IEEEkeywords}
Boolean Satisfiability, Gradient Descent, Multi-level Circuits, Verification, and Testing.
\end{IEEEkeywords}
\section{Introduction}
\section{Introduction}
\label{section:introduction}

% redirection is unique and important in VR
Virtual Reality (VR) systems enable users to embody virtual avatars by mirroring their physical movements and aligning their perspective with virtual avatars' in real time. 
As the head-mounted displays (HMDs) block direct visual access to the physical world, users primarily rely on visual feedback from the virtual environment and integrate it with proprioceptive cues to control the avatar’s movements and interact within the VR space.
Since human perception is heavily influenced by visual input~\cite{gibson1933adaptation}, 
VR systems have the unique capability to control users' perception of the virtual environment and avatars by manipulating the visual information presented to them.
Leveraging this, various redirection techniques have been proposed to enable novel VR interactions, 
such as redirecting users' walking paths~\cite{razzaque2005redirected, suma2012impossible, steinicke2009estimation},
modifying reaching movements~\cite{gonzalez2022model, azmandian2016haptic, cheng2017sparse, feick2021visuo},
and conveying haptic information through visual feedback to create pseudo-haptic effects~\cite{samad2019pseudo, dominjon2005influence, lecuyer2009simulating}.
Such redirection techniques enable these interactions by manipulating the alignment between users' physical movements and their virtual avatar's actions.

% % what is hand/arm redirection, motivation of study arm-offset
% \change{\yj{i don't understand the purpose of this paragraph}
% These illusion-based techniques provide users with unique experiences in virtual environments that differ from the physical world yet maintain an immersive experience. 
% A key example is hand redirection, which shifts the virtual hand’s position away from the real hand as the user moves to enhance ergonomics during interaction~\cite{feuchtner2018ownershift, wentzel2020improving} and improve interaction performance~\cite{montano2017erg, poupyrev1996go}. 
% To increase the realism of virtual movements and strengthen the user’s sense of embodiment, hand redirection techniques often incorporate a complete virtual arm or full body alongside the redirected virtual hand, using inverse kinematics~\cite{hartfill2021analysis, ponton2024stretch} or adjustments to the virtual arm's movement as well~\cite{li2022modeling, feick2024impact}.
% }

% noticeability, motivation of predicting a probability, not a classification
However, these redirection techniques are most effective when the manipulation remains undetected~\cite{gonzalez2017model, li2022modeling}. 
If the redirection becomes too large, the user may not mitigate the conflict between the visual sensory input (redirected virtual movement) and their proprioception (actual physical movement), potentially leading to a loss of embodiment with the virtual avatar and making it difficult for the user to accurately control virtual movements to complete interaction tasks~\cite{li2022modeling, wentzel2020improving, feuchtner2018ownershift}. 
While proprioception is not absolute, users only have a general sense of their physical movements and the likelihood that they notice the redirection is probabilistic. 
This probability of detecting the redirection is referred to as \textbf{noticeability}~\cite{li2022modeling, zenner2024beyond, zenner2023detectability} and is typically estimated based on the frequency with which users detect the manipulation across multiple trials.

% version B
% Prior research has explored factors influencing the noticeability of redirected motion, including the redirection's magnitude~\cite{wentzel2020improving, poupyrev1996go}, direction~\cite{li2022modeling, feuchtner2018ownershift}, and the visual characteristics of the virtual avatar~\cite{ogawa2020effect, feick2024impact}.
% While these factors focus on the avatars, the surrounding virtual environment can also influence the users' behavior and in turn affect the noticeability of redirection.
% One such prominent external influence is through the visual channel - the users' visual attention is constantly distracted by complex visual effects and events in practical VR scenarios.
% Although some prior studies have explored how to leverage user blindness caused by visual distractions to redirect users' virtual hand~\cite{zenner2023detectability}, there remains a gap in understanding how to quantify the noticeability of redirection under visual distractions.

% visual stimuli and gaze behavior
Prior research has explored factors influencing the noticeability of redirected motion, including the redirection's magnitude~\cite{wentzel2020improving, poupyrev1996go}, direction~\cite{li2022modeling, feuchtner2018ownershift}, and the visual characteristics of the virtual avatar~\cite{ogawa2020effect, feick2024impact}.
While these factors focus on the avatars, the surrounding virtual environment can also influence the users' behavior and in turn affect the noticeability of redirection.
This, however, remains underexplored.
One such prominent external influence is through the visual channel - the users' visual attention is constantly distracted by complex visual effects and events in practical VR scenarios.
We thus want to investigate how \textbf{visual stimuli in the virtual environment} affect the noticeability of redirection.
With this, we hope to complement existing works that focus on avatars by incorporating environmental visual influences to enable more accurate control over the noticeability of redirected motions in practical VR scenarios.
% However, in realistic VR applications, the virtual environment often contains complex visual effects beyond the virtual avatar itself. 
% We argue that these visual effects can \textbf{distract users’ visual attention and thus affect the noticeability of redirection offsets}, while current research has yet taken into account.
% For instance, in a VR boxing scenario, a user’s visual attention is likely focused on their opponent rather than on their virtual body, leading to a lower noticeability of redirection offsets on their virtual movements. 
% Conversely, when reaching for an object in the center of their field of view, the user’s attention is more concentrated on the virtual hand’s movement and position to ensure successful interaction, resulting in a higher noticeability of offsets.

Since each visual event is a complex choreography of many underlying factors (type of visual effect, location, duration, etc.), it is extremely difficult to quantify or parameterize visual stimuli.
Furthermore, individuals respond differently to even the same visual events.
Prior neuroscience studies revealed that factors like age, gender, and personality can influence how quickly someone reacts to visual events~\cite{gillon2024responses, gale1997human}. 
Therefore, aiming to model visual stimuli in a way that is generalizable and applicable to different stimuli and users, we propose to use users' \textbf{gaze behavior} as an indicator of how they respond to visual stimuli.
In this paper, we used various gaze behaviors, including gaze location, saccades~\cite{krejtz2018eye}, fixations~\cite{perkhofer2019using}, and the Index of Pupil Activity (IPA)~\cite{duchowski2018index}.
These behaviors indicate both where users are looking and their cognitive activity, as looking at something does not necessarily mean they are attending to it.
Our goal is to investigate how these gaze behaviors stimulated by various visual stimuli relate to the noticeability of redirection.
With this, we contribute a model that allows designers and content creators to adjust the redirection in real-time responding to dynamic visual events in VR.

To achieve this, we conducted user studies to collect users' noticeability of redirection under various visual stimuli.
To simulate realistic VR scenarios, we adopted a dual-task design in which the participants performed redirected movements while monitoring the visual stimuli.
Specifically, participants' primary task was to report if they noticed an offset between the avatar's movement and their own, while their secondary task was to monitor and report the visual stimuli.
As realistic virtual environments often contain complex visual effects, we started with simple and controlled visual stimulus to manage the influencing factors.

% first user study, confirmation study
% collect data under no visual stimuli, different basic visual stimuli
We first conducted a confirmation study (N=16) to test whether applying visual stimuli (opacity-based) actually affects their noticeability of redirection. 
The results showed that participants were significantly less likely to detect the redirection when visual stimuli was presented $(F_{(1,15)}=5.90,~p=0.03)$.
Furthermore, by analyzing the collected gaze data, results revealed a correlation between the proposed gaze behaviors and the noticeability results $(r=-0.43)$, confirming that the gaze behaviors could be leveraged to compute the noticeability.

% data collection study
We then conducted a data collection study to obtain more accurate noticeability results through repeated measurements to better model the relationship between visual stimuli-triggered gaze behaviors and noticeability of redirection.
With the collected data, we analyzed various numerical features from the gaze behaviors to identify the most effective ones. 
We tested combinations of these features to determine the most effective one for predicting noticeability under visual stimuli.
Using the selected features, our regression model achieved a mean squared error (MSE) of 0.011 through leave-one-user-out cross-validation. 
Furthermore, we developed both a binary and a three-class classification model to categorize noticeability, which achieved an accuracy of 91.74\% and 85.62\%, respectively.

% evaluation study
To evaluate the generalizability of the regression model, we conducted an evaluation study (N=24) to test whether the model could accurately predict noticeability with new visual stimuli (color- and scale-based animations).
Specifically, we evaluated whether the model's predictions aligned with participants' responses under these unseen stimuli.
The results showed that our model accurately estimated the noticeability, achieving mean squared errors (MSE) of 0.014 and 0.012 for the color- and scale-based visual stimili, respectively, compared to participants' responses.
Since the tested visual stimuli data were not included in the training, the results suggested that the extracted gaze behavior features capture a generalizable pattern and can effectively indicate the corresponding impact on the noticeability of redirection.

% application
Based on our model, we implemented an adaptive redirection technique and demonstrated it through two applications: adaptive VR action game and opportunistic rendering.
We conducted a proof-of-concept user study (N=8) to compare our adaptive redirection technique with a static redirection, evaluating the usability and benefits of our adaptive redirection technique.
The results indicated that participants experienced less physical demand and stronger sense of embodiment and agency when using the adaptive redirection technique. 
These results demonstrated the effectiveness and usability of our model.

In summary, we make the following contributions.
% 
\begin{itemize}
    \item 
    We propose to use users' gaze behavior as a medium to quantify how visual stimuli influences the noticebility of redirection. 
    Through two user studies, we confirm that visual stimuli significantly influences noticeability and identify key gaze behavior features that are closely related to this impact.
    \item 
    We build a regression model that takes the user's gaze behavioral data as input, then computes the noticeability of redirection.
    Through an evaluation study, we verify that our model can estimate the noticeability with new participants under unseen visual stimuli.
    These findings suggest that the extracted gaze behavior features effectively capture the influence of visual stimuli on noticeability and can generalize across different users and visual stimuli.
    \item 
    We develop an adaptive redirection technique based on our regression model and implement two applications with it.
    With a proof-of-concept study, we demonstrate the effectiveness and potential usability of our regression model on real-world use cases.

\end{itemize}

% \delete{
% Virtual Reality (VR) allows the user to embody a virtual avatar by mirroring their physical movements through the avatar.
% As the user's visual access to the physical world is blocked in tasks involving motion control, they heavily rely on the visual representation of the avatar's motions to guide their proprioception.
% Similar to real-world experiences, the user is able to resolve conflicts between different sensory inputs (e.g., vision and motor control) through multisensory integration, which is essential for mitigating the sensory noise that commonly arises.
% However, it also enables unique manipulations in VR, as the system can intentionally modify the avatar's movements in relation to the user's motions to achieve specific functional outcomes,
% for example, 
% % the manipulations on the avatar's movements can 
% enabling novel interaction techniques of redirected walking~\cite{razzaque2005redirected}, redirected reaching~\cite{gonzalez2022model}, and pseudo haptics~\cite{samad2019pseudo}.
% With small adjustments to the avatar's movements, the user can maintain their sense of embodiment, due to their ability to resolve the perceptual differences.
% % However, a large mismatch between the user and avatar's movements can result in the user losing their sense of embodiment, due to an inability to resolve the perceptual differences.
% }

% \delete{
% However, multisensory integration can break when the manipulation is so intense that the user is aware of the existence of the motion offset and no longer maintains the sense of embodiment.
% Prior research studied the intensity threshold of the offset applied on the avatar's hand, beyond which the embodiment will break~\cite{li2022modeling}. 
% Studies also investigated the user's sensitivity to the offsets over time~\cite{kohm2022sensitivity}.
% Based on the findings, we argue that one crucial factor that affects to what extent the user notices the offset (i.e., \textit{noticeability}) that remains under-explored is whether the user directs their visual attention towards or away from the virtual avatar.
% Related work (e.g., Mise-unseen~\cite{marwecki2019mise}) has showcased applications where adjustments in the environment can be made in an unnoticeable manner when they happen in the area out of the user's visual field.
% We hypothesize that directing the user's visual attention away from the avatar's body, while still partially keeping the avatar within the user's field-of-view, can reduce the noticeability of the offset.
% Therefore, we conduct two user studies and implement a regression model to systematically investigate this effect.
% }

% \delete{
% In the first user study (N = 16), we test whether drawing the user's visual attention away from their body impacts the possibility of them noticing an offset that we apply to their arm motion in VR.
% We adopt a dual-task design to enable the alteration of the user's visual attention and a yes/no paradigm to measure the noticeability of motion offset. 
% The primary task for the user is to perform an arm motion and report when they perceive an offset between the avatar's virtual arm and their real arm.
% In the secondary task, we randomly render a visual animation of a ball turning from transparent to red and becoming transparent again and ask them to monitor and report when it appears.
% We control the strength of the visual stimuli by changing the duration and location of the animation.
% % By changing the time duration and location of the visual animation, we control the strengths of attraction to the users.
% As a result, we found significant differences in the noticeability of the offsets $(F_{(1,15)}=5.90,~p=0.03)$ between conditions with and without visual stimuli.
% Based on further analysis, we also identified the behavioral patterns of the user's gaze (including pupil dilation, fixations, and saccades) to be correlated with the noticeability results $(r=-0.43)$ and they may potentially serve as indicators of noticeability.
% }

% \delete{
% To further investigate how visual attention influences the noticeability, we conduct a data collection study (N = 12) and build a regression model based on the data.
% The regression model is able to calculate the noticeability of the offset applied on the user's arm under various visual stimuli based on their gaze behaviors.
% Our leave-one-out cross-validation results show that the proposed method was able to achieve a mean-squared error (MSE) of 0.012 in the probability regression task.
% }

% \delete{
% To verify the feasibility and extendability of the regression model, we conduct an evaluation study where we test new visual animations based on adjustments on scale and color and invite 24 new participants to attend the study.
% Results show that the proposed method can accurately estimate the noticeability with an MSE of 0.014 and 0.012 in the conditions of the color- and scale-based visual effects.
% Since these animations were not included in the dataset that the regression model was built on, the study demonstrates that the gaze behavioral features we extracted from the data capture a generalizable pattern of the user's visual attention and can indicate the corresponding impact on the noticeability of the offset.
% }

% \delete{
% Finally, we demonstrate applications that can benefit from the noticeability prediction model, including adaptive motion offsets and opportunistic rendering, considering the user's visual attention. 
% We conclude with discussions of our work's limitations and future research directions.
% }

% \delete{
% In summary, we make the following contributions.
% }
% % 
% \begin{itemize}
%     \item 
%     \delete{
%     We quantify the effects of the user's visual attention directed away by stimuli on their noticeability of an offset applied to the avatar's arm motion with respect to the user's physical arm. 
%     Through two user studies, we identified gaze behavioral features that are indicative of the changes in noticeability.
%     }
%     \item 
%     \delete{We build a regression model that takes the user's gaze behavioral data and the offset applied to the arm motion as input, then computes the probability of the user noticing the offset.
%     Through an evaluation study, we verified that the model needs no information about the source attracting the user's visual attention and can be generalizable in different scenarios.
%     }
%     \item 
%     \delete{We demonstrate two applications that potentially benefit from the regression model, including adaptive motion offsets and opportunistic rendering.
%     }

% \end{itemize}

\begin{comment}
However, users will lose the sense of embodiment to the virtual avatars if they notice the offset between the virtual and physical movements.
To address this, researchers have been exploring the noticing threshold of offsets with various magnitudes and proposing various redirection techniques that maintain the sense of embodiment~\cite{}.

However, when users embody virtual avatars to explore virtual environments, they encounter various visual effects and content that can attract their attention~\cite{}.
During this, the user may notice an offset when he observes the virtual movement carefully while ignoring it when the virtual contents attract his attention from the movements.
Therefore, static offset thresholds are not appropriate in dynamic scenarios.

Past research has proposed dynamic mapping techniques that adapted to users' state, such as hand moving speed~\cite{frees2007prism} or ergonomically comfortable poses~\cite{montano2017erg}, but not considering the influence of virtual content.
More specifically, PRISM~\cite{frees2007prism} proposed adjusting the C/D ratio with a non-linear mapping according to users' hand moving speed, but it might not be optimal for various virtual scenarios.
While Erg-O~\cite{montano2017erg} redirected users' virtual hands according to the virtual target's relative position to reduce physical fatigue, neglecting the change of virtual environments. 

Therefore, how to design redirection techniques in various scenarios with different visual attractions remains unknown.
To address this, we investigate how visual attention affects the noticing probability of movement offsets.
Based on our experiments, we implement a computational model that automatically computes the noticing probability of offsets under certain visual attractions.
VR application designers and developers can easily leverage our model to design redirection techniques maintaining the sense of embodiment adapt to the user's visual attention.
We implement a dynamic redirection technique with our model and demonstrate that it effectively reduces the target reaching time without reducing the sense of embodiment compared to static redirection techniques.

% Need to be refined
This paper offers the following contributions.
\begin{itemize}
    \item We investigate how visual attractions affect the noticing probability of redirection offsets.
    \item We construct a computational model to predict the noticing probability of an offset with a given visual background.
    \item We implement a dynamic redirection technique adapting to the visual background. We evaluate the technique and develop three applications to demonstrate the benefits. 
\end{itemize}



First, we conducted a controlled experiment to understand how users perceived the movement offset while subjected to various distractions.
Since hand redirection is one of the most frequently used redirections in VR interactions, we focused on the dynamic arm movements and manually added angular offsets to the' elbow joint~\cite{li2022modeling, gonzalez2022model, zenner2019estimating}. 
We employed flashing spheres in the user's field of view as distractions to attract users' visual attention.
Participants were instructed to report the appearing location of the spheres while simultaneously performing the arm movements and reporting if they perceived an offset during the movement. 
(\zhipeng{Add the results of data collection. Analyze the influence of the distance between the gaze map and the offset.}
We measured the visual attraction's magnitude with the gaze distribution on it.
Results showed that stronger distractions made it harder for users to notice the offset.)
\zhipeng{Need to rewrite. Not sure to use gaze distribution or a metric obtained from the visual content.}
Secondly, we constructed a computational model to predict the noticing probability of offsets with given visual content.
We analyzed the data from the user studies to measure the influence of visual attractions on the noticing probability of offsets.
We built a statistical model to predict the offset's noticing probability with a given visual content.
Based on the model, we implement a dynamic redirection technique to adjust the redirection offset adapted to the user's current field of view.
We evaluated the technique in a target selection task compared to no hand redirection and static hand redirection.
\zhipeng{Add the results of the evaluation.}
Results showed that the dynamic hand redirection technique significantly reduced the target selection time with similar accuracy and a comparable sense of embodiment.
Finally, we implemented three applications to demonstrate the potential benefits of the visual attention adapted dynamic redirection technique.
\end{comment}

% This one modifies arm length, not redirection
% \citeauthor{mcintosh2020iteratively} proposed an adaptation method to iteratively change the virtual avatar arm's length based on the primary tasks' performance~\cite{mcintosh2020iteratively}.



% \zhipeng{TO ADD: what is redirection}
% Redirection enables novel interactions in Virtual Reality, including redirected walking, haptic redirection, and pseudo haptics by introducing an offset to users' movement.
% \zhipeng{TO ADD: extend this sentence}
% The price of this is that users' immersiveness and embodiment in VR can be compromised when they notice the offset and perceive the virtual movement not as theirs~\cite{}.
% \zhipeng{TO ADD: extend this sentence, elaborate how the virtual environment attracts users' attention}
% Meanwhile, the visual content in the virtual environment is abundant and consistently captures users' attention, making it harder to notice the offset~\cite{}.
% While previous studies explored the noticing threshold of the offsets and optimized the redirection techniques to maintain the sense of embodiment~\cite{}, the influence of visual content on the probability of perceiving offsets remains unknown.  
% Therefore, we propose to investigate how users perceive the redirection offset when they are facing various visual attractions.


% We conducted a user study to understand how users notice the shift with visual attractions.
% We used a color-changing ball to attract the user's attention while instructing users to perform different poses with their arms and observe it meanwhile.
% \zhipeng{(Which one should be the primary task? Observe the ball should be the primary one, but if the primary task is too simple, users might allocate more attention on the secondary task and this makes the secondary task primary.)}
% \zhipeng{(We need a good and reasonable dual-task design in which users care about both their pose and the visual content, at least in the evaluation study. And we need to be able to control the visual content's magnitude and saliency maybe?)}
% We controlled the shift magnitude and direction, the user's pose, the ball's size, and the color range.
% We set the ball's color-changing interval as the independent factor.
% We collect the user's response to each shift and the color-changing times.
% Based on the collected data, we constructed a statistical model to describe the influence of visual attraction on the noticing probability.
% \zhipeng{(Are we actually controlling the attention allocation? How do we measure the attracting effect? We need uniform metrics, otherwise it is also hard for others to use our knowledge.)}
% \zhipeng{(Try to use eye gaze? The eye gaze distribution in the last five seconds to decide the attention allocation? Basically constructing a model with eye gaze distribution and noticing probability. But the user's head is moving, so the eye gaze distribution is not aligned well with the current view.)}

% \zhipeng{Saliency and EMD}
% \zhipeng{Gaze is more than just a point: Rethinking visual attention
% analysis using peripheral vision-based gaze mapping}

% Evaluation study(ideal case): based on the visual content, adjusting the redirection magnitude dynamically.

% \zhipeng{(The risk is our model's effect is trivial.)}

% Applications:
% Playing Lego while watching demo videos, we can accelerate the reaching process of bricks, and forbid the redirection during the manipulation.

% Beat saber again: but not make a lot of sense? Difficult game has complicated visual effects, while allows larger shift, but do not need large shift with high difficulty




\section{Related Work}
\section{Related Work}
\label{lit_review}

\begin{highlight}
{

Our research builds upon {\em (i)} Assessing Web Accessibility, {\em (ii)} End-User Accessibility Repair, and {\em (iii)} Developer Tools for Accessibility.

\subsection{Assessing Web Accessibility}
From the earliest attempts to set standards and guidelines, web accessibility has been shaped by a complex interplay of technical challenges, legal imperatives, and educational campaigns. Over the past 25 years, stakeholders have sought to improve digital inclusion by establishing foundational standards~\cite{chisholm2001web, caldwell2008web}, enforcing legal obligations~\cite{sierkowski2002achieving, yesilada2012understanding}, and promoting a broader culture of accessibility awareness among developers~\cite{sloan2006contextual, martin2022landscape, pandey2023blending}. 
Despite these longstanding efforts, systemic accessibility issues persist. According to the 2024 WebAIM Million report~\cite{webaim2024}, 95.9\% of the top one million home pages contained detectable WCAG violations, averaging nearly 57 errors per page. 
These errors take many forms: low color contrast makes the interface difficult for individuals with color deficiency or low vision to read text; missing alternative text leaves users relying on screen readers without crucial visual context; and unlabeled form inputs or empty links and buttons hinder people who navigate with assistive technologies from completing basic tasks. 
Together, these accessibility issues not only limit user access to critical online resources such as healthcare, education, and employment but also result in significant legal risks and lost opportunities for businesses to engage diverse audiences. Addressing these pervasive issues requires systematic methods to identify, measure, and prioritize accessibility barriers, which is the first step toward achieving meaningful improvements.

Prior research has introduced methods blending automation and human evaluation to assess web accessibility. Hybrid approaches like SAMBA combine automated tools with expert reviews to measure the severity and impact of barriers, enhancing evaluation reliability~\cite{brajnik2007samba}. Quantitative metrics, such as Failure Rate and Unified Web Evaluation Methodology, support large-scale monitoring and comparative analysis, enabling cost-effective insights~\cite{vigo2007quantitative, martins2024large}. However, automated tools alone often detect less than half of WCAG violations and generate false positives, emphasizing the need for human interpretation~\cite{freire2008evaluation, vigo2013benchmarking}. Recent progress with large pretrained models like Large Language Models (LLMs)~\cite{dubey2024llama,bai2023qwen} and Large Multimodal Models (LMMs)~\cite{liu2024visual, bai2023qwenvl} offers a promising step forward, automating complex checks like non-text content evaluation and link purposes, achieving higher detection rates than traditional tools~\cite{lopez2024turning, delnevo2024interaction}. Yet, these large models face challenges, including dependence on training data, limited contextual judgment, and the inability to simulate real user experiences. These limitations underscore the necessity of combining models with human oversight for reliable, user-centered evaluations~\cite{brajnik2007samba, vigo2013benchmarking, delnevo2024interaction}. 

Our work builds on these prior efforts and recent advancements by leveraging the capabilities of large pretrained models while addressing their limitations through a developer-centric approach. CodeA11y integrates LLM-powered accessibility assessments, tailored accessibility-aware system prompts, and a dedicated accessibility checker directly into GitHub Copilot---one of the most widely used coding assistants. Unlike standalone evaluation tools, CodeA11y actively supports developers throughout the coding process by reinforcing accessibility best practices, prompting critical manual validations, and embedding accessibility considerations into existing workflows.
% This pervasive shortfall reflects the difficulty of scaling traditional approaches---such as manual audits and automated tools---that either demand immense human effort or lack the nuanced understanding needed to capture real-world user experiences. 
%
% In response, a new wave of AI-driven methods, many powered by large language models (LLMs), is emerging to bridge these accessibility detection and assessment gaps. Early explorations, such as those by Morillo et al.~\cite{morillo2020system}, introduced AI-assisted recommendations capable of automatic corrections, illustrating how computational intelligence can tackle the repetitive, common errors that plague large swaths of the web. Building on this foundation, Huang et al.~\cite{huang2024access} proposed ACCESS, a prompt-engineering framework that streamlines the identification and remediation of accessibility violations, while López-Gil et al.~\cite{lopez2024turning} demonstrated how LLMs can help apply WCAG success criteria more consistently---reducing the reliance on manual effort. Beyond these direct interventions, recent work has also begun integrating user experiences more seamlessly into the evaluation process. For example, Huq et al.~\cite{huq2024automated} translate user transcripts and corresponding issues into actionable test reports, ensuring that accessibility improvements align more closely with authentic user needs.
% However, as these AI-driven solutions evolve, researchers caution against uncritical adoption. Othman et al.~\cite{othman2023fostering} highlight that while LLMs can accelerate remediation, they may also introduce biases or encourage over-reliance on automated processes. Similarly, Delnevo et al.~\cite{delnevo2024interaction} emphasize the importance of contextual understanding and adaptability, pointing to the current limitations of LLM-based systems in serving the full spectrum of user needs. 
% In contrast to this backdrop, our work introduces and evaluates CodeA11y, an LLM-augmented extension for GitHub Copilot that not only mitigates these challenges by providing more consistent guidance and manual validation prompts, but also aligns AI-driven assistance with developers’ workflows, ultimately contributing toward more sustainable propulsion for building accessible web.

% Broader implications of inaccessibility—legal compliance, ethical concerns, and user experience
% A Historical Review of Web Accessibility Using WAVE
% "I tend to view ads almost like a pestilence": On the Accessibility Implications of Mobile Ads for Blind Users

% In the research domain, several methods have been developed to assess and enhance web accessibility. These include incorporating feedback into developer tools~\cite{adesigner, takagi2003accessibility, bigham2010accessibility} and automating the creation of accessibility tests and reports for user interfaces~\cite{swearngin2024towards, taeb2024axnav}. 

% Prior work has also studied accessibility scanners as another avenue of AI to improve web development practices~\cite{}.
% However, a persistent challenge is that developers need to be aware of these tools to utilize them effectively. With recent advancements in LLMs, developers might now build accessible websites with less effort using AI assistants. However, the impact of these assistants on the accessibility of their generated code remains unclear. This study aims to investigate these effects.

\subsection{End-user Accessibility Repair}
In addition to detecting accessibility errors and measuring web accessibility, significant research has focused on fixing these problems.
Since end-users are often the first to notice accessibility problems and have a strong incentive to address them, systems have been developed to help them report or fix these problems.

Collaborative, or social accessibility~\cite{takagi2009collaborative,sato2010social}, enabled these end-user contributions to be scaled through crowd-sourcing.
AccessMonkey~\cite{bigham2007accessmonkey} and Accessibility Commons~\cite{kawanaka2008accessibility} were two examples of repositories that store accessibility-related scripts and metadata, respectively.
Other work has developed browser extensions that leverage crowd-sourced databases to automatically correct reading order, alt-text, color contrast, and interaction-related issues~\cite{sato2009s,huang2015can}.

One drawback of collaborative accessibility approaches is that they cannot fix problems for an ``unseen'' web page on-demand, so many projects aim to automatically detect and improve interfaces without the need for an external source of fixes.
A large body of research has focused on making specific web media (e.g., images~\cite{gleason2019making,guinness2018caption, twitterally, gleason2020making, lee2021image}, design~\cite{potluri2019ai,li2019editing, peng2022diffscriber, peng2023slide}, and videos~\cite{pavel2020rescribe,peng2021say,peng2021slidecho,huh2023avscript}) accessible through a combination of machine learning (ML) and user-provided fixes.
Other work has focused on applying more general fixes across all websites.

Opportunity accessibility addressed a common accessibility problem of most websites: by default, content is often hard to see for people with visual impairments, and many users, especially older adults, do not know how to adjust or enable content zooming~\cite{bigham2014making}.
To this end, a browser script (\texttt{oppaccess.js}) was developed that automatically adjusted the browser's content zoom to maximally enlarge content without introducing adverse side-effects (\textit{e.g.,} content overlap).
While \texttt{oppaccess.js} primarily targeted zoom-related accessibility, recent work aimed to enable larger types of changes, by using LLMs to modify the source code of web pages based on user questions or directives~\cite{li2023using}.

Several efforts have been focused on improving access to desktop and mobile applications, which present additional challenges due to the unavailability of app source code (\textit{e.g.,} HTML).
Prefab is an approach that allows graphical UIs to be modified at runtime by detecting existing UI widgets, then replacing them~\cite{dixon2010prefab}.
Interaction Proxies used these runtime modification strategies to ``repair'' Android apps by replacing inaccessible widgets with improved alternatives~\cite{zhang2017interaction, zhang2018robust}.
The widget detection strategies used by these systems previously relied on a combination of heuristics and system metadata (\textit{e.g.,} the view hierarchy), which are incomplete or missing in the accessible apps.
To this end, ML has been employed to better localize~\cite{chen2020object} and repair UI elements~\cite{chen2020unblind,zhang2021screen,wu2023webui,peng2025dreamstruct}.

In general, end-user solutions to repairing application accessibility are limited due to the lack of underlying code and knowledge of the semantics of the intended content.

\subsection{Developer Tools for Accessibility}
Ultimately, the best solution for ensuring an accessible experience lies with front-end developers. Many efforts have focused on building adequate tooling and support to help developers with ensuring that their UI code complies with accessibility standards.

Numerous automated accessibility testing tools have been created to help developers identify accessibility issues in their code: i) static analysis tools, such as IBM Equal Access Accessibility Checker~\cite{ibm2024toolkit} or Microsoft Accessibility Insights~\cite{accessibilityinsights2024}, scan the UI code's compliance with predefined rules derived from accessibility guidelines; and ii) dynamic or runtime accessibility scanners, such as Chrome Devtools~\cite{chromedevtools2024} or axe-Core Accessibility Engine~\cite{deque2024axe}, perform real-time testing on user interfaces to detect interaction issues not identifiable from the code structure. While these tools greatly reduce the manual effort required for accessibility testing, they are often criticized for their limited coverage. Thus, experts often recommend manually testing with assistive technologies to uncover more complex interaction issues. Prior studies have created accessibility crawlers that either assist in developer testing~\cite{swearngin2024towards,taeb2024axnav} or simulate how assistive technologies interact with UIs~\cite{10.1145/3411764.3445455, 10.1145/3551349.3556905, 10.1145/3544548.3580679}.

Similar to end-user accessibility repair, research has focused on generating fixes to remediate accessibility issues in the UI source code. Initial attempts developed heuristic-based algorithms for fixing specific issues, for instance, by replacing text or background color attributes~\cite{10.1145/3611643.3616329}. More recent work has suggested that the code-understanding capabilities of LLMs allow them to suggest more targeted fixes.
For example, a study demonstrated that prompting ChatGPT to fix identified WCAG compliance issues in source code could automatically resolve a significant number of them~\cite{othman2023fostering}. Researchers have sought to leverage this capability by employing a multi-agent LLM architecture to automatically identify and localize issues in source code and suggest potential code fixes~\cite{mehralian2024automated}.

While the approaches mentioned above focus on assessing UI accessibility of already-authored code (\textit{i.e.,} fixing existing code), there is potential for more proactive approaches.
For example, LLMs are often used by developers to generate UI source code from natural language descriptions or tab completions~\cite{chen2021evaluating,GitHubCopilot,lozhkov2024starcoder,hui2024qwen2,roziere2023code,zheng2023codegeex}, but LLMs frequently produce inaccessible code by default~\cite{10.1145/3677846.3677854,mowar2024tab}, leading to inaccessible output when used by developers without sufficient awareness of accessibility knowledge.
The primary focus of this paper is to design a more accessibility-aware coding assistant that both produces more accessible code without manual intervention (\textit{e.g.,} specific user prompting) and gradually enables developers to implement and improve accessibility of automatically-generated code through IDE UI modifications (\textit{e.g.}, reminder notifications).

}
\end{highlight}



% Work related to this paper includes {\em (i)} Web Accessibility and {\em (ii)} Developer Practices in AI-Assisted Programming.

% \ipstart{Web Accessibility: Practice, Evaluation, and Improvements} Substantial efforts have been made to set accessibility standards~\cite{chisholm2001web, caldwell2008web}, establish legal requirements~\cite{sierkowski2002achieving, yesilada2012understanding}, and promote education and advocacy among developers~\cite{sloan2006contextual, martin2022landscape, pandey2023blending}. In the research domain, several methods have been developed to assess and enhance web accessibility. These include incorporating feedback into developer tools~\cite{adesigner, takagi2003accessibility, bigham2010accessibility} and automating the creation of accessibility tests and reports for user interfaces~\cite{swearngin2024towards, taeb2024axnav}. 
% % Prior work has also studied accessibility scanners as another avenue of AI to improve web development practices~\cite{}.
% However, a persistent challenge is that developers need to be aware of these tools to utilize them effectively. With recent advancements in LLMs, developers might now build accessible websites with less effort using AI assistants. However, the impact of these assistants on the accessibility of their generated code remains unclear. This study aims to investigate these effects.

% \ipstart{Developer Practices in AI-Assisted Programming}
% Recent usability research on AI-assisted development has examined the interaction strategies of developers while using AI coding assistants~\cite{barke2023grounded}.
% They observed developers interacted with these assistants in two modes -- 1) \textit{acceleration mode}: associated with shorter completions and 2) \textit{exploration mode}: associated with long completions.
% Liang {\em et al.} \cite{liang2024large} found that developers are driven to use AI assistants to reduce their keystrokes, finish tasks faster, and recall the syntax of programming languages. On the other hand, developers' reason for rejecting autocomplete suggestions was the need for more consideration of appropriate software requirements. This is because primary research on code generation models has mainly focused on functional correctness while often sidelining non-functional requirements such as latency, maintainability, and security~\cite{singhal2024nofuneval}. Consequently, there have been increasing concerns about the security implications of AI-generated code~\cite{sandoval2023lost}. Similarly, this study focuses on the effectiveness and uptake of code suggestions among developers in mitigating accessibility-related vulnerabilities. 


% ============================= additional rw ============================================
% - Paulina Morillo, Diego Chicaiza-Herrera, and Diego Vallejo-Huanga. 2020. System of Recommendation and Automatic Correction of Web Accessibility Using Artificial Intelligence. In Advances in Usability and User Experience, Tareq Ahram and Christianne Falcão (Eds.). Springer International Publishing, Cham, 479–489
% - Juan-Miguel López-Gil and Juanan Pereira. 2024. Turning manual web accessibility success criteria into automatic: an LLM-based approach. Universal Access in the Information Society (2024). https://doi.org/10.1007/s10209-024-01108-z
% - s
% - Calista Huang, Alyssa Ma, Suchir Vyasamudri, Eugenie Puype, Sayem Kamal, Juan Belza Garcia, Salar Cheema, and Michael Lutz. 2024. ACCESS: Prompt Engineering for Automated Web Accessibility Violation Corrections. arXiv:2401.16450 [cs.HC] https://arxiv.org/abs/2401.16450
% - Syed Fatiul Huq, Mahan Tafreshipour, Kate Kalcevich, and Sam Malek. 2025. Automated Generation of Accessibility Test Reports from Recorded User Transcripts. In Proceedings of the 47th International Conference on Software Engineering (ICSE) (Ottawa, Ontario, Canada). IEEE. https://ics.uci.edu/~seal/publications/2025_ICSE_reca11.pdf To appear in IEEE Xplore
% - Achraf Othman, Amira Dhouib, and Aljazi Nasser Al Jabor. 2023. Fostering websites accessibility: A case study on the use of the Large Language Models ChatGPT for automatic remediation. In Proceedings of the 16th International Conference on PErvasive Technologies Related to Assistive Environments (Corfu, Greece) (PETRA ’23). Association for Computing Machinery, New York, NY, USA, 707–713. https://doi.org/10.1145/3594806.3596542
% - Zsuzsanna B. Palmer and Sushil K. Oswal. 0. Constructing Websites with Generative AI Tools: The Accessibility of Their Workflows and Products for Users With Disabilities. Journal of Business and Technical Communication 0, 0 (0), 10506519241280644. https://doi.org/10.1177/10506519241280644
% ============================= additional rw ============================================
\section{Preliminary}
\section{Background}
\label{sec:background}

To present our work systematically, we formulate it as a two-level hierarchical solution of Markov Decision Processes (MDPs), corresponding to the two-phase pipeline. In contrast to some hierarchical methods targeting action levels \citep{mcgovern1998hierarchical, hauskrecht2013hierarchical}, we focus on the level of reward functions.

\textbf{Low-level MDP of Controlling Problem}
Specifically, the scene reconstruction phase is to construct the MDP $\mathcal{G} = \langle \mathcal{S}, \mathcal{A}, T, \mathcal{R}_{0|1} \rangle$ from videos, where $\mathcal{S} \in \mathbb{R}^m$ represents the states of the environment, $\mathcal{A}$ is the action space of the agent, and $T$ is the transition probability function. 
$\mathcal{R}_{0|1}$ is the 0-1 reward function that distinguish whether the trajectory is successful. $\mathcal{R}_{0|1}$ can be a scalar evaluation function. 
To solve this MDP, we will train a policy $\pi$ through reinforcement learning in the Isaac Gym simulation. To achieve high performance, we leverage the LLM to sample various reward functions to learn policies in a high-level manner, based on the evaluation results from $\mathcal{R}_{0|1}$ and some CoT \citep{wei2022chain} instructions.

\textbf{High-level MDP of Reward Designs} The aim of reward design is to create a shaped reward function that simplifies the optimization of a challenging given reward function, such as the sparse 0-1 reward function $\mathcal{R}_{0|1}$. Following the definition of Reward Design Problem (RDP) from previous works \citep{singh2009rewards, ma2023eureka}, we consider a high-level MDP $\hat{\mathcal{G}} = \langle \hat{\mathcal{S}}, \hat{\mathcal{A}}, \hat{T}, F \rangle$. Here, $\hat{\mathcal{A}}$ is the space of reward functions. Each time we choose an action $\hat{R} \in \mathcal{A}$ in the MDP $\hat{\mathcal{G}}$, we will train a policy $\pi$ by RL for the low-level MDP $\mathcal{G} = \langle \mathcal{S}, \mathcal{A}, T, \hat{\mathcal{R}} + \mathcal{R}_{0|1} \rangle$. The horizon of the MDP $\hat{\mathcal{G}}$ is the iteration number in the second phase.
$\hat{\mathcal{S}}$ includes the training and evaluation information during RL and the policy model $\pi$. $\hat{T}$ is the state transition function, and $F$ is the reward function that produces a scalar evaluation of any policy $\pi$. Specifically, $F$ is equal to $\mathcal{R}_{0|1}$. Thus, the high-level MDP's goal is to find a reward function $\hat{\mathcal{R}} \in \hat{A}$ to maximize the success rates of the low-level policies.

\section{Method}
\section{The E-3DGS Method}\label{sec:Method} 
Our aim is to learn a 3D representation of a static scene using only a color event stream, where each pixel observes changes in brightness corresponding to one of the red, green, or blue channels according to a Bayer pattern, with known camera intrinsics $K_t~\in~\mathbb{R}^{3 \times 3}$, and noisy initial poses~$P_t~\in~\mathbb{R}^{3 \times 4}$, at reasonably high-frequency time steps indexed by $t$. 
Following 3DGS~\cite{3dgs}, we represent our scene by anisotropic 3D Gaussians. Our methodology comprises a technique to initialize Gaussians in the absence of a Structure from Motion (SfM) point cloud, adaptive event frame supervision of 3DGS, and a pose refinement module. 
An overview of our method is provided in Fig.~\ref{fig:methodology}.


Our E-3DGS method is not restricted to scenes of a certain size and can handle unbounded environments. It does not rely on any assumptions regarding the background color, type of camera motion, or speed. Thus, it ensures robust performance across a wide range of scenarios. 

\subsection{Event Stream Supervision} 

There are two main categories of approaches to learning 3D scene representations from event streams. 
Some apply the loss to single events~\cite{robust_enerf} based on Eq.~\eqref{eq:egm}. Others use the sum of events~\( E_{\x}(t_1,t_2) \) from Eq.~\eqref{eq:egm_sum}. We choose the second approach, as rasterization in 3DGS is well suited to efficiently render entire images rather than individual pixels. 

To optimize our Gaussian scene representation using event data, we can make a logical equivalence between the observed event stream and the scene renderings. 
To do so, we replace the true logarithmic intensities~\( L_{\x} \) in Eq.~\eqref{eq:egm_sum} with the rendered logarithmic intensities~\(\hat{L}_{\x} \) from our scene, and the times $\tau$ with the camera poses $P_t$ that were used to render the scene at the respective time steps. 
Following the approach used in~\cite{eventnerf}, the log difference is then point-wise multiplied with a Bayer filter $F$ to obtain the respective color channel. We can finally calculate the error between the logarithmic change from our model and the actual change observed from the event stream, and define the following per-pixel loss: 
\begin{equation}
    \begin{split}
    &\mathcal{L}_{\x}\left(t_1, t_2\right) = \\
    &\left\| 
    F \odot (\hat{L}_{\x}(P_{t_2}) - \hat{L}_{\x}(P_{t_1})) 
    - F \odot E_{\x}\left(t_1, t_2\right)\right\|_1, 
    \end{split}
    \label{eq:L_recon_per_pixel} 
\end{equation}
where ``$\odot$'' denotes pixelwise multiplication. 


\subsection{Frustum-Based Initialization}
\label{sec:frustum_init}

In the original 3DGS \cite{3dgs}, the Gaussians are initialized using a point cloud obtained from applying SfM on the input images. 
The authors also experimented with initializing the Gaussians at random locations within a cube. While this worked for them with a slight performance drop, it requires an assumption about the extent of the scene. 

Applying SfM directly to event streams is more challenging than RGB inputs \cite{Kim2016} and exploring this aspect is not the primary focus of this paper. 
In the absence of an SfM point cloud, we use the randomly initialized Gaussians and extend this approach to unbounded scenes. 
To this end, we initialize a specified number of Gaussians (on the order of \qty{d4}{}) in the frustum of each camera. 
This gives two benefits: 1) All the initialized Gaussians are within the observable area, and 2) We only need one loose assumption about the scene, which is the maximum depth $z_\mathrm{far}$. 


\subsection{Adaptive Event Window}\label{subsec:adaptive_window}

Rudnev et al.~\cite{eventnerf} demonstrated in EventNeRF that using a fixed event window duration results in suboptimal reconstruction. They find that larger windows are essential for capturing low-frequency color and structure, and smaller ones are essential for optimization of finer high-frequency details. While they randomly sampled the event window duration, a drawback is that it does not consider the camera speed and event rate, thus the sampled windows may contain too many or too few events.  
As our dataset features variable camera speeds, we improve upon this by sampling the number of events rather than the window duration.  
To achieve this, for each time step we randomly sample a target number of events from within the range $[N_\mathrm{min}, N_\mathrm{max}]$. 
Given a time step~$t$, we search for a previous time step~$t_s$ such that the number of events in the event frame $E(t_s, t)$ is approximately equal to the desired number. 

When determining $N_\mathrm{max}$, we find that for values where details and low-frequency structure are optimal, 3DGS tends to get unstable and sometimes prunes away Gaussians in homogeneous areas.
While this can be mitigated by choosing a much larger $N_\mathrm{max}$, this again deteriorates the details. 
Therefore, we propose a strategy to incorporate both, small and large windows. For each $t$, we choose two earlier time steps~$t_{s_1}$ and~$t_{s_2}$. The ranges for sampling the event counts for both are empirically chosen to be $[\frac{N_\mathrm{max}}{10}, N_\mathrm{max}]$ and $[\frac{N_{max}}{300}, \frac{N_\mathrm{max}}{30}]$. We then render frames from our model at times $t$, $t_{s_1}$ and $t_{s_2}$, and use two concurrent losses for the event windows $E_{\x}\left(t_{s_1}, t\right)$ and $E_{\x}\left(t_{s_2}, t\right)$. 

\subsection{As-Isotropic-As-Possible Regularization} 
\label{ssec:IsotropicReg} 

In 3DGS, Gaussians are unconstrained in the direction perpendicular to the image plane. 
This lack of constraint can result in elongated and overfitted Gaussians. 
And while they may appear correct from the training views, they introduce significant artifacts when rendered from novel views by manifesting as floaters and distortions of object surfaces. 
We also observe that the lack of multi-view consistency and tendency to overfit destabilize the pose refinement. 

To mitigate these issues, we draw inspiration from Gaussian Splatting SLAM~\cite{3dgsslam} and SplaTAM~\cite{splatam}, and apply isotropic regularization:
\begin{equation}
    \mathcal{L}_{\text{iso}} = \frac{1}{|\mathcal{G}|} \sum_{g \in \mathcal{G}} \left\| S_g - \bar{S}_g \right\|_1
    \label{eq:L_iso} \mathrm{\,,}
\end{equation}
where~$\mathcal{G}$ is the set of Gaussians visible in the image. Eq.~\eqref{eq:L_iso} imposes a soft constraint on the Gaussians to be as isotropic as possible.
We find that it helps to improve pose refinement, minimizes floaters and enhances generalizability. 

\subsection{Pose Refinement} 
\label{sec:pose_refinement}

To obtain the most accurate results, we allow the poses to be refined during optimization
by modeling the refined pose as $P'_t = P^e_t P_t$, where  $P^e_t$ is an error correction transform. 
Instead of directly optimizing~$P^e_t$ as a~$3 \times 3$ matrix, following Hempel et al.~\cite{6d_rotation} we represent it as $[r_1\,\, r_2\,\, T]$, where $r_1$ and $r_2$ represent two rotation vectors of the rotation matrix~$R = [r_1\,\, r_2\,\, r_3]$, while $T$ is the translation.
We can then obtain the~$P^e_t$ matrix from the representation using Gram-Schmidt orthogonalization (see details in Supplement~\ref{sec:supp_pose_refinement}), hence ensuring that during optimization, our error correction transform always represents a valid transformation matrix. 
$P^e_t$ is initialized to be the identity transform. Since the loss function from Eq.~\eqref{eq:L_recon_per_pixel} depends on the camera pose as well, it allows us to use the same loss to backpropagate and obtain gradients for pose refinement. 

As our goal is to refine the estimated noisy poses rather than perform SLAM, this training signal is sufficient for our needs. Moreover, we observe that poses tend to diverge with 3DGS due to the periodic opacity reset.
To combat this, we impose a soft constraint with an additional pose regularization, that encourages the matrices~$P^e_t$ to stay close to the identity matrix $I$:
\begin{equation}
    \mathcal{L}_{\text{pose}} = \| P^e_{t_{s_1}} - I \|_2 + \| P^e_{t_{s_2}} - I \|_2 + \| P^e_{t} - I \|_2
    \label{eq:L_pose} \mathrm{\,,}
\end{equation}
with all terms weighted equally. 


\subsection{Optimization}
\label{ssec:Optimization} 

Eq.~\eqref{eq:L_recon_per_pixel} defines the reconstruction loss per pixel for a single event frame. However, naively averaging these per-pixel losses over whole images leads to problems. For small event windows, most pixels have no events, which are not very informative but will then make up the majority of the loss. 
To address this, we compute separate averages of the losses for pixels with events~$\mathcal{X}_\text{evs}$ and pixels without events~$\mathcal{X}_\text{noevs}$. 
These averages are then scaled by the hyperparameter~$\alpha=0.3$ to obtain the complete weighted reconstruction loss:
\begin{equation}
    \begin{split}
        \mathcal{L}_{\text{recon}}\left(t_s, t\right) = \,\,&
        \frac{\alpha}{|\mathcal{X}_{\text{noevs}}|} \cdot 
        \left(\sum_{\x\in \mathcal{X}_{\text{noevs}}} \mathcal{L}_{\x}\left(t_s, t\right)\right) + \\
        + \,\,& \,\, \frac{1 - \alpha}{|\mathcal{X}_{\text{evs}}|} \,\,\, \cdot 
        \left(\sum_{\x\in \mathcal{X}_{\text{evs}}} \mathcal{L}_{\x}\left(t_s, t\right)\right). 
    \end{split}
    \label{eq:L_recon}
\end{equation}
To obtain the final loss, we take a weighted sum of the reconstruction losses for the two event windows from Sec.~\ref{subsec:adaptive_window} along with the isotropic and pose regularization: 
\begin{equation}
    \begin{split}
        \mathcal{L} =\,\,\,\, & 
        \lambda_1 \mathcal{L}_{\text{recon}}\left(t_{s_1}, t\right) \,\,+  \,\,
        \lambda_2 \mathcal{L}_{\text{recon}}\left(t_{s_2}, t\right)  \\&
        +\,\, \lambda_\text{iso} \mathcal{L}_{\text{iso}} \,\,+ \,\,
        \lambda_\text{pose} \mathcal{L}_{\text{pose}}
    \end{split}
    \label{eq:loss} \mathrm{\,,}
\end{equation}
where $\lambda_1$, $\lambda_2$ and $\lambda_{\text{iso}}$ are hyper-parameters. In our experiments, we use  $\lambda_1=\lambda_2=0.65$, and $\lambda_{\text{iso}}$ is set to $10$ initially and reduced to $1$ after $\qty{d4}{}$ iterations. 






\section{Experiments}
\section{Experimental Evaluation}\label{section:experiments}
We already achieved our primary objective of deriving time-series-specific subsampling guarantees for DP-SGD adapted to forecasting.
Our main goal for this section is to investigate the trade-offs we discovered in discussing these guarantees.
In addition, we train common probabilistic forecasting architectures on standard datasets to verify the feasibility of training deep differentially private forecasting models while retaining meaningful utility.
The full experimental setup  is described in~\cref{appendix:experimental_setup}.
%An implementation will be made available upon publication.

\subsection{Trade-Offs in Structured Subsampling}

\begin{figure}
    \vskip 0.2in
    \centering
        \includegraphics[width=0.99\linewidth]{figures/experiments/eval_pld_deterministic_vs_random_top_level/daily_20_32_main.pdf}
        \vskip -0.3cm
        \caption{Top-level deterministic iteration (\cref{theorem:deterministic_top_level_wr}) vs top-level WOR sampling (\cref{theorem:wor_top_level_wr}) for $\numinstances=1$.
        Sampling is more private despite requiring more compositions.}
        \label{fig:deterministic_vs_random_top_level_daily_main}
    \vskip -0.2in
\end{figure}




For the following experiments, we assume that we have $N=320$ sequences, batch size $\batchsize = 32$, and noise scale $\sigma = 1$.
We further assume $L=10  (L_F + L_C) + L_F - 1$, so that 
the chance of bottom-level sampling a subsequence containing any specific element is 
$r=0.1$ when choosing $\numinstances = 1$ as the number of subsequences.
In~\cref{appendix:extra_experiments_eval_pld}, we repeat all experiments with a wider range of parameters.
All results are consistent with the ones shown here.

\textbf{Number of Subsequences $\bm{\numinstances}$.}
Let us begin with a trade-off inherent to bi-level subsampling:
We can achieve the same batch size $\batchsize$ with different $\numinstances$, each
leading to different top- and bottom-level amplification.
We claim that $\numinstances = 1$ (i.e., maximum bottom-level amplification) is preferable.
For a fair comparison, we compare our provably tight guarantee for $\numinstances=1$ (\cref{theorem:wor_top_level_wr})
with optimistic lower bounds for $\numinstances > 1$ (\cref{theorem:wor_top_wr_bottom_upper})
instead of our sound upper bounds (\cref{theorem:wor_top_level_wr_general}), i.e.,
we make the competitors stronger.
As shown in~\cref{fig:monotonicity_daily_main}, $\numinstances = 1$ only has smaller $\delta(\epsilon)$ for $\epsilon \geq 10^{-1}$ when considering a single training step.
However, after $100$-fold composition, $\numinstances = 1$ achieves smaller $\delta(\epsilon)$ even in $[10^{-3}, 10^{-1}]$ (see~\cref{fig:monotonicity_composed_daily_main}).
Our explanation is that $\numinstances > 1$ results in larger $\delta(\epsilon)$ for large $\epsilon$, i.e., is more likely to have a large privacy loss.
Because the privacy loss of a composed mechanism is the sum of component privacy losses~\cite{sommer2018privacy}, this is problematic when performing multiple training steps.
We shall thus later use $\numinstances=1$ for training.

%Intuitively, $\delta(\epsilon)$ can be interpreted as the probability that the log-likelihood ratio of $M_x$ and $M_{x'}$ (``privacy loss'') exceeds $\epsilon$.\footnote{For the formal relation between privay loss and privacy profiles, see~\cref{lemma:profile_from_pld} taken from~\cite{balle2018improving}}


\textbf{Step- vs Epoch-Level Accounting.}
Next, we show the benefit of top-level sampling sequences (\cref{theorem:wor_top_level_wr}) instead of deterministically iterating over them (\cref{theorem:deterministic_top_level_wr}), even though we risk privacy leakage at every training step.
For our parameterization and $\numinstances=1$, top-level sampling with replacement requires $10$ compositions per epoch.
As shown in~\cref{fig:deterministic_vs_random_top_level_daily_main}, the resultant epoch-level profile is nevertheless smaller, and remains so after $10$ epochs.
This is consistent with any work on DP-SGD (e.g., \cite{abadi2016deep}) that uses subsampling instead of deterministic iteration.

\textbf{Epoch Privacy vs Length.} In~\cref{appendix:extra_experiments_epoch_length} we additionally explore the fact that, if we wanted to use deterministic top-level iteration, 
the number of subsequences 
$\numinstances$ would affect epoch length.
As expected, we observe that composing many private mechanisms ($\numinstances=1$) is preferable to composing few much less private mechanisms ($\numinstances > 1$) 
when considering a fixed number of training steps.

\begin{figure}
    \vskip 0.2in
    \centering
        \includegraphics[width=0.99\linewidth]{figures/experiments/eval_pld_label_noise/daily_30_32_main.pdf}
        \vskip -0.3cm
        \caption{Varying label noise $\sigma_F$ for top-level WOR and bottom-level WR  (\cref{theorem:data_augmentation_general}) with $\sigma_C = 0, \numinstances=1$.
        Increasing $\sigma_F$ is equivalent to decreasing forecast length.
        }
        \label{fig:label_noise_daily_main}
    \vskip -0.2in
\end{figure}

\textbf{Amplification by Label Perturbation.}
Finally, because the way in which adding Gaussian noise to the context and/or forecast window 
amplifies privacy (\cref{theorem:data_augmentation_general}) 
may be somewhat opaque, let us consider top-level sampling without replacement, bottom-level sampling with replacement,
$\numinstances=1$, $\sigma_C=0$, and varying label noise standard deviations $\sigma_F$. 
As shown in~\cref{fig:label_noise_daily_main}, increasing $\sigma_F$ has the same effect as letting the forecast length $L_C$ go to zero, i.e., eliminates the risk of leaking private information if it appears in the forecast window.
Of course, this data augmentation 
will have an effect on model utility, which we investigate shortly.

\begin{figure*}
\centering
\vskip 0.2in
    \begin{subfigure}{0.49\textwidth}
        \includegraphics[]{figures/experiments/eval_pld_monotonicity_composed/daily_20_32_1_main.pdf}
        \caption{Training step $1$}\label{fig:monotonicity_daily_main}
    \end{subfigure}
    \hfill
    \begin{subfigure}{0.49\textwidth}
        \includegraphics[]{figures/experiments/eval_pld_monotonicity_composed/daily_20_32_100_main.pdf}
        \caption{Training step $100$}\label{fig:monotonicity_composed_daily_main}
    \end{subfigure}\caption{
    Top-level WOR and bottom-level WR sampling under varying number of subsequences.
    Under composition, even optimistic lower bounds (\cref{theorem:wor_top_wr_bottom_upper}) 
    indicate worse privacy for $\numinstances > 1$ than 
    our tight upper bound for $\numinstances=1$ (\cref{theorem:wor_top_level_wr}).}
    \label{fig:monotonicity_daily_main_container}
\vskip -0.2in
\end{figure*}


\subsection{Application to Probabilistic Forecasting}
While the contribution of our work lies in formally analyzing the privacy of DP-SGD adapted to forecasting, 
training models with this algorithm can serve as a sanity-check to verify that the guarantees are sufficiently strong to retain meaningful utility under non-trivial privacy budgets.


\begin{table}[b]
\vskip -0.38cm
\caption{Average CRPS on \texttt{traffic} for $\delta=10^{-7}$. Seasonal, AutoETS, and models with $\epsilon=\infty$ are without noise.}
\label{table:1_event_training_traffic_main}
\vskip 0.18cm
\begin{center}
\begin{small}
\begin{sc}
\begin{tabular}{lcccc}
\toprule
Model & $\epsilon = 0.5$ & $\epsilon = 1$ & $\epsilon = 2$ &  $\epsilon = \infty$ \\
\midrule
SimpleFF & $0.207$ & $0.195$ & $0.193$ & $0.136$ \\ 
DeepAR & $\mathbf{0.157}$ & $\mathbf{0.145}$ & $\mathbf{0.142}$ & $\mathbf{0.124}$ \\
iTransf. & $0.211$ & $0.193$ & $0.188$ & $0.135$ \\
DLinear & $0.204$ & $0.192$ & $0.188$ & $0.140$ \\
\midrule
Seasonal   & $0.251$ & $0.251$ & $0.251$ & $0.251$\\
AutoETS   & $0.407$ & $0.407$ & $0.407$ & $0.407$\\
\bottomrule
\end{tabular}
\end{sc}
\end{small}
\end{center}
\vskip -0.1in
\end{table}

\textbf{Datasets, Models, and Metrics.}
We consider three standard benchmarks: \texttt{traffic}, \texttt{electricity}, and \texttt{solar\_10\_minutes} as used in~\cite{Lai2018modeling}.
We further consider four common architectures: 
A two-layer feed-forward neural network (``SimpleFeedForward''), a recurrent neural network (``DeepAR''~\cite{salinas2020deepar}),
an encoder-only transformer (``iTransformer''~\cite{liu2024itransformer}), and a refined feed-forward network proposed to compete with attention-based models (``DLinear''~\cite{zeng2023transformers}).
We let these architectures parameterize elementwise $t$-distributions to obtain probabilistic forecasts.
We measure the quality of these probabilistic forecasts using continuous ranked probability scores (CRPS), which we approximate via mean weighted quantile losses (details in~\cref{appendix:metrics}).
As a reference for what constitutes ``meaningful utility'', we compare against seasonal na\"{i}ve forecasting and exponential smoothing (``AutoETS'') without introducing any noise.
All experiments are repeated with $5$ random seeds.


\textbf{Event-Level Privacy.} \cref{table:1_event_training_traffic_main} shows CRPS of all models on the \texttt{traffic} test set 
when setting $\delta=10^{-7}$, and training on the training set until reaching a pre-specified $\epsilon$
with $1$-event-level privacy. For the other datasets and standard deviations, see~\cref{appendix:privacy_utility_tradeoff_event_level_privacy}.
The column $\epsilon=\infty$ indicates non-DP training.
As can be seen, models can retain much of their utility and outperform the baselines, even for $\epsilon \leq 1$ which is generally considered a small privacy budget~\cite{ponomareva2023dp}.
For instance, the average CRPS of DeepAR on the traffic dataset is $0.124$ with non-DP training and $0.157$ for $\epsilon=0.5$.
Note that, since all models are trained using  our tight privacy analysis,
which specific model performs best  on which specific dataset is orthogonal to our contribution. 

\textbf{Other results.}
In~\cref{appendix:privacy_utility_tradeoff_user_level_privacy} we additionally train with $w$-event and $w$-user privacy.
In~\cref{appendix:privacy_utility_tradeoff_label_privacy}, we demonstrate that label perturbations can offer an improved privacy--utility trade-off. 
All results confirm that our guarantees for DP-SGD adapted to forecasting are strong enough to enable provably private training while retaining utility.

\section{Conclusion}

\section{Conclusion}
\label{sec:Conclusion}
In this paper, we proposed a complete real-time planning and control approach for continuous, reliable, and fast online generation of dynamically feasible Bernstein trajectories and control for FW aircrafts. The generated trajectories span kilometers, navigating through multiple waypoints. By leveraging differential flatness equations for coordinated flight, we ensure precise trajectory tracking. Our approach guarantees smooth transitions from simulation to real-world applications, enabling timely field deployment. The system also features a user-friendly mission planning interface. Continuous replanning  maintains the rajectory curvature 
$\kappa$ within limits, preventing abrupt roll changes.

Future works will include the ability to add  a higher-level kinodynamic path planner to optimize waypoint spatial allocation and improve replanning success, and enhancing the trajectory-tracking algorithm by refining the aerodynamic coefficient estimation. 

% \section*{Impact Statements}
% This paper presents work whose goal is to advance the field of Machine Learning, especially diffusion generative models. There are many potential societal consequences of our work, none which we feel must be specifically highlighted here.


% In the unusual situation where you want a paper to appear in the
% references without citing it in the main text, use \nocite
% \nocite{langley00}

\bibliography{icml2025/reference}
\bibliographystyle{icml2025}


%%%%%%%%%%%%%%%%%%%%%%%%%%%%%%%%%%%%%%%%%%%%%%%%%%%%%%%%%%%%%%%%%%%%%%%%%%%%%%%
%%%%%%%%%%%%%%%%%%%%%%%%%%%%%%%%%%%%%%%%%%%%%%%%%%%%%%%%%%%%%%%%%%%%%%%%%%%%%%%
% APPENDIX
%%%%%%%%%%%%%%%%%%%%%%%%%%%%%%%%%%%%%%%%%%%%%%%%%%%%%%%%%%%%%%%%%%%%%%%%%%%%%%%
%%%%%%%%%%%%%%%%%%%%%%%%%%%%%%%%%%%%%%%%%%%%%%%%%%%%%%%%%%%%%%%%%%%%%%%%%%%%%%%
\newpage
\appendix
\onecolumn
\appendix
% \setcounter{table}{0}
% \renewcommand*{\thetable}{\arabic{table}}
% \renewcommand*{\thefigure}{\arabic{figure}}
\section{Related algorithms and metric caculation}
\label{app:related_algo_metric}

\subsection{Performance-energy Consistency} In this paper, performance-energy consistency refers to the consistency between the results evaluated using an energy model and those evaluated using real-world metrics for the same sample. Specifically, the consistency requires that good samples are assigned low energy, while poor samples are assigned high energy. Performance-energy consistency measures the proportion of element pairs that maintain the same relative order in both permutations \( X \) and \( Y \), where \( X \) and \( Y \) represent the index arrays obtained by sorting the original energy values \( \mathbf{E} = (E_1, E_2, \dots, E_N) \) and performance metric values \( \mathbf{P} = (P_1, P_2, \dots, P_N) \), respectively, in ascending order. In this paper, the energy values are calculated by energy model $E_\theta(x_0)$ for samples $\x_0$. The performance metric values are calculated as the L2 distance between the generated samples $\x_0$ and the ground truth under the given condition.

Let \( X = (X_1, X_2, \dots, X_N) \) and \( Y = (Y_1, Y_2, \dots, Y_N) \) be the index arrays obtained by sorting the original energy values \( \mathbf{E} = (E_1, E_2, \dots, E_N) \) and performance metric values \( \mathbf{P} = (P_1, P_2, \dots, P_N) \), respectively, in ascending order. Specifically, \( X_i \) is the rank of the \( i \)-th sample in the sorted energy values \( \mathbf{E} \), and \( Y_i \) is the rank of the \( i \)-th sample in the sorted performance metric values \( \mathbf{P} \).

\textbf{Consistency Definition:}
The \textbf{consistency} is defined as the proportion of consistent pairs \( (i, j) \) where \( i < j \) and the relative order of \( i \) and \( j \) in \( X \) is the same as in \( Y \). Specifically:
\[
\text{Consistency} = \frac{1}{\binom{N}{2}} \sum_{i=1}^{N-1} \sum_{j=i+1}^{N} \mathbb{I}\left( (X_i < X_j \land Y_i < Y_j) \lor (X_i > X_j \land Y_i > Y_j) \right),
\]
where:
\begin{itemize}
    \item \( \binom{N}{2} = \frac{N(N-1)}{2} \) is the total number of pairs \( (i, j) \) with \( i < j \),
    \item \( \mathbb{I}[\cdot] \) is the indicator function, which evaluates to 1 if the condition inside the brackets holds (i.e., the relative order is consistent), and 0 otherwise.
\end{itemize}
\subsection{Adversarial sampling}
During the sampling process, energy optimization often gets trapped in local minima or incorrect global minima, making it difficult to escape and hindering the sampling of high-quality samples.
\subsection{Negative Sample Generation} Negative samples are generated by introducing noise into the positive sample \( x_0 \). In the Maze and Sudoku experiments, permutation noise is applied to the channel dimension to induce significant changes in the solution. Other noise types can be used, as this remains a hyperparameter choice. Specifically, we first randomly sample two scalars \( p_1 \) and \( p_2 \) from a uniform distribution in the interval \( [0, 1] \), i.e., \( p_1, p_2 \sim \text{Uniform}(0, 1) \) ($p_1<p_2$). Then, for each channel position of the positive sample \( x_0 \), we swap the channel positions with probabilities \( p_1 \) and \( p_2 \), resulting in \( x_0^{-} \) and \( x_0^{--} \), such that the L2 distance between \( x_0^{-} \) and \( x_0 \) is smaller than the L2 distance between \( x_0^{--} \) and \( x_0 \). For other noise types, such as Gaussian noise, we normalize the L2 norm of the noise and apply noise at different scales to ensure that the L2 distance from \( x_0^{-} \) to \( x_0 \) is smaller than the L2 distance from \( x_0^{--} \) to \( x_0 \).


\subsection{Linear-regression algorithm} Given three points \((x_1, y_1)\), \((x_2, y_2)\), and \((x_3, y_3)\), we wish to fit a line of the form ~\cite{lane2003introduction}:

\[
y = kx + b
\]
The mean of the \(x\)-coordinates and the mean of the \(y\)-coordinates are:
\[
\bar{x} = \frac{1}{3}(x_1 + x_2 + x_3), \quad \bar{y} = \frac{1}{3}(y_1 + y_2 + y_3)
\]
The slope \(k\) of the best-fit line is given by the formula:

\[
k = \frac{\sum_{i=1}^{3} (x_i - \bar{x})(y_i - \bar{y})}{\sum_{i=1}^{3} (x_i - \bar{x})^2}
\]
This formula represents the least-squares solution for the slope.
Once the slope \(k\) is determined, the intercept \(b\) can be calculated as:
\[
b = \bar{y} - k\bar{x}
\]
The equation of the best-fit line is:
\[
\hat{y} = kx + b
\]
\section{Details of experiments}
\label{app:Exp_detail}
\subsection{Detais of Sudoku experiments}
\label{app:Exp_sudoku}
For Sudoku experiment, the dataset, model architecture, and training configurations are adopted from \citet{du2024learning}. We mainly use solving success rate to evaluate different models. Model backbone and training configurations can be found in Fig. \ref{fig:sudoku_ebm} and Table \ref{tab:sudoku_exp_detail}, respectively. All the exploration hyperparameters $c$ are set as 100 for Sudoku task.
\begin{figure}[H]
\begin{minipage}{0.9\textwidth}
\centering
\small
\begin{tabular}{c}
    \toprule
    3x3 Conv2D, 384 \\
    \midrule
    Resblock 384 \\
    \midrule
    Resblock 384 \\
    \midrule
    Resblock 384 \\
    \midrule
    Resblock 384 \\
    \midrule
    Resblock 384 \\
    \midrule
    Resblock 384 \\
    \midrule
    3x3 Conv2D, 9 \\ 
    \bottomrule
\end{tabular}
\caption{The model architecture for \proj on Sudoku task. The energy value is computed using the L2 norm of the final predicted output similar to \citet{du2023reduce}, while the output is directly used as noise prediction for the diffusion baseline.}
\label{fig:sudoku_ebm}
\end{minipage}
\end{figure}
\begin{table}[ht]
  \begin{center}
    \caption{\textbf{Details of  training for Sudoku task}. }
    \vskip -0.15in
    \label{tab:2d_model_architecture}
    \begin{tabular}{l|c} % <-- Alignments: 1st column left, 2nd middle and 3rd right, with vertical lines in between
    \multicolumn{2}{l}{}\\
      \hline
       \multicolumn{1}{l|}{Training configurations } & \multicolumn{1}{l}{}\\
      \hline
      Number of training steps & 100000  \\
      Training batch size & 64 \\
      Learning rate & 0.0001 \\
      Diffusion steps & 10 \\
      Inner loop optimization steps & 20 \\
      Denoising loss type & MSE \\
      Optimizer & Adam \\
        \hline
    \end{tabular}
      \label{tab:sudoku_exp_detail}
  \end{center}
\end{table}
\subsection{Details of Maze experiments}
\label{app:Exp_maze}
The details of maze experiments and model backbone are provided in Table \ref{tab:maze_exp_detail} and Fig. \ref{fig:maze_ebm}, respectively. The key metric, the maze-solving success rate is defined as the proportion of model-generated paths that have no breakpoints, do not overlap with walls, and begin and end at the start and target points, respectively. Maze datasets are generated by \citet{ivanitskiy2023configurable}, and detailed hyperparameter configurations are in Table \ref{tab:maze_exp_detail}. All the exploration hyperparameters $c$ are set as 100 for Maze task.
\begin{figure}[H]
\begin{minipage}{0.9\textwidth}
\centering
\small
\begin{tabular}{c}
    \toprule
    3x3 Conv2D, 384 \\
    \midrule
    Resblock 384 \\
    \midrule
    Resblock 384 \\
    \midrule
    Resblock 384 \\
    \midrule
    Resblock 384 \\
    \midrule
    Resblock 384 \\
    \midrule
    Resblock 384 \\
    \midrule
    3x3 Conv2D, 9 \\ 
    \bottomrule
\end{tabular}
\caption{The model architecture for \proj on Maze task. The energy value is computed using the L2 norm of the final predicted output similar to \citet{du2023reduce}, while the output is directly used as noise prediction for the diffusion baseline.}
\label{fig:maze_ebm}
\end{minipage}
\end{figure}
\begin{table}[ht]
  \begin{center}
    \caption{\textbf{Details of Maze dataset, training}. }
    \vskip -0.15in
    \label{tab:2d_model_architecture}
    \begin{tabular}{l|c} % <-- Alignments: 1st column left, 2nd middle and 3rd right, with vertical lines in between
    \multicolumn{2}{l}{}\\
      \hline
      \multicolumn{1}{l|}{Dataset:} & \multicolumn{1}{l}{}\\ 
      \hline
      Size of training dataset with grid size 4 & 10219   \\
      Size of training dataset with grid size 5 & 9394   \\
      Size of training dataset with grid size 6 & 10295  \\
      Minimum length of solution path & 5 \\
      Algorithm to generate the maze & DFS \\
      Size of test dataset with grid size 6 & 837   \\
      Size of test dataset with grid size 8 & 888   \\
      Size of test dataset with grid size 10 & 948   \\
      Size of test dataset with grid size 12 & 960   \\
      Size of test dataset with grid size 15 & 975   \\
      Size of test dataset with grid size 20 & 978   \\
      Size of test dataset with grid size 30 & 994   \\
      \hline
       \multicolumn{1}{l|}{Training configurations } & \multicolumn{1}{l}{}\\
      \hline
      Number of training steps & 200000  \\
      Training batch size & 64 \\
      Learning rate & 0.0001 \\
      Diffusion steps & 10 \\
      Inner loop optimization steps & 20 \\
      Denoising loss type & MSE + MAE \\
      Optimizer & Adam \\
        \hline
    \end{tabular}
      \label{tab:maze_exp_detail}
  \end{center}
\end{table}

\section{Performance sensitivity to hyperparameters}
\label{app:hyperparameters_sensitivity}

% inner loop opt steps, mcts noise scale(original model, mixed trained model hMCTS & Random search) more visualizations?
In this subsection, we analyze the impact of several hyperparameters on the experimental results. As shown in Table \ref{tab:maze_noise_scale}, the influence of different noise scales on the performance of various methods is presented. The hMCTS denoising and random search require a relatively larger noise scale to better expand the search space and improve final performance, while the diffusion model with naive inference performs best with a smaller noise scale. As demonstrated in Table \ref{tab:maze_inner_loop_opt} and Fig. \ref{fig:maze_opt_step}, the effect of varying inner-loop optimization steps on the results is also analyzed. It can be observed that performance improves gradually with an increasing number of steps, and after 5 steps, the performance stabilizes and the improvement slows down. Therefore, we chose 5 inner-loop optimization steps for the Maze experiments in this paper.
\begin{figure}[h!]
\vskip 0.2in
\begin{center}
\centerline{\includegraphics[width=0.55\textwidth]{fig/maze_optimization_steps_vs_values.pdf}}
\caption{Visualization of success rate across different number of inner-loop optimization steps on Maze with grid size $\mathbf{15\times15}$. }
\label{fig:maze_opt_step}
\end{center}
\vskip -0.2in
\end{figure}
\begin{table}[ht]
\caption{Success rate across the different number of inner-loop optimization step on Maze with grid size \textbf{15}. }
\label{tab:maze_inner_loop_opt}
\vskip 0.15in
\begin{center}
\resizebox{0.85\textwidth}{!}{ % Resize the table to fit within a single column
\begin{tabular}{l|cccccccccc}
\toprule
 &\multicolumn{10}{c}{\textbf{Number of optimization step}} \\
\cmidrule(lr){2-11} 
\textbf{Methods}                & 1               & 2               & 3               & 4               & 5               & 6               & 7               & 8               & 9 &10        \\
\midrule
T-SCEND tr. (ours), Naive inference & 0.0000 & 0.1562 & 0.2109 & 0.2734 & 0.2812 & 0.2734 & 0.2812 & 0.2969 & 0.2969 & 0.2969\\
\bottomrule
\end{tabular}
}
\end{center}
\vskip -0.1in
\end{table}
\begin{table}[ht]
\caption{Success rate across different noise scales on Maze with grid size \textbf{15}. }
\label{tab:maze_noise_scale}
\vskip 0.15in
\begin{center}
\resizebox{1\textwidth}{!}{ % Resize the table to fit within a single column
\begin{tabular}{l|cccccccccc}
\toprule
 &\multicolumn{10}{c}{\textbf{Noise scale}} \\
\cmidrule(lr){2-11} 
\textbf{Methods}                & 0.1               & 0.2               & 0.3               & 0.4               & 0.5               & 0.6               & 0.7               & 0.8               & 0.9 &1.0        \\
\midrule
T-SCEND tr. (ours), hMCTS denoising (energy)               & 0.3828 & 0.4375 & 0.5312 & 0.6094 & 0.6562 & 0.6953 & 0.7031 & 0.7344 & 0.7734 & 0.7969 \\
T-SCEND tr. (ours), naive inference                    & 0.3125 & 0.2656 & 0.2578 & 0.2344 & 0.2422 & 0.2656 & 0.2578 & 0.2422 & 0.2500 & 0.2500 \\
T-SCEND tr. (ours), Random search(energy)      & 0.3906 & 0.4453 & 0.5312 & 0.5703 & 0.5938 & 0.6328 & 0.6641 & 0.6719 & 0.6797 & 0.6562 \\
\bottomrule
\end{tabular}
}
\end{center}
\vskip -0.1in
\end{table}
\section{Additional results}
\label{app:additional_results}
\begin{figure}[h!]
\vskip 0.2in
\begin{center}
\centerline{\includegraphics[width=0.6\textwidth]{fig/maze_success_rate_vs_ts.pdf}}
\caption{Visualization of Success rate across different MCTS start step $t_s$. }
\label{fig:maze_success_rate_vs_ts}
\end{center}
\vskip -0.2in
\end{figure}
The parameter \( t_s \) controls the proportion of the total inference budget allocated to MCTS denoising. When \( t_s = 9 \), it means only MCTS denoising is used, while \( t_s = 0 \) means only best-of-N random search is employed. For \( 0 < t_s < 9 \), hMCTS denoising is applied. As shown in Table \ref{tab:maze_mcts_start_step} and Fig. \ref{fig:maze_success_rate_vs_ts}, there is a noticeable peak in model performance as \( t_s \) varies.
\begin{table}[h!]
\caption{Success rate of hMCTS denoising on Maze with grid size \textbf{15} across different MCTS start steps. }
\label{tab:maze_mcts_start_step}
\vskip 0.15in
\begin{center}
\resizebox{\textwidth}{!}{ % Resize the table to fit within a single column
\begin{tabular}{l|cccccccccc}
\toprule
\cmidrule(lr){2-11} 
\textbf{Methods} & 0               & 1               & 2               & 3               & 4               & 5               & 6               & 7               & 8               & 9         \\
\midrule
Original, hMCTS denoising (energy)      & 0.0781 & 0.0703 & 0.0859& 0.0781 & 0.1250& 0.1484& 0.1250 & 0.0781 & 0.0625 & 0.0703\\
T-SCEND tr. (ours), hMCTS denoising (energy)   & 0.6562 & 0.6094& 0.6641 & 0.7969 & 0.7969 & 0.6406& 0.4922 & 0.4922 & 0.4609 & 0.4453 \\
\bottomrule
\end{tabular}
}
\end{center}
\vskip -0.1in
\end{table}

\begin{table}[h!]
\caption{Success rate of Random search for different training methods on Maze with grid size \textbf{15} and Sudoku harder dataset guided with ground truth accuracy. Untrained, Random search (gt) represents use ground truth to guide the random search.  Here, $L=N$. Bold font denotes the best model. }
\label{tab:maze_diffus_baseline_diversity}
\vskip 0.15in
\begin{center}
\resizebox{\textwidth}{!}{ % Resize the table to fit within a single column
\begin{tabular}{l|cccccc|ccccccc}
\toprule
\multicolumn{1}{c|}{} & \multicolumn{6}{c}{\textbf{Maze success rate}} & \multicolumn{7}{c}{\textbf{Sudoku success rate}}\\ 
\cmidrule(lr){2-14} 
\textbf{Methods} & \textbf{$N$=1} & \textbf{$N$=11} & \textbf{$N$=21} & \textbf{$N$=41} & \textbf{$N$=81} & \textbf{$N$=161} &\textbf{$N$=1} & \textbf{$N$=11} & \textbf{$N$=21} & \textbf{$N$=41} & \textbf{$N$=81} & \textbf{$N$=161} & \textbf{$N$=321} \\
\midrule
Untrained, Random search (gt) & 0.0000 & 0.0000 & 0.0000 & 0.0000 & 0.0000 & 0.0000 & 0.0000 & 0.0000 & 0.0000 & 0.0000 & 0.0000 & 0.0000 & 0.0000 \\ 
Original, Random search (gt) & 0.0625 & 0.1250 & 0.1094 & 0.1328 & 0.1719 & 0.1719 & 0.0859 & 0.1641 & 0.2188 & 0.2344 & 0.2422 & 0.2656 & 0.2969 \\
DDPM, Random search (gt) & 0.0312&0.1094&0.1587&0.1746&0.2031&0.2422& 0.0000          & 0.0000          & 0.0000          & 0.0000          & 0.0000          & 0.0000          & 0.0156 \\
T-SCEND tr. w/o LRNCL, Random search (gt) & \textbf{0.2500} & \textbf{0.5078} & \textbf{0.5938} & \textbf{0.6562} & \textbf{0.7109} & \textbf{0.7422} & \textbf{0.1094} & \textbf{0.2578} & \textbf{0.2969} & \textbf{0.3438} & \textbf{0.3750} & \textbf{0.3828} & \textbf{0.4219} \\ 
\bottomrule
\end{tabular}
}
\end{center}
\vskip -0.1in
\end{table}
\begin{table}[h!]
\caption{Success rate and element-wise accuracy of Random search for different training methods on Sudoku harder dataset guided with ground truth accuracy. Here, $L=N$. Bold font denotes the best model. }
\label{tab:sudoku_diffus_baseline_ddpm}
\vskip 0.15in
\begin{center}
\resizebox{1\textwidth}{!}{ % Resize the table to fit within a single column
\begin{tabular}{l|ccccccc|ccccccc}
\toprule
& \multicolumn{7}{c|}{\textbf{Success rate}} & \multicolumn{7}{c}{\textbf{Element-wise} accuracy}\\
\cmidrule(lr){2-15} 
Methods &\textbf{$N$=1} & \textbf{$N$=11} & \textbf{$N$=21} & \textbf{$N$=41} & \textbf{$N$=81} & \textbf{$N$=161} & \textbf{$N$=321} &
\textbf{$N$=1} & \textbf{$N$=11} & \textbf{$N$=21} & \textbf{$N$=41} & \textbf{$N$=81} & \textbf{$N$=161} & \textbf{$N$=321} \\
\midrule
DDPM, Random search, GT accuracy guided     & 0.0000          & 0.0000          & 0.0000          & 0.0000          & 0.0000          & 0.0000          & 0.0156          & 0.5071          & 0.6089          & 0.6316          & 0.6492          & 0.6691          & 0.6881          & 0.6999          \\
Original, Random search, GT accuracy guided & 0.0781          & 0.1641          & 0.2188          & 0.2344          & 0.2422          & 0.2656          & 0.2812          & \textbf{0.6650} & 0.7731          & 0.7952          & 0.8036          & 0.8217          & 0.8347          & 0.8491          \\
T-SCEND tr. w/o LRNCL, Random search, GT accuracy guided            & \textbf{0.1094} & \textbf{0.2578} & \textbf{0.2969} & \textbf{0.3438} & \textbf{0.3750} & \textbf{0.3828} & \textbf{0.4219} & 0.6442 & \textbf{0.7855} & \textbf{0.8096} & \textbf{0.8317} & \textbf{0.8466} & \textbf{0.8628} & \textbf{0.8854} \\ 
\bottomrule
\end{tabular}
}
\end{center}
\vskip -0.1in
\end{table}





% \begin{table}[]
% \caption{Success rate of mixed inference on Maze with grid size \textbf{15} across different MCTS start steps. Bold font denotes the best model.}
% \label{tab:maze_mcts_start_step}
% \begin{tabular}{lllllllllll}
% \cline{1-1}
% \multicolumn{1}{|l|}{Method} & \multicolumn{10}{c}{MCTS start step}                                                                                                                                              \\
% \multicolumn{1}{|l|}{}                        & 0               & 1               & 2               & 3               & 4               & 5               & 6               & 7               & 8               & 9               \\ \cline{1-1}
% Mixed inference, KL \& LRNCL ,Energy guided   & 0.6562 ± 0.4750 & 0.6094 ± 0.4879 & 0.6641 ± 0.4723 & 0.7969 ± 0.4023 & 0.7969 ± 0.4023 & 0.6406 ± 0.4798 & 0.4922 ± 0.4999 & 0.4922 ± 0.4999 & 0.4609 ± 0.4985 & 0.4453 ± 0.4970 \\
% Mixed inference, Original, Energy guided      & 0.0781 ± 0.2684 & 0.0703 ± 0.2557 & 0.0859 ± 0.2803 & 0.0781 ± 0.2684 & 0.1250 ± 0.3307 & 0.1484 ± 0.3555 & 0.1250 ± 0.3307 & 0.0781 ± 0.2684 & 0.0625 ± 0.2421 & 0.0703 ± 0.2557
% \end{tabular}
% \end{table}
\section{Limitations and future work}
\label{app:limit_future} 
Our inference framework primarily relies on MCTS, which presents two key limitations: (1) limited compatibility with parallel computing, and (2) challenges in effectively evaluating node quality during the early stages of denoising. Future work could explore integrating alternative search strategies, such as those proposed by \citet{wu2024inference}. Additionally, to enhance performance-energy consistency, we introduce linear-regression negative contrastive learning, which enforces a linear relationship between energy and the distance to real samples. Further investigation is needed to assess the broader implications of this constraint and explore alternative regularization approaches. Lastly, while our current implementation utilizes Gaussian noise for branching, other diffusion-based branching mechanisms remain an open area for exploration.
\section{Visualization of results}
\label{app:vis_results}
\subsection{Visualization of Maze experiments}
\label{app:maze_vis}
This section presents visualizations of the training in Fig. \ref{fig:maze_training_vis}, test Maze data in Fig. \ref{fig:maze_test_vis}, and samples generated by different methods in Fig. \ref{fig:maze_samples_diff}. In the visuals, black pixels denote walls, green represents the starting point, red represents the goal point, blue marks the solved path, and white represents the feasible area. All visualizations are based on a few representative samples. The results from the training and test sets clearly show that the tasks in the test set are notably more challenging than those in the training set. Visual comparisons of samples generated by different methods reveal that the originally trained model, regardless of the inference strategy, performs consistently worse than \proj.
\begin{figure}[tb]
\vskip 0.2in
\begin{center}
\centerline{\includegraphics[width=0.8\textwidth]{fig/maze_plot_multi_grid_size_appendix_train.pdf}}
\caption{Visualization of training maze dataset. }
\label{fig:maze_training_vis}
\end{center}
\vskip -0.2in
\end{figure}

\begin{figure}[ht]
\vskip 0.2in
\begin{center}
\centerline{\includegraphics[width=0.8\textwidth]{fig/maze_plot_multi_grid_size_appendix_test.pdf}}
\caption{Visualization of test maze dataset, where the blue paths are ground-truth solutions.}
\label{fig:maze_test_vis}
\end{center}
\vskip -0.2in
\end{figure}

\begin{figure}[ht]
\vskip 0.2in
\begin{center}
\centerline{\includegraphics[width=0.8\textwidth]{fig/maze_plot_diff.pdf}}
\caption{Visualization of samples generated by different training and inference methods.}
\label{fig:maze_samples_diff}
\end{center}
\vskip -0.2in
\end{figure}
\subsection{Visualization of Sudoku experiments}
\label{app:sudoku_vis}

\begin{figure}[ht]
% \vskip 0.2in
\begin{center}
\centerline{\includegraphics[width=0.7\textwidth]{fig/sudoku_train_test_samples.pdf}}
\caption{Visualization of training and test Sudoku dataset.}
\label{fig:sudoku_training_test_vis}
\end{center}
% \vskip -0.2in
\end{figure}

\begin{figure}[ht]

\vskip 0.2in
\begin{center}
\centerline{\includegraphics[width=0.7\textwidth]{fig/sudoku_plot_diff.pdf}}
\caption{Visualization of samples generated by different training and inference methods.}
\label{fig:sudoku_samples_diff}
\end{center}
\vskip -0.2in
\end{figure}

This section presents visualizations of the training and test Sudoku data in Fig.~\ref{fig:sudoku_training_test_vis}, and representative samples generated by different methods in Fig.~\ref{fig:sudoku_samples_diff}. In the
visuals, black numbers denote the condition, green numbers represent correct predictions, and red numbers represent wrong predictions. All visualizations are derived from a few representative samples. The comparison between the training and test sets clearly indicates that the tasks in the test set are significantly more difficult than those in the training set. When comparing the samples generated by different methods, it is evident that the originally trained model, regardless of the inference strategy, consistently underperforms compared to \proj.

% training dataset, landscape visualization, solution of different models
% \section{Analysis of failure case}
% \tao{TODO}
% \jiashu{TODO}
% \label{app:failure_analysis} 
% training dataset, test dataset, solutions of different models
%%%%%%%%%%%%%%%%%%%%%%%%%%%%%%%%%%%%%%%%%%%%%%%%%%%%%%%%%%%%%%%%%%%%%%%%%%%%%%%
%%%%%%%%%%%%%%%%%%%%%%%%%%%%%%%%%%%%%%%%%%%%%%%%%%%%%%%%%%%%%%%%%%%%%%%%%%%%%%%


\end{document}


% This document was modified from the file originally made available by
% Pat Langley and Andrea Danyluk for ICML-2K. This version was created
% by Iain Murray in 2018, and modified by Alexandre Bouchard in
% 2019 and 2021 and by Csaba Szepesvari, Gang Niu and Sivan Sabato in 2022.
% Modified again in 2023 and 2024 by Sivan Sabato and Jonathan Scarlett.
% Previous contributors include Dan Roy, Lise Getoor and Tobias
% Scheffer, which was slightly modified from the 2010 version by
% Thorsten Joachims & Johannes Fuernkranz, slightly modified from the
% 2009 version by Kiri Wagstaff and Sam Roweis's 2008 version, which is
% slightly modified from Prasad Tadepalli's 2007 version which is a
% lightly changed version of the previous year's version by Andrew
% Moore, which was in turn edited from those of Kristian Kersting and
% Codrina Lauth. Alex Smola contributed to the algorithmic style files.
