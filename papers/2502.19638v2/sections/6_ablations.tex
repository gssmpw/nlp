\section{Ablations}

\subsection{Number and Type of Calibration Images}

\begin{figure}[htbp]
    \vspace{-3mm}
    \centering
    \includegraphics[width=1.0\linewidth]{fig/ablations1.png}
    \caption{Ablation study on the number and type of calibration images used in SITR, showing their effect on (i) Classification accuracy for inter-sensor transfer, (ii) Classification accuracy for intra-sensor transfer, and (iii) Pose estimation error for inter-sensor transfer. }
    \label{fig:ablations_calibration}
    \vspace{-3mm}
\end{figure}

We conduct an ablation study to investigate the impact of the number and type of calibration images on the performance of \modelname. 
In the standard \modelname~setup, we press two objects—a ball and a cube corner—at nine locations roughly arranged in a 3x3 grid pattern across the sensor surface. 
To explore variations, we retrained \modelname~using different subsets of these calibration images and evaluated performance across all downstream tasks.

We test on five calibration configurations: No calibration images (0); Ball pressed at 4 corners (4); Ball pressed in a 3x3 grid (9); Ball and cube pressed at 4 corners ($8^*$); Ball and cube pressed in a 3x3 grid, which is the standard setup ($18^*$).
Fig. \ref{fig:ablations_calibration} illustrates how different numbers and types of calibration images impact \modelname's performance. We observe that increasing the number of calibration images increases performance across all tasks. However, the performance gains diminish as more images of the same object are added (as seen in the progression from cases (0) to (4) to (9)). Introducing a second calibration object with a distinct geometry, such as the cube (cases (4) to (8*)), results in a larger performance boost compared to simply adding more images of the same object (cases (4) to (9)). The effect of calibration images is particularly notable in the inter-sensor setting, where we see upwards of a 20\% increase in classification accuracy from case (0) to (18*). We choose case (18*) for SITR since increasing the number of calibration images does not incur additional inference costs, as calibration tokens are computed only once per sensor.

\subsection{Contrastive loss and temperature}

\begin{figure}[htbp]
    \vspace{-3mm}
    \centering
    \includegraphics[width=1.0\linewidth]{fig/ablations2.png}
    \caption{Ablation study examining the impact of SCL and varying contrastive temperature $\tau$ on SITR’s performance. Subplots (i) and (ii) show classification accuracy in inter-sensor and intra-sensor settings, respectively, while (iii) shows the effect on pose estimation RMSE.}
    \label{fig:ablations_temperature}
    \vspace{-3mm}
\end{figure}

We conduct an ablation study to assess the effect of SCL and varying contrastive temperatures $\tau$ on \modelname's performance. Specifically, we compared models with and without the SCL term and tested five contrastive temperatures: 0.25, 0.10, 0.07, 0.03, and 0.01.
No SCL corresponds to using only the normal map reconstruction loss during pre-training.
Results in Fig. \ref{fig:ablations_temperature} show that a contrastive temperature of 0.07 achieves the best classification performance in the intra-sensor setting, while 0.03 performs best for the inter-sensor setting. Lower or higher temperatures lead to reduced performance in both cases.
For the pose estimation task, the addition of SCL has a negligible impact on the RMSE. 
These results suggest that contrastive learning helps align features across sensors in classification tasks. However, in the pose estimation task, the model's performance is more dependent on the fine-grained geometry information from the contact surface. For SITR, we choose a temperature of 0.07 for its strong performance in the classification task. 