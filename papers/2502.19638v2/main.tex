
\documentclass{article} % For LaTeX2e
\usepackage{iclr2025_conference,times}
\usepackage{tabularx}
\usepackage{booktabs}
\usepackage{caption}
\usepackage{wrapfig}

% Optional math commands from https://github.com/goodfeli/dlbook_notation.
%%%%% NEW MATH DEFINITIONS %%%%%

% \usepackage{amsmath,amsfonts,bm}
\usepackage{amsmath,amsfonts}

\usepackage{pifont}


\newcommand{\R}{\mathbb{R}}


\def\va{{\mathbf{a}}}
\def\vg{{\mathbf{g}}}

% Sets
\def\sR{\mathbb{R}}
\def\sC{\mathbb{C}}
\def\sZ{\mathbb{Z}}
\def\sN{\mathbb{N}}
\def\sQ{\mathbb{Q}}

\def\sS{\mathcal{S}}



% Vectors
\def\vzero{{\mathbf{0}}}
\def\vone{{\mathbf{1}}}
\def\vmu{{\mathbf{\mu}}}
\def\vtheta{{\mathbf{\theta}}}
\def\va{{\mathbf{a}}}
\def\vb{{\mathbf{b}}}
\def\vc{{\mathbf{c}}}
\def\vd{{\mathbf{d}}}
\def\ve{{\mathbf{e}}}
\def\vf{{\mathbf{f}}}
\def\vg{{\mathbf{g}}}
\def\vh{{\mathbf{h}}}
\def\vi{{\mathbf{i}}}
\def\vj{{\mathbf{j}}}
\def\vk{{\mathbf{k}}}
\def\vl{{\mathbf{l}}}
\def\vm{{\mathbf{m}}}
\def\vn{{\mathbf{n}}}
\def\vo{{\mathbf{o}}}
\def\vp{{\mathbf{p}}}
\def\vq{{\mathbf{q}}}
\def\vr{{\mathbf{r}}}
\def\vs{{\mathbf{s}}}
\def\vt{{\mathbf{t}}}
\def\vu{{\mathbf{u}}}
\def\vv{{\mathbf{v}}}
\def\vw{{\mathbf{w}}}
\def\vx{{\mathbf{x}}}
\def\vy{{\mathbf{y}}}
\def\vz{{\mathbf{z}}}
\def\vzeta{{\mathbf{\zeta}}}

% Matrix
\def\mA{{\mathbf{A}}}
\def\mB{{\mathbf{B}}}
\def\mC{{\mathbf{C}}}
\def\mD{{\mathbf{D}}}
\def\mE{{\mathbf{E}}}
\def\mF{{\mathbf{F}}}
\def\mG{{\mathbf{G}}}
\def\mH{{\mathbf{H}}}
\def\mI{{\mathbf{I}}}
\def\mJ{{\mathbf{J}}}
\def\mK{{\mathbf{K}}}
\def\mL{{\mathbf{L}}}
\def\mM{{\mathbf{M}}}
\def\mN{{\mathbf{N}}}
\def\mO{{\mathbf{O}}}
\def\mP{{\mathbf{P}}}
\def\mQ{{\mathbf{Q}}}
\def\mR{{\mathbf{R}}}
\def\mS{{\mathbf{S}}}
\def\mT{{\mathbf{T}}}
\def\mU{{\mathbf{U}}}
\def\mV{{\mathbf{V}}}
\def\mW{{\mathbf{W}}}
\def\mX{{\mathbf{X}}}
\def\mY{{\mathbf{Y}}}
\def\mZ{{\mathbf{Z}}}
\def\mBeta{{\mathbf{\beta}}}
\def\mPhi{{\mathbf{\Phi}}}
\def\mLambda{{\mathbf{\Lambda}}}
\def\mSigma{{\mathbf{\Sigma}}}


% Expectation
% \def\eE{\mathop{\mathbb{E}}\limits}
\def\eE{\mathbb{E}}

% Probability
\def\pP{\mathbb{P}}

% Tilde
\def\tf{\tilde{f}}
\def\tS{\tilde{S}}
\def\wtF{\widetilde{\mathcal{F}}}
\def\whR{\widehat{R}}
\def\tvx{\tilde{\mathbf{x}}}
\def\ty{\tilde{y}}


\def\defeq{\overset{\textup{def}}{=}}
% \def\defeq{\overset{.}{=}}
\def\defone{\overset{\text{\ding{172}}}{=}}
\def\deftwo{\overset{\text{\ding{173}}}{=}}
\def\leqone{\overset{\text{\ding{172}}}{\leq}}
\def\leqtwo{\overset{\text{\ding{173}}}{\leq}}
\def\leqthree{\overset{\text{\ding{174}}}{\leq}}
\def\leqfour{\overset{\text{\ding{175}}}{\leq}}
\def\eqone{\overset{\text{\ding{172}}}{=}}
\def\eqtwo{\overset{\text{\ding{173}}}{=}}
\def\eqthree{\overset{\text{\ding{174}}}{=}}
\def\eqfour{\overset{\text{\ding{175}}}{=}}
\def\geqfive{\overset{\text{\ding{176}}}{\geq}}

\usepackage{hyperref}
\usepackage{url}
\usepackage{graphicx}
\usepackage{subcaption}
\usepackage{xspace}
\usepackage{listings}
\usepackage{subfiles}
\usepackage{amsmath}
\usepackage{amssymb}
\usepackage{array}
\usepackage{multirow}

%\usepackage[disable]{todonotes}
% \usepackage[colorinlistoftodos]{todonotes}
% \newcommand{\wenzhen}[1]{\todo[inline,color=red!40]{Wenzhen: #1}}
% \newcommand{\harsh}[1]{\todo[inline,color=yellow!40]{Harsh: #1}}
% \newcommand{\yuchen}[1]{\todo[inline,color=blue!40]{Yuchen: #1}}
% \newcommand{\Samuel}[1]{\todo[inline,color=green!20]{Sam: #1}}
% \newcommand{\ruihan}[1]{\todo[inline,color=purple!20]{Ruihan: #1}}
\newcommand{\ruihan}[1]{\textcolor{purple}{[Ruihan: #1]}}
\newcommand{\bluetext}[1]{\textcolor{blue}{#1}}
\newcommand{\markword}[1]{\textbf{[#1]}}

\newcommand{\modelname}{SITR\xspace}



\title{Sensor-Invariant Tactile Representation}

% Authors must not appear in the submitted version. They should be hidden
% as long as the \iclrfinalcopy macro remains commented out below.
% Non-anonymous submissions will be rejected without review.

\author{Harsh Gupta%
\thanks{~equal contribution } \; \; 
Yuchen Mo\hbox to 0pt{$^*$} \; \; 
Shengmiao Jin \; \; Wenzhen Yuan \\
University of Illinois Urbana-Champaign\\
\texttt{\{hgupt3,yuchenm7\}@illinois.edu}\\
}

% The \author macro works with any number of authors. There are two commands
% used to separate the names and addresses of multiple authors: \And and \AND.
%
% Using \And between authors leaves it to \LaTeX{} to determine where to break
% the lines. Using \AND forces a linebreak at that point. So, if \LaTeX{}
% puts 3 of 4 authors names on the first line, and the last on the second
% line, try using \AND instead of \And before the third author name.

\newcommand{\fix}{\marginpar{FIX}}
\newcommand{\new}{\marginpar{NEW}}

\iclrfinalcopy % Uncomment for camera-ready version, but NOT for submission.
\begin{document}


\maketitle

\begin{abstract}
High-resolution tactile sensors have become critical for embodied perception and robotic manipulation. 
However, a key challenge in the field is the lack of transferability between sensors due to design and manufacturing variations, which result in significant differences in tactile signals. 
This limitation hinders the ability to transfer models or knowledge learned from one sensor to another. 
To address this, we introduce a novel method to extract Sensor-Invariant Tactile Representations (SITR), enabling zero-shot transfer across optical tactile sensors. 
Our approach utilizes a transformer-based architecture trained on a diverse dataset of simulated sensor designs, allowing generalizability to new sensors in the real world with minimal calibration. 
Experimental results demonstrate our method’s effectiveness across various tactile sensing applications, facilitating data and model transferability for future advancements in the field.
\end{abstract}


\begin{figure}[ht]
\begin{center}
\includegraphics[width=1\linewidth]{fig/teaser.png}
\caption{
Vision-based tactile sensors vary in both optical design and physical properties. 
Even with the same contact object, a screw, the tactile images produced by each sensor differ significantly.
These variations highlight the challenge of transferring models from one sensor to another.
}
\label{fig:teaser}
\end{center}
\vspace{-8pt}
\end{figure}

\subfile{sections/1_introduction}

\subfile{sections/2_relatedworks}

\subfile{sections/3_method}

\subfile{sections/4_dataset.tex}

\subfile{sections/5_experiments}

\subfile{sections/6_ablations}

\subfile{sections/7_discussion_conclusion}

\subsubsection*{Acknowledgments}
The authors would like to thank Amin Mirzaee and Ruohan Zhang for their help with hardware design and prototyping. The authors thank Ruihan Gao for her assistance with revising the paper. We thank Arpit Agarwal for his insightful discussions on tactile simulation. This work was supported in part by the NSF award \#2024882, the Indira Gunda Saladi Engineering Research Prize, and the Illinois Office of Undergraduate Research.


\bibliography{iclr2025_conference}
\bibliographystyle{iclr2025_conference}

\appendix
\subfile{sections/9_appendix}


\end{document}
