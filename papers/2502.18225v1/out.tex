%Cirrhosis is the severe scarring (fibrosis) of the liver and a common endpoint of various chronic liver diseases. Early diagnosis is vital to prevent complications such as decompensation and cancer, which significantly decreases life expectancy. However, diagnosing cirrhosis in its early stages is challenging, and patients often present with life-threatening complications. Artificial intelligence algorithms can assist in the early diagnosis and staging by analyzing imaging findings and electronic health record (EHR) data. The lack of high-quality, large-scale magnetic resonance imaging (MRI) datasets for cirrhosis hinders data-intensive image analysis techniques, such as radiomics and deep learning. Additionally, assessing cirrhotic MRIs presents challenges due to the varied patterns of cirrhosis, the overlaps among stages, and changes in normal abdominal anatomy. The newly introduced large-scale multi-sequence \textit{CirrMRI600+} dataset includes cirrhotic MRIs from various stages and aims to bridge this gap. Utilizing this dataset, we developed a framework for radiomics and deep learning to predict the stages of cirrhosis. Our results indicate that both methods assist in predicting the stage of cirrhosis, with deep learning significantly outperforming traditional radiomics-based machine learning methods. To our knowledge, this is the first study to classify the stages of cirrhosis using radiomics and deep learning on a comprehensive MRI dataset.  %With the advent of large-scale MRI datasets and advances in deep learning, our study paves the way for AI-powered diagnostics that enable more targeted therapeutic approaches, improve patient outcomes, and revolutionize the field of hepatology. The source code will be available at \url{https://github.com/JunZengz/CirrhosisStage}.


%This advancement holds promise for refining clinical trial strategies and improving decision-making processes in healthcare settings.

%In this study, we evaluated the effectiveness of radiomics and deep learning algorithms in assessing the stages of cirrhosis. The main contributions of this work are as follows: 


%The goal of liver cirrhosis stage estimation is to accurately predict the stage for a given MRI cirrhotic image. Based on the severity of cirrhosis, MRI images are categorized into three stages: mild, moderate, and severe. The accurate estimation of liver cirrhosis stages can offer valuable clinical insights during the medical diagnostic process.In this study, we perform extensive studies of liver cirrhosis stage estimation on the large-scale MRI dataset CirrMRI600+. The imaging data are evaluated and staged by professional radiologists and are then utilized for the estimation of cirrhosis stages.We demonstrate the feasibility of using deep learning for this task with large-scale MRI data, aiming to propel advancements in liver cirrhosis stage estimation research with MRI. The main contributions of the work are as follows: 


%Liver Cirrhosis is a prevalent disease characterized by the gradual replacement of healthy liver cells with irreversible scar tissue as the condition progresses. This disease typically progresses from an early compensated stage to a late decompensated stage, and may ultimately lead to hepatocellular carcinoma (HCC). Early detection and treatment of cirrhosis can significantly reduce the risk of progression to advanced stages. 

%This study sets a foundation for future research in automated liver disease assessment, with potential extensions to other imaging modalities and related hepatic conditions. The insights gained regarding architectural choices, sequence-specific processing, and the limitations of current approaches provide valuable guidance for the continued development of AI-assisted medical imaging analysis.
%\vspace{-3mm}
%We conduct a comprehensive evaluation of advanced deep-learning models to estimate the stages of liver cirrhosis using a large MRI dataset. This challenging problem remains unresolved due to its morphological complexity and the subtle tissue changes between the cirrhosis stages. Leveraging newer architectures like MambaVision-S and various ResNet variants, we demonstrated that while models like MambaVision-S achieved the highest overall accuracy, there is a marked decrease in performance when estimating moderate stages of cirrhosis. This highlights the inherent difficulty in distinguishing advanced disease stages using current deep learning approaches. Our work sets a new benchmark in the field, emphasizing the need for specialized strategies to improve deep learning performance in complex medical imaging tasks like MRI-based cirrhosis staging.

%%we presented deep learning and radiomics based methods for liver cirrhosis stage estimation on a large-scale MRI dataset. Our experiments demonstrated that deep learning methods can be effectively applied to this task with the support of large amounts of MRI data, significantly outperforming traditional machine learning approaches. 
%It is important to note that live fibrosis can appear  similar to normal liver tissue in imaging examinations, and stages of the liver cirrhosis can be quite similar to each other, making it difficult to distinguish with the naked eye even for expert radiologists. These difficulties are reflected in the results. %The motivation for this study was to advance research in automatic liver cirrhosis stage estimation with MRI. %We anticipate that deep learning using MRI data will yield promising results in cirrhosis stage estimation, providing valuable insights for clinical diagnosis and enhancing patient prognosis.In the future, we plan to develop efficient deep-learning algorithms to further improve the model's capability in estimating liver cirrhosis stage. Additionally, we aim to extend the cirrhotic data to encompass more modalities and develop medical multi-modality foundation models to create more robust and interpretable tools for clinical diagnosis and treatment.

%we have analyzed the performance of various deep learning and machine learning methods for liver cirrhosis stage estimation. 
%This task presents significant challenges in clinical diagnosis and treatment, requiring the integration of multiple factors such as clinical symptoms and pathological reports to achieve a reliable diagnosis. 
 %Accurate diagnosis largely relies on clinical observations and the expertise of medical professionals. 

%One may argue whether pre-training medical images to computer vision models is suitable for this problem or not. Indeed, fine-tuning pre-trained models on the large-scale dataset Image1k is beneficial for most visual tasks, as it can absorb rich feature representations from existing natural categories. However, the unique characteristics of medical images mean that prior knowledge derived from natural images is not effectively transferable for estimating stages of liver cirrhosis.
%Training deep learning models on large-scale medical datasets is beneficial, as it can enhance clinical decision-making by providing valuable references. 



% \addtolength{\textheight}{-12cm}



%\section{Conclusion}
%\label{conclusion}




% \begin{table*}[t!]
% \centering
% \caption{Performance comparison of deep learning models on CirrMRI600+ T2-W MRI dataset using 2D images.}
% %Red font represents baseline models. Paras, P, Se, Sp, F1 means Parameter size, Precision, Sensitivity, Specificity and F1-score, respectively.  
%  \begin{tabular} {c|c|c|c|c|c|c|c}
% \toprule

% \textbf{{Dataset}}  & \textbf{{Model}}  &\textbf{AUC$\uparrow$} & \textbf{Param}  &\textbf{Precision(\%)$\uparrow$} & \textbf{Sensitivity(\%)$\uparrow$} & \textbf{Specificity(\%)$\uparrow$}  & \textbf{F1(\%)$\uparrow$}  
% \\ 
% \hline

%     \multirow{8}{*}{CirrMRI600+} 

%     &VGG-19    
%     &0.316 &139.58 &30.32 &30.60 &65.50 &30.06 \\
    
%     &Resnet50             
%     &0.360 &23.51 &31.60 &32.23 &67.07 &31.88 \\
    
%     &ConvNext-B  
%     &0.314 &87.57 &26.72 &31.38 &65.89 &26.65 \\
    
%     %&VIT-B/16   
%     % &0.366	&85.85	
%     % &24.27	&26.23	&60.64	&25.20	\\
    
%     %&Swin-B          
%     % &0.501	&86.75	
%     % &-	&-	&-	&-	&-	&-\\
    
%     &PVTv2-B3            
%     &0.366 &44.73 &32.96 &33.06 &66.70 &32.17 \\
    
%     &VMamba-S 
%     &0.295 &49.38 &28.74 &28.38 &63.55 &28.26 \\
    
%     &Vim-S 
%     &0.366 &25.41 &39.66 &36.67 &69.11 &36.14 \\

%     &MedMamba-S 
%     &0.347 &18.62 &18.62 &32.16 &67.08 &23.37 \\
    
%     &MambaVision-S 
%     &0.246 &49.37 &17.86 &21.15 &59.79 &19.11 \\
    
% \bottomrule
% \end{tabular}
% \label{tab:deeplearning}
% \end{table*}




% \begin{table*}[t!]
% \centering
% \caption{Performance of deep learning models for different liver cirrhosis stages on CirrMRI600+ T2-W MRI dataset.}  
%  \begin{tabular} {c|c|c|c|c|c}
% \toprule

% \textbf{{Liver cirrhosis stage}}  & \textbf{{Model}}  & \textbf{Precision (\%)$\uparrow$}  &\textbf{Sensitivity(\%)$\uparrow$} 
% &\textbf{Specificity(\%)$\uparrow$} 
% & \textbf{F1(\%)$\uparrow$} 
% \\
% \hline

%     \multirow{8}{*}{Mild} 

%     &VGG-19    
%     &32.09 &23.24 &68.62 &26.96 \\
    
%     &Resnet50             
%     &40.91 &43.78 &59.66 &42.30 \\
    
%     &ConvNext-B  
%     &19.57 &4.86 &87.24 &7.79 \\
    
%     % &VIT-B/16   
%     % &-	&-	
%     % &-	&-			\\
    
%     %&Swin-B          
%     % &0.501	&86.75	
%     % &-	&-	&-	&-	&-	&-\\
    
%     &PVTv2-B3            
%     &42.04 &35.68 &68.62 &38.60 \\
    
%     &VMamba-S 
%     &34.27 &26.49 &67.59 &29.88 \\
    
%     &Vim-S 
%     &56.00 &30.27 &84.83 &39.30 \\

%     &MedMamba-S 
%     &0.00 &0.00 &100.00 &0.00 \\
    
%     &MambaVision-S 
%     &27.65 &25.41 &57.59 &26.48 \\

%     \hline
    
%     \multirow{8}{*}{Moderate} 

%     &VGG-19    
%     &40.30 &44.02 &58.76 &42.08 \\
    
%     &Resnet50                   
%     &42.78 &43.48 &63.23 &43.13 \\
    
%     &ConvNext-B  
%     &40.84 &58.15 &46.74 &47.98 \\
    
%     %&VIT-B/16   
%     % &-	&-	
%     % &-	&-			\\
    
%     %&Swin-B          
%     % &0.501	&86.75	
%     % &-	&-	&-	&-	&-	&-\\
    
%     &PVTv2-B3         
%     &37.50 &52.17 &45.02 &43.64 \\
    
%     &VMamba-S 
%     &29.82 &36.96 &45.02 &33.01 \\
    
%     &Vim-S 
%     &42.25 &42.93 &62.89 &42.59 \\

%     &MedMamba-S 
%     &42.17 &80.43 &30.24 &55.33 \\
    
%     &MambaVision-S 
%     &25.93 &38.04 &31.27 &30.84 \\

%     \hline
%        \multirow{8}{*}{Severe} 

%     &VGG-19    
%     &18.57 &24.53 &69.11 &21.14 \\
    
%     &Resnet50                
%     &11.11 &9.43 &78.32 &10.20 \\
    
%     &ConvNext-B  
%     &19.76 &31.13 &63.69 &24.18 \\
    
%     % &VIT-B/16   
%     % &-	&-	
%     % &-	&-			\\
    
%     %&Swin-B          
%     % &0.501	&86.75	
%     % &-	&-	&-	&-	&-	&-\\
    
%     &PVTv2-B3            
%     &19.35 &11.32 &86.45 &14.29 \\
    
%     &VMamba-S 
%     &22.12 &21.70 &78.05 &21.90 \\
    
%     &Vim-S 
%     &20.74 &36.79 &59.62 &26.53 \\

%     &MedMamba-S 
%     &13.71 &16.04 &71.00 &14.78 \\
    
%     &MambaVision-S 
%     &0.00 &0.00 &90.51 &0.00 \\
    
% \bottomrule
% \end{tabular}
% \label{tab:deeplearningforallstages}
% \end{table*}



% \begin{table*}[t!]
% \centering
% \caption{Performance comparison of machine learning models on CirrMRI600+ T2-W MRI dataset using radiomics features.}
% %Red font represents baseline models. Paras, P, Se, Sp, F1 means Parameter size, Precision, Sensitivity, Specificity and F1-score, respectively.  
%  \begin{tabular} {c|c|c|c|c|c|c}
% \toprule

% \textbf{{Dataset}}  & \textbf{{Model}}  &\textbf{AUC$\uparrow$}   &\textbf{Precision(\%)$\uparrow$} & \textbf{Sensitivity(\%)$\uparrow$} & \textbf{Specificity(\%)$\uparrow$}  & \textbf{F1(\%)$\uparrow$}  
% \\ 
% \hline

%     \multirow{7}{*}{CirrMRI600+} 

%     &Decision Tree    
%     &0.321 &30.92 &30.54 &65.57 &30.58 \\
    
%     &Random Forest             
%     &0.300 &26.73 &27.11 &63.98 &26.90 \\
    
%     &KNeighbors  
%     &0.353 &32.94 &33.11 &66.37 &32.90 \\
    
%     %&VIT-B/16   
%     % &0.366	&85.85	
%     % &24.27	&26.23	&60.64	&25.20	\\
    
%     %&Swin-B          
%     % &0.501	&86.75	
%     % &-	&-	&-	&-	&-	&-\\
    
%     &SVC         
%     &0.277 &24.95 &25.12 &62.16 &24.86 \\
    
%     &GaussianNB 
%     &0.288 &24.30 &25.37 &62.81 &24.68 \\
    
%     &Logistic Regression 
%     &0.292 &26.62 &26.66 &63.31 &26.54 \\

%     &Gradient Boosting 
%     &0.296 &25.20 &26.22 &63.46 &25.64 \\
    
%     % &MambaVision-S 
%     % &0.478
%     % &37.27	&37.79	&69.24	&36.26	\\
    
% \bottomrule
% \end{tabular}
% \label{tab:machinelearning}
% \end{table*}





% \begin{table*}[t!]
% \centering
% \caption{Performance of machine learning models for different liver cirrhosis stages on CirrMRI600+ T2-W MRI dataset.}  
%  \begin{tabular} {c|c|c|c|c|c}
% \toprule

% \textbf{{Liver cirrhosis stage}}  & \textbf{{Model}}  & \textbf{Precision (\%)$\uparrow$}  &\textbf{Sensitivity(\%)$\uparrow$} 
% &\textbf{Specificity(\%)$\uparrow$} 
% & \textbf{F1(\%)$\uparrow$} 
% \\
% \hline

%     \multirow{7}{*}{Mild} 

%     &Decision Tree   
%     &39.35 &33.15 &67.47 &35.99 \\
    
%     &Random Forest             
%     &40.41 &42.39 &60.21 &41.38 \\
    
%     &KNeighbors
%     &43.96 &49.46 &59.86 &46.55 \\
    
%     % &VIT-B/16   
%     % &-	&-	
%     % &-	&-			\\
    
%     %&Swin-B          
%     % &0.501	&86.75	
%     % &-	&-	&-	&-	&-	&-\\
    
%     &SVC          
%     &36.51 &37.50 &58.48 &37.00 \\
    
%     &GaussianNB
%     &36.96 &36.96 &59.86 &36.96 \\
    
%     &Logistic Regression
%     &36.99 &34.78 &62.28 &35.85 \\

%     &Gradient Boosting 
%     &37.00 &40.22 &56.40 &38.54 \\
    
%     % &MambaVision-S 
%     % &45.38	&26.60	
%     % &76.62	&33.54			\\

%     \hline
    
%     \multirow{7}{*}{Moderate} 

%     &Decision Tree    
%     &35.94 &37.70 &57.59 &36.80 \\
    
%     &Random Forest                   
%     &29.35 &29.51 &55.17 &29.43 \\
    
%     &KNeighbors  
%     &30.73 &30.05 &57.24 &30.39 \\
    
%     %&VIT-B/16   
%     % &-	&-	
%     % &-	&-			\\
    
%     %&Swin-B          
%     % &0.501	&86.75	
%     % &-	&-	&-	&-	&-	&-\\
    
%     &SVC           
%     &24.64 &28.42 &45.17 &26.40 \\
    
%     &GaussianNB
%     &28.90 &34.43 &46.55 &31.42 \\
    
%     &Logistic Regression
%     &29.38 &33.88 &48.62 &31.47 \\

%     &Gradient Boosting
%     &31.09 &32.79 &54.14 &31.91 \\
    
%     % &MambaVision-S 
%     % &54.66	&70.54	
%     % &41.25	&61.59			\\

%     \hline
%        \multirow{7}{*}{Severe} 

%     &Decision Tree    
%     &17.46 &20.75 &71.66 &18.97 \\
    
%     &Random Forest                
%     &10.42 &9.43 &76.57 &9.90 \\
    
%     &KNeighbors  
%     &24.14 &19.81 &82.02 &21.76 \\
    
%     % &VIT-B/16   
%     % &-	&-	
%     % &-	&-			\\
    
%     %&Swin-B          
%     % &0.501	&86.75	
%     % &-	&-	&-	&-	&-	&-\\
    
%     &SVC           
%     &13.70 &9.43 &82.83 &11.17 \\
    
%     &GaussianNB
%     &7.04 &4.72 &82.02 &5.65 \\
    
%     &Logistic Regression
%     &13.48 &11.32 &79.02 &12.31 \\

%     &Gradient Boosting
%     &7.50 &5.66 &79.84 &6.45 \\
    
%     % &MambaVision-S 
%     % &11.76	&16.22	
%     % &89.86	&13.64			\\
    
% \bottomrule
% \end{tabular}
% \label{tab:machinelearningforallstages}
% \end{table*}

% \section*{Acknowledgments}
% \label{sec:acknowledgments}
%This research study was conducted retrospectively using human subject data made available in open access by Bilic et al.~\cite{bilic2023liver}. Ethical approval was not required as confirmed by the license attached with the open-access data.
%\vspace{-4mm}

% \subsection*{Conflicts of Interest}
%  The authors have no relevant financial or non-financial interests to disclose.

%\vspace{-7mm}