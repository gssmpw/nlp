\section{Introduction}\label{sec:introduction}

Throughout history, science has undergone a number of paradigm shifts, culminating in today's era of data-intensive exploration~\cite{hey2009jim}. Although new tools and frameworks have accelerated the pace of scientific discovery, its basic steps have remained unchanged for centuries. These include (1)~conception of a research question or problem, typically arising from a gap in disseminated knowledge; (2)~collection and study of existing literature or data relevant to the problem; (3)~formulation of a falsifiable hypothesis; (4)~design and execution of experiments to test this hypothesis; (5)~analysis and interpretation of the resulting data; and (6)~reporting on the findings, allowing for their exploitation in real-world applications or as a source of knowledge for a further iteration of the scientific cycle. A more detailed discussion of these steps is provided in Appendix \ref{sec:background}.

With the advent of large multimodal foundation models such as \href{https://chatgpt.com/}{ChatGPT}, \href{https://deepmind.google/technologies/gemini/}{Gemini}, \href{https://github.com/QwenLM/Qwen}{Qwen}, or \href{https://www.deepseek.com/}{DeepSeek}, many research fields and sectors of everyday life now stand at the threshold of an AI-based technological transformation. Science is no exception. A recent study analyzed approximately 148,000 papers from 22 non-CS fields that cited large language models (LLMs) between 2018 and 2024, reporting a growing prevalence of LLMs in these disciplines \cite{pramanick2024transformingscholarlylandscapesinfluence}. Additionally, a very recent survey among almost 5000 researchers in more than 70 countries, by the American Publishing Company Wiley, suggests that AI adoption is embraced by a majority of researchers who think that AI will become mainstream in science within the next two years, despite current usages often limited to forms of writing assistance.\footnote{\url{https://www.wiley.com/en-us/ai-study}, \url{https://www.nature.com/articles/d41586-025-00343-5}}

While science has traditionally relied on human ingenuity and labor for generating research ideas, formulating hypotheses, searching for relevant literature, conducting experiments, and reporting results, recent advancements in AI have introduced a surge of models and tools promising to assist researchers at every stage of this cycle. These include models like \href{https://elicit.com}{Elicit} or \href{https://ask.orkg.org/de}{ORKG ASK} for search; models like The AI Scientist \citep{lu2024aiscientist} for experimentation; and models like AutomaTikZ \citep{belouadi2024automatikz} and DeTikZify \citep{belouadi2024detikzify} for multimodal scientific content generation. Moreover, there is even research investigating the extent to which these AI models can evaluate scientific outcomes through automated peer review \cite{10.1613/jair.1.12862}. These AI-driven advancements promise to accelerate the scientific process, leading to unexpected discoveries, improved documentation, and more accessible research communication.\footnote{The benefits are expected to be particularly significant for non-native English speakers and those with lower technical skills, potentially increasing diversity and inclusivity in research.}

Despite this rapid progress, to our knowledge, there is no comprehensive survey covering the full breadth of AI-assisted tools, models, and functionalities available for supporting and improving the research cycle. Existing reviews are typically domain-specific, such as in the social sciences \cite{XU2024103665} or branches of physics \cite{Zhang2023ArtificialIF}.\footnote{We note two contemporaneous works developed completely independently from us \cite{zhang-etal-2024-comprehensive-survey,luo2025llm4srsurveylargelanguage}. Both are substantially narrower in scope than this survey; for example, \citet{luo2025llm4srsurveylargelanguage} neither cover multimodal approaches to scientific content synthesis nor search and also do not address ethical concerns in nearly the same depth as we do.} To address this gap, our survey provides an extensive overview of five central aspects of the research cycle where AI is making transformative contributions: (1) search and content summarization (Section \ref{sec:literature_search}); (2) scientific experimentation and research idea generation (Section \ref{sec:experiments}); (3) unimodal content generation, such as drafting titles, abstracts, suggesting citations, and assisting in text refinement (Section \ref{sec:textgeneration}); (4) multimodal content generation, including the creation and interpretation of figures, tables, slides, and posters (Section \ref{sec:multimodal}); and (5) AI-assisted peer review processes (Section \ref{sec:peer_review}). The recent Wiley survey underscores the significance of this endeavor, highlighting that ``63\% [of respondents indicated] a lack of clear guidelines and consensus on what uses of AI are accepted in their field and/or the need for more training and skills.''

When it comes to the use of AI tools in scientific research, ethical considerations are paramount. These tools exhibit various limitations, including (i) hallucinating and fabricating content, (ii) exhibiting bias, (iii) having limited reasoning abilities, (iv) lacking proper evaluation mechanisms, and (v) posing significant environmental costs. Additional concerns include risks of fake science, plagiarism, and diminished human oversight in scientific processes. The European Union has recently released guidelines on the responsible use of AI in science, emphasizing that, while [r]esearch is one of the sectors that could be most significantly disrupted by generative AI,'' and AI has great potential for accelerating scientific discovery and improving the effectiveness and pace of research and verification processes,'' it also ``raises questions about the ability of current models to combat deceptive scientific practices and misinformation.''\footnote{\url{https://research-and-innovation.ec.europa.eu/document/download/2b6cf7e5-36ac-41cb-aab5-0d32050143dc_en?filename=ec_rtd_ai-guidelines.pdf}} In this survey, we address these ethical considerations by including dedicated discussions within each of the five aspects of the research cycle and a standalone discussion in Section \ref{sec:ethics}.

%It is worth pointing out that our survey does not cover other aspects relating AI to science, e.g., of analysis of science with data science tools such as investigated in fields like the `science of science’ \cite{doi:10.1126/science.aao0185,wang2021science}.

As shown in Fig.~\ref{fig:overview}, the rest of this paper is structured as follows: \S\ref{sec:methodology} discusses the methodological approach of our survey. In \S\ref{sec:tasks}, each subsection describes the application of AI to an individual scientific task (literature search, experimental design, writing, etc.). For each task, we discuss important datasets, state-of-the-art methods and results, ethical concerns, domains of application, limitations, and future directions. \S\ref{sec:ethics} presents ethical considerations beyond individual tasks. Finally, \S\ref{sec:conclusion} summarizes the benefits and challenges of AI in science. %and identifies promising avenues for future work. %\todo{SE: make sure the conclusion contains what is promised here}

Resources related to this survey are available at \url{https://github.com/NL2G/TransformingScienceLLMs}. \se{A list of abbreviations and further material is available in our appendix.}

\begin{figure*}[htbp]
  \centering
  \includegraphics[width=0.85\textwidth]{image/overview_figure.pdf}
  \caption{Overview of the survey structure, including our survey methodology, five AI-assisted topics or tasks, and ethical considerations.} 
  \label{fig:overview}
\end{figure*}