The following sections \todo{SE: bullet points?}
provide an overview of these platforms and their distinctive functionalities.

\todo{SE: enumerate (a,b,c,d,...)?}
\begin{itemize}
\item \textbf{Elicit} is an AI research assistant designed to automate various research tasks, including searching for relevant papers, summarizing findings, extracting data, and synthesizing information across a vast database of over 125 million papers. It enables researchers to input specific questions and receive concise summaries, facilitating efficient literature exploration.
\item \textbf{Consensus} is an AI-powered academic search engine that allows users to query scientific literature using natural language. It searches through a repository of over 200 million research papers, providing instant insights with features like the \textit{Consensus Meter}, which aggregates and summarizes the degree of agreement across multiple studies on a given topic, and \textit{Pro Analysis}, which enables in-depth exploration of research trends, methodologies, and key findings. The platform leverages both custom %large language models (LLMs) 
\se{LLMs} 
and OpenAI technologies to deliver contextually relevant and reliable research findings. Additionally, Consensus offers advanced search filters that allow users to refine results based on study design, sample size, and access type, while quality indicators such as citation counts, journal impact, and study types help in identifying high-quality and credible sources. To further aid researchers, the \textit{Study Snapshot} feature provides a concise summary of study characteristics, including population size and methodology, directly within the search results. These capabilities position Consensus as an efficient tool for conducting literature reviews, facilitating evidence-based decision-making, and accelerating the research process across various scientific disciplines.  
\item \textbf{ResearchTrend.ai} is a platform focused on keeping users updated with the latest trends in %artificial intelligence 
\se{AI} 
research. It aggregates and presents information on AI-related papers, communities, and social events, enabling researchers to stay connected with advancements and collaborative opportunities in the AI field.
\item \textbf{OpenScholar} is an open-source platform designed to assist researchers in navigating and synthesizing scientific literature. It employs retrieval-augmented language models to answer user queries by searching for relevant papers and generating coherent responses based on the retrieved information. OpenScholar aims to enhance the efficiency of literature review processes and support evidence-based research practices.
\item \textbf{Typeset.io} is a comprehensive AI-powered search engine and research assistance platform that supports researchers throughout the entire research workflow, from literature discovery to manuscript preparation. The platform provides an extensive database of academic literature, enabling users to find relevant papers across multiple disciplines with AI-driven search capabilities that improve relevance through context-aware suggestions and topic-based exploration. Typeset.io offers nine key features that streamline research processes and enhance productivity. These features include \textit{Chat with PDF}, which allows users to interact with uploaded research papers by asking questions, extracting key insights, and obtaining explanations to better understand complex content. The \textit{Literature Review} feature helps users conduct structured literature reviews by summarizing key findings, identifying trends, and suggesting related works. The platform also offers \textit{AI Writer} and \textit{Paraphraser} tools to assist in drafting and refining academic content by generating text, improving readability, and ensuring compliance with academic writing standards. With the \textit{Find Topics} feature, users can discover emerging research topics and trends, helping them identify gaps and opportunities for further study. The \textit{Citation Generator} automates citation formatting according to various style guides, simplifying the referencing process. Additionally, the platform includes an \textit{Extract Data} tool that enables AI-driven extraction of quantitative and qualitative data from research papers, facilitating efficient information gathering. To ensure content authenticity, the \textit{AI Detector} identifies AI-generated text within research papers, promoting integrity in academic writing. A unique feature, \textit{PDF to Video}, converts research papers into video summaries, enabling quick comprehension of complex studies through visual storytelling. With this suite of features, Typeset.io provides an all-in-one solution for researchers looking to streamline their workflows and enhance the efficiency of their academic endeavors.
\end{itemize}