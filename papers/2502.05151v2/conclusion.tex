\section{Conclusion}\label{sec:conclusion}

%highlighting limitations and future work

In this paper, we surveyed approaches in the area of AI4Science, with a particular focus on recent large language model-based methods. We examined five key aspects of the research cycle: (1) search, (2) experimentation and research idea generation, (3) text-based content production, (4) multimodal content production, and (5) peer review. For each topic, we discussed relevant datasets, methods, and results, including evaluation strategies, while highlighting limitations and avenues for future research. Ethical concerns featured prominently in our survey, given the potential for misuse and challenges in maintaining scientific integrity in the face of AI-assisted content generation.

We hope that this survey inspires new initiatives in AI4Science, driving faster, more efficient, and more inclusive scientific discovery, experimentation, reporting and content synthesis---while upholding the highest ethical standards. %As the ultimate goal of science is to serve humanity, our hope is that these advancements will directly benefit society by accelerating knowledge creation and ensuring more accessible and trustworthy research.
As the ultimate goal of science is to serve humanity, we hope these advancements will accelerate knowledge creation and enhance the accessibility and reliability of research, leading to improved healthcare, medical treatments, economic processes, among a myriad of other societal benefits.
