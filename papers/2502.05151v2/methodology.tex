\section{Survey Methodology}\label{sec:methodology}

% how we carried out the survey

% \begin{itemize}
% \item Length: maybe 3-4 paragraphs?
% \item Audience, purpose/motivation, and enumerated contributions:
%   \begin{itemize}
%   \item thorough review of state of the art
%   \item connection to past/ongoing work in other disciplines
%   \item provide definitions
%   \item provide requirements
%   \item summarize evaluations
%   \item summarize data sets
%   \item identify future directions, open questions, opportunities, etc.
%   \end{itemize}

% \item Survey structure/type, including a preview of the organization of the rest of the article

% \item Scope of topic and inclusion/exclusion criteria for individual articles

% \item Contrast with previous surveys (if any)
%   \begin{itemize}
%   \item different structure?
%   \item broader/narrower topic?
%   \item different aims?
%   \item different conclusions?
%   \item different inclusion criteria?
%   \item outdated?
%   \end{itemize}
% \end{itemize}

This article offers a detailed, disciplinarily contextualized survey of state-of-the-art AI applications in scientific research, spanning every stage from the initial conception of ideas to the dissemination of results.  It is intended primarily to help researchers in fields within AI (natural language processing (NLP), computer vision (CV), etc.)\ quickly familiarize themselves with the transdisciplinary foundations of and latest developments in this broad-ranging and rapidly evolving research area.  Some of the material will also be useful to  policymakers, practitioners, and research collaborators in adjacent fields, including human--computer interaction, library and information science, communication studies, metascience, science journalism, and research ethics.  

We believe our contribution to be timely because, despite the growing interest in the topic, its researchers are only just now coalescing into a community with dedicated dissemination venues.  Recent examples include the workshops Natural Scientific Language Processing and Research Knowledge Graphs (NSLP)~\cite{rehm2024natural}, Foundation Models for Science (\href{https://fm-science.github.io/}{FM4Science}), AI \& Scientific Discovery (\href{https://ai-and-scientific-discovery.github.io/}{AISD}), and Towards a Knowledge-grounded Scientific Research Lifecycle (\href{https://sites.google.com/view/ai4research2024}{AI4Research}), all of which held their first editions in 2024 or 2025.  The few existing reviews of AI-for-science literature have addressed only isolated topics or application domains.  The earliest examples (e.g., \cite{langley2000computational,dzeroski2007computational}), now long out of date, tend to be organized into case studies of AI for specialized tasks such as equation or drug discovery.  More recent surveys, such as \cite{hastings2023ai}, cover a wider variety of application domains but focus on a narrower sector of the scientific life cycle, and are pitched more towards the potential users of the AI tools than towards AI researchers aiming to understand and advance the underlying data sets, methodologies, and evaluation metrics.

Given our topic's wide scope, rapid progress, and dependence on knowledge and methods from different domains, we have opted to take a \emph{narrative approach} to our survey.  This methodology allows for greater freedom in the selection and framing of relevant papers~\cite{king2005understanding}, which promotes ``breadth of literature coverage and flexibility to deal with evolving knowledge and concepts''~\cite{byrne2016improving} as well as the ability to ``bridge related areas of work, provoke thoughts, inspire new theoretical models, and direct future efforts in a research domain''~\cite{pare2015synthesizing}. \emph{Systematic reviews}, while regarded as more objective, are better suited to relatively narrow topics with well-defined, empirical research questions~\cite{pare2015synthesizing}.  Accordingly, we have adopted no fixed inclusion or exclusion criteria for the studies referenced in this survey, but have rather selected them on the basis of our own relevance judgements.  In assembling the co-authors for this survey, we have therefore endeavoured to include researchers actively publishing in each of the various subtopics we cover.
