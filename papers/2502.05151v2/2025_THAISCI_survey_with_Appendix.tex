%% Commands for TeXCount
%TC:macro \cite [option:text,text]
%TC:macro \citep [option:text,text]
%TC:macro \citet [option:text,text]
%TC:envir table 0 1
%TC:envir table* 0 1
%TC:envir tabular [ignore] word
%TC:envir displaymath 0 word
%TC:envir math 0 word
%TC:envir comment 0 0
%%
%%
%% The first command in your LaTeX source must be the \documentclass
%% command.
%%
%% For submission and review of your manuscript please change the
%% command to \documentclass[manuscript, screen, review]{acmart}.
%%
%% When submitting camera ready or to TAPS, please change the command
%% to \documentclass[sigconf]{acmart} or whichever template is required
%% for your publication.
%%
%%
\documentclass[manuscript,screen,nonacm]{acmart}

\usepackage[capitalise]{cleveref}
\usepackage{acronym}
\usepackage[breakable]{tcolorbox}
\usepackage[size=tiny,disable]{todonotes}
\usepackage[normalem]{ulem}
\usepackage{xcolor}
\usepackage{soul}
%%% new commands
%\newcommand{\mybox}[1]{\begin{tcolorbox}[colback=blue!10, colframe=blue, arc=3mm, %boxrule=0.5mm, width=\textwidth]
%#1
%\end{tcolorbox}}
\definecolor{boxback}{HTML}{dde5ef}
\definecolor{boxframe}{HTML}{4b72a6}
\newcommand{\mybox}[1]{\begin{tcolorbox}[breakable, colback=boxback, colframe=boxframe, arc=2mm, boxrule=0.3mm, width=\textwidth]
#1
\end{tcolorbox}}
%
\newcommand{\limitations}[1]{\begin{tcolorbox}[breakable, colback=red!30, colframe=blue, arc=3mm, boxrule=0.5mm, width=\textwidth]
#1
\end{tcolorbox}}

\usepackage{adjustbox}
\usepackage{smartdiagram}
\usepackage{multirow}
\usepackage{graphicx}
\usepackage{booktabs}
\usepackage{xcolor}
\usepackage{colortbl}

\usepackage[edges]{forest}
\definecolor{hidden-draw}{RGB}{106,142,189} 
\definecolor{hidden-blue}{RGB}{194,232,247} 
\definecolor{hidden-orange}{RGB}{217, 232, 252} 
\definecolor{layer-1}{HTML}{6a60a9}
\definecolor{layer-2}{HTML}{a5d296}
\definecolor{layer-3}{HTML}{FDD692}
\definecolor{layer-4}{HTML}{6AAFE6}
\definecolor{layer-5}{HTML}{FADAD8}
\definecolor{search-attribute-AI}{HTML}{D8E6E7}
\definecolor{search-attribute-Sum}{HTML}{f8ecc9}
\definecolor{search-attribute-Citation}{HTML}{f4f7f7}
\definecolor{search-attribute-Pers}{HTML}{d8e9ef}
\newcommand{\badge}[2]{
\begin{tikzpicture}
\node[
    draw, 
    fill=#2, 
    rounded corners=2pt, 
    inner xsep=8pt, 
    inner ysep=3pt, 
    minimum height=15pt,
    font=\sffamily
] {#1};
\end{tikzpicture}
}



\usepackage{makecell} 
\newcommand{\boldparagraph}[1]{\vspace{0.2cm}\noindent{\bf #1:} }
\newcommand{\collaborators}[1]{\textcolor{red}{#1}}
% was blue
\newcommand{\se}[1]{\textcolor{black}{#1}}
\usepackage{xspace}
\newcommand{\eg}{e.g.,\xspace}

\usepackage{jabbrv} % For ISO 4 abbreviations

%% Rights management information.  This information is sent to you
%% when you complete the rights form.  These commands have SAMPLE
%% values in them; it is your responsibility as an author to replace
%% the commands and values with those provided to you when you
%% complete the rights form.
\setcopyright{rightsretained}
\copyrightyear{2025}
\acmYear{2025}
\acmDOI{XXXXXXX.XXXXXXX}

%%
%% Submission ID.
%% Use this when submitting an article to a sponsored event. You'll
%% receive a unique submission ID from the organizers
%% of the event, and this ID should be used as the parameter to this command.
%%\acmSubmissionID{123-A56-BU3}

%%
%% The majority of ACM publications use numbered citations and
%% references.  The command \citestyle{authoryear} switches to the
%% "author year" style.
%%
%%\citestyle{acmauthoryear}

\begin{document}

%%
%% The "title" command has an optional parameter,
%% allowing the author to define a "short title" to be used in page headers.
\title[Transforming Science with Large Language Models]
{%What can AI/NLP do for science [production]: [A Survey/Review on AI for Science]}
%What can AI do for Scientific Content Synthesis? A Survey on AI for Science
Transforming Science with Large Language Models: 
A Survey on AI-assisted Scientific Discovery, Experimentation, Content Generation, and Evaluation 
}

%%
%% The "author" command and its associated commands are used to define
%% the authors and their affiliations.
%% Of note is the shared affiliation of the first two authors, and the
%% "authornote" and "authornotemark" commands
%% used to denote shared contribution to the research.

% Order of authors to be determined later.  Possible orderings:

% 1. Greatest to least contribution
% 2. Alphabetical order
% 3. Least to most seniority

\author{Steffen Eger}
\email{steffen.eger@utn.de}
\orcid{0000-0003-4663-8336}
\affiliation{%
  \institution{University of Technology Nuremberg (UTN)}
  \city{Nuremberg}
  \country{Germany}
}

\author{Yong Cao}
\email{yong.cao@uni-tuebingen.de}
\orcid{0000-0002-3889-0382}
\affiliation{%
  \institution{University of Tübingen, Tübingen AI Center}
  \city{Tübingen}
  \country{Germany}
}



% \author{Christine Bauer}
% \orcid{0000-0001-5724-1137}
% \email{christine.bauer@plus.ac.at}
% \affiliation{%
%   \institution{Paris Lodron University Salzburg}
%   \department{Department of Artificial Intelligence and Human Interfaces}
%   \city{Salzburg}
%   \country{Austria}
% }

\author{Jennifer D'Souza}
\email{jennifer.dsouza@tib.eu}
\orcid{0000-0002-6616-9509}
\affiliation{%
  \institution{TIB Leibniz Information Centre for Science and Technology}
  \city{Hannover}
  \country{Germany}
}


\author{Andreas Geiger}
\email{a.geiger@uni-tuebingen.de}
\orcid{0000-0002-8151-3726}
\affiliation{%
  \institution{University of Tübingen, Tübingen AI Center}
  \city{Tübingen}
  \country{Germany}
}


\author{Christian Greisinger}
\email{christian.greisinger@utn.de}
\orcid{}
\affiliation{%
  \institution{University of Technology Nuremberg (UTN)}
  \city{Nuremberg}
  \country{Germany}
}
\author{Stephanie Gross}
\email{stephanie.gross@ofai.at}
\orcid{0000-0002-9947-9888}
\affiliation{%
  \institution{Austrian Research Institute for Artificial Intelligence}
  \city{Vienna}
  \country{Austria}
}


\author{Yufang Hou}
\email{yufang.hou@it-u.at}
\orcid{0000-0003-2897-6075}
\affiliation{%
  \institution{IT:U Interdisciplinary Transformation University Austria}
  \city{Linz}
  \country{Austria}
}
\author{Brigitte Krenn}
\email{brigitte.krenn@ofai.at}
\orcid{0000-0003-1938-4027}
\affiliation{%
  \institution{Austrian Research Institute for Artificial Intelligence}
  \city{Vienna}
  \country{Austria}
}

\author{Anne Lauscher}
\email{anne.lauscher@uni-hamburg.de}
\orcid{0000-0001-8590-9827}
\affiliation{%
  \institution{University of Hamburg}
  \city{Hamburg}
  \country{Germany}
}


\author{Yizhi Li}
\email{yizhi.li-2@manchester.ac.uk}
\orcid{0000-0002-3932-9706}
\affiliation{%
  \institution{University of Manchester}
  \city{Manchester}
  \country{United Kingdom}
}
\author{Chenghua Lin}
\email{chenghua.lin@manchester.ac.uk}
\orcid{0000-0003-3454-2468}
\affiliation{%
  \institution{University of Manchester}
  \city{Manchester}
  \country{United Kingdom}
}
\author{Nafise Sadat Moosavi}
\email{n.s.moosavi@sheffield.ac.uk}
\orcid{0000-0002-8332-307X}
\affiliation{%
  \institution{University of Sheffield}
  \city{Sheffield}
  \country{United Kingdom}
}



\author{Wei Zhao}
\email{wei.zhao@abdn.ac.uk}
\orcid{0000-0001-7249-0094}
\affiliation{%
  \institution{University of Aberdeen}
  \city{Aberdeen}
  \country{United Kingdom}
}


\author{Tristan Miller}
\email{Tristan.Miller@umanitoba.ca}
\orcid{0000-0002-0749-1100}
\affiliation{%
  \institution{University of Manitoba}
  \city{Winnipeg}
  \state{Manitoba}
  \country{Canada}
}


%%
%% By default, the full list of authors will be used in the page
%% headers. Often, this list is too long, and will overlap
%% other information printed in the page headers. This command allows
%% the author to define a more concise list
%% of authors' names for this purpose.
\renewcommand{\shortauthors}{Eger et al.}

%%
%% The abstract is a short summary of the work to be presented in the
%% article.
\begin{abstract}
  %Abstract goes here
With the advent of large multimodal language models, science is now at a threshold of an AI-based technological transformation. Recently, a plethora of new AI models and tools have been proposed, promising to empower researchers and academics worldwide to conduct their research more effectively and efficiently. This includes all aspects of the research cycle, especially (1) searching for relevant literature; (2) generating research ideas and conducting experimentation; generating (3) text-based and (4) multimodal content (e.g., scientific figures and diagrams); and (5) AI-based automatic peer review. In this survey, we provide an in-depth overview over these recent advances, which promise to fundamentally alter the scientific research process for good. Our survey covers the five aspects outlined above, indicating relevant datasets, methods and results (including evaluation) as well as limitations and scope for future research. Ethical concerns regarding shortcomings of these tools and potential for misuse (fake science, plagiarism, harms to research integrity) take a particularly prominent place in our discussion. We hope that our survey will not only become a reference guide for newcomers to the field but also a catalyst for new AI-based initiatives in the area of ``AI4Science''.
\end{abstract}

%%
%% The code below is generated by the tool at http://dl.acm.org/ccs.cfm.
%% Please copy and paste the code instead of the example below.
%%
\begin{CCSXML}
<ccs2012>
   <concept>
       <concept_id>10003456.10003457.10003580.10003587</concept_id>
       <concept_desc>Social and professional topics~Assistive technologies</concept_desc>
       <concept_significance>300</concept_significance>
       </concept>
   <concept>
       <concept_id>10010405.10010432</concept_id>
       <concept_desc>Applied computing~Physical sciences and engineering</concept_desc>
       <concept_significance>300</concept_significance>
       </concept>
   <concept>
       <concept_id>10010405.10010444</concept_id>
       <concept_desc>Applied computing~Life and medical sciences</concept_desc>
       <concept_significance>300</concept_significance>
       </concept>
   <concept>
       <concept_id>10010405.10010455</concept_id>
       <concept_desc>Applied computing~Law, social and behavioral sciences</concept_desc>
       <concept_significance>300</concept_significance>
       </concept>
   <concept>
       <concept_id>10010147.10010178.10010179</concept_id>
       <concept_desc>Computing methodologies~Natural language processing</concept_desc>
       <concept_significance>500</concept_significance>
       </concept>
<concept>
<concept_id>10002944.10011122.10002945</concept_id>
<concept_desc>General and reference~Surveys and overviews</concept_desc>
<concept_significance>500</concept_significance>
</concept>
<concept>
<concept_id>10010147.10010178</concept_id>
<concept_desc>Computing methodologies~Artificial intelligence</concept_desc>
<concept_significance>500</concept_significance>
</concept>
 </ccs2012>
\end{CCSXML}

\ccsdesc[300]{Social and professional topics~Assistive technologies}
\ccsdesc[300]{Applied computing~Physical sciences and engineering}
\ccsdesc[300]{Applied computing~Life and medical sciences}
\ccsdesc[300]{Applied computing~Law, social and behavioral sciences}
\ccsdesc[500]{Computing methodologies~Natural language processing}
\ccsdesc[500]{General and reference~Surveys and overviews}
\ccsdesc[500]{Computing methodologies~Artificial intelligence}

%%
%% Keywords. The author(s) should pick words that accurately describe
%% the work being presented. Separate the keywords with commas.
%\keywords{Do, Not, Us, This, Code, Put, the, Correct, Terms, for,
%  Your, Paper}
\keywords{Language Language Models, Science, AI4Science, Search, Experimentation, Idea Generation, Multimodal Content Generation, Evaluation, Peer Review}

\iffalse COMMENT THIS BACK IN FOR JOURNAL SUBMISSION
\received{20 February 2007}
\received[revised]{12 March 2009}
\received[accepted]{5 June 2009}
\fi
%\todo{SE: Comment some things back in for journal submission}

%%
%% This command processes the author and affiliation and title
%% information and builds the first part of the formatted document.
\maketitle

%embed acronym definitions
\newacronym{rl}{RL}{Reinforcement Learning}
\newacronym{drl}{DRL}{Deep Reinforcement Learning}
\newacronym{mdp}{MDP}{Markov Decision Process}
\newacronym{ppo}{PPO}{Proximal Policy Optimization}
\newacronym{sac}{SAC}{Soft Actor-Critic}
\newacronym{epvf}{EPVF}{Explicit Policy-conditioned Value Function}
\newacronym{unf}{UNF}{Universal Neural Functional}

%\section{Introduction}

%Why is the topic important for the future of science, etc.
\todo{SE: Cover letter: \url{https://docs.google.com/document/d/1LrnFDk2z3oyKmEuJ2KCrgTOK86_OeVoQDQ5FZED0M3U/edit?usp=sharing}}




\section{Introduction}
\label{section:intro}



The landscape of large language models (LLMs) is rapidly evolving, with modern architectures capable of generating text from vast contexts. Recent advances have led to a significant increase in context window sizes, with Llama 3 \citep{dubey2024llama}, DeepSeekv3 \citep{liu2024deepseek}, and Gemini \citep{team2024gemini} supporting windows of at least 128k. 
However, long context modeling still poses significant challenges \citep{hsieh2024ruler} in terms of both accuracy  and  the substantial cost of processing long contexts in terms of memory and run-time compute. 
In spite of their importance, our current comprehension of the attention mechanism in long-context tasks remains incomplete. This work aims to address some of these knowledge gaps.

Despite the overwhelming complexity of state-of-the-art models, certain simple behaviors in the attention mechanism are strikingly consistent. In particular, many forms of sparse behaviors have been consistently observed, and exploited by numerous methods for efficient inference (see Section~\ref{sec:relatedworks}). 
Among them, \citet{xiao2023efficient} showed that even when computing the attention only using tokens close to the current token plus initial ``sink'' tokens, as illustrated in Figure~\ref{fig:gaussian},  the model is still capable of generating fluent text. We refer to these tokens as local window, and always implicitly include the   initial tokens as they play a crucial role as an attention ``sink'' (see also \citet{chen2024magicpig,gu2024attention,sun2024massive}). 

However, such a local window approximation, if applied to every attention head simultaneously, necessarily harms the capabilities of LLMs to retrieve and process long-context information (see e.g., \citet{xiao2024duoattention}).
Instead, to overcome such limitations, we aim to identify the heads whose output can be well-approximated via a local window attention, and apply the approximation to those only. If a head can be approximated via a local approximation, we call it a \textbf{local head}, and otherwise it is a \textbf{long-context head}. 
In particular, we ask:     Which heads can be approximated using a local window with minimal impact on downstream task performance?

 \begin{figure*}[ht]
\centering
\begin{subfigure}[b]{\linewidth}
\centering
        \includegraphics[width=1.0\linewidth]{plots/Figure1.pdf}\\
        \begin{subfigure}[b]{\linewidth}
    \end{subfigure}
\end{subfigure}
    \vspace{-1cm}
     \caption{
    \small{ 
    \textbf{Attention sparsity and its impact on efficiency.} 
    \textit{Left:} Attention scores are split into \textit{bulk} ($A^{\text{bulk}}$) for distant tokens and \textit{local window} ($A^{\text{local}}$) for nearby ones. A head is considered local if most of its attention mass falls within the local window. The static criterion pre-assigns local heads, while the adaptive oracle query-dependently compares bulk and local contributions but is computationally expensive. Our approximation models $A^{\text{bulk}}$ using a Gaussian distribution for efficiency.
    \textit{Middle:} Oracle-based classification with $\tau = 0.6$ (see Figure~\ref{fig:compare-approx} for the threshold) reveals three types of heads: consistently local (heads labeled more than $95\%$ of the times as local), often long-context (less than $50\%$), and varying, which switch behavior dynamically.
    \textit{Right:} Comparison of three methods: Static (green) removes a fixed fraction of heads, the adaptive oracle (blue) dynamically selects heads but is costly, and our adaptive method (purple) achieves near-oracle performance with significantly lower cost. As sparsity increases, static pruning degrades performance, while our adaptive method remains robust.
    These results show that \textit{most attention heads do not need to attend to the entire context}, enabling significant efficiency gains with \textit{query-adaptive} head classification.} }
    \label{fig:gaussian}
\end{figure*}



Two approaches to this problem can be distinguished:
  \textit{Static} criteria label the heads -- local vs long-context --  once for all queries, while \textit{query-adaptive} criteria change the labels from query to query.  Static criteria, as used by \citet{xiao2024duoattention,tang2024razorattention}, have the advantage that all key-value pairs (except for the few in the local window) of local heads can be discarded, thus saving memory. While recent works \citep{wu2024retrieval,tang2024razorattention,hong2024token} 
 provide some evidence that a \textit{fixed} small subset of the heads are particularly relevant for processing long-context information, the following question remains unclear:
 \begin{center}
     \textit{How much sparsity (measured as the average percentage of local heads) can we gain using query-adaptive criteria compared to static criteria?}
     \end{center} 


\paragraph{Contribution 1.} We present an extensive analysis comparing a query-adaptive oracle criterion, which selects local heads independently for each token, with static criteria. We make two observations: first, we find that static criteria can label up to 60\% of the heads as local heads without impacting downstream task evaluations, which confirms the intuition from \citep{wu2024retrieval}. Nevertheless, we find that a query-adaptive oracle criterion allows to label a substantially higher percentage of heads as local heads (up to 90\%) without sacrificing performance (see Figure~\ref{fig:gaussian}).


Unfortunately, the oracle requires the computation of the full attention scores. This leads to the following question:
\begin{center}
    \textit{ For each query, can we determine which heads are long-context and which are local without computing the full attention scores?}
\end{center}

The relevance of this question is twofold: on one hand, answering it helps guide further research toward developing more compute-efficient attention mechanisms. On the other hand, it advances our understanding of the inner workings of attention mechanisms, which is central to mechanistic interpretability (see, e.g., \citet{olsson2022context}). 

\paragraph{Contribution 2.} We address this question by proposing a novel query-adaptive attention criterion (QAdA) based on second-order statistics of the attention scores (briefly summarized in Figure~\ref{fig:gaussian}).
Our experiments on three families of LLMs, Llama \citep{dubey2024llama}, Qwen \citep{bai2023qwen} and Mistral \citep{jiang2023mistral} applied to a variety of standard long-context benchmarks, as well as hard  reasoning tasks embedded in long-context prompts, show that this relatively simple criterion allows to efficiently identify long-context heads: our method increased sparsity at a smaller loss in downstream performance than oracle static approaches. 
Along with our other experiments, it sheds light onto certain simple behaviors of attention heads in long-context settings. 


 



\section{Research Methodology}~\label{sec:Methodology}

In this section, we discuss the process of conducting our systematic review, e.g., our search strategy for data extraction of relevant studies, based on the guidelines of Kitchenham et al.~\cite{kitchenham2022segress} to conduct SLRs and Petersen et al.~\cite{PETERSEN20151} to conduct systematic mapping studies (SMSs) in Software Engineering. In this systematic review, we divide our work into a four-stage procedure, including planning, conducting, building a taxonomy, and reporting the review, illustrated in Fig.~\ref{fig:search}. The four stages are as follows: (1) the \emph{planning} stage involved identifying research questions (RQs) and specifying the detailed research plan for the study; (2) the \emph{conducting} stage involved analyzing and synthesizing the existing primary studies to answer the research questions; (3) the \emph{taxonomy} stage was introduced to optimize the data extraction results and consolidate a taxonomy schema for REDAST methodology; (4) the \emph{reporting} stage involved the reviewing, concluding and reporting the final result of our study.

\begin{figure}[!t]
    \centering
    \includegraphics[width=1\linewidth]{fig/methodology/searching-process.drawio.pdf}
    \caption{Systematic Literature Review Process}
    \label{fig:search}
\end{figure}

\subsection{Research Questions}
In this study, we developed five research questions (RQs) to identify the input and output, analyze technologies, evaluate metrics, identify challenges, and identify potential opportunities. 

\textbf{RQ1. What are the input configurations, formats, and notations used in the requirements in requirements-driven
automated software testing?} In requirements-driven testing, the input is some form of requirements specification -- which can vary significantly. RQ1 maps the input for REDAST and reports on the comparison among different formats for requirements specification.

\textbf{RQ2. What are the frameworks, tools, processing methods, and transformation techniques used in requirements-driven automated software testing studies?} RQ2 explores the technical solutions from requirements to generated artifacts, e.g., rule-based transformation applying natural language processing (NLP) pipelines and deep learning (DL) techniques, where we additionally discuss the potential intermediate representation and additional input for the transformation process.

\textbf{RQ3. What are the test formats and coverage criteria used in the requirements-driven automated software
testing process?} RQ3 focuses on identifying the formulation of generated artifacts (i.e., the final output). We map the adopted test formats and analyze their characteristics in the REDAST process.

\textbf{RQ4. How do existing studies evaluate the generated test artifacts in the requirements-driven automated software testing process?} RQ4 identifies the evaluation datasets, metrics, and case study methodologies in the selected papers. This aims to understand how researchers assess the effectiveness, accuracy, and practical applicability of the generated test artifacts.

\textbf{RQ5. What are the limitations and challenges of existing requirements-driven automated software testing methods in the current era?} RQ5 addresses the limitations and challenges of existing studies while exploring future directions in the current era of technology development. %It particularly highlights the potential benefits of advanced LLMs and examines their capacity to meet the high expectations placed on these cutting-edge language modeling technologies. %\textcolor{blue}{CA: Do we really need to focus on LLMs? TBD.} \textcolor{orange}{FW: About LLMs, I removed the direct emphase in RQ5 but kept the discussion in RQ5 and the solution section. I think that would be more appropriate.}

\subsection{Searching Strategy}

The overview of the search process is exhibited in Fig. \ref{fig:papers}, which includes all the details of our search steps.
\begin{table}[!ht]
\caption{List of Search Terms}
\label{table:search_term}
\begin{tabularx}{\textwidth}{lX}
\hline
\textbf{Terms Group} & \textbf{Terms} \\ \hline
Test Group & test* \\
Requirement Group & requirement* OR use case* OR user stor* OR specification* \\
Software Group & software* OR system* \\
Method Group & generat* OR deriv* OR map* OR creat* OR extract* OR design* OR priorit* OR construct* OR transform* \\ \hline
\end{tabularx}
\end{table}

\begin{figure}
    \centering
    \includegraphics[width=1\linewidth]{fig/methodology/search-papers.drawio.pdf}
    \caption{Study Search Process}
    \label{fig:papers}
\end{figure}

\subsubsection{Search String Formulation}
Our research questions (RQs) guided the identification of the main search terms. We designed our search string with generic keywords to avoid missing out on any related papers, where four groups of search terms are included, namely ``test group'', ``requirement group'', ``software group'', and ``method group''. In order to capture all the expressions of the search terms, we use wildcards to match the appendix of the word, e.g., ``test*'' can capture ``testing'', ``tests'' and so on. The search terms are listed in Table~\ref{table:search_term}, decided after iterative discussion and refinement among all the authors. As a result, we finally formed the search string as follows:


\hangindent=1.5em
 \textbf{ON ABSTRACT} ((``test*'') \textbf{AND} (``requirement*'' \textbf{OR} ``use case*'' \textbf{OR} ``user stor*'' \textbf{OR} ``specifications'') \textbf{AND} (``software*'' \textbf{OR} ``system*'') \textbf{AND} (``generat*'' \textbf{OR} ``deriv*'' \textbf{OR} ``map*'' \textbf{OR} ``creat*'' \textbf{OR} ``extract*'' \textbf{OR} ``design*'' \textbf{OR} ``priorit*'' \textbf{OR} ``construct*'' \textbf{OR} ``transform*''))

The search process was conducted in September 2024, and therefore, the search results reflect studies available up to that date. We conducted the search process on six online databases: IEEE Xplore, ACM Digital Library, Wiley, Scopus, Web of Science, and Science Direct. However, some databases were incompatible with our default search string in the following situations: (1) unsupported for searching within abstract, such as Scopus, and (2) limited search terms, such as ScienceDirect. Here, for (1) situation, we searched within the title, keyword, and abstract, and for (2) situation, we separately executed the search and removed the duplicate papers in the merging process. 

\subsubsection{Automated Searching and Duplicate Removal}
We used advanced search to execute our search string within our selected databases, following our designed selection criteria in Table \ref{table:selection}. The first search returned 27,333 papers. Specifically for the duplicate removal, we used a Python script to remove (1) overlapped search results among multiple databases and (2) conference or workshop papers, also found with the same title and authors in the other journals. After duplicate removal, we obtained 21,652 papers for further filtering.

\begin{table*}[]
\caption{Selection Criteria}
\label{table:selection}
\begin{tabularx}{\textwidth}{lX}
\hline
\textbf{Criterion ID} & \textbf{Criterion Description} \\ \hline
S01          & Papers written in English. \\
S02-1        & Papers in the subjects of "Computer Science" or "Software Engineering". \\
S02-2        & Papers published on software testing-related issues. \\
S03          & Papers published from 1991 to the present. \\ 
S04          & Papers with accessible full text. \\ \hline
\end{tabularx}
\end{table*}

\begin{table*}[]
\small
\caption{Inclusion and Exclusion Criteria}
\label{table:criteria}
\begin{tabularx}{\textwidth}{lX}
\hline
\textbf{ID}  & \textbf{Description} \\ \hline
\multicolumn{2}{l}{\textbf{Inclusion Criteria}} \\ \hline
I01 & Papers about requirements-driven automated system testing or acceptance testing generation, or studies that generate system-testing-related artifacts. \\
I02 & Peer-reviewed studies that have been used in academia with references from literature. \\ \hline
\multicolumn{2}{l}{\textbf{Exclusion Criteria}} \\ \hline
E01 & Studies that only support automated code generation, but not test-artifact generation. \\
E02 & Studies that do not use requirements-related information as an input. \\
E03 & Papers with fewer than 5 pages (1-4 pages). \\
E04 & Non-primary studies (secondary or tertiary studies). \\
E05 & Vision papers and grey literature (unpublished work), books (chapters), posters, discussions, opinions, keynotes, magazine articles, experience, and comparison papers. \\ \hline
\end{tabularx}
\end{table*}

\subsubsection{Filtering Process}

In this step, we filtered a total of 21,652 papers using the inclusion and exclusion criteria outlined in Table \ref{table:criteria}. This process was primarily carried out by the first and second authors. Our criteria are structured at different levels, facilitating a multi-step filtering process. This approach involves applying various criteria in three distinct phases. We employed a cross-verification method involving (1) the first and second authors and (2) the other authors. Initially, the filtering was conducted separately by the first and second authors. After cross-verifying their results, the results were then reviewed and discussed further by the other authors for final decision-making. We widely adopted this verification strategy within the filtering stages. During the filtering process, we managed our paper list using a BibTeX file and categorized the papers with color-coding through BibTeX management software\footnote{\url{https://bibdesk.sourceforge.io/}}, i.e., “red” for irrelevant papers, “yellow” for potentially relevant papers, and “blue” for relevant papers. This color-coding system facilitated the organization and review of papers according to their relevance.

The screening process is shown below,
\begin{itemize}
    \item \textbf{1st-round Filtering} was based on the title and abstract, using the criteria I01 and E01. At this stage, the number of papers was reduced from 21,652 to 9,071.
    \item \textbf{2nd-round Filtering}. We attempted to include requirements-related papers based on E02 on the title and abstract level, which resulted from 9,071 to 4,071 papers. We excluded all the papers that did not focus on requirements-related information as an input or only mentioned the term ``requirements'' but did not refer to the requirements specification.
    \item \textbf{3rd-round Filtering}. We selectively reviewed the content of papers identified as potentially relevant to requirements-driven automated test generation. This process resulted in 162 papers for further analysis.
\end{itemize}
Note that, especially for third-round filtering, we aimed to include as many relevant papers as possible, even borderline cases, according to our criteria. The results were then discussed iteratively among all the authors to reach a consensus.

\subsubsection{Snowballing}

Snowballing is necessary for identifying papers that may have been missed during the automated search. Following the guidelines by Wohlin~\cite{wohlin2014guidelines}, we conducted both forward and backward snowballing. As a result, we identified 24 additional papers through this process.

\subsubsection{Data Extraction}

Based on the formulated research questions (RQs), we designed 38 data extraction questions\footnote{\url{https://drive.google.com/file/d/1yjy-59Juu9L3WHaOPu-XQo-j-HHGTbx_/view?usp=sharing}} and created a Google Form to collect the required information from the relevant papers. The questions included 30 short-answer questions, six checkbox questions, and two selection questions. The data extraction was organized into five sections: (1) basic information: fundamental details such as title, author, venue, etc.; (2) open information: insights on motivation, limitations, challenges, etc.; (3) requirements: requirements format, notation, and related aspects; (4) methodology: details, including immediate representation and technique support; (5) test-related information: test format(s), coverage, and related elements. Similar to the filtering process, the first and second authors conducted the data extraction and then forwarded the results to the other authors to initiate the review meeting.

\subsubsection{Quality Assessment}

During the data extraction process, we encountered papers with insufficient information. To address this, we conducted a quality assessment in parallel to ensure the relevance of the papers to our objectives. This approach, also adopted in previous secondary studies~\cite{shamsujjoha2021developing, naveed2024model}, involved designing a set of assessment questions based on guidelines by Kitchenham et al.~\cite{kitchenham2022segress}. The quality assessment questions in our study are shown below:
\begin{itemize}
    \item \textbf{QA1}. Does this study clearly state \emph{how} requirements drive automated test generation?
    \item \textbf{QA2}. Does this study clearly state the \emph{aim} of REDAST?
    \item \textbf{QA3}. Does this study enable \emph{automation} in test generation?
    \item \textbf{QA4}. Does this study demonstrate the usability of the method from the perspective of methodology explanation, discussion, case examples, and experiments?
\end{itemize}
QA4 originates from an open perspective in the review process, where we focused on evaluation, discussion, and explanation. Our review also examined the study’s overall structure, including the methodology description, case studies, experiments, and analyses. The detailed results of the quality assessment are provided in the Appendix. Following this assessment, the final data extraction was based on 156 papers.

% \begin{table}[]
% \begin{tabular}{ll}
% \hline
% QA ID & QA Questions                                             \\ \hline
% Q01   & Does this study clearly state its aims?                  \\
% Q02   & Does this study clearly describe its methodology?        \\
% Q03   & Does this study involve automated test generation?       \\
% Q04   & Does this study include a promising evaluation?          \\
% Q05   & Does this study demonstrate the usability of the method? \\ \hline
% \end{tabular}%
% \caption{Questions for Quality Assessment}
% \label{table:qa}
% \end{table}

% automated quality assessment

% \textcolor{blue}{CA: Our search strategy focused on identifying requirements types first. We covered several sources, e.g., ~\cite{Pohl:11,wagner2019status} to identify different formats and notations of specifying requirements. However, this came out to be a long list, e.g., free-form NL requirements, semi-formal UML models, free-from textual use case models, UML class diagrams, UML activity diagrams, and so on. In this paper, we attempted to primarily focus on requirements-related aspects and not design-level information. Hence, we generalised our search string to include generic keywords, e.g., requirement*, use case*, and user stor*. We did so to avoid missing out on any papers, bringing too restrictive in our search strategy, and not creating a too-generic search string with all the aforementioned formats to avoid getting results beyond our review's scope.}


%% Use \subsection commands to start a subsection.



%\subsection{Study Selection}

% In this step, we further looked into the content of searched papers using our search strategy and applied our inclusion and exclusion criteria. Our filtering strategy aimed to pinpoint studies focused on requirements-driven system-level testing. Recognizing the presence of irrelevant papers in our search results, we established detailed selection criteria for preliminary inclusion and exclusion, as shown in Table \ref{table: criteria}. Specifically, we further developed the taxonomy schema to exclude two types of studies that did not meet the requirements for system-level testing: (1) studies supporting specification-driven test generation, such as UML-driven test generation, rather than requirements-driven testing, and (2) studies focusing on code-based test generation, such as requirement-driven code generation for unit testing.




%\section{Background}\label{sec:backgrnd}

\subsection{Cold Start Latency and Mitigation Techniques}

Traditional FaaS platforms mitigate cold starts through snapshotting, lightweight virtualization, and warm-state management. Snapshot-based methods like \textbf{REAP} and \textbf{Catalyzer} reduce initialization time by preloading or restoring container states but require significant memory and I/O resources, limiting scalability~\cite{dong_catalyzer_2020, ustiugov_benchmarking_2021}. Lightweight virtualization solutions, such as \textbf{Firecracker} microVMs, achieve fast startup times with strong isolation but depend on robust infrastructure, making them less adaptable to fluctuating workloads~\cite{agache_firecracker_2020}. Warm-state management techniques like \textbf{Faa\$T}~\cite{romero_faa_2021} and \textbf{Kraken}~\cite{vivek_kraken_2021} keep frequently invoked containers ready, balancing readiness and cost efficiency under predictable workloads but incurring overhead when demand is erratic~\cite{romero_faa_2021, vivek_kraken_2021}. While these methods perform well in resource-rich cloud environments, their resource intensity challenges applicability in edge settings.

\subsubsection{Edge FaaS Perspective}

In edge environments, cold start mitigation emphasizes lightweight designs, resource sharing, and hybrid task distribution. Lightweight execution environments like unikernels~\cite{edward_sock_2018} and \textbf{Firecracker}~\cite{agache_firecracker_2020}, as used by \textbf{TinyFaaS}~\cite{pfandzelter_tinyfaas_2020}, minimize resource usage and initialization delays but require careful orchestration to avoid resource contention. Function co-location, demonstrated by \textbf{Photons}~\cite{v_dukic_photons_2020}, reduces redundant initializations by sharing runtime resources among related functions, though this complicates isolation in multi-tenant setups~\cite{v_dukic_photons_2020}. Hybrid offloading frameworks like \textbf{GeoFaaS}~\cite{malekabbasi_geofaas_2024} balance edge-cloud workloads by offloading latency-tolerant tasks to the cloud and reserving edge resources for real-time operations, requiring reliable connectivity and efficient task management. These edge-specific strategies address cold starts effectively but introduce challenges in scalability and orchestration.

\subsection{Predictive Scaling and Caching Techniques}

Efficient resource allocation is vital for maintaining low latency and high availability in serverless platforms. Predictive scaling and caching techniques dynamically provision resources and reduce cold start latency by leveraging workload prediction and state retention.
Traditional FaaS platforms use predictive scaling and caching to optimize resources, employing techniques (OFC, FaasCache) to reduce cold starts. However, these methods rely on centralized orchestration and workload predictability, limiting their effectiveness in dynamic, resource-constrained edge environments.



\subsubsection{Edge FaaS Perspective}

Edge FaaS platforms adapt predictive scaling and caching techniques to constrain resources and heterogeneous environments. \textbf{EDGE-Cache}~\cite{kim_delay-aware_2022} uses traffic profiling to selectively retain high-priority functions, reducing memory overhead while maintaining readiness for frequent requests. Hybrid frameworks like \textbf{GeoFaaS}~\cite{malekabbasi_geofaas_2024} implement distributed caching to balance resources between edge and cloud nodes, enabling low-latency processing for critical tasks while offloading less critical workloads. Machine learning methods, such as clustering-based workload predictors~\cite{gao_machine_2020} and GRU-based models~\cite{guo_applying_2018}, enhance resource provisioning in edge systems by efficiently forecasting workload spikes. These innovations effectively address cold start challenges in edge environments, though their dependency on accurate predictions and robust orchestration poses scalability challenges.

\subsection{Decentralized Orchestration, Function Placement, and Scheduling}

Efficient orchestration in serverless platforms involves workload distribution, resource optimization, and performance assurance. While traditional FaaS platforms rely on centralized control, edge environments require decentralized and adaptive strategies to address unique challenges such as resource constraints and heterogeneous hardware.



\subsubsection{Edge FaaS Perspective}

Edge FaaS platforms adopt decentralized and adaptive orchestration frameworks to meet the demands of resource-constrained environments. Systems like \textbf{Wukong} distribute scheduling across edge nodes, enhancing data locality and scalability while reducing network latency. Lightweight frameworks such as \textbf{OpenWhisk Lite}~\cite{kravchenko_kpavelopenwhisk-light_2024} optimize resource allocation by decentralizing scheduling policies, minimizing cold starts and latency in edge setups~\cite{benjamin_wukong_2020}. Hybrid solutions like \textbf{OpenFaaS}~\cite{noauthor_openfaasfaas_2024} and \textbf{EdgeMatrix}~\cite{shen_edgematrix_2023} combine edge-cloud orchestration to balance resource utilization, retaining latency-sensitive functions at the edge while offloading non-critical workloads to the cloud. While these approaches improve flexibility, they face challenges in maintaining coordination and ensuring consistent performance across distributed nodes.






\section{AI Support for Individual Topics and Tasks\todo{JD: might we consider renaming this to something like "AI assistance in five stages of the research lifecycle"}}\label{sec:tasks}

The advancement of artificial intelligence in the legal domain has led to the development of various tools that assist in legal research, document retrieval, and automated legal reasoning. Several studies have explored the use of Natural Language Processing (NLP)\cite{khurana2023natural}, machine learning models, and vector-based search mechanisms to enhance the efficiency of legal chatbots. The primary focus of this literature review is on retrieval-augmented generation (RAG) models, FAISS-based document retrieval, deep learning for legal applications, and the use of large language models (LLMs) in legal AI.  

Recent research on Retrieval-Augmented Generation (RAG)\cite{gao2023retrieval} for legal AI has demonstrated its potential in enhancing legal text retrieval and summarization. S. S. Manathunga, Y. and A. Illangasekara\cite{manathunga2023retrieval} proposed a RAG-based model that improves legal text summarization by dynamically fetching relevant documents before generating responses. Similarly, Lee and Ryu \cite{ryu-etal-2023-retrieval} explored the application of RAG in case law retrieval, demonstrating its superiority over traditional keyword-based search engines. The introduction of RAG has significantly improved response accuracy by grounding AI-generated text in authoritative legal documents, reducing hallucinations in AI-driven legal assistance.  

% \begin{figure}[h]
%     \centering
%     \includegraphics[width=8cm]{FAISS.png}
%     \caption{Faiss: Efficient Similarity Search and Clustering of Dense Vectors}
%     \label{Overall Result of comparing FAISS and Chroma with different number of top documents}
% \end{figure}

The efficiency of FAISS (Facebook AI Similarity Search) in legal document retrieval has also been widely studied. Zhao et al. \cite{devlin-etal-2019-bert} implemented FAISS to enhance large-scale legal question answering systems, achieving significant improvements in retrieval speed and relevance. N. Goyal and D. Chen \cite{inbook} demonstrated that FAISS-based vector search mechanisms outperform conventional database searches in legal information retrieval, reducing query response time while maintaining high accuracy. The integration of FAISS with transformer-based models, as seen in the work of Hsieh and Wu, further enhances semantic retrieval, ensuring that chatbot responses align with actual legal texts.  

Transformer-based models such as BERT and GPT-based architecture have also contributed to the evolution of AI-driven legal research. Devlin et al. introduced BERT (Bidirectional Encoder Representations from Transformers), which significantly improved the understanding of legal language. RoBERTa, an optimized version of BERT, was later developed by Liu et al. \cite{liu2019roberta} to enhance contextual understanding and document similarity matching in legal queries. These models have been integrated into legal chatbots for contract analysis and legal decision-making, as demonstrated in the studies of Li et al. and Jin and Liu, where fine-tuned transformers improved legal text comprehension and summarization.  
The role of deep learning in legal AI has also been investigated extensively. Radford et al. introduced GPT-3, which paved the way for legal AI assistants capable of generating human-like responses. However, researchers such as Firth and Lee emphasized the limitations of LLMs in legal reasoning, arguing that these models require external verification mechanisms to prevent misinformation. The use of contrastive learning and fine-tuning for legal text retrieval has been explored by Arabi and Akbari \cite{article}, who demonstrated that embedding-based retrieval significantly improves chatbot response accuracy.  

Another significant area of research involves evaluating AI-generated legal responses using automated metrics. Zhang and Wu introduced BLEU\cite{10.3115/1073083.1073135} and ROUGE\cite{lin-2004-rouge} scores as a means to evaluate AI-generated legal text summaries, ensuring their quality and relevance. Similarly, Zhao et al. \cite{yuan2024rag} examined the effectiveness of RAG-based models in handling complex legal queries, highlighting the importance of legal consistency scores (LCS) in evaluating AI-driven responses.  

The practical applications of legal AI chatbots have been studied extensively in the context of access to justice and AI ethics. Wang and Cheng et al. \cite{xue2024bias} highlighted the potential of AI-driven legal assistants in bridging the justice gap, particularly in countries where legal resources are not easily accessible. Chan conducted a systematic review of retrieval-based legal chatbots, noting that while these systems improve accessibility, they also raise ethical concerns regarding legal misinformation and bias. Research by Min \cite{Min2023ARTIFICIALIA} explored methods for bias detection and mitigation in legal AI, ensuring fairness in AI-generated legal advice.  

Comparative studies between rule-based legal bots, keyword-driven legal search engines, and AI-powered legal chatbots further illustrate the superiority of retrieval-augmented approaches. In a study conducted by Zeng \cite{zeng2024scalable}, FAISS-based retrieval mechanisms significantly outperformed traditional Boolean keyword searches, reducing irrelevant document retrieval by 40\%. Singh \cite{10760929} further demonstrated that AI-powered legal research tools using NLP provide faster and more contextually accurate responses compared to standard legal databases.  

Despite these advancements, challenges remain in AI-driven legal research. Existing chatbots still struggle with multi-jurisdictional legal queries, as noted by Weichbroth \cite{Weichbroth2025AIAT}, who emphasized the need for jurisdiction-aware legal AI models. Additionally, legal AI models often lack the ability to process long-context legal arguments effectively, a limitation discussed by Gupta, who proposed memory-based retrieval techniques to improve long-form legal text processing.  

Research continues to refine AI-driven legal assistance, particularly in retrieval-augmented generation, FAISS-based search, transformer models, and deep learning techniques for legal research. However, further improvements are needed in bias mitigation, jurisdiction-specific adaptations, and long-context legal understanding. Future developments in multilingual legal AI, enhanced retrieval mechanisms, and AI-powered contract analysis will be crucial in making legal AI tools more accessible, reliable, and widely applicable in legal practice.
\section{Experiments}
\label{sec:experiments}
The experiments are designed to address two key research questions.
First, \textbf{RQ1} evaluates whether the average $L_2$-norm of the counterfactual perturbation vectors ($\overline{||\perturb||}$) decreases as the model overfits the data, thereby providing further empirical validation for our hypothesis.
Second, \textbf{RQ2} evaluates the ability of the proposed counterfactual regularized loss, as defined in (\ref{eq:regularized_loss2}), to mitigate overfitting when compared to existing regularization techniques.

% The experiments are designed to address three key research questions. First, \textbf{RQ1} investigates whether the mean perturbation vector norm decreases as the model overfits the data, aiming to further validate our intuition. Second, \textbf{RQ2} explores whether the mean perturbation vector norm can be effectively leveraged as a regularization term during training, offering insights into its potential role in mitigating overfitting. Finally, \textbf{RQ3} examines whether our counterfactual regularizer enables the model to achieve superior performance compared to existing regularization methods, thus highlighting its practical advantage.

\subsection{Experimental Setup}
\textbf{\textit{Datasets, Models, and Tasks.}}
The experiments are conducted on three datasets: \textit{Water Potability}~\cite{kadiwal2020waterpotability}, \textit{Phomene}~\cite{phomene}, and \textit{CIFAR-10}~\cite{krizhevsky2009learning}. For \textit{Water Potability} and \textit{Phomene}, we randomly select $80\%$ of the samples for the training set, and the remaining $20\%$ for the test set, \textit{CIFAR-10} comes already split. Furthermore, we consider the following models: Logistic Regression, Multi-Layer Perceptron (MLP) with 100 and 30 neurons on each hidden layer, and PreactResNet-18~\cite{he2016cvecvv} as a Convolutional Neural Network (CNN) architecture.
We focus on binary classification tasks and leave the extension to multiclass scenarios for future work. However, for datasets that are inherently multiclass, we transform the problem into a binary classification task by selecting two classes, aligning with our assumption.

\smallskip
\noindent\textbf{\textit{Evaluation Measures.}} To characterize the degree of overfitting, we use the test loss, as it serves as a reliable indicator of the model's generalization capability to unseen data. Additionally, we evaluate the predictive performance of each model using the test accuracy.

\smallskip
\noindent\textbf{\textit{Baselines.}} We compare CF-Reg with the following regularization techniques: L1 (``Lasso''), L2 (``Ridge''), and Dropout.

\smallskip
\noindent\textbf{\textit{Configurations.}}
For each model, we adopt specific configurations as follows.
\begin{itemize}
\item \textit{Logistic Regression:} To induce overfitting in the model, we artificially increase the dimensionality of the data beyond the number of training samples by applying a polynomial feature expansion. This approach ensures that the model has enough capacity to overfit the training data, allowing us to analyze the impact of our counterfactual regularizer. The degree of the polynomial is chosen as the smallest degree that makes the number of features greater than the number of data.
\item \textit{Neural Networks (MLP and CNN):} To take advantage of the closed-form solution for computing the optimal perturbation vector as defined in (\ref{eq:opt-delta}), we use a local linear approximation of the neural network models. Hence, given an instance $\inst_i$, we consider the (optimal) counterfactual not with respect to $\model$ but with respect to:
\begin{equation}
\label{eq:taylor}
    \model^{lin}(\inst) = \model(\inst_i) + \nabla_{\inst}\model(\inst_i)(\inst - \inst_i),
\end{equation}
where $\model^{lin}$ represents the first-order Taylor approximation of $\model$ at $\inst_i$.
Note that this step is unnecessary for Logistic Regression, as it is inherently a linear model.
\end{itemize}

\smallskip
\noindent \textbf{\textit{Implementation Details.}} We run all experiments on a machine equipped with an AMD Ryzen 9 7900 12-Core Processor and an NVIDIA GeForce RTX 4090 GPU. Our implementation is based on the PyTorch Lightning framework. We use stochastic gradient descent as the optimizer with a learning rate of $\eta = 0.001$ and no weight decay. We use a batch size of $128$. The training and test steps are conducted for $6000$ epochs on the \textit{Water Potability} and \textit{Phoneme} datasets, while for the \textit{CIFAR-10} dataset, they are performed for $200$ epochs.
Finally, the contribution $w_i^{\varepsilon}$ of each training point $\inst_i$ is uniformly set as $w_i^{\varepsilon} = 1~\forall i\in \{1,\ldots,m\}$.

The source code implementation for our experiments is available at the following GitHub repository: \url{https://anonymous.4open.science/r/COCE-80B4/README.md} 

\subsection{RQ1: Counterfactual Perturbation vs. Overfitting}
To address \textbf{RQ1}, we analyze the relationship between the test loss and the average $L_2$-norm of the counterfactual perturbation vectors ($\overline{||\perturb||}$) over training epochs.

In particular, Figure~\ref{fig:delta_loss_epochs} depicts the evolution of $\overline{||\perturb||}$ alongside the test loss for an MLP trained \textit{without} regularization on the \textit{Water Potability} dataset. 
\begin{figure}[ht]
    \centering
    \includegraphics[width=0.85\linewidth]{img/delta_loss_epochs.png}
    \caption{The average counterfactual perturbation vector $\overline{||\perturb||}$ (left $y$-axis) and the cross-entropy test loss (right $y$-axis) over training epochs ($x$-axis) for an MLP trained on the \textit{Water Potability} dataset \textit{without} regularization.}
    \label{fig:delta_loss_epochs}
\end{figure}

The plot shows a clear trend as the model starts to overfit the data (evidenced by an increase in test loss). 
Notably, $\overline{||\perturb||}$ begins to decrease, which aligns with the hypothesis that the average distance to the optimal counterfactual example gets smaller as the model's decision boundary becomes increasingly adherent to the training data.

It is worth noting that this trend is heavily influenced by the choice of the counterfactual generator model. In particular, the relationship between $\overline{||\perturb||}$ and the degree of overfitting may become even more pronounced when leveraging more accurate counterfactual generators. However, these models often come at the cost of higher computational complexity, and their exploration is left to future work.

Nonetheless, we expect that $\overline{||\perturb||}$ will eventually stabilize at a plateau, as the average $L_2$-norm of the optimal counterfactual perturbations cannot vanish to zero.

% Additionally, the choice of employing the score-based counterfactual explanation framework to generate counterfactuals was driven to promote computational efficiency.

% Future enhancements to the framework may involve adopting models capable of generating more precise counterfactuals. While such approaches may yield to performance improvements, they are likely to come at the cost of increased computational complexity.


\subsection{RQ2: Counterfactual Regularization Performance}
To answer \textbf{RQ2}, we evaluate the effectiveness of the proposed counterfactual regularization (CF-Reg) by comparing its performance against existing baselines: unregularized training loss (No-Reg), L1 regularization (L1-Reg), L2 regularization (L2-Reg), and Dropout.
Specifically, for each model and dataset combination, Table~\ref{tab:regularization_comparison} presents the mean value and standard deviation of test accuracy achieved by each method across 5 random initialization. 

The table illustrates that our regularization technique consistently delivers better results than existing methods across all evaluated scenarios, except for one case -- i.e., Logistic Regression on the \textit{Phomene} dataset. 
However, this setting exhibits an unusual pattern, as the highest model accuracy is achieved without any regularization. Even in this case, CF-Reg still surpasses other regularization baselines.

From the results above, we derive the following key insights. First, CF-Reg proves to be effective across various model types, ranging from simple linear models (Logistic Regression) to deep architectures like MLPs and CNNs, and across diverse datasets, including both tabular and image data. 
Second, CF-Reg's strong performance on the \textit{Water} dataset with Logistic Regression suggests that its benefits may be more pronounced when applied to simpler models. However, the unexpected outcome on the \textit{Phoneme} dataset calls for further investigation into this phenomenon.


\begin{table*}[h!]
    \centering
    \caption{Mean value and standard deviation of test accuracy across 5 random initializations for different model, dataset, and regularization method. The best results are highlighted in \textbf{bold}.}
    \label{tab:regularization_comparison}
    \begin{tabular}{|c|c|c|c|c|c|c|}
        \hline
        \textbf{Model} & \textbf{Dataset} & \textbf{No-Reg} & \textbf{L1-Reg} & \textbf{L2-Reg} & \textbf{Dropout} & \textbf{CF-Reg (ours)} \\ \hline
        Logistic Regression   & \textit{Water}   & $0.6595 \pm 0.0038$   & $0.6729 \pm 0.0056$   & $0.6756 \pm 0.0046$  & N/A    & $\mathbf{0.6918 \pm 0.0036}$                     \\ \hline
        MLP   & \textit{Water}   & $0.6756 \pm 0.0042$   & $0.6790 \pm 0.0058$   & $0.6790 \pm 0.0023$  & $0.6750 \pm 0.0036$    & $\mathbf{0.6802 \pm 0.0046}$                    \\ \hline
%        MLP   & \textit{Adult}   & $0.8404 \pm 0.0010$   & $\mathbf{0.8495 \pm 0.0007}$   & $0.8489 \pm 0.0014$  & $\mathbf{0.8495 \pm 0.0016}$     & $0.8449 \pm 0.0019$                    \\ \hline
        Logistic Regression   & \textit{Phomene}   & $\mathbf{0.8148 \pm 0.0020}$   & $0.8041 \pm 0.0028$   & $0.7835 \pm 0.0176$  & N/A    & $0.8098 \pm 0.0055$                     \\ \hline
        MLP   & \textit{Phomene}   & $0.8677 \pm 0.0033$   & $0.8374 \pm 0.0080$   & $0.8673 \pm 0.0045$  & $0.8672 \pm 0.0042$     & $\mathbf{0.8718 \pm 0.0040}$                    \\ \hline
        CNN   & \textit{CIFAR-10} & $0.6670 \pm 0.0233$   & $0.6229 \pm 0.0850$   & $0.7348 \pm 0.0365$   & N/A    & $\mathbf{0.7427 \pm 0.0571}$                     \\ \hline
    \end{tabular}
\end{table*}

\begin{table*}[htb!]
    \centering
    \caption{Hyperparameter configurations utilized for the generation of Table \ref{tab:regularization_comparison}. For our regularization the hyperparameters are reported as $\mathbf{\alpha/\beta}$.}
    \label{tab:performance_parameters}
    \begin{tabular}{|c|c|c|c|c|c|c|}
        \hline
        \textbf{Model} & \textbf{Dataset} & \textbf{No-Reg} & \textbf{L1-Reg} & \textbf{L2-Reg} & \textbf{Dropout} & \textbf{CF-Reg (ours)} \\ \hline
        Logistic Regression   & \textit{Water}   & N/A   & $0.0093$   & $0.6927$  & N/A    & $0.3791/1.0355$                     \\ \hline
        MLP   & \textit{Water}   & N/A   & $0.0007$   & $0.0022$  & $0.0002$    & $0.2567/1.9775$                    \\ \hline
        Logistic Regression   &
        \textit{Phomene}   & N/A   & $0.0097$   & $0.7979$  & N/A    & $0.0571/1.8516$                     \\ \hline
        MLP   & \textit{Phomene}   & N/A   & $0.0007$   & $4.24\cdot10^{-5}$  & $0.0015$    & $0.0516/2.2700$                    \\ \hline
       % MLP   & \textit{Adult}   & N/A   & $0.0018$   & $0.0018$  & $0.0601$     & $0.0764/2.2068$                    \\ \hline
        CNN   & \textit{CIFAR-10} & N/A   & $0.0050$   & $0.0864$ & N/A    & $0.3018/
        2.1502$                     \\ \hline
    \end{tabular}
\end{table*}

\begin{table*}[htb!]
    \centering
    \caption{Mean value and standard deviation of training time across 5 different runs. The reported time (in seconds) corresponds to the generation of each entry in Table \ref{tab:regularization_comparison}. Times are }
    \label{tab:times}
    \begin{tabular}{|c|c|c|c|c|c|c|}
        \hline
        \textbf{Model} & \textbf{Dataset} & \textbf{No-Reg} & \textbf{L1-Reg} & \textbf{L2-Reg} & \textbf{Dropout} & \textbf{CF-Reg (ours)} \\ \hline
        Logistic Regression   & \textit{Water}   & $222.98 \pm 1.07$   & $239.94 \pm 2.59$   & $241.60 \pm 1.88$  & N/A    & $251.50 \pm 1.93$                     \\ \hline
        MLP   & \textit{Water}   & $225.71 \pm 3.85$   & $250.13 \pm 4.44$   & $255.78 \pm 2.38$  & $237.83 \pm 3.45$    & $266.48 \pm 3.46$                    \\ \hline
        Logistic Regression   & \textit{Phomene}   & $266.39 \pm 0.82$ & $367.52 \pm 6.85$   & $361.69 \pm 4.04$  & N/A   & $310.48 \pm 0.76$                    \\ \hline
        MLP   &
        \textit{Phomene} & $335.62 \pm 1.77$   & $390.86 \pm 2.11$   & $393.96 \pm 1.95$ & $363.51 \pm 5.07$    & $403.14 \pm 1.92$                     \\ \hline
       % MLP   & \textit{Adult}   & N/A   & $0.0018$   & $0.0018$  & $0.0601$     & $0.0764/2.2068$                    \\ \hline
        CNN   & \textit{CIFAR-10} & $370.09 \pm 0.18$   & $395.71 \pm 0.55$   & $401.38 \pm 0.16$ & N/A    & $1287.8 \pm 0.26$                     \\ \hline
    \end{tabular}
\end{table*}

\subsection{Feasibility of our Method}
A crucial requirement for any regularization technique is that it should impose minimal impact on the overall training process.
In this respect, CF-Reg introduces an overhead that depends on the time required to find the optimal counterfactual example for each training instance. 
As such, the more sophisticated the counterfactual generator model probed during training the higher would be the time required. However, a more advanced counterfactual generator might provide a more effective regularization. We discuss this trade-off in more details in Section~\ref{sec:discussion}.

Table~\ref{tab:times} presents the average training time ($\pm$ standard deviation) for each model and dataset combination listed in Table~\ref{tab:regularization_comparison}.
We can observe that the higher accuracy achieved by CF-Reg using the score-based counterfactual generator comes with only minimal overhead. However, when applied to deep neural networks with many hidden layers, such as \textit{PreactResNet-18}, the forward derivative computation required for the linearization of the network introduces a more noticeable computational cost, explaining the longer training times in the table.

\subsection{Hyperparameter Sensitivity Analysis}
The proposed counterfactual regularization technique relies on two key hyperparameters: $\alpha$ and $\beta$. The former is intrinsic to the loss formulation defined in (\ref{eq:cf-train}), while the latter is closely tied to the choice of the score-based counterfactual explanation method used.

Figure~\ref{fig:test_alpha_beta} illustrates how the test accuracy of an MLP trained on the \textit{Water Potability} dataset changes for different combinations of $\alpha$ and $\beta$.

\begin{figure}[ht]
    \centering
    \includegraphics[width=0.85\linewidth]{img/test_acc_alpha_beta.png}
    \caption{The test accuracy of an MLP trained on the \textit{Water Potability} dataset, evaluated while varying the weight of our counterfactual regularizer ($\alpha$) for different values of $\beta$.}
    \label{fig:test_alpha_beta}
\end{figure}

We observe that, for a fixed $\beta$, increasing the weight of our counterfactual regularizer ($\alpha$) can slightly improve test accuracy until a sudden drop is noticed for $\alpha > 0.1$.
This behavior was expected, as the impact of our penalty, like any regularization term, can be disruptive if not properly controlled.

Moreover, this finding further demonstrates that our regularization method, CF-Reg, is inherently data-driven. Therefore, it requires specific fine-tuning based on the combination of the model and dataset at hand.
\subsection{Text-based Content Generation}
\label{sec:textgeneration}

%\collaborators{Stephanie, Wei, Steffen, Chenghua, Brigitte}

%Provide a concise description of the task here, indicate why it is important, and provide any necessary background information/references to contextualize the following subsections.
\mybox{Under text-based content generation for science, we subsume different tasks generating specific text-based subparts of a scientific paper, such as automatically generating (i) the title, (ii) the abstract, (iii) the related work section, as well as (iv) citation generation. Also, frameworks aiming to automate the full paper writing process will be discussed, as well as using AI systems for subtasks such as proof-reading, paraphrasing, and press release generation.}

\subsubsection{Data}

%Give an overview of the most important curated/annotated datasets, or sources of raw data, that are used (or potentially useful for) this task.

Open access research articles are a valuable data source for text-based content generation. These include scientific publisher repositories offering at least some open access content (e.g., \href{https://www.nature.com/nature-portfolio/for-authors/nature-research-journals}{Nature portfolio}, \href{https://www.tandfonline.com/}{Taylor \& Francis}) as well as preprint repositories (e.g., \href{https://arxiv.org/}{arXiv}, \href{https://www.biorxiv.org/}{bioRxiv}).
These open access repositories can be leveraged to develop datasets with pairs of titles and abstracts or abstract and conclusion/future work pairs. \citet{wang-etal-2019-paperrobot} for example extract (i) title to abstract pairs, (ii) abstract to conclusion and future work pairs, and (iii) conclusion and future work to title pairs from PubMed. Annotated, task-specific datasets for scientific text generation %, see 
\se{are presented in}
Table \ref{tab:data_text_generation}.  %\se{include}:


\begin{table}[th!]
\small
    \centering
    \begin{tabular}{p{2.8cm} p{4.8cm}p{2.8cm}p{3cm}}
    \toprule
       \textbf{Dataset}  & \textbf{Size} & \textbf{Sources} & \textbf{Application} \\
       \midrule
	   Abstract-title humor annotated dataset \cite{chen-eger-2023-transformers} & 2,638 humor annotated titles & ML \& NLP domain & Title generation\\
	   PaperRobot \cite{wang-etal-2019-paperrobot} & 27,001 title-abstract pairs; 27,001 abstract-conclusion \& future work pairs; 20,092 conclusion \& future work-title pairs & PubMed & Title generation, abstract generation, conclusion \& future work generation\\
        ScisummNet \cite{yasunaga2019scisummnet} & 1,000 papers + 20 citation sentences each & ACL Anthology Network & Related work generation\\
CORWA \cite{li-etal-2022-corwa} & 927 related work sections & NLP domain & Related work generation\\
		CiteBench \cite{funkquist-etal-2023-citebench} & 358,765 documents + citations & multiple, e.g., arXiv.org & Related work generation\\
		SciTechNews \cite{cardenas-etal-2023-dont} & 2,431 papers + press releases & ACM TechNews & Press release generation\\
\bottomrule
    \end{tabular}
    \caption{Annotated or task-specific datasets for scientific text generation.}
    \label{tab:data_text_generation}
\end{table}

\iffalse
\todo{SE: In face of Table 4, this itemize can now be removed, right? SG: yes}
\begin{itemize}
    \item Abstract to title humor annotated dataset \cite{chen-eger-2023-transformers}: 2,638 manually humor annotated titles (funny, not funny) from machine learning and natural language processing papers. Task: abstract to humorous title generation.
    \item PaperRobot, the PubMed term, abstract, conclusion, title dataset \citet{wang-etal-2019-paperrobot} contains three subsets: 27,001 title to abstract pairs, 27,001 abstract to conclusion and future work pairs, and 20,092 conclusion and future work to title pairs, all from publications from PubMed. Tasks: title generation, abstract generation, conclusion and future work generation
    \item ScisummNet \cite{yasunaga2019scisummnet} is a scientific article summarization dataset consisting of 1000 highly cited papers in computational linguistics and 20 sampled and cleaned citation sentences for each of those papers from the ACL Anthology Network. Task: related work generation.  
    \item CORWA is a dataset on citation oriented related work annotation \cite{li-etal-2022-corwa} and contains 927 manually annotated related work sections from the NLP domain. The data is annotated for the role of each related work sentence (discourse tag), the span of text whose information is directly derived from a specific cited paper (citation span detection), and whether a cited work is discussed in detail or high level (citation type recognition). Paper are from the NLP domain. Task: related work generation.
    \item The CiteBench dataset \cite{funkquist-etal-2023-citebench} is a citation text generation benchmark that brings together four existing task
designs on citation text generation by casting them into a single, general task definition, and unifying the respective datasets from \citet{aburaed:20}, \citet{chen-etal-2021-capturing}, \citet{lu-etal-2020-multi-xscience}, and \citet{xing-etal-2020-automatic}. Task: related work generation
\item The SciTechNews dataset \cite{cardenas-etal-2023-dont} consists of
2,431 scientific papers paired with their corresponding press release snippets mined from ACM TechNews. These papers are from a diverse pool of 
domains, including Computer Science, Physics, Engineering, and Biomedical. Task: press release generation
\end{itemize}
\fi

\subsubsection{Methods and Results}

%Describe the state-of-the-art methods and their results, noting any significant qualitative/quantitative differences between them where appropriate.
%Survey paper: \citet{zhang2024systematic}

\iffalse
\noindent\textbf{Title Generation}
\begin{itemize}
    \item Abstract-to-title: \citet{chen-eger-2023-transformers}
    \item A2T: \url{https://www.researchgate.net/profile/Vishal-Lodhwal-2/publication/369741619_Survey_Paper_Automatic_Title_Generation_for_Text_with_RNN_and_Pre-trained_Transformer_Language_Model/links/642fd66e20f25554da158ea3/Survey-Paper-Automatic-Title-Generation-for-Text-with-RNN-and-Pre-trained-Transformer-Language-Model.pdf}
    \item A2T: \citet{mishra2021automatic}
    \item Title-2-abstract: \citet{wang-etal-2019-paperrobot} 
\end{itemize}
\fi 
In the following, we survey approaches to generating %salient text-based parts of scientific papers, 
textual content for science, 
such as title, abstract, related work and bibliography. 
An overview of these processes \se{is given} in Appendix \ref{ax:content_generation}. 
%\todo{SG: can you please also add 'Paper Content' as basis for abstract generation in the figure?}. 

\paragraph{Title Generation.} %Several works have explored title generation. 
%Several works have considered title generation of scientific papers.
Generating adequate titles for scientific papers is an important task because titles are the first access point of a paper and can attract substantial reader interest; titles can also influence the reception of a paper \citep{letchford2015advantage}.  Consequently, several works have targeted generating titles automatically, often using paper abstracts as input. For example, \citet{mishra2021automatic} use a pipeline of three modules, viz.\ generation by transformer based (GPT2) models, selection (from multiple candidates) and refinement.  \citet{chen-eger-2023-transformers} also leverage transformers for title generation from abstracts but they in addition allow for generation of  humorous titles (which may be even more impactful) when an input flag is set appropriately. To achieve this, they annotate a training dataset of humorous titles from the fields of machine learning and NLP. %natural language processing. 
They explore different models including BART, GPT2, and T5 besides the more recent ChatGPT-3.5 LLM, finding that none of them can adequately generate humorous titles. They also explore generating titles from full texts instead of abstracts, with mixed results. \citet{wang-etal-2019-paperrobot} address the problem differently by drafting title names based on future work sections of previous related papers.
%\citet{wang-etal-2019-paperrobot} %address the problem more comprehensively, 
%consider %ing 
%paper part generation only as a subproblem of a more general `paper robot'. %\todo{BK: what does it mean to consider "paper part generation only as a subproblem", a subproblem of what?}  
%However, instead of generating titles from abstracts, they reverse the problem, generating abstracts from titles, in order to incrementally build up the paper drafting process, leveraging transformers and knowledge bases. \todo{SE: Perhaps then this should go to the next paragraph? SG: Moved it to 'abstract generation'}

\paragraph{Abstract Generation}
There are several approaches trying to assess the capabilities of proprietary LLMs to generate abstracts based on context information such as paper titles, journal names, keywords or the full text of the paper. \citet{hwang2024can} assess the ability of GPT 3.5 and GPT 4 to generate abstracts based on a full text. The results are manually evaluated using the Consolidated Standards of Reporting Trials for abstracts, a standard published with an aim to enhance the overall quality of scientific abstracts \cite{hopewell2008consort}. 
While the readability of abstracts generated by GPT is rated higher, their overall quality is inferior to the original abstracts. Also, minimal errors are reported in the AI generated abstracts. %\todo{SE: why are we switching to the past tense now? SG: sorry, I am so used to writing in past tense that I mixed it up}
\citet{wang-etal-2019-paperrobot} generate abstracts from titles, leveraging transformers and knowledge bases. Also generating abstracts from titles, \citet{gao2023comparing} collect 50 research abstracts from five medical journals and apply ChatGPT to generate research abstracts based on their titles and the name of one of the five journals. The original and the generated abstracts are then evaluated with AI output detectors and with blinded human reviewers to identify which of the abstracts are automatically generated. Human reviewers are able to identify 68\% of the generated abstracts as being automatically generated, but also incorrectly identify 14\% of original abstracts as being LLM generated.  
Applying AI output detectors, most generated abstracts can be identified by the GPT-2 Output Detector assigning a median of 99.98\% generated scores to generated abstracts and a median 0.02\% to original abstracts. 
However, \citet{anderson2023ai} have shown that after automatically paraphrasing AI generated text, the performance of AI detectors such as GPT-2 Output Detector decrease drastically. 
\citet{farhat2023trustworthy} evaluate the performance of ChatGPT generating abstracts based on 3 keywords, the name of a database (Scopus or web of science) and the task to analyze bibliographic data  in the domain indicated by the keywords. % domain in existing literatureconduct a bibliometric analysis. \todo{SE: don't understand this sentence... "the task to conduct a bibliometric analysis?}
%Bibliometrics is the application of statistical methods to identify prolific authors, top avenues, leading countries of a particular domain in existing literature. \todo{SE: remove sentence?}
The authors then compare the generated abstract to an actual abstract on the same topic. %\todo{SE: why past tense?}
After a manual comparison of the results, the authors come to the conclusion that at the time the study was conducted, ChatGPT is not a trustworthy tool for retrieving and assessing bibliographic data. %However, they emphasize the usefulness as a writing assistant tool for improving readability, language enhancement, rephrasing/paraphrasing and proofreading.    

\paragraph{Long Text Generation}  %\todo{SE: this paragraph should maybe be included in the section description: that we are doing this and why this is important? SG: I added it to the section description}
Some approaches aim at automating the full paper writing process. The \textbf{AI Scientist} \cite{lu2024aiscientist} presents a comprehensive framework designed to support the entire scientific research cycle, encompassing tasks such as idea generation, hypothesis formulation, experimental planning, and execution. While its primary focus is not on long-form text generation, AI Scientist is able to generate entire scientific papers. By incorporating structured scientific knowledge (e.g. experimental results), the framework can draft papers that adhere to domain-specific requirements, involving the integration of relevant citations and conforming to disciplinary norms. Despite its ability to produce comprehensive paper drafts, the framework does not explicitly address the challenge of maintaining coherence across extended narratives, and their dependencies. 
\textbf{LongWriter} \cite{bai2024longwriter} and \textbf{LongEval} \cite{wu2025longeval} directly address the challenge of generating extended text by introducing architectural modifications aimed at enhancing coherence and structural consistency in long-form outputs. The framework employs hierarchical attention mechanisms to ensure thematic consistency across long text and applies fine-tuning strategies to align outputs with user prompts. LongWriter conducts experiments on several domains, including academic and monograph texts. For academic content, the model %demonstrated its ability to 
can generate structured arguments and effectively incorporate domain-specific terminologies. However, noticeable issues remain around factual consistency, the integration of citations, and redundancy in the generated text. %\todo{SE: In this whole paragraph, to save space, we could remove the line breaks and bold the method names such as LongWriter.} 
However, by conducting experiments on various models in academic, wikipedia and blog domains, LongEval shows that the larger models trained with general instruction data performs similar to those specifically trained (e.g., LongWriter).
%LongCitez \cite{zhang2024longcite} 
%LongCite \cite{zhang2024longciteenablingllmsgenerate} \todo{SE: there was no reference and a paper called LongCitez seemingly doesn't exist. I inserted a paper called LongCite}
%emphasizes the integration of citation-based context in long-form text generation. By training models on citation-rich datasets, LongCitez ensures that generated content aligns with existing scientific discourse and appropriately references relevant literature. 
\textbf{LongReward} %\cite{zhang2024longreward}
\cite{zhang2024longrewardimprovinglongcontextlarge}
leverages reinforcement learning to improve long-text generation. The model employs custom reward signals that prioritize coherence, factual accuracy, and linguistic quality. These reward mechanisms are particularly relevant for scientific text generation, where accuracy and adherence to domain-specific conventions are crucial.


\paragraph{Related Work Generation} %\citet{li-ouyang-2024-related,li2022generating,hu-wan-2014-automatic,shah2021generatingrelatedwork} 

Already in the past, there has been a substantial body of work on related work generation through text summarization, most of which differ in their approach (extractive or abstractive) and the length of citation text (sentence-level or paragraph-level). Extractive approaches focus on selecting sentences from cited papers and reordering the extracted sentences to form a paragraph of related work. For instance, \citet{hoang-kan-2010-towards} propose an extractive summarization approach that selects sentences describing the cited papers to generate the related work section of a target paper. This approach relies on the full text of the target paper. Subsequent extractive approaches differ from this approach in how they order the extracted sentences: While \citet{wang-etal-2018-neural-related}, \citet{chen2019automatic}, and \citet{wang2019toc} assume that the sentence order is given,  \citet{hu2014automatic} and \citet{deng2021automatic} take advantage of an automatic approach to reorder sentences based on topic coherence. However, extractive approaches often struggle to produce coherent text, as they simply concatenate sentences without ensuring a cohesive narrative flow. In contrast, abstractive related work generation leverages devices of rewriting and restructuring to generate a summary of a cited paper. Most of the abstractive approaches are based on language models and focus on either generating (a) a single sentence from a single reference 
or (b) a paragraph from multiple references. Typically, the abstractive process is repeated multiple times until a related work section is complete.
% Typically, the abstract of a reference is given as input. 
% For instance, 
\citet{abura2020automatic} introduce an abstractive summarization approach to generate citation sentences in a single-reference setup. Their approach has been trained on the \textbf{ScisummNet} corpus with paper abstracts as inputs and citation sentences as outputs. \citet{li-etal-2022-corwa} further extend this idea to a multiple-reference setup, namely generating a paragraph of citation sentences from various cited papers. Their approach has been trained on the \textbf{CORWA} corpus to generate both citation and transition sentences. Additionally, instead of using paper abstracts as inputs, \citet{li2024cited} propose to retrieve relevant sentences from cited papers to generate citation sentences. More recently, works such as \citet{sahinuc-etal-2024-systematic} 
% argue that citation intents play an important role for related work generation and 
explore 
% systematically assess the impact of 
instruction promoting with %large language models, 
\se{LLMs}, 
which is alternative to extractive and abstractive approaches, to generate citation sentences.
% on citation text generation outputs.
Overall, 
% both extractive and abstractive approaches are widely used for generating citation sentences. 
extractive approaches, while factual, often lack fluency and coherence. In contrast, abstractive approaches and instruction prompting, which are based on (large) language models, do not struggle with these issues, however, they suffer from factual errors, known as hallucination. %\todo{SE: limitation section?} SG: I think it fits better here... However, if you prefer to move it to the limitation section, it is also fine with me.
 


\paragraph{Citation Generation} %\citet{li-ouyang-2024-related,huang2023citation,li2024citation,farhat2023trustworthy}, Hallucinations in citation-enhanced generation \citet{li2024citation}
Bibliographic references in scientific papers are important components for ensuring the scientific integrity of the authors. However, in many cases, cited articles of bibliographic references generated by LLMs such as ChatGPT are reported not to exist, that is, are hallucinated or incorrect \cite{li-ouyang-2024-related,huang2023citation,li2024citation,farhat2023trustworthy}. Most of the studies reporting hallucinated or erroneous bibliographic references are case studies presenting one or more examples. 
\citet{walters2023fabrication}, however, present a study in which they use ChatGPT-3.5 and ChatGPT-4 to produce 84 documents (short reviews of the literature) on 42 multidisciplinary topics. The resulting documents contain 636 bibliographic citations, which are further analyzed for errors and hallucinations. Their results show that 55\% of the GPT-3.5 citations but only 18\% of the GPT-4 citations are fabricated. Of the actual existing (non-fabricated) GPT-3.5 citations, 43\% include substantive citation errors, and of the non-fabricated GPT-4 citations it is 24\%. 
%\todo{SE: why are we switching to past tense now again? SG: sorry}
Even though this is a major improvement from GPT-3.5 to GPT4, problems with fabrication and errors in bibliographic citations remain. %\todo{SE: Any evidence? SG: The evidence is in the preceding sentence - I tried to make it clearer} 
Therefore, for generated citations and references, it is of particular importance to ensure their accuracy and completeness. %\todo{SE: Actually, this can be highlighted as a limitation of these studies (e.g., based on GPT3.5). LLMs are advancing rapidly, conclusions are quickly outdated SG: added it to the limitations section}

\paragraph{Proof-reading and Paraphrasing.} %\citet{huang2023role,salvagno2023can,kim2023using,castellanos2023good}
LLMs such as ChatGPT have been reported to provide useful assistance for scientific writing with regards to proof-reading and language review in order to enhance the readability of the paper. Subtasks these models can provide support for during the writing process include providing suggestions for improving the writing style, or proof-reading \cite{salvagno2023can}. Additionally, some authors emphasize that LLMs %such as ChatGPT \todo{SE: do we always need to say ``LLMs such as ChatGPT''?}
can be helpful especially for non-native English speakers with regards to grammar, sentence structure, vocabulary and even translation, i.e., providing an English editing service \cite{huang2023role,castellanos2023good,kim2023using}. Most papers on this topic are case studies, illustrating their research questions with one or more examples and their results are qualitatively evaluated by a human expert (typically the author of the paper). \citet{hassanipour2024ability} evaluate the effectiveness of ChatGPT in rephrasing not for improving the writing style, but for reducing plagiarism in the process of scientific paper writing. The results showed that even with explicit instructions to paraphrase or reduce plagiarism, the plagiarism rate remained relatively high.

%Hallucinations in scientific writing \citet{alkaissi2023artificial}, hallucinations in systematic reviews \citet{chelli2024hallucination}

\paragraph{Press Release Generation.}  Several studies attempt to generate press release articles for the general public based on scientific papers. \citet{cao-etal-2020-expertise} construct a manually annotated dataset for expertise-style transfer in the medical domain and apply various style transfer and sentence simplification models to convert expert-level language into layman’s terms. \citet{goldsack-etal-2022-making} develop standard seq-to-seq models to generate news summaries for scientific articles. Lastly, \citet{cardenas-etal-2023-dont} propose a framework that integrates metadata from scientific papers and scientific discourse structures to model journalists’ writing strategies. %\todo{SE: the last sentence miraculously switches back to present tense}

% %%%%%%%%%%%% Moved to appendix  %%%%%%%%%%%%%%%%
\subsubsection{Ethical Concerns}

%Identify and discuss any ethical issues related to the (mis)use of the data or the application of the methods, as well as strategies for mitigations.

In scientific work, authorship and plagiarism in AI generated texts are major concerns. In general, it is a challenge to distinguish between AI generated and human generated texts. %Although there is a number of tools to detect AI-generated text (e.g., GPTZero or Hive), \citet{anderson2023ai} have shown that after applying automatic paraphrasing, the detection of human generated text using GPT-2 Output Detector increased, e.g., the probability of the text being generated by a human from 0.02\% to 99.52\%.
Although there is a number of tools to detect AI-generated text (e.g., GPTZero or Hive), \citet{anderson2023ai} show that after applying automatic paraphrasing to AI generated text, the probability of a text to be human generated, increases. %todo{SE: repeated text}
%identified by GPT-2 Output Detector to be written by a human, increased (e.g., from 0.02\% to 99.52\%).
Therefore it is not possible to reconstruct if a text is an original work from a scientist or has been generated by an AI. 
In addition, it is also found that ChatGPT generated texts easily pass plagiarism detectors \cite{else2023chatgpt,altmae2023artificial}. 
Moveover, \citet{macdonald2023can} raise the concern that the frequent use of LLMs for drafting research articles might lead to similar paragraphs and structure of many papers in the same field. This again raises the question whether there should be a threshold for the acceptable amount of AI-generated content in scientific work \cite{macdonald2023can}.
%%%%%%%%%%%%%%%%%%%%%%%%%%%%%%%%%%%%%%%%%%%%%%%%%%%%%


\subsubsection{Domains of Application}

%\todo{Indicate whether any of the data, methods, ethical concerns, etc. are specific to a given domain (biology, health, computer science, etc.).}

%\todo{SE: shouldn't this section be about different domains?}

Text-based content generation is relevant for all scientific domains. \citet{liang2024mapping} conduct a large-scale analysis across 950,965 paper published between January 2020 and February 2024 to measure the prevalence of LLM modified content over time. The papers they investigated were published on (i) arXiv including the five areas Computer Science, Electrical Engineering and Systems Science, Statistics, Physics, and Mathematics, (ii) bioRxiv, and (iii) Nature portfolio. Their results show the largest and fastest growth in Computer Science with up to 17.5\% of the papers containing LLM modified content %\todo{SE: what does this number mean? SG: tried to clarify it}
and the least LLM modifications in Mathematics papers (up to 6.3\%). However, according to the Natural Language Learning \& Generation arXiv report from September 2024, top-cited papers show notably fewer markers of AI-generated content compared to random samples \cite{Leiter2024NLLGQA}.

\subsubsection{Limitations and Future Directions}

%\todo{Summarize the limitations of current approaches; point out any notable gaps in the research and future directions.}

Numerous studies have investigated text-based content generation for the scientific domain and have shown their potential to assist scientists in different phases of writing a paper. While for some tasks such as proof-reading and paraphrasing, its capabilities are well established, others pose limitations. Therefore it is crucial that automatically generated text is always assessed by a human expert. Factual consistency and truthfulness are issues which need to be reviewed by a human in the loop %\todo{SE: here it is not an adjective, so I would remove the hyphens}
for all types of text-based generated content. Current proprietary LLMs for example struggle in particular with generating existing and correct bibliographic citations. However, LLMs are advancing rapidly and studies evaluating LLMs are quickly outdated. Still, several ethical issues arise when text-generating systems are included in the scientific writing process, such as authorship, plagiarism, bias, and truthfulness. Therefore, in future research a focus on trustworthy, ethical AI systems is required. 

%\subsubsection{AI use case}

%\todo{Optional: describe which portions of your section (figures, tables, text, etc.) have been assisted by AI and how.}

%\subsubsection{Limitations and future directions}
% \newpage
\section{Multimodal Deepfakes}
\label{sec:multimodal_deepfakes_intro} 

The concept of multi-modality in deep learning involves integrating and processing data from various sources simultaneously. These sources can encompass text, images, audio, video, and sensor data. By leveraging different data types, multi-modal deep learning models can capture more comprehensive and diverse information, resulting in enhanced performance for tasks that require understanding the relationships between different data types \cite{gao2020survey, summaira2021recent, jabeen2023review}. In the realm of deepfakes, multi-modality entails using various types of data, such as images, audio, and video, to create highly realistic synthetic media that convincingly mimics real-world content, including visuals and sounds \cite{khalid2021fakeavceleb, hou2024polyglotfake}. Through aligning and synchronizing these modalities, deepfakes can produce seamless and coherent fake content, such as matching a person's lip movements to a different audio track or convincingly cloning their voice \cite{pei2024deepfake, prajwal2020lip, cheng2022videoretalking, lomnitz2020multimodal}. While this technology offers positive applications in entertainment, media, and education, such as creating special effects and developing realistic training simulations, it also poses significant ethical and security challenges \cite{pandey2021deepfakes}. These include the potential for misuse in misinformation, impersonation, and fraud. Detecting and preventing malicious deepfakes is a burgeoning area of research aimed at ensuring the responsible use of this powerful technology. As multi-modal deepfakes continue to evolve, it is crucial to balance innovation with ethical considerations to mitigate risks and maximize benefits \cite{khalid2021evaluation, liu2023magnifying, cheng2023voice}. In this section, we will investigate state-of-the-art methodologies for the generation (refer Section \ref{subsec:multimodal_generation}) and the detection of multi-modal deepfakes (refer Section \ref{subsec:multimodal_detection}). We will analyze the advanced and innovative techniques outlined in the existing literature, alongside the datasets utilized for deepfake detection and generation.

\subsection{Multimodal Deepfake Generation}
\label{subsec:multimodal_generation}

Combining audio and video deepfakes involves a sophisticated process of synchronizing lip movements with synthetic speech to produce seamless and coherent content \cite{liz2024generation}. This multi-modal approach, which integrates both visual and auditory elements, significantly enhances the realism of the generated media. By ensuring that the audio matches the lip movements and facial expressions perfectly, these deepfakes become more lifelike and convincing, making detection increasingly challenging \cite{hou2024polyglotfake}. Audio-visual multimodal deepfakes can be categorized into three main types based on the modality being faked \cite{khalid2021fakeavceleb}. This categorization helps in understanding the different ways in which deepfakes can manipulate audio and visual components to create convincing forgeries. Understanding these categories is important for recognizing and combating the potential misuse of deepfake technology.

\paragraph{Fake Video and Real Audio}
\label{para:multimodal_fakeV_realA}

Fake video and real audio deepfakes involve manipulating visual content while retaining the original audio, creating a synthetic video that depicts events or actions that never actually occurred. By keeping the original audio, which includes the true voice, tone, and speech patterns of the person, these deepfakes gain an added layer of credibility and can be particularly persuasive. The process of creating fake videos with real audio often involves altering the appearance, expressions, or actions of individuals in the video \cite{karras2019style, nirkin2019fsgan, korshunova2017fast}. For instance, face swapping \cite{korshunova2017fast} can replace the subject’s face with someone else's, or visual effects can be used to create entirely new scenes that seem authentic. The combination of real audio and fake video poses significant challenges for detection, as the genuine audio can make the fabricated visuals appear more believable. Detecting these deepfakes involves analyzing inconsistencies between the audio and video elements. Techniques such as temporal analysis of facial expressions, detecting unnatural movements, and scrutinizing visual artifacts are crucial \cite{kaur2020deepfakes, liu2023ti2net}.

\paragraph{Real Video and Fake Audio}
\label{para:multimodal_realV_fakeA}

In this type of deepfake, the video remains unaltered, while the audio is synthetically generated to mislead the audience about what is being said. This method generally involves using text-to-speech models and voice cloning techniques to create synthetic speech that closely mimics the vocal characteristics of a specific person \cite{deng2020unsupervised, kinnunen2017asvspoof, polyak2019tts}. By manipulating the audio, these deepfakes can make it seem like the person in the video is saying something they never actually said. This technique is particularly dangerous because the genuine video lends credibility to the fake audio, making it more convincing and harder to detect. Examples of using real video with fake audio include creating false news reports, tampering with evidence in legal cases, and producing deceitful content for political propaganda \cite{sankaranarayanan2021presidential}. Detecting such deepfakes requires a comprehensive analysis of audio-visual synchronization to identify discrepancies between lip movements and speech \cite{agarwal2020detecting}. Additionally, it demands the development of robust audio forensic techniques to scrutinize voice patterns and identify synthetic anomalies \cite{almutairi2022review}.

\paragraph{Fake Video and Fake Audio}
\label{para:multimodal_fakeV_fakeA}

These deepfakes manipulate visual content to depict events or actions that never occurred, while also synthesizing audio to accompany the fabricated visuals.  This allows for a wider range of possible sample alterations and a variety of manipulation techniques. For example, a deepfake could depict a person giving a speech they never delivered, with the voice and the video being entirely fabricated. Achieving realism in both modalities requires training models on large datasets of real audio and video to learn the nuances of human speech, facial expressions, and body movements. The challenge lies in creating seamless synchronization between the audio and video components to make the deepfake indistinguishable from genuine content. Furthermore, standardized multimodal deepfake datasets serve as benchmarks for evaluating the performance of detection algorithms \cite{dolhansky2020deepfake, khalid2021fakeavceleb, hou2024polyglotfake}. They offer a common ground for researchers to compare different approaches, facilitating the identification of the most effective methods for detecting deepfakes \cite{liu2023magnifying, cheng2023voice, feng2023self}. This benchmarking is vital for pushing the boundaries of deepfake detection technology, ensuring that the algorithms can generalize well across different types of deepfakes and are not limited to specific scenarios or formats.

\subsection{Multimodal Deepfake Detection Datasets}
\label{subsec:multimodal_datasets}

Multimodal deepfake datasets are crucial for advancing the understanding and detection of deepfakes, which involve the manipulation of multiple types of data such as audio, video, and text to create convincingly fake content. These datasets offer diverse examples of synthetic media that combine various modalities, accurately reflecting the complex, real-world scenarios where deepfakes are likely to be encountered. This diversity is indispensable for training sophisticated detection techniques, primarily deep learning-based networks, to recognize and detect deepfakes across various scenarios and formats. By including examples that span different combinations of audio, video, and text manipulations, multimodal datasets allow researchers to develop and refine algorithms that can analyze the consistency and coherence between these modalities.

Multimodal deepfake datasets are essential for advancing both the understanding and detection of deepfakes, which involve manipulating multiple types of data, such as audio and video, to create compelling fake content. These provide diverse examples of synthetic media, combining various modalities like audio, video, and text reflecting real-world scenarios where deepfakes are likely to be used. This diversity is essential for training sophisticated detection techniques (mainly deep learning-based networks) to recognize and detect deepfakes across different scenarios and formats. These multimodal datasets enable researchers to develop algorithms that analyze the consistency and coherence between modalities. Furthermore, standardized multimodal deepfake datasets offer benchmarks for evaluating the performance of detection algorithms. They provide a common ground for comparing different approaches and identifying the most effective methods for detecting multimodal deepfakes.

\begin{table*}[htbp]
\caption{\textcolor{black}{Multimodal Deepfake Datasets}}
\centering
\resizebox{\textwidth}{!}{%
\begin{tabular}{|P{67pt}|P{42pt}|P{40pt}|P{35pt}|P{35pt}|P{100pt}|P{100pt}|}
\hline
\multirow{2}{*}{Dataset} & \multirow{2}{*}{Multilingual} & \multirow{2}{*}{Subjects} & \multicolumn{2}{c|}{Samples} & \multicolumn{2}{c|}{Manipulation Techniques} \\\cline{4-7}
&  &  & Real & Fake & Video & Audio\\\hline\hline
DFDC \cite{dolhansky2020deepfake} & No  & 960 & 104,500 & 23,654 & DFAE, MM/NN face swap, NTH, FSGAN \cite{nirkin2019fsgan}, StyleGAN \cite{karras2019style} & TTS-Skins \cite{polyak2019tts}\\\hline
FakeAVCeleb \cite{khalid2021fakeavceleb} & No  & 500 & 500 & 19,500 & FaceSwap \cite{korshunova2017fast}, Wav2Lip \cite{prajwal2020lip}, FSGAN & SV2TTS \cite{jia2018transfer}\\\hline
VideoSham \cite{mittal2023video} & No & - & \textcolor{black}{413} & \textcolor{black}{413} & \multicolumn{2}{c|}{Manual manipulations by professional video editors (6 types)} \\\hline
PDD \cite{sankaranarayanan2021presidential} & No & 2 & 16 & 16 & Wav2Lip & Manipulated content recorded by voice actors \\\hline
LAV-DF \cite{cai2022you} & No & 153 & 36,431 & 99,873 & Wav2Lip & SV2TTS \\\hline
MMDFD \cite{asha2023mmdfd} & No  & 50 & 1,500 & 5,000 & FaceSwap, FSGAN, Wav2Lip, DeepFaceLab \cite{perov2020deepfacelab} & SV2TTS (AurisAI \cite{aurisaiAurisFree} for Text)\\\hline
DefakeAVMiT \cite{10081373} & No  & 43 & \multicolumn{2}{c|}{6,480} & FaceSwap, DeepFaceLab, EVP \cite{ji2021audio}, Wav2Lip, PC-AVS \cite{zhou2021pose}  & SV2TTS, Voice Replay \cite{kinnunen2017asvspoof}, AV exemplar autoencoders \cite{deng2020unsupervised} \\\hline
PolyGlotFake \cite{hou2024polyglotfake} & Yes  & 766 & 766 & 14,472 & VideoRetalking \cite{cheng2022videoretalking}, Wav2Lip & XTTS \cite{Gölge2024coqui}, Bark \cite{Kucsko2024suno} + FreeVC \cite{li2023freevc}, Tacotron \cite{wang2017tacotron} + FreeVC, MicrosoftTTS \cite{microsoftTextSpeech} +FreeVC, Vall-E-X \cite{wang2023neural}\\\hline
\end{tabular}}
\label{tab:multimodal_datasets}
\end{table*}

As illustrated in Table \ref{tab:multimodal_datasets}, the manipulation of modalities to generate multimodal deepfakes was accomplished using state-of-the-art deepfake generation techniques available at the time of the dataset release. The quality of the synthetic content is contingent upon the strengths and limitations of these techniques. Advances in sophisticated deep learning methodologies over time have yielded increasingly realistic fake content. Consequently, the techniques employed for multimodal deepfake generation are further elaborated in Section \ref{subsec:multimodal_generation}.

\subsection{Multimodal Deepfake Generation}
\label{subsec:multimodal_generation}

In the following section, we will delve into the latest and most widely utilized techniques for producing synthetic audio-visual content by synchronizing lip and/or face movements, drawing from state-of-the-art research in the field.

\paragraph{LipGAN}
\label{para:multimodal_lipgan}

Prajwal \textit{et al.} \cite{kr2019towards} proposed a GAN-based lip synchronization model called LipGAN. This model is capable of handling faces in random poses without the need for realignment to a template pose. Furthermore, LipGAN enables the generation of realistic talking face videos through an automated pipeline for face-to-face translation from any audio, without dependence on language.

\paragraph{Wav2Lip}
\label{para:multimodal_wav2lip}

Most state-of-the-art methods excel at generating accurate lip movements for static images or videos of specific individuals seen during the training phase. However, these methods often fail to accurately morph lip movements for arbitrary identities in dynamic, unconstrained talking face videos, resulting in significant portions of the video being out-of-sync with the new audio. As discussed by Prajwal et al. in \cite{\cite{prajwal2020lip}}, this failure can be primarily attributed to limitations in both training objectives (i.e., loss functions used) and lip-sync discriminators. Typically, the face reconstruction loss is computed for the detected facial region to ensure correct pose generation and identity preservation. However, the lip region constitutes a very small proportion of the face region (or the entire frame), causing the total reconstruction loss (often L1 distance) to be less impacted by the lip region due to its limited spatial extent. This can adversely affect the learning process, where the reconstruction of the surrounding image is prioritized over the lip region. To address this issue, specific discriminators, such as those used in LipGAN \cite{kr2019towards}, are employed to evaluate lip-sync accuracy. Prajwal et al. \cite{prajwal2020lip} emphasized the importance of short temporal context in detecting lip-sync discrepancies and incorporated this concept into the development of the Wav2Lip framework. They demonstrated that considering a short temporal context significantly improves lip-sync accuracy. Additionally, they noted that introducing artifacts such as pose variations during GAN training can negatively impact lip-sync performance. This degradation occurs because the discriminator may fail to focus on the correspondence between the video and lip movements, underscoring the need for discriminators specifically designed to evaluate lip-sync quality. Prajwal \textit{et al.} utilises a customised SyncNet-based \cite{chung2017out} discriminator to mitigate the limitations mentioned above, which is substantially more accurate than the previous methodologies. Furthermore, to increase the quality of the morphed regions in the reconstructed images, they have also utilised a visual-quality discriminator. This technique has been used by most of the above multimodal datasets to generate faked audio-visual content.

\paragraph{FaceRetalking}
\label{para:multimodal_faceretalking}

Cheng \textit{et al.} \cite{cheng2022videoretalking} highlighted that the state-of-the-art techniques often omit the original lip motion changes or retiming the background to avoid unnatural movements between the head pose and lip \cite{prajwal2020lip, song2022everybody}. However, in the FaceRatalking method, the lower half face (not only the lip) Cheng \textit{et al.} emphasized that existing state-of-the-art approaches often overlook authentic lip motion alterations or adjust the background timing to prevent unnatural movements between head pose and lip motion. In contrast, the FaceRatalking technique not only modifies the lip region but also encompasses editing of the entire lower half of the face, incorporating facial movements through an innovative face reenactment process. Additionally, they identified an information leakage in conditional in-painting-based methods when the original frame was utilized as the conditional image for lip synchronization. To remedy this, they introduced a semantic-guided reenactment network to alter the expression of the entire lower half of the face, producing an enriched frame with a consistent expression, which then served as the basis for subsequent lip synthesis. The lip synthesis network in the FaceRetalking approach incorporates a conditional inpainting-inspired network \cite{prajwal2020lip}. This network leverages pre-processed frames from the face reenactment network as the identity and structure reference, along with the audio and the masked original frames as the condition resulting in a highly effective method for synthesizing a lip-syncing video based on the input audio. They argued that even though the synthesized videos accurately depict lip movements, the visual quality is limited due to low-resolution training data. To tackle this issue, they developed an identity-preserving face enhancement network to improve output quality through progressive training. When compared to Wav2Lip \cite{prajwal2020lip} which mainly focuses only on lip synchronization, FaceRetalking provides a broader facial synthesis that includes expressions and head movements, enabling more realistic fake content.

\paragraph{Diff2Lip}
\label{para:multimodal_diff2lip}

In their work, Mukhopadhyay \textit{et al.} \cite{mukhopadhyay2024diff2lip} proposed Diff2Lip, which is an audio-conditioned diffusion model used for inpainting producing precise and natural lip sync by focusing on the fine details of lip movements. This model can achieve lip synchronization in real-world scenarios while preserving identity, pose, emotions, and image quality. The Diff2Lip model takes three inputs: a masked input frame (providing pose context), a reference frame (containing identity and mouth region textures), and an audio frame (used to drive the lip shape). It then outputs the lip-synced mouth region. The multimodal conditional diffusion implemented in the model allows for a fine balance between all contextual input information, effectively avoiding lip-sync problems. However, when compared with the FaceRetalking model, the FaceRetalking may provide a better integrated facial performance, enhancing overall expressiveness and realism beyond just the lips. However, through the leverage of the diffusion process, Diff2Lip can enhance the naturalness and fluidity of lip movements over time compared to similar lip synthesis methods such as Wav2Lip.


\subsection{Performance Evaluation in Deepfake Generation}
\label{subsec:multimodal_lossfunc}


Performance evaluation is a critical step in multimodal deepfake generation, providing a comprehensive framework to ensure that the generated content is realistic, high-quality, coherent, and consistent across different modalities. Various performance evaluation metrics are employed in the literature, encompassing both application-specific and generalized measures. In this section, we discuss some of the most widely considered performance metrics and their roles in assessing the quality and consistency of multimodal deepfake generation models.

\paragraph{The Fréchet Inception Distance (FID)}

The Fréchet Inception Distance (FID) serves as a crucial metric for evaluating the quality of generated images \cite{nunn2021compound, singh2020using, yu2021artificial}. By comparing the feature distribution of generated images with real images, it provides valuable insights into both the fidelity and diversity of the generated images. Lower FID values signify higher similarity to real images, indicating superior quality. An FID of 0 implies that the generated images are indistinguishable from real images in terms of their feature distributions. However, the FID score is sensitive to the choice of feature extractor. The FID score can be calculated as in Equation \ref{eq:multimodal_fid} where $\mu_r$, $\mu_g$, $\Sigma_r$, $\Sigma_g$ and $T_r$ represent the mean of real image features, mean of generated image features, covariance or real image features, covariance of generated image features and Trace of the matrix respectively \cite{nunn2021compound}. 

\begin{equation}
    FID = ||\mu_r - \mu_g||^2 + Tr(\Sigma_r+\Sigma_g - 2(\Sigma_r\Sigma_g)^{0.5})
    \label{eq:multimodal_fid}
\end{equation}

\paragraph{Structural Similarity Index (SSIM)}

SSIM compares two images by analyzing their structure, luminance, and contrast. Its goal is to provide a more accurate measure of perceptual image quality compared to simpler metrics like Mean Squared Error (MSE). SSIM is specifically designed to take human perception into account when assessing image quality, making it a more reliable method \cite{sun2020landmark, dagar2022literature, husseini2023comprehensive}. It does not consider higher-order image statistics, which may be important for certain aspects of image quality. However, it's important to note that SSIM is a pixel-wise image similarity metric that compares two images and may not be the best choice for capturing variability in video generation \cite{shrivastava2021diverse}. The SSIM can be calculated as in Equation \ref{eq:multimodal_ssim} where $c_1$, and $c_2$ represent constants to stabilize the division with a weak denominator \cite{wang2004image}.

\begin{equation}
    SSIM(x,y) = \frac{(2\mu_x \mu_y + c_1)(2\sigma_{xy}+c_2}{(\mu_x^2+\mu_y^2+c1)(\sigma_x^2+\sigma_y^2+c_2)}
    \label{eq:multimodal_ssim}
\end{equation}

\paragraph{Lip Movement Distance (LMD)}

Lip Movement Distance (LMD) is an important metric used to assess how well the movement of lips matches the corresponding audio, especially in the context of creating deepfakes. It measures the spatial difference between the lip positions in the generated frames and the actual frames, providing a quantitative evaluation of the accuracy of lip movements to the spoken audio \cite{chen2018lip}. However, the accuracy of LMD depends on the reliability of the facial landmark detection model used to extract lip positions. Additionally, since LMD focuses on spatial alignment, it may not fully capture the temporal dynamics and smoothness of lip movements over time. The LMD over $N$ frames can be calculated as in Equation \ref{eq:multimodal_lmd} where $L_t^{gen}$, and $L_t^{gt}$ represent lip landmarks for the generated and ground truth images respectively \cite{chen2018lip}.

\begin{equation}
    LMD = \frac{1}{N} \Sigma_{t=1}^n ||L_t^{gen} - L_t^{gt}||
    \label{eq:multimodal_lmd}
\end{equation}

\paragraph{LSE-C and LSE-D}

LSE-C and LSE-D are two important metrics used to assess the performance of the Wav2Lip model \cite{prajwal2020lip}, which is used to achieve synchronisation of audio-manipulated talking face videos. This model ensures that the lip movements in a video align with the corresponding audio. The LSE-D error (Lip-Sync Error-Distance), measures the misalignment between audio and visual streams in terms of lip synchronization, and a lower LSE-D denotes a higher audio-visual match \cite{prajwal2020lip}. LSE-C (Lip-Sync Error-Confidence) is the confidence score. A higher score suggests a better audio-visual correlation and more accurate alignment of lip movements with audio. Prajwal \textit{et al.} calculated these error metrics based on SyncNet \cite{chung2017out} extracted features from both the audio and visual inputs where the visual features are typically derived from the region around the lips, while audio features are extracted from the corresponding audio segment. However, poor feature extraction can lead to inaccurate error measurement, and may not capture the naturalness of continuous speech synchronization. These losses have been widely used in later research to validate the performance of lip-speech synchronisation \cite{wang2022attention, zhang2022meta, lu2022visualtts}. 

\paragraph{Peak Signal-to-Noise Ratio (PSNR)}

PSNR, which stands for Peak Signal-to-Noise Ratio, is a metric used to measure the quality of a reconstructed or generated image in comparison to a reference image \cite{huang2020fakeretouch, huang2020fakepolisher, wang2021faketagger}. It quantifies the level of distortion or noise introduced during the generation process, with higher PSNR values indicating better quality and less distortion. PSNR is easy to compute and understand, unlike most other metrics, providing a straightforward measure of image and video quality. However, it's important to note that PSNR does not always align well with human visual perception, and high PSNR values do not guarantee that the image will look good to human observers. PSNR can be calculated as in Equation \ref{eq:multimodal_psnr} where $MAX$ and $MSE$ refer to the maximum pixel value (can be either 1 or 255 depending on whether the input image is in double-precision floating-point or 8-bit unsigned integer format) and the Mean Squared Error (MSE) between the reference video frame (or image) and the generated frame.

\begin{equation}
    MSE = \frac{\Sigma_{M,N}[I_1(m,n)-I_2(m,n)]^2}{M*N}
    \label{eq:multimodal_psnr}
\end{equation}

\begin{equation}
    PSNR = 10.log_{10} (\frac{MAX^2}{MSE})
    \label{eq:multimodal_psnr}
\end{equation}

% \paragraph{Lip Sync Error Rate (LSER)}

% Lip sync error rate is the measure of how often or to what extent the audio track of a video does not match up correctly with the visual track of the speaker's lip movements. Various techniques can be used to calculate this metric, all of which assess how well the created lip movements synchronize with the accompanying audio, ensuring that the visual and audio components are convincingly synchronized. 

The performance metrics mentioned above can be divided into two main categories: lip-sync rate-related measures and realism-related measures. Lip-sync is essential for applications that require precise audio-visual alignment, focusing on synchronizing audio and lip movements. Realism is important for applications that require high visual fidelity and overall believability of the generated content. It takes into account not only lip synchronization, but also other factors such as facial expressions, eye movements, skin texture, and lighting. Lip-sync rate can be measured using metrics like LSE-C, LSE-D, and LMD, while realism can be measured through PSNR, SSIM, FID, and perceptual human evaluations (such as mean opinion score (MOS) \cite{lu2022visualtts, hou2024polyglotfake}).

\paragraph{BRISQUE}

BRISQUE is an image quality assessment (IQA) model that can evaluate the quality of an image without needing a reference image \cite{mittal2012no}. It works by analyzing natural scene statistics in the spatial domain, directly examining pixel values without transforming the image into a different domain, such as the frequency domain. The model employs statistical features derived from natural scene statistics to capture deviations from natural image properties, indicating distortions and quality degradation. This makes it useful for detecting deepfakes, as it does not rely on a real image for comparison once it's deployed in the world \cite{hou2024polyglotfake, yang2020deepfake}. 


\subsection{Multimodal Deepfake Generation Tools}

\textcolor{black}{A Deepfake Generation Tool is a sophisticated software or system designed to synthesize and create realistic media content across multiple modalities—primarily video and audio. These tools leverage advanced artificial intelligence (AI) and deep learning techniques to generate highly convincing synthetic content, often using generative models like GANs (Generative Adversarial Networks) or diffusion models. Deepfake generation tools can create content that closely mimics real media, making it challenging to distinguish between authentic and synthetic.}

\begin{table*}[htbp]
\caption{Top Tools to Detect Multimodal Deepfakes}
\centering
\resizebox{\textwidth}{!}{%
\begin{tabular}{|c|c|c|c|c|}
\hline
Method & Free & Open-source & URL\\\hline\hline
Synthesia  & \xmark & \xmark & https://www.synthesia.io/ \\\hline
AI Studios by DeepBrain AI  & \xmark & \xmark & https://www.aistudios.com/ \\\hline
Creative Reality Studio by D-ID  & \xmark & \xmark & https://www.d-id.com/creative-reality-studio/ \\\hline
Elai.io  & \xmark & \xmark & https://elai.io/ \\\hline
\end{tabular}}
\label{tab:tool_det_multimodal_deepfakes}
\end{table*}

\textcolor{black}{The rise of accessible deepfake generation tools has revolutionized content creation in several areas, particularly in Training & Education, E-commerce, Social Media, and Customer Support. These platforms use advanced AI to generate realistic videos based on text prompts, featuring avatars that can speak, express emotions, and perform subtle facial gestures. Designed mainly for professional use, these tools allow users to create customized video content at scale without the need for complex equipment, technical expertise, or high production costs. One of the most notable features of these tools is their user-friendly interface, which enables non-technical users and beginners to quickly produce sophisticated video content. Most of these platforms utilize deep learning and computer vision techniques to animate pre-existing avatars according to user-provided text prompts. The avatars are often photorealistic digital representations of real individuals, providing a high level of realism.}

\subsection{Multimodal Deepfake Detection}
\label{subsec:multimodal_detection}

Multimodal deepfake detection is an important area of research that focuses on identifying synthetic media manipulating multiple modalities, such as video, audio, and text, to create highly realistic and deceptive content \cite{raza2023multimodaltrace, katamneni2023mis, 10081373}. Unlike traditional deepfakes that target a single modality, multimodal deepfakes integrate alterations across several types of data, making them more sophisticated and harder to detect. Detection approaches typically use machine learning and deep learning algorithms, which are trained to recognize subtle anomalies in the synchronized behaviour of different modalities. For example, discrepancies between lip movements and speech, unnatural facial expressions, or inconsistencies in lighting and shadows can signal the presence of a deepfake \cite{lewis2020deepfake}. As deepfake technology continues to evolve, detection methods must also evolve to ensure the integrity and authenticity of digital media in an increasingly digital and interconnected world. In this section, we highlight the latest advancements in multi-modal deepfake detection technology.

\Rotatebox{90}{
\centering
\renewcommand{\arraystretch}{1}
\caption{\textcolor{black}{Multimodal Deepfake Detection Approaches}}
\begin{tabular}{|P{70pt}|P{120pt}|P{120pt}|P{130pt}|P{85pt}|}
\hline
\multirow{2}{*}{Method} & \multicolumn{2}{c|}{Feature Extractors} & \multirow{2}{*}{Technical Novelty} & \multirow{2}{*}{Performance} \\\cline{2-3}
& Audio & Video & & \\\hline\hline
Multimodaltrace \cite{raza2023multimodaltrace} & Resnet-1D on positive frequencies of FFT features & Multilayered 3D ResNet on normalised stacked video frames & Independent and joint feature learning through IntrAmodality Mixer Layer (IAML) and IntErModality Mixer Layer (IEML) & FakeAVCeleb: 92.9\% PDD: 70\% \\\hline
MIS-AVoiDD \cite{katamneni2023mis} & MFCC & MTCNN for face detection and Xception-based for feature extraction & Joint use of modality-invariant and specific representations to ensure both common and unique patterns of real or fake content are preserved and fused &  FakeAVCeleb: 96.2\%(Accuracy) 0.973(AUC) \\\hline
AVoiD-DF \cite{10081373} & transformer network on Mel-spectrograms of audio & transformer network on visual frames & Multimodal temporal \& spatial encoder (TSE) with multimodal joint decoder (MMD)  &  FakeAVCeleb: 83.7\%(Accuracy) 0.892(AUC) \\\hline
PVASS-MDD \cite{yu2023pvass} & VGGish network on log mel-spectrograms of audio \cite{hershey2017cnn} & MTCNN for face extraction and a Swin-Transformer \cite{liu2022video} & Cross-modal predictive VA alignment module  &  FakeAVCeleb: 84.3\%(Accuracy) 0.875(AUC) \\\hline
Emotions Don't Lie \cite{mittal2020emotions} & 13 MFCC features from pyAudioAnalysis \cite{giannakopoulos2015pyaudioanalysis} + DCNNs for modality encoding and perceived emotion encoding & facial features from OpnFace (430-D) \cite{amos2016openface} + DCNNs for modality encoding and perceived emotion encoding & Comparison of affective cues corresponding to perceived emotion to infer whether the video is manipulated & DFDC: 84.4\% \\\hline
AVAD \cite{feng2023self} & \multicolumn{2}{C{160pt}|}{Audio-visual synchronisation model  as in \cite{chen2021audio}. CNN-based feature encoders for visual frames and audio spectrograms and transformer as synchronisation module.} & Video forensics posed as an audio-visual anomaly detection problem and learning only on real videos  &  FakeAVCeleb: 87.9\%(AP) 0.900(AUC) \\\hline
VFD \cite{cheng2023voice} &  \multicolumn{2}{C{160pt}|}{Deep forward convolutional projection on the spectrogram and visual frames + transformer-like network to learn identity-related features} & A face-voice matching technique that measures homogeneity between the audio and video to identify deepfakes  &  FakeAVCeleb: 81.52\%(Accuracy) 0.8611(AUC) DFDC: 80.96\%(Accuracy) 0.8513(AUC)  \\\hline
Capsule Forensics (score fusion) \cite{muppalla2023integrating} & Capsule network on Mel-spectrograms of audio & Capsule network on MTCNN extracted face regions &  Multimodal score-fusion capable of identifying inconsistencies across
various deepfake types and artifacts within each modality  &  FakeAVCeleb: 99.2\%(Accuracy) 0.993(AUC) \\\hline
FCMT + DDIC \cite{liu2023magnifying} &  Audio Forgery Clues Magnification Transformer (FCMT) & Video FCMT &  FCMT module to capture intra-modal artifacts from different modalities by magnifying forgery clues + image spatial artifacts magnification with DDIC  &  FakeAVCeleb: 99.13\%(Accuracy) 0.9927(AUC) DFDC: 98.45\%(Accuracy) 0.9903(AUC)  \\\hline
\end{tabular}
\label{tab:multimodal_detectiondatasets}
}

To effectively detect realistic deepfakes, it is crucial to address both audio and video manipulation. This can be achieved either by independently detecting audio and video cues in deepfakes or through a combined approach that leverages joint audiovisual representation learning. Raza \textit{et al.} \cite{raza2023multimodaltrace} introduced a unified multimodal framework called "Multimodaltrace" which extracts learned feature representations from both audio and visual data, processing these elements separately before integrating them using an innovative multimodal fusion technique. Furthermore, they proposed a novel reformulation of the audiovisual deepfake detection problem, framing it as a multi-label classification task. This new approach predicts confidence levels across both audio and visual modalities, offering a more nuanced and effective method for identifying deepfakes. In their study, Katamneni et al.\cite{katamneni2023mis} focused on fusing modality invariant and specific feature representations for audio and visual streams. This method is similar to previous approaches but employs a different combination of regularization and learning objectives (modality invariant loss, modality-specific loss, and orthogonal loss), leading to improved results. Yang \textit{et al.} \cite{10081373} have proposed an innovative approach for detecting deepfakes by using audio-visual joint learning (AVoiD-DF) which leverages audio-visual inconsistencies for multi-modal forgery detection. The process begins by embedding temporal-spatial information in a Temporal-Spatial Encoder (TSE) to obtain temporal-spatial inconsistency between audio-visual signals (real and fake can exist across frames along the temporal dimension). It is then followed by a Multi-Modal Joint Decoder (MMD) to fuse multi-modal features and learn inherent relationships concurrently. Finally, a Cross-Modal Classifier is developed to detect manipulation by detecting inter-modal and intra-modal disharmony. Furthermore, to test the effectiveness of the proposed deepfake detection model in real-world scenarios, the researchers introduced DefakeAVMiT \cite{10081373}, a multimodal deepfake dataset where various forgery techniques have been applied to different modalities.

In the PVASS-MDD framework proposed by Yu \textit{et al.} \cite{yu2023pvass}, there are two main modules: an auxiliary PVASS stage that focuses on exploring common correspondences between video and audio (AV) and a cross-modal predictive VA alignment module (MDD). The PVASS module works by iteratively predicting audio features using visual features and then reconstructing visual features based on audio features and prediction errors to eliminate discrepancies between video and audio. In the MDD stage, the frozen PVASS network from the first stage is used to align the VA features of real videos, enabling the detection network to better learn the inconsistencies between video and audio in deepfake videos. This MDD stage, with the assistance of PVASS, can extract more accurate VA inconsistencies for multimodal deepfake detection. When it comes to detecting deepfakes, Feng \textit{et al.} \cite{feng2023self} took a different approach compared to other techniques. Instead of treating it as a classification problem, they looked at it as an anomaly detection problem. They analyzed the distribution of audio-visual examples and flagged those with low probability. They focused on subtle properties that manipulated videos are unlikely to accurately capture. They used three unique features for audio-visual anomaly detection: discrete time delay, time-delay distribution, and audio-visual network activations. They found that time-delay distribution is more meaningful for anomaly detection than time-delay alone. They also studied the effect of feature activations within the audio-visual synchronization network on anomaly detection. The results showed that manipulated videos can be detected by identifying unlikely sequences of these features based on a learned distribution. Mittal \textit{et al.} \cite{mittal2020emotions} proposed an innovative method for detecting alterations in videos, such as deepfakes. This approach utilizes both audio (speech) and video (face) data and extracts emotional features from both modalities. The method uses a Siamese network-based architecture (triplet learning) to process real and deepfake videos at the same time during training. It generates modality and perceived emotion embedding vectors for the subject's face and speech, which are then used to distinguish between real and fake content. Through experiments, the study demonstrated that the perceived emotion cues from both modalities play a crucial role in detecting deepfake content by assessing the similarity between modality signals. 

Cheng \textit{et al.} \cite{cheng2023voice} investigated using voice-face matching to detect deepfake videos. Their empirical results indicated that the identities behind voices and faces are often mismatched in deepfake videos and that voices and faces have some level of homogeneity. They detected deepfakes by examining the intrinsic correlation of facial and audio information, without using any additional auxiliary data such as more modalities or visual features. Muppalla \textit{et al.} \cite{muppalla2023integrating} utilised capsule networks to extract robust features from audio spectrograms and face visuals followed by multimodal fusion and classification for deepfake detection. They utilised both score-fusion and feature-fusion approaches, which substantially improved over the state-of-the-art methods. In their recent work, Liu \textit{et al.} \cite{liu2023magnifying} introduced an innovative multimodal Deepfake detection framework that enhances intra-modal and cross-modal forgery clues. The framework consists of several key modules. Firstly, the Forgery Clues Magnification Transformer (FCMT) module is proposed to capture temporal intra-modal defects by magnifying forgery clues based on sequence-level relationships. Additionally, a Distribution Difference Inconsistency Computing (DDIC) module, based on Jensen–Shannon divergence, is used to adaptively align multimodal information for further magnifying the cross-modal inconsistency. The framework also explores spatial artifacts by connecting multi-scale feature representation to provide comprehensive information. Finally, a feature fusion module is designed to adaptively fuse features to generate a more discriminative feature. Experimental results showed that the proposed framework outperforms independently trained models and demonstrated a superior generalisation on unseen types of Deepfake. The overall performance of the selected deepfake detection techniques on FakeAVCeleb dataset \cite{khalid2021fakeavceleb} is illustrated in Figure \ref{fig:multimodal_summary} in terms of accuracy and area under the curve (AUC) score.

\begin{figure}
    \centering
    \includegraphics[width=.5\linewidth]{figures_new/FakeAVCeleb.pdf}
    \caption{\textcolor{black}{Performance variation of recent state-of-the-art methods in detecting deepfakes on FakeAVCeleb dataset in terms of Accuracy and AUC.}}
    \label{fig:multimodal_summary}
\end{figure}

\textcolor{black}{Detecting deepfakes that involve manipulation across multiple types of media is a crucial task, and it is challenging to detect these using single-modal approaches highlighting the importance of addressing both audio and video manipulation in detection. While many methods have been proposed in the literature using different feature extraction and fusion techniques, most of them share similar architectural patterns. However, some state-of-the-art techniques have introduced novel approaches for deepfake detection. These advanced methods analyze audiovisual features iteratively, treating detection as an anomaly detection problem, and identifying inconsistencies using unique features like time-delay distribution and network activations. In addition, leveraging emotional features from both audio and visual sources, along with effective voice and face-matching techniques, greatly improves the chances of identifying deepfakes. More refined techniques also enhance the ability to detect forgery by emphasizing signs of manipulation and aligning the information from different media types. This leads to better performance and accuracy when detecting deepfakes that haven't been seen before. These developments in multimodal deepfake detection provide a more thorough and reliable way to spot manipulated content. Such improvements are essential for maintaining the trustworthiness of digital media and reducing the risks posed by misinformation in our increasingly complex online world.}

\subsection{Multimodal Deepfake Detection Tools}

\textcolor{black}{A Deepfake Detection Tool is an advanced software solution designed to identify and flag manipulated or synthetically generated digital content, including videos, audio, images, and increasingly, text. As deepfake generation techniques have become more sophisticated, detection tools have had to evolve as well, often utilizing state-of-the-art artificial intelligence (AI) and deep learning (DL) methods to keep pace with the capabilities of AI-generated forgeries. These detection tools analyze various features within digital media, employing algorithms that can identify subtle anomalies in pixel patterns, audio signals, or inconsistencies in images and videos that often occur during the deepfake creation process. Such tools not only serve as protective measures for individuals and organizations but are also gaining traction in sectors like law enforcement, cybersecurity, and media integrity, where they help maintain trustworthy sources of information. Table \ref{tab:tool_det_multimodal_deepfakes} provides a comparative overview of several currently available deepfake detection tools.}

\begin{table*}[htbp]
\caption{Top Tools to Detect Multimodal Deepfakes}
\centering
\resizebox{\textwidth}{!}{%
\begin{tabular}{|c|c|c|c|c|}
\hline
Method & Free & Open-source & URL\\\hline\hline
Sentinel  & \xmark & \xmark & https://thesentinel.ai/ \\\hline
Sensity  & \xmark & \xmark & https://sensity.ai/  \\\hline
Audio Visual Forensics & \cmark & \cmark & https://github.com/cfeng16/audio-visual-forensics \\\hline
Deepware  & \xmark & \xmark & https://deepware.ai/  \\\hline
Reality-Defender  & \xmark & \xmark & https://www.realitydefender.com/ \\\hline
Phoneme-Viseme Mismatch Detector  & \xmark & article & https://ieeexplore.ieee.org/document/9151013 \\\hline
\end{tabular}}
\label{tab:tool_det_multimodal_deepfakes}
\end{table*}


\subsection{Future Trends in Multimodal Deepfake Generation and Detection}

\textcolor{black}{In recent years, the creation of multimodal deepfakes has progressed rapidly, creating an ongoing challenge between making realistic deepfakes and developing methods to detect them. Diffusion models are one recent approach in image generation, which can be adapted for multimedia deepfake creation. These models can manipulate visual features and, when combined with audio signal manipulation techniques, allow for the generation of synchronized, multimodal content that includes both audio and visual modifications \cite{du2024dfadd, firc2024diffuse, av2024latent, bhattacharyya2024diffusion}. Despite the advances in state-of-the-art deepfake generation methods, issues with synchronization across different modalities (such as audio and video) persist. These synchronization inconsistencies are detectable by advanced detection systems and can help identify manipulated content \cite{liz2024generation, ivanovska2024vulnerability, mubarak2023survey}. To improve both deepfake generation and detection methods, high-quality datasets with a wide range of audio and video manipulations are essential. Adding datasets with content from multiple languages (most datasets are currently English-only) and diverse demographic representation would also be beneficial. This would support the development of more robust and generalized generation and detection models. Another key area in deepfake detection is explainability \cite{haq2024multimodal, tsigos2024towards}. Currently, most detection methods rely on deep learning, but what these models specifically learn to distinguish real from fake samples remains unclear. Future research could focus on understanding and interpreting these learning patterns, which could enhance both the effectiveness and trustworthiness of detection systems. In the future, multimodal deepfakes are likely to become increasingly realistic and harder to detect as techniques evolve. This makes it critical for research to keep pace, continually improving both generation and detection methods to address these advancements.}


\subsection{Combating Multimodal Deepfakes in Multimodal Biometrics}
\label{subsec:multimodal_biometrics}
With the advances in biometric evaluation techniques multimodal biometric recognition has also been recently introduced. Therefore, it is important to investigate the ability of multimodal deepfakes, especially the voice and face multimodal systems to thwart multimodal biometric recognition. In the following subsection, we discuss the summary of the findings of our evaluation, and a detailed discussion is provided in Sec. III of supplementary material
\subsubsection{Efficacy of multimodal deepfakes to fool multimodal biometrics systems}
Table III in Sec. III of supplementary material discusses the effectiveness of state-of-the-art multimodal deepfakes to thwart multimodal biometric recognition. Please note that this evaluation was conducted using voice and face multimodal systems and we considered a framework where voice modality and face modality are individually validated biometrically and the final decision is generated by fusing the individual decisions. While the current state-of-the-art multimodal deepfake generation methods failed to thwart the overall framework, especially due to their poor performance in manipulating the voice modality, the advances in multimodal deepfake technology could soon surpass the multimodal biometric reconnection and become a threat to multimodal authentication systems.


\subsubsection{Measures for revealing true identity:} To the best of our knowledge, there is no method to recover true identity from multimodal deepfakes. 



% \newpage

% \subsubsection{Universal multimodal deepfake detectors ? \textcolor{black}{I think this must come under unimodal techniques. Couldn't find any one detector that works on both audio and video. But there are many scenarios with identifying multiple types of attacks using single modality with one network}}

% \hspace{2mm}
% \subsection{Combating multimodal deepfakes in multimodal biometrics}
% \subsubsection{Efficacy of multimodal deepfakes to fool multimodal biometrics systems}
% \subsubsection{Measures for revealing true identity}
\subsection{Peer Review}
\label{sec:peer_review}

%\todo{CG: Correctly locate figures/tables after finishing writing}
\begin{figure*}[htbp]
  \centering
  \includegraphics[width=0.95\textwidth]{image/peer_review.pdf}
  \caption{Process of AI-enhanced peer review. In the analysis step, the LLM reviewer examines research manuscripts and evaluates peer reviews to assess scientific rigor. The review step involves providing feedback on the paper and verifying scientific claims. Finally, the gathered information is synthesized to generate a final meta-review. } 
  \label{fig:peer_review_overview}
\end{figure*}
% : analysis, where an LLM assesses manuscripts and reviews for rigor; review, where it provides feedback and verifies claims; and synthesis, where it generates a final meta-review.
% In the analysis step, the LLM reviewer examines research manuscripts and evaluates peer reviews to assess scientific rigor. The review step involves providing feedback on the paper and verifying scientific claims. Finally, the gathered information is synthesized to generate a final meta-review.
%Provide a concise description of the task here, indicate why it is important, and provide any necessary background information/references to contextualize the following subsections.

The highest standard in scientific quality control is \emph{peer reviewing}. In this process, the authors present their scientific argument (e.g., the findings of a study, a grant proposal, etc.), in form of a manuscript to their peers, who then assess its scientific validity and excellence. Often, this process has multiple stages, as shown in Fig. \ref{fig:peer_review_overview}. For instance, in the ACL Rolling Review system,\footnote{\url{https://aclrollingreview.org}} \emph{reviewers} write detailed assessments. Afterwards, the \emph{authors} may rebut the reviewers' arguments and clarify questions to convince them to raise their scores. Finally, a meta-reviewer re-evaluates the whole scientific discussion and gives a final acceptance recommendation (which the overall program chairs may or may not adhere to). During this process, multiple (potentially multi-modal) artifacts are involved and created, mainly \emph{the manuscript} under review, \emph{the written reviews}, \emph{the author-reviewer discussion texts}, and \emph{the meta-review}. 
In general, peer review is considered a challenging, and  subjective process, where reviewers are prone to unfair biases like sexism and racism, often relying on quick, simple heuristics~\cite[e.g.,][]{strauss2023racism, regner2019committees}. At the same time, we are faced with an exploding number of submissions in some fields like AI~\cite{kunzli2022not}, pushing peer reviewing systems to the limits of their capacities.

To counteract this problematic situation, researchers have worked on several problems under the umbrella of AI-supported peer review. Related overviews on the topic (or on some of its aspects) are given by \cite{kousha,drori2024human,staudinger-etal-2024-analysis,lin2023automated,checco,kuznetsov2024can}.  %\cite{
%drori2024human %}, \cite{
%staudinger-etal-2024-analysis, %}, %\cite{
%lin2023automated, 
%}, and \cite{
%checco}, 
pointing to the high relevancy of this problem. %Kutznetsov et al. \cite{kuznetsov2024can} provide an extensive discussion, outlining the potential of applying natural language processing techniques along all stages of the peer reviewing process. 
Here, we focus on existing works targeting the most established tasks, following the same structure as before, and provide an update on the recently published literature.

\subsubsection{Data}
%Give an overview of the most important curated/annotated datasets, or sources of raw data, that are used (or potentially useful for) this task.
% TODO: add all of the raw data mentioned here to the table below
Peer reviewing data is generally scarce, given that the scientific communities do not always make all reviewing artifacts publicly available under openly accessible licenses, with some exceptions like \href{https://iclr.cc}{ICLR}. %Accordingly, these exceptions make up the majority basis for several annotated datasets (see below). 
%Other works have also collected raw datasets, possibly further enriched and extended. For instance, 
As some exceptions, 
the PeerRead \cite{kang-etal-2018-dataset} collection of data from various sources (e.g., ACL, ICRL) and CiteTracked \cite{Plank2019CiteTrackedAL} %is
are 
published along with citation information. %which the authors specifically collected. 
As a prime example of how larger-scale open publishing of raw peer reviewing data may work, \citet{dycke-etal-2023-nlpeer} recently published the NLPeer corpus based on ARR reviews, for which they explicitly obtained the consent of the respective actors involved. 
%
%
% some works are missing given that almost every work also introduced or extended a data set
%
\begin{table}[th!]
\small
    \centering
    % \begin{tabular}{l l l l}
    \begin{tabular}{p{2.5cm} p{2.5cm}p{2.3cm}p{6.2cm}}
    \toprule
       \textbf{Dataset}  & \textbf{Size} & \textbf{Sources} & \textbf{Application} \\
       \midrule
         %\multicolumn{3}{c}{\textbf{Style Analysis, e.g., uncertainty detection}} \\ \midrule
        % uncertainty detection
        HedgePeer \cite{10.1145/3529372.3533300} & 2,966 documents & ICLR 2018 reviews & Uncertainty detection\\
PolitePeer \cite{politepeer} &2,500 sentences & Various, %sources, 
e.g., ICLR & Politeness Analysis\\
 %        \multicolumn{3}{c}{\textbf{Comparison Analyses}} \\ \midrule
COMPARE \cite{singh2021compare} & 1,800 sentences & ICLR & Comparison Analysis\\

ReAct \cite{Choudhary_2021} & 1,250 comments & ICLR & Actionability Analysis\\
MReD \cite{shen-etal-2022-mred} & 7,089 meta-reviews & ICLR & Meta-review analysis and generation\\ 
CiteTracked \cite{Plank2019CiteTrackedAL} & 3,427 papers and 12k reviewss & NeurIPS & citation prediction \\
MOPRD \cite{Lin_2023} & 6,578 papers & PeerJ & Review Comment Generation \\ 
Revise and Resubmit \cite{10.1162/coli_a_00455} & 5.4k papers & F1000Research & Tagging, Linking, Version Alignment  \\
ORB \cite{szumega2023open} & 92,879 reviews & OpenReview, SciPost & Acceptance Prediction\\ 
ARIES \cite{d2023aries} & 3.9k comments & OpenReview & Feedback-Edits Alignment, Revision Generation  \\ 
DISAPERE \cite{kennard-etal-2022-disapere} & 506 review-rebuttal pairs & ICLR & review action analysis, polarity prediction, review aspect \\
PeerReviewAnalyze \cite{10.1371/journal.pone.0259238} & 1,199 reviews & ICLR & Review Paper Section Correspondence, Paper Aspect Category Detection, Review Statement Role Prediction, Review Statement Significance Detection, and Meta-Review Generation\\
%\multicolumn{3}{c}{\textbf{Argumentation Analysis}} \\ \midrule
JitsuPeer \cite{purkayastha-etal-2023-exploring} & 9,946 review and 11,103 rebuttal sentences& ICLR & Argumentation Analysis, Canonical Rebuttal Scoring, Review Description Generation, End2End Canonical Rebuttal Generation\\

\bottomrule
    \end{tabular}
    \caption{Annotated or task-specific datasets for analyzing peer reviewing.}
    \label{tab:data_peer_reviewing}
    \vspace{-5mm}
\end{table}
%
%
%
For several tasks around peer review analyses, researchers have created annotated datasets. An overview of annotated and/ or task-specific datasets focusing on diverse aspects of peer review is provided in Table~\ref{tab:data_peer_reviewing}. %\todo{SE: fix link - CL: Done} 
%For instance, \cite
Most recently, researchers focused on curating resources for supporting more complex tasks, like understanding the effect of peer review feedback on revisions of the manuscript~\cite{d2023aries} or on identifying the underlying attitudes that cause a specific criticism in peer review~\cite{purkayastha-etal-2023-exploring}.


\subsubsection{Methods and Results}

%Describe the state-of-the-art methods and their results, noting any significant qualitative/quantitative differences between them where appropriate.
%Generally, the overall trends for computational approaches targeting the processing of peer reviews follow the  overall trends in NLP.  %natural language processing. 
Initial works were mostly based on more traditional machine learning methods and targeted simpler analyses involving sentence classification tasks. Later, deep learning approaches (also based on pre-trained language models) and more complex analyses, e.g., argumentation analyses, were defining the state-of-the-art in computational peer review processing. Nowadays, researchers started exploring %the use of large language models 
LLMs 
in prompting-based frameworks for complex tasks like peer review generation and meta-review generation. \todo{SE: add citations here?}
% First simpler analyses, then more like relationship exploitation, then LLMs


\paragraph{Analysis of Peer Reviews}
%Preceding 
Prior 
works have analyzed peer reviews for a multitude of aspects, like uncertainty~\cite{10.1145/3529372.3533300}, politeness~\cite{politepeer}, and sentiment~\cite{Chakraborty_2020}. 
%\cite{10.1145/3529372.3533300} uncertainty detection with base models like scibert and xlnet, also try mtl with sentiment but that doesn't work
%\cite{politepeer} politeness detection wiht scibert, toxicbert, word2vec
%\cite{Chakraborty_2020} aspect-based sentiment analysis traditional approaches like svm and mulitnominal naive bayes vs. deep learning like cnn, scibert embeddings
However, given that science as a whole and especially peer review relies to a large extent on convincing %our 
peers, large efforts have 
%, in particular, 
been spent on understanding arguments or argument-related aspects (e.g., substantiation of arguments) in peer review artifacts~\cite[e.g., ][]{Fromm2021,hua-etal-2019-argument}.
Here, most approaches leveraged pre-trained language models. For instance, \citet{hua-etal-2019-argument} work on mining the arguments in peer reviews using conditional random fields, %and %bi-directional long short-term memory networks (LSTMs) 
LSTMs, 
and BERT. In contrast, \citet{guo-etal-2023-automatic}  and \citet{Fromm2021} fully rely on (domain adjusted) pre-trained language models for argument mining like SciBERT, ArgBERT, and PeerBERT. \citet{cheng-etal-2020-ape}  leverage multi-task learning approaches based on LSTMs and BERT. In a similar vein,  \citet{purkayastha-etal-2023-exploring} study the generation of rebuttals for author-reviewer discussions based on Jiu-Jitsu argumentation, a specific  theory in argumentation theory. \todo{SE: where's the recent SOTA?}
%Given that science as a whole and especially peer review relies to a large extent on convincing our peers, large efforts have also been spent on understanding arguments in peer review artifacts.
%\cite{Fromm2021} argument mining, argbert, peerbert
% \cite{cheng-etal-2020-ape} argument pair extraction, mtl with lstms and bert
%\cite{guo-etal-2023-automatic} analysis of substantiation (also new dataset), SciBERT, RoBERTa, SpanBERT
% \cite{hua-etal-2019-argument} argument mining, crf, bilstm crf, elmo

\paragraph{Paper Feedback and Automatic Reviewing}
Several works have explored methods to provide general feedback on scientific publications to fully or partially automate peer reviews.  %i.e., peer feedback in its institutionalized form. Note, that relevant to this task, other works also focused on educational contexts (e.g., \cite{nguyen-litman-2014-improving}, \cite{su-etal-2023-reviewriter}, and \cite{bjet}).
%Originally, researchers mostly focused on score prediction given a particular reviewing artefact. 
For instance, \citet{li-etal-2020-multi-task} propose a multi-task learning approach for peer review score prediction, where different aspect score prediction tasks (e.g., novelty) can inform each other. \citet{ghosal-etal-2019-deepsentipeer} leverage the concept of sentiment to predict scores based on review texts. In a similar vein, \citet{10.1007/978-3-030-91669-5_33} leverage paper-review interactions to predict final decisions of a review process. %The idea is to envision a collaborative effort, in which reviews are human-written, and the AI supports the final decision.
%Importantly, 
\citet{wang-etal-2020-reviewrobot} focus on explainability during review score prediction for several review categories by constructing knowledge graphs (e.g., one which represents the background of a paper). More recent works have included the generation of feedback texts into the problem setup.
%Interestingly, 
\citet{bartoli2020} frame the problem as exploring the potential of GPT-2 for conducting academic fraud by generating fake reviews. 
In contrast, \citet{10.1613/jair.1.12862} ask whether it would be possible to automate reviewing leveraging %the potential 
targeted summarization models, \se{a recently trending topic}. 
%and similarly, many of the most recent works, do not study automatically generating reviews under the perspective of academic fraud. 
%Researchers are exploring more and more to what extent LLMs can be leveraged to automate peer reviews, or to support reviews in this process. 
For instance, \citet{liu2023reviewergpt} explore prompting-based review generation with several LLMs like GPT-4, Vicuna, Llama. They find that GPT-4 performs best among the models tested and that task granularity matters. Similarly, \citet{robertson2023gpt4} find GPT-4 to be ``slightly'' helpful for peer reviewing, and \citet{liang2023can} demonstrate in a comparative study that users of a GPT-4-based peer review system found the feedback to be (very) helpful in more than half of the cases. 
 \citet{d2024marg} show a multi-agent approach with LLMs that engage in a discussion to produce better results than a single model. %\todo{YC: The citation format of this paragraph is different from other sections. It's better to keep consistent, e.g. Chan et al. [29]. SE: please fix, not comment}




%\cite{Biswas2023ChatGPTAT}













\paragraph{Scientific Rigor} 

%Despite its fundamental importance, existing guidelines or definitions for rigor are often vague and general, such as the NIH's suggestion to justify the methodology, identify potential weaknesses, and address limitations~\cite{johnson2020review,wilson2021three}. 
%\citet{sansbury2022rigor} highlight the importance of rigor in study design and conduct, statistical procedures, data preparation, and availability. In addition, there exist many domain-specific requirements for rigor proposed by researchers. For example, \citet{lithgow2017long} believe stricter variability control is necessary for animal research, 
%following strict handling protocols, and adhering to precise methodological guidelines. However, these %aforementioned 
%criteria are predominantly developed in a top-down fashion, relying heavily on domain experts' experience. 

Several attempts have been made to computationally analyze the rigor of scientific papers. For example, \citet{Wael} investigate how researchers use the word ``rigor'' in information system literature but discovered that the exact meaning was ambiguous in current research. Additionally, various automated tools have been proposed to assess the rigor of academic papers. \citet{phillips2017online} develop an online software that spots genetic errors in cancer papers, while \citet{sun2022assessing} use knowledge graphs to assess the credibility of papers based on meta-data such as publication venue, affiliation, and citations. 
However, these methods are neither domain-specific nor do they provide sufficient guidance for authors to improve their narrative and writing.
In contrast, SciScore \cite{SciScore_2024} is an online system that uses language models to produce rigor reports for paper drafts, helping authors identify weaknesses in their presentation. 
%However, they rely on existing rigor checklists suggested by NIH and MDAR \cite{chambers2019towards}, which are not easily scalable or transferable to other domains. %\todo{SE: could go to limitations - CL: Done} 
More recently, \citet{james-etal-2024-rigour} propose a bottom-up, data-driven framework that automates the identification and definition of rigor criteria while assessing their relevance in scientific texts. Their framework integrates three key components: rigor keyword extraction, detailed definition generation, and the identification of salient criteria. Additionally, its domain-agnostic design allows for flexible adaptation across different fields.
%Furthermore, it is domain-agnostic, allowing for tailoring to the evaluation of scientific rigor across different areas by accommodating distinct salient criteria specific to each field. 

\paragraph{Scientific Claim Verification} 

The increasing number of publications requires the development of automated methods for verifying the validity and reliability of research claims. Scientific fact verification, which aims to assess the accuracy of scientific statements, often relies on external knowledge to support or refute claims \citep{vladika2023scientific, dmonte2024claim}. Several datasets have been developed to address this including SciFact-Open \citep{wadden2022scifact}, which provides scientific claims and supporting evidence from abstracts. However, they are limited to the use of abstracts as the primary source of evidence. As the statements in abstract can also be inaccurate (e.g. overstated claims), it is important to evaluate the evidence in the main body of the paper to determine if the statements made in the abstract are well supported. On the other side, \citet{glockner-etal-2024-missci,glockner2024groundingfallaciesmisrepresentingscientific} %and \citet{glockner2024groundingfallaciesmisrepresentingscientific} 
propose a theoretical argumentation model to reconstruct fallacious reasoning of false claims that misrepresent scientific publications.
% 
The need to contextualize claims with supporting evidence is highlighted by \citet{chan2024overview}, who introduce a dataset of claims extracted from lab notes. Unlike other datasets, this resource contains claims ``actually in use'', providing a more realistic understanding of how researchers interact with scientific findings. The authors annotate these claims with links to figures, tables, and methodological details, and develop associated tasks to improve retrieval. While this provides valuable resources for context-based verification, it primarily focuses on factual verification and does not evaluate the potential for overstated claims.
% 
Beyond factual correctness, there is a growing recognition for the need to analyze how researchers present their findings. This includes the detection of overstatements, where authors exaggerate their achievements, and understatements, where the true impact of the research is downplayed \citep{kao2024we}. Such analysis goes beyond the simple fact of a claim and is necessary to understand the presentation of a claim. \citet{schlichtkrull2023intended} present a qualitative analysis of how intended uses of fact verification are described in highly-cited NLP papers, particularly focusing on the introductions of the papers, to understand how these elements are framed. The work suggests that claims should be supported by relevant prior work and empirical results.

% Related to the area of scientific rigor analysis as well as ultimately, to the task of peer review feedback generation, is scientific claim verification. The idea is, to either verify claims made about a field based on the related literature or  ...
% \citet{wadden-etal-2022-scifact}

% \citet{papadopoulos2023red}

% \citet{wadden-etal-2022-multivers} 

% \cite{wadden-etal-2020-fact}

\paragraph{Meta Review Generation}
%While, for long, meta reviews have not been the focus of NLP/AI research, lately, more works look at tasks related to this particular artifact in the peer reviewing process. %As such, 
\citet{9651825} tackle meta-review generation using a multi-encoder transformer network, and \citet{li-etal-2023-summarizing} use a multi-task learning approach for refining pre-trained language models for the task. \citet{stappen2020uncertainty} explore the aggregation of reviews for providing additional computational decision support to editors based on uncertainty-aware methods like soft labeling. Both \citet{zeng2023meta} and \citet{santu2024prompting} rely on %large language models 
LLMs 
which they specifically prompt for the task.
%\cite{9651825} meta review generation, multi-encoder transformer network
% \cite{li-etal-2023-summarizing} % pretrained lms, multi-task training
%\cite{santu2024prompting}
% \cite{zeng2023meta}
%\todo{YC: The citation format of this paragraph is different from other sections. It's better to keep consistent, e.g. Chan et al. [29]. SE: Please fix it, no need for a comment ...}


% What to do with those? --> these are nnot inside the text yet
%\cite{nguyen-etal-2016-instant} traditional feature engineering with logistic regression for solution prediction
%\cite{kumar-etal-2023-reviewers} reviewer disagreement prediction, utilize Multi- Instance Multi-Label Learning Network 
% \cite{SUN2024101501} citation prediction based on textual features from peer review

% %%%%%%%%%%%% Moved to appendix  %%%%%%%%%%%%%%%%
\subsubsection{Ethical Concerns}
%Identify and discuss any ethical issues related to the (mis)use of the data or the application of the methods, as well as strategies for mitigations.
Given the critical role of scientific peer review for science, and, accordingly, for society as a whole, ethical considerations around AI-supported peer review are of utmost importance. As the general concerns around unfair biases in AI and the resulting harms apply~\cite{kuznetsov2024can}, research on safe peer-reviewing support needs to be prioritized. For instance, \citet{10.1001/jama.2023.24641} recently showed that %large language models 
LLMs exhibit 
%showed 
affiliation biases when reviewing abstracts. In this context, any AI-support for peer reviewing needs to be critically evaluated~\cite{schintler2023critical}, and solutions that target only a particular aspect in a collaborative environment that leaves the scientific autonomy to the human expert, may need to be preferred over end-to-end reviewing systems.
%%%%%%%%%%%%%%%%%%%%%%%%%%%%%%%%%%%%%%%%%%%%%%%%%%%%%

%\cite{10.1001/jama.2023.24641}
%\cite{schintler2023critical}


\subsubsection{Domains of Application}
%Indicate whether any of the data, methods, ethical concerns, etc. are specific to a given domain (biology, health, computer science, etc.).
Generally, peer review comes in many variations. For instance, the specific aspects to review for, how much textual content to produce, the specific scoring schemes, and the envisioned stages and dynamics of the reviewer and reviewer-author discussions may change. Thus, while none of the studies presented above targets a problem that is truly unique to any scientific domain, the particularities will likely be very different for each specific community and existing systems will need to be carefully evaluated before deployment.
 \todo{SE: but aren't most models and datasets originate from AI/NLP?}

\subsubsection{Limitations and Future Directions}
%Summarize the limitations of current approaches; point out any notable gaps in the research and future directions.
For existing studies on peer review, in particular, the variety of scientific domains that have been studied is still limited. As most of the works rely on data from OpenReview, most studies focus on peer review within the ICLR and ACL communities~\cite[e.g.,][]{Choudhary_2021,kennard-etal-2022-disapere}. \todo{SE: see above}
To the best of our knowledge, for some domains, no single data set (yet, a data set further enriched with annotations or other additional information) exists (e.g., legal studies). 
Furthermore, scientific rigor, a critical aspect of peer review, remains underexplored. Most existing studies rely on predefined rigor checklists, such as those suggested by the NIH and MDAR \cite{chambers2019towards}, which are not easily scalable or transferable across different domains. 
Given these gaps, future research could benefit from exploring new domains of peer review, developing domain adaptation approaches, and advancing models for assessing scientific rigor. Additionally, in light of the ethical concerns discussed earlier, it is crucial to prioritize research on trustworthy AI support for peer review - ensuring that human experts retain autonomy in the review process. \todo{SE: what about the quality of models?}

%Variety of domains and data sources

%\subsubsection{AI use case}

%Optional: describe which portions of your section (figures, tables, text, etc.) have been assisted by AI and how.
%\subsection{Other miscellaneous aspects}

Provide a concise description of the task here, indicate why it is important, and provide any necessary background information/references to contextualize the following subsections.

\subsubsection{Data}

Give an overview of the most important curated/annotated datasets, or sources of raw data, that are used (or potentially useful for) this task.

\subsubsection{Methods and results}

Describe the state-of-the-art methods and their results, noting any significant qualitative/quantitative differences between them where appropriate.



\subsubsection{Ethical concerns}

Identify and discuss any ethical issues related to the (mis)use of the data or the application of the methods, as well as strategies for mitigations.

\subsubsection{Domains of application}

Indicate whether any of the data, methods, ethical concerns, etc. are specific to a given domain (biology, health, computer science, etc.).

\subsubsection{Limitations and future directions}

Summarize the limitations of current approaches; point out any notable gaps in the research and future directions.

\subsubsection{AI use case}

Optional: describe which portions of your section (figures, tables, text, etc.) have been assisted by AI and how.

%\section{Domains to Study Personalization}
\label{sec:domains}

To study personalization with \methodname\, we construct a benchmark across 3 domains ranging from generating personalized movie reviews (\textbf{Reviews}), generating personalized responses based off a user's education background (\textbf{ELIX}), and personalizing for general question answering (\textbf{Roleplay}). We open-source preference datasets and evaluation protocols from each of these tasks for future work looking to study personalization (sample in supplementary).

\noindent \textbf{Reviews.} The Reviews task is inspired by the IMDB dataset~\citep{maas-etal-2011-learning}, containing reviews for movies. We curate a list of popular media such as movies, TV shows, anime, and books for a language model to review. We consider two independent axes of variation for users: sentiment (positive and negative) and conciseness (concise and verbose). Here being able to pick up the user is crucial as the users from the same axes (e.g positive and negative) would have opposite preferences, making this \emph{difficult} to learn with any population based RLHF method. We also study the steerability of the model considering the axes of verbosity and sentiment in tandem (e.g positive + verbose). 

\noindent \textbf{ELIX.} The Explain Like I'm X (ELIX) task is inspired by the subreddit "Explain Like I'm 5" where users answer questions at a very basic level appropriate for a 5 year old. Here we study the ability of the model to personalize a pedagogical explanation to a user's education background. We construct two variants of the task. The first variant is \textbf{ELIX-easy} where users are one of 5 education levels (elementary school, middle school, high school, college, expert) and the goal of the task is to explain a question such as ``How are beaches formed?'' to a user of that education background. The second, more realistic variant is \textbf{ELIX-hard}, which consists of question answering at a high school to university level. Here, users may have different levels of expertise in different domains. For example, a PhD student in Computer Science may have a very different educational background from an undergraduate studying studying Biology, allowing for preferences from diverse users (550 users). 

\noindent \textbf{Roleplay.} The Roleplay task tackles general question answering across a wide set of users, following PRISM~\citep{kirk2024prismalignmentdatasetparticipatory} and PERSONA Bench~\citep{castricato2024personareproducibletestbedpluralistic} to study personalization representative of the broad human population. We start by identifying three demographic traits (age, geographic location, and gender) that humans differ in that can lead to personalization. For each trait combination, we generate 30 personas, leading to 1,500 total personas. To more accurately model the distribution of questions, we split our questions into two categories: global and specific. Global questions are general where anyone may ask it, but specific questions revolve around a trait, for example an elderly person asking about retirement or a female asking about breast cancer screening.

One crucial detail for each task is the construction of a preference dataset that spans multiple users. But how should one construct such a dataset that is realistic and effective?

\begin{AIbox}{Takeaways from Personalization Domains}
We propose a benchmark consisting of 3 domains, where personalization can be studied: (1) \textbf{Reviews}, studying the generation ability of models for reviews of movies, TV shows, and books that are consistent with a user’s writing style, (2) \textbf{Explain Like I'm X (ELIX)}: studying the generation ability of models for responses that are consistent with a user’s education level, and (3) \textbf{Roleplay}: studying the generation ability of models for responses that are consistent with a user's description, with effective transferability to a real human-study.
\end{AIbox}
Our study was approved by the IRB of our institution.
Participants electronically signed a consent form describing the nature of our study and the data we would collect: their answers to the questionnaires, their demographic information provided by the platform, and their interactions with the study platform. All data was stored pseudonymously.
While our initial study description did not explicitly mention participants they would be exposed to phishing, this is a commonly used method in most phishing studies~\cite{resnik2018ethics,thomopoulos2023methodologies} to avoid excessive priming.
The participants were debriefed after completing the study with the full description, and is confirmed to incur only minimal risks~\cite{finn2007designing}, also confirmed by our IRB classifying our study as minimal risk.
Participants were appropriately remunerated for their time with a payment matching the highest minimum wage in their country.

We took further countermeasures to ensure participants' safety: the discomfort of being exposed to phishing emails was mitigated by the roleplay setting and their assigned fictitious identity.
Furthermore, their task was limited to clicking on links---there was no interaction with simulated phishing websites or other potentially harmful content.
Additionally, the phishing URLs we provided did not offer an easy way for participants to actually visit them (as our environment was preventing navigation); however, to protect participants that might transcribe or copy-paste them into their browsers, we constantly monitored all URLs to ensure they were offline during the duration of the study.

\section{Conclusion}
In this work, we propose a simple yet effective approach, called SMILE, for graph few-shot learning with fewer tasks. Specifically, we introduce a novel dual-level mixup strategy, including within-task and across-task mixup, for enriching the diversity of nodes within each task and the diversity of tasks. Also, we incorporate the degree-based prior information to learn expressive node embeddings. Theoretically, we prove that SMILE effectively enhances the model's generalization performance. Empirically, we conduct extensive experiments on multiple benchmarks and the results suggest that SMILE significantly outperforms other baselines, including both in-domain and cross-domain few-shot settings.

%%
%% The acknowledgments section is defined using the "acks" environment
%% (and NOT an unnumbered section). This ensures the proper
%% identification of the section in the article metadata, and the
%% consistent spelling of the heading.
%%%%%%%%%%%%%%%%%%%%%%%%% comment out when submit for reviewing %%%%%%%%%%%%%%%%%%%%%%%%
\begin{acks}
%This publication has been supported by the EXDIGIT (Excellence in Digital Sciences and Interdisciplinary Technologies) project, funded by Land Salzburg under grant number 20204-WISS/263/6-6022.
Yong Cao was supported by a VolkswagenStiftung Momentum grant. Jennifer D'Souza was supported by the \href{https://scinext-project.github.io/}{SCINEXT project} (BMBF, German Federal Ministry of Education and Research, Grant ID: 01lS22070). The NLLG Lab at UTN gratefully acknowledges support from the Federal Ministry of Education and Research (BMBF) via the research grant ``Metrics4NLG'' and the German Research Foundation (DFG) via the Heisenberg Grant EG 375/5-1. The work of Anne Lauscher is supported by the Excellence Strategy of the German Federal Government and the Federal States. Our AI use cases are document in the supplemental material. 
\end{acks}
%%%%%%%%%%%%%%%%%%%%%%%%% End Comment %%%%%%%%%%%%%%%%%%%%%%%%
%%
%% The next two lines define the bibliography style to be used, and
%% the bibliography file.
% \bibliographystyle{ACM-Reference-Format}
\bibliographystyle{ACM-Reference-Format-ISO4} % For ISO 4 abbreviations
\bibliography{2025_THAISCI_survey_wo_url}

\clearpage

\subsection{Lloyd-Max Algorithm}
\label{subsec:Lloyd-Max}
For a given quantization bitwidth $B$ and an operand $\bm{X}$, the Lloyd-Max algorithm finds $2^B$ quantization levels $\{\hat{x}_i\}_{i=1}^{2^B}$ such that quantizing $\bm{X}$ by rounding each scalar in $\bm{X}$ to the nearest quantization level minimizes the quantization MSE. 

The algorithm starts with an initial guess of quantization levels and then iteratively computes quantization thresholds $\{\tau_i\}_{i=1}^{2^B-1}$ and updates quantization levels $\{\hat{x}_i\}_{i=1}^{2^B}$. Specifically, at iteration $n$, thresholds are set to the midpoints of the previous iteration's levels:
\begin{align*}
    \tau_i^{(n)}=\frac{\hat{x}_i^{(n-1)}+\hat{x}_{i+1}^{(n-1)}}2 \text{ for } i=1\ldots 2^B-1
\end{align*}
Subsequently, the quantization levels are re-computed as conditional means of the data regions defined by the new thresholds:
\begin{align*}
    \hat{x}_i^{(n)}=\mathbb{E}\left[ \bm{X} \big| \bm{X}\in [\tau_{i-1}^{(n)},\tau_i^{(n)}] \right] \text{ for } i=1\ldots 2^B
\end{align*}
where to satisfy boundary conditions we have $\tau_0=-\infty$ and $\tau_{2^B}=\infty$. The algorithm iterates the above steps until convergence.

Figure \ref{fig:lm_quant} compares the quantization levels of a $7$-bit floating point (E3M3) quantizer (left) to a $7$-bit Lloyd-Max quantizer (right) when quantizing a layer of weights from the GPT3-126M model at a per-tensor granularity. As shown, the Lloyd-Max quantizer achieves substantially lower quantization MSE. Further, Table \ref{tab:FP7_vs_LM7} shows the superior perplexity achieved by Lloyd-Max quantizers for bitwidths of $7$, $6$ and $5$. The difference between the quantizers is clear at 5 bits, where per-tensor FP quantization incurs a drastic and unacceptable increase in perplexity, while Lloyd-Max quantization incurs a much smaller increase. Nevertheless, we note that even the optimal Lloyd-Max quantizer incurs a notable ($\sim 1.5$) increase in perplexity due to the coarse granularity of quantization. 

\begin{figure}[h]
  \centering
  \includegraphics[width=0.7\linewidth]{sections/figures/LM7_FP7.pdf}
  \caption{\small Quantization levels and the corresponding quantization MSE of Floating Point (left) vs Lloyd-Max (right) Quantizers for a layer of weights in the GPT3-126M model.}
  \label{fig:lm_quant}
\end{figure}

\begin{table}[h]\scriptsize
\begin{center}
\caption{\label{tab:FP7_vs_LM7} \small Comparing perplexity (lower is better) achieved by floating point quantizers and Lloyd-Max quantizers on a GPT3-126M model for the Wikitext-103 dataset.}
\begin{tabular}{c|cc|c}
\hline
 \multirow{2}{*}{\textbf{Bitwidth}} & \multicolumn{2}{|c|}{\textbf{Floating-Point Quantizer}} & \textbf{Lloyd-Max Quantizer} \\
 & Best Format & Wikitext-103 Perplexity & Wikitext-103 Perplexity \\
\hline
7 & E3M3 & 18.32 & 18.27 \\
6 & E3M2 & 19.07 & 18.51 \\
5 & E4M0 & 43.89 & 19.71 \\
\hline
\end{tabular}
\end{center}
\end{table}

\subsection{Proof of Local Optimality of LO-BCQ}
\label{subsec:lobcq_opt_proof}
For a given block $\bm{b}_j$, the quantization MSE during LO-BCQ can be empirically evaluated as $\frac{1}{L_b}\lVert \bm{b}_j- \bm{\hat{b}}_j\rVert^2_2$ where $\bm{\hat{b}}_j$ is computed from equation (\ref{eq:clustered_quantization_definition}) as $C_{f(\bm{b}_j)}(\bm{b}_j)$. Further, for a given block cluster $\mathcal{B}_i$, we compute the quantization MSE as $\frac{1}{|\mathcal{B}_{i}|}\sum_{\bm{b} \in \mathcal{B}_{i}} \frac{1}{L_b}\lVert \bm{b}- C_i^{(n)}(\bm{b})\rVert^2_2$. Therefore, at the end of iteration $n$, we evaluate the overall quantization MSE $J^{(n)}$ for a given operand $\bm{X}$ composed of $N_c$ block clusters as:
\begin{align*}
    \label{eq:mse_iter_n}
    J^{(n)} = \frac{1}{N_c} \sum_{i=1}^{N_c} \frac{1}{|\mathcal{B}_{i}^{(n)}|}\sum_{\bm{v} \in \mathcal{B}_{i}^{(n)}} \frac{1}{L_b}\lVert \bm{b}- B_i^{(n)}(\bm{b})\rVert^2_2
\end{align*}

At the end of iteration $n$, the codebooks are updated from $\mathcal{C}^{(n-1)}$ to $\mathcal{C}^{(n)}$. However, the mapping of a given vector $\bm{b}_j$ to quantizers $\mathcal{C}^{(n)}$ remains as  $f^{(n)}(\bm{b}_j)$. At the next iteration, during the vector clustering step, $f^{(n+1)}(\bm{b}_j)$ finds new mapping of $\bm{b}_j$ to updated codebooks $\mathcal{C}^{(n)}$ such that the quantization MSE over the candidate codebooks is minimized. Therefore, we obtain the following result for $\bm{b}_j$:
\begin{align*}
\frac{1}{L_b}\lVert \bm{b}_j - C_{f^{(n+1)}(\bm{b}_j)}^{(n)}(\bm{b}_j)\rVert^2_2 \le \frac{1}{L_b}\lVert \bm{b}_j - C_{f^{(n)}(\bm{b}_j)}^{(n)}(\bm{b}_j)\rVert^2_2
\end{align*}

That is, quantizing $\bm{b}_j$ at the end of the block clustering step of iteration $n+1$ results in lower quantization MSE compared to quantizing at the end of iteration $n$. Since this is true for all $\bm{b} \in \bm{X}$, we assert the following:
\begin{equation}
\begin{split}
\label{eq:mse_ineq_1}
    \tilde{J}^{(n+1)} &= \frac{1}{N_c} \sum_{i=1}^{N_c} \frac{1}{|\mathcal{B}_{i}^{(n+1)}|}\sum_{\bm{b} \in \mathcal{B}_{i}^{(n+1)}} \frac{1}{L_b}\lVert \bm{b} - C_i^{(n)}(b)\rVert^2_2 \le J^{(n)}
\end{split}
\end{equation}
where $\tilde{J}^{(n+1)}$ is the the quantization MSE after the vector clustering step at iteration $n+1$.

Next, during the codebook update step (\ref{eq:quantizers_update}) at iteration $n+1$, the per-cluster codebooks $\mathcal{C}^{(n)}$ are updated to $\mathcal{C}^{(n+1)}$ by invoking the Lloyd-Max algorithm \citep{Lloyd}. We know that for any given value distribution, the Lloyd-Max algorithm minimizes the quantization MSE. Therefore, for a given vector cluster $\mathcal{B}_i$ we obtain the following result:

\begin{equation}
    \frac{1}{|\mathcal{B}_{i}^{(n+1)}|}\sum_{\bm{b} \in \mathcal{B}_{i}^{(n+1)}} \frac{1}{L_b}\lVert \bm{b}- C_i^{(n+1)}(\bm{b})\rVert^2_2 \le \frac{1}{|\mathcal{B}_{i}^{(n+1)}|}\sum_{\bm{b} \in \mathcal{B}_{i}^{(n+1)}} \frac{1}{L_b}\lVert \bm{b}- C_i^{(n)}(\bm{b})\rVert^2_2
\end{equation}

The above equation states that quantizing the given block cluster $\mathcal{B}_i$ after updating the associated codebook from $C_i^{(n)}$ to $C_i^{(n+1)}$ results in lower quantization MSE. Since this is true for all the block clusters, we derive the following result: 
\begin{equation}
\begin{split}
\label{eq:mse_ineq_2}
     J^{(n+1)} &= \frac{1}{N_c} \sum_{i=1}^{N_c} \frac{1}{|\mathcal{B}_{i}^{(n+1)}|}\sum_{\bm{b} \in \mathcal{B}_{i}^{(n+1)}} \frac{1}{L_b}\lVert \bm{b}- C_i^{(n+1)}(\bm{b})\rVert^2_2  \le \tilde{J}^{(n+1)}   
\end{split}
\end{equation}

Following (\ref{eq:mse_ineq_1}) and (\ref{eq:mse_ineq_2}), we find that the quantization MSE is non-increasing for each iteration, that is, $J^{(1)} \ge J^{(2)} \ge J^{(3)} \ge \ldots \ge J^{(M)}$ where $M$ is the maximum number of iterations. 
%Therefore, we can say that if the algorithm converges, then it must be that it has converged to a local minimum. 
\hfill $\blacksquare$


\begin{figure}
    \begin{center}
    \includegraphics[width=0.5\textwidth]{sections//figures/mse_vs_iter.pdf}
    \end{center}
    \caption{\small NMSE vs iterations during LO-BCQ compared to other block quantization proposals}
    \label{fig:nmse_vs_iter}
\end{figure}

Figure \ref{fig:nmse_vs_iter} shows the empirical convergence of LO-BCQ across several block lengths and number of codebooks. Also, the MSE achieved by LO-BCQ is compared to baselines such as MXFP and VSQ. As shown, LO-BCQ converges to a lower MSE than the baselines. Further, we achieve better convergence for larger number of codebooks ($N_c$) and for a smaller block length ($L_b$), both of which increase the bitwidth of BCQ (see Eq \ref{eq:bitwidth_bcq}).


\subsection{Additional Accuracy Results}
%Table \ref{tab:lobcq_config} lists the various LOBCQ configurations and their corresponding bitwidths.
\begin{table}
\setlength{\tabcolsep}{4.75pt}
\begin{center}
\caption{\label{tab:lobcq_config} Various LO-BCQ configurations and their bitwidths.}
\begin{tabular}{|c||c|c|c|c||c|c||c|} 
\hline
 & \multicolumn{4}{|c||}{$L_b=8$} & \multicolumn{2}{|c||}{$L_b=4$} & $L_b=2$ \\
 \hline
 \backslashbox{$L_A$\kern-1em}{\kern-1em$N_c$} & 2 & 4 & 8 & 16 & 2 & 4 & 2 \\
 \hline
 64 & 4.25 & 4.375 & 4.5 & 4.625 & 4.375 & 4.625 & 4.625\\
 \hline
 32 & 4.375 & 4.5 & 4.625& 4.75 & 4.5 & 4.75 & 4.75 \\
 \hline
 16 & 4.625 & 4.75& 4.875 & 5 & 4.75 & 5 & 5 \\
 \hline
\end{tabular}
\end{center}
\end{table}

%\subsection{Perplexity achieved by various LO-BCQ configurations on Wikitext-103 dataset}

\begin{table} \centering
\begin{tabular}{|c||c|c|c|c||c|c||c|} 
\hline
 $L_b \rightarrow$& \multicolumn{4}{c||}{8} & \multicolumn{2}{c||}{4} & 2\\
 \hline
 \backslashbox{$L_A$\kern-1em}{\kern-1em$N_c$} & 2 & 4 & 8 & 16 & 2 & 4 & 2  \\
 %$N_c \rightarrow$ & 2 & 4 & 8 & 16 & 2 & 4 & 2 \\
 \hline
 \hline
 \multicolumn{8}{c}{GPT3-1.3B (FP32 PPL = 9.98)} \\ 
 \hline
 \hline
 64 & 10.40 & 10.23 & 10.17 & 10.15 &  10.28 & 10.18 & 10.19 \\
 \hline
 32 & 10.25 & 10.20 & 10.15 & 10.12 &  10.23 & 10.17 & 10.17 \\
 \hline
 16 & 10.22 & 10.16 & 10.10 & 10.09 &  10.21 & 10.14 & 10.16 \\
 \hline
  \hline
 \multicolumn{8}{c}{GPT3-8B (FP32 PPL = 7.38)} \\ 
 \hline
 \hline
 64 & 7.61 & 7.52 & 7.48 &  7.47 &  7.55 &  7.49 & 7.50 \\
 \hline
 32 & 7.52 & 7.50 & 7.46 &  7.45 &  7.52 &  7.48 & 7.48  \\
 \hline
 16 & 7.51 & 7.48 & 7.44 &  7.44 &  7.51 &  7.49 & 7.47  \\
 \hline
\end{tabular}
\caption{\label{tab:ppl_gpt3_abalation} Wikitext-103 perplexity across GPT3-1.3B and 8B models.}
\end{table}

\begin{table} \centering
\begin{tabular}{|c||c|c|c|c||} 
\hline
 $L_b \rightarrow$& \multicolumn{4}{c||}{8}\\
 \hline
 \backslashbox{$L_A$\kern-1em}{\kern-1em$N_c$} & 2 & 4 & 8 & 16 \\
 %$N_c \rightarrow$ & 2 & 4 & 8 & 16 & 2 & 4 & 2 \\
 \hline
 \hline
 \multicolumn{5}{|c|}{Llama2-7B (FP32 PPL = 5.06)} \\ 
 \hline
 \hline
 64 & 5.31 & 5.26 & 5.19 & 5.18  \\
 \hline
 32 & 5.23 & 5.25 & 5.18 & 5.15  \\
 \hline
 16 & 5.23 & 5.19 & 5.16 & 5.14  \\
 \hline
 \multicolumn{5}{|c|}{Nemotron4-15B (FP32 PPL = 5.87)} \\ 
 \hline
 \hline
 64  & 6.3 & 6.20 & 6.13 & 6.08  \\
 \hline
 32  & 6.24 & 6.12 & 6.07 & 6.03  \\
 \hline
 16  & 6.12 & 6.14 & 6.04 & 6.02  \\
 \hline
 \multicolumn{5}{|c|}{Nemotron4-340B (FP32 PPL = 3.48)} \\ 
 \hline
 \hline
 64 & 3.67 & 3.62 & 3.60 & 3.59 \\
 \hline
 32 & 3.63 & 3.61 & 3.59 & 3.56 \\
 \hline
 16 & 3.61 & 3.58 & 3.57 & 3.55 \\
 \hline
\end{tabular}
\caption{\label{tab:ppl_llama7B_nemo15B} Wikitext-103 perplexity compared to FP32 baseline in Llama2-7B and Nemotron4-15B, 340B models}
\end{table}

%\subsection{Perplexity achieved by various LO-BCQ configurations on MMLU dataset}


\begin{table} \centering
\begin{tabular}{|c||c|c|c|c||c|c|c|c|} 
\hline
 $L_b \rightarrow$& \multicolumn{4}{c||}{8} & \multicolumn{4}{c||}{8}\\
 \hline
 \backslashbox{$L_A$\kern-1em}{\kern-1em$N_c$} & 2 & 4 & 8 & 16 & 2 & 4 & 8 & 16  \\
 %$N_c \rightarrow$ & 2 & 4 & 8 & 16 & 2 & 4 & 2 \\
 \hline
 \hline
 \multicolumn{5}{|c|}{Llama2-7B (FP32 Accuracy = 45.8\%)} & \multicolumn{4}{|c|}{Llama2-70B (FP32 Accuracy = 69.12\%)} \\ 
 \hline
 \hline
 64 & 43.9 & 43.4 & 43.9 & 44.9 & 68.07 & 68.27 & 68.17 & 68.75 \\
 \hline
 32 & 44.5 & 43.8 & 44.9 & 44.5 & 68.37 & 68.51 & 68.35 & 68.27  \\
 \hline
 16 & 43.9 & 42.7 & 44.9 & 45 & 68.12 & 68.77 & 68.31 & 68.59  \\
 \hline
 \hline
 \multicolumn{5}{|c|}{GPT3-22B (FP32 Accuracy = 38.75\%)} & \multicolumn{4}{|c|}{Nemotron4-15B (FP32 Accuracy = 64.3\%)} \\ 
 \hline
 \hline
 64 & 36.71 & 38.85 & 38.13 & 38.92 & 63.17 & 62.36 & 63.72 & 64.09 \\
 \hline
 32 & 37.95 & 38.69 & 39.45 & 38.34 & 64.05 & 62.30 & 63.8 & 64.33  \\
 \hline
 16 & 38.88 & 38.80 & 38.31 & 38.92 & 63.22 & 63.51 & 63.93 & 64.43  \\
 \hline
\end{tabular}
\caption{\label{tab:mmlu_abalation} Accuracy on MMLU dataset across GPT3-22B, Llama2-7B, 70B and Nemotron4-15B models.}
\end{table}


%\subsection{Perplexity achieved by various LO-BCQ configurations on LM evaluation harness}

\begin{table} \centering
\begin{tabular}{|c||c|c|c|c||c|c|c|c|} 
\hline
 $L_b \rightarrow$& \multicolumn{4}{c||}{8} & \multicolumn{4}{c||}{8}\\
 \hline
 \backslashbox{$L_A$\kern-1em}{\kern-1em$N_c$} & 2 & 4 & 8 & 16 & 2 & 4 & 8 & 16  \\
 %$N_c \rightarrow$ & 2 & 4 & 8 & 16 & 2 & 4 & 2 \\
 \hline
 \hline
 \multicolumn{5}{|c|}{Race (FP32 Accuracy = 37.51\%)} & \multicolumn{4}{|c|}{Boolq (FP32 Accuracy = 64.62\%)} \\ 
 \hline
 \hline
 64 & 36.94 & 37.13 & 36.27 & 37.13 & 63.73 & 62.26 & 63.49 & 63.36 \\
 \hline
 32 & 37.03 & 36.36 & 36.08 & 37.03 & 62.54 & 63.51 & 63.49 & 63.55  \\
 \hline
 16 & 37.03 & 37.03 & 36.46 & 37.03 & 61.1 & 63.79 & 63.58 & 63.33  \\
 \hline
 \hline
 \multicolumn{5}{|c|}{Winogrande (FP32 Accuracy = 58.01\%)} & \multicolumn{4}{|c|}{Piqa (FP32 Accuracy = 74.21\%)} \\ 
 \hline
 \hline
 64 & 58.17 & 57.22 & 57.85 & 58.33 & 73.01 & 73.07 & 73.07 & 72.80 \\
 \hline
 32 & 59.12 & 58.09 & 57.85 & 58.41 & 73.01 & 73.94 & 72.74 & 73.18  \\
 \hline
 16 & 57.93 & 58.88 & 57.93 & 58.56 & 73.94 & 72.80 & 73.01 & 73.94  \\
 \hline
\end{tabular}
\caption{\label{tab:mmlu_abalation} Accuracy on LM evaluation harness tasks on GPT3-1.3B model.}
\end{table}

\begin{table} \centering
\begin{tabular}{|c||c|c|c|c||c|c|c|c|} 
\hline
 $L_b \rightarrow$& \multicolumn{4}{c||}{8} & \multicolumn{4}{c||}{8}\\
 \hline
 \backslashbox{$L_A$\kern-1em}{\kern-1em$N_c$} & 2 & 4 & 8 & 16 & 2 & 4 & 8 & 16  \\
 %$N_c \rightarrow$ & 2 & 4 & 8 & 16 & 2 & 4 & 2 \\
 \hline
 \hline
 \multicolumn{5}{|c|}{Race (FP32 Accuracy = 41.34\%)} & \multicolumn{4}{|c|}{Boolq (FP32 Accuracy = 68.32\%)} \\ 
 \hline
 \hline
 64 & 40.48 & 40.10 & 39.43 & 39.90 & 69.20 & 68.41 & 69.45 & 68.56 \\
 \hline
 32 & 39.52 & 39.52 & 40.77 & 39.62 & 68.32 & 67.43 & 68.17 & 69.30  \\
 \hline
 16 & 39.81 & 39.71 & 39.90 & 40.38 & 68.10 & 66.33 & 69.51 & 69.42  \\
 \hline
 \hline
 \multicolumn{5}{|c|}{Winogrande (FP32 Accuracy = 67.88\%)} & \multicolumn{4}{|c|}{Piqa (FP32 Accuracy = 78.78\%)} \\ 
 \hline
 \hline
 64 & 66.85 & 66.61 & 67.72 & 67.88 & 77.31 & 77.42 & 77.75 & 77.64 \\
 \hline
 32 & 67.25 & 67.72 & 67.72 & 67.00 & 77.31 & 77.04 & 77.80 & 77.37  \\
 \hline
 16 & 68.11 & 68.90 & 67.88 & 67.48 & 77.37 & 78.13 & 78.13 & 77.69  \\
 \hline
\end{tabular}
\caption{\label{tab:mmlu_abalation} Accuracy on LM evaluation harness tasks on GPT3-8B model.}
\end{table}

\begin{table} \centering
\begin{tabular}{|c||c|c|c|c||c|c|c|c|} 
\hline
 $L_b \rightarrow$& \multicolumn{4}{c||}{8} & \multicolumn{4}{c||}{8}\\
 \hline
 \backslashbox{$L_A$\kern-1em}{\kern-1em$N_c$} & 2 & 4 & 8 & 16 & 2 & 4 & 8 & 16  \\
 %$N_c \rightarrow$ & 2 & 4 & 8 & 16 & 2 & 4 & 2 \\
 \hline
 \hline
 \multicolumn{5}{|c|}{Race (FP32 Accuracy = 40.67\%)} & \multicolumn{4}{|c|}{Boolq (FP32 Accuracy = 76.54\%)} \\ 
 \hline
 \hline
 64 & 40.48 & 40.10 & 39.43 & 39.90 & 75.41 & 75.11 & 77.09 & 75.66 \\
 \hline
 32 & 39.52 & 39.52 & 40.77 & 39.62 & 76.02 & 76.02 & 75.96 & 75.35  \\
 \hline
 16 & 39.81 & 39.71 & 39.90 & 40.38 & 75.05 & 73.82 & 75.72 & 76.09  \\
 \hline
 \hline
 \multicolumn{5}{|c|}{Winogrande (FP32 Accuracy = 70.64\%)} & \multicolumn{4}{|c|}{Piqa (FP32 Accuracy = 79.16\%)} \\ 
 \hline
 \hline
 64 & 69.14 & 70.17 & 70.17 & 70.56 & 78.24 & 79.00 & 78.62 & 78.73 \\
 \hline
 32 & 70.96 & 69.69 & 71.27 & 69.30 & 78.56 & 79.49 & 79.16 & 78.89  \\
 \hline
 16 & 71.03 & 69.53 & 69.69 & 70.40 & 78.13 & 79.16 & 79.00 & 79.00  \\
 \hline
\end{tabular}
\caption{\label{tab:mmlu_abalation} Accuracy on LM evaluation harness tasks on GPT3-22B model.}
\end{table}

\begin{table} \centering
\begin{tabular}{|c||c|c|c|c||c|c|c|c|} 
\hline
 $L_b \rightarrow$& \multicolumn{4}{c||}{8} & \multicolumn{4}{c||}{8}\\
 \hline
 \backslashbox{$L_A$\kern-1em}{\kern-1em$N_c$} & 2 & 4 & 8 & 16 & 2 & 4 & 8 & 16  \\
 %$N_c \rightarrow$ & 2 & 4 & 8 & 16 & 2 & 4 & 2 \\
 \hline
 \hline
 \multicolumn{5}{|c|}{Race (FP32 Accuracy = 44.4\%)} & \multicolumn{4}{|c|}{Boolq (FP32 Accuracy = 79.29\%)} \\ 
 \hline
 \hline
 64 & 42.49 & 42.51 & 42.58 & 43.45 & 77.58 & 77.37 & 77.43 & 78.1 \\
 \hline
 32 & 43.35 & 42.49 & 43.64 & 43.73 & 77.86 & 75.32 & 77.28 & 77.86  \\
 \hline
 16 & 44.21 & 44.21 & 43.64 & 42.97 & 78.65 & 77 & 76.94 & 77.98  \\
 \hline
 \hline
 \multicolumn{5}{|c|}{Winogrande (FP32 Accuracy = 69.38\%)} & \multicolumn{4}{|c|}{Piqa (FP32 Accuracy = 78.07\%)} \\ 
 \hline
 \hline
 64 & 68.9 & 68.43 & 69.77 & 68.19 & 77.09 & 76.82 & 77.09 & 77.86 \\
 \hline
 32 & 69.38 & 68.51 & 68.82 & 68.90 & 78.07 & 76.71 & 78.07 & 77.86  \\
 \hline
 16 & 69.53 & 67.09 & 69.38 & 68.90 & 77.37 & 77.8 & 77.91 & 77.69  \\
 \hline
\end{tabular}
\caption{\label{tab:mmlu_abalation} Accuracy on LM evaluation harness tasks on Llama2-7B model.}
\end{table}

\begin{table} \centering
\begin{tabular}{|c||c|c|c|c||c|c|c|c|} 
\hline
 $L_b \rightarrow$& \multicolumn{4}{c||}{8} & \multicolumn{4}{c||}{8}\\
 \hline
 \backslashbox{$L_A$\kern-1em}{\kern-1em$N_c$} & 2 & 4 & 8 & 16 & 2 & 4 & 8 & 16  \\
 %$N_c \rightarrow$ & 2 & 4 & 8 & 16 & 2 & 4 & 2 \\
 \hline
 \hline
 \multicolumn{5}{|c|}{Race (FP32 Accuracy = 48.8\%)} & \multicolumn{4}{|c|}{Boolq (FP32 Accuracy = 85.23\%)} \\ 
 \hline
 \hline
 64 & 49.00 & 49.00 & 49.28 & 48.71 & 82.82 & 84.28 & 84.03 & 84.25 \\
 \hline
 32 & 49.57 & 48.52 & 48.33 & 49.28 & 83.85 & 84.46 & 84.31 & 84.93  \\
 \hline
 16 & 49.85 & 49.09 & 49.28 & 48.99 & 85.11 & 84.46 & 84.61 & 83.94  \\
 \hline
 \hline
 \multicolumn{5}{|c|}{Winogrande (FP32 Accuracy = 79.95\%)} & \multicolumn{4}{|c|}{Piqa (FP32 Accuracy = 81.56\%)} \\ 
 \hline
 \hline
 64 & 78.77 & 78.45 & 78.37 & 79.16 & 81.45 & 80.69 & 81.45 & 81.5 \\
 \hline
 32 & 78.45 & 79.01 & 78.69 & 80.66 & 81.56 & 80.58 & 81.18 & 81.34  \\
 \hline
 16 & 79.95 & 79.56 & 79.79 & 79.72 & 81.28 & 81.66 & 81.28 & 80.96  \\
 \hline
\end{tabular}
\caption{\label{tab:mmlu_abalation} Accuracy on LM evaluation harness tasks on Llama2-70B model.}
\end{table}

%\section{MSE Studies}
%\textcolor{red}{TODO}


\subsection{Number Formats and Quantization Method}
\label{subsec:numFormats_quantMethod}
\subsubsection{Integer Format}
An $n$-bit signed integer (INT) is typically represented with a 2s-complement format \citep{yao2022zeroquant,xiao2023smoothquant,dai2021vsq}, where the most significant bit denotes the sign.

\subsubsection{Floating Point Format}
An $n$-bit signed floating point (FP) number $x$ comprises of a 1-bit sign ($x_{\mathrm{sign}}$), $B_m$-bit mantissa ($x_{\mathrm{mant}}$) and $B_e$-bit exponent ($x_{\mathrm{exp}}$) such that $B_m+B_e=n-1$. The associated constant exponent bias ($E_{\mathrm{bias}}$) is computed as $(2^{{B_e}-1}-1)$. We denote this format as $E_{B_e}M_{B_m}$.  

\subsubsection{Quantization Scheme}
\label{subsec:quant_method}
A quantization scheme dictates how a given unquantized tensor is converted to its quantized representation. We consider FP formats for the purpose of illustration. Given an unquantized tensor $\bm{X}$ and an FP format $E_{B_e}M_{B_m}$, we first, we compute the quantization scale factor $s_X$ that maps the maximum absolute value of $\bm{X}$ to the maximum quantization level of the $E_{B_e}M_{B_m}$ format as follows:
\begin{align}
\label{eq:sf}
    s_X = \frac{\mathrm{max}(|\bm{X}|)}{\mathrm{max}(E_{B_e}M_{B_m})}
\end{align}
In the above equation, $|\cdot|$ denotes the absolute value function.

Next, we scale $\bm{X}$ by $s_X$ and quantize it to $\hat{\bm{X}}$ by rounding it to the nearest quantization level of $E_{B_e}M_{B_m}$ as:

\begin{align}
\label{eq:tensor_quant}
    \hat{\bm{X}} = \text{round-to-nearest}\left(\frac{\bm{X}}{s_X}, E_{B_e}M_{B_m}\right)
\end{align}

We perform dynamic max-scaled quantization \citep{wu2020integer}, where the scale factor $s$ for activations is dynamically computed during runtime.

\subsection{Vector Scaled Quantization}
\begin{wrapfigure}{r}{0.35\linewidth}
  \centering
  \includegraphics[width=\linewidth]{sections/figures/vsquant.jpg}
  \caption{\small Vectorwise decomposition for per-vector scaled quantization (VSQ \citep{dai2021vsq}).}
  \label{fig:vsquant}
\end{wrapfigure}
During VSQ \citep{dai2021vsq}, the operand tensors are decomposed into 1D vectors in a hardware friendly manner as shown in Figure \ref{fig:vsquant}. Since the decomposed tensors are used as operands in matrix multiplications during inference, it is beneficial to perform this decomposition along the reduction dimension of the multiplication. The vectorwise quantization is performed similar to tensorwise quantization described in Equations \ref{eq:sf} and \ref{eq:tensor_quant}, where a scale factor $s_v$ is required for each vector $\bm{v}$ that maps the maximum absolute value of that vector to the maximum quantization level. While smaller vector lengths can lead to larger accuracy gains, the associated memory and computational overheads due to the per-vector scale factors increases. To alleviate these overheads, VSQ \citep{dai2021vsq} proposed a second level quantization of the per-vector scale factors to unsigned integers, while MX \citep{rouhani2023shared} quantizes them to integer powers of 2 (denoted as $2^{INT}$).

\subsubsection{MX Format}
The MX format proposed in \citep{rouhani2023microscaling} introduces the concept of sub-block shifting. For every two scalar elements of $b$-bits each, there is a shared exponent bit. The value of this exponent bit is determined through an empirical analysis that targets minimizing quantization MSE. We note that the FP format $E_{1}M_{b}$ is strictly better than MX from an accuracy perspective since it allocates a dedicated exponent bit to each scalar as opposed to sharing it across two scalars. Therefore, we conservatively bound the accuracy of a $b+2$-bit signed MX format with that of a $E_{1}M_{b}$ format in our comparisons. For instance, we use E1M2 format as a proxy for MX4.

\begin{figure}
    \centering
    \includegraphics[width=1\linewidth]{sections//figures/BlockFormats.pdf}
    \caption{\small Comparing LO-BCQ to MX format.}
    \label{fig:block_formats}
\end{figure}

Figure \ref{fig:block_formats} compares our $4$-bit LO-BCQ block format to MX \citep{rouhani2023microscaling}. As shown, both LO-BCQ and MX decompose a given operand tensor into block arrays and each block array into blocks. Similar to MX, we find that per-block quantization ($L_b < L_A$) leads to better accuracy due to increased flexibility. While MX achieves this through per-block $1$-bit micro-scales, we associate a dedicated codebook to each block through a per-block codebook selector. Further, MX quantizes the per-block array scale-factor to E8M0 format without per-tensor scaling. In contrast during LO-BCQ, we find that per-tensor scaling combined with quantization of per-block array scale-factor to E4M3 format results in superior inference accuracy across models. 



\end{document}
\endinput
