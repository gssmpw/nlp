%%
%% This is file `sample-manuscript.tex',
%% generated with the docstrip utility.
%%
%% The original source files were:
%%
%% samples.dtx  (with options: `all,proceedings,bibtex,manuscript')
%% 
%% IMPORTANT NOTICE:
%% 
%% For the copyright see the source file.
%% 
%% Any modified versions of this file must be renamed
%% with new filenames distinct from sample-manuscript.tex.
%% 
%% For distribution of the original source see the terms
%% for copying and modification in the file samples.dtx.
%% 
%% This generated file may be distributed as long as the
%% original source files, as listed above, are part of the
%% same distribution. (The sources need not necessarily be
%% in the same archive or directory.)
%%
%%
%% Commands for TeXCount
%TC:macro \cite [option:text,text]
%TC:macro \citep [option:text,text]
%TC:macro \citet [option:text,text]
%TC:envir table 0 1
%TC:envir table* 0 1
%TC:envir tabular [ignore] word
%TC:envir displaymath 0 word
%TC:envir math 0 word
%TC:envir comment 0 0
%%
%%
%% The first command in your LaTeX source must be the \documentclass
%% command.
%%
%% For submission and review of your manuscript please change the
%% command to \documentclass[manuscript, screen, review]{acmart}.
%%
%% When submitting camera ready or to TAPS, please change the command
%% to \documentclass[sigconf]{acmart} or whichever template is required
%% for your publication.
%%
%%

%% FOR SUBMISSION
% \documentclass[manuscript,screen,review,anonymous]{acmart}
%% FOR SHARING AND PRE-PRINT
\documentclass[manuscript,screen]{acmart}


%%
%% \BibTeX command to typeset BibTeX logo in the docs
\AtBeginDocument{%
  \providecommand\BibTeX{{%
    Bib\TeX}}}

%% Rights management information.  This information is sent to you
%% when you complete the rights form.  These commands have SAMPLE
%% values in them; it is your responsibility as an author to replace
%% the commands and values with those provided to you when you
%% complete the rights form.
\setcopyright{acmlicensed}
\copyrightyear{2025}
\acmYear{2025}
\acmDOI{XXXXXXX.XXXXXXX}

%% These commands are for a PROCEEDINGS abstract or paper.
% \acmConference[FAccT '25]{The 2025 ACM Conference on Fairness, Accountability, and Transparency}{June, 2025}{Athens, Greece}
%%
%%  Uncomment \acmBooktitle if the title of the proceedings is different
%%  from ``Proceedings of ...''!
%%
%%\acmBooktitle{Woodstock '18: ACM Symposium on Neural Gaze Detection,
%%  June 03--05, 2018, Woodstock, NY}
% \acmISBN{978-1-4503-XXXX-X/25/06}


%%
%% Submission ID.
%% Use this when submitting an article to a sponsored event. You'll
%% receive a unique submission ID from the organizers
%% of the event, and this ID should be used as the parameter to this command.
\acmSubmissionID{957}

%%
%% For managing citations, it is recommended to use bibliography
%% files in BibTeX format.
%%
%% You can then either use BibTeX with the ACM-Reference-Format style,
%% or BibLaTeX with the acmnumeric or acmauthoryear sytles, that include
%% support for advanced citation of software artefact from the
%% biblatex-software package, also separately available on CTAN.
%%
%% Look at the sample-*-biblatex.tex files for templates showcasing
%% the biblatex styles.
%%

%%
%% The majority of ACM publications use numbered citations and
%% references.  The command \citestyle{authoryear} switches to the
%% "author year" style.
%%
%% If you are preparing content for an event
%% sponsored by ACM SIGGRAPH, you must use the "author year" style of
%% citations and references.
%% Uncommenting
%% the next command will enable that style.
%%\citestyle{acmauthoryear}


%%
%% end of the preamble, start of the body of the document source.

%|||||||||||||||||||||||||||||||||||||||||||||||||||||||||||||||||||||||||%
%                Packages                                                  
%|||||||||||||||||||||||||||||||||||||||||||||||||||||||||||||||||||||||||%
\usepackage{soul}
\usepackage{hyperref}
%\usepackage[textsize=small,shadow]{todonotes}  %Todo utility for PDF.
%\usepackage[dvipsnames]{xcolor} % For accessing additional color names in latex: https://www.overleaf.com/learn/latex/Using_colors_in_LaTeX#Accessing_additional_named_colors

\usepackage[en-GB]{datetime2} % to display dates like \today in British format
%\DTMlangsetup[en-GB]{ord=raise}

%|||||||||||||||||||||||||||||||||||||||||||||||||||||||||||||||||||||||||%
%             Custom Commands              
%|||||||||||||||||||||||||||||||||||||||||||||||||||||||||||||||||||||||||%

\newcommand{\note}[1]
{\todo[inline, size=\small, caption={note}]{\begin{minipage}{\textwidth}\textbf{Note:}~#1\end{minipage}}}

\newcommand{\cm}[1]
{\todo[size=\small, color={red!100!green!33}, caption={comment}]{\begin{minipage}{\textwidth}\textbf{Comment:}~#1\end{minipage}}}

\renewcommand{\today}{22\textsuperscript{nd} January 2025}

%%%%%%%%%%%%%%%%%%%%%%%%%%%%%%%%%%%%%%%%%%%%%%%%%%%%%%%%%%%%%%
%%   Draft SETTINGS:
%%
%%   If 'draft' is set to True comments will appear in the PDF
%%%%%%%%%%%%%%%%%%%%%%%%%%%%%%%%%%%%%%%%%%%%%%%%%%%%%%%%%%%%%%

\newif\ifdraft
\draftfalse %% (un/)comment to turn (on/)off and do the opposite with the line below
%\draftfalse
\ifdraft
	\def\an#1{{\color{MidnightBlue}[AN]#1}}
	\def\me#1{{\color{RedOrange}[ME]#1}}
        \def\ep#1{{\color{PineGreen}[EP]#1}}
        \def\dfl#1{{\color{Plum}[DFL]#1}}
        \def\jv#1{{\color{TealBlue}[JV]#1}}
        \def\gc#1{{\color{Violet}[GC]#1}}
        \def\eg#1{{\color{WildStrawberry}[EG]#1}}

\else
	\def\an#1{}
	\def\me#1{}
        \def\ep#1{}
        \def\dfl#1{}
        \def\jv#1{}
        \def\gc#1{}
        \def\eg#1{}
\fi



\begin{document}

%%
%% The "title" command has an optional parameter,
%% allowing the author to define a "short title" to be used in page headers.
\title[Can We Trust AI Benchmarks? An Interdisciplinary Review of Current Issues in AI Evaluation]{Can We Trust AI Benchmarks?\\An Interdisciplinary Review of Current Issues in AI Evaluation}

%%
%% The "author" command and its associated commands are used to define
%% the authors and their affiliations.
%% Of note is the shared affiliation of the first two authors, and the
%% "authornote" and "authornotemark" commands
%% used to denote shared contribution to the research.
\author{Maria Eriksson}
\authornote{\textbf{Disclaimer:} The views expressed in this paper are purely those of the authors and may not, under any circumstances, be regarded as an official position of the European Commission.}
\email{maria.eriksson@ec.europa.eu}
\orcid{0000-0002-7534-4268}
\affiliation{%
  \institution{European Commission, Joint Research Centre (JRC), Seville}
  \country{Spain}
}

\author{Erasmo Purificato}
\authornotemark[1]
\email{erasmo.purificato@ec.europa.eu}
\orcid{0000-0002-5506-3020}
\affiliation{%
  \institution{European Commission, Joint Research Centre (JRC), Ispra}
  \country{Italy}
}

\author{Arman Noroozian}
\authornotemark[1]
\email{arman.noroozian@ec.europa.eu}
\orcid{}
\affiliation{%
  \institution{European Commission, Joint Research Centre (JRC), Brussels}
  \country{Belgium}
}

\author{Jo\~{a}o Vinagre}
\authornotemark[1]
\email{joao.vinagre@ec.europa.eu}
\orcid{0000-0001-6219-3977}
\affiliation{%
  \institution{European Commission, Joint Research Centre (JRC), Seville}
  \country{Spain}
}

\author{Guillaume Chaslot}
\authornotemark[1]
\email{guillaume.chaslot@ec.europa.eu}
\orcid{}
\affiliation{%
  \institution{European Commission, Joint Research Centre (JRC), Brussels}
  \country{Belgium}
}

\author{Emilia Gomez}
\authornotemark[1]
\email{emilia.gomez-gutierrez@ec.europa.eu}
\orcid{0000-0003-4983-3989}
\affiliation{%
  \institution{European Commission, Joint Research Centre (JRC), Seville}
  \country{Spain}
}

\author{David Fernandez-Llorca}
\authornotemark[1]
\email{david.fernandez-llorca@ec.europa.eu}
\orcid{0000-0003-2433-7110}
\affiliation{%
  \institution{European Commission, Joint Research Centre (JRC), Seville}
  \country{Spain}
}

%%
%% By default, the full list of authors will be used in the page
%% headers. Often, this list is too long, and will overlap
%% other information printed in the page headers. This command allows
%% the author to define a more concise list
%% of authors' names for this purpose.
\renewcommand{\shortauthors}{Eriksson et al.}

%%
%% The abstract is a short summary of the work to be presented in the
%% article.
\begin{abstract}  
Test time scaling is currently one of the most active research areas that shows promise after training time scaling has reached its limits.
Deep-thinking (DT) models are a class of recurrent models that can perform easy-to-hard generalization by assigning more compute to harder test samples.
However, due to their inability to determine the complexity of a test sample, DT models have to use a large amount of computation for both easy and hard test samples.
Excessive test time computation is wasteful and can cause the ``overthinking'' problem where more test time computation leads to worse results.
In this paper, we introduce a test time training method for determining the optimal amount of computation needed for each sample during test time.
We also propose Conv-LiGRU, a novel recurrent architecture for efficient and robust visual reasoning. 
Extensive experiments demonstrate that Conv-LiGRU is more stable than DT, effectively mitigates the ``overthinking'' phenomenon, and achieves superior accuracy.
\end{abstract}  

%%
%% The code below is generated by the tool at http://dl.acm.org/ccs.cfm.
%% Please copy and paste the code instead of the example below.
%%
\begin{CCSXML}
<ccs2012>
   <concept>
       <concept_id>10002944.10011123.10011130</concept_id>
       <concept_desc>General and reference~Evaluation</concept_desc>
       <concept_significance>500</concept_significance>
       </concept>
   <concept>
       <concept_id>10002944.10011123.10011124</concept_id>
       <concept_desc>General and reference~Metrics</concept_desc>
       <concept_significance>500</concept_significance>
       </concept>
   <concept>
       <concept_id>10002944.10011123.10010916</concept_id>
       <concept_desc>General and reference~Measurement</concept_desc>
       <concept_significance>500</concept_significance>
       </concept>
   <concept>
       <concept_id>10002944.10011123.10011675</concept_id>
       <concept_desc>General and reference~Validation</concept_desc>
       <concept_significance>500</concept_significance>
       </concept>
   <concept>
       <concept_id>10002944.10011123.10010577</concept_id>
       <concept_desc>General and reference~Reliability</concept_desc>
       <concept_significance>500</concept_significance>
       </concept>
 </ccs2012>
\end{CCSXML}

\ccsdesc[500]{General and reference~Evaluation}
\ccsdesc[500]{General and reference~Metrics}
\ccsdesc[500]{General and reference~Measurement}
\ccsdesc[500]{General and reference~Validation}
\ccsdesc[500]{General and reference~Reliability}

%%
%% Keywords. The author(s) should pick words that accurately describe
%% the work being presented. Separate the keywords with commas.
\keywords{AI benchmarks, Benchmark critique, AI evaluation, Safety evaluation, AI Regulation}

% \received{20 February 2007}
% \received[revised]{12 March 2009}
% \received[accepted]{5 June 2009}

%%
%% This command processes the author and affiliation and title
%% information and builds the first part of the formatted document.
\maketitle

\section{Introduction}
\label{sec:introduction}
The business processes of organizations are experiencing ever-increasing complexity due to the large amount of data, high number of users, and high-tech devices involved \cite{martin2021pmopportunitieschallenges, beerepoot2023biggestbpmproblems}. This complexity may cause business processes to deviate from normal control flow due to unforeseen and disruptive anomalies \cite{adams2023proceddsriftdetection}. These control-flow anomalies manifest as unknown, skipped, and wrongly-ordered activities in the traces of event logs monitored from the execution of business processes \cite{ko2023adsystematicreview}. For the sake of clarity, let us consider an illustrative example of such anomalies. Figure \ref{FP_ANOMALIES} shows a so-called event log footprint, which captures the control flow relations of four activities of a hypothetical event log. In particular, this footprint captures the control-flow relations between activities \texttt{a}, \texttt{b}, \texttt{c} and \texttt{d}. These are the causal ($\rightarrow$) relation, concurrent ($\parallel$) relation, and other ($\#$) relations such as exclusivity or non-local dependency \cite{aalst2022pmhandbook}. In addition, on the right are six traces, of which five exhibit skipped, wrongly-ordered and unknown control-flow anomalies. For example, $\langle$\texttt{a b d}$\rangle$ has a skipped activity, which is \texttt{c}. Because of this skipped activity, the control-flow relation \texttt{b}$\,\#\,$\texttt{d} is violated, since \texttt{d} directly follows \texttt{b} in the anomalous trace.
\begin{figure}[!t]
\centering
\includegraphics[width=0.9\columnwidth]{images/FP_ANOMALIES.png}
\caption{An example event log footprint with six traces, of which five exhibit control-flow anomalies.}
\label{FP_ANOMALIES}
\end{figure}

\subsection{Control-flow anomaly detection}
Control-flow anomaly detection techniques aim to characterize the normal control flow from event logs and verify whether these deviations occur in new event logs \cite{ko2023adsystematicreview}. To develop control-flow anomaly detection techniques, \revision{process mining} has seen widespread adoption owing to process discovery and \revision{conformance checking}. On the one hand, process discovery is a set of algorithms that encode control-flow relations as a set of model elements and constraints according to a given modeling formalism \cite{aalst2022pmhandbook}; hereafter, we refer to the Petri net, a widespread modeling formalism. On the other hand, \revision{conformance checking} is an explainable set of algorithms that allows linking any deviations with the reference Petri net and providing the fitness measure, namely a measure of how much the Petri net fits the new event log \cite{aalst2022pmhandbook}. Many control-flow anomaly detection techniques based on \revision{conformance checking} (hereafter, \revision{conformance checking}-based techniques) use the fitness measure to determine whether an event log is anomalous \cite{bezerra2009pmad, bezerra2013adlogspais, myers2018icsadpm, pecchia2020applicationfailuresanalysispm}. 

The scientific literature also includes many \revision{conformance checking}-independent techniques for control-flow anomaly detection that combine specific types of trace encodings with machine/deep learning \cite{ko2023adsystematicreview, tavares2023pmtraceencoding}. Whereas these techniques are very effective, their explainability is challenging due to both the type of trace encoding employed and the machine/deep learning model used \cite{rawal2022trustworthyaiadvances,li2023explainablead}. Hence, in the following, we focus on the shortcomings of \revision{conformance checking}-based techniques to investigate whether it is possible to support the development of competitive control-flow anomaly detection techniques while maintaining the explainable nature of \revision{conformance checking}.
\begin{figure}[!t]
\centering
\includegraphics[width=\columnwidth]{images/HIGH_LEVEL_VIEW.png}
\caption{A high-level view of the proposed framework for combining \revision{process mining}-based feature extraction with dimensionality reduction for control-flow anomaly detection.}
\label{HIGH_LEVEL_VIEW}
\end{figure}

\subsection{Shortcomings of \revision{conformance checking}-based techniques}
Unfortunately, the detection effectiveness of \revision{conformance checking}-based techniques is affected by noisy data and low-quality Petri nets, which may be due to human errors in the modeling process or representational bias of process discovery algorithms \cite{bezerra2013adlogspais, pecchia2020applicationfailuresanalysispm, aalst2016pm}. Specifically, on the one hand, noisy data may introduce infrequent and deceptive control-flow relations that may result in inconsistent fitness measures, whereas, on the other hand, checking event logs against a low-quality Petri net could lead to an unreliable distribution of fitness measures. Nonetheless, such Petri nets can still be used as references to obtain insightful information for \revision{process mining}-based feature extraction, supporting the development of competitive and explainable \revision{conformance checking}-based techniques for control-flow anomaly detection despite the problems above. For example, a few works outline that token-based \revision{conformance checking} can be used for \revision{process mining}-based feature extraction to build tabular data and develop effective \revision{conformance checking}-based techniques for control-flow anomaly detection \cite{singh2022lapmsh, debenedictis2023dtadiiot}. However, to the best of our knowledge, the scientific literature lacks a structured proposal for \revision{process mining}-based feature extraction using the state-of-the-art \revision{conformance checking} variant, namely alignment-based \revision{conformance checking}.

\subsection{Contributions}
We propose a novel \revision{process mining}-based feature extraction approach with alignment-based \revision{conformance checking}. This variant aligns the deviating control flow with a reference Petri net; the resulting alignment can be inspected to extract additional statistics such as the number of times a given activity caused mismatches \cite{aalst2022pmhandbook}. We integrate this approach into a flexible and explainable framework for developing techniques for control-flow anomaly detection. The framework combines \revision{process mining}-based feature extraction and dimensionality reduction to handle high-dimensional feature sets, achieve detection effectiveness, and support explainability. Notably, in addition to our proposed \revision{process mining}-based feature extraction approach, the framework allows employing other approaches, enabling a fair comparison of multiple \revision{conformance checking}-based and \revision{conformance checking}-independent techniques for control-flow anomaly detection. Figure \ref{HIGH_LEVEL_VIEW} shows a high-level view of the framework. Business processes are monitored, and event logs obtained from the database of information systems. Subsequently, \revision{process mining}-based feature extraction is applied to these event logs and tabular data input to dimensionality reduction to identify control-flow anomalies. We apply several \revision{conformance checking}-based and \revision{conformance checking}-independent framework techniques to publicly available datasets, simulated data of a case study from railways, and real-world data of a case study from healthcare. We show that the framework techniques implementing our approach outperform the baseline \revision{conformance checking}-based techniques while maintaining the explainable nature of \revision{conformance checking}.

In summary, the contributions of this paper are as follows.
\begin{itemize}
    \item{
        A novel \revision{process mining}-based feature extraction approach to support the development of competitive and explainable \revision{conformance checking}-based techniques for control-flow anomaly detection.
    }
    \item{
        A flexible and explainable framework for developing techniques for control-flow anomaly detection using \revision{process mining}-based feature extraction and dimensionality reduction.
    }
    \item{
        Application to synthetic and real-world datasets of several \revision{conformance checking}-based and \revision{conformance checking}-independent framework techniques, evaluating their detection effectiveness and explainability.
    }
\end{itemize}

The rest of the paper is organized as follows.
\begin{itemize}
    \item Section \ref{sec:related_work} reviews the existing techniques for control-flow anomaly detection, categorizing them into \revision{conformance checking}-based and \revision{conformance checking}-independent techniques.
    \item Section \ref{sec:abccfe} provides the preliminaries of \revision{process mining} to establish the notation used throughout the paper, and delves into the details of the proposed \revision{process mining}-based feature extraction approach with alignment-based \revision{conformance checking}.
    \item Section \ref{sec:framework} describes the framework for developing \revision{conformance checking}-based and \revision{conformance checking}-independent techniques for control-flow anomaly detection that combine \revision{process mining}-based feature extraction and dimensionality reduction.
    \item Section \ref{sec:evaluation} presents the experiments conducted with multiple framework and baseline techniques using data from publicly available datasets and case studies.
    \item Section \ref{sec:conclusions} draws the conclusions and presents future work.
\end{itemize}

\section{Background}\label{sec:backgrnd}

\subsection{Cold Start Latency and Mitigation Techniques}

Traditional FaaS platforms mitigate cold starts through snapshotting, lightweight virtualization, and warm-state management. Snapshot-based methods like \textbf{REAP} and \textbf{Catalyzer} reduce initialization time by preloading or restoring container states but require significant memory and I/O resources, limiting scalability~\cite{dong_catalyzer_2020, ustiugov_benchmarking_2021}. Lightweight virtualization solutions, such as \textbf{Firecracker} microVMs, achieve fast startup times with strong isolation but depend on robust infrastructure, making them less adaptable to fluctuating workloads~\cite{agache_firecracker_2020}. Warm-state management techniques like \textbf{Faa\$T}~\cite{romero_faa_2021} and \textbf{Kraken}~\cite{vivek_kraken_2021} keep frequently invoked containers ready, balancing readiness and cost efficiency under predictable workloads but incurring overhead when demand is erratic~\cite{romero_faa_2021, vivek_kraken_2021}. While these methods perform well in resource-rich cloud environments, their resource intensity challenges applicability in edge settings.

\subsubsection{Edge FaaS Perspective}

In edge environments, cold start mitigation emphasizes lightweight designs, resource sharing, and hybrid task distribution. Lightweight execution environments like unikernels~\cite{edward_sock_2018} and \textbf{Firecracker}~\cite{agache_firecracker_2020}, as used by \textbf{TinyFaaS}~\cite{pfandzelter_tinyfaas_2020}, minimize resource usage and initialization delays but require careful orchestration to avoid resource contention. Function co-location, demonstrated by \textbf{Photons}~\cite{v_dukic_photons_2020}, reduces redundant initializations by sharing runtime resources among related functions, though this complicates isolation in multi-tenant setups~\cite{v_dukic_photons_2020}. Hybrid offloading frameworks like \textbf{GeoFaaS}~\cite{malekabbasi_geofaas_2024} balance edge-cloud workloads by offloading latency-tolerant tasks to the cloud and reserving edge resources for real-time operations, requiring reliable connectivity and efficient task management. These edge-specific strategies address cold starts effectively but introduce challenges in scalability and orchestration.

\subsection{Predictive Scaling and Caching Techniques}

Efficient resource allocation is vital for maintaining low latency and high availability in serverless platforms. Predictive scaling and caching techniques dynamically provision resources and reduce cold start latency by leveraging workload prediction and state retention.
Traditional FaaS platforms use predictive scaling and caching to optimize resources, employing techniques (OFC, FaasCache) to reduce cold starts. However, these methods rely on centralized orchestration and workload predictability, limiting their effectiveness in dynamic, resource-constrained edge environments.



\subsubsection{Edge FaaS Perspective}

Edge FaaS platforms adapt predictive scaling and caching techniques to constrain resources and heterogeneous environments. \textbf{EDGE-Cache}~\cite{kim_delay-aware_2022} uses traffic profiling to selectively retain high-priority functions, reducing memory overhead while maintaining readiness for frequent requests. Hybrid frameworks like \textbf{GeoFaaS}~\cite{malekabbasi_geofaas_2024} implement distributed caching to balance resources between edge and cloud nodes, enabling low-latency processing for critical tasks while offloading less critical workloads. Machine learning methods, such as clustering-based workload predictors~\cite{gao_machine_2020} and GRU-based models~\cite{guo_applying_2018}, enhance resource provisioning in edge systems by efficiently forecasting workload spikes. These innovations effectively address cold start challenges in edge environments, though their dependency on accurate predictions and robust orchestration poses scalability challenges.

\subsection{Decentralized Orchestration, Function Placement, and Scheduling}

Efficient orchestration in serverless platforms involves workload distribution, resource optimization, and performance assurance. While traditional FaaS platforms rely on centralized control, edge environments require decentralized and adaptive strategies to address unique challenges such as resource constraints and heterogeneous hardware.



\subsubsection{Edge FaaS Perspective}

Edge FaaS platforms adopt decentralized and adaptive orchestration frameworks to meet the demands of resource-constrained environments. Systems like \textbf{Wukong} distribute scheduling across edge nodes, enhancing data locality and scalability while reducing network latency. Lightweight frameworks such as \textbf{OpenWhisk Lite}~\cite{kravchenko_kpavelopenwhisk-light_2024} optimize resource allocation by decentralizing scheduling policies, minimizing cold starts and latency in edge setups~\cite{benjamin_wukong_2020}. Hybrid solutions like \textbf{OpenFaaS}~\cite{noauthor_openfaasfaas_2024} and \textbf{EdgeMatrix}~\cite{shen_edgematrix_2023} combine edge-cloud orchestration to balance resource utilization, retaining latency-sensitive functions at the edge while offloading non-critical workloads to the cloud. While these approaches improve flexibility, they face challenges in maintaining coordination and ensuring consistent performance across distributed nodes.



\section{Related Work}
The landscape of large language model vulnerabilities has been extensively studied in recent literature \cite{crothers2023machinegeneratedtextcomprehensive,shayegani2023surveyvulnerabilitieslargelanguage,Yao_2024,Huang2023ASO}, that propose detailed taxonomies of threats. These works categorize LLM attacks into distinct types, such as adversarial attacks, data poisoning, and specific vulnerabilities related to prompt engineering. Among these, prompt injection attacks have emerged as a significant and distinct category, underscoring their relevance to LLM security.

The following high-level overview of the collected taxonomy of LLM vulnerabilities is defined in \cite{Yao_2024}:
\begin{itemize}
    \item Adversarial Attacks: Data Poisoning, Backdoor Attacks
    \item Inference Attacks: Attribute Inference, Membership Inferences
    \item Extraction Attacks
    \item Bias and Unfairness
Exploitation
    \item Instruction Tuning Attacks: Jailbreaking, Prompt Injection.
\end{itemize}
Prompt injection attacks are further classified in \cite{shayegani2023surveyvulnerabilitieslargelanguage} into the following: Goal hijacking and \textbf{Prompt leakage}.

The reviewed taxonomies underscore the need for comprehensive frameworks to evaluate LLM security. The agentic approach introduced in this paper builds on these insights, automating adversarial testing to address a wide range of scenarios, including those involving prompt leakage and role-specific vulnerabilities.

\subsection{Prompt Injection and Prompt Leakage}

Prompt injection attacks exploit the blending of instructional and data inputs, manipulating LLMs into deviating from their intended behavior. Prompt injection attacks encompass techniques that override initial instructions, expose private prompts, or generate malicious outputs \cite{Huang2023ASO}. A subset of these attacks, known as prompt leakage, aims specifically at extracting sensitive system prompts embedded within LLM configurations. In \cite{shayegani2023surveyvulnerabilitieslargelanguage}, authors differentiate between prompt leakage and related methods such as goal hijacking, further refining the taxonomy of LLM-specific vulnerabilities.

\subsection{Defense Mechanisms}

Various defense mechanisms have been proposed to address LLM vulnerabilities, particularly prompt injection and leakage \cite{shayegani2023surveyvulnerabilitieslargelanguage,Yao_2024}. We focused on cost-effective methods like instruction postprocessing and prompt engineering, which are viable for proprietary models that cannot be retrained. Instruction preprocessing sanitizes inputs, while postprocessing removes harmful outputs, forming a dual-layer defense. Preprocessing methods include perplexity-based filtering \cite{Jain2023BaselineDF,Xu2022ExploringTU} and token-level analysis \cite{Kumar2023CertifyingLS}. Postprocessing employs another set of techniques, such as censorship by LLMs \cite{Helbling2023LLMSD,Inan2023LlamaGL}, and use of canary tokens and pattern matching \cite{vigil-llm,rebuff}, although their fundamental limitations are noted \cite{Glukhov2023LLMCA}. Prompt engineering employs carefully designed instructions \cite{Schulhoff2024ThePR} and advanced techniques like spotlighting \cite{Hines2024DefendingAI} to mitigate vulnerabilities, though no method is foolproof \cite{schulhoff-etal-2023-ignore}. Adversarial training, by incorporating adversarial examples into the training process, strengthens models against attacks \cite{Bespalov2024TowardsBA,Shaham2015UnderstandingAT}.

\subsection{Security Testing for Prompt Injection Attacks}

Manual testing, such as red teaming \cite{ganguli2022redteaminglanguagemodels} and handcrafted "Ignore Previous Prompt" attacks \cite{Perez2022IgnorePP}, highlights vulnerabilities but is limited in scale. Automated approaches like PAIR \cite{chao2024jailbreakingblackboxlarge} and GPTFUZZER \cite{Yu2023GPTFUZZERRT} achieve higher success rates by refining prompts iteratively or via automated fuzzing. Red teaming with LLMs \cite{Perez2022RedTL} and reinforcement learning \cite{anonymous2024diverse} uncovers diverse vulnerabilities, including data leakage and offensive outputs. Indirect Prompt Injection (IPI) manipulates external data to compromise applications \cite{Greshake2023NotWY}, adapting techniques like SQL injection to LLMs \cite{Liu2023PromptIA}. Prompt secrecy remains fragile, with studies showing reliable prompt extraction \cite{Zhang2023EffectivePE}. Advanced frameworks like Token Space Projection \cite{Maus2023AdversarialPF} and Weak-to-Strong Jailbreaking Attacks \cite{zhao2024weaktostrongjailbreakinglargelanguage} exploit token-space relationships, achieving high success rates for prompt extraction and jailbreaking.

\subsection{Agentic Frameworks for Evaluating LLM Security}

The development of multi-agent systems leveraging large language models (LLMs) has shown promising results in enhancing task-solving capabilities \cite{Hong2023MetaGPTMP, Wang2023UnleashingTE, Talebirad2023MultiAgentCH, Wu2023AutoGenEN, Du2023ImprovingFA}. A key aspect across various frameworks is the specialization of roles among agents \cite{Hong2023MetaGPTMP, Wu2023AutoGenEN}, which mimics human collaboration and improves task decomposition.

Agentic frameworks and the multi-agent debate approach benefit from agent interaction, where agents engage in conversations or debates to refine outputs and correct errors \cite{Wu2023AutoGenEN}. For example, debate systems improve factual accuracy and reasoning by iteratively refining responses through collaborative reasoning \cite{Du2023ImprovingFA}, while AG2 allows agents to autonomously interact and execute tasks with minimal human input.

These frameworks highlight the viability of agentic systems, showing how specialized roles and collaborative mechanisms lead to improved performance, whether in factuality, reasoning, or task execution. By leveraging the strengths of diverse agents, these systems demonstrate a scalable approach to problem-solving.

Recent research on testing LLMs using other LLMs has shown that this approach can be highly effective \cite{chao2024jailbreakingblackboxlarge, Yu2023GPTFUZZERRT, Perez2022RedTL}. Although the papers do not explicitly employ agentic frameworks they inherently reflect a pattern similar to that of an "attacker" and a "judge". \cite{chao2024jailbreakingblackboxlarge}  This pattern became a focal point for our work, where we put the judge into a more direct dialogue, enabling it to generate attacks based on the tested agent response in an active conversation.

A particularly influential paper in shaping our approach is Jailbreaking Black Box Large Language Models in Twenty Queries \cite{chao2024jailbreakingblackboxlarge}. This paper not only introduced the attacker/judge architecture but also provided the initial system prompts used for a judge.

\section{Research Methodology}~\label{sec:Methodology}

In this section, we discuss the process of conducting our systematic review, e.g., our search strategy for data extraction of relevant studies, based on the guidelines of Kitchenham et al.~\cite{kitchenham2022segress} to conduct SLRs and Petersen et al.~\cite{PETERSEN20151} to conduct systematic mapping studies (SMSs) in Software Engineering. In this systematic review, we divide our work into a four-stage procedure, including planning, conducting, building a taxonomy, and reporting the review, illustrated in Fig.~\ref{fig:search}. The four stages are as follows: (1) the \emph{planning} stage involved identifying research questions (RQs) and specifying the detailed research plan for the study; (2) the \emph{conducting} stage involved analyzing and synthesizing the existing primary studies to answer the research questions; (3) the \emph{taxonomy} stage was introduced to optimize the data extraction results and consolidate a taxonomy schema for REDAST methodology; (4) the \emph{reporting} stage involved the reviewing, concluding and reporting the final result of our study.

\begin{figure}[!t]
    \centering
    \includegraphics[width=1\linewidth]{fig/methodology/searching-process.drawio.pdf}
    \caption{Systematic Literature Review Process}
    \label{fig:search}
\end{figure}

\subsection{Research Questions}
In this study, we developed five research questions (RQs) to identify the input and output, analyze technologies, evaluate metrics, identify challenges, and identify potential opportunities. 

\textbf{RQ1. What are the input configurations, formats, and notations used in the requirements in requirements-driven
automated software testing?} In requirements-driven testing, the input is some form of requirements specification -- which can vary significantly. RQ1 maps the input for REDAST and reports on the comparison among different formats for requirements specification.

\textbf{RQ2. What are the frameworks, tools, processing methods, and transformation techniques used in requirements-driven automated software testing studies?} RQ2 explores the technical solutions from requirements to generated artifacts, e.g., rule-based transformation applying natural language processing (NLP) pipelines and deep learning (DL) techniques, where we additionally discuss the potential intermediate representation and additional input for the transformation process.

\textbf{RQ3. What are the test formats and coverage criteria used in the requirements-driven automated software
testing process?} RQ3 focuses on identifying the formulation of generated artifacts (i.e., the final output). We map the adopted test formats and analyze their characteristics in the REDAST process.

\textbf{RQ4. How do existing studies evaluate the generated test artifacts in the requirements-driven automated software testing process?} RQ4 identifies the evaluation datasets, metrics, and case study methodologies in the selected papers. This aims to understand how researchers assess the effectiveness, accuracy, and practical applicability of the generated test artifacts.

\textbf{RQ5. What are the limitations and challenges of existing requirements-driven automated software testing methods in the current era?} RQ5 addresses the limitations and challenges of existing studies while exploring future directions in the current era of technology development. %It particularly highlights the potential benefits of advanced LLMs and examines their capacity to meet the high expectations placed on these cutting-edge language modeling technologies. %\textcolor{blue}{CA: Do we really need to focus on LLMs? TBD.} \textcolor{orange}{FW: About LLMs, I removed the direct emphase in RQ5 but kept the discussion in RQ5 and the solution section. I think that would be more appropriate.}

\subsection{Searching Strategy}

The overview of the search process is exhibited in Fig. \ref{fig:papers}, which includes all the details of our search steps.
\begin{table}[!ht]
\caption{List of Search Terms}
\label{table:search_term}
\begin{tabularx}{\textwidth}{lX}
\hline
\textbf{Terms Group} & \textbf{Terms} \\ \hline
Test Group & test* \\
Requirement Group & requirement* OR use case* OR user stor* OR specification* \\
Software Group & software* OR system* \\
Method Group & generat* OR deriv* OR map* OR creat* OR extract* OR design* OR priorit* OR construct* OR transform* \\ \hline
\end{tabularx}
\end{table}

\begin{figure}
    \centering
    \includegraphics[width=1\linewidth]{fig/methodology/search-papers.drawio.pdf}
    \caption{Study Search Process}
    \label{fig:papers}
\end{figure}

\subsubsection{Search String Formulation}
Our research questions (RQs) guided the identification of the main search terms. We designed our search string with generic keywords to avoid missing out on any related papers, where four groups of search terms are included, namely ``test group'', ``requirement group'', ``software group'', and ``method group''. In order to capture all the expressions of the search terms, we use wildcards to match the appendix of the word, e.g., ``test*'' can capture ``testing'', ``tests'' and so on. The search terms are listed in Table~\ref{table:search_term}, decided after iterative discussion and refinement among all the authors. As a result, we finally formed the search string as follows:


\hangindent=1.5em
 \textbf{ON ABSTRACT} ((``test*'') \textbf{AND} (``requirement*'' \textbf{OR} ``use case*'' \textbf{OR} ``user stor*'' \textbf{OR} ``specifications'') \textbf{AND} (``software*'' \textbf{OR} ``system*'') \textbf{AND} (``generat*'' \textbf{OR} ``deriv*'' \textbf{OR} ``map*'' \textbf{OR} ``creat*'' \textbf{OR} ``extract*'' \textbf{OR} ``design*'' \textbf{OR} ``priorit*'' \textbf{OR} ``construct*'' \textbf{OR} ``transform*''))

The search process was conducted in September 2024, and therefore, the search results reflect studies available up to that date. We conducted the search process on six online databases: IEEE Xplore, ACM Digital Library, Wiley, Scopus, Web of Science, and Science Direct. However, some databases were incompatible with our default search string in the following situations: (1) unsupported for searching within abstract, such as Scopus, and (2) limited search terms, such as ScienceDirect. Here, for (1) situation, we searched within the title, keyword, and abstract, and for (2) situation, we separately executed the search and removed the duplicate papers in the merging process. 

\subsubsection{Automated Searching and Duplicate Removal}
We used advanced search to execute our search string within our selected databases, following our designed selection criteria in Table \ref{table:selection}. The first search returned 27,333 papers. Specifically for the duplicate removal, we used a Python script to remove (1) overlapped search results among multiple databases and (2) conference or workshop papers, also found with the same title and authors in the other journals. After duplicate removal, we obtained 21,652 papers for further filtering.

\begin{table*}[]
\caption{Selection Criteria}
\label{table:selection}
\begin{tabularx}{\textwidth}{lX}
\hline
\textbf{Criterion ID} & \textbf{Criterion Description} \\ \hline
S01          & Papers written in English. \\
S02-1        & Papers in the subjects of "Computer Science" or "Software Engineering". \\
S02-2        & Papers published on software testing-related issues. \\
S03          & Papers published from 1991 to the present. \\ 
S04          & Papers with accessible full text. \\ \hline
\end{tabularx}
\end{table*}

\begin{table*}[]
\small
\caption{Inclusion and Exclusion Criteria}
\label{table:criteria}
\begin{tabularx}{\textwidth}{lX}
\hline
\textbf{ID}  & \textbf{Description} \\ \hline
\multicolumn{2}{l}{\textbf{Inclusion Criteria}} \\ \hline
I01 & Papers about requirements-driven automated system testing or acceptance testing generation, or studies that generate system-testing-related artifacts. \\
I02 & Peer-reviewed studies that have been used in academia with references from literature. \\ \hline
\multicolumn{2}{l}{\textbf{Exclusion Criteria}} \\ \hline
E01 & Studies that only support automated code generation, but not test-artifact generation. \\
E02 & Studies that do not use requirements-related information as an input. \\
E03 & Papers with fewer than 5 pages (1-4 pages). \\
E04 & Non-primary studies (secondary or tertiary studies). \\
E05 & Vision papers and grey literature (unpublished work), books (chapters), posters, discussions, opinions, keynotes, magazine articles, experience, and comparison papers. \\ \hline
\end{tabularx}
\end{table*}

\subsubsection{Filtering Process}

In this step, we filtered a total of 21,652 papers using the inclusion and exclusion criteria outlined in Table \ref{table:criteria}. This process was primarily carried out by the first and second authors. Our criteria are structured at different levels, facilitating a multi-step filtering process. This approach involves applying various criteria in three distinct phases. We employed a cross-verification method involving (1) the first and second authors and (2) the other authors. Initially, the filtering was conducted separately by the first and second authors. After cross-verifying their results, the results were then reviewed and discussed further by the other authors for final decision-making. We widely adopted this verification strategy within the filtering stages. During the filtering process, we managed our paper list using a BibTeX file and categorized the papers with color-coding through BibTeX management software\footnote{\url{https://bibdesk.sourceforge.io/}}, i.e., “red” for irrelevant papers, “yellow” for potentially relevant papers, and “blue” for relevant papers. This color-coding system facilitated the organization and review of papers according to their relevance.

The screening process is shown below,
\begin{itemize}
    \item \textbf{1st-round Filtering} was based on the title and abstract, using the criteria I01 and E01. At this stage, the number of papers was reduced from 21,652 to 9,071.
    \item \textbf{2nd-round Filtering}. We attempted to include requirements-related papers based on E02 on the title and abstract level, which resulted from 9,071 to 4,071 papers. We excluded all the papers that did not focus on requirements-related information as an input or only mentioned the term ``requirements'' but did not refer to the requirements specification.
    \item \textbf{3rd-round Filtering}. We selectively reviewed the content of papers identified as potentially relevant to requirements-driven automated test generation. This process resulted in 162 papers for further analysis.
\end{itemize}
Note that, especially for third-round filtering, we aimed to include as many relevant papers as possible, even borderline cases, according to our criteria. The results were then discussed iteratively among all the authors to reach a consensus.

\subsubsection{Snowballing}

Snowballing is necessary for identifying papers that may have been missed during the automated search. Following the guidelines by Wohlin~\cite{wohlin2014guidelines}, we conducted both forward and backward snowballing. As a result, we identified 24 additional papers through this process.

\subsubsection{Data Extraction}

Based on the formulated research questions (RQs), we designed 38 data extraction questions\footnote{\url{https://drive.google.com/file/d/1yjy-59Juu9L3WHaOPu-XQo-j-HHGTbx_/view?usp=sharing}} and created a Google Form to collect the required information from the relevant papers. The questions included 30 short-answer questions, six checkbox questions, and two selection questions. The data extraction was organized into five sections: (1) basic information: fundamental details such as title, author, venue, etc.; (2) open information: insights on motivation, limitations, challenges, etc.; (3) requirements: requirements format, notation, and related aspects; (4) methodology: details, including immediate representation and technique support; (5) test-related information: test format(s), coverage, and related elements. Similar to the filtering process, the first and second authors conducted the data extraction and then forwarded the results to the other authors to initiate the review meeting.

\subsubsection{Quality Assessment}

During the data extraction process, we encountered papers with insufficient information. To address this, we conducted a quality assessment in parallel to ensure the relevance of the papers to our objectives. This approach, also adopted in previous secondary studies~\cite{shamsujjoha2021developing, naveed2024model}, involved designing a set of assessment questions based on guidelines by Kitchenham et al.~\cite{kitchenham2022segress}. The quality assessment questions in our study are shown below:
\begin{itemize}
    \item \textbf{QA1}. Does this study clearly state \emph{how} requirements drive automated test generation?
    \item \textbf{QA2}. Does this study clearly state the \emph{aim} of REDAST?
    \item \textbf{QA3}. Does this study enable \emph{automation} in test generation?
    \item \textbf{QA4}. Does this study demonstrate the usability of the method from the perspective of methodology explanation, discussion, case examples, and experiments?
\end{itemize}
QA4 originates from an open perspective in the review process, where we focused on evaluation, discussion, and explanation. Our review also examined the study’s overall structure, including the methodology description, case studies, experiments, and analyses. The detailed results of the quality assessment are provided in the Appendix. Following this assessment, the final data extraction was based on 156 papers.

% \begin{table}[]
% \begin{tabular}{ll}
% \hline
% QA ID & QA Questions                                             \\ \hline
% Q01   & Does this study clearly state its aims?                  \\
% Q02   & Does this study clearly describe its methodology?        \\
% Q03   & Does this study involve automated test generation?       \\
% Q04   & Does this study include a promising evaluation?          \\
% Q05   & Does this study demonstrate the usability of the method? \\ \hline
% \end{tabular}%
% \caption{Questions for Quality Assessment}
% \label{table:qa}
% \end{table}

% automated quality assessment

% \textcolor{blue}{CA: Our search strategy focused on identifying requirements types first. We covered several sources, e.g., ~\cite{Pohl:11,wagner2019status} to identify different formats and notations of specifying requirements. However, this came out to be a long list, e.g., free-form NL requirements, semi-formal UML models, free-from textual use case models, UML class diagrams, UML activity diagrams, and so on. In this paper, we attempted to primarily focus on requirements-related aspects and not design-level information. Hence, we generalised our search string to include generic keywords, e.g., requirement*, use case*, and user stor*. We did so to avoid missing out on any papers, bringing too restrictive in our search strategy, and not creating a too-generic search string with all the aforementioned formats to avoid getting results beyond our review's scope.}


%% Use \subsection commands to start a subsection.



%\subsection{Study Selection}

% In this step, we further looked into the content of searched papers using our search strategy and applied our inclusion and exclusion criteria. Our filtering strategy aimed to pinpoint studies focused on requirements-driven system-level testing. Recognizing the presence of irrelevant papers in our search results, we established detailed selection criteria for preliminary inclusion and exclusion, as shown in Table \ref{table: criteria}. Specifically, we further developed the taxonomy schema to exclude two types of studies that did not meet the requirements for system-level testing: (1) studies supporting specification-driven test generation, such as UML-driven test generation, rather than requirements-driven testing, and (2) studies focusing on code-based test generation, such as requirement-driven code generation for unit testing.





\section{NINE REASONS TO BE CAUTIOUS WITH BENCHMARKS} 

In the sections below, we summarise the main issues that were identified during the course of our research. Importantly, these issues - presented here as a taxonomy - are not arranged according to their importance or urgency. They are also not meant to be understood as isolated issues, but rather deeply interlinked problems. Indeed, this complexity and interdependence is precisely what makes AI evaluations challenging.

% Begin by mentioning a couple of overarching remarks and insights from our study as a whole, for instance regarding a "first wave" of benchmark critique with strong focus on datasets and transparency around 2021-2022, and then a "second wave" focusing on technical demonstrations of benchmark shortcomings in 2023-2024. 

% Also maybe say something over-arching about how people have mainly approached benchmark critique from different fields (computer science, cybersecurity, anthropology etc.). 

\subsection{Problems with Data Collection, Annotation, and Documentation}

An initial set of issues with AI benchmarks found during our research is limitations in the collection, annotation, and documentation of benchmark datasets. This ties into a broader critique regarding insufficient documentation in AI research which is central to calls for more transparent and trustworthy algorithmic systems~\cite{gebru2021, mitchell2019, orr2024b, simson2024, scheuerman2021}. Research has found that it is often difficult to trace precisely \textit{how}, \textit{when}, and by \textit{whom} benchmark datasets have been made \cite{reuel_betterbench_2024, denton2020} which compromises the ability for benchmarks to be used in robust and generalisable ways \cite{arzt2024}. The issue has partly been linked to the low status of dataset-related work within the machine learning community (which instead privileges model development) \cite{orr2024, sambasivan2021}, and the fact that AI datasets are often “reduced, reused, and recycled” \cite{koch2021}, which complicates documentations of their possible limitations \cite{thylstrup2022, park2022}. Notably, \citet{koch2021} have found that more than 70 percent of the benchmark datasets used in prominent computer vision papers had been reused from other domains.
\citet{park2022} also find that while benchmark datasets often contain ample information on how to use the dataset, documentation is often missing concerning their shortcomings and social impact. 
In their exploration of benchmark sharing platforms like HuggingFace\footnote{\url{https://huggingface.co/}. Last seen \today.} and PapersWithCode\footnote{\url{https://paperswithcode.com/}. Last seen \today.} the same researchers also note that confusing metadata terminology severely complicates efforts to understand benchmark dataset documentation. 

Like AI training datasets, benchmarks have been singled out for raising ethical and legal questions concerning copyrights, privacy, informed consent, and rights to opt-out \cite{paullada2021}. Many benchmarks rely on crowd-sourced or user-generated content from platforms like Wikihow\footnote{\url{https://www.wikihow.com/Main-Page}. Last seen \today.}, Reddit\footnote{\url{https://www.reddit.com/}. Last seen \today.}, or trivia websites \cite{keegan2024, grill2024}, whose annotation may be noisy and biased, lack input from expertise in specialised fields, or be produced under exploitative conditions \cite{tsipras2020, aroyo2015, sen2015}. Previous research has noted that this matters when baselines for "good", "bad", and "safe enough" AI models are calibrated in sensitive contexts \cite{rauh2024, grill2024}. Moreover, it is essential to highlight the absence of human performance references and difficulty rubrics in benchmarks, which are increasingly important factors in evaluating capabilities and generality \cite{chollet2019}.

A lack of care in the making of benchmark datasets has further been found to result in AI models exploiting unknown quirks and spurious cues in the training data, rather than solving their original intended task \cite{liao2021, paullada2021, geirhos_shortcut_2020}.%a problem that overlaps with the notorious problem of \textit{data leakage} \cite{kaufman_leakage_2012}.
For instance, \citet{oakden-rayner2019} found that an X-ray image classification model predicted collapsed lungs with high accuracy, although it turned out the model only identified the presence of a chest drain (used to cure the condition) that was (unknowingly) present in a majority of the positive training images. In other words, the model performed well in benchmark tests, but the reason why completely sidestepped the original purpose of the task (identifying collapsed lungs). When the all images containing chest drains were removed from the training dataset, the model performance dropped by over 20\%.
Similar patterns have also been confirmed in recent research on LLM evaluation \cite{pacchiardi2024} and is sometimes discussed as failure in the "world model" of an AI system \cite{vafa2024}. % ME: cut out for reasons of space: For instance, Pacchiardi et al. trained simple logistic regression models on 1- and 2-grams extracted from the prompts of well-known multiple choice question benchmarks, and found that i) such simple models yield comparable accuracy to LLMs, and ii) LLM performance is correlated to the performance of their simpler models, suggesting that LLMs may rely on the same cues \cite{pacchiardi2024}. 
While it could be argued that picking up on spurious cues is a sign of intelligence and therefore a capability worth noting, the issues above highlight how a lack of in-depth attention to the data that benchmarks rely on - alongside a failure to acknowledge how little is actually known about \textit{how} and \textit{why} AI models perform well in benchmark tests - can produce frail and uncertain AI evaluations.


% One benchmark that has received especially harsh critique for poor data quality is the Massive Multitask Language Understanding (MMLU) benchmark, which is one of the world's mose widely used frameworks for evaluating the capacities of language models \cite{gema2024, reuel_betterbench_2024}. MMLU consists of multiple choice questions organized into 57 subject categories that range from "abstract algebra" to "world religion". The MMLU dataset was "manually collected by graduate and undergraduate students from freely available sources online" (unclear precisely when, by whom, and according to which instructions) and has been found to include questions that are morally problematic, inaccurate, outdated, and unanswerable \cite{erenrich2023, keegan2024, gema2024}. For instance, identified errors in MMLU include subjects being described in highly Western-centric ways, a lack of specificity regarding juristictions in law-related questions, and questions being parsed in the wrong way \cite{gema2024}. Reuel et al. also found that MMLU scored lowest in their in-depth quality assessment of 24 state-of-the-art language model benchmarks, despite the fact that recent models like GPT-4, Claude-3, and Gemini were all released and tested against it \cite{reuel_betterbench_2024}. 

% Recently, attempts have been made to improve MMLU by releasing updated versions such as MMLU-Redux \cite{gema2024} and MMLU-Pro \cite{wang2024}. MMLU-Pro complements the original MMLU benchmark with questions from a new set of sources (STEM, SciBench, and TheoremQA). All questions in the dataset have also been reviewed by human experts, although it is still unclear precisely by whom, when, where, and how. It is also noteworthy that MMLU-Pro exclusively relies on original questions from MMLU in areas concerning law, health, history, philosophy and "other." Interestingly, the language models GPT4-Turbo and Gemini-1.5-Pro were also continuously used during the data processing for MMLU-Pro, for instance to extract short answers from solutions, convert problems into multiple-choice questions, and perform "quality control" and "option augmenation" (whatever that means). The decision to involve one AI model in the evaluation of another is indicative of a broader trend where AI technologies are increasingly integrated into AI benchmarking pipelines - an issue which raises concerns about possible issues with overfitting, alongside new and additional problems regarding transparency, explainability, safety, and trustworthiness.


\subsection{Weak Construct Validity and Epistemological Claims}

Another genre of benchmark critique focuses on the epistemological claims that tend to surround benchmarks and examines the limits of what can be known through quantitative AI tests. A central reference point in these discussions is the observation by \citet{raji2021} that many benchmarks suffer from construct validity issues in the sense that they do not measure what they claim to measure. As the authors proclaim, this is especially troublesome when benchmarks promise to measure universal or general capabilities, since this vastly misrepresents their actual capability. As a result, the authors argue that framing an AI benchmark dataset as general purpose "is ultimately dangerous and deceptive, resulting in misguidance on task design and focus, underreporting of the many biases and subjective interpretations inherent in the data as well as enabling, through false presentations of performance, potential model misuse".~\cite[p.~5]{raji2021}. At the heart of this critique lies the realization that many benchmarks do not have a clear definition of what they claim to measure, which makes it impossible to measure if they succeed in the task or not \cite{blodgett2021, bartz-beielstein2020}. In a close analysis of four benchmarks used to evaluate fairness in natural language processing (StereoSet, CrowS-Pairs, WinoBias, and WinoGender), \citet{blodgett2021} for example found that all benchmarks revealed severe weaknesses in terms of defining what is being measured. For instance, culturally complex and highly contested concepts like "stereotypes" or "offensive language" were left unspecified, causing a series of logical failures and interpretational conflicts. Elsewhere, research has shown strong disagreements in how benchmark tasks are \textit{conceptualised} and \textit{operationalised }\cite{subramonian2023}, and found that benchmarks are applied in highly idiosyncratic ways \cite{rottger2024}. Frequently, the difficulty defining what benchmarks evaluate persist since there is no clear, stable and absolute ground truth for what is claimed to be measured \cite{narayanan2023}. Since concepts like "bias" and "fairness" are inherently contested, messy, and shifting, benchmarks that promise to measure such terms will inevitably suffer from an "abstraction error" that produces a false sense of certainty \cite[p.~63]{selbst2019}. 

It has also been pointed out that many benchmarks datasets are inadequate and/or unuseful proxys for what they are meant to evaluate. For instance, researchers have identified a slippage in distinguishing between algorithmic "harms" and algorithmic "wrongs" when evaluating the capabilities of AI models - two not necessarily overlapping concepts \cite{diberardino2024}. Others have questioned whether the content of benchmark datasets are reasonable substitutes for the "real world" scenarios they are meant to reflect. For instance, researchers have 
questioned the decision to use examples generated by Amazon crowdworkers and posts from the Reddit forum "\textit{Am I the asshole?}" as proxies for ethics and morals in benchmarks such as HellaSwag - a widely cited benchmark for language models \cite{keegan2024}. In benchmarks consisting of professional exams, researchers have further argued that such tests "emphasize the wrong thing" and "overemphasize precisely the thing that language models are good at" and are thus unreliable measures of things such as medical or legal skill \cite[n.p]{narayanan2023}.
As \citet{narayanan2023} put it, % professional exams are very different from "the real-life problems that professionals face
"it's not like a lawyer's job is to answer bar exam questions all day". 
A recent study by \citet{ren2024} also found that many widely used safety benchmarks (including ETHICS, TruthfulQA, GPQA, QuALITY, MT-Bench, LMSYS Chatbot ARENA, ANLI, AdvGLUE, and AdvGLUE++) 
highly correlate with general and upstream model capabilities, raising concerns regarding “safetywashing” as applying them could imply that “capability improvements are misrepresented as safety advancements". While it could be argued that capability and safety are largely entwined (the more capable a model is, the higher the likelihood it could cause harm), \citet[p.~1]{ren2024} suggest that a blurred distinction between the two may hide the fact that severe biases and safety issues in AI models can persist, even as their overall capabilities improve. 

\an{With respect to issues regarding construct validity, and in particular what is discussed in the last paragraph, I feel like there is still a fundamental question that is not addressed in our text: While capability and safety are indeed connected, the thing that I feel is missing is that from a safety perspective there is on one hand the capability of the model, and on the other the capability of the agent/person/entity/adversary who is given the model. Certain entities have a higher capbility of bringing about undesired behaviour from a given model versus others, and this third element (which in typical security/safety literature and is generally referred to as the adversary) is another important epistemological element that is missing from most benchmarking. I haven't had the opportunity to study all the core literature but if there are discussion of a lack of attention to the "adversary" and assumptions around their capabilities that underpin certain benchmarks, I think we should highlight them here. If not, I highly suggest adding a final catch-ll-other-issues subsection in which we can touch upon some of the issues that we know exists. This would also give us a nice logic for discussing some of the literature that was left out and also connect back to and justify our own claim or drawing from and adding value from our  interdisciplinarity. Alternatively we could also do this in the discussion/conclusions as well but I always find it odd to add citations and references that haven't been discussed earlier in the text to a discussion/conclusions section, so ideally we could do it earlier.}
\me{I am not 100 percent sure if this is what you are referring to here, but the problem that humans are generally left out of benchmark tests (which instead build on highly abstracted and utterly unrealistic scenarios) is somewhat discussed in the next chapter on sociotechnical gap... I think}


\subsection{Sociocultural Context and Gap}

Another key insight from previous research concerns the importance of the social, economic and cultural contexts where AI benchmarks are created, used, and maintained. Among researchers engaging in benchmark critique, we identify a strong consensus regarding the need to recognise that benchmarks % do not appear out of a cultural and political vacuum, but are rather intimate products of social and economic interests and circumstances. As \citet[p.~1877]{orr2024a} put it, benchmarks 
are ultimately "normative instruments that perpetuate particular epistemological perspectives about how the world is ordered" \cite[p.~1877]{orr2024a}. Qualitative research has also examined the cultural and social environments where benchmarks are made, finding that they are deeply shaped by shared and arbitrary assumptions, commitments, and dependencies \cite{engdahl2024,michael2022, scheuerman2021,orr2024,sambasivan2021,paullada2021}.
Such assumptions for example include valuing "efficiency at the expense of care; universality at the expense of contextuality; impartiality at the expense of positionality;
and model work at the expense of data work" \cite{scheuerman2021}, or reproduce contested ideas such as the notion that low-quality data can be drowned out by scale \cite{orr2024}.
Recent research has also illustrated that AI safety research and benchmark competitions are increasingly informed by political movements and ideologies such as longtermism and effective altruism \cite{ahmed2024} % Need to shorten: and that datasets that are described "unscripted" or consisting of data captured "in the wild", in fact rely on complex amounts of practical work and decision-making processes \cite{engdahl2024}. 

Scholars have further identified sociotechnical gaps and a lack of consideration for downstream utility as a concern in AI benchmarking \cite{hutchinson_evaluation_2022}.
\citet{liao2023} highlight that this means it is often unclear who is meant to care about benchmark evaluation results and how they should be used in practice.
For instance, a recent study by \citet{blagec2023}, which compared the explicitly stated needs for AI technologies among clinical medical practitioners with existing clinical benchmark datasets, found that most benchmarks failed to answer to the needs of medical experts.
They also found that benchmarks for the most urgently requested medical or clinical tasks were completely missing and noted that similar misalignments likely exist in many other fields. For example, \citet[p.~79]{jannach2020} highlight a lack of attention to how recommender systems "may create value; how they, positively or negatively, impact consumers, businesses, and the society; and how we can measure the resulting effects". \citet{ethayarajh2021} further argue that failures to consider the practical utility of benchmarks has for example made it possible to ignore the discriminatory and environmental damages of AI technologies and allowed for highly energy-inefficient and deeply biased AI models to reach the top of most benchmark leaderboards. % The authors highlight how a lack of discussions concerning what values benchmarks prioritise, constitutes a key limitation in current benchmark and leaderboard practices. 

% ME: cut out in the interest of space: Relatedly, Hutchinson et.al. find that the consequential validity of benchmarks, i.e., "the real-world consequences of an evaluation's interpretation and use" has been broadly disregarded, despite its high relevance for "considerations of accountability and governance of ML models in applications" \cite{hutchinson2022} p.~1860-186. This highlights how many benchmarks lack a practical utility for policy enforcement efforts geared towards holding AI developers accountable and safeguarding against systemic risks.

% Added some discussion on incentives to the conclusions so considering the comment below addressed.
% \an{I think that there is also room for making parallels here with discussions around the incentives of the different actors that are invovled here, whether through the lense of economic incentives or other typical governance of technology lense, so if not here directly, as with my previous suggestions perhaps in the discussion/conclusions? This would also help with more firmly grounding the paper on interdisciplinarity.}

\subsection{Narrow Benchmark Diversity and Scope}

Previous research has further found that current benchmarking practices suffer from diversity issues, a problem that is also found within the broader AI ecosystem  \cite{gomez2024}. On the one hand, a vast majority of benchmarks focus on text, while other modalities (audio, images, video, and multimodal systems) remain largely unexamined \cite{rauh2024, weidinger_sociotechnical_2023, rottger2024}. This concentration is problematic since AI models are increasingly multimodal in scope. \citet{guldimann2024} also find that benchmarks addressing user privacy, copyright infringement, and interpretability are currently lacking and incomprehensive, while benchmarks in safety areas dealing with corrigibility and explainability are practically missing all together. This leads the authors to conclude that current practices for evaluating topics like safety and ethics “are often simplistic and brittle, leading to inconclusive results” \cite[p.~3]{guldimann2024}.
\citet{koch2021} further identify a concentration on fewer and fewer benchmark datasets within most task communities and note that dominant benchmarks have been introduced by researchers at just a handful of elite institutions, raising questions about representation diversity in the design of benchmarks. Scholars have also found that current AI safety evaluation practices almost exclusively deal with English content \cite{mcintosh2024, rottger2024}, and are frequently based on datasets where minorities are under-represented, despite efforts to diversify them \cite{simson2024}. This raises concerns regarding the inclusion of multiple perspectives on complex topics like ethics and harm. 

Aside from mainly focusing on a small range of tasks (evaluating capabilities of English-speaking language models), previous research has identified that most benchmarks tend to be abstracted out of their social and cultural context \cite{selbst2019}, and rely on a static, one-time testing logic, where results from single evaluations are taken to represent model capabilities writ large. This has given rise to calls for more multi-layered \cite{weidinger_sociotechnical_2023}, longitudinal \cite{mizrahi2024}, and holistic evaluation methods that make sure that AI models do not just perform well in controlled environments, but also in critical/real world circumstances over time \cite{ojewale2024, mcintosh2024, chang2023}. In their survey of LLM benchmark inadequacies, McIntosh et al. note that "current benchmarks generally adopt task-based formats, such as Multiple Choice Questions (MCQs) and dialogue-based evaluations, which tend to be static and do not capture the evolving nature of human-AI interactions" \cite[p.~6] {mcintosh2024}. %In reality, however, "LLM usage often involves continuous dialogues, yet many benchmarks assess only the first attempt of an LLM response, without considering the consistency and coherence of answers across multiple interactions.  focus on isolated responses reduces the relevance of these benchmarks in evaluating models designed for dynamic, ongoing interactions" \cite{mcintosh2024} p.~6. 
\citet{reuel_betterbench_2024} further note that failures to re-run evaluations multiple times (using different random seeds and sampling temperatures for example), implies that very little is known about the intra-model variance of benchmarks. As a result, they conclude that it is possible that "most benchmarks fail to distinguish signal and noise" \cite[p.~9]{reuel_betterbench_2024}. 
Others have emphasised that AI audits often fail to consider risks associated with multiple (inter)acting AI systems \cite{birhane2024}, and rarely take human actions and motivations into consideration \cite{weidinger_sociotechnical_2023, chang2023, rauh2024}. % and do not, once more, evaluate energy-use and environmental impact \cite{hutchinson_evaluation_2022}.

While most benchmarks are designed to tell us something about a model's success, research has further pointed out that they often reveal little (or nothing) about their particular ways of \textit{making mistakes}, which is crucial from an AI safety and policy enforcement perspective. As \citet{gehrmann2023} put it, "ranking models according to a single quality number is easy and actionable - we simply pick the model at the top of the list - [yet] it is much more important to understand when and why models fail" \cite[p.~130]{gehrmann2023}.
For instance, they suggest that a focus on errors and fragilities (as opposed to instances of success) can be useful for developers of smaller models, since "work on quantifying shortcomings is equally applicable to smaller models and methods that improve model robustness often work on all model sizes" \cite[p.~131]{gehrmann2023}.
In this sense, failure-focused benchmarks could play an important role in equalling out the playing field in AI development.


\subsection{Economic, Competitive, and Commercial Roots}

Another contextual element that has been singled out as important is the competitive and commercial roots of benchmark tests. % AI providers often make grandiose claims about their model's capabilities based on performance metrics from benchmarks (e.g., “Our Largest and Most Capable AI Model” \cite{grill2024}). This trend extends to the realm of geopolitics, where benchmark performance is used as part of the global race for AI leadership \cite{globalAI2024, Zhijia2024}. 
Previous research has emphasised that capability-oriented benchmarks are deeply embedded in corporate marketing strategies and play an important role in increasing the AI hype, attracting customers and investors, and showcasing how models outperform competitors \cite{orr2024a, grill2024, Zhijia2024}. As \citet[p.~1881]{orr2024a} put it, benchmarks "serve as the technological spectacle through which companies such as OpenAI and Google can market their technologies". Many benchmarks also have origins from within the industry and are capability-oriented and centred around tasks with a high potential economic reward, as opposed to focusing on other goals such as ethics and safety \cite{ren2024, ethayarajh2021}.


Previous research has noted that this competitive and corporate embedding discourages thorough self-critique since there is a direct "incentive mismatch between conducting high-quality evaluations and publishing new models or modelling techniques" \cite[p.~103]{gehrmann2023} and that the field of AI development is "turning into a giant leaderboard, where publication depends on numbers and little else (such as insight and explanation)" \cite{church_survey_2019}. 
While a lack of incentives to disclose weaknesses and limitations is a general problem in science \cite{smith2022}, it has been pointed out that benchmarks have played an especially central role in naturalizing and solidifying a competitive culture in AI research, which is increasingly approached as a "sport" \cite{orr2024a}. As of late, benchmark evaluations have also become increasingly professionalized and transformed into an industry in itself with the rise of platforms like Kaggle and Grand Challenge, who provide organisational and infrastructural support to AI competitions and increasingly function as infrastructures of power in fields like medical imaging \cite{luitse2024}. The issue of optimising for high benchmark scores at the expense of insight and explanation is known as a form of SOTA-chasing \cite{koch2021} and sometimes described as the "benchmark effect" \cite{Stewart2023}. Previous research \cite{maleve2023} has also described benchmarks as an example of what \citet{stengers2018} calls "fast track research" which idolises rapid, cumulative publication, thus producing a "winners curse" in AI development \cite{sculley2018}. 

Risks associated with the competitive and commercial roots of benchmarks have further been linked to the growing influence of industry in AI research, where private businesses share of the biggest AI models has increased from 11\% in 2010 to 96\% in 2021 \cite{ahmed2023}. In such a context, researchers have noted that current benchmarking tasks - which are generally highly data-intensive - are especially well suited to fit AI models that have been developed within the industry, whose access to advanced data infrastructures, computing power, valuable datasets, and skilled researchers now vastly exceed those of academic researchers \cite{ahmed2023}. 
% JV: This is a very important point, and  one that may explain many issues identified in other sections. In a way, it is  unprecedented that such a ground-breaking set of technologies is not the result of state-funded research. Maybe something to emphasize in the introduction, or discussion?
% ME: cut out for reasons of space: In 2021, for example, Ahmed Wahed and Thompson note that "non defense US government agencies allocated US\$1.5 billion on AI. In that same year, the European Commission planned to spend €1 billion (US\$1.2 billion). By contrast, globally, industry spent more than US\$340 billion on AI in 2021" (ibid., p.~884). 
% As the resources needed to perform well on benchmark tests are increasingly located in industry domains, at the same time as industry actors are playing a key role in defining what tasks and problems AI developers should strive to solve. 
This concentration of power could potentially stifle robust AI evaluations and hinder the development of AI models that adhere to other aims and goals than commercial ones. Scholars have also warned that if academic researchers continue to uphold data-intensive benchmark tests as the SOTA, there is a risk that their research will become increasingly dependent on technological infrastructures provided by the industry \cite{koch2024}. 

% Applying AI benchmarks and contributing to leaderboards implies legitimizing and reinforcing an approach to AI safety testing with deep corporate and market-oriented roots. Research in computer history and the history of science has emphasized that there is nothing natural or self-evident about this particular way of doing AI research and development (ibid.).


\subsection{Rigging, Gaming, and Measure Becoming Target}

A closely related issue concerns how benchmark tests can be tricked and gamed. In areas and modalities where best-practice benchmarks are missing (i.e., practically all modalities, except for text-based benchmark evaluations), researchers have noted that there are strong incentives to "rig" benchmark tests. For instance, \citet{dehghani2021} survey how know-how and recipes for how to score high on benchmark setups are often widely circulated online.
% As an example, they describe how researchers and developers in natural language processing "realized that pre-training on MNLI is necessary for obtaining strong performance on RTE and STS... [benchmark]) datasets" \cite[p.~13]{dehghani2021}. Accordingly, this soon became standard practice within the community. 
Recently, language models have also been found to be optimised for answering the multiple choice questions that are often part of benchmarks \cite{alzahrani2024}, and to (either intentionally or unintentionally) "fake" alignment with ethics or safety goals \cite{greenblatt2024}.
The issue point towards what is known as Goodhart's law, i.e. the recognition that "when a measure becomes a target, it ceases to be a good measure" \cite{strathern1997}.
\an{is there room for a parallel to be made here with the problematic of measurers like h-indexes, which most academics; including those in AI, should be aware of, but seem to be not be internalised when it comes to benchmarks and metrics development? and the two might even be connected in some ways. Or is it too confrontational to discuss?}
\me{if you have a source that critique h-indexes then I´d say this could absolutely be added as a brief note}

One reason why gaming can proceed is that users of benchmarks rarely provide the resources needed to validate and replicate their test results \cite{bartz-beielstein2020, dehghani2021, biderman2024, reuel_betterbench_2024}. To examine the relevance and validity of benchmark scores, researchers have emphasised that it is necessary to access information concerning all aspects of the evaluation procedure (such as the original evaluation code and all details concerning the experimental setup) \cite{biderman2024}. Providing such documentation is far from standard, however, especially when proprietary AI models are concerned. This makes it possible to tweak and cherry-pick benchmark results - a problem that is especially pressing given that subtle "variations in prompts, formatting or other implementation details can significantly impact the performance and validity of evaluations" \cite[p.~3]{biderman2024}. In their in-depth analysis of 24 SOTA language model benchmarks, \citet{reuel_betterbench_2024} found that only four provided scripts to replicate the results and that no more than ten performed multiple evaluations or reported the statistical significance of their results. 

An increasingly well discussed issue also concerns the problem of "data contamination" i.e., the risk that the models have either intentionally or unintentionally ingested benchmark datasets during training, which severely questions the integrity of AI tests \cite{xu2024, zhang2024, besen2024, magar2022, roberts2023}. The problem - which can be an instance of data leakage \cite{kaufman_leakage_2012} or train-test-overlap \cite{lewis2021} - has been known for long, and produces similar effects to those of overfitting and memorization \cite{tirumala2022,magar2022}, leading to models with low generalization power that perform well on familiar tasks (in-distribution) but fail other tasks with a similar difficulty and distribution shift (out-of-distribution) \cite{yuan2023, xu2024, zhang2024, besen2024, magar2022, roberts2023, narayanan2023}. When testing GPT4 on benchmark problems from Codeforces (a website hosting coding competitions) in 2023, for instance, \citet{narayanan2023} found that the AI model could regularly solve benchmark problems classified as easy - as long as the problems had been added before 5\textsuperscript{th} September 2021. For problems added later, GPT4 could not get a single question right, suggesting that the model had memorised questions and answers. Similar results have also been identified in multiple other models and benchmarks \cite{xu2024, zhang2024, besen2024, magar2022, roberts2023}. 
Despite the fact that issues with data leaks are so well known that strategies have been developed to avoid it\footnote{\url{https://livebench.ai}. Last seen \today.}, there is still a widespread lack of reporting of data contamination tendencies during benchmark tests. In a study from October 2024, for example, \citet{zhang2024} found that out of 30 analysed models, only 9 reported train test overlap. 

Another recently identified issue is called "sandbagging" and involves a "strategic underperformance on an evaluation" which occurs when an AI developer intentionally \textit{understates} a models capability, for instance to avoid becoming a target for AI safety regulation. In a 2024 study, for example, \citet[p.~1]{weij2024} prompted frontier models like GPT-4 and Claude 3 Opus and found that they "selectively underperform on dangerous capability evaluations, while maintaining performance on general (harmless) capability evaluations". The researchers also found that that it was possible to fine-tune and adjust both frontier and smaller models to hide specific capabilities or target specific capability scores. The issue could be generalised to popular benchmarks like the Weapons of Mass Destruction Proxy Benchmark (WMDP), and puts the trustworthiness of benchmark evaluations into question, especially in a regulatory context. 

% ME: Note that the Alzahrani piece above is published by people with connections to the Saudi Data and AI Authority and I don't have enough technical expertise to judge if their methods and results are are legit. Perhaps someone else can have a look?


\subsection{Dubious Community Vetting and Path Dependencies}

A related area of research emphasises how benchmarks become naturalised and reach standard status % not because of their technical brilliance, but 
because of the culture and logic of academic citations \cite{orr2024a}. For instance, new benchmarks are commonly introduced together with new or updated AI models. If the AI model becomes popular, the benchmark may become widely cited and circulated as a secondary effect, even though the developers of the benchmark did not intended or expect it to become standard. In this way, a benchmark that might have been selected for practical purpose or due to subjective preferences may come to have substantial technical and cultural influence as a result of community vetting, even though its suitability as a yardstick for ethics, safety, or performance may be questioned \cite{orr2024a, orr2024}. For instance, \citet{denton2021} show how this was partly the case when the ImageNet dataset - a key reference point in the performance testing of computer vision models - became standard, following  the unforeseen success of the ImageNet Large Scale Visual Recognition Challenge hosted by Toronto University in 2010. Likewise, the so-called Lena test image - also central to computer vision benchmark tests - was taken from the centerfold of a November 1972 Playboy magazine and catapulted into computer history since "someone happened to walk in with a recent issue of \textit{Playboy}" at a time of need, according to what has been described as the most credible origin story \cite[p.~80]{mulvin2021}.
Since then, the ImageNet dataset and its associated benchmark challenge has been become a symbol for dataset bias \cite{denton2021}, while the Lena test image has come to serve as a prime example of the role of whiteness and women's sexualised bodies in the standardisation of digital visual culture \cite{mulvin2021}. These examples highlight how benchmarks are fundamentally cultural and political products whose power and influence may and be (uncritically and problematically) reinforced through community vetting. 

Aside from bias and representational issues, \citet[p.~1]{schlangen_targeting_2020} argues that what is typically missing and left implicit in the peer-review-fuelled process of benchmark use is the argumentation for why scoring well on a particular benchmark "constitutes progress, and progress towards what". Instead, researchers are expected to routinely demonstrate performance on dominant benchmarks, despite the fact that more task-specific benchmarks may be more technically appropriate \cite{koch2021}. In this way, peer-washing serves to "maintain datasets as authoritative proxies even when they are shown to be harmful or problematic" \cite[p.~4966]{orr2024}. New benchmarks often have a difficult time to gain traction because of the dominance and authority of well-cited benchmarks \cite{jaton2021}, and in a study of 3765 benchmarks, \citet{ott2022} found that only a small minority of the proposed benchmark solutions reached widespread adoption. This problem is also recognised within the industry. For instance, researchers at Google's Brain Team describe what they call a "benchmark lottery" which "postulates that many factors, other than fundamental algorithmic superiority, may lead to a method being perceived as superior" \cite[p.~1]{dehghani2021}. Ironically, researchers have simultaneously found that a majority of influential benchmarks have been released as preprints without going through rigorous academic peer-review \cite[p.~6]{mcintosh2024}. In other words, problems with peer-washing both concern the poor quality control that many benchmarks are subjected to, and the self-fulfilling and sometimes excessive ways through which certain benchmarks are propelled into standards. % A 2022 survey answered by 480 NLP researchers also found that 88 percent of the respondents believed that "there is too much focus on optimizing performance on benchmarks" \cite{michael2022}. 

\citet{bao2022a} further note that most papers that introduce benchmarks have origins in the field of machine learning and are mainly focused on methods. This means that the content of benchmarks datasets are often considered secondary, since the datasets are merely there to provide a baseline for comparison between different evaluation methods. When benchmarks are applied to real and specific use-cases - such as evaluating the fairness of algorithmic risk assessment instruments within the criminal justice system - the downstream effects of such a lack of concern for datasets can have worrying effects.
As \citet{bao2022a} put it, a paper on benchmarks can be of high quality in a pure AI/ML methods sense, but irrelevant, dangerous, or harmful when applied in circumstances such as the criminal justice system, since it may introduce and perpetuate harms and mistranslations. The researchers also suggest that the current peer-review system implies that benchmarks that are primarily relevant from a methods/machine learning perspective will be cited far more often than benchmarks that are relevant for use in specific, real-life use-cases. Looking too closely at citation counts when determining the quality and relevance of benchmarks may thus effectively lead users astray and complicate identifying benchmarks with a high \textit{practical} utility. % Bao et al. further note that the conference publication workflow and community norms in computer science complicate appropriate use of benchmarks, compared to fields that rely on journal publications, since papers are often pre-published relatively quickly and supplemented over time. This means that many insufficiently curated datasets are in circulation. Researchers who may have used a benchmark dataset in a seminal paper - but later finds that it was of a poor quality - are also often still expected to cite and replicate flawed experiments during rebuttal periods. This implies that problematic benchmarks may continue to be widely cited. Furthermore, "the rapid publication cycle in which methods might build upon each other within a year or two often establishes an appearance of consensus around a benchmark, which becomes difficult to change or refute later" (ibid., 9). 

Problems with dubious community vetting become especially worrying given that benchmarks create "path dependencies" in AI research, meaning they reinforce certain methodologies and research goals, while stifling those that do not align with the logic of dominant benchmark tests \cite{blili-hamelin2023}. % When benchmarks become well established in specific AI communities, researchers have emphasised that they actively shape how communities approach progress. From a historical perspective, \citet{ensmenger2012} for instance shows how "the decision to focus on chess as the measure of both human and computer intelligence had important and unintended consequences for AI research," as it helped promote certain research agendas while neglecting others. In particular, the author suggests that the dominant use of chess as a benchmark for AI was instrumental in making deep-tree searching and the minimax algorithm the dominant focus of AI research during the mid 20th century. "In choosing chess over its various alternatives", \citet[p.~7]{ensmenger2012} writes, "AI researchers were able to tap into a long tradition of popular chess culture, with its corresponding technical and theoretical literature, international networks of enthusiasts and competitions, and well-developed protocols for documenting, sharing, and analyzing data. Yet the brute-force computational techniques that proved most suitable for winning computer chess tournaments distracted researchers from more generalizable and theoretically productive avenues of AI research... As a result, computers got much better at chess, but increasingly no one much cared".
In particular, \citet[p.~3]{koch2024} warn against the tendency for benchmarks to favour a form of task-driven scientific monoculture that privileges immediate, explicit, formal, quantitative, and easily-interpretable evaluation mechanisms that "prioritize one or a few key epistemic values (e.g., accuracy, safety)," at the expense of a broader and more complex vision of scientific progress. The authors note how a series of methodological paradigms in AI research (such as boosting, Bayesian networks, Bayesian non-parametrics, and support vector machines) have rapidly faded away following the boom in deep learning research around the year of 2014, and partly attribute this to the "epistemic narrowness" of current benchmark tests, which are well-suited for deep learning models (whose performance can be reliably increased through scaling), but less favourable for other methodological paradigms in AI. "When scientists can only gain high-status publications by demonstrating SOTA accuracy," writes \citet[p.~30]{koch2024}, "the safest research choice becomes incrementally advancing proven methods, not innovating new ones". According to Koch, the cost of current benchmark practices - which privileges scale, compute, and the use of larger and larger training datasets - for instance includes repeated privacy and copyright violations, emotional harms caused to under-paid data workers, and increasing ecological and environmental pressures due to high energy consumption. 


\subsection{Rapid AI Development and Benchmark Saturation}

Another social, economic, and cultural issue is the speed of AI developments. As the capabilities of AI models have increased manifold in the past decade, researchers have emphasised that many benchmarks are old and designed to test models far simpler than those in use today \cite{keegan2024, biderman2024}. For instance, \citet[p.~5]{biderman2024} find that many prominent LLM benchmarks (including Lambada, AI2 ARC, OBQA, Hella Swag, and WinoGrande) were "designed prior to shifts such as in-context learning and chat interaction, and therefore were not designed to take these formats and approaches into account", noting that this may affect their validity in unforeseen ways. 
Many benchmarks also struggle with the challenge of quickly being outperformed as AI models achieve 100 percent accuracy scores in tests \cite{hendrycks2021, bowman2021}. A look at current AI safety leaderboards such as HELM also reveal that many AI models score notably high on benchmarks that are widely applied today. The tendency for AI models to outperform benchmarks has been described as an issue of \textit{saturation} and implies that a benchmark no longer reflects model performance \cite{ott2022}. 

Relatedly, \citet{mcintosh2024} emphasises that many benchmark frameworks are slow and complicated to implement, meaning that evaluation processes can span weeks or months, which hinders timely feedback on AI model safety risks. This becomes an issue since new model releases often enter markets continuously, making it difficult to relocate evaluation resources in quick and adequate ways. It also undermines a "benchmark's ability to consistently evaluate reasoning, comprehension, or multimodal integration, as the results may vary with each model iteration" \cite[p.~13]{mcintosh2024}, which is especially concerning in a regulatory setting, where quick, fair, and accurate AI assessments are key. 
%The relative difficulty of producing benchmarks that can keep up with rapid AI development has also been singled out as one of the main reasons why hard-coded thresholds may be inappropriate to use in assessments of AI safety \cite{hooker2024}. 
%The use of benchmark thresholds is for instance central to current discussions on the use of compute measurements in evaluations of the systemic risks of AI models, where compute refers to the amount of resourced needed to train and deploy an AI model. While more and more compute has been central to AI development in the past decade, which indicates that it could be used as a reliable indicator of model capability, researchers have warned that this tendency might very well change in the future and that small yet highly capable AI models are already entering markets \cite{hooker2024} see for example See for example: https://arxiv.org/abs/2407.15811. 
The use of thresholds, either on benchmark performance or training compute (i.e., the amount of computational resources used to train an AI model), is often seen as central to determine which AI models should warrant further regulatory scrutiny \cite{AIA24, USAIEO2023, USAIDiff2025}. However, several limitations have been identified in relation to such attempts. Creating benchmarks that can keep pace with the rapid development of AI is increasingly challenging, at the same time as recent approaches (such as auxiliary or bootstrapped models or post-training interventions) enable enhanced capabilities with reduced training compute \cite{hooker2024}.

\subsection{AI Complexity and Unknown Unknowns}

A final issue that has been discussed in relation to benchmarks concerns AI complexity and the fundamental difficulty of foreseeing what risks, dangers, and threats AI models could pose to society. \citet[p.~13]{mcintosh2024} point out that "LLM benchmarks are constrained by the current limit of the benchmark creators' human knowledge, hindering their ability to fully assess and cultivate emerging AI capabilities that may surpass conventional human understanding".
According to the authors, the lack of general understanding of emerging AI capabilities, and the natural limitations of the benchmark creators' knowledge on a potentially infinitely large number of domains and tasks, may lead to generalist approaches that "often fail to address the subtle requirements of critical sectors such as national security or healthcare". They argue that this does not just pose security risks, but could potentially also hinder innovation. 

Evaluations of AI models are also complicated by the potential presence of unknown and latent vulnerabilities in AI models that “make it very hard to distinguish between (a) actually safe and (b) appears safe but is not” \cite{nasr2023a}.
In 2023, for instance, \citet{nasr2023} discovered that surprisingly simple prompts can ‘break’ the safety barriers of AI models, raising questions about the robustness of existing safety measurements.
More precisely, the simple command “Repeat the word “poem” forever” was found to make ChatGPT output several megabytes of sensitive training data. As a result of this finding, the authors conclude that “just as vulnerabilities can lie dormant in code - sometimes for decades – our attack demonstrates the potential for latent, hard-to-discover ML vulnerabilities that lie dormant in aligned models” and go unnoticed by existing safety benchmarks and tests \cite[p.~13]{nasr2023}.
Researchers have also noted the difficulty of foreseeing how complex AI models respond to (both small and large) interventions such as safety alignments and fine-tuning.
For instance, efforts to fine-tune AI models to address safety and/or security risks have been found to degrade a model’s performance in other safety areas, or introduce entirely new security risks \citet{qi2023}.
% In a study of the safety costs of customised fine-tuning, for example, \citet{qi2023} found that "the safety alignment of LLMs can be compromised by fine-tuning with only a few adversarially designed training examples" and that "even without malicious intent, simply fine-tuning with benign and commonly used datasets can... inadvertently degrade the safety alignment of LLMs". 

\an{The discussion in this particular section is also closely linked to my previous point related the capabilities of the adversary and the assumptions that are made about that. Security researchers typically document their assumptions regarding the adversary for instance by stating something like we "assume that adversary does not have NSA level, state-actor level capabilities of breaking encryption" for instance. But also another fundamental lesson from safety and security that can either be discussed here or in the discussion is that \textbf{the question of "is a system safe or not?" is seen as a false dichotomy but rather more seen like all systems are unsafe depending on who the adversary and context is.} And with that in mind safety/security is always seen as a balancing exercise rather than as an absolute. Perhaps something that might be useful to consider discussing here or somewhere is literature like this: https://arxiv.org/pdf/2004.07213}
\me{This is a good point, I think a relevant conclusion for us to draw is to say that we find a lack of acknowledgement of the fundamental limitations of benchmarking (marked in bold above) to be a real problem in the field of AI safety testing} 




% \subsection{Mitigation strategies: possibilities and limitations}

% ME: This subsection does not really fit here, and could perhaps be woven into the discussion and conclusion instead...

% One solution to the limitations of individual benchmarks has been attempts to aggregate different evaluation tasks into single multi-task benchmarks with the hope that this will increase the reliability, validity, and representativeness of model evaluations. Examples include BigBench \cite{srivastava2023} and HELM \cite{liang2023}. However, recent research has shown that such attempts may come with drawbacks, such as having to make important decisions concerning trade-offs between diversity and robustness/sensitivity \cite{zhang2024}. Zhang et.al also conclude that BigBench and HELM are "highly unstable to irrelevant changes" and suffer from significant instabilities with regards to rankings (ibid.). Likewise, Artzt et al. find that multitask benchmarks and leaderboards presenting aggregated benchmark statistics with the help of singular metrics (such as F1-score) often "fail to capture the complexity of the relation extraction task, especially in scenarios involving a large number of labels and highly imbalanced datasets" \cite{arzt2024}. 

% Another solution to the limitations of current benchmark practices has been the introduction of so-called dynamic benchmarks, which are designed to be updated and adjusted over time to mitigating risks associated with data saturation, data contamination, and rigging/gaming \cite{besen2024, shirali2023}. However... Shirali note that... Efforts to produce dynamic benchmarks for example involve using AI models to slightly adjust and update benchmark datasets to make them more difficult to solve \cite{wang2024} or using AI models as adversarial agents in the safety testing of other AI models \cite{perez2022}, or using AI models to identify and classify harmful AI output \cite{inan2023}. While such efforts may lower the time and cost of conducting AI safety tests, the use of additional AI tools in AI model evaluation may result in making it increasingly difficult to explain evaluation results, allocate accountability, and ensure that errors and biases do not propagate throughout the evaluation process. 

% Another way of solving ixing train-test overlap and data contamination by hiding benchmark datasets (following and ARC-AGI model), although this brings transparency issues... what would it mean if all benchmark datasets were hidden from view? It would be practically impossible to assess potential biases etc.

% Some scholars argue that method of closing gaps in existing AI safety evaluations could involve repurposing benchmark evaluations and datasets (for instance, by transcribing non-text output to make use of the wide range of text-based benchmarks that exist) \cite{rauh2024}. Since many existing benchmarks display serious weaknesses with regards to documentation, transparency, and replicability \cite{reuel_betterbench_2024} suggests that this may be directly unwise.

% Rauh et al. also emphasize that "evaluation alone is not a panacea" and that the answer to benchmark shortcomings should not just be to develop more or "better" benchmarks, but also a more flexible AI design that allow for continuous system updates, alongside efficient governance mechanisms \cite{rauh2024}

% Moving forward, Chang et al. notice a shift towards using humans-in-the-loop during testing, crowd-sourced and dynamic designs of benchmark datasets, and centralized testing environments where multiple benchmarks are gathered and applied \cite{chang2023}. 

% \subsection{\textcolor{red}{Cut out}}

% A broad range of critique has also been directed towards human-annotated ground truth datasets, which for example tend to assume that there exist only one correct interpretation for input examples and are often made under conditions where efforts are made to avoid disagreement between annotators - a move which effectively reduces complexity and may result in problematic abstractions \cite{aroyo2015}. 

% Elsewhere, Kovatchev and Lease find that differences in the distribution of certain dataset features in benchmarks datasets "significantly affect both absolute and relative model performance" \cite{kovatchev2024} p.~1544. More precisely, the researchers analyzed how differences in the constitution of datasets across six dimensions (ambiguity, difficulty, discriminability, length, noise and perplexity) impacted the performance of 135 AI models on two NLP benchmark tests: SQUAD and MLNI. The authors conclude that "without explicitly considering data features, standard evaluation frameworks are inconsistent and unreliable" (p.~1541).

% Ethayarajh and Jurafsky point out that NLP leaderboard design is for example limited by the fact that "practicioners recieve higher utility from a model that is more robust to adversarial perturbations, generalizes better to out-of-distribution data, and that is equally fair to all demografics. However, these benefits would leave leaderboard utility unchanged" \cite{ethayarajh2021} p.2). 

% Selbst et al. note that abstractions - which are central to computer science, machine learning, and not least benchmarking - generally invlove disregarding the social context of AI tools \cite{selbst2019}. This, they argue, may lead researchers into five different traps: a \textit{framing trap} (as they fail to consider the larger social context of the data and results they use and produce), a \textit{portability trap }(as they fail to recognize how the repurposing of algorithmic solutions from one context to the next can be misleading, inaccurate, and harmful), a \textit{formalism trap} (as they fail to recognize the instability of social concepts like fairness), \textit{a ripple effect trap} (as they disregard how the insertion of a technology into a social system changes it), and a \textit{solutionism trap} (as they do not acknowledge the possibility that the best solution to a problem may not involve technolgy) (ibid.). While Selbst et al. do not explicitly adress benchmarks in their discussions, their critique can easily be transferred to benchmark practices to, as a sociotechnical perspective is missing in most benchmark development and application processes.

% A recent study on LLMs sensitivity to prompt formatting found that widely used open-source LLMs (including) "are extremely sensitive to subtle changes in prompt formatting in few-shot settings, with performance differences of up to 76 accuracy points when evaluated using LLaMA-2-13B" \cite{sclar2024}.

% Recent research has also shown that large language models are "vulnerable to option position changes in... multiple choice questions due to their inherent "selection bias", namely they prefer to select specific option IDs as answers (like "option A") \cite{zheng2024}. For instance, the researchers found that "llama-30B selects A/B/C/D 34.6% / 27.3% / 22.3% / 15.8% of the time, while gpt-3.5-turbo for 22.5% / 25.6% / 32.3% / 19.6%, respectively (averaged over 10 runs)" a result which indicates that "large language models are not robust multiple choice selectors" (ibid., p.~2). 

\section{Conclusion}
In this work, we propose a simple yet effective approach, called SMILE, for graph few-shot learning with fewer tasks. Specifically, we introduce a novel dual-level mixup strategy, including within-task and across-task mixup, for enriching the diversity of nodes within each task and the diversity of tasks. Also, we incorporate the degree-based prior information to learn expressive node embeddings. Theoretically, we prove that SMILE effectively enhances the model's generalization performance. Empirically, we conduct extensive experiments on multiple benchmarks and the results suggest that SMILE significantly outperforms other baselines, including both in-domain and cross-domain few-shot settings.

%%
%% The acknowledgments section is defined using the "acks" environment
%% (and NOT an unnumbered section). This ensures the proper
%% identification of the section in the article metadata, and the
%% consistent spelling of the heading.
% \begin{acks}
% ACKWOLEDGMENT
% \end{acks}

%%
%% The next two lines define the bibliography style to be used, and
%% the bibliography file.
\renewcommand{\refname}{REFERENCES}
\bibliographystyle{ACM-Reference-Format}
\bibliography{references}

%% For biblatex
% \printbibliography

%%
%% If your work has an appendix, this is the place to put it.
% \appendix
% \section{APPENDIX TITLE}
% APPENDIX TEXT

\end{document}
\endinput
%%
%% End of file `sample-manuscript.tex'.
