\begin{abstract}

Quantitative Artificial Intelligence (AI) Benchmarks have emerged as fundamental tools for evaluating the performance, capability, and safety of AI models and systems. Currently, they shape the direction of AI development and are playing an increasingly prominent role in regulatory frameworks. As their influence grows, however, so too does concerns about \textit{how} and with what effects they evaluate highly sensitive topics such as capabilities, including high-impact capabilities, safety and systemic risks. This paper presents an interdisciplinary meta-review of about 100 studies that discuss shortcomings in quantitative benchmarking practices, published in the last 10 years. It brings together many fine-grained issues in the design and application of benchmarks (such as biases in dataset creation, inadequate documentation, data contamination, and failures to distinguish signal from noise) with broader sociotechnical issues (such as an over-focus on evaluating text-based AI models according to one-time testing logic that fails to account for how AI models are increasingly multimodal and interact with humans and other technical systems). Our review also highlights a series of systemic flaws in current benchmarking practices, such as misaligned incentives, construct validity issues, unknown unknowns, and problems with the gaming of benchmark results. Furthermore, it underscores how benchmark practices are fundamentally shaped by cultural, commercial and competitive dynamics that often prioritise state-of-the-art performance at the expense of broader societal concerns. By providing an overview of risks associated with existing benchmarking procedures, we problematise disproportionate trust placed in benchmarks and contribute to ongoing efforts to improve the accountability and relevance of quantitative AI benchmarks within the complexities of real-world scenarios.



% \textcolor{red}{Paper Length: Submissions must be no longer than 14 pages, excluding references. An additional (15th) page is permitted for authors who wish to include endmatter statements on the last page of the paper (and this extra page can only be used for these statements).}

\end{abstract}
